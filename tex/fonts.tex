\defaultfontfeatures{ 
  Path=fonts/,
  Scale=MatchLowercase,
  Extension=.otf,
}

% 
% Text fonts
%
\setmainfont{Alegreya}[
  ItalicFont=*-Italic,
  BoldFont=*-Bold,
  BoldItalicFont=*-BoldItalic,
  SlantedFont=Alegreya,
  SlantedFeatures={FakeSlant=0.13},
  Scale=0.93925,
]

\setsansfont{AlegreyaSans}[
  UprightFont=*-Regular,
  ItalicFont=*-Italic,
  BoldFont=*-Bold,
  BoldItalicFont=*-BoldItalic,
]

\newfontfamily{\scmain}{AlegreyaSC}[
  UprightFont=*-Regular,
  BoldFont=*-Bold,
  LetterSpace=10,
  WordSpace=2.5,
]

\newfontfamily{\scsans}{AlegreyaSansSC}[
  UprightFont=*-Regular,
  BoldFont=*-Bold,
  LetterSpace=10,
  WordSpace=3,
]

\newfontfamily{\letterspacefont}{AlegreyaSans}[
  UprightFont=*-ExtraBold,
  BoldFont=*-Black,
  LetterSpace=15,
  WordSpace=4
]

\newfontfamily{\tablefont}{AlegreyaSans}[
  UprightFont=*-Regular,
  Numbers={Monospaced,Lining}
]

\newfontfamily{\greekfont}{Alegreya}[
  Script=Latin,
]

% 
% Math fonts
%
\defaultfontfeatures{ 
  Path=fonts/,
  Scale=MatchLowercase,
  Extension=.otf,
}

\usepackage{unicode-math}

\setmathfont{STIX2Math}[% operators
  StylisticSet=01 ,
  Scale=MatchLowercase,
]

\setmathfont{Alegreya.otf}[% numbers
  range={up},
  Script=Latin,
  script-features={},
  sscript-features={}
]

\setmathfont{Alegreya-Italic.otf}[% italic letters
  range={it},
  Script=Latin,
  script-features={},
  sscript-features={}
]

%% Alllow cyrilic letters in math
\DeclareSymbolFont{cyrletters}{\encodingdefault}{\familydefault}{m}{it}
%% All letters
\newcommand{\makecyrmathletter}[1]{%
  \begingroup\lccode`a=#1\lowercase{\endgroup
  \Umathcode`a}="0 \csname symcyrletters\endcsname\space #1
}
\count255="409
\loop\ifnum\count255<"44F
  \advance\count255 by 1
  \makecyrmathletter{\count255}
\repeat

%% Fake slant fot г, д, п, т

\DeclareSymbolFont{cyrletterssl}{\encodingdefault}{\familydefault}{m}{sl}
\newcommand{\makecyrmathlettersl}[1]{%
  \begingroup\lccode`a=#1\lowercase{\endgroup
  \Umathcode`a}="0 \csname symcyrletterssl\endcsname\space #1
}
\makecyrmathlettersl{"433} % г
\makecyrmathlettersl{"434} % д
\makecyrmathlettersl{"43F} % п
\makecyrmathlettersl{"442} % т
