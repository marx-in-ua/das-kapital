\parcont{}  %% абзац починається на попередній сторінці
\index{iii2}{0066}  %% посилання на сторінку оригінального видання
загострювати; таким штучним піднесенням — що в рішучий момент стає дуже
значне — попиту на грошовий кредит, тобто на платіжні засоби, при одночасному
обмеженні їхнього подання, — він підносить рівень проценту до нечуваної досі
висоти; отже, замість усувати кризи, він, навпаки, підносить їх до тієї точки,
коли мусить розбитися в щент або ввесь промисловий світ або банковий акт.
Двічі, 25 жовтня 1847 року та 12 листопада 1857 року, криза дійшла до такої
висоти; тоді уряд звільнив банк від обмеження в справі видання банкнот, припинивши
чинність акту 1844 року, й обидва рази цього було досить, щоб перебороти
кризу. В 1847 році досить було певности, що тепер можна знову мати
банкноти під першорядне забезпечення, для того щоб знову витягти на денне
світло та пустити в циркуляцію 4--5 мільйонів нагромаджених банкнот; в
1857 році видано на неповний мільйон банкнот понад дозволену законом кількість,
але тільки на цілком короткий час.

Треба ще згадати й про те, що законодавство 1844 року виявляє ще сліди
спогаду про перші двадцять років віку, про час, коли банк спинив платежі готівкою
та знецінились банкноти. Ще дуже помітний є страх, що банкноти можуть
втратити довір’я до себе; цілком зайвий страх, бо вже в 1825 році видання
знайденого старого запасу однофунтівок-банкнот, що їх забрано було з
циркуляції, перебороло кризу, довівши тим способом, що вже тоді навіть за часів
найпоширенішого та найбільшого, недовір’я, все ж довір’я до банкнот лишилось
непохитним. І це цілком зрозуміло; аджеж фактично ціла нація з її кредитом
стоїть за цими знаками вартости — Ф.~Е].

Послухайте ще кілька свідчень про вплив банкового акту. Дж.~Ст.~Міл
гадає, що банковий акт 1844 року затримав надмірну спекуляцію. Цей премудрий
чолов’яга, на щастя, свідчив 12 червня 1857 року. Чотири місяці по тому
вибухла криза. Він буквально ґратулює «банкових директорів та комерційну
публіку взагалі» з тим, що вони «розуміють тепер краще, ніж раніш, природу
торговельної кризи й ту дуже велику шкоду, що її заподіюють вони самі собі
та публіці, підтримуючи надмірну спекуляцію». (В.~C. 1857, № 2031).

Премудрий Міл думає, що коли однофунтівки-банкноти видається «як позики
фабрикантам та ін., що виплачують заробітні плати\dots{} то ці банкноти
можуть дійти до рук інших людей, що витрачують їх з метою споживання, і в
цьому випадку банкноти самі усталюють попит на товари й можуть тимчасово
сприяти підвищепню цін». Отже, пан Міл припускає, що фабриканти будуть
платити вищу заробітну плату, виплачуючи її в паперових грошах замість золота?
Або може він гадає, що коли фабрикант одержить свою позику стофунтівками-банкнотами
та зміняє їх на золото, то ця заробітна плата становитиме менший
попит, ніж тоді, коли її одразу виплатиться однофунтівками-банкнотами? Хіба
він не знає того, що, напр., в певних гірничих округах заробітну плату платилось
банкнотами місцевих банків, так що кілька робітників одержувало разом
одну п’ятифунтівку-банкноту? Хіба це збільшує їхній попит? Або може банкіри
дрібними банкнотами визичають фабрикантам легше та більші суми грошей,
ніж великими?

[Цього дивного страху Міля перед однофунтівками-банкнотами не можна
було б пояснити, коли б увесь його твір про політичну економію не виявляв
еклектизму, що не лякається жодних суперечностей. З одного боку, в багатьох
речах він визнає рацію Тукові проти Оверстона, з другого боку, він гадає, що
товарові ціни визначається кількістю наявних грошей. Отже, він ніяк не переконався
того, що за кожну видану банкноту — припускаючи всі інші умови
однакові — до скарбниці банку надходить один соверен; він боїться, що маса
засобів циркуляції може збільшитися, а тому й знецінитися, тобто піднести
товарові ціни. Оце те — та й більш нічого, — що ховається за вищенаведеними
побоюваннями. — Ф.~Е].
