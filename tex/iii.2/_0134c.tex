\parcont{}  %% абзац починається на попередній сторінці
\index{iii2}{0134}  %% посилання на сторінку оригінального видання
акрі землі того самого роду лишився незмінний; припущено тільки, — і це кожного
даного моменту відбувається в усякій країні, — що землі різних родів перебувають
у певному відношенні до всієї оброблюваної землі; і що відношення, — а це постійно
відбувається в двох країнах при порівнянні їх одна з однією, або в тій самій
країні за різних часів, — в якому вся оброблювана земельна площа розподіляється
між різними родами землі, змінюється.

Порівнюючи І$а$ з І, ми бачимо, що коли площа оброблюваних земель усіх
чотирьох розрядів зростає в однаковій пропорції, то з подвоєнням кількости
оброблюваних акрів подвоюється вся продукція, а також рента в збіжжі і грошах.

Але порівнюючи послідовно І$b$ з І$с$, ми знайдемо, що в обох випадках площа
оброблюваної землі збільшується утроє. В обох випадках вона збільшується з
4 до 12 акрів, але найбільше збільшення в I$b$ відбувається в розрядах
$а$ і $b$, в яких $а$ не дає жодної ренти, а $b$ дає найменшу
диференційну ренту, а саме з 8 новооброблюваннх акрів на землю $а$ і $b$
припадає по 3, разом 6, тимчасом,
як на $c$ і $d$ припадає лише по 1 акрові, разом 2. Іншими словами: \sfrac{3}{4} приросту
припадають на $а$ і $b$ і лише \sfrac{1}{4} на $c$ і $d$. Коли таке припустити, то в
I$b$ порівняно з І, потроєному збільшенню площі обробленої землі не відповідає
таке саме потроєне збільшення продукту, бо кількість його збільшилась з 10
не до 30, а лише до 26. З другого боку, тому, що значна частина усього приросту
постала на землі $А$, що не дає ренти, а більша частина приросту на
кращих землях постала на розряді $В$, то рента в збіжжі збільшується лише
з 6 до 14 кварт., а грошова рента — з 18 до 42\pound{ ф. стерл}.

Коли ми, навпаки, порівняємо І$с$ з І, де земля, яка не дає ренти, зовсім
не збільшується в розмірі, а земля, що дає мінімальну ренту, збільшується
лише незначно, тоді як найбільший приріст припадає на $C$ і $D$, то ми побачимо,
що при потроєному збільшенні обробленої землі продукція зросла з 10 до
36 квартерів, тобто більш, ніж у три рази; рента в збіжжі збільшилася з 6 до
24 квартерів, або в чотири рази; і так само збільшилась грошова рента
з 18 до 72\pound{ ф. стерл}.

В усіх цих випадках по самій суті справи ціна хліборобського продукту
лишається незмінною; в усіх випадках загальна сума ренти зростає з поширенням
культури, оскільки воно відбувається не виключно на гіршій землі, що
не дає жодної ренти. Але зріст цей різний. В тій самій мірі, в якій поширення
відбувається на кращих землях і в якій, отже, маса продукту зростає не тільки
пропорційно поширенню земельної площи, але швидше, — зростає і рента в
збіжжі і в грошах. В тій самій мірі, в якій поширення відбувається переважно
на найгіршій землі і на близьких до неї родах землі (при чому припускається,
що розряд найгіршої землі лишається той самий), в цій самій мірі загальна
сума ренти збільшується непропорційно поширенню культивованої площі. Отже,
в двох країнах, де земля $А$, що не дає ренти, однакова якістю, сума ренти
буде перебувати у зворотному відношенні до тієї відповідної частини, яку в загальній
площі обробленої землі становлять найгірші і менш родючі землі, а тому
\parbreak{}  %% абзац продовжується на наступній сторінці
