\parcont{}  %% абзац починається на попередній сторінці
\index{iii2}{0036}  %% посилання на сторінку оригінального видання
є лише фіктивна, тобто лише титул на вартість, цілком так, як от знаки
вартости. Оскільки гроші функціонують в кругообороті капіталу, являють вони,
щоправда, в певний момент грошовий капітал; проте вони не перетворюються на
позичковий грошовий капітал, а їх або вимінюють на елементи продуктивного капіталу,
або ж виплачують як засіб циркуляції, реалізуючи дохід, отже й не можуть
вони перетворитися на позичковий капітал для свого державця. Але оскільки
вони перетворюються на позичковий капітал, і та ж сама сума грошей повторно
представляє позичковий капітал, то очевидно, що вони лише на одному пункті
існують як металеві гроші; на всіх інших пунктах вони існують лише в формі
вимоги на капітал. Нагромадження цих вимог, згідно з припущенням, виникає
з дійсного нагромадження, тобто з перетворення вартости товарового капіталу
і~\abbr{т. ін.} на гроші; проте нагромадження цих вимог або титулів, як таке, є відмінне
так від дійсного нагромадження, що з нього воно постав, як і від майбутнього
нагромадження (нового процесу продукції), що обслуговується за посередництвом
визичання грошей.

Prima facie позичковий капітал існує завжди у формі грошей\footnote{
В.~А. 1857. Свідчення банкіра Twells’a: «4516. Як банкір, чи ви робите операції із капіталом
чи з грішми? — Ми провадимо операції з грішми. — 4517. В якій формі платять вклади в ваш
банк? — Грішми. — 4518. Як виплачуєте ви її? — Грішми. — Чи можна отже сказати, що вони є дещо
інше, ніж гроші? — Ні».

Оверстон (див. розд. XXVI) раз-у-раз плутається між «capital» та «money». Value of money
означає в нього також і процент, але остільки, оскільки він визначається масою грошей; value of
capital означає процент, оскільки він визначається попитом на продуктивний капітал та зиском, що
його він
дає. Він каже: «4140. Уживати слово капітал дуже небезпечно. — 4148. Вивіз золота з Англії становить
зменшення кількости грошей в країні й він мусить, природна річ, взагалі викликати збільшений
попит на грошовому ринку» [отже, за Оверстоном, не на ринку капіталу]. — «4112. В міру того, як
гроші
відходять з країни, меншає їхня кількість в країні. Це зменшення кількости грошей, що лишаються в
країні, породжує зріст вартости грошей» [первісно за його теорією це означало піднесення вартости
грошей як грошей проти вартостей товарів, піднесення, викликане скороченням циркуляції; при чому,
отже, це піднесення вартости грошей \deq{} спадові вартости товарів. А що в проміжний час навіть для
нього безперечно
доведено, що маса грошей в циркуляції не визначає цін, то й має тепер зменшення грошей як
засобів циркуляції підвищувати їхню вартість як капіталу, що дає процент, а разом з тим підвищувати
й рівень проценту]. «І це піднесення вартости решти грошей затримує їхній відплив та триває далі
доти, доки поверне воно назад стільки грошей, скільки треба, щоб знову відновити рівновагу». —
Продовження
про суперечності Оверстона буде далі.
}, пізніше
у формі вимоги на гроші, бо гроші, що в них він спочатку існує, існують тепер
тільки в руках позикоємця у дійсній грошовій формі. Для позикодавця позичковий
капітал перетворився у вимогу на гроші, у титул власности. Тому та сама маса
дійсних грошей може становити дуже різні маси грошового капіталу. Просто
гроші — чи становлять вони реалізований капітал, чи реалізований дохід, —
стають позичковим капіталом за допомогою простого акту визичання, за допомогою
перетворення їх у вклад, якщо розглядати загальну форму за розвинутої
кредитової системи. Вклад є грошовий капітал для вкладника. Однак в руках
банкіра він може бути тільки потенціяльним грошовим капіталом, що лежить
без діла в його касі, замість лежати в касі його власника\footnote{
Тут постає плутанина, бо і те й це є «гроші», і вклад як вимога до банкіра на платіж,
і складені гроші в руках банкіра. Банкір Twells наводить такий приклад перед банковою комісією
1857 року: «Я починаю своє підприємство з \num{10.000}\pound{ ф. ст}. На 5000\pound{ ф. ст.} я купую товарів та
беру їх до себе на склад. Другі 5000\pound{ ф. ст.} я складаю в банкіра, щоб брати по потребі. Однак я
розглядаю цілу суму все ще, як свій капітал, дарма що 5000\pound{ ф. ст.} з неї перебувають у формі вкладу
або грошей. (4528)». З цього розгортаються тепер такі цікаві дебати: «4531. Отже, ви дали комусь
іншому свої 5000\pound{ ф. ст.} в банкнотах? — Так. — 4532. Отож ця особа має тепер вклад в 5000\pound{ ф. ст.} —
Так. — 4533. І ви маєте вклад в 5000\pound{ ф. ст.}? — Цілком слушно. — 4534. Вона має 5000 ф. ст; грішми,
і ви маєте 5000\pound{ ф. ст.} грішми? — Так. — 4535. Але ж кінець-кінцем це — не що інше як гроші? —
Ні». Плутанина виникає почасти з такої причини: $А$, що склав вклад в 5000\pound{ ф. ст.}, може брати
з неї собі частину, порядкує ними так само, як коли б він ще їх мав при собі. Остільки вони
функціонують для нього як потенціяльні гроші. Але в усіх випадках, коли він бере собі якусь частину,
він нищить свій вклад pro tanto. Якщо він бере з банку дійсні гроші, — а його гроші вже визичено
далі, — то йому платять не його власними грішми, а грішми, що їх склав хтось інший. Коли він
платить $В$ борг чеком на свого банкіра, і $В$ складає цей чек вкладом у свого банкіра, а банкір того
вкладника $А$ має теж чек на банкіра вкладника $В$, так що обидва банкіри тільки вимінюють ці чеки,
то гроші, складені $А$, двічі виконали грошову функцію; поперше, в руках того, хто одержав гроші,
складені
$А$; подруге, в руках самого $А$. В другій функції це є вирівнюванння боргових вимог (боргова вимога $А$
до
свого банкіра та боргова вимога останнього до банкіра $В$) без посередництва грошей. Тут вклад діє
двічі як гроші, а саме одного разу як дійсні гроші, а потім як вимога на гроші. Просто вимоги на
гроші можуть заступати місце грошей лише через вирівнювання боргових вимог.
}.

\index{iii2}{0037}  %% посилання на сторінку оригінального видання
Зі зростом матеріяльного багатства зростає кляса грошових капіталістів;
з одного боку, більшає число й багатство капіталістів, що відходять від справ, —
рантьє; з другого боку, складаються сприятливі обставини для розвитку кредитової
системи, а разом з цим більшає число банкірів, грошових позикодавців, фінансистів
і~\abbr{т. ін.} — З розвитком вільного грошового капіталу збільшується кількість
процентодайних паперів, державних паперів, акцій тощо, як це ми вже раніше
розвинули. Але одночасно з цим зростає попит на вільний грошовий капітал, при
чому Jobbers (маклери), що спекулюють тими паперами, відіграють головну ролю на
грошовому ринку. Коли б усі купівлі та продажі цих паперів означали лише
дійсне приміщення капіталу, то мали б рацію сказати, що вони не можуть
впливати на попит позичкового капіталу, бо, коли $А$ продає свій папір, він
забирає саме стільки грошей, скільки $B$ приміщує в той папір. Тимчасом навіть
тоді, коли папір хоч і існує, але немає того капіталу (принаймні, як грошового
капіталу), що його той папір первісно представляв, — навіть тоді він, цей папір,
завжди породжує pro tanto попит на такий грошовий капітал. Але в усякому
разі це — той грошовий капітал, що ним порядкував спочатку $В$, а тепер
порядкує $А$.

В.~А. 1857. № 4886: «Чи на вашу думку слушно зазначено причини,
що визначають норму дисконту, коли я кажу, що її регулюється масою капіталу
на ринку, уживаного для дисконту торговельних векселів, у відміну від
інших родів цінних паперів? — [Chapman:] Ні; я тримаюсь тієї думки, що на
рівень проценту впливають усі ті цінні папери, які легко перетворюються
на гроші (all convertible securities of a current character); було б неслушно
обмежувати це питання лише на дисконті векселів; бо, коли є великий попит
на гроші під [заставу] консолів або навіть посвідок державної скарбниці — як
це нещодавно дуже часто траплялося — та ще й за процент, куди вищий за
торговельний процент, то було б абсурдом казати, що це не зачіпає нашого торговельного
світу; це зачіпає його дуже й дуже. — 4890. Коли на ринку є добрі
та ходові цінні папери, що їх банкіри визнають за такі, і коли власники
хочуть узяти в позику гроші під ці папери, то, звичайно, це матиме свій вплив на
торговельні векселі; я не можу, напр., сподіватися, що якась особа дасть мені
свої гроші під торговельний вексель за 5\%, коли вона того самого часу може
визичити їх за 6\% під консолі і~\abbr{т. ін.}; таким самим способом це впливатиме й на
нас; ніхто не може від мене вимагати, щоб я дисконтував його векселі за 5\sfrac{1}{2}\%,
коли я маю змогу визичити свої гроші за 6\% — 4892. Про людей, що на 2000\pound{ ф. ст.}, або на 5000\pound{ ф. ст.}, або ж на \num{10.000}\pound{ ф. ст.} купують цінні папери, вважаючи
їх за добре приміщення капіталу, ми не кажемо, що вони значно впливають
на грошовий ринок. Коли ви питаєте мене про рівень проценту під [заставу]
консолів, то я кажу про людей, що роблять операції на сотні тисяч, про так
званих Jobbers’iв, що підписують або купують на ринку громадські позики на
великі суми, а потім мусять тримати ці папери, поки матимуть змогу збути їх
з зиском; ці люди мусять позичати для цього гроші».

З розвитком кредитової справи утворюються великі концентровані грошові
ринки, як от Лондон, що одночасно є головний центр торговлі цими паперами.
Банкіри дають банді цих торговців до розпорядку маси грошового капіталу публіки,
і так зростає це кодло цих грачів. «На фондовій біржі гроші звичайно
\parbreak{}  %% абзац продовжується на наступній сторінці
