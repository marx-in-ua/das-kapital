\parcont{}  %% абзац починається на попередній сторінці
\index{iii2}{0028}  %% посилання на сторінку оригінального видання
позик, але фігурує вона як їхній резерв, і здебільша й як резерв Англійського
банку, де ті вклади складено. Насамкінець, той самий пан каже: floating
capital — це bullion, тобто зливки золота та металеві гроші. (503). Взагалі предивна
річ, як в отих теревенях грошового ринку про кредит усі категорії політичної
економії набувають інший зміст та іншу форму. Floating capital є тут
вираз для circulating capital,\footnote*{
circulating capital — оборотний капітал. Пр. Ред.
} що певне, є цілком інша річ, і money\footnote*{
money — гроші Пp. Ред.
} є capital,
і bullion\footnote*{
bullion — зливки Пр. Ред.
} є capital, і банкноти в circulation, і капітал є a commodity,\footnote*{
commodity — товар Пр. Ред.
} і борги є
commodities, і fixed capital\footnote*{
fixed capital — основний капітал. Пр. Ред.
} є гроші, приміщені в паперах, що їх тяжко продати!

«Лондонські акційні банки\dots{} збільшили свої вклади від \num{8.850.774} ф. ст.
в 1847~\abbr{р.} до \num{43.100.724} ф. ст. 1857 році\dots{} Відомості та свідчення, подані
комісії, дають змогу зробити висновки, що значну частину цієї величезної суми
добуто з джерел, раніше не використовуваних для цієї мети; та що звичай відкривати
рахунок в банкіра та складати в нього гроші поширився на численні
джерела, раніше не використовувані для цієї мети; що звичай відкривати рахунок
в банкіра та складати в нього гроші поширився на численні кляси, які
раніше не приміщували свого капіталу (!) таким способом. Пан Rodwell, президент
асоціації провінціяльних приватних банків» [у відміну від акційних
банків], «делеґований нею, щоб скласти свідчення перед комісією, повідомляє,
що в околиці Іпсвіча цей звичай серед фармерів та дрібних торговців тієї
округи за останній час збільшився учетверо; що майже всі фармери, навіть ті,
що платять оренди тільки 50 ф. ст., мають тепер вклади в банках. Маса
цих вкладів, природно, знаходить собі шлях до вжитку в підприємствах та
тяжить особливо до Лондону, центру комерційної діяльности, де вона знаходить
собі ужиток насамперед в дисконті векселів та в інших позиках клієнтам лондонських
банкірів. Однак, значна частина їх, що на неї сами банкіри не мають
безпосереднього попиту, йде до рук billbrokers’iв, які дають банкірам замість
неї торговельні векселі, вже раз дисконтовані ними, billbrokers’ами, для людей
з Лондону та провінції». (В. C. 1858, р. 8).

Коли банкір дає позику billbrokers’oвi під вексель, що його цей останній
вже раз дисконтував, то фактично він ще раз редисконтує його; однак в дійсності
дуже багато з цих векселів вже були billbrokers’oм редисконтовані, й на
ті самі гроші, що ними банкір редисконтує векселі billbrokers’a, цей останній
редисконтує нові векселі. Ось до чого це доводить: «Поширення фіктивних кредитів
утворювалося дисконтуванням векселів (Akkomodationswechsel) та бланковим
кредитом, що дуже полегшувалось поведінкою провінціяльних акційних банків,
які дисконтували такі векселі, редисконтуючи їх потім у billbrokers’iв на лондонському
ринку і то лише на основі кредитоспроможности самого банку, не
зважаючи на іншу якість векселя». (І. c.).

Про це редисконтування та про те сприяння кредитовим спекуляціям, що
його учинює це лише технічне збільшення позичкового грошового капіталу, цікаве
таке місце з Economist’а: «Протягом багатьох років в деяких округах країни
капітал» [власне позичковий грошовий капітал] «нагромаджувався швидше, ніж
його можна було уживати, тимчасом коли в інших округах засоби його приміщення
зростали швидше, ніж сам капітал. Отже, тимчасом коли в хліборобських округах
банкіри не знаходили нагоди примістити вклади в своїх власних околицях з
зиском та з певністю, в банкірів по промислових округах та торговельних містах
був більший попит на капітал, ніж вони того капіталу могли постачити. Вплив
такого відмінного по різних округах становища зумовив останніми роками
\parbreak{}  %% абзац продовжується на наступній сторінці
