
\index{iii2}{0245}  %% посилання на сторінку оригінального видання
Отже, вся частина вартости товарів, в якій реалізується вся праця робітників,
приєднувана протягом одного дня або одного року, сукупна вартість річного
продукту, яку створює ця праця, розпадається на вартість заробітної плати,
зиск і ренту. Бо вся ця праця розпадається на потрібну працю, що нею робітник
створює ту частину вартости продукту, якою він сам оплачується, тобто заробітну
плату, і на неоплачену додаткову працю, що нею він створює ту частину
вартости продукту, яка становить додаткову вартість, і опісля розпадається на
зиск і ренту. Крім цієї праці, робітник не виконує жодної іншої праці, і крім
усієї вартости продукту, яка набуває форми заробітної плати, зиску, ренти, він
не створює жодної вартости. Вартість річного продукту, що в ній втілюється праця,
новоприлучена ним протягом року, дорівнює заробітній платі або вартості змінного
капіталу плюс додаткова вартість, яка й собі розпадається на форми зиску
і ренти.

Отже, вся частина вартости річного продукту, створювана робітником протягом
року, виражається в річній сумі вартости трьох доходів: вартости заробітної
плати, зиску і ренти. Тому, очевидно, що в створюваній за рік вартості
продукту, вартість сталої частини капіталу не репродукується, бо заробітна
плата дорівнює лише вартості змінної частини капіталу, авансованої на продукцію,
а рента та зиск дорівнюють лише додатковій вартості, створеному надмірові
вартости над усією вартістю авансованого капіталу, що дорівнює вартості
сталого капіталу плюс вартість змінного капіталу.

Для тих труднощів, які треба тут розв’язати, цілком байдуже, що частину
додаткової вартости, перетвореної на форми зиску і ренти, не споживається як
дохід, а служить вона для акумуляції. Та її частина, яка зберігається як фонд акумуляції, служить
для створення нового, додаткового капіталу, але не для покриття
старої складової частини колишнього капіталу, витраченої чи то на
робочу силу, чи то на засоби праці. Отже заради спрощення тут можна вважати,
що доходи цілком пішли на особисте споживання. Труднощі є подвійні.
З одного боку: вартість річного продукту, що в ньому споживаються ці
доходи, — заробітна плата, зиск, рента, — містить у собі певну частину вартости,
рівну частині вартости сталої частини капіталу, що ввійшла в нього. Вона містить
в собі цю частину вартости, крім частини вартости, яка зводиться до заробітної
плати, і крім тієї частини вартости, яка розпадається на зиск і ренту. Отже,
вартість річного продукту = заробітній платі + зиск + рента $+ C$, при чому
останнє становить сталу частину його вартости. Яким же чином випродукована
за рік вартість, яка дорівнює лише заробітній платі + зиск + рента, може
купити продукт, що його вартість = (заробітна плата + зиск + рента) $+ C$?
Яким чином випродукована за рік вартість може купити продукт, що має
більшу вартість, ніж вона сама?

З другого боку: коли ми лишимо осторонь ту частину сталого капіталу,
що не ввійшла в продукт, і яка тому, хоч і зі зменшеною вартістю, існує далі
і після річної продукції товарів, як і до неї; отже, коли ми на час абстрагуємося
від застосованого у продукції, але не зужиткованого основного капіталу, то
виявиться, що стала частина авансованого капіталу в формі сирових і допоміжних
матеріялів цілком увійшла в новий продукт, тимчасом як частина засобів продукції
цілком зужиткована, а друга — лише почасти використана, так що тільки
частину її вартости зужитковано у продукції. Вся ця зужиткована у продукції частина
сталого капіталу мусить бути покрита in natura. Припускаючи, що всі інші
умови, особливо продуктивна сила праці, не змінились, для її покриття потрібна
буде та сама кількість праці, як і давніш, тобто вона мусить бути
покрита еквівалентом вартости. Коли ж цього не станеться, то сама репродукція
не може відбуватись у колишньому маштабі. Але хто повинен провадити ці
праці і хто їх провадить?
