\parcont{}  %% абзац починається на попередній сторінці
\index{iii2}{0139}  %% посилання на сторінку оригінального видання
ринок в готовому вигляді ті продукти, от-як одіж, знаряддя тощо, які в інших
обставинах їм довелося б виробляти самим. Тільки на такій базі південні штати
Союзу і могли зробити бавовну своїм головним продуктом. Поділ праці на світовому
ринку дає їм цю можливість. Коли тому здається, що вони, беручи
на увагу їхню молодість і відносну нечисленність людности, продукують дуже
великий надмірний продукт, то це не в наслідок родючости їхнього ґрунту і не
в наслідок продуктивности їхньої праці, а однобічної форми їхньої праці, отже,
і того надмірного продукту, в якому ця праця виявляється.

Але далі, відносно менш родюча орна земля, що почала оброблятись уперше
і ще не була зачеплена будь-якою культурою, за більш-менш сприятливих
кліматичних умов має, принаймні у верхніх шарах, так багато легко розчинюваних
речовин, живних для рослин, що вона довгий час дає урожай, без будьякого
добрива, до того ж при найповерховішім обробітку. Щодо західніх прерій,
то сюди приєднується ще те, що на їх обробіток майже не треба підготовчих
витрат, бо вони придатні для оброблення вже з природи.\footnote{
[Саме швидкий зріст оброблення таких прерій і степових місцевостей за останнього часу і звів до
рівня дитячого жарту славнозвісну засаду Мальтуса: «Людність тисне на засоби існування», і в
протилежність цьому породив скарги аграріїв на те, що хліборобство, а разом з ним і Німеччина
загинуть, коли насильницькими заходами не усунути засобів існування, що тиснуть на людність. Але
обробіток цих степів, прерій, пампасів і льяносів тощо ще тільки починається; тому його
революціонізаційний вплив на европейське сільське господарство згодом дасться в знаки цілком інакше,
ніж до цього часу. — \emph{Ф. Е.}].
}. В менш родючих
краях цього роду надмір походить не з високої родючости ґрунту, тобто не з
високого продукту на акр, а з великої кількости акрів, які можуть бути поверхово
оброблені, бо сама ця земля або нічого не коштує хліборобові, або проти
старих країн коштує надзвичайно мало. Наприклад, там, де існує відчастинна
(Métairie) оренда, як в декотрих частинах Ныо-Йорку, Мічіґену, Канади, тощо.
Одна родина поверхово обробляє, скажімо, 100 акрів, і хоч кількість продукту,
що дає акр, невелика, з 100 акрів це дає значний надмір для продажу. До
цього приєднується ще майже дарове утримання худоби на природних пасовиськах,
без штучних лук. Переважне значіння має тут не якість, а кількість землі.
Можливість такого поверхового обробітку, природно, більш або менш швидко
вичерпується, в зворотному відношенні до родючости нової земли і в прямому відношенні до вивозу її
продукту. «А проте така земля дає чудові перші врожаї, навіть
пшениці; той, хто бере перший взяток з землі, може послати на ринок великий
надмір пшениці» (1. c, р. 224). В країнах старої культури відносини власности,
ціна необроблюваної землі, визначувана ціною оброблюваної тощо, унеможливлюють
таке екстенсивне господарство.

Але, що — всупереч думці Рікардо — ця земля не повинна неодмінно бути
дуже родюча, а також не має потреби в тому, щоб оброблювались лише землі,
однакові своєю родючістю, це видно з такого: в штаті Мічіґені 1848 року засіяно
пшеницею 463.900 акрів і випродуковано 4.739.300 бушелів, або пересічно по
10\sfrac{1}{5} бушеля на акр; по вирахуванні насіння це дає менш за 9 бушелів на
акр. З 29 округ штату 2 продукували пересічно 7 бушелів, 3--8, 2--9, 7--10, 6--11, 3--12, 4--13 бушелів, і
лише одна округа — 16 бушелів і ще одна — 18 бушелів на акр (1. c., р. 226).

Для хліборобської практики більша родючість ґрунту збігається з можливістю
більшого негайного використання цієї родючости. Можливість ця може бути
більша щодо бідного з природи ґрунту, ніж щодо ґрунту з природи багатого;
але це той сорт землі, що до нього колоніст візьметься насамперед і за браком
капіталу мусить до нього взятись.

\emph{Нарешті}, поширення культури на більші площі — лишаючи осторонь щойно розглянений випадок, коли
доводиться звертатися до землі гіршої якости,
\parbreak{}  %% абзац продовжується на наступній сторінці
