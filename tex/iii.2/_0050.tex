
\index{iii2}{0050}  %% посилання на сторінку оригінального видання
В інших випадках абсолютний розмір циркуляції не впливає на рівень
проценту, поперше, тому, що — припускаючи економію та швидкість циркуляції
як сталі — той розмір циркуляції визначається цінами товарів та кількістю операцій
(при чому, здебільша, один момент паралізує вплив другого) та, насамкінець,
станом кредиту, тимчасом коли навпаки сам той розмір ніяк не визначає
цих факторів; і, подруге, тому, що товарові ціни та процент не мають між собою
ніякого неминучого зв’язку.

За тих часів, коли мав силу Bank Restriction Act (1797--1820 p. p.),
був надмір засобів циркуляції, рівень проценту був завжди далеко вищий, ніж
тоді, коли відновили платежі готівкою. Він швидко впав пізніше, коли обмежили
видання банкнот та підвищились вексельні курси. В 1822,1823, 1832 роках
загальний розмір циркуляції був низький, рівень проценту теж низький. В 1824,
1825, 1836 роках розмір циркуляції був високий, рівень проценту піднісся.
Улітку 1830 року циркуляція була висока, рівень проценту низький. Від часу
відкриття нових покладів золота розмір циркуляції грошей поширився по цілій
Европі, рівень проценту підвищився. Отже рівень проценту не залежить від
кількости грошей, що перебувають в циркуляції.

Ріжниця між випуском засобів циркуляції та визичанням капіталу найкраще
виявляється в дійсному процесі репродукції. Розглядаючи його, ми бачили (Книга II,
відділ III), як обмінюється різні складові частини продукції. Напр., змінний капітал
речово складається з життьових засобів робітників, з частини їхнього власного
продукту. Але його виплачують їм частинами у грошах. Ці гроші мусить
авансувати капіталіст, і від організації кредитової справи дуже залежить, чи
зможе він ближчого тижня знову виплатити новий змінний капітал старими
грішми, що він їх платив минулого тижня. Те саме бачимо ми в актах обміну
між різними складовими частинами сукупного суспільного капіталу, напр., між
засобами спожитку та засобами продукції тих засобів спожитку. Гроші, як ми
бачили, мусять для циркуляції авансуватися однією або обома особами, що обмінюються.
Потім гроші лишаються в циркуляції, але по закінченні обміну вони
раз-у-раз вертаються назад до того, хто їх авансував, бо він авансував їх зверх
свого дійсно занятого промислового капіталу (див. Книга II, 20 розділ). За розвинутої
кредитової справи, коли гроші концентруються в руках банків, ці останні,
принаймні, номінально, являють ту установу, яка авансує гроші. Це авансування
стосується тільки до тих грошей, що перебувають в циркуляції. Це — авансування
засобів циркуляції, а не авансування капіталів, що їх циркуляція обумовлюється
цим авансуванням.

Chapman: «5062. Може надійти час, коли банкноти в руках публіки становитимуть
дуже велику суму, а проте їх не можна добути». Гроші є й підчас
паніки; але кожен стережеться перетворювати їх на позичковий капітал, на
позичкові гроші; кожен міцно тримає їх для дійсної платіжної потреби.

«5099. Чи посилають банки сільських округ свої надміри вільних грошей
до вас та до інших лондонських фірм? — Так, — 5100. З другого боку, чи дисконтують
у вас фабричні округи Ланкашайру та Іїоркшайру векселі для своїх
промислових цілей? — Так. — 5101. Отже, цим способом зайві гроші однієї частини
країни стають пожиточні для потреб другої частини країни? — Цілком слушно».

Chapman каже, що звичай банків уживати свій надмірний грошовий капітал
на короткий час на купівлю консолів та посвідок державної скарбниці, цей звичай
за останній час дуже обмежився від того часу, коли стало звичаєм визичати ці
гроші at call (з дня на день, маючи змогу кожного часу вимагати їх назад).
Сам він вважає купівлю таких паперів для свого підприємства за незвичайно
недоцільну. Тому він приміщує гроші в добрі векселі, що для частини їх щодня
надходить реченець, так що він завжди знає, на скільки вільних грошей вік
має рахувати кожного дня. (5001--5005).
\parbreak{}  %% абзац продовжується на наступній сторінці
