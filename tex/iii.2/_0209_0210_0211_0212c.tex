\parcont{}  %% абзац починається на попередній сторінці
\index{iii2}{0209}  %% посилання на сторінку оригінального видання
виявиться, що в наслідок збільшення маси ренти підвищився і її рівень. Той
самий акр, що давав 2\pound{ ф. стерл.} ренти, дає тепер 4\pound{ ф. стерл.}\footnote{
Одна з заслуг Родбертуса, що до його важливої праці про ренту ми вернемося в книзі IV,
є в тому, що він розвинув цей пункт. Але, поперше, він помиляється, припускаючи, ніби для капіталу
зріст зиску завжди виявляється як і зріст капіталу, так що при збільшенні маси зиску відношення
лишається
те саме. А проте, це невірно, бо коли склад капіталу змінюється, норма зиску, не зважаючи на
незмінну експлуатацію праці, може підвищитись саме тому, що відносна вартість сталої частини
капіталу
проти змінної знизилася. — Подруге, він помиляється, трактуючи це відношення грошової ренти до
кількісно певної дільниці землі, наприклад, до одного акра, як щось таке, що взагалі припускає
тисячна
економія в її дослідженнях про підвищення або пониження ренти. Це знов невірно. Вона постійно
розглядає норму ренти у відношенні до продукту, — оскільки вона розглядає ренту в її натуральній
формі, — і у відношенні до авансованого капіталу, — оскільки вона розглядає ренту як грошову ренту,
—
бo це в дійсності є раціональні вирази.
}.

Відношення певної частини додаткової вартости, грошової ренти, — бо гроші
є самостійний вираз вартости, — до землі само по собі є безглузде й іраціональне;
бо це не співмірні величини, що тут виміряються одна одною, певна
споживна вартість, дільниця землі на стільки-от квадратових футів з одного
боку, і вартість, точніше, додаткова вартість — з другого. В дійсності це не виражає
нічого іншого, а тільки те, що в даних відносинах власність на стільки-от квадратових
футів землі дає земельному власникові можливість уловлювати певну кількість
неоплаченої праці, реалізованої капіталом, який риється на цих квадратових футах,
як свиня у картоплі (в рукопису тут стоїть в дужках, але закреслене: Лібіх). Але
prima facie цей вираз є те саме, як коли б ми здумали говорити про відношення
п’ятифунтової банкноти до діяметра землі. Однак, до посередництва тих іраціональних
форм, в яких виступають і на практиці резюмуються певні економічні
відносини, практичним носіям цих відносин у їхньому житті-бутті немає
жодного діла; а що вони привикли рухатися в цих посередницьких відносинах,
то їхній розум ані трохи не спотикається на них. Цілковита суперечність для них
не має рішуче нічого таємничого. У формах проявлення, відчужених від внутрішнього
зв’язку і безглуздих, коли їх узяти самих по собі, вони почувають себе
так само вдома, як риба у воді. Тут справедливе те, що Геґель сказав про
відомі математичні формули: те, що звичайний людський розум вважає за
і раціональне, є раціональне, а раціональне для нього є сама і раціональність.

Отже, коли розглядати справу у відношенні до самої площі землі, то підвищення
маси ренти виражається цілком так само, як підвищення норми ренти;
а звідси труднощі, що постають, коли умови, які пояснювали б один випадок,
відсутні в іншому випадку.

Але ціна землі може підвищитись навіть тоді, коли ціна продукту землі
зменшується.

В цьому випадку в наслідок дальшого диференціювання може збільшитися
диференційна рента, а тому й ціна кращих земель. Або ж коли цього немає, то
при збільшеній продуктивній силі праці ціна хліборобського продукту може понизитись,
але так, що це буде більш, ніж урівноважено збільшенням продукції. Припустімо,
що квартер коштував 60\shil{ шил.} Коли на тім самім акрі при тому самому
капіталі будуть випродуковані 2 квартери замість одного, і квартер понизиться
до 40 шил, то 2 квартери дадуть 80 шил, так що вартість продукту того самого
капіталу на тому самому акрі підвищиться на одну третину, хоч ціна акра
понизилась на одну третину. Як це можливо без того, щоб продукт продавався
вище його ціни продукції або вартости, було показано при дослідженні диференційної
ренти. В дійсності це можливо тільки в два способи. Або гіршу
землю вилучається з конкуренції, але ціна кращої землі зростає, коли
диференційна рента зростає, отже, коли загальне поліпшення діє нерівномірно
на різні роди землі. Або ж на найгіршій землі та сама ціна продукції
(і та сама вартість, коли виплачується абсолютну ренту) в наслідок підвищення
\index{iii2}{0210}  %% посилання на сторінку оригінального видання
продуктивности праці виражається у збільшеній масі продукту. Продукт
становить тепер ту саму вартість, що й давніш, але ціна його складових
частин понизилася, тимчасом, як число цих частин збільшилося. Коли вживається
той самий капітал, це неможливе, бо в цьому випадку та сама вартість
виражається в якій завгодно масі продукту. Але це можливе, коли витрачено
додатковий капітал на гіпс, гуано тощо, коротко кажучи, на такі поліпшення,
що вплив їхній триває багато років. Умова цього є в тому, щоб ціна одного квартера
хоч і знизилась, але не в такому самому відношенні, як зростає число квартерів.

III.~Ці різні умови підвищення ренти, а тому і ціни землі взагалі або окремих
родів землі, можуть почасти конкурувати між собою, почасти вони виключають
одна одну і можуть діяти лише навперемінки. Але з вище розвинутого, випливає,
що з підвищення ціни землі не можна без дальших околичностей робити
висновку, що рента підвищилась, і з підвищення ренти, яке завжди спричинює
підвищення ціни землі, не можна без дальших околичностей робити висновку,
що продукт землі збільшився\footnote{
Про падіння земельних цін при підвищенні ренти як про факт дивись Passy.
}.

\pfbreak

Замість звернутися до дійсних природних причин виснаження ґрунту, які,
проте, в наслідок стану хліборобської хемії в той час були невідомі усім економістам,
що писали про диферецційну ренту, — по допомогу звернулися до того
поверхового погляду, що в просторово обмежений лан не можна вкласти необмежену
масу капіталу; паприклад, Westminster Rewiew заперечує Річардові
Джонсові, що не можливо було б прогодувати цілу Англію обробітком Soho Square.
Хоч це вважається за особливу невигоду хліборобства, але справедливе як раз
зворотне. У хліборобстві можна продуктивно провадити послідовні приміщення
капіталу тому, що сама земля діє як знаряддя продукції, тимчасом як цього зовсім
немає, або є лише в дуже вузьких межах у випадку з фабрикою, де земля
функціонує лише як фундамент, як місце, як просторова операційна база. Правда,
можна — так і робить велика промисловість — саме на відносно невеликім, проти
парцельованого ремесла, просторі концентрувати велику продукційну споруду.
Але за даного ступеня розвитку продуктивної сили завжди потрібен певний простір,
і будування в висоту теж має свої певні практичні межі. Поширення продукції
за ці межі потребує і поширення простору землі. Основний капітал, вкладений
у машини тощо, не поліпшується споживанням, а навпаки, зношується. Внаслідок
нових винаходів і тут можуть статися окремі поліпшення, але, припускаючи
даний ступінь розвитку продуктивної сили, машина при споживанні може
лише погіршуватись. При швидкому розвитку продуктивної сили всю сукупність
старих машин доводиться заміняти вигіднішими, отже, вони гинуть. Навпаки,
земля, коли вона правильно обробляється, дедалі поліпшується. Та перевага
землі, що послідовні приміщення капіталу можуть дати вигоду без втрати колишніх,
одночасно має в собі можливість різної продуктивности цих послідовних
приміщень капіталу.

\section{Генеза капіталістичної земельної ренти}

\subsection{Вступ.}

Треба з’ясувати собі, в чому власне є труднощі трактування земельної
ренти з погляду сучасної економії, як теоретичного виразу капіталістичного
способу продукції. Цього ще не розуміє навіть величезне число новітніх письменників,
про що свідчить всяка нова спроба з’ясувати земельну ренту «по
новому». Новіша тут майже завжди є в повороті до давно вже побореного погляду.
\index{iii2}{0211}  %% посилання на сторінку оригінального видання
Трудність не в тому, щоб взагалі з’ясувати створений хліборобським
капіталом додатковий продукт і відповідну до нього додаткову вартість. Це
питання, радше, є вже розв’язане аналізою додаткової вартости, створюваної
всяким продуктивним капіталом, хоч би в яку сферу він був вкладений. Трудність
є в тому, що треба показати, звідки після того, як додаткова вартість
вирівнялась між різними капіталами на пересічний зиск, на відповідну до
їхніх відносних величин пропорційну частину всієї додаткової вартости, створеної
всім суспільним капіталом у всіх сферах продукції, — звідки після цього вирівняння,
після того як розподіл усієї додаткової вартости, яка взагалі може
бути розподілена, вже очевидно стався — звідки ж тут після цього береться
ще й та надмірна частина цієї додаткової вартости, яку капітал, вкладений
в землю, виплачує в формі земельної ренти земельному власникові.
Цілком лишаючи осторонь практичні мотиви, які спонукали сучасних економістів
як оборонців промислового капіталу проти земельної власности
досліджувати це питання, — мотиви, які ми накреслимо ближче в розділі
про історію земельної ренти, — це питання становило для них, як для теоретиків,
переважний інтерес. Визнати, що появлення ренти на капітал, вкладений
в хліборобство, завдячує особливій дії самої сфери приміщення, властивостям,
належним земній корі, як такій, це значило б відмовитись від самого
поняття вартости, отже, відмовитися від усякої можливости наукового
пізнання в цій галузі. Саме звичайне спостереження, що ренту виплачується
з ціни продукту землі, а це так і є навіть в тому випадку, коли її виплачується
в натуральній формі, скоро тільки орендар здобуває свою ціну продукції,
— показує, оскільки безглуздо надмір цієї ціни над звичайною ціною
продукції, отже, відносну дорожнечу хліборобського продукту, пояснювати надміром
природної продуктивности хліборобської промисловости над продуктивністю
інших галузей промисловости; бо, навпаки, що продуктивніша праця, то
дешевша кожна складова частина її продукту, тому що тим більша маса споживних
вартостей, в якій репрезентована та сама кількість праці, отже, та сама вартість.

Отже, при аналізі ренти вся трудність була в тому, що треба було
пояснити надмір хліборобського зиску над пересічним зиском, з’ясувати не додаткову
вартість, а властиву цій сфері продукції надмірну додаткову вартість,
отже, знов таки не «чистий продукт», а надмір цього чистого продукту над
чистим продуктом інших галузей промисловости. Сам пересічний зиск є продукт,
витвір процесу соціального життя, що відбувається в цілком певних історичних
продукційних відносинах, продукт, що має своєю передумовою, як ми
бачили, дуже широкосяжні посередницькі ланки. Для того, щоб взагалі можна було
говорити про надмір над пересічним зиском, сам цей пересічний зиск мусить
взагалі скластися як маштаб і — як це відбувається за капіталістичного способу
продукції, — як регулятор продукції. Отже, в таких суспільних формах, де ще
немає капіталу, який виконує ту функцію, що вимушує всю додаткову працю
і привласнює в першу чергу собі всю додаткову вартість, отже, де капітал ще
не упідлеглив собі суспільної праці, або упідлеглив її лише місцями, — взагалі
не може бути мови про ренту в сучасному значенні, про ренту як надмір
над пересічним зиском, тобто над пропорційною частиною всякого індивідуального
капіталу в додатковій вартості, спродукованій усім суспільним капіталом. Те
що, наприклад, пан Раssy (дивись далі) говорить вже про ренту в первісному стані
як про надмір над зиском, як про надмір над історично-певного суспільною
формою додаткової вартости, так що за п. Раssy ця форма могла б, мабуть,
існувати і без суспільства, — свідчить лише про його наївність.

Для колишніх економістів, які взагалі лише починали аналізу капіталістичного
способу продукції, ще нерозвиненого за їхнього часу, аналіза ренти
або взагалі не становила жодних труднощів, або становила лише труднощі цілком
\index{iii2}{0212}  %% посилання на сторінку оригінального видання
іншого характеру. Петті, Кантільйон, взагалі письменники, що ближче стоять
до доби февдалізму, беруть земельну ренту як нормальну форму додаткової
вартости взагалі, тимчасом як зиск для них ще не визначився і поєднується
з заробітною платою, або, щонайбільше, виступає як та частина цієї
додаткової вартости, що її капіталіст витискує з земельного власника. Отже, вони
виходять з такого стану, коли, поперше. хліборобська людність становить ще рішучу
переважну частину нації, і коли, подруге, земельний власник ще є тією
особою, яка, користуючись монополією земельної власности, у першу чергу привласнює
надмірну працю беспосередніх продуцентів, коли, отже, земельна власність
все ще є головна умова продукції. Для них ще не могло існувати такої
постави питання, що, навпаки, з погляду капіталістичного способу продукції,
намагається дослідити, яким чином земельна власність досягає того, що
віднімає від капіталу частину спродукованої ним (тобто вичавленої з безпосередніх
продуцентів) і в першу чергу привласненої вже ним додаткової вартости.

\emph{У фізіократів} труднощі вже іншого характеру. Як дійсно перші систематичні
тлумачі капіталу, вони намагалися аналізувати природу додаткової вартости
взагалі. Для них ця аналіза збігається з аналізою ренти, однісінької
форми, в якій для них існує додаткова вартість. Капітал, що дає ренту, або
хліборобський капітал, є для них однісінький капітал, що продукує додаткову
вартість, і пущена ним в рух хліборобська праця є однісінька, що створює додаткову
вартість, отже, з капіталістичного погляду цілком послідовно однісінька
продуктивна праця. Продукцію додаткової вартости вони цілком слушно вважають
за визначальний момент. Їм, залишаючи осторонь інші заслуги, про які мова
буде в книзі ІV, належить насамперед та велика заслуга, що від торговельного
капіталу, який функціонує тільки в сфері циркуляції, вони звернулись до продуктивного
капіталу, протилежно до меркантильної системи, яка за своїм грубим
реалізмом була справжньою вульґарною економією тієї доби, що її практичними
інтересами було відсунуто цілком на задній плян початки наукової аналізи
у Петті та його послідовників. Між іншим, тут, при критиці меркантильної системи,
мова йде лише про її погляди на капітал та додаткову вартість. Вже
давніш ми відзначали, що продукцію на світовий ринок і перетворення продукту
на товар, а тому і на гроші, монетарна система справедливо проголосила за передумову
і умову капіталістичної продукції. В її продовженні, в меркантильній системі,
переважну ролю відіграє вже не перетворення товарової вартости на гроші, а створення
додаткової вартости, але розглядається воно з іраціонального погляду сфери
циркуляції, до того ж так, що ця додаткова вартість виступає в формі додаткових
грошей, в надмірі торговельного балансу. Разом з тим справді характеристичне
для заінтересованих купців і фабрикантів того часу і адекватне тому періодові
капіталістичного розвитку, який вони репрезентують, є те, що при перетворенні
хліборобських февдальних громад на промислові, і при відповідній промисловій
боротьбі націй на світовому ринку, справа залежить від прискореного розвитку
капіталу, що досягається не так званим природним шляхом, а примусовими заходами.
Величезна ріжниця є в тому, чи перетворюється національний капітал на промисловий
поступово і повільно, чи це перетворення прискорюється в часі, в наслідок податків,
що ними вони в формі охоронних мит оподатковували переважно земельних
власників, середніх і дрібних селян і ремесло, в наслідок прискореної експропріяції
самостійних безпосередній продуцентів, в наслідок насильницької прискореної
акумуляції і концентрації капіталів, коротко, в наслідок прискореного
створення умов капіталістичного способу продукції. Разом з тим це становить
величезну ріжницю в капіталістичній і промисловій експлуатації природної національної
продуктивної сили. Тому національний характер меркантильної системи
в устах її оборонців є не просто фраза. З тієї притоки, що їх ніби цікавить тільки
багатство нації та допоміжні ресурси держави, вони в дійсності проголошують
\parbreak{}  %% абзац продовжується на наступній сторінці
