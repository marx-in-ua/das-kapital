\parcont{}  %% абзац починається на попередній сторінці
\index{iii2}{0255}  %% посилання на сторінку оригінального видання
продукту зможе цілком покрити in natura сталу частину. Надмір може тоді
придатись до створення нового додаткового капіталу, або можна буде значнішій
частині продукту надати форми засобів споживання, абож зменшити додаткову
працю. Навпаки, коли продуктивна сила праці зменшується, то на покриття старого
капіталу мусить піти значніша частина продукту; додатковий продукт зменшується.

Зворотне перетворення зиску або взагалі будь-якої форми додаткової вартости
на капітал показує, — коли ми абстрагуємось від певної історичної форми
і розглядатимемо це перетворення як просте створення нових засобів продукції, —
що все ще зберігається те становище, коли робітник, крім праці для придбання
безпосередніх засобів існування, витрачає працю на продукцію засобів продукції.
Перетворення зиску на капітал визначає не що інше, як застосування
частини надмірної праці на створення нових додаткових засобів продукції. Та
обставина, що це відбувається в формі перетворення зиску на капітал визначає лише,
що не робітник, а капіталіст порядкує надмірною працею. Що ця надмірна праця
мусить спочатку перейти стадію, в якій вона виступає як дохід (тимчасом як,
наприклад, у дикуна вона виступає як надмірна праця, що спрямована безпосередньо
на продукцію засобів продукції), це визначає лише, що ця праця або
її продукт привласнюється не тим, хто працює. Але в дійсності на капітал
перетворюється не зиск як такий. Перетворення додаткової вартости на капітал
визначає лише, що додаткову вартість і додатковий продукт не споживаються
індивідуально капіталістом як дохід. Такого перетворення в дійсності
зазнає вартість, зрічевлена праця зглядно продукт, що в ньому безпосередньо втілюється ця вартість,
або на яку її обмінюється після попереднього перетворення
на гроші. Так само і тоді, коли зиск знову перетворюється на капітал, не ця
певна форма додаткової вартости, не зиск являє собою джерело нового капіталу.
Додаткова вартість при цьому тільки перетворюється з однієї форми на другу.
Але не це перетворення форми робить її капіталом. Як капітал тепер функціонують
товар і його вартість. Але та обставина, що вартість товару не оплачена, —
а тільки в наслідок цього вона стає додатковою вартістю, — немає жодного значіння
для зрічевлення праці, для самої вартости.

Непорозуміння виявляться в різних формах. Наприклад, в тому, що товари,
що з них складається сталий капітал, також мають в собі елементи заробітної
плати, зиску й ренти. Або в тому, що те, що становить для одного дохід, для
іншого становить капітал, і тому це є просто суб’єктивні відношення. Приміром,
пряжа прядуна має частину вартости, яка становить для нього зиск. Отже, ткач,
купуючи пряжу, реалізує зиск прядуна, але для нього самого ця пряжа є лише
частина його сталого капіталу.

Крім того, що вже давніш сказано про відношення доходу і капіталу,
тут слід зауважити: те, що розглядуване з боку вартости, входить разом з пряжею,
як складова частина, в капітал ткача, є саме вартість пряжі. Яким чином
частини цієї вартости розклались для самого прядуна на капітал і дохід, іншими
словами на оплачену і неоплачену працю, — це цілком байдуже для визначення
самої вартости товару (коли абстрагуватись від змін, зумовлюваних пересічним
зиском). Але тут на заднім пляні завжди чигає уявлення, ніби зиск, взагалі
додаткова вартість, є такий надмір над вартістю товару, що створюється в наслідок
лише додачі, взаємного шахрайства, баришів від продажу. Коли оплачується
ціну продукції або навіть вартість товару, оплачується звичайно і ті
складові частини вартости товару, що для його продавця виступають у формі
доходу. Про монопольні ціни тут звичайно немає мови.

Подруге, цілком справедливо, що складові частини товарів, що з них складається
сталий капітал, зводяться, подібно до всякої іншої товарової вартости,
до частин вартости, які для продуцентів і власників засобів продукції розкладаються
на заробітну плату, зиск і ренту. Це є лише капіталістична форма
\parbreak{}  %% абзац продовжується на наступній сторінці
