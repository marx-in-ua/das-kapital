\parcont{}  %% абзац починається на попередній сторінці
\index{iii2}{0261}  %% посилання на сторінку оригінального видання
відхиляється від цього фізичного мінімуму; вона змінюється залежно від клімату
й рівня суспільного розвитку; вона залежить не тільки від фізичних, але
і від історично розвинених суспільних потреб, які стають другою природою.
Але в кожній країні за кожного даного періоду ця реґуляційна пересічна заробітна
плата є дана величина. Таким чином, вартість усіх інших доходів набуває
певної межі. Вона завжди дорівнює вартості, що в ній втілюється ввесь
робочий день (який тут збігається з пересічним робочим днем, бо він охоплює
сукупну масу праці, пущену в рух сукупним суспільним капіталом) мінус та
його частина, яка втілюється в заробітній платі. Отже, її межу дано межею
тієї вартости, що в ній виражається неоплачена праця, тобто кількістю цієї
неоплаченої праці. Коли та частина робочого дня що її робітник витрачає на
репродукцію вартости своєї заробітної плати, знаходить свою крайню межу
в фізичному мінімумі заробітної плати, то друга частина робочого дня, — та,
в якій визначається його додаткова праця, отже, і та частина вартости, яка
виражає додаткову вартість, — знаходить свою межу в фізичному максимумі
робочого дня, тобто в тій сукупній кількості щоденного робочого часу, що
її робітник взагалі може дати при умові збереження і репродукції своєї робочої
сили. А що в цьому дослідженні мова йде про розподіл тієї вартости, що
в ній визначається сукупна праця, новодолучена протягом року, то робочий
день можна розглядати тут як величину сталу, і ми розглядаємо його таким,
незалежно від того, наскільки він відхиляється від свого фізичного максимуму.
Абсолютна межа тієї частини вартости, що становить додаткову вартість і
розпадається на зиск і земельну ренту, є таким чином дана; вона визначається
надміром неоплаченої частини робочого дня понад оплачену його
частину, отже, тією частиною вартости сукупного продукту, що в ній реалізується
ця додаткова праця. Коли ми назвемо, як я це зробив, зиском додаткову
вартість, обмежену цими межами і обчислену на сукупний авансований
капітал, то зиск цей, розглядуваний щодо його абсолютної величини, дорівнює
додатковій вартості і, отже, межі його визначені так само закономірно,
як і межі цієї останньої. Але висота норми зиску також є величина,
включена у певні межі, визначувані вартістю товарів. Вона є відношення
сукупної додаткової вартости до сукупного суспільного капіталу, авансованого
на продукцію. Коли цей капітал \deq{} 500 (скажімо, мільйонів), а додаткова
вартість \deq{} 100, то 20\% є абсолютною межею норми зиску. Розподіл
суспільного зиску відповідно до цієї норми між капіталами, приміщеними в різних
сферах продукції, породжує відхильні від вартости товарів ціни продукції,
які й є дійсно реґуляційними пересічними ринковими цінами. Проте, відхил цей
не знищує ані визначення цін вартостями, ані закономірних меж зиску. Коли
вартість товару дорівнює зужиткованому при його продукції капіталові $k$ плюс
вміщена в ньому додаткова вартість, то ціна продукції дорівнює зужиткованому
при його продукції капіталові $k$ плюс додаткова вартість, що припадає на цей
товар, відповідно до загальної норми зиску, наприклад, 20\% на капітал, авансований
для продукції цього товару, як на дійсно зужиткований, так і на просто
вжитий. Але ця добавка в 20\% сама визначається додатковою вартістю, створеною
сукупним суспільним капіталом, і її відношенням до вартости цього капіталу;
саме тому вона становить 20\%, а не 10 або 100. Таким чином перетворення
вартости на ціни продукції, не знищує меж зиску, але тільки змінює
розподіл його між різними окремими капіталами, що з них складається суспільний
капітал, — розподіляє його між ними рівномірно, в тій пропорції, що в ній
вони становлять частини вартости цього сукупного капіталу. Ринкові ціни
то підносяться вище, то падають нижче від цієї реґуляційної ціни продукції,
але ці коливання взаємно знищуються. Якщо ми розглянемо таблицю цін за
довший період та усунемо ті випадки, коли в наслідок зміни продуктивности
\parbreak{}  %% абзац продовжується на наступній сторінці
