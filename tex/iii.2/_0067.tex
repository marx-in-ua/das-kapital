
\index{iii2}{0067}  %% посилання на сторінку оригінального видання
Про розділ банку на два відділи та про надмірне піклування в справі забезпечення
розміну банкнот Тук висловлюється так перед C. D 1848/57:

Більші коливання рівня проценту в 1847 році проти років 1837 та 1839
завдячували лише розділові банку на два відділи. (3010). — Забезпечености банкнот
не було порушено ані в 1825, ані в 1837 та 1839 роках. (3015). — Попит
на золото в 1825 році мав на меті тільки заповнити порожняву, що утворилася
в наслідок цілковитого дискредитування однофунтівок-банкнот провінціяльних
банків; цю порожняву можна було заповнити тільки золотом, поки Англійський
банк не почав теж видавати однофунтівки-банкноти. — (3022). В листопаді та
грудні 1825 року не було ані найменшого попиту на золото для вивозу. (3023).

«Щодо дискредитування банку всередині країни та закордоном, то припинення
виплати дивідендів та вкладів мало б куди тяжчі наслідки, ніж припинення
оплати банкнот (3028)».

«3035. Чи не сказали б ви, що кожна обставина, яка, кінець-кінцем,
загрожує небезпекою розмінові банкнот, могла б в момент комерційного пригнічення
породити нові та серйозні труднощі? — Аж ніяк».

Протягом 1847 року «збільшене видання банкнот, може бути, допомогло б
знову поповнити золотий скарб банку, як це сталося в 1825 році». (3058).

Newmarch свідчить перед В А. 1857: «1357. Перший лихий вплив\dots{}
цього відокремлення обох відділів [банку] та розділу золотого запасу на дві частини,
розділу, що неминуче випливав з такого відокремлення, був той, що банкові
операції Англійського банку, отже, цілу ту ділянку його операцій, що ставить
його в безпосередній зв’язок з торговлею країни, провадилось далі лише за допомогою
половини суми попереднього запасу. В наслідок цього розділу запасу
дійшло до того, що банк мусів підвищувати норму свого дисконту, скоро запас
банкового відділу зменшувався хоч трохи. Тому цей зменшений запас зумовлював
ряд раптових змін у нормі дисконту. — 1358. Таких змін, починаючи від
1844 року [до червня 1857 року], було, може, з 60, тимчасом коли протягом
такого самого часу перед 1844 роком вони ледве чи становили тузінь». Особливий
інтерес має теж свідчення Palmer’a, що від 1811 року був директором, а
деякий час управителем англійського банку, перед C. D. комісією лордів (1848--57).

«828. В грудні 1825 року банк ще зберіг приблизно 1.100.000 ф. ст.
золота. Він мусів би тоді, безперечно, цілком збанкрутувати, коли б тоді був
цей акт (1844 року). В грудні він видав, на мою думку, 5 або 6 мільйонів
банкнот протягом одного тижня, й це значно полегшило тодішню паніку.

«825. Перший період [від 1 липня 1825 року], коли сучасне банкове
законодавство збанкрутувало б, якщо банк спробував би довести до кінця вже
розпочаті операції, був 28 лютого 1837 року; в ті часи в розпорядженні банку
було 3.900.000 до 4 мільйонів ф. ст., і він зберіг би тоді лише 650.000 ф. ст.
в запасі. Другий такий період був у 1839 році й тривав від 9 липня до 5 грудня.
— 826. Яка була сума запасу в цьому випадку? — 5 вересня запас
складався з дефіциту в цілому на суму 200.000 ф. ст. (the reserve was minus
altogether 200.000 ф. ст). На 5 листопада запас зріс приблизно до 1--1\sfrac{1}{2}
мільйонів. — 830. Акт 1844 року заважав би банкові підтримувати торговлю
з Америкою. — 831. Три головні американські фірми збанкрутували\dots{} Майже
кожну фірму, що провадила американські операції, позбавлено кредиту, і
коли б у ті часи банк не прийшов на поміч, то я не думаю, щоб більше, як
1 або 2 фірми, могли витримати, — 836. Скруту 1837 року не можна рівняти
з скрутою 1847 року. В 1837 році вона обмежилася головне на американських
операціях». — 838. (На початку червня 1837 року дирекція банку дискутувала
питання, як зарадити тій скруті). «В цій справі декотрі з панів боронили думку\dots{}
що найправильнішим принципом було б підвищити рівень проценту, через що
товарові ціни впали б; коротко, зробити гроші дорожчими, а товари дешевшими,
\parbreak{}  %% абзац продовжується на наступній сторінці
