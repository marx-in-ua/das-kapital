\parcont{}  %% абзац починається на попередній сторінці
\index{iii2}{0086}  %% посилання на сторінку оригінального видання
раніше. Але, може бути, я й тепер витрачу 2000\pound{ ф. ст.}, куплю на них подвійну
кількість товарів проти попередньої та поширю своє підприємство, авансуючи
той самий капітал, що його, можливо, мені доведеться у когось позичити. Я купую
тепер, як і раніше, на 2000\pound{ ф. ст}. Отже, мій попит на грошовому ринку лишається
однаковий, дарма що мій попит на товаровому ринку зі спадом цін зростає.
Але коли цей останній спадає, тобто продукція — що перечило б всім законам
Economist'a — не поширюється зі спадом товарових цін, то попит на
позичковий грошовий капітал зменшився б, дарма що зиск збільшився б; але
це збільшення зиску утворило б попит на позичковий капітал. Проте низькі
товарові ціни можуть виникати з трьох причин Поперше, через брак попиту.
В цьому разі рівень проценту низький тому, що продукція в занепаді, а не
тому, що товари дешеві, бо ця дешевина — лише вияв того занепаду. Або
низькі товарові ціни можуть, бути від того, що подання надмірно велике проти
попиту. Це може статися в наслідок такого переповнення ринків і~\abbr{т. ін.},
що приводить до кризи, та підчас самої кризи може збігатися з високим рівнем
проценту; або це може статися тому, що вартість товарів впала, отже, той
самий попит можна задовольнити за нижчі ціни. Чого в останьому випадку рівень
проценту має спадати? Тому що зиск зростає? Коли тому, що менше грошового
капіталу треба на те, щоб одержати той самий продуктивний або товаровий
капітал, то це доводило б тільки те, що зиск та процент стоять один до одного
у зворотному відношенні. В усякому випадку загальна теза Economist’a — помилкова.
Низькі грошові ціни товарів та низький рівень проценту не неминуче збігаються.
А інакше в найубогіших країнах, де грошові ціни продуктів найнижчі,
мусив би й рівень проценту бути найнижчий, а в найбагатших країнах, де
грошові ціни хліборобських продуктів найвищі, мусив би й рівень проценту
бути найвищий. Загалом Economist визнає: коли вартість грошей спадає, то це
не має жодного впливу на рівень проценту. 100 ф. ст, як і давніш, даватимуть
105\pound{ ф. ст.}; коли 100\pound{ ф. ст.} варті менше, то менше варті й 5\pound{ ф. ст.} проценту.
На відношення не впливає піднесення вартости або знецінення первісної суми.
Коли певну кількість товарів розглядати як вартість, то вона дорівнює певній
сумі грошей. Коли її вартість зростає, то і вона дорівнює більшій грошовій
сумі; навпаки буває, коли її вартість спадає. Коли вона = 2000, то 5\% =
100; коли вона = 1000, то 5\% = 50. Але це нічого не змінює у нормі проценту.
Правильне лише те, що потрібна більша грошова позика, коли треба
2000\pound{ ф. ст.}, щоб продати ту саму кількість товарів, ніж, тоді, коли треба
тільки 1000\pound{ ф. ст}. Але в даному разі це виявляє лише зворотне відношення
між зиском і процентом. Бо зиск зростає, а процент спадає з дешевиною елементів
сталого й змінного капіталу. Але може траплятися й навпаки, і це буває
часто. Напр., бавовна може бути дешевою, бо нема жодного попиту на
пряжу та тканини; вона може бути відносно дорогою, бо великий зиск у бавовняній
промисловості породжує великий попит на неї. З другого боку, зиск промисловців
може бути високий саме тому, що ціна на бавовну низька. Таблиця
Hubbard’a доводить, що рівень проценту й товарові ціни мають рухи, цілком незалежні
одні від одних, тимчасом як рухи рівня проценту точно відповідають
рухам металевого скарбу та вексельних курсів.

«Отже, якщо товарів є понад міру, грошовий процент мусить бути низький»,
каже Economist. Саме протилежне відбувається підчас кризи; товарів є понад міру, їх
не можна перетворити на гроші, й тому рівень проценту високий: підчас другої фази
циклу панує великий попит на товари, а тому зворотний приплив капіталів відбувається
легко, але одночасно підносяться товарові ціни та з причини легкости повороту
капіталу рівень проценту є низький. «Коли їх [товарів] обмаль, він мусить
бути високий». Знову таки протилежне буває підчас полегшення стану по кризі. Товарів
обмаль, абсолютно кажучи, а не проти попиту; а рівень проценту низький.
