
\index{iii2}{0048}  %% посилання на сторінку оригінального видання
[Поки стан справ є такий, що зворотний приплив зроблених авансувань
відбувається реґулярно, отже й кредит лишається незахитаним, пошир та
скорочення циркуляції реґулюється просто потребами промисловців та купців. Що
золото, принаймні в Англії, не має ваги для гуртової торговлі, а циркуляцію
золота — якщо не вважати на сезонні коливання — можна розглядати для довшого
часу як досить сталу величину, то й становить циркуляція банкнот Англійського
банку досить точне мірило ступеня цих змін. За тихих часів по кризі розмір
циркуляції є найменший, з новим оживленням попиту постає й більша потреба
на засоби циркуляції, ця потреба зростає з розвитком розцвіту; найвищої точки
кількість засобів циркуляції доходить в період надмірного напруження та надмірної
спекуляції, — тоді вибухає криза й за ніч зникають з ринку банкноти,
що їх ще вчора було багато, а з ними зникають і дисконтери векселів, і ті хто
дають гроші під цінні папери, і купці товарів. Англійський банк має допомагати,
— але й його сили незабаром вичерпані, банковий акт 1844 року змушує
його обмежувати циркуляції своїх банкнот саме тоді, коли весь світ криком
вимагає банкнот, коли державці товарів не можуть їх продавати, а проте повинні
платити та готові на всякі жертви, аби тільки одержати банкноти. «Підчас
переляку», каже вищезгаданий банкір Wright (1. c. № 2930), «країна потребує
удвоє більшої циркуляції, ніж за звичайних часів, бо банкіри й інші скупчують
собі про запас засоби циркуляції».

Скоро вибухав криза, справа вже тільки в платіжних засобах. А що
в надході цих платіжних засобів кожен залежить від іншого та ніхто не знає,
чи той інший в стані буде платити в реченець, то й настає справжня гонитва
за тими платіжними засобами, що є на ринку, тобто за банкнотами. Кожен
скупчує тих банкнот як скарб, скільки тільки він їх може одержати, і таким
чином банкноти зникають з циркуляції того самого дня, коли їх потребують
найбільше. Samuel Gurney (C. D. 1848/57, № 1116) визначає число банкнот,
прихованих під замок в момент паніки, в жовтні 1847~\abbr{р.} на суму 4--5 мільйонів
ф. ст. — Ф. Е.]

Щодо цього особливо цікаві свідчення перед банковою комісією 1857 року
спільника Gurney’ового, вже згаданого Chapman’а. Я подаю тут головний
зміст їх у зв’язному викладі, хоч в них розглядаються деякі пункти, що їх ми
дослідимо тільки пізніше. Пан Chapman дає таке свідчення.

«4963. Я не вагаючися скажу, що я не вважаю за доладне, коли грошовий
ринок має бути під владою будь-якого індивідуального капіталіста (а їх
в Лондоні є досить), що в стані утворювати величезну недостачу грошей та скруту
тоді, коли циркуляція є саме дуже низька\dots{} Це можливо\dots{} є не один капіталіст,
що може витягти з циркуляції банкнот на 1 чи 2 міл. ф. ст., якщо він
може тим досягти певної мети». 4995. Якийсь великий спекулянт може продати
консолів на 1 чи 2 міл. й таким способом забрати гроші з ринку. Дещо подібне
сталося зовсім недавно, «і це утворює незвичайно гостру скруту».

4967. Певна річ, банкноти тоді є непродуктивні. «Але нема чого тим
журитися, якщо таким способом можна осягнути великої мети; його велика
мета — збити ціни на фонди, утворити грошову скруту, а зробити це — цілком
в його силі». Приклад: одного ранку був великий попит на гроші на фондовій
біржі; ніхто не знав причини; хтось запропонував Chapman’oвi, щоб останній
позичив йому 50.000 ф. ст. з 7\%. Chapman здивувався, бо в нього рівень
проценту був значно нижчий; він згодився. Скоро по тому той чоловік прийшов
знову, взяв" знову 50.000 ф. ст. з 7\sfrac{1}{2}\%, потім 100.000 ф. ст. з 8\% і хотів
взяти ще більшу суму з 8\sfrac{1}{2}\%. Але тоді самого Chapman’a охопила тривога.
Потім виявилося, що раптом забрано з ринку значну суму грошей. Однак, каже
Chapman, «я проте визичив значну суму з 8\%; йти далі я боявся; я не знав,
що з того вийде».
