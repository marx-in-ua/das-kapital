
\index{iii2}{0203}  %% посилання на сторінку оригінального видання
З поступом культури до нового оброблення може притягатись, — як ми це
вже показали, досліджуючи диференційну ренту — землю такої самої і навіть
кращої якости, так само як і землю гіршої якости.

\emph{Поперше}, тому, що при диференційній ренті (і при ренті взагалі, бо
навіть при не диференційній ренті завжди постає питання, чи дозволяє, з одного
боку, родючість землі взагалі, а з другого її положення, обробляти її,
одержуючи при реґуляційній ринковій ціні зиск і ренту), діють в зворотному
напрямі дві умови, які то взаємно паралізують одна одну, то беруть
перевагу одна над однією. Підвищення ринкової ціни — припускаючи, що витрати
продукції потрібні для оброблення не понизились, іншими словами, що
не завоювання технічного характеру становлять новий момент, який зумовлює
нове оброблення, — може призвести до оброблення родючішої землі, яка давніш
через своє положення виключалась з числа конкурентних земель. Абож для
менш родючої землі це може остільки підвищити вигоди положення, що ними
вирівнюється меншу родючість. Або, і без підвищення ринкової ціни, в наслідок
поліпшення засобів комунікації, положення може змінитись так, що кращі землі
ввійдуть у конкуренцію, що спостерігається у великому маштабі в степових
штатах Північної Америки. Та і в країнах старої цивілізації це відбувається
постійно, хоч і не в такому самому маштабі як у колоніях, де, як справедливо
відзначів Wakefield, вирішна роля належить положенню. Отже, поперше,
протилежні дії положення і родючости, і змінливість чинника положення, який
постійно вирівнюється, постійно проробляє проґресивні зміни, що спрямовуються
до вирівнювання, — це призводить до того, що в конкуренцію з уже оброблюваними
землями навперемінку вступають дільниці землі однакової, кращої й
гіршої якости.

\emph{Друге}. З розвитком природничих наук і агрономії змінюється і родючість
землі, бо змінюються засоби, що з допомогою їх уможливлюється негайне
використання елементів ґрунту. Таким чином, у Франції і східніх графствах
Англії легкі ґрунти, які раніш вважалися за кепські, ще зовсім недавно були
перетворені в першорядні (див. Passy). З другого боку, земля, що її за хемічним
складом не вважалось за погану, але яка лише ставила певні механічнофізичні
перешкоди обробіткові, перетворюється на добру землю, скоро знаходять
засоби для того, щоб подолати ці перешкоди.

\emph{Третє}. У всіх країнах старої цивілізації старі історичні і традиційні
відносини, наприклад, у формі державних земель, громадських земель тощо,
цілком випадково відволікли великі дільниці землі від обробітку, до якого їх
притягають лише поступово. Порядок, в якому їх притягається до оброблення,
не залежить ані від їхньої якости, ані від їхнього положення, а лише від цілком
зовнішніх обставин. Коли простежити історію англійських громадських земель,
простежити, як вони законами про обгороджування (Enclosure Bills) поступово перетворювалися
на приватну власність і оброблялися, то виявилося б, що не може
бути нічого безглуздішого за те фантастичне припущення, ніби вибором цього
порядку керував якийсь сучасний хліборобський хемік, наприклад, Лібіх, призначаючи
певні лани через їхні хемічні якості під культуру та виключаючи
інші. Тут, радше, переважне значення відігравала нагода, яка робить з людини
злодія; більш або менш зовнішньо пристойні юридичні зачіпки для привласнення,
що їх могли використати великі лендлорди.

\emph{Четверте}. Залишаючи осторонь той факт, що досягнутий в кожний
певний момент ступінь розвитку в зрості людности і капіталу кладе поширенню
хліборобської культури певну, хоча б і елястичну межу; залишаючи осторонь
діяння таких випадковостей, які справляють тимчасовий вплив на ринкову
ціну, от як ряд сприятливих або несприятливих діб року, — просторове поширеная
хліборобської культури залежить від загального стану ринку капіталів і
\parbreak{}  %% абзац продовжується на наступній сторінці
