\parcont{}  %% абзац починається на попередній сторінці
\index{iii2}{0230}  %% посилання на сторінку оригінального видання
масами, бо тут число дрібних покупців велике, а число великих покупців мале
(Bandes Noires, Rubichon; Newmann). З усіх цих причин тут ціна землі підвищується
при відносно високому розмірі проценту. Відносно низькому процентові,
що його селянин добуває тут з капіталу, витраченого на купівлю землі (Mounier),
на протилежному боці тут відповідає високий лихварський розмір проценту,
що його сам селянин повинен виплачувати гіпотечному кредиторові. Ірляндська
система показує те саме, тільки в іншій формі.

Тому ціна землі, цей елемент, що є чужий продукції самій по собі, може
тут підвищитись до такої висоти, що продукція стає неможлива (Dombasle).

Що ціна землі відіграє тут таку ролю, що купівля і продаж землі, циркуляція
землі як товару розвивається до такого розміру, це практично є наслідок
розвитку капіталістичного способу продукції, оскільки за цього способу продукції
товар стає загальною формою усіх продуктів і всіх знарядь продукції. З другого
боку, циркуляція землі як товару розвивається тільки там, де капіталістичний
спосіб продукції набув лише обмеженого розвитку і не розгорнув усіх своїх
особливостей, бо розвиток циркуляції землі як товару ґрунтується саме на тому,
що хліборобство вже не упідлеглене, або ще не упідлеглене капіталістичному
способові продукції, а упідлеглене способові продукції, перейшлому від загинулих
форм суспільства. Отже, вади капіталістичного способу продукції з його
залежністю продуцента від грошової ціни його продукту збігаються тут з вадами,
що випливають з недостатнього розвитку капіталістичного способу продукції.
Селянин стає купцем і промисловцем, не маючи умов, при яких він може продукувати
свій продукт як товар.

Конфлікт між ціною землі, як елементом витрат продукції для продуцента,
і неелементом ціни продукції продукту (навіть коли рента входить визначально
в ціну хліборобського продукту, капіталізована рента, що авансується на 20
і більше років, ні в якому разі не входить визначально в його ціну) є лише
одна з форм, що в них взагалі виявляється суперечність приватної власности на
землю з раціональним хліборобством, з нормальним суспільним користуванням
землею. Але з другого боку, приватна власність на землю, отже, експропріація
землі у безпосередніх продуцентів — приватна власність одних, яка має
за передумову, що немає землі в інших, — є основа капіталістичного способу
продукції.

Тут, при дрібній культурі, ціна землі, форма і наслідок приватної власности
на землю, виступає як межа самої продукції. При великому хліборобстві
і при великій земельній власності, яка ґрунтується на капіталістичному способі
провадження господарства, власність теж виступає як межа, бо вона обмежує
орендаря в продуктивному приміщенні капіталу, яке кінець-кінцем іде на
користь не йому, а земельному власникові. При обох формах, замість свідомого
раціонального оброблення землі, як спільної довічної власности, невідчужуваної
умови існування і репродукції для ланцюга послідовних людських поколінь, виступає
визиск і марнотратство сил землі (не кажучи вже про те, що визиск
ставлять в залежність не від досягненого суспільством рівня розвитку, а від випадкових,
неоднакових обставин окремих продуцентів). При дрібній власності
це відбувається в наслідок браку засобів і знаннів, потрібних для застосування
суспільної продуктивної сили праці. При великій власності — в наслідок визиску
цих засобів для можливо швидшого збагачення орендарів і власників. І в тому,
і в тому разі — в наслідок залежности від ринкової ціни.

Вся критика дрібної земельної власности кінець-кінцем зводиться до критики
приватної власности, як межі і перешкоди для хліборобства. Так само вся
протилежна критика великої земельної власности. Звичайно ми тут для обох
випадків залишаємо осторонь побічні політичні міркування. Ця межа і перешкода,
що її всяка приватна власність на землю протиставить хліборобській продукції
\parbreak{}  %% абзац продовжується на наступній сторінці
