\parcont{}  %% абзац починається на попередній сторінці
\index{iii2}{0215}  %% посилання на сторінку оригінального видання
капітал, — то вони входять в рахунок, виражені як рахункові гроші, і віднімаються
як складові частини витрат продукції. Зношування машин і взагалі
основного капіталу доводиться покривати грішми. Нарешті, маємо зиск, який
обчислюється на суму цих витрат, що виражені в дійсних грошах або в рахункових.
Цей зиск втілюється в певній частині гуртового продукту, яка визначається
його ціною. А та частина, що залишається після цього, становить ренту.
Коли рента продуктами, встановлена контрактом, більша за цю. визначену ціною
рештку, то це вже буде не рента, а вирахування з зиску. Вже в наслідок самої
цієї можливості рента продуктами, що не відповідає ціні продукту, що, отже,
може становити і більше і менше, ніж дійсна рента і яка тому може становити
вирахування не тільки з зиску, але і з тих складових елементів, що ними покривається
капітал, — вже з самої цієї можливості вона становить архаїчну
форму. В дійсності ця рента продуктами, оскільки вона є рента не тільки з
назви, але й по суті, визначається виключно надміром ціни продукту над його
ціною продукції. Річ тільки в тому, що ця змінна величина припускається нею
як стала. Але ж це є таке узяте з минулого уявлення, що продукту in natura,
поперше, досить для того, щоб прохарчувати робітників, далі, дати капіталістичному
орендареві більше їжі, ніж йому потрібно, і що надмір над цим становить
природну ренту. Цілком так само, як коли фабрикант фабрикує \num{200.000}
ліктів ситцю. Цих ліктів досить для того, щоб не тільки одягти його робітників,
але й більше, ніж одягти його дружину і всіх його нащадків і його самого, залишити
крім того ситець на продаж і, нарешті, виплачувати ситцем величезну
ренту. Така собі звичайна річ! Досить тільки з \num{200.000} ліктів ситцю вирахувати
ціну їхньої продукції, і тоді мусить залишитися надмір ситцю, що становить
ренту. Наприклад, з \num{200.000} ліктів ситцю вирахувати ціну їхньої продукції
в \num{10.000}\pound{ ф. ст.}, не знаючи продажної ціни ситцю, з ситцю вирахувати
гроші, з споживної вартости як такої, вирахувати мінову вартість, і потім визначити
надмір ліктів ситцю над фунтами стерлінґів, — це дійсно наївна уява.
Це гірше, ніж квадратура круга, в основі якої принаймні лежить уява про
межі, що в них зливаються пряма лінія і крива. Але саме такий є рецепт п.
Passy. Вирахуйте гроші з ситцю, перше ніж у голові або в дійсності! ситець
перетворився на гроші! Надмір становить ренту, але вона має стати обмацальною
naturaliter\footnote*{
Лат., з самої своєї природи, природно. \emph{Прим. Ред.}
} (див. напр. Карла Арнда), а не через «софістичну» чортівню!
До цього безглуздя, до вирахування ціни продукції з стількох-от шефелів пшениці,
до вирахування грошової суми з міри об’єму зводиться вся реставрація
натуральної ренти.

\subsubsection{Відробітна рента}

Коли розглядати земельну ренту в її найпростішій формі, у формі \emph{відробітної
ренти}, коли безпосередній продуцент частину тижня обробляє фактично
належну йому землю знаряддями праці (плуг, худоба, тощо), що фактично
або юридично належать йому ж, а інші дні тижня працює в маєтку землевласника,
для землевласника, задурно, то тут справа ще цілком ясна, рента і додаткова вартість
тут тотожні. Рента, а не зиск, — ось та форма, що в ній тут виражається неоплачена
додаткова праця. В якій мірі робітник (self sustaining serf)\footnote*{
Англ. раб, що сам себе утримує. \emph{Прим. Ред.}
} може одержати тут
надмір над доконечними засобами свого існування, тобто надмір понад те, що
при капіталістичному способі продукції ми назвали б заробітною платою, це
залежить за інших незмінних умов від того відношення, в якому його робочий
час ділиться на робочий час для нього самого і панщизняний робочий час для
\parbreak{}  %% абзац продовжується на наступній сторінці
