\parcont{}  %% абзац починається на попередній сторінці
\index{iii2}{0234}  %% посилання на сторінку оригінального видання
певного матеріяльного продукту, пшениці. Але вона не має ніякого чинення до продукції
\emph{вартости пшениці}. Оскільки вартість втілюється у пшеницю, пшениця
розглядається лише як певна кількість зрічевленої суспільної праці, цілком
байдуже до тієї особливої речовини, в якій втілюється ця праця або до особливої
споживної вартости цієї речовини. Це не суперечить тому, що 1) в інших рівних
умовах дешевина або дорожнеча пшениці залежать від продуктивности землі. Продуктивність
хліборобської праці зв’язана з природними умовами, і залежно від продуктивности
останніх та сама кількість праці втілюється в більший або меншій кількості
продукту, споживних вартостей. Наскільки велика кількість праці, втілена в
одному шефелі, це залежить від того, яку кількість шефелів дає дана кількість праці.
Від продуктивности землі тут залежить, в якій кількості продукту втілюється вартість;
але цю вартість дано незалежно від такого поділу. Вартість втілюється
у споживній вартості; а споживна вартість є умова створення вартости; але
було б безглуздям створювати таке протиставлення, де на однім боці ставиться
споживну вартість, землю, а на другому боці — вартість, і до того ще особливу
частину вартости. 2) \emph{[Тут рукопис уривається]}

\subsection*{ІII.}

Вульґарна економія в дійсності не робить нічого іншого, як тільки подоктринерському
тлумачить, систематизує і виправдує уявлення аґентів буржуазної
продукції, захоплених відносинами цієї продукції. Отже, нас не повинно
дивувати те, що якраз у формі виявлення економічних відносин, яка відчужена
від них, і в якій вони prima facie набувають вульґарного і цілком суперечливого
характеру, — а всяка наука була б зайва, коли б форма виявлення й суть
речей безпосередньо збігалися, — що якраз тут вульґарна економія почуває себе
цілком вдома, і що ці відносини здаються їй то самоочевиднішими, що більше
захований в них внутрішній зв’язок і що звичайнішими здаються вони для
буденної уяви. Тому в неї немає ніякого передчуття того, що триєдиність,
з якої вона виходить: земля — рента, капітал — процент, праця — заробітна плата
або ціна праці, є три prima facie неможливі сполучення. Насамперед перед
нами споживна вартість, \emph{земля}, яка не має вартости, і мінова вартість, \emph{рента},
таким чином соціальне відношення, взяте як річ, є поставлене в певне
відношення до природи; отже, дві неспівмірні величини ставляться в певне
відношення одна до однієї. Потім \emph{капітал — процент}. Коли під капіталом
розуміти певну суму вартости, самостійно визначену в грошах, то prima facie
є безглуздя, ніби вартість має більшу вартість, ніж вона варта. Саме у формі:
капітал — процент відпадає всяке посередництво, і капітал зводиться до своєї
найзагальнішої, але тому й нез’ясуванної з себе самої та абсурдної формули.
Саме тому вульґарний економіст і дає перевагу формулі: капітал — процент,
з таємничою властивістю вартости бути нерівною самій собі, над формулою:
капітал — зиск, бо ця вже ближче підходить до дійсного капіталістичного відношення.
А потім, турботно відчуваючи, що 4 не є 5, і тому 100 талярів не
можуть бути 110 талярами, вульґарний економіст від капіталу як вартости
вдається до речевої субстанції капіталу, до його споживної вартости як умови
продукції для праці, до машин, сирового матеріялу тощо. Таким чином, знову
щастить, замість незрозумілого першого відношення, за яким 4 \deq{} 5, створити
цілком неспівмірне відношення між споживною вартістю, річчю, на одному боці,
і певним суспільним продукційним відношенням, додатковою вартістю на другому
боці, як у випадку з земельною власністю. Скоро вульґарний економіст
доходить і до цієї неспівмірности, йому все стане ясним, він уже не почуває
потреби міркувати далі. Бо він дійшов саме до «раціонального» для буржуазного
уявлення. Нарешті, \emph{праця — заробітна плата}, ціна праці, як показано
\parbreak{}  %% абзац продовжується на наступній сторінці
