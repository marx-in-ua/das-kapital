\parcont{}  %% абзац починається на попередній сторінці
\index{iii2}{0123}  %% посилання на сторінку оригінального видання
до життя цю природну умову підвищеної продуктивної сили праці, в такий
спосіб як кожен капітал може воду перетворити в пару. Ця природна умова
трапляється в природі лише місцями, і там, де її немає, її неможливо створити
певного витратою капіталу. Вона зв’язана не з продуктами, створюваними працею,
як машини, вугілля тощо, а з певними природними відносинами певної частини
землі. Та частина фабрикантів, що їм належать водоспади, усувають ту частину
фабрикантів, у яких немає водоспадів, від застосування цієї природної сили,
бо земля — і тим паче земля, що має водну силу, — обмежена. Це не виключає
того, що хоч кількість природних водоспадів у певній країні обмежена, кількість
водяної сили, яку може використовувати промисловість, може бути збільшена.
Водоспад можна штучно відвести, щоб цілком використати його рушійну силу;
коли вже є водоспад, водяне колесо можна удосконалити, щоб більше використати
силу води; там, де для подачі води звичайне колесо непридатне, можна застосувати
турбіни і~\abbr{т. ін.} Посідання цією природною силою становить монополію
в руках її посідача, таку умову високої продуктивної сили вкладеного капіталу,
яку не можна створити продукційним процесом самого капіталу\footnote{
Див. про надзиск „Inquiry“ (проти Мальтуса).
}; ця природна
сила, яка може бути так монополізована, завжди зв’язана з землею. Така природна
сила не належить ні до числа загальних умов згаданої сфери продукції, ні до
числа таких її умов, що їх можна створити як загальні умови.

Тепер, коли ми собі уявимо що водоспади разом з прилежною до них землею
перебувають в руках осіб, які вважаються власниками цих частин землі,
землевласниками, то ми побачимо, що вони не дозволяють прикладати капітал
до водоспаду, використовувати його з допомогою капіталу. Вони можуть
дозволити і не дозволити використання водоспаду. Але капітал не може створити
водоспад із себе. Тому надзиск, що постає з цього використання водоспаду, постає
не з капіталу, а з застосування капіталом цієї природної сили, яку монополізувати
можна і яка монополізована. В таких обставинах надзиск перетворюється на земельну
ренту, тобто він дістається власникові водоспаду. Коли фабрикант виплачує
йому за його водоспад 10\pound{ ф. ст.} на рік, то його зиск становить 15\pound{ ф. ст.}; 15\% на
ті 100\pound{ ф. ст.}, що їх тепер досягають його витрати продукції; і він опиняється
тепер цілком в такому самому становищі, може в кращому, ніж усі інші капіталісти
його сфери продукції, що працюють з допомогою пари. Справа ані трохи
не відмінилась би від того, коли б капіталіст сам був власником водоспаду. Він,
як і раніш, одержував би надзиск в 10\pound{ ф. ст.} не як капіталіст, а як власник водоспаду,
і саме тому, що цей надмір постає не з його капіталу, як такого, а
з порядкування такою природною силою, що її можна відділити від його капіталу,
що її можна монополізувати, та яка обмежена в своїх розмірах, — саме тому, цей
надмір переворюється на земельну ренту.

\emph{Перше}: Ясно, що ця рента завжди становить диференційну ренту, бо
вона не ввіходить визначально в загальну ціну продукції товару, а навпаки,
має її за передумову. Вона завжди виникає з ріжниці між індивідуальною
ціною продукції, для окремого капіталу, який порядкує монополізованою природною
силою, і загальною ціною продукції для капіталу, взагалі вкладеного у згадану
сферу продукції.

\emph{Друге}: Ця земельна рента постає не з абсолютного підвищення продуктивної
сили застосованого капіталу — зглядно привласненої ним праці, —
що взагалі могло б призвести лише до зменшення вартости товарів; а
з більшої відносної продуктивности певних окремих капіталів, приміщених
в певну сферу продукції, порівняно з тими приміщенями капіталу, які усунені
від цих виключних, створених природою сприятливих умов підвищення
продуктивної сили. Коли б, наприклад, не зважаючи на те, що вугілля має вартість,
\index{iii2}{0124}  %% посилання на сторінку оригінального видання
а сила води не має вартости, користання парою все ж давало б рішучі
переваги, недосяжні при використанні сили води, і коли б ці переваги більше
ніж компенсували силу води, то сила води не мала б застосування і не могла б
породити жодного надзиску, а, отже, і ренти.

\emph{Третє}: Сила природи не є джерело надзиску, а лише його природна
база, бо це є природна база виключно підвищеної продуктивної сили праці. Так
взагалі споживна вартість є носій мінової вартости, а не причина її. Коли б
ту саму споживну вартість можна було створювати без праці, вона б не мала
жодної мінової вартости, але як і давніш, мала б свою природну корисність
як споживна вартість. Але, з другого боку, без споживної вартости, отже,
без такого природного носія праці, річ не має жодної мінової вартости. Коли б
різні вартості не вирівнювались у ціни продукції і різні індивідуальні ціни
продукції не вирівнювались би в загальну ціну продукції, яка реґулює ринок,
то звичайне підвищення продуктивної сили праці в наслідок використання водоспаду,
лише знизило б ціну товарів, продукованих з допомогою водоспаду, але
не підвищило б тієї частини зиску, що міститься в цих товарах; так само, як,
з другого боку, ця підвищена продуктивна сила праці взагалі не перетворювалась
би на додаткову вартість, коли б капітал продуктивну силу вживаної ним
праці, природну і суспільну, не привлащував би як свою власну.

\emph{Четверте}: Земельна власність на водоспад сама по собі не має ніякого
чинення до створення цієї частини додаткової вартости (зиску), а тому і взагалі
ціни товару, який продукується з допомогою водоспаду. Цей надзиск існував
би і тоді коли б не існувало земельної власности, коли б, наприклад,
земля, до якої належить водоспад, використовувалась фабрикантом, як безгосподарна
земля. Отже, земельна власність не створює тієї частини вартости,
яка перетворюється в надзиск, а лише дає земельному власникові, власникові
водоспаду, можливість перекласти цей надзиск з кишені фабриканта у свою
власну. Земельна власність є причина не створення цього надзиску, а його
перетворення у форму земельної ренти, отже, привласнення цієї частини зиску,
зглядно ціни товару, власником землі або водоспаду.

\emph{П’яте}: Ясно, що ціна водоспаду, отже, ціна, яку одержав би земельний
власник, коли б він продав його третій особі, або самому фабрикантові,
спочатку не входить у ціну продукції товарів, хоч входить в індивідуальні
витрати продукції даного фабриканта; бо рента виникає тут з ціни продукції
товарів того самого роду, продукованих з допомогою парових машин,
з ціни продукції, що реґулюється незалежно від водоспаду. Але, далі, ця ціна
водоспаду взагалі є іраціональний вираз, що за ним ховається реальне економічне
відношення. Водоспад, як земля взагалі, як усі сили природи, не має
жодної вартости, бо в ньому не зрічевлено жодної праці, а тому не має він
жодної ціни, яка нормально є не що інше, як виражена в грошах вартість.
Де немає вартости, там ео ipso\footnote*{
Тим самим. \emph{Пр.~Ред.}
} нічого виражати в грошах. Ця ціна є не
що інше, як капіталізована рента. Земельна власність дає власникові можливість
захоплювати ріжницю між індивідуальним зиском і пересічним зиском,
захоплюваний в такий спосіб зиск, що відновляється щорічно, може бути капіталізований
і тоді виступає як ціна самої сили природи. Коли надзиск, що
його дає фабрикантові використання водоспаду, становить 10\pound{ ф. ст.} на рік, а
пересічний процент 5\%, то ці 10\pound{ ф. ст.} на рік становлять проценти з капіталу в
200\pound{ ф. ст.} і ця капіталізація річних 10\pound{ ф. ст.}, що водоспад дає змогу власникові
його захоплювати їх у фабриканта, виступає тоді, яв капітальна вартість самого
водоспаду. Те, що водоспад не має вартости і що ціна його є звичайний
відбиток захоплюваного надзиску, капіталістично обчисленого, це одразу
\parbreak{}  %% абзац продовжується на наступній сторінці
