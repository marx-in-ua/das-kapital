\parcont{}  %% абзац починається на попередній сторінці
\index{iii2}{0193}  %% посилання на сторінку оригінального видання
пересічний зиск. Отже, припустімо, що в наявності є умови для нормального
використування капіталу на землі кляси $А$. Чи досить цього? Чи дійсно можна
буде тоді вкласти цей капітал? Чи, може, ринкова ціна мусить підвищитись
до такого ступеня, щоб ренту давала й найгірша земля $А$? Чи покладає, отже,
монополія земельного власника приміщенню капіталу таку межу, якої не було б
з суто-капіталістичного погляду без існування цієї монополії? Вже з умов поставленого
питання випливає, що коли, наприклад, на старих заорендованих
дільницях вкладено додаткові капітали, які при даній ринковій ціні не дають
жодної ренти, а дають лише пересічний зиск, то ця обставина ніяк не розв'язує
того питання, чи можна тепер в дійсності вкласти капітал у землю кляси $А$,
яка теж стала б давати пересічний зиск, але жодної ренти. В цьому якраз і є
питання. Що додаткові вкладання капіталу, які не дають ренти, не задовольняють
попиту, — це доводиться доконечністю притягнення нової землі кляси $А$. Якщо
додаткове оброблення землі $А$ відбувається тільки тоді, коли вона дає ренту,
отже, більше, ніж ціну продукції, то можливі тільки два випадки. Або ринкова
ціна мусить стояти на такому рівні, щоб навіть останні додаткові вкладання
капіталу на старих заорендованих дільницях давали надзиск, чи потрапляє він
в кишеню орендаря, чи власника. Це підвищення ціни і цей надзиск від останніх
додаткових вкладень капіталу були б тоді наслідком того, що земля $А$ не може
бути оброблювана, коли вона не дає ренти. Бо якби для оброблення було б
досить ціни продукції, тобто одержувати просто пересічний зиск, то ціна не
підвищилася б до такої міри, і конкуренція нових дільниць землі вже почалася
б, скоро вони стали б давати тільки ці ціни продукції. З додатковими
приміщеннями капіталу на старих заорендованих дільницях, що не дають ренти,
тоді почали б конкурувати приміщення капіталу на землі $А$, що так само не
дають ренти. — Абож останні приміщення капіталу на старих заорендованих
дільницях не дають ренти, але ринкова ціна піднеслась однак досить високо
для того, щоб земля $А$ почала оброблятися і давати ренту. В цьому випадку
додаткове вкладання капіталу, що не дає ренти, було можливе лише тому, що
земля $А$ не може оброблятись, поки ринкова ціна не дозволить їй давати ренту.
Без цієї умови культура її почалася б уже при нижчому рівні ціни; і ті пізніші
вкладання капіталу на старих заорендованих дільницях, які для того,
щоб давати звичайний зиск без ренти, потребують високої ринкової ціни, не
могли б статись. Адже і при високій ринковій ціні вони дають лише пересічний
зиск Отже при нижчій ціні, яка при культурі землі $А$ стала б реґуляційною,
як ціна продукції на ній, вони не давали б цього зиску, отже, прицьому
припущенні вони взагалі не могли б статись. Щоправда, рента з землі
$А$ була б таким чином диференційною рентою порівняно з цими приміщеннями
капіталу на старих заорендованих дільницях, що не дають ренти. Але
що дільниці землі $А$ створюють таку диференційну ренту, це є лише наслідок
того, що вони взагалі неприступні для оброблення, хіба тільки тоді коли
даватимуть ренту; отже, наслідок того, що виникає доконечність цієї ренти, яка
сама по собі не зумовлюється хоч би якою ріжницею між родами землі, і яка
створює межу для можливого приміщення додаткових капіталів на старих заорендованих
дільницях. В обох випадках рента з землі $А$ була б не звичайним
наслідком підвищення ціни збіжжя, а навпаки: та обставина, що найгірша
земля мусить давати ренту для того, щоб її взагалі дозволили обробляти, була б
за причину підвищення ціни збіжжя до такого пункту, на якому постане змога
здійснити цю умову.

Диференційна рента має ту особливість, що земельна власність тут лише
уловлює той надзиск, що його інакше захопив би орендар, і за певних обставин,
поки не скінчиться термін його орендного договору, дійсно захоплює. Земель а
власність є тут лише за причину перенесення певної, виниклої без її допомоги
\parbreak{}  %% абзац продовжується на наступній сторінці
