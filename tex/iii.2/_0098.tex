\parcont{}  %% абзац починається на попередній сторінці
\index{iii2}{0098}  %% посилання на сторінку оригінального видання
capitaliste industriel\footnote*{
Промисловий капіталіст. Пр.~Ред.
} Проте, помилково було б розглядати ті засоби, що ними
порядкує новітнє банкірство, як засоби гультяїв. Поперше, це — та частина
капіталу, що її тримають промисловці та капіталісти в даний момент без діла
у грошовій формі як грошовий запас або капітал, який лише мають примістити;
отже, це — гулящий капітал, а не капітал гулящих багатіїв. Подруге, це — та
частина всяких доходів та ощаджень, що назавжди або тимчасово призначена
для нагромадження. І обидві ці частині істотні для характеру банкової системи.

Однак, ніколи не треба забувати, поперше, що гроші — в формі благородних
металів — лишаються основою, що від неї кредитова справа з природи
своєї \emph{ніколи} не може звільнитися. Подруге, що кредитова система має собі
за передумову монополію приватних осіб на суспільні засоби продукції (в формі
капіталу та земельної власності!), що сама вона, з одного боку, є іманентна
форма капіталістичного способу продукції, а з другого боку, рушійна сила його
розвитку до його найвищої та останньої можливої форми.

Банкова система щодо форм організації та централізації, як то висловлено
ще в 1697 році в творі «Some Thoughts of the Interest of England» є найбільш
штучний та найбільш розвинутий продукт, що його взагалі може породити
капіталістичний спосіб продукцїї. Відси величезна влада такої установи,
як Англійський банк, над торговлею та промисловістю, дарма що дійсний рух
їх лишається поза його межами, та що він ставиться до того руху пасивно.
Звичайно в банковій системі дано форму загального рахівництва та розподілу
засобів продукції у суспільному маштабі, але тільки форму. Ми бачили, що пересічний
зиск поодинокого капіталіста або кожного окремого капіталу визначається
не додатковою працею, що його присвоює собі цей капітал безпосередньо,
а кількістю усієї тієї додаткової праці, яку присвоює собі весь капітал, та з якої
кожен окремий капітал одержує свій дивіденд лише пропорційно тій частині, що
її він становить в усьому капіталі. Цей суспільний характер капіталу цілком
здійснюється лише за посередництвом повного розвитку кредитової та банкової
системи. З другого боку, ця система йде далі. Вона дає промисловим та торговельним
капіталістам до розпорядку ввесь той вільний і навіть потенціяльний
капітал суспільства, що не діє ще активно, так що ані позикодавець, ані уживач
цього капіталу не є його власники або продуценти. Тим способом знищує
вона приватний характер капіталу, і тому має в собі, але й тільки в собі, знищення
самого капіталу. Банкова справа забирає з рук приватних капіталістів
та лихварів розподіл капіталу, — осібну справу, — як суспільну функцію. Але
з тієї причини банк та кредит одночасно стають і наймогутнішим засобом, що
заганяє розвиток капіталістичної продукції поза її власні межі, і одним з найдужчих
факторів криз та спекуляції.

Далі, замінюючи гроші різними формами кредитової циркуляції, банкова
система показує, що гроші дійсно є не що інше, як тільки осібний вислів суспільного
характеру праці та її продуктів, але вислів цей, як суперечний базі приватної
продукції, кінець-кінцем, мусить завжди видаватися як річ, як осібний
товар побіч інших товарів.

Насамкінець, немає жодного сумніву, що кредитова система служитиме за
могутню підойму підчас переходу від капіталістичного способу до способу продукції
асоційованої праці; однак лише як певний елемент у зв’язку з іншими
великими органічними переворотами в самому способі продукції. Навпаки, ілюзії
щодо чудотворної сили кредитової та банкової справи, ілюзії в соціалістичному
розумінні, постають з повного нерозуміння капіталістичного способу продукції
та кредитової справи, як однієї з його форм. Скоро засоби продукції перестали
перетворюватись на капітал (а це включає і знищення приватної
\parbreak{}  %% абзац продовжується на наступній сторінці
