\parcont{}  %% абзац починається на попередній сторінці
\index{iii2}{0195}  %% посилання на сторінку оригінального видання
Wakefield\footnote{
Wakefeld: England and America. London 1833. Порів, також книгу 1, розд. XXV.
} і вже задовго до нього відкрили фізіократ Мірабо-батько та інші
старі економісти. Тут цілком байдуже, чи привласнюють колоністи собі землю
просто, чи вони під виглядом номінальної ціпи землі в дійсності виплачують державі
лише податок за правний юридичний титул на землю. Байдуже також і те, що вже
осілі колоністи є юридичні власники землі. Фактично земельна власність не
становить тут межі для приміщення капіталу або також праці без капіталу;
захоплення частини землі вже осілими колоністами не виключає для нових
прихідців можливости зробити нову землю сферою приміщення їхнього капіталу
або їхньої праці. Тому тоді, коли доводиться досліджувати як земельна власність
впливає на ціни продуктів землі і на ренту там, де ця власність обмежує
землю як сферу приміщення капіталу, було б в найбільшій мірі недоладним
посилатися на вільні буржуазні колонії, де немає ані капіталістичного способу
продукції в хліборобстві, ані відповідної йому форми земельної власности, і де
остання взагалі фактично не існує. Так робить, наприклад, Рікардо, в розділі
про земельну ренту. Спочатку він говорить, що хоче дослідити вплив привласнення
землі на вартість продуктів землі і безпосередньо після цього бере, як
ілюстрацію, колонії, при чому припускає, що земля існує там в порівняно
первісних умовах і експлуатація її не обмежується монополією земельної
власности.

Сама юридична власність на землю не створює земельної ренти для власника.
Але дає йому, певно, силу усувати свою землю від експлуатації доти,
доки економічні відносини дозволять таке використання її, яке дасть йому
певний надмір, при чому байдуже, чи застосовуватиметься землю для власне
хліборобства, чи для інших продукційних цілей, як будівлі тощо. Він не може
збільшити або зменшити абсолютного розміру цієї сфери підприємств, але, певна
річ, може зробити це щодо тієї кількости її, яка перебуває на ринку. Звідси,
як відзначив уже Фур’є, той характеристичний факт, що в усіх цивілізованих
країнах порівняно значна частина землі завжди усунена від оброблення.

Отже, припускаючи такий випадок, що попит потребує обробітку нових
земель, скажімо, менш родючих, ніж оброблювані до того часу, то чи стане
тоді земельний власник здавати ці землі в оренду даром, тому що ринкова
ціна продукту землі піднеслась досить високо, так що приміщення капіталу
в ту землю дає орендареві ціну продукції, а тому й звичайний зиск? Ні в якому
разі. Вкладання капіталу мусить дати йому ренту. Він здає в оренду лише тоді,
коли йому може бути виплачена орендна плата. Отже, щоб можна було виплачувати
земельному власникові ренту, ринкова ціна мусить піднестись вище
ціни продукції, до $Р + r$. А що, згідно з припущенням, земельна власність
без здачі в оренду нічого не дає, економічно є безвартісна, то невеликого підвищення
ринкової ціни над ціною продукції досить для того, щоб дати на ринок
нову землю найгіршого роду.

Тепер постає таке питання: чи випливає з земельної ренти з найгіршої
землі, ренти, яка не може бути виведена з ріжниці родючости, те, що ціна
продукту землі неминуче є монопольною ціною в звичайному значінні, або
ціною, до складу якої рента входить в такій самій формі, як податок, з тією
тільки ріжницею, що цей податок стягає земельний власник замість держави?
Що такий податок має свої певні економічні межі, це зрозуміло само собою.
Він обмежується додатковими приміщеннями капіталу на старих заорендованих
дільницях, конкуренцією закордонних продуктів землі — припускаючи вільний
довіз їх — конкуренцією земельних власників між собою, нарешті, потребою
і платоспроможністю споживачів. Але тут мова йде не про те. Мова йде про
те, чи входить рента, виплачувана найгіршою землею, в ціну її продукту, яка
\parbreak{}  %% абзац продовжується на наступній сторінці
