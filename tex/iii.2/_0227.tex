\parcont{}  %% абзац починається на попередній сторінці
\index{iii2}{0227}  %% посилання на сторінку оригінального видання
французькі й західньо-німецькі селяни. Про колонії ми тут не говоримо, бо
незалежний селянин розвивається там в інших умовах.

Вільна власність селянина, який сам господарює, очевидно, є найнормальніша
форма земельної власности для дрібної продукції; тобто для такого способу
продукції, за якого посідання землі є умова власности робітника на продукт
його власної праці, і за якого, чи буде хлібороб вільним власником, чи підлеглим,
він в усякому разі завжди повинен сам, самостійно являючи відокремленого
робітника, разом з своєю родиною, продукувати засоби свого існування. Власність
на землю так само потрібна для цілковитого розвитку цього способу продукції,
як власність на знаряддя для вільного розвитку ремісничої продукції. Вона
становить тут базу для розвитку особистої самостійности. Вона є доконечний
переходовий пункт для розвитку самого хліборобства. Причини, з яких вона
гине, показують, які є її межі. Вони такі: знищення сільської хатньої промисловости,
що становить нормальне доповнення до неї, в наслідок розвитку
великої промисловости; поступове зубожіння і виснаження землі, що підлягає
цій культурі; узурпація великими земельними власниками громадської власности,
яка всюди являє собою друге доповнення парцелярного господарства і
тільки й дає йому можливість держати худобу; конкуренція великого господарства,
чи провадиться його як плянтаторське господарство, чи капіталістично.
Поліпшення в хліборобстві, які призводять, з одного боку, до пониження цін
хліборобських продуктів, і, з другого боку, потребують збільшення витрат і
достатніших речових умов продукції, також сприяють загибелі цієї власности
на землю, як це було, наприклад, за першої половини XVIII століття в Англії.

Парцелярна власність за своєю природою виключає: розвиток суспільних
продуктивних сил праці, суспільні форми праці, суспільну концентрацію капіталів,
скотарство в великому маштабі, проґресивне застосування науки.

Лихварство і податкова система всюди мусять її зубожувати. Витрата капіталу
на купівлю землі відтягає цей капітал від культури. Безконечне розпорошення
засобів продукції і відокремлення самих продуцентів. Величезне марнотратство
людської сили. Проґресивне погіршення умов продукції і подорожчання
засобів продукції — доконечний закон парцелярної власности. Урожайні роки для
цього способу продукції є нещастям\footnote{
Див. тронну промову французького короля у Туке.
}.

Одно з специфічних лих дрібного хліборобства, коли воно зв’язане з вільною
власністю на землю, постає з того, що обробник витрачає капітал на купівлю
землі. (Те саме має силу і щодо переходової форми, коли великий поміщик
витрачає капітал, поперше, на купівлю землі і, подруге, на те, щоб як
свій власний орендар господарювати на ній). При тій рухливості, якої набуває
тут земля як простий товар, зростає число перемін у володінні нею\footnote{Див. Meunier та Rubichon.}, так що
для кожного нового покоління з кожним поділом спадщини, земля, з погляду
селянина, знову виступає у вигляді витрати капіталу, тобто стає купленою
ним землею. Ціна землі становить тут переважний елемент індивідуальних фалшивих
витрат продукції, або витрат продукції продукту для поодиноких продуцентів.

Ціна землі є не що інше, як капіталізована і тому антиципована рента.
Коли хліборобство провадиться капіталістично, так що земельний власник одержує
тільки ренту, а орендар нічого не платить за землю, крім цієї щорічної
ренти, то ясно, що хоч витрата капіталу самим земельним власником на купівлю
землі є для нього процентодайним приміщенням капіталу, проте капітал
цей, не має ніякого чинення до капіталу приміщеного у саме хліборобство. Він
не являє собою ані частини основного капіталу, що тут функціонує, ані частини
\parbreak{}  %% абзац продовжується на наступній сторінці
