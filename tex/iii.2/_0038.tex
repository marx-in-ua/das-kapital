\parcont{}  %% абзац починається на попередній сторінці
\index{iii2}{0038}  %% посилання на сторінку оригінального видання
дешевші, ніж де-інде», каже в 1848 році тодішній управитель Англійського
банку перед таємним комітетом лордів (C.~D. 1848, printed 1857, № 219).

Розглядаючи капітал, що дає процент, ми виявили вже тоді, що пересічний
процент для довшої низки років визначається — за інших незмінних обставин
— пересічною нормою зиску; не підприємецького бариша, бо той бариш
сам є не що інше, а тільки зиск мінус процент.

Те, що і для коливань торговельного проценту — того проценту, що в межах
торговельного світу обраховується позикодавцями грошей для дисконтування
та позик, — настає така фаза протягом промислового циклу, коли рівень проценту
перевищує свій мінімум, досягаючи середньої пересічної висоти (що її він
потім згодом перевищує), при чому цей рух є наслідок піднесення зиску, — і це
вже згадано та згодом ще докладніше дослідиться.

Тимчасом тут треба зауважити дві речі:

\emph{Поперше}. Коли рівень проценту протягом довшого часу (ми кажемо тут
про рівень проценту в даній країні, як от в Англії, де середній рівень проценту
для довшого часу є даний і де він також виявляється і в проценті, що його платять
за позики на довші реченці і що його можна назвати приватним процентом)
тримається високо, то це є prima facie доказ того, що протягом цього
часу норма зиску є висока, однак це не доводить неминуче, що норма підприємецького
бариша висока. Ця остання ріжниця більш або менш відпадає для
капіталістів, що працюють переважно своїм капіталом; вони реалізують високу
норму зиску, бо вони процент платять самим собі. Можливість високого рівня
проценту протягом довшого часу — ми не кажемо тут про фазу дійсного пригнічення
— дається високою нормою зиску. Однак можливо, що ця висока норма
зиску, після відлічення високої норми проценту, лишить тільки низьку норму
підприємецького бариша. Ця остання може зменшитися, тимчасом коли висока
норма зиску існуватиме й далі. Це можливо тому, що підприємства, скоро вони
вже почали працювати, мусять провадитися й далі. У цій фазі багато підприємців
працюють самим лише кредитовим капіталом (чужим капіталом); і висока
норма зиску може бути подекуди спекулятивна у сподіванці на добрі справи
в майбутньому. Високу норму проценту можуть платити при високій нормі зиску,
але при чимраз меншому підприємецькому бариші. Її можуть виплачувати —
і це буває, почасти, за часів спекуляції — не з зиску, а з самого позиченого
чужого капіталу, і це може тривати деякий час.

\emph{Подруге}. Вираз, що попит на грошовий капітал, а тому й норма проценту
зростають в наслідок того, що норма зиску висока, не є тотожній з тим, що попит
на промисловий капітал зростає, а тому й норма проценту є висока.

Підчас кризи попит на позичковий капітал, а разом з ним і норма проценту
досягають свого максимуму; норма зиску, а з нею і попит на промисловий
капітал майже зовсім зникають. Тоді кожен позичає тільки на те, щоб
платити, щоб ліквідувати свої попередні зобов’язання. Навпаки, підчас оживлення
по кризі вимагають позичкового капіталу на те, щоб купувати та щоб перетворювати
грошовий капітал в продуктивний, або комерційний капітал. І тоді
вимагає його або промисловий капіталіст або купець. Промисловий капіталіст
витрачає його на засоби продукції та на робочу силу.

Чимраз більший попит на робочу силу сам по собі ніколи не може бути
основою для піднесення рівня проценту, оскільки останній визначається нормою
зиску. Вища заробітна плата ніколи не є основа вищого зиску, хоч — якщо
розглядати осібні фази промислового циклу — вона й може бути одним з його
наслідків.

Попит на робочу силу може більшати тому, що визиск праці відбувається
серед особливих сприятливих обставин, але чимраз більший попит на робочу
силу, а тому й на змінний капітал сам по собі не збільшує зиску, а зменшує
\parbreak{}  %% абзац продовжується на наступній сторінці
