\parcont{}  %% абзац починається на попередній сторінці
\index{iii2}{0186}  %% посилання на сторінку оригінального видання
для всіх кращих земель при витраті в 2\sfrac{1}{2}  ф. стерл. Частина їхнього колишнього
надпродукту входить тепер в створення їхнього потрібного продукту, так
само, як частина їхнього колишнього надзиску — в створення пересічного зиску.

Коли, навпаки, обчислити так само, як на кращих землях, де пересічне обчислення
нічого не змінює в абсолютній величині надпродукту, зглядно надзиску, бо для
них межа витрати капіталу дана загальною ціною продукції, то квартер від першої
витрати капіталу коштує 3 ф. стерл., а 2 квартерп від кожної другої витрати
лише по 1\sfrac{1}{2} ф. стерл.. Отже, на $А$ постала б збіжжева рента в 1 квартер
і грошова рента в 3 ф. стерл., але ці 3 квартери продавалися б по старій
ціні, разом за 9 ф. стерл. Коли б сталася третя витрата капіталу в 2\sfrac{1}{2}  ф.
стерл. з такою самою продуктивністю, як друга, то були б випродуковані тепер
разом 5 квартерів за 9 ф. стерл. ціни продукції. Коли б індивідуальна
пересічна ціна продукції на $А$ лишилась реґуляційною, то квартер довелося б
тепер продавати за 1\sfrac{4}{5}  ф. стерл. Пересічна ціна знову понизилася б не в
наслідок нового підвищепня продуктивности третьої витрати капіталу, а лише в
наслідок нової додаткової витрати капіталу з такою самою додатковою продуктивностю
як друга. Замість підвищити ренту, як це було б на землях, що дають
ренту, послідовні витрати капіталу вищої, але відносно незмінної продуктивности
на землі $А$, відповідно понизили б ціну продукції, а разом з тим, в інших
рівних умовах, і диференційну ренту на всіх інших родах землі. Навпаки,
коли б перша витрата капіталу, яка продукує 1 квартер за 3 ф. стерл. ціни
продукції, залишилася сама по собі міродайною, то 5 квартерів були б продані
за 15 ф. стерл., і диференційна рента від пізніших витрат капіталу на землі
$А$ становила б 6 ф. стерл.. Приєднання додаткового капіталу до акра землі $А$,
хоч би в якій формі відбулося воно, було б тут поліпшенням, і додатковий
капітал зробив би продуктивнішою і первісну частину капіталу. Було б безглуздям
сказати, що \sfrac{1}{3}  капіталу випродукувала 1 квартер, а інші \sfrac{2}{3} випродукували
4 квартери. 9 ф. стерл. на акр завжди продукували б 5 квартерів, тимчасом
як 3 ф. стерл. продукували б тільки 1 квартер. Чи постала б тут рента, надзиск,
чи ні, це цілком залежало б від обставин. Нормально, реґуляційна ціна
продукції мусула б понизитись. Так буде в тому випадку, коли це поліпшене,
а тому сполучене із збільшеними витратами оброблення відбувається на землі $А$
лише тому, що воно також відбувається і на кращих родах землі, що, отже,
відбувається загальна революція в хліборобстві; так що тепер, коли мова йде
про природну родючість землі $А$, то припускається, що на неї витрачено 6, зглядно
9 ф. стерл. замість 3 ф. стерл. Це особливо мало б силу тоді, коли більшість
оброблених акрів землі $А$, які постачають головну масу подання в даній країні,
підпали б цій новій методі. Але коли б поліпшення охопило спочатку лише
невелику частку площі $А$, то ця краще оброблювана частка давала б надзиск,
що його землевласник швидко подбав би перетворити цілком або почасти
на ренту і фіксувати як ренту. Таким чином, коли б попит розвивався рівнобіжно
з ростучим поданням, то поступово, в міру того, як земля $А$ в усій своїй площі
помалу підпадала б під нову методу обробітку, могла б створитися рента
на всій землі якости $А$, і додаткова продуктивність цілком або почасти, залежно
від умов ринку, була б конфіскована. Таким чином, вирівнянню ціни продукції
з землі $А$ у пересічну ціну її продукту, одержуваного з неї при збільшеній витраті
капіталу, могло б перешкодити фіксування надзиску від цієї збільшеної витрати
капіталу у формі ренти. В цьому випадку це знову було б, як ми бачили це
давніш на кращих землях за низхідної продуктивної сили додаткових капіталів,
перетворення надзиску в земельну ренту, тобто втручання земельної власности,
яке підвищило б ціну продукції, замість того, щоб диференційна рента була
просто наслідком ріжниць між індивідуальною і загальною ціною продукції. Це
перешкодило б для землі $А$ збігові обох цін, бо перешкодило б реґулюванню
\parbreak{}  %% абзац продовжується на наступній сторінці
