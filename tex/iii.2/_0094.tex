\parcont{}  %% абзац починається на попередній сторінці
\index{iii2}{0094}  %% посилання на сторінку оригінального видання
зливаються в одне, це доводять новітні грошові кризи. Але те саме лихварство стає
головним засобом, щоб далі розвивати потребу в грошах як платіжному засобі,
чим раз дужче обтяжуючи продуцента боргами та знищуючи тим самим в нього
звичайні платіжні засоби, так що тягар процентів робить для нього неможливою
навіть його звичайну репродукцію. Тут лихварство виростає з грошей, як платіжного
засобу, та поширює цю функцію грошей, — своє справжнє поле.

Розвиток кредитової справи постає як реакція проти лихварства. Але цього
не треба розуміти хибно й ніяк не можна розуміти в дусі античних письменників,
батьків церкви, Лютера або давніших соціялістів. Це означає ані менше ані
більше як тільки підпорядкування капіталу, що дає процент, умовам та потребам
капіталістичного способу продукції.

Взагалі та в цілому капітал, що дає процент, за новітньої кредитової системи
пристосовується до умов капіталістичної продукції. Лихварство, як таке, не
тільки існує далі, але у народів розвинутої капіталістичної продукції визволяється
із меж, поставлених йому всім давнім законодавством. Капітал, що дає
процент, зберігає форму лихварського капіталу проти таких осіб та кляс або
серед таких відносин, що серед них немає позик або не може бути позик в розумінні
капіталістичного способу продукції; коли позичають з причини індивідуальної
потреби, як от в льомбарді; коли позичається багатству, що споживає,
для марнотратства; або коли продуцент є некапіталістичний продуцент, дрібний
селянин, ремесник і~\abbr{т. ін.} отже, як безпосередній продуцент, щє є власник своїх
власних умов продукції; насамкінець, коли сам капіталістичний продуцент оперує
в такому малому маштабі, що наближається до тих продуцентів, які працюють сами.

Те, чим капітал, що дає процент, — оскільки він становить істотний момент
капіталістичного способу продукції — відрізняється від лихварського капіталу,
ніяк не є в самій природі або в самому характері цього капіталу. Ріжницю становлять
лише змінені умови, що серед них він функціонує, а тому й цілком
перетворена постать довжника, що протистоїть грошовому позикодавцеві. Навіть
тоді, коли як промисловець або як купець одержує кредит незаможна особа, то
це робиться в тій надії, що він, функціонуватиме як капіталіст, і за допомогою
позиченого капіталу присвоюватиме неоплачену працю. Кредит дається йому як
потенціяльному капіталістові. І ця обставина, що їй так дуже радіють економісти
апологети, а саме, що людина без майна, але з енерґією, солідністю, здібністю
та знанням справ може перетворитися таким способом на капіталіста — аджеж,
взагалі, за капіталістичного способу продукції торговельну цінність кожного
індивідуума оцінюється більш або менш правильно, — ця обставина, хоч і як
вона раз-у-раз виводить на поле бою проти наявних поодиноких капіталістів,
небажаний їм шерег нових шукачів щастя, зміцнює панування самого капіталу,
поширює базу того панування, даючи йому змогу рекрутувати собі раз-у-раз
нові сили з суспільної основи. Цілком так само, як та обставина, що католицька
церква в середні віки творила свою гієрархію з найкращих голів народу, не
вважаючи на стан, рід та майно, була головним засобом до зміцнення попів та
пригнічення мирян. Що більше панівна кляса є здібна вбирати в себе найвизначніших
людей з кляс поневолених, то міцніше та небезпечніше є її панування.

Тому ініціятори новітньої кредитової системи замість взагалі клясти капітал,
що дає процент, виходять, навпаки, з його виразного визнання.

Ми не говоримо тут про реакцію проти лихварства, що силкувалася боронити
від нього убогих, як от Monts de-piété\footnote*{
Дослівно — «захисток милосердя», назва льомбардів у романських країнах. \emph{Пр.~Ред.}
} (в 1350 році в Сарлені у Франш-Конте,
пізніше в Перуджії й Савоні в Італії, в роках 1400 та 1479). Вони
гідні уваги лише тому, що виявляють ту іронію історію, яка побожні бажання
обертає при реалізації їх у пряму протилежність. За помірним цінуванням
\parbreak{}  %% абзац продовжується на наступній сторінці
