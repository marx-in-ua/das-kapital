
\index{iii2}{0163}  %% посилання на сторінку оригінального видання

\begin{table}[H]
  \begin{center}
    \emph{Таблиця VIa}
    \footnotesize

  \begin{tabular}{c@{  } c@{  } c@{  } c@{  } c@{  } c@{  } c@{  } c@{  } c}
    \toprule
      \multirowcell{2}{Земля} &
      \multirowcell{2}{Акри} &
      Капітал &
      Зиск &
      \multirowcell{2}{\makecell{Продукт з акра\\ в квартерах}} &
      \makecell{Продажна \\ ціна} &
      %% TODO: Якась проблема з роздільними лініями починаючи зі здобуток
      \makecell{Здобу-\\ток} &
      \multicolumn{2}{c}{Рента} &\\

      \cmidrule(r){3-3}
      \cmidrule(r){4-4}
      \cmidrule(r){6-6}

      \cmidrule(r){7-7}
      \cmidrule(r){8-8}
      \cmidrule(r){9-9}

       &  & ф. ст. & ф. ст. & & ф. ст. & ф. ст. & Кварт. & ф. ст.   \\
      \midrule
       A & 1 & 2\sfrac{1}{2} \dplus{} 2\sfrac{1}{2} \deq{} 5 & 1 & 1 \dplus{} \phantom{0}3\phantom{\sfrac{1}{2}} \deq{} \phantom{0}4\phantom{\sfrac{1}{2}}   & 1\sfrac{1}{2} & \phantom{0}6\phantom{\sfrac{3}{4}} & \phantom{0}0\phantom{\sfrac{1}{2}}\footnotemarkZ{}  & \phantom{0}0\phantom{\sfrac{1}{2}} \\
       B & 1 & 2\sfrac{1}{2} \dplus{} 2\sfrac{1}{2} \deq{} 5 & 1 & 2 \dplus{} \phantom{0}2\sfrac{1}{2} \deq{} \phantom{0}4\sfrac{1}{2}                       & 1\sfrac{1}{2} & \phantom{0}6\sfrac{3}{4}           & \phantom{00}\sfrac{1}{2}                            & \phantom{00}\sfrac{3}{4}           \\
       C & 1 & 2\sfrac{1}{2} \dplus{} 2\sfrac{1}{2} \deq{} 5 & 1 & 3 \dplus{} \phantom{0}5\phantom{\sfrac{1}{2}} \deq{} \phantom{0}8\phantom{\sfrac{1}{2}}   & 1\sfrac{1}{2} & 12\phantom{\sfrac{3}{4}}           & \phantom{0}4\phantom{\sfrac{1}{2}}                  & \phantom{0}6\phantom{\sfrac{1}{2}} \\
       D & 1 & 2\sfrac{1}{2} \dplus{} 2\sfrac{1}{2} \deq{} 5 & 1 & 4 \dplus{} 12\phantom{\sfrac{1}{2}} \deq{} 16\phantom{\sfrac{1}{2}}                       & 1\sfrac{1}{2} & 24\phantom{\sfrac{3}{4}}           & 12\phantom{\sfrac{1}{2}}                            & 18\phantom{\sfrac{1}{2}}           \\
     \cmidrule(r){1-1}
     \cmidrule(r){2-2}
     \cmidrule(r){3-3}
     \cmidrule(r){4-4}
     \cmidrule(r){5-5}
     \cmidrule(r){6-6}

     \cmidrule(r){8-8}
     \cmidrule(r){9-9}

      Разом & 4 & \phantom{2\sfrac{1}{2} \dplus{} 2\sfrac{1}{2} \deq{}}20 & & \phantom{2 \dplus{} 12\sfrac{1}{2} \deq{}}32\sfrac{1}{2} & & & 16\sfrac{1}{2} & 24\sfrac{3}{4}\\
  \end{tabular}

  \end{center}
\end{table}
\footnotetextZ{В німецькому тексті тут очевидно помилково стоїть «6» \emph{Прим. Ред.}} % текст примітки прямо під заголовком

Нарешті, грошова рента підвищилася б, коли б у кращі земельні дільниці,
при тому самому відносному підвищенні родючости, вкладено було більше
додаткового капіталу, ніж у землю $А$, або коли б додаткові вкладання капіталу в кращі
земельні дільниці впливали, підвищуючи норму продуктивности. В обох випадках
ріжниці зростали б.

Грошова рента понижується, коли поліпшенння, що сталося в наслідок
додаткової витрати капіталу, зменшує всі ріжниці, або частину їх, впливаючи
більше на $А$, ніж на $В$ і $C$. Вона понижується то більше, що незначніше
підвищення продуктивности кращих земельних дільниць. Від відносної неоднаковости
впливу залежить, чи підвищиться збіжжева рента, чи понизиться або
залишиться без зміни.

Грошова рента підвищується, а також і збіжжева рента, або тоді, коли за
незмінної відносної ріжниці в додатковій родючості різних земель більше вкладається
додаткового капіталу в землю, що дає ренту, ніж у землю $А$, що не дає
ренти, і більше у землю, що дає вищу, ніж у землю, що дає нижчу ренту;
абож тоді, коли родючість, при однаковому додатковому капіталі, більше зростає
на кращій і найкращій землі, ніж на землі $А$, причому грошова і збіжжева
рента підвищується саме у такому відношенні, в якому це збільшення родючости
на вищих розрядах землі вище, ніж на нижчих.

Але за всяких обставин рента відносно підвищується, коли підвищена продуктивність
є наслідок додаткової витрати капіталу, а не просто наслідок
збільшеної родючости за незмінної витрати капіталу. Це є абсолютний погляд,
який показує, що тут, як і в усіх давніших випадках, рента і збільшена рента з акра
(подібно до того, як при диференційній ренті І висота пересічної ренти на всю
оброблювану площу) є наслідок збільшеної витрати капіталу на землю, при
чому байдуже, чи функціонує ця витрата з сталою нормою продуктивности за
сталих або понижених цін, чи з низхідною нормою продуктивности за сталих або
за понижених цін, чи з висхідною нормою продуктивности за понижених цін.
Бо наше припущення: стала ціна за сталої, низхідної або висхідної норми продуктивности додаткового
капіталу, і низхідна ціна, за сталої, низхідної і висхідної
норми продуктивности, зводиться ось до чого: стала норма продуктивности додаткового капіталу при
сталій або низхідній ціні, низхідна норма продуктивности
при сталій або низхідній ціні, висхідна норма продуктивности за сталої
\parbreak{}  %% абзац продовжується на наступній сторінці
