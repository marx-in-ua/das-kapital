\parcont{}  %% абзац починається на попередній сторінці
\index{iii2}{0153}  %% посилання на сторінку оригінального видання
капітал. Це включає, за незмінних ріжниць між родами землі, зріст надпродукту,
пропорційний зростові вкладеного капіталу. Отже, випадок цей виключає всяку
додаткову витрату на землю $А$, яка вплинула б на диференційну ренту. На цій землі
норма надзиску \deq{} 0; отже, вона лишається \deq{} 0, бо ми припустили, що продуктивна
сила додаткового капіталу, а тому і норма надзиску лишаються сталими.

Але реґуляційна ціна продукції може за цих умов лише знизитися, бо замість
ціни продукції з $А$ реґуляційною стає ціна продукції ближчої якістю землі
$В$ або взагалі з будь-якої землі, кращої, ніж $А$; отже, коли б ціна продукції
на землі $C$ зробилась регуляційною, то капітал був би вилучений з $А$,
або навіть з $А$ і $В$, і таким чином всі землі, гірші, ніж $C$, випали б з конкуренції
земель, на яких сіють пшеницю. Умова, потрібна для цього за даних
припущень, є в тому, щоб надпродукт з додаткових капіталовкладень задовольняв
потребам, і щоб тому продукція на гіршій землі $А$ тощо зробилася
зайвою для поновлення подання.

Отже, візьмімо, наприклад, таблицю II, але змінимо її так, щоб замість 20
квартерів, потребу задовольняли 18 квартерів. Земля $А$ відпала б; $D$, а
разом з нею ціна продукції в 30\shil{ шил.} за кв. стала б реґуляційною. Диференційна
рента набуває тоді такої форми:

\begin{table}[H]
  \begin{center}
    \emph{Таблиця ІV}
    \footnotesize

  \begin{tabular}{c c c c c c c c c c c}
    \toprule
      \multirowcell{2}{\makecell{Рід \\землі}} &
      \multirowcell{2}{\rotatebox[origin=c]{90}{Акри}} &
      \rotatebox[origin=c]{90}{Капітал} &
      \rotatebox[origin=c]{90}{Зиск} &
      \rotatebox[origin=c]{90}{\makecell{Ціна про- \\ дукції}} &
      \multirowcell{2}{\rotatebox[origin=c]{90}{\makecell{Продукт \\ в кварт.}}} &
      \rotatebox[origin=c]{90}{\makecell{Продажна \\ ціна}} &
      \rotatebox[origin=c]{90}{Здобуток} &
      \multicolumn{2}{c}{Рента} &
      \multirowcell{2}{\makecell{Норма \\надзиску}} \\

      \cmidrule(rl){3-3}
      \cmidrule(rl){4-4}
      \cmidrule(rl){5-5}
      \cmidrule(rl){7-7}
      \cmidrule(rl){8-8}
      \cmidrule(rl){9-10}

       &  &  ф. ст. & ф. ст. & ф. ст. & & ф. ст. & ф. ст. & Кварт. & ф. ст. &  \\
      \midrule

      B & 1 &  \phantom{0}5 & 1 & \phantom{0}6 & \phantom{0}4 & 1\sfrac{1}{2} & \phantom{0}6 & 0 & \phantom{0}0 & \phantom{00}0\% \\ % ця мітка у заголовку \\
      C & 1 &  \phantom{0}5 & 1 & \phantom{0}6 & \phantom{0}6 & 1\sfrac{1}{2} & \phantom{0}9 & 2 & \phantom{0}3 & \phantom{0}60\%\\
      D & 1 &  \phantom{0}5 & 1 & \phantom{0}6 & \phantom{0}8 & 1\sfrac{1}{2} & 12           & 4 & \phantom{0}6 & 120\%\\
     \cmidrule(rl){1-1}
     \cmidrule(rl){2-2}
     \cmidrule(rl){3-3}
     \cmidrule(rl){4-4}
     \cmidrule(rl){5-5}
     \cmidrule(rl){6-6}
     \cmidrule(rl){8-8}
     \cmidrule(rl){9-9}
     \cmidrule(rl){10-10}

     Разом & 3 & 15 & 3 & 18 & 18 & & 27 & 6 & 9 &\\
  \end{tabular}

  \end{center}
\end{table}

Отже, вся рента порівняно з таблицею II знизилась би з 36\pound{ ф. стерл.}
до 9, а в збіжжі з 12 кварт, до 6; вся продукція знизилася б лише на 2
квартери, з 20 до 18. Норма надзиску, обчислена у відношенні до капіталу,
знизилася б наполовину, з 180 до 90\%\footnote*{
До 60\%, тобто знизилася б втроє, бо в таблиці II вона \deq{} \frac{36}{20} × 100 \deq{} 180\%, а в тaблиці
IV вона $= \frac{9}{15} × 100 \deq{} 60\%$. \emph{Прим. Ред.}
}. Отже, пониженню ціни продукції
тут відповідає зменшення збіжжевої і грошової ренти.

Порівняно з таблицею І, відбувається лише зменшення грошової ренти;
збіжжева рента в обох випадках дорівнює 6 квартерам; але тільки в одному
випадку вона \deq{} 18\pound{ ф. стерл.}, а в другому \deq{} 9\pound{ ф. стерл}. Для земель $C$ і $D$
збіжжева рента проти таблиці І лишилась та сама\footnote*{
Те, що сказано тут, правильне лише для землі $C$, але неправильне для землі $D$, бо в табл. І земля
$D$ дає 3 кв. ренти, а в табл. IV земля $D$ дає 4 кв. ренти. Те, що тут сказано, було б правильне, коли
взяти загальну ренту з земель $B$, $C$ і $D$. \emph{Прим. Ред.}
}. В дійсності, в наслідок
того, що додаткова продукція, досягнена з допомогою додаткового капіталу рівної
продуктивности, витиснула з ринку продукт $А$ і разом з тим усунула землю $А$
з числа конкурентних аґентів продукції — в наслідок цього в дійсності створилася
нова диференційна рента І, в якій краща земля $В$ від грає ту саму ролю,
яку давніш відігравала гірша земля $А$. В наслідок цього, з одного боку, відпадає
рента з $В$; з другого боку, згідно з припущенням, вкладення додаткового
\parbreak{}  %% абзац продовжується на наступній сторінці
