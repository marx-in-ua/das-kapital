\parcont{}  %% абзац починається на попередній сторінці
\index{iii2}{0182}  %% посилання на сторінку оригінального видання
капіталу, яка продукувала б квартер дорожче за 3\pound{ ф. стерл.}, призвела б до
скорочення його зиску. Це, за недостатньої продуктивности, перешкоджає вирівнянню індивідуальної
пересічної ціни.

Візьмімо цей випадок у колишньому прикладі, коли ціна продукції на
землі А в 3\pound{ ф. стерл.} за квартер реґулює ціну для землі В.

\begin{table}[h]
  \begin{center}
    \footnotesize

  \begin{tabular}{c@{  } c@{  } c@{  } c@{  } c@{  } c@{  } c@{  } c@{  } c@{  } c@{  } c}
    \toprule
      Капітал &
      Зиск &
      \makecell{Ціна \\ продукції} &
      Здобуток &
      \makecell{Ціна \\ продукції \\ за кварт.} &
      \multicolumn{2}{c}{Продажна ціна} &
      Надзиск &
      Витрата \\

      \cmidrule(r){1-1}
      \cmidrule(r){2-2}
      \cmidrule(r){3-3}
      \cmidrule(r){4-4}
      \cmidrule(r){5-5}
      \cmidrule(r){6-7}
      \cmidrule(r){8-8}
      \cmidrule(r){9-9}\pound{ ф. ст.} & ф. ст. & ф. ст. & Кварт & ф. ст. & \makecell{за квартер \\ ф. ст.} & \makecell{разом \\ ф. ст.} & ф. ст. & ф. ст.   \\
      \midrule
      \phantom{0}2\sfrac{1}{2}           & \phantom{1}\sfrac{1}{2} & \phantom{0}3 & 2\phantom{\sfrac{1}{2}} & 1\sfrac{1}{2}           & 3 & \phantom{0}6\phantom{\sfrac{1}{2}} & 3\phantom{\sfrac{1}{2}} & \textemdash \\
      \phantom{0}2\sfrac{1}{2}           & \phantom{1}\sfrac{1}{2} & \phantom{0}3 & 1\sfrac{1}{2}           & 2\phantom{\sfrac{1}{2}} & 3 & \phantom{0}4\sfrac{1}{2}           & 1\sfrac{1}{2}           & \textemdash \\
      \phantom{0}5\phantom{\sfrac{1}{2}} & 1\phantom{\sfrac{1}{2}} & \phantom{0}6 & 1\sfrac{1}{2}           & 4\footnotemarkZ{}\phantom{1}  & 3 & \phantom{0}4\sfrac{1}{2}           & \textemdash             & 1\sfrac{1}{2}           \\
      \phantom{0}5\phantom{\sfrac{1}{2}} & 1\phantom{\sfrac{1}{2}} & \phantom{0}6 & 1\phantom{\sfrac{1}{2}} & 6\phantom{\sfrac{1}{2}} & 3 & \phantom{0}3\phantom{\sfrac{1}{2}} & \textemdash             & 3\phantom{\sfrac{1}{2}} \\

     \cmidrule(r){1-1}
     \cmidrule(r){2-2}
     \cmidrule(r){3-3}
     \cmidrule(r){7-7}
     \cmidrule(r){8-8}
     \cmidrule(r){9-9}

       15\phantom{\sfrac{1}{2}} & 3\phantom{\sfrac{1}{2}} & 18 & & & & 18\phantom{\sfrac{1}{1}} & 4\sfrac{1}{2} & 4\sfrac{1}{2} \\
  \end{tabular}

  \end{center}
\end{table}
\footnotetextZ{В німецькому тексті тут стоїть «3». Очевидна помилка. \emph{Прим. Ред}}

Ціна продукції 3\sfrac{1}{2} квартерів з перших двох витрат капіталу так само
становить для орендаря 3\pound{ ф. стерл.} за квартер, бо йому доводиться виплачувати
ренту в 4\sfrac{1}{2}\pound{ ф. стерл.}, при чому ріжниця між його індивідуальною ціною продукції
і загальною ціною продукції іде, таким чином, не в його кишеню. Отже,
надмір ціни продукту перших двох витрат не може йому покрити дефіциту
в продуктах третьої і четвертої витрати капіталу.

1\sfrac{1}{2} квартера від витрати капіталу 3) коштують орендареві, включаючи
і зиск, 6\pound{ ф. стерл.}; але за реґуляційної ціни в 3\pound{ ф. стерл.} за квартер він може
продати їх лише за 4\sfrac{1}{2}\pound{ ф. стерл}. Отже, він втратив би не тільки ввесь
зиск, але й понад нього \sfrac{1}{2}\pound{ ф. стерл.} або  10\% вкладеного капіталу в 5\pound{ ф. стерл}.
Його втрата з зиску і капіталу при витраті капіталу 3) дорівнювала б  1\sfrac{1}{2}\pound{ ф. стерл.}, а при витраті капіталу 4) — 3\pound{ ф. стерл.}, разом 4\sfrac{1}{2}\pound{ ф. стерл.}, якраз
стільки, скільки становить рента від продуктивніших витрат капіталу, що їхня
індивідуальна ціна продукції саме тому не може справити вирівнювального
впливу на індивідуальну пересічну ціну продукції всього продукту землі В, що
надмір його доводиться виплатити як ренту третій особі.

Коли б для задоволення попиту довелося випродукувати додаткові 1\sfrac{1}{2}
квартери з допомогою третьої витрати капіталу, то регуляційна ринкова ціна
мусила б піднестись до 4\pound{ ф. стерл.} за квартер. В наслідок цього підвищення
реґуляційної ринкової ціни рента на землі В для першої і другої витрати капіталу
підвищилась би, а на землі А створилася б рента.

Отже, хоч диференційна рента є лише формальне перетворення надзиску
в ренту, і хоч власність на землю дає тут власникові лише можливість перемістити
надзиск з рук орендаря в свої, проте виявляється, що послідовна витрата
капіталу на ту саму земельну площу або, що сходить на те саме, збільшення
капіталу, витраченого на тій самій земельній площі, за низхідної норми
продуктивности капіталу і незмінної реґуляційної ціни геть швидше доходить
до своєї межі, отже, досягає в дійсності більш або менш штучної межі, в наслідок
просто формального перетворення надзиску в земельну ренту, що є наслідком
земельної власности. Отже підвищення загальної ціни продукції, яке стає
тут доконечним за вужчих меж, ніж в інших умовах, є тут не тільки за причину
\parbreak{}  %% абзац продовжується на наступній сторінці
