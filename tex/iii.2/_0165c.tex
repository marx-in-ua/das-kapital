\looseness=1
У вищенаведеному випадку ми припускали, що продуктивна сила другого
капіталовкладення вища, ніж первісна продуктивність першого вкладення. Справа
не зміниться, коли ми припустимо для другого капіталовкладення лише таку саму
продуктивність, що її мала первісна продуктивність першого вкладення, як от у
таблиці VIII.

\vspace{-\bigskipamount}
\begin{table}[H]
  \centering
  \caption*{Таблиця VIII}

  \footnotesize
  \setlength{\tabcolsep}{4.5pt}
  \settowidth\rotheadsize{\theadfont Продажна}

  \begin{tabular}{l c r c c r c c c c c}
    \toprule
      \thead[tl]{Рід\\землі} &
      &
      \thead[t]{Капітал} &
      \rothead{Зиск} &
      \rothead{Ціна\\продукції} &
      \thead[t]{Продукт} & % \\ в кварт.}}}
      \rothead{Продажна\\ціна} &
      \rothead{Здобуток} &
      \multicolumn{2}{c}{Рента} &
      \rothead{Норма\\надзиску} \\

    \cmidrule(rl){2-11}
      & акри  & \poundsign{} & \poundsign{} & \poundsign{} & кв. & \poundsign{} & \poundsign{} & кв. & \poundsign{} & \% \\

    \midrule
      A & 1 & 2\tbfrac{1}{2} \dplus{} 2\tbfrac{1}{2} \deq{} 5 & 1 & 6 & \phantom{0}\tbfrac{1}{2} \dplus{} 1 \deq{} 1\tbfrac{1}{2}                                 & 4 & \phantom{0}6 & 0\phantom{\tbfrac{1}{2}} & \phantom{0}0 & \phantom{00}0 \\
      B & 1 & 2\tbfrac{1}{2} \dplus{} 2\tbfrac{1}{2} \deq{} 5 & 1 & 6 & 1\phantom{\tbfrac{1}{2}} \dplus{} 2 \deq{} 3\phantom{\tbfrac{1}{2}}                       & 4 & 12           & 1\tbfrac{1}{2}           & \phantom{0}6 & 120 \\
      C & 1 & 2\tbfrac{1}{2} \dplus{} 2\tbfrac{1}{2} \deq{} 5 & 1 & 6 & 1\tbfrac{1}{2} \dplus{} 3 \deq{} 4\tbfrac{1}{4}                                           & 4 & 18           & 3\phantom{\tbfrac{1}{2}} & 12           & 240\\
      D & 1 & 2\tbfrac{1}{2} \dplus{} 2\tbfrac{1}{2} \deq{} 5 & 1 & 6 & 2\phantom{\tbfrac{1}{2}} \dplus{} 4 \deq{} 6\phantom{\tbfrac{1}{2}} & 4 & 24           & 4\tbfrac{1}{2}           & 18           & 360\\
     \midrule
       Разом & & 20 & & & 15\pF{} & & 60 & 9\pF{} & 36 & 240\\
  \end{tabular}
\end{table}
\vspace{-\bigskipamount}

\noindent{}І тут ціна продукції, яка підвищується в тому самому відношенні, зумовлює
те, що зменшення продуктивности цілком урівноважується так щодо здобутку,
як і щодо грошової ренти.

\looseness=1
У своєму чистому вигляді третій випадок виступає лише за низхідної продуктивности
другого капіталовкладення, тимчасом як продуктивність першого вкладення,
як це всюди припускалось для першого і другого випадків, лишається сталою.
Диференційна рента І тут не зачіпається, зміна відбувається лише з тією частиною,
що походить з диференційної ренти II.~Ми подаємо два приклади: в
першому продуктивність другого капіталовкладення зводиться до \sfrac{1}{2}, у другому
— до \sfrac{1}{4} продуктивности першого вкладення.

\begin{table}[H]
  \centering
  \caption*{Таблиця IX}

  \footnotesize
  \setlength{\tabcolsep}{4.5pt}
  \settowidth\rotheadsize{\theadfont Продажна}

  \begin{tabular}{l c r c c r c c c c c}
    \toprule
      \thead[tl]{Рід\\землі} &
      &
      \thead[t]{Капітал} &
      \rothead{Зиск} &
      \rothead{Ціна\\продукції} &
      \thead[t]{Продукт} & % \\ в кварт.}}}
      \rothead{Продажна\\ціна} &
      \rothead{Здобуток} &
      \multicolumn{2}{c}{Рента} &
      \rothead{Норма\\надзиску} \\

    \cmidrule(rl){2-11}
      & акри  & \poundsign{} & \poundsign{} & \poundsign{} & кв. & \poundsign{} & \poundsign{} & кв. & \poundsign{} & \% \\

    \midrule
      A & 1 & 2\tbfrac{1}{2} \dplus{} 2\tbfrac{1}{2} \deq{} 5 & 1 & 6 & 1 \dplus{} \phantom{0}\tbfrac{1}{2} \deq{} 1\tbfrac{1}{2}                                 & 4 & \phantom{0}6 & 0\phantom{\tbfrac{1}{2}} & \phantom{0}0 & \phantom{00}0 \\
      B & 1 & 2\tbfrac{1}{2} \dplus{} 2\tbfrac{1}{2} \deq{} 5 & 1 & 6 & 2 \dplus{} 1\phantom{\tbfrac{1}{2}} \deq{} 3\phantom{\tbfrac{1}{2}}                       & 4 & 12           & 1\tbfrac{1}{2}           & \phantom{0}6 & 120 \\
      C & 1 & 2\tbfrac{1}{2} \dplus{} 2\tbfrac{1}{2} \deq{} 5 & 1 & 6 & 3 \dplus{} 1\tbfrac{1}{2} \deq{} 4\tbfrac{1}{2}                                           & 4 & 18           & 3\phantom{\tbfrac{1}{2}} & 12           & 240\\
      D & 1 & 2\tbfrac{1}{2} \dplus{} 2\tbfrac{1}{2} \deq{} 5 & 1 & 6 & 4 \dplus{} 2\phantom{\tbfrac{1}{2}} \deq{} 6\phantom{\tbfrac{1}{2}} & 4 & 24           & 4\tbfrac{1}{2}           & 18           & 360\\

     \midrule
        Разом & & 20 & & & 15\pF{} & & 60 & 9\pF{} & 36 & 240\\
  \end{tabular}
\end{table}
\vspace{-\bigskipamount}

\noindent{}Таблиця IX та сама, що й таблиця VIII, тільки в таблиці VIII зменшення
продуктивности припадає на перше, в таблиці IX — на друге капіталовкладення.
