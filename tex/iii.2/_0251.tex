\parcont{}  %% абзац починається на попередній сторінці
\index{iii2}{0251}  %% посилання на сторінку оригінального видання
геть усе споживання одержувачів цих доходів, товари, які, крім цих трьох складових
частин вартости, мають у собі ще одну лишню складову частину вартости,
а саме сталий капітал? Яким чином можуть вони на вартість, що має в
собі три складові частини, купити таку, що має в собі чотири складові частини?\footnote{
Прудон виявляє цілковиту нездібність зрозуміти це в своїй обмеженій формулі: l’ouvrier ne peut
pas racheter son propre produit (робітник не може викупити свого власного продукту), бо в продукті
міститься процент, який додається до prix-de-revient (покупної ціни). Але як напоумляє його кращому
п. Eugène Forcade? «Коли б заперечення Прудона було слушне, воно стосувалося б не тільки profits du
capital (зиск з капіталу), але знищило б саму можливість промисловости. Коли робітник мусить платити
100 за річ, за яку він одержав лише 80, коли заробітна плата може викупити в продукті лише вартість,
вкладену в нього нею самою, то це рівнозначно твердженню, що робітник не може нічого викупити, що
заробітна плата нічого не може оплатити. Справді, в покупній ціні завжди є щось більше, ніж
заробітна плата робітника, а в продажній ціні щось більше, ніж зиск підприємця, наприклад, ціна
сирового матеріялу, часто виплачена закордон\dots{} Прудон забув про безупинний зріст національного
капіталу; він забув, що цей зріст відбувається для всіх робітників, для робітників підприємств так
само, як для ремесників» (Revue des deux Mondes, 1848, t. 24, p. 998). Mu маємо тут перед собою
оптимізм буржуазного безглуздя в найвідповіднішій йому формі глибокодумности. Поперше, п. Forcade
вважає, що робітник не міг би жити, коли б не одержував, крім вартости, яку продукує, ще вищої
вартости, тимчасом як, навпаки, капіталістичний спосіб продукції був би неможливий, коли б робітник
дійсно
одержував вартість, яву він продукує. Подруге, він правильно узагальнює трудність, висловлену
Прудоном лише з певного обмеженого погляду. Ціна товару має в собі надмір не тільки над заробітною
платою, але також і над зиском, а саме сталу частину вартости. Таким чином і капіталіст, згідно з
міркуванням Прудона не міг би викупити товари на свій зиск. Як же розв’язує Forcade загадку?
Безглуздою фразою про зріст капіталу. Отже, постійний зріст капіталу повинен між іншим виявлятися і
в тому, що аналізи товарової ціни, неможлива для економіста при капіталі в 100, стає зайвою при
капіталі в \num{10.000}. Що сказали б ми про хеміка, який на запитання: чим пояснюється, що в
хліборобському продукті міститься більше вуглецю, ніж у самому ґрунті, — відповів би: це пояснюється
постійним зростом хліборобської продукції. Добромисне бажання розкрити в буржуазному світі найкращий
з можливих світів заміняє у вульґарній економії усяку доконечність любови до істини і прагнення до
наукового дослідження.
}.

Ми дали аналізу в книзі II, відділ III.

2) Нерозуміння способу, в який праця, долучаючи нову вартість, зберігає
стару вартість у новій формі, не продукуючи цієї останньої вартости наново.

3) Нерозуміння загального зв’язку процесу репродукції, розглядуваного не
з погляду поодинокого капіталу, а з погляду сукупного капіталу; нерозуміння
труднощів, які є в тому, яким чином продукт, що в ньому реалізується заробітна
плата і додаткова вартість, отже, вся вартість, створена всією новодолученою
протягом року працею, може покривати свою сталу частину вартости, і ще одночасно
зводитися до вартости, що обмежена самими лише доходами; яким чином,
далі, зужиткований у продукції сталий капітал може бути речево і за
вартістю покритий новим, хоч загальна сума новодолученої праці реалізується
лише в заробітній платі і додатковій вартості, і вичерпно визначається в сумі
вартости обох. Саме в цьому і є головна трудність, в аналізі репродукції і в відношенні
її різних складових частин, з боку так їхнього речового характеру, як
і відношень їхньої вартости.

4) Але сюди приєднуються дальші труднощі, які ще збільшуються, скоро
різні складові частини додаткової вартости виступають у формі самостійних один
проти одного доходів. Труднощі ці є в тому, що тверді призначення доходу і
капіталу взаємно міняються, міняють своє місце, так що здаються лише відносними
призначеннями з погляду поодинокого капіталіста, які, як здається, зникають,
скоро ми подивимось на сукупний процес продукції. Наприклад, дохід
робітників і капіталістів кляси І, яка продукує сталий капітал, покриває вартість,
і речовину сталого капіталу кляси капіталістів II, яка продукує засоби споживання.
Отже, можна поминути труднощі з допомогою того уявлення, що те, що
для одного — дохід, для другого — капітал, а тому ці визначення не мають жодного
чинення до дійсного відокремлення складових частин вартости товару. Далі:
товари, призначені кінець-кінцем до того, щоб правити за речові елементи, на
\parbreak{}  %% абзац продовжується на наступній сторінці
