Другий випадок за низхідної ціни продукції.

Варіянт 1: за незмінної продуктивности другої витрати капіталу: земля
$А$ випадає з конкуренції, земля $В$ стає землею, що не дає ренти.

\begin{table}[H]
  \centering
  \footnotesize
  \caption*{Таблиця XVI}

  \begin{tabular}{lcccccc}
    \toprule
      \thead[tl]{Рід\\землі} &
      Ціна продукції &
      Продукт &
      \thead[t]{Продажна\\ціна} &
      \thead[t]{Здо-\\буток} &
      Рента &
      \thead[t]{Підвищення\\ренти} \\

    \cmidrule(r){2-6}
      & \shil{Шил.} & бушелі & \shil{Шил.} & \shil{Шил.} & \shil{Шил.} & \\

    \midrule
      B & 60 \dplus{} 60 \deq{} 120 & 12 \dplus{} 12 \deq{} 24 & 5 & 120  & \phantom{00}0 & \phantom{01 × }0 \\
      C & 60 \dplus{} 60 \deq{} 120 & 14 \dplus{} 14 \deq{} 28 & 5 & 140  & \phantom{0}20 & \phantom{1 ×} 20 \\
      D & 60 \dplus{} 60 \deq{} 120 & 16 \dplus{} 16 \deq{} 32 & 5 & 160  & \phantom{0}40 & 2 × 20 \\
      E & 60 \dplus{} 60 \deq{} 120 & 18 \dplus{} 18 \deq{} 36 & 5 & 180  & \phantom{0}60 & 3 × 20 \\

    \cmidrule(r){6-7}
      & & & & & 120 & 6 × 20 \\
  \end{tabular}
\end{table}

Варіант 2: за низхідної продуктивности другої витрати капіталу; земля
$А$ випадає з конкуренції, земля $В$ стає землею, що не дає ренти.

\begin{table}[H]
  \centering
  \footnotesize
  \caption*{Таблиця XVII}

  \begin{tabular}{lcccccc}
    \toprule
      \thead[tl]{Рід\\землі} &
      Ціна продукції &
      Продукт &
      \thead[t]{Продажна\\ціна} &
      \thead[t]{Здо-\\буток} &
      Рента &
      \thead[t]{Підвищення\\ренти} \\

    \cmidrule(r){2-6}
      & \shil{Шил.} & бушелі & \shil{Шил.} & \shil{Шил.} & \shil{Шил.} & \\

    \midrule
      B & 60 \dplus{} 60 \deq{} 120 & 12 \dplus{} \phantom{0}9\phantom{\tbfrac{1}{2}} \deq{} 21\phantom{\tbfrac{1}{2}} & 5\tbfrac{5}{7} & 120  & \phantom{00}0 & \phantom{01 × }0 \\
      C & 60 \dplus{} 60 \deq{} 120 & 14 \dplus{} 10\tbfrac{1}{2} \deq{} 24\tbfrac{1}{2}                               & 5\tbfrac{5}{7} & 140  & \phantom{0}20 & \phantom{1 ×} 20 \\
      D & 60 \dplus{} 60 \deq{} 120 & 16 \dplus{} 12\phantom{\tbfrac{1}{2}} \deq{} 28\phantom{\tbfrac{1}{2}}           & 5\tbfrac{5}{7} & 160  & \phantom{0}40 & 2 × 20 \\
      E & 60 \dplus{} 60 \deq{} 120 & 18 \dplus{} 13\tbfrac{1}{2} \deq{} 31\tbfrac{1}{2}                               & 5\tbfrac{5}{7} & 180  & \phantom{0}60 & 3 × 20 \\

    \cmidrule(r){6-7}
      & & & & & 120 & 6 × 20 \\
  \end{tabular}
\end{table}

Варіант 3: за висхідної продуктивности другої витрати капіталу; земля
$А$ залишається конкурентною. Земля $В$ дає ренту.

\begin{table}[H]
  \centering
  \footnotesize
  \caption*{Таблиця XVIII}

  \begin{tabular}{lcccccc}
    \toprule
      \thead[tl]{Рід\\землі} &
      Ціна продукції &
      Продукт &
      \thead[t]{Продажна\\ціна} &
      \thead[t]{Здо-\\буток} &
      Рента &
      \thead[t]{Підвищення\\ренти} \\

    \cmidrule(r){2-6}
      & \shil{Шил.} & бушелі & \shil{Шил.} & \shil{Шил.} & \shil{Шил.} & \\

    \midrule
      A & 60 \dplus{} 60 \deq{} 120 & 10 \dplus{} 15 \deq{} 25 & 4\tbfrac{4}{5} & 120  & \phantom{00}0 & \phantom{00 × 0}0 \\
      B & 60 \dplus{} 60 \deq{} 120 & 12 \dplus{} 18 \deq{} 30 & 4\tbfrac{4}{5} & 144  & \phantom{0}24 & \phantom{01 × }24 \\
      C & 60 \dplus{} 60 \deq{} 120 & 14 \dplus{} 21 \deq{} 35 & 4\tbfrac{4}{5} & 168  & \phantom{0}48 & \phantom{0}2 × 24 \\
      D & 60 \dplus{} 60 \deq{} 120 & 16 \dplus{} 24 \deq{} 40 & 4\tbfrac{4}{5} & 192  & \phantom{0}72 & \phantom{0}3 × 24 \\
      E & 60 \dplus{} 60 \deq{} 120 & 18 \dplus{} 27 \deq{} 45 & 4\tbfrac{4}{5} & 216  & \phantom{0}96 & \phantom{0}4 × 24 \\

    \cmidrule(r){6-7}
      & & & & & 240 & 10 × 24 \\
  \end{tabular}
\end{table}
