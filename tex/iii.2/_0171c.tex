
\index{iii2}{0171}  %% посилання на сторінку оригінального видання
Другий випадок за низхідної ціни продукції.

Варіянт 1: за незмінної продуктивности другої витрати капіталу: земля
$А$ випадає з конкуренції, земля $В$ стає землею, що не дає ренти.

\begin{table}[H]
  \begin{center}
    \emph{Таблиця XVI}
    \footnotesize

  \begin{tabular}{c@{  } c@{  } c@{  } c@{  } c@{  } c@{  } c}
    \toprule
      \multirowcell{2}{\makecell{Рід\\ землі}} &
      Ціна продукції &
      Продукт &
      \makecell{Продажна \\ ціна} &
      \makecell{Здо-\\буток} &
      Рента &
      \multirowcell{2}{Підвищення ренти} \\

      \cmidrule(r){2-2}
      \cmidrule(r){3-3}
      \cmidrule(r){4-4}
      \cmidrule(r){5-5}
      \cmidrule(r){6-6}

       & Шил. & Бушелі & Шил. & Шил. & Шил. &  \\
      \midrule
      B & 60 + 60 = 120 & 12 + 12 = 24 & 5 & 120  & \phantom{00}0 & \phantom{01 × }0 \\
      C & 60 + 60 = 120 & 14 + 14 = 28 & 5 & 140  & \phantom{0}20 & \phantom{1 ×} 20 \\
      D & 60 + 60 = 120 & 16 + 16 = 32 & 5 & 160  & \phantom{0}40 & 2 × 20 \\
      E & 60 + 60 = 120 & 18 + 18 = 36 & 5 & 180  & \phantom{0}60 & 3 × 20 \\

     \cmidrule(r){6-6}
     \cmidrule(r){7-7}

      & & & & & 120 & 6 × 20 \\
  \end{tabular}

  \end{center}
\end{table}

Варіант 2: за низхідної продуктивности другої витрати капіталу; земля
$А$ випадає з конкуренції, земля $В$ стає землею, що не дає ренти.

\begin{table}[H]
  \begin{center}
    \emph{Таблиця XVII}
    \footnotesize

  \begin{tabular}{c@{  } c@{  } c@{  } c@{  } c@{  } c@{  } c}
    \toprule
      \multirowcell{2}{\makecell{Рід\\ землі}} &
      Ціна продукції &
      Продукт &
      \makecell{Продажна \\ ціна} &
      \makecell{Здо-\\буток} &
      Рента &
      \multirowcell{2}{Підвищення ренти} \\

      \cmidrule(r){2-2}
      \cmidrule(r){3-3}
      \cmidrule(r){4-4}
      \cmidrule(r){5-5}
      \cmidrule(r){6-6}

       & Шил. & Бушелі & Шил. & Шил. & Шил. &  \\
      \midrule
      B & 60 + 60 = 120 & 12 + \phantom{0}9\phantom{\sfrac{1}{2}} = 21\phantom{\sfrac{1}{2}} & 5\sfrac{5}{7} & 120  & \phantom{00}0 & \phantom{01 × }0 \\
      C & 60 + 60 = 120 & 14 + 10\sfrac{1}{2} = 24\sfrac{1}{2}                               & 5\sfrac{5}{7} & 140  & \phantom{0}20 & \phantom{1 ×} 20 \\
      D & 60 + 60 = 120 & 16 + 12\phantom{\sfrac{1}{2}} = 28\phantom{\sfrac{1}{2}}           & 5\sfrac{5}{7} & 160  & \phantom{0}40 & 2 × 20 \\
      E & 60 + 60 = 120 & 18 + 13\sfrac{1}{2} = 31\sfrac{1}{2}                               & 5\sfrac{5}{7} & 180  & \phantom{0}60 & 3 × 20 \\

     \cmidrule(r){6-6}
     \cmidrule(r){7-7}

      & & & & & 120 & 6 × 20 \\
  \end{tabular}

  \end{center}
\end{table}

Варіант 3: за висхідної продуктивности другої витрати капіталу; земля
$А$ залишається конкурентною. Земля $В$ дає ренту.

\begin{table}[H]
  \begin{center}
    \emph{Таблиця XVIII}
    \footnotesize

  \begin{tabular}{c@{  } c@{  } c@{  } c@{  } c@{  } c@{  } c}
    \toprule
      \multirowcell{2}{\makecell{Рід\\ землі}} &
      Ціна продукції &
      Продукт &
      \makecell{Продажна \\ ціна} &
      \makecell{Здо-\\буток} &
      Рента &
      \multirowcell{2}{Підвищення ренти} \\

      \cmidrule(r){2-2}
      \cmidrule(r){3-3}
      \cmidrule(r){4-4}
      \cmidrule(r){5-5}
      \cmidrule(r){6-6}

       & Шил. & Бушелі & Шил. & Шил. & Шил. &   \\
      \midrule
      A & 60 + 60 = 120 & 10 + 15 = 25 & 4\sfrac{4}{5} & 120  & \phantom{00}0 & \phantom{00 × 0}0 \\
      B & 60 + 60 = 120 & 12 + 18 = 30 & 4\sfrac{4}{5} & 144  & \phantom{0}24 & \phantom{01 × }24 \\
      C & 60 + 60 = 120 & 14 + 21 = 35 & 4\sfrac{4}{5} & 168  & \phantom{0}48 & \phantom{0}2 × 24 \\
      D & 60 + 60 = 120 & 16 + 24 = 40 & 4\sfrac{4}{5} & 192  & \phantom{0}72 & \phantom{0}3 × 24 \\
      E & 60 + 60 = 120 & 18 + 27 = 45 & 4\sfrac{4}{5} & 216  & \phantom{0}96 & \phantom{0}4 × 24 \\

     \cmidrule(r){6-6}
     \cmidrule(r){7-7}

      & & & & & 240 & 10 × 24 \\
  \end{tabular}

  \end{center}
\end{table}
