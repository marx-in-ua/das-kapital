\parcont{}  %% абзац починається на попередній сторінці
\index{iii2}{0053}  %% посилання на сторінку оригінального видання
обмін векселів, акцептованих великими фірмами, на гроші; подбати про те,
щоб Англійський банк серед усяких обставин редисконтував ті векселі bill-brokers’ам.
І це тоді, коли в 1857 році троє таких bill-brokers’ів збанкрутували
приблизно на 8 мільйонів, маючи проти цих боргів мізерний власний капітал. —
«5177. Чи хочете ви цим сказати, що, на вашу думку, їх (акцепти Baring'a або
Loyd’a) малося б обов’язково дисконтувати, от як тепер банкноти Англійського
банку мусять обов’язково заміняти на золото? — Я тієї думки, що це була б
дуже прикра справа, коли б їх не сила було дисконтувати; це було б незвичайне
становище, коли б хтось мусів припинити платежі, тому, що він, маючи акцепти
Smith, Payne and C° або Jones, Loyd and C° не має змоги їх дисконтувати. — 5178.
Хіба акцепт Baring’a не є зобов’язання заплатити певну суму грошей, коли тому
векселеві надійде реченець платежа? — Де цілком слушно; але панове Baring’и.
коли вони беруть на себе таке зобов’язання, як і кожен купець, що бере таке
зобов’язання, і в сні не думають, що їм доведеться те зобов’язання оплатити
соверенами; вони сподіваються, що вони його оплатять в розрахунковій палаті. —
5180. Чи думаєте ви, що треба вигадати механізм такого роду, щоб за його
допомогою публіка мала право одержувати гроші за вексель перед реченцем його
оплати таким способом, щоб хтось інший мусів би його дисконтувати? —
Ні, цього не мусів би робити акцептант; але, коли ви маєте тут на думці, що
нам не повинно давати змогу дисконтувати торговельні векселі, то мусимо ми змінити
ввесь лад речей. — 5182. Отже, ви думаєте, що він [торговельний вексель] мусить
мати змогу перетворюватись на гроші так само, як банкнота Англійського банку
мусить мати змогу обмінюватись на золото? — Цілком слушно, в певних умовах
5184. Отже, ви тієї думки, що установи currency треба так зорганізувати, щоб торговельний
безперечної солідности вексель можна було повсякчас так само легко
обмінювати на гроші, як і банкноту? — Я тієї думки. — 5185. Ви не йдете так
далеко, щоб сказати, що Англійський банк, або когось іншого треба постановою
закону примусити міняти вексель на гроші? — Звичайно, я йду досить далеко,
щоб сказати, що, складаючи закон до реґулювання currency, ми маємо вжити
заходів проти можливости такого стану речей, коли внутрішніх, безперечно
солідних та законно складених, торговельних векселів не можна перетворити
на гроші». — Де є розмін торговельного векселя поряд розміну банкноти.

«5189. Торговці грішми країни в дійсності репрезентують тільки публіку»
— як сказав пан Chapman пізніше перед асизами в справі Davison’a.
Дивись Great City Frauds.

«5196. Щочверть року» [коли виплачують дивідендні «нам\dots{} абсолютно
потрібно вдаватись до Англійського банку. Коли ви заберете з циркуляції 6 або
7 мільйонів державних доходів, маючи платити дивіденди, то мусить бути хтось,
хто дав би цю суму до розпорядку на проміжний час». — [Отже, в цьому випадку
мовиться про подання грошей, а не капіталу або позичкового капіталу].

«5169. Кожен, хто знає наш торговельний світ, мусить знати, що, коли
ми опиняємося в такому стані, що посвідок державної скарбниці не сила продати,
що облігації ост-індської компанії є цілком ні до чого, що найкращих
торговельних векселів не можна дисконтувати, — кожен мусить знати, що за таких
часів мусить панувати великий неспокій серед тих, кого справи доводять до
такого становища, коли вони на просту вимогу мусять даної хвилини робити
платежі звичайними в країні засобами циркуляції, а це трапляється з усіма
банкірами. А наслідок цього той, що кожен подвоює свої запаси. Тепер уявіть
собі, як це вплине на цілу країну, коли кожен провінціяльний банкір — а їх є
приблизно 500 — дасть своєму лондонському кореспондентові доручення переказати
йому 5000\pound{ ф. ст.} банкнотами. Навіть, коли ми візьмемо за пересічну
отаку малу суму — а це вже цілком абсурдна річ, — матимемо ми 2\sfrac{1}{2} міл. ф. ст.,
що їх треба забрати з циркуляції. Чим можна їх замінити?».
