\parcont{}  %% абзац починається на попередній сторінці
\index{iii2}{0219}  %% посилання на сторінку оригінального видання
вищий ступінь розвитку його праці і суспільства взагалі; і відрізняється вона
від попередньої форми тим, що додаткову працю доводиться виконувати вже
не в її натуральному вигляді, а тому вже не під безпосереднім наглядом і
примусом земельного власника, або його представника; навпаки, безпосередній
продуцент повинен виконувати її на свою власну відповідальність, примушуваний
силою відносин замість безпосереднього примусу, і постановою закону замість
нагая. Додаткова продукція, в розумінні продукції понад доконечні потреби
безпосереднього продуцента, і продукція на фактично йому самому належному
полі продукції, ним самим експлуатованій землі, замість продукції в панському
маєтку біля і поза своїм, як було давніш, стали тут уже само собою зрозумілим
правилом. При цих відносинах безпосередній продуцент більш або менш
порядкує застосуванням усього свого робочого часу, хоч частина цього робочого
часу, первісно майже вся надмірна частина його, як і давніш даром належить
земельному власникові; ріжниця тільки в тому, що останній уже не
одержує його безпосередньо в його власній і натуральній формі, а одержує в
натуральній формі того продукту, в якому цей час реалізується. Обтяжливі і
залежно від реґулювання панщинної праці більш або менш перешкідні перерви,
зумовлювані працею на земельного власника (порівн. книга перша, розд. VIII, 2,
фабрикант і маґнат) відпадають, коли рента продуктами є в чистому вигляді
або зводиться, принаймні, до нечисленних коротких перерв протягом року,
коли поряд з рентою продуктами й далі тривають певні панщини. Праця продуцента
на самого себе і його праця на земельного власника обмацально
вже більше не відокремлюються в часі і просторі. Ця рента продуктами в її
чистому вигляді, хоч її уламки можуть доходити до розвиненіших способів
продукції і продукційних відносин, як і давніш, має своєю передумовою натуральне
господарство, тобто припускає, що умови господарювання цілком або в
переважній частині продукуються в самому господарстві, покриваються і репродукуються
безпосередньо з його гуртового продукту. Далі, вона має своєю передумовою
сполучення сільської домашньої промисловости з хліборобством; додатковий
продукт, що створює ренту, є продукт цієї об’єднаної хліборобсько-промислової
родинної праці, однаково, чи має в собі рента продуктами в більшій або
меншій мірі промислові продукти, як це часто було за середньовіччя, чи вона
виплачується лише в формі власне хліборобського продукту. При цій формі
ренти, рента продуктами, що в ній втілюється додаткова праця, ніяк не потребує
того, щоб вичерпувалось всю надмірну працю сільської родини. Навпаки, продуцентові
дається тут, порівняно з відробітною рентою, більшу волю для того, щоб
здобути час для надмірної праці, продукт якої належить йому самому, цілком так
само, як продукт його праці, що задовольняє його доконечні потреби. Так само
разом з цією формою постають більші ріжниці в економічному становищі окремих
безпосередніх продуцентів. Принаймні, є можливість для цього, а також та
можливість, що цей безпосередній продуцент здобуде засоби для того, щоб і
собі безпосередньо визискувати чужу працю. Проте, тут, де ми розглядаємо
чисту форму ренти продуктами, це нас не стосується; як і взагалі ми не можемо
розглядати безконечно різних комбінацій, в яких різні форми ренти
можуть сполучатися, фалшуватися і з’єднуватися. Через те, що ця форма
ренти, рента продуктами, зв’язана з певним характером продукту і самої продукції,
через доконечне для неї сполучення сільського господарства і домашньої
промисловости, через те, що з нею сільська родина набуває майже цілком
самодостатьного характеру, через її незалежність від ринку, від продукційного
і історичного руху частини суспільства, що стоїть поза нею, коротко кажучи,
через характер натурального господарства взагалі, ця форма цілком придатна
для того, щоб бути за базу застійних станів суспільства, як це ми спостерігаємо,
наприклад, в Азії. Тут, як і при найдавнішій формі відробітної ренти,
\parbreak{}  %% абзац продовжується на наступній сторінці
