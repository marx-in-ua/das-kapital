\parcont{}  %% абзац починається на попередній сторінці
\index{iii2}{0226}  %% посилання на сторінку оригінального видання
що рента існує, незалежно від будь-яких ріжниць у родючості і положенні
землі, — якраз тут, у пересічному доводиться припускати, що абсолютної ренти
не існує, і що, отже, найгірша земля не дає жодної ренти; бо абсолютна рента
має своєю передумовою або реалізований надмір вартости продукту над його ціною
продукції, або надмірну монопольну ціну, що перевищує вартість продукту. А що
сільське господарство провадиться тут переважно як хліборобство за-для безпосередніх
засобів існування, і що земля становить для більшости людности
конче потрібне поле для застосування її праці і капіталу, то регуляційна ринкова
ціна продукту лише за виключних обставин досягне розмірів його вартости;
а вартість ця, в наслідок переваги елементу живої праці, буде, взагалі кажучи,
стояти вище, ніж ціна продукції, хоч цей надмір вартости над ціною продукції
в свою чергу буде обмежуватись тим, що в країнах парцелярного господарства
і нехліборобський капітал має низький склад. За межу експлуатації для парцелярного
селянина, з одного боку, не є пересічний зиск на капітал, оскільки
сам він є дрібний капіталіст, ані доконечність ренти, з другого боку, оскільки
він сам є земельний власник. За абсолютну межу для нього, як для дрібного
капіталіста, є лише заробітна плата, яку він виплачує сам собі по вирахуванні
власне витрат. Поки ціна продукту покриває заробітну плату для нього, він
оброблятиме свою землю, при чому він часто спускається до фізичного мінімуму
заробітної плати. Щодо нього як земельного власника, то для нього відпадає
межа, яку кладе власність, бо вона може виявитись лише в протилежність
відокремленому від неї капіталові (включаючи і працю), ставлячи перешкоду для
застосування його. Певна річ, що процент на ціну землі, який до того ж здебільша
доводиться виплачувати третій особі, гіпотечному кредиторові, становить
межу. Але цей процент може виплачуватися саме з тієї частини додаткової
праці, яка за капіталістичних відносин становила б зиск. Отже, рента, антиципована
в земельній ціні і в виплачуваному на неї проценті, не може бути чимось
іншим, як лише частиною капіталізованої додаткової праці селянина, понад
працею, потрібною для його існування, причому ця додаткова праця не реалізується
в частині вартости товару, рівній усьому пересічному зискові, і тим
паче вона не реалізується в надмірі над додатковою працею, в пересічному зиску
реалізованою, тобто в надзиску. Рента може становити вирахування з пересічного
зиску, або навіть однісіньку частину його, яка тільки й реалізується. Отже,
для того, щоб парцелярний селянин міг обробляти свою землю, або купити
землю для оброблення, немає потреби, як за нормального капіталістичного
способу продукції, в тому, щоб ринкова ціна хліборобського продукту піднеслась
так високо, щоб давати йому пересічний зиск, а тим паче надмір над цим
пересічним зиском, фіксований у формі ренти. Отже, немає потреби в тому,
щоб ринкова ціна підвищилась, або до рівня вартости, або до рівня ціни продукції
продукту селянина. Це є одна з причин, що пояснює, чому в країнах,
де панує парцелярна власність, ціна збіжжя стоїть нижче, ніж у країнах капіталістичного
способу продукції. Частину додаткової праці селян, що працюють
в найнейсприятливіших умовах, даром дається суспільству і не бере вона участи
в регулюванні цін продукції або в створенні вартости взагалі. Отже, ця порівняно
низька ціна є наслідок убозтва продуцентів, але ніяк не продуктивности
їхньої праці.

Ця форма вільної парцелярної власности селян, що сами господарюють, як
панівна, нормальна форма створює, з одного боку, економічну основу суспільства
за найкращих часів класичної давнини, а, з другого боку, ми подибуємо
її в сучасних народів як одну з форм, що постають з розпаду февдально,
земельної власности. Такі уeomanry\footnote*{
Селяни-землевласники. \emph{Пр.~Ред.}
} в Англії, селянський стан у Швеції
\parbreak{}  %% абзац продовжується на наступній сторінці
