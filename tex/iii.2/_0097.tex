\parcont{}  %% абзац починається на попередній сторінці
\index{iii2}{0097}  %% посилання на сторінку оригінального видання
Банк той не робив блискучих справ. Механізм був занадто складний, а риск
підчас знецінення товарів занадто великий.

Спиняючись над дійсним змістом тих творів, що теоретично складалися
поряд формування новітньої банкової справи в Англії й тому формуванню
сприяли, ми не знайдемо в ньому нічого, крім домагання підпорядкувати
капіталістичному способові продукції, як одну з його умов, капітал, що дає процент,
і взагалі засоби продукції, що їх можна визичати. Коли ж спинитися над
самою тільки фразеологією, то схожість її аж до висловів з банковими та кредитовими
ілюзіями сен-сімоністів часто може здивувати нас.

Цілком так, як у фізіократів «cultivateur» означає не дійсного хлібороба-селянииа,
а великого орендаря, у Сен Сімона та іноді — в його учнів «travailleur»
(працівник) означає не робітника, а промислового та торговельного капіталіста.
«Un travailleur a besoin d’aides, de seconds, \emph{d'ouvriers}; il les cherche
intelligents, habiles, dévoués,; il les met à l’oeuvre, et leurs travaux sont productifs»\footnote*{
«Travailleur має потребу в помічниках, заступниках, \emph{робітниках}; він шукає людей розумних,
вправних, відданих; він ставить їх на працю й їхня праця продуктивна» Пр.~Ред.
}
(Religion saint-simonienne. Économie politique et Politique. Paris 1831 p, 104). Взагалі
не треба забувати, що тільки в своєму останньому творові, «Nouveau Christianisme»,
Сен-Сімон виступає безпосередньо як оборонець робітничої кляси та
проголошує її емансипацію за кінцеву мету своїх намагань. Всі його попередні
твори в дійсності є лише вихваляння новітнього буржуазного суспільства проти
февдального, або вихваляння промисловців та банкірів проти маршалів та юристів,
що фабрикували закони за наполеонових часів. Яка ріжниця, коли порівняти ці його
твори з одночасними творами Оуена!\footnote{
Коли б Маркс мав змогу обробити манускрипт, він, безумовно, значно змінив би це місце.
Воно інспіроване ролею ех-сен-сімоністів за другої імперії у Франції, де саме тоді, коли Маркс
писав вищенаведене, кредитові фантазії цієї школи, які мали визволити світ, силою історичної
іронії реалізувалися в формі спекуляції нечуваної досі сили. Пізніше Маркс говорив лише з
здивованням
про геній та енциклопедичну голову Сен-Сімона. Якщо останній в своїх раніших творах ігнорував
протилежність між буржуазією та пролетаріатом, що саме вперше тоді народжувався у Франції,
якщо він залічував частину буржуазії, зайнятої в продукції, до travailleurs, то це відповідає
поглядові
Фур’є, що хотів помирити капітал та працю, і пояснюється економічним і політичним станом тодішньої
Франції. Коли Оуен в цьому мав ширший погляд, то тому, що жив в іншому оточенні, серед промислової
революції та класових суперечностей, уже дуже загострених. — Ф.~Е.
} І в його наступників, як уже показує оте
цитоване місце, промисловий капіталіст лишається travailleur par excellence. Читаючи
критично їхні твори, ми не дивуватимемося тому, що реалізацією їхніх кредитових\footnote*{
В нім. тексті тут стоїть очевидно помилково «критичних». Пр.~Ред.
}
та банкових мрій був crédit mobilier, заснований ех-сен-сімоністом Емілем
Перейром, форма, що проте могла стати панівною тільки в такій країні, як
Франція, де ані кредитова система, ані велика промисловість не розвинулись
до новітньої висоти. В Англії та Америці щось подібне було б неможливе. —
В дальших місцях «Doctrine de St.~Simon. Exposition. Première année. 1828--29,
3-е éd. Paris 1831» міститься вже зародок crédit mobilier. Звичайно,
банкір може давати позики дешевше, ніж капіталіст та приватний лихвар.
Отже, ці банкіри «можуть постачати промисловцям знаряддя далеко дешевше,
тобто \emph{за нижчі проценти}, ніж то могли б зробити земельні власники
та капіталісти, що легше можуть помилитися, вибираючи позикоємців».
(р. 202). Але самі автори долучають в примітці: «Користь, що мусила б поставати
з посередництва банкіра між неробами та travailleurs (працівниками),
часто зрівноважується й навіть нищиться тому, що наше дезорганізоване суспільство
дає нагоду егоїзмові виявляти себе в різних формах шахрайства та
шарлатанства; часто банкіри втручаються між travailleurs і нероб, щоб визискувати
і тих і цих на шкоду суспільству». Travailleur стоїть тут замість
\parbreak{}  %% абзац продовжується на наступній сторінці
