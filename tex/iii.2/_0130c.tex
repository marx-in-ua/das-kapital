
\index{iii2}{0130}  %% посилання на сторінку оригінального видання
Перше і головне припущення є, що поліпшення в хліборобстві нерівномірно
впливає на землі різних родів, і тут воно більше впливає на кращі землі
$C$ і $D$, ніж на $А$ і $В$. Досвід довів, що, звичайно, справа так і стоїть, хоч може
статись і зворотне. Коли б поліпшення більше впливало на гірші землі, ніж на
кращі, то рента з останніх понизилася б замість підвищитись. — Але з абсолютним
зростом родючости всіх родів землі у таблиці одночасно припускається зріст
вищої відносної родючості кращих родів землі $C$ і $D$, а тому зріст ріжниці в продукті за однакової
величини застосованого капіталу, а тому і зріст диференційної ренти.
Друге припущення є в тому, що з зростанням всього продукту відповідно зростає і загальна потреба в
ньому. \emph{Поперше}, не слід уявляти собі це зростання раптовим; воно відбувається поступово, доти, доки
не встановиться ряд III. \emph{Подруге}, невірно, нібито споживання потрібних засобів існування не зростає разом з їхнім
здешевленням. Скасування хлібних законів в Англії (дивись Newman) довело зворотне, і протилежне
уявлення постало лише тому, що великі і раптові ріжниці в урожаях, які пояснюються тільки погодою,
спричиняють то неспіврозмірне пониження, то неспіврозмірне підвищення цін збіжжя.
Коли в цьому разі раптове і скороминуще здешевлення не встигає справити повного впливу на поширення
споживання, то зворотне явище спостерігається в тому випадку, коли здешевлення випливає із зменшення
самої регуляційної ціни продукції, отже, коли воно має тривалий характер. \emph{Потретє}, частина збіжжя
може бути спожита у вигляді горілки або пива. А зростаюуче споживання обох цих продуктів ніяк не
обмежено вузькими межами. \emph{Почетверте}, тут справа залежить почасти від приросту людности, почасти
від експорту збіжжя в тих країнах, що вивозять збіжжя — як от Англія, до і
пізніше половини XVIII століття, і де тому потребу реґулюється межами не самого
тільки національного споживання. \emph{Нарешті}, збільшення і здешевлення
продукції пшениці може мати своїм наслідком, що замість жита або вівса за
головний засіб харчування маси народу стане пшениця, так що вже в наслідок
самого цього ринок для неї зросте подібно до того, як при зменшенні кількості
продукту і збільшенні його ціни може постати зворотне явище. — При цих припущеннях, отже, і при
взятих числових відношеннях, ряд III дає той наслідок,
що ціна знижується з 60 до 30 шил. за квартер, отже на 50\%; що продукція проти ряду І зростає з 10
до 23 квартерів, отже, на 130\%; що рента з землі $В$ лишається незмінною, рента з землі $C$
подвоюється, а з $D$ більше, ніж подвоюється, і що загальна сума ренти підвищується з 18 до 22\pound{ ф.
стерл.}, отже, на 22\sfrac{1}{9}\%.

З порівняння цих трьох таблиць (при чому ряд I треба брати подвійно:
у висхідному напрямку від $А$ до $D$ і в низхідному від $D$ до $А$), що їх можна
розглядати або як дані ступені хліборобства, за даного стану суспільства, наприклад,
один поряд одного в трьох різних країнах або як такі, що йдуть одна по одній в різних періодах
розвитку тієї самої країни, — з цього порівняння випливає:
