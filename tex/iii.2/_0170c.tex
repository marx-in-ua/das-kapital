
\index{iii2}{0170}  %% посилання на сторінку оригінального видання

\begin{table}[H]
  \begin{center}
    \emph{Таблиця XIII}
    \footnotesize

  \begin{tabular}{c@{  } c@{  } c@{  } c@{  } c@{  } c@{  } c}
    \toprule
      \multirowcell{2}{\makecell{Рід\\ землі}} &
      Ціна продукції &
      Продукт &
      \makecell{Продажна \\ ціна} &
      \makecell{Здо-\\буток} &
      Рента &
      \multirowcell{2}{Підвищення ренти} \\

      \cmidrule(r){2-2}
      \cmidrule(r){3-3}
      \cmidrule(r){4-4}
      \cmidrule(r){5-5}
      \cmidrule(r){6-6}

       & Шил. & Бушелі & Шил. & Шил. & Шил. & &   \\
      \midrule
      A & \phantom{60 \dplus{} 60 \deq{} 0}60 & \phantom{12 \dplus{} 10\sfrac{1}{3} \deq{}} 10\phantom{\sfrac{2}{3}}           & 6 & \phantom{0}60 & \phantom{00}0 & \phantom{0 × 0}0 \\
      B & 60 \dplus{} 60 \deq{} 120           & 12 \dplus{} \phantom{0}8\phantom{\sfrac{1}{3}} \deq{} 20\phantom{\sfrac{2}{3}} & 6 & 120           & \phantom{00}0 & \phantom{0 × 0}0 \\
      C & 60 \dplus{} 60 \deq{} 120           & 14 \dplus{} \phantom{0}9\sfrac{1}{3} \deq{} 23\sfrac{1}{3}                     & 6 & 140           & \phantom{0}20 & \phantom{1 × }20 \\
      D & 60 \dplus{} 60 \deq{} 120           & 16 \dplus{} 10\sfrac{2}{3} \deq{} 26\sfrac{2}{3}                               & 6 & 160           & \phantom{0}40 & 2 × 20 \\
      E & 60 \dplus{} 60 \deq{} 120           & 18 \dplus{} 12\footnotemarkZ{}\phantom{/}= 30\phantom{\sfrac{2}{3}}                  & 6 & 180           & \phantom{0}60 & 3 × 20 \\

     \cmidrule(r){6-6}
     \cmidrule(r){7-7}

      & & & & & 120 & 6 × 20 \\
  \end{tabular}

  \end{center}
\end{table}
\footnotetextZ{В німецькому тексті тут стоїть «20». Очевидна помилка. Прим. Ред.} % текст примітки прямо під заголовком

2) Кола земля $В$ не стає землею, що зовсім не дає ренти.

\begin{table}[H]
  \begin{center}
    \emph{Таблиця XIV}
    \footnotesize

  \begin{tabular}{c@{  } c@{  } c@{  } c@{  } c@{  } c@{  } c}
    \toprule
      \multirowcell{2}{\makecell{Рід\\ землі}} &
      Ціна продукції &
      Продукт &
      \makecell{Продажна \\ ціна} &
      \makecell{Здо-\\буток} &
      Рента &
      \multirowcell{2}{Підвищення ренти} \\

      \cmidrule(r){2-2}
      \cmidrule(r){3-3}
      \cmidrule(r){4-4}
      \cmidrule(r){5-5}
      \cmidrule(r){6-6}

       & Шил. & Бушелі & Шил. & Шил. & Шил. & &   \\
      \midrule
      A & \phantom{60 \dplus{} 60 \deq{} 0}60 & \phantom{12 \dplus{} 10\sfrac{1}{3} \deq{}} 10\phantom{\sfrac{2}{3}}           & 6 & \phantom{0}60 & \phantom{00}0 & \phantom{4 ×}0\phantom{ \dplus{} 3 × 21}\\
      B & 60 \dplus{} 60 \deq{} 120           & 12 \dplus{} \phantom{0}9\phantom{\sfrac{1}{3}} \deq{} 21\phantom{\sfrac{2}{3}} & 6 & 126           & \phantom{00}6 & \phantom{4 ×}6\phantom{ \dplus{} 3 × 21}\\
      C & 60 \dplus{} 60 \deq{} 120           & 14 \dplus{} 10\sfrac{1}{2} \deq{} 24\sfrac{1}{2}                               & 6 & 147           & \phantom{0}27 & \phantom{4 ×}6 \dplus{} 21\phantom{1 × } \\
      D & 60 \dplus{} 60 \deq{} 120           & 16 \dplus{} 12\phantom{\sfrac{2}{3}} \deq{} 28\phantom{\sfrac{2}{3}}           & 6 & 168           & \phantom{0}48 & \phantom{4 ×}6 \dplus{} 2 × 21 \\
      E & 60 \dplus{} 60 \deq{} 120           & 18 \dplus{} 13\sfrac{1}{2}= 31\sfrac{1}{2}                                & 6 & 189           & \phantom{0}69 & \phantom{4 ×}6 \dplus{} 3 × 21 \\

     \cmidrule(r){6-6}
     \cmidrule(r){7-7}

      & & & & & 150 & 4 × 6 \dplus{} 6 × 21 \\
  \end{tabular}

  \end{center}
\end{table}

Варіянт 3: за висхідної продуктивности другої витрати капіталу; на землі
$А$ тут теж не робиться другої витрати.

\begin{table}[H]
  \begin{center}
    \emph{Таблиця XV}
    \footnotesize

  \begin{tabular}{c@{  } c@{  } c@{  } c@{  } c@{  } c@{  } c}
    \toprule
      \multirowcell{2}{\makecell{Рід\\ землі}} &
      Ціна продукції &
      Продукт &
      \makecell{Продажна \\ ціна} &
      \makecell{Здо-\\буток} &
      Рента &
      \multirowcell{2}{Підвищення ренти} \\

      \cmidrule(r){2-2}
      \cmidrule(r){3-3}
      \cmidrule(r){4-4}
      \cmidrule(r){5-5}
      \cmidrule(r){6-6}

       & Шил. & Бушелі & Шил. & Шил. & Шил. & &   \\
      \midrule
      A & \phantom{60 \dplus{} 60 \deq{} 0}60 & \phantom{12 \dplus{} 10\sfrac{1}{3} \deq{}} 10\phantom{\sfrac{2}{3}}           & 6 & \phantom{0}60 & \phantom{00}0 & \phantom{4 ×0}0\phantom{ \dplus{} 3 × 27}\\
      B & 60 \dplus{} 60 \deq{} 120           & 12 \dplus{} 15\phantom{\sfrac{1}{3}} \deq{} 27\phantom{\sfrac{2}{3}}           & 6 & 162           & \phantom{0}42 & \phantom{4 ×}42\phantom{ \dplus{} 3 × 27}\\
      C & 60 \dplus{} 60 \deq{} 120           & 14 \dplus{} 17\sfrac{1}{2} \deq{} 31\sfrac{1}{2}                               & 6 & 189           & \phantom{0}69 & \phantom{4 ×}42 \dplus{} 27\phantom{1 × } \\
      D & 60 \dplus{} 60 \deq{} 120           & 16 \dplus{} 20\phantom{\sfrac{2}{3}} \deq{} 36\phantom{\sfrac{2}{3}}           & 6 & 216           & \phantom{0}96 & \phantom{4 ×}42 \dplus{} 2 × 27 \\
      E & 60 \dplus{} 60 \deq{} 120           & 18 \dplus{} 22\sfrac{1}{2}= 40\sfrac{1}{2}                                & 6 & 243           & 123           & \phantom{4 ×}42 \dplus{} 3 × 27 \\

     \cmidrule(r){6-6}
     \cmidrule(r){7-7}

      & & & & & 330 & 4 × 42 \dplus{} 6 × 27 \\
  \end{tabular}

  \end{center}
\end{table}
