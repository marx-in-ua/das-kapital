\parcont{}  %% абзац починається на попередній сторінці
\index{iii2}{0121}  %% посилання на сторінку оригінального видання
самим ціну продукції товару. Для промисловця справа стоїть так, що для нього
витрати продукції товару менші. Йому доводиться менше платити за зрічевлену
працю, а також менше платити заробітної плати за живу робочу силу, якої
у нього застосовується менше. А що витрати продукції його товару менші,
то й його індивідуальна ціна продукції менша. Витрати продукції становлять для
нього 90 замість 100. Отже, його індивідуальна ціна продукції була б замість
115 лише 103\sfrac{1}{2} (100: 115 \deq{} 90: 103\sfrac{1}{2)}. Ріжниця між його індивідуальною ціною
продукції і загальною обмежена ріжницею між його індивідуальними витратами
продукції і загальними. Це — одна з величин, що становлять межі його надпродукту\footnote{Терміни Surplusprodukt (надпродукт) і Mehrprodukt (додатковий продукт) Маркс взагалі вживає,
як тотожні. Про специфічне значення терміну «надпродукт» тут і далі, коли йдеться
про рентодайний капітал, див. кінець розд. 41. \emph{Прим. Ред.}}. Друга — це є величина загальної ціни продукції, в якій бере участь
загальна норма зиску, як один з регуляційних чинників. Коли б вугілля подешевшало,
то ріжниця між його індивідуальними і загальними витратами продукції
зменшилася б, а тому зменшився б і його надзиск. Коли б він мусив продавати
товар по його індивідуальній вартості, або по ціні продукції, визначуваній
його індивідуальною вартістю, то ріжниця відпала б. Вона є наслідок, з одного
боку, того, що товар продається по своїй загальній ринковій ціні, по ціні,
в яку конкуренція вирівнює індивідуальні ціни, а з другого боку — того, що
більша індівидуальна продуктивна сила праці, приведеної ним в рух, іде на користь
не робітникам, а як взагалі всяка продуктивна сила праці, тому, хто їх
застосовує; що вона виступає як продуктивна сила капіталу.

А що однією межею цього надзиску є висота загальної ціни продукції,
а висота загальної норми зиску є один з її чинників, то цей надзиск може
виникнути лише з ріжниці між загальною і індивідуальною ціною продукції,
отже, з ріжниці між індивідуальною і загальною нормою зиску. Надмір над цією
ріжницею має за свою передумову продаж продуктів дорожче, а не по ціні
продукції, регульованій ринком.

\emph{Подруге}. До цього часу надзиск фабриканта, що вживає як рушійну силу
природний водоспад замість пари, аж ніяк не відрізняється від усякого іншого
надзиску. Всякий нормальний надзиск, тобто такий, що виникає не від випадкових
операцій продажу або від коливань ринкової ціни, визначається ріжницею між
індивідуальною ціною продукції товарів цього окремого капіталу і загальною
ціною продукції, яка реґулює ринкові ціни товарів капіталу цієї сфери продукції
взагалі, або що реґулює ринкові ціни товарів усього капіталу, вкладеного в цю
сферу продукції.

Але звідси починається ріжниця.

Якій обставині завдячує фабрикант в даному разі своїм надзиском, тим надміром,
що його дає йому особисто ціна продукції, реґульована загальною нормою зиску?

Насамперед — природній силі, рушійній силі водоспаду, який є даний природою
і який сам не є продукт праці, а тому не має вартости, як от вугілля,
що перетворює воду в пару і яке само є продукт праці, тому має вартість,
та мусить бути оплачене еквівалентом, потребує витрат. Водоспад — такий природний
аґент продукції, що на створення його не треба жодної праці.

Але це не все. Фабрикант, що працює з паровою машиною, теж вживає
природні сили, які нічого не коштують йому, але роблять працю продуктивнішою
і — оскільки вони таким чином здешевлюють виготовлення засобів існування,
потрібних для робітників, — збільшують додаткову вартість, а тому і зиск; які,
отже, цілком так само монополізуються капіталом, як суспільні природні сили
праці, що постають з кооперації, поділу праці тощо. Фабрикант оплачує вугілля,
а не здібність води змінювати свій аґреґатний стан, переходити в пару, не
еластичність пари тощо. Ця монополізація сил природи, тобто спричиненого ними
\parbreak{}  %% абзац продовжується на наступній сторінці
