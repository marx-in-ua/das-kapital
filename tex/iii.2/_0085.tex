\index{ііі2}{0085}  %% посилання на сторінку оригінального видання
4)~Коли товарів припливає багато, то число продавців більшатиме проти числа покупців, та,
в міру того як кількість товарів переважає потреби безпосереднього спожитку,
доводиться чим раз більшу частину їх зберігати для пізнішого споживання. За цих
обставин державець товарів продаватиме їх на умовах пізнішої оплати або
в кредит на менш вигідних для себе умовах, ніж тоді, коли б він мав певність,
що весь його запас продасться протягом кількох тижнів.

До тези 1) треба зауважити, що сильний приплив благородного металу
може відбуватись одночасно з обмеженням продукції, як це завжди
буває безпосередньо по кризі. В дальшій фазі благородний метал може припливати
з тих країн, що переважно продукують благородний метал; довіз
інших товарів підчас цього періоду звичайно урівноважується вивозом. В цих
обох фазах рівень проценту низький та підноситься лише поволі; чому це так,
ми вже бачили. Цей низький рівень проценту в усіх випадках можна пояснити,
не вдаючися до абиякогось впливу абияких «великих запасів всякого роду».

І як має відбуватись такий вплив? Низька ціна, напр., бавовни, робить можливими
високий зиск прядунів і~\abbr{т. ін.} Чого ж рівень проценту низький? Звичайно,
не тому, що зиск, який можна здобути з допомогою узятого в позику
капіталу, є високий. Але тому й тільки тому, що серед наявних обставин попит
на позичковий капітал не зростає пропорційно ростові цього зиску; отже, позичковий
капітал має інший рух, ніж промисловий капітал. Що хоче довести
Economist, так це саме протилежне: що рух позичкового капіталу ідентичний
з рухом промислового капіталу.

Теза 2) — коли абсурдне припущення, що є запас на два роки наперед,
ми обмежимо до того, щоб воно мало якийсь глузд, — має собі за передумову
переповнення товарового ринку. Це спричинилося б до спаду цін. За пак бавовни
треба було б платити менше. Відси ніяк не випливає, що гроші на купівлю бавовни,
можна було б діставати в позику дешевше. Це залежить від стану грошового
ринку. Якщо їх можна дістати в позику дешевше, то тільки тому, що комерційний
кредит є в такому стані, що він мусить менше, ніж звичайно удаватись до
банкового кредиту. Товари, що переповнюють ринок, це — життьові засоби або засоби
продукції. Низька ціна обох підвищує зиск промислового капіталіста. Чому
має він знижувати процент, якщо не з причини протилежности, — а не тотожности,
— між багатістю промислового капіталу та попитом на грошові позики?
Обставини складаються так, що купець і промисловець можуть легше давати
кредит один одному; з причини такого полегшення комерційного кредиту промисловець
і купець менше потребують банкового кредиту; тому рівень проценту
може бути низький. Цей низький рівень проценту не має нічого до діла
з припливом благородного металу, хоч і обидва можуть відбуватися один поряд
одного, й ті самі причини, що породжують низькі ціни на довізні речі, можуть
зумовити також надмір довізного благородного металу. Коли б імпортовий ринок
дійсно був переповнений, то це доводило б зменшення попиту на імпортові
товари, чого при низьких цінах не можна було б пояснити, хіба тільки тим,
що це — наслідок обмеження тубільної промислової продукції; але цього знову
не можна було б пояснити при надмірно великому довозі за низькі ціни. Маємо
чисте безглуздя, як наслідок довести, що спад цін \deq{} спадові рівня проценту.
Обидва явища можуть одночасно існувати одне порядок одного. Але в такому
разі існують вони, як вияв протилежности напрямків, що в них відбувається рух
промислового капіталу та рух позичкового грошового капіталу, а не яв вияв
ідентичности тих напрямків.

Чому — як це сказано в тезі 3) — грошовий процент має бути низький,
коли товарів є понад міру, також не можна зрозуміти, і з нижченаведених
дальших міркувань. Коли товари дешеві, то для того, щоб купити певну кількість
їх, мені треба, скажемо, 1000\pound{ ф. ст.} замість 2000\pound{ ф. ст.}, як то було
\parbreak{}  %% абзац продовжується на наступній сторінці
