\parcont{}  %% абзац починається на попередній сторінці
\index{iii2}{0122}  %% посилання на сторінку оригінального видання
підвищення робочої сили, спільна всякому капіталові, що працює з паровими
машинами. Вона може збільшити ту частину продукту праці, яка становить
додаткову вартість, проти тієї частини, що перетворюється в заробітну плату.
Оскільки вона це робить, вона підвищує загальну норму зиску, але' вона не
утворює надзиску, бо цей надзиск зводиться саме до надміру індивідуального
зиску над пересічним зиском. Те, що застосування певної природної сили — водоспаду
— утворює тут надзиск, не може, отже, постати з того лише факту, що
підвищення продуктивної сили праці завдячує тут застосуванню природної сили.
Для цього мусять постати дальші модифікаційні обставини.

Навпаки. Звичайне застосування сил природи в промисловості може вплинути
на висоту загальної норми зиску, бо воно впливає на масу праці, яку
треба зужити на продукцію потрібних засобів існування. Але воно само по собі не
створює жодного відхилу від загальної норми зиску, а тут справа саме йде
про такий відхил. Далі: надзиск, що його взагалі реалізує індивідуальний капітал
в якійсь окремій сфері продукції, — бо відхили норм зиску між окремими сферами
продукції невпинно вирівнюються в пересічну норму зиску, — постає, коли лишити
осторонь випадкові відхили, від зменшення витрат продукції, отже, видатків на
продукцію; а це зменшення в свою чергу завдячує або тій обставині, що
капітал застосовується в більших масах ніж пересічно і тому faux frais\footnote*{
Faux frais — французький термін, що значить «фалшиві» витрати, тобто непродуктивні. \emph{Пр.~Ред.}
}
продукції зменшуються, тимчасом як загальні причини підвищення продуктивної
сили праці (кооперація, поділ праці тощо), можуть діяти в підвищеній мірі, з
більшою інтенсивністю, бо вони діють на ширшому полі праці; або ж зменшення
витрат продукції завдячує тій обставині, що — коли лишити осторонь той розмір,
в якому функціонує капітал — вживається кращих методів праці, нових винаходів,
удосконалених машин, хемічних таємниць фабрикації тощо, коротко, нових
удосконалених, вищих за пересічний рівень засобів продукції і методів продукції.
Зменшення витрат продукції і надзиск, що випливає з цього, постають тут
з того способу, що ним застосовується капітал, що функціонує. Вони постають
або в наслідок того, що капітал виключно великими масами концентрується в
одних руках, — обставина, яка відпадає, скоро тільки вживаються пересічно
рівновеликі маси капіталу, — або з того, що капітал певної величини функціонує
особливо продуктивним способом, — обставина, яка відпадає, скоро тільки
виключний спосіб продукції набуває загального поширення або випереджується
ще розвиненішим способом.

Отже, причина надзиску постає тут з самого капіталу (залічуючи до
нього і пущену ним у рух працю); чи з ріжниць в розмірі застосованого
капіталу, чи з доцільнішого способу його застосування; само по собі ніщо не
перешкоджає тому, щоб увесь капітал в певній сфері продукції застосувався
однаковим способом. Конкуренція між капіталами намагається навпаки дедалі
більше зрівнювати ці ріжниці; визначення вартости суспільно потрібним робочим
часом здійснюється у здешевленні товарів і в спонукуванні продукувати товари
в однаково сприятливих умовах. Але з надзиском фабриканта, що застосовує
водоспад, справа стоїть інакше. Підвищена продуктивна сила застосовуваної
ним праці постає ані з самого капіталу і праці, ані з простого застосування
природної сили, відмінної від капіталу і праці, та долученої до капіталу. Вона
постає з більшої природної продуктивної сили праці, в зв’язку з використанням
природної сили, але не такої природної сили, що її може мати всякий капітал
в даній сфері продукції, як наприклад, еластичність пари; отже, не такої природної
сили, що її застосування розуміється само собою, скоро тільки капітал
взагалі застосовується в цій сфері. А такої монополізованої природної сили, якою,
як-от водоспадом, можуть порядкувати лише ті, що порядкують окремими дільницями
землі, разом з їхніми належностями. Від капіталу ніяк не залежить покликати
\parbreak{}  %% абзац продовжується на наступній сторінці
