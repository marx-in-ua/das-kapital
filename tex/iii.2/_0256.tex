\parcont{}  %% абзац починається на попередній сторінці
\index{iii2}{0256}  %% посилання на сторінку оригінального видання
виразу того факту, що всяка товарова вартість є лише міра вміщеної в товарі
суспільно потрібної праці. Але вже в першій книзі показано, що це ані трохи
не перешкоджає розпаданню товарового продукту всякого капіталу на окремі
частини, що з них одна становить виключно сталу частину капіталу, друга —
змінну частину капіталу і третя — тільки додаткову вартість.

Шторх висловлює не тільки свою думку, але й думку багатьох інших,
коли говорить: «Les produits vendables qui constituent le revenu national doivent
être considérés dans l’économie politique de deux manières différentes: relativement aux
individus comme des valeurs; et relativement à la nation comme des biens; car le
revenu d’une nation ne s’apprécie pas comme celui d’un individu, d’après sa valeur,
mais d’après son utilité ou d’après les besoins auxquels il peut satisfaire» (Consid.
sur le revenu national, p. 19)\footnote*{
«Продукти, що належать продажеві й становлять національний дохід, в політичній економії слід
розглядати подвійно: як вартості щодо осіб і як блага щодо нації; бо дохід нації визначається не
так, як дохід окремого індивідуума, не за його вартістю, а за його корисністю або за тими потребами,
що їх він може задовольнити».
}.

Поперше, це цілком помилкова абстракція, коли націю, що її спосіб продукції
ґрунтується на вартості, що (націю) далі, капіталістично організовану,
розглядають як спільний організм, який працює тільки для задоволення національних
потреб.

Подруге, по знищенні капіталістичного способу продукції, але при збереженні
суспільної продукції, визначення вартости лишається панівним в тому
розумінні, що реґулювання робочого часу і розподіл суспільної праці між різними
галузями продукції, нарешті, бухгальтерія щодо всього, цього стають,
важливіші, ніж колибудь.

\section{Позірна роля конкуренції}

Вже показано, що вартість товарів, або реґульована їхньою сукупною вартістю
ціна продукції, розпадається на:

1)~Частину вартости, яка покриває сталий капітал, або репрезентує минулу
працю, зужитковану в формі засобів продукції при виготовленні товару; одним
словом вартість або ціну, що її мали ці засоби продукції, які ввійшли в процес
продукції товару. Ми говоримо тут завжди не про окремі товари, а про товаровий
капітал, тобто про ту форму, що її набирає продукт капіталу за певний
період, наприклад, за рік; — про товаровий капітал, що його окремий товар
становить лише один з його елементів і який, проте, врешті теж, за своєю
вартістю, розпадається на такі самі складові частини, що й товаровий капітал.

2)~Частину вартости, яка становить змінний капітал, що виміряє собою
дохід робітника і перетворюється для нього на його заробітну плату, отже, на
заробітну плату, яку робітник репродукував у цій змінній частині вартости;
коротко, це — та частина вартости, що в ній втілюється оплачена частина
праці, новодолученої в продукції товару до першої сталої частини вартости.

3)~Додаткову вартість, тобто ту частину вартости товарового продукту,
що в ній втілюється неоплачена або додаткова праця. Ця остання частина
вартости набуває в свою чергу самостійних форм, які одночасно є форми
доходу: форму зиску на капітал (процент на капітал як такий, підприємницький
бариш з капіталу як капіталу, що функціонує) і форму земельної ренти.
\parbreak{}  %% абзац продовжується на наступній сторінці
