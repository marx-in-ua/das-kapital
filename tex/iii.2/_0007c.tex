
% \index{iii2}{0007}  %% посилання на сторінку оригінального видання

% \chapter{Розпад зиску на процент і підприємецький бариш. Капітал, що дає процент. (Продовження)}

\section{Складові частини банкового капіталу}

Тепер треба ближче придивитись до того, з чого складається банковий
капітал.

Ми щойно бачили, що Фулартон та інші перетворюють ріжницю між грішми
як засобом циркуляції та грішми як платіжним засобом (теж як світовими
грішми, оскільки доводиться вважати на відплив золота) у ріжницю між циркуляцією
(currency) та капіталом.

Особлива роля, що її тут відіграє капітал, приводить до того, що так само
ретельно, як освічена економія силкувалася втовкмачити, що гроші — не капітал,
так само ретельно ця економія банкірів втовкмачує, що гроші дійсно є капітал
par excellence.

Однак в дальших дослідах ми покажемо, що грошовий капітал переплутують
при цьому з moneyed capital в розумінні капіталу, що дає процент,
тимчасом коли в першому розумінні грошовий капітал є завжди тільки переходова
форма капіталу, як форма, відмінна від інших форм капіталу, від
товарового капіталу та продуктивного капіталу.

Банковий капітал складається 1)~з готівки, золота або банкнот, 2)~з
цінних паперів. Ці останні ми знову можемо поділити на дві частини: торговельні
папери, поточні векселі, що їм від часу до часу надходить реченець, та
що на їхнє дисконтування власне сходить праця банкіра; та громадські
цінні папери, як от державні папери, посвідки державної скарбниці, акції
всякого роду, — коротко кажучи, папери, що дають процент, проте істотно відрізняються
від векселів. Сюди можна зарахувати й гіпотеки. Капітал, що складається
з цих речових складових частин, знову поділяється на капітал, приміщений
самим банкіром, та на вклади, що являють його banking capital або
позичений капітал. В банків, що видають банкноти, сюди долучаються й ці
банкноти. Вклади та банкноти ми лишимо тимчасом осторонь. Аджеж очевидно,
що у дійсних складових частинах банкірського капіталу — грошах, векселях,
процентних паперах — нічого не зміняється від того, чи ці різні елементи
\parbreak{}  %% абзац продовжується на наступній сторінці
