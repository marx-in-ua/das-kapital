\parcont{}  %% абзац починається на попередній сторінці
\index{iii2}{0120}  %% посилання на сторінку оригінального видання
цінам продукції. Тоді постає питання як за такого припущення може розвинутися
земельна рента, тобто яким чином частина зиску може перетворитись на земельну
ренту, а тому частина ціни товару — дістатися земельному власникові.

Щоб показати загальний характер цієї форми земельної ренти, припустімо,
що переважна більшість фабрик в якійсь країні рухається паровими машинами,
а певна меншість — природними водоспадами. Припустімо, що ціна продукції
у тих галузях промисловости становить 115 за таку масу товарів, на яку капіталу
споживається 100. Ці 15\% зиску обчислюються не тільки на спожитий капітал
в 100, але на ввесь капітал, вкладений у продукцію цієї товарової вартости.
Цей процес продукції, як показано давніш, визначається не індивідуальними
витратами продукції кожного окремого промислового продуцента, а тими витратами
продукції, яких товар коштує пересічно за пересічних умов для всього капіталу
певної сфери продукції. Це в дійсності ринкова ціна продукції; пересічна ринкова
ціна на відміну від її коливань. Природа вартости товарів, визначення вартости
не тим індивідуальним робочим часом, який потрібен для продукції певної
кількости товарів або окремих товарів певному поодинокому продуцентові, а
суспільно потрібним робочим часом, тим робочим часом, який в даних пересічних
суспільних умовах продукції потрібен для того, щоб випродукувати всю суспільно-потрібну
кількість різних товарів, що є на ринку, — ця природа вартости,
цей спосіб її визначення, взагалі знаходить собі вираз у вигляді ринкової ціни
і, далі, у вигляді реґуляційної ринкової ціни, або ринкової ціни продукції.

А що тут цілком байдуже, які певні числові відношення ми візьмемо, то ми
припустімо далі, що витрати продукції на фабриках, що їх рухає сила води,
становлять лише 90 замість 100. А що ціна продукції маси цих товарів, яка
реґулює ринок, дорівнює 115, з зиском в 15\%, то фабриканти, що рухають
свої машини силою води, теж продаватимуть по 115, тобто по пересічній ціні,
що реґулює ринкову ціну. Тому їхній зиск становив би 25 замість 15; реґуляційна
ціна продукції дозволила б їм одержувати надзиск в 10\% — не тому, що вони
продають свої товари дорожче за ціни продукції, а тому, що вони продають
їх по ціні продукції, тому, що їхні товари продукуються, або що їхній капітал
функціонує у виключно сприятливих умовах, в умовах, що стоять вище від пересічного
рівня, який панує у цій сфері.

Зараз же виявляються двоякі обставини.

\emph{Поперше}. Надзиск продуцентів, що вживають природний водоспад, як
рушійну силу, спочатку має такий самий характер, як усякий надзиск (а ми
вже з’ясували цю категорію, коли досліджували ціни продукції), що не є
випадковим наслідком операцій в процесі циркуляції, випадкових коливань ринкових
цін. Отже, цей надзиск також дорівнює ріжниці між індивідуальною ціною продукції
цих продуцентів, поставлених в сприятливі умови, і загальною суспільною ціною
продукції, що реґулює ринок усієї цієї сфери продукції. Ця ріжниця дорівнює
надмірові загальної ціни продукції товару над його індивідуальною ціною продукції.
Цей надмір реґулюють дві межі: індивідуальні витрати продукції, а тому
індивідуальна ціна продукції, з одного боку; загальна ціна продукції, з другого. —
Вартість товару, продукованого з допомогою водоспаду, менша тому, що для
його продукції потрібно меншої загальної кількости праці, саме менше праці, яка
входить у продукцію в зрічевленій формі, як частина сталого капіталу. Застосована
тут праця продуктивніша, її індивідуальна продуктивна сила більша,
ніж продуктивна сила праці, застосованої в масі фабрик такого самого роду.
Її більша продуктивна сила виявляється в тому, що для продукції тієї самої
маси товарів потрібно меншої кількости сталого капіталу, меншої кількости
зрічевленої праці, ніж іншим; та крім того, меншої кількости і живої праці,
бо опалювати водяне колесо не доводиться. Ця більша індивідуальна продуктивна
сила застосовуваної праці зменшує вартість, а також витрати продукції, а тим
\parbreak{}  %% абзац продовжується на наступній сторінці
