\parcont{}  %% абзац починається на попередній сторінці
\index{iii2}{0185}  %% посилання на сторінку оригінального видання
з тією самою витратою капіталу, перетворюється на надпродукт, який репрезентує
надзиск, а тому й ренту. Коли припустити, що норма зиску лишається
та сама, то орендар міг би купити на свій зиск меншу кількість збіжжя.
Норма зиску може лишитись та сама, коли заробітна плата не підвищиться,
або тому, що її понижено до фізичного мінімуму, отже, нижче нормальної вартости
робочої сили; або тому, що інші речі споживання робітників, давані мануфактурою,
стали порівняно дешевші; або тому, що робочий день став довший
або зробився інтенсивніший, і тому норма зиску в нехліборобських галузях
продукції, яка проте, реґулює хліборобський зиск, лишилась незмінна, якщо
тільки не підвищилась; абож тому, що хоч у хліборобстві й витрачається такий
самий капітал, але більш сталого і менше змінного.

Ми тут розглянули перший спосіб, у який може постати рента на землі
$А$, що до цього часу була найгірша, без того, щоб притягалось до оброблення
ще гіршу землю; а саме, коли вона постає в наслідок ріжниці індивідуальної
ціни продукції на цій землі, — ціни продукції, що до цього часу була за
реґуляційну проти тієї нової, вищої ціни продукції, по якій останній додатковий
капітал, витрачений з недостатною продуктивною силою на кращій землі,
дав потрібну додаткову кількість продукту.

Коли додаткова продукція мусила б постачатись землею $А_{-1}$, яка може дати
квартер лише за 4 ф. стерл., то рента з акра на $А$ підвищилася б до 1 ф. стерл. Але в цьому випадку
земля $А_{-1}$ пересунулася б на місце $А$, як
найгірша з культивованих земель, а земля $А$ вступила б як нижчий член в
ряд родів землі, що дають ренту. Диференційна рента I змінилася б. Отже,
цей випадок лежить поза аналізою диференційної ренти II, яка виникає з різної
продуктивности послідовних витрат капіталу на тій самій дільниці землі.

Але, крім того, диференційна рента на землі $А$, може постати ще двояким
способом:

Коли за незмінної ціни, — будь-якої ціни, хоч би вона і була знижена
проти колишньої, — додаткова витрата капіталу породжує додаткову продуктивність,
що prima facie до певної межі завжди мусить статися якраз на найгіршій
землі.

Подруге, тоді, коли навпаки, продуктивність послідовних витрат капіталу
на землі $А$ понижується.

В обох випадках припускається, що стан попиту потребує збільшення
продукції.

Але, з погляду диференційної ренти, тут виступає специфічна трудність
в зв’язку з раніш викладеним законом, що за ним визначальною для всієї продукції
(або для всієї витрати капіталу) завжди є індивідуальна пересічна ціна
продукції одного квартера. Але для землі $А$, у протилежність кращим родам
землі, ціна продукції, яка обмежує для нових витрат капіталу вирівняння індивідуальної
з загальною ціною продукції, дана не поза нею. Бо індивідуальна ціна
продукції на $А$ і є та сама загальна ціна продукції, що реґулює ринкову ціну.

Припустімо:

1) За висхідної продуктивної сили послідовних витрат
капіталу на одному акрі землі $А$, з авансованим капіталом в 5 ф. стерл.,
відповідно 6 ф. стерл. ціни продукції, можна випродукувати замість 2 квартерів
З квартери. Перша витрата капіталу в 2\sfrac{1}{2} ф. стерл. дає 1 квартер, друга — 2 квартери. В цьому
випадку 6 ф. стерл. ціни продукції дають 3 квартери,
отже, квартер коштуватиме пересічно 2 ф. стерл.; отже, коли 3 квартери
будуть продані по 2 ф. стерл., то $А$, як і давніш, не дасть ренти, але зміниться
лише основа диференційної ренти II; за реґуляційну ціну продукції стали
2 ф. стерл. замість 3 ф. стерл.; на найгіршій землі капітал в 2\sfrac{1}{2}  ф. стерл.
продукує тепер пересічно 1\sfrac{1}{2}  замість 1 квартера, і це тепер офіційна родючість
\parbreak{}  %% абзац продовжується на наступній сторінці
