\parcont{}  %% абзац починається на попередній сторінці
\index{iii2}{0138}  %% посилання на сторінку оригінального видання
номінальна ціна для необроблюваних земель, таким чином вони стають товаром,
джерелом багатства для своїх власників. Цим самим пояснюється, чому зростає ціна
землі всього краю, включаючи сюди і необроблювану землю (Opdyke).
Земельна спекуляція, наприклад, у Сполучених Штатах, ґрунтується лишена цьому
відбитому впливові, що його капітал і праця справляють на необроблювану землю.

\emph{Друге}. Поступ в поширенні оброблюваної землі взагалі відбувається або в
напрямку до гіршої землі, або на різних даних родах землі в різних пропорціях,
залежно від того, що є зручного в наявності. Перехід до гіршої землі звичайно
ніколи не відбувається з доброї волі, а може статися лише — припускаючи капіталістичний спосіб
продукції — в наслідок підвищення цін, а в умовах всякого
способу продукції — лише в наслідок доконечности. Але це не має безумовного значіння. Гіршій землі
дається перевагу над відносно кращою землею в наслідок
її положення, яке має переважне значіння за всякого поширення культури
в молодих країнах; далі і тому, що, хоч земля якогось краю взагалі родючіша,
проте місцями землі кращої й гіршої якости можуть примхливо чергуватися, і
гіршу землю доводиться обробляти вже через те, що вона має зв’язок з кращою.
Коли гірша земля вклинюється в кращу, то це дає їй перевагу положення
над землею родючішою, але не зв’язаною з уже оброблюваною або такою, що має
оброблятися.

Так штат Мічіґен серед західніх штатів один з перших почав вивозити
хліб. В цілому ґрунт його убогий. Але сусідство його з штатом Нью-Йорком і
водне сполучення з допомогою озер і канала Ірі від самого початку давало йому
перевагу над родючішими з природи штатами, що лежать далі на захід.
Приклад цього штату порівняно з штатом Нью-Йорком показує нам також
перехід від кращої землі до гіршої. Ґрунт штату Нью-Йорку, особливо західня
частина, незрівняно родючіший і особливо для сіяння пшениці. Хижацьким обробітком
цей родючий ґрунт перетворено на неродючий, і ґрунт Мічіґену став тепер
родючіший.

«1836 року пшеничне борошно з Буффало вивозили на Захід водою, переважно
з пшеничної округи Нью-Йорку й Верхньої Канади. Тепер, лише через
12 років, величезні запаси пшениці й борошна привозять озером Ірі і каналом
Ірі з Заходу в Буффало й сумежну гавань Блекрок для вивозу водою на
Схід. Експорт пшениці й борошна особливо посилився через голод в Европі
1847 року. Через це пшениця в західній частині Нью-Йорку подешевшала, і
сіяти її стало менш вигідно; це спонукало нью-йоркських фармерів узятися
більше до скотарства, молочарства, садівництва тощо, до галузей, в яких, на,
їхню думку, північно-захід не зуміє безпосередньо конкурувати з ними» (F.~W.~Johnston, Notes on North America, London 1851. I, p. 222).

\emph{Третє}. Неправильно припускали, що ґрунт колоній і взагалі молодих
країн, які можуть вивозити збіжжя за дешевшу ціну, неодмінно має тому більшу
природну родючість. Збіжжя продається в даному разі не тільки нижче його
вартості, але й нижче його ціни продукції, а саме нижче ціни продукції, визначуваної
в старіших країнах пересічною нормою зиску.

Коли ми, як говорить Johnston (стор. 223), «з цими новими штатами, з
яких щороку в Буффало йде такий великий довіз пшениці, звикли поєднувати
уяву про велику природну родючість і безкраї простори багатії землі», то це
залежить насамперед від економічних умов. Уся людність такої країни, як, наприклад,
Мічіґен, працювала спочатку майже виключно в хліборобстві, а саме,
виробляла хліборобські продукти масового споживання, бо тільки їх можна було
вимінювати на промислові товари й тропічні продукти. Ввесь його надмірний продукт
з’являється тому в вигляді збіжжя. Це вже від самого початку відрізняє
колоніяльні держави, засновані на базі сучасного світового ринку, від
колишніх і особливо від колоній античного часу. Вони одержують через світовий
\parbreak{}  %% абзац продовжується на наступній сторінці
