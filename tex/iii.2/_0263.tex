\parcont{}  %% абзац починається на попередній сторінці
\index{iii2}{0263}  %% посилання на сторінку оригінального видання
вартість між двома посідачами цього самого чинника продукції. Але та обставина,
що тут немає певної закономірної межі для поділу пересічного зиску
на окремі частини, не знищує меж його самого як частини товарової вартости,
так само як та обставина, що два спільника якогось підприємства з якихось
зовнішніх обставин ділять поміж собою зиск не нарівно, ніяк не зачіпає меж
цього зиску.

Отже, коли та частина товарової вартости, що в ній визначається праця,
новодолучена до вартости засобів продукції, розпадається на різні частини, що
набувають у формі доходів самостійного вигляду одна проти однієї, то звідси ще
зовсім не випливає, що заробітну плату, зиск і ренту треба розглядати, як
конституційні елементи, з сполучення або підсумовування яких виникає реґуляційна
ціна (natural price, prix nécessaire) самих товарів, — так що при цьому
вже не товарова вартість по вирахуванні з неї сталої частини вартости була б
первісно даною одиницею, яка поділяється на зазначені три частини, а навпаки,
ціна кожної з цих трьох частин являла б собою величину, визначувану самостійно,
і лише з комплексу цих трьох незалежних величин складалася б ціна
товару. В дійсності вартість товару є величина наперед дана, сукупність загальної
вартости заробітної плати, зиску й ренти, хоч би які були відносні величини їх
поміж себе. Навпаки, при зазначеному помилковому погляді заробітна плата,
зиск і рента є три самостійні величини вартости, що їх сукупна величина
створює, обмежує і визначає величину товарової вартости.

Насамперед, ясно, що коли б заробітна плата, зиск і рента конституювали
ціну товарів, то це в однаковій мірі мало б силу так до сталої частини товарової
вартости, як і до другої її частини, що в ній визначається змінний капітал
і додаткова вартість. Отже, ця стала частина може тут бути залишена
осторонь, бо вартість товарів, з яких вона складається, так само зводиться до
суми вартости заробітної плати, зиску й ренти. Як уже зазначено, цей погляд
заперечує також саме існування такої сталої частини вартости.

Ясно далі, що тут відпадає саме розуміння вартости. Лишається тільки
уявлення про ціну, в тому розумінні, що посідачам робочої сили, капіталу
й землі виплачується певну суму грошей. Але що таке гроші? Гроші не річ,
а певна форма вартости, отже в свою чергу мають своєю передумовою вартість.
Отже, припустімо, що певну кількість золота або срібла виплачується за ті елементи
продукції, або що їх у гадці прирівнюється до цієї кількости. Але ж
золото й срібло (і освічений економіст пишається з цього відкриття) сами є
товаром, як усякі інші товари. Отже, ціну золота й срібла також визначається
заробітною платою, зиском і рентою. Отже, ми не можемо заробітну плату, зиск
і ренту визначити тим, що їх можна прирівняти до певної кількости золота
і срібла, бо вартість цього золота і срібла, якою ми хочемо виміряти їх, як їхнім
еквівалентом, має ще й собі лише визначитись саме ними, незалежно від золота
й срібла, тобто незалежно від вартости всякого товару, яка сама є продукт зазначених
трьох чинників. Отже, сказати, що вартість заробітної плати, зиску
й ренти є в тому, що вони дорівнюють певній кількості золота та срібла, значить
лише сказати, що вони дорівнюють певній кількості заробітної плати,
зиску й ренти.

Візьмімо насамперед заробітну плату. Бо і з цього погляду праця мусить
бути за вихідний пункт. Отже, чим визначається реґуляційна ціна заробітної
плати, та ціна, що навколо неї коливаються її ринкові ціни?

Скажімо, попитом і поданням робочої сили. Але про який попит на робочу
силу іде тут мова? Про попит від капіталу. Отже, попит на працю рівнозначний
поданню капіталу. Щоб говорити про подання капіталу, ми мусимо
насамперед знати, що таке капітал. З чого складається капітал? Коли взяти
його найпростішу форму: з грошей і товарів. Але гроші є лише форма товару-
\parbreak{}  %% абзац продовжується на наступній сторінці
