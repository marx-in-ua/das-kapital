\parcont{}  %% абзац починається на попередній сторінці
\index{iii2}{0177}  %% посилання на сторінку оригінального видання
дукції з землі $В$ це становить округло 1\sfrac{1}{2} квартера. Надзиск з $В$ визначається,
отже, у відповідній частині продукту з $В$, в цих 1\sfrac{1}{2} квартерах, які
становлять ренту, визначену в збіжжі, і які продаються по загальній ціні продукції
за 4\sfrac{1}{2}\pound{ ф. стерл}. Але, навпаки, надмірний продукт з акра землі $В$, надмірний
проти продукту з акра землі $А$, не можна просто вважати за надзиск,
а тому й за надпродукт. Згідно з припущенням акр землі $В$ продукує 3\sfrac{1}{2} квартери,
акр землі $А$ лише 1 квартер. Надмірний продукт з землі $В$ є, отже,
2\sfrac{1}{2} квартери, але надпродукт є лише 1\sfrac{1}{2} квартери; бо в землю
$В$ вкладено удвоє більший капітал, ніж у землю $А$, і тому вся ціна продукції тут удвоє
більша. Коли б у землю $А$ також було вкладено 5\pound{ ф. стерл.} і норма продуктивности
лишилася б без зміни, то продукт становив би 2 квартери замість одного,
і таким чином виявилося б, що дійсний надпродукт можна знайти порівнянням
не 3\sfrac{1}{2} і 1, а 3\sfrac{1}{2} і 2; що, отже, він дорівнює не 2\sfrac{1}{2},
а лише 1\sfrac{1}{2} квартерам.
Але далі, якби в землю $В$ було вкладено третю порцію капіталу в 2\sfrac{1}{2}\pound{ ф. стерл.},
що дала б лише 1 квартер, так що він коштував би 3\pound{ ф. стерл.}, як на землі $А$, то
його продажна ціна в 3\pound{ ф. ст.} покрила б тільки ціну продукції, дала б лише
пересічний зиск, але не дала б надзиску, а отже і нічого, що могло б перетворитися
на ренту. Продукт з акра будь-якого роду землі, порівняно з продуктом
з акра землі $А$, не показує ані того, чи є він продукт однакової або більшої
витрати капіталу, ані того, чи надмірний продукт покриває тільки ціну продукції,
чи завдячує він своїм виникненням вищій продуктивності додаткового капіталу.

\emph{Друге}: З щойно викладеного випливає, що при низхідній нормі продуктивности
додаткових витрат капіталу, за межу котрих, — оскільки мова йде
про створення нового надзиску, — є така витрата капіталу, що покриває лише
ціну продукції, тобто що продукує квартер так само дорого, як рівна витрата
капіталу на землі $А$, отже, згідно з припущенням, за 3\pound{ ф. стерл.}, — випливає,
що за межу, на якій загальна витрата капіталу на акр землі $В$ перестала б
давати ренту, є та, коли індивідуальна пересічна ціна продукції продукту з
акра землі $В$ підвищилася б до рівня ціни продукції з акра землі $А$.

Коли на $В$ робляться лише такі додаткові витрати капіталу, що оплачують
ціну продукції і, отже, не створюють надзиску, а тому й нової ренти, то хоч
це й підвищує індивідуальну пересічну ціну продукції квартера, проте, не зачіпає
надзиску, що створився від попередніх витрат капіталу, евентуально ренти. Бо
пересічна ціна продукції завжди лишається нижча від ціни продукції на $А$, а коли
надмір ціни з квартера і зменшується, то кількість квартерів збільшується
у тому самому відношенні, так що загальний надмір ціни лишається без зміни.

В наведеному випадку дві перші витрати капіталу в 5\pound{ ф. стерл.} на землі $В$
продукують 3\sfrac{1}{2} квартери, отже, згідно з припущенням, 1\sfrac{1}{2}
квартери ренти \deq{} 4\sfrac{1}{2}\pound{ ф. стерл}. Коли сюди прилучиться третя витрата
капіталу в 2\sfrac{1}{2}\pound{ ф. стерл.},
що продукує лише 1 додатковий квартер, то вся ціна продукції (включаючи 20\%
зиску) 4\sfrac{1}{2} квартерів \deq{} 9\pound{ ф. стерл.}; отже, пересічна ціна за
квартер \deq{} 2\pound{ ф. стерл}. Отже, пересічна ціна продукції за квартер на землі $В$
піднеслась з 1\sfrac{5}{7}\pound{ ф. стерл.}
до 2\pound{ ф. стерл.}, надзиск з квартера порівняно з регуляційною ціною $А$ упав
з 1\sfrac{2}{7}\pound{ ф. стерл.} до 1\pound{ ф. стерл}. Але 1×4\sfrac{1}{2} \deq{} 4\sfrac{1}{2}\pound{ ф.
стерл.}, цілком так само,
як раніш $1\sfrac{2}{7} × 3\sfrac{1}{2} \deq{} 4\sfrac{1}{2}$\pound{ ф. стерл}.

Коли ми припустимо, що на $В$ було б зроблено ще четверту і п’яту додаткові
витрати капіталу по 2\sfrac{1}{2}\pound{ ф. стерл.}, які продукують квартер лише по його
загальній ціні продукції, то весь продукт з акра становив би тепер 6\sfrac{1}{2} квартерів,
а ціна його продукції була б 15\pound{ ф. стерл}. Пересічна ціна продукції
квартера для $В$ знову підвищилась би з 2
\footnote*{В німецькому тексті тут стоїть «з 1\pound{ ф. стерл.}» Очевидна помилка,
бо у вищенаведеному прикладі пересічна ціна продукції квартера для $В$
становила не 1\pound{ ф. стерл.}, а 2\pound{ ф. стерл.} \emph{Прим. Ред.}}
до 2\sfrac{4}{13}\pound{ ф. стерл.}, а надзиск з квартера
\index{iii2}{0178}  %% посилання на сторінку оригінального видання
порівняно з реґуляційною ціною продукції на землі $А$ знову зменшився
б з 1 ф. стерл, до \sfrac{9}{13}\pound{ ф. стерл}. Але ці \sfrac{9}{13}\pound{ ф. стерл.} тут слід
помножити на 6\sfrac{1}{2} квартерів замість колишніх 4\sfrac{1}{2}
$А$ $\sfrac{9}{13}×6\sfrac{1}{2} \deq{} 1× 4\sfrac{1}{2} \deq{} 4\sfrac{1}{2}$\pound{ ф. стерл}.

Звідси насамперед випливає, що за цих обставин не потрібно жодного підвищення
реґуляційної ціни продукції для того, щоб уможливити додаткові витрати
капіталу на рентодайних землях, навіть в такому розмірі, що додатковий
капітал зовсім перестає давати надзиск і дає ще лише пересічний зиск. З
цього випливає далі, що тут сума надзиску на акр лишається без зміни,
хоч би як дуже зменшувався надзиск з квартера; це зменшення завжди урівноважується
відповідним збільшенням квартерів, продукованих на акрі. Для того,
щоб пересічна ціна продукції піднеслась до рівня загальної ціни продукції (отже,
тут досягла б 3\pound{ ф. стерл.} на землі $В$), мусять бути зроблені такі додаткові витрати
капіталу, продукт яких мав би вищу ціну продукції, ніж реґуляційна ціна
в 3\pound{ ф. стерл}. Але ми побачимо, що тільки цього ще не досить, щоб підвищити
пересічну ціну продукції квартера на землі $В$ до рівня загальної ціни продукції
в 3\pound{ ф. стерл}.

Припустімо, що на землі $В$ було випродуковано:

1) 3\sfrac{1}{2} квартери, що їхня ціна продукції, як і давніш, 6\pound{ ф. стерл.}; отже,
дві витрати капіталу по  2\sfrac{1}{2}\pound{ ф. стерл.} кожна, при чому обидві дають надзиски,
але низхідної висоти.

2) 1 квартер за 3\pound{ ф. стерл.}; витрата капіталу, при якій індивідуальна
ціна продукції дорівнювала б реґуляційній ціні продукції.

3) 1 квартер за 4\pound{ ф. стерл.}; витрата капіталу, при якій індивідуальна
ціна продукції на 25\% вища за реґуляційну ціну.

Ми мали б тоді 5\sfrac{1}{2} квартерів з акра за 13\pound{ ф. стерл.} при витраті капіталу
в 10\pound{ ф. стерл.}; первісна витрата капіталу зросла б учетверо, але продукт
першої витрати капіталу не збільшився б і втроє.

5\sfrac{1}{2} квартерів за 13\pound{ ф. стерл.} дають пересічну ціну продукції в 2\sfrac{4}{11}\pound{ ф. стерл.}
за квартер, отже, за реґуляційної ціни продукції в 3\pound{ ф. стерл.} надмір
в \sfrac{7}{11}\pound{ ф. стерл.} з квартера, який може перетворитися на ренту.
5\sfrac{1}{2} квартерів, продані по реґуляційній ціні в 3\pound{ ф. стерл.} дають
16\sfrac{1}{2}\pound{ ф. стерл}. За вирахуванням
ціни продукції в 13\pound{ ф. стерл.} залишається 3\sfrac{1}{2}\pound{ ф. стерл.} надзиску, або
ренти, так що ці 3\sfrac{1}{2}\pound{ ф. стерл.}, рахуючи по теперішній пересічній ціні
продукції квартера з землі $В$, тобто по 2\sfrac{4}{11}\pound{ ф. стерл.} за квартер,
репрезентують 1\sfrac{25}{52}\footnote*{
В німецькому тексті тут стоїть: «1\sfrac{5}{72}». Очевидна помилка. \emph{Прим. Ред.}
} квартера. Грошова рента понизилася б на 1\pound{ ф. стерл.}, збіжжева
рента приблизно на \sfrac{1}{2} квартера, але, не зважаючи на те, що четверта додаткова
витрата капіталу на $В$ не тільки не створює надзиску, але дає менше, ніж
пересічний зиск, — як і давніш, існує надзиск і рента. Коли ми припустимо, що,
крім витрати капіталу 3), і витрата 2) продукує по ціні, що перебільшує реґуляційну
ціну продукції, то вся продукція становитиме: 3\sfrac{1}{2} квартери за
6\pound{ ф. ст.} \dplus{} 2 квартери за 8\pound{ ф. ст.}, разом 5\sfrac{1}{2} квартерів за 14\pound{ ф. ст.}
ціни продукції. Пересічна ціна продукції квартера була б 2\sfrac{6}{11}\pound{ ф. ст.},
що давало б надмір в \sfrac{5}{11}\pound{ ф. ст}. Ці 5\sfrac{1}{2}  квартери, продані по
3\pound{ ф. ст.}, дають 16\sfrac{1}{2}\pound{ ф. ст.}; за вирахуванням
14\pound{ ф. ст.} ціни продукції, лишається 2\sfrac{1}{2}\pound{ ф. ст.} на ренту. За теперішньої
пересічної ціни продукції на землі $В$ це становило б \sfrac{55}{56} квартера.
Отже, рента все ще одержується, хоч і в меншому розмірі, ніж давніш.

В усякому разі це показує, що на кращих земельних дільницях при додаткових
витратах капіталу, що їхній продукт коштує дорожче, ніж реґуляційна
ціна продукції, рента, принаймні в межах допустимих практикою, мусить не
зникнути, а лише зменшитися, і саме відповідно до того, з одного боку, яку
\parbreak{}  %% абзац продовжується на наступній сторінці
