\parcont{}  %% абзац починається на попередній сторінці
\index{iii2}{0132}  %% посилання на сторінку оригінального видання
умова її виникнення є в тому, що підвищення абсолютної родючості всієї земельної площі не знищує
тієї неоднаковості, але збільшує її, або залишає без зміни, абож лише зменшує.

Від початку і до половини XVIII століття панувало в Англії, не зважаючи на пониження ціни золота і
срібла, безперервне падіння цін збіжжя одночасно (коли розглядати весь період) з ростом ренти,
загальної суми ренти, розміру оброблюваної земельної площі, хліборобської продукції і людности. Де
відповідає таблиці І, комбінованій з таблицею II, у висхідному напрямку, але так, що гірша земля $А$
або поліпшується або виключається з числа земель, оброблюваних під збіжжя; це, звичайно, не значить,
що вона не буде використана для інших сільськогосподарських або промислових цілей.

Від початку XIX століття (треба точніше подати час) до 1815 року безперервне підвищення цін збіжжя
одночасно з постійним зростом ренти, загальної суми ренти, розміру оброблюваної земельної площі,
хліборобської продукції і людності. Це відповідає таблиці І у низхідному напрямку. (Тут слід навести
цитату щодо обробітку гірших земель за того часу).

За доби Петті і Давенанта скарги сільської людности і земельних власників на поліпшення і поширення
обробітку; пониження ренти на кращих землях, підвищення загальної суми ренти в наслідок поширення
площі землі, що дає ренту.

(До цих трьох пунктів навести потім дальші цитати; також щодо ріжниці у родючості різних частин
обробленої землі в країні).

Щодо диференційної ренти слід взагалі зауважити, що ринкова вартість завжди перевищує загальну ціну
продукції даної маси продуктів. Для прикладу візьмімо таблицю І. 10 кватерів всього продукту
продаються за 600\shil{ шил.}, бо ринкова ціна визначається ціною продукції на $А$, яка становить 60\shil{ шил.} за
квартер. Але дійсна ціна продукції є:

\begin{table}[H]
  \centering
  \small
  \begin{tabular}{l r@{~}r l l}
    А & 1 кварт. \deq{} & 60\shil{шил.} & & 1 кварт. \deq{} 60\shil{шил.} \\
    В & 2 кварт. \deq{} & 60\shil{шил.} & & 1 кварт. \deq{} 30\shil{шил.} \\
    C & 3 кварт. \deq{} & 60\shil{шил.} & & 1 кварт. \deq{} 20\shil{шил.} \\
    D & 4 кварт. \deq{} & 60\shil{шил.} & & 1 кварт. \deq{} 15\shil{шил.} \\
    \midrule
      &10 кварт. \deq{} &240\shil{шил.} & ~пересічно & 1 кварт. \deq{} 24\shil{шил.} \\
  \end{tabular}
\end{table}

\noindent{}Дійсна ціна продукції 10 квартерів є 240\shil{ шил.}; вони продаються за 600, тобто на 250\% дорожче.
Дійсна пересічна ціна 1 квартера є 24\shil{ шил.}: ринкова ціна — 60\shil{ шил.}, тобто теж на 250\% дорожча.

Тут маємо визначення за посередництвом ринкової вартости в тому її вигляді, як вона на базі
капіталістичного способу продукції пробивається за посередництвом конкуренції; ця остання породжує
фалшиву соціяльну вартість. Це постає з закону ринкової вартости, якому підпорядковані продукти
хліборобства.
Визначення ринкової вартости продуктів, отже, і хліборобських продуктів, є суспільний акт, хоч і акт
суспільно несвідомий і ненавмисний, акт, що неминуче ґрунтується на міновій вартості продукту, а не
на землі і ріжницях її родючості. Коли уявити собі, що капіталістична форма суспільства знищена і
суспільство організоване як свідома і плянова асоціація, то ці 10 квартерів являтимуть собою
кількість самостійного робочого часу, рівну тому, що міститься в 240\shil{ шил.} Отже, суспільство не
купувало б цього хліборобського продукту за таку кількість робочого часу, яка в 2\sfrac{1}{2}, раза більша
за робочий час, який дійсно міститься в цьому продукті. Тим самим відпала б база класу власників
землі. Це впливало б цілком так само, як здешевлення продукту на таку суму в наслідок чужоземного
довозу. Тому, оскільки справедливо було б сказати, що — в умовах збереження сучасного способу
продукції, але припускаючи, що диференційна рента діставатиметься державі — ціни земельних
продуктів, за інших незміних умов, залишились би тими самими, так само помилково було б
\parbreak{}  %% абзац продовжується на наступній сторінці
