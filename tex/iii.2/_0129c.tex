\parcont{}  %% абзац починається на попередній сторінці
\index{iii2}{0129}  %% посилання на сторінку оригінального видання
що перша земля А поліпшилась в наслідок постійного раціонального обробітку,
або що вона при незмінності витрат стала продуктивніше оброблятися,
наприклад, в наслідок заведення конюшини тощо, так що її продукт, за незмінної
величини авансованого капіталу, збільшився до 1\sfrac{1}{3} кварт. Припустімо
далі, що землі В, C, В, як і давніш, дають ту саму кількість продукту, але що
почали оброблятися нові землі А' пересічної між А і В родючости, далі В', В", що
містяться своєю родючістю між В і C; в цьому випадку постали б такі явища:

\emph{Перше}: ціна продукції квартера пшениці, або її реґуляційна ринкова
ціна, зменшилась би з 60 до 45 шил., або па 25\%.

\emph{Друге}: відбувся б одночасний перехід від родючішої землі до менш
родючої, і від менш родючої землі до родючішої. Земля А' родючіша, ніж А, але
менш родюча, ніж В, C, D, що оброблялись до цього часу; а В', В" родючіші, ніж
А, А' і В, але менш родючі, ніж C і D. Отже, перехід від однієї землі до другої
відбувався б у всіх напрямках; відбувся б перехід не до абсолютно
менш родючої землі проти А тощо, а до відносно менш родючої, порівняно
з землями C і D, які до цього часу були найродючіші; з другого боку, перехід
відбувався б не до абсолютно родючішої землі, а до відносно родючішої проти
земель А, — або А і В, — які до цього часу були найменш родючі.

\emph{Третє}: Рента з В знизилася б; а також рента з C і D; але загальна
сума ренти, визначена в збіжжі, піднеслась би з 6 до 7\sfrac{2}{3} кв.; маса землі, що
обробляється і дає ренту, збільшилася б, а також збільшилася б і маса продукту
з 10 до 17 квар. Зиск, хоч він і лишився без перемін для А, визначений у
збіжжі, підвищився б; але можливо, що навіть норма зиску підвищилася б, бо
підвищилася б відносна додаткова вартість. В цьому випадку в наслідок здешевлення
засобів існування зменшилася б заробітна плата, отже, витрата на змінний капітал,
отже, і загальні витрати. Вся сума ренти, визначена в грошах, знизилась би з 360 до 345шил.

Подаємо нову послідовність переходу
(див. табл. II).

\begin{table}[h]
  \begin{center}
  \footnotesize
    \emph{Таблиця II}

  \begin{tabular}{c c c c c c c с c}
    \toprule
      \multirowcell{2}{\makecell{Рід \\землі}} &
      \multicolumn{2}{c}{Продукт} &
      \multirowcell{2}{\makecell{Витрата \\капіталу}} &
      \multicolumn{2}{c}{Зиск} &
      \multicolumn{2}{c}{Рента} &
      \multirowcell{2}{\makecell{Ціна про-\\дукції \\квартера}}
      \\
    \cmidrule(rl){2-3}
    \cmidrule(l){5-6}
    \cmidrule(l){7-8}
    &
    \makecell{Квар-\\тери} &
    \makecell{Ши-\\лінґи} &
    &
    \makecell{Квар-\\тери} &
    \makecell{Ши-\\лінґи} &
    \makecell{Квар-\\тери} &
    \makecell{Ши-\\лінґи} &
    \\
    \midrule
     А\phantom{''}   &  1\sfrac{1}{3}            & \phantom{0}60 & 50  &  \phantom{0}\sfrac{2}{9} & \phantom{0}10  &  \textemdash             & \textemdash    & 45\phantom{\sfrac{1}{1}} шил. \\
     А'\phantom{'}   &  1\sfrac{2}{3}            & \phantom{0}75 & 50  &  \phantom{0}\sfrac{5}{9} & \phantom{0}25  &  \phantom{0}\sfrac{1}{3} & \phantom{0}15  & 36\phantom{\sfrac{1}{1}} \ditto{шил.} \\
     B\phantom{''}   &  2\phantom{\sfrac{1}{1}}  & \phantom{0}90 & 50  &  \phantom{0}\sfrac{8}{9} & \phantom{0}40  &  \phantom{0}\sfrac{2}{3} & \phantom{0}30  & 30\phantom{\sfrac{1}{1}} \ditto{шил.} \\
     В'\phantom{'}   &   2\sfrac{1}{2}           & 105           & 50  &  1\sfrac{2}{9}           & \phantom{0}55  &  1\phantom{\sfrac{1}{1}}                       & \phantom{0}45  & 25\sfrac{2}{7} \ditto{шил.} \\
     В''             &   2\sfrac{2}{3}           & 120           & 50  &  1\sfrac{5}{9}           & \phantom{0}70  &  1\sfrac{1}{3}           & \phantom{0}60  & 22\sfrac{1}{2} \ditto{шил.} \\
     C\phantom{''}   &  3\phantom{\sfrac{1}{1}}  & 135           & 50  &  1\sfrac{8}{9}           & \phantom{0}85  &  1\sfrac{2}{3}           & \phantom{0}75  & 20\phantom{\sfrac{1}{1}} \ditto{шил.} \\
     D\phantom{''}   &  4\phantom{\sfrac{1}{1}}  & 180           & 50  &  2\sfrac{8}{9}           & 130            &  2\sfrac{2}{3}           & 120            & 15\phantom{\sfrac{1}{1}} \ditto{шил.} \\
     \cmidrule(rl){2-2}
     \cmidrule(l){7-8}
     Разом & 17 & &    &       &      &   7\sfrac{2}{3} &     345 \\
  \end{tabular}
  \end{center}
\end{table}

Нарешті, коли б, як і давніш, оброблялись тільки землі А, В, C, D, але продуктивність їхня зросла б
остільки, що земля А замість 1 квартера давала б 2, В замість 2 квартерів — 4,
C замість 3 квартерів — 7 і D замість 4 квартерів — 10, отже, коли б ті самі
причини по-різному вплинули б на різні землі, то вся продукція підвищилася
б з 10 до 23 квартерів. Припустімо, що попит в наслідок приросту
людности і пониження ціни поглинув би ці 23 квартери, в такому разі ми
мали б такий результат (див. табл. III).

Числові відношення тут, як і в попередніх таблицях, довільні, але припущення
цілком раціональні.

