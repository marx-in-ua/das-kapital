\parcont{}  %% абзац починається на попередній сторінці
\index{iii2}{0129}  %% посилання на сторінку оригінального видання
що перша земля $А$ поліпшилась в наслідок постійного раціонального обробітку,
або що вона при незмінності витрат стала продуктивніше оброблятися,
наприклад, в наслідок заведення конюшини тощо, так що її продукт, за незмінної
величини авансованого капіталу, збільшився до 1\sfrac{1}{3} кварт. Припустімо
далі, що землі $В$, $C$, $D$, як і давніш, дають ту саму кількість продукту, але що
почали оброблятися нові землі $А'$ пересічної між $А$ і $В$ родючости, далі $В'$, $В''$, що
містяться своєю родючістю між $В$ і $C$; в цьому випадку постали б такі явища:

\emph{Перше}: ціна продукції квартера пшениці, або її реґуляційна ринкова
ціна, зменшилась би з 60 до 45\shil{ шил.}, або па 25\%.

\emph{Друге}: відбувся б одночасний перехід від родючішої землі до менш
родючої, і від менш родючої землі до родючішої. Земля $А'$ родючіша, ніж $А$, але
менш родюча, ніж $В$, $C$, $D$, що оброблялись до цього часу; а $В'$, $В''$ родючіші, ніж
$А$, $А'$ і $В$, але менш родючі, ніж $C$ і $D$. Отже, перехід від однієї землі до другої
відбувався б у всіх напрямках; відбувся б перехід не до абсолютно
менш родючої землі проти $А$ тощо, а до відносно менш родючої, порівняно
з землями $C$ і $D$, які до цього часу були найродючіші; з другого боку, перехід
відбувався б не до абсолютно родючішої землі, а до відносно родючішої проти
земель $А$, — або $А$ і $В$, — які до цього часу були найменш родючі.

\emph{Третє}: Рента з $В$ знизилася б; а також рента з $C$ і $D$; але загальна
сума ренти, визначена в збіжжі, піднеслась би з 6 до 7\sfrac{2}{3} кв.; маса землі, що
обробляється і дає ренту, збільшилася б, а також збільшилася б і маса продукту
з 10 до 17 квар. Зиск, хоч він і лишився без перемін для $А$, визначений у
збіжжі, підвищився б; але можливо, що навіть норма зиску підвищилася б, бо
підвищилася б відносна додаткова вартість. В цьому випадку в наслідок здешевлення
засобів існування зменшилася б заробітна плата, отже, витрата на змінний капітал,
отже, і загальні витрати. Вся сума ренти, визначена в грошах, знизилась би з 360 до 345\shil{ шил.}

Подаємо нову послідовність переходу.

\begin{table}[H]
  \centering
  \small
  \caption*{Таблиця II}

  \begin{tabular}{l c c c c c c c c}
    \toprule
      \multirowcell{2}[0ex][l]{Рід\\землі} &
      \multicolumn{2}{c}{Продукт} &
      \multirowcell{2}[0ex][c]{Витрата\\капіталу} &
      \multicolumn{2}{c}{Зиск} &
      \multicolumn{2}{c}{Рента} &
      \multirowcell{2}[0ex][c]{Ціна продукції\\квартера, ш.}
      \\
    \cmidrule(rl){2-3}
    \cmidrule(l){5-6}
    \cmidrule(l){7-8}
    &
    \makecell{кварт.} &
    \makecell{ш.} &
    &
    \makecell{кварт.} &
    \makecell{ш.} &
    \makecell{кварт.} &
    \makecell{ш.} &
    \\
    \midrule
     А   & 1\tbfrac{1}{3}            & \phantom{0}60 & 50  &  \phantom{0}\tbfrac{2}{9} & \phantom{0}10  &  \textemdash             & \textemdash    & 45\phantom{\tbfrac{1}{1}} \\
     А'  & 1\tbfrac{2}{3}            & \phantom{0}75 & 50  &  \phantom{0}\tbfrac{5}{9} & \phantom{0}25  &  \phantom{0}\tbfrac{1}{3} & \phantom{0}15  & 36\phantom{\tbfrac{1}{1}} \\
     B   & 2\phantom{\tbfrac{1}{1}}  & \phantom{0}90 & 50  &  \phantom{0}\tbfrac{8}{9} & \phantom{0}40  &  \phantom{0}\tbfrac{2}{3} & \phantom{0}30  & 30\phantom{\tbfrac{1}{1}} \\
     В'  & 2\tbfrac{1}{2}           & 105           & 50  &  1\tbfrac{2}{9}           & \phantom{0}55  &  1\phantom{\tbfrac{1}{1}}                       & \phantom{0}45  & 25\tbfrac{2}{7} \\
     В'' & 2\tbfrac{2}{3}           & 120           & 50  &  1\tbfrac{5}{9}           & \phantom{0}70  &  1\tbfrac{1}{3}           & \phantom{0}60  & 22\tbfrac{1}{2} \\
     C   & 3\phantom{\tbfrac{1}{1}}  & 135           & 50  &  1\tbfrac{8}{9}           & \phantom{0}85  &  1\tbfrac{2}{3}           & \phantom{0}75  & 20\phantom{\tbfrac{1}{1}} \\
     D   & 4\phantom{\tbfrac{1}{1}}  & 180           & 50  &  2\tbfrac{8}{9}           & 130            &  2\tbfrac{2}{3}           & 120            & 15\phantom{\tbfrac{1}{1}} \\
     \cmidrule(rl){2-2}
     \cmidrule(l){7-8}
     Разом & \hang{r}{1}7\phantom{\tbfrac{1}{1}} & &    &       &      &   7\tbfrac{2}{3} &     345 \\
  \end{tabular}
\end{table}

\noindent{}Нарешті, коли б, як і давніш, оброблялись тільки землі $А$, $В$, $C$, $D$, але продуктивність їхня зросла б
остільки, що земля $А$ замість 1 квартера давала б 2, $В$ замість 2 квартерів — 4,
$C$ замість 3 квартерів — 7 і $D$ замість 4 квартерів — 10, отже, коли б ті самі
причини по-різному вплинули б на різні землі, то вся продукція підвищилася
б з 10 до 23 квартерів. Припустімо, що попит в наслідок приросту
людности і пониження ціни поглинув би ці 23 квартери, в такому разі ми
мали б такий результат.

\begin{table}[H]
  \centering
  \small
  \caption*{Таблиця III}

  \begin{tabular}{l c c c c c c c c}
    \toprule
      \multirowcell{2}[0ex][l]{Рід\\землі} &
      \multicolumn{2}{c}{Продукт} &
      \multirowcell{2}[0ex][c]{Витрата\\капіталу} &
      \multicolumn{2}{c}{Зиск} &
      \multicolumn{2}{c}{Рента} &
      \multirowcell{2}[0ex][c]{Ціна продукції\\квартера, ш.}
      \\
    \cmidrule(rl){2-3}
    \cmidrule(l){5-6}
    \cmidrule(l){7-8}
    &
    \makecell{кварт.} &
    \makecell{ш.} &
    &
    \makecell{кварт.} &
    \makecell{ш.} &
    \makecell{кварт.} &
    \makecell{ш.} &
    \\
    \midrule
      А  &  \phantom{0}2  &  \phantom{0}60  & 50 & \phantom{0}\tbfrac{1}{3}  & \phantom{0}10  & \phantom{0}0 & \phantom{00}0  &  30\phantom{\tbfrac{1}{1}}\\
      B  &  \phantom{0}4  &  120            & 50 & 2\tbfrac{1}{3}            & \phantom{0}70  & \phantom{0}2 & \phantom{0}60  &  15\phantom{\tbfrac{1}{1}}\\
      C  &  \phantom{0}7  &  210            & 50 & 5\tbfrac{1}{3}            & 160            & \phantom{0}5 & 150            &  \phantom{0}8\tbfrac{4}{7} \\
      D  &  10              &  300            & 50 & 8\tbfrac{1}{3}            & 250            & \phantom{0}8 & 240            &  \phantom{0}6\phantom{\tbfrac{1}{1}} \\
      \cmidrule(rl){2-2}
      \cmidrule(l){7-8}
      Разом & 23          &                 &    &                          &                & 15           & 450           & \\
  \end{tabular}
\end{table}

\noindent{}Числові відношення тут, як і в попередніх таблицях, довільні, але припущення
цілком раціональні.
 