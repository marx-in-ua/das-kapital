\parcont{}  %% абзац починається на попередній сторінці
\index{iii2}{0029}  %% посилання на сторінку оригінального видання
виникнення та незвичайно швидке поширення нової кляси фірм, що, працюючи
коло розподілу капіталу, в дійсності є банкіри великого маштабу, дарма що
їх звичайно звуть billbrokers’ами. Операції цих фірм сходять на те, щоб,
узявши на певний умовний реченець та за певні умовні проценти надмірний
капітал в банках тих округ, де того капіталу не можна було ужити, а так
само й ті засоби акційних товариств та великих торговельних фірм, що лежать
тимчасово без діла, визичати ці гроші за вищий процент банкам тих округ, де
є більший попит на капітал; звичайно шляхом редисконту векселів своїх клієнтів\dots{}
Таким чином Lombardstreet стала великим центром, де відбувається перенесення
капіталу, що лежить без діла, з однієї частини країни, де його не можна ужити
з користю, до іншої, де є попит на нього: і це має силу так для різних частин
країни, як і для поодиноких осіб, що перебувають в аналогічному становищі.
Первісно ці операції обмежувались майже тільки на позичанні та визичанні
під застави, звичайні у банковій практиці. Але відповідно до того, як капітал
країни швидко зростав і все більше економізувався за допомогою засновуваних
банків, фонди, що були до розпорядку цих дисконтових фірм, так збільшувались,
що ці фірми почали давати позики спочатку під dock warrants (посвідки
на товари, складені в доках), а потім і під накладні, що представляли ще
зовсім не прибулі продукти, дарма що іноді, хоч і нерегулярно, під ці накладні
вже видано було векселі на товарового маклера. Така практика скоро змінила
увесь характер англійської кредитової справи. Полегкості, що їх, отже, давала Lombardstreet,
утворили дуже міцну позицію товарових маклерів на Mincing Lane’i; ці
маклери своєю чергою знову давали всі ці пільги купцям-імпортерам; ці останні
почали так дуже користуватися з того, що тимчасом, коли 25 років тому кредитування
під накладну або навіть під dock warrants якогось купця підтяло б
його кредит, за останні роки ця практика стала така загальна, що її можна
вважати за правило, а не за рідкий виняток, як то було 25 років тому. Ця
система навіть так поширилася, що великі суми беруть на Lombardstreet’i під
векселі, видані під хліб \emph{на пні} по далеких колоніях. В наслідок таких пільг
купці-імпортери поширювали свої закордонні операції та твердо приміщували
свій текучій (floating) капітал, що за його допомогою вони досі провадили своє
підприємство, в найнепевніших з усіх підприємств, в колоніяльних плянтаціях.
що їх вони мало або й зовсім не могли контролювати. Отже, ми бачимо безпосередній
зв’язок кредитів. Капітал країни, що назбирався в наших хліборобських
округах, складається невеликими сумами як вклади до провінціяльних банків,
та централізується для вжитку на Lombardstreet’i. Але використовується його,
поперше, для поширу підприємств в наших гірничих та промислових округах за
допомогою редисконту векселів, виданих на тамтешні банки; а потім також для того,
щоб дати більші полегкості імпортерам закордонних продуктів, а саме позиками
під dock warrants та під накладні; в наслідок цього «лояльний» купецький
капітал фірм в закордонній та колоніяльній торговлі міг звільнятися та уживатися
для найнепевніших родів приміщення в заокеанських плянтаціях». (Economist,
1847~\abbr{р.}, р. 1334). Оце — те «прегарне» переплутування кредитів. Сільський
вкладник уявляє собі, що склав вклад тільки у свого банкіра, він уявляє собі
далі, що банкір, визичаючи гроші, робить це відомим йому приватним особам
Він не має ані найменшої уяви про те, що цей банкір дає свої вклади до розпорядку
якомусь лондонському billbroker’ові, що над його операціями обидва
вони не мають ані найменшого контролю.

Як великі громадські підприємства, напр., будівництво залізниць, можуть
на певний час збільшувати позичковий капітал, коли виплачувані за акції
суми лишаються завжди до вільного розпорядку банків протягом певного часу,
аж до того часу, коли дійсно вживається ті суми, — це ми вже бачили.
\pfbreak
