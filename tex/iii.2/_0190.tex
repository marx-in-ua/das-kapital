\parcont{}  %% абзац починається на попередній сторінці
\index{iii2}{0190}  %% посилання на сторінку оригінального видання
по якій може бути приставлено ввесь продукт, і в цьому розумінні вона реґулює
ціну цього всього продукту.

Проте, \emph{подруге}, хоч у цьому випадку загальна ціна продукту землі
істотно модифікувалася б, цим зовсім не був би скасований закон диференційної
ренти. Бо коли ціна продукту кляси $А$, а разом з тим і загальна ринкова
ціна $= Р \dplus{} r$, то ціна кляс $B$, $C$, $D$ і~\abbr{т. ін.} теж була б
$= P \dplus{} r$. А що
$Р - Р'$ для кляси $B$ $= d$, то $(Р \dplus{} r) - (Р' \dplus{} r)$ теж було б $= d$, а для $C$
$P - Р'' \deq{} (Р \dplus{} r) - (Р'' \dplus{} r)$ було б $= 2d$, як для $D$, нарешті,
$Р - Р''' \deq{} (Р \dplus{} r) - (Р''' \dplus{} r) \deq{} 3d$ і~\abbr{т. д.} Отже, диференційна рента лишилася б та сама, що
давніш і реґулювалася б тим самим законом, хоч рента мала б у собі елемент незалежний
від цього закону, і хоч вона взагалі підвищилась би одночасно з ціною продукту
землі. Звідси випливає: хоч би як завжди стояла справа з рентою з найменш
родючих родів землі, закон диференційної ренти від цього не тільки не залежить,
але навіть єдиний спосіб зрозуміти саму диференційну ренту відповідно до її
характеру є в тому, що рента кляси землі $А$ припускається рівною нулеві. Чи
вона дійсно $= 0$, чи $> 0$, це байдуже, оскільки справа йде про диференційну
ренту, і насправді не береться на увагу.

Отже, закон диференційної ренти не залежить від наслідку дальшого
дослідження.

Тепер, коли ставити далі питання про підставу того припущення, що продукт
землі найгіршого роду $А$ не дає ренти, то відповідь неминуче така:
коли ринкова ціна продукту землі, скажімо, збіжжя, досягає такої висоти, що
додатково авансований капітал, укладений в землю кляси $А$, оплачує звичайну
ціну продукції, тобто дає капіталові звичайний пересічний зиск, то цієї умови
досить для приміщення додаткового капіталу в землю кляси $А$. Тобто, капітаталістові
досить цієї умови для того, щоб укладати новий капітал з звичайним
зиском і використовувати його нормальним способом.

Тут слід зауважити, що і в цьому випадку ринкова ціна мусить стояти
вище, ніж ціна продукції на $А$. Бо скоро створюється додаткове подання, відношення
попиту і подання очевидно зміниться. Давніш подання було недостатнє,
тепер воно достатнє. Отже, ціна мусить понизитись. Але для того, щоб вона могла
понизитись, вона давніш мусила стояти на вищому рівні, ніж ціна продукції на $А$.
Але те, що кляса $А$, яка наново вступає в обробіток, менш родюча, призводить до
того, що ціна не впаде знову до такого низького рівня, як в той час, коли ринок
реґулювала ціна продукції кляси $В$. Ціна продукції на $А$ становить межу не для
тимчасового, а для відносно перманентного підвищення ринкової ціни. — Навпаки,
коли новооброблювана земля родючіша, ніж кляса $А$, яка до того часу була за
реґуляційну, і проте, її досить лише для покриття додаткового попиту, то ринкова
ціна залишається без зміни. Але дослідження того, чи дає ренту нижча
кляса землі, і в цьому випадку збігається з тим, яким ми зайняті тепер, бо
і тут припущення, що кляса землі $А$ не дає ренти, з’ясовувалося б тим, що
капіталістичному орендареві досить ринкової ціни, щоб нею точно покрити
зужиткований капітал плюс пересічний зиск; коротко кажучи, тим, що ринкова
ціна дає йому ціну продукції його товару.

В усякому разі капіталістичний орендар, може за цих відношень обробляти
землю кляси $А$, оскільки він вирішує справи як капіталіст. Умова для
нормального збільшення вартости капіталу на землі роду $А$ є тепер в наявності.
Але з тієї передумови, що орендар міг би вкладати тепер капітал у землі
роду $А$, за умов відповідних пересічним відношенням зростання вартости капіталу,
хоч він і не мав би можливости платити ренту, — зовсім не випливає
висновок, що ця земля, належна до кляси $А$, так і буде без дальших околичностей
віддана орендареві. Та обставина, що орендар міг би використати свій
капітал з звичайним зиском, коли йому не доводиться платити ренти, для
\parbreak{}  %% абзац продовжується на наступній сторінці
