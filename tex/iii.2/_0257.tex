\parcont{}  %% абзац починається на попередній сторінці
\index{iii2}{0257}  %% посилання на сторінку оригінального видання
що дістається власникові тієї землі, яка бере участь у процесі продукції. Складові
частини 2) і 3), тобто та складова частина вартости, що завжди набуває
форми доходу: заробітної плати (цю форму вона завжди набирає лише після
того, як попередньо проробить форму змінного капіталу), зиску й ренти —
відрізняється від сталої частини 1) тим, що в ній міститься вся та вартість, в
якій зрічевлюється праця, новодолучена до цієї сталої частини, до засобів продукції
товару. Коли ми лишимо осторонь сталу частину вартости, то буде слушним
сказати, що вартість товару, оскільки вона таким чином являє собою новодолучену
працю, завжди розпадається на три частини, які становлять три форми
доходу — на заробітну плату, зиск і ренту\footnote{
При розпаданні вартости, долученої до сталої частини капіталу, на заробітну плату, зиск і ренту,
ці останні, зрозуміло, являють собою частини вартости. Їх можна, звичайно, уявити собі як втілені в
безпосередньому продукті, що в ньому визначається ця вартість, тобто в безпосередньому продукті,
який випродукували робітники й капіталісти певної окремої сфери продукції, наприклад,
бавовнянопрядіння, отже, в пряжі. Але в дійсності вони втілені в цьому продукті не більше і не
менше, ніж у будь-якому іншому товарі, будь-якій іншій складовій частині речового багатства такої ж
самої вартости. Адже на практиці заробітну плату виплачується грішми, що становлять чистий вираз
вартости, так само як і процент і рента. І справді для капіталіста перетворення його продукту на
чистий вираз вартости має велику вагу; його припускається вже при самому розподілі. Чи
перетворюються ці вартості зворотно на той самий продукт, на той самий товар, що з його продукції
вони виникли, чи купує
робітник, зворотно частину безпосередньо випродукованого ним продукту, чи ж вій купує продукт іншої
і якісно відмінної праці, — все це не має чинення до розглядуваного питання. Пан Ротбертус цілком
даремно розпалюється з приводу цього.
}, що їхні відповідні величини вартости,
тобто ті частки, які ці величини становлять від усієї вартости, визначаються
різними, своєрідними, зазначеними вище законами. Але було б неправильно
сказати, навпаки, що вартість заробітної плати, норма зиску і норма ренти
становлять самостійні конституційні елементи вартости, що з їх сполучення виникає,
— коли лишити осторонь сталу складову частину, — вартість товару; інакше
кажучи, було б неправильно сказати, що вони становлять композиційні складові
частини товарової вартости або ціни продукції\footnote{
Досить зауважити, що ті самі загальні правила, які реґулюють вартість сирового продукту і
мануфактурних товарів, є застосовні також і до металів; їхня вартість залежить не від норми зиску не
від норми заробітної плати, не від ренти, виплачуваної з копалень, але від усієї кількости праці,
потрібної для того, щоб здобути метал і приставити його на ринок». (Ricardo, Princ., chap. III, р.
77),
}.

Одразу легко помітити, в чому тут ріжниця.

Припустімо, що вартість продукту якогось капіталу в 500 \deq{} 400c \dplus{}
100v \dplus{} 150m \deq{} 650; при чому ці 150m розпадаються і собі на 75 зиску \dplus{}
75 ренти. Припустімо далі, щоб уникнути непотрібних труднощів, що це —
капітал середнього складу, та що його ціна продукції збігається з його вартістю;
збіг цей завжди буває, коли продукт цього окремого капіталу може розглядатись
як продукт відповідної до його величини частини сукупного капіталу.

Тут заробітна плата, вимірювана змінним капіталом, становить 20\% авансованого
капіталу; додаткова вартість, обчислена на сукупний капітал — 30\%,
а саме 15\% зиску і 15\% ренти. Вся складова частина вартости товару, що
в ній зрічевлюється новодолучена праця, дорівнює 100v \dplus{} 150m \deq{} 250. Величина
її не залежить від того, як розпадається вона на заробітну плату, зиск і ренту.
Ми бачимо з співвідношення цих частин, що робоча сила, оплачена сумою
грошей в 100, скажімо, в 100 фунтів стерл., дала кількість праці, що визначається
сумою грошей в 250 фунтів стерл. Ми бачимо звідси, що робітник
виконав додаткової праці в півтора раза більше, ніж праці для самого себе.
Коли робочий день \deq{} 10 годинам, він працював 4 години для себе і 6 для
капіталіста. Тому праця робітника, оплаченого 100 фунтів стерл., визначається
в грошовій вартості в 250 фунтів стерл. Крім цієї вартости в 250 фунтів стерл.,
немає чого ділити між робітником і капіталістом, між капіталістом і земельним
власником. Це є вся вартість, новодолучена до вартости засобів продукції,
\parbreak{}  %% абзац продовжується на наступній сторінці
