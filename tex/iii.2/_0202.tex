\parcont{}  %% абзац починається на попередній сторінці
\index{iii2}{0202}  %% посилання на сторінку оригінального видання
отже, для цивілізованих народів взагалі головний засіб існування. Вже А.~Сміт
показав, — і це одна з його заслуг, — що в скотарстві і взагалі пересічно в
усіх капіталах, вкладених в землю не для продукції головних засобів існування,
наприклад, збіжжя, ціну визначається зовсім інакше. Саме її визначається
тут таким чином, що ціна продукту землі, яку, скажімо, як штучні луки використовується
для скотарства, але яку з такою самою зручністю можна було б
повернути на орну землю певної гідности, — що ціна продукту такої землі мусить,
підвищитись остільки, щоб давала вона таку саму ренту, як орна земля такої
самої якости; отже, рента з орної землі бере тут участь у визначенні ціни худоби,
чому Рамзей слушно відзначив, що ціна худоби в такий спосіб штучно
підвищується рентою, економічним виразом земельної власности, отже, земельною
власністю.

«В наслідок поширення культури, необроблюваних пустирів уже не вистачає
для того, щоб поповняти подання худоби, яка йде на заріз. Значну частину
оброблюваних земель доводиться повертати під розведення та відгодовування
худоби, що ціна її тому мусить бути така висока, щоб оплатити не
тільки вжиту на це працю, але й ренту, яку міг би здобувати землевласник,
і зиск, який міг би здобувати орендар з цієї землі, коли б вона була
оброблена як орна земля. Худоба вигодована на найнепридатніших до оброблення
торфовищах, буде продана, відповідно до ваги і якости, по тій самій ціні, поякій
на тому самому ринку продається худобу, вигодовану на землі якнайкраще
культивованій. Власники цих торфовищ виграють від цього і підвищують ренту
своїх земель відповідно до цін худоби». (A.~Smith, Book 1, chap. XI, part. I).
Отже, і тут на відміну від збіжжевої ренти диференційна рента йде на користь
гіршій землі.

Абсолютна рента пояснює деякі явища, які з першого погляду справляють
таке вражіння, ніби рента є наслідок просто монопольної ціни. Щоб почати
з прикладу А.~Сміта, уявімо, наприклад, власника ростучого без усякої
людської допомоги лісу, отже, такого лісу, що існує не як продукт лісництва,
наприклад, в Норвегії. Коли ренту виплачує йому капіталіст, що займається
рубанням лісу, бо на нього є попит в Англії, або коли власник сам як капіталіст
теж береться до рубання, то в дереві йому, крім зиску на авансований
капітал, виплачується більшу або меншу ренту. У відношенні до цього чисто
природного продукту це видається чисто монопольною надвишкою. Але в дійсності
капітал складається тут майже тільки з змінного капіталу, витрачуваного
на працю, і тому він пускає в рух більшу кількість додаткової праці, ніж інший
капітал рівної величини. Отже, у вартості дерева міститься більший надмір
неоплаченої праці або додаткової вартости, ніж у продукті капіталів вищого
складу. Тому з дерева може виплачуватись пересічний зиск, і значний надмір у
формі ренти може припадати власникові лісу. Навпаки, доводиться визнати, що
при тій легкості, з якою може поширюватися рубання лісу, тобто при тій
швидкості, з якою може збільшуватися ця продукція, треба дуже значного збільшення
попиту для того, щоб ціна дерева зрівнялась з його вартістю, і щоб тому
ввесь надмір неоплаченої праці (над тією її частиною, яка дістається капіталістові
як пересічний зиск) дістався б власникові у формі ренти.

Ми припускали, що новопритягнена до обробітку земля своює якістю ще
гірша, ніж та найгірша, що оброблялась останнього часу. Коли вона краща, то
вона дає диференційну ренту. Але тут ми досліджуємо саме той випадок, коли
рента являє собою не диференційну ренту. Тут можливі тільки два випадки.
Або новопритягнена до обробітку земля гірша, абож вона такої самої якости,
як остання з оброблюваних земель. Коли вона гірша, то цей випадок ми вже
дослідили. Отже, лишається дослідити ще тільки той випадок, коли вона такої
самої якости.
