\parcont{}  %% абзац починається на попередній сторінці
\index{iii2}{0144}  %% посилання на сторінку оригінального видання
1\sfrac{1}{2} ф. стерл. за квартер становило б 12 ф. стерл., тимчасом як грошова рента
з D давніш дорівнювала 9 ф. стерл. Це слід відзначити. Коли зробити розрахунок
на акр, то висота ренти піднеслась би на 33\sfrac{1}{3}\%, не зважаючи на зменшування
норми надзиску з 2 додаткових капіталів по 2\sfrac{1}{2} ф. стерл. кожен.

Звідси видно, до яких надзвичайно складних комбінацій призводить диференційна
рента взагалі, і особливо в її формі II в зв’язку з формою І, тимчасом
як Рікардо, наприклад, трактує її цілком однобічно і як річ просту. Буває, наприклад,
як вище наведено, падіння реґуляційної ринкової ціни і одночасно зріст
ренти на родючих землях, так що зростає як абсолютний продукт, так і
абсолютний надпродукт. (При диференційній ренті І по низхідній лінії може
зростати відносний надпродукт, а тому й рента з акра, хоч абсолютний надпродукт
з акра лишається той самий або навіть зменшується). Але одночасно
зменшується продуктивність капіталовкладень, що їх роблять одне по одному
на тій самій землі, хоч значна частина їх і припадає на родючіші землі. Коли
дивитися з одного погляду — з погляду кількости продукту і цін продукції —
продуктивність праці зросла. Але з другого погляду вона зменшилась, бо норма
надзиску і надпродукт на акр для різних капіталовкладень на тій самій землі
зменшились.

Диференційна рента II за зменшеної продуктивности послідовних приміщень
капіталу тільки тоді була б неодмінно зв’язана з подорожченням ціни продукції
і абсолютним зменшенням продуктивности, коли б ці приміщення капіталу
могли бути зроблені виключно на гіршій землі А. Коли акр землі А, що
при вкладенні капіталу в 2\sfrac{1}{2} ф. стерл. давав 1 квартер по ціні продукції
в 3 ф. стерл., при дальшому вкладенні в 2\sfrac{1}{2} ф. стерл., тобто при загальному
вкладенні в 5 ф. стерл., дає сукупно лише 1\sfrac{1}{2} квартера, то ціна продукції цих
1\sfrac{1}{2}, квартерів = 6 ф. стерл., а тому 1 квартера = 4 ф. стерл. Всяке пониження
продуктивности при ростучому вкладенні капіталу було б тут відносним зменшенням
продукту з акра, тимчасом як на землі кращих сортів воно є лише
зменшення надмірного надпродукту.

Але сама природа справи призводить до того, що з розвитком інтенсивної
культури, тобто послідовних вкладень капіталу в ту саму землю, ці вкладення
відбуваються переважно або в більшій мірі на землях кращих родів. (Ми не говоримо
тут про ті тривалі поліпшення, за допомогою яких землі, що були непридатні,
перетворюються в придатні). Зменшувана продуктивність послідовних
витрат капіталу мусить, отже, діяти переважно вищезазначеним чином. Найкращу
землю вибирається тут тому, що вона дає найбільшу надію на рентабельність
від застосованого на ній капіталу, бо має в собі найбільшу кількість природних
елементів родючости, що їх треба лише використати.

Коли після скасування хлібних законів культура в Англії зробилась ще
інтенсивніша, масу земель, на яких до того сіяли пшеницю, було використано
з іншою метою, а саме, як пасовиська; навпаки, родючі простори землі, найпридатніші
для сіяння пшениці, були дреновані та іншим способом поліпшені.
Таким чином капітал, що вживався у виробленні пшениці, був сконцентрований
на меншій дільниці.

В цьому випадку — а всі можливі додаткові норми, що містяться між найбільшою
кількістю надпродукту з кращої землі і кількістю продукту землі
А, що не дає ренти, відповідають тут не відносному, а абсолютному збільшенню
надпродукту на акр — новоутворений надзиск (евентуальна рента) становить не
перетворену на ренту частину колишнього пересічного зиску (частина продукту,
в якій раніше виявлявся пересічний зиск), а додатковий надзиск, який з цієї
форми перетворився на ренту.

Навпаки, тільки в тому випадку, коли попит на збіжжя збільшився
так, що ринкова ціна перевищила б ціну продукції на землі А, і тому
\parbreak{}  %% абзац продовжується на наступній сторінці
