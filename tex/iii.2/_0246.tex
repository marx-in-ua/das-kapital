
\index{iii2}{0246}  %% посилання на сторінку оригінального видання
Щодо першої трудности: хто повинен оплатити вміщену в продукті сталу
частину вартости і чим? — то припускається, що вартість сталого капіталу, зужиткованого у продукції,
знову з’являється як частина вартости продукту. Це не
суперечить засновкам другої трудности. Бо вже в книзі І, розділ  V (процес праці і
процес зростання вартости) показано, яким чином в наслідок простого долучення
нової праці, хоч вона і не репродукує старої вартости, а створює лише додаток
до неї, створює лише додаткову вартість, все таки разом з тим зберігається
у продукті стара вартість; але одночасно показано, що це відбувається як наслідок
праці не остільки, оскільки вона є вартостетворча, тобто праця взагалі, а в її
функції як певної продуктивної праці. Таким чином, не треба жодної новодолучуваної
праці для того, щоб зберегти вартість сталої частини в тому продукті,
на який витрачається дохід, тобто вся створена протягом року вартість. Але,
зрозуміла річ, потрібна новодолучувана праця для того, щоб покрити вартість
і споживну вартість сталого капіталу, зужиткованого протягом минулого
року. Без такого покриття репродукція взагалі неможлива.

Вся новодолучена праця втілюється у новоствореній протягом року вартості,
яка і собі цілком сходить на три види доходу: заробітну плату,
зиск і ренту. — Отже, з одного боку, не лишається надмірної суспільної праці
для покриття зужиткованого сталого капіталу, що підлягає відновленню почасти
in natura і в його вартості, почасти тільки в його вартості (оскільки справа
йде просто про зношування основного капіталу). З другого боку, вартість, що
створена річною працею і розпадається на форми заробітної плати, зиску і ренти,
і яка в цьому вигляді підлягає витрачанню, є недостатня для того, щоб оплатити
або купити сталу частину капіталу, яка теж мусить міститися в продукті,
крім новоствореної вартости.

Ми бачимо, що поставлену тут проблему вже розв’язано при дослідженні
репродукції сукупного суспільного капіталу, книга II, відділ III.~Тут ми вертаємось
до цього насамперед тому, що там додаткова вартість ще не була розгорнута
в тих її формах, яких вона набуває як дохід: зиск (підприємницький
бариш плюс процент) і рента, а тому і не могла бути досліджена в цих формах; потім також і тому, що
якраз з формою заробітної плати, зиску і ренти
сполучається неймовірний прогріх в аналізі, який проходить через усю політичну
економію, починаючи від А.~Сміта.

Ми поділили там увесь капітал на дві великі кляси: кляса І, що створює
засоби продукції, кляса II, що продукує засоби індивідуального споживання.
Та обставина, що деякі продукти можуть так само правити за речі особистого
користування, як і засоби продукції (кінь, збіжжя тощо) зовсім не знищує абсолютної
правдивости цього поділу. Справді, він не гіпотеза, а лише вираз факту.
Візьмімо річний продукт якоїсь країни. Частина продукту, хоч яка б була здатність
його правити за засіб продукції, входить в індивідуальне споживання.
Це — продукт, на який витрачається заробітну плату, зиск і ренту. Продукт цей
становить продукт певного підрозділу суспільного капіталу. Можливо, що цей
самий капітал продукує також і продукти, що належать до кляси І.~Оскільки
це так, продуктивно спожиті продукти, належні до кляси І, постачаються не
тією частиною цього капіталу, що зужиткована на продукт кляси II, на продукт,
який дійсно дістається індивідуальному споживанню. Весь той продукт
II, що входить в індивідуальне споживання, і на який тому витрачається
дохід, є формою буття зужиткованого на нього капіталу плюс випродукований
надмір. Отже, це — продукт капіталу, вкладеного тільки в продукцію
засобів споживання. І в цьому ж розумінні підрозділ І річного продукту, який
править за засоби репродукції, — сирового матеріялу і знарядь праці, — хоч би
яка взагалі була здатність цього продукту naturaliter правити за засоби споживання,
— є продукт капіталу, вкладеного виключно в продукцію засобів
\parbreak{}  %% абзац продовжується на наступній сторінці
