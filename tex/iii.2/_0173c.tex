
В. Коли гірша (позначувана літерою а) земля стає землею, яка реґулює
ціну і через те $А$ починає давати ренту. Де не виключає можливости незмінюваної
продуктивности другої витрати для всіх варіянтів.

Варіянт 1: Незмінювана продуктивність другої витрати капіталу.

\begin{table}[H]
  \begin{center}
    \emph{Таблиця XXII}
    \footnotesize

  \begin{tabular}{c@{  } c@{  } c@{  } c@{  } c@{  } c@{  } c}
    \toprule
      \multirowcell{2}{\makecell{Рід\\ землі}} &
      Ціна продукції &
      Продукт &
      \makecell{Продажна \\ ціна} &
      \makecell{Здо-\\буток} &
      Рента &
      \multirowcell{2}{Підвищення ренти} \\

      \cmidrule(r){2-2}
      \cmidrule(r){3-3}
      \cmidrule(r){4-4}
      \cmidrule(r){5-5}
      \cmidrule(r){6-6}

       & Шил. & Бушелі & Шил. & Шил. & Шил. &  \\
      \midrule
      a & \phantom{60 \dplus{} 60 \deq{} }120 & \phantom{10 \dplus{} 10 \deq{} }16 & 7\tbfrac{1}{2} & 120  & \phantom{00}0  & \phantom{01 × }0 \\
      A & 60 \dplus{} 60 \deq{} 120           & 10 \dplus{} 10 \deq{} 20            & 7\tbfrac{1}{2} & 150  & \phantom{0}30 & \phantom{1 ×} 30 \\
      B & 60 \dplus{} 60 \deq{} 120           & 12 \dplus{} 12 \deq{} 24            & 7\tbfrac{1}{2} & 180  & \phantom{0}60 & 2 × 30 \\
      C & 60 \dplus{} 60 \deq{} 120           & 14 \dplus{} 14 \deq{} 28            & 7\tbfrac{1}{2} & 210  & \phantom{0}90 & 3 × 30 \\
      D & 60 \dplus{} 60 \deq{} 120           & 16 \dplus{} 16 \deq{} 32            & 7\tbfrac{1}{2} & 240  & 120           & 4 × 30 \\
      E & 60 \dplus{} 60 \deq{} 120           & 18 \dplus{} 18 \deq{} 36            & 7\tbfrac{1}{2} & 270  & 150           & 5 × 30 \\

     \cmidrule(r){6-6}
     \cmidrule(r){7-7}

      & & & & & 450 & \hang{r}{1}5 × 30 \\
  \end{tabular}

  \end{center}
\end{table}

Варіянт 2: Низхідна продуктивність другої витрати капіталу.

\begin{table}[H]
  \begin{center}
    \emph{Таблиця XXIII}
    \footnotesize

  \begin{tabular}{c@{  } c@{  } c@{  } c@{  } c@{  } c@{  } c}
    \toprule
      \multirowcell{2}{\makecell{Рід\\ землі}} &
      Ціна продукції &
      Продукт &
      \makecell{Продажна \\ ціна} &
      \makecell{Здо-\\буток} &
      Рента &
      \multirowcell{2}{Підвищення ренти} \\

      \cmidrule(r){2-2}
      \cmidrule(r){3-3}
      \cmidrule(r){4-4}
      \cmidrule(r){5-5}
      \cmidrule(r){6-6}

       & Шил. & Бушелі & Шил. & Шил. & Шил. &  \\
      \midrule
      a & \phantom{60 \dplus{} 60 \deq{} }120 & \phantom{10 \dplus{} 10\tbfrac{1}{2} \deq{} }15\phantom{\tbfrac{1}{2}}  & 8 & 120 & \phantom{00}0 & \phantom{5 × 0}0 \phantom{+ 01 × 28} \\
      A & 60 \dplus{} 60 \deq{} 120           & 10 \dplus{} \phantom{0}7\tbfrac{1}{2} \deq{} 17\tbfrac{1}{2}                       & 8 & 140 & \phantom{0}20 & \phantom{5 × }20 \phantom{+ 01 × 28} \\
      B & 60 \dplus{} 60 \deq{} 120           & 12 \dplus{} \phantom{0}9\phantom{\tbfrac{1}{2}} \deq{} 21\phantom{\tbfrac{1}{2}}   & 8 & 168 & \phantom{0}48 & \phantom{5 × }20 \dplus{} \phantom{01 × }28\\
      C & 60 \dplus{} 60 \deq{} 120           & 14 \dplus{} 10\tbfrac{1}{2} \deq{} 24\tbfrac{1}{2}                      & 8 & 194 & \phantom{0}76 & \phantom{5 × }20 \dplus{} \phantom{0}2 × 28 \\
      D & 60 \dplus{} 60 \deq{} 120           & 16 \dplus{} 12\phantom{\tbfrac{1}{2}} \deq{} 28\phantom{\tbfrac{1}{2}}  & 8 & 224 & 104           & \phantom{5 × }20 \dplus{} \phantom{0}3 × 28 \\
      E & 60 \dplus{} 60 \deq{} 120           & 18 \dplus{} 13\tbfrac{1}{2} \deq{} 31\tbfrac{1}{2}                      & 8 & 252 & 132           & \phantom{5 × }20 \dplus{} \phantom{0}4 × 28 \\

     \cmidrule(r){6-6}
     \cmidrule(r){7-7}

      & & & & & 380 & 5 × 20 \dplus{} 10 × 28 \\
  \end{tabular}

  \end{center}
\end{table}

Варіянт 3: Висхідна продуктивність другої витрати капіталу.

\begin{table}[H]
  \begin{center}
    \emph{Таблиця XXIV}
    \footnotesize

  \begin{tabular}{c@{  } c@{  } c@{  } c@{  } c@{  } c@{  } c}
    \toprule
      \multirowcell{2}{\makecell{Рід\\ землі}} &
      Ціна продукції &
      Продукт &
      \makecell{Продажна \\ ціна} &
      \makecell{Здо-\\буток} &
      Рента &
      \multirowcell{2}{Підвищення ренти} \\

      \cmidrule(r){2-2}
      \cmidrule(r){3-3}
      \cmidrule(r){4-4}
      \cmidrule(r){5-5}
      \cmidrule(r){6-6}

       & Шил. & Бушелі & Шил. & Шил. & Шил. &  \\
      \midrule
      a & \phantom{60 \dplus{} 60 \deq{} }120 & \phantom{10 \dplus{} 10\tbfrac{1}{2} \deq{} }16\phantom{\tbfrac{1}{2}}  & 7\tbfrac{1}{2} & 120\phantom{\tbfrac{1}{2}} & \phantom{00}0\phantom{\tbfrac{1}{2}} & \phantom{5 × 15 \dplus{} 15 × }0\phantom{\tbfrac{3}{4}} \\
      A & 60 \dplus{} 60 \deq{} 120           & 10 \dplus{} 12\tbfrac{1}{2} \deq{} 22\tbfrac{1}{2}                      & 7\tbfrac{1}{2} & 168\tbfrac{3}{4}           & \phantom{0}48\tbfrac{3}{4}           & \phantom{5 × }15 \dplus{} \phantom{1 × }33\tbfrac{3}{4} \\
      B & 60 \dplus{} 60 \deq{} 120           & 12 \dplus{} 15\phantom{\tbfrac{1}{2}} \deq{} 27\phantom{\tbfrac{1}{2}}  & 7\tbfrac{1}{2} & 202\tbfrac{1}{2}           & \phantom{0}82\tbfrac{1}{2}           & \phantom{5 × }15 \dplus{} 2 × 33\tbfrac{3}{4} \\
      C & 60 \dplus{} 60 \deq{} 120           & 14 \dplus{} 17\tbfrac{1}{2} \deq{} 31\tbfrac{1}{2}                      & 7\tbfrac{1}{2} & 236\tbfrac{1}{4}           & 116\tbfrac{1}{4}                     & \phantom{5 × }15 \dplus{} 3 × 33\tbfrac{3}{4} \\
      D & 60 \dplus{} 60 \deq{} 120           & 16 \dplus{} 20\phantom{\tbfrac{1}{2}} \deq{} 36\phantom{\tbfrac{1}{2}}  & 7\tbfrac{1}{2} & 270\phantom{\tbfrac{1}{2}} & 150\phantom{\tbfrac{1}{2}}           & \phantom{5 × }15 \dplus{} 4 × 33\tbfrac{3}{4} \\
      E & 60 \dplus{} 60 \deq{} 120           & 18 \dplus{} 22\tbfrac{1}{2} \deq{} 40\tbfrac{1}{2}                      & 7\tbfrac{1}{2} & 303\tbfrac{3}{4}           & 183\tbfrac{3}{4}                     & \phantom{5 × }15 \dplus{} 5 × 33\tbfrac{3}{4} \\

     \cmidrule(r){6-6}
     \cmidrule(r){7-7}

      & & & & & 581\tbfrac{3}{4} & 5 × 15 \dplus{} 15 × 33\tbfrac{3}{4} \\
  \end{tabular}

  \end{center}
\end{table}
