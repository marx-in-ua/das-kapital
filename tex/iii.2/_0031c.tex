\parcont{}  %% абзац починається на попередній сторінці
\index{iii2}{0031}  %% посилання на сторінку оригінального видання
золотий запас. Тому, коли золото відпливає, це зменшує в країні суму незайнятого
капіталу і з цієї причини підноситься вартість решти капіталу. — 1364. Золотий
запас Англійського банку становить справді центральний резерв або той скарб
готівкою, що на його основі провадиться всі справи країни\dots{} Це — той скарб
або той резервуар, що на ньому завжди відбивається вплив закордонного вексельного
курсу». (Report on Bank Acts 1857).

\pfbreak

Маштаб для нагромадження дійсного, тобто продуктивного та товарового
капіталу, дає статистика вивозу та довозу. І тут раз-у-раз виявляється, що
для періодів розвитку англійської промисловости (1815--1870), які перебігають
десятирічними циклами, максимум останнього періоду розцвіту \emph{перед} кризою
кожного разу є знову мінімум ближчого наступного періоду розцвіту, щоб потім
піднестися до далеко вищого нового максимуму.

Дійсна або декларована вартість продуктів, вивезених з Великобританії та
Ірляндії в 1824 році, в році розцвіту, становила \num{40.396.300}\pound{ ф. ст}. Потім з кризою
1825 року розмір вивозу спадає нижче цієї суми, коливаючись між 35 та 39 мільйонами
на рік. Коли повернувся розцвіт в 1834 році, він підноситься понад попередній
найвищий рівень до \num{41.649.191}\pound{ ф. ст.}, досягаючи в 1836 році нового
максимуму в \num{53.368}. 571\pound{ ф. ст}. В 1837 році він знову спадає до 42 міл.,
так що новий мінімум уже вищий за старий максимум, коливаючися потім між
50 та 53 мільйонами. Поворот розцвіту підносить суму вивозу в 1844 році до
58\sfrac{1}{2} мільйонів, при чому максимум 1836 року вже знову далеко перевищено.
Року 1845 він досягає \num{60.111.082}\pound{ ф. ст.}; потім в 1846 році спадає до суми щось
понад 57 мільйонів, в 1847 році майже 59 мільйонів, в 1848 році майже 53 мільйони,
в 1849 році підноситься до 63\sfrac{1}{2} мільйонів, в 1853 році майже 99 мільйонів,
в 1854 році 97 мільйонів, в 1855 році 94\sfrac{1}{2} мільйони, в 1856 році майже
116 мільйонів, досягаючи в 1857 році максимуму в 122 мільйони. В 1858 році
він спадає до 116 мільйонів, але вже в 1859 році підноситься до 13 (1 мільйонів.
в 1860 році майже 136 мільйонів, в 1861 році лише 125 мільйонів (тут
новий мінімум знову вищий за попередній максимум), в 1863 році 146\sfrac{1}{2} мільйонів.

Те саме можна було б, звичайно, виявити й щодо довозу, який є покажчик
поширу ринку; тут маємо ми до діла тільки з маштабом продукції. [Само
собою зрозуміло, це має силу щодо Англії тільки для часу фактичної промислової
монополії: але це має силу й щодо всіх країн з сучасною великою промисловістю,
поки світовий ринок ще поширюється. — \emph{Ф.~Е.}].

\subsection{Перетворення капіталу або доходу на гроші, що перетворюються
в позичковий капітал}

Ми розглядаємо тут нагромадження грошового капіталу, оскільки воно не
означає ані припинення в потоці комерційного кредиту, ані економізування, чи
то засобів, що є дійсно в циркуляції, чи то запасного капіталу аґентів, що зайняті
у репродукції.

Опріч цих двох випадків, нагромадження грошового капіталу може поставати
через надзвичайний приплив золота, як це було в 1852 та 1853 роках
в наслідок відкриття нових, австралійських та каліфорнійських покладів золота.
Де золото складалось до Англійського банку. Вкладники брали замість нього
банкноти, що їх вони безпосередньо по тому не складали в банкірів. Тим способом
кількість засобів циркуляції незвичайно збільшувалася. (Свідчення Weguelin’a,
В.~C. 1857, № 1329). Банк силкувався ці вклади використувати, знизивши
дисконт до 2\%. Маса золота, нагромаджена в банку, зросла протягом
шести місяців 1853 року до 22--23 мільйонів.
