\parcont{}  %% абзац починається на попередній сторінці
\index{iii2}{0160}  %% посилання на сторінку оригінального видання
буде вирівнювати або загострювати ріжниці, диференційна рента з кращих земель,
а разом з тим і загальна сума ренти знизиться або підвищиться, як це було
вже в випадку з диференційною рентою І.~В решті, це залежить від величини земельної
площі й капіталу, вилучених разом з $А$, і від відносного розміру авансованого
капіталу, потрібного за висхідної продуктивности для того, щоб дати
додаткову кількість продукту для покриття попиту.

Єдиний пункт, на дослідженні якого тут варто спинитися, і який взагалі
вертає нас до дослідження того, як цей диференційний зиск перетворюється
на диференційну ренту, є такий:

У першому випадку, коли ціна продукції лишається та сама, додатковий
капітал, вкладений в землю $А$, не справляє впливу на диференційну ренту, як
таку, бо земля $А$, як і давніш, не дає ренти, ціна її продукту лишається та
сама, і продовжує реґулювати ринок.

У другому випадку, варіянт І, коли ціна продукції за незмінної норми продуктивности
понижується, земля $А$ неодмінно відпадає, і ще в більшій мірі це
відбувається у варіянті II (низхідна ціна продукції за низхідної норми продуктивности),
бо в противному разі додатковий капітал, вкладений у землю $А$,
мусив би підвищити ціну продукції. Але тут, у варіянті III другого випадку,
коли ціна продукції понижується, бо продуктивність додаткового капіталу підвищується,
цей додатковий капітал за певних умов може бути вкладений так
в землю $А$, як і в землі кращої якости.

Припустімо, що додатковий капітал в 2\sfrac{1}{2}\pound{ ф. стерл.}, вкладений в землю
$А$, продукує 1\sfrac{1}{5} кварт. замість 1 квартера.

\begin{table}[h]
  \begin{center}
    \emph{Таблиця VI}
    \footnotesize

  \begin{tabular}{c@{  } c@{  } c@{  } c@{  } c@{  } c@{  } c@{  } c@{  } c@{  } c@{  } c}
    \toprule
      \multirowcell{2}{\makecell{Рід\\ землі}} &
      \multirowcell{2}{Акри} &
      Капітал &
      Зиск &
      \makecell{Ціна\\ продук.} &
      \multirowcell{2}{\makecell{Продукт в\\ квартерах}} &
      \makecell{Продажна \\ ціна} &
      \makecell{Здо-\\буток} &
      \multicolumn{2}{c}{Рента} &
      \multirowcell{2}{\makecell{Норма \\надзиску}} \\

      \cmidrule(r){3-3}
      \cmidrule(r){4-4}
      \cmidrule(r){5-5}

      \cmidrule(r){7-7}
      \cmidrule(r){8-8}
      \cmidrule(r){9-9}
      \cmidrule(r){10-10}

       &  & ф. ст. & ф. ст. & ф. ст. & & ф. ст. & ф. ст. & Кварт. & ф. ст. &   \\
      \midrule
       A & 1 & 2\sfrac{1}{2} \dplus{} 2\sfrac{1}{2} \deq{} 5 & 1 & 6 & 1 \dplus{} 1\sfrac{1}{5} \deq{} 2\sfrac{1}{5} & 2\sfrac{8}{11} & \phantom{0}6 & 0\phantom{\sfrac{1}{2}} & \phantom{0}0 & \phantom{00}0\% \\
       B & 1 & 2\sfrac{1}{2} \dplus{} 2\sfrac{1}{2} \deq{} 5 & 1 & 6 & 2 \dplus{} 2\sfrac{2}{5} \deq{} 4\sfrac{2}{5} & 2\sfrac{8}{11} & 12           & 2\sfrac{1}{5}           & \phantom{0}6 & 120\% \\
       C & 1 & 2\sfrac{1}{2} \dplus{} 2\sfrac{1}{2} \deq{} 5 & 1 & 6 & 3 \dplus{} 3\sfrac{3}{5} \deq{} 6\sfrac{3}{5} & 2\sfrac{8}{11} & 18           & 4\sfrac{2}{5}           & 12           & 240\%\\
       D & 1 & 2\sfrac{1}{2} \dplus{} 2\sfrac{1}{2} \deq{} 5 & 1 & 6 & 4 \dplus{} 4\sfrac{4}{5} \deq{} 8\sfrac{4}{5} & 2\sfrac{8}{11} & 24           & 6\sfrac{3}{5}           & 18           & 360\%\\
     \cmidrule(r){1-1}
     \cmidrule(r){2-2}
     \cmidrule(r){3-3}
     \cmidrule(r){4-4}
     \cmidrule(r){5-5}
     \cmidrule(r){6-6}

     \cmidrule(r){8-8}
     \cmidrule(r){9-9}
     \cmidrule(r){10-10}
     \cmidrule(r){11-11}

      Разом & 4 & \phantom{2\sfrac{1}{2} \dplus{} 2\sfrac{1}{2} \deq{}}20 & 4 & 24 & \phantom{2 \dplus{} 1\sfrac{1}{2} \deq{}}22\phantom{\sfrac{1}{2}} & & 60 & 13\sfrac{1}{5} & 36 & 240\%\footnotemarkZ{}\\
  \end{tabular}

  \end{center}
\end{table}
\footnotetextZ{Тут пересічну норму надзиску обчислено не до всього вкладеного капіталу, а тільки до капіталу, вкладеного в рентодайні дільниці $В$, $C$ і $D$. \emph{Прим. Ред.}} % текст примітки прямо під заголовком

Цю таблицю слід порівняти, крім основної таблиці І, і з таблицею II, в якій
подвоєне вкладення капіталу сполучається з сталою продутивністю, пропорційною
капіталовкладенню.

Згідно з припущенням, регуляційна ціна продукції понижується. Коли б
вона залишалася сталою, 3\pound{ ф. стерл.}, то найгірша земля $А$, що давніш, при
капіталовкладенні лише в 2\sfrac{1}{2}\pound{ ф. стерл.}, не давала ренти, тепер почала б давати
ренту, хоч ніякої нової найгіршої землі не було б притягнено до оброблення;
це сталося б саме в наслідок того, що продуктивність на ній збільшилася б, але
лише для частини капіталу, а не для первісно вкладеного капіталу. Перші 3\pound{ ф.
стерл.} ціни продукції дають 1 квартер; другі — 1\sfrac{1}{5} квартера; але ввесь продукт в
2\sfrac{1}{5} квартери продається тепер по його пересічній ціні. А що норма продуктивности
зростає з додатковим капіталовкладенням, то це включає й поліпшення.
