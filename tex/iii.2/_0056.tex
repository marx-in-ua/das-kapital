\parcont{}  %% абзац починається на попередній сторінці
\index{iii2}{0056}  %% посилання на сторінку оригінального видання
в 1848 році доводить лише, що позика капіталу навіть тоді, коли банк переводить
її, видаючи нові банкноти, не свідчить ще невідмінно про збільшення
кількости банкнот в циркуляції.

«3099. Чи думаєте ви, що, прим., Англійський банк міг би значно поширити
свої позики, не збільшуючи видання банкнот? — Є понад міру фактів, що
це доводять. Один з найяскравіших прикладів цього трапився в 1835 році, коли
банк зужив вест-індські вклади та позику в ост-індської кампанії для збільшення
позик публіці; в той самий час сума банкнот в руках публіки фактично
трохи зменшилася\dots{} Дещо подібне було помітно в 1846 році за того періоду,
коли залізничні вклади складалось до банку; цінні папери (дисконтовані та
заставлені) піднеслися приблизно до 30 міл., однак це не мало помітного
впливу на суму банкнот в руках публіки».

Але поряд банкнот гуртова торговля має другий і для неї далеко важливіший
засіб циркуляції: вексель. Пан Chapman показав нам, оскільки важливо
для регулярного розвитку справ, щоб добрі векселі приймалось на оплату скрізь
та серед усяких обставин: коли вже й це не має сили, так що має лихові
зарадити, ох біда! Отож як відносяться ці два засоби циркуляції один до одного?

Про це каже Gilbart: «Обмеження суми банкнот в циркуляції реґулярно
збільшує суму векселів в ній. Векселі бувають двоякого роду — торговельні
векселі та банкірські векселі\dots{} якщо грошей стає обмаль, то позикодавці
грошей кажуть: «Виставляйте на нас вексель, а ми його акцептуємо», і коли
якийсь провінціяльний банкір дисконтує вексель, то він дає клієнтові не гроші
готівкою, а свою власну трату на 21 день на свого лондонського аґента. Ці
векселі правлять за засіб циркуляції». (G.~W.~Gilbart, An Inquiry into the Causes
of the Pressure etc. p. 31).

У трохи зміненій формі це потверджує Newmarch В.~А. 1857, № 1426:

«Між коливаннями в сумі векселів, що перебувають в циркуляції, та
коливаннями банкнот, що перебувають в циркуляції, немає жодного зв’язку\dots{}
однісінький, більш менш рівномірний результат є той\dots{} що, скоро на грошовому
ринку настає найменша скрута, оскільки її виявляє піднесення норми дисконту,
обсяг циркуляції векселів значно збільшується й навпаки».

Однак, векселі, видані за таких часів, ніяк не є лише короткотермінові
банкові векселі, що про них згадує Gilbart. Навпаки; це — здебільша акомодаційні
векселі, що не представляють жодних дійсних операцій, або представляють
тільки такі операції, які лише на те розпочато, щоб мати змогу виставити під них
вексель; прикладів тих двох родів векселів ми подали досить. Тому то Economist
(Wilson), порівнюючи забезпеченість таких векселів з забезпеченістю банкнот,
каже: «Банкноти, що їх оплачують негайно пред’явникові, ніколи не можуть у
надмірному числі лишатися поза банком, бо надмір їх раз-у-раз припливатиме
назад до банку для розміну, тимчасом коли двомісячні векселі можна видавати
дуже надмірно, бо немає засобу контролювати видання їх, поки надійде реченець
їхньої оплати, а тоді може бути їх знову вже замінено на інші векселі. Щоб
нація визнавала забезпеченість векселів в циркуляції, що їх мається оплатити
у якийсь майбутній реченець, та мала б, навпаки, сумнів щодо циркуляції
паперових грошей, розмінюваних пред’явникові негайно — це для нас цілком
незрозуміло». (Economist, 1847~\abbr{р.}, 572).

Таким чином, кількість векселів в циркуляції, як і кількість банкнот, визначається
тільки потребами обороту; за п’ятидесятих років в звичайні часи у
Сполученому Королівстві в циркуляції було поряд 39 мільйонів банкнот приблизно
на 300 міл. векселів, з того на 100--120 міл. на самий лише Лондон. Обсяг
вексельної циркуляції не має впливу на обсяг циркуляції банкнот, та зазнає
впливу цієї останньої лише підчас недостачі грошей, коли кількість векселів
більшає, а їхня якість гіршає. Насамкінець, під час кризи вексельна циркуляція
\parbreak{}  %% абзац продовжується на наступній сторінці
