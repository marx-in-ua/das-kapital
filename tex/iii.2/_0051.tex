\parcont{}  %% абзац починається на попередній сторінці
\index{iii2}{0051}  %% посилання на сторінку оригінального видання
Навіть зріст вивозу видається більш або менш для кожної країни, а найбільше
для тієї країни, що дає кредит, чимраз більшим попитом на внутрішньому
грошовому ринку, чимраз більшим попитом, що однак відчувається як
такий лише підчас пригнічення. Коли вивіз більшає, британські фабриканти виставляють
звичайно довготермінові векселі на купців-експортерів під відправлені
на комісію британські фабрикати. (5126) «5127. Чи не часто складаються такі
угоди, щоб від часу до часу поновлювати ці векселі? — [Chapman], Це — така
справа, яку вони тримають в таємниці від нас; ми б не прийняли такого векселя\dots{}
Певне, це може траплятись, але про дещо подібне я не можу нічого
сказати». [Безгріховний Chapman] — «5123. Якщо дуже збільшується вивіз, як
от, напр., за останній тільки рік на 20 міл. ф. ст., то чи не приводить це само
собою до великого попиту на капітал для дисконтування векселів, що представляють
цей вивіз? — Безперечно. — 5130. Що Англія звичайно дає закордонові
кредит на ввесь свій вивіз, то чи може це зумовити поглинення відповідного
додаткового капіталу протягом того часу, коли той кредит триває? —
Англія дає величезний кредит; але зате вона бере на кредит свої сирові
матеріяли. Америка завжди бере векселі на нас на 60 днів, а інші країни — на
90 днів. З другого боку, ми даємо кредит; коли ми відправляємо товари до
Німеччини, то даємо 2 або 3 місяці кредиту».

Вілсон питає Chapman’a (5131), чи не виписують векселі на Англію під
ці імпортові сирові матеріяли та колоніяльні товари одночасно з їхнім навантаженням
та чи не прибувають уже вони навіть одночасно з накладними? На
думку Chapman’a це так, але він нічого не знає про ці «купецькі» операції, —
треба, мовляв, спитати більш тямущих людей. — При експорті до Америки, каже
Chapman, «товари символізувалися в транзиті»; ця нісенітниця має значити, що
англійський купець-експортер виписує чотиримісячний вексель під товари на
одну з великих американських банкових фірм в Лондоні, а ця банкова фірма
одержує покриття з Америки.

«5136. Чи не провадять звичайно справ з далекими країнами через такого
купця, що чекає на поворот свого капіталу, поки товари спродано? — Може й
є фірми, що, маючи значне приватне багатство, в стані витратити свій власний
капітал на товари, не беручи позик під ці товари; однак, здебільша, ці товари
перетворюються на позики за допомогою акцептів добре відомих фірм. — 5137.
Ці фірми закладено\dots{} в Лондоні, Ліверпулі та де-інде. — 5138. Отже, немає ніякої
ріжниці, чи фабрикантові доведеться витрачати свої власні гроші, чи він знайде
в Лондоні або Ліверпулі купця, що ті гроші визичить; це завжди є позика, що
її зроблено в Англії? — Цілком слушно. Фабрикант має з цим дещо до діла лише
в небагатьох випадках» (навпаки, в 1847 році майже в усіх випадках). «Торговець
фабрикатами, напр., менчестерськими, купує товари та відправляє їх
кораблями за посередництвом поважної лондонської фірми; скоро лондонська фірма
пересвідчилася, що все навантажено на кораблі за умовою, торговець виписує
шестимісячний вексель на лондонську фірму під ці товари, відправлені до Індії,
Китаю або до якої іншої країни; тоді приходить банкова система, дисконтуючи
йому цей вексель; отже на той час, коли він мас платити за ці товари, він
має вже гроші напоготові в наслідок дисконтування тих векселів. — 5139. А коли
він і має гроші, то банкір все таки мусить позичати їх йому? — \emph{Банкір має
вексель; банкір купив вексель;} він уживає свій банковий капітал у цій формі,
а саме у формі дисконтування торгових векселів». [Отже й Chapman розглядає
дисконтування векселів не як позику, а як купівлю товару. — Ф.~Е.] — «5140.
Але це являє проте завжди частину попиту на грошовому ринку Лондону? —
Безперечно; в цьому головна робота грошового ринку й Англійського банку.
Англійський банк так само, як і ми, радіє, одержуючи ці векселі; він знає, що
вони є добре приміщення грошей. — 5141. Чи зростає також попит на грошовому
\parbreak{}  %% абзац продовжується на наступній сторінці
