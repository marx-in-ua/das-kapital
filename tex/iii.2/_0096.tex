\parcont{}  %% абзац починається на попередній сторінці
\index{iii2}{0096}  %% посилання на сторінку оригінального видання
Child, батько нормального англійського приватного банкірства. Він деклямує
проти монополії лихварів цілком так, як велика фірма одягу, Moses and Son,
реклямуе себе як борця проти монополії «приватних кравців». Цей Josiah Child
в одночасно й батько англійського Stockjobberei (маклерства в торговлі процентовими
паперами). Так він, цей самодержець Ост-індської компанії, боронить її
монополію в ім’я вільної торговлі. Проти Thomas Manley («Interest of Money
mistaken») він каже: «Як оборонець боягузливої та полохливої банди лихварів
ставить він свою головну батарею в тому пункті, що його я проголосив за
найнедолужніший\dots{} він навіть відкидає те, що низький рівень проценту є причина
багатства, запевнюючи, що він лише наслідок багатства». (Traités sur la
Commerce etc. 1669. Trad. Amsterdam et Berlin, 1754). «Коли торговля збагачує
країну й коли зниження проценту збільшує торговлю, то зниження проценту
або обмеження лихварства, безперечна річ, є головна плодотворча причина багатства
нації. Аж ніяк не безглуздо казати, що та сама річ одночасно серед певних
обставин може бути причиною, а серед інших — наслідком». (І. c., р. 55.). «Яйце —
причина курки, а курка — причина яйця. Зменшення проценту може викликати
збільшення багатства, а збільшення багатства — ще більше зменшення проценту».
(I. c., р. 156). «Я — оборонець промисловости, а мій супротивник боронить
лінощі та гультяйство» (р. 179).

Ця люта боротьба з лихварством, це домагання підпорядкувати капітал, що
дає процент, промисловому капіталові є лише предтеча органічних витворів, що
породжують ці умови капіталістичної продукції у формі новітньої банкової справи,
яка з одного боку, вириває в лихварського капіталу його монополію, концентруючи
та кидаючи до грошового ринку всі мертві грошові запаси, а, з другого
боку, обмежує саму монополію благородних металів, утворюючи кредитові гроші.

Так само, як тут у Child’a, здибуємо у всіх творах про банкову справу
в Англії за останню третину 17 та на початку 18 віку ворожнечу до лихварства,
домагання емансипувати торговлю, промисловість та державу від лихварства.
А одночасно й величезні ілюзії щодо чудотворного впливу кредиту, позбавлення
благородних металів їхньої монополії, заміну їх паперовими грішми
і~\abbr{т. ін.} Шотландець William Patterson, фундатор Англійського банку та Шотландського
банку, є, справді, Law Перший.

Проти Англійського банку «всі золотарі та позикодавці під заставу почали
люто вити» (Macaulay, History of England, IV, p. 499). — «Протягом перших
10 років банк мав боротися з великими труднощами; велика ворожнеча
зовні; його банкноти приймалось далеко нижче від номінальної вартости\dots{} золотарі
(що в їхніх руках торговля благородними металами була за базу для
примітивної банкової справи) провадили значні інтриґи проти банку, бо через
нього зменшились їхні операції, знизився їхній процент дисконту, а їхні
операції з урядом перейшли до рук цього їхнього супротивника». (J.~Francis,
І. c., р. 37).

Вже перед тим, як засновано Англійський банк, в 1683 році постав плян
організації національного кредитового банку, що його мета між іншим була:
«щоб ділові люди, коли вони мають значну кількість товарів, мали змогу за
підтримкою цього банку покласти свої товари на склад, та одержати певний
кредит під ці складені запаси, давати працю своїм службовцям та збільшувати
своє підприємство, поки знайдуть добрий ринок, замість продавати з втратою».
По довгій праці цей кредитовий банк відкрили в Devonschire House на Bishopsgade
Street'i. Він давав позики промисловцям та купцям векселями під забезпечення
товарами на складах, позичаючи \sfrac{3}{4} вартости тих товарів. Щоб зробити
ці векселі здатними до циркуляції в кожній ділянці підприємств певне число
людей, об’єднувалося в товариство, що від нього кожен державець таких векселів
мав одержувати за них товари так легко, наче він пропонував платіж готівкою
\parbreak{}  %% абзац продовжується на наступній сторінці
