
\index{iii2}{0167}  %% посилання на сторінку оригінального видання

\begin{table}[h]
  \begin{center}
    \emph{Таблиця Xa}
    \footnotesize

  \begin{tabular}{c@{  } c@{  } c@{  } c@{  } c@{  } c@{  } c@{  } c@{  } c@{  } c@{  } с}
    \toprule
      \multirowcell{2}{\makecell{Рід\\ землі}} &
      \multirowcell{2}{Акри} &
      Капітал &
      Зиск &
      \makecell{Ціна\\ продук.} &
      \multirowcell{2}{\makecell{Продукт в\\ квартерах}} &
      \makecell{Продажна \\ ціна} &
      \makecell{Здо-\\буток} &
      \multicolumn{2}{c}{Рента} &
      \multirowcell{2}{Підвищення} \\

      \cmidrule(r){3-3}
      \cmidrule(r){4-4}
      \cmidrule(r){5-5}
      \cmidrule(r){7-7}
      \cmidrule(r){8-8}
      \cmidrule(r){9-9}
      \cmidrule(r){10-10}

       &  & ф. ст. & ф. ст. & ф. ст. & & ф. ст. & ф. ст. & Кварт. & ф. ст. &   \\
      \midrule
      a & 1 & \phantom{2\sfrac{1}{2} + }5\phantom{\sfrac{1}{2}} & 1 & 6 & \phantom{1\sfrac{1}{2} + 3 = }1\sfrac{1}{8}           & 5\sfrac{1}{3} & \phantom{0}6\phantom{\sfrac{1}{5}} & 0\phantom{\sfrac{1}{2}}  & \phantom{0}0\phantom{\sfrac{1}{1}} & 0\phantom{\sfrac{1}{5} + 3 × 7\sfrac{1}{5}} \\
      A & 1 & 2\sfrac{1}{2} + 2\sfrac{1}{2}                     & 1 & 6 & 1 + \phantom{0}\sfrac{1}{4} = 1\sfrac{1}{4}           & 5\sfrac{1}{3} & \phantom{0}6\sfrac{2}{3}           & \phantom{0}\sfrac{1}{8}  & \phantom{00}\sfrac{2}{3}           & \sfrac{2}{3}\phantom{ + 3 × 7\sfrac{1}{5}} \\
      B & 1 & 2\sfrac{1}{2} + 2\sfrac{1}{2}                     & 1 & 6 & 2 + \phantom{0}\sfrac{1}{2} = 2\sfrac{1}{2}           & 5\sfrac{1}{3} & 13\sfrac{1}{3}                     & 1\sfrac{3}{8}            & \phantom{0}7\sfrac{1}{3}           & \sfrac{2}{3} + 6\sfrac{2}{3}\phantom{ 1 ×} \\
      C & 1 & 2\sfrac{1}{2} + 2\sfrac{1}{2}                     & 1 & 6 & 3 + \phantom{0}\sfrac{3}{4} = 3\sfrac{3}{4}           & 5\sfrac{1}{3} & 20\phantom{\sfrac{3}{5}}           & 2\sfrac{5}{8}            & 14\phantom{\sfrac{3}{5}}           & \sfrac{2}{3} + 2 × 6\sfrac{2}{3}\\
      D & 1 & 2\sfrac{1}{2} + 2\sfrac{1}{2}                     & 1 & 6 & 4 + 1\phantom{\sfrac{0}{0}} = 5\phantom{\sfrac{0}{0}} & 5\sfrac{1}{3} & 26\sfrac{2}{3}                     & 3\sfrac{7}{8}            & 20\sfrac{2}{3}                     & \sfrac{2}{3} + 3 × 6\sfrac{2}{3}\\

     \cmidrule(r){5-5}
     \cmidrule(r){6-6}
     \cmidrule(r){8-8}
     \cmidrule(r){9-9}
     \cmidrule(r){10-10}

      Разом & & & & 30 & \phantom{2 + 1\sfrac{1}{2} =}13\sfrac{5}{8} & & 72\sfrac{2}{3} & 8\phantom{\sfrac{1}{2}} & 42\sfrac{2}{3} & \\
  \end{tabular}

  \end{center}
\end{table}

Приєднанням землі \emph{а} породжується нову диференційну ренту І; на цій
новій основі розвивається потім диференційна рента II теж у зміненому вигляді.
Земля \emph{а} має в кожній з трьох вищенаведених таблиць ріжну родючість; ряд
відповідно висхідних ступенів родючости починається лише з А. Відповідно до
цього розміщується і ряд висхідних рент. Рента з найгіршої рентодайної землі,
що раніш ренти не давала, становить постійну величину, яка просто приєднується
до всіх вищих рент; лише за вирахуванням цієї сталої величини ясно виступає
при порівнянні вищих рент ряд ріжниць і його паралелізм з рядом, що
визначає родючість різних земель. У всіх таблицях різні ступені родючости, починаючи
з А до D, стосуються один до одного, як 1: 2 : 3 : 4, і відповідно до
цього ренти стосуються одна до однієї:

\begin{tabular}{l}
в VIIa, як 1 : 1 + 7 : 1 + 2 × 7 : 1 + 3 × 7,\\
в VIIIa, як 1\sfrac{1}{5}:1\sfrac{1}{5} + 7\sfrac{1}{5} : 1\sfrac{1}{5} + 2 × 7\sfrac{1}{5} : 1\sfrac{1}{5} + 3 × 7\sfrac{1}{5},\\
в Xa, як \sfrac{2}{3} : \sfrac{2}{3} + 6\sfrac{2}{3} : \sfrac{2}{3} + 2 × 6\sfrac{2}{3} : \sfrac{2}{3} + 3 × 6\sfrac{2}{3}.\\
\end{tabular}

Коротко: коли рента з А = n, а рента з землі безпосередньо вищої родючости
$= n + m$, то ряд буде такий: $n: n + m: n + 2m : n + З m$ і~\abbr{т. д.} — Ф. Е.]

\pfbreak

[А що вищенаведений третій випадок в рукопису не був опрацьований —
там є лише його заголовок, — то завдання редактора було по змозі доповнити
це, як зроблено вище. Але йому лишається ще зробити загальні висновки, що
випливають з усього попереднього дослідження диференційної ренти II в її трьох
головних випадках і дев’ятьох похідних випадках. Але для цієї мети наведені
в рукопису випадки придаються лише дуже мало. По-перше, в них порівнюються
дільниці землі, що з них здобутки для площ однакової величини стосуються
як 1: 2 : 3 : 4; отже, беруться ріжниці, що вже від самого початку дуже перебільшені,
і які в дальшому розвитку зроблених на цій основі припущень і обчислень
призводять до цілком насильницьких числових відношень. Але подруге,
\parbreak{}  %% абзац продовжується на наступній сторінці
