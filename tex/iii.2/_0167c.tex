\begin{table}[H]
  \centering
  \caption*{Таблиця Xa}

  \footnotesize
  \setlength{\tabcolsep}{4.5pt}
  \settowidth\rotheadsize{\theadfont Продажна}

  \begin{tabular}{l c r c c r c c c c c}
    \toprule
      \thead[tl]{Рід\\землі} &
      &
      \thead[t]{Капітал} &
      \rothead{Зиск} &
      \rothead{Ціна\\продукції} &
      \thead[t]{Продукт} & % \\ в кварт.}}}
      \rothead{Продажна\\ціна} &
      \rothead{Здобуток} &
      \multicolumn{2}{c}{Рента} &
      \thead[t]{Підвищення} \\

    \cmidrule(rl){2-11}
      & акри  & \poundsign{} & \poundsign{} & \poundsign{} & кв. & \poundsign{} & \poundsign{} & кв. & \poundsign{} & \\

    \midrule
      a & 1 & \phantom{2\tbfrac{1}{2} \dplus{} }5\phantom{\tbfrac{1}{2}} & 1 & 6 & \phantom{1\tbfrac{1}{2} \dplus{} 3 \deq{} }1\tbfrac{1}{8}           & 5\tbfrac{1}{3} & \phantom{0}6\phantom{\tbfrac{1}{5}} & 0\phantom{\tbfrac{1}{2}}  & \phantom{0}0\phantom{\tbfrac{1}{1}} & 0\phantom{\tbfrac{1}{5} \dplus{} 3 × 7\tbfrac{1}{5}} \\
      A & 1 & 2\tbfrac{1}{2} \dplus{} 2\tbfrac{1}{2}                     & 1 & 6 & 1 \dplus{} \phantom{0}\tbfrac{1}{4} \deq{} 1\tbfrac{1}{4}           & 5\tbfrac{1}{3} & \phantom{0}6\tbfrac{2}{3}           & \phantom{0}\tbfrac{1}{8}  & \phantom{00}\tbfrac{2}{3}           & \tbfrac{2}{3}\phantom{ \dplus{} 3 × 7\tbfrac{1}{5}} \\
      B & 1 & 2\tbfrac{1}{2} \dplus{} 2\tbfrac{1}{2}                     & 1 & 6 & 2 \dplus{} \phantom{0}\tbfrac{1}{2} \deq{} 2\tbfrac{1}{2}           & 5\tbfrac{1}{3} & 13\tbfrac{1}{3}                     & 1\tbfrac{3}{8}            & \phantom{0}7\tbfrac{1}{3}           & \tbfrac{2}{3} \dplus{} 6\tbfrac{2}{3}\phantom{ 1 ×} \\
      C & 1 & 2\tbfrac{1}{2} \dplus{} 2\tbfrac{1}{2}                     & 1 & 6 & 3 \dplus{} \phantom{0}\tbfrac{3}{4} \deq{} 3\tbfrac{3}{4}           & 5\tbfrac{1}{3} & 20\phantom{\tbfrac{3}{5}}           & 2\tbfrac{5}{8}            & 14\phantom{\tbfrac{3}{5}}           & \tbfrac{2}{3} \dplus{} 2 × 6\tbfrac{2}{3}\\
      D & 1 & 2\tbfrac{1}{2} \dplus{} 2\tbfrac{1}{2}                     & 1 & 6 & 4 \dplus{} 1\phantom{\tbfrac{2}{3}} \deq{} 5\phantom{\tbfrac{2}{3}} & 5\tbfrac{1}{3} & 26\tbfrac{2}{3}                     & 3\tbfrac{7}{8}            & 20\tbfrac{2}{3}                     & \tbfrac{2}{3} \dplus{} 3 × 6\tbfrac{2}{3}\\

    \midrule
      Разом & & & & \hang{r}{3}0 & \phantom{2 \dplus{} 1\tbfrac{1}{2} \deq{}}13\tbfrac{5}{8} & & 72\tbfrac{2}{3} & 8\phantom{\tbfrac{1}{2}} & 42\tbfrac{2}{3} & \\
  \end{tabular}
\end{table}

\noindent{}Приєднанням землі \emph{а} породжується нову диференційну ренту І; на цій
новій основі розвивається потім диференційна рента II теж у зміненому вигляді.
Земля \emph{а} має в кожній з трьох вищенаведених таблиць ріжну родючість; ряд
відповідно висхідних ступенів родючости починається лише з $А$. Відповідно до
цього розміщується і ряд висхідних рент. Рента з найгіршої рентодайної землі,
що раніш ренти не давала, становить постійну величину, яка просто приєднується
до всіх вищих рент; лише за вирахуванням цієї сталої величини ясно виступає
при порівнянні вищих рент ряд ріжниць і його паралелізм з рядом, що
визначає родючість різних земель. У всіх таблицях різні ступені родючости, починаючи
з $А$ до $D$, стосуються один до одного, як $1: 2 : 3 : 4$, і відповідно до
цього ренти стосуються одна до однієї:


\begin{center}
в VIIa, як $1 : 1 \dplus{} 7 : 1 \dplus{} 2 × 7 : 1 \dplus{} 3 × 7$,

в VIIIa, як $1\sfrac{1}{5}:1\sfrac{1}{5} \dplus{} 7\sfrac{1}{5} : 1\sfrac{1}{5} \dplus{} 2 × 7\sfrac{1}{5} : 1\sfrac{1}{5} \dplus{} 3 × 7\sfrac{1}{5}$,

в Xa, як $\sfrac{2}{3} : \sfrac{2}{3} \dplus{} 6\sfrac{2}{3} : \sfrac{2}{3} \dplus{} 2 × 6\sfrac{2}{3} : \sfrac{2}{3} \dplus{} 3 × 6\sfrac{2}{3}$.

\end{center}

\noindent{}Коротко: коли рента з $А \deq{} n$, а рента з землі безпосередньо вищої родючости
$= n \dplus{} m$, то ряд буде такий: $n: n \dplus{} m: n \dplus{} 2m : n \dplus{} З m$ і~\abbr{т. д.} —~Ф.~Е.]

\pfbreak

[А що вищенаведений третій випадок в рукопису не був опрацьований —
там є лише його заголовок, — то завдання редактора було по змозі доповнити
це, як зроблено вище. Але йому лишається ще зробити загальні висновки, що
випливають з усього попереднього дослідження диференційної ренти II в її трьох
головних випадках і дев’ятьох похідних випадках. Але для цієї мети наведені
в рукопису випадки придаються лише дуже мало. Поперше, в них порівнюються
дільниці землі, що з них здобутки для площ однакової величини стосуються
як $1: 2 : 3 : 4$; отже, беруться ріжниці, що вже від самого початку дуже перебільшені,
і які в дальшому розвитку зроблених на цій основі припущень і обчислень
призводять до цілком насильницьких числових відношень. Але подруге,
\parbreak{}  %% абзац продовжується на наступній сторінці
