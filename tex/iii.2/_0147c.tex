\parcont{}  %% абзац починається на попередній сторінці
\index{iii2}{0147}  %% посилання на сторінку оригінального видання
давніш 6\pound{ ф. стерл.} ренти до 2\sfrac{1}{2}\pound{ ф. стерл.} капіталу. Тут не виникає нових
ріжниць між витраченими капіталами, але виникають нові надзиски тільки
тому, що додатковий капітал вкладається в якусь з земель, що дають ренту,
або в усі землі, даючи при цьому пропорційно своїй величині той самий
продукт. Коли б подвійна витрата капіталу була зроблена, наприклад, лише
на $C$, то диференційна рента між $C$, $В$ і $D$, обчислена на капітал, залишалася б
та сама; бо хоч її маса з $C$ і подвоїлася б, але подвоївся б і вкладений
капітал.

Звідси видно, що за незмінної ціни продукції, незмінної норми зиску і
незмінних ріжниць (а тому і за незмінної норми надзиску або ренти, обчислених
на капітал), висота ренти, визначеної в продукті і в грошах, може підвищитись
з акра, а тому може підвищитись і ціна землі.

Те саме може статися при зменшуваних нормах надзиску, отже, і ренти,
тобто при зменшуваній продуктивності додаткових вкладень капіталу, що все
ще дають ренту. Коли б другі вкладення капіталу в 2\sfrac{1}{2}\pound{ ф. стерл.} не дали
подвоєного продукту, а дали б на $В$ лише З\sfrac{1}{2} квартери, на $C$ — 5 і на $D$ —
6 квартерів, то диференційна рента на $В$ для другого вкладення капіталу в 2\sfrac{1}{2}\pound{ф. стерл.}
була б лише \sfrac{1}{2} квартера замість 1, на C — 1 замість 2 і на $D$ — 2 замість
3 квартерів. Відношення між рентою і капіталом для обох послідовних
витрат було б таке:

\begin{table}[H]
  \centering
  \small
  \begin{tabular}{l l l}

  & Перша витрата & Друга витрата \\

$В$: & Рента 3\pound{ ф. стерл.}, капітал 2\sfrac{1}{2}\pound{ ф. стерл.}
      & Рента 1\sfrac{1}{2}\pound{ ф. стерл.}, капітал 2\sfrac{1}{2}\pound{ ф. стерл.} \\

$C$: & \ditto{Рента} 6\ditto{\pound{ ф. стерл.}, капітал} 2\sfrac{1}{2}
      & \ditto{Рента} З\phantom{\sfrac{1}{2}}\ditto{\pound{ ф. стерл.}, капітал} 2\sfrac{1}{2} \\

$D$: & \ditto{Рента} 9\ditto{\pound{ ф. стерл.}, капітал} 2\sfrac{1}{2}
      & \ditto{Рента} 6\phantom{\sfrac{1}{2}}\ditto{\pound{ ф. стерл.}, капітал} 2\sfrac{1}{2} \\
  \end{tabular}
\end{table}

\noindent{}Не зважаючи на таку понижену норму відносної продуктивности капіталу,
а тому і надзиску, обчисленого на капітал, збіжжева і грошова рента підвищилася
б для $В$ з 1 до 1\sfrac{1}{2} квартерів (з 3 до 4\sfrac{1}{2}\pound{ ф. стерл.}), для $C$ з 2 до 3 квартерів (з 6 до 9\pound{ ф. стерл.}) і для $D$ з 3 до 5 квартерів (з 9 до 15\pound{ ф. стерл.})
В цьому випадку ріжниці для додаткових капіталів порівняно з капіталом,
вкладеним в $А$, зменшилися б, ціна продукції лишилася б та сама, але рента
на акр, а тому і ціна землі на акр підвищилася б.

Щодо комбінацій диференційної ренти II, що має за свою передумову, як
свою базу диференційну ренту І, то вони такі.

\section{Диференційна рента II.~Перший випадок: стала ціна продукції}

Таке припущення включає й те, що ринкова ціна, як і давніше, регулюється
капіталом, вкладеним в найгіршу землю $А$.

I.~Коли додатковий капітал, вкладений в якусь із земель $В$, $C$, $D$, що
дають ренту, продукує лише стільки, скільки продукує такий самий капітал на
землі $А$, тобто коли при регуляційній ціні продукції він дає лише пересічний
зиск, не даючи, отже, жодного надзиску, то вплив справлений ним на ренту, дорівнює
нулеві. Все лишається, як було давніш. Це те саме, як коли б перше-ліпше
число акрів якости $А$, найгіршої землі, було приєднано до вже оброблюваної
площі.

II.~Додаткові капітали дають на землях усіх родів додаткові продукти в кількості,
пропорційній величині цих капіталів; тобто — величина продукції зростає,
\parbreak{}  %% абзац продовжується на наступній сторінці
