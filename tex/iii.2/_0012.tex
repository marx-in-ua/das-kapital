\parcont{}  %% абзац починається на попередній сторінці
\index{iii2}{0012}  %% посилання на сторінку оригінального видання
боку, як уже згадано, їх визичається у формі капіталу, що дає процент, отже
вони не лежать в касах банків, а фігурують тільки в їхніх книгах, як їх боргові
зобов’язання щодо вкладників. З другого боку, вони функціонують як такі прості
записи по книгах, оскільки боргові зобов’язання вкладників навзаєм вирівнюються
чеками на їхні вклади та навзаєм списуються; при цьому цілком байдуже,
чи ті вклади лежать у того самого банкіра, так що він сам змінює записи
в різних рахунках, чи це роблять різні банки, що навзаєм вимінюють свої
чеки, виплачуючи один одному тільки ріжниці.

З розвитком капіталу, що дає процент, та кредитової системи ввесь капітал
здається подвоєним, а подекуди навіть потроєним в наслідок різних способів,
що ними той самий капітал або й лише та сама боргова вимога в різних
руках з’являється під різними формами.\footnote{
[Це подвоєння та потроєння капіталу розвинулося за останні роки значно далі, напр., через
фінансові трести, що вже мають окрему рубрику в лондонських біржових звітах. Утворюється товариство
для купівлі певного роду процентних паперів, напр., закордонних державних паперів, англійських
міських або американських державних боргових посвідок, залізничних акцій і~\abbr{т. ін.} Капітал,
напр., в 2 мільйони ф. ст. складається підпискою на акції; дирекція купує відповідні цінності, або
більш чи менш жваво спекулює ними та розподіляє річну суму процентів як дивіденд. між акційниками,
відрахувавши видатки. — Далі, у поодиноких акційних товариств постав звичай ділити звичайні акції
на дві кляси, preferred та deferred. Акції preferred одержують стале опроцевтовання, напр., 5\%,
припускаючи, що цілий зиск дає змогу це робити; коли ж потім лишається ще деяка зайвина, то її
одержують акції deferred. От таким способом «солідні» капіталовкладення в preferred більш або менш
відокремлюються від власної спекуляції, що панує в deferred. А що поодинокі великі підприємства
не, хочуть скоритися цій новій моді, то і сталося, що утворювалися товариства, які приміщують один
або декілька мільйонів ф. ст. в акціях тих підприємств, а потім на номінальну вартість цих акцій
видають
нові акції, але вже одну половину preferred, а другу deferred. Кількість первісних акцій подвоюється
в цьому разі, бо вони є за підставу до нового видання акцій. — Ф. E.]
} Найбільша частина цього «грошового
капіталу» є суто-фіктивна. Усі вклади, крім запасного фонду, є не що інше,
як тільки боргові вимоги до банкіра, але вони ніколи не лежать у нього. Оскільки
вони придаються до переказових операцій, вони функціонують, як капітал для
банкірів після того, як ці їх визичили. Банкіри платять один одному взаємними
переказами на неіснуючі вклади, навзаєм списуючи ці вимоги.

А. Сміт каже так про ролю, що її відіграє капітал при визичанні грошей:
«Навіть у грошових операціях готівкою гроші однак є лише посвідка, що переносить
з одних рук до інших ті капітали, для котрих їхні власники не знаходять
жодного вжитку. Ці капітали можуть бути майже необмежено більші за ту
грошову суму, що є за знаряддя їхньої передачі; ті самі монети одна по одній
служать в багатьох різних позиках, так само як і в багатьох різних купівлях.
Напр., А позичає V 1000 ф. ст., що на них W одразу ж купує в В товарів
на 1000 ф. ст. ІЦо В сам не має жодного вжитку для тих грошей, то й визичає
він X ті самісінькі монети, що на них X знову негайно купує в С товарів
на 1000 ф. ст. Тим самим способом та з тієї самої причини С визичає гроші
Y, що знову купує на них товари в D. Оттак ті самі золоті монети або папери
можуть протягом небагатьох днів упосереднювати три різні позики та три різні
купівлі, що з них кожна вартістю дорівнює цілій сумі цих монет. Що саме ті троє
власників грошей А, В й С передали тим трьом позикоємцям W, X та Y, так це
спроможність робити ці купівлі. У цій спроможності є так цінність, як і користь цих
позик. Капітал, позичений цими трьома власниками грошей, дорівнює вартості
товарів, що їх можна купити на ті гроші, і втроє більший за вартість грошей,
що на них робиться ті купівлі. Проте всі ці позики можуть бути цілком забезпечені,
бо товари, куплені різними винуватцями на ті позики, так ужито, що
вони у свій час повертають ту саму вартість у золотих або паперових грошах
разом з зиском. І так само, як ті самі монети можуть упосереднювати різні позики,
утроє або навіть утридцятеро більшої вартости, ніж вони сами, так
\parbreak{}  %% абзац продовжується на наступній сторінці
