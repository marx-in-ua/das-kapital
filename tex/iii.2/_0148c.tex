\parcont{}  %% абзац починається на попередній сторінці
\index{iii2}{0148}  %% посилання на сторінку оригінального видання
залежно від специфічної родючости землі кожного типу, пропорційно величині
додаткового капіталу. В XXXIX розділі ми виходили з такої таблиці І:

\begin{table}[H]
  \centering
  \caption*{Таблиця І}
  \footnotesize

  \settowidth\rotheadsize{\theadfont Продажна}
  \begin{tabular}{l c r c c c c c c c c}
    \toprule
      \thead[tl]{Рід\\землі} &
      &
      \rothead{Капітал} &
      \rothead{Зиск} &
      \rothead{Ціна\\продукції} &
      \rothead{Продукт} & % \\ в кварт.}}} \\ в кварт.}}}
      \rothead{Продажна\\ціна} &
      \rothead{Здобуток} &
      \multicolumn{2}{c}{Рента} &
      \rothead{Норма\\надзиску} \\

      \cmidrule(rl){2-11}

       & акри  & \makecell{\poundsign{}} & \poundsign{} & \poundsign{} & кв. & \poundsign{} & \poundsign{} & кв. & \poundsign{}  & \% \\
      \midrule

      A & 1 &  \phantom{0}2\tbfrac{1}{2} & \tbfrac{1}{2} & \phantom{0}3 & \phantom{0}1 & 3 & \phantom{0}3 & 0 & \phantom{0}0 & \phantom{00}0\\
      B & 1 &  \phantom{0}2\tbfrac{1}{2} & \tbfrac{1}{2} & \phantom{0}3 & \phantom{0}2 & 3 & \phantom{0}6 & 1 & \phantom{0}3 & 120 \\
      C & 1 &  \phantom{0}2\tbfrac{1}{2} & \tbfrac{1}{2} & \phantom{0}3 & \phantom{0}3 & 3 & \phantom{0}9 & 2 & \phantom{0}6 & 240 \\
      D & 1 &  \phantom{0}2\tbfrac{1}{2} & \tbfrac{1}{2} & \phantom{0}3 & \phantom{0}4 & 3 & 12           & 3 & \phantom{0}9 & 360 \\
     \midrule

     Разом & 4 & \hang{r}{1}0\pF{} & & 12 & 10 & & 30 & 6 & 18 &\\
  \end{tabular}
\end{table}

% REMOVED: 
%\hang{l}{\footnotemarkZ{}}
% \footnotetextZ{В німецькому тексті тут стоїть «12\%, 24\%, 36\%». Очевидна помилка. % \emph{Прим. Ред.}}

\noindent{}Тепер ця таблиця перетворюється на:

\begin{table}[H]
  \centering
  \caption*{Таблиця ІІ}
  \footnotesize

  \settowidth\rotheadsize{\theadfont Продажна}
  \begin{tabular}{l c r c c c c c c c c}
    \toprule
      \thead[tl]{Рід\\землі} &
      &
      \thead[t]{Капітал} &
      \rothead{Зиск} &
      \rothead{Ціна\\продукції} &
      \rothead{Продукт} & % \\ в кварт.}}} \\ в кварт.}}}
      \rothead{Продажна\\ціна} &
      \rothead{Здобуток} &
      \multicolumn{2}{c}{Рента} &
      \rothead{Норма\\надзиску} \\

      \cmidrule(rl){2-11}

       & акри  & \poundsign{} & \poundsign{} & \poundsign{} & кв. & \poundsign{} & \poundsign{} & кв. & \poundsign{}  & \% \\
      \midrule

      A & 1 & 2\tbfrac{1}{2} \dplus{} 2\tbfrac{1}{2} \deq{} 5 & 1 & 6 & \phantom{0}2 & 3 & \phantom{0}6 & \phantom{0}0 & \phantom{0}0 & \phantom{00}0 \\
      B & 1 & 2\tbfrac{1}{2} \dplus{} 2\tbfrac{1}{2} \deq{} 5 & 1 & 6 & \phantom{0}4 & 3 & 12           & \phantom{0}2 & \phantom{0}6 & 120 \\ % ця мітка у заголовку \\
      C & 1 & 2\tbfrac{1}{2} \dplus{} 2\tbfrac{1}{2} \deq{} 5 & 1 & 6 & \phantom{0}6 & 3 & 18           & \phantom{0}4 & 12 & 240\\
      D & 1 & 2\tbfrac{1}{2} \dplus{} 2\tbfrac{1}{2} \deq{} 5 & 1 & 6 & \phantom{0}8 & 3 & 25           & \phantom{0}6 & 18 & 360\\
     \midrule

     Разом & 4 & 20 & & & 20 & & 60 & 12 & 36 & \\
  \end{tabular}
\end{table}

\noindent{}Тут немає потреби в тому, щоб капітал вкладати у кожний з типів землі
в подвоєному розмірі, як це є в таблиці. Закон лишається той самий, скоро
тільки на якийсь один або декілька родів землі, що дають ренту, вжито додатковий капітал, хоч би в
якому розмірі. Треба лише, щоб продукція на землях
кожного роду збільшувалася в тому самому відношенні, в якому збільшується
капітал. Рента підвищується тут виключно в наслідок збільшення вкладеного
в землю капіталу і відповідно до цього збільшення капіталу. Це збільшення
продукту і ренти, в наслідок збільшення вкладеного капіталу і пропорційно
йому, є,  щодо кількости продукту і ренти, цілком таке саме, як у тому випадку,
коли оброблювана площа рівних за якістю дільниць землі, що дають ренту,
збільшилася б, обробляючись з такою самою витратою капіталу, з якою давніш оброблялись земельні
дільниці тієї самої якости. В випадку, поданому в таблиці II, наприклад, наслідок був би той самий,
коли б додатковий капітал
в 2\sfrac{1}{2}\pound{ ф. стерл.} на акр було вкладено в другі акри земель $B$, $C$ і $D$.
