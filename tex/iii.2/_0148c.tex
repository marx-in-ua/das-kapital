\parcont{}  %% абзац починається на попередній сторінці
\index{iii2}{0148}  %% посилання на сторінку оригінального видання
залежно від специфічної родючости землі кожного типу, пропорційно величині
додаткового капіталу. В XXXIX розділі ми виходили з такої таблиці І:

\begin{table}[h]
  \begin{center}
    \emph{Таблиця І}
    \footnotesize

  \begin{tabular}{c c c c c c c c c c c}
    \toprule
      \multirowcell{2}{\makecell{Рід \\землі}} &
      \multirowcell{2}{\rotatebox[origin=c]{90}{Акри}} &
      \rotatebox[origin=c]{90}{Капітал} &
      \rotatebox[origin=c]{90}{Зиск} &
      \rotatebox[origin=c]{90}{\makecell{Ціна про- \\ дукції}} &
      \multirowcell{2}{\rotatebox[origin=c]{90}{\makecell{Продукт \\ в кварт.}}} &
      \rotatebox[origin=c]{90}{\makecell{Продажна \\ ціна}} &
      \rotatebox[origin=c]{90}{Здобуток} &
      \multicolumn{2}{c}{Рента} &
      \multirowcell{2}{\makecell{Норма \\надзиску}} \\

      \cmidrule(rl){3-3}
      \cmidrule(rl){4-4}
      \cmidrule(rl){5-5}
      \cmidrule(rl){7-7}
      \cmidrule(rl){8-8}
      \cmidrule(rl){9-10}

       &  &  ф. ст. & ф. ст. & ф. ст. & & ф. ст. & ф. ст. & Кварт. & ф. ст. &  \\
      \midrule

      A & 1 &  \phantom{0}2\sfrac{1}{2} & \sfrac{1}{2} & \phantom{0}3 & \phantom{0}1 & 3 & \phantom{0}3 & 0 & \phantom{0}0 & \phantom{00}0\\
      B & 1 &  \phantom{0}2\sfrac{1}{2} & \sfrac{1}{2} & \phantom{0}3 & \phantom{0}2 & 3 & \phantom{0}6 & 1 & \phantom{0}3 & 120\% \footnotemarkZ{}\\ % ця мітка у заголовку \\
      C & 1 &  \phantom{0}2\sfrac{1}{2} & \sfrac{1}{2} & \phantom{0}3 & \phantom{0}3 & 3 & \phantom{0}9 & 2 & \phantom{0}6 & 240\%\\
      D & 1 &  \phantom{0}2\sfrac{1}{2} & \sfrac{1}{2} & \phantom{0}3 & \phantom{0}4 & 3 & 12           & 3 & \phantom{0}9 & 360\%\\
     \cmidrule(rl){1-1}
     \cmidrule(rl){2-2}
     \cmidrule(rl){3-3}
     \cmidrule(rl){5-5}
     \cmidrule(rl){6-6}
     \cmidrule(rl){8-8}
     \cmidrule(rl){9-9}
     \cmidrule(rl){10-10}

     Разом & 4 & 10 & & 12 & 10 & & 30 & 6 & 18 &\\
  \end{tabular}

  \end{center}
\end{table}
\footnotetextZ{В німецькому тексті тут стоїть «12\%, 24\%, 36\%». Очевидна помилка. \emph{Прим. Ред.}} % текст примітки прямо під заголовком

Тепер ця таблиця перетворюється на:
\begin{table}[h]
  \begin{center}
    \emph{Таблиця ІI}
    \footnotesize

  \begin{tabular}{c c c c c c c c c c c}
    \toprule
      \multirowcell{2}{\makecell{Рід \\землі}} &
      \multirowcell{2}{\rotatebox[origin=c]{90}{Акри}} &
      Капітал &
      \rotatebox[origin=c]{90}{Зиск} &
      \rotatebox[origin=c]{90}{\makecell{Ціна про- \\ дукції}} &
      \multirowcell{2}{\rotatebox[origin=c]{90}{\makecell{Продукт \\ в кварт.}}} &
      \rotatebox[origin=c]{90}{\makecell{Продажна \\ ціна}} &
      \rotatebox[origin=c]{90}{Здобуток} &
      \multicolumn{2}{c}{Рента} &
      \multirowcell{2}{\rotatebox[origin=c]{90}{\makecell{Норма \\ надзиску}}} \\

      \cmidrule(rl){3-3}
      \cmidrule(rl){4-4}
      \cmidrule(rl){5-5}
      \cmidrule(rl){7-7}
      \cmidrule(rl){8-8}
      \cmidrule(rl){9-10}

       &  &  ф. ст. & ф. ст. & ф. ст. & & ф. ст. & ф. ст. & Кварт. & ф. ст. &  \\
      \midrule

      A & 1 & 2\sfrac{1}{2} \dplus{} 2\sfrac{1}{2} \deq{} 5 & 1 & 6 & \phantom{0}2 & 3 & \phantom{0}6 & \phantom{0}0 & \phantom{0}0 & \phantom{00}0\phantom{\%}\\
      B & 1 & 2\sfrac{1}{2} \dplus{} 2\sfrac{1}{2} \deq{} 5 & 1 & 6 & \phantom{0}4 & 3 & 12           & \phantom{0}2 & \phantom{0}6 & 120\% \\ % ця мітка у заголовку \\
      C & 1 & 2\sfrac{1}{2} \dplus{} 2\sfrac{1}{2} \deq{} 5 & 1 & 6 & \phantom{0}6 & 3 & 18           & \phantom{0}4 & 12 & 240\%\\
      D & 1 & 2\sfrac{1}{2} \dplus{} 2\sfrac{1}{2} \deq{} 5 & 1 & 6 & \phantom{0}8 & 3 & 25           & \phantom{0}6 & 18 & 360\%\\
     \cmidrule(rl){1-1}
     \cmidrule(rl){2-2}
     \cmidrule(rl){3-3}
     \cmidrule(rl){6-6}
     \cmidrule(rl){8-8}
     \cmidrule(rl){9-9}
     \cmidrule(rl){10-10}

     Разом & 4 & \phantom{2\sfrac{1}{2} \dplus{} 2\sfrac{1}{2} \deq{}}20 & & & 20 & & 60 & 12 & 36 &\\
  \end{tabular}

  \end{center}
\end{table}

Тут немає потреби в тому, щоб капітал вкладати у кожний з типів землі
в подвоєному розмірі, як це є в таблиці. Закон лишається той самий, скоро
тільки на якийсь один або декілька родів землі, що дають ренту, вжито додатковий капітал, хоч би в
якому розмірі. Треба лише, щоб продукція на землях
кожного роду збільшувалася в тому самому відношенні, в якому збільшується
капітал. Рента підвищується тут виключно в наслідок збільшення вкладеного
в землю капіталу і відповідно до цього збільшення капіталу. Це збільшення
продукту і ренти, в наслідок збільшення вкладеного капіталу і пропорційно
йому, є,  щодо кількости продукту і ренти, цілком таке саме, як у тому випадку,
коли оброблювана площа рівних за якістю дільниць землі, що дають ренту,
збільшилася б, оброблючись з такою самою витратою капіталу, з якою давніш оброблялись земельні
дільниці тієї самої якости. В випадку, поданому в таблиці II, наприклад, наслідок був би той самий,
коли б додатковий капітал,
в 2\sfrac{1}{2}\pound{ ф. стерл.} на акр було вкладено в другі акри земель $B$, $C$ і $D$.
