\parcont{}  %% абзац починається на попередній сторінці
\index{iii2}{0073}  %% посилання на сторінку оригінального видання
це мірило є більш або менш точне, це, природно, залежить перш за все від
того, оскільки банкову справу взагалі централізовано. Бо від цього залежить,
оскільки нагромаджений в, так званому, національному банку благородний метал
являє взагалі національний металевий скарб. Але, коли припустити що це є так,
все ж це мірило не є точне тому, що додатковий довіз серед певних обставин
вбирається внутрішньою циркуляцією та чимраз більшим ужитком золота та
срібла для речей розкошів; а крім того й тому, що без додаткового довозу
втягувалось би золоту монету з банку до внутрішньої циркуляції, і таким
чином золотий скарб міг би зменшуватися теж без одночасного збільшення вивозу.

\emph{Почетверте}. Вивіз металу набирає форми відпливу (drain), коли рух
його зменшення триває протягом довшого часу, так що це зменшення виявляється
як тенденція руху, й знижує металевий запас банку значно нижче від його пересічної
висоти, аж до рівня середнього мінімуму цього запасу. Цей останній усталюється
більш або менш свавільно, оскільки законодавство про покриття банкнот і~\abbr{т. ін.}
металевою готівкою в кожному поодинокому випадку різно визначає той запас.
Про кількісні межі, що їх може досягти такий відплив в Англії, Newmarch
каже перед В.~А. 1857, Evid. № 1494: «Коли керуватися досвідом, то дуже
неймовірна річ, щоб відплив металу в наслідок будь-якого коливання в закордонних
операціях перевищив 3 або 4 мільйони ф. ст.». В 1847 році найнижчий
рівень золотого запасу Англійського банку був 23 жовтня, на \num{5.198.156}\pound{ ф. ст.}
менший проти 26 грудня 1846 року та на \num{6.453.748}\pound{ ф. ст.} менший проти найвищого
рівня його в 1846 році (29 серпня).

\emph{Поп’яте}. Призначення металевого запасу так званого національного
банку, призначення, що однак само лише ніяк не реґулює величини металевого
скарбу, бо він може зростати з причини самого тільки припинення внутрішніх
та зовнішніх операцій, — це призначення є трояке: 1)~запасний фонд для міжнародніх
платежів, або коротше запасний фонд світових грошей. 2)~Запасний
фонд для внутрішньої металевої циркуляції, що, періодично, то ширшає, то
вужчає. 3)~Запасний фонд для сплати вкладів та розміну банкнот, який має
зв’язок з банковою функцією та не має нічого до діла з функціями грошей, як
просто грошей. Тому на запасний фонд можуть впливати ті умови, що зачіпають
кожну поодиноку з цих трьох функцій; отже, як на інтернаціональний
фонд на нього може вплинути платіжний балянс, хоч і якими причинами
той балянс не визначався б і хоч яке не було б відношення того балянсу
до балянсу торговельного; як на запасний фонд внутрішньої металевої циркуляції
на нього може впливати пошир або скорочення тієї циркуляції. Третя
функція його як функція ґарантійного фонду, хоч і не визначає самостійного
руху металевого запасу, впливає однак двояко. Коли видають банкноти, що мають
замінити металеві гроші (отже, й срібні монети по країнах, де за мірило вартости
є срібло) у внутрішній циркуляції, то функція запасного фонду означена вище
під пунктом 2) відпадає. І певна частина благородного металу, що виконувала
цю функцію, перейде на довгий час закордон. В цьому разі не витягується металевої монети з запасу
банка для внутрішньої циркуляції й тому разом з тим відпадає
потреба тимчасово посилювати металевий запас, імобілізуючи частину металевої
монети, що є в циркуляції. Далі: коли треба серед усяких обставин тримати певний
мінімум металевого скарбу для оплати вкладів та для розміну банкнот, то це
своєрідно позначається на наслідках відпливу або припливу золота; це впливає
на ту частину скарбу, що її банк зобов’язаний тримати серед усяких обставин,
або на ту частину, що її іншого часу банк силкується позбутися як некорисної.
За суто металевої циркуляції та концентрації банкової справи банк мав би
розглядати свій металевий скарб теж як ґарантію для оплати вкладів в нього,
й при відпливі металу могла б постати така сама паніка, як та, що була в Гамбурзі
в 1857 році.
