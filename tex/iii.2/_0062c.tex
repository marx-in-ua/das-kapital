\parcont{}  %% абзац починається на попередній сторінці
\index{iii2}{0062}  %% посилання на сторінку оригінального видання
реґулюватись імпортом та експортом благородних металів або вексельним курсом.
Помилкове припущення Рікардо, ніби золото є тільки монета, що отже все імпортоване
золото збільшує кількість грошей в циркуляції, а тому й підносить ціни,
а все експортоване золото зменшує кількість монет, а тому й знижує ціни —
це теоретичне припущення стає тут \emph{практичною спробою лишати стільки
монети в циркуляції, скільки є в наявності золота в кожний даний
момент}. Лорд Оверстон (банкір Jones Loyd), полковник Torrens, Normam, Clay,
Arbuthnot та деякі інші письменники, відомі в Англії під назвою школи «Currency
Principle», не тільки проповідували цю доктрину, але й за допомогою банкових
актів сера Піля з 1844 та 1845 р. р. зробили її основою англійського та шотляндського
банкового законодавства. Її ганебне теоретичне й практичне фіяско після
експериментів найбільшого національного маштабу ми зможемо розглянути лише
в науці про кредит» .

Критику цієї школи подали Томас Тук, Джемс Вілсон (в Economist’i 1844--47 р. р.)
та Джон Фулартон. Але як хибно також і вони розуміли природу золота
та як неясно було їм відношення між грішми та капіталом, це ми бачили не
раз, а саме в розділі XXVIII цієї книги. Тут подамо ще деякі матеріяли з дебатів
комісії нижчої палати в 1857 році про банкові акти Піля. (В. С. 1857) — Ф. Е.].

I. G. Hubbard, колишній управитель Англійського банку, свідчить: «2400. —
Вивіз золота\dots{} абсолютно не впливає на товарові ціни. Навпаки, він дуже
значно впливає на ціни цінних паперів, бо в міру того, як змінюється рівень
проценту, неминуче справляється незвичайний вплив на вартість товарів, що втілюють
цей процент». Він подає про роки 1834--43 та 1844--56 дві таблиці,
які доводять, що рух цін п’ятнадцятьох найзначніших торговельних речей був
цілком незалежний від відпливу та припливу золота й від рівня проценту. Але
зате ці таблиці доводять щільний зв’язок між відпливом та припливом золота,
що дійсно «є представник нашого капіталу, який шукає приміщення», і рівнем
проценту. — «В 1847 році дуже велику суму американських цінних паперів переслали
назад до Америки, так само й російські цінні папери — до Росії, а інші
континентальні папери — до тих країн, звідки ми імпортували збіжжя».

15 головних товарів, покладених в основу далі поданих таблиць Hubbard’a,
є такі: бавовна, бавовняна пряжа, бавовняні тканини, вовна, сукно,
льон, полотно, індиґо, чавун, бляха, мідь, волове сало, цукор, кава, шовк.
\begin{table}[h]
  \begin{center}
  \caption*{І. 1834\textendash{}1843}
\begin{tabular} {l r c c c c}
  \toprule
      \multirowcell{2}{\makecell{Час}} &
      \multirowcell{2}{\makecell{Металева\\ готівка\\ банку\\ ф. ст.}} &
      \multirowcell{2}{\makecell{Ринкова\\ норма \\ дисконту \\ в\%}} &
      \multicolumn{3}{c}{З 15 головних товарів} \\
    \cmidrule(l){4-6}

    & & & \makecell{Піднеслися \\ ціною } & Впали & Без змін \\
    %TODO Щось пішло не так, довелось додати пустий рядок
    & & & & & \\
    \midrule
1834, 1    березня  & 9104000  &  2\sfrac{3}{4} & \textemdash & \textemdash & \textemdash \\
1835, 1    \ditto{березня}        & 6274000  &  3\sfrac{3}{4} &   7           &         7     &   1 \\
1836, 1    \ditto{березня}         & 7918000  &  3\sfrac{1}{4} &   11          &         3     &   1 \\
1837, 1    \ditto{березня}         & 4079000  &  5 \phantom{\sfrac{1}{4}}            &   5           &         9     &   1 \\
1838, 1    \ditto{березня}         & 10471000 &  2\sfrac{3}{4} &   4           &        11     &   \textemdash \\
1839, 1    вересня  & 2684000  &  6 \phantom{\sfrac{1}{4}}             &   8           &         5     &  2 \\
1840, 1    червня   & 4571000  &  4\sfrac{3}{4} &   5           &         9     &   1 \\
1840, 1    грудня   & 3642000  &  5\sfrac{3}{4} &   7           &         6     &  2 \\
1841, 1    \ditto{грудня}        & 4873000  &  5 \phantom{\sfrac{1}{4}}             &   3           &        12     &  \textemdash \\
1842, 1    \ditto{грудня}        & 10603000 &  2\sfrac{1}{2} &   2           &        13     &  \textemdash \\
1843, 1    червня   & 11566000 &  2\sfrac{1}{4} &   1           &        14     &  \textemdash \\
\end{tabular}
\end{center}
\end{table}

