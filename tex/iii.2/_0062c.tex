\parcont{}  %% абзац починається на попередній сторінці
\index{iii2}{0062}  %% посилання на сторінку оригінального видання
реґулюватись імпортом та експортом благородних металів або вексельним курсом.
Помилкове припущення Рікардо, ніби золото є тільки монета, що отже все імпортоване
золото збільшує кількість грошей в циркуляції, а тому й підносить ціни,
а все експортоване золото зменшує кількість монет, а тому й знижує ціни —
це теоретичне припущення стає тут \emph{практичною спробою лишати стільки
монет в циркуляції, скільки є в наявності золота в кожний даний
момент}. Лорд Оверстон (банкір Jones Loyd), полковник Torrens, Normam, Clay,
Arbuthnot та деякі інші письменники, відомі в Англії під назвою школи «Currency
Principle», не тільки проповідували цю доктрину, але й за допомогою банкових
актів сера Піля з 1844 та 1845~\abbr{рр.} зробили її основою англійського та шотляндського
банкового законодавства. Її ганебне теоретичне й практичне фіяско після
експериментів найбільшого національного маштабу ми зможемо розглянути лише
в науці про кредит».

\looseness=1
Критику цієї школи подали Томас Тук, Джемс Вілсон (в Economist’i 1844--47~\abbr{рр.})
та Джон Фулартон. Але як хибно також і вони розуміли природу золота
та як неясно було їм відношення між грішми та капіталом, це ми бачили не
раз, а саме в розділі XXVIII цієї книги. Тут подамо ще деякі матеріяли з дебатів
комісії нижчої палати в 1857 році про банкові акти Піля. (В.~C. 1857) — \emph{Ф.~Е.}].

\looseness=1
I.~G.~Hubbard, колишній управитель Англійського банку, свідчить: «2400. —
Вивіз золота\dots{} абсолютно не впливає на товарові ціни. Навпаки, він дуже
значно впливає на ціни цінних паперів, бо в міру того, як змінюється рівень
проценту, неминуче справляється незвичайний вплив на вартість товарів, що втілюють
цей процент». Він подає про роки 1834--43 та 1844--56 дві таблиці,
які доводять, що рух цін п’ятнадцятьох найзначніших торговельних речей був
цілком незалежний від відпливу та припливу золота й від рівня проценту. Але
зате ці таблиці доводять щільний зв’язок між відпливом та припливом золота,
що дійсно «є представник нашого капіталу, який шукає приміщення», і рівнем
проценту. — «В 1847 році дуже велику суму американських цінних паперів переслали
назад до Америки, так само й російські цінні папери — до Росії, а інші
континентальні папери — до тих країн, звідки ми імпортували збіжжя».

15 головних товарів, покладених в основу далі поданих таблиць Hubbard’a,
є такі: бавовна, бавовняна пряжа, бавовняні тканини, вовна, сукно,
льон, полотно, індиґо, чавун, бляха, мідь, волове сало, цукор, кава, шовк.

\begin{table}[H]
  \centering
  \caption*{І. 1834\textendash{}1843}
\begin{tabular} {l r c c c c}
  \toprule
      \multirowcell{2}[-2ex][l]{Час} &
      \multirowcell{2}[0.5ex][c]{\makecell{Металева\\ готівка\\ банку, \pound{ф. ст.}}} &
      \multirowcell{2}[0.5ex][c]{\makecell{Ринкова\\ норма \\ дисконту, \%}} &
      \multicolumn{3}{c}{З 15 головних товарів} \\
    \cmidrule(l){4-6}
    & & & \makecell{Піднеслися \\ ціною } & Впали & Без змін \\
    \midrule
1834, 1  березня  & \num{9.104.000}  &  2\tbfrac{3}{4} & \textemdash & \textemdash & \textemdash \\
1835, 1  березня   & \num{6.274.000}  &  3\tbfrac{3}{4} & 7     &   7   & 1 \\
1836, 1  березня   & \num{7.918.000}  &  3\tbfrac{1}{4} & \hang{r}{1}1    &   3   & 1 \\
1837, 1  березня   & \num{4.079.000}  &  5\phantom{\tbfrac{1}{4}}    & 5     &   9   & 1 \\
1838, 1  березня   & \num{10.471.000} &  2\tbfrac{3}{4} & 4     &    \hang{r}{1}1   & \textemdash \\
1839, 1  вересня           & \num{2.684.000}  &  6\phantom{\tbfrac{1}{4}}     & 8     &   5   &  2 \\
1840, 1  червня            & \num{4.571.000}  &  4\tbfrac{3}{4} & 5     &   9   & 1 \\
1840, 1  грудня            & \num{3.642.000}  &  5\tbfrac{3}{4} & 7     &   6   &  2 \\
1841, 1  грудня    & \num{4.873.000}  &  5\phantom{\tbfrac{1}{4}}     & 3     &    \hang{r}{1}2   &  \textemdash \\
1842, 1  грудня    & \num{10.603.000} &  2\tbfrac{1}{2} & 2     &    \hang{r}{1}3   &  \textemdash \\
1843, 1  червня             & \num{11.566.000} &  2\tbfrac{1}{4} & 1     &    \hang{r}{1}4   &  \textemdash \\
\end{tabular}
\end{table}
