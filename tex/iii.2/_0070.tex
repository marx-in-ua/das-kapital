\parcont{}  %% абзац починається на попередній сторінці
\index{iii2}{0070}  %% посилання на сторінку оригінального видання
за нею в наші часи провадиться значну частину операцій. Такі люди охоче
гублять 20, 30 та 40\% на одній відправі товару кораблем; ближча операція
може вернути їм ті втрати. Коли їм раз-по-раз не щастить, тоді вони гинуть;
і саме такі випадки ми часто бачили останніми часами; торговельні фірми
збанкрутували, не залишивши жодного шилінґа в активі.

«4791. Нижчий рівень проценту [протягом останніх 10 років], звичайно,
має для банкірів несприятливий вплив, але, не подаючи вам для огляду торговельних
книг, мені було б дуже тяжко пояснити вам, оскільки теперішній
зиск [його власний] вищий від попереднього. Коли рівень проценту низький у
наслідок надмірного видання банкнот, то в нас є багато вкладів; коли рівень
проценту високий, то це дає нам безпосередній бариш. — 4794. Коли гроші можна
мати за помірний процент, то ми маємо більший попит на них; ми більше
визичаємо; такий вплив має це [для нас, банкірів] у цьому випадку. Якщо він
підноситься, то ми одержуємо за ті позики більше, ніж то годиться; ми одержуємо
більше, ніж повинні б мати».

Ми бачили, що всі експерти вважають кредит банкнот Англійського банку
за непохитний. А проте, банковий акт покладає цілком точно суму 9--10
мільйонів золотом для забезпечення розміну тих банкнот. Святість та непорушність
цього скарбу здійснюється, отож, цілком інакше, ніж у давніх збирачів
скарбів. W. Brown (Liverpool) свідчить перед C. D. 1847/58, 2311 так: «Щодо
тієї користи, яку ці гроші [металевий скарб в емісійному відділі] давали в ті
часи, так їх так само добре можна було б кинути в море; аджеж не можна було
навіть найменшої частини їх ужити, не ламаючи того парламентського акту».

Підприємець — будівничий Е. Capps, що його ми вже раніше згадували,
той самий, що з його свідчень узято характеристику сучасної лондонської системи
будівництва (Книга II, розд. XII), так резюмує свій погляд на банковий
акт 1844 року (В. А. 1857): «5508. Отже, ви взагалі тієї думки, що сучасна
система [банкового законодавства] дуже зручна установа для того, щоб періодично
кидати зиски промисловости до грошової торби лихваря? — Я такої думки.
Я знаю, що в будівельній справі ця система мала такий вилив».

Як згадано, шотландські банки примушено банковим актом 1845 року до
такої системи, що наближалась до англійської. На них поклали обов’язок тримати
золото в запасі на покриття банкнот, що вони видаватимуть понад суму,
усталену для кожного банку. Який вилив це мало, про це подаємо тут кілька
свідчень перед В. C. 1857.

Kennedy, управитель одного шотландського банку: «3375. Чи перед заведенням
банкового акту 1845 було в Шотландії дещо таке, що можна було б
назвати золотою циркуляцією? — Нічого подібного. — 3376. Чи від того часу
збільшилась кількість золота в циркуляції? — Ані найменше; люди не хочуть
мати золота (the people dislike gold)» — 3450. Ті приблизно 900.000 ф. ст. золота,
що їх шотландські банки мусять тримати, починаючи від 1845 року, на його думку,
тільки шкодять та «непожиточно поглинають рівну собі частину капіталу Шотландії».

Далі, Anderson, управитель Union Bank of Scotland: «3558. Єдиний значний
попит на золото, що його Англійський банк мав з боку шотландських
банків, був з нагоди закордонних вексельних курсів? — Це так; і цей попит
не зменшився від того, що ми тримаємо золото в Едінбурзі. — 3590. Поки ми
тримаємо ту саму суму цінних паперів в Англійському банкові» [або в приватних
банках Англії], «ми маємо ту саму силу, що й раніш, до того, щоб
викликати відплив золота з Англійського банку».

Насамкінець, ще одна стаття з Economist’a (Wilson): «Шотландські банки
тримають у своїх лондонських аґентів вільні суми готівкою; останні тримають
ті суми в англійському банкові. Це дає шотландським банкам змогу порядкувати
металевим скарбом банку в межах цих сум, а той скарб є завжди тут, на тому
\parbreak{}  %% абзац продовжується на наступній сторінці
