\parcont{}  %% абзац починається на попередній сторінці
\index{iii2}{0181}  %% посилання на сторінку оригінального видання
створення пересічного зиску. А проте, ввесь капітал, витрачений на акр землі $В$,
продовжує давати надзиск, хоч він зменшується разом з збільшенням маси капіталу
недостатньої продуктивности і з збільшенням ступеня цієї недостатньої
продуктивности. За ростучого капіталу і збільшуваної продукції, рента з акра
зменшується тут абсолютно, а не тільки відносно проти ростучого розміру вкладеного
капіталу, як у другому випадку.

Знищитись рента може тільки тоді, коли індивідуальна пересічна ціна
продукції всього продукту на кращій землі $В$ збігається з регуляційною ціною,
коли весь надзиск перших продуктивніших витрат капіталу піде на створення
пересічного зиску.

За крайню межу зменшення ренти з акра є той пункт, на якому вона
зникає. Але цей пункт настає не безпосередньо після того, як додаткові витрати
капіталу починають продукувати з недостатньою продуктивністю, а по тому, як
додаткова витрата частини капіталу з недостатньою продуктивністю набуває таких
розмірів, що її вплив знищує надмірну продуктивність перших витрат капіталу,
і продуктивність усього вкладеного капіталу стає рівною продуктивності капіталу
на $А$, а тому індивідуальна пересічна ціна квартера з $В$ — рівною індивідуальній
пересічний ціні квартера з $А$.

І в цьому випадку реґуляційна ціна продукції в 3\pound{ ф. стерл.} з квартера
лишилася б та сама, хоч рента і зникла б. Тільки за цим пунктом ціна продукції
мусила б підвищитись в наслідок збільшення чи то ступеня недостатньої
продуктивности додаткового капіталу, чи то величини самого додаткового
капіталу тієї самої недостатньої продуктивности. Коли б, наприклад, вище, в таблиці
на стор. 179, на тій самій землі продукувалось замість 1\sfrac{1}{2} квартера 2\sfrac{1}{2} квартера
по 4\pound{ ф. стерл.} за квартер, то ми мали б в цілому 7 квартерів, вся ціна
продукції яких дорівнювала б 22\pound{ ф. стерл}. Квартер коштував би 3\sfrac{1}{7}\pound{ ф. стерл.},
тобто він коштував би на \sfrac{1}{7} вище за загальну ціну продукції, і остання
мусила б підвищитись.

Отже, ще довгий час можна було б вживати додатковий капітал з недостатньою
продуктивністю і навіть з дедалі більшою недостатньою продуктивністю,
поки індивідуальна пересічна ціна квартера на кращих земельних дільницях
не стане рівна загальній ціні продукції, поки не зникне цілком надмір останньої
над першою, а разом з тим і надзиск і рента.

Навіть і в цьому випадку з зникненням ренти з кращих земель індивідуальна
пересічна ціна їхнього продукту тільки збіглася б з загальною ціною продукції,
отже, все ще не потрібне було б підвищення цієї останньої.

У вищенаведеному прикладі на кращій землі $В$, що має, проте останнє
місце в ряді кращих земель, або таких, що дають ренту, 3\sfrac{1}{2} квартери продукуються
капіталом в 5\pound{ ф. стерл.}, який має додаткову продуктивність, і 2\sfrac{1}{2} квартери
продукуються капіталом в 10\pound{ ф. стерл.}, що має недостатню продуктивність,
разом 6 квартерів, отже \sfrac{5}{12} цієї кількости, продукується цими
останніми частинами капіталу, витраченими з недостатньою продуктивністю.
І тільки в цьому пункті індивідуальна пересічна ціна продукції 6 квартерів
підвищується до 3\pound{ ф. стерл.} за квартер, отже, збігається з загальною ціною
продукції.

Але за законом власности на землю, останні 2\sfrac{1}{2} квартери не могли б бути
випродуковані в такий спосіб по 3\pound{ ф. стерл.} за квартер, за винятком того
випадку, коли вони могли б бути випродуковані на нових 2\sfrac{1}{2} акрах землі
$А$. Випадок, коли додатковий капітал продукує вже тільки по загальній ціні
продукції, був би за межу. За нею мусили б припинитись додаткові витрати,
капіталу на тій самій землі.

Скоро лише орендареві довелося за дві перші витрати капіталу заплатити
4\sfrac{1}{2}\pound{ ф. стерл.} ренти, то він мусить продовжувати її виплачувати, і кожна витрата
\parbreak{}  %% абзац продовжується на наступній сторінці
