\parcont{}  %% абзац починається на попередній сторінці
\index{iii2}{0214}  %% посилання на сторінку оригінального видання
в тих хліборобських господарствах старовини, в яких виявляється найбільша
аналогія з капіталістичним сільським господарством, у Картагені та Римі,
навіть в них більше схожости з господарством плянтацій, ніж з формою, відповідною
до дійсного капіталістичного способу експлуатації\footnote{
А.~Сміт показує, до якої міри за його часу (та й для нашого часу це має силу щодо
плантаторського господарства в тропічних та субтропічних країнах) рента й зиск ще не відокремились,
бо земельний власник є одночасно і капіталіст, як був, наприклад, Катон у своїх маєтках. Але таке
відокремлення є саме передумова капіталістичного способу продукції, що його поняттю до того ж
взагалі суперечить така база, як рабство.
}. Формальної аналогії,
— яка однак в усіх істотних пунктах виступає цілком облудною для того,
хто зрозумів капіталістичний спосіб продукції і хто не відкриває, як пан Момзен\footnote{
У своїй римській історії п. Момзен бере слово капіталіст зовсім не в розумінні сучасної економії
і сучасного суспільства, а в дусі популярної уяви, яка все ще розповсюджується не в Англії або в
Америці, а на континенті, як старовинна традиція зниклих відносин.
}
капіталістичного способу продукції вже в усякому грошовому господарстві —
ми взагалі не знайдемо в старовину в континентальній Італії, хіба тільки в Сіцілії,
бо остання існувала як хліборобська країна, що виплачувала дань Римові,
і де тому хліборобство по суті провадилось на експорт. Тут трапляються орендарі
в сучасному розумінні.

Неправильне розуміння природи ренти ґрунтується на тій обставині, що
рента в натуральній формі з натурального господарства середньовіччя, і в цілковитій
суперечності умовам капіталістичного способу продукції, перейшла в новітній
час почасти у вигляді церковної десятини, почасти як дивовижність, увічнена
старовинними договорами. Через це здається, що рента виникає не з ціни
хліборобського продукту, а з маси продукту, тобто не з суспільних відносин,
а з землі. Вже давніш ми показали, що хоч додаткова вартість втілюється у
надпродукті, проте додатковий продукт в розумінні звичайного збільшення маси
продукту не втілює, навпаки, додаткової вартости. Він може являти собою мінус
вартости. Інакше бавовняна промисловість 1860 року проти 1840 мусила б
втілювати величезну додаткову вартість, тимчасом як ціна пряжі, навпаки, понизилась. В наслідок ряду
неврожайних років рента може зрости надзвичайно,
бо ціна збіжжя підвищується, хоч ця додаткова вартість втілюється в абсолютно
зменшеній масі подорожалої пшениці. Навпаки, в наслідок ряду урожайних
років рента може понизитись, бо ціна падає, хоч зменшена рента втілюється в
більшій масі порівняно дешевої пшениці. Тепер щодо ренти продуктами слід
насамперед зазначити, що вона являє собою просто традицію, перетягнуту з віджилого
способу продукції, що скніє як руїна останнього, і суперечність цієї
традиції з капіталістичним способом продукції виявляється в тому, що вона
сама собою зникає з приватних договорів, і що там, де могло втручитися законодавство,
як у випадку з церковними десятинами в Англії, вона була ґвалтовно
знесена як безглуздя. Але, подруге, там, де вона на базі капіталістичного
способу продукції і далі існує, вона була й могла бути не чим іншим, як
середньовічно замаскованим виразом грошової ренти. Хай, наприклад квартер
пшениці доходить до 40\shil{ шил.} Частина цього квартера мусить покрити заробітну
плату, що міститься в ньому, та мусить бути продана, щоб можна було знову
його витрачати; друга частина квартера мусить бути продана для того, щоб виплатити
частину податків, яка припадає на нього. Там, де капіталістичний
спосіб продукції розвинений, а з ним і поділ суспільної праці, насіння і навіть
частина добрива входять у репродукцію, як товари, отже, мусять бути куплені
для покриття; щоб здобути грошей на це, знов таки частина квартера мусить
бути продана. Оскільки ж їх в дійсності не доводиться купувати як товари,
а можуть бути взяті вони з самого продукту in natura, щоб знову
ввійти як умови продукції в його репродукцію, — як це трапляється не тільки
в хліборобстві, але і в багатьох галузях продукції, що продукують сталий
\parbreak{}  %% абзац продовжується на наступній сторінці
