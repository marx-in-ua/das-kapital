\parcont{}  %% абзац починається на попередній сторінці
\index{iii2}{0250}  %% посилання на сторінку оригінального видання
дійсно цілком сходить на суму вартости, складеної з заробітної плати плюс
зиск плюс рента, тобто на всю вартість трьох доходів, хоч вартість цієї частини
продукту цілком так само, як і тієї, що не входить в дохід, містить частину
вартости = C, рівну вартості сталого капіталу, що міститься в цих частинах,
отже, prima facie не може обмежуватись вартістю доходу, — ця обставина є, з одного
боку, практично безперечним фактом, з другого боку, так само безперечною
теоретичною суперечністю. Цю трудність найпростіше оминають, кажучи, що
товарова вартість лише з подоби, з погляду окремого капіталіста, має в собі
якусь іншу частину вартости, відмінну від частини, сущої в формі доходу.
Фраза: те, що для одного є доходом, для другого становить капітал, позбавляє
потреби всякого дальшого думання. Яким чином, коли вартість усього продукту
належить споживанню у формі доходів, може бути покритий старий капітал; і
яким чином вартість продукту кожного індивідуального капіталу може бути
рівна сумі вартости трьох доходів плюс C, сталий капітал, а вся разом складена
сума вартости продукту всіх капіталів дорівнює сумі вартости трьох доходів
плюс 0, — все це звичайно виступає при цьому як нерозв’язна загадка, все це
мусить бути пояснено тим, що аналіза взагалі нездібна виявити прості елементи
ціни та мусить задовольнятись обертанням в порочному колі і відсуванням
задачі до безконечности. Таким чином, те, що з’являється як сталий капітал,
може бути розкладене на заробітну плату, зиск, ренту, а товарові вартості,
що в них репрезентовані заробітна плата, зиск, рента, в свою чергу визначаються
заробітною платою, зиском, рентою і так далі до безконечности\footnote{
«В усякому суспільстві ціна кожного товару кінець-кінцем зводиться до однієї, або другої або до
всіх цих трьох частин [тобто заробітної плати, зиску, ренти]\dots{} Четверта частина, як можна було
припустити, потрібна для покриття капіталу орендаря, або зношування і полагодження і для відновлення
робочої худоби та інших знарядь хліборобства. Але слід зауважити, що ціна хоч би якого
хліборобського знаряддя, наприклад, робочого коня, в свою чергу складається з цих самих трьох
частин: ренти
з тієї землі, що на ній він виріс, праці, витраченої на догляд його і на його годівлю, і зиску
фармера, що авансує й ренту за землю і заробітну плату за працю. Тому, хоч ціна збіжжя покриє як
ціну, так і утримання коня, а проте, вся ціна розпадається безпосередньо або кінець-кінцем на ті
самі три частини: ренту, працю [треба сказати заробітну плату] і зиск». (А. Сміт). Ми покажемо
пізніш, що А. Сміт сам розуміє всю суперечність і недостатність цього викруту, бо що ж інше, як не
викрут відсилати
нас від Понтія до Пілата, ніде не показуючи нам тієї дійсної витрати капіталу, що при ній ціна
продукту кінець-кінцем без дальшого, відсування, без рештки розпадається на ці три частини.
}.

Фалшива в своїй основі догма, що вартість товарів кінець-кінцем може
бути розкладена на заробітну плату + зиск + рента, набуває ще й такого виразу,
ніби споживач кінець-кінцем мусить оплатити всю вартість сукупного
продукту; або що грошова циркуляція між продуцентами й споживачами кінець-кінцем
мусить дорівнювати грошевій циркуляції між самими продуцентами
(Tooke); засади, що так само неправдиві, як та основна засада, на яку вони
спираються.

Труднощі, які призводять до цієї помилкової і prima facie абсурдної аналізи,
коротко кажучи такі:

1) Нерозуміння основного відношення між сталим та змінним капіталом,
отже й природи додаткової вартости, а разом з тим і всієї бази капіталістичного
способу продукції. Вартість кожної частини продукту капіталу, кожного
окремого товару має в собі частину вартости = сталому капіталові, частину
вартости = змінному капіталові (перетвореному на заробітну плату робітників)
і частину вартости = додатковій вартості (пізніше поділяється на зиск і ренту).
Отже, яким же чином можливо, щоб робітник на свою заробітну плату, капіталіст
на свій зиск, земельний власник на свою ренту могли купити товари, що
з них кожен має в собі не тільки одну з цих складових частин, але всі три, і
яким чином можливо, щоб сума вартости заробітної плати, зиску, ренти, отже,
всіх трьох джерел доходу, разом узятих, могла купити товари, що становлять
\parbreak{}  %% абзац продовжується на наступній сторінці
