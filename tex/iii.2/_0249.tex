
\index{iii2}{0249}  %% посилання на сторінку оригінального видання
Гуртовий дохід є частина вартости і вимірювана нею частина брутто-продукту
або гуртового продукту, яка залишається за вирахуванням тієї частини
вартости і вимірюваної нею частини продукту сукупної продукції, що покриває
авансований і зужиткований у продукції сталий капітал. Отже, гуртовий дохід
дорівнює заробітній платі (або тій частині продукту, що має призначення знову
стати доходом робітника) \dplus{} зиск \dplus{} рента. Навпаки, чистий дохід це є додаткова
вартість, отже, додатковий продукт, що залишається за вирахуванням заробітної
плати; отже, він в дійсності є додаткова вартість, яка реалізована капіталом і
повинна бути поділена з земельним власником, він є вимірюваний цією додатковою
вартістю додатковий продукт.

Ми щойно бачили, що вартість кожного окремого товару і вартість усього
товарового продукту кожного окремого капіталу розпадається на дві частини:
одну, що просто покриває сталий капітал, і другу, що певна частина її, хоч і
припливає назад як змінний капітал, отже, припливає і в формі капіталу,
а проте все ж таки призначена на те, щоб цілком перетворитись на гуртовий
дохід і набути форми заробітної плати, зиску й ренти, що сума їх становить
гуртовий дохід. Ми бачили далі, що те саме має силу і щодо вартости сукупного
річного продукту суспільства. Ріжниця між продуктом поодинокого капіталіста
і продуктом суспільства є лише ось у чому: розглядуваний з погляду
окремого капіталіста, чистий дохід відрізняється від гуртового доходу, бо останній
має в собі заробітну плату, а перший виключає її. Коли розглядати дохід
усього суспільства, то національний дохід складається з заробітної плати плюс
зиск плюс рента, отже, з гуртового доходу. Проте, і це є абстракція в тому
розумінні, що все суспільство, на основі капіталістичної продукції, стає на капіталістичний
погляд і тому за чистий дохід вважає лише дохід, що зводиться
до зиску й ренти.

Навпаки, фантазія, як наприклад, у п. Сея, ніби ввесь здобуток, сукупний
гуртовий продукт перетворюється для нації на чистий здобуток або не відрізняється
від нього, що, отже, ріжниця ця з національного погляду перестає існувати,
— ця фантазія є лише доконечне і крайнє виявлення абсурдної догми, яка
проходить через усю політичну економію від А.~Сміта і є в тому, що вартість
товарів кінець-кінцем без остачі розпадається на доходи, заробітну плату,
зиск і ренту\footnote{
Рікардо робить таке дуже вдале зауваження щодо безглуздого Сея: «Про чистий і гуртовий продукт
п. Сей говорить таке: «Вся випродукована вартість є гуртовий продукт; ця вартість за вирахуванням з
неї витрат продукції, є чистий продукт» (Vol. II, p. 491). Отже, чистого продукту бути не може, бо
витрати продукції за п. Сеєм складаються з ренти, заробітної плати й зиску. На стор. 508 він
говорить: «Отже вартість продукту, вартість продуктивних послуг, вартість витрат продукції — все
це однорідні вартості, коли тільки речі віддається на їхній природний перебіг». Коли з усього
вирахувати все, то в наслідку нічого не лишиться» (Ricardo, Principles, chap. XXII, p. 512,
примітка). А втім, ми пізніш побачимо, що й Рікардо ніде не заперечив неправдивої аналізи товарової
ціни у Сміта, зведення її до суми вартости доходів. Він не турбується про цю аналізу і в своєму
дослідженні визнає її за правдиву остільки, оскільки він «абстрагується» від сталої частини вартости
товарів. Час від часу він сам підпадає під владу такого самого способу уявлення.
}.

Природна річ, дуже легко зрозуміти, коли справа йде про кожного окремого
капіталіста, що частина його продукту мусить знову перетворитись на
капітал (навіть лишаючи осторонь поширення репродукції або акумуляцію) і до
того не тільки на змінний капітал, призначення якого в свою чергу знову перетворитись
на дохід для робітника, тобто в форму доходу, але й на сталий
капітал, що ніколи не може перетворитись на дохід. Найпростіше спостереження
процесу продукції унаочнює це. Трудність починається лише тоді, коли процес
продукції розглядається в цілому. Та обставина, що вартість усієї частини продукту,
споживаної як дохід у формі заробітної плати, зиску й ренти (причому
цілком байдуже, чи буде це особисте, чи продуктивне споживання), при аналізі
\parbreak{}  %% абзац продовжується на наступній сторінці
