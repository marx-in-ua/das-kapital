
\index{iii2}{0058}  %% посилання на сторінку оригінального видання
Проте банки мають ще й інші засоби утворювати капітал. За тим самим
Newmarch’oм провінціяльні банки, як уже вище згадано, мають звичай відправляти
свої зайві фонди (тобто банкноти Англійського банку) лондонським billbrokers'aм,
що шлють їм натомість дисконтовані векселі. Цими векселями банк
обслуговує своїх клієнтів, бо він має за правило не видавати векселів, одержаних
від своїх місцевих клієнтів, щоб комерційні операції цих клієнтів не стали відомі
в їхній власній місцевості. Оці, одержані з Лондону, векселі служать не тільки
до того, щоб видавати їх клієнтам, які мають робити платежі безпосередньо
в Лондоні, якщо вони не вважатимуть за краще доручити банкові зробити
власний переказ на Лондон; ці векселі служать ще й до сплочування платежів
у провінції, бо передатний напис банкіра забезпечує їм місцевий кредит. Таким
чином, напр., в Ланкашайрі, витиснули ті векселі з циркуляції всі власні банкноти
місцевих банків та чималу частину банкнот Англійського банку (ibidem,
1568--74).

Отже, ми бачимо, як банки утворюють кредит та капітал: 1) виданням
власних банкнот; 2) виданням переказів на Лондон реченцем до 21 дня, переказів,
що їх однак в момент їхнього видання одразу оплачується банкам готівкою;
3) платежем дисконтованими векселями, що їхня кредитоздібність перед
усім та найголовніше, принаймні для відповідної місцевої округи, забезпечується
передатним написом банку.

Сила Англійського банку виявляється в реґулюванні ним ринкової норми
рівня проценту. Підчас нормального перебігу справ може трапитися, що Англійському
банкові не сила буде припинити помірний відплив золота з свого металевого
скарбу, підвищенням норми дисконту\footnote{
На загальних зборах акційний Union Bank of London 17 січня 1894 президент пан Ritchie
оповідав, що Англійський банк підвищив в 1893 році дисконт від 2\sfrac{1}{2} (липень) до 3 та 4\% в серпні,
та, згубивши проте протягом чотирьох тижнів повних 4\sfrac{1}{2} міл. ф. ст. золотом, до 5\%, після чого
золото почало припливати назад і банкову норму дисконту знизили в вересні до 4\%, а в жовтні до 3\%.
Але на ринку цю банкову норму не визнали. «Коли банкова норма була 5\%, то ринкова норма була З \sfrac{1}{2},
а норма для грошей була 2\sfrac{1}{2}\%; коли банкова норма впала до 4\%, то норма дисконту була 2\sfrac{3}{8}\%, а
грошова норма 1\sfrac{3}{4}\%; коли банкова норма була 3\%, то норма дисконту була 1\sfrac{1}{2}, а грошова норма
трохи нижча». (Daily News 18 січня 1894~\abbr{р.}) — Ф.~Е).
}, бо потребу на платіжні засоби
задовольняють приватні й акційні банки та bill-brokers’и, що за останні тридцять
років набули значної сили на полі капіталу. Тоді має він уживати інших
засобів. Але для критичних моментів все ще має силу те, про що банкір Glyn
(з фірми Glyn, Mills, Currie and C°) свідчив перед C.~D. 1848/57: «1709. Підчас
великої скрути в країні Англійський банк диктує рівень проценту — 1710. Підчас
надзвичайної скрути\dots{}, коли приватні банкірі або brokers’и порівняно обмежують
дисконтові операції, ці операції випадають Англійському банкові, й тоді він має
силу усталювати ринкову норму рівня проценту».

Звичайно, як офіційна установа, що має державну охорону та державні
привилеї, не сміє банк використовувати немилосердно цю свою силу, так як
можуть собі дозволити це приватні підприємства. Тому й Hubbard ось що каже
перед банковою комісією В.~А. 1857; «2844 [питання]: А хіба не правда, що
коли норма дисконту є найвища, то Англійський банк обслуговує найдешевше,
а коли вона найнижча, тоді brokers’и обслуговують найдешевше? — [Hubbard]:
Так завжди буває, бо Англійський банк ніколи не знижує норми проценту так
низько, як його конкуренти, а коли норма є найвища, ніколи не підносить її
цілком так високо, як вони».

А проте серйозною подією в комерційному житті буває, коли банк підчас
скрути починає — уживаючи ходячого вислову — наганяти рівень проценту, тобто
ще вище підносити рівень проценту, що вже піднявся вище від пересічного. «Скоро
Англійський банк починає наганяти рівень проценту, припиняються всі закупи
для вивозу закордон\dots{} експортери чекають, поки спад цін дійде найнижчої точки
\parbreak{}  %% абзац продовжується на наступній сторінці
