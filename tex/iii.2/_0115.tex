\parcont{}  %% абзац починається на попередній сторінці
\index{iii2}{0115}  %% посилання на сторінку оригінального видання
але це утома переситу — і подивіться, як вони квапляться з місця на місце,
ніби все сходить на те, щоб знайти нове задоволення» (Morning Star, 15 грудня
1865 року).

Далі показано, як додаткова праця, а тому і додатковий продукт взагалі
сплутується з земельною рентою, цією частиною додаткового продукту, яка —
принаймні на базі капіталістичного способу продукції — є специфічно визначена
і кількісно і якісно. За природну базу додаткової праці взагалі, тобто за
природну умову, без якої вона неможлива, є те, що природа при витраті робочого
часу, що не поглинає всього робочого дня, дає потрібні засоби існування
чи в продуктах землі, рослинних або тваринних, чи в продуктах рибальства тощо.
Ця природна продуктивність хліборобської праці (сюди належить і праця простого
збирання, ловів, рибальства, скотарства) є базою всякої додаткової праці, бо вся
праця насамперед первісно спрямована на привласнення і продукцію їжі. (Але
тварина дає одночасно хутро, що зберігає тепло в холодному підсонні; крім того
печерні житла тощо).

Таке саме сплутування додаткового продукту і земельної ренти, тільки
інакше висловлене, трапляється в п. Dove. Первісно хліборобська праця і промислова
праця не відділені одна від однієї: друга приєднується до першої.
Додаткова праця і додатковий продукт хліборобського плем’я, домової громади
або родини має в собі так хліборобську, як і промислову працю. Обидві ідуть
пліч-о-пліч. Полювання, рибальство, хліборобство неможливі без відповідних
знарядь. Ткацтво, прядіння тощо спочатку ведуться, як допоміжні при хліборобстві
роботи.

Давніш ми показали, що так само як праця окремого робітника розпадається
на потрібну і додаткову працю, так само і всю працю робітничої кляси можна
поділити так, що та частина, яка продукує сукупні засоби існування для робітничої
кляси (включаючи сюди і потрібні для цього засоби продукції), виконує потрібну
працю для всього суспільства. Працю, виконувану всією іншою частиною
робітничої кляси, можна розглядати як додаткову працю. Але потрібна праця
має в собі не тільки хліборобську працю, але також і ту працю, яка продукує
всі інші продукти, що доконечно входять в пересічне споживання робітника.
Крім того, з суспільного погляду одні виконують тільки потрібну працю лише
тому, що інші виконують тільки додаткову працю, і навпаки. Це — лише
поділ праці між ними. Так само стоїть справа і з поділом праці між хліборобськими
і промисловими робітниками взагалі. Суто-промисловому характерові
праці на одному боці відповідає суто-хліборобський на другому. Ця сутохліборобська
праця аж ніяк не дана природою, але вона є сама продукт
суспільного розвитку, до того ж продукт дуже новий, геть не всюди досягнений,
і відповідає цілком певному ступеневі в розвитку продукції. Так само, як частина
хліборобської праці зрічевлюється в продуктах, які або служать тільки розкошам,
або становлять сировий матеріял для промисловости, але ніяк не входять
в харч, не говорячи вже про харч мас, — так з другого боку частина промислової
праці зрічевлюється в продуктах, які правлять за потрібні засоби споживання
так хліборобських, як і нехліборобських робітників. Було б помилково цю
промислову працю — з суспільного погляду — розглядати як додаткову працю. Вона
в певній частині така ж потрібна праця, як потрібна частина хліборобської
праці. Вона теж є тільки усамостійнена форма певної частини тієї промислової
праці, яка давніш природно була з’єднана з хліборобською працею, потрібне взаємне
доповнення відокремленої тепер від неї суто-хліборобської праці. (Розглядаючи
справу з суто-матеріяльного боку, наприклад, 500 ткачів за механічними
варстатами продукують в незрівняно більшій мірі додаткову тканину, тобто
продукують більше, ніж потрібно для їхнього власного одягу)

Нарешті, розглядаючи форми прояву земельної ренти, тобто орендні гроші,
\parbreak{}  %% абзац продовжується на наступній сторінці
