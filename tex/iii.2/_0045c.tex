
Як значно зменшено ужиток грошей в гуртовій торговлі до невеличкого
мінімуму, про це свідчить таблиця, надрукована в книзі І, розд. III,
що подана банковій комісії фірмою Morrison Dillon and C°, однією з тих найбільших
лондонських фірм, де дрібний торговець може закупити ввесь свій запас
товарів усякого роду.

За свідченням W.~Newmarch’a перед банковою комісією 1857~\abbr{р.} № 1741,
заощадженню засобів циркуляції сприяли ще й інші обставини: поштовий тариф
в 1\pens{ пенс} (das Penny-Briefporto), залізниці, телеграфи, коротко — поліпшені засоби
комунікації; так що Англія тепер має змогу при приблизно тій самій циркуляції
банкнот робити у п’ятеро-шестеро більш операцій. Але до цього значно
спричинилося теж виключення з циркуляції банкнот, більших за 10\pound{ ф. ст.} 1 це
здається йому природним поясненням того, що в Шотляндії та Ірландії, де в
циркуляції є навіть банкноти в 1\pound{ ф. ст.}, циркуляція банкнот зросла приблизно
на 31\% (1747). Вся сума банкнот, що є в циркуляції у Сполученому Королівстві,
включаючи й банкноти в 1\pound{ ф. ст.}, становить, хай 39 мільйонів ф. ст.
(1749). Сума золота в циркуляції — 70 мільйонів ф. ст. (1750). В Шотляндії
в 1834 році було в циркуляції банкнот \num{3.120.000}\pound{ ф. ст.}; в 1844 році —
\num{3.020.000}\pound{ ф. ст.}; в 1854 році — \num{4.050.000}\pound{ ф. ст.} (1752).

Вже з цього випливає, що збільшення числа банкнот в циркуляції ніяк
не в волі банків, що видають банкноти, поки ці банкноти можна кожного часу
розмінювати на грші (Goeld). [Про паперові гроші, що їх не можна розміняти,
тут взагалі немає мови; нерозмінні банкноти можуть лише там правити за загальний
засіб циркуляції, де їх фактично підпирає державний кредит, як от, напр., тепер
в Росії. Отже вони підлягають уже розвинутим законам (книга І, розд. III, 2, c:
монета, знак вартости) про нерозмінні державні паперові гроші. — Ф.~Е.].

Число банкнот в циркуляції реґулюється потребами обороту, й кожна
зайва банкнота повертається негайно назад до того, хто її видав. Оскільки в
Англії взагалі лише банкноти Англійського банку обертаються як законний
платіжний засіб, то й можемо ми тут знехтувати незначною, а до того ще й лише
місцевою циркуляцією банкнот провінціяльних банків.
