
\index{iii2}{0045}  %% посилання на сторінку оригінального видання
\begin{table}[h]
  \small
  \begin{tabular}{l c c c c c c c}
  \toprule
Рік  &  \makecell{Банкноти\\ в 5\textendash{}10 \\ф. ст.} & Відсотки  &
\makecell{Банкноти\\ в 20\textendash{}100\\ ф. ст.} & Відсотки  &  \makecell{Банкноти\\ в 200\textendash{}\\1000
ф. ст.} & Відсотки &   \makecell{Разом\\  фунтів \\ ст.} \\
  \midrule
1844    &     \phantom{0}9263  &  45,7  &  5735 & 28,3 & 5253 &   26,0 &   20241\\
1845    &     \phantom{0}9698  &  46,9  &  6082 & 29,3 & 4942 &   28,6 &   20723\\
1846    &     \phantom{0}9918  &  48,9  &  5771 & 28,5 & 4590 &   22,6 &   20286\\
1847    &     \phantom{0}9591  &  50,1  &  5498 & 28,7 & 4066 &   21,2 &   19155\\
1848    &     \phantom{0}8732  &  48,3  &  5046 & 27,9 & 4307 &   23,8 &   18085\\
1849    &     \phantom{0}8692  &  47,2  &  5234 & 28,5 & 4777 &   24,3 &   18403\\
1850    &     \phantom{0}9164  &  47,2  &  5587 & 28,8 & 4646 &   24,0 &   19398\\
1851    &     \phantom{0}9362  &  48,8  &  5554 & 28,5 & 4557 &   23,4 &   19473\\
1852    &     \phantom{0}9839  &  45,0  &  6161 & 28,2 & 5856 &   26,8 &   21856\\
1853    &     10699 &  47,3  &  6393 & 28,2 & 5541 &   24,5 &   22653\\
1854    &     10565 &  51,0  &  5910 & 28,5 & 4234 &   20,5 &   20709\\
1855    &     10628 &  53,6  &  5706 & 28,9 & 3459 &   17,5 &   19793\\
1856    &     10680 &  54,4  &  5645 & 28,7 & 3324 &   16,9 &   19648\\
1857    &     10659 &  54,7  &  5567 & 28,6 & 3241 &   16,7 &   19467\\
\end{tabular}
\end{table}
Як значно зменшено ужиток грошей в гуртовій торговлі до невеличкого
мінімуму, про це свідчить таблиця, надрукована в книзі І, розд. III,
що подана банковій комісії фірмою Morrison Dillon and C°, однією з тих найбільших
лондонських фірм, де дрібний торговець може закупити ввесь свій запас
товарів усякого роду.

За свідченням W. Newmarch’a перед банковою комісією 1857~\abbr{р.} № 1741,
заощадженню засобів циркуляції сприяли ще й інші обставини: поштовий тариф
в 1 пенс (das Penny-Briefporto), залізниці, телеграфи, коротко — поліпшені засоби
комунікації; так що Англія тепер має змогу при приблизно тій самій циркуляції
банкнот робити у п’ятеро-шестеро більш операцій. Але до цього значно
спричинилося теж виключення з циркуляції банкнот, більших за 10 ф. ст. 1 це
здається йому природним поясненням того, що в Шотляндії та Ірландії, де в
циркуляції є навіть банкноти в 1 ф. ст., циркуляція банкнот зросла приблизно
на 31\% (1747). Вся сума банкнот, що є в циркуляції у Сполученому Королівстві,
включаючи й банкноти в 1 ф. ст., становить, хай 39 мільйонів ф. ст.
(1749). Сума золота в циркуляції-70 мільйонів ф. ст. (1750). В Шотляндії
в 1834 році було в циркуляції банкнот 3.120.000 ф. ст.; в 1844 році —
3.020.000 ф. ст.; в 1854 році — 4.050.000 ф. ст. (1752).

Вже з цього випливає, що збільшення числа банкнот в циркуляції ніяк
не в волі банків, що видають банкноти, поки ці банкноти можна кожного часу
розмінювати на грші (Goeld). [Про паперові гроші, що їх не можна розміняти,
тут взагалі немає мови; нерозмінні банкноти можуть лише там правити за загальний
засіб циркуляції, де їх фактично підпирає державний кредит, як от, напр., тепер
в Росії. Отже вони підлягають уже розвинутим законам (книга І, розд. III, 2, c:
монета, знак вартости) про нерозмінні державні паперові гроші. — Ф. Е.].

Число банкнот в циркуляції реґулюється потребами обороту, й кожна
зайва банкнота повертається негайно назад до того, хто її видав. Оскільки в
Англії взагалі лише банкноти Англійського банку обертаються як законний
платіжний засіб, то й можемо ми тут знехтувати незначною, а до того ще й лише
місцевою циркуляцією банкнот провінціяльних банків.
