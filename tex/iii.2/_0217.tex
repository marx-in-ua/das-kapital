\parcont{}  %% абзац починається на попередній сторінці
\index{iii2}{0217}  %% посилання на сторінку оригінального видання
специфічної форми держави. Це не перешкоджає тому, що та сама економічна
база — та сама з боку головних умов — в наслідок безконечно різних емпіричних
обставин, природних умов, расових відносин, історичних впливів, що діють
зовні тощо, може показувати у своєму вияві безконечні варіації і ґрадації, що
їх зрозуміти можливо лише з допомогою аналізи цих емпірично даних обставин.

Щодо відробітної ренти, найпростішої і найпервісної форми ренти, то очевидно
таке: рента є тут первісною формою додаткової вартости й збігається з
нею. Але збіг додаткової вартости з неоплаченою чужою працею не потребує
тут дальшої аналізи, тому що вона існує тут ще в своїй очевидній, обмацальній
формі, бо праця безпосереднього продуцента на самого себе тут ще відокремлена
в просторі і часі від його праці на земельного власника, а остання виступає
безпосередньо в грубій формі примусової праці на другу особу. Так само «властивість»
землі давати ренту сходить тут до обмацально розкриваної таємниці, бо до
природи, що дає ренту, належить також прикріплена до землі людська робоча сила,
і відносини власности, які примушують власника робочої сили напружувати і витрачати
її поза межами того, що потрібно було б для задоволення його власних
доконечних потреб. Рента становить беспосереднє привласнення земельним
власником цієї надмірної витрати робочої сили, бо крім цього безпосередній
продуцент не виплачує йому жодної ренти. Тут, де не тільки тотожні додаткова
вартість і рента, але додаткова вартість ще обмацально має форму додаткової праці,
цілком ясно виступають і природні умови або межі ренти, бо це є природні
умови і межі додаткової праці взагалі. Беспосередній продуцент
мусить 1)~мати достатню робочу силу і 2)~природні умови його праці, отже,
в першу чергу оброблюваної землі, мусять бути досить сприятливі, одним
словом, природна продуктивність його праці мусить бути досить велика для
того, щоб у нього лишалася можливість витрачати надмірну працю понад працю
потрібну для задоволення його власних доконечних потреб. Ця можливість не
створює ренти, її створює лише примус, що перетворює можливість на дійсність.
Але сама можливість зв’язана з суб’єктивними й об’єктивними природними
умовами. В цьому теж немає рішуче нічого таємничого. Коли робоча сила незначна
і природні умови праці мізерні, то додаткова праця незначна, але в
такому випадку незначні, з одного боку, потреби продуцентів, з другого боку,
відносна кількість визискувачів додаткової праці, і, нарешті, незначний додатковий
продукт, що в ньому реалізується ця мало продуктивна додаткова праця для
цього відносно незначного числа визискувачів-власників.

Нарешті, при відробітній ренті ясно само собою, що, за інших незмінних
умов, від відносних розмірів додаткової або панщинної праці цілком залежить,
в якій мірі в безпосереднього продуцента з’явиться можливість поліпшувати
своє власне становище, збагачуватися, продукувати певний надмір понад доконечні
засоби існування, або, коли ми антиципуємо капіталістичний спосіб виразу,
чи з’явиться у нього і в якій мірі можливість продукувати хоч би якийсь
зиск для себе самого, тобто надмір над його заробітною платою, продукуваною
ним самим. Рента тут нормальна, всежеруща, так би мовити, законна форма
додаткової праці: вона далека від того, щоб становити надмір над зиском, тобто
далека від того, щоб бути тут за надмір над якимось іншим надміром понад
заробітну плату; тут не тільки розмір такого зиску, але й саме його існування
залежить, за інших незмінних умов, від розміру ренти, тобто додаткової праці,
примусово виконуваної для власника.

Деякі історики висловили своє здивування перед тим, що хоч беспосередній
продуцент не є власник, а лише посідач, і вся його додаткова праця de jure
дійсно належить земельному власникові, — що за цих умов взагалі може відбуватись
самостійний розвиток майна і, кажучи відносно, багатства у зобов’язаних
\parbreak{}  %% абзац продовжується на наступній сторінці
