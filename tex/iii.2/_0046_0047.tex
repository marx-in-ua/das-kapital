
\index{iii2}{0046}  %% посилання на сторінку оригінального видання
Пан Neave, управитель Англійського банку, дає таке свідчення перед
банковою Комісією 1858 року: «№ 947. (Питання): Хоч і яких заходів ви
раз-у-раз уживаєте, проте сума банкнот в руках публіки, як ви кажете,
лишається однакова; тобто приблизно 20 міл. ф. ст.? — За звичайних часів,
здається, потреби публіки вимагають приблизно 20 міл. В певні часи, що періодично
повторюються протягом року, сума та підноситься на 1 або 1\sfrac{1}{2} міл.
Коли публіка потребує більше, вона може, як я сказав, завжди одержати
потрібну кількість в Англійському банку. — 948. Ви сказали, що підчас паніки
публіка не дозволяла вам зменшувати суму банкнот; чи не буде ваша ласка
обґрунтувати цю вашу думку? — Під час паніки публіка, так мені здається, має
повну змогу добувати собі банкноти; і, природно, поки банк має зобов’язання,
публіка може, на основі цього зобов’язання, брати з банку банкноти. — 949.
Отже, здається, що повсякчас потрібно банкнот Англійського банку приблизно
на 20 міл. ф. ст.? — 20 міл. ф. ст. банкнотами в руках публіки; це число
змінюється. Буває 18\sfrac{1}{2}, 19, 20 міл. і~\abbr{т. ін.}; але пересічно ви можете сказати —
19--20 мільйонів».

Свідчення Тома Тука перед комісією лордів про Commercial Distress\footnote*{
Commercial Distress — торговельна криза. \Red{Прим. Ред.}
}
(C.~D. 1848/57); № 3094: «Банк не має сили своєвільно збільшувати суму
банкнот в руках публіки; він має силу зменшувати суму банкнот в руках
публіки; але лише за допомогою дуже ґвалтовної операції».

І.~C.~Wright, банкір в Нотігемі протягом останніх 30 років, з’ясувавши
докладно неможливість того, щоб провінціяльні банки могли колинебудь зберігати
в циркуляції більше банкнот, ніж того потребує та хоче публіка, каже про банкноти
Англійського банку (C.~D. 1848/57) № 2844: «Я не знаю ніяких меж»
(щодо видання банкнот) «для Англійського банку, але всякий надмір циркуляції
переходитиме у вклади й так прийматиме іншу форму».

Те саме має силу для Шотляндії, де циркулюють майже самі лише паперові
гроші, бо там, як і в Ірляндії, дозволено й однофунтівки, а також
тому що, «the Scotch hate gold\footnote*{
Шотляндець ненавидить золото. \Red{Прим. Ред.}
}». Кеннеді, директор одного шотляндського банку,
заявляє, що банки не спромоглися б навіть зменшити циркуляції своїх банкнот,
та дотримується він «тієї думки, що, доки операції всередині країни для свого
здійснення вимагають банкнот або золота, доти банкіри мусять постачати стільки
засобів циркуляції, скільки того потребують ці операції, — чи на вимогу своїх
вкладників, чи то якось інакше\dots{} Шотляндські банки можуть обмежити свої
операції, але вони не можуть контролювати видання своїх банкнот» (ib. № 3446 —
48). Так само висловлюється Андерсон, директор Union Bank of Scotland, ib.
№ 3578: «Чи заважає система взаємного обміну банкнотами» [між шотляндськими
банками]» надмірному виданню банкнот з боку якогось поодинокого банку? —
Так; але ми маємо більш дійсний засіб, ніж обмін банкнотами» [який в дійсності
не має з цим нічого до діла, але, що правда, забезпечує здатність банкнот кожного
банку обертатись по всій Шотляндії], «а саме — і це є загальний звичай
в Шотляндії — мати рахунок в банку; кожен, хто має сякі-такі гроші, має
й рахунок в якомусь банку й щодня складає до банку гроші, що їх він сам
безпосередньо не потребує, так що наприкінці кожного операційного дня всі
гроші є в банках опріч тих, що їх кожен має в кишені».

Так само і в Ірляндії; див. свідчення управителя Ірляндського банку,
Mac Donnall’я, та директора провінціяльного банку Ірляндії, Murray’я, перед
тією самою комісією.

Так само як циркуляція банкнот не залежить від волі Англійського банку,
вона не залежить і від стану того золотого скарбу в коморах банку, що забезпечує
\index{iii2}{0047}  %% посилання на сторінку оригінального видання
розмінність цих банкнот. «18 вересня 1846 року циркуляція банкнот
Англійського банку становила \num{20.900.000}\pound{ ф. ст.}, а його металевий скарб —
\num{16.273.000}\pound{ ф. ст.}; 5 квітня 1847~\abbr{р.} циркуляція — \num{20.815.000}\pound{ ф. ст.}, а металевий
скарб — \num{10.246.000}\pound{ ф. ст}. Отже, не зважаючи на експорт 6 мільйонів ф. ст. благородного
металю, не настало зменшення циркуляції». (I.~G.~Kinnear, The Crisis
and the Currency, Ld. 1847, p 5). Однак, само собою зрозуміла річ, що це мав силу
тільки в тих умовах, що тепер панують в Англії, та й то лише остільки, оскільки
законодавство не визначить якогось іншого відношення між виданням банкнот
та металевим скарбом.

Отже, тільки потреби самої лише торговлі (des Geschäfts selbst) мають
вплив на кількість грошей — банкнот та золота — в циркуляції. Тут, насамперед,
треба звернути увагу на періодичні коливання, що-повторюються кожного року,
хоч і який був би загальний стан справ, так що протягом останніх 20 років
«одного певного місяця циркуляція є висока, другого — низька, а третього певного
місяця доходиться середньої точки». (Newmarch, Б.~А. 1857, № 1650).

Напр., у серпні місяці кожного року кілька мільйонів, здебільша золотом,
переходять з Англійського банку у внутрішню циркуляцію на оплату видатків
в зв’язку з жнивами; що тут головна справа у виплаті заробітної плати, то
в Англії в цій справі менше вживають банкнот. До кінця року ці гроші знову
припливають до банку. В Шотландії замість золотих соверенів є майже самі лише
банкноти однофунтівки; тому тут у відповідному випадку поширюється циркуляція
банкнот, і то саме двічі на рік, у травні та листопаді, від 3 до 4 мільйонів;
по 14 днях починається вже зворотний приплив, а за місяць він майже закінчується
(Anderson, 1., c; № 3595--3600).

Циркуляція банкнот Англійського банку щочверть року зазнає ще й
тимчасових коливань, бо виплачується щочверть року «дивіденди», тобто проценти
на державні борги, через що спочатку банкноти вилучаються з циркуляції,
а по тому знову їх викидається проміж публіку, але вони дуже скоро припливають
назад. Weguelin (В.~А. 1857, № 38) подає суму викликаного цим коливанням
циркуляції банкнот в 2\sfrac{1}{2} мільйони. Навпаки, пан Chapman з відомої фірми
Overend Gurney~\& Co обчислює суму порушення на грошовому ринку, викликану
тим явищем, далеко вище. «Коли ви податками заберете з циркуляції 6 або
7 мільйонів, щоб ними виплатити дивіденди, то мусить же бути хтось, хто дав
би цю суму до розпорядку на проміжний час». (В.~А. 1857, № 5196).

Далеко значніші та тримаються довший час ті коливання суми засобів
циркуляції, що відповідають різним фазам промислового циклу. Про це послухаймо
іншого Associé\footnote*{
Спільник. \Red{Пр.~Ред.}
} тієї фірми шановного квакера Samuel Gurney’я (C.~D.
1848/57, № 2645): «Наприкінці жовтня (1847~\abbr{р.}) в руках публіки було банкнот
на \num{20.800.000}\pound{ ф. ст}. Тоді на грошовому ринку було дуже тяжко одержувати
банкноти. Це постало з загальної опаски, що в наслідок обмеження банковим
актом 1844 не можна буде добувати банкноти. Тепер [березень 1848 року] сума
банкнот в руках публіки становить\dots{} \num{17.700.000}\pound{ ф. ст.}, але що тепер немає
ніякої комерційної паніки, то і є ця сума далеко більша за ту, що потрібна.
В Лондоні немає жодного банкіра або торговця грішми, що не мав би банкнот
більше, ніж він може їх ужити. — 2650. Сума банкнот\dots{} опріч тих, що є на
схові в Англійському банку, являє собою цілком недостатній покажчик активного
стану циркуляції, коли одночасно теж не взяти на увагу\dots{} стану торговельного
світу та кредиту. — 2651. Почуття, що тепер при сучасній сумі циркуляції в
руках публіки є надмір банкнот, постає в значній мірі з нашого сучасного становища,
з великого застою в справах. За високих цін та жвавих справ ця
кількість, \num{17.700.000}\pound{ ф. ст.}, викликала б у нас почуття недостачі».
