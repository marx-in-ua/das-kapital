\parcont{}  %% абзац починається на попередній сторінці
\index{iii2}{0205}  %% посилання на сторінку оригінального видання
це розходження, або чи існує воно взагалі, це залежить від відносного розвитку хліборобства
проти промисловости. З самої природи справи ця ріжниця з поступом
хліборобства мусить зменшуватись, коли пропорція, в якій змінна частина
капіталу зменшується проти сталої, не буде для промислового капіталу ще
більша, ніж для хліборобського.

Ця абсолютна рента відіграє ще значнішу ролю у власне видобувній
промисловості, де один елемент сталого капіталу, сировий матеріял, цілком відпадає,
і де, — за винятком галузей, в яких частина, що складається з машин і
іншого основного капіталу, дуже значна, — безумовно переважає найнижчий
склад капіталу. Якраз тут, де видається, що рента завдячує своїм походженням
виключно монопольній ціні, потрібні екстраординарно сприятливі ринкові відносини,
щоб можна було продавати товари по їхній вартості, або щоб рента
зробилася рівна всьому надмірові додаткової вартости товару над його ціною
продукції. Так стоїть справа наприклад, з рентою з придатних до рибальства
вод, каменярень, дико ростучих лісів тощо\footnote{
Рікардо розправляється з цим дуже поверхово. Дивись місце, спрямоване проти А.~Сміта
про ренту з лісів у Норвегії, Principles, ch. II, на самому початку.
}.

\section{Рента з будівельних дільниць. Рента з рудень. Ціна землі}

Диференційна рента з’являється і підлягає тим самим законам, що й хліборобська
диференційна рента, всюди, де взагалі існує рента. Всюди, де природні
сили можуть бути монополізовані і забезпечують промисловцеві, що застосовує
їх, певний надзиск, — чи то буде водоспад, чи багата копальня, чи багата на
рибу вода, чи доброго положення будівельне місце, — особа, визнана через
свій титул на частину землі власником цих речей природи, уловлює з капіталу,
що функціонує, цей надзиск у формі ренти. Щодо землі, призначеної для будівельних
цілей, то А.~Сміт показав, яким чином рента з цієї землі, як і рента
з усіх нехліборобських дільниць, реґулюється в своїй основі власне хліборобською
рентою (Book I, chap. XI, 2 і 3). Ця рента характеризується поперше, тим
переважним впливом, що його тут справляє на диференційну ренту положення
(дуже важливе, наприклад, при обробленні винограду і для будівельних дільниць
по великих містах); подруге, очевидністю цілковитої пасивности власника, що його
активність є (особливо щодо копалень) просто в експлуатації поступу суспільного
розвитку, до якого він нічого не додає від себе і в якому він нічим не
ризикує, — хоча б так, як це все-таки робить промисловий капіталіст; і нарешті,
перевагою монопольної ціни в багатьох випадках, особливо у випадках
найбезсоромнішої експлуатації убозтва (бо убозтво є багатше джерело для
домової ренти, ніж яким будьколи були копальні Потозі для Еспанії\footnote{
Laing, Newmann.
},
і потворна могутність, яку дає ця земельна власність, коли вона, сполучаючись
з промисловим капіталом в одних руках, уможливлює собі у боротьбі з робітниками
за заробітну плату практично усувати їх з землі, як з їхнього
житла\footnote{
Crowlington Strike. Engels, Lage der arbeitenden Klasse in England, S. 307 (Видання
1892 року, ст. 259).
}. Одна частина суспільства вимагає тут від другої дані за право
жити на землі, як і взагалі в земельну власність включається право власників
визискувати земну кулю, надра землі, повітря, а разом з тим усе
потрібне для збереження і розвитку життя. Будівельну ренту доконечно підвищує
не тільки зріст людности і ростуча разом з цим потреба в житлах, але
\parbreak{}  %% абзац продовжується на наступній сторінці
