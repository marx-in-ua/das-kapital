\parcont{}  %% абзац починається на попередній сторінці
\index{iii2}{0233}  %% посилання на сторінку оригінального видання
вартість не може бути землею. Абсолютна родючість землі не призводить ні до
чого іншого, як тільки до того, що певна кількість праці дає певний, зумовлений
природною родючістю землі, продукт. Ріжниця в родючості землі призводить
до того, що ті самі кількості праці й капіталу, отже, та сама вартість, виражається
в різних кількостях хліборобських продуктів; отже, що ці продукти
мають різні індивідуальні вартості. Вирівнювання цих індивідуальних вартостей
за ринковими вартостями призводить до того, що «advantages of fertile over inferior
soil\dots{} are transferred from the cultivator or consumer to the landlord» (Ricardo,
Principles, p. 6)\footnote*{
«Вигоди, одержувані від родючішого ґрунту проти гіршого\dots{} переносяться від обробника або
споживача до лендлорда».
}.

І, нарешті, як третій в цій спілці простий привид, — праця «взагалі»,
яка є не що інше, як абстракція і взята сама по собі взагалі не існує, або,
коли ми\dots{} (нерозбірливо) візьмемо, продуктивна діяльність людини взагалі,
з допомогою якої людина упосереджує обмін речовин з природою, не тільки
оголена від усякої суспільної форми і характеристичної визначености, але навіть
і просто в її природному бутті, незалежно від суспільства, абстраговано від
хоч би яких суспільств, і як вияв життя та процес життя, спільна ще несуспільній
людині взагалі з людиною, що має будь-яке суспільне визначення.

\subsubsection{}

Капітал — процент; земельна власність, приватна власть на землю, до того ж
сучасна, відповідна капіталістичному способові продукції, — рента; наймана праця
— заробітна плата. Отже, в цій формі повинен бути зв’язок між джерелами
доходу. Як капітал, так само й наймана праця і земельна власність є історично
визначені суспільні форми; одна — праці, друга — монополізованої землі, і до
того обидві є форми, відповідні капіталові і належні тій самій економічній
формації суспільства.

Перше, що впадає на очі в цій формулі, є те, що поряд з капіталом,
поряд з цією формою одного елементу продукції, належного певному способові
продукції, певній історичній структурі суспільного процесу продукції, поряд
з елементом продукції, що поєднався з певною соціяльною формою і репрезентований
цією соціяльною формою, без дальших околичностей, ставляться: земля
на одному боці, праця — на другому, два елементи реального процесу праці,
які в цій речовій формі спільні всім способам продукції, є речові елементи
всякого процесу продукції, і не мають ніякого чинення до його суспільної форми.

Подруге. У формулі: капітал — процент, земля — земельна рента, праця —
заробітна плата, капітал, земля, праця виступають відповідно як джерела проценту
(замість зиску), земельної ренти і заробітної плати як своїх продуктів,
витворів; перші — основа, другі — наслідок, перші — причина, другі — дія; і до
того ж таким чином, що кожне окреме джерело стоїть до свого продукту в такому
відношенні, як до чогось від нього відштовхнутого і ним спродукованого.
Усі три доходи, процент (замість зиску), рента, заробітна плата, є три частини
вартости продукту, отже, взагалі частини вартости, або в грошовому виразі
певні частини грошей, частини ціни. Хоч формула: капітал — процент і є найіраціональніша формула
капіталу, а проте це — його формула. Але яким чином
земля може створити вартість, тобто суспільно визначену кількість праці і навіть
ту особливу частину вартости її власних продуктів, яка становить ренту. Земля
діє, наприклад, як аґенг продукції при створенні певної споживної вартости,
\parbreak{}  %% абзац продовжується на наступній сторінці
