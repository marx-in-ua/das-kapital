\parcont{}  %% абзац починається на попередній сторінці
\index{iii2}{0168}  %% посилання на сторінку оригінального видання
вони спричинюють зовсім фалшиве уявлення. Коли для ступенів родючости, що
стосуються один до одного, як 1: 2 : 3 : 4 тощо, виникають ренти ряду 0 : 1 : 2 : 3
тощо, то зараз же постає спокуса вивести другий ряд з першого і пояснити
подвоєння, потроєння тощо рент, подвоєнням потроєнням тощо всього здобутку.
Але це було б цілком помилково. Ренти стосуються як 0 : 1 : 2 : 3 : 4 навіть
тоді, коли ступені родючості стосуються як n : n + 1 : n + 2 : n + 3 : n + 4;
ренти стосуються одна до однієї, не як ступені родючости, а як ріжниці родючости,
виходячи з землі, що не дає ренти, як нулевої точки.

Таблиці оригіналу потрібно було навести для пояснення тексту. Але щоб
здобути наочну основу для наведених нижче наслідків дослідження, я далі
даю новий ряд таблиць, що в них здобуток показано в бушелях (\sfrac{1}{8}  квартера,
або 36, 35 літра) і шилінґах (= марці).

Перша таблиця (XI) відповідає давнішій таблиці І. Вона дає здобутки
і ренти для земель п’ятьох якостей A — E, при \emph{першій} витраті капіталу в 50\shil{ шил.}, що разом з 10\shil{ шил.} зиску = 60\shil{ шил.} усієї ціни продукції на акр. Здобутки
збіжжя взято низькі: 10, 12, 14, 16, 18 бушелів з акра. Регуляційна
ціна продукції, яка тут складається, є 6\shil{ шил.} за бушель.

Дальші 13 таблиць відповідають трьом випадкам диференційної ренти II,
розгляненим в цьому і в обох попередніх розділах, при чому припускається, що
\emph{додаткова} витрата капіталу на тій самій землі рівна 50\shil{ шил.} на акр за сталої,
низхідної і висхідної ціни продукції. Кожен з цих випадків знову таки
подається так, як він складається 1) за сталої, 2) за низхідної, 3) за висхідної
продуктивности другої витрати капіталу проти першої. При цьому постають ще
деякі особливо наочні варіянти.

В випадку І: стала ціна продукції, ми маємо:

Варіянт 1: незмінна продуктивність другої витрати капіталу (таблиця XII).

Варіянт 2: низхідна продуктивність. Це може статися лише тоді, коли на землі
А не робиться жодної другої витрати. А саме або:

а) так, що земля В теж не дає ренти (таблиця XIII), або

б) так, що земля В не стає землею, що зовсім не дає ренти (таблиця XIV).

Варіянт 3: висхідна продуктивність (таблиця ХV). І цей випадок виключає
другу витрату капіталу на землю А.

В випадку II: низхідна ціна продукції, ми маємо:

Варіянт 1: незмінна продуктивність другої витрати (таблиця ХVI).

Варіянт 2: низхідна продуктивність (таблиця XVII). Обидва варіянти призводять
до того, що земля А вилучається з числа конкурентних земель, земля
В перестає давати ренту і регулює ціну продукції.

Варіянт 3: висхідна продуктивність (таблиця XVIII). Тут земля А лишається
регуляційною.

У випадку III: висхідна ціна продукції, можливі дві видозміни: земля
А може лишитися землею, що не дає ренти і яка реґулює ціни, або ж в конкуренцію
вступає земля гіршої якости, ніж А, і починає регулювати ціну, так що А
тоді дає ренту.

Перша видозміна: земля А залишається реґуляційною.

Варіянт 1: незмінна продуктивність другої витрати (таблиця XIX). Це припустиме
лише за тієї передумови, що продуктивність першої витрати
зменшується.

Варіянт 2: низхідна продуктивність другої витрати (таблиця XX); це не виключає
того, що продуктивність першої витрати не зміниться.
