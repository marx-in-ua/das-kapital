\parcont{}  %% абзац починається на попередній сторінці
\index{iii2}{0269}  %% посилання на сторінку оригінального видання
для нього — поодинокого капіталіста — входить в склад витрат продукції, продукованих
ним товарів. Так само стоїть справа для хліборобського капіталіста з
земельною рентою у формі встановленої контрактом орендної плати, і в формі
комірного за промислові будівлі для інших підприємців. А що ці частини, на які
розпадається додаткова вартість, з’являються для кожного окремого капіталіста
як дані елементи його витрат продукції, то й видаються вони, навпаки, чинниками,
що створюють додаткову вартість: чинниками, що створюють одну
частину товарової ціни, подібно до того, як заробітна плата створює її другу
частину. Таємниця того, чому ці продукти розпаду товарової вартости завжди
здаються передумовами самого створення вартости, є просто в тому, що капіталістичний
спосіб продукції, як і всякий інший, невпинно репродукує не тільки
матеріяльний продукт, але й суспільно-економічні відносини, економічно певні
форми його утворення. Тому наслідок цього процесу продукції так само постійно
набуває вигляду його передумов, як його передумови — вигляду його наслідку.
І саме ця невпинна репродукція тих самих відносин антиципується окремим
капіталістом як сам собою зрозумілий факт, що не підлягає жодному
сумнівові. Поки капіталістична продукція як така продовжує існувати, одна частина
новодолученої праці постійно перетворюється на заробітну плату, друга
на зиск (процент і підприємницький бариш), третя — на ренту. При складанні
контрактів між власниками різних елементів продукції це є передумова, і ця
передумова правильна, хоч би як коливалась в кожному окремому випадку відносна
величина зазначених трьох частин. Та певна форма, в якій протистоять
одна одній частини вартости, є передумова, бо вона постійно репродукується, і
вона постійно репродукується, бо вона постійно є передумова.

Правда, досвід і зовнішній вигляд явищ показують також, що ринкові
ціни, вплив яких видається капіталістові дійсно єдиним чинником, що визначає
вартості, — що ці ринкові ціни, розглядувані з боку їхньої величини, зовсім не
залежать від цих антиципацій капіталіста, зовсім не рівняються за тим, високий
чи низький процент, висока чи низька рента, зумовлені контрактом. Але
ринкові ціни є сталі лише в зміні, і їхня пересічна за довші періоди саме і дає
відповідні пересічні для заробітної плати, зиску й ренти як сталі величини,
отже, кінець-кінцем панівні над ринковими цінами.

З другого боку, дуже простою здається така думка: коли заробітна плата,
зиск і рента є вартостетворчі чинники, тому що вони видаються передумовами
продукції вартости, і для окремого капіталіста входять як такі передумови в
витрати продукції і ціни продукції, то і стала частина капіталу, що її вартість
входить в продукцію кожного товару як дана величина, є вартостетворчий чинник.
Але стала частина капіталу є не що інше, як сума товарів, а тому і товарових
вартостей. Отже, це сходить на вульґарну тавтологію, що товарова вартість
є витворець і причина товарової вартости.

\looseness=1
Але коли б капіталіст мав якийсь інтерес поміркувати над цим, —
а його міркування як капіталіста визначається виключно його інтересами
і мотивами, що випливають з цих інтересів, — то досвід покаже йому, що
продукт, який він сам продукує, входить в інші сфери продукції як стала
частина капіталу, а продукти цих інших сфер продукції, входять в його продукт
як стала частина капіталу. А що для нього, оскільки в нього відбувається
нова продукція, новостворена вартість складається, як здається, з трьох
величин — заробітної плати, зиску й ренти, — то це, як здається, має силу
і щодо сталої частини, яка складається з продуктів інших капіталістів:
а тому ціна сталої частини капіталу і тим самим і сукупна вартість товарів,
зводиться кінець-кінцем, що правда, не послідовним шляхом, до суми вартости,
яка постає з складання заробітної плати, зиску й ренти, як самостійних
\parbreak{}  %% абзац продовжується на наступній сторінці
