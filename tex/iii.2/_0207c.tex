\parcont{}  %% абзац починається на попередній сторінці
\index{iii2}{0207}  %% посилання на сторінку оригінального видання
якости, вино, яке взагалі може продукуватися лише в порівняно невеликій кількості,
має монопольну ціну. В наслідок цієї монопольної ціни, надмір якої над вартістю
продукту визначається тільки багатством і смаком вельможних споживачів вина,
винороб реалізував значний надзиск. Цей надзиск, який тут випливає з монопольної
ціни, перетворюється на ренту і дістається в цій формі земельному власникові, в
наслідок його титулу на цю дільницю землі, що має особливі властивості. Отже,
ренту тут створює монопольна ціна. Навпаки, рента створила б монопольну ціну,
коли б в наслідок тієї межі, яку покладає земельна власність нерентодайному приміщенню
капіталу на необробленій землі, коли б, у наслідок цієї межі, збіжжя продавалось
не тільки вище його ціни продукції, але й вище його вартости. Що самий
тільки титул власности певного числа осіб на земну кулю дає їм можливість привласнювати
собі частину додаткової праці суспільства, як дань, до того ж привласнювати
її собі з розвитком продукції в постійно ростучому маштабі, — це затушковується
тією обставиною, що капіталізована рента, отже, саме ця капіталізована
дань виступає, як ціна землі, і тому земля може продаватись, як усякий інший
об’єкт торговлі. Тому покупцеві здається, що він одержав своє домагання на ренту
не даром, і не даром одержав до розпорядку працю, риск і підприємецький дух
капіталу, а заплатив за це відповідний еквівалент. Рента, як відзначено вже
давніш, здається йому тільки процентом на капітал, за який він купив землю,
а тому і домагання на ренту. Цілком так само рабовласникові, що купив негра,
здається, що він придбав свою власність на негра не через інститут рабства
як такий, а через купівлю та продаж товарів. Але ж самий титул актом продажу
не породжується, а лише переноситься. Титул мусить існувати до того,
як його можна продати, і продаж, так само як і ряд продажів і їхнє постійне
повторення, не можуть створити цього титулу. Що взагалі створило його, так
це продукційні відносини. Скоро вони досягають такого пункту, де вони мусять
змінити свою шкуру, відпадає матеріяльне джерело титулу, економічно
та історично виправдуваного й виниклого з процесу суспільної продукції
життя, а разом з ним відпадають і засновані на ньому операції. З погляду
вищої економічної формації суспільства, приватна власність окремих індивідуумів
на земну кулю буде здаватись цілком такою самою безглуздою, як приватна
власність однієї людини на іншу людину. Навіть ціле суспільство, нація
і навіть усі одночасно сущі суспільства, узяті разом, не є власники землі. Вони є
лише її посідачі, лише користувачі з неї і як boni patres familias\footnote*{
Pater familias (у римлян) — голова родини; поняття це ширше, ніж в українській мові — бо
«родина» у римлян складалась з членів власне сем’ї, родини разом з усією челяддю, рабами тощо. Boni
patres familias — порядні голови родин. \emph{Прим. Ред.}
} вони мусять залишити її наступним поколінням поліпшеною.

\pfbreak

В дальшому дослідженні ціни землі ми залишаємо осторонь усі коливання
конкуренції, всяку спекуляцію землею, а також дрібну земельну власність, за
якої земля становить головне знаряддя продуцентів, бо вони вимушені купувати її
за всяку ціну.

I.~Ціна землі може підвищитись, хоч рента її не підвіщується; саме:

1) в наслідок просто пониження розміру проценту, яке впливає так, що
ренту продається дорожче, а тому капіталізована рента, ціна землі зростає;

2) тому що зростає процент на долучений до землі капітал.

II.~Ціна землі може підвіщитися тому, що зростає рента.

Рента може зростати тому, що підвищується ціна продукту землі; в цьому
випадку завжди підвищується норма диференційної ренти, незалежно від того,
чи буде рента з найгіршої з оброблюваних земель велика, мала, чи її зовсім
не буде. Під нормою ми розуміємо відношення тієї частини додаткової вартости,
\parbreak{}  %% абзац продовжується на наступній сторінці
