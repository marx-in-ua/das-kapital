\parcont{}  %% абзац починається на попередній сторінці
\index{iii2}{0052}  %% посилання на сторінку оригінального видання
ринку в тій самій мірі, в якій зростає експорт? — Оскільки зростає розцвіт
країни, то і ми» [Chapman’и] «причетні до нього. — 5142. Отже, коли ці різні
ділянки приміщення капіталу раптом поширюються, то природний наслідок нього
є піднесення рівня проценту? — В цьому немає жодного сумніву».

5143. Для Chapman’a «не цілком зрозуміло те, що при нашому великому
вивозі ми мали такий великий ужиток для золота».

5144. Шановний Wilson питає: «Чи не може бути так, що ми даємо більші
кредити під наш вивіз, аніж беремо під наш довіз? — Я сам маю сумнів щодо цього
пункту. Коли хтось акцептує вексель під свої менчестерські товари, відправлені
до Індії, то ви не можете акцептувати його менше, ніж на 10 місяців.
Ми маємо — і це цілком певна річ — платити Америці за її бавовну трохи раніше,
ніж Індія заплатить нам за товари; але дослідити, як це впливає, є досить
тяжка справа. — 5145. Коли б ми мали збільшення вивозу мануфактурних
товарів, як це було в минулому році, на 20 міл. ф. ст., то раніше ми все ж
мусили мати дуже значне збільшення довозу сирових матеріялів «[і вже тому надмірний
експорт є тотожній з надмірним імпортом, а надмірна продукція — з надмірною
торговлею]», щоб випродукувати цю збільшену кількість товарів? — Річ
безперечна; ми мусили б оплатити дуже значний балянс; тобто протягом того
часу балянс мусив бути несприятливий для нас, але вексельний курс в наших
розрахунках з Америкою протягом довшого часу є сприятливий для нас і ми
одержували протягом довшого часу чимало благородних металів з Америки».

5148. Wilson питає архілихваря Chapman’a, чи не вважає він свої високі
проценти за ознаку великого розцвіту та високого рівня зисків. Chapman, очевидно,
здивований наївністю цього сикофанта, природно, відповідає на це питання
позитивно, проте він досить щирий, щоб зробити таке застереження: «Є декотрі,
що не можуть собі дати іншої ради; вони мають виконати певні зобов'язання
й мусять їх виконати, чи матимуть від того зиск, чи ні; але коли він»
[високий рівень проценту] «тримається довго, то це свідчило б про розцвіт». Обидва
забувають, що такий процент може бути також ознакою того, що пройдисвіти
кредиту — як то було в 1857 році — роблять стан країни непевним, та мають змогу
платити високий процент, бо вони платять його з чужої кишені (при цьому однак
вони сприяють усталенню такого рівня проценту для всіх), а тимчасом живуть
розкішно за рахунок майбутніх зисків. Проте саме це може бути одночасно й для
фабрикантів і т. ін. справою, що справді даватиме дуже добрі зиски. Зворотний
приплив капіталів через систему позик стає цілком оманливим. Це пояснює й
далі наведене, що проте відносно Англійського банку не потребує жодного пояснення,
бо за високого рівня проценту він дисконтує за нижчий процент, ніж інші.

«5156. Я можу напевно сказати, каже Chapman, що в даний момент, після
того як ми протягом такого довгого часу мали високий рівень проценту, суми
нашого дисконту досягають свого максимуму». [Це сказав Chapman 21 липня
1857 року, кілька місяців перед крахом].

«5157. В 1852 році» [коли процент був низький] «вони були далеко не
такі великі». Бо в дійсності в ті часи справи були далеко здоровіші.

«5159. Коли б на ринку був великий надмір грошей\dots{} а банковий дисконт
був низький, ми мали б зменшення числа векселів\dots{} в 1852 році ми
перебували в цілком іншій фазі. Вивіз та довіз країни були тоді ніщо проти
того, що маємо за наших часів. — 5161. За цієї високої норми дисконту обсяг
наших дисконтових операцій є такого ж розміру, як і року 1854». [Коли процент
був 5—5\sfrac{1}{2}].

Надзвичайно цікаве в свідченнях Chapman’а те, що ці люди дійсно вважають
гроші публіки за свою власність і думають, що мають право на повсякчасний
обмін дисконтованих ними векселів на гроші. Наївність в питаннях
та відповідях велика. Законодавство, мовляв, має обов’язок уможливити постійний
\parbreak{}  %% абзац продовжується на наступній сторінці
