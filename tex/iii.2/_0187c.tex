\parcont{}  %% абзац починається на попередній сторінці
\index{iii2}{0187}  %% посилання на сторінку оригінального видання
ціни продукції пересічною ціною продукції з $А$; отже, це тримало б ціну продукції
на вищому рівні, ніж це потрібно, і таким чином створило б ренту.
Навіть при вільному довозі хліба з-за кордону такий результат міг би скластись
або триматись, бо орендарі вимушені були б для землі, яка при зовні
визначеній ціні продукції могла б конкурувати у продукції збіжжя, не даючи
ренти, дати інше призначення, наприклад, призначити її під пасовисько, і таким
чином лише рентодайні землі були б зайняті під збіжжя, тобто лише землі,
на яких індивідуальна пересічна ціна продуктції за квартер була б нижча від
ціни продукції, визначуваної зовні. В цілому можна визнати, що в даному випадку
ціна продукції понизиться, але не до рівня пересічної ціни, і буде стояти
вище від неї, але нижче ціни продукції на гірше оброблюваній землі $А$, так
що конкуренцію нової землі $А$ буде обмежено.

\emph{2) За низхідної продуктивної сили додаткових капіталів}

Припустімо, що земля $А_{-1}$ могла б випродукувати додатковий квартер
лише за 4\pound{ ф. стерл.}, а земля $А$ за 3\sfrac{3}{4}, отже дешевше ніж $А_{-1}$ але на \sfrac{3}{4}\pound{ ф. стерл.} дорожче, ніж квартер, випродукований першою витратою капіталу на
$А$. В цьому випадку вся ціна двох випродукованих на $А$ квартерів була б \deq{}
6\sfrac{3}{4}\pound{ ф. стерл.}; отже, пересічна ціна за квартер \deq{} 3\sfrac{3}{8}\pound{ ф. стерл}. Ціна продукції
пидвищилася б, але лише на \sfrac{3}{8}\pound{ ф. стерл.}, тимчасом як коли б додатковий
капітал був витрачений на новій землі, яка продукує квартер за 3\sfrac{3}{4}\pound{ ф. стерл.}, вона підвищилася б на дальші \sfrac{3}{8}\pound{ ф. стерл.} до 3\sfrac{3}{4}\pound{ ф. стерл.}, і цим
було б спричинене відповідне підвищення усіх інших диференційних рент.

Ціна продукції в 3\sfrac{3}{8} ф, стерл. за квартер на землі $А$ таким чином
вирівнялася б за пересічною ціною продукції на тій самій землі за збільшеної
витрати капіталу і стала б реґуляційною; отже, вона не дала б ренти, бо не
було б надзиску.

Але коли б цей квартер, випродукований другою витратою капіталу, був проданий
за 3\sfrac{3}{4}\pound{ ф. стерл.}, то земля $А$ дала б тепер ренту в \sfrac{3}{4}\pound{ ф. стерл.},
дала б її також і на всі акри $А$, на яких не зроблено додаткової витрати і
які, отже, як і давніш, продукують квартер за 3\pound{ ф. стерл}. Поки існують ще
необроблені дільниці землі $А$, ціна могла б лише тимчасове підвищитись до
3\sfrac{3}{4}\pound{ ф. стерл}. Конкуренція нових дільниць $А$ підтримувала б ціну продукції
на 3\pound{ ф. стерл.}, поки не були б вичерпані всі землі $А$, що їхнє сприятливе положення
дає їм можливість продукувати квартер дешевше, ніж за 3\sfrac{3}{4}\pound{ ф. стерл}.
Отже, доводиться припустити це, хоч власник землі, коли один акр землі дає ренту,
не відступить орендареві другого акра без ренти.

Чи вирівняється ціна продукції відповідно до пересічної ціни, чи зареґуляційну
зробиться індивідуальна ціна продукції другої витрати капіталу в 3\sfrac{3}{4}\pound{ ф. стерл.},
це залежить знов таки від того, більшого чи меншого загального поширення набула
друга витрата капіталу на наявній землі $А$. За реґуляційну ціну стає 3\sfrac{3}{4}\pound{ ф. стерл.} тільки в тому випадку, коли у землевласника є досить часу для того,
щоб фіксувати як ренту той надзиск, який одержувано б при ціні в 3\sfrac{3}{4}\pound{ ф.
стерл.} за квартер, поки не задовольниться попиту.

\pfbreak

Щодо низхідної продуктивности землі за послідовних витрат капіталу,
слід подивитися Лібіха. Ми бачили, що послідовне зменшення додаткової продуктивної
сили витрат капіталу постійно збільшує ренту з акра, коли ціна
продукції не змінюється, і що воно може призвести до цього навіть за низхідної
ціни продукції.

Але взагалі треба відзначити таке:

З погляду капіталістичного способу продукції відносне подорожчання
продукту відбувається завжди, коли для одержання того самого продукту
\parbreak{}  %% абзац продовжується на наступній сторінці
