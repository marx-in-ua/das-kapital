\parcont{}  %% абзац починається на попередній сторінці
\index{iii2}{0083}  %% посилання на сторінку оригінального видання
до Індії. Але тут надходять вимоги від India House. India House оповіщає, що
може видати тратти на різні президентства в Індії на суму \num{3.250.000}\pound{ ф. ст.}
[Цю суму бралося на лондонські видатки ост-індської компанії та на виплату
дивідентів акційникам]. І це не тільки ліквідує балянс в \num{2.250.000}\pound{ ф. ст.}, що
постав у процесі торговлі, а й ще дає мільйон надміру». (1917).

1922. [Wood:] «Отже, тоді вплив цих тратт India House не в тому, що
вони збільшують вивіз до Індії, а в тому, що вони його pro tanto зменшують?»
[Має означати, зменшування потреби покривати довіз з Індії вивозом туди на ту
саму суму.] Пан Newmarch пояснює це тим, що англійці на ці \num{3.700.000}\pound{ ф. ст.}
імпортують до Індії «добрий уряд». (1925). Wood, що як міністр у справах Індії дуже
добре знав той імпортований англійцями сорт «доброго уряду», слушно й іронічно
каже, 1926: «В такому разі вивіз, що його, як ви кажете, зумовлюють
тратти India House, є вивіз доброго уряду, а не товарів». А що Англія
«цим способом» багато експортує для «доброго уряду» й для приміщення
капіталу по закордонних країнах, — отже одержує довіз, цілком незалежний від
звичайного розвитку справ, данину, частиною за експортований «добрий уряд»,
частиною як дохід від капіталу, приміщеного в колоніях або де інде, данину,
за яку вона не має платити жодного еквіваленту — то й очевидно, що вексельні
курси не порушуються, коли Англія цю данину просто з’їдає без відповідного
свого експорту; отже, очевидно, й те, що курси не порушується, коли цю данину
знову приміщується продуктивно або непродуктивно не в Англії, а закордоном,
коли, напр., за неї посилається зброю до Криму. А до того, оскільки довіз з закордону
ввіходить в дохід Англії — звичайно його доводиться оплачувати, або як
данину, де не треба жодного еквіваленту, або обміном на цю неоплачену данину
або ж звичайним у торговлі способом, — Англія може або спожити той довіз, або
знову примістити його як капітал. Ні те, ні це не порушує вексельних курсів,
і цього недобачає премудрий Вілсон. Чи певну частину доходу являє тубільний,
чи чужоземний продукт — а в цьому останньому випадку передумовою є
лише обмін тубільних продуктів на закордонні, — споживання цього доходу, однаково
продуктивне чи непродуктивне, нічого не зміняє у вексельних курсах,
хоч і робить зміни у продукції. Згідно з цим треба розглядати нижченаведене.

1934 Wood питає його, як відправа військових, запасів до Криму мала б
вплинути на вексельний курс з Турцією. Newmarch відповідає: «Я не розумію,
чому проста відправа військових припасів мала б неминуче вплинути на вексельний
курс, але відправа благородного металу напевно вплинула б на той
курс». Отже, тут він відрізняє капітал у грошовій формі від іншого капіталу.
Але ось Вілсон питає:

«1935. Коли ви організуєте експорт якогось товару в широкому маштабі,
при чому не відбувається еквівалентного йому імпорту» [Пан Вілсон забуває,
що відносно до Англії відбувається дуже значний імпорт, при чому ніколи не
було відповідного йому експорту, хіба тільки в формі експорту «доброго уряду»
або капіталу, раніше експортованого для приміщення; в усякому разі це не той
імпорт, що ввіходить у регулярний торговельний рух. Але цей імпорт знову
обмінюється, напр., на американський продукт, а те, що американський
продукт експортується без відповідного імпорту, нічого не зміняє в тому, що
вартість цього імпорту можна спожити без еквівалентного експорту закордон;
його одержали без еквівалентного експорту, а тому він і може споживатись, не
ввіходячи в торговельний балянс], «то ви не оплачуєте закордонного боргу, що
його ви зробили з причини свого довозу». [Але коли ви цей імпорт вже
раніш оплатили, напр., закордоним кредитом, то ви таким способом не зробили
ніякого боргу, й це питання нічогісінько не має до діла з міжнароднім балянсом;
воно сходить до продуктивного або непродуктивного витрачання продуктів,
однаково, чи ті спожиті продукти становлять тубільний, чи закордонний виріб]
\parbreak{}  %% абзац продовжується на наступній сторінці
