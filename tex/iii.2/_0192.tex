\parcont{}  %% абзац починається на попередній сторінці
\index{iii2}{0192}  %% посилання на сторінку оригінального видання
капіталу відпадає, і саме через договір з самим земельним власником. Але він не платить
ренти за ці дільниці тільки тому, що він платить ренту за землю, до
якої вони належать. Тут припускається якраз така комбінація, коли доводиться
звернуть до гіршого роду землі $А$ не як до самостійного нового поля продукції,
яке покрило б недостатнє подання, а як до такого, що становить лише
неподільну смугу в кращій землі. А випадок, який ми маємо дослідити, є якраз той,
коли доводиться самостійно провадити господарство на дільницях землі роду $А$,
отже, коли вони мусять за наявности загальних передумов капіталістичного способу
продукції здаватися в оренду як самостійні дільниці.

\emph{Третє}: Орендар може вкласти додатковий капітал у ту саму орендовану
дільницю, хоч за сущих ринкових цін одержувана в такий спосіб додаткова
продукція дає йому лише ціну продукції, звичайний зиск, але не дає йому
змоги платити додаткову ренту. Таким чином, на одну частину капіталу, вкладеного
в землю, він виплачує земельну ренту, на другу — ні. Як мало це припущення
розв’язує проблему, видно ось з чого: коли ринкова ціна (і разом
з цим родючість землі) дає йому можливість на додатковий капітал одержувати
додатковий здобуток, який подібно до старого капіталу дає йому, крім ціни продукції,
певний надзиск, то він бо скінчення терміну орендного договору залишає
його в себе. Але чому? Тому, що поки триває термін орендного договору, відпадає
та межа для примінення його капіталу у землю, яку створює земельна
власність. Проте, та звичайна обставина, що для забезпечення йому цього надзиску
мусить розпочатися самостійний обробіток додаткової гіршої землі і її самостійне
заорендування, незаперечно доводить, що приміщення додаткового капіталу
у стару землю не досить для створення потрібного підвищеного подання.
Одно припущення виключає друге. Правда, тепер можна було б сказані: сама
рента з найгіршого роду землі $А$ є диференційна рента, чи то порівняно з землею,
яка обробляється самим власником (проте, це трапляється у чистому вигляді
лише як випадковий виняток), чи то порівняно з додатковим приміщенням
капіталу на тих старих заорендованих дільницях землі, що не дають ренти.
Але це була б 1) така диференційна рента, що виникала б не з ріжниці родючости
різних родів землі, а тому не мала б за свою передумову того, що земля
роду $А$ не дає ренти, і що продукти її продається по ціні продукції; і 2) та обставина,
чи дають ренту додаткові приміщення капіталу на тій самій заорендованій
дільниці, чи ні, цілком також байдужа щодо того, чи виплачує ренту новооброблювана
земля кляси $А$, чи ні, так само як наприклад, для заснування нового самостійного
фабричного підприємства байдуже, чи вкладе інший фабрикант тієї
самої галузі підприємств у процентні папери частину свого капіталу, не бувши в
стані її цілком використати у своєму підприємстві, чи він зробить ряд таких окремих
розширень, що не дають йому повного зиску, а проте дають більше за процент.
Це для нього справа другорядна. Навпаки, додаткові нові підприємства мусять
давати пересічний зиск і споруджуються в надії на пересічний зиск. В усякому
разі, додаткові приміщення капіталу на старих заорендованих дільницях
землі і додаткове оброблення нової землі роду $А$ становлять межі одне для одного.
Межа, до якої в ту саму заорендовану дільницю може вкладатись додатковий
капітал за менш сприятливих умов продукції, визначається конкурентними
новими приміщеннями у землю кляси $А$; з другого боку, рента, яку
може давати земля цієї кляси, обмежується конкурентними додатковими приміщеннями
капіталу на старих заорендованих землях.

Проте, всі ці фалшиві викрути не розв’язують проблеми, яка в простій
поставі така: припустімо, що ринкова ціна збіжжя (яке в цьому дослідженні
є для нас за представника всякого продукту землі) достатня для того, щоб
можна було почати оброблення частин землі кляси $А$, і щоб капітал, вкладений
у ці нові лани, здобув ціну продукції продукту, тобто покриття капіталу плюс
\parbreak{}  %% абзац продовжується на наступній сторінці
