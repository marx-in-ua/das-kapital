\parcont{}  %% абзац починається на попередній сторінці
\index{iii2}{0107}  %% посилання на сторінку оригінального видання
побачимо, як визначається її ціну, — її вартість тепер уже стала вища. Він продає
не тільки землю, але поліпшену землю, долучений до землі капітал, що йому
нічого не коштував. Це одна з таємниць — цілком не вважаючи на рух власне
земельної ренти — чимраз більшого збагачування земельних власників, невпинного бубнявіння їхніх рент
та зросту грошової вартости їхніх земель з поступом
економічного розвитку. Так земельні власники ховають до своєї кешені цей
результат суспільного розвитку, що склався без їхньої допомоги, — fruges consumere nati\footnote*{
Народжені, щоб споживати плоди. \Red{Прим. Ред.}
}. Але
одночасно це є одна з найбільших перешкод для раціонального хліборобства, бо фармер уникає всяких
поліпшень та витрат, що їхнього
повного повороту не можна сподіватися протягом часу його оренди; і ми
бачимо, що цю обставину чим далі більш проголошують за таку перешкоду
так само в минулому віці James Anderson, що власне винайшов новітню теорію
ренти, та одночасно був практиком-фармером і видатним для свого часу агрономом, як в наші дні
противники сучасної побудови земельної власности в Англії.

A.~A.~Walton в «History of the Landed Tenures of Greath Britain and
Ireland» 1865, на стор. 96, 97 говорить про це так: «Всі намагання численних
сільсько-господарських установ нашої країни не в стані дати дуже значних або
дійсно помітних результатів щодо дійсного поступу поліпшеного обробітку землі,
поки такі поліпшення збільшують вартість земельної власности та висоту ренти
земельного власника до далеко вищого ступеня, ніж поліпшують стан фармера
або сільського робітника. Загалом кажучи, фармери точнісінько так само, як
і земельний власник, або його скарбник-управитель або навіть сам президент
сільсько-господарського товариства, знають, що добрий дренаж, добре угноєння та добре
господарювання, разом з збільшеним ужитком праці для
ґрунтовного очищення й оброблення землі, даватимуть дивовижні результати
як щодо поліпшення ґрунту, так і щодо піднесення продукції. Але все це
потребує значних витрат, а фармери так само добре знають, що хоч і як
вони поліпшуватимуть землю та підвищуватимуть її вартість, однаково головну користь від цього
пізніше пожнуть земельні власники в формі підвищеної ренти та збільшеної вартости землі\dots{} Вони
досить мудрі, щоб
примітити те, що ті промовці [землевласники та їхні управителі на сільськогосподарських
бенкетах] надиво завжди забувають їм сказати, а власне, що
левова пайка від усіх пороблених фармером поліпшень завжди мусить іти,
кінець-кінцем, до кешені земельного власника\dots{} Хоч і як попередній фармер
поліпшив орендовану землю, його наступник завжди бачитиме, що земельний
власник підвищить ренту відповідно до піднесеної попередніми поліпшеннями
вартости землі».

\disablefootnotebreak{}
У власне хліборобстві цей процес виявляється ще не так ясно, як
от при використуванні землі для будівництва. Переважну частину землі, що її
продають в Англії для будівельних цілей, а не як freehold\footnote*{
Форма власності на землю, при якій власник має право на необмежений термін володіння і використання цієї землі \Red{Прим. Ред.}
} земельні власники
винаймають на 99 років або, коли можна, на коротший час. Коли мине цей
час, будівлі з самою землею дістаються земельному власникові. «Вони [орендарі]
зобов’язуються по закінченні контракту наймів — по тому, як вони аж до цього
моменту платили прибільшену земельну ренту — передати дім великому земельному
власникові в доброму для житла стані. Ледве закінчився цей контракт,
як от приходить аґент або інспектор того земельного власника, оглядає ваш
дім, дбаючи про те, щоб ви довели його до доброго стану, потім забирає його
у володіння землевласника, анексуючи його до земель останнього. Факт той, що,
коли ця система в своїй повній силі лишиться ще протягом довшого часу; то
вся домовласність в королівстві, тав само як і сільська землевластність, буде
\parbreak{}  %% абзац продовжується на наступній сторінці
