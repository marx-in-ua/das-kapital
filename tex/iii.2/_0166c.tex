
\index{iii2}{0166}  %% посилання на сторінку оригінального видання

\begin{table}[H]
  \begin{center}
    \emph{Таблиця X}
    \footnotesize

  \begin{tabular}{c@{  } c@{  } c@{  } c@{  } c@{  } c@{  } c@{  } c@{  } c@{  } c@{  } c}
    \toprule
      \multirowcell{2}{\makecell{Рід\\ землі}} &
      \multirowcell{2}{Акри} &
      Капітал &
      Зиск &
      \makecell{Ціна\\ продук.} &
      \multirowcell{2}{\makecell{Продукт в\\ квартерах}} &
      \makecell{Продажна \\ ціна} &
      \makecell{Здо-\\буток} &
      \multicolumn{2}{c}{Рента} &
      \multirowcell{2}{\makecell{Норма \\ренти}} \\

      \cmidrule(r){3-3}
      \cmidrule(r){4-4}
      \cmidrule(r){5-5}
      \cmidrule(r){7-7}
      \cmidrule(r){8-8}
      \cmidrule(r){9-9}
      \cmidrule(r){10-10}

       &  & ф. ст. & ф. ст. & ф. ст. & & ф. ст. & ф. ст. & Кварт. & ф. ст. &   \\
      \midrule
      A & 1 & 2\sfrac{1}{2} + 2\sfrac{1}{2} = 5 & 1 & 6 & 1 + \phantom{0}\sfrac{1}{4} = 1\sfrac{1}{4}            & 4\sfrac{4}{5} & \phantom{0}6 & 0\phantom{\sfrac{1}{2}} & \phantom{0}0 & \phantom{00}0\% \\
      B & 1 & 2\sfrac{1}{2} + 2\sfrac{1}{2} = 5 & 1 & 6 & 2 + \phantom{0}\sfrac{1}{2} = 2\sfrac{1}{2}            & 4\sfrac{4}{5} & 12           & 1\sfrac{1}{4}           & \phantom{0}6 & 120\% \\
      C & 1 & 2\sfrac{1}{2} + 2\sfrac{1}{2} = 5 & 1 & 6 & 3 + \phantom{0}\sfrac{3}{4} = 3\sfrac{3}{4}            & 4\sfrac{4}{5} & 18           & 2\sfrac{1}{2}           & 12           & 240\%\\
      D & 1 & 2\sfrac{1}{2} + 2\sfrac{1}{2} = 5 & 1 & 6 & 4 + 1\phantom{\sfrac{0}{0}} = 5\phantom{\sfrac{0}{0}}  & 4\sfrac{4}{5} & 24           & 3\sfrac{3}{4}           & 18           & 360\%\\

     \cmidrule(r){3-3}
     \cmidrule(r){5-5}
     \cmidrule(r){6-6}
     \cmidrule(r){8-8}
     \cmidrule(r){9-9}
     \cmidrule(r){10-10}
     \cmidrule(r){11-11}

      Разом & & \phantom{2\sfrac{1}{2} + 2\sfrac{1}{2} =}20 & & 24 & \phantom{2 + 1\sfrac{1}{2} =}12\sfrac{1}{2} & & 60 & 7\sfrac{1}{2} & 36 & 240\%\\
  \end{tabular}

  \end{center}
\end{table}

В цій таблиці загальний здобуток, сума грошової ренти і норма ренти
теж лишаються такі самі, як у таблицях, II, VII і VIII, бо продукт і продажна
ціна знов таки змінились у зворотному відношенні, а капіталовкладення лишилось
те саме.

Але як стоїть справа в іншому випадку, можливому за висхідної ціни
продукції, а саме в тому випадку, коли гірша земля, яку до цього часу не
варто було обробляти, тепер починає оброблятись.

Припустімо, що така земля, яку ми позначимо \emph{а}, вступає в конкуренцію.
Тоді земля $А$, що не давала до цього часу ренти, почала б давати ренту, і
вищенаведені таблиці VIII, VIII і X набули б такого вигляду:

\begin{table}[H]
  \begin{center}
    \emph{Таблиця VIIa}
    \footnotesize

  \begin{tabular}{c@{  } c@{  } c@{  } c@{  } c@{  } c@{  } c@{  } c@{  } c@{  } c@{  } c}
    \toprule
      \multirowcell{2}{\makecell{Рід\\ землі}} &
      \multirowcell{2}{Акри} &
      Капітал &
      Зиск &
      \makecell{Ціна\\ продук.} &
      \multirowcell{2}{\makecell{Продукт в\\ квартерах}} &
      \makecell{Продажна \\ ціна} &
      \makecell{Здо-\\буток} &
      \multicolumn{2}{c}{Рента} &
      \multirowcell{2}{Підвищення} \\

      \cmidrule(r){3-3}
      \cmidrule(r){4-4}
      \cmidrule(r){5-5}
      \cmidrule(r){7-7}
      \cmidrule(r){8-8}
      \cmidrule(r){9-9}
      \cmidrule(r){10-10}

       &  & ф. ст. & ф. ст. & ф. ст. & & ф. ст. & ф. ст. & Кварт. & ф. ст. &   \\
      \midrule
      a & 1 & \phantom{2\sfrac{1}{2} + }5\phantom{\sfrac{1}{2}} & 1 & 6 & \phantom{1\sfrac{1}{2} + 3\sfrac{3}{4} = }1\sfrac{1}{2}                     & 4 & \phantom{0}6 & 0\phantom{\sfrac{1}{2}} & \phantom{0}0 & 0\phantom{+ 3 × 7} \\
      A & 1 & 2\sfrac{1}{2} + 2\sfrac{1}{2}                     & 1 & 6 & \phantom{0}\sfrac{1}{2} + 1\sfrac{1}{4} = 1\sfrac{3}{4}                     & 4 & \phantom{0}7 & \phantom{}\sfrac{1}{4}  & \phantom{0}1 & 1\phantom{+ 3 × 7} \\
      B & 1 & 2\sfrac{1}{2} + 2\sfrac{1}{2}                     & 1 & 6 & 1\phantom{\sfrac{0}{0}} + 2\sfrac{1}{2} = 3\sfrac{1}{2}                     & 4 & 14           & 2\phantom{\sfrac{1}{2}} & \phantom{0}8 & 1 + 7\phantom{ × 7} \\
      C & 1 & 2\sfrac{1}{2} + 2\sfrac{1}{2}                     & 1 & 6 & 1\sfrac{1}{2} + 3\sfrac{3}{4} = 5\sfrac{1}{4}                               & 4 & 21           & 3\sfrac{3}{4}           & 15           & 1 + 2 × 7\\
      D & 1 & 2\sfrac{1}{2} + 2\sfrac{1}{2}                     & 1 & 6 & 2\phantom{\sfrac{0}{0}} + 5\phantom{\sfrac{0}{0}} = 7\phantom{\sfrac{0}{0}} & 4 & 28           & 5\sfrac{1}{2}           & 22           & 1 + 3 × 7\\

     \cmidrule(r){5-5}
     \cmidrule(l){6-6}
     \cmidrule(r){8-8}
     \cmidrule(r){9-9}
     \cmidrule(r){10-10}

      Разом & & & & 30 & \phantom{2 + 1\sfrac{1}{2} =}19\phantom{\sfrac{1}{2}} & & 76 & 11\sfrac{1}{2} & 46 & \\
  \end{tabular}

  \end{center}
\end{table}

\begin{table}[H]
  \begin{center}
    \emph{Таблиця VIIIa}
    \footnotesize

  \begin{tabular}{c@{  } c@{  } c@{  } c@{  } c@{  } c@{  } c@{  } c@{  } c@{  } c@{  } c}
    \toprule
      \multirowcell{2}{\makecell{Рід\\ землі}} &
      \multirowcell{2}{Акри} &
      Капітал &
      Зиск &
      \makecell{Ціна\\ продук.} &
      \multirowcell{2}{\makecell{Продукт в\\ квартерах}} &
      \makecell{Продажна \\ ціна} &
      \makecell{Здо-\\буток} &
      \multicolumn{2}{c}{Рента} &
      \multirowcell{2}{Підвищення} \\

      \cmidrule(r){3-3}
      \cmidrule(r){4-4}
      \cmidrule(r){5-5}
      \cmidrule(r){7-7}
      \cmidrule(r){8-8}
      \cmidrule(r){9-9}
      \cmidrule(r){10-10}

       &  & ф. ст. & ф. ст. & ф. ст. & & ф. ст. & ф. ст. & Кварт. & ф. ст. &   \\
      \midrule
      a & 1 & \phantom{2\sfrac{1}{2} + }5\phantom{\sfrac{1}{2}} & 1 & 6 & \phantom{1\sfrac{1}{2} + 3 = }1\sfrac{1}{4}           & 4\sfrac{4}{5} & \phantom{0}6\phantom{\sfrac{1}{5}} & 0\phantom{\sfrac{1}{2}} & \phantom{0}0             & 0\phantom{\sfrac{1}{5} + 3 × 7\sfrac{1}{5}} \\
      A & 1 & 2\sfrac{1}{2} + 2\sfrac{1}{2}                     & 1 & 6 & \phantom{0}\sfrac{1}{2} + 1 = 1\sfrac{1}{2}           & 4\sfrac{4}{5} & \phantom{0}7\sfrac{1}{5}           & \phantom{}\sfrac{1}{4}  & \phantom{0}1\sfrac{1}{5} & 1\sfrac{1}{5}\phantom{ + 3 × 7\sfrac{1}{5}} \\
      B & 1 & 2\sfrac{1}{2} + 2\sfrac{1}{2}                     & 1 & 6 & 1\phantom{\sfrac{0}{0}} + 2 = 3\phantom{\sfrac{1}{2}} & 4\sfrac{4}{5} & 14\sfrac{2}{5}                     & 1\phantom{\sfrac{3}{4}} & \phantom{0}8\sfrac{2}{5} & 1\sfrac{1}{5} + 7\sfrac{1}{5}\phantom{ × 7} \\
      C & 1 & 2\sfrac{1}{2} + 2\sfrac{1}{2}                     & 1 & 6 & 1\sfrac{1}{2} + 3 = 4\sfrac{1}{2}                     & 4\sfrac{4}{5} & 21\sfrac{3}{5}                     & 2\sfrac{1}{4}           & 15\sfrac{3}{5}           & 1\sfrac{1}{5} + 2 × 7\sfrac{1}{5}\\
      D & 1 & 2\sfrac{1}{2} + 2\sfrac{1}{2}                     & 1 & 6 & 2\phantom{\sfrac{0}{0}} + 4 = 6\phantom{\sfrac{0}{0}} & 4\sfrac{4}{5} & 28\sfrac{4}{5}                     & 4\sfrac{3}{4}           & 22\sfrac{4}{5}           & 1\sfrac{1}{5} + 3 × 7\sfrac{1}{5}\\

     \cmidrule(r){2-2}
     \cmidrule(r){5-5}
     \cmidrule(r){6-6}
     \cmidrule(r){8-8}
     \cmidrule(r){9-9}
     \cmidrule(r){10-10}

      Разом & 5 & & & 30 & \phantom{2 + 1\sfrac{1}{2} =}16\sfrac{1}{4} & & 78\phantom{\sfrac{1}{5}} & 9\phantom{\sfrac{1}{2}} & 48 & \\
  \end{tabular}

  \end{center}
\end{table}

