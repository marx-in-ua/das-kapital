
\begin{table}[H]
  \centering
  \caption*{Таблиця X}

  \footnotesize
  \setlength{\tabcolsep}{4.5pt}
  \settowidth\rotheadsize{\theadfont Продажна}

  \begin{tabular}{l c r c c r c c c c c}
    \toprule
      \thead[tl]{Рід\\землі} &
      &
      \thead[t]{Капітал} &
      \rothead{Зиск} &
      \rothead{Ціна\\продукції} &
      \thead[t]{Продукт} & % \\ в кварт.}}}
      \rothead{Продажна\\ціна} &
      \rothead{Здобуток} &
      \multicolumn{2}{c}{Рента} &
      \rothead{Норма\\надзиску} \\

    \cmidrule(rl){2-11}
      & акри  & \poundsign{} & \poundsign{} & \poundsign{} & кв. & \poundsign{} & \poundsign{} & кв. & \poundsign{} & \% \\

    \midrule
      A & 1 & 2\tbfrac{1}{2} \dplus{} 2\tbfrac{1}{2} \deq{} 5 & 1 & 6 & 1 \dplus{} \phantom{1}\tbfrac{1}{4} \deq{} 1\tbfrac{1}{4}            & 4\tbfrac{4}{5} & \phantom{0}6 & 0\phantom{\tbfrac{1}{2}} & \phantom{0}0 & \phantom{00}0\\
      B & 1 & 2\tbfrac{1}{2} \dplus{} 2\tbfrac{1}{2} \deq{} 5 & 1 & 6 & 2 \dplus{} \phantom{1}\tbfrac{1}{2} \deq{} 2\tbfrac{1}{2}            & 4\tbfrac{4}{5} & 12           & 1\tbfrac{1}{4}           & \phantom{0}6 & 120\\
      C & 1 & 2\tbfrac{1}{2} \dplus{} 2\tbfrac{1}{2} \deq{} 5 & 1 & 6 & 3 \dplus{} \phantom{1}\tbfrac{3}{4} \deq{} 3\tbfrac{3}{4}            & 4\tbfrac{4}{5} & 18           & 2\tbfrac{1}{2}           & 12           & 240\\
      D & 1 & 2\tbfrac{1}{2} \dplus{} 2\tbfrac{1}{2} \deq{} 5 & 1 & 6 & 4 \dplus{} 1\phantom{\tbfrac{1}{1}} \deq{} 5\phantom{\tbfrac{1}{1}}  & 4\tbfrac{4}{5} & 24           & 3\tbfrac{3}{4}           & 18           & 360\\

    \midrule
      Разом & & 20 & & \hang{r}{2}4 & 12\tbfrac{1}{2} & & 60 & 7\tbfrac{1}{2} & 36 & 240\\
  \end{tabular}
\end{table}

\noindent{}В цій таблиці загальний здобуток, сума грошової ренти і норма ренти
теж лишаються такі самі, як у таблицях, II, VII і VIII, бо продукт і продажна
ціна знов таки змінились у зворотному відношенні, а капіталовкладення лишилось
те саме.

Але як стоїть справа в іншому випадку, можливому за висхідної ціни
продукції, а саме в тому випадку, коли гірша земля, яку до цього часу не
варто було обробляти, тепер починає оброблятись.

Припустімо, що така земля, яку ми позначимо \emph{а}, вступає в конкуренцію.
Тоді земля $А$, що не давала до цього часу ренти, почала б давати ренту, і
вищенаведені таблиці VII, VIII і X набули б такого вигляду:

\begin{table}[H]
  \begin{center}
    \emph{Таблиця VIIa}
    \footnotesize

  \begin{tabular}{c@{  } c@{  } c@{  } c@{  } c@{  } c@{  } c@{  } c@{  } c@{  } c@{  } c}
    \toprule
      \multirowcell{2}{\makecell{Рід\\ землі}} &
      \multirowcell{2}{Акри} &
      Капітал &
      Зиск &
      \makecell{Ціна\\ продук.} &
      \multirowcell{2}{\makecell{Продукт в\\ квартерах}} &
      \makecell{Продажна \\ ціна} &
      \makecell{Здо-\\буток} &
      \multicolumn{2}{c}{Рента} &
      \multirowcell{2}{Підвищення} \\

      \cmidrule(r){3-3}
      \cmidrule(r){4-4}
      \cmidrule(r){5-5}
      \cmidrule(r){7-7}
      \cmidrule(r){8-8}
      \cmidrule(r){9-9}
      \cmidrule(r){10-10}

       &  & ф. ст. & ф. ст. & ф. ст. & & ф. ст. & ф. ст. & Кварт. & ф. ст. &   \\
      \midrule
      a & 1 & \phantom{2\sfrac{1}{2} \dplus{} }5\phantom{\sfrac{1}{2}} & 1 & 6 & \phantom{1\sfrac{1}{2} \dplus{} 3\sfrac{3}{4} \deq{} }1\sfrac{1}{2}                     & 4 & \phantom{0}6 & 0\phantom{\sfrac{1}{2}} & \phantom{0}0 & 0\phantom{+ 3 × 7} \\
      A & 1 & 2\sfrac{1}{2} \dplus{} 2\sfrac{1}{2}                     & 1 & 6 & \phantom{0}\sfrac{1}{2} \dplus{} 1\sfrac{1}{4} \deq{} 1\sfrac{3}{4}                     & 4 & \phantom{0}7 & \phantom{}\sfrac{1}{4}  & \phantom{0}1 & 1\phantom{+ 3 × 7} \\
      B & 1 & 2\sfrac{1}{2} \dplus{} 2\sfrac{1}{2}                     & 1 & 6 & 1\phantom{\sfrac{0}{0}} \dplus{} 2\sfrac{1}{2} \deq{} 3\sfrac{1}{2}                     & 4 & 14           & 2\phantom{\sfrac{1}{2}} & \phantom{0}8 & 1 \dplus{} 7\phantom{ × 7} \\
      C & 1 & 2\sfrac{1}{2} \dplus{} 2\sfrac{1}{2}                     & 1 & 6 & 1\sfrac{1}{2} \dplus{} 3\sfrac{3}{4} \deq{} 5\sfrac{1}{4}                               & 4 & 21           & 3\sfrac{3}{4}           & 15           & 1 \dplus{} 2 × 7\\
      D & 1 & 2\sfrac{1}{2} \dplus{} 2\sfrac{1}{2}                     & 1 & 6 & 2\phantom{\sfrac{0}{0}} \dplus{} 5\phantom{\sfrac{0}{0}} \deq{} 7\phantom{\sfrac{0}{0}} & 4 & 28           & 5\sfrac{1}{2}           & 22           & 1 \dplus{} 3 × 7\\

     \cmidrule(r){5-5}
     \cmidrule(l){6-6}
     \cmidrule(r){8-8}
     \cmidrule(r){9-9}
     \cmidrule(r){10-10}

      Разом & & & & 30 & \phantom{2 \dplus{} 1\sfrac{1}{2} \deq{}}19\phantom{\sfrac{1}{2}} & & 76 & 11\sfrac{1}{2} & 46 & \\
  \end{tabular}

  \end{center}
\end{table}

\begin{table}[H]
  \begin{center}
    \emph{Таблиця VIIIa}
    \footnotesize

  \begin{tabular}{c@{  } c@{  } c@{  } c@{  } c@{  } c@{  } c@{  } c@{  } c@{  } c@{  } c}
    \toprule
      \multirowcell{2}{\makecell{Рід\\ землі}} &
      \multirowcell{2}{Акри} &
      Капітал &
      Зиск &
      \makecell{Ціна\\ продук.} &
      \multirowcell{2}{\makecell{Продукт в\\ квартерах}} &
      \makecell{Продажна \\ ціна} &
      \makecell{Здо-\\буток} &
      \multicolumn{2}{c}{Рента} &
      \multirowcell{2}{Підвищення} \\

      \cmidrule(r){3-3}
      \cmidrule(r){4-4}
      \cmidrule(r){5-5}
      \cmidrule(r){7-7}
      \cmidrule(r){8-8}
      \cmidrule(r){9-9}
      \cmidrule(r){10-10}

       &  & ф. ст. & ф. ст. & ф. ст. & & ф. ст. & ф. ст. & Кварт. & ф. ст. &   \\
      \midrule
      a & 1 & \phantom{2\sfrac{1}{2} \dplus{} }5\phantom{\sfrac{1}{2}} & 1 & 6 & \phantom{1\sfrac{1}{2} \dplus{} 3 \deq{} }1\sfrac{1}{4}           & 4\sfrac{4}{5} & \phantom{0}6\phantom{\sfrac{1}{5}} & 0\phantom{\sfrac{1}{2}} & \phantom{0}0             & 0\phantom{\sfrac{1}{5} \dplus{} 3 × 7\sfrac{1}{5}} \\
      A & 1 & 2\sfrac{1}{2} \dplus{} 2\sfrac{1}{2}                     & 1 & 6 & \phantom{0}\sfrac{1}{2} \dplus{} 1 \deq{} 1\sfrac{1}{2}           & 4\sfrac{4}{5} & \phantom{0}7\sfrac{1}{5}           & \phantom{}\sfrac{1}{4}  & \phantom{0}1\sfrac{1}{5} & 1\sfrac{1}{5}\phantom{ \dplus{} 3 × 7\sfrac{1}{5}} \\
      B & 1 & 2\sfrac{1}{2} \dplus{} 2\sfrac{1}{2}                     & 1 & 6 & 1\phantom{\sfrac{0}{0}} \dplus{} 2 \deq{} 3\phantom{\sfrac{1}{2}} & 4\sfrac{4}{5} & 14\sfrac{2}{5}                     & 1\phantom{\sfrac{3}{4}} & \phantom{0}8\sfrac{2}{5} & 1\sfrac{1}{5} \dplus{} 7\sfrac{1}{5}\phantom{ × 7} \\
      C & 1 & 2\sfrac{1}{2} \dplus{} 2\sfrac{1}{2}                     & 1 & 6 & 1\sfrac{1}{2} \dplus{} 3 \deq{} 4\sfrac{1}{2}                     & 4\sfrac{4}{5} & 21\sfrac{3}{5}                     & 2\sfrac{1}{4}           & 15\sfrac{3}{5}           & 1\sfrac{1}{5} \dplus{} 2 × 7\sfrac{1}{5}\\
      D & 1 & 2\sfrac{1}{2} \dplus{} 2\sfrac{1}{2}                     & 1 & 6 & 2\phantom{\sfrac{0}{0}} \dplus{} 4 \deq{} 6\phantom{\sfrac{0}{0}} & 4\sfrac{4}{5} & 28\sfrac{4}{5}                     & 4\sfrac{3}{4}           & 22\sfrac{4}{5}           & 1\sfrac{1}{5} \dplus{} 3 × 7\sfrac{1}{5}\\

     \cmidrule(r){2-2}
     \cmidrule(r){5-5}
     \cmidrule(r){6-6}
     \cmidrule(r){8-8}
     \cmidrule(r){9-9}
     \cmidrule(r){10-10}

      Разом & 5 & & & 30 & \phantom{2 \dplus{} 1\sfrac{1}{2} \deq{}}16\sfrac{1}{4} & & 78\phantom{\sfrac{1}{5}} & 9\phantom{\sfrac{1}{2}} & 48 & \\
  \end{tabular}

  \end{center}
\end{table}

