\parcont{}  %% абзац починається на попередній сторінці
\index{iii2}{0021}  %% посилання на сторінку оригінального видання
в своїй History of Prices, коли подає історію кожної кризи, знову віддається цій
ілюзії. Справи завжди цілком здорові та перебіг їхній якнайкращий, поки раптом
настає катастрофа.

\pfbreak

Тепер ми вертаємося до нагромадження грошового капіталу.

Не кожне збільшення грошового капіталу, що призначений для позик, свідчить
про дійсне нагромадження капіталу або пошир процесу репродукції. Найвиразніше
це виявляється в тій фазі промислового циклу, що настає безпосередньо по пережитій
кризі, коли позичковий капітал масами лежить без діла. В такі моменти,
коли процес продукції обмежений (після кризи 1847 року продукція в англійських
промислових округах зменшилась па третину), коли ціни товарів доходять
своєї найнижчої точки, коли дух підприємливости паралізований, — тоді панує
низький рівень проценту, що свідчить тут лише про збільшення позичкового
капіталу саме з причини скорочення та бездіяльности промислового капіталу.
Те, що зі спадом товарових цін, зі зменшенням оборотів та зі скороченням
капіталу, витраченого на заробітну плату, треба менше засобів циркуляції;
з другого боку, те, що по ліквідації закордонних боргів, почасти за допомогою
відпливу золота, а почасти через банкрутства, не треба додаткових грошей
в їх функції як світових грошей; насамкінець те, що обсяг операцій дисконтування
векселів меншає разом зі зменшенням числа та валюти цих самих
векселів, — все це є річ очевидна. Тому попит на грошовий позичковий капітал,
чи то як на засоби циркуляції, чи то як на платіжні засоби (про нові приміщення
капіталу ще немає мови) меншає, і тому капіталу, призначеного для позик,
стає порівняно багато. Але серед таких обставин і подання позичкового грошового
капіталу, як пізніше виявиться, позитивно зростає.

Так, по кризі 1847 року панувало «обмеження оборотів та сила зайвих
грошей» (Comm. Distress, 1847--48, Evid. № 1664). Рівень проценту був дуже
низький з причини «майже цілковитого знищення торговлі та майже цілковитої
неможливости приміщувати продуктивно гроші». (І. c, р. 45. Свідчення Hodgson’a,
директора Royal Bank of Liverpool). Якої нісенітниці понавигадували ці панове
(a Hodgson ще один з кращих), щоб пояснити це собі, можна побачити з такої
фрази: «Скрута (1847~\abbr{р.}) виникла з дійсного зменшення грошового капіталу
в країні, викликаного почасти потребою оплачувати золотом девізні товари
з усіх частин світу, а почасти перетвором капіталу циркуляції (floating capital)
на основний». Як перетвір капіталу циркуляції на основний має зменшувати
грошовий капітал країни, цього годі зрозуміти, бо, напр., будуючи залізниці,
куди в ті часи головне приміщувалося капітал, не вживають золота або папірців
до будування віядуків чи колії, а гроші за залізничні акції, оскільки вони
вкладалися просто на оплату тих акцій, функціонували цілком так само, як
і всякі інші гроші, вкладені до банків, і навіть, як уже вище показано, на
деякий час збільшували позичковий грошовий капітал; оскільки ж гроші дійсно
витрачалось на будування, вони оберталися в країні як купівний та платіжний
засоби. На грошовому капіталі міг би той перетвір відбитися лише остільки,
оскільки основний капітал не є річ до вивозу, оскільки, отже, через неможливість
вивозу відпадає і той вільний капітал, що утворюється з зворотного припливу грошей,
одержаних за вивезені продукти, отже відпадають і зворотні припливи готівкою
або зливками золота. Але в ті часи навіть англійські експортові товари масами
лежали непродані на складах по закордонних ринках. Для купців та фабрикантів
в Менчестері і~\abbr{т. д.}, що частину нормального капіталу своїх підприємств твердо
примістили в залізничні акції, і тому, провадячи своє підприємство, стали
залежати від позичкового капіталу, — для них їхній floating capital дійсно став
зафіксований, і тому довелося їм зазнати наслідків цього. Але те саме сталося б,
\parbreak{}  %% абзац продовжується на наступній сторінці
