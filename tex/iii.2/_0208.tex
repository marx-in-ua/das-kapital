\parcont{}  %% абзац починається на попередній сторінці
\index{iii2}{0208}  %% посилання на сторінку оригінального видання
яка перетворюється на ренту, до авансованого капіталу, який продукує продукт
землі. Це відношення відрізняється від відношення додаткового продукту до
всього продукту, бо весь продукт має в собі не весь авансований капітал, саме
не має в собі основного капіталу, який продовжує існувати поряд з продуктом.
Навпаки, воно припускає, що на тих родах землі, які дають диференційну
ренту, дедалі ростуча частина продукту перетворюється в надмірний  надпродукт.
На найгіршій землі підвищення ціни хліборобського продукту вперше створює
ренту, а тому і ціну землі.

Але рента може зростати і без підвищення ціни хліборобського продукту.
Остання може лишитися сталою або навіть понизитися.

Коли вона лишається сталою, то рента може зрости або тільки тому
(залишаючи осторонь монопольні ціни), що при колишньому розмірі капіталу,
вкладеного у старі землі, починають оброблятись нові землі кращої якости, але
їх лише вистачає на те, щоб покрити вирослий попит, так що реґуляційна
ринкова ціна залишається без зміни. В цьому випадку ціна старих земель не
підвищується, але для землі, наново взятої під оброблення, ціна підвищується
понад рівень ціни старої землі.

Або ж рента підвищується тому, що при незмінній відносній продуктивності
і незмінній ринковій ціні зростає маса капіталу, що експлуатує землю.
Тому, хоч рента у відношенні до авансованого капіталу лишається та сама, її
маса, наприклад, подвоюється, бо сам капітал подвоївся. А щоб не сталося
пониження ціни, то друге приміщення капіталу дає, так само як і перше,
надзиск, який по закінчені терміну оренди теж перетворюється на ренту. Маса
ренти тут збільшується тому, що збільшується маса капіталу, який створює
ренту. Твердження, що різні послідовні приміщення капіталу на тій самій дільниці
землі можуть створити ренту лише тоді, коли продукт їхній неоднаковий
і тому постає диференційна рента, сходить на твердження, що, коли два капітали
по \num{1.000}\pound{ ф. стерл.}, вкладено в два лани однакової продуктивности, то
лише один з них може дати ренту, хоч обидва ці лани належать до кращої
кляси землі, яка дає диференційну ренту. (Отже, загальна маса ренти, вся
рента певної країни, збільшується з масою вкладеного капіталу, при чому
необов’язково, щоб тут зростала ціна одиниці земельної площі, або норма
ренти, або навіть маса ренти на одиницю площі; в цьому випадку маса
всієї ренти зростає з просторовим поширенням культури. Це може навіть
бути поєднане з падінням ренти на окремих володіннях). Інакше це твердження
звелося б до другого твердження, а саме, що приміщення капіталуодне
поряд одного у дві різні дільниці землі підлягає іншим законам, ніж послідовне
приміщення капіталу на тій самій дільниці землі, тимчасом як в дійсності
диференційну ренту висновують саме з тотожності закону в обох випадках,
з приросту продуктивности приміщення капіталу на тім самім лані, як і на
різних ланах. Єдина модифікація, що існує тут, і якої не помічають, є в тому,
що послідовні приміщення капіталів, коли їх вживають до просторово різних
земель, наражаються на таку межу, як земельна власність, тим часом як при послідовних
приміщеннях капіталу в ту саму землю цього не буває. Звідси і та
протилежна дія, в наслідок якої ці різні форми приміщення капіталу на практиці
взаємно обмежують одна одну. Тут ніколи не постає ріжниці з самого капіталу.
Коли склад капіталу лишається той самий, так само, як норма додаткової
вартости, то норма зиску лишається незмінна, так що при подвоєнні
капіталу маса зиску подвоюється. Так само за припущених відношень норма
ренти лишається та сама. Коли капітал в \num{1.000}\pound{ ф. стерл.} дає ренту в х, то
капітал в \num{2.000}\pound{ ф. стерл.} за припущених обставин дає ренту в 2х. Але, коли
обчислити ренту у відношенні до земельної площі, яка лишилася без зміни, бо,
згідно з припущенням, подвоєний капітал працює на тому самому лані, то
\parbreak{}  %% абзац продовжується на наступній сторінці
