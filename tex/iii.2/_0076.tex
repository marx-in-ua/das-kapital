\parcont{}  %% абзац починається на попередній сторінці
\index{iii2}{0076}  %% посилання на сторінку оригінального видання
кожен — а найперше лицарі кредиту — силкується дисконтувати майбутнє та
мати до свого розпорядку стільки засобів кредиту, скільки він може добути в
даний момент. Таким чином, щойно наведені причини сходять на те, що сама
лише кількість — чи то довезеного, чи то вивезеного благородного металу, впливає
не сама по собі, а вона впливає, поперше, специфічним характером благородного
металу як капіталу в грошовій формі, а, подруге, вона впливає як те перо,
що його — долучивши до тягара на терезах — вистачить, щоб остаточно схилити
шальку, що коливається, на один бік; вона впливає тому, що постає серед
обставин, коли справу вирішує будь-який надмір в той або в інший бік. Без цих
причин аж ніяк не сила було б зрозуміти, як то міг би відплив золота, прим.,
в 5--8 міл. ф. ст. — а це є та межа, що її досі знає досвід — вчинити якийсь
значний вплив; це невеличке збільшення або зменшення капіталу, що є незначним
навіть проти 70 міл. ф. ст. золотом, які пересічно перебувають у циркуляції
у Англії, становить, в дійсності, без краю малу величину в продукції такого
обсягу, як англійська\footnote{
Дивись, напр., сміху гідну відповідь Weguelin’a, коди він каже, що відплив 5 міл. золота
є зменшення капіталу на таку саму суму, й коли він тим хоче пояснити явища, що не постають при
безкраю
більших піднесеннях цін, або знеціненнях, поширах та скороченнях дійсного промислового капіталу.
З другого боку, не менш сміху гідна спроба пояснити ці явища, як безпосередні симптоми поширу або
скорочення маси реального капіталу (розглядаючи його за його матеріяльними елементами).
}. Але саме розвиток кредитової та банкової системи, що,
з одного боку, змушує ввесь грошовий капітал служити продукції (або — що
сходить на те саме — перетворювати усякий грошовий дохід на капітал), а з
другого боку, на певній фазі циклу зводить металевий запас до такого мінімуму,
коли той запас уже не може виконувати тих функцій, що йому випадають, —
саме цей розвиток кредитової та банкової системи породжує таку надмірну чутливість
усього цього організму. На менш розвинутих щаблях продукції зменшення
або збільшення скарбу проти його пересічної міри є річ порівняно байдужа.
Так само, з другого боку, навіть дуже значний відплив золота відносно
не матиме впливу, якщо він постає не підчас критичного періоду промислового
циклу.

Даючи це пояснення, ми не вважали на ті випадки, коли відплив металу
постає в наслідок неврожаїв і~\abbr{т. ін.} Тут значне та раптове порушення рівноваги
продукції, — що виявом його є відплив металу, — не потребує дальшого пояснення
його впливу. Цей вплив є то більший, що більше таке порушення постає
підчас такого періоду, коли продукція працює під високим тиском.

Далі ми не вважали на функції металевого скарбу, як ґарантії розмінности
банкнот та як осі всієї кредитової системи. А металевий запас, своєю чергою,
є вісь банку\footnote{
Newmarch (В.~А. 1857): «1364. Металевий резерв Англійського банку є справді\dots{} центральний
резерв або центральний металевий скарб, що на його основі провадяться всі справи країни.
Він є, сказати б, вісь, що навколо неї мають обертатись всі справи країни; всі інші банки країни
розглядають Англійський банк, як центральний скарб або резервуар, звідки вони мають добувати свої
запаси металевих грошей; і вплив закордонних вексельних курсів завжди позначається саме на цьому
скарбі та цьому резервуарі».
}. Перетвір кредитової системи на систему монетарну є річ неминуча,
як я це змалював вже в книзі І, розд, III, розглядаючи платіжний
засіб. Що потрібні найбільші жертви реальним багатством для того, щоб в критичний
момент зберегти металеву базу, — це визнали й Тук, і Лойд-Оверстон.
Суперечка точиться лише коло деякого плюса або мінуса, та коло більш або
менш раціонального поводження з цією неминучою річчю\footnote{
«Отже, на практиці обидва вони, Тук і Лойд, зустріли б надмірно великий попит на золото,
заздалегідь обмежуючи кредит, підвищуючи рівень проценту та зменшуючи визичання капіталу. Тільки
Лойд своєю ілюзією спричинився б до тяжких і навіть небезпечних [законодатних] обмежень та
приписів».
Economist, 1847. р. 1417.
}. Певну кількість
металу, незначну проти цілої продукції, визнано за вісь тієї системи. Звідси —
незалежно від жахливого унаочнення цього його характеру, як осі підчас
\parbreak{}  %% абзац продовжується на наступній сторінці
