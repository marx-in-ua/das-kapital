
\index{iii2}{0030}  %% посилання на сторінку оригінального видання
Проте маса позичкового капіталу в ділком відмінна від кількости циркуляції.
Кількість циркуляції ми тут розуміємо, як суму всіх банкнот і всіх металевих
грошей — залічуючи сюди й зливки благородних металів, — що є в певній
країні та перебувають в циркуляції. Частина цієї кількости становить резерв
банків, раз-у-раз змінюючись своєю величиною.

«12 листопаду 1857 року» [дата припинення чинности банкового акту
1844 року] «ввесь резерв Англійського банку, разом з усіма філіями, становив
тільки \num{580.751} ф. ст.; одночасно сума вкладів становила 22\sfrac{1}{2} мільйонів ф. ст.,
і з них щось 6\sfrac{1}{2} мільйонів належало лондонським банкірам». (В. А. 1858, р. LVII).

Зміни рівня проценту (не вважаючи на ті зміни, що відбуваються протягом
довших періодів, або на ріжниці рівня проценту по різних країнах; перші
зумовлено змінами в загальній нормі зиску, другі — ріжницями в нормах зиску
та в розвитку кредиту) залежать від подання позичкового капіталу (вважаючи
усі інші обставини, стан довір’я тощо за однакові), тобто капіталу, визичуваного
у формі грошей, металевих грошей та банкнот; у відміну від промислового
капіталу, що як такий, у товаровій формі, визичається за посередництвом
комерційного кредиту між самими аґентами репродукції.

А проте маса цього позичкового грошового капіталу відмінна й незалежна
від маси грошей, що перебувають в циркуляції.

Коли б, напр., 20 ф. ст. визичалось п’ять разів на день, то визичалось би
грошовий капітал в 100 ф. ст., і це разом з тим значило б, що ці 20. ф ст.,
опріч того, функціонували б принаймні 4 рази як купівний або платіжний
засіб; бо, коли б при цьому не було посередництва купівлі та продажу, так
що ті гроші не представляли б, принаймні, чотири рази перетвореної форми
капіталу (тобто товару, залічуючи до нього й робочу силу), то вони становили б
не капітал в 100 ф. ст., а тільки п’ять вимог, кожну на 20 ф. ст.

В країнах розвинутого кредиту ми можемо припустити, що ввесь вільний до
визичання грошовий капітал є в формі вкладів у банках та у грошових позикодавців.
Це має силу, принаймні для кредитових операцій взагалі та в цілому. А до цього
коли справи добрі, раніше ніж розбушувалася власно спекуляція, при легкому кредиті
та чим раз більшому довір’ї більша частина функцій циркуляції виконується
простою передачею кредиту без посередництва металевих або паперових грошей.

Сама можливість великих сум вкладів при відносно малій кількості засобів
циркуляції залежить виключно від:

1) числа купівель та платежів, що їх виконує та сама грошова монета;

2) числа її поворотів до банку як вкладу, так що її повторна функція як
купівного та платіжного засобу здійснюється за посередництвом поновного перетвору
її у вклад. Напр., дрібний торговець щотижня складає грішми 100 ф. ст.
у свого банкіра; тими грішми банкір виплачує частину вкладу фабриканта;
цей останній оплачує ними своїх робітників; робітники платять ними дрібному
торговцеві, що знову складає їх до банку. Отже, 100 ф. ст., складені дрібним
торговцем до банку, придались, поперше, на те, щоб виплатити фабрикантові
його вклад, подруге, щоб виплатити заробіток робітникам, потрете, щоб платити
ними самому дрібному торговцеві, почетверте, щоб скласти до банку дальшу
частину грошового капіталу того самого дрібного торговця; бо наприкінці
20 тижнів, коли б він сам не мав удаватися до тих грошей, він склав би таким
чином у банкіра тими самими 100 ф. ст. вклад в \num{2.000} ф. ст.

В якій мірі цей грошовий капітал буває незайнятий, це виявляється тільки
у відпливі та припливі запасного фонду банків. Звідси пан Weguelin, управитель
Англійського банку в 1857 році, робить той висновок, що золото становить
«єдиний» запасний фонд Англійського банку: «1258. На мою думку, норма
дисконту фактично визначається сумою незайнятого капіталу, що є в країні. Суму
незайнятого капіталу представляє резерв Англійського банку, що фактично є
\parbreak{}  %% абзац продовжується на наступній сторінці
