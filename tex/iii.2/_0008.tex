\parcont{}  %% абзац починається на попередній сторінці
\index{iii2}{0008}  %% посилання на сторінку оригінального видання
представляють його власний капітал, чи вклади, капітал інших людей. Цей
розподіл лишився б і тоді, коли б банкір провадив своє підприємство на власний
капітал, і тоді, коли б він провадив його тільки на капітал з вкладів.

Форма капіталу, що дає процент, приводить до того, що кожний певний
та реґулярний грошовий дохід здається процентом від капіталу, однаково, чи
постає той дохід від капіталу, чи ні. Спочатку грошовий дохід перетворюється
на процент, а потім з процентом знайдеться й капітал, що з нього постає той
процент. Так само за наявности капіталу, що дає процент, кожна сума вартости
здається капіталом, скоро її не витрачається як дохід; а саме основною (principal)
сумою протилежно до того можливого або дійсного проценту, що його вона
може давати.

Справа проста. Нехай пересічний рівень проценту буде 5\% річно. Отже,
сума в 500\pound{ ф. ст.}, коли її перетворити на капітал, що дає процент, давала б
щороку 25\pound{ ф. стерл}. Тимто кожен певний щорічний дохід в 25\pound{ ф. стерл.} розглядають,
як процент від капіталу в 500\pound{ ф. ст}. Однак, це є суто-ілюзорне
уявлення, окрім того випадку, коли джерело тих 25\pound{ ф. ст.} доходу — все одно,
чи воно є простий титул власности, або боргова вимога, чи воно становить
дійсний елемент продукції, як от шматок землі, — можна безпосередньо передавати,
або воно набуває таку форму, що в ній можна його передавати. Візьмімо
для прикладу державний борг та заробітну плату.

Держава має щороку виплачувати своїм кредиторам певний процент за
позичений капітал. Кредитор не може тут вимагати від свого винуватця, щоб він
повернув гроші перед реченцем, але може лише продати свою вимогу, своє
право володіння тією вимогою. Сам капітал спожито, витрачено державою. Його
вже немає. Що ж має кредитор держави, так це 1)~боргову посвідку держави,
напр., на 100\pound{ ф. ст.}; 2)~ця боргова посвідка дає йому право на щорічні
державні доходи, тобто на щорічну суму податків, на певну суму їх, припустімо,
на 5\pound{ ф. ст.} або 5\%; 3)~він може продати цю боргову посвідку на 100\pound{ ф. ст.}
іншим особам, кому схоче. Якщо рівень проценту є 5\%, і крім того є предумова,
що держава є надійний довжник, то державець $А$ звичайно може продати ту боргову
посвідку $В$ за 100\pound{ ф. ст.}; бо для $В$ однаково, чи визичає він комусь 100\pound{ ф. ст.}
за 5\% річно, чи, заплативши 100\pound{ ф. ст.}, він забезпечує собі щорічну данину
від держави на суму 5\pound{ ф. ст}. Але в усіх цих випадках капітал, що за його
парость (процент) вважається державний платіж, лишається ілюзорним, фіктивним
капіталом. Не тому тільки, що суми, визиченої державі, взагалі вже немає.
Її, ту суму, взагалі ніколи не призначалось витрачати, приміщувати як капітал,
а проте тільки приміщення її як капіталу могло б перетворити її на самозбережну
вартість. Для первісного кредитора $А$ та частина річних податків, що припадає
йому, становить процент від його капіталу, так само як для лихваря є процент
та частина майна марнотратника, що йому припадає, дарма що в обох випадках
позичену грошову суму витрачалося не як капітал. Змога продати боргову
посвідку держави для $А$ являє змогу повернути назад основну суму. Щодо $В$,
то з його приватного погляду, його капітал приміщено, як капітал, що дає
процент. По суті ж справи $В$ тільки заступив місце $А$, купивши в нього
державну боргову посвідку. Хоч як ці операції можуть раз-у-раз множитися,
однак капітал державного боргу лишається суто-фіктивним, а від того моменту,
коли б ці боргові посвідки не сила було продати, зникла б і зовнішня подоба
цього капіталу. Проте, як це ми зараз побачимо, цей фіктивний капітал
має свій власний рух.

Тепер розгляньмо робочу силу протилежно до капіталу державного боргу,
де неіснуюча величина (Minus) здається капіталом, — як і взагалі капітал, що дає
процент, є матір усіх безглуздих форм, так що, напр., борги в уявленні банкіра
можуть здаватися товарами. Заробітну плату вважається тут за процент, а тому
\parbreak{}  %% абзац продовжується на наступній сторінці
