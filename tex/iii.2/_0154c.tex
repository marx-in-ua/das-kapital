\parcont{}  %% абзац починається на попередній сторінці
\index{iii2}{0154}  %% посилання на сторінку оригінального видання
капіталу нічого не змінило в ріжницях між $В$, $C$ і $D$. Тому частина продукту,
що перетворюється на ренту, зменшується.

Коли б вищенаведений наслідок — задоволення попиту при виключенні
землі $А$ — був спричинений тим, що більше, ніж подвійна кількість капіталу
вкладалася б в землю $C$ або $D$ або в обидві разом, то справа набула б іншого
вигляду. Наприклад, коли б третє вкладення капіталу було зроблено на $C$:

\begin{table}[H]
  \centering
  \caption*{Таблиця ІVa}
  \footnotesize

  \settowidth\rotheadsize{\theadfont Продажна}
  \begin{tabular}{l c r c c c c c c c c}
    \toprule
      \thead[tl]{Рід\\землі} &
      &
      \rothead{Капітал} &
      \rothead{Зиск} &
      \rothead{Ціна\\продукції} &
      \rothead{Продукт} & % \\ в кварт.}}} \\ в кварт.}}}
      \rothead{Продажна\\ціна} &
      \rothead{Здобуток} &
      \multicolumn{2}{c}{Рента} &
      \rothead{Норма\\надзиску} \\

      \cmidrule(rl){2-11}

       & акри  & \makecell{\poundsign{}} & \poundsign{} & \poundsign{} & кв. & \poundsign{} & \poundsign{} & кв. & \poundsign{}  & \% \\
      \midrule

      B & 1 &  \phantom{0}5\phantom{\tbfrac{1}{2}} & 1\phantom{\tbfrac{1}{2}} & \phantom{0}6 & \phantom{0}4 & 1\tbfrac{1}{2} & \phantom{0}6\phantom{\tbfrac{1}{2}} & 0 & \phantom{0}0\phantom{\tbfrac{1}{2}}   & \phantom{00}0 \\
      C & 1 &  \phantom{0}7\tbfrac{1}{2}           & 1\tbfrac{1}{2}           & \phantom{0}9 & \phantom{0}9 & 1\tbfrac{1}{2} & 13\tbfrac{1}{2}                     & 3 & \phantom{0}4\tbfrac{1}{2}            & \phantom{0}60\\
      D & 1 &  \phantom{0}5\phantom{\tbfrac{1}{2}} & 1\phantom{\tbfrac{1}{2}} & \phantom{0}6 & \phantom{0}8 & 1\tbfrac{1}{2} & 12\phantom{\tbfrac{1}{2}}           & 4 & \phantom{0}6\phantom{\tbfrac{1}{2}}  & 120\\
     \midrule

     Разом & 3 & 17\tbfrac{1}{2} & 3\tbfrac{1}{2} & 21 & 21 & & 30\tbfrac{1}{2} & 7 & 10\tbfrac{1}{2} &\\
  \end{tabular}
\end{table}

\noindent{}Продукт з $C$ збільшився тут проти таблиці ІV з 6 кватерів до 9, надпродукт
— з 2 квартерів до 3, грошова рента зросла з 3\pound{ ф. стерл.} до 4\sfrac{1}{2}\pound{ ф.
стерл}. Проти таблиці II, де грошова рента була 12\pound{ ф. стерл.} і таблиці І,
де вона була 6\pound{ ф. стерл.}, вона навпаки зменшилась. Загальна сума ренти визначена
в збіжжі \deq{} 7 квартерів, зменшилась проти таблиці II (12 квартерів),
збільшилась проти таблиці І (6 квартерів); визначена в грошах (10\sfrac{1}{2}\pound{ ф. стерл.})
зменшилася проти обох (18\pound{ ф. стерл.} і 36\pound{ ф. стерл.}).

Коли б у землю $В$ було вкладено третій капітал в 2\sfrac{1}{2}\pound{ ф. стерл.}, то хоч
це й змінило б масу продукції, але не зачепило б ренти, бо згідно з припущенням
послідовні вкладення капіталу не вносять жодної ріжниці в землю того
самого роду, а земля $В$ ренти не дає. Навпаки, коли ми припустимо, що третій капітал вкладається в землю $D$,
замість $C$, то ми матимемо:

\begin{table}[H]
  \centering
  \caption*{Таблиця ІVb}
  \footnotesize

  \settowidth\rotheadsize{\theadfont Продажна}
  \begin{tabular}{l c r c c c c c c c c}
    \toprule
      \thead[tl]{Рід\\землі} &
      &
      \rothead{Капітал} &
      \rothead{Зиск} &
      \rothead{Ціна\\продукції} &
      \rothead{Продукт} & % \\ в кварт.}}} \\ в кварт.}}}
      \rothead{Продажна\\ціна} &
      \rothead{Здобуток} &
      \multicolumn{2}{c}{Рента} &
      \rothead{Норма\\надзиску} \\

      \cmidrule(rl){2-11}

       & акри  & \makecell{\poundsign{}} & \poundsign{} & \poundsign{} & кв. & \poundsign{} & \poundsign{} & кв. & \poundsign{}  & \% \\
      \midrule

      B & 1 &  \phantom{0}5\phantom{\tbfrac{1}{2}} & 1\phantom{\tbfrac{1}{2}} & \phantom{0}6 & \phantom{0}4 & 1\tbfrac{1}{2}  & \phantom{0}6 & 0 & \phantom{0}0 & \phantom{00}0 \\
      C & 1 &  \phantom{0}5\phantom{\tbfrac{1}{2}} & 1\phantom{\tbfrac{1}{2}} & \phantom{0}6 & \phantom{0}6 & 1\tbfrac{1}{2}  & \phantom{0}9 & 2 & \phantom{0}3 & \phantom{0}60\\
      D & 1 &  \phantom{0}7\tbfrac{1}{2}           & 1\tbfrac{1}{2}           & \phantom{0}9 & \phantom{0}12 & 1\tbfrac{1}{2} & 18           & 6 & \phantom{0}9 & 120\\
     \midrule

     Разом & 3 & 17\tbfrac{1}{2} & 3\tbfrac{1}{2} & 21 & 22 & & 33 & 8 & 12 &\\
  \end{tabular}
\end{table}

\noindent{}Тут загальна кількість продукту \deq{} 22 кварт., більша ніж удвоє проти
загальної кількости продукту таблиці І, хоч авансований капітал є лише 17\sfrac{1}{2}\pound{ ф. стерл.} проти 10\pound{ ф. стерл.}, отже, не подвоївся. Далі, загальна кількість продукту
на 2 квартерн більша, ніж загальна кількість продукту у таблиці II, хоч
в останній авансований капітал більший, а саме 20\pound{ ф. стерл.}.
