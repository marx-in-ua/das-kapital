\parcont{}  %% абзац починається на попередній сторінці
\index{iii2}{0224}  %% посилання на сторінку оригінального видання
рента, ціна землі, а тому і її відчужуваність і відчуження, і що тому не тільки
колишні зобов’язані до виплати ренти можуть перетворитись на незалежних
селян-власників, але й міські і інші посідачі грошей можуть купувати дільниці
землі для того, щоб здавати їх в оренду або селянам або капіталістам
і користуватись рентою як формою проценту на свій в такий спосіб приміщений
капітал; отже, що і ця обставина сприяє перетворенню колишнього способу
експлуатації, відносин між власником і дійсним обробником, а також самої ренти.

\subsubsection{Відчастинне (métairie)\footnote*{
Фр. основне значіння — маєток, хутір, фарма, в даному разі мовиться про відчастинне
господарство (рос. издольное). Прим. Ред.
} господарство і селянська парцелярна
власність}

Тут ми підійшли до кінця нашого ряду розвитку форм земельної ренти.

В усіх цих формах земельної ренти: відробітної ренти, ренти продуктами,
грошової ренти (як просто перетвореної форми ренти продуктами) дійсним
обробником і посідачем землі завжди припускається виплатник ренти, що його
неоплачена додаткова праця безпосередньо йде власникові землі. Це не тільки
можливо, але воно дійсно так і е, навіть при останній формі, при грошовій
ренті, — оскільки вона є в чистому вигляді, тобто як просто перетворена форма
ренти продуктами.

Як переходову форму від первісної форми ренти до капіталістичної ренти
можна розглядати métairie système, або систему відчастинного господарства, за якого
обробник (орендар) крім своєї праці (власної або чужої) дає частину капіталу
для господарювання, а земельний власник дає крім землі іншу частину потрібного
для господарювання капіталу (напр., худобу), і продукт ділиться в певних,
різних для різних країн пропорціях поміж орендарем та земельним власником.
З одного боку, в орендаря тут немає достатнього капіталу для цілковитого капіталістичного
господарювання. З другого боку, та частина, яку одержує тут
земельний власник, не є чиста форма ренти. В дійсності в ній може бути процент
на авансований земельним власником капітал і надмірна рента. Вона може
в дійсності також поглинути всю додаткову працю орендаря, або лишити йому
більшу або меншу частину цієї додаткової праці. Але істотне є в тому, що
рента тут уже більш не виступає, як нормальна форма додаткової вартости взагалі.
На одному боці орендар, чи вживає він тільки власної, чи також і чужої праці,
має домагання на певну частину продукту не тому, що він робітник, а тому,
що він посідач частини знарядь праці, капіталіст сам собі. На другому боці
земельний власник домагається своєї частини, ґрунтуючись не виключно на
своїй власності на землю, але як і позикодавець капіталу\footnote{
Порівн. Buret, Tocqueville, Sismondi.
}.

Рештки старовинної громадської власности на землю, що збереглись після
переходу до самостійного селянського господарства, наприклад, у Польщі та
Румунії, були там за привід для того, щоб здійснити перехід до нижчих форм
земельної ренти. Частина землі належить поодиноким селянам і вони обробляють
її самостійно. Друга частина обробляється спільно і створює додатковий
продукт, який придається почасти для покриття витрат громади, почасти як
резерв на випадок неврожаїв тощо. Ці дві останні частини додаткового продукту,
а кінець-кінцем і весь додатковий продукт, разом з землею, на якій він виростає,
помалу узурпується державними урядовцями і приватними особами,
і первісно вільні селяни-землевласники, що для них зберігається повинність
спільного обробітку цієї землі, перетворюються таким чином на панщанних,
або зобов’язаних до виплати ренти продуктами, тимчасом як узурпатори
\parbreak{}  %% абзац продовжується на наступній сторінці
