\parcont{}  %% абзац починається на попередній сторінці
\index{iii2}{0127}  %% посилання на сторінку оригінального видання
методів (кормові трави), почасти механічними засобами, які перетворюють
підґрунтя в верхній шар ґрунту, або змішують його з ним, або обробляють підґрунтя,
не переміщуючи його на поверхню.

Всі ці впливи на диференційну родючість різних земель сходять на те, що для
економічної родючости стан продуктивної сили праці, в даному разі здібність хліборобства
одразу використовувати природну родючість ґрунту, — здібність, яка різна
на різних ступенях розвитку, — становить так само момент так званої природної
родючости ґрунту, як і його хемічний склад і інші природні властивості.

Отже, ми припускаемо певний ступінь розвитку хліборобства. Ми припускаємо
далі, що ієрархія щодо родів ґрунту відповідає цьому ступеневі розвитку,
як це звичайно завжди буває щодо одночасних приміщень капіталу на різних
землях. В такому разі диференційна рента може бути представлена у висхід ій
або низхідній послідовності, бо, хоч певна послідовність дана для всієї сукупности
дійсно оброблюваних земель, проте завжди відбувається послідовний рух,
в якому складалась ця послідовність.

Припустімо землю чотирьох родів: $А$, $В$, $C$, $D$. Припустимо далі, що ціна
квартера пшениці = 3 ф. стерл. або 60 шил. А що рента є просто диференційна рента,
то ця ціна в 60 шил. за квартер з найгіршої землі дорівнює ціні продукції,
тобто дорівнює капіталові плюс пересічний зиск.

Хай $А$ буде ця найгірша земля, що на 50 шил. витрат дає 1 квартер = 60
шил.; отже, 10 шил. зиску, або 20\%.

Хай $В$ при цій самій витраті дає 2 кварт. = 120 шил. Це дало б 70 шил.
зиску, або 60 шил. надзиску.

Хай $C$ при такій самій витраті дає 3 кварт — 180 шил.; загальний
зиск = 130 шил.; надзиск = 120 шил.

Хай $D$ дає 4 кварт. = 240 шил. = 180 шил. надзиску.

Ми мали б тоді тоді таку послідовність:

Відповідні ренти були б для $D = 190 - 10$ шил. або ріжниця між $D$ та
$А$; для $C = 130 - 10$ шил. або ріжниця між $C$ та $А$; для $В = 70 - 10$ шил. або ріжниця
між $В$ та $А$; а загальна рента для $В$, $C$, $D$ = 6 кв. = 360 шил., дорівнювала б сумі ріжниць між
$D$ і $А$, $C$ і $А$, $В$ та $А$.

\begin{table}[h]
  \begin{center}

    \emph{Таблиця I}

  \begin{tabular}{ccccccсс}
    \toprule
      \multirowcell{2}{\makecell{Рід \\землі}} &
      \multicolumn{2}{c}{Продукт} &
      \multirowcell{2}{\makecell{Авансова-\\ний капітал}} &
      \multicolumn{2}{c}{Зиск} &
      \multicolumn{2}{c}{Рента}
      \\
    \cmidrule(rl){2-3}
    \cmidrule(l){5-6}
    \cmidrule(l){7-8}
    &
    \makecell{Квар-\\тери} &
    \makecell{Ши-\\лінги} &
    &
    \makecell{Квар-\\тери} &
    \makecell{Ши-\\лінги} &
    \makecell{Квар-\\тери} &
    \makecell{Ши-\\лінги} &
    \\
    \midrule
     A  &  1  &  \phantom{0}60 & 50 & \phantom{0}\sfrac{1}{6}   &  \phantom{0}10  &   \textemdash & \textemdash \\
     B  &  2  &  120           & 50 & 1\sfrac{1}{6}  &  \phantom{0}70  &   1           & \phantom{0}60 \\
     C  &  3  &  180           & 50 & 2\sfrac{1}{6}  &  130 &   2           & 120 \\
     D  &  4  &  240           & 50 & 3\sfrac{1}{6}  &  190 &   3           & 180 \\
     \cmidrule(rl){2-3}
     \cmidrule(l){7-8}
     Разом & 10 квар. & 600 ш. &    &       &      &   6 квар. &     360 ш. \\
  \end{tabular}
  \end{center}
\end{table}

Ця послідовність,
що становить за даних умов даний продукт, коли справу розглядати
абстрактно (а ми вже показали ті причини, що в наслідок їх така послідовність
може бути і в дійсності), може бути і в низхідному порядку (низхідному
від $D$ до $А$, від родючої землі до менш і менш родючої (так само як і в висхідному
порядку (висхідному від $А$ до $D$, від відносно неродючої до чимраз родючішої землі)
і, нарешті, перемінно, то в низхідному, то в висхідному порядку, наприклад,
від $D$ до $C$, від $C$ до $А$, від $А$ до $В$.

Процес, що відбувався при низхідній послідовності, був такий: ціна квартера
поступово підвищується, скажемо, з 15 шил. до 60. Скоро виявилося, що
випродукованих на $D$ 4 кв. (під ними можпа розуміти мільйони) уже не
\parbreak{}  %% абзац продовжується на наступній сторінці
