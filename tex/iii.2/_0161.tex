\parcont{}  %% абзац починається на попередній сторінці
\index{iii2}{0161}  %% посилання на сторінку оригінального видання
Воно може бути в тому, що на акр взагалі вживається більше капіталу (більше
добрива, більше механічної праці тощо) або також в тому, що взагалі лише
додатковий капітал дає змогу перевести відзначну, якісною стороною продуктивнішу,
витрату капіталу. В обох випадках при витраті 5 ф. стерл. на акр одержується
продукт в 2\sfrac{1}{2}  квартери, тоді як при витраті половини цього капіталу,
2\sfrac{1}{2}  ф. стерл., одержується продукт лише в один квартер. Продукт землі $A$,
залишаючи осторонь минущі ринкові відносини, можна було б і далі продавати
по вищій ціні продукції, замість продавати його по новій пересічній
ціні, лише доти, доки значна площа земель розряду $A$ і далі оброблялася б
з капіталом лише в 2\sfrac{1}{2}  ф. стерл. на акр. Але скоро нове відношення
в 5 ф. стерл. капіталу на акр, а разом з тим, поліпшене господарство набудуть
загального поширення, реґуляційна ціна продукції мусить понизитися до 2\sfrac{8}{11}  ф.
стерл. Ріжниця між обома частинами капіталу зникла б, і тоді дійсно акр землі $A$.
оброблюваний лише з капіталом в 2\sfrac{1}{2}  ф. стерл, оброблявся б ненормально,
невідповідно до нових умов продукції. Це вже було б ріжницею не між здобутком від
різних частин капіталу, вкладених у той самий акр, а між достатньою й недостатньою
загальною витратою капіталу на акр. Звідси видно, \emph{поперше}, що недостатність
капіталу в руках більшости орендарів (це мусить бути більшість, бо коли б це
була меншість, їй довелося б лише продавати нижче від своєї ціни продукції)
впливає цілком так само, як диференціювання самих земель в низхідному порядку.
Гірший спосіб обробітку на гіршій землі збільшує ренту з кращої землі;
він може навіть створити ренту з краще оброблюваної землі такої самої кепської
якости, яка взагалі ренти не дає. Звідси видно, \emph{подруге}, що диференційна рента,
оскільки вона виникає з послідовного капіталовкладення на тій самій земельній
площі, в дійсності перетворюється на пересічну величину, в якій уже не можна
розпізнати і відрізнити впливів різних капіталовкладень, і які тому не породжують
ренти на найгіршій землі, а 1) пересічну ціну всього продукту, скажімо
з одного акра $A$, перетворюють на нову регуляційну ціну і 2) виявляються, як
зміна загальної кількости капіталу на акр, що в нових умовах потрібна для
задовільного обробітку землі, і в якій так окремі послідовні капіталовкладення, як і
їхні відповідні впливи так поєднані, що їх не можна відрізнити. Так само стоїть
справа з поодинокими диференційними рентами кращих земель. Вони визначаються
в кожному випадку ріжницею пересічного продукту відповідного роду землі порівняно
з продуктом найгіршої землі за підвищеної витрати капіталу, що тепер
стала нормальною.

Жодна земля не дає будь-якого продукту без витрати капіталу. Отже,
навіть при звичайній диференційній ренті, при диференційній ренті І; коли говорять,
що 1 акр землі $А$, що реґулює ціну продукції, дає стільки й стільки продукту,
по такій-от ціні, і що кращі землі $B$, $C$, $D$ дають стільки й стільки диференційного
продукту, а тому за даної реґуляційної ціни стільки от грошової ренти, то
тут завжди припускається, що вжито певний капітал, який в даних умовах продукції
вважається за нормальний. Цілком так само, як у промисловості для
кожної галузі підприємств потрібен певний мінімум капіталу для того, щоб
можна було виготовляти товари по їхній ціні продукції.

Якщо цей мінімум змінюється в наслідок сполучених з поліпшеннями послідовних
капіталовкладень на тій самій землі, то це відбувається поступово.
Поки в певну кількість акрів, наприклад, землі $A$ не буде вкладено такого додаткового
капіталу, доти рента з краще оброблюваних акрів землі $A$ породжуватиметься
ціною продукції, яка лишилася незмінною, а рента з усіх кращих родів землі
$B$, $C$, $D$, підвищиться. Проте, скоро новий спосіб продукції так пошириться, що
зробиться нормальним, — ціна продукції понизиться; рента з найкращих дільниць
землі знову понизиться, і та частина землі $A$, в яку капітал вкладено в розмірі,
\parbreak{}  %% абзац продовжується на наступній сторінці
