\parcont{}  %% абзац починається на попередній сторінці
\index{iii2}{0033}  %% посилання на сторінку оригінального видання
споживається поволі, отже, щодо земельної ренти, заробітної плати в її вищих
формах, доходів непродуктивних кляс і~\abbr{т. ін.} Всі вони набирають на певний
час форму грошового доходу, і тому їх можна перетворити на вклади, отже
й на позичковий капітал. Щодо всякого доходу, чи призначено його до спожитку,
чи до нагромадження, скоро він існує у якійсь грошовій формі, має
силу те, що дохід являє частину вартости товарового капіталу, перетворену на
гроші, а тому й є він вияв та результат дійсного нагромадження, але не самий
продуктивний капітал. Коли прядун виміняв свою пряжу на бавовну, а ту частину,
що являє дохід, на гроші, то дійсною формою існування його промислового
капіталу є пряжа, що перейшла до рук ткача або, може, й до рук індивідуального
споживача, і то саме та пряжа є — чи буде вона для репродукції, чи
для спожитку — формою існування так капітальної вартости, як і додаткової
вартости, що міститься в тій пряжі. Величина перетвореної на гроші додаткової
вартости залежить від величини додаткової вартости, що міститься в пряжі.
Але скоро додаткова вартість перетворилась на гроші, ці гроші становлять лише
форму вартісного існування цієї додаткової вартости. І як така форма стають
вони моментом позичкового капіталу. Для цього треба тільки, щоб вони перетворились
на вклад, якщо сам власник уже не визичив їх. Навпаки, щоб
перетворитись знову на продуктивний капітал, мусять вони досягти вже певної
мінімальної межі.

\subsection{Грошовий капітал та дійсний капітал. III}

(Кінець)

Маса грошей, що так повинні знову перетворитись на капітал, є результат
масового процесу репродукції, але, розглядувана сама про себе, як позичковий
грошовий капітал, не є вона сама маса репродуктивного капіталу.

Найголовніше з досі розвинутого те, що пошир частини доходу, призначеної
до спожитку (при чому ми не вважаємо на робітника, бо його дохід = змінному
капіталові), виявляється передусім, як нагромадження грошового капіталу.
Отже, в нагромадження грошового капіталу ввіходить момент, істотно відмінний
від дійсного нагромадження промислового капіталу; бо частина річного
продукту, призначена до спожитку, ніяк не стає капіталом. Певна частина її
\emph{заміщує} капітал, тобто сталий капітал продуцентів засобів споживання, але
оскільки вона дійсно перетворюється на капітал, вона існує в натуральній формі
доходів продуцентів цього сталого капіталу. Ті самі гроші, що представляють
дохід, що правлять за простого посередника споживання, реґулярно перетворюються
на певний час у позичковий грошовий капітал. Оскільки ці гроші
представляють заробітну плату, є вони разом з тим грошова форма змінного
капіталу; а оскільки вони заміщують сталий капітал продуцентів засобів споживання,
вони є грошова форма, що її тимчасово набирає той сталий капітал,
та придається до купівлі натуральних елементів їхнього сталого капіталу, що його
треба замістити. А ні в тій, ні в цій формі вони сами про себе не виявляють
нагромадження, дарма що їхня маса зростає з обсягом процесу репродукції.
Але тимчасово вони виконують функцію позичкових грошей, отже, грошового
капіталу. Отже, з цього боку нагромадження грошового капіталу мусить
завжди відбивати більше нагромадження капіталу, ніж воно є в дійсності, бо
пошир індивідуального спожитку, здійснюючись за посередництвом грошей, видається
нагромадженням грошового капіталу, бо воно постачає грошову форму для
дійсного нагромадження, для грошей, що відкривають нові приміщення капіталу.
