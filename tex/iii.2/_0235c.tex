\parcont{}  %% абзац починається на попередній сторінці
\index{iii2}{0235}  %% посилання на сторінку оригінального видання
в книзі І, є вираз, що prima facie суперечить поняттю вартости, а також
і поняттю ціни, яка взагалі є лише певний вираз вартости; і «ціна праці»
є так само іраціональна, як жовтий логаритм. Але саме тут вульґарний
економіст лише найбільше і заспокоюється, бо він тут дійшов до глибокого погляду
буржуа, який вважає, що він платить гроші за працю, і що саме суперечність
формули поняттю вартости усуває для нього обов’язок зрозуміти останню.

\pfbreak

Ми\footnote{
Початок розділу XLVІІІ за рукописом.
} бачили, що капіталістичний процес продукції є історично певна
форма суспільного процесу продукції взагалі. Цей останній є і процес продукції
матеріяльних умов людського життя, і процес, що відбувається в специфічних
історико-економічних відносинах продукції, що продукує і репродукує сами
ці відносини продукції, а разом з тим і носіїв цього процесу, матеріяльні умови їхнього
існування і їхні взаємні відносини, тобто певні суспільно-економічні
форми останніх. Бо сукупність цих відносин, що в них носії цієї продукції перебувають
до природи і один до одного, відносин, що в них вони продукують, ця сукупність
саме і є суспільство, розглядуване з погляду його економічної структури.
Подібно до всіх його попередників капіталістичний процес продукції відбувається
в певних матеріяльних умовах, що є одночасно за носіїв певних суспільних
відносин, в які вступають індивідууми в процесі репродукції свого життя. Як
ті умови, так і ці відносини, є, з одного боку, передумови, з другого — наслідки
і витвори капіталістичного процесу продукції; вони ним продукуються й репродукуються.
Далі ми бачили: капітал, — а капіталіст є лише персоніфікований
капітал, функціонує в процесі продукції лише як носій капіталу — отже, капітал
висмоктує у відповідному йому суспільному процесі продукції певну кількість
додаткової праці з безпосередніх продуцентів або робітників, додаткову працю,
що він її одержує без еквіваленту, і яка за своєю суттю завжди лишається
примусовою працею, хоча б вона і здавалася наслідком вільної договірної угоди.
Ця додаткова праця втілюється у додатковій вартості, і ця додаткова вартість
існує у додатковому продукті. Додаткова праця взагалі, як праця понад міру
даної кількости потреб, мусить завжди існувати. Але в капіталістичній, як і в рабській
системі тощо вона має лише антагоністичну форму і доповнюється цілковитим
неробством певної частини суспільства. Певна кількість додаткової праці потрібна
як страхування проти випадковостей, в наслідок доконечного, відповідного розвиткові
потреб і поступові людности, проґресивного поширення процесу репродукції,
що з капіталістичного погляду називається нагромадженням. Одна з цивілізаторських
сторін капіталу є в тому, що він вимушує цю додаткову працю
таким способом і в таких умовах, які для розвитку продуктивних сил, суспільних
відносин і створення елементів вищої нової формації є вигідніші, ніж за колишніх
форм рабства, крепацтва тощо. Він приводить, таким чином, з одного боку, до
ступеня, на якому відпадає примус і монополізація суспільного розвитку (включаючи
сюди його матеріяльні й інтелектуальні вигоди) однією частиною суспільства
за рахунок іншої; з другого боку, він створює матеріяльні засоби
і зародок для відносин, які в вищій формі суспільства дадуть можливість сполучити
цю додаткову працю з значнішим обмеженням часу, присвяченого матеріяльній
праці взагалі. Бо додаткова праця, залежно від розвитку продуктивної
сили праці, може бути велика при малій загальній довжині робочого дня і відносно
мала при великій загальній довжині робочого дня. Коли потрібний робочий
час \deq{} 3 і додаткова праця \deq{} 3, то весь робочий день \deq{} 6, і норма додаткової
праці \deq{} 100\%. Коли потрібна праця \deq{} 9 і додаткова праця \deq{} 3, то ввесь
робочий день \deq{} 12, і норма додаткової праці \deq{} лише ЗЗ\sfrac{1}{3}\%. Але далі від продуктивности
праці залежить, яка кількість споживної вартости продукується
\parbreak{}  %% абзац продовжується на наступній сторінці
