\parcont{}  %% абзац починається на попередній сторінці
\index{iii2}{0019}  %% посилання на сторінку оригінального видання
не оплачував грішми бавовни, фабрикант ситцю — пряжі, купець — ситцю і~\abbr{т. ін.}
В перших актах процесу товар, бавовна, проходить свої різні фази продукції, і цей
перехід відбувається за посередництвом кредиту. Але, скоро бавовна одержала
вже в продукції свою останню форму як товар, цей самий товаровий капітал
проходить ще лише через руки різних купців, що упосереднюють транспорт до
далекого ринку; останній з них, кінець-кінцем, продає його споживачеві,
купуючи натомість інший товар, що ввіходить або в споживання або в процес
репродукції. Отже, тут слід відрізняти два періоди: протягом першого періоду
кредит упосереднює дійсні послідовні фази продукції того самого товару; протягом
другого він упосереднює лише перехід з рук одного купця до рук іншого, перехід,
що включає й транспорт, отже, акт $Т — Г$. Але й тут товар перебуває принаймні
завжди в акті циркуляції, отже фазі процесу репродукції.

Отже, те, що тут визичається, ніяк не є капітал незайнятий, а навпаки це
капітал, що в руках свого державця мусить змінити свою форму, капітал, що
існує в такій формі, в якій він для нього є простий товаровий капітал, тобто
капітал, що мусить відбути зворотне перетворення, а саме, принаймні насамперед,
перетворитись на гроші. Отже, це та метаморфоза товару, що її тут упосереднює
кредит; не тільки $Т — Г$, але й $Г — Т$ та дійсний процес продукції.
Багато кредиту в межах репродуктивного кругообороту — якщо не вважати на банкірський
кредит — не означає: багато незайнятого капіталу, що його пропонується
для позик та що шукає прибуткового приміщення, але означає: велику
зайнятість капіталу в процесі репродукції. Отже, кредит упосереднює тут,
1) оскільки вважати на промислових капіталістів, — перехід промислового капіталу
з однієї фази до другої, зв’язок сфер продукції, що одна до однієї приналежні
та одна до однієї втручаються; 2) оскільки вважати на купців, —
транспорт та перехід товарів з рук до рук аж до остаточного продажу їх за
гроші або обмін їх за якийсь інший товар.

Максимум кредиту дорівнює тут найповнішій зайнятості промислового капіталу,
тобто найвищому напруженню його репродукційної сили, не зважаючи на межі
споживання. Ці межі споживання поширюються напруженням самого процесу
репродукції; з одного боку, те напруження збільшує споживання доходу робітниками
та капіталістами, з другого боку, воно є тотожнє з напруженням продуктивного
споживання.

Поки процес репродукції тече реґулярно, а тому й зворотний приплив капіталу
лишається забезпеченим, цей кредит тримається твердо та поширюється, і пошир
його базується на поширі самого процесу репродукції. Скоро настає застій
в наслідок затриманого зворотного припливу капіталу, переповнення ринків, спаду
цін, то виявляється надмір промислового капіталу, але в такій формі, в якій він
не може виконувати своїх функцій. Маса товарового капіталу, але не сила
його продати. Маса основного капіталу, але з причин застою репродукції той
капітал здебільша незайнятий. Кредит скорочується, 1) бо цей капітал незайнятий,
тобто припинив свій рух на одній з фаз своєї репродукції, бо він не може
виконати своєї метаморфози, 2) бо зламано віру у поточність процесу репродукції;
3) бо попит на цей комерційний кредит меншає. Прядунові, що обмежує
свою продукцію, маючи на складі масу непроданої пряжі, не треба закупати
бавовну на кредит; купцеві не треба купувати товари на кредит, бо він їх
уже має більше, ніж досить.

Отже, коли настає порушення цього поширу або бодай лише нормального
напруження процесу репродукції, то разом з цим постає й недостача кредиту;
стає тяжче одержувати товари на кредит. Але особливо характеристичним є вимагання
платежу готівкою та обережність у продажі на кредит для тієї фази
промислового циклу, що постає по кризі. А підчас самої кризи, коли кожен
має що продати та, не маючи змоги продати, проте мусить продавати, щоб
\parbreak{}  %% абзац продовжується на наступній сторінці
