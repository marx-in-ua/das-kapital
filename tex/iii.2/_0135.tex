\parcont{}  %% абзац починається на попередній сторінці
\index{iii2}{0135}  %% посилання на сторінку оригінального видання
і в зворотному відношенні до маси продукту, що його дають за однакової величини
вкладеного капіталу рівновеликі загальні площі землі. Отже, відношення між
кількістю найгіршої оброблюваної землі і кількостю найліпшої, в межах усієї
земельної площі країни справляє на загальну суму ренти вплив зворотний
тому, що його справляє відношення між якістю найгіршої з оброблюваних земель
і якістю ліпшої і найліпшої на ренту з акра і, тому, за інших рівних
умов і на суму ренти. Сплутування цих двох моментів дало привід до всеможливих
безглуздих заперечень проти диференційної ренти.

Отже, загальна сума ренти зростає в наслідок простого поширення культури
і сполученого з ним збільшеного застосування капіталу і праці до землі.

Але найважливіший пункт є такий: хоч згідно з припущенням відношення
рент з різних родів землі, обчислених на акр, не змінюється, а тому
не змінюється і норма ренти щодо капіталу, вкладеного в кожен акр, проте
виявляється таке: коли ми порівняємо І$а$ з І — тим випадком, коли число оброблюваних
акрів і вкладений в них капітал збільшились пропорційно, — то ми знайдемо,
що так само як загальна продукція зросла пропорційно збільшенній площі
обробленої землі, тобто обидві подвоїлись, так само зросла і загальна сума ренти.
Вона збільшилась з 18 до 36\pound{ ф. стерл.}, цілком так само, як число акрів, що
збільшилося з 4 до 8.

Коли ми візьмемо загальну площу в 4 акри, то загальна сума ренти з них
становитиме 18\pound{ ф. стерл.}, отже, пересічна рента, враховуючи землю, яка не дає
ренти, становитиме 4\sfrac{1}{2}\pound{ ф. стерл}. Таке обчислення міг зробити б, наприклад,
якийсь земельний власник, котрому належали б усі 4 акри; і таким самим
чином обчислюється статистично пересічна рента усієї країни. Загальна сума
ренти в 18\pound{ ф. стерл.} постає при застосуванні капіталу в 10\pound{ ф. стерл}.
Відношення між обома цими числами ми називаємо нормою ренти: отже, тут
180\%.

Та сама норма ренти постає з І$а$, де замість 4 акрів обробляється 8,
але де землі всіх родів збільшились в однаковому відношенні. Загальна сума
ренти в 36\pound{ ф. стерл.} дає при 8 акрах і 20\pound{ ф. стерл.} застосованого капіталу
пересічну ренту в 4\sfrac{1}{2}\pound{ ф. стерл.} з акра і норму ренти в 180\%.

Якщо ми, навпаки, розглянемо І$b$, де приріст відбувся переважно на обох
гірших родах землі, то ми матимемо ренту в 42\pound{ ф. стерл.} з 12 акрів, тобто
пересічну ренту в 3\sfrac{1}{2}\pound{ ф. стерл.} з акра. Ввесь витрачений капітал є 30\pound{ ф. стерл.},
отже, норма ренти \deq{} 140\%. Отже, пересічна рента з акра зменшилась на 1\pound{ ф.
стерл.}, а норма ренти упала з 180 до 140\%. Отже, тут поруч з зростанням
загальної суми ренти з 18\pound{ ф. стерл.} до 42\pound{ ф. стерл.} відбувається зниження
пересічної ренти, обчислюваної так на акр, як і на капітал; зниження рівнобіжне,
але не пропорційне зростанню продукції. Це відбувається, не зважаючи на те,
що рента з усіх родів землі, обчислена так на акр, як і на витрачений капітал,
лишається та сама. Це відбувається тому, що \sfrac{3}{4} приросту припадає на
землю $А$, яка не дає ренти, і на землю $В$, яка дає лише мінімальну ренту.

Коли б у випадку І$b$ все поширення сталось лише на землі $А$, то ми
дали б 9 акрів на $А$, 1 на $В$, 1 на $C$ і 1 на $D$. Загальна сума ренти, як і давніш,
була б 18\pound{ ф. стерл.}, отже, пересічна рента з акра на цих 12 акрах
була б 1\sfrac{1}{2}\pound{ ф. стерл.}; 18\pound{ ф. стерл.} ренти на 30\pound{ ф. стерл.} витраченого капіталу
становили б норму ренти в 60\%. Середня рента, обчислена так на акр,
як і на застосований капітал, дуже зменшилася б, тоді як загальна сума ренти
не зросла б.

Порівняймо, нарешті, І$с$ з І і І$b$. В порівнянні з І земельна площа збільшилась
утроє і так само збільшився витрачений капітал. Загальна сума ренти
становить 72\pound{ ф. стерл.} з 12 акрів, отже, 6\pound{ ф. стерл.} з акра, проти 4\sfrac{1}{2}\pound{ ф.
стерл.} в випадку І. Норма ренти на витрачений капітал ($72\pound{ ф. стерл.}: 30\pound{ ф. стерл.}$)
\parbreak{}  %% абзац продовжується на наступній сторінці
