\parcont{}  %% абзац починається на попередній сторінці
\index{iii2}{0072}  %% посилання на сторінку оригінального видання
застосування цього акту довело на практиці саме протилежне. За єдиним винятком,
що про нього ми зараз скажемо, маса банкнотів англійського банку,
що були в циркуляції, після 1844 року ніколи не досягала того максимуму, що
його має право випускати банк. І з другого боку криза 1857 року довела, що
в певних умовах цього максимуму недосить. З 13 до 30 листопада 1857 року
в циркуляції було банкнот пересічно щодня на \num{488.830}\pound{ ф. ст.} понад цей
максимум. (В.~А. 1858~\abbr{р.} XI). Законний максимум був тоді \num{14.475.000}\pound{ ф. ст.}
плюс сума металевого скарбу в льохах банку.

Про відплив і приплив благородного металу треба зауважити ось що:

\emph{Поперше}. Треба відрізняти, з одного боку, переміщення металу туди
й сюди в межах країни, яка не продукує зовсім золота й срібла, а з другого
боку течію золота і срібла від джерел їхньої продукції через різні інші країни
та розподіл цього додаткового матеріялу між цими країнами.

З початку цього століття, — поки не почали впливати російські, каліфорнійські
та австралійські золоті копальні, подання цих металів було досить лише
для заміщення зношеної монети, для звичайного споживання металів як речей
розкошів і для вивозу срібла в Азію.

Однак з того часу, разом з розвитком азійської торгівлі Америки та Европи,
поперше надзвичайно збільшився вивіз срібла в Азію. Срібло, що його вивозилось
з Европи, здебільша заміщувалось додатковим золотом. Далі, частину золота,
що знову довозилось, вбирала внутрішня грошова циркуляція. Вважають, що до
1875 року щось 30 міл. золота додатково ввійшло у внутрішню циркуляцію Англії\footnote{Як впливало це на грошовий ринок, показують такі свідчення W.~Newmarch’a: «1509.
Наприкінці 1853 року серед публіки постали значні побоювання; в вересні Англійський банк тричі
раз за разом підвищив свій дисконтовий процент\dots{} в перші дні жовтня\dots{} виявився значний неспокій
та паніка серед публіки. Ці побоювання та цей неспокій, здебільша, усунулось перед кінцем,
листопада, і майже цілком їх усунулось в наслідок прибуття 5 міл. благородного металу з Австралії.
Те саме повторилося восени 1854 рову, коли надійшло, в жовтні та листопаді, майже 6 міл.
благородного металу. Те саме повторилося восени 1855 року — як відомо, це був час зворушення та
непокою — коли надійшло 8 міл. благородного металу протягом місяців вересня, жовтня та листопада.
Наприкінці 1856~\abbr{р.} ми бачимо те саме. Коротко, я міг би, цілком певно, апелювати до досвіду майже
кожного члена комісії: хіба ми не звикли вже при всякій фінансовій скруті дивитися на прибуття
корабля з золотом як на природний повний порятунок».}.
Потім, після 1845 року пересічна кількість металевих запасів збільшилась по всіх
центральних банках Европи та Північної Америки. Одночасно зріст внутрішньої
грошової циркуляції привів до того, що по паніці, протягом періоду млявих
справ, що настав по тій паніці, банковий запас зростав уже швидше в наслідок
збільшення маси золотої монети, витиснутої внутрішньою циркуляцією та іммобілізованої.
Нарешті, по відкритті нових покладів золота піднеслося споживання
благородного металу для речей розкошів, бо зросло багатство.

\emph{Подруге}. Благородний метал перебуває у невпинному потоці, припливаючи
від однієї до однієї між тими країнами, що не продукують золота та
срібла; та сама країна раз-у-раз імпортує той метал та раз-у-раз експортує
його. І тільки перевага руху в один або другий бік вирішує, що, кінець-кінцем,
відбувається відплив або приплив, бо ті рухи, що відбуваються, як коливання,
а часто й паралельно, здебільша, невтралізуються. Але з цієї причини,
коли вважають лише на цей результат, недобачають також сталости та паралельного
в цілому перебігу обох рухів. Справу завжди розуміють лише так, ніби
додатковий довіз та додатковий вивіз благородного металу є тільки наслідок та
вияв відношення між довозом та вивозом товарів, тимчасом коли це є одночасно
і вияв відношення між незалежними від товарової торговлі довозом та вивозом
самого благородного металу.

\emph{Потретє}. Перевага довозу над вивозом та навпаки виміряється в цілому
збільшенням або зменшенням металевого запасу в центральних банках. Оскільки
\parbreak{}  %% абзац продовжується на наступній сторінці
