\parcont{}  %% абзац починається на попередній сторінці
\index{iii2}{0064}  %% посилання на сторінку оригінального видання
утворити фонд для закупу шовку; і стан ринків в Індії такий (не вважаючи на
нагромадження там срібла), що для купця вигідніше посилати туди срібло, ніж
тканини або інші британські фабрикати. — 4338. Чи не було великого відпливу
з Франції, що в наслідок його ми одержували срібло? — Так, дуже значний відплив.
— 4344. Замість вивозити шовк з Франції та Італії, ми відправляємо його
туди великими партіями, так бенгальський, як і китайський».

Отож до Азії відправлялося срібло — грошовий металь цієї частини світу —
замість товарів не тому, що ціни на ці товари піднеслися в країні, що їх продукує
(Англія), а тому що впали — впали через надмірний імпорт — в тій країні,
куди їх імпортують; дарма що це срібло Англія одержувала з Франції та почасти
мусила його оплачувати золотом. За currency-теорією при такому імпорті
ціни в Англії мусіли б впасти, а в Індії та Китаю — піднестися.

Другий приклад. Перед комісією лордів (C.~D. 1848--1857) Wylie, один
з перших ліверпульських купців, свідчить так: «1994. Наприкінці 1845 року
не було вигіднішої справи, що давала б такі великі зиски [як бавовнопрядіння].
Запас бавовни був великий, і добру придатну бавовну можна було мати по
4\pens{ пенси} за фунт, а з такої бавовни можна було випрядати гарний secunda
mule twist № 40, що на нього теж мали витратити коло 4\pens{ пенсів}, отже разом
витрат щось коло 8\pens{ пенсів} у прядуна. Цю пряжу великими масами продавалося
в вересні та жовтні 1845 року — і складалися так само великі контракти на
постачання її — по 10\sfrac{1}{2} та 11\sfrac{1}{2}\pens{ пенсів} за фунт, і в деяких випадках прядуни
реалізували зиск, рівний купівельній ціні бавовни. — 1996. Справа була вигідна
до початку 1846 року. — 2000. З березня 1844 року запас бавовни [\num{627.042} паки]
становив подвійну кількість того, що він становить сьогодні [7 березня 1848 року,
коли його було \num{301.070} паків], а проте ціна була на 1\sfrac{1}{4}\pens{ пенси} за фунт вища».
[6\sfrac{1}{4}\pens{ пенсів} проти 5\pens{ пенсів}]. Одночасно пряжа — добрий secunda mule twist
№ 40 — впала від 11\sfrac{1}{2}--12\pens{ пенсів} до 9\sfrac{1}{2}\pens{ пенсів} в жовтні та до 7\sfrac{3}{4}\pens{ пенсів}
наприкінці грудня 1847 року; пряжу продавалося за купівельну ціну бавовни,
що з неї було її випрядено (ib, № 2021, 2023). Це виявляє ту заінтересовану
премудрість Оверстона, що гроші мають бути «дорогі», бо капітал «рідкий». 3-го березня
1844 року банковий рівень проценту був 3\%; в жовтні та листопаді
дійшов він до 8 та 9\% й 7 березня 1848 року становив ще 4\%. Ціни на бавовну
— в наслідок цілковитого спину в збуті та в наслідок паніки з відповідним
їй високим рівнем проценту — впали далеко нижче від тієї ціни на неї,
що відповідала станові подання. Наслідок цього було, з одного боку, величезне
зменшення довозу в 1848 році, а з другого боку, зменшення продукції в Америці;
відси новий зріст бавовняних цін в 1849 році.

За Оверстоном, товари були занадто дорогі тому, що занадто багато грошей
було в країні.

«2002. Недавнє погіршення стану бавовняної промисловости завдячує не
бракові сировини, бо ціна впала, дарма що запас бавовни-сировини значно
зменшився». Але в Оверстона маємо приємне переплутування ціни, відповідно
вартости товару, з вартістю грошей, власне з рівнем проценту. Відповідаючи на
питання 2026, Wylie подає свій загальний погляд на currency-теорію, що на
ній Cardwell та сер Charles Wood в травні 1847 року «заснували потребу перевести
банковий акт 1844 року в усій повноті його змісту»: «Ці принципи, на
мою думку, такі, що вони надаватимуть грошам штучну високу вартість, а всім
товарам штучну руйнаційно низьку вартість». — Далі він каже про вплив цього
банкового акту на загальний стан справ: «Що лише з великими втратами можна
було дисконтувати чотиримісячні векселі, — які є звичайні трати фабричних міст
на купців та банкірів за куплені товари, призначені для Сполучених Штатів, —
то і виконання замовлень дуже гальмувалося аж до урядового листа з 25 жовтня»
[припинення чинности банкового акту], «коли знову з’явилася змога дисконтувати
\index{iii2}{0065}  %% посилання на сторінку оригінального видання
ці чотиримісячні векселі». (2097). Отже, і в провінції припинення чинности
банкового акту вплинуло, як визволення. — «2102. Минулого жовтня»
[1847 року] «майже всі американські закупники, що купують тут товари, негайно
обмежили, скільки змоги було, свої замовлення: а як звістка про подорожчання
грошей дійшла до Америки, всі нові замовлення припинилися. — 2134.
Збіжжя та цукор являли спеціяльні випадки. На збіжжевий ринок вплинули
сподіванки на урожай, а на цукор вплинули величезні запаси та довізи — 2163.
Багато з наших платіжних зобов’язань до Америки\dots{} зліквідувалось примусовими
продажами товарів, що їх відправлено на комісію, а багато, побоююсь,
анульовано тутешніми банкрутствами. — 2196. Якщо я добре пригадую, на нашій
фондовій біржі в \emph{жовтні 1847 платилося до 70\%}».

[Криза 1837 року з її довготривалими лихими наслідками, що до них
року 1842 долучилося ще справжнє покриззя, та заінтересоване осліплення
промисловців та купців, що ніяк не хотіли бачити надпродукції — аджеж за
вульґарною політичною економією вона є безглуздя та неможливість! — кінець-кінцем
породили ту плутанину в головах, що дозволила школі-currency перетворити
свою догму на практику в національному маштабі. Банкове законодавство
1844--45 років було переведено.

Банковий акт 1844 року поділяє англійський банк на відділ видання банкнот
і на відділ банковий. Перший одержує забезпечень — здебільша з паперів державного
боргу — на 14 мільйонів та ввесь металевий скарб, що має складатися
щонайбільше на \sfrac{1}{4} з срібла, і видає банкноти на суму, рівну загальній сумі
тих забезпечень і того скарбу. Оскільки ці банкноти не є в руках публіки, вони
лежать в банковому відділі, являючи разом з невеликою кількістю монети (щось
з мільйон), потрібної до щоденного вжитку, завжди готовий запас банку. Емісійний
відділ видає публіці золото за банкноти та банкноти за золото; решту
зносин з публікою обслуговує банковий відділ. Приватні банки, що в 1844 році
мали право видавати власні банкноти в Англії та Велсі, зберігають це право,
проте, видання ними банкнот обмежено певним контингентом; коли якийсь з цих
банків перестає видавати власні банкноти, то Англійський банк може збільшити
суму своїх непокритих банкнот на \sfrac{2}{3} невикористуваного тим банком континґенту;
цим способом та сума протягом часу до 1892 року підвищилась від 14 до 16\sfrac{1}{2}
мільйонів ф. ст. (точно — \num{16.450.000}\pound{ ф. ст.}).

Отже, замість кожних п’ятьох фунтів золотом, що відпливають з банкового
скарбу, до емісійного відділу вертається банкнота-п’ятифунтівка й її
нищиться там; замість кожних п’ятьох соверенів, що надходять до скарбу, в
циркуляцію йде п’ятифунтівка-банкнота. Таким способом здійснюється па практиці
Оверстонова ідеальна паперова циркуляція, що регулюється точно за законами
металевої циркуляції, й тим, як стверджують теоретики currency, кризи на віки
вічні унеможливлено.

Але в дійсності розділ банку на два незалежні відділи відібрав у дирекції
змогу вільно порядкувати у рішучі моменти всіма вільними засобами, так що
могли трапитися випадки, коли банковому відділові загрожувало банкрутство, в
той час коли емісійний відділ мав незайманих кілька мільйонів золотом, та, крім
того, ще отих своїх 14 мільйонів забезпечень. І це могло то легше статись, що
майже в кожній кризі буває такий відтинок часу, коли постає великий відплив
золота за кордон, а покривати його доводиться, переважно, металевим скарбом
банку. А за кожні п’ять фунтів, що відпливають тоді за кордон, забирають з
внутрішньої циркуляції країни п’ятифунтівку-банкноту, отже, кількість засобів
циркуляції меншає саме тоді, коли їх найбільше уживається та є в них найбільша
потреба. Отже, банковий акт 1844 року провокує ввесь торговельний
світ безпосередньо до того, щоб на початку кризи своєчасно відкладати собі певний
запасний фонд банкнот, отже, до того, щоб ту кризу прискорювати та
\parbreak{}  %% абзац продовжується на наступній сторінці
