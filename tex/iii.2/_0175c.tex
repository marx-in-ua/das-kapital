\parcont{}  %% абзац починається на попередній сторінці
\index{iii2}{0175}  %% посилання на сторінку оригінального видання
землі $А$, перестала брати участь у конкуренції і земля $В$, і земля $C$ зробилася б
регуляційною землею, що не дає ренти.

Отже, що більше капіталу вживається на землі, що вищого розвитку досягли
у країні хліборобство і цивілізація взагалі, то вище підносяться ренти
з акра, так само як і загальна сума рент, то колосальніший стає податок, що
його виплачує суспільство великим земельним власникам у вигляді надзисків, — доки
всі роди землі, що вже підлягли обробленню, зберігають здатність до конкуренції.

Цей закон пояснює дивовижну живучість кляси великих землевласників.
Жодна інша кляса суспільства не живе так марнотратно, як ця; жодна інша
не заявляє такої претенсії на звичну «відповідну станові» розкіш, хоч би звідки
одержувано для цього гроші; жодна інша кляса не нагромаджує з таким легким
серцем боргів за боргами. А проте, вона завжди хоч і вскочить, а вискочить —
завдяки капіталові, що його інші люди вклали в землю і що дає їй ренти позавсяким
співвідношенням з зисками, що їх одержує з нього капіталіст. Але той самий закон
пояснює також, чому ця живучість великого землевласника поволі вичерпується.

Коли 1846 року скасовано було в Англії збіжжеві мита, англійські фабриканти
думали, що цим вони перетворили землевласницьку аристократію на
павперів. Замість цього вона забагатіла більше, ніж будь-коли раніш. Яким
чином це сталося? Дуже просто. Поперше, від цього часу до орендарів почали
ставити закріплену контрактом вимогу, за якою вони зобов’язувалися витрачати
щорічно по 12 ф. ст. замість 8 ф. ст. на акр, і по-друге, землевласники, що мали і в
нижній палаті дуже численних представників, асигнували собі велику державну
допомогу для дренування та інших перманентних поліпшень своїх земель. А що
цілковитого витиснення найгіршої землі не сталося, а відбулося, щонайбільше,
застосування її для іншої мети, та й то здебільша тимчасове, то ренти підвищились
відповідно до підвищенної витрати капіталу, і земельна аристократія
виграла від цього більше, ніж будь-коли раніш.

Але все минає. Трансатлантійські пароплави, а також північно південноамериканські
та індійські залізниці дали змогу цілком особливим країнам конкурувати
на європейських збіжжевих ринках. Це були, з одного боку, північноамериканські
прерії, арґентінські пампаси, степи вже від природи придатні для
обробітку плугом, незайманий ґрунт, що багато років давав багаті врожаї навіть
за примітивної культури і без добрива. Далі це були землі російських та
індійських комуністичних громад, які мусили продавати частину свого продукту,
до того ж дедалі більшу, щоб одержати гроші для виплати податків, що їх виплачувати
примушував, досить часто з допомогою катування, нещадний деспотизм
держави. Ці продукти продавалося безвідносно до цін продукції, продавалося
за ціну, яку пропонував торговець, бо селянин на строк виплати мусив
мати гроші хоч би за яку ціну. І з цією конкуренцією, — незайманої степової
землі, а також російських та індійських селян, що знемагають під податковим пресом,
— європейський орендар і селянин не міг упоратись при старих рентах. Частина
землі в Европі остаточно стала щодо продукції збіжжя, конкурентно неспроможною,
ренти всюди занепали; другий наш випадок, варіянт II: низхідна ціна
і низхідна продуктивність додаткових витрат капіталу зробився загальним
правилом для Европи, звідси лемент аґраріїв від Шотландії до Італії, від Південної
Франції до Східньої Прусії. На щастя, ще геть не всі степові землі
оброблено; їх ще надто досить для того, щоб зруйнувати все європейське велике
землеволодіння та крім того і дрібне. — Ф. Е.].

\pfbreak

Рубрики, під якими треба дослідити ренту, такі:

А. Диференційна рента.

1) Поняття диференційної ренти. Ілюстрація силою води. Перехід до власне
хліборобської ренти.
\parbreak{}  %% абзац продовжується на наступній сторінці
