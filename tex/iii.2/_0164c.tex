\parcont{}  %% абзац починається на попередній сторінці
\index{iii2}{0164}  %% посилання на сторінку оригінального видання
і низхідної ціни. Хоч в усіх цих випадках рента може залишитися без зміни
і може понизитись, вона понизилася б значніше, коли б додаткове вживання
капіталу, за інших незмінних обставин не зумовлювало збільшення родючости. Додаткове вкладення
капіталу тоді завжди є за причину відносної висоти ренти,
хоча б абсолютно вона й понизилась.

\section{Диференційна рента II. Третій випадок: висхідна ціна продукції}
\chaptermark{Диференційна рента II. Третій~випадок}
\vspace{8\bigskipamount}

[Підвищення ціни продукції має за свою передумову, що продуктивність землі
найгіршої якости, що не дає ренти, зменшується. Ціна продукції, взята нами за
реґуляційну, може піднестися вище від 3\pound{ ф. ст.} за кв., лише тоді, коли 2\sfrac{1}{2}\pound{ ф. ст.},
витрачені на $А$, продукуватимуть менш за 1 квартер, або 5\pound{ ф. ст.} менш за
2 квартери, або коли б довелося обробляти землю ще гіршої якости, ніж $А$.

За незмінної або навіть висхідної продуктивности другого вкладення капіталу
це було б можливе лише тоді, коли б продуктивність першого вкладення в 2\sfrac{1}{2}\pound{ф. ст.}
зменшилась. Цей випадок трапляється досить часто. Наприклад, коли виснажений
при поверховій оранці зверхній шар ґрунту дає при старій системі обробітку
дедалі менші врожаї, то витягнений на поверхню з допомогою глибшої
оранки нижній шар за раціонального обробітку починає давати вищі
урожаї, ніж давніш. Але цей сцеціяльний випадок, точно кажучи, сюди не
стосується. Пониження продуктивности першої витрати капіталу в 2\sfrac{1}{2}\pound{ ф. ст.} зумовлює для кращих
земель, навіть коли там припустити аналогічні відношення,
пониження диференційної ренти І; проте тут ми розглядаємо лише диференційну
ренту II.~Але тому, що даний спеціяльний випадок не може статися, коли не
припускається існування диференційної ренти II і тому, що він в дійсності
становить відбитий вплив модифікації диференційної ренти І на диференційну
ренту II, то ми наведемо приклад цього випадку.

\disablefootnotebreak{}
\begin{table}[H]
  \centering
  \caption*{Таблиця VII}

  \footnotesize
  \setlength{\tabcolsep}{4.5pt}
  \settowidth\rotheadsize{\theadfont Продажна}

  \begin{tabular}{l c r c c r c c c c c}
    \toprule
      \thead[tl]{Рід\\землі} &
      &
      \thead[t]{Капітал} &
      \rothead{Зиск} &
      \rothead{Ціна\\продукції} &
      \thead[t]{Продукт} & % \\ в кварт.}}}
      \rothead{Продажна\\ціна} &
      \rothead{Здобуток} &
      \multicolumn{2}{c}{Рента} &
      \rothead{Норма\\ренти} \\

    \cmidrule(rl){2-11}
      & акри  & \poundsign{} & \poundsign{} & \poundsign{} & кв. & \poundsign{} & \poundsign{} & кв. & \poundsign{} & \% \\

    \midrule
      A & 1 & 2\tbfrac{1}{2} \dplus{} 2\tbfrac{1}{2} \deq{} 5 & 1 & 6 & \phantom{0}\tbfrac{1}{2} \dplus{} 1\tbfrac{1}{4} \deq{} 1\tbfrac{3}{4}                      & 3\tbfrac{3}{7} & \phantom{0}6 & 0\phantom{\tbfrac{1}{2}} & \phantom{0}0 & \phantom{00}0\\
      B & 1 & 2\tbfrac{1}{2} \dplus{} 2\tbfrac{1}{2} \deq{} 5 & 1 & 6 & 1\phantom{\tbfrac{1}{1}} \dplus{} 2\tbfrac{1}{2} \deq{} 3\tbfrac{1}{2}                     & 3\tbfrac{3}{7} & 12           & 1\tbfrac{3}{4}           & \phantom{0}6 & 120 \\
      C & 1 & 2\tbfrac{1}{2} \dplus{} 2\tbfrac{1}{2} \deq{} 5 & 1 & 6 & 1\tbfrac{1}{2} \dplus{} 3\tbfrac{3}{4} \deq{} 5\tbfrac{1}{4}                               & 3\tbfrac{3}{7} & 18           & 3\tbfrac{1}{2}           & 12           & 240\\
      D & 1 & 2\tbfrac{1}{2} \dplus{} 2\tbfrac{1}{2} \deq{} 5 & 1 & 6 & 2\pF{} \dplus{} 5\pF{} \deq{} 7\pF{} & 3\tbfrac{3}{7} & 24           & 5\tbfrac{1}{4}           & 18           & 360\\

     \midrule

      Разом & & 20 & & & 17\tbfrac{1}{2} & & 60 & 10\tbfrac{1}{2} & 36 & 240\footnotemarkZ{}\\
 
  \end{tabular}
\end{table}
\footnotetextZ{Тут, як і далі в таблицях VIII, IX, і X пересічну норму ренти обчислено не до всього
вкладеного капіталу, а тільки до капіталу, вкладеного в рентодайні дільниці. \Red{Пр.~Ред.}} % текст примітки прямо під заголовком
\enablefootnotebreak{}
\vspace{-\bigskipamount}

\noindent{}Грошова рента, як і грошовий здобуток, лишаються ті самі, що і в таблиці II.~Підвищена реґуляційна ціна продукції точнісінько покриває те, що втрачено
на кількості продукту; а що ця ціна продукції і кількість продукту змінюються
в зворотному відношенні, то само собою зрозуміло, що здобуток їх лишається
той самий.
