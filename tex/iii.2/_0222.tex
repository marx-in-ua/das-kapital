\parcont{}  %% абзац починається на попередній сторінці
\index{iii2}{0222}  %% посилання на сторінку оригінального видання
збагачувало їх при звичайних для того часу довготермінових орендних договорах
коштом земельних власників.

Далі: скоро рента набуває форму грошової ренти, і разом з тим відносини,
між селянином, що виплачує ренту, і земельним власником набувають форму
договірних відносин — перетворення, що взагалі можливе лише при вже даному
відносно високому рівні розвитку світового ринку, торгівлі й мануфактури, —
неминуче починається і здача землі в оренду капіталістам, які до того часу
стояли за межами села, і які тепер переносять в село і в сільське господарство
придбаний в містах капітал і розвинений вже в містах капіталістичний спосіб
господарювання, створення продукту як лише товару і як просто засобу для
привласнення додаткової вартости. Загальним правилом ця форма може стати
лише в тих країнах, які за переходу з февдального в капіталістичний спосіб
продукції, панували на світовому ринку. З втручанням капіталістичного орендаря
між земельним власником і дійсно зайнятим хліборобом розриваються всі
відносини, що виникли з старого сільського способу продукції. Орендар стає
дійсним розпорядником цих хліборобських робітників і дійсним визискувачем
їхньої додаткової праці, тимчасом як земельний власник стоїть тепер
у безпосередніх відносинах, саме просто в грошових і договірних відносинах
лише до цього капіталістичного орендаря. Тим самим перетворюється
і природа ренти, не тільки фактично і випадково, як почасти було вже за
давніших форм, а нормально в її визнаній і панівній формі. Від нормальної
форми додаткової вартости і додаткової праці вона знижується до надміру цієї
додаткової праці над тією її частиною, яка привласнюється у формі зиску
капіталістом-визискувачем; так само, як уся додаткова праця, зиск і надмір
над зиском тепер здобувається безпосередньо капіталістом-визискувачем, забирається
в формі всього додаткового продукту і перетворюється на гроші. Як
ренту капіталіст-визискувач віддає земельному власникові тепер тільки надмірну
частину цієї додаткової вартости, яку він, в силу свого капіталу, здобув
через безпосередній визиск сільських робітників. Скільки саме він віддає земельному
власникові, це визначається пересічно, як межею, тим пересічним
зиском, який капітал дає в нехліборобських сферах продукції, і нехліборобськими
цінами продукції, що ним реґулюються. Отже, з нормальної форми додаткової
вартости і додаткової праці рента тепер перетворилась на характеристичний для цієї
окремої сфери продукції, — для хліборобської сфери, — надмір над тією частиною
додаткової праці, яку капітал забирає як частину, що безпосередньо й нормально
належить йому. Замість ренти тепер зиск став нормальною формою додаткової
вартости, і ренту вважається тепер лише усамостійненою за осібних
обставин формою не додаткової вартости взагалі, а певного її відростку, надзиску.
Немає потреби далі спинятись на тому, що цьому перетворенню відповідає
поступове перетворення в самому способі продукції. Це видно вже з того,
що нормальною для цього капіталістичного орендаря є продукція хліборобського
продукту як товару, і що тимчасом, як давніш, на товар перетворювався лише
надмір над його засобами існування, тепер лише відносно дуже незначна частина
цих товарів безпосередньо перетворюється на засоби існування для нього.
Вже не земля, а капітал безпосередньо упідлеглює тепер собі і своїй продуктивності
хліборобську працю.

Пересічний зиск і реґульована ним ціна продукції створюється поза відносинами
сільського господарства, у сфері міської торговлі й мануфактури. Зиск
селянина, що виплачує ренту, не бере участи у вирівнянні в пересічний зиск,
бо його відносини до земельного власника не є капіталістичні відносини.
Оскільки він одержує зиск, тобто реалізує — чи то власного працею, чи визиском
чужої праці — надмір над потрібними засобами свого існування, це відбувається
поза нормальними відносинами і, за інших рівних умов, не висота
\parbreak{}  %% абзац продовжується на наступній сторінці
