
\index{iii2}{0016}  %% посилання на сторінку оригінального видання
Що правда, титули власности на громадські підприємства, залізниці,
копальні й~\abbr{т. ін.}, як ми теж бачили, є в дійсності титули на дійсний капітал.
Однак, не дають вони права порядкувати тим капіталом. Його не можна вилучити
з підприємства. Вони дають лише право вимог на частину додаткової вартости,
що її той капітал має добути. Але ці титули стають теж паперовими
дублікатами дійсного капіталу, так як от, коли б накладна здобула вартість
поряд вантажу та одночасно з ним. Вони стають номінальними представниками
неіснуючих капіталів. Бо дійсний капітал існує поряд них та ніяк не переходить
з рук до рук від того, що ці дублікати переміняють своїх державців. Вони стають
формами капіталу, що дає процент, не тільки тому, що вони забезпечують певні
доходи, але й тому, що продаючи ті папери, можна одержати за них платіж
капітальними вартостями. Оскільки нагромадження цих паперів свідчить про
нагромадження залізниць, копалень, пароплавів і~\abbr{т. ін.}, воно свідчить про пошир
дійсного процесу репродукції, цілком так, як от збільшення податкового списку,
напр., на рухому власність, свідчитиме про зріст цього роду власности. Але як
дублікати, що самі можуть продаватись як товари, отже й самі обертаються як
капітальні вартості, ті папери ілюзорні, а величина їхньої вартости може спадати
та підноситися цілком незалежно від руху вартости того дійсного капіталу, що
на нього вони становлять титули. Величина їхньої вартости, тобто їхній курс на
біржі, зі спадом рівня проценту — оскільки цей спад, незалежний від особливих
рухів грошового капіталу, є простий наслідок тенденції норми зиску до зниження
— неминуче має тенденцію підноситися. Так що вже з цієї причини
в процесі розвитку капіталістичної продукції це уявлюване багатство ширшає
в вислові вартости щодо кожної його складової частини, порівнюючи з його первісною
номінальною вартістю\footnote{
Частина нагромадженого грошового позичкового капіталу в дійсності є простий вислів промислового
капіталу. Коли, напр., Англія в 1857 році примістила 80 мільйонів ф. ст. в американських залізницях
та інших підприємствах, то це приміщення майже геть усе сталося в формі вивозу англійських товарів,
що за них американцям не довелося платити ані копійки. Англійський експортер брав за ці товари
векселі на Америку, ті векселі скуповували англійські покупці акцій та відправляли до Америки для
частинної оплати тих акцій.
}.

Виграш та втрата через коливання цін цих титулів власности так само як
і їхня централізація в руках залізничих королів і~\abbr{т. ін.} стає по самій суті справи
чимраз більше результатом гри, що замість праці видається первісним способом
набувати власність на капітал і заступає також місце прямого насильства. Цей рід
уявлюваного грошового майна становить не тільки дуже значну частину грошового
майна приватних осіб, але й банкірського капіталу, про що вже згадувалося.

Можна було б — ми згадуємо це лише на те, щоб швидше довести справу
до кінця — нагромадження грошового капіталу розуміти також як нагромадження
багатства в руках банкірів (у позикодавців грошей з професії), як посередників
між приватними грошовими капіталістами, по один бік, та державою, громадами
й позикоємцями-репродуцентами, по другій бік; при цьому цілий величезний
пошир кредитової системи, взагалі сукупний кредит визискують вони, як свій приватний
капітал. Ці панове мають капітал та доходи завжди у грошовій формі
або у формі прямих вимог на гроші. Нагромадження майна цієї кляси може
відбуватися напрямком, дуже відмінним від дійсного нагромадження, доводячи
однак в усякому випадку, що ця кляса добру частину того останнього нагромадження
ховає собі до кишені.

Зведім дане питання до вужчих меж. Державні фонди, так само як і акції та
інші цінні папери усякого роду, є сфери приміщення для позичкового капіталу,
для капіталу, призначеного стати капіталом, що дає процент. Вони є форми визичення
того капіталу. Але сами вони не є той позичковий капітал, що його
в них приміщено. З другого боку, оскільки кредит відіграє безпосередню ролю
\parbreak{}  %% абзац продовжується на наступній сторінці
