
\index{iii2}{0010}  %% посилання на сторінку оригінального видання
Самостійний рух вартости цих титулів власности, не тільки державних
фондів, але й акцій, потверджує ілюзію, ніби вони становлять дійсний капітал
поряд того капіталу або тієї вимоги, що їхніми титулами можливо є вони.
Вони стають власне товарами, що їхня ціна має власний рух та власне усталення.
їхня ринкова вартість одержує відмінне від їхньої номінальної вартости
визначення, без того, щоб змінювалась вартість (хоч і змінюється зріст вартости)
дійсного капіталу. З одного боку, їхня ринкова вартість коливається разом з
висотою та певністю тих доходів, що на них вони дають правний титул. Коли
номінальна вартість якоїсь акції, тобто витрачена сума, що її первісно представляла
та акція, становить 100\pound{ ф. ст.}, а підприємство замість 5\% дає 10\%,
то її ринкова вартість серед решти однакових обставин та при рівні проценту
в 5\% зростає до 200\pound{ ф. ст.}, капіталізована бо з 5\%, вона становить тепер фіктивний
капітал в 200\pound{ ф. ст}. Хто купує її за 200\pound{ ф. ст.}, одержує 5\% доходу
від цього приміщення капіталу. Навпаки буває, коли дохід підприємства меншає.
Ринкова вартість цих паперів почасти спекулятивна, бо вона визначається не
тільки дійсним доходом, але й сподіваним, наперед обрахованим. Але, коли припустити,
що розмір зростання вартости дійсного капіталу є постійний, або коли там,
де жодного капіталу немає, як от в державних боргах, вважати щорічний дохід
за зафіксований законом та загалом за досить забезпечений, то ціна цих цінних
паперів підноситься або спадає зворотно проти піднесення або спаду
рівня проценту. Якщо рівень проценту зростає від 5\% до 10\%, то якийсь
цінний папер, що забезпечує 5\pound{ ф. ст.} доходу, становитиме капітал тільки
в 50\pound{ ф. ст}. Коли рівень проценту спаде до 2\sfrac{1}{2}\%, то той самий цінний
папер становитиме капітал в 200\pound{ ф. ст}. Його вартість є завжди тільки капіталізований
дохід, тобто дохід, обчислений від ілюзорного капіталу за наявним
рівнем проценту. Отже, підчас скрути на грошовому ринку ці цінні папери
падатимуть у ціні подвійно; поперше, тому, що рівень проценту зростає, а подруге
тому, що їх масами викидають на ринок, щоб реалізувати їх у грошах.
Цей спад ціни відбувається незалежно від того, чи розмір доходу, забезпечуваний
цими паперами своєму державцеві, є постійний, як от в державних фондах,
або чи зростання вартости того дійсного капіталу, що його вони представляють,
порушується перешкодами у процесі репродукції, як от це може бути в промислових
підприємствах. В останньому разі до вже згаданого знецінення долучається
тільки ще нове. Коли буря минулася, ці папери знову підносяться до своєї
колишньої висоти, оскільки вони не представляють зруйнованих або шахрайських
підприємств. їхнє знецінення підчас кризи впливає як міцний засіб до централізації
грошового майна\footnote{[Безпосередньо по лютневій революції, коли в Парижі товари й цінні папери були вкрай
знецінені та їх зовсім не можна було продати, один швайцарський купець в Ліверпулі, пан Р.~Цвільхенбарт
(що оповів про це моєму батькові) повернув на гроші все що міг, поїхав з готівкою до Парижу
та удався до Ротшільда з пропозицією зробити спільно ґешефт. Ротшільд пронизливо глянув на нього та,
кинувшись до нього, ухопив його за плечі: «Avez-vous de l’argent sur vous? — Oui M. le baron! —
Alors vous-êtes-mon homme! — («Маєте гроші? — Так, пане бароне! — Тоді ви мій чоловік!»).
І вони спільно зробили блискучий ґешефт. — Ф.~Е.].}.

Оскільки знецінення або піднесення вартости цих паперів не залежить
від руху вартости того дійсного капіталу, що його вони представляють, багатство
нації лишається по знеціненні або піднесенні вартости таке саме, як і перед
тим. «23 жовтня 1847 року державні фонди й акції каналів та залізниць знецінилися
на \num{114.725.255}\pound{ ф. ст.}» (Morris, управитель Англійського банку, свідчення
у звіті про Commercial Distress 1847--48~\abbr{р.}). Оскільки їхнє знецінення не
свідчило про дійсний спин продукції та комунікації на залізницях та каналах,
або про ліквідацію вже початих підприємств, або про марне кидання капіталу в
\parbreak{}  %% абзац продовжується на наступній сторінці
