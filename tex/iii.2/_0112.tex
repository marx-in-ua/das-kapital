\parcont{}  %% абзац починається на попередній сторінці
\index{iii2}{0112}  %% посилання на сторінку оригінального видання
аристократією і буржуазією самим собі, — з очевидністю, поза всяким сумнівом
доводять, що високі ренти і відповідний їм зріст земельних цін підчас антиякобінської
війни завдячують почасти тільки вирахуванню з заробітної плати
і її пониженню навіть нижче від фізичного рівня; тобто завдячують виплаті
частини нормальної заробітної плати земельному власникові. Різні обставини,
між іншим, знецінення грошей, спосіб додержування законів про бідних
у хліборобських округах і~\abbr{т. ін.} уможливили цю операцію в той самий час, коли
доходи орендарів колосально зростали і землевласники казково збагачувались.
А втім, одним з головних арґументів так орендарів, як і землевласників на
користь запровадження хлібних мит був той, що фізично неможливо ще більше
знизити заробітну плату сільських поденників. Це становище по суті не змінилось,
і в Англії, як в усіх європейських країнах, частина нормальної
заробітної плати по-старому входить до складу земельної ренти. Коли
граф Shaftesbury, в той час лорд Ashley, один з філантропів-аристократів,
був так надзвичайно зворушений становищем англійських фабричних робітників
і виступив їхнім парламентським оборонцем в справі аґітації за десятигодинний
день — на помсту за це оборонці промисловців опублікували статистичні
дані про заробітну плату сільських поденників у належних йому селах (див.
книга І, розділ XXIII, 5, е: британський хліборобський пролетаріат), які
ясно показали, що частина земельної ренти цього філантропа постає просто
з грабунку, що за нього чинять його орендарі над заробітною платою
хліборобських робітників. Ця публікація ще й тим цікава, що наведені в ній
факти можуть сміливо стати поряд з усім найгіршим, що розкрили комісії
1814 і 1815~\abbr{рр.} Скоро тільки обставини примушують до тимчасового підвищення
заробітної плати хліборобських робітників, так орендарі починають кричати, що
підвищення заробітної плати до нормального рівня, якого вона досягає в інших
галузях промисловости, є річ неможлива, яка неминуче зруйнує їх, коли
одночасно не знизити земельної ренти. Отже, тут робиться визнання, що під
ім’ям земельної ренти орендарі роблять вирахування з заробітної плати і виплачують
його землевласникові. Наприклад, 1849--1859~\abbr{р.} заробітна плата хліборобських
робітників в Англії підвищилась в наслідок збігу потужних обставин
як от: еміграція з Ірландії, що припинила приплив звідти хліборобських
робітників; надзвичайне вбирання хліборобської людности фабричною промисловістю;
попит на салдатів для війни; надзвичайна еміграція в Австралію
і Сполучені Штати (в Каліфорнію) та інші причини, на яких тут не доводиться
спинятися докладніше. Одночасно за цей період, за винятком 1854--1856~\abbr{рр.}, років
з кепськими урожаями, пересічні ціни збіжжя понизились більше, ніж на 16\%.
Орендарі волали про пониження рент. В окремих випадках вони досягли цього.
Взагалі ж вони з цією вимогою зазнали поразки. Вони вдалися до пониження
витрат продукції, між іншим засобом масового вживання парових
локомобілів і нових машин, які почасти замінили й витиснули з господарства
коней, а почасти, звільняючи хліборобських робітників, зумовили
штучне перелюднення, а тому і нове зниження заробітної плати. І це відбувалось,
не зважаючи на загальне відносне зменшення хліборобської людности за це
десятиліття, порівняно з ростом усієї людности, і не зважаючи на абсолютне зменшення
хліборобської людности в деяких суто-хліборобських округах\footnote{
John C.~Morton, The Forces used in Agriculture. Доповідь у Лондонському Society of Arts 1860
року, основана на автентичних документах, зібраних безпосередньо приблизно у 100 орендарів в 12
шотляндських і 35 англійських графствах.
}. Так
само, 12 жовтня 1865 року Fowcett, тоді професор політичної економії у Кембріджі,
що вмер 1884 року генерал-почтмайстером, говорив на Social Science
Congress: хліборобські поденники починають еміґрувати, і орендарі починають
\parbreak{}  %% абзац продовжується на наступній сторінці
