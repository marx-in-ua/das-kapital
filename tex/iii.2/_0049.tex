
\index{iii2}{0049}  %% посилання на сторінку оригінального видання
Не треба ніколи забувати, що хоч 19--20 мільйонів банкнотами, як
кажуть, перебуває в руках публіки досить стало, проте, та частина цих банкнот,
що є дійсно в циркуляції, з одного боку, і та частина їх, що лежить в банках
незайнята як запас, з другого боку, раз-у-раз та значно змінюються одна
проти однієї. Якщо цей запас великий, отже, рівень дійсної циркуляції низький,
то з погляду грошового ринку це значить, що сфера циркуляції є повна (the
circulation is full, money is plentifull); коли запас малий, отже коли рівень
дійсної циркуляції високий, то грошовий ринок зве його низьким (the circulation
is low, money is scarce); саме та частина являє низьку суму, що представляє
позичковий незайнятий капітал. Дійсний, від фаз промислового циклу незалежний,
пошир або скорочення циркуляції — так що однак та сума, що її потребує
публіка, лишається однаковою — буває лише з технічних причин, напр., коли настає
реченець платежа податків або процентів на державний борг. При платежі податків
банкноти та золото припливають до Англійського банку понад звичайну міру,
фактично скорочуючи циркуляцію, не зважаючи на потреби останньої. Навпаки
буває, коли виплачується дивіденди на державний борг. В першому випадку
роблять позики в банку на те, щоб добути засоби циркуляції. В останньому
випадку спадає рівень проценту в приватних банках з причини тимчасового
зросту їхніх резервів. Де не має нічого до діла з абсолютною масою засобів
циркуляції, а тільки має до діла з тією банковою фірмою, що пускає ці засоби
в циркуляцію і що з погляду її той процес видається вивласненням
позичкового капіталу, що й дає їй тому змогу ховати собі до кишені зиск
від того.

В одному випадку відбувається лише часове переміщення засобів циркуляції,
що його Англійський банк вирівнює тим способом, що незадовго перед реченцем
платежа чвертьрічних податків або виплати так само чвертьрічних дивідендів
він видає короткотермінові позики за низькі проценти; отож ці отак понад міру
видані банкноти спершу заповнюють ті прогалини, що їх викликав платіж
податків, тимчасом як їх зворотний платіж до банку зараз же по тому усовує
той надмір банкнот, що до його призводить виплата дивідендів публіці.

В другому випадку низький або високий рівень циркуляції завжди становить
лише інший розподіл тієї самої маси засобів циркуляції на активну циркуляцію
та вклади, тобто знаряддя позик.

З другого боку, коли, напр., через приплив золота до Англійського банку
більшає число банкнот, виданих за те золото, то ці останні допомагають
дисконтові поза банком та припливають назад на оплату позик, так що абсолютна
маса банкнот в циркуляції збільшується лише на короткий час.

Якщо циркуляція повна з причини поширу справ (що можливе й при порівняно
низьких цінах), то рівень проценту може бути, відносно високий з причини
попиту на позичковий капітал, що зумовлюється зростом зиску та збільшенням
змоги нових приміщень капіталу. Коли рівень циркуляції є низький в наслідок
скорочення справ або й великої поточности кредиту, то рівень проценту може бути
низький і при високих цінах (див. Hubbard).

Абсолютний розмір циркуляції впливає на рівень проценту, визначаючи
його, тільки підчас пригнічення. Тут попит на поширену циркуляцію означає
або лише попит на засоби до утворення скарбів (якщо не вважати на зменшену
швидкість, з якою гроші обертаються та з якою ті ж самі монети раз-у-раз
перетворюються на позичковий капітал) в наслідок відсутности кредиту, як от
в 1847 році, коли припинення банкового акту не викликало жодного поширу
циркуляції, але його вистачило, щоб нагромаджені скарбом банкноти знову
витягти на світ денний та кинути їх до циркуляції. Абож у певних обставинах
дійсно може бути потрібно більше засобів циркуляції, як от в 1857 році, коли
циркуляція по припиненні банкового акту дійсно зросла на деякий час.
