\parcont{}  %% абзац починається на попередній сторінці
\index{iii2}{0204}  %% посилання на сторінку оригінального видання
становища справ у країні. Для того, щоб додатковий капітал спрямувати до
хліборобства, за періодів скрути не досить того, що необроблений ґрунт може
дати орендареві пересічний зиск, — чи виплачує він ренту, чи ні. В інші періоди,
періоди достатку (der Plethora) капіталу він прямує до оброблення землі, навіть
без підвищення ринкової ціни, аби тільки взагалі були в наявності нормальні
умови. Тоді краща земля, ніж оброблювана до того часу, в дійсності могла б
бути виключена з конкуренції тільки або моментом її положення, або все ще
непоборними межами її винятковости, або випадковістю. Тому ми можемо зайнятися
тільки тими родами землі, що якістю однакові з останніми з оброблених
земель. Але між новою землею і останньою з оброблених все ще лишається
ріжниця витрат на оброблення нової землі, і від стану ринкових цін і відносин
кредиту залежить, чи будуть вони зроблені чи ні. Потім, скоро лише ця земля
дійсно візьме участь у конкуренції, ринкова ціна за інших незмінних відносин
знову понизиться до свого колишнього рівня, при чому новооброблювана земля
буде давати таку саму ренту, як відповідна їй стара. Засаду, що вона не даватиме
ренти, прихильники цієї засади доводять припущенням того, що треба ще довести,
а саме: що остання земля не дала ренти. Таким самим способом можна було б довести,
що останні з побудованих будинків, крім власне плати за наймання (Miethzins)\footnote*{
Звичайно плата за наймання чогось тут — комірна плата, як процент та амортизація
вкладеного у будівлю капіталу, на відміну від власне ренти. \Red{Прим. Ред.}
}
будівель, не дають жодної ренти, хоч і винаймається їх. Але факт такий,
що коли вони протягом довгого часу лишаються незайняті, вони дають ренту, ще
до того, як починають давати плату за наймання (Miethzins). Подібно до того,
як послідовні приміщення капіталу на певній дільниці землі можуть давати
відповідний додатковий здобуток, а тому і таку саму ренту, як перші, — цілком
так само лани такої якости, як останні з оброблених, можуть при рівних витратах
давати рівний здобуток. Інакше взагалі було б незрозуміло, яким чином
можна було б лани однакової якости брати під оброблення послідовно, а не
всі разом або, радше, не брати жодного, щоб не викликати конкуренції всіх,
інших. Земельний власник завжди готовий здобувати ренту, тобто одержувати
щось даром; але щоб задовольнити його бажання, капітал потребує певних
умов. Тому взаємна конкуренція між землями залежить не від того, що землевласник
хоче їхньої конкуренції, але від того, чи знайдеться капітал, який
схоче на нових ланах конкурувати з іншими.

Оскільки власне хліборобська рента є просто наслідок монопольної ціни,
вона може бути лише незначна, як і абсолютна рента може бути тут за нормальних
умов лише незначною, хоч би який був надмір вартости продукту
над його ціною продукції. Отже, суть абсолютної ренти є ось у чому: рівновеликі
капітали в різних сферах продукції, при рівній нормі додаткової вартости
або рівній експлуатації праці, продукують, залежно від їхнього різного пересічного
складу, різні маси додаткової вартости. В промисловості ці різні маси
додаткової вартости вирівнюються в пересічний зиск і рівномірно розподіляються
між окремими капіталами, як між відповідними частинами суспільного капіталу.
Земельна власність, коли для продукції потрібна земля чи то для хліборобства,
чи то для здобування сирових матеріялів, перешкоджає цьому вирівнюванню для
капіталів, приміщених у землю, і уловлює певну частину додаткової вартости,
яка інакше взяла б участь у вирівнюванні на загальну норму зиску. Отож,
рента становить частину вартости, точніше додаткової вартости товарів, алеж
тільки таку, що замість дістатися клясі капіталістів, яка здобула її з робітників,
дістається земельним власникам, які здобувають її з капіталістів. При
цьому припускається, що хліборобський капітал пускає в рух більше праці,
ніж рівновелика частина нехліборобського капіталу. До якої міри велике
\parbreak{}  %% абзац продовжується на наступній сторінці
