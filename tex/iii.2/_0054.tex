
\index{iii2}{0054}  %% посилання на сторінку оригінального видання
З другого боку, приватні капіталісти і~\abbr{т. ін.}, що мають гроші, не дають
їх ні за які проценти, бо вони за Chapman’ом кажуть: «5194. Краще зовсім не
мати жодних процентів, аніж не знати напевно, чи ми одержимо гроші, коли
вони будуть нам потрібні».

«5173. Наша система така: ми маємо на 300 міл. ф. ст. зобов’язань, що
їхньої оплати звичайними в країні грішми можуть вимагати кожного даного
моменту одразу; і цих грошей, коли ми уживемо їх усі на це, є на 23 міл.
ф. ст. або щось біля того; чи це не такий стан, що кожної хвилини може
кинути нас у конвульсію?» Відси підчас криз раптове перетворення кредитової
системи на систему монетарну.

Коли не вважати на паніку всередині країни підчас криз, то мова про
кількість грошей може бути лише остільки, оскільки це стосується металю, світових
грошей. Та саме це вилучає Chapman, він каже тільки про 23 міл.
\emph{банкнотами}.

Той самий Chapman: «5218. Первісною причиною розладу на грошовому
ринку» [квітень та пізніше жовтень 1847 року) «безперечно була маса грошей,
що в наслідок значного довозу в цім році потрібні були на те, щоб реґулювати
вексельні курси».

Поперше, цей запас грошей світового ринку було тоді зведено до свого
мінімуму. Подруге, він одночасно був за ґарантію розмінности кредитових грошей,
банкнот. Отже, він сполучав дві цілком відмінні функції, що однак обидві
випливають з природи грошей, бо дійсні гроші завжди є гроші світового ринку,
а кредитові гроші завжди спираються на гроші світового ринку.

В 1847 році, коли б не припинили чинности банкового акту 1844 року,
«розрахункові палати не могли б провадити свої справи» (5221).

Проте Chapman мав певне передчуття близької кризи: «5236. Бувають певні
ситуації на грошовому ринку (а сучасна не дуже далеко від того), коли з грішми
тяжко, і доводиться удаватися до банку».

«5239. Щодо сум, взятих нами з банку в п’ятницю, суботу та понеділок,
19, 20 та 22-го жовтня 1847 року, то ближчої середи ми були б тільки найбільше
вдячні, коли б спромоглися одержати векселі назад; гроші одразу ж
почали припливати до нас, скоро минулася паніка». — У вівторок 23 жовтня
саме припинили чинність банкового акту й тим перебороли кризу.

Chapman тієї думки, 5274, що сума всіх поточних векселів на Лондон
становить на кожний момент 100--120 міл. ф. ст. Ця сума не охоплює місцевих
векселів на провінціяльні пункти.

«5287. Хоч у жовтні 1856 року сума банкнот в руках публіки зросла%
\break
до \num{21.155.000}\pound{ ф. ст.}, проте незвичайно тяжко було діставати гроші; не зважаючи
на те, що публіка мала стільки грошей в своїх руках, ми не мали змоги запопасти
їх до своїх рук». І це саме в наслідок неспокою, породженого тим
скрутним станом, що в ньому деякий час (березень 1856 року) перебував
Eastern Bank.

5190--92. Скоро паніка минулася, «всі банкіри, що свій зиск добувають
з проценту, одразу ж почали пускати свої гроші в діло».

5302. Chapman пояснює неспокій, зумовлений зменшенням банкового
запасу, не страхом за вклади, а тим, що всі ті, хто може опинитися в такому
стані, що їм доведеться платити раптом великі грошові суми, знають дуже добре,
що підчас скрути на грошовому ринку вони можуть бути змушені вдатись до
банку, як до останнього джерела допомоги; а «коли банк має дуже малий запас,
він не радітиме цьому, а навпаки».

А втім, прегарна річ, як той запас зникає як фактична величина. Банкіри
тримають почасти в себе, почасти в Англійському банку певний мінімум, потрібний
\parbreak{}  %% абзац продовжується на наступній сторінці
