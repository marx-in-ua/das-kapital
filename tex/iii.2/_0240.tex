\parcont{}  %% абзац починається на попередній сторінці
\index{iii2}{0240}  %% посилання на сторінку оригінального видання
від усякої історичної його визначености. Ми маємо знову те саме тільки в іншій
формі, коли кажуть: продукт, в якому втілюється праця найманого робітника
на себе самого, як його здобуток, його дохід, це лише заробітна плата, та частина
вартости (а тому й суспільного продукту, вимірюваного цією вартістю),
яка становить його заробітну плату. Отже, коли наймана праця збігається з
працею взагалі, то й заробітна плата збігається з продуктом праці, і та частина
вартости, яка репрезентована заробітною платою, збігається з вартістю, взагалі
створеною працею. Але в наслідок цього і інші частини вартости, зиск і рента,
так само самостійно протиставляться заробітній платі, і їх доводиться виводити
з власних джерел, специфічно відмінних і незалежних від праці; їх доводиться
виводити з співдіющих елементів продукції, що посідачам їх вони припадають,
отже, зиск доводиться виводити з засобів продукції, речевих елементів капіталу,
а ренту з землі або природи, репрезентованої земельним власником (Рошер).

Тому земельна власність, капітал і наймана праця з джерел доходу в тому
розумінні, що капітал притягає в формі зиску до капіталіста ту частину додаткової
вартости, яку він здобуває з праці, монополія на землю притягає до
земельного власника іншу частину в формі ренти, а праця дає робітникові в
формі заробітної плати останню ще вільну частину вартости, з джерел доходу,
що за їх посередництвом одна частина вартости перетворюється на форму зиску,
друга на форму ренти і третя на форму заробітної плати, — перетворюються на
дійсні джерела, що з них виникають ці частини вартости і ті відповідні частини
продукту, що в них вони існують або на які вони можуть бути обмінені —
на джерела, з яких кінець-кінцем виникає сама вартість продукту\footnote{
«Заробітна плата, зиск і рента є три первісні джерела всякого доходу, так само як і всієї
мінової вартости» (А.~Smith). «Таким чином, причини матеріяльної продукції є одночасно джерела всіх
сущих основних доходів» (Storch, І, р. 259).
}.

Розглядаючи простіші категорії капіталістичного способу продукції, і навіть
товарової продукції, товар і гроші, ми вже зазначали той містифікаційний
характер, що перетворює суспільні відносини, що для них при продукції речеві
елементи багатства правлять за носіїв, на властивості самих цих речей (товари)
і ще яскравіше, саме продукційне відношення — на річ (гроші). Всі форми
суспільства, оскільки вони призводять до товарової продукції і грошової циркуляції,
беруть участь у цьому перекрученні. Але за капіталістичного способу
продукції й за капіталу, який є його панівною категорією, його визначальним
відношенням продукції, цей зачарований і перекручений світ розвивається геть
більше. Коли розглядати капітал, насамперед в безпосередньому процесі продукції,
— як висмоктувача додаткової праці, — то це відношення ще дуже просте;
і дійсний внутрішній зв’язок ще нав’язується носіям цього процесу, самим
капіталістам і ще усвідомлюється ними. Це переконливо доводиться упертою
боротьбою за межі робочого дня. Але навіть всередині цієї неускладненої сфери,
сфери безпосереднього процесу між працею й капіталом, справа не лишається
така проста. З розвитком відносної додаткової вартости за власне специфічного
капіталістичного способу продукції, в наслідок чого розвиваються і суспільні
продуктивні сили праці, — ці продуктивні сили і суспільні відносини праці виступають
у безпосередньому процесі праці в такому вигляді, як ніби з праці
вони перенесені в капітал. Тим самим капітал стає дуже таємничою істотою,
бо всі суспільні продуктивні сили праці виступають у такому вигляді,
ніби вони належать йому, а не праці як такій, і як такі сили, що народжуються
в його власних надрах. А потім втручається процес циркуляції, що
в його обмін речовин і зміну форм втягується всі частини капіталу, навіть
хліборобського капіталу, в тій самій мірі, в якій розвивається специфічно
капіталістичний спосіб продукції. Це є така сфера, в якій відносини первісної
\parbreak{}  %% абзац продовжується на наступній сторінці
