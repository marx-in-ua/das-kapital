\parcont{}  %% абзац починається на попередній сторінці
\index{iii2}{0133}  %% посилання на сторінку оригінального видання
казати, що вартість продуктів залишилась би та сама в умовах заміни капіталістичної
продукції асоціяцією. Тотожність ринкової ціни однорідних товарів є
спосіб, у який, на базі капіталістичного способу продукції і взагалі продукції,
що ґрунтується на обміні товарів між поодинокими продуцентами, пробивається
суспільний характер вартости. Те, що суспільство, розглядуване як споживач,
переплачує за хліборобські продукти, і що становить мінус у реалізації його
робочого часу в хліборобській продукції, це становить тепер плюс для однієї
частини суспільства, для земельних власників.

Друга обставина, важлива для того, про що говориться в дальшім розділі
під рубрикою II, така:

Справа не тільки в ренті з акра або з гектара, взагалі не тільки в ріжниці
між ціною продукції і ринковою ціною, або між індивідуальною й загальною
ціною продукції з акра, але також і в тому, скільки акрів кожного
роду землі обробляється. Тут важлива безпосередньо лише величина загальної
суми ренти, тобто сукупної ренти з усієї оброблюваної площі; але це дає
нам одночасно можливість перейти до з’ясовування того, як підвищується \emph{норма ренти},
хоч ціни не збільшуються, і хоч за низхідних цін не збільшуються
ріжниці у відносній родючості різних родів землі. Вище ми мали:
% (див. табл. I).

\begin{table}[H]
  \small
  \centering
  \caption*{Таблиця I}

  \begin{tabular}{l c c c c c}
    \toprule
      \makecell[l]{Рід\\землі} &
      \makecell{Акри} &
      \makecell{Ціна\\продукції,\pound{}} &
      \makecell{Продукт,\\кварт.} &
      \makecell{Рента\\в збіжжі, кварт.} &
      \makecell{Грошова\\рента,\pound{}}
      \\
     \midrule
     A & 1 & 3 & 1 & 0 & 0 \\
     B & 1 & 3 & 2 & 1 & 3 \\
     C & 1 & 3 & 3 & 2 & 6 \\
     D & 1 & 3 & 4 & 3 & 9 \\
     \midrule
     Сума & 4 & \textendash{} & \hang{r}{1}0 & 6 & \hang{r}{1}8 \\
  \end{tabular}
\end{table}

\noindent{}Припустімо тепер, що число оброблюваних акрів кожного розряду подвоїлося.
В такому разі ми матимемо:
% (див. табл. Iа).

\begin{table}[H]
  \small
  \centering
  \caption*{Таблиця Iа}

  \begin{tabular}{l c c c c c}
    \toprule
      \makecell[l]{Рід\\землі} &
      \makecell{Акри} &
      \makecell{Ціна\\продукції,\pound{}} &
      \makecell{Продукт,\\кварт.} &
      \makecell{Рента\\в збіжжі, кварт.} &
      \makecell{Грошова\\рента,\pound{}}
      \\
     \midrule
     A & 2 & 6 & 2 & 0 & \phantom{0}0 \\
     B & 2 & 6 & 4 & 2 & \phantom{0}6 \\
     C & 2 & 6 & 6 & 4 & 12 \\
     D & 2 & 6 & 8 & 6 & 18 \\
     \midrule
     Сума 
       & 8 & \textendash{} & \hang{r}{2}0 & \hang{r}{1}2 & 36 \\
  \end{tabular}
\end{table}

\noindent{}Ми припустимо ще 2 випадки; перший, коли продукція розширюється
на обох гірших родах землі. Отже, тоді матимемо:
%(див. табл. Іb).

\begin{table}[H]
  \small
  \centering
  \caption*{Таблиця Ib}
  \begin{tabular}{l c c c c c c}
    \toprule
      \multirowcell{2}[0ex][l]{Рід\\землі} &
      \multirowcell{2}{Акри} &
      \multicolumn{2}{c}{Ціна продукції,\pound{ ф. ст.}} &
      \multirowcell{2}{Продукт} &
      \multirowcell{2}{Рента\\в збіжжі, кварт.} &
      \multirowcell{2}{Грошова\\рента,\pound{}} \\
      \cmidrule(rl){3-4}

      &  &  на акр. & в сумі & &                    &  \\
      \midrule
      A & 4 & 3 & 12 & 4 & 0 &  \pZ{}0 \\
      B & 4 & 3 & 12 & 8 & 4 & 12 \\
      C & 2 & 3 & \pZ{}6 & 6 & 4 & 12 \\
      D & 2 & 3 & \pZ{}6  & 8 & 6 & 18 \\
     \midrule
     Сума & \hang{r}{1}2 & \textendash{} & 36 & \hang{r}{2}6 & \hang{r}{1}4 & 42 \\
  \end{tabular}
\end{table}

\noindent{}І, нарешті, коли маємо неоднакове поширення продукції і оброблюваної площі в чотирьох розрядах:
% (див. табл. Iс).

\begin{table}[H]
  \small
  \centering
  \caption*{Таблиця Iс}
  \begin{tabular}{l c c c c c c}
    \toprule
      \multirowcell{2}[0ex][l]{Рід\\землі} &
      \multirowcell{2}{Акри} &
      \multicolumn{2}{c}{Ціна продукції,\pound{ ф. ст.}} &
      \multirowcell{2}{Продукт} &
      \multirowcell{2}{Рента\\в збіжжі, кварт.} &
      \multirowcell{2}{Грошова\\рента,\pound{}} \\
      \cmidrule(rl){3-4}

      &  &  на акр. & в сумі & &                    &  \\
      \midrule
      A & 1 & 3 & \pZ{}3 & \pZ{}1 & \pZ{}0 &  \pZ{}0 \\
      B & 2 & 3 & \pZ{}6 & \pZ{}4 & \pZ{}2 & \pZ{}6 \\
      C & 5 & 3 & 15      & 15     & 10 & 30 \\
      D & 4 & 3 & 12      & 16     & 12 & 36 \\
     \midrule
     Сума & \hang{r}{1}2 & \textendash{} & 36 & 36 & 24 & 72 \\
  \end{tabular}
\end{table}

\noindent{}Насамперед, в усіх цих випадках І, І$а$, І$b$, І$с$ рента з одного акра лишається та сама; бо в
дійсності продукт однакової маси капіталу на кожному
\parbreak{}  %% абзац продовжується на наступній сторінці
