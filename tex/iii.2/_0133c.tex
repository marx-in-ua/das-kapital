\parcont{}  %% абзац починається на попередній сторінці
\index{iii2}{0133}  %% посилання на сторінку оригінального видання
казати, що вартість продуктів залишилась би та сама в умовах заміни капіталістичної
продукції асоціацією. Тотожність ринкової ціни однорідних товарів є
спосіб, у який, на базі капіталістичного способу продукції і взагалі продукції,
що ґрунтується на обміні товарів між поодинокими продуцентами, пробивається
суспільний характер вартости. Те, що суспільство, розглядуване як споживач,
переплачує за хліборобські продукти, і що становить мінус у реалізації його
робочого часу в хліборобській продукції, це становить тепер плюс для однієї
частини суспільства, для земельних власників.

Друга обставина, важлива для того, про що говориться в дальшім розділі
під рубрикою II, така:

Справа не тільки в ренті з акра або з гектара, взагалі не тільки в ріжниці
між ціною продукції і ринковою ціною, або між індивідуальною й загальною
ціною продукції з акра, але також і в тому, скільки акрів кожного
роду землі обробляється. Тут важлива безпосередньо лише величина загальної
суми ренти, тобто сукупної ренти з усієї оброблюваної площі; але це дає
нам одночасно можливість перейти до з’ясовування того, як підвищується норма
ренти, хоч ціни не збільшуються, і хоч за низхідних цін не збільшуються
ріжниці у відносній родючості різних родів землі. Вище ми мали:
% (див. табл. I).

\begin{table}[H]
  \small
  \centering
  \caption*{Таблиця I}

  \begin{tabular}{l c c c c c}
    \toprule
      \makecell[l]{Рід\\землі} &
      \makecell{Акри} &
      \makecell{Ціна\\продукції,\pound{}} &
      \makecell{Продукт,\\кварт.} &
      \makecell{Рента\\в збіжжі, кварт.} &
      \makecell{Грошова\\рента,\pound{}}
      \\
     \midrule
     A & 1 & 3 & 1 & 0 & 0 \\
     B & 1 & 3 & 2 & 1 & 3 \\
     C & 1 & 3 & 3 & 2 & 6 \\
     D & 1 & 3 & 4 & 3 & 9 \\
     \midrule
     Сума & 4 & \textendash{} & \hang{r}{1}0 & 6 & \hang{r}{1}8 \\
  \end{tabular}
\end{table}

\noindent{}Припустімо тепер, що число оброблюваних акрів кожного розряду подвоїлося.
В такому разі ми матимемо:
% (див. табл. Iа).

\begin{table}[H]
  \small
  \centering
  \caption*{Таблиця Iа}

  \begin{tabular}{l c c c c c}
    \toprule
      \makecell[l]{Рід\\землі} &
      \makecell{Акри} &
      \makecell{Ціна\\продукції,\pound{}} &
      \makecell{Продукт,\\кварт.} &
      \makecell{Рента\\в збіжжі, кварт.} &
      \makecell{Грошова\\рента,\pound{}}
      \\
     \midrule
     A & 2 & 6 & 2 & 0 & \phantom{0}0 \\
     B & 2 & 6 & 4 & 2 & \phantom{0}6 \\
     C & 2 & 6 & 6 & 4 & 12 \\
     D & 2 & 6 & 8 & 6 & 18 \\
     \midrule
     Сума 
       & 8 & \textendash{} & \hang{r}{2}0 & \hang{r}{1}2 & 36 \\
  \end{tabular}
\end{table}

\noindent{}Ми припустимо ще 2 випадки; перший, коли продукція розширюється
на обох гірших родах землі. Отже, тоді матимемо: (див. табл. Іb).

\begin{table}[H]
  \begin{center}
    \emph{Таблиця Ib.}
    \footnotesize

  \begin{tabular}{c c c c c c c}
    \toprule
      \multirowcell{2}{\makecell{Рід \\землі}} &
      \multirowcell{2}{\makecell{Акри}} &
      \multicolumn{2}{c}{Ціна продукції} &
      \multirowcell{2}{\makecell{Продукт}} &
      \multirowcell{2}{\makecell{Рента \\ в збіжжі}} &
      \multirowcell{2}{\makecell{Грошова \\рента}} \\
      \cmidrule(rl){3-4}

      &  &  На акр. & В сумі & &                    &  \\
      \midrule

      A & 4\phantom{акр.} &  3\pound{ ф. ст.}                 & 12\pound{ ф. ст.}         & 4  кварт.         & 0  кварт.         & 0\pound{ ф. ст.}\\
      B & 4\phantom{акр.} &  3  \ditto{ф.} \ditto{ст.} & 12  \ditto{ф.} \ditto{ст.} & 8  \ditto{кварт.} & 4  \ditto{кварт.} & 12  \ditto{ф.} \ditto{ст.}\\
      C & 2\phantom{акр.} &  3  \ditto{ф.} \ditto{ст.} & 6  \ditto{ф.} \ditto{ст.}  & 6  \ditto{кварт.} & 4  \ditto{кварт.} & 12  \ditto{ф.} \ditto{ст.}\\
      D & 2\phantom{акр.} &  3  \ditto{ф.} \ditto{ст.} & 6  \ditto{ф.} \ditto{ст.}  & 8  \ditto{кварт.} & 6  \ditto{кварт.} & 18  \ditto{ф.} \ditto{ст.}\\
     \cmidrule(rl){1-1}
     \cmidrule(rl){2-2}
     \cmidrule(rl){4-4}
     \cmidrule(rl){5-5}
     \cmidrule(rl){6-6}
     \cmidrule(rl){7-7}
     Сума & 12 акр. &                 & 36\pound{ ф. ст.}  & 26 кварт.        & 14  кварт.         & 42\pound{ ф. ст.} \\
  \end{tabular}
  \end{center}
\end{table}

І, нарешті, коли маємо неоднакове поширення продукції і оброблюваної площі в чотирьох розрядах:
(див. табл. Iс).

Насамперед, в усіх цих випадках І, Іа, Іb, Іс рента з одного акра лишається та сама; бо в
дійсності продукт однакової маси капіталу на кожному
\parbreak{}  %% абзац продовжується на наступній сторінці
