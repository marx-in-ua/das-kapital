\parcont{}  %% абзац починається на попередній сторінці
\index{iii2}{0084}  %% посилання на сторінку оригінального видання
«і тому цією операцією ви мусити порушити вексельний курс, бо закордонний
борг не оплачено в наслідок того, що ваш експорт не має відповідного імпорту.
— Це правило для всіх країн взагалі».

Лекція Вілсона сходить на те, що всякий експорт без відповідного імпорту
становить одночасно імпорт без відповідного експорту; бо в продукцію товарів,
що їх експортують, ввіходять чужоземні, отже, імпортовані товари. Перед
нами припущення, що всякий такий експорт ґрунтується на неоплаченому
імпорті або породжує його, — отже, породжує борг закордонові, або ґрунтується
на ньому. Це — помилкова річ, навіть, коли не вважати на ті дві обставини,
що 1)~Англія має даремний імпорт, не платячи за нього жодного еквівалента;
напр., частину свого індійського імпорту. Індійський імпорт вона може обмінювати
на американський імпорт, експортуючи останній без еквівалентного імпорту;
щож до вартости, то в усякім разі Англія експортувала тільки те, що їй нічого
не коштувало; 2)~Англія може й оплатила імпорт, напр., американський, що
утворює додатковий капітал; коли вона той імпорт споживає непродуктивно,
напр., на військові припаси, то це не утворює боргу проти Америки та не
впливає на вексельний курс з Америкою. Newmarch суперечить сам собі в
посвідченнях 1934 та 1935, й Wood звертає його увагу на це в 1938: «Коли
жодна частина товарів, ужитих на виготовлення речей, що їх ми вивозимо без
зворотного припливу» [військові видатки] «не походить з тієї країни, куди ці
речі експортуються, то яким способом це впливатиме на вексельний курс з цією
країною? Нехай торговля з Турцією перебуває у звичайному стані рівноваги;
яким способом вивіз військових припасів до Криму вплине на вексельний курс
між Англією та Турцією?» — Тут Newmarch втрачає свою рівновагу, забуваючи,
що саме на це просте питання він дав уже слушну відповідь під № 1934, він
каже: «Ми вже, мені здасться, вичерпали практичне питання, а тепер увіходимо
в дуже високу ділянку метафізичної дискусії».

[Вілсон має ще й інше формулювання того свого твердження, що на вексельний
курс впливає всяке перенесення капіталу з однієї країни до іншої, однаково,
чи відбувається воно у формі благородного металу, чи у формі товарів.
Вілсон, природно, знає, що на вексельний курс впливає рівень проценту, а
саме, відношення чинних норм проценту в тих двох країнах, що їхній взаємний
вексельний курс розглядається. Отже, коли він буде в стані довести, що надмір
капіталу взагалі, отже, передусім надмір товарів всякого роду, в тім і благородного
металу, має разом з іншими обставинами вплив на рівень проценту, визначаючи
його, то він буде уже на крок ближче до своєї мети; перенесення
значної частини цього капіталу з однієї країни до іншої мусить змінити рівень
проценту в обох країнах, і то саме в протилежному напрямку а тому другою
чергою мусить воно змінити й вексельний курс між обома країнами. — \emph{Ф.~Е.}].

В Economist’і, що його він тоді редаґував, за рік 1847, на стор. 475,
він пише:

1)~«Очевидно, що такий надмір капіталу, який виявляється у великих запасах
всякого роду, в тім і благородного металу, неминуче мусить привести не тільки
до низьких цін на товари взагалі, але й до нижчого рівня проценту за ужиток
капіталу.

2)~Коли ми маємо запас товарів, достатній для того, щоб обслужити
потреби країни протягом двох наступних років, то порядкування цими
товарами протягом даного періоду можна здобути за далеко нижчу норму, ніж
тоді, коли того запасу вистачить ледви чи на два місяці.

3)~Всякі позики грошей,
хоч і в якій формі їх робитиметься, являють лише передачу порядкування над
товарами від однієї особи до іншої. Тому, коли товарів є понад міру, грошовий
процент мусить бути низький, а коли товарів обмаль, він мусить бути високий.
