Таблиця X

Рід землі
Акри
Капіталовкладення  Ф. ст.
Зиск Ф. ст.
Ціна продукції Ф. ст.
Продукт в квартерах
Продажна ціна Ф. ст.
Здобуток Ф. ст.
Рента
Збіжж. Кварт.
Грош. Ф. ст.
Норма ренти

A 1 2 1/2 + 2 1/2 = 5  1  6  1 + 1/4 =1 1/4     4 4/5   6     0         0     0
B 1 2 1/2 + 2 1/2 = 5  1  6  2 + 1/2 = 2 1/2    4 4/5   12  1 1/4   6     120\%
C 1 2 1/2 + 2 1/2 = 5  1  6  3 + 3/4 =3  3/4    4 4/5   18  2 1/2   12   240\%
D 1 2 1/2 + 2 1/2 = 5  1  6  4 + 1 = 5              4 4/5    24  3 3/4   18  360\%
                           20     24                  12 1/2                  60  7 1/2   36  240\%

В цій таблиці загальний здобуток, сума грошової ренти і норма ренти
теж лишаються такі самі, як у таблицях, II, VII і VIII, бо продукт і продажна
ціна знов таки змінились у зворотному відношенні, а капіталовкладення лишилось
те саме.

Але як стоїть справа в іншому випадку, можливому за висхідної ціни
продукції, а саме в тому випадку, коли гірша земля, яку до цього часу не
варто було обробляти, тепер починає оброблятись.

Припустімо, що така земля, яку ми позначимо а, вступає в конкуренцію.
Тоді земля А, що не давала до цього часу ренти, почала б давати ренту, і
вищенаведені таблиці VIII, VIII і X набули б такого вигляду:

Таблиця VIIa

Рід землі
Акри
Капітал Ф.ст.
Зиск Ф.ст.
Ціна продукції Ф.ст.
Продукт в квартерах
Продажна ціна Ф.ст.
Здобуток Ф.ст.
Рента
Кварт.
Ф.ст.
Підвищення

а 1               5          1 6                           1 1/2      4  6     0           0    0
A 1  2 1/2 + 2 1/2  1  6   1/2 + 1 1/4 = 1 3/4      4  7    1/4        1     1
B 1  2 1/2 + 2 1/2  1  6   1 + 2 1/2 = 3 1/2          4  14  2           8     1+7
C 1  2 1/2 + 2 1/2  1  6   1 1/2 + 3 3/4 = 5 1/4   4  21  3 3/4    15   1+2×7
D 1  2 1/2 + 2 1/2  1  6   2 + 5 = 7                       4  28  5 1/2    22   1+3×7
                                30              19                             76  11 1/2  46

Таблиця VIIIa

Рід землі
Акри
Капітал Ф. ст.
Зиск Ф. ст.
Ціна продукції Ф. ст.
Продукт в квартерах
Продажна ціна Ф. ст.
Здобуток Ф. ст.
Рента Кварт. Ф. ст.
Підвищення

а  1                      5   1  6                     1 1/4     4 4/5   6            0            0
           0
A  1  2 1/2 + 2 1/2  1  6   1/2  + 1  = 1 1/2     4 4/5   7 1/5     1/4         1 1/5     1 1/5
B  1  2 1/2 + 2 1/2  1  6    1 + 2  = 3                4 4/5   14 2/5   1 3/4     8 2/5      1 1/5 +
7 1/5
C  1  2 1/2 + 2 1/2  1  6    1 1/2 + 3 = 4 1/2   4 4/5   21 3/5    2 1/4    15 3/5    1 1/5 + 2 × 7
1/5
D  1  2 1/2 + 2 1/2  1  6    2 + 4 = 6                4 4/5    28 4/5    4 3/4    22 4/5   1 1/5 + 3
× 7 1/5
     5                              30              16 1/4                     78           9       
   48
