
\index{iii2}{0169}  %% посилання на сторінку оригінального видання
Варіянт III: Висхідна продуктивність другої витрати (таблиця XXI); це знов
таки зумовлює низхідну продуктивність першої витрати.

Друга видозміна: земля гіршої якости, (позначувана: літерою а)
вступає в конкуренцію; земля $А$ дає ренту.

Варіянт 1: Незмінна продуктивність другої витрати (таблиця XXII).

Варіант 2: Низхідна продуктивність (таблиця XXIII).

Варіант 3: Висхідна продуктивність (таблиця XXIV).

Ці три варіянти відповідають загальним умовам проблеми і не дають
приводу до будь-яких зауважень.

Тепер ми наведемо таблиці:

\begin{table}[H]
  \begin{center}
    \emph{Таблиця XI}
    \footnotesize

  \begin{tabular}{c@{  } c@{  } c@{  } c@{  } c@{  } c@{  } c}
    \toprule
      \multirowcell{2}{\makecell{Рід\\ землі}} &
      Ціна продукції &
      Продукт &
      \makecell{Продажна \\ ціна} &
      \makecell{Здо-\\буток} &
      Рента &
      \multirowcell{2}{Підвищення ренти} \\

      \cmidrule(r){2-2}
      \cmidrule(r){3-3}
      \cmidrule(r){4-4}
      \cmidrule(r){5-5}
      \cmidrule(r){6-6}

       & Шил. & Бушелі & Шил. & Шил. & Шил. & &   \\
      \midrule
      A & 60 & 10 & 6 & 60  & \phantom{00}0 & \phantom{00 × 0}0 \\
      B & 60 & 12 & 6 & 72  & \phantom{0}12 & \phantom{01 × }12 \\
      C & 60 & 14 & 6 & 84  & \phantom{0}24 & \phantom{0}2 × 12           \\
      D & 60 & 16 & 6 & 96  & \phantom{0}36 & \phantom{0}3 × 12           \\
      E & 60 & 18 & 6 & 108 & \phantom{0}48 & \phantom{0}4 × 12           \\

     \cmidrule(r){6-6}
     \cmidrule(r){7-7}

      & & & & & 120 & 10 × 12 \\
  \end{tabular}

  \end{center}
\end{table}

За другої витрати капіталу на тій самій землі.

Перший випадок: за незмінної ціни продукції.

Варіянт 1: за незмінної продуктивности другої витрати капіталу.

\begin{table}[H]
  \begin{center}
    \emph{Таблиця XII}
    \footnotesize

  \begin{tabular}{c@{  } c@{  } c@{  } c@{  } c@{  } c@{  } c}
    \toprule
      \multirowcell{2}{\makecell{Рід\\ землі}} &
      Ціна продукції &
      Продукт &
      \makecell{Продажна \\ ціна} &
      \makecell{Здо-\\буток} &
      Рента &
      \multirowcell{2}{Підвищення ренти} \\

      \cmidrule(r){2-2}
      \cmidrule(r){3-3}
      \cmidrule(r){4-4}
      \cmidrule(r){5-5}
      \cmidrule(r){6-6}

       & Шил. & Бушелі & Шил. & Шил. & Шил. & &   \\
      \midrule
      A & 60 \dplus{} 60 \deq{} 120 & 10 \dplus{} 10 \deq{} 20 & 6 & 120  & \phantom{00}0 & \phantom{00 × 0}0 \\
      B & 60 \dplus{} 60 \deq{} 120 & 12 \dplus{} 12 \deq{} 24 & 6 & 144  & \phantom{0}24 & \phantom{01 × }24 \\
      C & 60 \dplus{} 60 \deq{} 120 & 14 \dplus{} 14 \deq{} 28 & 6 & 168  & \phantom{0}48 & \phantom{0}2 × 24 \\
      D & 60 \dplus{} 60 \deq{} 120 & 16 \dplus{} 16 \deq{} 32 & 6 & 192  & \phantom{0}72 & \phantom{0}3 × 24 \\
      E & 60 \dplus{} 60 \deq{} 120 & 18 \dplus{} 18 \deq{} 36 & 6 & 216  & \phantom{0}96 & \phantom{0}4 × 24 \\

     \cmidrule(r){6-6}
     \cmidrule(r){7-7}

      & & & & & 240 & 10 × 24 \\
  \end{tabular}

  \end{center}
\end{table}

Варіянт 2: за низхідної продуктивности другої витрати капіталу: на землі
$А$ не зроблено другої витрати.

1)~Коли земля $В$ стає землею, що не дає ренти.
