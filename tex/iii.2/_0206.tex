\parcont{}  %% абзац починається на попередній сторінці
\index{iii2}{0206}  %% посилання на сторінку оригінального видання
й розвиток основного капіталу, який або долучається до землі, або укорінюється
в ній, спирається на ній, як усі промислові будівлі, залізниці, товарові склепи,
фабричні будівлі, доки тощо. Сплутування комірної плати, оскільки вона становить
процент та амортизацію капіталу, вкладеного в будинок, з рентою просто за
землю, неможливе тут навіть при всій добрій волі, як у Кері, особливо тоді
коли, як в Англії, земельний власник і будівельний спекулянт є цілком різні
особи. Тут береться на увагу два елементи: з одного боку, експлуатація землі
з метою репродукції або видобувної промисловости; з другого, — простір, який
потрібен як елемент усякої продукції й усякої людської діяльности. І за те і за
друге земельна власність вимагає своєї дані. Попит на будівельні дільниці підвищує
вартість землі як простору і основи, тимчасом як через це і разом
з цим зростає попит на елементи землі, що правлять за будівельний матеріял\footnote{
«Забрукування лондонських вулиць дало можливість власникам деяких голих скель на шотляндському
березі здобувати ренту з абсолютно некорисного до того часу кам’янистого ґрунту». A. Smith, Book I,
chap. XI. 2.
}.

У книзі II, розд. XII в свідченнях Едварда Каппа, великого лондонського
будівельного спекулянта, перед банковою комісією 1857 року, ми бачили приклад
того, яким чином в швидко ростучих містах, особливо коли будівлю провадиться,
як у Лондоні, фабричним способом, за головний об’єкт будівельної
спекуляції є власне не будинок, а земельна рента. Він говорить там № 5435:

«Я вважаю, що людина, яка бажає поступувати на світі, навряд чи може
розраховувати, що вона буде поступувати, провадячи тільки солідну справу (fair
trade)\dots{} вона неминуче мусить, крім того, будувати з метою спекуляції і до
того ж у великому маштабі; бо підприємець здобуває дуже мало зиску з самих
будівель, свій головний зиск він здобуває з підвищених земельних рент. Припустімо,
що він орендує дільницю землі і виплачує за неї 300 ф. стерл. на рік;
коли він, дбало опрацювавши будівельний плян, збудує на цій дільниці будинки
належного розряду, то йому може пощастити здобути за це 400 або 450 ф. стерл.
на рік, і його зиск у багато більшій мірі був би у збільшеній земельній ренті
на 100 або 150 ф. стерл. на рік, ніж у зиску від будівель, який він у багатьох
випадках взагалі навряд чи бере на увагу». До того не слід забувати, що
по закінченні договору про винаймання, який найчастіше складається на 99 років,
земля з усіма будівлями на ній і з земельною рентою, яка за цей час
здебільша підвищується більше, ніж у два-три рази, знову повертається від
будівельного спекулянта або його правонаступника до первісного останнього
земельного власника.

Власне рента з копалень визначається цілком так само, як хліборобська
рента. «Бувають такі копальні, що їхній продукт навряд чи достатній для того,
щоб оплатити працю і покрити вкладений туди капітал разом з звичайним
зиском. Вони дають деякий зиск підприємцеві, але жодної ренти для земельного
власника. Їх міг би з вигодою обробляти тільки земельний власник, який сам,
бувши підприємцем, здобуває звичайний зиск на свій вкладений капітал. Багато
вугільних шахт у Шотландії розробляються в такий спосіб, і не могли б
розроблятися якось інакше. Земельний власник нікому не дозволяє розробляти
їх, коли йому не виплачують ренти, але ніхто не може виплачувати за них
ренти». (A. Smith, Book I, chap. XI, 2).

Треба відрізняти, чи випливає рента з монопольної ціни тому, що існує
незалежна від неї монопольна ціна продуктів або самої землі, або ж чи продаються
продукти по монопольній ціні тому, що існує рента. Коли ми говоримо
про монопольну ціну, ми взагалі маємо на думці таку ціну, яка визначається
тільки прагненням купити і платоспроможністю покупців, незалежно так від
тієї ціни, що визначається загальною ціною продукції, як і від тієї, що визначається
вартістю продукту. Виноградник, що продукує вино цілком виключної
\parbreak{}  %% абзац продовжується на наступній сторінці
