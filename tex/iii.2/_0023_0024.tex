\parcont{}  %% абзац починається на попередній сторінці
\index{iii2}{0023}  %% посилання на сторінку оригінального видання
час у позичковий капітал, — де збільшення так само мало свідчить про зростання
продуктивного капіталу, як і зріст вкладів в акційних банках Лондону, скоро
ці останні почали платити проценти на вклади. Поки розмір продукції лишається
той самий, це збільшення викликав тільки багатість позичкового грошового
капіталу проти капіталу продуктивного. Відси низький рівень проценту.

Коли процес репродукції знову досягає стану розцвіту, що йде попереду надмірного
напруження, то комерційний кредит досягає дуже великого поширу, що в такі
часи дійсно знову має «здорову» базу легкого зворотного припливу капіталів та поширу
продукції. При такому стані справ рівень проценту все ще низький, хоч він і
підноситься понад свій мінімум. Це — справді \emph{єдиний} момент, коли можна сказати,
що низький рівень проценту, а тому й відносна багатість позичкового капіталу
збігаються з дійсним поширом промислового капіталу. Легкість та реґулярність
зворотного припливу капіталів, сполучені з поширом комерційного кредиту,
забезпечують подання позичкового капіталу, дарма що попит збільшився, та
заважають зростові рівня проценту. З другого боку, тільки тепер з’являються
помітно лицарі, що працюють без запасного капіталу або взагалі без капіталу
й тому провадять свої операції цілком на основі грошового кредиту. Сюди
долучається тепер ще й значний пошир основного капіталу в усіх формах та
масове відкриття нових широко організованих підприємств. Тепер процент підноситься
до своєї пересічної висоти. Свого максимуму він досягає знову тоді, коли вибухає
нова криза, раптом спиняється кредит, припиняються платежі, паралізується
процес репродукції та, за згаданими раніше винятками, поряд майже абсолютного
браку позичкового капіталу настає надмір бездіяльного промислового капіталу.

Отже, рух позичкового капіталу, оскільки він виявляється в рівні проценту,
відбувається взагалі в напрямку, протилежному рухові промислового капіталу.
Фаза, коли низький рівень проценту, що однак є вищий за мінімум, збігається
з «поліпшенням» та з чим раз більшим — по кризі — довір’ям, і особливо фаза,
коли він досягає своєї пересічної висоти, середини, однаково віддаленої від його
мінімуму та максимуму, — тільки ці два моменти виявляють збіг багатости позичкового
капіталу зі значним поширом промислового капіталу. Але на початку
промислового циклу низький рівень проценту збігається зі зменшенням, а наприкінці
циклу високий рівень проценту з надміром промислового капіталу. Низький
рівень проценту, що супроводить «поліпшення», виявляє те, що комерційний
кредит тільки в незначній мірі потребує банкового кредиту, бо він ще стоїть
на своїх власних ногах.

Цей промисловий цикл має ту особливість, що після того, як уже дано перший
поштовх, той самий кругооборот мусить періодично репродукуватися\footnote{
[Як я вже зауважував в іншому місці, від часу останньої великої загальної кризи тут настала
зміна. Гостра форма періодичного процесу з її дотеперішнім десятирічним циклом, здається, відступила
місце більш хронічному, довгочасному чергуванню, що поширюється на різні індустріяльні країни в
різні
часи, чергуванню порівняно короткого та млявого поліпшення справ, та порівняно, довгого
нерозв'язного
пригнічення. Однак може й таке бути, що тут маємо ми лише збільшення часу тривання циклу. За
дитинства
світової торговлі, в роки 1815--47, можна виявити кризи, що повторювались приблизно через
кожні п’ять років; в роки 1847--67 цикл є виразно десятирічний; чи не перебуваємо ми в періоді
зародження нового світового краху нечуваної сили? Дещо, здається, вказує на це. Від часу останньої
загальної кризи 1867 року настали великі зміни. Колосальний пошир засобів комунікації — океанські
пароплави,
залізниці, електричні телеграфи, Суецький канал — уперше дійсно утворив світовий ринок. Побіч
Англії, що раніш монополізувала промисловість, постав ряд промислових країн — конкурентів; для
приміщення надмірного европейського капіталу одкрилися по всіх частинах світу безмежно більші та
різноманітніші ділянки, так що капітал розподіляється далеко більше, а місцеву надмірну спекуляцію
легше перемогти. В наслідок цього всього більшість старих огнищ криз і нагод до утворення криз
усунено
або дуже зменшено. Поряд цього конкуренція на внутрішньому ринку відступає перед картелями та
трестами, тимчасом коли на зовнішньому ринку її обмежує охоронне мито, що ним оточили себе всі
великі
промислові країни, крім Англії. Але це саме охоронне мито становить не що інше, а тільки озброєння
для остаточної загальної промислової війни, що має вирішити справу панування на світовому ринку.
Отак кожен з тих елементів, що діє проти повторювання колишніх криз, ховає в собі зародок далеко
більш могутньої майбутньої кризи. — Ф.~Е].
}. В стані
підупаду продукція знижується нижче від того щабля, що його вона досягла
за попереднього циклу, і що для нього тепер покладено технічну базу. В періоді
розцвіту — середньому періоді — продукція розвивається на цій базі далі. В період
надмірної продукції та спекуляції продукція напружує продуктивні сили до найвищої
точки, аж поза капіталістичні межі продукційного процесу.

\index{iii2}{0024}  %% посилання на сторінку оригінального видання
Що в період кризи бракує платіжних засобів, це очевидно само собою.
Перетворність\footnote*{
Нім. Konvertibilität, властивість і можливість перетворюватись, переходити
з одного стану в інший. \Red{Пр.~Ред.}
} векселів у гроші заступила місце метаморфози самих товарів,
і то саме за таких часів то більше, що більше частина торговельних фірм працює
тільки на кредит. Невігласне та недоладне банкове законодавство як от
років 1844--45, може зробити цю грошову кризу тяжчою. Але жодне банкове
законодавство не в стані тієї кризи усунути.

При такій системі продукції, коли всі взаємозв’язки процесу репродукції спирається
на кредиті, то якщо кредит раптом припинено та має силу лише платіж готівкою,
очевидно, мусить наступати криза, надзвичайно велика гонитва за платіжними
засобами. Тому на перший погляд вся криза здається лише кредитовою кризою
та грошовою кризою. І справді, справа — лише в перетворності векселів у гроші.
Але ці векселі представляють, здебільша, дійсні купівлі та продажі, що їхній пошир,
який значно пересягає межі суспільної потреби, кінець-кінцем, лежить в основі
всієї кризи. Однак поряд цього величезна маса цих векселів представляє лише
шахрайські операції, що виходячи тепер на денне світло, врахують; далі — спекуляції,
ведені та ще й нещасливо на чужий капітал; насамкінець — товарові
капітали, що знецінилися або й зовсім не можуть бути продані; або зворотний приплив
капіталів, що фактично вже ніколи не відбудеться. Всю цю штучну систему
ґвалтовного поширу процесу продукції не можна, звичайно, вилікувати тим, що
якийсь банк, напр., Англійський банк, дасть — у своїх паперах — усім спекулянтам
капітал, що його їм бракує, та купить усі знецінені товари за їхні старі
номінальні вартості. Проте все тут виявляється перекручено, бо в цьому паперовому
світі ніде не виступає реальна ціна та її реальні моменти, а тільки
зливки, металеві гроші, банкноти, векселі, цінні папери. Це перекручення
виявляється особливо в центрах, де, як от в Лондоні, зосереджено всі грошові
підприємства країни; весь процес стає незрозумілим; вже менше помічається це
в центрах продукції.

Проте, з приводу того надміру промислового капіталу, що виявляється підчас
криз, треба зауважити ось що: товаровий капітал сам про себе є одночасно
грошовий капітал, тобто певна сума вартости, висловлена в ціні товару. Як споживча
вартість, є він певна кількість певних речей споживання, що їх підчас
кризи є понад міру. Але як грошовий капітал сам про себе, як потенціяльний
грошовий капітал, він зазнає повсякчас поширу та скорочення. Напередодні кризи та
протягом її товаровий капітал скорочується в своїй властивості, як потенціяльний
грошовий капітал. Для своїх державців та їхніх кредиторів (а так само як і забезпечення
для векселів та позик) він становить менше грошового капіталу, ніж тоді,
коли його скуповували та коли робилось засновані на ньому дисконтові та заставні
операції. Коли такий має бути зміст твердження, що грошовий капітал певної
країни підчас скрути меншає, то є це тотожне з тим, що ціни товарів спали.
Однак такий крах цін тільки вирівнює їхнє колишнє набубнявіння (Aufblähung).

Доходи непродуктивних кляс та тих, що живуть з сталих доходів, лишаються,
здебільша, незмінні підчас такого набубнявіння цін (Preisaufblähung), що
розвивається поряд надмірної продукції та надмірної спекуляції. Тому їхня спожиткова
спроможність відносно меншає, а разом з тим меншає й їхня здібність повертати
\parbreak{}  %% абзац продовжується на наступній сторінці
