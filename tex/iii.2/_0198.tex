\parcont{}  %% абзац починається на попередній сторінці
\index{iii2}{0198}  %% посилання на сторінку оригінального видання
передумовою перетворення вартостей на ціни продукції і загальну норму зиску. Проте, як з’ясовано
давніше, ця передумова ґрунтується на раз-у-раз змінюваному пропорційному розподілі всього
суспільного капіталу між різними сферами продукції, на невпинній іміграції і еміграції капіталів, на
можливості переносити їх з одної сфери в іншу, коротко, на вільному русі їх між цими різними сферами
продукції, як між відповідними вільними
царинами для приміщення самостійних частин усього суспільного капіталу. При цьому припускається, що
жодне — за винятком хіба лише випадкового і тимчасового — обмеження не перешкоджає конкуренції
капіталів зводити вартість до розміру ціни продукції, напр., у такій сфері продукції, в якій
вартість товарів вища від їхньої ціни продукції, або в якій створена додаткова вартість більша, ніж
пересічний зиск, зводити тут вартість до розмірів ціни продукції і тим самим розподіляти надмірну
додаткову вартість цієї сфери продукції пропорційно між усіма сферами,
що їх експлуатує капітал. Але коли настає протилежне цьому, коли капітал
наражається на чужу силу, яку він зовсім не може подолати, або може подолати
лише почасти, і яка обмежує його вкладення в окремих сферах продукції,
допускає його лише на умовах, що цілком або почасти виключають
згадане загальне вирівнювання додаткової вартості на пересічний зиск, то очевидно, що в таких сферах
продукції через надмір товарової вартости над ціною продукції їхніх товарів постав би надзиск, який
міг би перетворитися на ренту, а вона як така, могла б усамостійнитися проти зиску. Але як така чужа
сила і обмеження при приміщенні у землю капіталові протистоїть земельна власність, або капіталістові
— земельний власник.

Земельна власність є тут за бар’єр, що не дозволяє вкладати нових капіталів
у необроблену до того часу або не здану в оренду землю, не стягуючи
при цьому мита, тобто не вимагаючи ренти, хоч ця новопритягнена до обробітку
земля належить до такого роду, який не дає диференційної ренти, і який, коли б не існувало земельної
власності, міг би оброблятися вже при такому незначному підвищенні ринкової ціни, коли реґуляційна
ринкова ціна виплачувала б обробникові цієї найгіршої землі лише ціну продукції. Проте, в наслідок
межі, що її ставить земельна власність, ринкова ціна мусить підвищитись до такого пункту, коли земля
може виплачувати надмірну понад ціну продукції, тобто ренту. А що, згідно з припущенням, вартість
товарів, продукованих хліборобським капіталом, вища від їхньої ціни продукції, то ця рента (за
винятком того випадку, який буде зараз досліджено), становить надмір вартости над ціною продукції,
або частину цього надміру. Чи рівна рента всій різниці між вартістю і ціною продукції, чи тільки
більшій або меншій частині цієї різниці, це цілком залежить від стану попиту та подання і від
розміру площі, новопритягненої до обробітку. Доки рента не дорівнює надмірної вартости хліборобських
продуктів над їхньою ціною продукції, частина цього надміру завжди братиме участь у загальному
вирівнюванні і пропорційному розподілі всієї додаткової вартости між різними поодинокими капіталами.
Скоро б рента стала рівною надмірові вартости над ціною продукції, то вся ця частина додаткової
вартости, що становить залишок над пересічним зиском, була б відтягнена від цього вирівнювання. Але
чи дорівнює ця абсолютна рента всьому надмірові вартости над ціною продукції, чи дорівнює лише
частині його, все
одно, хліборобські продукти продавалося б по монопольній ціні не тому, що
їхня ціна вища, ніж їхня вартість, а тому, що вона дорівнює їхній вартості,
або тому, що вона нижча, ніж їхня вартість, але вища, ніж їхня ціна продукції, їхнє монопольне
становище було б у тому, що вони у
протилежність іншим промисловим продуктам, вартість яких вища від загальної ціни продукції, не
нівелювались би на ціну продукції. А що частина вартости, як і ціни продукції, є фактично дана стала
величина, а саме, витрати продукції, зужиткований у
\parbreak{}  %% абзац продовжується на наступній сторінці
