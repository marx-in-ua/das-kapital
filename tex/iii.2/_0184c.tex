\parcont{}  %% абзац починається на попередній сторінці
\index{iii2}{0184}  %% посилання на сторінку оригінального видання
але ціну продукції в 3\sfrac{1}{2}\pound{ ф. стерл.} реґулювала б не найгірша земля $А$, а краща
земля $В$. Звичайно, при цьому припускається, що нова земля якости $А$ такого
ж зручного положення, як оброблювана до цього часу, є неприступна, і що довелося
би зробити другу витрату капіталу на вже оброблюваній дільниці $А$, але
з більшою ціною продукції, або довелось би притягнути до обробітку ще гіршу
землю $А_{-1}$. Коли в наслідок послідовних витрат капіталу починає діяти диференційна
рента II, то може статися, що межі підвищуваної ціни продукції реґулюватимуться
кращою землею, і гірша земля, база диференційної ренти І, тоді
теж може давати ренту. Таким чином, при наявності самої лише диференційної
ренти всі оброблювані землі почали б тоді давати ренту. Ми мали б тоді такі
дві таблиці, в яких під ціною продукції розуміється суму авансованого капіталу
плюс 20\%  зиску, отже на кожні 2\sfrac{1}{2}\pound{ ф. стерл.} капіталу по \sfrac{1}{2}\pound{ ф. стерл.}
зиску, разом 3\pound{ ф. стерл.}.
% (див. табл. І).

\vspace{-\medskipamount}
\begin{table}[H]
  \centering
  \footnotesize

  \settowidth\rotheadsize{\theadfont Продажна}
  \begin{tabular}{l c c c c c c c}
    \toprule
      \thead[tl]{Рід\\землі} &
      &
      \rothead{Ціна\\продукції} &
      \rothead{Продукт} & % \\ в кварт.}}}
      \rothead{Продажна\\ціна} &
      \rothead{Грошовий\\ціна} &
      \rothead{Збіжжева\\рента} &
      \rothead{Грошова\\рента} \\

      \cmidrule(rl){2-8}

       & акри &  \poundsign{} & кв. & \poundsign{} & \poundsign{} & \poundsign{} & \poundsign{} \\
      \midrule

      A & 1 &  \phantom{0}3 & \phantom{0}1\phantom{\tbfrac{1}{2}} & 3 & \phantom{0}3\phantom{\tbfrac{1}{2}} & \phantom{0}0\phantom{\tbfrac{1}{2}} & \phantom{0}0\phantom{\tbfrac{1}{2}} \\
      B & 1 &  \phantom{0}6 & \phantom{0}3\tbfrac{1}{2}           & 3 & 10\tbfrac{1}{2}                     & \phantom{0}1\tbfrac{1}{2}           & \phantom{0}4\tbfrac{1}{2} \\
      C & 1 &  \phantom{0}6 & \phantom{0}5\tbfrac{1}{2}           & 3 & 16\tbfrac{1}{2}                     & \phantom{0}3\tbfrac{1}{2}           & 10\tbfrac{1}{2} \\
      D & 1 &  \phantom{0}6 & \phantom{0}7\tbfrac{1}{2}           & 3 & 22\tbfrac{1}{2}                     & \phantom{0}5\tbfrac{1}{2}           & 16\tbfrac{1}{2} \\

     \midrule
     Разом & 4 & 21 & 17\tbfrac{1}{2} & & 52\tbfrac{1}{2} & 10\tbfrac{1}{2} & 31\tbfrac{1}{2} \\
  \end{tabular}
\end{table}
\vspace{-\medskipamount}

Таке становище речей перед новою витратою капіталу в
3\sfrac{1}{2}\pound{ ф. стерл.} на $В$, що дає тільки 1 квартер. Після цієї витрати
капіталу справа стоїть так:
%(див. табл. II).

\vspace{-\medskipamount}
\begin{table}[H]
  \centering
  \footnotesize

  \settowidth\rotheadsize{\theadfont Продажна}
  \begin{tabular}{l c c c c c c c}
    \toprule
      \thead[tl]{Рід\\землі} &
      &
      \rothead{Ціна\\продукції} &
      \rothead{Продукт} & % \\ в кварт.}}}
      \rothead{Продажна\\ціна} &
      \rothead{Грошовий\\ціна} &
      \rothead{Збіжжева\\рента} &
      \rothead{Грошова\\рента} \\

      \cmidrule(rl){2-8}

       & акри &  \poundsign{} & кв. & \poundsign{} & \poundsign{} & \poundsign{} & \poundsign{} \\
      \midrule

      A & 1 &  \phantom{0}3\phantom{\tbfrac{1}{2}} & \phantom{0}1\phantom{\tbfrac{1}{2}} & 3\tbfrac{1}{2} & \phantom{0}3\tbfrac{1}{2} & \phantom{00}\tbfrac{1}{7}   & \phantom{00}\tbfrac{1}{2} \\
      B & 1 &  \phantom{0}9\tbfrac{1}{2}           & \phantom{0}4\tbfrac{1}{2}           & 3\tbfrac{1}{2} & 15\tbfrac{3}{4}           & \phantom{0}1\tbfrac{11}{14} & \phantom{0}6\tbfrac{1}{4} \\
      C & 1 &  \phantom{0}6\phantom{\tbfrac{1}{2}} & \phantom{0}5\tbfrac{1}{2}           & 3\tbfrac{1}{2} & 19\tbfrac{1}{4}           & \phantom{0}3\tbfrac{11}{14} & 13\tbfrac{1}{4} \\
      D & 1 &  \phantom{0}6\phantom{\tbfrac{1}{2}} & \phantom{0}7\tbfrac{1}{2}           & 3\tbfrac{1}{2} & 26\tbfrac{1}{4}           & \phantom{0}5\tbfrac{11}{14} & 20\tbfrac{1}{4}           \\

     \midrule

     Разом & 4 & 24\tbfrac{1}{2} & 18\tbfrac{1}{2} & & 64\tbfrac{3}{4} & 11\tbfrac{1}{2} & 40\tbfrac{1}{4} \\
  \end{tabular}
\end{table}
\vspace{-\medskipamount}

[Це знов не зовсім вірно обчислено. Орендареві $В$ продукція цих 4\sfrac{1}{2} квартерів
коштує, поперше, 9\sfrac{1}{2}\pound{ ф. стерл.} ціни продукції і, подруге,
4\sfrac{1}{2}\pound{ ф. стерл.} ренти, разом 14\pound{ ф. стерл.}; пересічно за квартер 3\sfrac{1}{9}\pound{ ф. стерл}.
Ця пересічна ціна всієї його продукції стає через це за реґуляційну ринкову ціну. Тому рента на
$А$ становила б \sfrac{1}{9}\pound{ ф. стерл.} замість \sfrac{1}{2}\pound{ ф. стерл.}, а рента на $В$ лишалася б, як і давніш, 4\sfrac{1}{2}\pound{ ф.
стерл.}: 4\sfrac{1}{2} квартери по 3\sfrac{1}{2}\pound{ ф. стерл.} \deq{} 14\pound{ ф. стерл.}, звідси вирахувати
9\sfrac{1}{2}\pound{ ф. стерл.} ціни продукції, лишається надзиск в 4\sfrac{1}{2}\pound{ ф. стерл}. Бачимо: не
зважаючи на змінені числа, приклад показує, як з допомогою диференційноі
ренти II краща земля, що вже дає ренту, може стати за реґуляційну щодо ціни,
і через це вся земля, також і та, що до того часу не давала ренти, може перетворитися
в рентодайну. — \emph{Ф.~Е.}].

Збіжжева рента мусить підвищитись, скоро підвищується реґуляційна
ціна продукції збіжжя, отже, скоро підвищується ціна продукції квартера збіжжя
на реґуляційній землі, або реґуляційна витрата капіталу на одному з родів
землі. Це все одно, як коли б усі роди землі стали менш плодючі і продукували
б, наприклад, на кожні 2\sfrac{1}{2}\pound{ ф. стерл.} нових витрат капіталу лише по \sfrac{5}{7}
квартера замість 1 квартера. Весь надмір збіжжя, що його вони продукують
\parbreak{}  %% абзац продовжується на наступній сторінці
