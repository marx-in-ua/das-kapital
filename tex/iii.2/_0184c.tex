\parcont{}  %% абзац починається на попередній сторінці
\index{iii2}{0184}  %% посилання на сторінку оригінального видання
але ціну продукції в 3\sfrac{1}{2}\pound{ ф. стерл.} реґулювала б не найгірша земля $А$, а краща
земля $В$. Звичайно, при цьому припускається, що нова земля якости $А$ такого
ж зручного положення, як оброблювана до цього часу, є неприступна, і що довелося
би зробити другу витрату капіталу на вже оброблюваній дільниці $А$, але
з більшою ціною продукції, або довелось би притягнути до обробітку ще гіршу
землю $А_{-1}$. Коли в наслідок послідовних витрат капіталу починає діяти днференційна
рента II, то може статися, що межі підвищуваної ціни продукції реґулюватимуться
кращою землею і гірша земля, база диференційної ренти І, тоді
теж може давати ренту. Таким чином, при наявності самої лише диференційної
ренти всі оброблювані землі почали б тоді давати ренту. Ми мали б тоді такі
дві таблиці, в яких під ціною продукції розуміється суму авансованого капіталу
плюс 20\%  зиску, отже на кожні 2\sfrac{1}{2}\pound{ ф. стерл.} капіталу по \sfrac{1}{2}\pound{ ф. стерл.}
зиску, разом 3\pound{ ф. стерл.} (див. табл. І).

\begin{table}[h]
  \begin{center}
    \footnotesize

  \begin{tabular}{c c c c c c c c}
    \toprule
      \multirowcell{2}{\makecell{Рід \\землі}} &
      \multirowcell{2}{\rotatebox[origin=c]{90}{Акри}} &
      \rotatebox[origin=c]{90}{\makecell{Ціна про- \\ дукції}} &
      \multirowcell{2}{\rotatebox[origin=c]{90}{\makecell{Продукт \\ в кварт.}}} &
      \rotatebox[origin=c]{90}{\makecell{Продажна \\ ціна}} &
      \rotatebox[origin=c]{90}{\makecell{Грошовий \\ здобуток}} &
      \rotatebox[origin=c]{90}{\makecell{Збіжжева \\ рента}} &
      \rotatebox[origin=c]{90}{\makecell{Грошова \\ рента}} \\

      \cmidrule(rl){3-3}
      \cmidrule(rl){5-5}
      \cmidrule(rl){6-6}
      \cmidrule(rl){7-7}
      \cmidrule(rl){8-8}

       &  &  ф. ст. & & ф. ст. & ф. ст. & ф. ст. & ф. ст.  \\
      \midrule

      A & 1 &  \phantom{0}3 & \phantom{0}1\phantom{\sfrac{1}{2}} & 3 & \phantom{0}3\phantom{\sfrac{1}{2}} & \phantom{0}0\phantom{\sfrac{1}{2}} & \phantom{0}0\phantom{\sfrac{1}{2}} \\
      B & 1 &  \phantom{0}6 & \phantom{0}3\sfrac{1}{2}           & 3 & 10\sfrac{1}{2}                     & \phantom{0}1\sfrac{1}{2}           & \phantom{0}4\sfrac{1}{2} \\
      C & 1 &  \phantom{0}6 & \phantom{0}5\sfrac{1}{2}           & 3 & 16\sfrac{1}{2}                     & \phantom{0}3\sfrac{1}{2}           & 10\sfrac{1}{2} \\
      D & 1 &  \phantom{0}6 & \phantom{0}7\sfrac{1}{2}           & 3 & 22\sfrac{1}{2}                     & \phantom{0}5\sfrac{1}{2}           & 16\sfrac{1}{2}           \\

     \cmidrule(rl){1-1}
     \cmidrule(rl){2-2}
     \cmidrule(rl){3-3}
     \cmidrule(rl){4-4}
     \cmidrule(rl){6-6}
     \cmidrule(rl){7-7}
     \cmidrule(rl){8-8}

     Разом & 4 & 21 & 17\sfrac{1}{2} & & 52\sfrac{1}{2} & 10\sfrac{1}{2} & 31\sfrac{1}{2} \\
  \end{tabular}

  \end{center}
\end{table}

Таке становище речей перед новою витратою капіталу в
3\sfrac{1}{2}\pound{ ф. стерл.} на $В$, що дає тільки 1 квартер. Після цієї витрати
капіталу справа стоїть так: (див. табл. II).

\begin{table}[h]
  \begin{center}
    \footnotesize

  \begin{tabular}{c c c c c c c c}
    \toprule
      \multirowcell{2}{\makecell{Рід \\землі}} &
      \multirowcell{2}{\rotatebox[origin=c]{90}{Акри}} &
      \rotatebox[origin=c]{90}{\makecell{Ціна про- \\ дукції}} &
      \multirowcell{2}{\rotatebox[origin=c]{90}{\makecell{Продукт \\ в кварт.}}} &
      \rotatebox[origin=c]{90}{\makecell{Продажна \\ ціна}} &
      \rotatebox[origin=c]{90}{\makecell{Грошовий \\ здобуток}} &
      \rotatebox[origin=c]{90}{\makecell{Збіжжева \\ рента}} &
      \rotatebox[origin=c]{90}{\makecell{Грошова \\ рента}} \\

      \cmidrule(rl){3-3}
      \cmidrule(rl){5-5}
      \cmidrule(rl){6-6}
      \cmidrule(rl){7-7}
      \cmidrule(rl){8-8}

       &  &  ф. ст. & & ф. ст. & ф. ст. & ф. ст. & ф. ст.  \\
      \midrule

      A & 1 &  \phantom{0}3\phantom{\sfrac{1}{2}} & \phantom{0}1\phantom{\sfrac{1}{2}} & 3\sfrac{1}{2} & \phantom{0}3\sfrac{1}{2} & \phantom{00}\sfrac{1}{7}   & \phantom{00}\sfrac{1}{2} \\
      B & 1 &  \phantom{0}9\sfrac{1}{2}           & \phantom{0}4\sfrac{1}{2}           & 3\sfrac{1}{2} & 15\sfrac{3}{4}           & \phantom{0}1\sfrac{11}{14} & \phantom{0}6\sfrac{1}{4} \\
      C & 1 &  \phantom{0}6\phantom{\sfrac{1}{2}} & \phantom{0}5\sfrac{1}{2}           & 3\sfrac{1}{2} & 19\sfrac{1}{4}           & \phantom{0}3\sfrac{11}{14} & 13\sfrac{1}{4} \\
      D & 1 &  \phantom{0}6\phantom{\sfrac{1}{2}} & \phantom{0}7\sfrac{1}{2}           & 3\sfrac{1}{2} & 26\sfrac{1}{4}           & \phantom{0}5\sfrac{11}{14} & 20\sfrac{1}{4}           \\

     \cmidrule(rl){1-1}
     \cmidrule(rl){2-2}
     \cmidrule(rl){3-3}
     \cmidrule(rl){4-4}
     \cmidrule(rl){6-6}
     \cmidrule(rl){7-7}
     \cmidrule(rl){8-8}

     Разом & 4 & 24\sfrac{1}{2} & 18\sfrac{1}{2} & & 64\sfrac{3}{4} & 11\sfrac{1}{2} & 40\sfrac{1}{4} \\
  \end{tabular}

  \end{center}
\end{table}

[Це знов не зовсім вірно обчислено. Орендареві $В$ продукція цих 4\sfrac{1}{2} квартерів
коштує, по-перше, 9\sfrac{1}{2}\pound{ ф. стерл.} ціни продукції і, по-друге,
4\sfrac{1}{2}\pound{ ф. стерл.} ренти, разом 14\pound{ ф. стерл.}; пересічно за квартер 3\sfrac{1}{9}\pound{ ф. стерл}.
Ця пересічна ціна всієї його продукції стає через це за реґуляційну ринкову ціну. Тому рента на
$А$ становила б \sfrac{1}{9}\pound{ ф. стерл.} замість \sfrac{1}{2}\pound{ ф. стерл.}, а рента на $В$ лишалася б, як і давніш, 4\sfrac{1}{2}\pound{ ф.
стерл.}: 4\sfrac{1}{2} квартерн по 3\sfrac{1}{2}\pound{ ф. стерл.} = 14\pound{ ф. стерл.}, звідси вирахувати
9\sfrac{1}{2}\pound{ ф. стерл.} ціни продукції, лишається надзиск в 4\sfrac{1}{2}\pound{ ф. стерл}. Бачимо: не
зважаючи на змінені числа, приклад показує, як з допомогою диференційноі
ренти II краща земля, що вже дає ренту, може стати за регуляційну щодо ціни,
і через це вся земля, також і та, що до того часу не давала ренти, може перетворитися
в рентодайну. — Ф.~Е.].

Збіжжева рента мусить підвищитись, скоро підвищується реґуляційна
ціна продукції збіжжя, отже, скоро підвищується ціна продукції квартера збіжжя
на реґуляційній землі, або реґуляційна витрата капіталу на одному з родів
землі. Це все одно, як коли б усі роди землі стали менш плодючі і продукували
б, наприклад, на кожні 2\sfrac{1}{2}\pound{ ф. стерл.} нових витрат капіталу лише по \sfrac{5}{7}
квартера замість 1 квартера. Весь надмір збіжжя, що його вони продукують

Рід землі    Акри    Ціна продукції    Продукт в кварт. Продажна  ціна    Грошовий  здобуток
Збіжжева рента    Грошова  рента
        ф. стер. ф. стер. ф. стер. ф. стер. ф. стер
А                    1    3             1            31/2       31/2        1/7            1/2
В                    1    9 1/2       41/2    31/2       153/4        111/14    61/4
C                    1    6             51/2    31/2        191/2        311/14    131/4
D                    1    6             71/2    31/2        261/2        511/14     201/4
Разом           4    241/2    181/2           643/4    111/2            401/4
\parbreak{}  %% абзац продовжується на наступній сторінці
