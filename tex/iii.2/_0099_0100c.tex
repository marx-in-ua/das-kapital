\parcont{}  %% абзац починається на попередній сторінці
\index{iii2}{0099}  %% посилання на сторінку оригінального видання
земельної власности), кредит як такий не має вже ніякого сенсу, що однак
зрозуміли й сами сен-сімоністи. З другого боку, поки капіталістичний спосіб
продукції існує далі, існує далі як одна з його форм і капітал, що дає процент,
і дійсно являє базу його кредитової системи. Тільки той самий сенсаційний письменник,
Прудон, що, лишаючи товарову продукцію, хотів знищити гроші\footnote{
Karl Marx, Philosophie de la Misère, Bruxelles et Paris. 1847 — Karl Marx, Kritik der Polit.
Ökonomie p. 64.
},
був здатен вимріяти таке чудерство, як crédit gratuit\footnote*{
Безплатного кредиту. \Red{Пр.~Ред.}
}, цю ніби реалізацію побожного
бажання дрібно буржуазного погляду.

В «Religion Saint-simonienne, Economie et Politique», на стор. 45 сказано:
«В суспільстві, де одні мають знаряддя промисловости, не маючи здібности або
охоти уживати їх, а інші вправні промислові люди не мають жодного знаряддя
праці, там кредит має собі за мету перенести якомога найлегшим способом
ці знаряддя з рук перших їхніх власників до рук тих інших, що тямлять їх
уживати. Зауважмо, що за цим визначенням кредит є наслідок того способу, що
ним \emph{ту власність} усталилось». Отже, кредит відпадає разом з оцим усталенням
власности. Далі на стор. 98, сказано: Сучасні банки «дивляться на себе, як на
установи, призначені йти за тим рухом, що його породили підприємства, які діють
поза їх межами, а що сами вони імпульсу до цього руху не повинні давати; інакше
кажучи, банки виконують ролю капіталістів проти тих travailleurs, що їм вони позичають
капітали». В тій думці, що сами банки мають взяти на себе провід та відзначатися
«кількістю та корисністю підприємств, що ними вони порядкують, і кількістю
тих робіт, що переводиться за їх приводом» (р. 101), — в ній маємо заховану
ідею crédit mobilier. Так само Charles Peequeur вимагає, щоб банки (те, що сен-сімонисти
звуть Système général des banques) «порядкували продукцією». Взагалі Pecquer
є в основі сен-сімоніст, хоч і багато радикальніший. Він хоче, щоб «кредитова
установа\dots{} управляла цілим рухом національної продукції». — «Спробуйте
утворити національну кредитову установу, що позичала б засоби незаможному,
який має талант та заслуги, не зв’язуючи тих довжників між собою примусовою
взаємною солідарністю в продукції та споживанні, а навпаки позичала б
ті засоби так, щоб довжники сами визначали акти свого обміну та продукції.
Цим шляхом ви досягнете лише того, чого вже тепер досягають приватні банки, —
анархії, диспропорції між продукцією та спожитком, раптової руїни одних та
раптового збагачення інших; так що ваша установа ніколи не піде далі від того,
щоб утворити для одних певну суму добробуту, рівну сумі лиха, що припадає
іншим\dots{} найманим робітникам, яких ви підтримуєте позиками, ви тільки дасте
засіб до тієї самої взаємної конкуренції, яку тепер одні одним роблять їхні капіталістичні
хазяїни». (Ch.~Pecqueur, Théorie Nouvelle d’Econoiuie Soc. et Fol. Paris
1842 p. 434.).

\looseness=1
Ми бачили, що купецький капітал та капітал, що дає процент, є найдавніші
форми капіталу. Однак, з самої природи справи випливає те, що капітал, який дає
процент, видається в народній уяві як форма капіталу par excellence. В купецькому
капіталі маємо посередницьку діяльність, хоч і як її тлумачити, як шахрайство, чи
як працю, чи якось інак. Навпаки, в процентодайному капіталі виявляється у чистій
формі характер капіталу, що сам себе репродукує, вартість, що сама собою зростає,
продукція додаткової вартости, як певна таємна якість. Відси ж постає й те, що
навіть частина політикоекономів, особливо в країнах, де, як от у Франції,
промисловий капітал ще не цілком розвинувся, твердо вважають капітал, який
дає процент, за основну форму капіталу, розуміючи, напр. земельну ренту
лише як його іншу форму, бо і тут панує форма визичання. З цієї причини
цілком не розуміють внутрішньої диференціяції капіталістичного способу продукції,
\index{iii2}{0100}  %% посилання на сторінку оригінального видання
та цілком не розуміють того, що землю як і капітал визичають тільки
капіталістам. Звичайно, замість грошей можна визичати засоби продукції in natura,
як от машини, промислові будинки і~\abbr{т. ін.} Але тоді вони становлять певну
грошову суму, і те, що, крім проценту, платиться певну частку за зужиткування,
випливає з споживчої вартости, з особливої натуральної форми цих елементів
капіталу. Справу вирішує тут знову те, чи визичають їх безпосереднім продуцентам,
що має собі за передумову відсутність капіталістичного способу продукції,
принаймні в тій сфері, де трапляється це; або чи визичають їх промисловим
капіталістам, передумова, що може бути саме на базі капіталістичного
способу продукції. Ще більш недоречно та іраціонально притягувати сюди визичання
домів і~\abbr{т. ін.} для індивідуального споживання. Що робітничу клясу
обшахровують і в цій формі, та й ще страшенно обшахровують, це — відомий
факт; але те саме робить і дрібний крамар, що постачає тій клясі життьові
засоби. Це другоступневий визиск, який іде поряд первісного, що відбувається
безпосередньо в самому процесі продукції. Ріжниця між продажем та визичанням
є тут цілком байдужа й формальна річ, яка, як уже показано, тільки при
певному нерозумінні дійсного зв’язку здається істотною.

\plainbreak{3}

Лихварство, як і торговля, визискують даний спосіб продукції, не утворюючи
його та ставлячись до нього зовнішнім способом. Лихварство силкується
просто зберегти його, щоб мати змогу знову й знову визискувати його, воно консервативне
й робить той спосіб тільки злиденнішим. Що менше елементи продукції
ввіходять до процесу продукції, як товари, та виходять з нього, як товари,
то більше добування їх з грошей видається осібним актом. Що незначніша
є роля, що її відіграє циркуляція в суспільній репродукції, то більше розцвітає
лихварство.

\looseness=1
Те, що грошове майно розвивається, як осібне майно, означає відносно
лихварського капіталу, що він усі свої вимоги має в формі грошових вимог.
Він розвивається в країні то більше, що більше продукція у своїй масі обмежується
на натуральних відбутках і~\abbr{т. ін.}, отже, на продукції споживчої вартости.

Оскільки лихварство здійснює дві речі: поперше, взагалі утворює поряд
купецтва самостійне грошове майно, подруге, присвоює собі умови праці, тобто
руйнує власників колишніх умов праці, — остільки є воно могутня підойма до
утворення передумов для промислового капіталу.

\begin{center}
  \so{Процент у середні віки.}
\end{center}

«В середні віки людність була суто хліборобська. А серед такої людности,
як і за февдального режиму можуть бути лише невеликі торговельні зв’язки, а
тому й лише невеликий зиск. Тому закони про лихварство в середні віки були
цілком правні. Сюди долучається те, що у хліборобській країні рідко хто доходить
такого стану, щоб позичати гроші, хіба тільки тоді, коли він опинився
серед убозства та злиднів\dots{} Генріх VII обмежує процент на 10\%, Яків І —
на 8, Карл II — на 6. Анна — на 5\%\dots{} Тоді позикодавці були, як що й неправні,
то все ж фактичні монополісти, і тому треба було обмежити їх, як і інших монополістів\dots{}
В наші часи норма зиску реґулює норму проценту, а тоді норма проценту
реґулювала норму зиску. Коли грошовий позикодавець накидав купцеві тягар високої
норми проценту, купець мусив долучати вищу норму зиску до своїх товарів.
Тому велику суму грошей бралося з кешень покупців, щоб перекласти їх до кешень
грошових позикодавців». (Gilbart, and History Princ. of Banking, p. 164, 165).

«Мені кажуть, що тепер беруть на кожну ляйпцізьку марку 10 ґульденів
на рік, що становить 30 на 100; декотрі ще додають сюди наєнбурзьку марку,
\parbreak{}  %% абзац продовжується на наступній сторінці
