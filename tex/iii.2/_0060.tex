\parcont{}  %% абзац починається на попередній сторінці
\index{iii2}{0060}  %% посилання на сторінку оригінального видання
щоб його не побачило пильне та осудливе око його банкіра? З яким страхом
дбає він про те, щоб його банкір мав добру думку про нього, щоб здаватись
завжди вартим пошани! Насуплене чоло банкіра має на нього більший вплив,
ніж моральні проповіді його приятелів; чи не тремтить він від думки бути запідозреним
в тому, що він винен у шахрайстві або в найдрібнішому фальшивому
посвідченні, боючися, що це може збудити підозри і в наслідок цього можуть
або обмежити його банковий кредит, або й зовсім відмовити йому в кредиті!
Порада банкіра є для нього важливіша, ніж порада попа». (G.~М.~Bell, директор
одного шотляндського банку, The philosophy of Ioint Stock Banking. London 1840,
p. 46, 47).

\section{
Currency principle і англійське банкове законодавство
1844 року}

[В одному попередньому творі\footnote{
К.~Marx, Zur Kritik der Politischen Ökonomie, Berlin 1859, s. 150 ff. (K.~Маркс. До критики
політичної економії, вид. II.~ДВУ, 1926, ст. 179 і далі).
} досліджено теорію Рікардо про вартість
грошей у відношенні до цін товарів; отже ми можемо тут обмежитися на найпотрібнішому.
За Рікардо, вартість грошей — металевих — визначається зречевленим
у них робочим часом, але лише доти, доки кількість грошей є у правильному
відношенні до кількости та ціни товарів, що їх мають обертати. Якщо
кількість грошей підноситься понад це відношення, то їхня вартість знижується,
товарові ціни підносяться; коли ця кількість спада нижче від цього правильного
відношення, то їхня вартість підноситься, а товарові ціни спадають — за всіх інших
однакових умов. В першому випадку країна, де є цей надмір золота,
вивозитиме золото, що впало нижче від своєї вартости, та довозитиме товари;
в другому випадку золото припливатиме до країн, де його цінують вище від його
вартости, тимчасом коли товари, ціновані нижче від вартости, пливтимуть звідти
до інших ринків, де їх можна продати за нормальними цінами. Що серед цих
передумов «само золото, чи то в монеті, чи в зливках, може стати знаком металевої
вартости, більшої або меншої за його власну вартість, то само собою
зрозуміло, що таку саму долю матимуть і розмінні банкноти, що є в циркуляції.
Хоч банкноти й розмінні, отже й їхня реальна вартість відповідає їхній номінальній
вартості, проте вся маса грошей в циркуляції, золота й банкнот (the
aggregate currency consisting of metal and of convertible notes), може ціною піднестися
або знизитися, як до того, чи їхня загальна кількість — з вище розвинутих
причин — підноситься вище або спадає нижче від рівня, що його визначає
мінова вартість товарів, що є в циркуляції, та металева вартість золота\dots{} Це
знецінювання, не паперу проти золота, а золота й паперових грошей разом, або
загальної маси засобів циркуляції певної країни — є один з головних винаходів
Рікардо, що його лорд Оверстон та К° примусили служити собі та зробили фундаментальним
принципом банкового законодавства сера Роберта Піля з років
1844 та 1845». (І. c., р. 155).

Нам не треба повторювати тут наведений на тому самому місці доказ
хибности цієї теорії Рікардо. Нас цікавить лише той спосіб, що ним тези
теорії Рікардо обробила школа банкових теоретиків, яка диктувала згадані банкові
акти Піля.

«Торговельні кризи протягом XIX віку, особливо великі кризи 1825 та
1836 років, не викликали дальшого розвитку Рікардової теорії грошей, але, щоправда,
породили новий ужиток її. Це були вже не поодинокі економічні явища,
як от за часів Юма знецінення благородних металів в XVI і XVII віці, або як
\parbreak{}  %% абзац продовжується на наступній сторінці
