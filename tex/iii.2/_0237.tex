\parcont{}  %% абзац починається на попередній сторінці
\index{iii2}{0237}  %% посилання на сторінку оригінального видання
бо вона є передумова і умова експропріяції умов праці у робітників, — але
особливо й тому, що земельний власник виступає як персоніфікація однієї з
істотних умов продукції.

Нарешті, робітник, як власник і продавець своєї особистої робочої сили,
під ім’ям заробітної плати одержує частину продукту, в якій втілюється та
частина його праці, яку ми називаємо потрібною працею, тобто працею, потрібною
для збереження і репродукції цієї робочої сили, хоч би які були умови
цього збереження і репродукції, бідніші чи багатіші. сприятливіші чи несприятливіші.

Хоч би як по-різному виступали взагалі ці відносини, у всіх них є одно
спільне: капітал рік-у-рік дає капіталістові зиск, земля — земельному власникові
земельну ренту, і робоча сила — при нормальних відносинах і поки
вона лишається придатною робочою силою — робітникові заробітну плату.
Ці три частини вартости щороку продукованої сукупної вартости, і відповідні
їм частини щороку продукованого сукупного продукту можуть бути, — акумуляцію
ми тут покищо залишаємо осторонь, — щорічно спожиті їхніми відповідними
посідачами, при чому джерело їхньої репродукції не вичерпується. Вони являють
собою ніби призначені для щорічного споживаня овочі багатолітнього дерева, або
радше трьох дерев; вони становлять річний дохід трьох кляс, капіталіста, земельного
власника і робітника, доходи, що їх розподіляє капіталіст, який функціонує,
бо він безпосередньо висмоктує додаткову працю і застосовує працю взагалі.
Таким чином, для капіталіста його капітал, для земельного власника його земля
і для робітника його робоча сила, або радше сама його праця (бо він дійсно
продає тільки вияв своєї робочої сили, і ціна робочої сили, як показано раніш,
на базі капіталістичного способу продукції неминуче здається йому ціною праці)
виступають як три різні джерела їхніх специфічних доходів: зиску, земельної
ренти і заробітної плати. Вони й дійсно є такі в тому розумінні, що капітал
для капіталіста є багатолітня машина для висмоктування додаткової праці,
земля для земельного власника — багатолітній маґнет для притягнення частини
тієї додаткової вартости, яку висмоктав капітал, і нарешті, праця є постійно
самовідновна умова і постійно самовідновний засіб для того, щоб під титулом
заробітної плати добувати частину створеної робітником вартости, а тому й вимірювану
цією частиною вартости частину суспільного продукту, потрібні засоби
існування. Вони є такими далі в тому розумінні, що капітал фіксує частину
вартости, а тому і продукту річної праці в формі зиску, земельна власність
— другу частину в формі ренти, і наймана праця — третю частину в формі
заробітної плати, і саме через це перетворення, ці частини стають доходами
капіталіста, земельного власника і робітника, що, проте, не має ніякого чинення
до створення самої субстанції, яка перетворюєтся в ці різні категорії. Навпаки,
розподіл має за свою передумову наявність цієї субстанції, а саме сукупну вартість
річного продукту, що є не що інше, як зрічевлена суспільна праця. Проте,
аґентам продукції, носіям різних функцій процесу продукції, справа уявляється
не в цій, а навпаки, в перекрученій формі. Чому це так стається, ми розвинемо
в перебізі досліду. Цим аґентам продукції капітал, земельна власність і
праця уявляються трьома різними, незалежними джерелами, що з них, як таких,
походять три різні складові частини щорічно продукованої вартости, а
тому й продукту, в якому вона існує; що з них, отже, походять не тільки різні
форми цієї вартости як доходи, що дістаються окремим чиникам суспільного процесу
продукції, але й сама ця вартість, а тим самим і субстанція цих форм
доходу.

[Тут у рукопису не вистачає одного аркуша in folio]

\dots{} Диференційна рента зв’язана з відносною родючістю земель, отже,
з властивостями, що виникають з ґрунту як такого. Але, оскільки вона, поперше,
\parbreak{}  %% абзац продовжується на наступній сторінці
