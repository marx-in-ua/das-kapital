\parcont{}  %% абзац починається на попередній сторінці
\index{iii2}{0102}  %% посилання на сторінку оригінального видання
марнотратникам-вельможам, вони шукали й знаходили поза межами своєї країни
бланковий вексельний кредит, тобто такий кредит, що собі за основу не мав
ніякісінької товарової торговлі; кредит, що його закордонні трасати терпляче
акцептували доти, доки ще надходили римеси, утворені цим вексельовим шахрайством.
За це вони дуже тяжкого лиха зазнали з причини банкрутства такого
банкіра, як Тапер, та інших вельми поважних варшавських банкірів» (І.~G.~Büsch,
Teoretisch-praktische Darstellung der Handlung etc 3. Auflage. Hamburg 1808. Band
II, p. 232, 233.)

\subsubsection{Користь для церкви від заборони проценту}

«Проценти брати церква забороняла; але не забороняла продавати власність,
щоб зарадити собі в нужді; навіть не забороняла віддавати цю власність
на певний час, аж до оплати боргу, грошовому позикодавцеві, щоб він міг собі
мати в тій власності забезпечення, і також, щоб протягом того часу, поки та
власність перебуває в його руках, міг він користуючися мати винагородження
за визичені гроші. Сама церква або приналежні до неї комуни й ріа corpora\footnote*{
Дослівно «благочестиві, побожні тіла», тобто вірні, в церковних громадах,
об’єднані. \Red{Прим. Ред.}
} добували
собі від того значну користь, особливо підчас хрестових походів. Таким
чином значна частина національного доходу опинилася з цієї причини в володінні
так званої «мертвої руки», особливо тому, що єврей не міг лихварювати таким
способом, бо володіння такою нерушною заставою не сила було затаїти\dots{} Без
заборони проценту церкви й манастирі ніколи б не мали змоги стати такими
багатими». (l. c., р. 55.)

