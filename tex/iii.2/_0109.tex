\parcont{}  %% абзац починається на попередній сторінці
\index{iii2}{0109}  %% посилання на сторінку оригінального видання
в 200\pound{ ф. ст.} може розглядатися як процент на капітал в \num{4.000}\pound{ ф. ст}. Капіталізована
таким чином земельна рента і становить купувальну ціну або вартість землі,
категорія, що prima facie, так само як і ціна праці є іраціональна, бо земля не є продукт
праці, отже, не має жодної вартости. Але, з другого боку, за цією іраціональною
формою ховається дійсне продукційне відношення. Коли капіталіст купує
землю, що дає річну ренту в 200\pound{ ф. ст.} за \num{4.000}\pound{ ф. ст.}, то він одержує пересічний річний
процент в 5\% з \num{4.000}\pound{ ф. ст.} — цілком так само, якби він вклав цей капітал у
процентні папери, або безпосередньо віддав його в позику з 5\%. Це зростання вартости
капіталу в \num{4.000}\pound{ ф. ст.} на 5\%. При такому припущенні він повернув би собі
за 20 років купівельну ціну свого маєтку доходами з нього. Тому в Англії купівельна
ціна землі обчислюється за певним числом years’ purchase\footnote*{
Роки, що протягом їх оплачується купівельну ціну. Прим. Ред.
}, що є лише іншим
виразом капіталізування земельної ренти. На ділі, це купівельна ціна, — не
землі, а тієї земельної ренти, яву вона дає, — обчислена відповідно до звичайного
розміру проценту. Але ця капіталізація ренти має своєю передумовою
ренту, тимчасом як ренти не можна вивести й пояснити в зворотному порядку
з її власної капіталізації. її існування, незалежно від продажу, є тут
передумовою, вихідним пунктом.

З цього випливає, що, припускаючи земельну ренту незмінною щодо величини,
ціна землі може підвищуватись або падати в зворотному напрямку з
підвищенням і падінням розміру проценту. Коли б звичайний розмір проценту
знизився з 5 до 4\%, то річна земельна рента в 200\pound{ ф. ст.} становила б річне
зростання вартости з капіталу вже не в \num{4.000}\pound{ ф. ст.} а в \num{5.000}\pound{ ф. ст.} і таким чином
ціна тієї самої ділянки землі підвищилась би з \num{4.000} до \num{5.000}\pound{ ф. ст.} або з 20 years’
purchase до 25. Зворотне в зворотному випадку. Це незалежний від руху самої
земельної ренти і реґульований лише розміром проценту рух земельної ціни.
А що ми бачили, що з поступом суспільного розвитку норма зиску, а тому і
розмір проценту, оскільки він регулюється нормою зиску, має тенденцію
знижуватися; що, далі, навіть лишаючи осторонь норму зиску, розмір проценту
має тенденцію знижуватися в наслідок зросту позичкового грошового капіталу,
то з цього випливає, що ціна землі має тенденцію підвищуватись і незалежно
від руху земельної ренти та ціни земельних продуктів, частину якої становить
рента.

Сплутування самої земельної ренти з тією процентною формою, яку вона
набуває для покупця землі — сплутування, що ґрунтується на цілковитій
непізнанності природи земельної ренти, — мусить привести до найдивовижніших
помилкових висновків. Але що земельну власність у всіх старих країнах
вважається за особливо почесну форму власности, а купівлю земельної власности
— за особливо певне приміщення капіталу, то розмір проценту, з якого
обчислюється купувальна ціна земельної ренти, є звичайно нижчий, ніж його розмір
при інших способах приміщення капіталу, розрахованих на порівняно довший час,
так що, наприклад, покупець землі одержує на купівельну ціну її лише 4\%, тимчасом
як він одержав би на той самий капітал при іншому способі приміщення 5\%;
або, що сходить на те саме, він платить за земельну ренту більше капіталу,
ніж довелося б йому заплатити за такий самий річний грошовий дохід в інших
сферах приміщення капіталу. Пан Тьєр у своїй взагалі цілком кепській праці про
La Propriété (відбитку його промови проти Прудона, проголошеної 1849~\abbr{р.} на французьких
Національних Зборах) робить той висновок, що земельна рента низька,
тимчасом як це тільки доводить, що купувальна ціна її висока.

Та обставина, що капіталізована земельна рента видається ціною або
вартістю землі, і що земля, таким чином, продається і купується, як усякий
інший товар, є для деяких апологетів ґрунтом для виправдання земельної
\parbreak{}  %% абзац продовжується на наступній сторінці
