\parcont{}  %% абзац починається на попередній сторінці
\index{iii2}{0055}  %% посилання на сторінку оригінального видання
їм для своїх поточних операцій. Bill-brokers’и тримають «вільні банкові гроші
країни», не маючи запасу. І Англійський банк на забезпечення своїх боргів по
вкладам мав тільки запаси банкірів та інших осіб, разом з public deposits і~\abbr{т. ін.},
доводячи їх до найнижчого рівня, напр., до 2 міл. Тому, опріч цих 2 міл. в паперах,
підчас скрути (а вона зменшує запас, бо банкноти, що надходять замість
металю, який відпливає, доводиться анулювати), все це крутійство абсолютно
не має ніякого резерву, опріч металевого скарбу, й тому кожне зменшення останнього
в наслідок відпливу золота збільшує кризу.

«5306. Коли б не було грошей на те, щоб вирівнювати платежі в розрахунковій
палаті, то, на мою думку, нам не лишилося б нічого іншого, як
зійтися разом та проробити наші платежі прима-векселями, векселями на державну
скарбницю, на фірми Smith, Payne~\& Co і~\abbr{т. ін.} — 5307. Отже, коли б
уряд не подбав постачити вам засоби циркуляції, то ви самі утворили б їх
собі? — Що можемо ми зробити? Приходить публіка й забирає в нас з рук
засіб циркуляції; його не стало. — 5308. Отже ви ж зробили б в Лондоні тільки
те, що роблять в Менчестері щодня? — Так».

Дуже добра відповідь Chapman’a на питання, що його поставив Cayley
(Birmingham-man етвудівської школи) щодо оверстонівського уявлення про
капітал: «5315. Перед комісією говорилось, що підчас скрути, такої, як от та,
що була року 1847, шукають не грошей, а капіталу; яка ваша думка в цій
справі? — Я не розумію вас; ми робимо операції тільки грішми; я не розумію,
що ви маєте тут на думці. — 5316. Коли ви під цим» [комерційний капітал]
«розумієте кількість належних певній людині грошей, що їх вона має в
своєму підприємстві, — коли ви це звете капіталом, то та ж кількість грошей є
лише дуже незначна частина тих грошей, що ними він порядкує у своїх операціях,
користуючися з кредиту, що його дає йому публіка» — через посередництво
Chapman’ів.

«5339. Чи бракує в нас багатства, коли ми припиняємо свої платежі готівкою?
— Аж ніяк;\dots{} в нас не бракує багатства, але ми рухаємося в незвичайно
штучній системі, і коли в нас постає величезний загрозливий [superincumbent]
попит на наші засоби циркуляції, то можуть настати такі обставини, які
заважатимуть нам добувати ці засоби циркуляції. Чи має з тієї причини спаралізуватися
ціла комерційна промисловість країни? Чи маємо закрити всякі
шляхи до зайняття? — 5338. Коли б перед нами мало постати питання, що
маємо зберегти — платежі готівкою чи промисловість країни, то я вже знатиму,
що з них мені пустити в занепад».

Про нагромадження банкнот «з наміром загострити скруту та добути
користь від наслідків того» [5358] й він каже, що це може трапитися дуже
легко. Для цього досить було б трьох великих банків. «5383. Чи не мусить
бути відомо вам, як людині добре ознайомленій з великими підприємствами
нашої столиці, що капіталісти використовують ці кризи на те, щоб добувати
величезні зиски від руїни тих, що падають жертвою? — В цьому не може
бути жодного сумніву». І тут мусимо ми йняти віру панові Chapman’ові, дарма
що він кінець-кінцем, спробувавши «добути величезні зиски від руїни жертов»,
зламав собі шию, як комерсант. Бо коли його спільник Gurney каже, що кожна
зміна в справах корисна для тямущої людини, то Chapman каже: «Одна
частина суспільства не знає нічого про іншу; от, напр., фабрикант, що експортує
до континенту або імпортує собі сировину, нічого не знає про іншого,
що торгує зливками золота». (5046). І так саме трапилося, що одного дня
самі Gyrney та Chapman стали отими «нетямущими» та зробилися ганебними
банкрутами.

Ми вже бачили вище, що видання банкнот не в усіх випадках означає
позику капіталу. Нижче наведене свідчення Тука перед C.~Д. комісією лордів
\parbreak{}  %% абзац продовжується на наступній сторінці
