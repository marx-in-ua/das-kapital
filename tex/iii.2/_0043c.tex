\parcont{}  %% абзац починається на попередній сторінці
\index{iii2}{0043}  %% посилання на сторінку оригінального видання
(solvent) свого banking department’у\footnote*{
Banking department — банковий відділ. \emph{Пр. Ред}.
}, банк мусить робити все можливе для поповнення запасу цього
відділу; отже, скоро він бачить, що настає відплив, він мусить забезпечити собі запас та або
обмежити свої дисконтові операції, або продавати цінні
напери». — Запас, оскільки ми маємо на увазі лише banking department, є запас
тільки для вкладів. За Оверстоном, banking department має діяти лише як банкір,
не вважаючи на «автоматичне» видання банкнот. Але підчас дійсного пригнічення
банк незалежно від того запасу banking department'у, що складається лише з
банкнот, пильно стежить за металевим скарбом і мусить стежити, коли не хоче
збанкрутувати. Бо в тій самій мірі, в якій зникає металевий скарб, зникає й
запас банкнот, й ніхто цього не мав би знати краще, як пан Оверстон, що так
премудро улаштував це саме своїм банковим актом 1844 року.

\section{Засоби циркуляції при кредитовій системі.}

«Великий реґулятор швидкости циркуляції є кредит. Відси пояснюється,
чому гостра скрута на грошовому ринку звичайно збігається з переповненою
циркуляцією». (The Currency Question Reviewed, p. 65.). Це треба розуміти
двояко. З одного боку, всі методи, що заощаджують засоби циркуляції, засновані
на кредиті. Але, з другого боку візьмім, напр., банкноту в 500\pound{ ф. ст.} $А$ дає її
сьогодні $В$ на оплату векселя; $В$ складає її того самого дня у свого банкіра;
останній ще того самого дня дисконтує цією сумою вексель $C$; $C$ платить нею
своєму банкові, банк дає її bill-broker’ові позикою і~\abbr{т. ін.} Швидкість, з якою
обертається тут ця банкнота, обслуговуючи купівлі або платежі, залежить від
тієї швидкости, що з нею вона раз-у-раз вертається до когось у формі вкладу
та знову переходить до когось іншого у формі позики. Просте економізування
засобу циркуляції досягає найбільшого розвитку в Clearing House\footnote*{
Clearing House — розрахункова палата. \emph{Пр. Ред}.
}, у простій
виміні векселів, що їм надійшов реченець платежа, та в переважній функції
грошей як платіжного засобу для вирівнювання самих лише лишків. Але саме
існування цих векселів спирається знову ж на кредит, що його дають одні одним
промисловці та купці. Якщо цей кредит меншає, то меншає й число векселів,
особливо довготермінових, отже, меншає й чинність цієї методи вирівнювань.
І ця економія, що є в усуненні грошей з оборотів та спирається цілком на функції
грошей як платіжного засобу, функції, що своєю чергою спирається на кредит, —
ця економія може (якщо не вважати на більш або менш розвинуту техніку
концентрації цих платежів) бути лише двоякого роду: або взаємні боргові вимоги,
представлені векселями чи чеками, вирівнюються в того самого банкіра, що тільки
переписує вимогу з рахунку одного на рахунок іншого; або різні банкіри
вирівнюють їх між собою\footnote{
Пересічне число днів, що протягом їх лишалася в циркуляції банкнота:

\begin{tabular}{l c c c c c}
\toprule
\makecell{P і к}  &  \makecell{Банкнота \\ в 5\pound{ ф. ст.}}  &  \makecell{10\pound{ ф. ст.}}  &  \makecell{20\textendash{}100 \\ ф. ст.}  &  \makecell{200\textendash{}500 \\ ф. ст.} & \makecell{Банкнота \\ в 1000\pound{ ф. ст.}} \\
\cmidrule{1-6}
1798 \dotfill{} & \phantom{0}?\phantom{0}  &   236           &  209            &   31           &   22 \\
1818 \dotfill{} &                148       &   137           &  121            &   18           &   13 \\
1846 \dotfill{} & \phantom{0}79            &   \phantom{0}71 &   \phantom{0}34 &   12           &   \phantom{0}8 \\
1856 \dotfill{} & \phantom{0}70            &   \phantom{0}58 &   \phantom{0}27 &   \phantom{0}9 &   \phantom{0}7 \\
\end{tabular}

(Відомості, подані касиром Англійського банку, Маршалем, в Report on Bank Acts; 1857. II. Appendix
p. 301--302.
}. Концентрація 8--10 мільйонів векселів в руках
\parbreak{}  %% абзац продовжується на наступній сторінці
