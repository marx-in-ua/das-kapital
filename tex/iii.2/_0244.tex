\parcont{}  %% абзац починається на попередній сторінці
\index{iii2}{0244}  %% посилання на сторінку оригінального видання
тим, що норма зиску зростає, коли товар, проданий нижче від його вартости,
становить елемент сталого капіталу, або тим, що зиск і рента втілюються
в більшій кількості продукту, коли товар, проданий нижче від його вартости,
входить як річ особистого споживання в частину вартости, споживану як дохід.
А подруге, це знищується в пересічних коливаннях. В усякому разі, коли навіть
частина додаткової вартости, яка не реалізувалася в ціні товару, не бере участи
в створенні ціни, — сума пересічного зиску плюс рента в її нормальній формі
ніколи не може бути більша за всю додаткову вартість, хоч і може бути
менша за неї.

Її нормальна форма має своєю передумовою заробітну плату, відповідну
до вартости робочої сили. Навіть монопольна рента, оскільки вона не є вирахування
із заробітної плати, отже, не являє собою осібної категорії, посередньо
мусить завжди становити частину додаткової вартости; коли вона і не являє
собою частини надміру ціни над ціною продукції того самого товару, що вона
є його складова частина (як за диференційної ренти), або коли вона і не являє
собою надмірної частини додаткової вартости того самого товару, що вона є його
складова частина, над частиною його власної додаткової вартости, вимірюваної
пересічним зиском (як за абсолютної ренти), то все таки вона становить частину
додаткової вартости інших товарів, тобто товарів, обмінюваних на цей товар,
що має монопольну ціну. — Сума пересічного зиску плюс земельна рента ніколи
не можуть перебільшувати величини, що частинами її є ці зиски і рента, і що її
вже дано до цього поділу. Тому для нашого дослідження байдуже, чи реалізується
в ціні товарів уся додаткова вартість товарів, тобто вся додаткова праця,
що міститься в товарах, чи ні. Додаткова праця вже тому не реалізується цілком,
що при постійній зміні кількости праці, суспільно потрібної для продукції даного
товару, що виникає з постійної зміни продуктивної сили праці, частину
товарів завжди продукується в ненормальних умовах, а тому їх доводиться
продавати нижче від їхньої індивідуальної вартости. В усякому разі зиск
плюс рента дорівнюють усій реалізованій додатковій вартості (додатковій праці)
і для дослідження, про яке тут іде мова, реалізовану додаткову вартість можна
вважати за рівну всій додатковій вартості, бо зиск і рента є реалізована додаткова
вартість, отже, взагалі та додаткова вартість, що входить в ціни товарів,
отже, практично вся та додаткова вартість, яка є складова частина цієї ціни.

\looseness=1
З другого боку, заробітна плата, що становить третю своєрідну форму
доходу, завжди дорівнює змінній складовій частині капіталу, тобто тій складовій
частині, яку витрачається не на засоби праці, а на купівлю живої робочої
сили, на виплату робітникам. (Працю, оплачувану при витрачанні доходу, оплачується
з заробітної плати, зиску або ренти і тому вона не становить частини
вартости товарів, що ними її оплачується. Таким чином, її не береться на увагу
при аналізі вартости товарів і складових частин, на які вона розпадається).
Вартість змінного капіталу, отже, і ціна праці репродукується в певній частині
усього зрічевленого робочого дня робітників, в тій частині товарової вартости,
в якій робітник репродукує вартість своєї власної робочої сили або ціну своєї
праці. Весь робочий день робітника розпадається на дві частини. Одна частина
та, підчас якої він виконує кількість праці, потрібну для репродукції вартости
його власних засобів існування: оплачена частина всієї його праці, та частина
його праці, що потрібна для його власного збереження і репродукції. Вся решта
робочого дня, вся надмірна кількість праці, яку він виконує понад працю, реалізовану
в вартості його заробітної плати, є додаткова праця, неоплачена праця,
що втілюється в додатковій вартості усіх випродукованих ним товарів (і тому
в надмірній кількості товару), в додатковій вартості, яка й собі розпадається на
частини з різними назвами, на зиск (підприємницький бариш плюс процент)
та ренту.
