
\index{iii2}{0155}  %% посилання на сторінку оригінального видання
На землі $D$ збіжжева рента проти таблиці I зросла з 3
% REMOVED \footnote*{
%В німецькому тексті стоїть: «з 2 квартерів». Явна помилка, як це можна бачити з таблиці І \emph{Прим. Ред.}}
квартерів до 6
тимчасом як грошова рента лишилася, як і давніш, 9\pound{ ф. стерл}. Проти таблиці II
збіжжева рента з $D$ лишилася колишня, 6 квартерів, але грошова рента знизилась
з 18\pound{ ф. стер.} до 9\pound{ ф. стерл}.

Коли розглядати загальні суми ренти, то збіжева рента таблиці IVb \deq{} 8
квартерам, більша, ніж рента в таблиці І, що дорівнює 6 квартерам, і більша,
ніж рента в таблиці IVа, що дорівнює 7 квартерам; і навпаки, вона менша, ніж
рента в таблиці II \deq{} 12 кварт. Грошова рента в таблиці IVb \deq{} 12\pound{ ф. стерл.},
більша, ніж грошова рента в таблиці ІVа \deq{} 10\sfrac{1}{2}\pound{ ф. стерл.}, і менша від грошової
ренти таблиці І \deq{} 18\pound{ ф. стерл.} і таблиці II \deq{} 36\pound{ ф. стерл}.

Щоб по відпаданні ренти з $В$ в умовах таблиці IVb загальна сума ренти
дорівнювала такій у таблиці I, ми мусимо одержати ще на 6\pound{ ф. стерл.}
надпродукту, тобто 4 квартери по 1\sfrac{1}{2}\pound{ ф. стерл.}, що є новою ціною продукції.
Тоді ми знову маємо загальну суму ренти в 18\pound{ ф. стерл.}, як у таблиці І. Величина
потрібного на це додаткового капіталу буде різна залежно від того, чи
вкладемо ми його в $C$ або $D$, чи розподілимо його між обома родами землі.

На $C$ капітал в 5\pound{ ф. стерл.} дає 2 квартери надпродукту, отже, 10\pound{ ф. ст.}
додаткового капіталу дадуть 4 квартери додаткового надпродукту. На $D$ було б
досить додаткової витрати в 5\pound{ ф. стерл.}, щоб випродукувати 4 квартери додаткової
збіжжевої ренти при зробленому тут основному припущенні, що продуктивність
додаткових капіталовкладень лишається та сама. Тому здобуваємо
такі наслідки.

\begin{table}[H]
  \centering
  \caption*{Таблиця ІVc}
  \footnotesize

  \settowidth\rotheadsize{\theadfont Продажна}
  \begin{tabular}{l c r c c c c c c c c}
    \toprule
      \thead[tl]{Рід\\землі} &
      &
      \rothead{Капітал} &
      \rothead{Зиск} &
      \rothead{Ціна\\продукції} &
      \rothead{Продукт} & % \\ в кварт.}}} \\ в кварт.}}}
      \rothead{Продажна\\ціна} &
      \rothead{Здобуток} &
      \multicolumn{2}{c}{Рента} &
      \rothead{Норма\\надзиску} \\

      \cmidrule(rl){2-9}

       & акри  & \makecell{\poundsign{}} & \poundsign{} & \poundsign{} & кв. & \poundsign{} & \poundsign{} & кв. & \poundsign{}  & \% \\
      \midrule

      B & 1 &  \phantom{0}5\phantom{\tbfrac{1}{2}} & 1\phantom{\tbfrac{1}{2}} & \phantom{0}6 & \phantom{0}4 & 1\tbfrac{1}{2} & \phantom{0}6 & \phantom{0}0 & \phantom{0}0 & \phantom{00}0 \\
      C & 1 & 15\phantom{\tbfrac{1}{2}}            & 3\phantom{\tbfrac{1}{2}} & 18           & 18           & 1\tbfrac{1}{2} & 27                   & \phantom{0}6 & \phantom{0}9 & \phantom{0}60\\
      D & 1 &  \phantom{0}7\tbfrac{1}{2}           & 1\tbfrac{1}{2}           & \phantom{0}9 & 12           & 1\tbfrac{1}{2} & 18                   & \phantom{0}6 & \phantom{0}9 & 120\\
     \midrule

     Разом & 3 & 27\tbfrac{1}{2} & 5\tbfrac{1}{2} & 33 & 34 & & 51 & 12 & 18 &\\
  \end{tabular}
\end{table}

\begin{table}[H]
  \centering
  \caption*{Таблиця ІVd}
  \footnotesize

  \settowidth\rotheadsize{\theadfont Продажна}
  \begin{tabular}{l c r c c c c c c c c}
    \toprule
      \thead[tl]{Рід\\землі} &
      &
      \rothead{Капітал} &
      \rothead{Зиск} &
      \rothead{Ціна\\продукції} &
      \rothead{Продукт} & % \\ в кварт.}}} \\ в кварт.}}}
      \rothead{Продажна\\ціна} &
      \rothead{Здобуток} &
      \multicolumn{2}{c}{Рента} &
      \rothead{Норма\\надзиску} \\

      \cmidrule(rl){2-11}

       & акри  & \makecell{\poundsign{}} & \poundsign{} & \poundsign{} & кв. & \poundsign{} & \poundsign{} & кв. & \poundsign{}  & \% \\
      \midrule

      B & 1 & \phantom{0}5\phantom{\tbfrac{1}{2}} & 1\phantom{\tbfrac{1}{2}} & \phantom{0}6 & \phantom{0}4 & 1\tbfrac{1}{2} & \phantom{0}6 & \phantom{0}0 & \phantom{0}0 & \phantom{00}0\\
      C & 1 & \phantom{0}5\phantom{\tbfrac{1}{2}} & 1\phantom{\tbfrac{1}{2}} & \phantom{0}6 & \phantom{0}6 & 1\tbfrac{1}{2} & \phantom{0}9 & \phantom{0}2 & \phantom{0}3 & \phantom{0}60\\
      D & 1 & 12\tbfrac{1}{2}                     & 2\tbfrac{1}{2}           & 15           & 20           & 1\tbfrac{1}{2} & 30           & 10           & 15           & 120\\
     \midrule

     Разом & 3 & 22\tbfrac{1}{2} & 4\tbfrac{1}{2} & 27 & 30 & & 45 & 12 & 18 &\\
  \end{tabular}
\end{table}
