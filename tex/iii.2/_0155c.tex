
\index{iii2}{0155}  %% посилання на сторінку оригінального видання
На землі $D$ збіжжева рента проти таблиці I зросла з 3\footnote*{
В німецькому тексті стоїть: «з 2 квартерів». Явна помилка, як це можна бачити з таблиці І \emph{Прим. Ред.}
} квартерів до 6
тимчасом як грошова рента лишилася, як і давніш, 9\pound{ ф. стерл}. Проти таблиці II
збіжжева рента з $D$ лишилася колишня, 6 квартерів, але грошова рента знизилась
з 18\pound{ ф. стер.} до 9\pound{ ф. стерл}.

Коли розглядати загальні суми ренти, то збіжева рента таблиці IVb \deq{} 8
квартерам, більша, ніж рента в таблиці І, що дорівнює 6 квартерам, і більша,
ніж рента в таблиці IVа, що дорівнює 7 квартерам; і навпаки, вона менша, ніж
рента в таблиці II \deq{} 12 кварт. Грошова рента в таблиці IVb \deq{} 12\pound{ ф. стерл.},
більша, ніж грошова рента в таблиці ІVа \deq{} 10\sfrac{1}{2}\pound{ ф. стерл.}, і менша від грошової
ренти таблиці І \deq{} 18\pound{ ф. стерл.} і таблиці II \deq{} 36\pound{ ф. стерл}.

Щоб по відпаданні ренти з $В$ в умовах таблиці IVb загальна сума ренти
дорівнювала такій у таблиці I, ми мусимо одержати ще на 6\pound{ ф. стерл.}
надпродукту, тобто 4 квартери по 1\sfrac{1}{2}\pound{ ф. стерл.}, що є новою ціною продукції.
Тоді ми знову маємо загальну суму ренти в 18\pound{ ф. стерл.}, як у таблиці І.~Величина
потрібного на це додаткового капіталу буде різна залежно від того, чи
вкладемо ми його в $C$ або $D$, чи розподілимо його між обома родами землі.

На $C$ капітал в 5\pound{ ф. стерл.} дає 2 квартери надпродукту, отже, 10\pound{ ф. ст.}
додаткового капіталу дадуть 4 квартери додаткового надпродукту. На $D$ було б
досить додаткової витрати в 5\pound{ ф. стерл.}, щоб випродукувати 4 квартери додаткової
збіжжевої ренти при зробленому тут основному припущенні, що продуктивність
додаткових капіталовкладень лишається та сама. Тому здобуваємо
такі наслідки.

\begin{table}[H]
  \begin{center}
    \emph{Таблиця ІVc}
    \footnotesize

  \begin{tabular}{c c c c c c c c c c c}
    \toprule
      \multirowcell{2}{\makecell{Рід \\землі}} &
      \multirowcell{2}{\rotatebox[origin=c]{90}{Акри}} &
      \rotatebox[origin=c]{90}{Капітал} &
      \rotatebox[origin=c]{90}{Зиск} &
      \rotatebox[origin=c]{90}{\makecell{Ціна про- \\ дукції}} &
      \multirowcell{2}{\rotatebox[origin=c]{90}{\makecell{Продукт \\ в кварт.}}} &
      \rotatebox[origin=c]{90}{\makecell{Продажна \\ ціна}} &
      \rotatebox[origin=c]{90}{Здобуток} &
      \multicolumn{2}{c}{Рента} &
      \multirowcell{2}{\makecell{Норма \\надзиску}} \\

      \cmidrule(rl){3-3}
      \cmidrule(rl){4-4}
      \cmidrule(rl){5-5}
      \cmidrule(rl){7-7}
      \cmidrule(rl){8-8}
      \cmidrule(rl){9-10}

       &  &  ф. ст. & ф. ст. & ф. ст. & & ф. ст. & ф. ст. & Кварт. & ф. ст. &  \\
      \midrule

      B & 1 &  \phantom{0}5\phantom{\sfrac{1}{2}} & 1\phantom{\sfrac{1}{2}} & \phantom{0}6 & \phantom{0}4 & 1\sfrac{1}{2} & \phantom{0}6 & 0 & \phantom{0}0 & \phantom{00}0\% \\
      C & 1 & 15\phantom{\sfrac{1}{2}}            & 3\phantom{\sfrac{1}{2}} & 18           & 18           & 1\sfrac{1}{2} & 27           & 6 & \phantom{0}9 & \phantom{0}60\%\\
      D & 1 &  \phantom{0}7\sfrac{1}{2}           & 1\sfrac{1}{2}           & \phantom{0}9 & 12           & 1\sfrac{1}{2} & 18           & 6 & \phantom{0}9 & 120\%\\
     \cmidrule(rl){1-1}
     \cmidrule(rl){2-2}
     \cmidrule(rl){3-3}
     \cmidrule(rl){4-4}
     \cmidrule(rl){5-5}
     \cmidrule(rl){6-6}
     \cmidrule(rl){8-8}
     \cmidrule(rl){9-9}
     \cmidrule(rl){10-10}

     Разом & 3 & 27\sfrac{1}{2} & 5\sfrac{1}{2} & 33 & 34 & & 51 & 12 & 18 &\\
  \end{tabular}

  \end{center}
\end{table}

\begin{table}[H]
  \begin{center}
    \emph{Таблиця ІVd}
    \footnotesize

  \begin{tabular}{c c c c c c c c c c c}
    \toprule
      \multirowcell{2}{\makecell{Рід \\землі}} &
      \multirowcell{2}{\rotatebox[origin=c]{90}{Акри}} &
      \rotatebox[origin=c]{90}{Капітал} &
      \rotatebox[origin=c]{90}{Зиск} &
      \rotatebox[origin=c]{90}{\makecell{Ціна про- \\ дукції}} &
      \multirowcell{2}{\rotatebox[origin=c]{90}{\makecell{Продукт \\ в кварт.}}} &
      \rotatebox[origin=c]{90}{\makecell{Продажна \\ ціна}} &
      \rotatebox[origin=c]{90}{Здобуток} &
      \multicolumn{2}{c}{Рента} &
      \multirowcell{2}{\makecell{Норма \\надзиску}} \\

      \cmidrule(rl){3-3}
      \cmidrule(rl){4-4}
      \cmidrule(rl){5-5}
      \cmidrule(rl){7-7}
      \cmidrule(rl){8-8}
      \cmidrule(rl){9-10}

       &  &  ф. ст. & ф. ст. & ф. ст. & & ф. ст. & ф. ст. & Кварт. & ф. ст. &  \\
      \midrule

      B & 1 & \phantom{0}5\phantom{\sfrac{1}{2}} & 1\phantom{\sfrac{1}{2}} & \phantom{0}6 & \phantom{0}4 & 1\sfrac{1}{2} & \phantom{0}6 & \phantom{0}0 & \phantom{0}0 & \phantom{00}0\% \\
      C & 1 & \phantom{0}5\phantom{\sfrac{1}{2}} & 1\phantom{\sfrac{1}{2}} & \phantom{0}6 & \phantom{0}6 & 1\sfrac{1}{2} & \phantom{0}9 & \phantom{0}2 & \phantom{0}3 & \phantom{0}60\%\\
      D & 1 & 12\sfrac{1}{2}                     & 2\sfrac{1}{2}           & 15           & 20           & 1\sfrac{1}{2} & 30           & 10           & 15           & 120\%\\
     \cmidrule(rl){1-1}
     \cmidrule(rl){2-2}
     \cmidrule(rl){3-3}
     \cmidrule(rl){4-4}
     \cmidrule(rl){5-5}
     \cmidrule(rl){6-6}
     \cmidrule(rl){8-8}
     \cmidrule(rl){9-9}
     \cmidrule(rl){10-10}

     Разом & 3 & 22\sfrac{1}{2} & 4\sfrac{1}{2} & 27 & 30 & & 45 & 12 & 18 &\\
  \end{tabular}

  \end{center}
\end{table}

