\parcont{}  %% абзац починається на попередній сторінці
\index{iii2}{0015}  %% посилання на сторінку оригінального видання
і ми маємо підставу гадати, що в другого було 8--10 мільйонів; один мав 4,
другий 3\sfrac{1}{2}, третій більше за 8. Я кажу про вклади у brokers’ів». (Report,
of Committee on Rank Akts, 1857--58 p., 5, № 8).

«Лондонські billbrokers’и\dots{} провадили свої величезні операції без всякого
запасу готівкою; вони покладалися на одержування грошей від тих своїх векселів,
що їм раз-у-раз надходив реченець платежу, або ж в гіршому разі — на змогу
одержувати позики в Англійському банку, депонуючи в нього оті дисконтовані
ними векселі». — Дві фірми bill-brokers’ів у Лондоні припинили платежі в
1847 році; обидві вони пізніше відновили операції. В 1857 році вони знову
припинили платежі. Пасиви однієї фірми в 1847 році становили кругло
\num{2.683.000}\pound{ ф. ст.} при капіталі з \num{180.000}\pound{ ф. ст.}; її пасиви в 1857 році були \deq{}
\num{5.300.000}\pound{ ф. ст.}, тимчасом коли капітал становив, імовірно, не більше як чверть
того, що було в неї в 1847 році. Пасиви другої фірми були в обох випадках
в межах 3--4 мільйонів при капіталі не більшому, ніж \num{45.000}\pound{ ф. ст.}» (ibidem,
p. XXI, № 52).

\section{Грошовий капітал та дійсний капітал. I}

Єдино тяжкі питання, що до них наближаємось ми тепер у справі кредиту,
такі:

\emph{Поперше}. Нагромадження власно грошового капіталу. Оскільки воно
є й оскільки воно не є ознака дійсного нагромадження капіталу, тобто репродукції
у поширеному маштабі? Чи так звана plethora\footnote*{
Грецьке слово, що йому найближче відповідає — укр. повнява, або багатість. \Red{Пр.~Ред.}
} капіталу, вислів, що його
завжди уживається тільки про капітал, який дає процент, тобто про грошовий
капітал, — є лише осібний спосіб виражати промислову надмірну продукцію,
чи являє він осібне явище поряд неї? Чи ця plethora, це надмірне постачання
грошового капіталу збігається з наявністю грошових мас (зливків,
золотих монет та банкнот) у стані застою, так що цей надмір дійсних грошей
є вислів і форма вияву тієї plethor’и позичкового капіталу?

І, \emph{подруге}. Оскільки скрута на гроші, тобто недостача позичкового капіталу,
означає недостачу дійсного капіталу (товарового капіталу та продуктивного
капіталу)? Оскільки, з другого боку, вона, та скрута, збігається з недостачею
грошей як таких, з недостачею засобів циркуляції?

Оскільки ми досі розглядали своєрідну форму нагромадження грошового
капіталу та грошового майна взагалі, вона сходила на нагромадження вимог
власности на працю. Нагромадження капіталу державного боргу означає, як уже
виявилося, не що інше, а тільки збільшення кляси кредиторів держави, що мають
право брати собі наперед певні суми з податків\footnote{
«Державні фонди — не що інше, як уявлюваний капітал, що представмо частину річного доходу,
призначену для оплати боргу. Капітал, що є тій частині еквівалентний, вже витрачено; він визначив
суму позики, але не його представляють державні фонди; бо капіталу того вже більше немає. Тимчасом
нові багатства повинна утворити промислова праця; певну річну частину цих багатств наперед
призначається
тим, що визичили капітал, тепер уже знищений; цю частину багатств податками відбирають
від продуцентів їх, щоб віддати кредиторам держави, і відповідно до звичайного в даній країні
відношення
між капіталом та процентом припускається, що є уявлюваний капітал, рівновеликий тому, що міг би
дати таку річну ренту, яку мають одержувати кредитори». (Sismondi, Nouveaux Principes II, p. 230).
}. В тому факті, що навіть нагромадження
боргів може здаватися нагромадженням капіталу, виявляється довершення
того перекручування, що відбувається в кредитовій системі. Ці боргові
посвідки, видані за первісно позичений та вже давно витрачений капітал, ці
паперові дублікати знищеного капіталу функціонують для своїх державців як
капітал остільки, оскільки вони є товари, що їх можна продати, отже й оскільки
їх можна перетворити знову на капітал.
