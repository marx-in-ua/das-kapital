
1)~Що ряд у своєму закінченному вигляді, — хоч би який завжди був перебіг процесу його складання, —
завжди виступає як низхідний; бо при розгляді ренти завжди виходять спочатку від землі, яка дає
максимум ренти/і лише нарешті переходять до тієї, що не дає жодної ренти.

2)~Ціна продукції на найгіршій землі, що не дає ренти, завжди становить реґуляційну ринкову ціну,
хоч ціна ця в таблиці 1, коли вона склалася у висхідному порядку, тільки через те лишається
незмінна, що всю кращу землю оброблено. В цьому випадку ціна збіжжя, випродукованого на кращій
землі, є реґуляційною остільки, оскільки від кількости продукту, випродукованого на ній, залежить, в
якій мірі земля $А$ лишається реґуляційною. Коли б на землях $В$, $C$, $D$ продукувалось понад потребу, то
земля $А$ перестала б бути реґуляційною. Це й штовхнуло Шторха на те, що він за реґуляційпі визнав
найкращі землі. В цьому розумінні англійські ціпи хліба реґулюються американськими.

3)~Диференційна рента постає з ріжниці в природній родючості різного ґрунту (тут положення землі ще
не береться на увагу), — ріжниці даної для кожного даного ступеня розвитку культури, отже, з
обмежености кількости кращих земель, і з тієї обставини, що однакові капітали доводиться вкладати в
неоднакові землі, які, отже, при витраті однакового капіталу дають неоднакову кількість продукту.

4)~Диференційна рента і ґрадація диференційної ренти можуть однаково виникнути так у низхідному
порядку в наслідок переходу від кращої землі до гіршої, як і навпаки, від гіршої до кращої, або в
наслідок переходу впереміжку в усіх напрямках. (Ряд І може скластися в наслідок переходу так від І)
до $А$, як і від $А$ до $D$. Ряд II охоплює обидва види руху).

5)~Залежно від способу виникнення, диференційна рента може постати за сталої, висхідної і низхідної
ціни хліборобського продукту. За низхідної ціни загальна продукція і загальна сума ренти може
підвищитись, і земельні дільниці, що не давали до цього часу ренти, можуть почати давати ренту, не
зважаючи на те, що гірша земля $А$ витиснена кращою або сама поліпшилась, і що рента з інших кращих і
навіть найкращих земель понижується (таблиця II);
цей процес може бути також зв’язаний з пониженням загальної суми ренти (в грошах). Нарешті, за
низхідних цін, зумовлених загальним поліпшенням обробітку, в зв’язку з чим кількість і ціна продукту
з найгіршої землі понижується, — рента з частини кращих земель може лишитись незмінною або
понизитися, але рента з найліпших земель може зрости. Коли ріжниця мас продукту дана, то
диференційна рента з усякої землі проти найгіршої землі без сумніву залежить від ціни, наприклад,
квартера пшениці. Але коли ціна є дана, то диференційна рента залежить від розміру ріжниці між
масами продукту, і коли з підвищенням абсолютної родючости всіх земель родючість земель відносно
більше підвищується, ніж родючість гірших, то разом з цим зростає і величина цієї ріжниці. Так
(таблиця І) при ціні в 60\shil{ шил.} рента з $B$ визначається ріжницею в продукті проти $А$, тобто, надміром в
3 квартери; тому рента $= 3 × 60 \deq{} 180$ шил. Але в таблиці III, де ціна \deq{} 30\shil{ шил.}, вона визначається
масою надмірного продукту з землі $D$ порівняно з $А \deq{} 8$ кварт.; але $8 × 30 \deq{} 240$ шил.

Разом з тим відпадає та перша фалшива передумова диференційної ренти, — яка ще панує в Веста,
Мальтуса, Рікардо, нібито диференційна рента неодмінно має за передумову перехід до земель дедалі
гіршої якости, або постійно зменшувану родючість ґрунту. Як ми вже бачили, диференційна рента може
постати при переході до земель дедалі ліпшої якости; вона може постати, коли нижчий ступінь займе
краща земля, замість колишньої гіршої; вона може постати в зв’язку з ростучим поступом хліборобства.
Умовою її виникнення є лише неоднаковість родів землі. Оскільки береться на увагу розвиток
продуктивності,
\parbreak{}  %% абзац продовжується на наступній сторінці
