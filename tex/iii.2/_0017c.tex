\parcont{}  %% абзац починається на попередній сторінці
\index{iii2}{0017}  %% посилання на сторінку оригінального видання
в процесі репродукції, промисловець чи купець, якщо він хоче дисконтувати
вексель або зробити позику, не потребує ані акцій, ані державних паперів. Чого
потребує він, так це грошей. Отже він заставляє або продає ті цінні папери, якщо
йому не сила якось інакше добути ті гроші. Про нагромадження цього позичкового
капіталу й мова тут мовиться, і то власне спеціяльно про нагромадження позичкового
грошового капіталу. Мова тут не про позики домів, машин або іншого
основного капіталу. І не про ті позики мова, що їх промисловці та купці дають
одні одним товарами в межах процесу репродукції; хоч і цей пункт нам ще
доведеться дослідити докладніше; тут мовитиметься лише про ті грошові позики,
що їх роблять банкіри, як посередники, промисловцям та купцям.

\pfbreak

Отже, насамперед проаналізуймо комерційний кредит, тобто кредит, що його
дають один одному капіталісти, які зайняті в процесі репродукції. Він являє
основу кредитової системи. Його представник є вексель, боргова посвідка з визначеним
у ній реченцем платежу, document, of deferred payment. Кожен дає кредит
однією рукою та одержує кредит другою. Лишім насамперед цілком осторонь банкірський
кредит, що становить зовсім інший, істотно відмінний момент. Оскільки ці
векселі обертаються знову серед самих купців як засіб платежу за допомогою
передатних написів одного на одного, але без посередництва дисконту, є це не
що інше, як перенесення боргової вимоги від А на В, і абсолютно нічого не
змінює в цих взаєминах. Одна особа лише заступає місце другої. І навіть
в цьому разі ліквідація може відбутись без посередництва грошей. Напр., прядун
А має оплатити вексель бавовяному маклерові В, а цей останній — імпортерові С.
Коли ж С ще й експортує пряжу, що трапляється доволі часто, то він може
купити пряжу в А на вексель, а прядун А може покрити свій борг маклерові В
його власним векселем, що його С одержав, як плату від В; при цьому щонайбільше
доведеться оплатити сальдо грішми. В цьому разі ціла операція
упосереднює лише обмін бавовни на пряжу. Експортер представляє лише прядуна,
бавовняний маклер — плянтатора бавовни.

Отже в кругообороті цього суто-комерційного кредиту треба зауважити дві речі:

\emph{Поперше.} Виплата сальдо цих взаємних боргових вимог залежить від
зворотного припливу капіталу, тобто від Т — Г, того акту, що його лише відкладено
на певний реченець. Якщо прядун одержав вексель від фабриканта ситцю,
то цей останній може його оплатити тоді, коли ситець, що його він має на
ринку, буде тимчасом продано. Коли спекулянт збіжжям видав вексель на свого
фактора, то фактор може виплатити гроші тоді, коли тимчасом збіжжя буде продано
за сподівані ціни. Отже, ці платежі залежать від поточности репродукції,
тобто від процесу продукції. та споживання. А що кредити взаємні, то й залежить
платоспроможність одного від платоспроможности другого; бо, виставляючи
свій вексель, кожен може рахувати або на зворотний приплив капіталу у своєму
власному підприємстві, або на поворот капіталу у підприємстві третьої особи,
що має йому оплатити вексель у певний час. Коли ж не вважати на сподіваний
зворотний приплив капіталу, то платіж може бути зроблено тільки з запасного
капіталу, що його має в розпорядженні векселедавець на те, щоб виконати свої
зобов’язання, якщо зворотний приплив капіталу затримується.

\emph{Подруге.} Ця кредитова система не усуває потреби грошових платежів
готівкою. Насамперед доводиться чималу частину витрат оплачувати раз-у-раз
готівкою, заробітну плату, податки і~\abbr{т. ін.} А потім, напр., В, що одержав від
С платіж векселем, сам має, раніше, ніж цьому векселеві надійшов реченець,
оплатити D вексель, що йому вже надійшов реченець, і для цього мусить
він мати готівку. Такий досконалий кругооборот репродукції, як отой, що ми
вище припускали, від плянтатора бавовни до прядуна й навпаки, може
\parbreak{}  %% абзац продовжується на наступній сторінці
