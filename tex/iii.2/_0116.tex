\parcont{}  %% абзац починається на попередній сторінці
\index{iii2}{0116}  %% посилання на сторінку оригінального видання
що їх виплачується під титулом земельної ренти власникові землі за використання
ґрунту чи то з метою виробництва, чи то споживання, треба пам’ятати, що
ціна речей, які самі по собі не мають вартости, тобто не є продукти праці, як
земля, або, принаймні, не можуть бути відтворені працею, як старовинні речі,
художні вироби певних майстрів тощо, може визначатися дуже випадковими
комбінаціями.

Щоб продати річ, для цього не треба нічого іншого, як тільки того, щоб
вона могла зробитись об’єктом монополії і відчуження.

\pfbreak{} % see russ. book

Є три головні помилки, що їх при розгляді земельної ренти треба уникати
й що затемнюють аналізу.

1)~Сплутування різних форм ренти, відповідних різним ступеням розвитку
суспільного процесу продукції.

Хоч би яка була специфічна форма ренти, всім її типам є спільне те, що привласнення
ренти є економічна форма, що в ній реалізується земельна власність і що
земельна рента в свою чергу має за свою передумову земельну власність, власність
певних індивідуумів на певні дільниці землі, чи власником буде особа, що репрезентує
громаду як в Азії, Єгипті тощо, чи ця земельна власність буде привхідною
обставиною власности певних осіб на особи безпосередніх продуцентів, як за системи
рабства або кріпацтва, чи ж земельна власність буде суто приватною
власністю непродуцентів на природу, простим титулом власности на землю, чи,
нарешті, це буде таке відношення до землі, що як от у колоністів і дрібноселянських
землевласників, за ізольованої і соціяльно-нерозвиненої праці, виступає,
як відношення безпосередньо дане привласненням і виробництвом продуктів
на певних дільницях землі безпосередніми продуцентами.

Ця \emph{спільність} різних форм ренти — те, що вона являє собою економічну
реалізацію земельної власности, юридичної фікції, в силу якої ріжним індивідуумам
належить виключне володіння певними дільницями землі, — призводить до
того, що ріжниці форм не помічаються.

2)~Всяка земельна рента є додаткова вартість, продукт додаткової праці.
У своїй нерозвиненій формі, у формі натуральної ренти, вона ще є безпосередньо
додатковий продукт. Звідси та помилка, ніби та рента, що відповідає капіталістичному
способові продукції і яка завжди становить надлишок над зиском,
тобто над тією частиною вартости товару, що сама складається з додаткової вартости
(додаткової праці), — ніби ця особлива і специфічна складова частина додаткової
вартости буде пояснена тим, що будуть пояснені загальні умови існування додаткової
вартости і зиску взагалі. Ці умови такі: безпосередні продуценти мусять працювати
понад той час, який потрібен для репродукції їхньої власної робочої
сили, для репродукції їх самих. Вони взагалі мусять виконувати додаткову працю.
Це — суб’єктивна умова. А об’єктивна є в тому, щоб у них була і \emph{можливість}
виконувати додаткову працю; щоб природні умови були такі, щоб лише деякої
\emph{частини} робочого часу, яким вони порядкують, було досить для їхньої репродукції
і самозбереження як продуцентів; щоб продукція потрібних засобів їхнього
існування не забирала всієї їхньої робочої сили. Родючість природи становить
тут одну межу, один вихідний пункт, одну основу. З другого боку, розвиток суспільної
продуктивної сили праці становить тут другу межу. Розглядаючи справу
ще ближче, можна сказати: тому що продукція харчових засобів є найперша умова
життя продуцентів і всякої продукції взагалі, — праця застосована до цієї продукції,
отже, хліборобська праця в найширшому економічному розумінні мусить бути
остільки продуктивна, щоб продукцією харчових засобів для безпосередніх продуцентів
забирався не ввесь робочий час, що вони ним порядкують, отже, щоб була
можлива хліборобська додаткова праця, а тому і хліборобський додатковий продукт.
Розвиваючи далі: треба, щоб уся хліборобська праця — потрібна й додаткова
праця — деякої частини суспільства була достатня для того, щоб продукувати
\parbreak{}  %% абзац продовжується на наступній сторінці
