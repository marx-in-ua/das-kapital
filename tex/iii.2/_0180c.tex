\parcont{}  %% абзац починається на попередній сторінці
\index{iii2}{0180}  %% посилання на сторінку оригінального видання
в такий спосіб, що 2\sfrac{1}{2} акри землі $А$ були б оброблені наново з витратою капіталу
в 2\sfrac{1}{2}\pound{ ф. стерл.} на акр, то витрачений додатковий капітал становив би
лише 6\sfrac{1}{4}\pound{ ф. стерл}, отже, вся витрата на $А$ і $В$ для продукції цих 6 квартерів
становила б лише 11\sfrac{1}{4}\pound{ ф. стерл.} замість 15\pound{ ф. стерл.}, і вся їхня ціна
продукції, включаючи й зиск, становила б 13\sfrac{1}{2}\pound{ ф. стерл}. Ці 6 квартерів, як
і давніш, разом продавалося б за 18\pound{ ф. стерл.}, але витрата капіталу зменшилася
б на 3\sfrac{3}{4}\pound{ ф. стерл.}, і рента з $В$ становила б, як і давніше, 4\sfrac{1}{2}\pound{ ф. стерл.}
на акр. Інакше стояла б справа, коли б для продукції додаткових 2\sfrac{1}{2} квартерів
довелося вжити гіршої, ніж $А$, землі, $А_{-1}$, $А_{-2}$; так що ціна продукції
квартера для 1\sfrac{1}{2} квартера на землі $А_{-1} \deq{} 4$\pound{ ф. стерл.}, а для останнього
квартера на землі $А_{-2} \deq{} 6$\pound{ ф. стерл}. В цьому випадку 6\pound{ ф. стерл.}
зробилися б реґуляційною ціною продукції квартера. 3\sfrac{1}{2} квартери з землі $В$
були б продані за 21\pound{ ф. стерл.} замість 10\sfrac{1}{2}\pound{ ф. стерл.}, що дало б ренту в 15\pound{ ф.
стерл.} замість 4\sfrac{1}{2}\pound{ ф. стерл.}, а в збіжжі ренту в 2\sfrac{1}{2} кв. замість 1\sfrac{1}{2} квартера.
Так само квартер з землі $А$ тепер репрезентував би ренту в 3\pound{ ф. стерл.} \deq{}
\sfrac{1}{2} квартера.

Перше, ніж дослідити цей пункт далі, зробимо ще одне зауваження.

Пересічна ціна квартера з $В$ вирівнюється, збігається з загальною ціною
продукції в 3\pound{ ф. стерл.} з квартера, регульованою землею $А$, скоро частина
всього капіталу, що продукує надмірні 1\sfrac{1}{2} квартера зрівноважиться тією частиною
всього капіталу, яка недопродуковує 1\sfrac{1}{2} квартера. Як скоро вирівнюються
ці ціни, або скільки для цього треба витратити на $В$ капіталу з недостатньою
продуктивною силою, — це залежить, коли припустити додаткову
продуктивність перших витрат капіталу за дану, від відносно недостатньої продуктивности
наступних витрат капіталу проти продуктивности рівновеликої витрати
капіталу на найгіршій регуляційній землі $А$, або від індивідуальної ціни
продукції їхнього продукту порівняно з реґуляційною ціною.

\pfbreak

З попереднього насамперед випливає:

\emph{Перше.} Доти, доки додаткові капітали витрачаються на тій самій землі
з додатковою продуктивністю, хоча б і низхідною, абсолютно рента з акра
так збіжжева, як і грошова, зростає, хоч відносно проти авансованого капіталу
(отже, норма надзиску або ренти) вона зменшується. За межу тут є той додатковий
капітал, що дає лише пересічний зиск, або для продукту якого індивідуальна
ціна продукції збігається з загальною ціною продукції. Ціна продукції
за цих умов залишається та сама, коли тільки продукція на гірших землях
не стає зайвою в наслідок збільшеного подання. Навіть за низхідних цін ці
додаткові капітали можуть у певних межах все ще продукувати надзиск, хоч
і менш значний.

\emph{Друге.} Витрата додаткового капіталу, що продукує тільки пересічний
зиск і додаткова продуктивність якого, отже \deq{} 0, нічого не змінює у висоті
створеного надзиску, а тому і ренти. Індивідуальна пересічна ціна квартера на
кращих землях в наслідок цього зростає. Надмір з квартера зменшується, але
число квартерів, що дають такий зменшений надмір, збільшується, так що здобуток
лишається той самий.

\emph{Третє.} Додаткові витрати капіталу, які дають продукт, що його індивідуальна
ціна продукції перевищує регуляційну ціну, отже, додаткова продуктивність
яких не тільки \deq{} 0, але менша нуля, деякий мінус, тобто нижча,
ніж продуктивність рівної витрати капіталу на регуляційній землі $А$, дедалі
більше наближають індивідуальну пересічну ціну всього продукту з кращих
земель до загальної ціни продукції, отже, дедалі більше зменшують ріжницю
між обома, ту ріжницю, яка створює надзиск, зглядно ренту. Дедалі
більша частина того, що раніше становило надзиск або ренту, входить в
\parbreak{}  %% абзац продовжується на наступній сторінці
