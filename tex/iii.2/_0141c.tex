\parcont{}  %% абзац починається на попередній сторінці
\index{iii2}{0141}  %% посилання на сторінку оригінального видання
не дають надзиску, тимчасом як їхні інші частини дають надзиск, відповідний
ріжниці між їхнім продуктом і продуктом того вкладення, що не дає ренти.

Надзиски і різні норми надзиску з різних частин вартости капіталу створюються
в обох випадках рівномірно. А рента є не що інше, як форма цього
надзиску, який становить її субстанцію. Але в усякому разі другий спосіб являє
собою труднощі щодо перетворення надзиску в ренту, цієї зміни форми, яка містить
в собі перенесення надзиску від капіталістичного орендаря до земельного власника.
Звідси упертий опір офіційній хліборобській статистиці з боку англійських
орендарів. Звідси боротьба між ними й землевласниками за встановлення дійсних
наслідків приміщення їхніх капіталів (Morton). Рента встановлюється саме
при оренді земель, і після цього надзиски, що постають з послідовного приміщення
капіталу, потрапляють в кишеню орендаря весь час, поки триває орендний
договір. Звідси боротьба орендарів за тривалі орендні договори і навпаки,
в наслідок переваги сили лендлордів, збільшення числа контрактів, які можна
щороку скасовувати (tenancies at will).

\looseness=-1
Тому ясно з самого початку: хоч для закону створення надзиску нічого не
змінюється від того, чи вкладено рівні капітали з різними наслідками один поряд
одного в рівновеликі земельні дільниці, чи вкладено їх послідовно один за одним в ту
саму дільницю землі, — проте, це становить значну ріжницю для перетворення
надзиску в земельну ренту. Останній спосіб замикає це перетворення, з одного
боку, у вужчі, з другого — у мінливіші межі. Тому в країнах інтенсивної культури
(а економічно під інтенсивною культурою ми розуміємо не що інше, як
концентрацію капіталу на тій самій земельній площі, замість розподілу його
між земельними дільницями, що лежать одна біля однієї) праця таксатора, як
це зазначає Morton у своїх «Resources of States», стає дуже важливою, складною
і важкою професією. При триваліших поліпшеннях землі, коли минає термін
орендного договору, штучно підвищена диференційна родючість землі збігається
з її природною, а тому і оцінка розміру ренти збігається з оцінкою розміру
ренти від земель різної родючости взагалі. Навпаки, оскільки створення надзиску
визначається висотою капіталу, вкладеного в продукцію, висота ренти, одержуваної
при певній величині цього капіталу, приєднується до пересічної ренти
країни і тому дбають про те, щоб новий орендар порядкував капіталом, достатнім
для продовження культури з колишнім ступенем інтенсивности.

\plainbreak{3}

При розгляді диференційної ренти II треба відзначити ще такі пункти:

\emph{Поперше}. Її база і вихідний пункт, не тільки історично, але й оскільки
справа йде про рух за всякого даного моменту, є диференційна рента I, тобто
одночасний обробіток розміщених одна поряд однієї земельних дільниць, різних
своєю родючістю і положенням; отже одночасне вживання одної поряд однієї
різних складових частин усього хліборобського капіталу на земельних дільницях
різної якости.

Історично це само собою зрозуміло. В колоніях колоністам доводиться прикладати
лише незначний капітал; за головних аґентів продукції є праця і земля.
Кожен окремий голова родини намагається добитися для себе і своїх самостійного
поля дії, поряд з товаришами-колоністами. У власне хліборобстві це
взагалі мусило так відбуватися вже за докапіталістичних способів продукції. При
вівчарстві і взагалі скотарстві як самостійних галузях продукції земля експлуатується
більш або менш спільно, і з самого початку експлуатація має екстенсивний
характер. Капіталістичний спосіб продукції походить з давніших способів
продукції, за яких засоби продукції, фактично або юридично, становлять
власність самого обробника, словом, з ремісничої продукції в хліборобств. По
суті справи з ремісничої продукції лише поступово розвивається концентрація
засобів продукції і перетворення їх на капітал, що протистоїть безпосереднім
\parbreak{}  %% абзац продовжується на наступній сторінці
