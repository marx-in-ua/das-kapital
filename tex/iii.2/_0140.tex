\parcont{}  %% абзац починається на попередній сторінці
\index{iii2}{0140}  %% посилання на сторінку оригінального видання
ніж та, яка оброблялася до того часу, — на різних родах землі, починаючи з $А$
і до $D$, отже, наприклад, обробіток більших площ $В$ і $C$, зовсім не має за свою передумову
попереднього підвищення цін хліба, подібно до того, як щорічне поширення,
наприклад, бавовнопрядіння не потребує постійного підвищення цін пряжі. Хоч
значне підвищення або пониження ринкових цін впливає на розмір продукції,
проте, залишаючи це осторонь, і при пересічних цінах, що своїм рівнем не справляють
на продукцію ні пригнобного, ні особливо оживного впливу, у хліборобстві
(як і в усіх інших галузях продукції, проваджених капіталістично) постійно
відбувається та відносна перепродукція, яка сама по собі тотожня з
акумуляцією і яка, за інших способів продукції, безпосередньо спричинюється
зростом людности, а в колоніях — постійною іміграцією. Попит постійно зростає,
і передбачаючи це, постійно вкладають в нові землі все нові й нові капітали;
хоч, залежно від обставин, капітали вкладають на створення різних хліборобських
продуктів. До цього само по собі призводить наростання нових капіталів.
Щодо окремого капіталіста, то розміри своєї продукції він припасовує до розміру
капіталу, що він ним порядкує, оскільки сам він може ще його контролювати.
Він прагне лише того, щоб захопити якомога більше місця на ринку.
Коли настає перепродукція, то він обвинувачує в цьому не себе, а своїх конкурентів.
Окремий капіталіст може розширювати свою продукцію так привласнюючи
собі порівняно більшу відповідну частину даного ринку, як і розширюючи самий
ринок.

\section{Друга форма диференційної ренти (диференційна рента II)}

До цього часу ми розглядали диференційну ренту лише як наслідок різної
продуктивности однакових капіталовкладень на однакових площах землі з різною
родючістю, так що диференційна рента визначалась ріжницею між продуктом
капіталу, вкладеного в найгіршу землю, що не дає ренти, і продуктом капіталу,
вкладеного в кращу землю. При цьому ми мали одночасне приміщення капіталів
в різні дільниці землі, так що кожному новому приміщенню капіталу відповідало
поширення обробітку землі, збільшення оброблюваної площі. Але, кінець-кінцем,
диференційна рента по суті справи була лише наслідком різної
продуктивности рівних капіталів, вкладених в землю. Чи буде будь-яка ріжниця,
коли капітали різної продуктивности вкладаються один після одного в ту
саму дільницю землі, і коли вони вкладаються один поряд одного в різні дільниці
землі — чи буде якась ріжниця, коли тільки припустити, що наслідки ті самі?

Насамперед, не можна заперечувати, що, оскільки справа йде про створення
надзиску, цілком байдуже, чи дадуть 3\pound{ ф. ст.} ціни продукції, вкладені в акр
землі $А$, продукт в 1 квартер, так що 3\pound{ ф. ст.} будуть ціною продукції і реґуляційною
ринковою ціною одного квартера, тоді як 3\pound{ ф. ст.} ціни продукції на акрі землі
$В$ дадуть 2 квартери і, таким чином, надзиск в 3\pound{ ф. ст.}, а 3\pound{ ф. ст.} ціни продукції
на акрі землі $C$ дадуть 3 квартери і 6\pound{ ф. ст.} надзиску і, нарешті, 3\pound{ ф. ст.} ціни
продукції на акрі землі $D$ дадуть 4 квартери і 9\pound{ ф. ст.} надзиску; чи такий самий
наслідок буде досягнутий тим, що ці 12\pound{ ф. ст.} ціни продукції, або 10\pound{ ф. ст.},
капіталу будуть вкладені з таким самим успіхом, в такій самій послідовності,
в один і той самий акр. І в тому, і в тому випадку це капітал в 10\pound{ ф. ст.}
що частини його вартости, послідовно вкладені по 2\sfrac{1}{2}\pound{ ф. ст.}, — однаково, чи вкладаються
вони один поряд одного на 4 акрах землі різної родючости, чи послідовно на
тому самому акрі, — в наслідок того, що продукт їхній різний, однією своєю частиною
\parbreak{}  %% абзац продовжується на наступній сторінці
