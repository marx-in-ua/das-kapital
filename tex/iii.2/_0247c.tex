\parcont{}  %% абзац починається на попередній сторінці
\index{iii2}{0247}  %% посилання на сторінку оригінального видання
продукції. Незрівняно більша частина продуктів, що становлять сталий капітал
перебуває також у такій речевій формі, в якій вони не можуть увійти
в індивідуальне споживання. Оскільки це для неї можливе, оскільки, наприклад,
селянин може з’їсти своє збіжжя, призначене на насіння, або зарізати свою робочу
худобу, економічне обмеження щодо цієї частини призводить цілком до того
самого, як коли б ця частина існувала в формі непридатній для споживання.

Як уже сказано, при розгляді обох кляс, ми залишаємо осторонь основну
частину сталого капіталу, яка продовжує існувати in natura і за своєю вартістю,
незалежно від річного продукту обох кляс.

У клясі II, що на її продукти витрачається заробітну плату, зиск і ренту,
коротко кажучи, споживаються доходи, продукт за його вартістю в свою чергу
складається з трьох складових частин. Одна складова частина дорівнює вартості
зужиткованої в продукції сталої частини капіталу; друга складова частина дорівнює
вартості тієї авансованої на продукцію змінної частини капіталу, яка
витрачена на заробітну плату; нарешті, третя складова частина дорівнює випродукованій додатковій
вартості, отже, \deq{} зискові \dplus{} рента. Перша складова частина
продукту кляси II, вартість сталої частини капіталу, не може бути спожита
ні капіталістами, ні робітниками кляси II, ні земельними власниками. Вона не
становить частини їхнього доходу, а мусить бути покрита in natura, а щоб це
могло статись, вона мусить бути продана. Навпаки, дві інші складові частини
цього продукту дорівнюють вартості випродукованих у цій клясі доходів, \deq{} заробітній
платі \dplus{} зиск \dplus{} рента.

У клясі І продукт складається за формою з таких самих складових частин.
Але ту частину, яка становить тут дохід, заробітна плата \dplus{} зиск \dplus{} рента, коротко,
змінну частину капіталу \dplus{} додаткову вартість, споживається тут не
в натуральній формі продуктів цієї кляси І, а в продуктах кляси II.~Отже, вартість
доходів кляси І мусить бути спожита в натуральній формі тієї частини
продукту кляси II, яка становить належний покриттю сталий капітал цієї
кляси II.~Частина продукту кляси II, що мусить покрити сталий капітал цієї
кляси II, споживається в її натуральній формі робітниками, капіталістами та
земельними власниками кляси І.~Вони витрачають свої доходи на цей продукт II.~З другого боку, продукт кляси І, оскільки він становить дохід кляси І, продуктивно
споживається в його натуральній формі клясою II, що її сталий капітал
він покриває in natura. Нарешті, зужиткована частина сталого капіталу кляси
І покривається власними продуктами цієї кляси, які складаються саме з засобів
праці, сирових і допоміжних матеріялів тощо, почасти покривається за посередництвом
обміну капіталістів І між собою, почасти тим, що частина цих капіталістів
може безпосередньо застосувати свій власний продукт, як засіб продукції.

Візьмімо давнішу схему (книга II, розділ XX, II) простої репродукції:

\begin{center}
\[
 \left.\begin{aligned}
        \text{I. }\num{4.000} c \dplus{} \num{1.000} v \dplus{} \num{1.000} m \deq{} \num{6.000}\\
        \text{II. }\num{2.000} c \dplus{} \phantom{0.}500 v \dplus{} \phantom{0.}500 m \deq{} \num{3.000}
       \end{aligned}
 \right\}
  \deq{} \num{9.000}
\]
\end{center}
За нею в II продуценти та земельні власники споживають як дохід
$500 v \dplus{} 500 m \deq{} \num{1.000}$; залишається покрити $\num{2.000} c$. Не споживають
робітники, капіталісти та одержувачі ренти кляси І, що їхній дохід \deq{}
$\num{1.000} v \dplus{} \num{1.000} m \deq{} \num{2.000}$. Спожитий продукт кляси II споживається як дохід
клясою І, а частина доходу кляси І, представлена в неспоживному продукті,
споживається як сталий капітал в клясі II.~Отже, лишається дати звіт про
$\num{4.000} c$ кляси І.~Це покривається з власного продукту кляси І \deq{} \num{6.000}, або радше
$= \num{6.000} — \num{2.000}$, бо ці \num{2.000} вже перетворені на сталий капітал для кляси II.~Слід
зауважити, що числа звичайно взято довільно, отже, і відношення між вартістю
доходу І і вартістю сталого капіталу II є довільне. А проте, очевидно, що коли
процес репродукції відбувається нормально і за інших незмінних умов, тобто,
\parbreak{}  %% абзац продовжується на наступній сторінці
