\parcont{}  %% абзац починається на попередній сторінці
\index{iii2}{0009}  %% посилання на сторінку оригінального видання
й робочу силу — за капітал, що дає цей процент. Якщо, напр., заробітна плата
одного року = 50 ф. ст., а рівень проценту є 5\%, то річну робочу силу вважається
за рівну капіталові в 1000 ф. ст. Безглуздість капіталістичного способу
уявлення сягає тут своєї найвищої точки, бо замість пояснювати зріст вартости
капіталу визиском робочої сили, продуктивність робочої сили пояснюється, навпаки,
тим, що сама ця робоча сила є містична річ, капітал, що дає процент.
У другій половині XVII віку (напр., у Петті) це було улюбленим уявленням,
але й сьогодні його цілком серйозно додержуються почасти вульґарні економісти,
а почасти й насамперед німецькі статистики\footnote{
«Робітник мав капітальну вартість, що ії обчислюють, розглядаючи грошову вартість його
річного заробітку, як процент з капіталу\dots{} Коли\dots{} капіталізувати пересічну поденну заробітну
плату з 4\%, то ми одержимо, як пересічну вартість одного сільсько-господарського робітника чоловічої
статі: в німецькій Австрії 1500 талерів, в Прусії 1500, в Англії 3750, у Франції 2000, у
центральній Росії 750 талерів». (Von Reden, Vergleichende. Kulturstatistik, Berlin 1848, p. 134).
}. На жаль тут з’являються дві
обставини, що неприємно розбивають це безглузде уявлення, — поперше, та, що
робітник мусить працювати, щоб одержати цей процент, а, подруге, та, що він
не може перетворити капітальну вартість своєї робочої сили на готівку, передавши
власність на неї. Ще більше, річна вартість його робочої сили є рівна
його річній пересічній заробітній платі, а те, що він своєю працею має повернути
покупцеві цієї сили, становить саме цю вартість плюс додаткову вартість,
приріст її. При системі невільництва робітник має капітальну вартість, а саме
свою купівну ціну. І коли його наймають, то наймач має, поперше, платити
процент від його купівної ціни та, окрім того, повертати річне зношування
того капіталу.

Утворення фіктивного капіталу звуть капіталізуванням. Капіталізують кожен
дохід, що реґулярно повторюється, обчислюючи його за пересічним рівнем проценту,
яв дохід, що його давав би капітал, визичений за такий процент; напр.,
коли річний дохід = 100 ф. ст., а рівень проценту = 5\%, то ці 100 ф. ст.
становили б річний процент від 2000 ф. ст., і ці 2000 ф. ст. вважається отже
за капітальну вартість юридичного титулу власности на ці таки 100 ф. ст.
річно. Для того, хто купив цей титул власности, ці 100 ф. ст. річного доходу
становитимуть в цьому разі дійсно реалізацію його капіталу, як капіталу, приміщеного
з 5\%. Отже тут губиться останній слід всякого зв’язку з дійсним
процесом зростання вартости капіталу та зміцнюється уявлення про капітал, що
дає процент, як про автомат, що самозростає у своїй вартісті.

Навіть там, де боргова посвідка — цінний папер — не становить суто-ілюзорного
капіталу, як от в державних боргах, капітальна вартість цього
паперу є суто-ілюзорна. Ми бачили раніше, як кредитова справа утворює асоційований
капітал. Папери вважається за титули власности, що представляють
цей капітал. Акції залізничих, гірничих, пароплавних і~\abbr{т. ін.} товариств представляють
дійсний капітал, а саме капітал, приміщений та діющий в цих
підприємствах, або грошову суму, що її авансували спільники, щоб витратити
її в таких підприємствах як капітал. При цьому ніяк не виключається й те,
що ці акції є просте шахрайство. Але цей капітал не існує подвійно, одного
разу як капітальна вартість титулу власности, акцій, а другого разу як капітал,
що вже дійсно приміщений або що його має бути приміщено в ті підприємства.
Він існує лише в тій останній формі, і акція є не що інше, а
тільки титул власности, pro rata, на додаткову вартість, що її мається тим
титулом зреалізувати. $А$ може продати цей титул $В$, а $В$ — $C$. Ці операції ані
трохи не змінюють суті справи. $А$ або $В$ в такому випадку перетворив свій
титул в капітал, а $C$ свій капітал — у простий титул власности на додаткову
вартість, що її сподіваються одержати від акційного капіталу.
