
Для зручности порівняння поновимо насамперед таку таблицю:

\begin{table}[H]
  \centering
  \caption*{Таблиця І}
  \footnotesize

  \settowidth\rotheadsize{\theadfont Продажна}
  \begin{tabular}{l c r c c c c c c}
    \toprule
      \thead[tl]{Земля} &
      &
      \rothead{Капітал} &
      \rothead{Зиск} &
      \rothead{Ціна\\продукції} &
      \rothead{Продукт} &
      \multicolumn{2}{c}{Рента} &
      \rothead{Норма\\надзиску} \\

      \cmidrule(rl){2-9}

       & акри  & \makecell{\poundsign{}} & \poundsign{} & кв. & кв. & кв. & \poundsign{}  & \% \\
      \midrule

       A & 1 & 2\tbfrac{1}{2} & \tbfrac{1}{2} & 3\phantom{\tbfrac{1}{2}} & \phantom{0}1 & 0 & \phantom{0}0 & \phantom{00}0 \\
       B & 1 & 2\tbfrac{1}{2} & \tbfrac{1}{2} & 1\tbfrac{1}{2}           & \phantom{0}2 & 1 & \phantom{0}3 & 120 \\
       C & 1 & 2\tbfrac{1}{2} & \tbfrac{1}{2} & 1\phantom{\tbfrac{1}{2}} & \phantom{0}3 & 2 & \phantom{0}6 & 240\\
       D & 1 & 2\tbfrac{1}{2} & \tbfrac{1}{2} & \phantom{0}\tbfrac{3}{4} & \phantom{0}4 & 3 & \phantom{0}9 & 360\\
    \midrule
      Разом & 4 & \hang{r}{1}0\pF & & & 10 & 6 & 18 & \makecell[t]{180 \\ пересічно}\\
  \end{tabular}

\end{table}

\noindent{}Коли ми тепер припустимо, що цифра 16 квартерів, що їх даватимуть землі
$В$, $C$, $D$ за низхідної норми продуктивности, достатня для того, щоб вилучити
$А$ з числа оброблюваних земель, то таблиця III перетворюється на таку:

\begin{table}[H]
  \centering
  \caption*{Таблиця V}
  \footnotesize

  \settowidth\rotheadsize{\theadfont Продажна}
  \begin{tabular}{l c r c r c c c c c}
  \toprule

\thead[tl]{Земля} &
&
\thead[t]{Вкладення \\ капіталу} &
\rothead{Зиск} &
\thead[t]{Продукт} &
\rothead{Продажна\\ціна} &
\rothead{Здобуток} &
\multicolumn{2}{c}{Рента} &
\rothead{Норма\\надзиску} \\

  \cmidrule(rl){2-10}
  & акри  & \poundsign{} & \poundsign{} & кв. & \poundsign{} & \poundsign{} & кв. & \poundsign{}  & \% \\
  \midrule


       B & 1 & 2\tbfrac{1}{2} \dplus{} 2\tbfrac{1}{2} & 1 & 2 \dplus{} 1\tbfrac{1}{2} \deq{} 3\tbfrac{1}{2} 
            & 1\tbfrac{5}{7} & \phantom{0}6\phantom{\tbfrac{1}{2}} & 0\phantom{\tbfrac{1}{2}} & 0\phantom{\tbfrac{1}{2}} & \phantom{00}0\\
       C & 1 & 2\tbfrac{1}{2} \dplus{} 2\tbfrac{1}{2} & 1 & 3 \dplus{} 2\phantom{\tbfrac{1}{2}} \deq{} 5\phantom{\tbfrac{1}{2}} 
            & 1\tbfrac{5}{7} & \phantom{0}8\tbfrac{4}{7}           & 1\tbfrac{1}{2}           & 2\tbfrac{4}{7}           & \phantom{0}51\hang{l}{\tbfrac{2}{5}}\\
       D & 1 & 2\tbfrac{1}{2} \dplus{} 2\tbfrac{1}{2} & 1 & 4 \dplus{} 3\tbfrac{1}{2} \deq{} 7\tbfrac{1}{2}
            & 1\tbfrac{5}{7} & 12\tbfrac{6}{7}                     & 4\phantom{\tbfrac{1}{2}} & 6\tbfrac{6}{7}           & 137\hang{l}{\tbfrac{1}{5}}\\  

      \midrule

      Разом & 3 & 15 & &  \phantom{2 \dplus{} 1\sfrac{1}{2} \deq{}}16\phantom{\sfrac{1}{2}} & & 27\sfrac{3}{7} & 5\sfrac{1}{2} & 9\sfrac{3}{7} & \makecell[t]{94\hang{l}{\sfrac{3}{10}} \\ пересічно\footnotemarkZ{}}\\
  \end{tabular}
\end{table}
\footnotetextZ{Тут пересічну норму надзиску обчислено не до всього вкладеного капіталу, а тільки до капіталу, вкладеного в рентодайні дільниці $C$ і $D$. \Red{Прим. Ред.}}

\noindent{}Тут за низхідної норми продуктивности додаткових капіталів і за різного
ступеня цього зменшення на різних землях, реґуляційна ціна продукції знизилася
з 3\pound{ ф. стерл.} до 1\sfrac{5}{7}\pound{ ф. стерл}. Вкладення капіталу збільшилося наполовину з 10\pound{ ф.
стерл.} до 15\pound{ ф. стерл}. Грошова рента зменшилася майже удвоє, з 18 до 9\sfrac{3}{7}\pound{ ф.
стерл.}, але збіжжева рента лише на \sfrac{1}{2},
% REMOVED \footnote*{
% В німецькому тексті тут стоїть «\sfrac{1}{22}». Очевидна помилка. \emph{Прим. Ред.}}
з 6 квартерів до 5\sfrac{1}{2}. Весь продукт
збільшився з 10 до 16, або на 60\%.
% REMOVED \footnote*{
% В німецькому тексті тут помилково стоїть: «160\%». \emph{Прим. Ред.}}
Збіжжева рента становить небагато більше
від третини всього продукту. Авансований капітал відноситься до грошової ренти
як $15 : 9\sfrac{3}{7}$, тимчасом як давніш це відношення було $10:18$.

\paragraph{За висхідної норми продуктивности додаткових капіталів.}

Цей випадок відрізняється від варіянту І, наведеного на початку цього
розділу, де ціна продукції за незмінної норми продуктивности знижується, тільки
тим, що коли потрібна додаткова кількість продукту для того, щоб вилучити
землю $А$, то це відбувається тут швидше.

Так за низхідної, як і за висхідної продуктивности додаткових вкладень
капіталу може це різно впливати, залежно від того, як ці вкладення розподіляються
між різними родами землі. В міру того, як цей різний вплив
\parbreak{}  %% абзац продовжується на наступній сторінці
