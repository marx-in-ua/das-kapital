\parcont{}  %% абзац починається на попередній сторінці
\index{iii2}{0105}  %% посилання на сторінку оригінального видання
собі за передмову, з одного боку, визволення безпосереднього продуцента з стану
простої приналежности до землі (в формі підвладного, кріпака, невільника і~\abbr{т. ін.})
а, з другого боку, експропріяцію землі в маси народу.

В цьому розумінні монополія на земельну власність є історична передумова й
лишається постійною основою капіталістичного способу продукції, як і всіх попередніх
способів продукції, що спираються на визиск мас в тій або іншій
формі. Але та форма земельної власности, що її знаходить капіталістичний
спосіб продукції на початку свого розвитку, не відповідає йому. Форму, що йому
відповідає, утворює лише він сам, підпорядковуючи хліборобство капіталові:
а тому й февдальна земельна власність, власність клану, або дрібна селянська
власність з громадою марки, хоч і які різні їхні юридичні форми,
перетворюються на економічну форму, що відповідає цьому способові продукції.
Одним з великих результатів капіталістичного способу продукції є те, що, з одного
боку, він перетворює хліборобство з простої емпіричної та механічної
традиційної методи найнерозвинутішої частини суспільства на свідомий науковий
ужиток аґрономії, оскільки це взагалі можливо серед умов, даних приватною
власністю\footnote{
Цілком консервативні аґрикультурні хеміки, як от, напр., В.~Johnston, визнають, що дійсно
раціональне хліборобство скрізь надибує непереможні межі в приватній власності. Те саме визнають
письменники, оборонці ex professe монополії приватної власности на землю, як от напр., пан Charles
Comte у двотомній праці, що має собі за спеціяльну мету боронити приватну власність. «Народ» — каже
він — «не може досягнути того ступеня добробуту та сили, що визначається його природою, якщо
кожна частина тієї землі, що його годує, не одержить призначення, найбільш згідного з загальним
інтересом. Щоб значно розвинути свої багатства, мусила б по змозі єдина та передусім освічена воля
взяти до своїх рук розпорядок над кожним окремим кавалком своєї території та кожний кавалок зуживати
так, щоб тим допомагати поспіхові всіх інших. Але існування такої волі\dots{} не сила було б погодити
з поділом землі на приватні земельні ділянки\dots{} та з даною кожному власникові змогою майже абсолютно
порядкувати своїм майном». — Johnston, Comte і~\abbr{т. д.}, розглядаючи суперечність між власністю та
раціональною аґрономією, мають на оці тільки потребу обробляти землю певної країни як одну цілість.
Але залежність культури окремих продуктів землі від коливань ринкових цін, та невпинна зміна цієї
культури разом з тими коливаннями цін, увесь дух капіталістичної продукції, що простує до
безпосереднього
найближчого грошового зиску, — це все стоїть у суперечності до хліборобства, що йому
доводиться господарювати серед сукупних постійних життєвих умов послідовних різних людських
ґенерацій.
Яскравий приклад цього є ліси, що ними іноді господарюють до певної міри в дусі громадських
інтересів
тільки там, де ті ліси не становлять приватної власности, а підлягають державному управлінню.
}; що, з одного боку, він цілком звільняє земельну власність від
відносин панування та підлеглости, а з другого боку, цілком відлучає землю
як умову праці від земельної власности та земельного власника, для якого та
земля не становить нічого більше, крім певного грошового податку, що його
він бере від промислового капіталіста фармера за посередництвом своєї монополії:
остільки рве цей зв’язок земельного власника з землею, що земельний власник
цілий свій вік може прожити в Костянтинополі, дарма що його земельна власність
буде в Шотландії. Отак земельна власність одержує свою суто-економічну
форму, скидаючи з себе всі свої попередні політичні й соціяльні лямівки та
зв’язки, коротко — всі ті традиційні додатки, що їх, як ми пізніше побачимо, сами
капіталісти й їхні теоретичні проводирі в запалі своєї боротьби з земельною
власністю проголосили некорисною та безглуздою надмірністю. З одного боку,
раціоналізація хліборобства, що вперше дає змогу провадити його на суспільних
основах, з другого боку, доведення земельної власности до абсурду, — це великі
заслуги капіталістичного способу продукції. Як і всі свої інші історичні кроки
поступу, так само й ці, купив він передусім ціною повного зубожіння безпосередніх
продуцентів.

Раніше, ніж перейти до самої теми, треба зробити ще кілька попередніх
уваг, щоб уникнути непорозумінь.

Отже, передумова капіталістичного способу продукції така: дійсні хлібороби
— то наймані робітники, що мають працю від капіталіста, фармера, який
\parbreak{}  %% абзац продовжується на наступній сторінці
