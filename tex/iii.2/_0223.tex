\parcont{}  %% абзац починається на попередній сторінці
\index{iii2}{0223}  %% посилання на сторінку оригінального видання
цього зиску визначав ренту, а навпаки, сам він визначається рентою як своєю
межею. Висока норма зиску за середньовіччя завдячує своєю висотою не тільки
низькому складові капіталу, що в ньому переважає змінний, витрачуваний на
заробітну плату елемент. Вона завдячує своєю висотою гнобленню села, привласненню
частини ренти земельного власника і доходу його підлеглих. Коли за
середньовіччя село визискує місто політично, всюди де февдалізм не був зламаний
виключним розвитком міст, як в Італії, то місто всюди і без винятків
визискує село економічно своїми монопольними цінами, своєю системою податків,
своїм цеховим ладом, своїм безпосереднім купецьким обманом і своїм
лихварством.

Можна було б думати, що проста поява капіталістичного орендаря в сільсько\dash{}господарській
продукції дає доказ того, що ціна хліборобських продуктів,
які здавна в тій чи іншій формі виплачували ренту, мусить стояти вище, ніж
ціни продукції мануфактури, принаймні, за доби цієї появи; чи тому, що
вона досягла рівня монопольної ціни, чи тому, що вона підвищилась до
рівня вартости хліборобських продуктів, а їхня вартість в дійсності вища за
ціну продукції, реґульовану пересічним зиском. Бо, коли б цього не було, то
капіталістичний орендар за наявних цін хліборобських продуктів не міг би
спочатку реалізувати з ціни цих продуктів пересічний зиск, а потім з цієї
самої ціни ще виплатити в формі ренти надмір над цим зиском. З цього можна
було б зробити той висновок, що загальна норма зиску, яка характеризує капіталістичного
орендаря в його контракті з земельним власником, створилась без
долучення ренти і тому вона, починаючи відігравати реґуляційну ролю в сільському
господарстві, знаходить цей надмір готовим і виплачує його земельному
власникові. Таким традиційним способом пояснює собі справу, наприклад,
п. Родбертус. Але:

\emph{Поперше}. Цей вступ капіталу як самостійної і керівної сили в хліборобство
відбувається не разом і не всюди, а поступово і в окремих галузях продукції.
Він захоплює спочатку не власне хліборобство, а такі галузі продукції,
як скотарство, особливо вівчарство, що його головний продукт, вовна, з піднесенням
промисловости дає спочатку сталий надмір ринкової ціни над ціною
продукції, причому ці ціни лише згодом вирівнюються. Так було в Англії протягом
XVI століття.

\emph{Подруге}. Тому що ця капіталістична продукція спочатку постає лише
спорадично, то нічого не можна заперечити проти припущення, що вона спочатку
опановує лише такі комплекси земель, які в наслідок своєї специфічної
родючости, або в наслідок особливо сприятливого положення, в цілому можуть
виплачувати диференційну ренту.

\emph{Потретє}. Припустімо навіть, що ціни хліборобського продукту при
появі цього способу продукції, — що в дійсності припускає зріст значіння міського
попиту, — були вищі, ніж ціна продукції, як це без усякого сумніву було,
наприклад, за останньої третини XVII століття в Англії, то тоді, — скоро цей
спосіб продукції до певної міри виб’ється з простого упідлеглення хліборобства
капіталові, і скоро постане доконечне зв’язане з його розвитком поліпшення
в хліборобстві і пониження витрат продукції, — відбудеться процес вирівняння
в наслідок реакції, певного пониження ціни хліборобських продуктів, як це
було в першій половині XVIII століття в Англії.

Отже, цим традиційним способом не можна пояснити ренту, як надмір
над пересічним зиском. Хоч би в яких історично даних умовах рента з’явилась
спочатку, — скоро тільки вона пустила коріння, — вона може існувати вже
лише в вище викладених сучасних умовах.

На закінчення, щодо перетворення ренти продуктами на грошову ренту,
слід ще зауважити, що разом з цим стає істотним моментом капіталізована
\parbreak{}  %% абзац продовжується на наступній сторінці
