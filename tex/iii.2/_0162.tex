\parcont{}  %% абзац починається на попередній сторінці
\index{iii2}{0162}  %% посилання на сторінку оригінального видання
меншім від того, що став тепер пересічним, муситиме продавати продукти
нижче від своєї індивідуальної ціни продукції, отже, з зиском нижчим за
пересічний.

Це саме відбувається і за низхідної ціни продукції, навіть за низхідної продуктивности
додаткового капіталу, скоро лише в наслідок збільшеної витрати капіталу
весь потрібний продукт постачатимуть кращі землі, і, отже, застосовуваний капітал
буде вилучений, наприклад, з землі $А$, так що $А$ перестає конкурувати в продукції
цього певного продукту, наприклад, пшениці. Та кількість капіталу, яка
тепер пересічно вживається на кращій землі $В$, що зробилася реґуляційною
землею, стає тепер нормальною; і коли говориться про різну родючість земельних
дільниць, то припускається, що на акр вживається така нова нормальна
кількість капіталу.

З другого боку ясно, що цей пересічний розмір вкладуваного капіталу, як,
наприклад, в Англії 8\pound{ ф. стерл.} на акр до 1848 року і 12\pound{ ф. стерл.} після 1848 року, —
становить маштаб при складанні орендних договорів. Для орендаря, що витрачає
більше, надзиск, поки триває орендний договір, не перетворюється на ренту. Чи
станеться це по закінченні орендного договору, залежатиме від конкуренції орендарів,
які можуть робити таке саме надзвичайне авансування. При цьому не мається
на увазі перманентних поліпшень ґрунту, що за однакової або навіть зменшуваної
витрати капіталу продовжують забезпечувати збільшений продукт. Ці поліпшення,
хоч вони і є продуктом капіталу, проте, діють цілком так само, як
ріжниця в природних якостях землі.

Отже, ми бачимо, що при диференційній ренті II значення має такий момент,
який при диференційній ренті І як такій не виявляється, бо остання може
і далі існувати незалежно від будь-якої зміни нормальної витрати капіталу
на акр. Це є, з одного боку, згладжування наслідків різних витрат капіталу
на реґуляційній землі $А$, що продукт з неї виступає тепер просто як нормальний
пересічний продукт з акра. З другого боку, це зміна в нормальному мінімумі
або пересічній величині витрати капіталу на акр, так що ця зміна виступає як
властивість землі. Нарешті, це — ріжниця способу перетворення надзиску на форму
ренти.

Далі таблиця VI порівняно з таблицею І і II показує, що збіжжева рента
проти І більше ніж подвоїлась, проти II збільшилась на 1\sfrac{1}{5}  квартера; тимчасом
як грошова рента проти І подвоїлась, а проти II не змінилась. Вона значно
зросла б, коли б (за інших рівних умов) більша частина додаткового капіталу
припала на землю кращих родів, або коли б з, другого боку, дія додаткового
капіталу на $А$ була б менш значна і, отже, реґуляційна пересічна ціна квартера
з $А$ була б вища.

Коли б збільшення родючости, що відбувається в наслідок додаткової витрати
капіталу, різно впливало на різних родах землі, то це призвело б до зміни
диференційних рент з цих земель.

В усякому разі доведено, що коли нижчає ціна продукції, в наслідок
підвищення норми продуктивности додаткової витрати капіталу, —
отже, коли ця продуктивність зростає у більшому відношенні, ніж авансований
капітал, — рента з акра, наприклад, при подвоєній витрати капіталу може не
тільки подвоїтись, але й більше, ніж подвоїтись. Але вона може і знизитись,
коли в наслідок швидкого зростання продуктивности землі $А$ ціна продукції
зменшиться ще в значно більшій мірі.

Коли б ми припустили, що додаткові витрати капіталу, наприклад, на землях
$В$ і $C$ збільшили продуктивність не в такій мірі, як на землі $А$, так що для земель
$В$ і $C$ відносні ріжниці зменшуються і приріст продукту не компенсує пониження
ціни, то проти таблиці II рента на $D$ підвищилася б, на $В$ і $C$ знизилася~б.
