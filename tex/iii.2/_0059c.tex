\parcont{}  %% абзац починається на попередній сторінці
\index{iii2}{0059}  %% посилання на сторінку оригінального видання
й тільки тоді, а не раніше, починають вони закупати. Але коли цієї точки досягнено,
курси вже знову уреґульовано, — золото перестають вивозити раніше, ніж досягнено
найнижчої точки спаду цін. Закупи товарів для експорту, можливо, й повернуть
назад частину відправленого за кордон золота, але вони з’являються занадто
пізно для того, щоб стати на заваді тому відпливові» (G. W. Gilbart, An Inquiry
into the Causes of the Pressure on the Money Market. London 1840, p. 37). «Інший
вплив реґулювання засобів циркуляції за допомогою закордонного вексельного
курсу є в тому, що підчас скрути воно приводить до величезного рівня проценту».
(1. с., р. 40). «Витрати, що постають з віднови вексельних курсів,
падають на продуктивну промисловість країни, тимчасом коли протягом цього
процесу зиск Англійського банку позитивно зростає тому, що він провадить
далі свої операції, маючи менше благородного металю» (І. с., р. 52).

Але, каже наш приятель Samuel Gurney, «ці великі коливання рівня
проценту корисні для банкірів і торговців грішми — всякі коливання в справах
корисні для того, хто тямить у справах». І хоч панове Gurneys’и безсоромно збирають
вершки, використовуючи скрутний стан справ, тимчасом як Англійський
банк не може собі дозволити того так само вільно, як вони, проте й йому випадають
гарненькі зиски — не кажучи вже про той приватний зиск, що сам собою падає
до рук панів директорів в наслідок того, що вони мають незвичайну добру
нагоду раз-у-раз знайомитися з загальним станом справ. За відомостями Комісії
лордів з 1817 року ці зиски Англійського банку після того, як відновлено
платежі готівкою, становили за цілий період 1797--1817~\abbr{рр.}:

\begin{center}
\begin{tabular}{l r}
  Bonuses and increased dividends\dotfill{} & 7.451.136\\

  New stock divided among proprietors\dotfill{} & 7.276 500\\

  Increased value of capital\dotfill{} & 14 553.000\\
  \cmidrule(rl){1-2}
  \makecell{Разом} & 29.280.636\\
\end{tabular}
\end{center}

на капітал 11.642.100 ф. ст. за 19 років (D. Hardcastle Banks and Bankers. 2-nd.
ed. London 1843, p. 120). Цінуючи за тим самим принципом загальний бариш
Ірляндського банку, що теж припинив року 1797 платежі готівкою, ми одержим
такий результат:

\begin{center}
\begin{tabular}{l r}
Dividends as by returns due 1821 & 4.736.085\\

Declared bonus\dotfill{} & 1.225.000\\

Increased assets\dotfill{} & 1.214 800\\

Increased value of capital\dotfill{} & 4.185 000\\
\cmidrule(rl){1-2}
\makecell{Разом} & 11.360.885 \\
\end{tabular}
\end{center}

на капітал з 3 міл. ф. ст. (ibidem, р. 163).

А ще кажуть про централізацію! Кредитова система, яка має свій центральний
пункт в так званих національних банках та в великих грошових позикодавцях
і лихварях, що є навколо тих банків, становить величезну централізацію
та дає цій клясі паразитів казкову силу не тільки періодично нищити
промислових капіталістів, але й найнебезпечнішим способом втручатися до дійсної
продукції — і ця банда не знає нічого в продукції та не має з нею нічого до
діла. Акти 1844 та 1845 років — то докази чимраз більшої сили цих бандитів,
до яких прилучаються фінансисти та stock-jobbers’и\footnote*{
Stock-jobber, протилежно до звичайного jobber’a, біржового маклера в широкому розумінні, є
біржовик, який грає на біржі, на цінних державних паперах, акціях тощо. \emph{Пр. Ред}.
}.

Але коли хтось ще має сумнів щодо того, що ці шановні бандити визискують
національну й інтернаціональну продукцію лише в інтересі самої продукції та самих
визискуваних, то того навчить краще оцей малюнок високої моральної гідности,
банкірів: «Банкові заклади — то установи релігійні й моральні. Як часто молодий
купець відвертався від товариства галасливих гультяїв-приятелів, боючися
\parbreak{}  %% абзац продовжується на наступній сторінці
