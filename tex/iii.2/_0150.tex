\parcont{}  %% абзац починається на попередній сторінці
\index{iii2}{0150}  %% посилання на сторінку оригінального видання
додаткові земельні дільниці, що їх різні ступені родючости розподіляються між
$D$ і $А$, між родючістю кращої і гіршої землі. Коли послідовні вкладення капіталу
відбуваються виключно на землі $D$, то вони можуть включати ріжниці, що
існують між $D$ і $А$, далі ріжниці між $D$ і $C$ так само як і ріжниці між $D$ і $В$.
Коли ж усі вони відбуваються на землі $C$, то лише ріжниці між $C$ і $А$ або $В$;
коли на $В$ — то лише ріжниці між $В$ і $А$.

Але закон такий: рента на землях усіх цих родів абсолютно зростає, хоч
і не пропорційно додатково вкладеному капіталові.

Норма надзиску зменшується так у відношенні до додаткового капіталу,
як і у відношенні до всього вкладеного в землю капіталу; але абсолютна величина
надзиску збільшується; цілком так само як зменшення норми надзиску
на капітал взагалі здебільша зв’язане зі збільшенням абсолютної маси зиску.
Так, пересічний надзиск з капіталу вкладеного в $В \deq{} 90\%$ на капітал, тимчасом
як при першому вкладенні капіталу він \deq{} 120\%. Але загальний надзиск
збільшується з 1 квартера до 1\sfrac{1}{2} кв. і з 3\pound{ ф. стерл.} до 4\sfrac{1}{2} . Вся рента,
розглядувана сама по собі, — а не в відношенні до подвоєного розміру авансованого
капіталу — абсолютно зросла. Ріжниці між рентами різних родів землі і
їхнє відношення одна до однієї можуть тут змінюватися; але ця зміна ріжниці
є тут наслідок, а не причина збільшення рент однієї проти однієї.

ІV.~Випадок, коли додаткові вкладення капіталу на кращих землях породжують
більшу кількість продукту, ніж первісні, не потребує дальшої аналізи.
Само собою зрозуміло, що за такого припущення ренти з кожного
акра підвищуються і при тому в більшій пропорції, ніж додатковий капітал,
хоч би в який рід землі він був вкладений. В цьому випадку додаткове
капіталовкладення зв’язано з поліпшенням. Це буває тоді, коли додаткове
вкладення меншого капіталу впливає так само або більш продуктивно, ніж зроблене
давніш додаткове вкладення більшого капіталу. Випадок цей не зовсім тотожній
з давнішим, і ріжниця між ними має важливе значення при всіх
вкладеннях капіталу. Коли, наприклад, 100 дають зиску 10, а 200 при певній
формі вживання — зиск в 40, то зиск збільшується з 10\% до 20\%, і остільки
це є те саме, як коли б 50, при ефективнішій формі вживання, дали зиск в 10
замість 5. Ми припускаємо тут, що зиск є зв’язаний з відповідним збільшенням
продукту. Але ріжниця є в тому, що в одному випадку я мушу подвоїти капітал,
тоді як в другому досягаю подвоєного ефекту при колишньому капіталі.
Зовсім не є те саме, чи продукую я: 1)~колишній продукт, витрачаючи половину
колишньої кількости живої й зрічевленої праці, чи 2)~подвоєний продукт
за колишньої кількости праці, чи 3)~почвірний продукт за подвійної кількости
праці. В першому випадку праця — в живій або зрічевленій формі — стає вільна
і може бути вжита якось інакше; зростає можливість порядкувати працею
і капіталом. Звільнення капіталу (і праці) само по собі є збільшення багатства;
воно впливає цілком так само, як коли б цей додатковий капітал був здобутий
з допомогою акумуляції, але воно заощаджує працю акумуляції.

Припустімо, що капітал в 100 випродукував продукт в 10 метрів. В 100 є так
сталий капітал, як жива праця й зиск. Таким чином, метр коштує 10. Коли я тепер з
таким самим капіталом в 100 можу випродукувати 20 метрів, то метр коштуватиме 5.
Коли навпаки, я можу з капіталом в 50 випродукувати 10 метрів, то метр також
коштуватиме 5, але в цьому випадку звільняється капітал в 50, якщо колишнє подання
товару достатнє. Коли я мушу вкласти капітал в 200, щоб випродукувати
40 метрів, то метр також коштуватиме 5. Визначення вартості або також ціни так
само мало дозволяє помітити тут будь-яку ріжницю, як і маса продукту, що є пропорційна авансованому
капіталові. Але в першому випадку звільняється капітал; у
другому — заощаджується додатковий капітал, коли б треба було приблизно подвоїти
продукцію; в третьому випадку збільшений продукт можна одержати лише тоді, коли
\parbreak{}  %% абзац продовжується на наступній сторінці
