\parcont{}  %% абзац починається на попередній сторінці
\index{iii2}{0259}  %% посилання на сторінку оригінального видання
втім відношенні, як норма додаткової вартости. Ця остання з \frac{150}{100} понижується
\frac{100}{150}, отже з 150\% до 66\sfrac{2}{3}\%, тимчасом як норму зиску понижується
лише з \frac{150}{500} до \frac{100}{550}, або з 30\% до 18 2\sfrac{2}{11}\%. Таким чином, відносне пониження
норми зиску більше, ніж пониження маси додаткової вартости, але менше, ніж
пониження норми додаткової вартости. Далі виявляється, що вартості і маси продуктів
лишаються незмінні, коли вживається ту саму кількість праці, що і давніш,
хоча б авансований капітал в наслідок збільшення його змінної складової частини
і збільшився. Це збільшення авансованого капіталу позначилося б, звичайно,
дуже чутливо на капіталісті, що починає нове підприємство. Але з погляду
репродукції в цілому збільшення змінного капіталу визначає не що інше, а тільки
те, що значніша частина вартости, новоствореної новодолученою працею, перетворюється на заробітну
плату, тобто насамперед на змінний капітал, замість
перетворюватись на додаткову вартість і додатковий продукт. Отже, вартість
продукту лишається незмінна, бо вона обмежена, з одного боку, вартістю
сталого капіталу \deq{} 400, з другого боку — числом 250, що в ньому визначається
новодолучена праця. Але обидві ці величини лишились незмінні. Продукт цей,
оскільки він сам знову входить в сталий капітал, в даній величині вартости
являє ту саму, що й давніш, масу споживної вартости; отже, та сама маса
елементів сталого капіталу зберігає ту саму вартість. Інакше стояла б справа,
коли б заробітна плата підвищилась не тому, що робітник одержував би більшу
частину своєї власної праці, але коли б він одержував більшу частину своєї
власної праці тому, що понизилась продуктивність праці. В цьому випадку сукупна
вартість, що в ній втілюється та сама кількість праці, оплаченої і неоплаченої,
лишилася б незмінна; але маса продукту, що в ній втілюється вся
кількість праці, зменшилася б, і, отже, зросла б ціна кожної даної частини
продукту, бо кожна така частина являла б більшу кількість праці. Підвищена
заробітна плата в 150 являла б не більше продукту, ніж колишня заробітна
плата в 100; понижена додаткова вартість в 100 являла б лише \sfrac{2}{3} колишнього
продукту, 66\sfrac{2}{3}\% тієї маси додаткових вартостей, які давніш визначалися
в 100. В цьому випадку подорожчав би і сталий капітал, оскільки в нього
входить цей продукт. Але це не було б наслідком підвищення заробітної плати;
— навпаки підвищення заробітної плати було б наслідком подорожчання
товарів і наслідком пониженої продуктивности тієї самої кількости праці. Тут
постає ілюзія, ніби підвищення заробітної плати удорожчує продукт; але тут
підвищення це є не причина, а результат зміни вартости товарів в наслідок
пониженої продуктивности праці.

Коли, навпаки, за інших рівних умов, коли, отже, та сама кількість
вжитої праці визначається, як і давніш, в 250, — вартість застосованих нею
засобів продукції підвищиться або знизиться, то й вартість тієї самої маси продуктів
підвищиться або знизиться на ту саму величину. $450 c \dplus{} 100v \dplus{} 150  m$
дає вартість продукту $= 700$; навпаки, $350 c \dplus{} 100 v \dplus{} 150 m$ дає для вартости
тієї самої маси продукту лише 600, замість колишніх 650. Отже, коли зростає
або зменшується авансований капітал, пущений в рух тією самою кількістю
праці, тоді, за інших рівних умов, зростає або зменшується і вартість продукту,
коли це збільшення або зменшення авансового капіталу походить із зміни
величини вартости сталої частини капіталу. Навпаки, вона не змінюється, коли
збільшення або зменшення авансованого капіталу походить із зміни величини
вартости змінної частини капіталу при незмінній продуктивності праці. Збільшення
або зменшення вартости сталого капіталу не компенсується жодним протилежним
рухом. Але збільшення або зменшення вартости змінного капіталу, при незмінній
\parbreak{}  %% абзац продовжується на наступній сторінці
