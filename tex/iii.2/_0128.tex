\parcont{}  %% абзац починається на попередній сторінці
\index{iii2}{0128}  %% посилання на сторінку оригінального видання
досить, так ціна пшениці почала підноситись доти, поки $C$ не набуло змоги
покрити недостачу подання. Тобто ціна мусила піднестись до 20\shil{ шил.} за
квартер. Скоро тільки ціна пшениці піднеслась до 30\shil{ шил.} за квартер, ак
в число оброблюваних земель могла б увійти земля $В$. а якби вона піднеслась
до 60\shil{ шилінґів}, до числа оброблюваних земель могла б увійти й земля $А$, це
не призвело б до того, що на застосований тут капітал довелось би задовольнятися
нормою зиску, нижчою за 20\%. Таким чином, для $D$ створилась би
рента спочатку в 5\shil{ шил.} з квартера \deq{} 20\shil{ шил.} з 4 кв., що тут продукується,
а потім в 45\shil{ шил.} з квартера \deq{} 180\shil{ шил.} з 4 квартерів.

Коли норма зиску з $D$ спочатку також була \deq{} 20\%, то і загальний зиск
з 4 кв. був також лише 10\shil{ шил.}, що проте, при ціні збіжжя в 15\shil{ шил.}, становило
більшу кількість збіжжя, ніж при ціні в 60\shil{ шил.} А що збіжжя входить
у репродукцію робочої сили і частина кожного квартера мусить покривати заробітну
плату, а друга — сталий капітал, то за такого припущення додаткова
вартість була вища, а тому, за інших незміних умов, вища була і норма зиску.
(Справу про норму зиску треба ще дослідити осібно і детальніше).

Коли, навпаки, послідовність була зворотна, коли процес починався з $А$,
то, — якщо довелося б ввести в обробіток нові лани, — ціна квартера спочатку
піднеслась би вище за 60\shil{ шил.}; але тому, що потрібне подання в 2 кварт. постачало
б $В$, то ціна знову понизилась би до 60\shil{ шил.}; хоч $В$ і продукує квартер
за 30\shil{ шилінґів}, проте, продається він за 60, бо його подання вистачало б
якраз тільки для того, щоб покрити попит. Так створилася б рента спочатку в
60\shil{ шил.} для $В$, і таким самим способом для $C$ і $D$, припускаючи завжди, що
ринкова ціна залишається 60\shil{ шил.}, хоч дійсна вартість, по якій $C$ і $D$ дають
квартер пшениці, дорівнює 20 і 15\shil{ шил.}; бо як і давніш потрібно подання одного
квартера, що його постачає $А$, для задоволення загальної потреби. В цьому випадку
підвищення попиту понад ту потребу, яку спочатку задовольняло $А$, потім
$А$ і $В$, могло б привести не до послідовного обробітку $В$, $C$ і $D$, а до поширення
площі обробітку взагалі і можливо, що родючіші землі входили б в обробіток
лише пізніше.

В першому ряді із збільшенням ціни рента стала б підвищуватись, а норма
зиску зменшуватись. Це зменшення могло б цілком або почасти паралізуватися
протидіющими обставинами; на цьому пункті згодом спинимося докладніше.
Не слід забувати, що загальна норма зиску визначається не додатковою вартістю
в усіх сферах продукції рівномірно. Не хліборобський зиск визначає промисловий,
а навпаки. Але про це далі.

У другому ряді норма зиску на витрачений капітал лишилась би та
сама; маса зиску визначилась би в меншій кількості збіжжя; але відносна ціна
його проти інших товарів підвищилась би. Але збільшення зиску, там де воно
відбувається, відокремлюється в формі ренти від зиску, замість того, щоб потрапити
до кишені промислових орендарів і визначитися як зиск, що зростає.
А ціна хліба за такого припущення лишилась би незмінною.

Розвиток і зріст диференційної ренти залишаються однакові так за незмінних
цін, як і за таких, що підвищуються, і так само за безперервного поступу
від гірших земель до кращих, як і за безперервного реґресу від крайніх
до гірших земель.

До цього часу ми вважали: 1) що ціна при одній послідовності підвищується,
при другій — лишається незмінна, і 2) що постійно відбувається перехід
від кращих земель до гірших або навпаки — від гірших до кращих.

Але припустімо, що потреба в хлібі піднялась з первісних 10 до 17 кв.;
далі, що найгірша земля $А$ витиснута другою землею $А$, яка при ціні продукції
в 60\shil{ шил.} (50\shil{ шил.} витрат, плюс 10\shil{ шил.}, що становлять 20\% зиску) дає
1\sfrac{1}{3} кварт., так що ціна продукції одного квартера \deq{} 45\shil{ шил.}; абож припустімо,
\parbreak{}  %% абзац продовжується на наступній сторінці
