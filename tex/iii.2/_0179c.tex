\parcont{}  %% абзац починається на попередній сторінці
\index{iii2}{0179}  %% посилання на сторінку оригінального видання
частку всього витраченого капіталу становить цей менш продуктивний капітал,
а з другого боку, відповідно до зменшення його продуктивности. Пересічна
ціна продукту цього менш продуктивного капіталу все ще була б нижча
від регуляційної ціни, і тому все ще лишався б надзиск, який міг би перетворитися
на ренту.

Припустімо тепер, що пересічна ціна квартера з землі $В$ збігається з загальною
ціною продукції, в наслідок чотирьох послідовних витрат капіталу
(2\sfrac{1}{2}, 2\sfrac{1}{2}, 5 і 5\pound{ ф. ст.}) з низхідною продуктивністю:

\begin{table}[H]
  \begin{center}
    \footnotesize

  \begin{tabular}{c@{  } c@{  } c@{  } c@{  } c@{  } c@{  } c@{  } c@{  } c@{  } c@{  } c}
    \toprule
      \multicolumn{2}{c}{Капітал} &
      Зиск &
      Здобуток &
      \multicolumn{2}{c}{Ціна продукції} &
      \makecell{Продажна \\ ціна} &
      Здобуток &
      \multicolumn{2}{c}{Додаток для ренти} \\

      \cmidrule(r){1-2}
      \cmidrule(r){3-3}
      \cmidrule(r){4-4}
      \cmidrule(r){5-6}
      \cmidrule(r){7-7}
      \cmidrule(r){8-8}
      \cmidrule(r){9-10}

        & ф. ст. & ф. ст. & Кварт & \makecell{за квартер \\ ф. ст.} & \makecell{разом \\ ф. ст.} & ф. ст. & ф. ст. & Кварт. & ф. ст.   \\
      \midrule
      1) & \phantom{0}2\sfrac{1}{2}           & \phantom{1}\sfrac{1}{2} & 2\phantom{\sfrac{1}{2}} & 1\sfrac{1}{2}           & \phantom{0}3 & 3 & \phantom{0}6\phantom{\sfrac{1}{2}} & \phantom{-}1\phantom{\sfrac{1}{2}} & \phantom{-}3\phantom{\sfrac{1}{2}} \\
      2) & \phantom{0}2\sfrac{1}{2}           & \phantom{1}\sfrac{1}{2} & 1\sfrac{1}{2}           & 2\phantom{\sfrac{1}{2}} & \phantom{0}3 & 3 & \phantom{0}4\sfrac{1}{2}           & \phantom{-1}\sfrac{1}{2}           & \phantom{-}1\sfrac{1}{2} \\
      3) & \phantom{0}5\phantom{\sfrac{1}{2}} & 1\phantom{\sfrac{1}{2}} & 1\sfrac{1}{2}           & 4\phantom{\sfrac{1}{2}} & \phantom{0}6 & 3 & \phantom{0}4\sfrac{1}{2}           & -\phantom{1}\sfrac{1}{2}           & -1\sfrac{1}{2}           \\
      4) & \phantom{0}5\phantom{\sfrac{1}{2}} & 1\phantom{\sfrac{1}{2}} & 1\phantom{\sfrac{1}{2}} & 6\phantom{\sfrac{1}{2}} & \phantom{0}6 & 3 & \phantom{0}3\phantom{\sfrac{1}{2}} & -1\phantom{\sfrac{1}{2}}           & -3\phantom{\sfrac{1}{2}} \\
     \cmidrule(r){2-2}
     \cmidrule(r){3-3}
     \cmidrule(r){4-4}
     \cmidrule(r){6-6}
     \cmidrule(r){8-8}
     \cmidrule(r){9-9}
     \cmidrule(r){10-10}

       & 15\phantom{\sfrac{1}{2}} & 3\phantom{\sfrac{1}{2}} & 6\phantom{\sfrac{1}{2}} & & 18 & & 18\phantom{\sfrac{1}{1}} & \phantom{-}0\phantom{\sfrac{1}{2}} & \phantom{-}0\phantom{\sfrac{1}{2}} \\
  \end{tabular}

  \end{center}
\end{table}

Тут орендар продає кожен квартер по його індивідуальній ціні продукції,
і тому все число квартерів продає він по їхній пересічній ціні продукції квартера,
яка збігається з регуляційною ціною в 3\pound{ ф. стерл}. Він одержує тому на свій
капітал в 15\pound{ ф. стерл.}, як і давніш, 20\% зиску \deq{} 3\pound{ ф. стерл}. Але рента зникла.
Куди ж дівся надмір при цьому вирівнянні індивідуальних цін продукції кожного
квартера з загальною ціною продукції?

Надзиск з перших 2\sfrac{1}{2}\pound{ ф. стерл.} був 3\pound{ ф. стерл.}; з других 2\sfrac{1}{2}\pound{ ф. стерл.}
він був 1\sfrac{1}{2}\pound{ ф. стерл.}; разом надзиск на  \sfrac{1}{3} авансованого капіталу, тобто на
5\pound{ ф. стерл.} \deq{} 4\sfrac{1}{2}\pound{ ф. стерл.} \deq{} 90\%.

При витраті капіталу 3) 5\pound{ ф. стерл.} не тільки не дають надзиску, але
продукт їхній в 1\sfrac{1}{2} квартера, проданий по загальній ціні продукції, дає мінус в
1\sfrac{1}{2}\pound{ ф. стерл}. Нарешті, при витраті капіталу 4) теж в 5\pound{ ф. стерл.}, продукт
їхній в 1 кв., проданий по загальній ціні продукції, дає мінус в 3\pound{ ф. стерл}. Отже,
обидві витрати капіталу, взяті разом, дають мінус в 4\sfrac{1}{2}\pound{ ф. стерл.}, рівний надзискові
в 4\sfrac{1}{2}\pound{ ф. стерл.}, який постав від витрат капіталу 1) і 2).

Надзиск і мінус-зиск урівноважуються. Тому рента зникає. Але в дійсності
це можливе тому, що елементи додаткової вартости, які раніш становили
надзиск або ренту, входять тепер в створення пересічного зиску. Орендар одержує
цей пересічний зиск в розмірі 3\pound{ ф. стерл.} на 15\pound{ ф. стерл.}, або в розмірі
20\% коштом ренти.

Вирівняння індивідуальної пересічної ціни продукції з землі $В$ з загальною
ціною продукції $А$, яка регулює ринкову ціну, має за передумову, що ріжниця,
на яку індивідуальна ціна продукту перших витрат капіталу нижча,
ніж регуляційна ціна, дедалі більше зрівноважується і нарешті знищується
ріжницею, на яку продукт пізніших витрат капіталу починає перебільшувати
регуляційну ціну. Те, що являє собою надзиск, поки продукт перших витрат
капіталу продається сам по собі, в такий спосіб поступово стає частиною їхньої
пересічної ціни продукції, і разом з тим входить в створення пересічного зиску,
аж поки, нарешті, не буде зовсім поглинуте цим пересічним зиском.

Коли б замість вкладати в землю $В$ 15\pound{ ф. стерл.} капіталу, в неї було вкладено
лише 5\pound{ ф. стерл.} і додаткові 2\sfrac{1}{2} квартери останньої таблиці були випродуковані
\parbreak{}  %% абзац продовжується на наступній сторінці
