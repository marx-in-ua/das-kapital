
\index{iii2}{0157}  %% посилання на сторінку оригінального видання
Причина ж того, що не зважаючи на пониження ціни на 1\sfrac{1}{2}\pound{ ф. стерл.}
за квартер, отже на 50\%, і не зважаючи на зменшення площі конкурентної
землі з 4 до 3 акрів, загальна сума грошової ренти лишається та сама, а збіжжева
рента подвоюється, тимчасом як збіжжева й грошова рента, обчислена на акр, підвищується,
— причина цього в тому, що вироблено більше квартерів надпродукту.
Ціна збіжжя знижується на 50\%, надпродукт зростає на 100\%.
Але для досягнення такого наслідку вся продукція, згідно з нашими умовами,
мусить збільшитись утроє, а капітал, вкладений у кращу землю, мусить більше
ніж подвоїтись. В якому відношенні він мусить збільшуватись, залежить насамперед
від того, як розподіляється додаткові вкладення капіталу між кращими та
найкращими землями, припускаючи завжди, що продуктивність капіталу на
кожній категорії землі зростає пропорційно його величині.

Коли б пониження ціни продукції було менш значне, то потрібно було б
менше додаткового капіталу, щоб випродукувати ту саму грошову ренту. Коли б
подання збіжжя потрібне для того, щоб вилучити $А$ з числа оброблюваних земель,
— а це залежить не тільки від кількости продукту з акра землі $А$, але
також і від того, яку частину всієї оброблюваної земельної площі становить $А$, —
отже, коли б потрібне для цього подання було більше, отже, коли б також
потрібно було і більшої маси додаткового капіталу на кращій, ніж $А$, землі, то,
за інших незмінних відношень грошова і збіжжева ренти зросли б ще більше,
не зважаючи на те, що земля $В$ перестала б давати грошову і збіжжеву ренту.

Коли б капітал, що перестав функціонувати на землі $А$, дорівнював 5\pound{ ф.
стерл.}, то для цього випадку треба було б взяти для порівняння обидві таблиці:
II і ІV$d$. Весь продукт збільшився б з 20 до 30 квартерів. Грошова рента
зменшилася б удвоє, вона дорівнювала б 18\pound{ ф. стерл.} замість 36\pound{ ф. стерл.},
збіжжева рента залишилась би та сама \deq{} 12 квартерів.

Коли б можна було випродукувати на землі $D$ 44 квартери загального
продукту \deq{} 66\pound{ ф. стерл.}, вкладаючи капітал в 27\sfrac{1}{2}\pound{ ф. стерл.}, — що відповідало б
колишньому припущенню для $D$: 4 квартери на 2\sfrac{1}{2}\pound{ ф. стерл.} капіталу, —
то загальна сума\footnote*{
Тут очевидно справа йде про загальну грошову ренту. \Red{Прим. Ред.}
} ренти знову досягла б тієї висоти, яку вона мала в таблиці
II, і таблиця набула б такого вигляду:

\vspace{\bigskipamount}
\begin{table}[H]
  \centering
  \small
  \begin{tabular}{l c c c c}
  \toprule
  Рід землі & Капітал, \poundsign{} & Продукт, кв. & \makecell{Збіжжева \\ рента, кв.} & \makecell{Грошова\\рента, \poundsign{}} \\
  \midrule
  B &    \phantom{0}5\phantom{\tbfrac{1}{2}} & \phantom{0}4  & \phantom{0}0  & \phantom{0}0\\
  C &    \phantom{0}5\phantom{\tbfrac{1}{2}} & \phantom{0}6  & \phantom{0}2  & \phantom{0}3\\
  D &   27\tbfrac{1}{2}                      & 44            & 22            & 33\\
  \midrule
  Разом & 37\tbfrac{1}{2} &      54  &  24  &  36\\
  \end{tabular}
\end{table}

\looseness=1
\noindent{}Уся продукція була б 54 квартери проти 20 квартерів у таблиці II, грошова рента була
б та сама \deq{} 36\pound{ ф. стерл}. Але весь капітал був би 37\sfrac{1}{2}\pound{ ф. стерл.}, тимчасом як у таблиці II
він був \deq{} 20\pound{ ф. стерл}. Весь авансований капітал майже подвоївся б, тимчасом як продукція майже
потроїлася б; збіжжева рента збільшилася б удвоє, грошова рента залишилася б та сама.
Отже, коли ціна, за незмінної продуктивности, знижується в наслідок приміщення
додаткового грошового капіталу у кращі землі, що дають ренту, отже
в усі землі кращі від $А$, то весь капітал має тенденцію зростати не в такій
самій пропорції, як продукція і збіжжева рента; так що зростання збіжжевої ренти
може урівноважити падіння грошової ренти, яке постає в наслідок пониження ціни.
Той самий закон виявляється і в тому, що авансований капітал мусить бути більший
відповідно до того, як його вживається більше на землі $C$, ніж на $D$, — на землі,
що дає менше ренти, ніж на тій, яка дає більше ренти. Це визначає лише ось що:
\parbreak{}  %% абзац продовжується на наступній сторінці
