\parcont{}  %% абзац починається на попередній сторінці
\index{iii2}{0025}  %% посилання на сторінку оригінального видання
ту частину сукупної продукції, що нормально мусіла б увіходити в їхній спожиток.
Навіть тоді, коли їхній попит лишається номінально однаковим, у дійсності він
меншає.

Щодо довозу та вивозу треба зауважити, що всі країни одна по одній
заплутуються у кризу й тоді виявляється, що всі вони, за небагатьма винятками,
занадто багато вивозили й довозили, що, отже, \emph{платіжний балянс в несприятливий
для всіх}, що, отже, у дійсності справа не в платіжному балянсі.
Напр., Англія терпить від відпливу золота. Вона довозила понад міру. Але одночасно
всі інші країни переповнені англійськими товарами. Отже й вони довозили
понад міру, або до них довозили понад міру. (Звичайно, є ріжниця між країною,
що вивозить в кредит, та країнами, що не вивозять або лише вивозять мало
на кредит. Але тоді останні довозять на кредит; і цього не буває тільки тоді,
коли товари відправляють туди на комісію). Криза може насамперед вибухнути
в Англії, в країні, що дає найбільше та бере найменше того кредиту, бо платіжний
балянс, балянс платежів, що їм надійшов реченець та що їх доводиться
негайно ліквідувати, є для неї \emph{несприятливий}, дарма що загальний торговельний
балянс є для неї \emph{сприятливий}. Це останнє пояснюється почасти тим
кредитом, що його вона дає, а почасти масою капіталів, визичених закордонові,
так що відбувається масовий зворотний приплив товарами, крім зворотного припливу,
від властивих торговельних операцій. (Іноді криза поставала спочатку в
Америці, країні, що одержує від Англії найбільший торговельний та капітальний
кредит). Крах в Англії, що починається з відпливу золота й відбувається поряд того
відпливу, вирівнює платіжний балянс Англії, почасти в наслідок банкрутства її
імпортерів (про це нижче), почасти тому, що викидають частину її товарового капіталу
за дешеві ціни закордон, а почасти тому, що продається чужі цінні папери,
купується англійські папери й~\abbr{т. ін.} Тепер приходить черга на якусь іншу
країну. Платіжний балянс за даної хвилини був для неї сприятливий; але той
час, що в нормальних обставинах лежить між платіжним балянсом та балянсом
торговельним, тепер відпав, або його скорочено кризою; усі платежі треба виконати
одразу. Та сама справа повторюється тепер тут. Англія має тепер зворотний
приплив золота, друга країна — відплив золота. Те, що в одній країні виявляється
як надмірний довіз, у другій видається надмірним вивозом, і навпаки. Але надмірний
довіз та надмірний вивіз були по всіх країнах (ми кажемо тут не про
неврожаї і~\abbr{т. ін.}, а про загальні кризи); інакше кажучи, була надмірна продукція,
що їй сприяв кредит в супроводі загального набубнявіння цін.

В 1857 році криза вибухла в Сполучених Штатах. Постав відплив
золота з Англії до Америки. Але, скоро набубнявіння цін в Америці спало,
настала криза в Англії та відплив золота з Америки до Англії. Те саме трапилось
і між Англією та континентом. Платіжний балянс підчас загальної кризи
є для кожної нації несприятливий, принаймні для кожної комерційно розвинутої
нації, але завжди в однієї нації постає він по другій — як от стрільба
зводом — скоро черга платежу прийшла на неї; і криза, що вже вибухла в
якійсь країні, напр., в Англії, стискує чергу цих реченців у цілком короткі періоди.
Тоді виявляється, що всі ці нації одночасно понад міру вивезли (отже
й по надміру виробили) та понад міру довозили (отже понад міру наторгували),
що у всіх них ціни були понад усяку міру піднесено, а кредит понад міру був
напружений. І в усіх них настає та сама катастрофа. Явище відпливу золота
приходить по черзі до всіх них та саме своєю загальністю показує, 1) що
відплив золоту б лише феномен кризи, а не її причина; 2) що та послідовність,
з якою він настає в різних націй, показує лише, коли прийшла на кожну з них
черга закінчити свій рахунок з небом, коли в них надійшов реченець кризи
та коли заховані елементи тієї кризи дійшли в них вибуху.

Для англійських економістів-письменників характеристично — а варта уваги
\parbreak{}  %% абзац продовжується на наступній сторінці
