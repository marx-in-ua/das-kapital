\parcont{}  %% абзац починається на попередній сторінці
\index{iii2}{0197}  %% посилання на сторінку оригінального видання
значну ролю. Однак, в гірничій промисловості друга частина сталого капіталу, основний капітал,
відіграє значну ролю. Проте, і тут перебіг розвитку може вимірятися відносним зростом сталого
капіталу порівняно з змінним.

Коли склад капіталу у власне хліборобстві нижчий, ніж склад пересічного
суспільного капіталу, то це prima facie було б виразом того, що в країнах
розвиненої продукції хліборобство не досягло такого ступеня розвитку, як обробна промисловість.
Такий факт, залишаючи осторонь всі інші і до того ж почасти вирішені економічні обставини, мав би
для себе пояснення вже в давнішому і швидшому розвитку механічних наук і особливо в їхньому
застосуванні порівняно з пізнішим і почасти зовсім недавнім розвитком хімії, геології, фізіології і
особливо знов таки в їхньому застосуванні до хліборобства. Проте, це безперечний і давно відомий\footnote{Див. Dombasle і R.Jones}
факт, що проґрес самого хліборобства постійно визначається у відносному зрості сталої частини
капіталу проти змінної. Чи в певній країні капіталістичної продукції, наприклад, в Англії, склад
хліборобського капіталу є нижчий, ніж склад пересічного суспільного капіталу, — це питання, що його
можна розв’язати лише статистично, і яке детально розглядати було б зайвим з огляду на нашу мету. В
усякому разі теоретично усталено, що тільки при цьому припущенні вартість хліборобських продуктів
може бути вища від їхньої ціни продукції; тобто, що додаткова вартість, породжувана в хліборобстві
капіталом певної величини, або, що сходить на те саме, додаткова праця (отже, і вжита жива праця
взагалі), пущена ним в рух і упідлеглена йому, більша, ніж при рівновеликому капіталі пересічного
суспільного складу.

Отже, для форми ренти, що її ми досліджуємо тут, і яка може постати лише
при цьому припущенні, досить, коли ми зробимо це припущення. Де ця гіпотеза
відпадає, там відпадає і відповідна їй форма ренти.

Проте, простий факт надміру вартости хліборобських продуктів над їхньою ціною продукції сам по
собі ні в якому разі недостатній для того, щоб пояснити існування земельної ренти, незалежної від
різниці у родючості різних родів землі, або послідовних приміщень капіталу на тій самій землі,
коротко, такої ренти, яка в понятті відмінна від диференційної ренти і яку ми можемо тому позначити
як \emph{абсолютну ренту}. Цілий ряд мануфактурних продуктів має ту властивість, що їхня вартість вища від
їхньої ціни продукції і, не зважаючи на це, вони не дають такого надміру над пересічним зиском або
такого надзиску, що міг би перетворитись на ренту. Навпаки. Існування і поняття ціни продукції і
загальної норми зиску, яку вона включає, ґрунтуються на тому, що окремі товари продаються не по
їхній вартості. Ціни продукції виникають з вирівняння товарових вартостей, яке по покритті
відповідних капітальних вартостей, з ужиткованих в різних сферах продукції, розподіляє всю додаткову
вартість не в тій пропорції, що в ній її створено в окремих сферах продукції, і скільки її тому є в
продуктах останніх, а пропорційно величині авансованих капіталів. Тільки таким чином виникає
пересічний зиск і ціна продукції товарів, для якої пересічний зиск є характеристичним елементом. Це
є постійна тенденція капіталів, через конкуренцію здійснювати це вирівнювання в розподілі додаткової
вартости, створеної усім капіталом, і перемагати всі перешкоди цьому
вирівнюванню. Звідси і тенденція їхня допускати тільки такі надзиски, як вони виникають за всяких
обставин, не з різниці між вартостями і цінами продукції товарів, а радше, з різниці між загальною
ціною продукції, що реґулює ринок, і відмінними від неї індивідуальними цінами продукції; такі
надзиски, які тому, постають не з різниці між двома різними сферами продукції, а в межах кожної
сфери продукції, які, отже, не зачіпають загальних цін продукції різних сфер, тобто, загальної норми
зиску, а радше мають своєю
\parbreak{}  %% абзац продовжується на наступній сторінці
