\parcont{}  %% абзац починається на попередній сторінці
\index{iii2}{0158}  %% посилання на сторінку оригінального видання
щоб грошова рента лишилась та сама або підвищилась, мусить бути випродукована
певна додаткова кількість надпродукту, а для цього треба то менш
капіталу, що більша родючість земель; які дають надпродукт. Коли б ріжниця
між $В$ і $C$, $C$ і $D$ була ще більша, то потрібно було б ще менш додаткового
капіталу. Певне відношення залежить: 1)~від відношення, в якому понижується
ціна, отже, від ріжниці між землею $В$, що тепер не дає ренти, і $А$, яка давніш
не давала ренти; 2)~від відношення ріжниць між кращими, ніж $В$, землями; 3)~від
маси нововкладуваного додаткового капіталу і 4)~від його розподілу між землями
різної якости.

В дійсності бачимо, що закон не виражає нічого іншого, як те, що вже
було розвинено при дослідженні першого випадку: саме, що, коли ціна продукції
є дана, хоч би яка була її величина, рента може підвищуватися в наслідок
додаткового вкладення капіталу. Бо в наслідок вилучення $А$ тепер дана нова диференційна
рента І, за якої земля $В$ є тепер найгірша земля, і 1\sfrac{1}{2}\pound{ ф. стерл.} за
квартер становлять нову ціну продукції. Це однаково має силу так щодо таблиці
IV, як і щодо таблиці II.~Це той самий закон, але за вихідний
пункт береться землю $В$ замість $А$, і ціну продукції в 1\sfrac{1}{2}\pound{ ф. стерл.} замість
3\pound{ ф. стерл}.

Справа важлива тут лише от чим: оскільки така кількість додаткового
капіталу потрібна була для того, щоб капітал з $А$ відтягти від землі, і обслугувати
постачання без його участи, то й виявляється, що це може супроводитись
незмінною, висхідною або низхідною рентою з акра, якщо не на всіх землях, то
принаймні на деяких, і пересічно для всіх оброблюваних земель. Ми бачили, що
збіжжева рента і грошова рента не співрозмірні. Тільки за традицією збіжжева
рента все ще продовжує відігравати ролю в економії. З однаковим успіхом можна
було б довести, що, наприклад, фабрикант на свій зиск в 5\pound{ ф. стерл.} може купити
геть більшу кількість своєї власної пряжі, ніж давніше на зиск в 10\pound{ ф. стерл}.
Але в усякому разі це доводить, що панове земельні власники, коли вони одночасно
власники або учасники мануфактур, цукроварень, гуралень то що, з пониженням
грошової ренти все таки можуть дуже значно вигравати, як продуценти
свого власного сирового матеріялу\footnote{
У вищенаведених таблицях від IVа до ІVb треба було б виправити в розрахунку помилку, що
проходить через них. Хоч це не зачіпає теоретичних засад, виведених з даних таблиць, але іноді
приводить до неймовірних числових відношень продукції з акра. Але й це по суті не має значення. У
всіх мапах, що змальовують рельєф і висоту профілю місцевості, беруть значно більший маштаб для
вертикалей, ніж для горизонталей. А хто все таки почуватиме себе ображеним у своїх аграрних
почуттях, тому дається на волю помножити число акрів на перше-ліпше число. Можна також у таблиці І
замінити 1, 2, 3, 4 квартери з акра 10, 12, 15, 16 бушелями (8 бушелів \deq{} 1 квартер), з тим
розрахунком, щоб виведені з цього числа інших таблиць не виходили з меж імовірности; тоді виявиться,
що наслідок — відношення підвищення ренти до збільшення капіталу — зводиться цілком до того самого.
Це й зроблено в тих таблицях, що їх редактор додає до найближчого розділу. — \emph{ Ф.~Е.}
}.

\paragraph{За низхідної норми продуктивности додаткових капіталів}
Це не викликає нічого нового остільки, оскільки ціна продукції і тут, як
в щойно розгляненому випадку, може лише понизитись, коли в наслідок додаткових
вкладень капіталу на землях кращої якости, ніж $А$, продукт з $А$ зробиться
зайвий і тому капітал буде вилучений з $А$, або земля $А$ буде застосована до вироблення
іншого продукту. Випадок цей ми вже вичерпно дослідили. Ми показали,
що збіжжева і грошова ренти з акра можуть при цьому випадку зрости,
зменшитися або лишитися без зміни.
