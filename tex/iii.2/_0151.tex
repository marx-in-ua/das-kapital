\parcont{}  %% абзац починається на попередній сторінці
\index{iii2}{0151}  %% посилання на сторінку оригінального видання
зростає авансований капітал, хоч не в такому самому відношенні, яке потрібне
було б, коли б більшу кількість продукту довелося виготовляти за колишньої
продуктивної сили. (Стосується до відділу I).

З погляду капіталістичної продукції, у відношенні не до збільшення додаткової
вартости, а до зменшення витрат продукції, — а заощадження витрат
навіть на елемент, що створює додаткову вартість, на працю, робить капіталістові
цю послугу і створює для нього зиск, доки регуляційна ціна продукції
лишається та сама, — вживання сталого капіталу завжди дешевше, ніж
вживання змінного. Справді це має за свою передумову відповідний капіталістичному
способові продукції розвиток кредиту і багатість позикового капіталу. З одного
боку, я вживаю 100\pound{ ф. стерл.} додаткового сталого капіталу, коли 100\pound{ ф.
стерл.} становлять продукт, випродукований 5 робітниками протягом року; з
другого боку — 100\pound{ ф. стерл.} як змінний капітал. Коли норма додаткової вартости
\deq{} 100\%, то вартість випродукована 5 робітниками \deq{} 200\pound{ ф. стерл.};
навпаки, вартість 100\pound{ ф. стерл.} сталого капіталу \deq{} 100\pound{ ф. стерл.}, а як капіталу,
можливо, \deq{} 105\pound{ ф. стерл.}, коли рівень проценту \deq{} 5\%. Ті самі грошові суми,
залежно від того чи авансовано їх для продукції як вартісні величини сталого,
чи змінного капіталу, виражають, розглядувані в їхньому продукті, дуже неоднакові
вартості. Далі, щодо витрат продукції товарів, з погляду капіталіста,
то ріжниця є ще в тому, що з цих 100\pound{ ф. стерл.} сталого капіталу, оскільки
вони вкладені в основний капітал, в вартість товару входить лише спрацьовування,
тоді як ці 100\pound{ ф. стерл.}, витрачені на заробітну плату, мусять бути
цілком репродуковані у вартості товару.

У колоністів і взагалі самостійних дрібних продуцентів, які зовсім не
порядкують капіталом, або можуть ним порядкувати тільки за високі проценти,
частина продукту, відповідна заробітній платі, становить їхній дохід, тоді як
для капіталістів вона є авансування капіталу. Тому перший дивиться на цю
витрату праці як на доконечну передумову трудового продукту, про який насамперед
і йдеться. Щождо надмірної праці, витрачуваної ним понад цю потрібну
працю, то вона в усякому разі реалізується в надмірному продукті; і оскільки
він може продати або сам застосувати його, цей продукт розглядає він як
щось, що йому нічого не коштувало, бо він не коштував зрічевленої праці.
Тільки витрата такої праці має для нього значіння відчуження багатства. Звичайно,
він намагається продавати якомога дорожче; але навіть продаж нижче
вартости і нижче капіталістичної ціни продукції все ще має для нього значіння
зиску, оскільки цей зиск не антиципований заборгованістю, гіпотеками тощо.
Навпаки, для капіталістів витрата так змінного капіталу, як і сталого, однаково
є авансування капіталу. Відносно більше авансування сталого капіталу
зменшує за інших незмінних обставин витрати продукції, а також в дійсності і
вартість товарів. Тому, хоч зиск походить лише з додаткової праці, отже, лише
з вживання змінного капіталу, проте поодинокому капіталістові може здаватися,
що жива праця є найдорожчий елемент витрат продукції, який найбільш слід звести
до мінімуму. Це лише капіталістично перекручена форма тієї істини, що відносно
більше вживання зрічевленої праці порівняно з живою, свідчить про підвищення
продуктивности суспільної праці і збільшення суспільного багатства. От в якому фальшивому
вигляді, яким поставленим шкереберть здається все з погляду конкуренції.

Коли припустити незмінні ціни продукції, то додаткові капітали можуть
вкладатися з незмінною, висхідною або низхідною продуктивністю на кращих
землях, тобто на всіх землях, починаючи з $В$ і вище. На самій $А$ це було б
можливо при нашому припущенні або тільки за незмінної продуктивності, за
якої земля, як і давніш, не дає ренти, або ж і тоді, коли продуктивність зростає;
одна частина вкладеного в землю $А$ капіталу давала б тоді ренту,
друга — ні. Але це було б неможливо, коли припустити, що продуктивна сила
\parbreak{}  %% абзац продовжується на наступній сторінці
