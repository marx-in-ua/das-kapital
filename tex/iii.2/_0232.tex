
\chapter{Доходи та їхні джерела}

\section{Триєдина формула}

\subsection*{І.}

Капітал\footnote{Дальші три уривки містяться в різних місцях рукопису відділу VІ. — Ф.~Е.}
— зиск (підприємницький бариш плюс процент), земля — земельна
рента, праця — заробітна плата, це триєдина форма, яка охоплює всі таємниці
суспільного процесу продукції.

А що далі, як це показано раніш, процент виступає як специфічний,
характеристичний продукт капіталу, а підприємницький бариш протилежно
до цього як незалежна від капіталу заробітна плата, то зазначена триєдина
форма найближче зводиться до такої:

Капітал — процент, земля — земельна рента, праця — заробітна плата, де зиск,
ця форма додаткової вартости, що специфічно характеризує капіталістичний спосіб
продукції, щасливо усувається.

При ближчому розгляді цієї економічної триєдиности ми відкриваємо таке:

Поперше, позірні джерела багатства, що ним можна щороку порядкувати,
належать до цілком різних сфер і не мають найменшої схожости між собою.
Взаємне відношення між ними приблизно таке, як наприклад, між нотаріяльними
оплатами, червоними буряками і музикою.

Капітал, земля, праця! Але капітал — це не річ, а певне, суспільне,
належне певній історичній формації суспільства продукційне відношення, яке
виявляється в речі і надає цій речі специфічного суспільного характеру. Капітал
не є сума матеріяльних і випродукованих засобів продукції. Капітал, це —
перетворені на капітал засоби продукції, які сами по собі так само не є капітал,
як золото або срібло сами по собі не є гроші. Це є монополізовані певною
частиною суспільства засоби продукції, усамостійнені проти живої робочої сили
продукти й умови діяльности самої цієї робочої сили, які в наслідок цієї протилежности
персоніфікуються в капіталі. Це не тільки продукти робітників,
перетворені на самостійні сили, продукти як поневільники і покупці своїх продуцентів,
але також і суспільні сили і майбутня\dots{} [? нерозбірливо] форма цієї
праці, — сили, що протистоять робітникам, являючи властивості їхнього продукту.
Отже, ми маємо тут певну, на перший погляд дуже містичну, суспільну форму
одного з чинників певного історично створеного суспільного процесу продукції.

А тепер, поряд з цим земля, неорганічна природа як така, rudis indigestaque
moles\footnote*{
Лат. груба, необроблена груда. \Red{Пр.~Ред.}
} у всій її перевісній дикості. Вартість є праця. Тому додаткова
\parbreak{}  %% абзац продовжується на наступній сторінці
