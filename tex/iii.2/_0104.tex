\parcont{}  %% абзац починається на попередній сторінці
\index{iii2}{0104}  %% посилання на сторінку оригінального видання
постають з приміщення капіталу в сільському господарстві. Без цього аналіза
капіталу була б неповна. Отже, ми обмежуємося тільки на приміщенні капіталу
у власне хліборобство, тобто в продукцію тієї головної рослинної речовини, що
з неї живе людність. Ми можемо сказати — пшениці, бо вона становить головний
харчовий засіб сучасних капіталістично розвинутих народів. (Або замість
хліборобства, копальні, бо закони — ті самі).

Одна з найбільших заслуг А. Сміта в тому, що він з’ясував, як земельна
рента капіталу, ужитого до продукції інших сільсько-господарських продуктів,
напр., льону, фарбівних рослин, продуктів самостійного скотарства і~\abbr{т. ін.}, визначається
тією земельною рентою, що її дає капітал, приміщений у продукції
головного харчового засобу. Дійсно, по ньому в цьому напрямі не зроблено жодного
поступу. Те, що ми мали б пригадати, обмежуючи або доповнюючи його
аналізу, належить не сюди, а до самостійного розгляду земельної власности.
Тому про земельну власність — оскільки вона не має відношення до землі, призначеної
до продукції пшениці — ми не будемо говорити ex professo, але іноді
будемо вертатися до неї лише для ілюстрації.

Для повноти треба зауважити, що під землею ми розуміємо тут і воду і~\abbr{т. ін.},
наскільки вона має власника, є принадлежність землі.

Земельна власність має собі за передумову монополію певних осіб щодо
порядкування певними частинами земної кулі, як сферами лише їхньої приватної
волі, з вилученням всіх інших\footnote{
Нічого не може бути комічнішого за геґелівську теорію приватної земельної власности. Людина
як особа мусить здійснити свою волю, як душу зовнішньої природи, і тому мусить взяти цю
природу в володіння, як свою приватну власність. Коли це є означення «особи», людини, як особи,
то відси виходило б, що кожна людина мусить бути земельним власником для того, щоб здійснити себе,
як особу. Вільна приватна власність на землю — продукт новішого розвитку за Геґелем є не певне
суспільне відношення, а відношення людини, як особи, до «природи», абсолютне право людини на
присвоювання
всіх речей (Hegel, Philosophie des Rechts. Berlin 1840, ст. 79). Але передусім ясно, що поодинока
особа не може закріпити за собою власність на землю лише своєю «волею» проти чужої волі, що
так само хоче втілитися на тому самому клаптикові земної кулі. До цього потрібні цілком інші речі, ,
ніж добра воля. Далі абсолютно не можна зрозуміти, де «особа» має покласти межу здійсненню своєї
волі, чи реалізується буття її волі в цілій країні, чи їй для цього потрібна ціла купа країн, щоб
присвоєнням
їх «маніфестувати величчя моєї волі проти речі». Тут же Геґель і зайшов цілком у безвихідь.
«Заволодіння чимось — має цілком поодинокий характер; я беру у своє володіння не більше від того,
чого
я можу доторкнутися своїм тілом; але подруге ясно, що зовнішні речі мають більший пошир, ніж
я можу охопити. Отож, коли я щось маю в своєму володінні, то з ним у зв’язку перебуває щось інше.
Я здійснюю акт заволодіння за допомогою руки, але обсяг його можна поширити» (р. 90). Але з цим
іншим є знову дещо інше в зв’язку й оттак зникає та межа, що має визначити, як далеко має розлитись
моя воля, як душа, по землі. «Коли я дечим володію, то розум зараз переходить до того, що не
тільки те, чим я безпосередньо володію, але й те, що є з ним у зв'язку, належить мені. Тут мусить
зробити свої постанови позитивне право, бо з поняття нічого більше не можна вивести». (р. 91). Це —
незвичайно наївне визнання з боку «поняття», і доводить, що поняття, яке з самого початку робить
помилку, вважаючи цілком визначену та властиву буржуазному суспільству юридичну уяву про земельну
власність за абсолютну, «нічого» не розуміє у дійсних формах цієї земельної власности. Одночасно тут
є й визнання того, що зі зміною потреб суспільного, тобто економічного розвитку, «позитивне право»
може й мусить зміняти свої постанови.
}. Коли взяти це за передумову, то
справа в тому, щоб з’ясувати економічну вартість, тобто реалізацію вартости
цієї монополії на базі капіталістичної продукції. Юридичною владою тих осіб
уживати чи зловживати частинами земної кулі справи ані трохи не розв’язано.
Уживання тих частин залежить цілком від економічних умов, незалежних від
волі цих осіб. Сама юридична уява не означає нічого більше, як тільки те, що
земельний власник може робити з землею те саме, що й кожен власник
товарів зі своїм товаром; і ця уява — юридична уява вільної приватної земельної
власности — постає в стародавньому світі лише за часів розкладу органічного'
суспільного ладу, а в новому світі лише з розвитком капіталістичної продукції.
В Азію її лише подекуди імпортували європейці. У відділі про первісне нагромадження
(Книга І, розд. XXIV) ми бачили, що цей спосіб продукції має
\parbreak{}  %% абзац продовжується на наступній сторінці
