\parcont{}  %% абзац починається на попередній сторінці
\index{iii2}{0260}  %% посилання на сторінку оригінального видання
продуктивності праці, компенсується зворотним рухом додаткової вартости, так що
вартість змінного капіталу плюс додаткова вартість, отже, вартість новодолучена
працею до засобів продукції і новостворена в продукті лишається незмінна.

Навпаки, коли збільшення або зменшення вартости змінного капіталу або
заробітної плати є наслідок подорожчання або пониження ціни товарів, тобто
наслідок зменшення або збільшення продуктивности праці, вжитої в цій сфері
приміщення капіталу, то це впливає на вартість продукту. Але підвищення
або пониження заробітної плати є тут не причина, а тільки наслідок.

Навпаки, коли б у вищенаведеному прикладі при незмінному сталому
капіталі $= 400 c$, зміна $100 v + 150 m$ на $150 v + 100 m$, отже, підвищення
змінного капіталу, було наслідком пониження продуктивної сили праці не в даній
окремій галузі продукції, наприклад, у бавовнопрядінні, але, скажімо, в хліборобстві,
що постачає робітникові харчові продукти, — отже, було б наслідком
подорожчання цих харчових продуктів, то вартість продуктів не змінилася б.
Вартість в 650, як і давніш, визначалася б тією самою масою бавовняної пряжі.

З викладеного вище випливає далі таке: коли в наслідок економії тощо,
зменшуються витрати сталого капіталу в тих галузях продукції, що їхні продукти
входять в споживання робітника, то це, так само, як і безпосереднє зростання
продуктивности самої ужитої праці, може призвести до зменшення заробітної
плати, бо це здешевлює засоби існування робітника, а тому це може
призвести до підвищення додаткової вартости; так що норма зиску зростає тут
з двох причин, а саме: з одного боку, тому, що зменшується вартість сталого
капіталу, і, з другого боку, тому, що збільшується додаткова вартість. Розглядаючи
перетворення додаткової вартости на зиск, ми припускали, що заробітна
плата не понижується, а лишається сталою, бо там нам треба було дослідити
коливання норми зиску, незалежно від зміни норми додаткової вартости. Крім
того, розвинуті нами там закони мають загальний характер, вони мають силу
і для тих приміщень капіталу, що їхні продукти не входять в споживання робітника,
і що зміни вартости їхнього продукту не впливають тому на заробітну плату.

\pfbreak

Отже, відокремлення і розпад вартости, яку новоприєднувана праця щорічно
знову долучає до засобів продукції, або до сталої частини капіталу, на різні форми
доходу: на заробітну плату, зиск і ренту — ані трохи не змінює межі самої вартости,
тієї суми вартости, що розподіляється між цими різними категоріями; так
само як зміна відношення між цими окремими частинами не може змінити суми їх,
цієї даної величини вартости. Дане число 100 залишається завжди тим самим, чи
розкладемо ми його на $50 + 50$ чи на $20 + 70 + 10$, чи на $40 + 30 + 30$. Та
частина вартости продукту, що розпадається на ці доходи, є визначена, як і стала
частина вартости капіталу, вартістю товарів, тобто кількістю праці, зрічевленою
в них в кожному даному випадку. Отже, поперше, дано величину вартости
товарів, яка розподіляється на заробітну плату, зиск і ренту; дано, отже, абсолютну
межу суми окремих частин вартости цих товарів. Подруге, щодо самих
цих категорій, то дано також їхні пересічні і регуляційні межі. Заробітна плата
становить базу цього останнього обмеження. Вона, з одного боку, реґулюється
природним законом; її мінімальна межа дана фізичним мінімумом засобів існування,
потрібних робітникові для збереження і репродукції його робочої сили;
дана, отже, як певна кількість товарів. Вартість цих товарів визначається робочим
часом, потрібним для їхньої репродукції; отже, тієї частиною праці, новодолученої
до засобів продукції, або тією частиною кожного робочого дня, яку
робітник витрачає на продукцію і репродукцію еквіваленту вартості цих потрібних
засобів існування. Коли, наприклад, пересічна вартість його засобів
існування за день дорівнює 6 годинам пересічної праці, то він мусить пересічно
працювати на себе 6 годин на день. Дійсна вартість його робочої сили
\parbreak{}  %% абзац продовжується на наступній сторінці
