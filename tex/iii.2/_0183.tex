\parcont{}  %% абзац починається на попередній сторінці
\index{iii2}{0183}  %% посилання на сторінку оригінального видання
підвищення диференційної ренти, але саме існування диференційної ренти як
ренти є разом з тим причина ранішого й швидшого підвищення загальної ціни
продукції, щоб таким чином забезпечити збільшене подання продукту, яке стало доконечним.

Треба зауважити далі таке:

Через додаткове капіталовкладення в землю $В$ не могла б підвищитися
реґуляційна ціна до 4\pound{ ф. стерл.}, як це наведено вище, коли б земля $А$, в наслідок
другої витрати капіталу, давала додаткову продукцію дешевше від 4\pound{ ф.
стерл.}, або коли б вступила в конкуренцію нова гірша, ніж $А$, земля, що на
ній ціна продукції була б хоч і вища 3, але нижча 4\pound{ ф. стерл}. Таким чином,
ми бачимо, як диференційна рента І і диференційна рента II, тимчасом як
перша є за базу для другої, одночасно правлять одна для однієї за межу, що
спричинює то послідовні витрати капіталу на тій самій земельній дільниці, то
витрати капіталу одну біля однієї на новій додатковій землі. Так само вони
обмежують одна одну і в інших випадках, коли, наприклад, черга доходить до
кращих земель.

\section{Диференційна рента і з найгіршої з оброблюваних земель}

Припустімо, що попит на збіжжя підвищується, і що подання може бути
задоволене лише через послідовні витрати капіталу з недостатньою продуктивністю
на землях, що дають ренту, або через додаткову витрату капіталу
теж з низхідною продуктивністю на землі $А$, або через витрату капіталу на
нових землях гіршої якости, ніж $А$.

Візьмімо землю $В$ як представницю земель, що дають ренту.

Щоб уможливити додаткову продукцію 1 квартера на землі $В$ (який
тут може становити 1 мільйон квартерів, як кожен акр — мільйон акрів), додаткове
капіталовкладення вимагає підвищення ринкової ціни понад 3\pound{ ф. стерл.}
за квартер, що були до цього часу за реґуляційну ціну. На землях $C$ і $D$ і~\abbr{т.
ін.} родах землі з найвищою рентою, теж може бути випродуковано додатковий
продукт, але лише з низхідною додатковою продуктивною силою; проте,
припускається, що 1 квартер землі $В$ потрібен для задоволення попиту.
Коли цей один квартер можна дешевше випродукувати з допомогою додаткового
капіталу на $В$, ніж з допомогою рівної витрати додаткового капіталу
на $А$, або спускаючись до землі $А_{-1}$, яка може випродукувати квартер, наприклад,
лише за 4\pound{ ф. стерл.}, тимчасом як додатковий капітал на $А$ міг би випродукувати
квартер уже за 3\sfrac{3}{4}\pound{ ф. стерл.}, то додатковий капітал, витрачений на
$В$, почав би реґулювати ринкову ціну.

Земля $А$, як і давніш, випродукувала 1 квартер за 3\pound{ ф. стерл.}. $В$ теж,
як і давніш, випродукувала в цілому 3\sfrac{1}{2} квартера, що їхня індивідуальна ціна
продукції становить разом 6\pound{ ф. стерл}. Тепер, коли б на землі $В$ потрібно було
додаткової витрати в 4\pound{ ф. стерл.} ціни продукції (включаючи і зиск), щоб випродукувати ще 1 квартер,
тимчасом як на $А$ його можна випродукувати за
3\sfrac{3}{4}\pound{ ф. стерл.}, то, зрозуміла річ, він був би випродукований на $А$, а не на $В$.
Отже, припустімо, що він може бути випродукований на $В$ з 3\sfrac{1}{2}\pound{ ф. стерл.}
додаткової ціни продукції. В цьому випадку 3\sfrac{1}{2}\pound{ ф. стерл.} були б реґуляційною ціною
для всієї продукції. Тоді $В$ продав би свій теперішній продукт в 4\sfrac{1}{2}  квартери за
15\sfrac{3}{4}\pound{ ф. стерл}. З цього на ціну продукції перших 3\sfrac{1}{2}  квартерів припадає
6\pound{ ф. стерл.} і на останній квартер 3\sfrac{1}{2}\pound{ ф. стерл.}, разом 9\sfrac{1}{2}\pound{ ф. стерл}. Лишається
надзиск, для ренти \deq{} 6\sfrac{1}{4}\pound{ ф. стерл}, проти лише 4\sfrac{1}{2}\pound{ ф. стерл.} колишніх.
В цьому випадку акр землі $А$ також дав би ренту в  \sfrac{1}{2}\pound{ ф. стерл.};
\parbreak{}  %% абзац продовжується на наступній сторінці
