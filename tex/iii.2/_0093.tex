\parcont{}  %% абзац починається на попередній сторінці
\index{iii2}{0093}  %% посилання на сторінку оригінального видання
І саме функція їх як платіжного засобу розвиває процент, а тому й грошовий
капітал. Чого хоче марнотратне та коруптивне багатство, так це грошей
як грошей, грошей як засобу, щоб усе купувати. (А також, щоб платити борги).
На що потребує грошей дрібний продуцент, так це головним чином для платежа.
(Перетвір натуральних відбутків і повинностей поміщикам і державі на грошову
ренту та грошові податки відіграє тут чималу ролю). В обох випадках грошей
потребують як грошей. В другого боку, утворення скарбів стає лише тепер реальним,
здійснюючи свою мрію в лихварстві. Чого вимагає власник скарбів так це не капіталу,
а грошей як грошей; але за допомогою проценту він перетворює цей грошовий
скарб, як такий, на капітал, — на засіб, що за його допомогою він цілком або
почасти захоплює додаткову працю, а так само й частину самих умов продукції,
хоч номінально вони й далі протистоять йому як чужа власність.
Лихварство живе наче в шпарах продукції, як боги Епікура в міжсвітових просторах.
Гроші то тяжче мати, що менше товарова форма є загальна форма
продукту. Тому лихвар не знає ніякісіньких меж, окрім дієздатности або здатности
до опору з боку тих, що потребують грошей. В дрібно-селянській та
дрібно-буржуазній продукції грошей уживають, як купівного засобу переважно
тоді, коли випадково або в наслідок незвичайного зрушення робітник втрачає
умови продукції (що їхніми власником він ще є переважно за цих способів продукції)
або, принаймні, тоді, коли тих умов продукції не можна повернути у звичайному перебігу
репродукції. Життьові засоби та сирові матеріяли є головна частина цих умов
продукції. Подорожчання їх може унеможливити покриття їх з виручки за продукт,
так само як прості неврожаї можуть заважати селянинові повернути своє насіння in
natura. Ті самі війни, що ними римські патриції руйнували плебеїв, примушуючи
їх до військової повинности, які заважали їм в репродукції умов їхньої праці,
а тому й робили з них злидарів (а зубожіння, занепад або втрата умов
репродукції є тут домінантна форма), — ці самі війни наповнювали патриціям
комори та льохи здобичею — міддю, грішми тих часів. Замість давати плебеям
безпосередньо потрібні їм товари, збіжжя, коні, рогатизну, вони визичали
їм цю самим їм непотрібну мідь та використовували цей стан, щоб видушувати
величезні лихварські проценти, перетворюючи таким чином плебеїв на своїх
довжників — рабів. За Карла Великого франкських селян теж зруйнували війни,
так що їм не лишалося нічого іншого, як з довжників зробитися кріпаками.
В Римській імперії, як відомо, дуже часто траплялося, що голод примушував
вільних людей продавати своїх дітей та самих себе багатшим у невільництво. Оце
взагалі щодо цього поворотного пункту. Коли ж розглядати поодинокі випадки,
то збереження або втрата умов продукції дрібним продуцентом залежить від
тисячі пригод, і кожна така пригода або втрата означає зубожіння, та стає
тим пунктом, де може приміститися лихвар-паразит. Досить того, щоб у дрібного
селянина здохла корова, і він стає вже нездатним знову розпочати свою репродукцію
у колишньому маштабі. Отак опиняється він в руках лихваря, а, опинившися
в тих руках, він ніколи вже з них не визволиться.

Однак функція грошей, як платіжного засобу, становить притаманне, велике
та своєрідне поле для лихварства. Всяка грошова, повинність, що для неї має
надійти певний реченець, земельний чинш, данина, податок і~\abbr{т. ін.}, приносить
з собою потребу платити гроші. Тому, починаючи від стародавніх римлян аж
до новітніх часів, лихварство в широких розмірах прилучається до відкупників
податків, fermiers généraux, receveurs généraux\footnote*{
Великих фармерів, великих збирачів податку. \Red{Прим. Ред.}
}. Потім з розвитком торговлі та
з загальним поширом товарової продукції розвивається відокремлення актів купівлі
та платежа щодо часу. Гроші треба постачити у певний реченець. Як це може
привести до такого стану, що навіть і досі грошовий капіталіст та лихвар
\parbreak{}  %% абзац продовжується на наступній сторінці
