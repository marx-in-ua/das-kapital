
\index{iii2}{0189}  %% посилання на сторінку оригінального видання
Одно з найкумедніших явищ є в тому, що всі противники Рікардо, які
заперечують визначення вартости виключно працею, в справі з диференційною
рентою, що випливає з ріжниць землі, надають ваги тому, що тут вартість
визначається природою, а не працею; і одночасно приписують це визначення
положенню, або, і ще більше, процентові на капітал, вкладений в землю при
обробітку. Та сама праця дає однакову вартість для продукту, створеного
протягом даного часу; але величина або кількість цього продукту, отже, і та
частина вартости, яка припадає на відповідну частину цього продукту за даної
кількости праці, залежить єдино від кількости продукту, а це знову від продуктивности
даної кількости праці, не від величини цієї кількости. Чи завдячує
ця продуктивність своїм походженням природі, чи суспільству — цілком байдуже.
Тільки в тому разі, коли вона сама коштує праці, отже, капіталу, вона
збільшує ціну продукції новою складовою частиною, чого природа сама по собі
не робить.

\section{Абсолютна земельна рента}

Аналізуючи диференційну ренту, ми виходили з припущення, що найгірша
земля не виплачує земельної ренти, або, висловлюючись загальніше, що земельну
ренту виплачує тільки така земля, для продукту якої індивідуальна ціна продукції
нижча від ціни продукції, що реґулює ринок, так що в такий спосіб
виникає надзиск, що перетворюється на ренту. Потрібно насамперед зауважити,
що закон диференційної ренти, як диференційної ренти, зовсім не залежить від
правильности чи неправильности того припущення.

Коли загальну ціну продукції, що реґулює ринок, ми назвемо $Р$, то $Р$ для
продукту найгіршого роду землі $А$ збігається з індивідуальною ціною продукції
на цій землі; тобто вона оплачує зужиткований у продукції сталий і змінний капітал
плюс пересічній зиск (= підприємницькому баришеві плюс процент).

Рента тут дорівнює нулеві. Індивідуальна ціна продукції найближчого
кращого роду землі $В \deq{} Р'$, і $Р>Р'$; тобто $Р$ оплачує більше, ніж дійсну
ціну продукції продукту на клясі землі $В$. Хай тепер $Р — Р' \deq{} d$; тому
$d$, надмір $Р$ над $Р'$, є той надзиск, що його добуває орендар з цієї кляси $В$.
Це $d$ перетворюється на ренту, яку доводиться виплачувати власникові землі.
Хай для третьої кляси землі $C$ за дійсну ціну продукції буде $Р''$, і хай
$Р - Р'' \deq{} 2d$; отже, ці $2d$ перетворюються на ренту; так само для четвертої кляси
$D$ індивідуальна ціна продукції хай буде $Р'''$, а $Р - Р''' \deq{} 3d$, які перетворюються
на земельну ренту і~\abbr{т. д.} Даймо тепер, що припущення, ніби для
кляси землі $А$ рента $= 0$, а тому ціна її продукту $= Р \dplus{} 0$, помилкове. Хай,
навпаки, і вона дає ренту $= r$. В цьому випадку маємо двоякі наслідки.

\emph{Поперше}: ціна продукту землі кляси $А$ не реґулювалася б ціною продукції
на цій землі, а мала б деякий надмір над цією ціною, вона була б
$= P - r$. Бо, коли припускається нормальний перебіг капіталістичного способу
продукції, отже, коли припускається, що надмір $r$, виплачуваний від орендаря
земельному власникові, не становить вирахування ані з заробітної плати, ані
з пересічного зиску на капітал, то орендар може виплачувати його лише тому,
що його продукт продається понад ціну продукції, що він, отже, дав би йому
надзиск, коли б не доводилося відступати цей надмір у формі ренти земельному
власникові. Реґуляційна ринкова ціна всього наявного на ринку продукту
всіх родів землі була б тоді не та ціна продукції, яку дає капітал взагалі
у всіх сферах продукції, тобто не ціна рівна витратам плюс пересічний
зиск, а була б ціною продукції плюс рента, $Р \dplus{} r$, не $Р$. Бо ціна продукту
кляси $А$ визначає взагалі межу реґуляційної загальної ринкової ціни, тієї ціни,
\parbreak{}  %% абзац продовжується на наступній сторінці
