
\index{iii2}{0149}  %% посилання на сторінку оригінального видання
Цей випадок не припускає далі жодного продуктивнішого застосування
капіталу, а лише застосування більшого капіталу до тієї самої площі і з тими
самими наслідками, як і до того часу.

Усі відносні величини тут лишаються ті самі. Звичайно, коли розглядати
не відносні ріжниці, а суто аритметичні, то диференційна рента з різних земельможе
змінитися. Припустімо, наприклад, що додатковий капітал вкладено лише
в $В$ і $D$. Тоді ріжниця між $D$ і $А$ \deq{} 7 квартерам, давніш вона \deq{} 3; ріжниця
між $В$ і $А$ \deq{} 3 кварт., давніш вона \deq{} 1; ріжниця між $C$ і $В \deq{} - 1$, давніш
вона $= \dplus{} 1$ і~\abbr{т. ін.} Але ця аритметична ріжниця, вирішальна щодо диференційної
ренти І, оскільки в ній виражається ріжниця в продуктивності за однакового
розміру вкладеного капіталу, тут цілком не має ваги, бо вона є лише
наслідок того, чи вкладено, чи ні різні додаткові капітали, за незмінної ріжниці
для кожної рівної частини капіталу на ріжних дільницях.

IIІ.~Додаткові капітали дають надмірний продукт і створюють тому надзиски,
але при понижуваній нормі, не пропорційно їхньому збільшенню.

\begin{table}[h]
  \begin{center}
    \emph{Таблиця ІII}
    \footnotesize

  \begin{tabular}{c@{ } c@{ } c@{ } c@{ } c@{ } c@{ } c@{ } c@{ } c@{ } c@{ } c}
    \toprule
      \multirowcell{2}{\makecell{Рід \\землі}} &
      \multirowcell{2}{\rotatebox[origin=c]{90}{Акри}} &
      Капітал &
      \rotatebox[origin=c]{90}{Зиск} &
      \rotatebox[origin=c]{90}{\makecell{Ціна про- \\ дукції}} &
      \multirowcell{2}{\makecell{Продукт в\\ квартерах}} &
      \rotatebox[origin=c]{90}{\makecell{Продажна \\ ціна}} &
      \rotatebox[origin=c]{90}{Здобуток} &
      \multicolumn{2}{c}{Рента} &
      \multirowcell{2}{\makecell{Норма \\надзиску}} \\

      \cmidrule(r){3-3}
      \cmidrule(r){4-4}
      \cmidrule(r){5-5}
      \cmidrule(r){7-7}
      \cmidrule(r){8-8}
      \cmidrule(r){9-10}

       &  &  ф. ст. & ф. ст. & ф. ст. & & ф. ст. & ф. ст. & Кварт. & ф. ст. &  \\
      \midrule

      A & 1 & \phantom{2\sfrac{1}{2} \dplus{} 2\sfrac{1}{2} \deq{}} 2\sfrac{1}{2} & \phantom{0}\sfrac{1}{2} & \phantom{0}3 & \phantom{2 \dplus{} 1\sfrac{1}{2} \deq{}} 1\phantom{\sfrac{1}{2}} & 3 & \phantom{0}3\phantom{\sfrac{1}{2}} &\phantom{0} 0\phantom{\sfrac{1}{2}} & \phantom{0}0\phantom{\sfrac{1}{2}} & \phantom{00}0\phantom{\%} \\
      B & 1 & 2\sfrac{1}{2} \dplus{} 2\sfrac{1}{2} \deq{} 5\phantom{\sfrac{1}{2}} & 1\phantom{\sfrac{1}{2}} & \phantom{0}6 & 2 \dplus{} 1\sfrac{1}{2} \deq{} 3\sfrac{1}{2}           & 3           & 10\sfrac{1}{2}                     & \phantom{0}1\sfrac{1}{2}           & \phantom{0}4\sfrac{1}{2}           & 90\% \\
      C & 1 & 2\sfrac{1}{2} \dplus{} 2\sfrac{1}{2} \deq{} 5\phantom{\sfrac{1}{2}} & 1\phantom{\sfrac{1}{2}} & \phantom{0}6 & 3 \dplus{} 2\phantom{\sfrac{1}{2}} \deq{} 5\phantom{\sfrac{1}{2}} & 3 & 15\phantom{\sfrac{1}{2}}           & \phantom{0}3\phantom{\sfrac{1}{2}} & \phantom{0}9\phantom{\sfrac{1}{2}} & 180\%\\
      D & 1 & 2\sfrac{1}{2} \dplus{} 2\sfrac{1}{2} \deq{} 5\phantom{\sfrac{1}{2}} & 1\phantom{\sfrac{1}{2}} & \phantom{0}6 & 4 \dplus{} 3\sfrac{1}{2} \deq{} 7\sfrac{1}{2}           & 3           & 22\sfrac{1}{2}                     & \phantom{0}5\sfrac{1}{2}           & 16\sfrac{1}{2}                     & 330\%\\
     \cmidrule(r){1-1}
     \cmidrule(r){3-3}
     \cmidrule(r){4-4}
     \cmidrule(r){5-5}
     \cmidrule(r){6-6}
     \cmidrule(r){8-8}
     \cmidrule(r){9-9}
     \cmidrule(r){10-10}

     Разом &  & \phantom{2\sfrac{1}{2} \dplus{} 2\sfrac{1}{2} \deq{}} 17\sfrac{1}{2} & 3\sfrac{1}{2} & 21 & \phantom{2 \dplus{} 1\sfrac{1}{2} \deq{}}17\phantom{\sfrac{1}{2}} & & 51\phantom{\sfrac{1}{2}}  & 10 & 30\phantom{\sfrac{1}{2}} &\\
  \end{tabular}

  \end{center}
\end{table}

При цьому третьому припущені знов таки байдуже, чи повторні додаткові
капітали вкладаються рівномірно або нерівномірно на землі різних родів або ні
в однакових чи неоднакових відношеннях відбувається зменшення продукції
надзиску; чи всі додаткові капітали вкладаються в той самий сорт землі, щодає
ренту, чи розподіляються вони рівномірно або нерівномірно, між землями
різної якости, що дають ренту. Всі ці обставини байдужі для закону, що його тут
розвиваємо. Єдине наше припущення є в тому, що додатковий капітал,
вкладений в будь-який сорт землі, що дає ренту, дає надзиск, але в зменшуваній
пропорції проти розміру збільшення капіталу. Межі цього зменшення
коливаються в прикладах вищенаведеної таблиці, між 4 квартерами \deq{} 12\pound{ ф. стерл.},
продуктом першого капіталовкладення на найкращій землі $В$ і 1 квартером
\deq{} 3\pound{ ф. стерл.}, продуктом такого самого вкладення капіталу на найгіршій
землі $А$. Продукт з найкращої землі при витраті капіталу і становить максимальну
межу, а продукт з найгіршої землі $А$, що не дає ні ренти, ні надзиску,
становить, за однакового вкладення капіталу, мінімальну межу продукту,
який дають послідовні вкладення капіталу па будь-якого роду землях, що дають надзиск за зменшуваної
продуктивности послідовних вкладень капіталу. Як
припущення ІІ відповідає тому, що нові однакові якістю дільниці землі кращих
родів приєднується до оброблюваної площі, так що кількість якогось роду обробленої
землі збільшується, так припущення ІІІ відповідає тому, що оброблюються
\parbreak{}  %% абзац продовжується на наступній сторінці
