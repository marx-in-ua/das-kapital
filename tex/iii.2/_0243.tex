\parcont{}  %% абзац починається на попередній сторінці
\index{iii2}{0243}  %% посилання на сторінку оригінального видання
цілком як вдома серед цих відокремлених від дійсних відносин та іраціональних
форм: капітал-процент, земля-рента, праця-заробітна плата, бо це є саме ті
форми ілюзії, що в них вони живуть і з якими щодня мають справу. Тому
так само природно те, що вульґарна економія, яка є не що інше, як дидактичний,
більш або менш доктринерський переклад повсякденних уявлень дійсних аґентів
продукції, і яка лише вносить певний розумний порядок в ці уявлення, що вона
саме в цій триєдиності, в якій згладжено всякий внутрішній зв’язок, знаходить
природну, безперечну базу для свого безпідставного пишання. Одночасно ця
формула відповідає інтересам владущих кляс, бо вона проклямує і підносить
до догми природну доконечність і вічне виправдання джерел їхнього доходу.

Коли ми описуємо зречевлювання відносин продукції і їх усамостійнення
проти аґентів продукції, ми не спиняємося на тому, яким чином, в наслідок
явищ світового ринку, його коньюнктури, руху ринкових цін, періодів кредиту,
циклів промисловости і торговлі, чергування розцвіту й кризи — яким чином,
в наслідок цих явищ внутрішні зв’язки видаються діячам продукції як непоборні,
стихійно владущі над ними закони природи, і виявляються проти них як сліпа
доконечність. Не спиняємось тому, що дійсний рух конкуренції лежить поза
нашим пляном і тому, що ми тепер повинні описувати тільки внутрішню
організацію капіталістичного способу продукції, його, так би мовити, ідеальну
пересічну.

В колишніх суспільних формах ця економічна містифікація виступає лише
переважно щодо грошей і капіталу, що дає процент. З самої природи справи
вона виключена, поперше, там, де переважає продукція ради споживної вартости,
ради безпосереднього власного споживання; подруге, там, де, як за античної доби
і середньовіччя, рабство або кріпацтво становить широку основу суспільної продукції:
панування умов продукції над продуцентами замасковується тут відносинами
панування і упідлеглення, які виявляються і помічаються як безпосередні
рушійні пружини продукційного процесу. В первісних громадах, в яких
панує примітивний комунізм, і навіть в античних міських громадах, сама ця
громада з її умовами являє собою базу продукції так само, як репродукція цієї
бази є її остаточна мета. Навіть у середньовічних цехах ані праця, ані капітал
не являються незв’язаними, навпаки, їхні взаємовідносини визначаються корпораційним
ладом і відносинами, що стоять в зв’язку з останнім та відповідними
до них уявленнями про професійний обов’язок, становище майстра тощо.
Тільки за капіталістичного способу продукції\dots{}

\section{До аналізи процесу продукції}

Для дальшого ось дослідження, ми можемо лишити осторонь ріжницю між
ціною продукції і вартістю, бо ріжниця ця взагалі відпадає, коли, як ми це
робимо тут, будемо розглядати вартість сукупного річного продукту праці, отже,
продукту сукупного суспільного капіталу.

Зиск (підприємницький бариш плюс процент) і рента є не що інше, як
своєрідні форми, що їх набувають окремі частини додаткової вартости товарів.
Величина додаткової вартости є межа суми величин частин, на які додаткова
вартість може розпадатися. Пересічний зиск плюс рента дорівнюють
тому додатковій вартості. Можливо, що частина додаткової праці, яка міститься
в товарах, а тому й додаткової вартости, не входять безпосередньо у вирівнювання
зиску на пересічний зиск, так що частина товарової вартости взагалі не
реалізується в ціні відповідного товару. Але, поперше, це компенсується або
\parbreak{}  %% абзац продовжується на наступній сторінці
