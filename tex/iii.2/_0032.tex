
\index{iii2}{0032}  %% посилання на сторінку оригінального видання
Нагромадження у всіх капіталістів, що дають у позику гроші, відбувається,
зрозуміло, завжди безпосередньо у грошовій формі, тимчасом коли ми бачили, що
дійсне нагромадження у промислових капіталістів відбувається звичайно через
збільшення елементів самого репродуктивного капіталу. Отже, розвиток кредитової
справи та величезна концентрація підприємств, що визичають гроші, в руках
великих банків мусять уже сами про себе прискорювати нагромадження позичкового
капіталу як форму, відмінну від дійсного нагромадження. Тимто цей
швидкий розвиток позичкового капіталу є результат дійсного нагромадження,
бо він є наслідок розвитку процесу репродукції, а зиск, що являє джерело нагромадження
для цих грошових капіталістів, є тільки сума, відлічена з додаткової
вартости, що її добувають капіталісти репродукції (одночасно й привлащена
частина проценту від \emph{чужих} заощаджень). Позичковий капітал нагромаджується
одночасно і за рахунок промисловців і за рахунок купців. Ми бачили, що
підчас несприятливих фаз промислового циклу рівень проценту може підноситись
остільки високо, що він часово цілком поглинає зиск тих поодиноких ділянок
промисловости, які є в особливо несприятливому стані. Одночасно спадають ціни
державних фондів та інших цінних паперів. Це є той момент, коли грошові
капіталісти масами скуповують ці знецінені папери, що підчас пізніших фаз
незабаром підносяться знову до своєї нормальної висоти та понад неї. Тоді
папери ці збувають, і таким способом привлащується частина грошового капіталу
публіки. Та частина, що її не збувають, дає вищі проценти, бо її куплено
за нижчу ціну. Але увесь той зиск, що його добувають грошові капіталісти та
перетворюють його знову на капітал, вони перетворюють насамперед у позичковий
грошовий капітал. Отже, нагромадження його, як відмінне від дійсного нагромадження,
хоч і породжене ним, виявляється вже, скоро ми розглядаємо лише грошових
капіталістів, банкірів і~\abbr{т. ін.}, як нагромадження цієї осібної кляси капіталістів.
І воно мусить зростати з кожним поширом кредитової справи, що відбувається
одночасно з дійсним поширом процесу репродукції.

Коли рівень проценту низький, то це знецінення грошового капіталу відбивається
переважно на вкладниках, а не на банках. Перед розвитком акційних
банків в Англії \sfrac{3}{4} усіх вкладів лежали в банках, не даючи процентів. Якщо тепер
платять за них процент, то він є принаймні на 1\% нижчий від звичайного
рівня проценту.

Щодо нагромадження грошей рештою капіталістичних кляс, то ми не
вважатимемо на ту частину грошей, що приміщується у процентових паперах
та нагромаджується в цій формі. Ми розглядаємо лише ту частину, що її кидають
на ринок в формі позичкового грошового капіталу.

Тут ми маємо, поперше, ту частину зиску, що її не витрачається як дохід,
а призначається до нагромадження, але що для неї промислові капіталісти
зразу не мають ужитку в своїх власних підприємствах. Цей зиск існує
безпосередньо в товаровому капіталі, що його вартости частину він складає, та
реалізується разом з ним у грошах. Отже, коли його (покищо ми не вважаємо
на купця, бо про нього буде мова окремо) не перетворюють назад на елементи
продукції товарового капіталу, то він мусить на деякий час затвердіти у формі
грошей. Ця маса зростає з масою самого капіталу, навіть коли норма зиску
меншає. Частина, що має витрачатись як дохід, поволі споживається, однак
протягом проміжного часу вона, як вклад, являє позичковий капітал у банкіра.
Отже, навіть зріст частини зиску, витрачуваної як дохід, виявляється у повільному
нагромадженні позичкового капіталу, нагромадженні, раз-у-раз повторюваному.
І так само стоїть справа з другою частиною, призначеною до нагромадження.
Отже, з розвитком кредиту та його організації навіть зріст доходів,
тобто споживання промислових та торговельних капіталістів, виявляється як нагромадження
позичкового капіталу. І це має силу щодо всіх доходів, оскільки їх
\parbreak{}  %% абзац продовжується на наступній сторінці
