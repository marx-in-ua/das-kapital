\parcont{}  %% абзац починається на попередній сторінці
\index{iii2}{0199}  %% посилання на сторінку оригінального видання
продукції капітал $= k$, то ріжниця їхня є в другій, змінній частині, в додатковій
вартості, що в ціні нродукції $= р$, зискові, тобто дорівнює всій додатковій
вартості, обчисленій на суспільний капітал і на кожен окремий капітал, як на
пропорційну частину суспільного капіталу; але у вартості товару вона дорівнює
дійсній додатковій вартості, породженій цим окремим капіталом, та становить
інтеґральну частину породжених ним товарових вартостей. Коли вартість товару
вища за його ціну продукції, то ціна продукції $= k + р$, вартість $= k + p + d$,
так що $р + d =$ додатковій вартості, що міститься в ньому. Отже, ріжниця між
вартістю і ціною продукції $= d$, надмірові додаткової вартости, продукованої цим
капіталом понад ту, яка припадає йому відповідно до загальної норми зиску.
З цього випливає, що ціна хліборобських продуктів може бути вища від їхньої
ціни продукції, хоч вона не досягатиме їхньої вартости. З цього випливає
далі, що до певного пункту може відбуватися тривале підвищення ціни
хліборобських продуктів, перше ніж їхня ціна досягне їхньої вартости. З цього
випливає також, що тільки в наслідок монополії земельної власности надмір
вартости хліборобських продуктів над їхньої ціною продукції може стати моментом,
що визначає їхню загальну ринкову ціну. З цього випливає, нарешті,
що в цьому випадку не подорожчання продукту є причина ренти, а рента
є причиною подорожчання продукту. Коли ціна продукту з одиниці площі найгіршої
землі $= Р + r$, то всі диференційні ренти збільшуються відповідними кратними
$r$, бо, згідно з припущенням, за реґуляційну ринкову ціну стає $Р + r$.

Коли б пересічний склад нехліборобського суспільного капіталу
$= 85c + 15v$ і норма додаткової вартости = 100\%, то ціна продукції дорівнювала
б 115. Коли б склад хліборобського капіталу $= 75c + 25v$, то вартість
продукту, при тій самій нормі додаткової вартости, і реґуляційна ринкова
вартість дорівнювала б 125. Коли б хліборобський продукт вирівнявся з нехліборобським
до пересічної ціни (для короткости ми припускаємо, що в обох галузях
продукції загальна кількість капіталу однакова), то вся додаткова вартість
дорівнювала б 40, тобто 20\% на капітал в 200. Продукт так одного, як і другого
продавалось би за 120. Отже, при вирівнянні за цінами продукції пересічні ринкові
ціни нехліборобського продукту стояли б вище, а хліборобського продукту
нижче від їхньої вартости. Коли б хліборобські продукти продавалось по їхній
повній вартості, то вони були б на 5 вище, а промислові продукти
на 5 нижче, ніж по вирівнянні. Коли ринкові відносини не дозволяють продавати
хліборобські продукти по їхній повній вартості, виторговувати весь надмір
над ціною продукції, то це призводить до середнього між обома крайніми
пунктами стану; промислові продукти будуть продаватися трохи вище від їхньої
вартости, а хліборобські продукти трохи вище від їхньої ціни продукції.

Хоч земельна власність може нагнати ціну хліборобських продуктів вище
від їхньої ціни продукції, проте не від земельної властности, а від загального стану
ринку залежить, в якій мірі ринкова ціна, піднявшись над ціною продукції, наближається
до вартости, і, отже, в якій мірі додаткова вартість, створена в хліборобстві
понад даний пересічний зиск, перетворюється на ренту, абож бере участь у загальному
вирівнянні додаткової вартости в пересічний зиск. В усякому випадку
ця абсолютна рента, що виникає з надміру вартости над ціною продукції, становить
просто частину хліборобської додаткової вартости, перетворення цієї додаткової
вартости на ренту, захоплювання цієї додаткової вартости земельним власником;
цілком так само, як диференційна рента виникає з перетворення надзиску в ренту, захопленпя його
земельною власністю, при загальній регуляційній ціні продукції. Ці обидві форми ренти є єдино
нормальні. Рента, крім цих форм, може ґрунтуватися лише на власне монопольній ціні, яку не
визначається ані ціною продукції, ані вартістю товарів, а потребою і виплатоспроможністю покупців, і
розгляд якої стосуються до вчення про конкуренцію, де досліджується справжній рух ринкових цін.
