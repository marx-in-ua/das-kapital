\parcont{}  %% абзац починається на попередній сторінці
\index{iii2}{0146}  %% посилання на сторінку оригінального видання
нового надзиску, при чому воно може відбутися одночасно на землях D, C, В, А.
Коли ж, навпаки, з обробітку буде витиснута гірша земля А, то реґуляційна
ціна продукції понизиться, і від відношення між зменшеною ціною одного
квартера і збільшеним числом квартерів, що створюють надзиск, залежить, чи
підвищується чи понижується визначений в грошах надзиск, а отже, і диференційна
рента. Але в усякому разі тут виявляється та варта уваги обставина,
що за зменшуваних надзисків з послідовних капіталовкладень, ціна продукції
може зменшуватися замість підвищуватись, як це здається на перший погляд.

Ці додаткові приміщення капіталу з зменшуванними додатковими здобутками
цілком відповідають тому випадкові, коли в землі, що за своєю родючістю
містяться між А і В, В і С, С і D, було б, наприклад, вкладено чотири нові
самостійні капітали по 2\sfrac{1}{2} ф. стерл., які давали б відповідно 1\sfrac{1}{2} квартери,
2\sfrac{1}{3}, 2\sfrac{2}{3} і 3 квартери. Для всіх цих чотирьох додаткових капіталів на всіх
цих родах землі створились би надзиски, потенціяльні ренти, хоч норма надзиску
порівняно з тією, що дає таке саме капіталовкладення на щоразу кращій землі,
і зменшилася б. Та цілком байдуже було б, чи вкладено ці чотири капітали
в землю D і т. ін., чи розподілені вони між D і А.

Ми підходимо тепер до посутньої ріжниці між обома формами диференційної
ренти.

Коли справа йде про диференційну ренту І, то за незмінної ціни продукції
і незмінних ріжниць, разом з загальною сумою ренти може підвищитись
пересічна рента на акр, або пересічна норма ренти на капітал; але пересічність
є лише абстракція. Дійсний рівень ренти, обчислений на акр або на капітал,
тут лишається той самий.

Навпаки, розмір ренти обчислений на акр в тих самих обставинах, може
підвищитись, хоч норма ренти, обчислена на витрачений капітал, лишається
та сама.

Припустімо, що продукція подвоюється в наслідок того, що в землі А,
В, С, D вкладалося б по 5 ф. стерл. капіталу замість 2\sfrac{1}{2} ф. стерл., тобто
в цілому 20 ф. стерл. замість 10 ф. стерл., з незмінною відносною родючістю.
Це було б цілком те саме, як коли б замість одного акра кожного з цих
родів землі оброблялося 2 акри, а витрати лишалися б ті самі. Норма зиску
залишалася б та сама, так само як і її відношення до надзиску або ренти. Але,
коли б земля А почала давати тепер 2 квартери, В — 4, С — 6, D — 8, то ціна
продукції, як і давніш, дорівнювала б 3 ф. стерл. за квартер, бо цей приріст
завдячував би своїм походженням не подвоєній родючості за незмінного розміру
капіталу, а незмінній відносній родючості за подвоєного розміру капіталу. Ці
два квартери з А коштували б тепер 6 ф. стерл., як давніш 1 квартер коштував
3 ф. стерл. Зиск на всіх чотирьох родах землі подвоївся б, але тільки
тому, що подвоївся б витрачений капітал. Але в тому самому відношенні подвоїлася
б рента, вона дорівнювала б 2 квартерам для В замість 1 квартера, 4 квартерам
для C замість 2 і 6 квартерам для D замість 3, і відповідно до цього
грошова рента для В, С, D дорівнювала б відповідно 6 ф. стерлінґів, 12 ф.
стерл., 18 ф., стерл. Так само як продукт з акра, подвоїлася б і грошова
рента з акра, отже, і ціна землі, що в ній капіталізується ця грошова рента. За
таким розрахунком підвищується рівень збіжжевої і грошової ренти, а тому і
ціна землі, бо маштаб, що ним виміряється ця ціна, акр, є земельна площа
сталої величини. Навпаки, у пропорційній висоті ренти не сталося жодної зміни,
коли обчислювати її як норму ренти щодо витраченого капіталу. Загальна сума
ренти в 36 стосується до витраченого капіталу в 20, як загальна сума ренти
в 18 до витраченого капіталу в 10. Це саме має силу і для відношення грошової
ренти з земель кожного роду до вкладеного в них капіталу; так, наприклад,
12 ф. стерл. ренти з землі C стосуються до 5 ф. стерл. капіталу, як
\parbreak{}  %% абзац продовжується на наступній сторінці
