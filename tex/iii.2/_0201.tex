\parcont{}  %% абзац починається на попередній сторінці
\index{iii2}{0201}  %% посилання на сторінку оригінального видання
хоч у технічному відношенні між ужитою живою працею та масою і природою
застосованих умов праці не сталося жодної зміни. З другого боку, коли розглядати
справу виключно з погляду складу за вартістю, капітал порівняно низького
органічного складу в наслідок простого підвищення вартостей його сталих частин
міг би справити таке вражіння, ніби він піднісся на один ступінь з капіталом
вищого органічного складу. Хай дано капітал $= 60c \dplus{} 40v$, тому що він
застосовує багато машин і сирового матеріялу, порівняно з живою працею, і
другий капітал $= 40c \dplus{} 60v$, тому що він вживав багато живої праці (60\%),
мало машин (скажімо, 10\%) і відносно до робочої сили мало, до того ж
ще і дешевого, сирового матеріялу (скажімо, 30\%); таким чином в наслідок простого
підвищення вартости сирових і допоміжних матеріялів з 30 до 80, склад
міг би зрівнятися так, що в другому капіталі на 10 в машинах припадало б 80
в сировому матеріялі і 60 робочої сили, тобто $90c \dplus{} 60v$, що, визначене в процентах,
теж дорівнювало б $60c \dplus{} 40v$, при чому не сталося б жодної зміни в технічному
складі. Отже, капітали однакового органічного складу можуть мати
різний вартісний склад, і капітали однакового процентного вартісного складу
можуть стояти на різних ступенях органічного складу, отже, виражати різні ступені
розвитку суспільної продуктивної сили праці. Отже, сама лише обставина,
що за вартісним складом хліборобський капітал стояв би на загальному рівні, ще
не доводила б того, що суспільна продуктивна сила праці досягла у нього
такого самого рівня. Вона могла б лише показувати, що власний продукт цього
капіталу, який знову таки становить частину умов його продукції, є дорожчий,
або що допоміжні матеріяли, от як добриво, котрі давніш були просто під руками,
тепер доводиться довозити здалека тощо.

Але, залишаючи це осторонь, треба взяти на увагу своєрідний характер
хліборобства.

Припустімо, що вживання в хліборобстві машин, які зберігають працю,
хемічних допоміжних засобів тощо, набирають тут ширшого розміру, отже, що сталий
капітал технічно зростає не тільки щодо вартости, але й щодо маси, порівняно
з масою ужитої робочої сили; в такому разі в хліборобстві (як і в гірничій
промисловості) справа йде не тільки про суспільну, але і про природну продуктивність
праці, яка залежить від природних умов праці. Можливо, що збільшення
суспільної продуктивної сили в хліборобстві лише компенсує, або навіть
не зовсім компенсує зменшення природної сили — ця компенсація завжди може
впливати лише протягом деякого часу, — так що, не зважаючи на технічний
розвиток, продукт не здешевлюється, а лише гальмується його ще більше подорожчання.
Можливо також, що при висхідній ціні збіжжя абсолютна маса
продукту зменшується, тимчасом як відносний надпродукт зростає; це можливо
саме при відносному збільшенні сталого капіталу, що складається переважно
з машин або худоби, при чому доводиться покривати тільки його зношування,
і при відповідному зменшенні змінної частини капіталу, яка витрачається на
заробітну плату, що її постійно доводиться покривати з продукту цілком.

Але можливо також, що з поступом хліборобства потрібно буде лише помірне
підвищення ринкової ціни над пересічною для того, щоб могла оброблятись
і одночасно давати ренту така земля гіршої якости, яка при нижчому рівні
технічних допоміжних засобів потребувала б вищого підвищення ринкової ціни.

Може здатися, що та обставина, що, наприклад, у скотарстві, коли воно провадиться
в великих розмірах, маса вжитої робочої сили дуже мала проти сталого
капіталу, який є в самій худобі, може здатися, що ця обставина цілком суперечить
тому, що хліборобський капітал, обчислений у процентах, пускає в рух робочої
сили більше, ніж нехліборобський пересічний суспільний капітал. Але тут слід
відзначити, що при розгляді ренти ми виходимо, як з вирішної, з тієї частини
хліборобського капіталу, яка продукує основний рослинний засіб харчування,
\parbreak{}  %% абзац продовжується на наступній сторінці
