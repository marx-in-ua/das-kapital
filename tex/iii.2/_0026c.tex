\parcont{}  %% абзац починається на попередній сторінці
\index{iii2}{0026}  %% посилання на сторінку оригінального видання
економічна література, від 1830~\abbr{р.} починаючи, сходить переважно на літературу
про currency, кредит, кризи, — що вони розглядають вивіз благородних металів підчас
кризи, не вважаючи на зміну вексельного курсу, тільки з погляду Англії, як
суто-національне явище, та цілком заплющують очі на той факт, що, коли їхній
банк підчас кризи підвищує рівень проценту, то й усі інші європейські банки
роблять те-саме, і що, коли сьогодні у них лунають скарги з приводу відпливу
золота, то завтра ті скарги залунають в Америці, позавтра в Німеччині та Франції.

В 1847 році «довелося покривати поточні зобов’язання Англії» [здебільша, за
збіжжя]. «На нещастя, їх ліквідували здебільша банкрутствами». [Багата Англія
полегшила собі своє становище проти континенту та Америки банкрутством]. «Але
оскільки їх не ліквідувалося банкрутствами, їх покривали вивозом благородних
металів». (Report of Committee on Bank Acts, 1857). Отже, оскільки кризу в Англії
загострює банкове законодавство, це законодавство є засіб до того, щоб підчас
голоду обшахровувати нації, що вивозять збіжжя, спочатку на їхньому збіжжі,
а потім на грошах за їхнє, збіжжя. Отож, заборона вивозу збіжжя в такі часи
для країн, що сами більш або менш терплять від дорожнечі, є дуже раціональний
засіб проти цього плану Англійського банку — «покривати зобов’язання» за
довезене збіжжя «за допомогою банкрутств». Далеко краще буде тоді, щоб продуценти
збіжжя та спекулянти втратили частину свого зиску на користь своєї
країни, аніж свій капітал на користь Англії.

Із сказаного виявляється, що товаровий капітал підчас кризи та взагалі
підчас застою в справах великою мірою втрачає свою властивість представляти
потенціяльний грошовий капітал. Те саме має силу й щодо фіктивного
капіталу, процентових паперів, оскільки вони сами обертаються на біржі, як
грошові капітали. З підвищенням проценту спадає їхня ціна. Далі вона спадає
з причини загальної недостачі кредиту, що змушує власників тих паперів збувати
їх масами на ринку, щоб добути собі грошей. Насамкінець, вона спадає
в акцій, почасти в наслідок зменшення доходів, що на них вони, ті акції, є
посвідками, почасти в наслідок шахрайського характеру підприємств, що їх вони
досить часто представляють. Підчас криз цей фіктивний грошовий капітал незвичайно
меншає, а разом з цим меншає в його власників і спроможність позичати
під нього гроші на ринку. Однак, зменшення грошових назов цих цінних
паперів у курсовому бюлетені не має нічого до діла з дійсним капіталом, що
його вони представляють, але зате дуже багато — з платоспроможністю його
власників.

\section[Грошовий капітал та дійсний капітал. II. (Продовження)]{Грошовий капітал та дійсний капітал. II. \\ (Продовження)}

Ми все ще не кінчили з питанням, якою мірою нагромадження капіталу в
формі позичкового грошового капіталу збігається з дійсним нагромадженням,
з поширом процесу репродукції.

Перетвір грошей на позичковий грошовий капітал — куди простіша історія,
ніж перетвір грошей в продуктивний капітал. Але ми маємо тут відрізняти дві
речі:

1) Простий перетвір грошей на позичковий капітал;

2) Перетвір капіталу або доходу на гроші, що перетворюються на позичковий
капітал.

Тільки останній пункт може мати в собі позитивне нагромадження позичкового
капіталу, нагромадження, що є у зв’язку з дійсним нагромадженням
промислового капіталу.
