\parcont{}  %% абзац починається на попередній сторінці
\index{iii2}{0101}  %% посилання на сторінку оригінального видання
так що стає 40 на 100; тільки я не знаю, чи це так. Сором тобі, куди ж це
до біса, кінець-кінцем, має довести\elli{?..} Хто має тепер в Ляйпціґу 100 флоринів,
той щороку бере собі 40, це значить за один рік пожерти селянина або городянина.
Якщо він має 1000 флоринів, то щороку він бере собі 400, — це значить
за один рік пожерти лицаря або багатого дворянина. Коли він має \num{10.000}, то
бере собі щороку 4000; це значить за рік пожерти багатого графа. Коли він
має \num{100.000} — як то мусить бути у великих торговців — то бере собі щороку
\num{40.000}: це значить за рік пожерти великого, багатого князя. Коли він має
\num{1.000.000}, то бере собі щороку \num{400.000}, це значить за рік пожерти великого
короля. І не зазнає він при тому жодної небезпеки, ані щодо життя свого, ані
щодо своїх товарів, нічого не робить, сидить на печі та лузає насіння: і отакий
тихенький розбійник, сидячи дома, міг би за десять років зжерти цілий світ.
(Це з «Bücher von Kaufhandel und Wucher» з року 1524. Luther’s Werke, Wittenberg
1589.6 Teil.)

«15 років тому я писав проти лихварства, бо воно вже так дуже зміцнилося,
що я не мав надії на поліпшення. Від того часу воно так піднеслося, що
не хоче вже ніде бути вадою, гріхом або соромом, але бундючно вихваляється
чеснотою та честю, ніби воно має до людей велику любов та робить їм християнську
послугу. Чим же маємо зарадити лихові, коли ганьба стала честю, а
вада чеснотою» (An die Pfarherrn wider den Wucher zu predigen. Wittenberg 1540).

\pfbreak

«Євреї, ломбардці, лихварі та кровопійці були наші перші банкіри, наші
первісні банкові баришники, їхній характер можна було назвати майже безсоромним\dots{}

До них долучилися потім лондонські золотарі. Загалом\dots{} наші первісні
банкіри були\dots{} дуже поганим товариством, вони були жадобливими лихварями,
кам’яносердими кровососами». (І. Hardcastle, Bank and Bankers, 2-nd ed. London
1843, p. 19, 20)

«Отже, приклад, що його подала Венеція (утворення банку), швидко викликав
наслідування; всі міста коло моря і взагалі міста, що здобули собі
славу своєю незалежністю та своєю торговлею, заснували свої перші банки
їхні кораблі часто примушували чекати на свій поворот і це неминучо приводило
до звички кредитування, що її ще дужче зміцнили відкриття Америки та
торговля з нею». (Це головний пункт.) Перевіз товарів на кораблях змушував
брати великі позики, що бувало вже в стародавні часи в Атенах та Греції.
В 1380 році ганзейське місто, Брюґґе, мало страхову палату». (М. Augier, І. c., р.
202, 203.)

Як значно в останній третині 17 віку, перед розвитком новітньої кредитової
системи, переважало навіть в Англії визичання грошей земельним
власникам, отже й взагалі багатству, що розкошує, це можна побачити між
іншим з праці сера Dudley North’a, не тільки одного з перших англійських
купців, але й одного з найвидатніших теоретиків-економістів свого часу: «З тих
грошей, що їх наш народ віддає на проценти, далеко менше від десятої частини
віддається комерсантам, щоб вони на ті гроші могли провадити свої справи;
здебільша, ті гроші визичається на купівлю речей розкошів та на видатки людей,
що, хоч і є великі землевласники, але витрачають гроші швидше, ніж їх дає
їм їхня земельна власність; а що вони бояться продавати свої маєтки, то
й охотніше переобтяжують вони їх гіпотечними боргами». (Discourses upon Trade.
London 1691. p. 6,7.)

В XVIII віці в Польші: «Варшава робила великі вексельні операції, що
однак за свою основу та за мету мали користь варшавських лихварів банкірів.
Щоб добути собі гроші, що їх вони могли визичати за 8 і більше процентів
\parbreak{}  %% абзац продовжується на наступній сторінці
