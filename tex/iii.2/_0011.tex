\parcont{}  %% абзац починається на попередній сторінці
\index{iii2}{0011}  %% посилання на сторінку оригінального видання
підприємства, справді безвартісні, — нація не стала ані на шеляг біднішою
від того, що луснули ці мильні баньки номінального грошового капіталу.

Всі ці папери становлять справді не що інше, як нагромаджені вимоги,
правні титули на майбутню продукцію, їхня грошова чи капітальна вартість
або зовсім не представляє жодного капіталу, як от в державних боргах, або
реґулюється незалежно від вартости того дійсного капіталу, що його вони представляють.

По всіх країнах капіталістичної продукції є величезна маса того так званого,
капіталу, що дає процент, або moneyed capital у цій формі. І під нагромадженням
грошового капіталу, здебільша, треба розуміти не що інше, як нагромадження
цих вимог на продукцію, нагромадження ринкової ціни, ілюзорної
капітальної вартости цих вимог.

Отож частину банкірського капіталу приміщено в цих, так званих, процентних
паперах. Це саме та частина запасного капіталу, що не функціонує
у дійсному банковому підприємстві. Найзначніша його частина складається з
векселів, тобто з платіжних зобов’язань промислових капіталістів або купців.
Для позикодавця грошей ці векселі є папери, що дають процент; тобто, купуючи
їх, він відраховує процент за той час, на протязі якого вони ще мають обертатися.
Це є те, що звуть дисконтуванням. Отже, від рівня проценту в даний момент
залежить, скільки відраховується від тієї суми, що її представляє вексель.

Насамкінець, остання частина капіталу банкіра складається з його грошового
запасу золотом або банкнотами. Вклади, коли не складено умови про
довший час тримання їх в банку, є кожної хвилі до розпорядку вкладників.
Вони перебувають у постійній флюктуації\footnote*{
Від лат. слова «fluctus», гра хвиль, хвилювання, почережне піднесення й спад. Пр.~Ред.
}. Але, забрані одним вкладником, вони
повертаються іншим, так що підчас нормального розвитку справ загальна пересічна
сума їх коливається мало.

В країнах розвинутої капіталістичної продукції запасні фонди банків
завжди виражають пересічну кількість грошей, що є в вигляді скарбу, а частина
цього скарбу сама знову ж складається з паперів, з простих посвідок на золото,
що сами не становлять жодних вартостей. Тому, більша частина банкірського
капіталу є суто-фіктивна та складається з боргових вимог (векселів), державних
паперів (що представляють минулий капітал) та акцій (посвідок на майбутній
дохід). При цьому не слід забувати, що грошова вартість капіталу,
представлена цими паперами в панцерних скринях банкіра, є цілком фіктивна
навіть тоді, коли вони є посвідки на забезпечені доходи (як от в державних
паперах) або титули власности на дійсний капітал (як от в акціях), і цю грошову
вартість реґулюється незалежно від вартости дійсного капіталу, що його
вони принаймні почасти представляють; або що там, де вони представляють просту
вимогу на доходи, а не капітал, вимога на той самий дохід висловлюється у
раз-у-раз мінливому фіктивному грошовому капіталі. Опріч того, сюди долучається
ще й те, що цей фіктивний капітал банкіра здебільша є не його капітал,
а капітал публіки, яка склала його до банку або на проценти або без процентів.

Вклади завжди робиться грішми, золотом або банкнотами, або посвідками
на них. За винятком запасного фонду, — а він залежно від потреби дійсної циркуляції
меншає або більшає, — ці вклади в дійсності завжди перебувають, з одного
боку, в руках промислових капіталістів та купців, що з тих вкладів дисконтують
свої векселі та добувають собі позики; з другого боку, в руках торговців
цінними паперами (біржових маклерів), або в руках приватних осіб, що продали
свої цінні папері, або в руках уряду (як от буває при посвідках державної
скарбниці та нових позиках). Самі вклади відіграють подвійну ролю. З одного
\parbreak{}  %% абзац продовжується на наступній сторінці
