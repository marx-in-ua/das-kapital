\parcont{}  %% абзац починається на попередній сторінці
\index{iii2}{0265}  %% посилання на сторінку оригінального видання
створити цю рівновагу, але не ту норму зиску, яка складається при цій рівновазі.
Коли така рівновага вже постала, то чому загальна норма зиску 10 або
20 або 100\%? В наслідок конкуренції. Але, якраз навпаки, конкуренція усунула
причини, що зумовили відхил від 10 або 20 або 100\%. Вона призвела
до встановлення товарової ціни, при якій кожен капітал дає однаковий зиск,
пропорційно до своєї величини. Але величина самого цього зиску є незалежна
від конкуренції. Конкуренція лише знов і знов приводить усі відхили до цієї
величини. Одна людина конкурує з іншими, і конкуренція примушує її продавати
свої товари по тій самій ціні, як усі інші. Але чому ж ця ціна 10 або
20, або 100?

Отже, не лишається нічого іншого, як з’ясувати норму зиску, а тому й
зиск, як визначувану якимось незрозумілим чином додачу до ціни товару, яка
до цього пункту визначалась заробітною платою. Одним-одно, про що говорить
нам конкуренція, це те, що ця норма зиску мусить бути певного величиною.
Але це ми знали і давніш, коли говорили про загальну норму зиску та про
«потрібну ціну» зиску.

Немає жодної потреби цей безглуздий процес міркувань наново простежувати
щодо земельної ренти. І без того ясно, що переведений хоч трохи послідовно,
він призводить лише до того, що зиск і рента видаються просто додачами,
визначуваними цілком незрозумілими законами, до ціни товарів, визначуваної
в першу чергу заробітною платою. Коротко кажучи, фактично конкуренція
бере на себе пояснити всю беззмістовність економістів, тимчасом як здавалося б,
навпаки, економісти повинні були б пояснити конкуренцію.

Коли лишити осторонь ту фантазію, що зиск і рента, як складові частини
ціни, створюються циркуляцією, тобто продажем, — адже в дійсності циркуляція
ніколи не може дати того, що їй самій попереду не дано, — то все питання зведеться
просто ось до чого:

Хай ціна товару, визначувана заробітною платою, = 100; норма зиску
10\% на заробітну плату, і рента 15\% на заробітну плату. Тоді ціна товару,
визначувана сумою заробітної плати, зиску й ренти, = 125. Ці 25 додачі не
можуть походити з продажу товару. Бо всі, хто продає один одному товари,
продають за 125 те, що кожному з них коштувало 100 заробітної плати; а це
цілком те саме, як коли б вони всі продавали за 100. Отже, ця операція мусить
розглядатись незалежно від процесу циркуляції.

Коли ці три особи ділять між собою самий товар, який тепер коштує
125, — справа ані трохи не зміниться, якщо капіталіст спочатку продасть
товар за 125, а потім виплатить робітникові 100, собі самому 10 і землевласникові
15, — то робітник одержує \sfrac{4}{5} = 100 вартости і продукту. Капіталіст
одержує \sfrac{2}{25} вартости і продукту, землевласник \sfrac{3}{25}. Продавши за 125, замість
100, капіталіст віддає робітникові лише \sfrac{4}{5}, того продукту, що в ньому втілюються
праця робітника. Отже, це було б цілком те саме, як коли б він дав
робітникові 80, залишивши в себе 20, що з них 8 припало б йому і 12 земельному
власникові. Тоді він продав би товар по його вартості, бо в дійсності
надбави до ціни становлять підвищення, незалежні від вартости товару, яка, при
зробленому вище припущенні, визначається вартістю заробітної плати. Таким
чином, манівцями це призвело б до такого висновку, що при даному уявленні
заробітна плата, 100, дорівнює вартості продукту, тобто = сумі грошей, що в
ній визначається ця певна кількість праці, але що вартість ця відрізняється від
реальної заробітної плати і, отже, дає деякий надмір. Надмір цей, згідно з розглядуваною
теорією, реалізується номінальною додачею до ціни. Отже коли б
заробітна плата дорівнювала 110, замість 100, то зиск мусив би = 11, і земельна
рента 16\sfrac{1}{2}, отже, ціна товару = 137\sfrac{1}{2}. Відношення лишилися б при
цьому без зміни. Але тому, що при цьому поділ здійснювався б завжди через
\parbreak{}  %% абзац продовжується на наступній сторінці
