\parcont{}  %% абзац починається на попередній сторінці
\index{iii2}{0111}  %% посилання на сторінку оригінального видання
принаймні, своїм власним капіталом. Таке постійне пограбування становить об’єкт
боротьби за ірландське земельне законодавство, яке пропонують звести до того,
щоб примусити земельного власника, який відмовляє орендареві, винагородити
його за зроблені ним поліпшення ґрунту або за долучений до землі капітал.
Пальмерстон звичайно давав на це цинічну відповідь: «Палата громад — палата
земельних власників».

Ми не говоримо також про ті виключні відносини, коли навіть в країнах
капіталістичної продукції земельний власник може вичавлювати високу оредну
плату, що аніяк не відповідає продуктові землі, як наприклад здача в оренду
в англійських промислових округах дрібних клаптиків землі фабричним робітникам
чи то під малесенькі садочки чи то для аматорського обробітку землі на
дозвіллі (Rероrts of Inspectors of Factories).

Ми говоримо про хліборобську ренту в країнах розвиненої капіталістичної
продукції. Наприклад, серед англійських орендарів є певна кількість дрібних
капіталістів, які вихованням, освітою, традиціями, конкуренцією та іншими обставинами
призначені й примушені до того, щоб вкладати свій капітал у
хліборобство, як орендарі. Вони примушені задовольнятися зиском меншим,
ніж пересічний і віддавати частину його у формі ренти землевласникові. Це —
однісінька умова, за якої їм тільки й може бути дозволено вкладати свій
капітал у ґрунт, у хліборобство. А що земельні власники всюди мають значний,
в Англії навіть переважний, вплив на законодавство, то й можуть вони використати
цей вплив для того, щоб ошукувати цілу клясу орендарів. Наприклад,
хлібні закони 1816 року — податок на хліб, як відомо, накладений на
країну з тією метою, щоб забезпечити для неробів землевласників дальше існування
збільшених рент, що надзвичайно зросли підчас анти-якобінської війни —
за винятком окремих виключно урожайних років, впливали правда так, що тримали
ціну сільсько-господарських продуктів вище від того рівня, до якого вони
упали б при вільному довозі хліба. Проте, вони не мали таких наслідків, щоб
утримати хлібні ціни на такій висоті, яку землевласники-законодавці декретували,
як свого роду нормальну ціну, так щоб вони становили законну межу довозу
закордонного збіжжя, але орендні договори складалося під вражінням цих
нормальних цін. Коли ілюзії зникли, складено новий закон з новими нормальними
цінами, які, проте, були так само простим безсилим виразом загребущої
землевласницької фантазії, як і старі. Таким способом орендарів ошукували
від 1815 до 30-х р. р. Звідси agricultural distress\footnote*{
Agricultural distress — пригнічений стан хліборобства Пр.~Ред.
} як постійна тема протягом
усього цього часу. Звідси — протягом цього періоду експропріяція і руйнування
цілого покоління орендарів, заміщення їх новою клясою капіталістів\footnote{
Див. Anti-Corn-Law Prize-Essays. Тимчасом хлібні закони все-таки тримали ціни на штучно
підвищеному рівні. Це сприяло кращим орендарям. Вони вигравали від інертности, до якої охоронні
мита схиляли переважну масу орендарів, що покладались — підставно чи ні, — на виключну
пересічну ціну.
}.

Але далеко загальніший і важливий факт являє собою пониження заробітної
плати власне хліборобських робітників нижче за її нормальний пересічний
рівень, так що частина заробітної плати віднімається у робітника, становить
собою складову частину орендної плати і таким чином під маскою земельної ренти
дістається землевласникові замість робітника. Наприклад, в Англії і Шотландії,
за винятком небагатьох графств, що перебувають у сприятливім становищі,
це — загальне явище. Праці уряджених перед запровадженням хлібних законів
в Англії парламентських слідчих комісій про висоту заробітної плати — до цього
часу найцінніші і майже цілком невикористані матеріяли з історії заробітної
плати XIX століття і одночасно ганебний пам’ятник, поставлений англійською
\parbreak{}  %% абзац продовжується на наступній сторінці
