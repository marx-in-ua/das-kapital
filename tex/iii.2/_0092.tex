\parcont{}  %% абзац починається на попередній сторінці
\index{iii2}{0092}  %% посилання на сторінку оригінального видання
змушує репродукцію відбуватися серед чимраз наймізерніших умов. Звідси народна
ненависть до лихваря, найбільша в античному світі, де власність продуцента на
умови його продукції являла одночасно базу політичних відносин, самостійности
громадянина.

Поки панує невільництво або поки додатковий продукт поїдає февдал та
його почет, а рабовласник або февдал перебуває в залежності від лихваря,
спосіб продукції лишається теж тим самим: тільки він стає тяжчим для робітника.
Обтяжений боргами рабовласник або февдал висисає більше, бо з нього
самого висисають більше. Або, кінець-кінцем, він відступає своє місце лихвареві,
що сам стає земельним власником або рабовласником, як от верхівці
у стародавньому Римі. Місце колишнього визискувача, що його експлуатація
була більш або менш патріярхальною, бо здебільша становила вона знаряддя
політичної влади, заступає жорстокий, до грошей жадобливий вискочень. Але сам
спосіб продукції не змінюється.

Лихварство впливає революційно за всяких передкапіталістнчних способів
продукції лише оскільки воно руйнує та нищить ті форми власности, що
на їхній міцній базі та невпинній репродукції, яка відбувається в тій самій
формі, спираються політичні відносини. За азіятських форм лихварство може
триматися довго, не викликаючи нічого іншого, опріч економічного занепаду та
політичної розпусти. Тільки там і тоді, де й коли в наявності є решта умов капіталістичного
способу продукції, лихварство є один з засобів до утворення
нового способу продукції, руйнуючи февдалів та дрібну продукцію з одного боку,
централізуючи умови праці в капітал з другого боку.

В середні віки в жодній країні не панував загальний рівень проценту.
Церква взагалі забороняла всякі процентові операції. Закони й суди лише мало
забезпечували стягання боргів. То вище бувала норма проценту в поодиноких
випадках. Невеличка грошова циркуляція, потреба більшість платежів робити
готівкою, змушували до грошових позик, і то більше, що менше була розвинута
вексельна справа. Панувала значна ріжниця і в рівні проценту, і в понятті про
лихварство. За часів Карла Великого вважалося за лихварство, коли хтось брав
100\%. В Ліндав, що коло Боденського озера, в 1348 році місцеві громадяни брали
216\sfrac{2}{3}\%. В Цюриху рада визначила 43\sfrac{1}{3}\% як законний процент. В Італії часами
доводилося платити 40\%, дарма, що в 12--14 віці звичайна норма проценту
не перевищувала 20\%. Верона визначила 12\sfrac{1}{2}\% як законний процент. Імператор
Фрідріх II усталив 10\%, але це тільки для євреїв. Про християн він
не мав охоти говорити. В надрайнській Німеччині вже в 13 віці 10\% становило
звичайний процент. (Hülmann, Geschichte des Städtewesens. II. p. 55--57).

Лихварський капітал має властивий капіталові спосіб експлуатації без
властивого йому способу продукції. Це відношення повторюється і в межах буржуазної
економіки в відсталих ділянках промисловости або в таких, що чинять
опір переходові до новітнього способу продукції. Коли ми захочемо порівняти,
напр., англійський рівень проценту з індійським, то нам доведеться взяти не
рівень проценту Англійського банку, а, напр., процент, що його беруть ті, хто
визичають дрібні машини дрібним продуцентам домашньої промисловости.

Протилежно до багатства, що споживає, лихварство має історичну вагу,
як процес самого поставання капіталу. Лихварський капітал та купецьке
майно упосереднюють утворення незалежного від земельної власности грошового
майна. Що менше розвинувся товаровий характер продукту, що менше мінова
вартість підбила під себе продукцію в цілій її широті й глибині, то більше гроші видаються
власне багатством, як таким, загальним багатством, протилежно до його
обмеженого способу вияву у споживчих вартостях. На це спирається утворення
скарбів. Коли абстрагуватись від грошей, як світових грошей та скарбу, то
власне у формі платіжного засобу, вони виступають як абсолютна форма товару.
\parbreak{}  %% абзац продовжується на наступній сторінці
