
Третій випадок: За висхідної ціни продукції.

А.~Коли земля $А$ не дає ренти й продовжує реґулювати ціну.

Варіянт 1: За незмінної продуктивности другої витрати капіталу, що зумовлює
низхідну продуктивність першої витрати.

\begin{table}[H]
  \begin{center}
    \emph{Таблиця XIX\footnotemarkZ{}}
    \footnotesize

  \begin{tabular}{c@{  } c@{  } c@{  } c@{  } c@{  } c@{  } c}
    \toprule
      \multirowcell{2}{\makecell{Рід\\ землі}} &
      Ціна продукції &
      Продукт &
      \makecell{Продажна \\ ціна} &
      \makecell{Здо-\\буток} &
      Рента &
      \multirowcell{2}{Підвищення ренти} \\

      \cmidrule(r){2-2}
      \cmidrule(r){3-3}
      \cmidrule(r){4-4}
      \cmidrule(r){5-5}
      \cmidrule(r){6-6}

       & Шил. & Бушелі & Шил. & Шил. & Шил. &  \\
      \midrule
      A & 60 \dplus{} 60 \deq{} 120 & 5 \dplus{} 12\sfrac{1}{2} \deq{} 17\sfrac{1}{2}                      & 6\sfrac{6}{7} & 120  & \phantom{00}0 & \phantom{01 × }0 \\
      B & 60 \dplus{} 60 \deq{} 120 & 6 \dplus{} 15\phantom{\sfrac{1}{2}} \deq{} 21\phantom{\sfrac{1}{2}}  & 6\sfrac{6}{7} & 144  & \phantom{0}24 & \phantom{1 ×} 24 \\
      C & 60 \dplus{} 60 \deq{} 120 & 7 \dplus{} 17\sfrac{1}{2} \deq{} 24\sfrac{1}{2}                      & 6\sfrac{6}{7} & 168  & \phantom{0}48 & 2 × 24 \\
      D & 60 \dplus{} 60 \deq{} 120 & 8 \dplus{} 20\phantom{\sfrac{1}{2}} \deq{} 28\phantom{\sfrac{1}{2}}  & 6\sfrac{6}{7} & 192  & \phantom{0}72 & 3 × 24 \\
      E & 60 \dplus{} 60 \deq{} 120 & 9 \dplus{} 22\sfrac{1}{2} \deq{} 31\sfrac{1}{2}                      & 6\sfrac{6}{7} & 216  & \phantom{0}96 & 4 × 24 \\

     \cmidrule(r){6-6}
     \cmidrule(r){7-7}

      & & & & & 240 & 10 × 24 \\
  \end{tabular}

  \end{center}
\end{table}

\footnotetextZ{Це є таблиця висхідної продуктивности другої витрати капіталу. Порівн. табл. XXI. \Red{Прим. Ред}}

Варіянт 2: За низхідної продуктивности другої витрати капіталу, що не виключає
незмінюваної продуктивности першої витрати.

\begin{table}[H]
  \begin{center}
    \emph{Таблиця XX}
    \footnotesize

  \begin{tabular}{c@{  } c@{  } c@{  } c@{  } c@{  } c@{  } c}
    \toprule
      \multirowcell{2}{\makecell{Рід\\ землі}} &
      Ціна продукції &
      Продукт &
      \makecell{Продажна \\ ціна} &
      \makecell{Здо-\\буток} &
      Рента &
      \multirowcell{2}{Підвищення ренти} \\

      \cmidrule(r){2-2}
      \cmidrule(r){3-3}
      \cmidrule(r){4-4}
      \cmidrule(r){5-5}
      \cmidrule(r){6-6}

       & Шил. & Бушелі & Шил. & Шил. & Шил. &  \\
      \midrule
      A & 60 \dplus{} 60 \deq{} 120 & 10 \dplus{} 5 \deq{} 15  & 8 & 120  & \phantom{00}0 & \phantom{01 × }0 \\
      B & 60 \dplus{} 60 \deq{} 120 & 12 \dplus{} 6 \deq{} 18  & 8 & 144  & \phantom{0}24 & \phantom{1 ×} 24 \\
      C & 60 \dplus{} 60 \deq{} 120 & 14 \dplus{} 7 \deq{} 21  & 8 & 168  & \phantom{0}48 & 2 × 24 \\
      D & 60 \dplus{} 60 \deq{} 120 & 16 \dplus{} 8 \deq{} 24  & 8 & 192  & \phantom{0}72 & 3 × 24 \\
      E & 60 \dplus{} 60 \deq{} 120 & 18 \dplus{} 9 \deq{} 27  & 8 & 216  & \phantom{0}96 & 4 × 24 \\

     \cmidrule(r){6-6}
     \cmidrule(r){7-7}

      & & & & & 240 & 10 × 24 \\
  \end{tabular}

  \end{center}
\end{table}

Варіянт 3: За висхідної продуктивности другої витрати капіталу, що, за даних
припущень, обумовлює низхідну продуктивність першої витрати.

\begin{table}[H]
  \begin{center}
    \emph{Таблиця XXI}
    \footnotesize

  \begin{tabular}{c@{  } c@{  } c@{  } c@{  } c@{  } c@{  } c}
    \toprule
      \multirowcell{2}{\makecell{Рід\\ землі}} &
      Ціна продукції &
      Продукт &
      \makecell{Продажна \\ ціна} &
      \makecell{Здо-\\буток} &
      Рента &
      \multirowcell{2}{Підвищення ренти} \\

      \cmidrule(r){2-2}
      \cmidrule(r){3-3}
      \cmidrule(r){4-4}
      \cmidrule(r){5-5}
      \cmidrule(r){6-6}

       & Шил. & Бушелі & Шил. & Шил. & Шил. &  \\
      \midrule
      A & 60 \dplus{} 60 \deq{} 120 & 5 \dplus{} 12\sfrac{1}{2} \deq{} 17\sfrac{1}{2}                      & 6\sfrac{6}{7} & 120  & \phantom{00}0 & \phantom{01 × }0 \\
      B & 60 \dplus{} 60 \deq{} 120 & 6 \dplus{} 15\phantom{\sfrac{1}{2}} \deq{} 21\phantom{\sfrac{1}{2}}  & 6\sfrac{6}{7} & 144  & \phantom{0}24 & \phantom{1 ×} 24 \\
      C & 60 \dplus{} 60 \deq{} 120 & 7 \dplus{} 17\sfrac{1}{2} \deq{} 24\sfrac{1}{2}                      & 6\sfrac{6}{7} & 168  & \phantom{0}48 & 2 × 24 \\
      D & 60 \dplus{} 60 \deq{} 120 & 8 \dplus{} 20\phantom{\sfrac{1}{2}} \deq{} 28\phantom{\sfrac{1}{2}}  & 6\sfrac{6}{7} & 192  & \phantom{0}72 & 3 × 24 \\
      E & 60 \dplus{} 60 \deq{} 120 & 9 \dplus{} 22\sfrac{1}{2} \deq{} 31\sfrac{1}{2}                      & 6\sfrac{6}{7} & 216  & \phantom{0}96 & 4 × 24 \\

     \cmidrule(r){6-6}
     \cmidrule(r){7-7}

      & & & & & 240 & 10 × 24 \\
  \end{tabular}

  \end{center}
\end{table}
