
\index{iii2}{0172}  %% посилання на сторінку оригінального видання
Третій випадок: За висхідної ціни продукції.

А.~Коли земля $А$ не дає ренти й продовжує реґулювати ціну.

Варіянт 1: За незмінної продуктивности другої витрати капіталу, що зумовлює
низхідну продуктивність першої витрати.

 % ця мітка у заголовку
 % текст примітки прямо під заголовком

\begin{table}[h]
  \begin{center}
    \emph{Таблиця XIX\footnotemarkZ{}}
    \footnotesize

  \begin{tabular}{c@{  } c@{  } c@{  } c@{  } c@{  } c@{  } c}
    \toprule
      \multirowcell{2}{\makecell{Рід\\ землі}} &
      Ціна продукції &
      Продукт &
      \makecell{Продажна \\ ціна} &
      \makecell{Здо-\\буток} &
      Рента &
      \multirowcell{2}{Підвищення ренти} \\

      \cmidrule(r){2-2}
      \cmidrule(r){3-3}
      \cmidrule(r){4-4}
      \cmidrule(r){5-5}
      \cmidrule(r){6-6}

       & Шил. & Бушелі & Шил. & Шил. & Шил. &  \\
      \midrule
      A & 60 + 60 = 120 & 5 + 12\sfrac{1}{2} = 17\sfrac{1}{2}                      & 6\sfrac{6}{7} & 120  & \phantom{00}0 & \phantom{01 × }0 \\
      B & 60 + 60 = 120 & 6 + 15\phantom{\sfrac{1}{2}} = 21\phantom{\sfrac{1}{2}}  & 6\sfrac{6}{7} & 144  & \phantom{0}24 & \phantom{1 ×} 24 \\
      C & 60 + 60 = 120 & 7 + 17\sfrac{1}{2} = 24\sfrac{1}{2}                      & 6\sfrac{6}{7} & 168  & \phantom{0}48 & 2 × 24 \\
      D & 60 + 60 = 120 & 8 + 20\phantom{\sfrac{1}{2}} = 28\phantom{\sfrac{1}{2}}  & 6\sfrac{6}{7} & 192  & \phantom{0}72 & 3 × 24 \\
      E & 60 + 60 = 120 & 9 + 22\sfrac{1}{2} = 31\sfrac{1}{2}                      & 6\sfrac{6}{7} & 216  & \phantom{0}96 & 4 × 24 \\

     \cmidrule(r){6-6}
     \cmidrule(r){7-7}

      & & & & & 240 & 10 × 24 \\
  \end{tabular}

  \end{center}
\end{table}

\footnotetextZ{Це є таблиця висхідної продуктивности другої витрати капіталу. Порівн. табл. XXI. \emph{Прим. Ред}}

Варіянт 2: За низхідної продуктивности другої витрати капіталу, що не виключає
незмінюваної продуктивности першої витрати.

\begin{table}[h]
  \begin{center}
    \emph{Таблиця XX}
    \footnotesize

  \begin{tabular}{c@{  } c@{  } c@{  } c@{  } c@{  } c@{  } c}
    \toprule
      \multirowcell{2}{\makecell{Рід\\ землі}} &
      Ціна продукції &
      Продукт &
      \makecell{Продажна \\ ціна} &
      \makecell{Здо-\\буток} &
      Рента &
      \multirowcell{2}{Підвищення ренти} \\

      \cmidrule(r){2-2}
      \cmidrule(r){3-3}
      \cmidrule(r){4-4}
      \cmidrule(r){5-5}
      \cmidrule(r){6-6}

       & Шил. & Бушелі & Шил. & Шил. & Шил. &  \\
      \midrule
      A & 60 + 60 = 120 & 10 + 5 = 15  & 8 & 120  & \phantom{00}0 & \phantom{01 × }0 \\
      B & 60 + 60 = 120 & 12 + 6 = 18  & 8 & 144  & \phantom{0}24 & \phantom{1 ×} 24 \\
      C & 60 + 60 = 120 & 14 + 7 = 21  & 8 & 168  & \phantom{0}48 & 2 × 24 \\
      D & 60 + 60 = 120 & 16 + 8 = 24  & 8 & 192  & \phantom{0}72 & 3 × 24 \\
      E & 60 + 60 = 120 & 18 + 9 = 27  & 8 & 216  & \phantom{0}96 & 4 × 24 \\

     \cmidrule(r){6-6}
     \cmidrule(r){7-7}

      & & & & & 240 & 10 × 24 \\
  \end{tabular}

  \end{center}
\end{table}

Варіянт 3: За висхідної продуктивности другої витрати капіталу, що, за даних
припущень, обумовлює низхідну продуктивність першої витрати.

\begin{table}[h]
  \begin{center}
    \emph{Таблиця XXI}
    \footnotesize

  \begin{tabular}{c@{  } c@{  } c@{  } c@{  } c@{  } c@{  } c}
    \toprule
      \multirowcell{2}{\makecell{Рід\\ землі}} &
      Ціна продукції &
      Продукт &
      \makecell{Продажна \\ ціна} &
      \makecell{Здо-\\буток} &
      Рента &
      \multirowcell{2}{Підвищення ренти} \\

      \cmidrule(r){2-2}
      \cmidrule(r){3-3}
      \cmidrule(r){4-4}
      \cmidrule(r){5-5}
      \cmidrule(r){6-6}

       & Шил. & Бушелі & Шил. & Шил. & Шил. &  \\
      \midrule
      A & 60 + 60 = 120 & 5 + 12\sfrac{1}{2} = 17\sfrac{1}{2}                      & 6\sfrac{6}{7} & 120  & \phantom{00}0 & \phantom{01 × }0 \\
      B & 60 + 60 = 120 & 6 + 15\phantom{\sfrac{1}{2}} = 21\phantom{\sfrac{1}{2}}  & 6\sfrac{6}{7} & 144  & \phantom{0}24 & \phantom{1 ×} 24 \\
      C & 60 + 60 = 120 & 7 + 17\sfrac{1}{2} = 24\sfrac{1}{2}                      & 6\sfrac{6}{7} & 168  & \phantom{0}48 & 2 × 24 \\
      D & 60 + 60 = 120 & 8 + 20\phantom{\sfrac{1}{2}} = 28\phantom{\sfrac{1}{2}}  & 6\sfrac{6}{7} & 192  & \phantom{0}72 & 3 × 24 \\
      E & 60 + 60 = 120 & 9 + 22\sfrac{1}{2} = 31\sfrac{1}{2}                      & 6\sfrac{6}{7} & 216  & \phantom{0}96 & 4 × 24 \\

     \cmidrule(r){6-6}
     \cmidrule(r){7-7}

      & & & & & 240 & 10 × 24 \\
  \end{tabular}

  \end{center}
\end{table}
