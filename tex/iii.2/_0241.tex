\parcont{}  %% абзац починається на попередній сторінці
\index{iii2}{0241}  %% посилання на сторінку оригінального видання
продукції вартости відступають цілком на задній плян. Вже в безпосередньому
процесі продукції капіталіст діє одночасно як товаропродуцент, як керівник
товарової продукції. Тому цей процес продукції зовсім не уявляється йому просто
процесом продукції додаткової вартости. Але хоч би яка була та додаткова
вартість, яку капітал у безпосередньому процесі продукції висмоктував і втілював
в товари, вартість і додаткова вартість, що міститься в товарах, мусить реалізуватись
лише в процесі циркуляції. І справа набуває такого вигляду, ніби вартість,
яка покриває вартості, авансовані на продукцію, і особливо додаткова
вартість, що міститься в товарах, не просто реалізуються в циркуляції, але виникають
з неї; цю ілюзію особливо зміцнюють дві обставини: поперше, зиск,
одержуваний при відчуженні, залежить від обману, хитрощів, знання справи,
спритности й тисячі ринкових коньюнктур; а подруге, та обставина, що тут
поряд з робочим часом виступає другий визначальний елемент, час циркуляції.
Хоч він функціонує тільки як негативна межа створення вартости і додаткової
вартости, але має таку подобу, ніби він є так само позитивна причина
їх створення, як сама праця, і ніби він додає незалежне від праці визначення,
що походить з природи капіталу. У книзі II нам, природно, довелось подати цю
сферу циркуляції лише в її відношенні до визначень форм, які вона породжує,
показати дальший розвиток структури капіталу, який відбувається в цій сфері.
Але в дійсності ця сфера є сфера конкуренції, над якою, коли розглядати кожен
окремий випадок, панує випадковість; отже, сфера, в якій внутрішній закон,
що пробивається серед цих випадковостей і регулює їх, стає видимим лише тоді,
коли сполучити ці випадковості в велику масу, в якій, отже, він лишається
невидимим і незрозумілим для самих окремих аґентів продукції. Але далі: дійсний
процес продукції, як єдність безпосереднього процесу продукції і процесу
циркуляції, породжує нові витвори, в яких дедалі більше втрачається нитка
внутрішнього зв’язку, відносини продукції взаємно усамостійнюються, і складові
частини вартости костеніють у самостійних одна проти однієї формах.

Як ми бачили, перетворення додаткової вартости на зиск визначається так
процесом циркуляції, як і процесом продукції. Додаткова вартість, у формі зиску,
відноситься вже не до витраченої на працю частини капіталу, з якої вона виникає,
а до всього капіталу. Норма зиску регулюється власними законами, що
допускають і навіть зумовлюють її зміну за незмінної норми додаткової
вартости. Все це дедалі більше затушковує справжню природу додаткової вартости,
а тому й дійсний механізм капіталу. Ще в більшій мірі стається це
в наслідок перетворення зиску на пересічний зиск і вартостей на ціни продукції,
на регуляційні пересічні ринкових цін. Тут втручається складний суспільний
процес, процес вирівнювання капіталів, який відриває відносні пересічні
ціни товарів від їхніх вартостей, і пересічні зиски в різних сферах продукції
(залишаючи цілком осторонь індивідуальні вкладання капіталу в кожній окремій
сфері продукції) від дійсної експлуатації праці окремими капіталами. Тут не
тільки так здається, але й дійсно пересічна ціна товарів відмінна від їхньої
вартости, отже, від реалізованої в них праці, і пересічний зиск окремого капіталу
відмінний від додаткової вартости, яку цей капітал здобув з зайнятих ним
робітників. Вартість товарів виявляється безпосередньо лише в тому впливі, що
його справляють зміни продуктивної сили праці на пониження та підвищення цін
продукції, на їхній рух, а не на їхні кінцеві межі. Зиск, як здається, визначається
безпосередньою експлуатацією праці лише випадково, лише остільки,
оскільки ця експлуатація дає капіталістові можливість за наявности реґуляційних
ринкових цін, які видаються незалежними від цієї експлуатації, реалізувати
зиск, що відхиляється від пересічного зиску. Щождо самих нормальних пересічних
зисків, то вони здаються іманентними капіталові, незалежно від експлуатації;
ненормальна експлуатація, а також пересічна експлуатація за сприятливих
\parbreak{}  %% абзац продовжується на наступній сторінці
