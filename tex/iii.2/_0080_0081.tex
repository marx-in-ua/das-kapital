\parcont{}  %% абзац починається на попередній сторінці
\index{iii2}{0080}  %% посилання на сторінку оригінального видання
Рівень проценту не підвищився. І очевидно, що — оскільки братимемо на увагу
дійсний капітал, тобто тут товари — вплив на грошовий ринок є той самий,
все одно, чи ці товари призначено для закордону, чи для внутрішнього споживання.
Ріжниця була б тільки тоді, коли б приміщення англійського капіталу
закордоном впливали на комерційний експорт Англії, обмежуючи його, — експорт,
що його доводиться оплачувати, що, отже, тягне за собою зворотний приплив
капіталу, — або тільки тоді, коли б ці приміщення капіталу взагалі були вже
симптомом надмірного напруження кредиту та початку спекулятивних операцій.

Далі питає Вілсон, а відповідає Newmarch.

«1786. Раніше ви казали про попит на срібло для Східньої Азії, що, на
вашу думку, вексельні курси з Індією сприятливі для Англії, дарма що до Східньої
Азії невпинно відправлялось чималі металеві скарби; маєте підстави до цього? —
Звичайно\dots{} Я гадаю, що дійсна вартість вивозу Сполученого Королівства до
Індії становила в 1851 році 7.420.000 ф. ст.; до цього треба додати суму
векселів India House, тобто суму тих фондів, що їх витягає Ост-індська компанія
з Індії для покриття своїх власних видатків. Ці тратти становили в тім році
3.200.000 ф. ст.; так що цілий вивіз Сполученого Королівства до Індії становив
10.620.000 ф. ст. В 1855 році дійсна вартість товарового експорту піднеслася
до 10.350.000 ф. ст.; тратти India House становили 3.700.000 ф. ст.;
отже, цілий вивіз 14.050.000 ф. ст. Для 1851 року, думається мені, в нас немає
ніякого засобу визначити дійсну вартість довозу товарів з Індії до Англії; але
для 1854 та 1855 років ми можемо це зробити. В 1855 році дійсна вартість
цілого довозу товарів з Індії до Англії становила 12.670.000 ф. ст., і сума ця —
коли її порівняти з отими 14.050.000 ф. ст., лишає сприятливий для Англії
балянс в безпосередній торговлі між обома країнами на 1.380.000 ф. ст.».

На це Вілсон зауважує, що на вексельні курси впливає й посередня торговля.
Напр., вивіз з Індії до Австралії й Північної Америки покривається траттами
на Лондон і тому він впливає на вексельний курс цілком так само, як
коли б товари йшли безпосередньо з Індії до Англії. Далі, коли взяти Індію та
Китай разом, то балянс буде несприятливий для Англії, бо Китай раз-у-раз
має робити значні платежі Індії за опій, а Англія — Китаєві, й, таким чином, цим
кружним шляхом суми з Англії йдуть до Індії (1787, 88.).

1789. Тепер Вілсон запитує, чи не буде вплив на вексельний курс той
самий, всеодно, чи капітал «йтиме закордон у формі залізничих шин та локомотивів,
чи в формі металевих грошей». На це Newmarch відповів цілком слушно:
12 міл. ф. ст., останніми роками відправлені до Індії для будування залізниць,
послужили для купівлі річної ренти, що її Індія має платити Англії у певні
реченці. «Якщо мати на увазі безпосередній вплив на ринок благородного металу,
то приміщення тих 12 міл. ф. ст. може чинити такий вплив лише остільки,
оскільки доводилося відправляти метал для дійсного приміщення його у формі
грошей».

1797. [Weguelin питає:] «Коли не відбувається жодного зворотного припливу
за це залізо (шини), то як можна казати, що це впливає на вексельний курс? —
Я не думаю, що частина витрати, відправлена закордон у формі товарів, впливала
на стан вексельного курсу\dots{} на стан курсу між двома країнами, можна
сказати, впливає виключно тільки кількість зобов’язань або векселів, що їх
подають в одній країні, проти тієї кількости, що її подають у другій країні; така
є раціональна теорія вексельного курсу. Щождо відправи тих 12 мільйонів, то
їх передусім підписано тут; коли б ця операція була такого роду, що всі ці
12 міл. було б складено металевими грішми в Калькуті, Бомбеї та Мадрасі\dots{}
то цей раптовий попит надзвичайно вплинув би на ціну срібла й на вексельний
курс, цілком так само, як коли б Ост-індська компанія оповістила завтра, що
вона свої тратти збільшить від 3 до 12 міл. Але половину цих 12 міл. витрачено\dots{}
\index{iii2}{0081}  %% посилання на сторінку оригінального видання
на закуп товарів в Англії\dots{} залізних шин та дерева й інших матеріялів\dots{}
це — витрачання англійського капіталу в самій Англії на певний сорт
товарів, що його відправляється до Індії, й на тому кінець справи. — 1798.
[Weguelin:] Але продукція цих товарів з заліза та дерева, товарів, потрібних для
залізниць, породжує значне споживання закордонних товарів, а це могло б всеж
вилинути на вексельний курс? — Звичайно.

Вілсон гадає, що залізо представляє, здебільша, працю, а заробітна плата,
виплачена за цю працю, представляє здебільша імпортовані товари (1799), й
потім питає далі:

«1801. Але загалом кажучи: якщо товари, випродуковані через споживу
цих імпортованих товарів, вивозиться так, що ми не одержуємо за них назад
жодного еквіваленту, чи то продуктами, чи якось інак, — то чи не впливатиме
це на курс у несприятливому для нас напрямі? — Цей принцип точно висловлює те,
що відбувалося в Англії підчас великого будування залізниць (1845). Протягом
трьох або чотирьох, а то й п’ятьох послідованих років ви витратили на залізниці
30 міл. ф. ст. й майже всю цю суму на заробітну плату. Протягом трьох років,
будуючи залізниці, локомотиви, вагони та станції, ви годували більше число
люду, ніж по всіх фабричних округах разом. Ці люди\dots{} витрачали свою заробітну
плату на купівлю чаю, цукру, горілки та інших закордонних товарів; ці
товари доводилось імпортувати; але безперечно, що протягом того часу, коли робилось
ці великі витрати, вексельні курси між Англією та іншими країнами не
дуже були порушені. Не було відпливу благородного металу, навпаки, радше був
приплив його».

1802. Вілсон обстоює ту думку, що при рівновазі торговельного балансу
та паритетному курсі між Англією та Індією надзвичайна відправа заліза та локомотивів
«мусить впливати на вексельний курс з Індією». Newmarch не може
цього зрозуміти, бо поки шини відправляють до Індії, як приміщення капіталу,
Індія не має цього оплачувати в цій або тій формі; він додає до цього: «Я згоден
з тим принципом, що жодна країна не може протягом довшого часу мати
несприятливий вексельний курс з усіма тими країнами, з якими вона торгує;
несприятливий вексельний курс з однією країною неминуче породжує сприятливий
курс з якоюсь іншою». На це Вілсон відповідає йому такою тривіальністю:
«1803. Хіба ж перенесення капіталу не буде однаковим, чи відправлено той
капітал у тій або цій формі? — Так, оскільки мати на увазі боргове зобов’язання. —
1804. Отже, чи відправите ви благородний метал, чи товари, вплив залізничного
будівництва в Індії на ринок капіталу тут у нас був би однаковий і підвищить
вартість капіталу так само, як коли б усе відправлялося у формі благородного
металу?»

Якщо ціни на залізо не піднеслися, то це було в усякому разі доказом того, що
«вартість» уміщеного в шинах «капіталу» не збільшилася. «Вартість», про яку тут
мовиться, є вартість грошового капіталу — рівень проценту. Вілсон хотів би ототожнити
грошовий капітал з капіталом взагалі. Тут передусім той простий факт, що в
Англії було підписано 12 міл. на індійські залізниці. Це — справа, що безпосередньо
не має нічого до діла з вексельними курсами, і призначення тих 12 міл. є теж річ
байдужа для грошового ринку. Якщо грошовий ринок перебуває в сприятливому
стані, то це може взагалі не породжувати жодного впливу на нього, так само
як-от підписки на англійські залізниці в 1844 та 1845 роках не справили
впливу на грошовий ринок. Коли грошовий ринок перебуває вже до певної міри
у скрутному стані, то така підписка, звичайно, могла вилинути на рівень проценту,
але однак лише в напрямі піднесення, а це, за теорією Вілсона, мусило б вплинути на курс сприятливо
для Англії, тобто загальмувати тенденцію до вивозу благородного металу, якщо не до Індії, так бодай
куди інде. Пан Вілсон стрибає від однієї
справи до іншої. В питанні 1802 він каже, що вексельні курси порушилось би;
\parbreak{}  %% абзац продовжується на наступній сторінці
