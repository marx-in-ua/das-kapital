\parcont{} (абзац починається на попередній сторінці ¶)
\index{iii2}{0271}  %% посилання на сторінку оригінального видання
й ренти входять в розрахунки, як сталі й реґуляційні величини, — сталі не в тому
розумінні, що величини ці не зміняються, а в тому розумінні, що в кожному окремому
випадку вони є дані і становлять сталу межу для ринкових цін, які безупинно
коливаються. Наприклад, при конкуренції на світовому ринку справа
йде виключно про те, чи можна при даній заробітній платі, даному проценті й
даній ренті продати товар по даній загальній ринковій ціні або нижче цієї ціни
з вигодою, тобто реалізуючи при цьому відповідний підприємницький бариш.
Коли в одній країні заробітна плата і ціна землі низькі, а процент на капітал
високий, бо капіталістичний спосіб продукції тут взагалі нерозвинений,
тимчасом як в іншій країні заробітна плата і ціна землі номінально високі, а
процент на капітал низький, то капіталіст у першій країні вживає більше праці
й землі, в другій порівняно більше капіталу. Оцінюючи, в якій мірі можлива
конкуренція між обома цими капіталістами, обидва зазначені чинники треба
взяти на увагу як визначальні елементи. Отже досвід показує тут теоретично, а
заінтересовані розрахунки капіталіста показують практично, що ціни товарів
визначаються заробітною платою, процентом і рентою, ціною праці, капіталу
й землі, і що ці елементи ціни дійсно є реґуляційні, цінотворчі чинники.

Природна річ, при цьому завжди лишається один елемент, який є не
передумова, а наслідок ринкової ціни товарів, — саме, надмір над витратами продукції, що постають з
складання зазначених вище елементів: заробітної плати,
проценту й ренти. Цей четвертий елемент, як здається, визначається в кожному
окремому випадку конкуренцією, а в пересічному з цих випадків — пересічним
зиском, який знов таки регулюється тією самою конкуренцією, тільки за довший
період.

\emph{Поп’яте}. На базі капіталістичного способу продукції розпад вартости, що
в ній втілюється новодолучена праця, на доходи в формі заробітної плати, зиску
й земельної ренти є остільки сам собою зрозумілий, що цю методу застосовується
також там, де немає навіть умов існування цих форм доходу (ми не говоримо
вже про колишні історичні періоди, що їх ми наводили при дослідженні
земельної ренти). Це значить, що під зазначені форми доходу підводиться за
аналогією все що завгодно.

Коли самостійний робітник — візьмімо дрібного селянина, бо тут є застосовні
всі три форми доходу — працює на себе самого і продає свій власний продукт, то
його розглядається, поперше, як свого власного працедавця (капіталіста), що вживає
самого себе як робітника; подруге, як свого власного земельного власника, що править
для самого себе за орендаря. Як найманому робітникові, він виплачує собі заробітну
плату, як капіталістові дає собі зиск, як земельному власникові платить
собі ренту. Припускаючи, що капіталістичний спосіб продукції і відповідні
йому відносини є загальною соціяльною базою, це підведення є слушне
остільки, оскільки самостійний робітник не своїй праці, а своїй власності на
засоби продукції, — що взагалі кажучи, набули тут форми капіталу, — завдячує
тим, що він має змогу привласнити свою власну додаткову працю. І далі,
оскільки він продукує свій продукт як товар і тому залежить від ціни останнього
(а коли навіть цього й немає, то ціну все таки треба взяти на увагу),
маса додаткової праці, що він її може реалізувати як вартість, залежить не від
її власної величини, а від загальної норми зиску; так само той надмір, що його він
можливо одержує понад певну частку додаткової вартости, визначувану загальною
нормою зиску, залежить знов таки не від кількости витраченої ним праці, але
може бути привласнений ним лише через те, що він є власник землі. А що
такі форми продукції, які цілком не відповідають капіталістичному способові
продукції, можуть бути підведені — і до певної міри не без слушности — під капіталістичні форми
доходу, то тим дужче зміцнюється ілюзія, ніби капіталістичні
відносини є природні відносини всякого способу продукції.
\parbreak{}  %% абзац продовжується на наступній сторінці
