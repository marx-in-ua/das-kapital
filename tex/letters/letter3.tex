\documentclass{kapital}
%% should be a class option
\renewcommand{\parbreak}{\unskip\ignorespaces}
\renewcommand{\parcont}{\unskip\ignorespaces}

%% proper quote marks 
\newunicodechar{„}{«}
\newunicodechar{“}{»}

%% ditto mark
\renewcommand{\dittomark}{~}

%% overfull boxes
\vfuzz=12pt

\usepackage{pgffor}
\renewcommand{\labelitemi}{\textemdash{}}
\pagestyle{empty}

\def\names{
Дорога Олено,
Дорогий Дмитре,
Дорогий Дмитре,
Дорогий Дмитре,
Дорогий Вадиме,
Дорогий Ернесте,
Дорога Яно,
Дорогий Валентине,
Дорогий Володимире,
Дорогий Володимире,
Дорогий Миколо,
Дорогий Юрій,
Дорога Юліє,
Дорогий Вікторе,
Вітаємо,
Вітаємо,
Вітаємо,
Дорогий Сергію,
Дорога Катерино,
Дорога Вікторіє%
}

\begin{document}

\foreach \name in \names {
  \section*{\name{}!}
  \thispagestyle{empty}

  \subsubsection*{Термін}

  Будь ласка, вичитайте текст і надішліть нам свої правки до 4 жовтня.
  При читанні першого тому лише кожен третій надіслав нам свої правки.
  Встановіть собі планку в 5 або 10 сторінок на день і притримуйтесь цього 
  графіку. Не відкладайте читання на останній день — цей підхід тут навряд чи спрацює.

  \subsubsection*{Яким чином надсилати правки}

  Ідеальний варіант виглядає так:
  \begin{itemize}
  \item Вносите правки на папері
  \item Скануйте сторінки з правками
  \item Створюєте альбом google photo і розшарюєте його з marx.ukr@gmail.com
  \item Напишіть нам у FaceBook, що ви впорались
  \end{itemize}

  \noindent{}Життя не ідеальне, тому якщо у вас не вийде надіслати правки за ідеальним
  сценарієм відправте їх нам як завгодно: починаючи від фото фрагментів 
  сторінки повідомленнями через FaceBook закінчуючи відсилкою через Нову Пошту. 
  Ми все одно будемо вдячні.

  \subsubsection*{Помилки та орфографія}

  Умовно на кожних 10 сторінках тексту залишилось близько 4 помилок. 
  Розподілені вони нерівномірно: може бути 10 сторінок без жодної помилки, 
  а за ними – сторінка-катастрофа з 12 помилками. Отож, будьте пильними.

  \smallskip
  \noindent{}Наш спільний проект — повторне видання текстів 30-х років, тому орфографія
  може здаватися незвичною. Але це — не помилки. Найчастіше вам будуть
  зустрічатися:
  \begin{itemize}
  \item я замість а в запозичених словах (\emph{соціялний, клясичний, Голляндія})
  \item ґ замість г в запозичених словах (\emph{проґрес, реґулюють, континґент})
  \item архаїчні граматичні форми (працю \emph{вимірюється} безпосередньо часом, при \emph{машиновому} вибиванні, раптом \emph{порушиться} звичну рівновагу)
  \end{itemize}

  \noindent{}Якщо у вас є бажання звіритися з оригінальним виданням — 
  скан книжки знаходиться за посиланням \underline{bit.ly/marx\_book}.

  \subsubsection*{Зворотній зв'язок}


  Якщо у вас залишилися питання, то найбільш швидкий спосіб отримати 
  на нього відповідь це — телеграм, пишіть @toha\_kabina. Якщо вам 
  більш до вподоби Фейсбук то пишіть в повідомлення в 
  групу(\underline{fb.me/marx.in.ua}).

  \subsubsection*{Дякуємо!}

  Дякуємо, що долучилися до завершального раунду коректури другого 
  тому Марксового «Капіталу». Ми займаємось цією справою вже близько 
  5 років і кожний, хто долучився надає нам наснаги рухатись далі.

  \subsubsection*{Як це працює}
  
  Ми використовуємо пошук за зразком. Тобто, коли ми бачимо помилку,
  ми намагаємось знайти схожі помилки в інших місцях. Це означає, 
  що одна помилка, яку ви знайшли може привести до знаходження 
  10 аналогічних помилок. 
  
  \bigskip{}

  \noindent{}Волонтери «Капіталу»,

  \medskip{}

  \noindent{}Денис Потапов

  \noindent{}Ірина Зробок

  \noindent{}Антон Потапов
}

\end{document}

