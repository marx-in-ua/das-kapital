\documentclass{kapital}
%% should be a class option
\renewcommand{\parbreak}{\unskip\ignorespaces}
\renewcommand{\parcont}{\unskip\ignorespaces}

%% proper quote marks 
\newunicodechar{„}{«}
\newunicodechar{“}{»}

%% ditto mark
\renewcommand{\dittomark}{~}

%% overfull boxes
\vfuzz=12pt
\hfuzz=1pt

\newcommand{\Year}{2023}
% \newcommand{\Year}{2024}
% \newcommand{\Year}{2025}

\newcommand{\City}{Київ}

\usepackage{pgffor}
\renewcommand{\labelitemi}{\textemdash{}}
\pagestyle{empty}



  Вітаємо!

\begin{document}

  \thispagestyle{empty}

  \subsubsection*{Термін}

  Будь ласка, вичитайте текст і надішліть нам свої правки не пізніше ніж через 30 днів після отримання фрагменту.
  При читанні першого тому лише кожен третій надіслав нам свої правки.
  Встановіть собі планку в 5 або 10 сторінок на день і притримуйтесь цього 
  графіку. Не відкладайте читання на останній день — цей підхід тут навряд чи спрацює.

  \subsubsection*{Яким чином надсилати правки}

  Ідеальний варіант виглядає так:
  \begin{itemize}
  \item Вносите правки на папері
  \item Скануйте сторінки з правками
  \item Створюєте альбом google photo і розшарюєте його з marx.ukr@gmail.com
  \item Напишіть нам у FaceBook, що ви впорались
  \end{itemize}

  \noindent{}Життя не ідеальне, тому якщо у вас не вийде надіслати правки за ідеальним
  сценарієм відправте їх нам як завгодно: починаючи від фото фрагментів 
  сторінки повідомленнями через FaceBook закінчуючи відсилкою через Нову Пошту. 
  Ми все одно будемо вдячні.

  \subsubsection*{Помилки та орфографія}

  Умовно на кожних 10 сторінках тексту залишилось близько 4 помилок. 
  Розподілені вони нерівномірно: може бути 10 сторінок без жодної помилки, 
  а за ними – сторінка-катастрофа з 12 помилками. Отож, будьте пильними.

  \smallskip
  \noindent{}Наш спільний проект — повторне видання текстів 30-х років, тому орфографія
  може здаватися незвичною. Але це — не помилки. Найчастіше вам будуть
  зустрічатися:
  \begin{itemize}
  \item я замість а в запозичених словах (\emph{соціялний, клясичний, Голляндія})
  \item ґ замість г в запозичених словах (\emph{проґрес, реґулюють, континґент})
  \item архаїчні граматичні форми (працю \emph{вимірюється} безпосередньо часом, при \emph{машиновому} вибиванні, раптом \emph{порушиться} звичну рівновагу)
  \end{itemize}

  \noindent{}Якщо у вас є бажання звіритися з оригінальним виданням — 
  скан книжки знаходиться за посиланням \underline{bit.ly/marx\_book}.

  \subsubsection*{Не зациклюйтесь}

  Наприклад, ви побачили в нашому виданні £, а в оригіналі всюди ф. стер. Відмітьте
  цю помилку 2 рази й рухайтесь далі. Бо це насправді не помилка, а зміна правопису.
  Навіть якби це була помилка - ми б знайшли всі подібні помилки регулярними виразами.
  Так що не зациклюйтесь на однакових помилках.

  \subsubsection*{Зворотній зв'язок}


  Якщо у вас залишилися питання, то найбільш швидкий спосіб отримати 
  на нього відповідь це — телеграм, пишіть @toha\_kabina. Якщо вам 
  більш до вподоби Фейсбук то пишіть в повідомлення в 
  групу(\underline{fb.me/marx.in.ua}).

  \subsubsection*{Дякуємо!}

  Проєкт Марксів «Капітал» українською триває вже 8 років і, можливо, триватиме ще стільки ж. Ми вдячні, що ви
  розділили з нами частину цього шляху.
  
  \bigskip{}

  \noindent{}Волонтери «Капіталу»,

  \medskip{}

  \noindent{}Денис Потапов

  \noindent{}Ірина Зробок

  \noindent{}Антон Потапов

\end{document}

