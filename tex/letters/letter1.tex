\documentclass{kapital}
%% should be a class option
\renewcommand{\parbreak}{\unskip\ignorespaces}
\renewcommand{\parcont}{\unskip\ignorespaces}

%% proper quote marks 
\newunicodechar{„}{«}
\newunicodechar{“}{»}

%% ditto mark
\renewcommand{\dittomark}{~}

%% overfull boxes
\vfuzz=12pt

\usepackage{pgffor}
\renewcommand{\labelitemi}{\textemdash{}}
\pagestyle{empty}

\def\names{
Дорогий Василю,
Дорога Ірино%
}

\begin{document}

\foreach \name in \names {
  \section*{\name{}!}
  \thispagestyle{empty}

  Ми вдячні вам за те, що погодились взяти участь у тестовому читанні Марксового «Капіталу».
  Насолоджуйтесь! Але пам'ятайте, що в тексті ще залишились помилки які ми хочемо знайти з вашою допомогою.

  \subsubsection*{Помилки та орфографія}

  На кожні 10 сторінок залишилось близько 4 помилки. Розподілені вони не рівномірно, може бути 10 сторінок без жодної помилки, а за ними сторінка-катастрофа з 12 помилками. Будьте пильні.

  Наш спільний проект — повторне видання текстів 30-х років. І орфографія може здаватися незвичною. Але це — не помилки. Найчастіше вам будуть зустрічатися:
  \begin{itemize}
  \item я замість а в запозичених словах (\emph{соціялний, клясичний, Голляндія})
  \item ґ замість г в запозичених словах (\emph{проґрес, реґулюють, континґент})
  \item архаїчні граматичні форми (працю \emph{вимірюється} безпосередньо часом, при \emph{машиновому} вибиванні, раптом \emph{порушиться} звичну рівновагу)
  \end{itemize}

  \noindent{}Якщо у вас є бажання звіритися з оригінальним виданням — скан книжки знаходиться за посиланням додати сюди посилання.

  \subsubsection*{Зворотній зв'язок}

  Найшвидший шлях отримати відповідь на будь-які запитання написати нам у фейсбуці лінк. Відмічайте помилки, які ви помітили в тексті. Та повідомте нам:

  \begin{itemize}
  \item Сфотографувати або відсканувати сторінки з помилками. Відправте їх нам на пошту marx.ukr@gmail.com (або завантажте через google photo).
  \item Відправити назад сторінки з помилками новою поштою. На ім’я Антон Потапов, 093~770~01~45, м.~Київ, Відділення~№58
  \end{itemize}

  \subsubsection*{Дякуємо!}

  Ми почали цей проект 3 роки назад. До проекту вже долучилось 80 волонтерів та волонтерок. Разом вони витратили на проект майже 1000 годин. Тож Марксів «Капітал» показав, яка велика кількість інтелектуалів та інтелектуалок серед нас, що готові діяти спільно. 

  \bigskip{}

  \noindent{}Волонтери Марксового «Капіталу»,

  \medskip{}

  \noindent{}Денис Потапов

  \noindent{}Ірина Зробок

  \noindent{}Антон Потапов
}

\end{document}

