\index{franko}{0064}

\vspace{\medskipamount}
\subsection{Вивласненє хліборобів}
\vspace{\medskipamount}

\looseness=1
В Англії щезло кріпацтво дійсно в послідній части \RNum{14} віку. Огромна більшість людности тоді, а ще
більше в \RNum{15} віці, се були свобідні хлібороби, дрібні посідачі ґрунтів, ґазди, — хоть власність їх і
була прикрита різними феодальними прикривками\footnote{
Ще при кінци \RNum{17} віку звиж \sfrac{4}{5} усеї англійської людности були самостійні ґазди-хлібороби, як се
стверджує Маколєй. (Macaulay: „The History of England“, Lond. 1854, v. I, p. 413). Я покликуюсь на
Маколєя тим радше, що він сістематично фальшує історію і подібні факти стараєсь о кілько мож
„обкроювати“.
}. В більших панських добрах замісць давнійших
кріпаків-совтисів (bailiff) настали тепер свобідні арендаторі. Наємні робітники до хліборобства, се
були по части самостійні ґазди-хлібороби, котрі при вільнім часі йшли до пана на заробок, а по части
була се відрубна, стосунково і абсолютно мала верства властивих наймитів. І ті послідні на ділі були
також самостійними ґаздами, бо крім платні одержували від пана також поле коло 4 екрів завбільшки і
коттедж (хату). Притім порівно с прочими ґаздами вони допущені були до вживаня громадського ґрунту,
т. є. толоки, де паслась їх худоба, і ліса, відки вони брали топливо, дерево, торф і пр.\footnote{
Не тре забувати, що навіть і кріпак був не тілько властивцем — хоть за оплатою — тих часток
ґрунту, котрі належали до єго дому, але був також співвластивцем громадських ґрунтів. (Порівн., що
каже Мірабó про шльонських хліборобів в книжці: „De la Monarchie Prussiennе“, Londres 1788).
} У всіх краях Европи ціхує феодальну продукцію поділ ґрунту поміж як мож найбільше підданих. Сила
феодального пана, як і сила кождого короля, полягала не в великости єго доходів, а в многоті єго
підданих, а многота сеся залежала від многоти самостійних ґаздів, осілих на єго добрах\footnote{
Японія зі своїм чисто феодальним упорядкованєм ґрунтової власности і розвитим дрібним
ґосподарством хліборобським вказує далеко вірнійший образ середновікової Европи, ніж усі накупі наші
історії, звичайно закаламучені буржоазними пересудами. Се, бач, дуже вигідна річ —
„ліберальствувати“ на кошт середних віків!}. Хоть
затим англійський край по норманськім завойованю поділено на величезні баронства, с котрих одно
нераз містило в собі 900 анґльосаских льордств, то прецінь край той був покритий дрібними
хліборобськими ґаздівствами, серед котрих тілько декуди розлягалися великі панські добра. Такі
стосунки при рівночаснім росцвіті міст, котрий наступив в \RNum{15} віці, сприяли заможности люду, яку
описує канцлєр державний Фортеске в своїх „Laudes Legum Angliae“, але при них не можливе було
капіталістичне богацтво.
