\parcont{}
\index{franko}{0073}
цілі німецькі князівства), а також окремою формою ґрунтової власности, котру так насильно перемінюют
в приватну власність. Ті ґрунти, то була власність повіту (clan), — начальник або „великий чоловік“
був тілько титулярним властивцем, як представник повіту, так само, як королева англійська є
титулярною властителькою всего ґрунту Англії. Тот переворот, котрий в Шотляндії почався по посліднім
повстаню претендента, мож слідити в перших єго початках у письмах Джемса Стеєрта і Джемса Андерсона\footnote{
Стеєрт каже: „Рента в тих околицях (він хибно називає рентою тоту оплату, яку обивателі повіту
(taksmen) складали начальникови повіту) зовсім незначна в стосунку до обширности піль, але що до
числа осіб, котрих удержує одна аренда, мож сміло твердити, що оден кусник ґрунту в шотлянських
горах виживлює десять раз більше людей, ніж так само заобширний ґрунт в найбогатших рівнинах“.
}. В \RNum{18} віці заборонено притім Ґелям, прогнаним з ґрунтів, виселюватись в чужі краї, щоб їх таким
способом силою попхнути до Ґлязґова і других фабричних міст\footnote{
1860 виводжено тих насильно вивласнених хліборобів до Канади, отуманивши їх фальшивими
обіцянками. Деякі повтікали в гори і на сусідні пусті острови. Поліція пустилася за ними в погоню,
прийшло до бійки і втікачі здужали вирватися та порозбігатись.
}. За примір
методи пануючої в девятнайцятім віці\footnote{
„В шотляндських горах“, каже Бюкенен, коментатор А.~Сміта, 1814, „день в день насильно затираєся
давний власностевий порядок\dots{} Сільский льорд, без згляду на дідичних арендаторів (знов хибно
названі тексмени), винаймає ґрунт тому, хто найбільше платит, а коли той належит до меліораторів
(imprower), то зараз заводит новий спосіб управи поля. Ґрунт, давнійше покритий дрібними
властивцями, був в стосунку до своєї плодовитости досить заселений; при новім сістемі поліпшеної
управи і побільшеної ренти одержуєсь як мож найбільше плодів як мож найменьшим коштом, і для того
віддалюются робітники, котрі стали тепер непотрібними. Ті вигнанці з рідних хат шукают відтак
утриманя в фабричних містах і т. д. (David Buchanan: „Observations on A.~Smith’s Wealth of Nations.
Edinb. 1814“.) „Шотляндські маґнати вивласнили цілі родини, немов хопту випололи: вони так обійшлися
з селами й людністю, як Інди розїдлі пімстою з дикими звірями по норах\dots{} Чоловіка продают
за овече руно, за волове стегно, ба ні, ще за меньшу дрібницю\dots{} Підчас нападу на північні
провінції Хіни була на раді Монголів така думка, щоб усіх мешканців витратити а їх край перемінити
в степ. Тоту раду богато північно-шотляндських маґнатів дословно виповнили в своїм власнім краю і на
своїх власних земляках“. (Джордж Ензер: „An Inquiry concerning the Population of Nations. Lond.
1818“. Стор. 215, 216.)
} досить буде ту навести „обчищуваня“ герцоґині Созерлєнд.
Тота в економії вишколена особа постановила зараз в початку свого панованя взятися до радікального
ліку економічного, і ціле ґрафство, в котрім задля давнійших подібних процесів осталось
\index{franko}{0074}
було всего лиш \num{15000} люда, перемінити в толоку для овець. Від 1814 до 1820 сістематично
прогонювано та нищено тих \num{15000} мешканців, т. є. майже 3000 родин. Всі їх села поруйновано і
попалено, всі їх поля пороблено толоками. Англійських жовнірів викомендерувано там для еґзекуції, і
між ними а мешканцями прийшло до бійки. Одна
стара баба згоріла враз іс хатою, с котрої не хтіла вступитися. І таким способом присвоїла собі
вельможна герцоґиня \num{794000} екрів ґрунту, котрі споконвіку належали до
повіту. Вигнаним мешканцям визначила вона на морськім узберіжю около 6000 екрів, по 2 екри на
родину. Тих 6000 екрів лежали доси пусто і не давали властительці ніякого
доходу. Герцоґиня так далеко зайшла в своїй щедрости, що винаймила екр пересічно по 2\shil{ шілінґи}
6\pens{ пенсів} для тих самих селян, котрі много сот літ проливали кров свою за
вельможну герцоґську родину. Увесь зрабований ґрунт повіту поділила герцоґиня на 29 великих аренд
для випасаня овець; в кождій аренді осіла тілько одна родина, переважно англійські наємні
арендаторі. 1825 р. замісць \num{15000} Ґелів на їх ґрунтах жило вже \num{181000} овець. А родини, вивержені на
морський беріг, старалися жити риболовством. З них поробилися земноводяні, і вони жили, як каже
писатель, на половину в воді, а на половину на березі, тілько що ні ту ні там не могли найти
достаточного прожитку\footnote{
Коли теперішна герцоґиня Созерлєнд витала в Льондоні з великою парадою міссіс Бічер Стоу,
авторку „Хати дядька Томи“, щоб виставити на показ свою прихильність для муринів-невольників в
американській републіці — чого вона і єї співарістократки певно не булиб зробили підчас домашної
війни американської, бо тоді кожде „шляхотне“ англійське серце було прихильне плянтаторам — в той
сам час описав
я в газеті „New-York-Tribune“ побут невольників созерлєндських. (Деякі місця тої статі навів Керей в
своїй „The Slave Trade. London 1853“.). Мою статю перепечатала одна шотляндська ґазета і викликала
дуже чемну перепалку між тою ґазетою а підхлібниками та похвальками герцоґів Созерлєндів.
}.

Але небораки Ґелі мусіли щераз відпокутувати свою романтичну наклінність для „великих мужів“, т. є.
для начальників повітових (Сlanchef). Запах риб, котрими прокормлювались земноводяні Ґелі, ударив
великим мужам в ніс. Вони завітрили тут щось зисковного і заарендували морське узберіжє великим
льондонським гендлярам риб. Ґелів другий раз вигнано на штири вітри\footnote{
Цікаву історію того рибного торгу найде читатель у д. Девіда Оркуарта в єго книжці: „Portofolio.
New Series“. Сеніор в одній іс своїх посмертних статей називає „процедуру в Созерлєндшайрі“ одним з
найблагодатнійших очищень від віків.
}.

Аж вкінци одну часть пасовиськ назад перемінено
\parbreak{}
