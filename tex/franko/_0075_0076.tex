\parcont{}
\index{franko}{0075}
в місця для польованя. Звісна річ, що в Англії нема правдивих гір. Дичина в маґнацьких парках, се
констітуційна домашна худоба, товста, як льондонські ольдермени. Шотляндія, се затим послідне місце
для „шляхотного занятя“ — стрілецтва. „В шотляндських горах“, каже Сомерс 1848, „ліси дуже стали
обширні. Ось з одного боку від Ґейка маєте новий ліс Ґлєнфшай, а з другого боку також новий ліс
Ордверікай. Поряд з ними бачите Блік-Моунт, огромну пустиню, свіжо заложено. Від сходу до заходу,
від околиць Обердіна аж до урвищ Обена тягнутся тепер без перерви ліси. А й по других частях
Шотляндії находятся нові ліси, як ось в Льоч Орчейґ, Ґлєнджеррі, Ґлєнморістен і др\dots{} Переміна
ґрунтів в овечі толоки прогнала Шотляндців на неврожайні пустарі. Тепер серни та лиси починают
витискати овець, а Шотляндців кидати ще в страшнійшу нужду\dots{} „Ліси для дичини“\footnote{
Шотляндські „ліси для дичини“ (deer forests) не мают і одного деревця. Се звичайні голі толоки,
с котрих вигнано вівці, а на їх місце нагнано оленів, — тай ось і преславетний „ліс для дичини“. Про
засів та плеканє лісів і казки нема!
} а народи не
можут істнувати побіч себе. Або одні, або другі мусят уступити місця. Нехай тілько число і розмір
польовань в слідуючих 25 роках зможеся так, як в минувших, то певно не здиблете й одного Шотляндця в
єго ріднім краю. А се змаганє між шотляндськими маґнатами походит по части з моди, панської бути,
забагів на польованє і т. д., а по части вони займаются дичиною для зиску. Бо кусень гористого
простору, затичений для польованя, нераз далеко більше дає зиску, ніж колиб був толокою. Той, кому
забаглося польованя і шукає такого обшару, платит за него тілько, накілько статчит єго кішеня,
— (а се вже певно, що бідному чоловікови не до польованя!)\dots{} Відти то сплило на шотляндські гори
тілько недолі, кілько єї сплило на Англію ізза політики норманських
королів. Оленям віддано огромні простори до волі, а людей зігнано в тісні і чим раз тіснійші
закамарки\dots{} Одну вільність за другою видирано народови\dots{} І притиск той ще
день-денно змагаєся. Властивці з засади і загалом вимітают і виганяют народ, мов сіно косят, — мов
ті австральські та американські осадники відвічні ліси витинают, і тота операція поступає чим раз
далі, спокійно, с холодною розвагою та обрахунком\footnote{
Robert Somers: „Letters from the Highlands: or, the Famine of 1847. Lond. 1848“, стор. 12--28. Ті
листи надрукувала зразу ґазета Таймc. Англійські економісти, розумієся, зараз ростолкували, що
Шотляндці бідуют і мрут з голоду задля — перелюдненя. Сяк чи так, а їсти не було що. І в Німеччині
не чужа тота операція „Clearing of Estates“, — єї ту прізвано „Bauernlegen“ (обалюванє мужиків), і
вона особливо далася чути по 30-тилітній війні. Ще 1790 в одній части Саксонії вона викликала хлопські бунти. А найбільше
ширилося „обалюванє мужиків“ в східній Німеччині. В найбільшій части німецьких провінцій аж Фрідріх
II запевнив селянам право власности. По завойованю Шльонська він присилував дідичів до відбудованя
хат, стоділ і т. д., а також до заосмотреня мужицьких осад худобою та ґосподарськими знадобами.
Фрідріх II потребував жовнірів для війська і оподаткованих людей для побільшеня державного
скарбу. Впрочім селянам і під Фрідріхом II жилось далеко не гарно, як се мож побачити с письм єго
головного похвальця, Мірабо.

В цвітню 1872, 18 літ по виданю згаданої ту книжки Сомерса читав проф. Лєоні Лєві в „Society of
Arts“ відчит про переміну овечих пасовиск в ліси для дичини, де вказує дальший розвиток спустошеня в
шотлянських височинах. Між їншим каже він: „Обезлюдненє і переміна ґрунтів в голі толоки, се був
найвигіднійший для панів спосіб — получити доходи без видатків\dots{} Теперже в височинах звичайно
пороблено с толок „deer forest“-и. Дичина прогнала овець, так як недавно вівці прогнали були людей. Мож вандрувати від
дібр ґрафа Дельгаузі в Форфершайрі аж до Джона o’Ґротса, не виходячи зовсім з лісів. В многих іс
тих лісів замешкуют лиси, дикі коти, куни, тхорі, ласиці та альпейські заяці; від недавна
росплодилися там також крілики, вивірки та щурі. Огромні простори, котрі в шотлянській статистиці
значились „надзвичайно врожайні і розляглі пастівники“, позбавлені тепер всякої управи і поправи і служат виключно для мисливської
забави кількох осіб, тай то лиш короткий час в році!“.

Льондонський „Economist“ з 2 червця 1866 каже: „Одна шотлянська ґазета доносит послідного тижня між
їншими новинами ось що: Одна з найкращих овечих аренд в Созерлєндшайрі, за котру недавно, за упливом
біжучого арендового контракту, давано річної ренти 1200\pound{ фунтів штерлінґів}, тепер зістає перемінена в
„deer forest“! Феодальні інстінкти проявляются й тепер так само, як тоді, коли норманський
завойовник зруйнував 36 сіл, щоб закласти Ню-форс (Новий ліс)\dots{} Два мілійони екрів, самих
найурожайнійших в Шотляндії, опустошуются тепер до крихітки. Природна трава Ґлєн-Тільту належала до
найпоживнійших в ґрафстві Перз; теперішний дір-форст Бен-Альдер був найкращим пасовиском в цілім
лісистім Бедноч; одна часть теперішного Блік-Моунт-форста була найліпшим на всю Шотляндію пасовиском
для чорномордих овець. Про обсяг ґрунтів опустошеннх для стрілецької примхи мож виробити собі
яке-таке понятє, зваживши, що вони обіймают далеко більше простору, ніж ціле ґрафство Перз. Що через
те насильне опустошенє стратив край на жерелах продукції, мож оцінити с того, що форс Бен-Альдер міг
% REMOVED \footnote*{В рукописі: між.}
прокормити \num{15000} овець і що він становит лиш \sfrac{1}{30} всіх „диких лісів“ шотлянських.
Весь той „дикий“ ґрунт зовсім не продуктівний\dots{} На одно б вийшло, як би був запався в фалі
Північного моря. Сильна рука праводавства повинна би прецінь зупинити розріст і творенє таких
самовільних пустинь“.
}.

\index{franko}{0076}
Рабунок дібр церковних, злодійське загарбуванє державних маєтків, крадіж громадських ґрунтів,
безправна \parbreak{}
