ванєм тих, „котрих піхто не хоче взяти на службу. Єлисаветина
5 устава про учеників ремісницьких, уст. 3. надає
мировим судям власть становити де в яких реміслах плату
і змінювати ї відповідно до пори року і ціни товарів. Яков
I ростягнув ту саму реґуляцію робітницької плати на ткачів,
прядільників і на всі можливі розряди робітників\footnote{
З одної примітки до устави 2. за Якова І, розд. 6. видно, що
деякі суконники позваляли собі самі яко мирові суды урядово діктувати
платну тарифу в своїх варстатах. — В Німеччині, а іменно по 30-літній
війні, виходнт богато устав для знижуваня робучої плати. „Помічникам
на безлюдних ґрунтах дуже прикро давалась чути недостача слуг і робітників.
Всім мужикам-ґаздам заказано приймати в комірне мужчин та
женщин вільного стану; про всіх таких комірників повинно доноситися
урядови, а той запирає їх в тюрму, скоро не хотят стати слугами, хоть би
й без того мали яке їнше вдержанє, хоть би працювали у  мужиків за поденщину
або навіть торгували грішми та збіжєм. (Цісарські прівілєї та
ухвали для Шльонська, І, стор. 125). Через цілих сто літ роздаются в приписах
князів та поміщиків раз-відразу гіркі наріканя на злосливих
і здуфалих слуг, що не хотят піддатися важким условинам, не хотят вдоволюватися
платою правом приписаною. Виходят накази, щоб поєдинчпй
поміщик не смів своїм слугам платити більше, ніж кілько весь краєвий
збір покладе в таксу. А прецінь условини служби по війні нераз ще
бувают ліпші, ніж були 100 літ опісля. В р. 1052 діставали ще слуги на
Шльонську по два рази до тижня мясо; а ще в нашім столітю іменно
там були такі округи, де слуги діставали мясо хіба три рази до року.
І поденщина (плата за день роботи) по 30-літній війні була більша ніж
в слідуючих столітях“ (Ґустав Фрейтаг).
}, Джордж
II ростягнув устави протів робітницьких товариств на всі
мануфактури. В властивій порі мануфактуровій капіталістична
продукція була вже досить сильною, щоб правну
реґуляцію робучої плати зробити непотрібною, а то й неможливою,
але все такі ще на всякий злучай не закидувано
того перестарілого оружя. Ще 8. устава Джорджа II заказує
давати кравецьким челядникам в Льондоні і околици більше
понад 2 шіллінґи і півосьма пенса денної плати, окрім хіба
в разах загальної жалоби. ІЦе 13 уст. Джорджа III, розд.
68. повіряє мировим судям реґульованє робучої плати у виробників
шовку. Ще 1796 тре було двох декретів висших
судів для рішеня, чи накази мирових судьїв що до робучої
плати мают вагу і для нерільничих робітників. Ще 1799.
потвердила ухвала парляменту, що плата копальників шотландських
уреґульована уставою Єлисавети і двома шотляндськими
актами з р. 1661 і 1671. А який між тим переворот
доконався у всіх обставинах, доказала подія нечувана
в англійській палаті панів. Ту, де від звиш 400 літ
фабриковано устави виключно о тім, понад яку міру не
може ніяк переступити робуча плата, — ту поставив 1799