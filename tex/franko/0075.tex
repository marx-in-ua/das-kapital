в місця для польованя. Звісна річ, що в Англії нема правдивих гір. Дичина в маґнацьких парках, се
констітуційна домашна худоба, товста, як льондонські ольдермени. Шотляндія, се затим послідне місце
для „шляхотного занятя“ — стрілецтва. „В шотляндських горах“, каже Сомерс 1848, „ліси дуже стали
обширні. Ось з одного боку від Ґейка маєте новий ліс Ґлєнфшай, а з другого боку також новий ліс
Ордверікай. Поряд з ними бачите Блік-Моунт, огромну пустиню, свіжо заложено. Від сходу до заходу,
від околиць Обердіна аж до урвищ Обена тягнутся тепер без перерви ліси. А й по других частях
Шотляндії находятся нові ліси, як ось в Льоч Орчейґ, Ґлєнджеррі, Ґлєнморістен і др\dots Переміна
ґрунтів в овечі толоки прогнала Шотляндців на неврожайні пустарі. Тепер серни та лиси починают
витискати овець, а Шотляндців кидати ще в страшнійшу нужду\dots „Ліси для дичини“19), а народ(и) не
можут істнувати побіч себе. Або одні, або другі мусят уступити місця. Нехай тілько число і розмір
польовань в слідуючих 25 роках зможеся так, як в минувших, то певно не здиблете й одного Шотляндця в
єго ріднім краю. А се змаганє між шотляндськими маґнатами походит по части з моди, панської бути,
забагів на польованє і т. д., а по части вони займаются дичиною для зиску. Бо кусень гористого
простору, затичений для польованя, нераз далеко більше дає зиску, ніж колиб був толокою. Той, кому
забаглося польованя і шукає такого обшару, платит за него тілько, накілько статчит єго кішеня,
— (а се вже певно, то бідному чоловікови не до польованя!)\dots Відти то сплило на шотляндські гори
тілько недолі, кілько єї сплило на Англію ізза політики норманських
королів. Оленям віддано огромні простори до волі, а людей зігнано в тісні і чим раз тіснійші
закамарки\dots Одну вільність за другою видирано народови\dots І притиск той ще
день-денно змагаєся. Властивці з засади і загалом вимітают і виганяют народ, мов сіно косят, — мов
ті австральські та американські осадники відвічні ліси витинают, і тота операція поступає чим раз
далі, спокійно, с холодною розвагою та обрахунком20).

19) Шотляндські „ліси для дичини“ (deer forests) не мают і одного деревця. Се звичайні голі толоки,
с котрих вигнано вівці, а на їх місце нагнано оленів, — тай ось і преславетний „ліс для дичини“. Про
засів та плеканє лісів і казки нема!

20) Robert Somers: „Letters from the Highlands; or, the Famine of 1847. Lond. 1848“, стор. 12—28. Ті
листи надрукувала зразу ґазета Таймc. Англійські економісти, розумієся, зараз ростолкували, що
Шотляндці бідуют і мрут з голоду задля — перелюдненя. Сяк чи так, а їсти не було що. І в Німеччині
не чужа тота операція „Clearing of Estates“, — єї ту прізвано „Bauernlegen“ (обалюванє мужиків), і
вона особливо далася чути
