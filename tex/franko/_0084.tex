\parcont{}
\index{franko}{0084}
пануючих. Навіть террорістична управа єго не ткнула. Аж
зовсім уже недавно вичеркнуто єго с карної устави. Нічо
нема цікавійшого, як повід до того горожанського замаху
державного. Шапельє, справоздавець в тім ділі, каже: „Хоть
воно й пожадана річ, щоби плата робуча піднеслась вище,
ніж як тепер стоїт, — щоб той, хто її побирає, не стояв
в такій безвихідній залежности, в яку втискає го недостача
конечного прожитку і котра майже тілько важит, що залежність
невільника“, — то прецінь не сміют робітники
порозуміватися о своїх справах, не сміют спільно лучитися
і вменьшувати через те свою „безвихідну, майже невольничу
залежність“, — не сміют сего всего, бо через те „нарушили б
свободу своїх колишних майстрів, а теперішних
предприємців“ (свободу держаня робітника в неволи!), „і позаяк
стоваришенє робітників супротів самовладства давнійших
майстрів цехових, се — згадайте що? — се як раз приверненє
тих давних цехів, знесених французькою констітуцією“.
(Гл.~Buchez et Roux: „Histoire Parlamentaire“, т. X,
стор. 195)

\vspace{-\smallskipamount}
\subsection{Як повстали капіталістичні арендаторі?}
\vspace{-\smallskipamount}

Ми бачили, якими насильними способами витворений
зістав свобідний і голий пролєтаріят; яким кровавим примусом
перемінено його в наємних робітників; ми бачили всі
ті брудні підходи та державні укази, котрі поліційною силою
вбільшували визискуванє праці, прискорювали нагромадженє
капіталу. Питанє тепер, відки взялися перші капіталісти?
Бо вивласненє
% REMOVED \footnote*{В рукописі: вивласненье.}
мужицтво
могло витворити наразі тілько великих властивців ґрунтових.
Що до повставаня арендаторів, то се можем, так сказать, рукою
намацати, бо те повставанє відбувалося звільна, в протягу
многих століть. Самі кріпаки і побіч них дрібні, свобідні
властивці ґрунтові находилися в дуже різнородних обставинах
що до свого посіданя, і для того то й надана воля
застала їх в дуже різнородних економічних обставинах.
Першою формою арендатора в Англії є сам ще кріпосний
Bailiff. Єго становище подібне до того, яке в стародавнім
Римі займав villicus, тілько що обсяг єго діяльности тіснійший.
В другій половині \RNum{14} столітя на єго місце настає свобідний
арендатор, котрому дідич дає насінє, тягло і прилади
ґосподарські. Єго положінє мало чим різнится від
положіня мужика. Тілько він більше послугуєся наємною
працею. Швидко він стає Metair-ом, т. є. половинником. Він
дає одну часть робучого капіталу, а дідич другу. Всі плоди
ділятся між ними обома в певній, контрактом означеній
пропорції. Ся форма в Англії швидко щезає, уступаючи
місце властивим арендаторам, котрі помножуют свій власний
\parbreak{}
