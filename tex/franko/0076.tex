Рабунок дібр церковних, злодійське загарбуванє державних маєтків, крадіж громадських ґрунтів,
безправна

по 30-тилітній війні. Ще 1790 в одній части Саксонії вона викликала хлопські бунти. А найбільше
ширилося „обалюванє мужиків“ в східній Німеччині. В найбільшій части німецьких провінцій аж Фрідріх
II запевнив селянам право власности. По завойованю Шльонська він присилував дідичів до відбудованя
хат, стоділ і т. д., а також до заосмотреня мужицьких осад худобою та ґосподарськими знадобами.
Фрідріх II. потребував жовнірів для війська і оподаткованих людей для побільшеня державного
скарбу. Впрочім селянам і під Фрідріхом II. жилось далеко не гарно, як се мож побачити с письм єго
головного похвальця, Мірабо.

В цвітню 1872, 18 літ по виданю згаданої ту книжки Сомерса читав проф. Лєоні Лєві в „Society of
Arts“ відчит про переміну овечих пасовиск в ліси для дичини, де вказує дальший розвиток спустошеня в
шотлянських височинах. Між їншим каже він: „Обезлюдненє і переміна ґрунтів в голі толоки, се був
найвигіднійший для панів спосіб — получити доходи без видатків\dots Тепер же в височинах звичайно
пороблено с толок
„deer forest-и“. Дичина прогнала овець так, як недавно вівці прогнали були людей. Мож вандрувати від
дібр ґрафа Дельгаузі в Форфершайрі аж до Джона o' Ґротса, не виходячи зовсім з лісів. В многих іс
тих лісів замешкуют лиси, дикі коти, куни, тхорі, ласиці та альпейські заяці; від недавна
росплодилися там також крілики, вивірки та щурі. Огромні простори, котрі в шотлянській статистиці
значились „надзвичайно врожайні
і розляглі пастівники“, позбавлені тепер всякої управи і поправи і служат виключно для мисливської
забави кількох осіб, тай то лиш короткий час в році!“

Льондонський „Economist“ з 2. червця 1866 каже: „Одна шотлянська ґазета доносит послідного тижня між
їншими новинами ось що: Одна з найкращих овечих аренд в Созерлєндшайрі, за котру недавно, за упливом
біжучого арендового контракту, давано річної ренти 1200 фунтів штерлінґів, тепер зістає перемінена в
„deer forest!“. Феодальні інстінкти проявляются й тепер так само, як тоді, коли норманський
завойовник зруйнував 36 сіл, щоб закласти Ню-форс (Новий ліс)\dots Два мілійони екрів, самих
найурожайнійших в Шотляндії, опустошуются тепер до крихітки. Природна трава Ґлєн-Тільту належала до
найпоживнійших в ґрафстві Перз; теперішний дір-форст Бен-Альдер був найкращим пасовиском в цілім
лісистім Бедноч; одна часть теперішного Блік-Моунт-форста була найліпшим на всю Шотляндію пасовиском
для чорномордих овець. Про обсяг ґрунтів опустошеннх для стрілецької примхи мож виробити собі
яке-таке понятє, зваживши, що вони обіймают далеко більше простору, ніж ціле ґрафство Перз. Що через
те насильне опустошенє стратив край на жерелах продукції, мож оцінити с того, що форс Бен-Альдер міг
(в рукоп. „між“) прокормити 15000 овець і що він становит лиш \sfrac{1}{30} всіх „диких лісів“ шотлянських.
Весь той „дикий“ ґрунт зовсім не продуктівний\dots На одно б вийшло, як би був запався в фалі
Північного моря. Сильна рука праводавства повинна би прецінь зупинити розріст і творенє таких
самовільних
пустинь“.
