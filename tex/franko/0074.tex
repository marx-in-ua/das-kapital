лось було всего лиш 15000 люда, перемінити в толоку для овець. Від 1814 до 1820 сістематично
прогонювано та нищено тих 15000 мешканців, т. є. майже 3000 родин. Всі їх села поруйновано і
попалено, всі їх поля пороблено толоками. Англійських жовнірів викомендерувано там для еґзекуції, і
між ними а мешканцями прийшло до бійки. Одна
стара баба згоріла враз іс хатою, с котрої не хтіла вступитися. І таким способом присвоїла собі
вельможна герцоґиня 794000 екрів ґрунту, котрі споконвіку належали до
повіту. Вигнаним мешканцям визначила вона на морськім узберіжю около 6000 екрів, по 2 екри на
родину. Тих 6000 екрів лежали доси пусто і не давали властительці ніякого
доходу. Герцоґиня так далеко зайшла в своїй щедрости, що винаймила екр пересічно по 2 шілінґи 6
пенсів для тих самих селян, котрі много сот літ проливали кров свою за
вельможну герцоґську родину. Увесь зрабований ґрунт повіту поділила герцоґиня на 29 великих аренд
для випасаня овець; в кождій аренді осіла тілько одна родина, переважно англійські наємні
арендаторі. 1825 р. замісць 15000 Ґелів на їх ґрунтах жило вже 181000 овець. А родини, вивержені на
морський беріг, старалися жити риболовством. З них поробилися земноводяні, і вони жили, як каже
писатель, на половину в воді, а на половину на березі, тілько що ні ту ні там не могли найти
достаточного прожитку17).

Але небораки Ґелі мусіли ще раз відпокутувати свою романтичну наклінність для „великих мужів“, т. є.
для начальників повітових (Сlanchef). Запах риб, котрими прокормлювались земноводяні Ґелі, ударив
великим мужам в ніс. Вони завітрили тут щось зисковного і заарендували морське узберіжє великим
льондонським гендлярам риб. Ґелів другий раз вигнано на штири вітри 18).

Аж вкінци одну часть пасовиськ назад перемінено

17) Коли теперішна герцоґиня Созерлєнд витала в Льондоні з великою парадою міссіс Бічер Стоу,
авторку „Хати дядька Томи“, щоб виставити на показ свою прихильність для муринів-невольників в
американській републіці — чого вона і єї співарістократки певно не булиб зробили підчас домашної
війни американської, бо тоді кожде „шляхотне“ англійське серце було прихильне плянтаторам — в той
сам час описав
я в газеті „New-York-Tribune“ побут невольників созерлєндських. (Деякі місця тої статі навів Керей в
своїй „The Slave Trade. London 1853“.). Мою статю перепечатала одна шотляндська ґазета і викликала
дуже чемну перепалку між тою ґазетою а підхлібниками та похвальками герцоґів Созерлєндів.

18) Цікаву історію того рибного торгу найде читатель у д. Девіда Оркуарта в єго книжці: „Portofolio.
New Series“. Сеніор водній іс своїх посмертних статей називає „процедуру в Созерлєндшайрі“ одним з
найблагодатнійших очищень від віків.
