
\setcounter{chapter}{23}
\section[Початок і історичний розвиток
капіталістичної продукції в Англії]{Початок і історичний розвиток капіталістичної продукції в Англії\footnotemarkZ{}}
\markboth{Початок і історичний розвиток
капіталістичної продукції в Англії}{Фрагмент «Капіталу» у~перекладі Івана~Франка}

\footnotetextZ{
Вперше надруковано в журн. «Культура», 1926, № 4--9, с. 61--87.

\nopagebreak[4]
Подається за автографом перекладача: відділ рукописних фондів і текстології Інституту літератури ім. Т.~Г.~Шевченка НАН України. — Ф. 3. — Од. зб. 448. — 14 арк. Кінець автографа не зберігся. 

\nopagebreak[4]
Переклад зроблено з другого німецького видання: \textgerman{Das Kapital. Kritik der politischen Oekonomie. Von Karl Marx. Erster Band. Zweite verbesserte Aufgabe. Hamburg. Verlag von Otto Meissner, 1872.} Про це є згадка І. Франка на початку тексту перекладу «Гл[яди] К. Marx. Das Kapital, 2 вид. з р. 1872, стор. 742--794».}

\subsection{Первісне нагромадженє капіталу}

Ми бачили, що гроші стают капіталом тоді, коли служат
до купованя робучої сили. Ми бачили, що капітал
раз~у~раз намагає — творити надзвишку вартости, а надзвишка
вбільшує капітал. Між тим щоб капітал міг нагромаджуватись,
мусит уже вперед витворюватись надзвишка;
щоб могла витворюватись надзвишка, мусит істнувати капіталістична
продукція, а щоб тота істнувала, мусит уже
вперед більша маса капіталу бути нагромаджена в руках
поєдинчих богатирів. Здаєсь затим, що весь той процес
полягає на якімось „первіснім“ нагромадженю, котре мало
місце перед капіталістичною продукцією, котре, значит, не
було випливом капіталістичної продукції, а єї жерелом.

\index{franko}{0062}
Тото первісне нагромадженє капіталу („previous accumulation“,
як каже А.~Сміт) грає в суспільній економії
майже таку саму ролю, як „гріхопаденіє“ в теольоґії. Адам
зїв яблоко і через те стягнув гріх на рід людський. Початок
гріха обяснений казкою про давнину. Колись-колись
в давнину були з одного боку пильні вибранці, а з другого —
ліниві нероби. Через те сталося, що перші нагромадили
богацтво, а другі зійшли на таке, що остаточно не мали
вже що продавати крім себе самих. І від того гріхопаденія
почалася бідність великої маси, котра ще й доси, хоть і як
тяжко працює, не має що продавати крім себе самих, —
і богацтво деяких, що й доси змагаєся, хоть самі вони
давно перестали працювати\footnote{
Такі безглузді дитиньства плете ще д. Тйер (звісний французький
муж стану) дотепним колись французам с повагою великого мудрця —
для оборони святої власности. Ну і справді, — скоро діло йде о власність,
то святий обовязок кождого — міцно стояти на становищи букваря,
ще й других переконувати, що те становище для всякого „віка
і возраста“ єдино відповідне і належне.
}. В правдивій історії грали, як
звісно, завойованя, гнет, рабунки, вбійства, — одним словом,
усілякі насиля велику ролю. Але в сумирній політичній
економії з давен-давна — все іділлія. Право і „праця“, се
здавна були єдині способи до збогаченя, тілько, розумієся,
завсігди с тим застереженєм, що аж „сего року воно щось
не так“. Але на ділі способи первісного нагромадженя капіталу
були всякі, які хочете, — тілько не іділлічні.

Гроші і товар не є зразу капіталом, таксамо, як не
є ним зразу средства продукційні і знадоби до житя. Вони
мусят бути перемінені в капітал. Але та переміна може настати
тілько серед певних обставин, котрі зводятся ось на
що: двоякі дуже відмінні посідачі товарів мусят стати супротів
себе і зіткнутися с собою, — з одного боку властивці
грошей, средств продукційних і знадіб до житя, котрим
о то йде, щоб свою суму вартостей побільшити купівлею
чужої робучої сили; а з другого боку вільні робітники, продавці
власної робучої сили і, значит, продавці \so{праці}.
Вільні вони мусят бути в двоякім значіню, т. є. щоб ані
самі вони беспосередно не були средствами продукційними,
як невольники, кріпаки і т. д., ані шоб вони самі не посідали
средств продукційних, як ґазди-селяне, дрібні властивці
ґрунтові і т. д. Такий розділ товарив між дві крайности
— се основні вимінки для капіталістичної продукції.
Без відділеня робітників від власности не може настати
капіталістична продукція. Але скоро вона раз настала, то
не тілько підтримує те відділенє, але й сама доводит до
него раз~у~раз на~ново і раз~у~раз на більший розмір. Коли
затим спитаємо: де є жерело капіталістичного ладу? то
\parbreak{}
