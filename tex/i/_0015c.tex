\index{i}{0015}  %% посилання на сторінку оригінального видання
\subsection{Відносна форма вартости}

\subsubsection{Зміст відносної форми вартости}

Щоб знайти, яким чином простий вираз вартости одного товару
міститься у вартостевому відношенні двох товарів, мусимо
розглянути це відношення передусім цілком незалежно від його
кількісного боку. Здебільша роблять якраз навпаки й бачать у
вартостевому відношенні лише ту пропорцію, в якій певні кількості
двох сортів товару рівні одна одній. При цьому недобачають,
що величини різних речей стають кількісно порівнянними
лише після того, як їх вже зведено до тієї самої одиниці. Тільки
як вирази тієї самої одиниці вони є однойменні, а через те й спільномірні
величини.\footnote{
Ті нечисленні економісти, що, як, приміром, S. Bailey, працювали
коло аналізи форми вартости, не могли дійти якогобудь результату,
поперше, тому, що вони сплутували форму вартости з вартістю, а, подруге,
тому, що вони, під грубим впливом практичного буржуа, бралися насамперед
виключно до кількісної визначености мінового відношення. «Панування
кількости\dots{} конституює вартість». («The command of quantity\dots{}
constitutes value»). (\emph{S. Bailey}: «Money and its Vicissitudes», London
1837, p. 11).
}

Чи 20 метрів полотна = 1 сурдутові, чи 20, чи $x$ сурдутам,
інакше кажучи, чи дана кількість полотна варта багато чи небагато
сурдутів — всяка така пропорція завжди включає й те, що
полотно й сурдути як величини вартости є вирази тієї самої одиниці,
речі тієї самої природи. Полотно = сурдутові — це основа
рівнання.

Але ці два якісно урівняні один з одним товари відіграють
неоднакову ролю. Тільки вартість полотна тут виражається. Але
як? Через його відношення до сурдута як до його «еквіваленту»,
тобто до чогось, «на що можна обміняти полотно». В цьому відношенні
сурдут фігурує як форма існування вартости, як предмет
вартости (Wertding), бо тільки як такий він є те саме, що й полотно.
З другого боку, власне вартостеве існування полотна
тільки тут виявляється, або набуває самостійного виразу, бо
тільки як вартість полотно можна відносити до сурдута як до
чогось рівновартого, або вимінного на нього. Так, масляна кислота
є тіло, відмінне від пропілово-муравельного етеру. Однак, обоє
вони складаються з тих самих хемічних субстанцій — вуглеця
(С), водня (Н) і кисня (О), і до того ж в однаковому відсотковому
складі, а саме С\textsubscript{4}Н\textsubscript{8}O\textsubscript{2}.
Коли хто порівняв би пропілово-муравельний
етер до масляної кислоти, то в цьому відношенні пропілово-муравельний
етер, поперше, фігурував би лише як форма
існування С\textsubscript{4}Н\textsubscript{8}O\textsubscript{2},
а, подруге, тим було б сказано, що й масляна
кислота теж складається з С\textsubscript{4}Н\textsubscript{8}O\textsubscript{2}.
Отже, порівнянням пропілового
етеру до масляної кислоти було б виражено їхню хемічну
субстанцію на відміну від їхньої тілесної форми.

Коли ми кажемо: як вартості, товари є лише згустки людської
праці, — то наша аналіза зводить їх на абстрактну вартість, але
\parbreak{}  %% абзац продовжується на наступній сторінці
