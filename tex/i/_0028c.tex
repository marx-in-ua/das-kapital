\parcont{}  %% абзац починається на попередній сторінці 
\index{i}{0028}  %% посилання на сторінку оригінального видання 
конкретний корисний рід прані, що міститься в кожному осібному
товаровому еквіваленті, є лише осібна форма виявлення людської
праці, отже, форма, що не вичерпує всіх форм людської праці.
Правда, ця остання має свою повну або вичерпну форму виявлення
в сукупності тих осібних форм виявлення. Але в такому разі вона
не має жодної однорідної форми виявлення.

Однак розгорнута відносна форма вартости складається лише
з суми простих відносних виразів вартости або з рівнань першої
форми, як от:\footnote{
метрів полотна = 10 фунтам чаю й т. ін.

Але кожне з цих рівнань містить у собі й зворотне тотожне
рівнання:
} метрів полотна =\footnote{
сурдут = 20 метрам полотна
10 фунтів чаю = 20 метрам полотна й т. ін.

Дійсно, коли хтось обмінює своє полотно на багато інших товарів
і таким чином виражає вартість полотна в ряді інших товарів,
то й усі інші посідачі товарів неминуче мусять обмінювати
свої товари на полотно й таким чином виражати вартість своїх
різних товарів у тому самому третьому товарі, в полотні. — Отже,
якщо ми обернемо ряд: 20 метрів полотна = 1 сурдутові, або =
10 фунтам чаю, або = і т. ін., тобто, якщо виразимо те зворотне
відношення, яке по суті вже міститься в цьому ряді, то матимемо:

С. загальна форма вартости
} сурдутові\footnote{
квартер пшениці =
} сурдут =

10 фунтів чаю =

40 фунтів кави =

2    унції золота =\footnote{
. Змінений характер форми вартости

Тепер товари виражають свої вартості, по-перше, просто, бо
вони їх виражають в якомусь одним-одному товарі, а по-друге,
однорідно, бо вони їх виражають в тому самому товарі. їхня форма
вартости є проста й спільна їм всім, тим то й загальна.

Форми перша, А, і друга, В, досягали лише того, що виражали
вартість якогось товару як щось відмінне від його власної
споживної вартости або від його товарового тіла.

Перша форма, А, подавала такі рівнання вартости, як ось: 1 сурдут
= 20 метрам полотна, 10 фунтів чаю = 1/2 тонни заліза й т. ін.
Вартість сурдута виражається як щось рівне полотну, вартість
}/2 тонни заліза =

х товару А =

і т. под. товарів =

20 метрам полотна
\parbreak{}  %% абзац продовжується на наступній сторінці
