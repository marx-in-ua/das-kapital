\parcont{}  %% абзац починається на попередній сторінці
\index{i}{0647}  %% посилання на сторінку оригінального видання
ринку, створеного великими відкриттями кінця XV століття.
Але середньовіччя залишило по собі дві різні форми капіталу,
що визрівають серед якнайрізніших суспільно-економічних
формацій і що їх перед епохою капіталістичної продукції вважається
за капітал quand même\footnote*{
— не зважаючи ні на що. \emph{Ред.}
} — лихварський капітал і купецький
капітал. «За теперішніх часів усе суспільне багатство
попадає спочатку до рук капіталіста\dots{} він платить ренту земельному
власникові, заробітну плату робітникові, податки й десятину
збирачеві й затримує для себе самого велику, в дійсності
найбільшу частину річного продукту праці, частину, що день-у-день
зростає. Капіталіста можна тепер розглядати як первісного
власника цілого суспільного багатства, хоч жодний закон
не передав йому права на цю власність\dots{} Цю зміну щодо власности
зумовлено тим, що брали проценти на капітал\dots{} і не менш
дивно, що законодавці цілої Европи хотіли перешкодити цьому
законами проти лихварства\dots{} влада капіталіста над цілим багатством
країни — це цілковита революція в праві власности;
алеж яким законом, або якою низкою законів зумовлено цю
революцію?».\footnote{
«The Natural and Artificial Rights of Property Contrasted», London
1832, p. 98, 99. Автор цієї анонімної праці — Т. Годжскін.
} Авторові слід було б пригадати, що революцій
не роблять за допомогою законів.

Перетворенню грошового капіталу, утвореного лихварством
і торговлею, на промисловий капітал заважав февдальний режим
на селі, цеховий режим у місті.\footnote{
Ще 1794 р. дрібні виробники сукна Лідсу послали до парляменту
депутацію з проханням видати закона, що заборонив би купцям ставати
фабрикантами. (Dr. Aikn: «Description of the Country from thirty to forty
miles round Manchester», London 1795).
} Ці обмеження впали з розпуском
февдальних дружин, з експропріяцією й частинним зігнанням
сільської людности. Нова мануфактура постала в морських
експортових гаванях або в таких пунктах усередині
країни, які були поза контролем старих міст і їхнього цехового
ладу. Звідси, в Англії, люта боротьба corporate towns\footnote*{
— міст з цеховим корпоративним ладом. \emph{Ред.}
} проти
цих нових розсадників промисловости.

Відкриття копалень золота й срібла в Америці, винищення,
поневолення й поховання живцем тубільної людности в копальнях,
розпочате завоювання й розграбовування Східньої Індії,
перетворення Африки на загороду для комерційного полювання
на чорношкірих — така була світова зоря капіталістичної ери
продукції. Оці ідилічні процеси є головні моменти первісної
акумуляції капіталу. За ними слідом іде торговельна війна европейських
націй, арена її — ціла земна куля. Ця війна починається
відпадом Нідерляндії від Еспанії, набирає велетенських розмірів
в англійській антиякобінській війні й ще тепер триває у війнах
проти Китаю за опій і т. ін.
