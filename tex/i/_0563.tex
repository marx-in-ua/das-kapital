
\index{i}{0563}  %% посилання на сторінку оригінального видання
Перед цим іще одно слово про офіціяльний павперизм, тобто
про ту частину робітничої кляси, яка втратила умову свого існування,
продаж робочої сили, і животіє з громадської милостині.
(Офіціяльна статистика налічувала в Англії\footnote{
До Англії завжди залічують Велз, до Великобрітанії — Англію,
Велз і Шотляндію, до Об’єднаного королівства — ці три країни й Ірляндію.
} 1855~\abbr{р.} \num{851.369} павперів,
1856~\abbr{р.} — \num{877.767}, 1865~\abbr{р.} — \num{971.433}. У наслідок бавовняного
голоду число їх зросло 1863 і 1864~\abbr{рр.} до \num{1.079.382}
й \num{1.014.978}. Криза 1866~\abbr{р.}, що найтяжче вразила Лондон, створила
в цьому центрі світового ринку, більшому числом жителів
за королівство Шотляндію, приріст павперів на 19,5\% для
1866~\abbr{р.} порівняно з 1865~\abbr{р.} і на 24,4\% порівняно з 1864~\abbr{р.} і ще
більший приріст для перших місяців 1867~\abbr{р.} порівняно з
1866~\abbr{р.} Аналізуючи статистику павперизму, треба звернути увагу
на два пункти. З одного боку, рух зменшення і збільшення маси
павперів відбиває в собі періодичні зміни промислового циклу.
З другого боку, офіціяльна статистика стає чимраз більш фалшивим
показником дійсних розмірів павперизму в міру того,
як з акумуляцією капіталу розвивається клясова боротьба, а
тому й почуття самоповаги в робітників. Наприклад, варварство
в поводженні з павперами, про що останніми двома роками так
голосно кричала англійська преса («Times», «Pall Mall Gazette»
і~\abbr{т. д.}) — явище старе. Енґельс констатує в 1844~\abbr{р.} цілком
ті самі страхіття й цілком те саме минуще лицемірне обурення
«сенсаційної літератури». Але страшне збільшення числа випадків
голодної смерти («death by starvation») в Лондоні протягом
останніх десятьох років безумовно свідчить про щораз більше зростання
огиди робітників до рабства робітних домів\footnote{
Своєрідне освітлення проґресові, який стався від часів А.~Сміта,
дає та обставина, що для нього слово workhouse, робітний дім, деколи ще
рівнозначне слову manufactory, мануфактура. Наприклад, у нього на
початку розділу про поділ праці: «Тих робітників, що працюють у різних
галузях праці, часто можна скупчувати в тому самому робітному домі»
(«Those employed in every different branch of the work can often be collected
into the same workhouse»);
}, цих карних
закладів для бідности.

\subsubsection{Погано оплачувані верстви брітанської промислової робітничої
кляси}

Звернімося тепер до погано оплачуваних верств промислової
робітничої кляси. Підчас бавовняного голоду 1862~\abbr{р.} Privy
Council\footnote*{
Таємна державна рада. \emph{Ред.}
} доручив докторові Смітові розслідити стан харчування
збіднілих бавовняних робітників Ланкашіру й Чешіру. Довголітнє
попереднє спостереження привело його до результату, що
«для того, щоб запобігти недугам від голоду» (starvation diseases),
щоденні харчі жінки мусять мати в собі пересічно щонайменш
\num{3.900} ґранів вуглецю і 180 ґранів азоту, щоденні харчі
чоловіка — пересічно щонайменше \num{4.300} ґранів вуглецю і 200 ґранів
\parbreak{}  %% абзац продовжується на наступній сторінці
