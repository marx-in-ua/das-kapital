\parcont{}  %% абзац починається на попередній сторінці
\index{i}{0023}  %% посилання на сторінку оригінального видання
виразну форму, що в ній здійснюється абстрактна людська праця.

Отже, це є друга особливість еквівалентної форми, а саме, що конкретна праця стає тут формою
виявлення своєї протилежности,
абстрактної людської праці.

Але що ця конкретна праця, кравецтво, виступає тут лише як вираз безріжницевої (unterschiedsloser)
людської праці, то вона має форму рівности іншій праці, праці, що міститься в полотні, а тому, хоч і
є вона приватна праця, як і всяка інша праця, що продукує товари, вона все ж становить працю в
безпосередньо суспільній формі. Саме через це вона йвиражається в продукті, безпосередньо вимінному
на інший товар. Отже, це є третя особливість еквівалентної форми, а саме, що приватна праця стає
формою своєї протилежности, працею в безпосередньо суспільній формі.

Обидві останні, щойно розглянуті особливості еквівалентної форми стануть ще зрозуміліші, коли ми
звернемось до великого дослідника, який перший проаналізував форму вартости, так само як і багато
інших форм мислення, суспільних форм і форм природних. Це є Арістотель.

Насамперед Арістотель ясно каже, що грошова форма товару є тільки далі розвинений вигляд простої
форми вартости, тобто виразу вартости одного товару в будь-якому іншому товарі, бо він каже:

\noindent{}\begin{minipage}{\textwidth}
\begin{center}
«5 ліжок \deq{} 1 будинкові»
(«\textgreek{Κλίναι πέντε άνι\dots{} οσου αί πέντε χλίναι}»)

«не відрізняється» від:

«5 ліжок \deq{} такій і такій кількості грошей»

(«\textgreek{Κλίναι πέντε άντί\dots{} όσον αί πέντε χλίναι}»).
\end{center}
\medskip
\end{minipage}

\noindent{}Далі він розуміє, що вартостеве відношення, в якому міститься цей вираз вартости, із свого боку має
за умову, що будинок якісно прирівнюється до ліжка та що ці почуттєво різні речі без такої рівности
їхньої суті не могли б відноситися одна до однієї як спільномірні величини. «Обмін, — каже він, — не
може бути без рівности, а рівність — без спільномірности» («\textgreek{οΰτ’ ίσότης μη σΰσηςσυμμετρίας}»). Але
тут він збивається й відмовляється від дальшої аналізи форми вартости. «Однак справді неможливо
(«\textgreek{τη μεν όυν αληυεία άδύνατον}»), щоб такі різні речі були спільномірні», тобто якісно однакові. Таке
порівняння може бути лише чимось чужим для справжньої природи речей, отже, лише «чимось таким, до
чого необхідно вдаватися для практичних потреб».

Отже, Арістотель сам каже нам, чому крахує його дальша аналіза, а саме в наслідок відсутности
поняття вартости. Чим є те рівне, тобто та спільна субстанція, яку репрезентує будинок супроти ліжка
у виразі вартости ліжка? Щось таке «не може справді існувати», каже Арістотель. Чому? Будинок
репрезентує супроти ліжка щось рівне, оскільки він репрезентує те, що є дійсно рівне в одному й
другому, в ліжку й будинку. А~це є — людська праця.
