Хоч машина неминуче й витискує робітників по тих галузях
промисловосте, де її заводять, все ж вона може викликати збільшення
праці по інших галузях праці. Але це діяння не має нічого
спільного з так званою теорією компенсації. А що кожний машиновий
продукт, наприклад, один метр машинової тканини, дешевший,
ніж витиснутий ним однорідний ручний продукт, то звідси випливає
такий абсолютний закон: якщо загальна кількість товарів,
випродукованих машиновим способом, лишається рівна загальній
кількості заміщуваних нею товарів, випродукованих ремісничим
або мануфактурним способом, то загальна сума вжитої праці
зменшується. Те збільшення праці, яке потрібне для продукції
самих засобів праці, машин, вугілля й т. д., мусить бути менше
від праці, заощадженої через уживання машин. У противному
разі машиновий продукт був би так само дорогий або й дорожчий,
ніж ручний продукт. Алеж фактично замість лишатися однаковою
вся маса машинового продукту, що його продукує зменшене
число робітників, зростає далеко понад загальну масу витиснутого
ремісничого продукту. Припустімо, що 400.000 метрів машинової
тканини продукується меншим числом робітників, ніж 100.000
метрів ручної тканини. У збільшеному вчетверо продукті міститься
вчетверо більше сировинного матеріялу. Отже, продукцію
сировинного матеріялу треба збільшити в чотири рази. Щождо
спожитих засобів праці, як от будівлі, вугілля, машини й т. д.,
то межі, в яких може зростати додаткова праця, потрібна на їхню
продукцію, змінюються відповідно до ріжниці поміж масою машинового
продукту й масою ручного продукту, яку може виготовити
те саме число робітників.

Отже, з поширенням машинового виробництва в якійсь одній
галузі промисловости більшає насамперед продукція в тих інших
галузях, які постачають їй її засоби продукції. Якою мірою
через те зростає маса занятих робітників, це залежить, за
даної довжини робочого дня й інтенсивности праці, від складу
вжитих капіталів, тобто від відношення між їхніми сталими та
змінними складовими частинами. Це відношення, з свого боку,
дуже варіює залежно від того обсягу, в якому машини вже захопили
або захоплюють ті галузі промисловости. Число робітників,
засуджених на працю по копальнях вугілля й металю, страшенно
зросло з розвитком англійської машинової системи, хоч це зростання
останніми десятиліттями уповільнюється в наслідок заведення
нових машин у гірництві. 217 Разом з машиною увіходить
у життя новий рід робітника — її продуцент. Ми вже знаємо, що
машинове виробництво завойовує в чимраз більшому розмірі й

Тьєр переплутав тут прядільну машину з механічним ткацьким варстатом,
та «нас це, однак, мало обходить».

217    За переписом 1861 р. (Vol. II. London 1863) число робітників,
що працювали по копальнях вугілля Англії та Велзу, становило 246.613,
з них 73.545 молодші й 173.067 понад 20 років віку. До першої рубрики
належать 835 від п’яти до десяти років, 30.701 від десяти до п’ятнадцяти,
42.010 від п’ятнадцяти до дев’ятнадцяти років. Число заня-
