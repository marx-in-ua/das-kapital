ліст, чи, як це маємо в акційних товариствах, як комбінований
капіталіст.

Кооперація в процесі праці в тій формі, в якій ми находимо
її переважно на початках людської культури, у мисливських народів\footnoteA{
Linguet у своїй «Théorie des Lois civiles», мабуть, має рацію, коли
він називає полювання першою формою кооперації, а полювання на людей
(війну) першою формою полювання.
} або хоч би в рільничих індійських громадах, базується
з одного боку, на спільному володінні умовами продукції, з другого
боку — на тому, що поодинокий індивід не відірвався ще від
пуповиння племени або громади й прив’язаний до них так само
міцно, як окрема бджола до бджоляного вулика. І те і друге відрізняє
цю кооперацію від капіталістичної кооперації. Спорадичне
вживання кооперації у великому маштабі в античному світі, в
середньовіччі та в сучасних колоніях базується на безпосередніх
відносинах панування та підлеглости, здебільша на рабстві.
Навпаки, капіталістична форма кооперації з самого початку має
за свою передумову вільного найманого робітника, що продає
свою робочу силу капіталістові. Однак історично вона розвивається
протилежно до селянського господарства та незалежного
ремества, однаково, чи це останнє має цехову форму, чи ні.\footnote{
Дрібне селянське господарство і незалежне ремество, що обидва
почасти становлять базу февдального способу продукції, а почасти після
його розкладу існують поруч капіталістичної продукції, разом з тим становлять
економічну основу клясичної громади за її найліпших часів, за
тих часів, коли первісна східня громадська власність уже розпалася, а
рабство не встигло ще серйозно опанувати продукцію.
}
Супроти них капіталістична кооперація виступає не як осібна
історична форма кооперації; навпаки, сама кооперація виступає
супроти них як певна історична форма, властива капіталістичному
процесові продукції, як специфічна форма, що відрізняє
його від інших способів продукції.

Подібно до того, як суспільна продуктивна сила праці, що
розвивається в наслідок кооперації, з’являється як продуктивна
сила капіталу, так само й сама кооперація з’являється як специфічна
форма капіталістичного процесу продукції, протилежно до
процесу продукції поодиноких незалежних робітників або й дрібних
майстрів. Це — перша зміна, якої зазнає дійсний процес
праці через свою підлеглість капіталові. Ця зміна відбувається
спонтанно. Її передумова, одночасна експлуатація значного
числа найманих робітників у тому самому процесі праці, становить
вихідний пункт капіталістичної продукції, який збігається
з існуванням самого капіталу. Тим то, якщо капіталістичний
спосіб продукції, з одного боку, являє собою історичну доконечність
для перетворення процесу праці на суспільний процес, то,
з другого боку, ця суспільна форма процесу праці являє собою
методу, що її застосовує капітал на те, щоб із більшим зиском
експлуатувати процес праці, збільшуючи його продуктивну силу.

У своїй розглянутій досі простій формі кооперація збігається
з продукцією у великому маштабі, але не становить тривалої