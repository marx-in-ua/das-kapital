\parcont{}  %% абзац починається на попередній сторінці
\index{i}{0034}  %% посилання на сторінку оригінального видання
хоч які будуть її зміст і форма, в суті своїй є витрата людського
мозку, нервів, м’яснів, чуттєвих органів і~\abbr{т. ін.} Подруге, що торкається
того, що лежить в основі визначення величини вартости,
часу тривання цієї витрати, або кількости праці, то кількість
праці навіть наочно відрізняється від якости праці. За всіх суспільних
формацій (Zustände, états sociaux) робочий час, що його
коштувала продукція засобів існування, мусив цікавити людей,
хоч і неоднаково на різних ступенях розвитку\footnote{
Примітка до другого видання. У стародавніх германців величину
одного морґа землі обчислювалось працею одного дня; звідси назва морга
Tagwerk (праця одного дня), (також Tagwanne) (jurnale або jurnalis,
terra jurnalis, jurnalis або diornalis), Mannwerk, Mannskraft, Mannsmaad,
Mannshauet і~\abbr{т. ін.} Див. \emph{Georg Ludwig von Maurer}: «Einleitung zur Geschichte
der Mark — Hof usw. Verfassung», München 1859, p. 129 і далі.
}. Нарешті, скоро
люди в будь-який спосіб працюють один на одного, то і праця
їхня набуває суспільної форми.

Звідки ж випливає загадковий характер продукту праці,
скоро тільки він набирає форми товару? Ясна річ, із самої цієї
форми. Рівність людських праць набирає речової форми однакової
вартостевої предметности (\textgerman{Wertgegenständlichkeit}) продуктів
праці;\footnote*{
У французькому виданні це речення зредаґовано так: «Характер
рівности людських праць набирає форми вартостевої властивости продуктів
праці» («Le caractère d’égalité des travaux acquiert la forme de
a qualité de valeur des produits du travail»).
} виміряння витрати людської робочої сили часом її тривання
набирає форми величини вартости продуктів праці, нарешті,
відносини між продуцентами, що в них виявляються ті
суспільні визначення їхніх праць, набирають форми суспільного
відношення продуктів праці.

Отже, таємність товарової форми є попросту в тому, що форма
ця відбиває людям суспільний характер їхньої власної праці як
предметний характер самих продуктів праці, як суспільні природні
властивості цих речей; тим то й суспільне відношення продуцентів
до всієї сукупної праці вона відбиває як суспільне відношення
предметів, яке існує поза ними. В наслідок такого quid
pro quo продукти праці стають товарами, почуттєво-надпочуттєвими,
або суспільними речами. Так, світловий вплив якоїсь речі
на зоровий нерв виявляється не як суб’єктивне подразнення самого
зорового нерву, а як предметна форма речі, що є поза оком.
Але при баченні світло дійсно кидається однією річчю, зовнішнім
предметом, на другу річ, на око. Це є фізичне відношення між
фізичними речами. Навпаки, товарова форма й вартостеве відношення
продуктів праці, що в ньому вона виявляється, не мають
абсолютно нічого спільного з їхньою фізичною природою і речовими
відношеннями, що з неї випливають. Це є лише певне суспільне
відношення самих людей, яке для них набирає тут фантасмагоричної
форми відношення речей. Тим то, щоб знайти для
цього аналогію, ми мусимо кинутися в туманну царину релігійного
світу. Тут продукти людської голови видаються обдарованими
власним життям, самостійними постатями, які стоять у певних
\index{i}{0035}  %% посилання на сторінку оригінального видання
відносинах одні до одних і до людей. Так само стоїть справа
в товаровому світі з продуктами людських рук. Це я зву фетишизмом,
що пристає до продуктів праці, скоро їх продукується
як товари, і що, отже, є невіддільний від товарової продукції.

Цей фетишистичний характер товарового світу випливає, як
показала вже попередня аналіза, із своєрідного суспільного
характеру праці, що продукує товари.

\looseness=-1
Предмети споживання стають взагалі товарами лише тому, що
вони є продукти приватних праць, виконуваних незалежно одна
від однієї. Комплекс цих приватних праць становить сукупну
працю суспільства. А що продуценти ввіходять у суспільний контакт
лише через обмін продуктів своєї праці, то й специфічні суспільні
характери їхніх приватних праць виявляються лише в
межах цього обміну. Інакше кажучи, приватні праці фактично
виявляються як члени сукупної праці суспільства лише через ті
відносини, в які обмін ставить продукти праці, а за допомогою цих
продуктів і продуцентів. Тим то для продуцентів суспільні відносини
їхніх приватних праць з’являються тим, чим саме вони
й є, тобто не безпосередньо суспільними відносинами осіб у самій
їхній праці, а, навпаки, речовими відносинами осіб і суспільними
відносинами речей.

Лише в межах обміну їх продукти праці набирають суспільно
однакової вартостевої предметности (Wertgegenständlichkeit), відокремленої
від їхніх почуттєво різних споживних предметностей
(Gebrauchsgegenstandlichkeit)\footnote*{
У французькому виданні це місце зредаґовано так: «Лише в процесі
обміну їх продукти праці набирають як вартості тотожного й одноманітного
суспільного існування, що є відмінне від їхнього матеріяльного
й різноманітного існування як предметів споживання. («Le Capital
etc.», v. I., ch. I., p. 29). \emph{Ред.}
}. Це розщеплення продукту праці
на корисний предмет і предмет вартости реалізується на практиці
лише тоді, коли обмін набирає вже достатнього поширення й значення
для того, щоб корисні речі продукувалося для обміну, коли,
отже, вартостевий характер речей береться на увагу вже за самої
продукції їх. Від цього моменту приватні праці продуцентів
дійсно набувають двоїстого суспільного характеру. З одного боку,
вони мусять як певні корисні праці задовольняти певні суспільні
потреби й таким чином довести, що вони є члени сукупної праці,
члени природно вирослої системи суспільного поділу праці. З другого
боку, вони задовольняють лише різноманітні потреби своїх
власних продуцентів, оскільки кожна окрема корисна приватна
праця є вимінна на кожний інший рід корисної приватної праці,
тобто є йому рівнозначна. Рівність toto coelo\footnote*{
всіма сторонами. \emph{Ред.}
} різних праць може
бути лише в абстрагуванні від їхньої справжньої нерівности, у
зведенні до спільного характеру, що його вони мають як витрати
людської робочої сили, абстрактної людської праці. Мозок приватних
продуцентів відбиває двоїстий суспільний характер їхніх
приватних праць тільки в тих формах, які з’являються у практичних
\index{i}{0036}  %% посилання на сторінку оригінального видання
зносинах, у обміні продуктів; отже, суспільно-корисний
характер їхніх приватних праць мозок їхній відбиває в тій формі,
що продукт праці мусить бути корисний, і то саме для інших
людей, а суспільний характер рівности різнорідних праць він
відбиває у формі спільного характеру вартости цих матеріяльно
відмінних речей, продуктів праці.
\enablefootnotebreak{}

Отже, люди ставлять у взаємні відношення продукти своєї
праці як вартості не тому, що ці речі мають для них значення лише
речових оболонок однорідної людської праці. Навпаки. Прирівнюючи
в процесі обміну один до одного свої різнорідні продукти
як вартості, вони тим самим прирівнюють одну до однієї свої
різні праці як людську працю. Вони несвідомі цього, але вони
це роблять\footnote{
Примітка до другого видання. Тим то, коли Galiani каже: Вартість
є відношення між двома персонами — «La Richezza è una ragione
tra due persone», — то він мусив би був додати: відношення, заховане під
речовою оболонкою. (\emph{Galiani:} «Della Moneta», р. 220, vol. III збірника
Custodi: «Scrittori Classici Italiani di Economia Politica». Parte
Moderna. Milano 1801).
}. Отже, у вартости на чолі не написано, щó вона є.
Скорше вартість перетворює кожний продукт праці на суспільний
гієрогліф. Пізніше люди силкуються розшифрувати значення того
гієрогліфу, збагнути таємницю свого власного суспільного продукту,
бож визначення предметів споживання як вартостей є
так само їхній суспільний продукт, як і мова. Пізніше наукове
відкриття, що продукти праці, оскільки вони є вартості, є лише
речові вирази людської праці, витраченої на їхню продукцію,
становить епоху в історії розвитку людства, але ні в якому разі
не усуває предметної зовнішности суспільного характеру праці.
Те, що має силу лише для цієї осібної форми продукції, для товарової
продукції, а саме, що специфічний суспільний характер
незалежних одна від однієї праць полягає в їхній рівності як
людської праці взагалі і набирає форми характеру вартости продуктів
праці, — це для людей, захоплених відношенням товарової
продукції, видається, як перед цим відкриттям, так і після нього,
так само незмінним і так само природним, як і те, що форма повітря
як фізична тілесна форма й далі існує, хоч наукова аналіза
й розклала повітря на його складові елементи.

Осіб, що обмінюють продукти, передусім практично цікавить
питання, скільки чужих продуктів вони дістануть за свій власний
продукт, тобто в яких пропорціях обмінюються продукти. Скоро
ці пропорції набирають певної звичної сталости, тоді здається,
нібито вони випливають із самої природи продуктів праці; приміром,
здається, що одна тонна заліза й дві унції золота мають однакову
вартість, подібно до того, як 1 фунт золота і 1 фунт заліза, не
зважаючи на те, що їхні фізичні та хемічні властивості є неоднакові,
мають однакову вагу. В дійсності характер вартости продуктів
праці закріпляється лише в наслідок функціонування їх як
вартостей певної величини. Величини вартостей змінюються постійно
незалежно від волі, передбачення й діяльности осіб, що
\parbreak{}  %% абзац продовжується на наступній сторінці
