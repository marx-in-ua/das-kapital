Коли робітники взагалі не можуть безпосередньо співробітничати,
не бувши згуртовані, отже, коли згуртовання їх у певному
місці є умова їхньої кооперації, то наймані робітники не можуть
кооперувати без того, щоб той самий капітал, той самий капіталіст
не вживав їх одночасно, отже, і не купував одночасно їхні робочі
сили. Тому сукупна вартість цих робочих сил, або сума заробітної
плати робітників за день, тиждень і т. д., мусить вже бути нагромаджена
в кишені капіталіста раніш, ніж сами робочі сили будуть
сполучені у продукційному процесі. Заплатити 300 робітникам
відразу навіть хоч би й лише за один день — це вимагає
більшого авансування капіталу, аніж платити кільком робітникам
тиждень за тижнем протягом цілого року. Отже, число
кооперованих робітників, або маштаб кооперації, залежить насамперед
від величини того капіталу, що його поодинокий капіталіст
може авансувати на купівлю робочої сили, тобто від того обсягу,
в якому кожен поодинокий капіталіст порядкує засобами
існування багатьох робітників.

І щодо сталого капіталу справа стоїть так само, як і щодо
змінного. Приміром, видатки на сировинний матеріял для одного
капіталіста, що вживає 300 робітників, у тридцять разів більші,
ніж для кожного з тих 30 капіталістів, що кожний з них вживає
10 робітників. Правда, розмір вартости й маса матеріялу спільно
вживаних засобів праці зростають не в такій пропорції, як число
вживаних робітників, але все ж вони зростають дуже значно.
Отже, концентрація більших мас засобів продукції в руках поодиноких
капіталістів є матеріяльна умова кооперації найманих
робітників, а розмір кооперації, або маштаб продукції залежить
від розміру цієї концентрації.

Первісно певна мінімальна величина індивідуального капіталу
виступала як доконечна для того, щоб кількости одночасно
визискуваних робітників, а тому й маси продукованої додаткової
вартости вистачило для звільнення самого визискувача від ручної
праці, для перетворення дрібного майстра на капіталіста, отже,
і для того, щоб формально створити капіталістичне відношення.
Тепер вона виступає як матеріяльна умова для перетворення
багатьох розпорошених і один від одного незалежних індивідуальних
процесів праці на один комбінований суспільний процес
праці.

Так само командування капіталу над працею первісно виступало
лише як формальний наслідок того, що робітник, замість
працювати на себе, працює на капіталіста, а тому й під доглядом
капіталіста. З розвитком кооперації багатьох найманих робітників
командування капіталу розвивається на доконечність для
виконання самого процесу праці, на дійсну умову продукції.
Наказ капіталіста на полі продукції стає тепер так само доконечний,
як наказ генерала на полі бою.

Всяка безпосередньо суспільна або спільна праця у великому
маштабі потребує в більшій або меншій мірі керування, яке упосереднює
гармонію між індивідуальними діями та виконує за-
