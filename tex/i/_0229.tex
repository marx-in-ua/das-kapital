\parcont{}  %% абзац починається на попередній сторінці
\index{i}{0229}  %% посилання на сторінку оригінального видання
сцени протягом усього часу тривання драми, так і робітники належали
тепер до фабрики протягом 15 годин, не рахуючи часу на
дорогу до фабрики й назад. Таким чином години відпочинку перетворювалися
на години примусового безділля, що гнали молодого
робітника до шинку, а молоду робітницю в дім розпусти. За
кожної нової витівки, що її день-у-день вигадував капіталіст,
щоб тримати свої машини в русі 12 або 15 годин, не збільшуючи
робочого персоналу, робітник мусів проковтнути свою їжу то в
той, то в інший шматок часу. Під час агітації за десятигодинний
робочий день фабриканти кричали, що робітнича наволоч подає
петиції, сподіваючись дістати за десятигодинну працю дванадцятигодинну
заробітну плату. Тепер вони обернули медалю. Вони
виплачували десятигодинну заробітну плату за дванадцяти й
п’ятнадцятигодинне порядкування робочими силами!\footnote{
Див. «Reports etc. for 30 th April 1849», p. 6 і докладне пояснення
«shifting system»\footnote*{
— системи пересувань. \emph{Ред.}
}, яке фабричні інспектори Хоуелл і Савндер дають
у «Reports etc. for 31 st October 1848». Див. також петицію проти
«shift system», подану королеві духівництвом Ashton’a й околиць на весні
1849~\abbr{р.}
} Так ось
у чім була річ; це було фабрикантське видання десятигодинного
закону! Це були ті самі фритредери, сповнені благодаті й любови
до людства, що підчас аґітації проти хлібних законів цілих десять
років до останнього шага обчислювали робітникам, що за вільного
довозу хліба, при тих засобах, що їх має англійська промисловість,
цілком досить було б десяти годин праці, щоб збагатити
капіталістів\footnote{
Порівн., наприклад, «The Factory Question and the Ten Hours
Bill. By R. H. Greg. 1837».
}.

Дворічний бунт капіталу увінчався нарешті присудом однієї
з чотирьох вищих судових установ Англії, Court of Exchequer,
який в одному з випадків, що дійшов до нього, 8 лютого 1850~\abbr{р.}
вирішив, що хоч фабриканти й чинили проти змісту закону
1844~\abbr{р.}, але самий цей закон містить у собі деякі слова, що роблять
його безглуздим. «Цей вирок знищив закон про десятигодинну
працю»\footnote{
\emph{F. Engels}: «Die englische Zehnstundenbill» (у видаваній мною
«Neue Rheinische Zeitung». Politish-ökonomische Revue, Aprilheft
1850», p. 13). Той самий «високий» суд так само винайшов підчас американської
громадянської війни словесну зачіпку, яка перетворювала закон
проти озброєння піратських кораблів у його пряму протилежність.
}. Маса фабрикантів, що досі боялись застосовувати
систему змін для підлітків і робітниць, ухопилися за неї тепер
обома руками\footnote{
«Reports etc. for 30 th April 1850».
}.

Але за цією, здавалось, остаточною перемогою капіталу
настав зараз же поворот. Робітники досі ставили пасивний, хоч
і впертий і день-у-день відновлюваний опір. Тепер вони почали
голосно протестувати на загрозливих мітинґах у Ланкашірі і
Йоркшірі. Значить, так званий десятигодинний закон — це лише
ошуканство, парляментське шахрайство, а на ділі він ніколи не
існував! Фабричні інспектори пильно попереджали уряд, що
\parbreak{}  %% абзац продовжується на наступній сторінці
