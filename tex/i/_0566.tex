\parcont{}  %% абзац починається на попередній сторінці
\index{i}{0566}  %% посилання на сторінку оригінального видання
інших попередніх нестатків. Задовго перед тим, як недостатнє
харчування почне впливати на здоров’я, задовго перед тим, як
фізіолог почне лічити ґрани азоту й вуглецю, між якими коливається
життя й голодна смерть, домашнє господарство вже геть
чисто позбувається всякого матеріяльного комфорту. Одяг і
опалення стають ще злиденніші, ніж харч. Немає достатнього
захисту від суворої негоди; щодо житла, то обмеження доходить
такої міри, що воно породжує або загострює недуги; ледве помітні
сліди домового начиння або меблів, навіть чистота стає
дуже дорогою; підтримувати її важко. А проте, коли з самоповаги
ще й робляться спроби підтримувати її, то кожна така спроба
сполучена з додатковими муками голоду. Оселяються там, де
найдешевше можна найняти притулок, у кварталах, де санітарна
поліція має якнайменші успіхи, де каналізація найгірша, де
найнезначніша комунікація, найбільше бруду, де наймізерніше
й найгірше водопостачання, а по містах — де найбільша недостача
світла й повітря. Це — ті небезпеки для здоров’я, яких
неминуче зазнає бідність, коли вона сполучена з недостачею харчів.
Коли сума всіх цих злигоднів має страшенний вплив на
життя, то вже проста недостача харчів сама по собі жахлива\dots{}
Це — дуже болючі думки, особливо коли пригадати собі, що
бідність, про яку тут мовиться, це не є бідність з своєї вини,
викликана ледарством. Це — бідність робітників. І навіть щодо
міських робітників, ту працю, за яку вони купують собі цю мізерну
дещицю харчу, здебільша здовжують понад усяку міру.
А проте лише в дуже умовному розумінні можна сказати, що
з цієї праці можна себе самого утримати\dots{} Це номінальне утримання
самого себе у дуже великій мірі є не що інше, як коротший
або довший обхідний шлях до павперизму»\footnote{
Там же, стор. 14, 15.
}.

\looseness=1
Внутрішній зв’язок між муками голоду якнайпрацьовитіших
верств робітників і брутальним або рафінованим марнотратством
багатих, основаним на капіталістичній акумуляції, можна
розкрити лише через пізнання економічних законів. Інша справа
з житловими умовами. Кожний безсторонній спостерігач бачить,
що чим масовіший характер має централізація засобів продукції,
тим більше відповідне скупчення робітників на тому самому
просторі, отже, чим швидша капіталістична акумуляція, тим
злиденніші житлові умови робітників. «Поліпшення» (improvements)
міст через зламання погано побудованих кварталів,
будування палаців для банків, універсальних крамниць і~\abbr{т. д.},
будування вулиць для комерційних зносин і розкішних екіпажів,
заведення кінних залізниць і~\abbr{т. д.}, — ці «поліпшення»,
що супроводять проґрес багатства, очевидно, заганяють бідних
у щораз гірші й щораз густіш заповнювані закутки. З другого
боку, кожний знає, що дорожнеча помешкань є зворотно пропорційна
до їхньої якости, і що будівельні спекулянти з більшим
зиском і з меншими витратами експлуатують копальні бідности,
\parbreak{}  %% абзац продовжується на наступній сторінці
