продукті.\footnote{
«Заробітну плату, як і зиск, треба розглядати як частину готового
продукту» («Wages as well as profits are to be considered each oj
them as really a portion of the finished product»!. (G. Ramsay: «An Essay
on the Distribution of Wealth», Edinburgh 1836 p. 142). «Частина продукту,
що припадає робітникові у формі заробітної плати». (J. Mill:
«Elements of Political Economy». Переклад Parissot’a, Paris 1823 p. 34).
} Це — частина продукту, постійно репродукованого
самим робітником, яка постійно припливає до нього назад у
формі заробітної плати. Правда, капіталіст виплачує йому цю
товарову вартість грішми. Але ці гроші є лише перетворена форма
продукту праці [або, точніше, певної частини продукту праці].\footnote*{
Заведене у прямі дужки беремо з другого німецького видання. \emph{Ред.}
}
Тимчасом як робітник перетворює частину засобів продукції
на продукт, частина його попереднього продукту знов іеретворюється
на гроші. Його праця минулого тижня або останнього
півріччя є те, чим оплачують його сьогоднішню працю або працю
найближчого півріччя. Ілюзія, яку утворює грошова форма,
вмить зникає, скоро тільки ми замість поодинокого капіталіста
й поодинокого робітника розглядатимемо клясу капіталістів і
клясу робітників. Кляса капіталістів постійно дає клясі робітників
у формі грошей чеки на частину продукту, випродукованого
клясою робітників і присвоєного клясою капіталістів. Ці
чеки робітник так само постійно повертає клясі капіталістів і
таким чином відбирає від неї ту частину свого власного продукту,
що припадає йому самому. Товарова форма продукту і грошова
форма товару замасковують цей процес.

Отже, змінний капітал\footnote*{
У французькому виданні Маркс тут робить таку примітку: «Змінний
капітал тут розглядається виключно як фонд для оплати найманих
робітників. Відомо, що в дійсності він стає змінним лише з того моменту,
коли куплена ним робоча сила функціонує вже в процесі продукції».
\emph{Ред.}
} є лише осібна історична форма виявлення
фонду засобів існування або робочого фонду, що його
робітник потребує для свого утримання й своєї репродукції і
що його він за всяких систем суспільної продукції завжди мусить
сам продукувати й репродукувати. Робочий фонд постійно
припливає до нього у формі засобів платежу за його працю лише
тому, що його власний продукт постійно віддалюється від нього
у формі капіталу. Але ця форма виявлення робочого фонду нічого
не змінює в тому, що капіталіст авансує робітникові його
власну упредметнену працю.\footnote{
«Коли капітал вживають на авансування робітникам заробітної
плати, то він нічого не додає до фонду, призначеного для підтримання
праці» («When capital is employed in advancing to the workmen his wages
it adds nothing to the funds for the maintenance of labour»). (Cazenove
у примітці до його видання праці Малтуза «Definitions in Political Economy»,
London 1853, p. 22).
} Візьмімо селянина-кріпака. Він
працює своїми власними засобами продукції на своєму власному
полі, наприклад, три дні на тиждень. Три інші дні на тиждень
він одробляє панщину в панському маєтку. Він постійно репродукує
свій власний робочий фонд, і цей фонд ніколи не наби-