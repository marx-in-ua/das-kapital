Далі, вартість робочої сили не може падати, отже, додаткова
вартість не може підвищуватися без того, щоб не підвищувалася
продуктивна сила праці; наприклад, у вищенаведеному випадку
вартість робочої сили не може впасти з 3 до 2 шилінґів, якщо
підвищена продуктивна сила праці не дозволяє за 4 години продукувати
ту саму масу засобів існування, яка раніш потребувала
для своєї продукції 6 годин. Навпаки, вартість робочої сили не
може підвищитися з 3 шилінґів до 4 без того, щоб продуктивна
сила праці не знизилась, отже, без того, щоб не потрібно було
8 годин на продукцію тієї самої маси засобів існування, яку раніш
продукувалося за 6 годин. Звідси випливає, що збільшення продуктивности
праці знижує вартість робочої сили і тим підвищує
додаткову вартість, і, навпаки, зменшення продуктивности праці
підвищує вартість робочої сили та знижує додаткову вартість.

Формулюючи цей закон, Рікардо не помітив одного: хоч зміна
величини додаткової вартости або додаткової праці зумовлює
зворотну зміну величини вартости робочої сили або доконечної
праці, але звідси ні в якому разі не випливає, що вони змінюються
в тій самій пропорції. Вони більшають або меншають на ту саму
величину. Але пропорція, в якій кожна частина новоспродукованої
вартости або робочого дня більшає або меншає, залежить
від первісного поділу, що був перед зміною продуктивної сили
праці. Якщо вартість робочої сили була 4 шилінґи, або доконечний
робочий час — 8 годин, додаткова вартість — 2 шилінґи,
або додаткова праця — 4 години, і якщо, в наслідок підвищення
продуктивної сили праці, вартість робочої сили падає
до 3 шилінґів, або доконечна праця падає до 6 годин, то додаткова
вартість підвищується до 3 шилінґів, або додаткова праця
до 6 годин. Ту саму величину, 2 години, або 1 шилінґ, там додано,
тут однято. Але відносна зміна величин на обох сторонах
різна. Тимчасом як вартість робочої сили падає з 4 шилінґів
до 3, отже, на \sfrac{1}{4}, або на 25\%, додаткова вартість підвищується
з 2 шилінґів до 3, отже, на \sfrac{1}{2}, або 50\%. Звідси випливає, що
відносне збільшення або зменшення додаткової вартости, яке
постає в наслідок даної зміни продуктивної сили праці, то більше,
що менша, і то менше, що більша була первісно частина робочого
дня, яка виражається в додатковій вартості.

По-третє, збільшення або зменшення додаткової вартости є
завжди наслідок, але ніколи не причина відповідного зменшення
або збільшення вартости робочої сили.10

10    До цього третього закону Мак Куллох зробив, між іншим, безглуздий
додаток, ніби додаткова вартість може підвищуватися й без зниження
вартости робочої сили, в наслідок скасування податків, що їх раніш мав
платити капіталіст. Скасування таких податків аніскільки не змінює
тієї кількости додаткової вартости, яку промисловий капіталіст безпосередньо
витискує з робітника. Воно змінює лише відношення між тією
частиною додаткової вартости, яку капіталіст ховає собі до кишені, і
тією частиною, що її він мусить віддати третім особам. Отже, воно нічого не
змінює у відношенні між вартістю робочої сили й додатковою вартістю.
Таким чином, виняток Мак Куллоха доводить тільки його нерозуміння
