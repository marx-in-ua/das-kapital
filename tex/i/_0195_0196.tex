\parcont{}  %% абзац починається на попередній сторінці
\index{i}{0195}  %% посилання на сторінку оригінального видання
на той світ. Причина нещастя — недбальство залізничників. Вони
в один голос заявляють перед присяжними, що років 10--12 тому
їхня праця тривала лише 8 годин на день. За останні 5--6 років
робочий час було доведено до 14, 18 і 20 годин, а за особливо
жвавого припливу пасажирів, як от підчас періоду екскурсій,
робочий час триває часто без перерви 40--50 годин. Вони, мовляв,
звичайні люди, а не циклопи. На якомусь даному пункті
їхня робоча сила відмовляється служити. Їх обіймає одубілість,
мозок перестає думати, очі — бачити. Цілком «respectable Вгіtish,
Juryman»\footnote*{
поважний брітанський присяжний засідатель. \emph{Ред.}
} відповідає на це присудом, що відсилає їх до
карного суду за «manslaughter»\footnote*{
вбивство. \emph{Ред.}
}, і в додатковому пункті в лагідному
тоні висловлює побожне побажання, щоб пани-маґнати
від залізничного капіталу надалі були щедріші, купуючи потрібну
кількість «робочих сил», і щоб виявляли більше «поздержливости»,
або «самовідречення», або «ощадности», висисаючи
оплачену робочу силу\footnoteWithFotnote{
«Reynolds, Newspaper», січень 1866~\abbr{р.} Ця сама тижнева газета щотижня
подає цілу низку нових залізничних катастроф під «sensational headings\footnote*{
сенсаційними заголовками. \emph{Ред.}
}:
«Fearful and fatal accidents»\footnote*{
Жахливий і фатальний випадок. \emph{Ред.}
}, «Appalling tragedies»\footnote*{
Жахлива трагедія. \emph{Ред.}
} і~\abbr{т. ін.}
Це викликало таку відповідь одного робітника з північно-стафордшірської
залізниці: «Кожний знає, до яких наслідків доводить навіть хвилинне
ослаблення уваги в машиніста й кочегара. І як воно може бути інакше
за безмірного здовження праці, в найсуворішу годину, без перерви й відпочинку?
Візьміть як приклад такий випадок із щоденного життя. Минулого
понеділка кочегар почав свою денну працю дуже раннім ранком.
Він скінчив її після 14 годин 50 хвилин. Не встиг він іще напитись чаю,
як його знов покликали до праці. Таким чином, він повинен був працювати
без перерви 29 годин 15 хвилин. Дальші дні тижня він був занятий ось
як: середа — 15 годин, четвер — 15 годин 35 хвилин, п’ятниця — 14\sfrac{1}{2} годин,
субота — 14 годин 10 хвилин, разом 88 годин 40 хвилин за тиждень.
Уявіть же ви собі його здивовання, коли він одержав плату лише за 6 робочих
днів. Цей робітник був новак і спитав, що треба розуміти під робочим
днем. Відповідь: 13 годин, тобто 78 годин на тиждень\dots{} Але як же тоді
з виплатою за додаткові 10 годин 40 хвилин? Після довгих суперечок він
одержав 10\pens{ пенсів} (не цілих 10 срібних шагів) винагороди за цей час».
(Там же, число з 4 лютого 1866~\abbr{р.}).
}.

Із строкатої юрби робітників усіх професій, усякого віку й
статі, що насовуються на нас настирливіше, ніж тіні вбитих на
Одіссея, і обличчя яких на перший же погляд, коли навіть не
зазирати до тих Синіх Книг, що вони тримають їх під пахвами,
вказують на надмірну працю, — з цієї юрби робітників виберімо
собі ще дві постаті: кравчиху й коваля, яскравий контраст між
якими доводить, що перед капіталом усі люди рівні.

В останніх тижнях червня 1863~\abbr{р.} всі лондонські газети принесли
замітку під «сенсаційним» заголовком «Death from simple
Owerwork» («Смерть просто від надмірної праці»). Йшлося про
смерть двадцятилітньої кравчихи Мері Енні Волклей, що працювала
в дуже поважній придворній кравецькій майстерні, яку експлуатувала
\index{i}{0196}  %% посилання на сторінку оригінального видання
одна пані з добродушним іменням Еліза. Тут знову
викрито стару історію\footnote{
Порівн. \emph{F.~Engels}: «Die Lage der arbeitenden Klasse in England»,
Leipzig 1845, S. 253, 254. («Становище робітничої кляси в Англії»,
Партвидав «Пролетар» 1932~\abbr{р.}, стор. 210--212).
}, яку часто розказують, історію про те,
як ці дівчата працюють пересічно по 16\sfrac{1}{2} годин на добу, а в час
сезону часто й 30 годин без перерви, причому їхню «робочу силу»,
що відмовлялася служити, час від часу підтримували хересом,
портвайном і кавою. А був це саме найгарячіший час сезону. Треба
було вмить виготовити для благородних леді розкішні сукні на
баль у честь свіжоімпортованої принцеси Велзької. Мері Енні
Волклей працювала без перерви 26\sfrac{1}{2} годин разом із 60 іншими
дівчатами, по 30 в одній кімнаті, яка ледве чи давала третину
потрібної кубатури повітря; спати ж їм доводилося по дві в одному
ліжку в одній такій задушній конурі, де спальні повідгороджувано
різними дощаними перегородками\footnote{
Д-р Лісбі, лікар при Board of Health, пояснював тоді: «Спальня
дорослої людини мусить мати мінімум 300 кубічних футів повітря, а
мешкальна кімната — мінімум 500 кубічних футів». Д-р Річардсон,
головний лікар однієї лондонської лікарні, каже ось що: «Різні швачки,
модистки, кравчихи й звичайні швачки страждають від потрійного лиха:
надмірної праці, недостачі повітря й недостачі харчів або вад у травленні.
Загалом, така праця за всяких обставин більше личить жінкам, ніж чоловікам.
Але нещастя цього ремества в тому, що воно монополізоване, особливо
ж у столиці, від якихбудь 26 капіталістів, які, користуючися з
усіх ресурсів влади, що випливає з капіталу (that spring from capital),
видушують із праці економію (force economy out of labour; він хотів сказати:
заощаджують на видатках, марнуючи робочу силу). Їхню владу
відчуває на собі ціла ця кляса робітниць. Коли кравчисі пощастить придбати
невелике коло замовниць, то конкуренція примушує її дома запрацьовуватись
на смерть, щоб зберегти цих замовниць, і до такої надмірної
праці вона примушена з конечности приневолювати й своїх помічниць.
Коли не пощастить їй у справі, або не зможе вона влаштуватися самостійно,
то вона звертається до якогось підприємства, де працювати доводиться
не менше, але заробіток певніший. Таким чином вона потрапляє у стан
справжньої рабині, яку кидає туди й сюди кожна суспільна хвиля: то вмирає
вона від голоду вдома в маленькій кімнатці, або недалеко їй до цього,
то знов працює по 15, 16 та навіть 18 годин на добу в такому повітрі, що
його ледве можна знести, маючи таку харч, що коли б навіть була й добра,
то її не перетравлював би організм через відсутність свіжого повітря.
Такими жертвами живляться сухоти, що є не що інше, як хороба від недостачі
повітря». (\emph{Dr.~Richardson}: «Work and Overwork» y «Social Science
Review» 18 липня 1863).
}. І це була одна з кращих
модних кравецьких майстерень Лондону. Мері Енні Волклей
занедужала в п’ятницю й померла в неділю, не потурбувавшись
навіть, на велике здивовання пані Елізи, скінчити ще перед тим
останню сукню. Лікар, пан Кейз, покликаний запізно до її смертельної
постелі, такими сухими словами дав свої свідчення перед
«Coroner’s Jury»: «Мері Енні Волклей вмерла від надмірної праці
в переповненій майстерні й від того, що спала в занадто тісній,
погано провітрюваній спальні». Щоб подати лікареві лекцію,
як слід поводитися, «Coroner’s Jury» заявив на те: «Небіжчиця
вмерла від апоплексії, але є підстава побоюватися, що надмірна
праця в переповненій майстерні й~\abbr{т. ін.} прискорила її смерть».
\parbreak{}  %% абзац продовжується на наступній сторінці
