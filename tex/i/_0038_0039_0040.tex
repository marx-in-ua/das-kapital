
\index{i}{0038}  %% посилання на сторінку оригінального видання
Саме такого роду форми і становлять категорії буржуазної
політичної економії. Це — суспільно-визнані дійсними, отже,
об’єктивні форми мислення для продукційних відносин цього
історично визначеного суспільного способу продукції — товарової
продукції. Тим то ввесь містицизм товарового світу, всі чари й
примари, що обгортають туманом продукти праці, які продукується
на основі товарової продукції, все це одразу зникає, скоро
тільки ми звернемось до інших форм продукції.

А що політична економія закохана в робінзонадах\footnote{
Примітка до другого видання. І Рікардо не міг обійтися без робінзонади.
«Прарибалка і прамисливець у нього зараз же виступають
як посідачі товарів, що обмінюють рибу на дичину в пропорції, відповідній
до робочого часу, упредметненого в цих мінових вартостях. За цієї
нагоди він вдається в той анахронізм, що прарибалка й прамисливець
у своїх обрахунках знарядь праці користуються рахівничими таблицями,
що їх уживалось на лондонській біржі в 1817~\abbr{р.} «Паралелограми пана
Овена» є, здається, єдина суспільна форма, що її він знав крім буржуазної».
(\emph{Karl Marks}: «Zur Kritik der Politischen Oekonomie», S. 38, 39. —
\emph{K.~Маркс.} «До критики політичної економії», ДВУ 1926 року,
стор. 76, 77).
}, то нехай
і у нас передусім з’явиться Робінзон на його острові. Хоч і який
він з роду скромний, та все ж він має задовольняти різноманітні
потреби, і тому мусить він виконувати корисні праці різного
роду: виробляти знаряддя, фабрикувати меблі, приручати ляму,
рибалчити, полювати й~\abbr{т. ін.} Про молитву тощо ми тут не говоримо,
бо наш Робінзон находить у тому приємність і вважає подібну
діяльність за відпочинок. Не зважаючи на те, що його продуктивні
функції такі різні, він знає, що вони є тільки різні форми
діяльности того самого Робінзона, отже, лише різні роди людської
праці. Сама нужда примушує його точно розподілювати свій час
на різні функції. Чи та чи інша функція забере йому місця більше
чи менше в його цілій діяльності, це залежить від більших або
менших труднощів, які треба йому перебороти, щоб досягти наміченого
корисного ефекту. Досвід навчає його цього, і наш Робінзон,
що врятував з розбитого корабля годинник, головну книгу,
чорнило й перо, починає зразу ж, як чистокровний англієць, вести
книги про себе самого. Його інвентар складається з реєстра предметів
споживання, які він посідає, різних операцій, потрібних для
продукції предметів споживання, нарешті, робочого часу, якого
йому коштує пересічно певна кількість цих різних продуктів. Всі
відношення між Робінзоном і речами, які становлять його власноручно
створене багатство, тут такі прості й прозорі, що навіть
пан М.~Вірт зрозумів би їх, ані трохи не напружуючи розуму.
А проте в них містяться всі істотні визначення вартости.

\looseness=1
Перенесімось тепер з ясного острова Робінзона в темне середньовіччя
Европи. Замість незалежної людини ми бачимо тут усіх
залежними — кріпаків і сеньйорів, васалів і сюзеренів, парафіян
і попів. Особиста залежність характеризує тут суспільні відносини
матеріяльної продукції так само, як і інші збудовані на ній сфери
життя. Але саме тому, що відносини особистої залежности становлять
\index{i}{0039}  %% посилання на сторінку оригінального видання
основу даного суспільства, праці та її продуктам не потрібно
набирати відмінної від їхньої реальности фантастичної
форми. Вони увіходять у суспільний рух як натуральні служби
і натуральні повинності. За безпосередню суспільну форму праці
є тут її натуральна форма, її осібність, а не її загальність, як це
є на основі товарової продукції. Панщанну працю так само добре
вимірюється часом, як і працю, що продукує товари, але кожний
кріпак знає, що це певна кількість його особистої робочої сили,
яку він витрачає на службі своєму панові. Десятина, яку він має
віддавати попові, є для нього ясніша, ніж благословення попове.
Тому, хоч би й що думати про характеристичні маски, що в них
тут люди протистоять одні одним, в усякому разі суспільні відносини
осіб у їхній праці виявляються тут як їхні власні особисті
відносини, і не є вони переодягнуті в суспільні відносини речей,
продуктів праці.

Щоб розглянути спілкову, тобто безпосередньо усуспільнену
працю, нам не треба звертатись до її природно вирослої форми,
яку ми подибуємо на порозі історії всіх культурних народів\footnote{
Примітка до другого видання. «Останніми часами поширився
смішний забобон, нібито форма первісної громадської власности є специфічна
слов’янська, навіть виключно російська форма. Це є та праформа,
яку ми можемо довести в римлян, германців, кельтів, а в індійців ще й нині
ми знаходимо цілу низку різноманітних зразків цієї форми, хоч уже
почасти й зруйнованих. Глибше студіювання азійських, а особливо індійських
форм громадської власности показало б нам, як із різних форм
первісної громадської власности постають різні форми її розпаду. Так,
наприклад, різні ориґінальні типи римської й германської приватної
власности можна вивести з різних форм індійської громадської власности».
(\emph{К.~Marx}: «Zur Kritik der Politischen Oekonomie», S. 10. — \emph{K.~Маркс.}
«До критики політичної економії», ДВУ, 1926~\abbr{р.}, стор. 51).
}.
Ближчий приклад дає нам сільська патріярхальна індустрія селянської
родини, що для власних потреб продукує хліб, худобу,
пряжу, полотно, одяг і~\abbr{т. ін.} Ці різні речі протистоять цій родині
як різні продукти її родинної праці, але вони не протистоять
одна одній як товари. Різні праці, що витворюють ці продукти, —
рільництво, скотарство, прядільництво, ткацтво, кравецтво і~\abbr{т. ін.} —
є суспільні функції у своїй натуральній формі, бо це функції
родини, яка має свій власний природно вирослий поділ праці
так само, як і товарова продукція. Ріжниця статі й віку, як і
зміни природних умов праці, зумовлені зміною пір року, реґулюють
розподіл праці між членів родини й робочий час поодиноких
членів родини. Але витрата індивідуальних робочих сил,
вимірювана часом її тривання, вже від самого початку з’являється
тут як суспільне визначення самих праць, бо індивідуальні робочі
сили від самого початку функціонують тут лише як органи спільної
робочої сили родини.

Нарешті, уявімо собі, для різноманітности, товариство вільних
людей, що працюють спільними засобами продукції і свідомо
витрачають свої численні індивідуальні робочі сили як одну суспільну
робочу силу. Всі визначення робінзонової праці повторюються
\index{i}{0040}  %% посилання на сторінку оригінального видання
й тут, та тільки ж суспільно, а не індивідуально. Всі продукти Робінзона були виключно його
особистим продуктом, отже, і безпосередньо предметами споживання для нього самого. Сукупний продукт
товариства є суспільний продукт. Певна частина
цього продукту служить знову за засіб продукції. Вона лишається суспільною. Але другу частину члени
товариства споживають як засоби існування. Тим то вона мусить бути розподілена між ними. Спосіб
цього розподілу змінюватиметься разом із осібним характером самого суспільно-продукційного організму
й відповідного рівня історичного розвитку продуцентів. Лише на те, щоб провести паралелю з товаровою
продукцією, ми припускаємо, що пайку кожного продуцента в засобах існування визначає його робочий
час. Отже, робочий час відігравав би двоїсту ролю. Суспільний пляномірний розподіл робочого часу
реґулює правильне відношення різних функцій праці до різних потреб. А, з другого боку, робочий час
разом з тим служить за міру індивідуальної участи продуцента в спільній праці, а тому і в
індивідуально споживаній частині спільного продукту. Суспільні відношення людей до їхніх праць і
продуктів їхньої праці лишаються тут прозоро прості так в продукції, як і в розподілі.

\disablefootnotebreak{}
[Релігійний світ є лише рефлекс реального світу]\footnote*{
Заведене у прямі дужки ми беремо з французького видання: «Le monde religieux n’est que le reflet
du mond réel». \emph{Ред.}
}. Для суспільства товаропродуцентів, що його
загальносуспільні продукційні відносини є в тому, щоб до своїх продуктів ставитись як до товарів,
отже, як до вартостей, і в цій речовій формі відносити одну до однієї свої приватні праці як
однакову людську працю, — для такого суспільства за найвідповіднішу форму релігії є християнство з
його культом абстрактної людини, особливо християнство в його буржуазних формах розвитку,
протестантстві, деїзмі і~\abbr{т. ін.} За староазійського, античного й таких інших способів продукції
перетворення продукту на товар, а тому й існування людей як продуцентів товару, відіграє
підпорядковану ролю, яка проте стає то значніша, що сильніша стадія занепаду громадського ладу.
Власне торговельні народи існують тільки в межисвітових просторах старого світу, як боги Епікура,
або як євреї в порах польського суспільства. Ці старі суспільно-продукційні організми куди простіші
й прозористіші, ніж буржуазні, але вони спираються або на незрілість індивідуальної людини, що не
відірвалася ще від пуповиння природного родового зв’язку з іншими людьми, або на безпосередні
відносини панування й рабства. Їхнє існування обумовлено низьким ступенем розвитку продуктивних сил
праці й відповідно обмеженими відносинами людей в процесі творення їхнього матеріяльного життя, а
тому й відповідно обмеженими відносинами людей між собою й до природи. Ця реальна обмеженість
відбивається ідеально в старовинних природних і народніх релігіях. Релігійний рефлекс дійсного світу
може взагалі зникнути лише тоді, коли відносини
\parbreak{}  %% абзац продовжується на наступній сторінці
