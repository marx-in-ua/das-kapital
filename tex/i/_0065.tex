\parcont{}  %% абзац починається на попередній сторінці
\index{i}{0065}  %% посилання на сторінку оригінального видання
взагалі відбувається, тобто коли товар не є непродажний, то
завжди відбувається зміна його форми, дарма що в ненормальних
випадках у цій зміні форми субстанція — величина вартости —
може меншати або більшати.

Одному товаропосідачеві золото заміняє його товар, другому
товар — його золото. Отже, явище, що тут впадає на очі, є те, що
товар і золото, 20 метрів полотна і 2\pound{ фунти стерлінґів}, переходять
з рук до рук, з місця на місце, тобто обмін їх. Але на що обмінюється
товар? На свою власну загальну форму вартости. А золото?
На осібну форму своєї споживної вартости. Чому золото
виступає проти полотна як гроші? Тому, що ціна полотна —
2\pound{ фунти стерлінґів}, або його грошова назва, ставить його вже у
відношення до золота як до грошей. Товар скидає свою первісну
товарову форму через відчуження, тобто в той момент, коли споживна
вартість товару дійсно притягає до себе золото, в ціні товару
лише уявлюване. Отже, реалізація ціни, або лише ідеальної форми
вартости товару є разом з тим зворотна реалізація лише ідеальної
споживної вартости грошей, а перетворення товару на гроші є
разом з тим перетворення грошей на товар. Цей єдиний процес
є двобічний процес: з полюсу товаропосідача — продаж, з протилежного
полюсу посідача грошей — купівля. Або продаж є купівля,
$Т — Г$ є разом з тим $Г — Т$\footnote{
«Всякий продаж є купівля» («Toute vente est achat»). (\emph{Dr.~Quesnay}:
«Dialogues sur le Commerce et les Travaux des Artisans», Physiocrates,
ed. Daire, I.~Partie, Paris, 1846, p. 170), або, як Кене говорить у своїх
«Maximes Générales»: «Продати — значить купити» («Vendre est acheter»).
}.

\looseness=1
Ми не знаємо досі жодного іншого економічного відношення
людей, опріч відношення посідачів товарів, — відношення, в
якому вони лише привласнюють собі чужий продукт праці, відчужуючи
свій власний. Отже, один посідач товарів може протистояти
іншому лише як посідач грошей, або тому, що його продукт
праці має з природи грошову форму, тобто є грошовий матеріял,
золото і~\abbr{т. ін.}, або тому, що його власний товар змінив уже
шкуру, скинувши свою первісну споживну форму. Щоб функціонувати
як гроші, золото мусить, звичайно, з’явитися в якомусь
пункті на товаровому ринку. Цей пункт лежить біля самого
джерела його продукції — там, де золото, як безпосередній продукт
праці, обмінюється на інший продукт праці тієї самої вартости.
Але, починаючи від цього моменту, воно завжди репрезентує
зреалізовані ціни товарів\footnote{
«За ціну одного товару можна заплатити лише ціною іншого товару»
(«Le prix d’une marchandise ne pouvant être payé que par le prix
d’une autre marchandise»). (\emph{Mercier de la Rivière}: «L’Ordre naturel et
essentiel des sociétés politiques». Physiocrates, éd. Daire, II.~Partie, p. 554).
}. Коли залишити осторонь
обмін золота на товар біля джерела продукції золота, в руках
кожного посідача товарів, воно є преображена форма (entäusserte
Gestalt) його відчуженого товару, продукт продажу, або першої
метаморфози товару $Т — Г$\footnote{
«Щоб мати ці гроші, треба спочатку продати» («Pour avoir çet
argent, il faut avoir vendu»). (Там же, стор. 543).
}. Ідеальними грішми або мірою
\parbreak{}  %% абзац продовжується на наступній сторінці
