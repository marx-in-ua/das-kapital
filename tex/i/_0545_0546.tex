\parcont{}  %% абзац починається на попередній сторінці
\index{i}{0545}  %% посилання на сторінку оригінального видання
середніх потреб самозростання капіталу, є умова існування сучасної
промисловости.

«Припустімо, — каже Г. Мерівел, раніш професор політичної
економії в Оксфорді, а потім урядовець англійського міністерства
колоній, — припустімо, що нація з нагоди якоїсь кризи напружить
свої сили, щоб за допомогою еміґрації позбутися кількох сот
тисяч зайвих бідних. Який був би з цього наслідок? Такий, що
при першому ж відновленні попиту на працю була б недостача
робітників. Хоч і як швидко відбуватиметься репродукція людей,
вона в усякому разі потребує для заміни дорослих робітників
переміжну часу однієї ґенерації. Але зиски наших фабрикантів
залежать переважно від спроможности використовувати сприятливий
момент жвавого попиту й таким чином відшкодовувати
себе за часи застою. Цю спроможність фабрикантам забезпечує
тільки панування над машинами й ручною працею. Для них
повинні знайтись вільні руки; вони повинні бути здібні в разі
потреби дужче напружувати або зменшувати активність своїх
операцій відповідно до стану ринку, бо інакше вони ніяк не зможуть
серед шаленої конкуренції втримати ту перевагу, на якій
основано багатство цієї країни».\footnote{
\emph{Н. Merivale}: «Lectures on Colonization and Colonies», London 1841
and 1842, vol. I, p. 146.
} Навіть Малтуз у перелюдненні,
яке він з свого обмеженого погляду пояснює абсолютним
надмірним приростом робітничої людности, а не тим, що вона
стає відносно надмірною, визнає доконечність для сучасної промисловости.
Він каже: «Мудрі звички щодо шлюбу, доведені
до певної височини серед робітничої кляси якоїсь країни, яка
залежить головним чином від мануфактури й торговлі, були б
для цієї країни шкідливі\dots{} Відповідно до самої природи людности,
приріст робітників не може бути поданий на ринок у наслідок
особливого попиту раніше, ніж мине 16 або 18 років, а перетворення
доходу на капітал через заощадження може відбуватися
куди швидше; країні завжди загрожує, що її робочий фонд зростатиме
швидше, ніж людність».\footnote{
«Prudential habits with regard to marriage carried to a considerable
extent among the labouring class of a country mainly depending upon manufactures
and commerce might injure it\dots{} From the natute of a population,
an increase of labourers cannot be brought into market, in consequence
of a particular demand, till after the lapse of 16 or-18 years, and the conversion
of revenue into capital, by saving, may take place much more rapidly;
a country is always liable to an increase in the quantity of the funds
for the maintenance of labour faster than the increase of population»).
(\emph{Malthus}: «Principles of Political Economy», p. 254, 319, 320). У цій праці
Малтуз відкриває, нарешті, за допомогою Сісмонді, прегарну трійцю капіталістичної
продукції: перепродукцію — перелюднення — переспоживання,
три справді любісінькі почвари (three very delicate monsters, indeed)!
Порівн. \emph{F. Engels}: «Umrisse zu einer Kritik der Nationalökonomie» in
Deutsch-Französische Jahrbücher, herausgegeben von Arnold Rüge
und Karl Marx, Paris 1844, S. 107 ff.
} Оголосивши таким чином
постійну продукцію відносного перелюднення робітників доконечністю
капіталістичної акумуляції, політична економія цілком
\index{i}{0546}  %% посилання на сторінку оригінального видання
на кшталт старої діви, вкладає в уста свого «прегарного
ідеалу», капіталіста, такі слова, звернені до «зайвих» робітників,
викинутих на брук додатковим капіталом, що вони сами його
створили: «Ми, фабриканти, збільшуючи капітал, з якого ви
мусите жити, робимо для вас усе, що можемо; а ви мусите зробити
решту, пристосовуючи свою чисельність до засобів існування».\footnote{
\emph{Harriet Martineau}: «The Manchester Strike», 1842, p. 101.
}

Для капіталістичної продукції ні в якому разі недосить тієї
кількости вільної робочої сили, що її дає природний приріст
людности. Для свого вільного розвитку вона потребує промислової
резервної армії, незалежної від цієї природної межі.

Досі ми припускали, що збільшення або зменшення змінного
капіталу точно відповідає збільшенню або зменшенню числа
занятих робітників.

Однак і за незмінного або навіть зменшеного числа робітників
змінний капітал, що панує над ними, зростає, якщо індивідуальний
робітник дає більше праці і в наслідок цього зростає
його заробітна плата, хоч ціна праці лишається незмінна, а то
навіть падає, тільки повільніше, ніж зростає маса праці. Тоді
приріст змінного капіталу стає показником більшої кількости
праці, але не більшої кількости занятих робітників. Кожний
капіталіст має абсолютний інтерес у тому, щоб видушити певну
кількість праці з меншого, а не з більшого числа робітників,
хоча б останнє коштувало так само дешево, а то й дешевше.
В останньому випадку видатки на сталий капітал зростають
пропорційно до маси праці, пущеної в рух, у першому випадку
вони зростають далеко повільніше. Що більший маштаб продукції,
то вирішальніший є цей мотив. Його вага зростає з акумуляцією
капіталу.

Ми бачили, що розвиток капіталістичного способу продукції
і продуктивної сили праці — одночасно причина й наслідок
акумуляції — дають капіталістові спроможність, за однакової витрати
змінного капіталу, через екстенсивнішу або інтенсивнішу
експлуатацію індивідуальних робочих сил пускати в рух більше
праці. Далі ми бачили, що він за ту саму капітальну вартість
купує більше робочої сили, щораз більше витискуючи навчених
робітників менш навченими, дозрілих робітників — недозрілими,
чоловіків — жінками, дорослих — підлітками й дітьми.

Отже, з проґресом акумуляції більший змінний капітал, з
одного боку, пускає в рух більше праці, не наймаючи більшого
числа робітників, з другого боку, змінний капітал тієї самої
величини пускає в рух більше праці за тієї самої маси робочої
сили і, нарешті, витискуючи робочі сили вищої якости, пускає
в рух більше робочих сил нижчої якости.

Тому продукція відносного перелюднення або звільнення
робітників іде ще швидше, ніж технічний переворот процесу
продукції, і без того прискорюваний проґресом акумуляції, і
\parbreak{}  %% абзац продовжується на наступній сторінці
