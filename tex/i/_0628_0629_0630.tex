\parcont{}  %% абзац починається на попередній сторінці
\index{i}{0628}  %% посилання на сторінку оригінального видання
між 1801 і 1831~\abbr{рр.} загарбовано в нього і парляментськими актами подаровано
лендлордами лендлордам?

Нарешті, останнім великим процесом експропріяції земель у
рільників був так званий Clearing of Estates («очищення маєтків»,
у дійсності очищення їх від людей). «Очищення» — це кульмінаційний
пункт усіх розглянутих досі англійських метод експропріяції.
Як ми бачили в попередньому відділі, при розгляді сучасних відносин,
тепер, коли вже немає незалежних селян, яких треба було б проганяти,
справа доходить до «очищення» землі від котеджів, так що рільничі
робітники на оброблюваній ними землі навіть не находять уже місця,
потрібного для їхніх жител. А що означає «Clearing of Estates» у
власному значенні слова, це ми можемо пізнати лише в горішній Шотляндії,
цій обітованій країні сучасної літератури романів. Там цей процес
визначається своїм систематичним характером, тим широким маштабом,
що в ньому він воднораз відбувається (в Ірляндії землевласники за
одним разом очищають землю від декількох сел; у горішній
Шотляндії йдеться про очищення земельних просторів
завбільшки як німецькі герцоґства) — і, нарешті, особливою формою
загарбовуваної земельної власности.

Кельти горішньої Шотляндії складалися з кланів, з яких
кожний був власником заселеної ним землі. Представник клану,
його голова, або «великий чоловік», був тільки номінальним власником цієї землі,
цілком так само, як англійська королева є номінальна власниця всієї
національної території. Коли англійському урядові пощастило придушити
внутрішні війни поміж цими «великими чоловіками» і їхні
постійні напади на рівнини долішньої Шотляндії, то голови кланів
зовсім не покинули свого старого розбійницького ремества;
вони тільки змінили його форму. Своєю власною владою вони перетворили своє
номінальне право власности на приватне право власности,
а що вони при цьому наражались на опір з боку членів клану, то вони
вирішили просто прогнати їх силою. «Якийбудь англійський король міг би з
таким самим правом претендувати на те, щоб позаганяти своїх підданих у море»
— каже професор Ньюмен\footnote{
«А king of England migth as well claim to drive all his subjects into the sea».
(\emph{F.~W.~Newman}: «Lectures on Political Economy», London 1851, p. 132).
}. Цю революцію що почалась у Шотляндії після останнього повстання претендента,
можна простежити в її перших фазах у творах сера Джемса Стюарта\footnote{
Стюарт каже: «Ренти з цих земель (він помилково переносить цю економічну
категорію на данину, що її taskmen\footnote*{
васаль. \emph{Ред.}
} платить голові клану) є цілком незначні порівняно з розмірами останніх;
щождо числа осіб, які живуть з оренди, то мабуть виявиться, що шматок
землі в гірських місцевостях Шотляндії прогодовує вдесятеро більше людей,
аніж земля такої самої вартости в найбагатших провінціях».
(«Works etc., ed. by General Sir James Steuart, his son», London 1801, vol. I,
ch. 16, p. 104).
} й Джемса Андерсона\footnote{
\emph{James Anderson}: «Observation on the means of exciting a spirit of
National Indusrty etc.», Edinburgh 1777.
}. У XVIII столітті
\index{i}{0629}  %% посилання на сторінку оригінального видання
прогнаним із сел ґаелам одночасно заборонено еміґрувати,
щоб силоміць позаганяти їх у Ґлезґо та в інші фабричні міста\footnote{
В 1860~\abbr{р.} насильно експропрійованих вивезено до Канади, при
чому їм понадавано багато брехливих обіцянок. Декотрі з них повтікали
в гори й на сусідні острови. За ними погнались поліціянти, втікачі вступили
з ними в бійку і таки повтікали.
}.
Як приклад тієї методи, що панувала в XIX столітті\footnote{
«У гірських місцевостях, — писав у 1814~\abbr{р.} Б’юкенен, коментатор
А.~Сміта, — старі відносини власности день-у-день зазнають насильного
перевороту\dots{} Лендлорд, не звертаючи уваги на спадкових орендарів
(цю категорію тут теж ужито помилково), віддає землю тому, хто дає найбільшу
ціну, і коли цей останній є меліоратор (improver), то він одразу заводить
нову систему культури. Земля, раніше рясно вкрита дрібними рільниками,
була заселена відповідно до того, скільки вона давала продукту;
за нової системи поліпшеної культури й збільшених рент треба продукувати
якнайбільше продукту з якомога меншими витратами, і задля цієї
мети усувають усі ті руки, що поробилися некорисними\dots{} Викинуті з
рідних сел шукають собі засобів існування по фабричних містах тощо».
(\emph{David Buchanan}: «Observations on etc. A.~Smith’s Wealth of Nations»,
Edinburgh 1814, vol. IV, p. 144). «Шотляндські вельможі експропріювали
цілі родини так, наче виполювали бур’ян; вони поводилися з селами
та їхньою людністю так, як індійці у своїй помсті з лігвами хижих звірів.
Людину продають за смушок, за баранячу ногу, навіть за щось
менше\dots{} Підчас нападу на північні провінції Китаю на нараді монголів
було запропоновано винищити людність і її землю перетворити на пасовиська.
Цю пропозицію виконало багато лендлордів горішньої Щотляндії
у своїй власній країні проти своїх власних земляків». (\emph{George
Ensor}: «An Inquiry concerning the Population of Nations», London 1818,
p. 215, 216).
}, тут досить
навести «очищення», пороблені герцоґинею Сотерлендською.
Ця економічно освічена особа, скоро взяла в свої руки управління,
зараз же вирішила розпочати радикальне економічне лікування
краю й перетворити на пасовиська для овець ціле графство,
що його людність попередніми подібними процесами була
вже зменшена до \num{15.000} душ. Від 1814 до 1820~\abbr{р.} цих \num{15.000} жителів,
приблизно \num{3.000} родин, систематично проганяли й винищували.
Всі їхні села позруйновано й попалено, всі їхні поля
поперетворювано на пасовиська. Між брітанськими солдатами,
присланими для екзекуції, та місцевою людністю доходило до
боїв. Одна стара жінка згоріла в полум’ї своєї хати, не схотівши
залишити її. Таким чином ця мадам присвоїла собі \num{794.000} акрів
землі, що від незапам’ятних часів належала кланові. Для прогнаних
тубільців вона відвела на узмор’ї приблизно \num{6.000} акрів
землі, по 2 акри на родину. Ці \num{6.000} акрів до того часу лежали
пустирем, не приносячи їхній власниці ніякого доходу. У своїх
благородних почуттях герцоґиня пішла так далеко, що здала
цю землю в оренду пересічно по 2\shil{ шилінґи} 6\pens{ пенсів} ренти за акр
членам клану, які протягом століть проливали за її рід свою кров.
Всю землю, загарбану в клану, вона поділила на 29 великих
овечих фарм, посадивши на кожній одним-одну родину, здебільшого
англійських фармерських наймитів. У 1825~\abbr{р.} замість
\num{15.000} ґаелів там було вже \num{131.000} овець. Викинута на узмор’я
частина тубільців намагалася прожити з рибальства. Вони поробилися
\index{i}{0630}  %% посилання на сторінку оригінального видання
амфібіями й жили, як каже один англійський письменник,
наполовину на землі й наполовину на воді, але і з того
і з другого вони жили тільки наполовину\footnote{
Коли нинішня герцоґиня Сотерлендська з великою пишністю
приймала в Лондоні пані Бічер-Стов, авторку «Хижини дядька Тома»,
щоб показати свою симпатію до рабів-негрів американської республіки, —
підчас громадянської війни, коли кожне «благородне» англійське серце
співчувало рабовласникам, вона разом з іншими аристократками благорозумно
забула про цю симпатію, — я змалював у «New-York Tribune»
становище сотерлендських рабів. (Частину моєї статті Кері подав у витягах
у «The Slave Trade», London 1853, p. 202, 203). Мою статтю передруковано
в одній шотляндській газеті, і вона викликала «ввічливу» полеміку
поміж цією газетою і сикофантами Сотерлендів.
}.

Але бравим ґаелам довелося ще тяжче спокутувати своє гірсько\dash{}романтичне
ідолопоклонство перед «великими людьми»
клану. Запах риби лоскотав великим людям у носі. Вони занюхали
тут щось зисковне і заорендували узмор’я великим риботорговцям
з Лондону. Ґаелів прогнано вдруге\footnote{
Цікаві подробиці про цю торговлю рибою подає пан \emph{Давид Уркварт}
«Portfolio, New Series». — \emph{Н.~В.~Сеніор} y своєму вищецитованому посмертному
творі «Journals, Conversations and Essays relating to Ireland»,
London 1868, кваліфікує «процедуру в Sutherlandshire, як одне з найдобродійніших
очищень (clearings), що їх люди пам’ятають».
}.

Але кінець-кінцем частину овечих пасовиськ перетворено
на мисливські парки. Як відомо, в Англії нема справжніх лісів.
Дичина по парках вельмож — це конституційна домашня худоба,
гладка, як лондонський aldermen\footnote*{
член міської ради. \emph{Ред.}
}. Тим то Шотляндія є останнє
пристановище цієї «благородної пристрасти». «У гірських місцевостях,
— каже Сомерс в 1848~\abbr{р.}, — лісова площа значно
поширилась. Тут, по цей бік Gaick’a, ви бачите новий ліс Glenfeshie,
а там, по другий бік, новий ліс Ardverikie. Там же ви маєте
й Blak-Mount, величезну пущу, нещодавно тільки заведену.
Із сходу на захід, від околиць Aberdeen’a й до скель Oban’а,
тягнеться тепер безперервна смуга лісів, тимчасом як по інших
частинах гірського краю стоять нові ліси Loch Archaig, Glengarry,
Glenmoriston і ін. Перетворення земель ґаелів на пасовиська\dots{}
загнало їх на неродючі землі. Тепер олені й сарни (Rotwild)
починають витискувати овець, кидаючи цим ґаелів у ще
гірші злидні\dots{} Мисливські парки\footnoteA{
У шотляндських «deer forests» (мисливських парках) немає жодного
дерева. Овець виганяють геть, на їхнє місце в голі гори приганяють
оленів, і це називають «deer forest». Отже, тут немає навіть лісової культури.
} й народ не можуть існувати
одне побіч одного. В усякому разі хтобудь із них мусить очистити
місце. Якщо місця для полювання протягом найближчої чверти
віку зростатимуть щодо кількости й простору, як і минулої
чверти, то жодного ґаела не залишиться на його рідній землі.
Цей рух серед землевласників гірських місцевостей спричинено
почасти модою, аристократичними примхами, мисливським запалом
тощо, а почасти землевласники торгують дичиною, маючи
на меті виключно зиск. Бо це факт, що шматок гірської землі,
\parbreak{}  %% абзац продовжується на наступній сторінці
