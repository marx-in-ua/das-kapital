\parcont{}  %% абзац починається на попередній сторінці
\index{i}{0442}  %% посилання на сторінку оригінального видання
1) Скорочення робочого дня за даних умов, тобто, коли продуктивність
та інтенсивність праці не змінюються, залишає
вартість робочої сили, а тому й доконечний робочий час незмінним.
Воно зменшує додаткову працю й додаткову вартість. З абсолютною
величиною останньої падає і її відносна величина, тобто
її величина проти незмінної величини вартости робочоі сили.
Тільки через пониження її ціни нижче від її вартости може капіталіст
триматися без втрат.

В усіх звичайних запереченнях проти скорочення робочого
дня виходять з тієї гіпотези, що це явище відбувається при припущених
тут обставинах, тимчасом як у дійсності, навпаки, зміни
продуктивности й інтенсивности праці або передують скороченню
робочого дня, або безпосередньо настають після нього\footnote{
«Є обставини, що компенсують це\dots{} їх вивів на світ десятигодинний
фабричний закон» («There are compensating circumstances\dots{} which
the working of the Ten Hour’s Act has brought to light»). («Reports of
Insp. of Fact, for 1st December 1848», p. 7).
}.

2) Здовження робочого дня. Хай доконечний робочий час
буде 6 годин, або вартість робочої сили 3\shil{ шилінґи}, так само додаткова
праця — 6 годин, або додаткова вартість 3\shil{ шилінґи.}
Тоді цілий робочий день становить 12 годин і виражається у
продукті вартістю в 6\shil{ шилінґів.} Якщо робочий день здовжується
на дві години, а ціна робочої сили лишається незмінна, то з
абсолютною зростає й відносна величина додаткової вартости.
Хоч величина вартости робочої сили абсолютно лишається незмінна,
відносно вона спадає. За умов пункту 1) відносна величина
вартости робочої сили не могла змінятися без зміни її абсолютної
величини. Тут, навпаки, відносна зміна величини вартости
робочої сили є результат абсолютної зміни величини додаткової
вартости.

А що новоспродукована вартість, у якій виражається робочий
день, зростає разом з його здовженням, то ціна робочої сили
і додаткова вартість можуть зростати одночасно, чи то на однакову
чи на неоднакову величину. Отже, це одночасне зростання
можливе у двох випадках — при абсолютному здовженні робочого
дня і при ростущій інтенсивності праці без такого здовження.

Із здовженням робочого дня ціна робочої сили може впасти
нижче від її вартости, хоч би номінально ця ціна й лишилася
незмінна або навіть і зросла. Адже ж денну вартість робочої сили,
як ми собі пригадуємо, оцінюється за її нормальним пересічним
триванням або за нормальним періодом життя робітника та за
відповідним, нормальним, властивим людській натурі перетворенням
життєвої субстанції на рух\footnote{
«Кількість праці, витраченої людиною протягом 24 годин, можна
приблизно визначити, досліджуючи хемічні зміни, що відбулися в її
тілі, бо зміна форм матерії показує на попередню діяльність динамічної
сили» («The amount of labour which a man had undergone in the course
of 24 hours might approximative arrived at by an-examination of the
chemical changes which had taken place in his body, changed forms in
matter indicating the anterior exercise of dynamic force»). (Grove: «On
the Correlation of Physical Forces»).
}. До певного пункту збільшене
\index{i}{0443}  %% посилання на сторінку оригінального видання
зужитковання робочої сили, невіддільне від здовження
робочого дня, можна компенсувати збільшеним відживленням її.
Поза цим пунктом зужитковування зростає в геометричній проґресії
і одночасно руйнуються всі нормальні умови репродукції
та функціонування робочої сили. Ціна робочої сили і ступінь
її експлуатації перестають бути спільномірними величинами.

\vspace{\medskipamount}
\subsection{Одночасні зміни тривання праці, продуктивної сили праці
та~інтенсивности праці}
\vspace{\medskipamount}

\looseness=1
Тут, очевидно, можливе велике число комбінацій. Можуть
змінятися два фактори, а один лишатися сталим, або всі три фактори
можуть одночасно змінятися. Вони можуть змінятися в
однаковій або неоднаковій мірі, в тому самому або в протилежному
напрямі, і їхні зміни можуть тому почасти або цілком
навзаєм компенсуватися. А втім, аналіза всіх можливих випадків,
після висновків, поданих у пунктах І, II та III, легка. Результат
кожної можливої комбінації можна знайти, коли розглядати
почережно кожний з факторів як змінний, а інші як сталі.
Тому ми тут коротко зазначимо лише два важливі випадки.

1) Падуща продуктивна сила праці при одночасному здовженні
робочого дня.

Коли ми тут говоримо про падущу продуктивну силу праці,
то йдеться про галузі праці, що їхні продукти визначають вартість
робочої сили, отже, наприклад, про падущу продуктивну силу
праці в наслідок чимраз більшої неродючости ґрунту та відповідного
подорожчання продуктів землі. Припустімо, що робочий
день триває 12 годин, новоспродукована протягом нього вартість
становить 6\shil{ шилінґів}, з чого половина покриває вартість робочої
сили, а друга половина становить додаткову вартість. Отже,
робочий день розпадається на 6 годин доконечної праці і 6 годин
додаткової праці. Хай у наслідок подорожчання продуктів землі
вартість робочої сили підвищується з 3\shil{ шилінґів} до 4\shil{ шилінґів},
отже, доконечний робочий час — з 6 до 8 годин. Якщо робочий
день лишається незмінний, то додаткова праця спадає з б до 4 годин,
а додаткова вартість з 3 до 2\shil{ шилінґів.} Якщо робочий день
здовжується на 2 години, тобто з 12 до 14 годин, то додаткова
праця лишається 6 годин, а додаткова вартість 3\shil{ шилінґи}, але
величина цієї останньої падає порівняно з вартістю робочої сили,
вимірюваною доконечною працею. Якщо робочий день здовжується
на 4 години — з 12 до 16 годин, то відносні величини додаткової
вартости й вартости робочої сили, додаткової праці й
доконечної праці, лишаються незмінні, але абсолютна величина
додаткової вартости зростає з 3 до 4\shil{ шилінґів}, абсолютна величина
додаткової праці — з 6 до 8 робочих годин, отже, на \sfrac{1}{3},
або 33\sfrac{1}{3}\%. Отже, при зменшенні продуктивної сили праці
і одночасному здовженні робочого дня абсолютна величина
додаткової вартости може лишатись незмінна, тимчасом як її
відносна величина падає; її відносна величина може лишатись
\parbreak{}  %% абзац продовжується на наступній сторінці
