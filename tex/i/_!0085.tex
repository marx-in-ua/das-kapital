\parcont{}  %% абзац починається на попередній сторінці
\index{i}{*0085}  %% посилання на сторінку оригінального видання
майже вичерпано. Вже 1871~\abbr{р.} п. М.~Зібер, професор політичної
економії Київського університету, показав у своїй книзі «Теория
ценности и капитала Д.~Рикардо», що моя теорія вартости, грошей
і капіталу в своїх основних рисах є доконечний дальший
розвиток науки Сміта--Рікарда. Що вражає західньоевропейського
читача при читанні його чудової книги, так це послідовне
додержування суто теоретичного погляду.

Методу, застосовану в «Капіталі», мало зрозуміли, як це вже
доводять різні розуміння її, що суперечать одно одному.

Так, паризьке «Revue Positiviste» закидає мені, з одного боку,
що я розглядаю політичну економію метафізично, а з другого, —
вгадайте! — що я обмежуюсь тільки критичною аналізою даного,
замість складати рецепти (контівські?) для лябораторії (Garküche)
будучини. Щодо закидуваної мені метафізики професор Зібер зауважує:
«Оскільки ідеться про теорію у власному значенні слова,
метода Марксова є дедуктивна метода цілої англійської школи,
що її хиби й переваги спільні всім кращим теоретикам-економістам».
Пан М.~Бльок — «Les Théoriciens du Socialisme en Allemagne.
Extrait du Journal des Economistes, juillet et août 1872» —
відкриває, що моя метода аналітична, і каже, між іншим, ось що:
«Par cet ouvrage М.~Marx se classe parmi les esprits analitiques les
plus éminents»\footnote*{
«Цією працею Маркс зайняв місце серед найвидатніших аналітичних
мислителів». \Red{Ред.}
}. Німецькі рецензенти кричать, звичайно, про
геґелівську софістику. Петербурзький «Вестник Европы» у
статті\footnote*{
Стаття ця належить відомому економістові І.~І.~Кавфманові. \Red{Ред.}
}, яка розглядає виключно методу «Капіталу» (число за
травень 1872~\abbr{р.}, стор. 427--436), вважає мою методу досліду за
строго реалістичну, а методу викладу, на нещастя, за німецько-діялектичну.
Там сказано: «На перший погляд, коли зважати на
зовнішню форму викладу, Маркс є найбільший ідеаліст-філософ
і саме в німецькому, тобто в поганому розумінні цього слова.
В дійсності ж він безмірно більший реаліст, ніж усі його попередники
в справі економічної критики\dots{} Ідеалістом його аж ніяк не
можна назвати». Не можу краще відповісти авторові, як хіба
декількома витягами з його власної критики, що до того можуть
зацікавити декого з моїх читачів, яким російський ориґінал є
неприступний.

Навівши цитату з моєї передмови до «\textgerman{Zur Kritik der Politischen
Oekonomie}», Berlin, 1859, стор. IV--VII, де я викладаю
матеріялістичні основи своєї методи, пан автор каже далі ось що:

«Для Маркса важливо лише одно: знайти закон тих явищ, що
коло розсліду їх він працює. І при цьому для нього важливий не
лише закон, що ними керує, поки вони мають певну форму й перебувають
у тому взаємовідношенні, яке спостерігається протягом
даного часу. Для нього передусім важливий іще закон їхньої
змінности, їхнього розвитку, тобто переходу з однієї форми в
іншу, з одного порядку взаємовідношень в інший. Скоро він уже
\parbreak{}  %% абзац продовжується на наступній сторінці
