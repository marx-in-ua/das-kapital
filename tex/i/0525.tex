дукція безупинно репродукує саме капіталістичне відношення,
капіталістів на одному боці, робітників на другому, так і репродукція
в поширеному маштабі, або акумуляція, репродукує
капіталістичне відношення в поширеному маштабі — більше капіталістів
або більших капіталістів на цьому полюсі, більше
найманих робітників на тому. Репродукція робочої сили, що невпинно
мусить входити до складу капіталу як засіб збільшення
його вартости і не може відокремитись від нього, — робочої сили,
що її підлеглість капіталові лише маскується зміною індивідуальних
капіталістів, яким вона продає себе, становить, в дійсності,
момент репродукції самого капіталу. Отже, акумуляція
капіталу є збільшення пролетаріяту.\footnote{
К. Marx: «Lohnarbeit und Kapital». (K. Маркс: «Наймана праця
і капітал», Партвидав «Пролетар» 1932). — «За однакового пригнічення
мас що більше в країні пролетарів, то вона багатша» («A égalité
d’oppression des masses, plus un pays a de prolétaires et plus il est riche»).
(Colins: «L’Economie Politique, Source des Révolutions et des Utopies
prétendues Socialistes», Paris 1857, vol. III, p. 331). Під «пролетарем»
y політичній економії треба розуміти не що інше, як найманого робітника,
який продукує «капітал» і збільшує його вартість і якого викидають
на брук, скоро тільки він стає зайвим для потреб самозростання
«пана капіталу», як називає цю персону Пекер. «Хоробливий пролетар
пралісу» — це лише чемна фантазія Рошера. Пралісовик є власник того
пралісу й поводиться з пралісом як із своєю власністю, так само безцеремонно
як оранґутанґ. Отже, він не є пролетар. Він був би ним тільки
тоді, коли б праліс експлуатував його, а не він — цей праліс. Щождо
стану його здоров'я, то він витримає порівняння не тільки з сучасним
пролетарем, а й з сифілітичними й золотушними «порядними особами».
А втім, під пралісом пан Вільгельм Рошер розуміє, певно, свою рідну
Lüneburger Heide.
}

Клясична політична економія так добре розуміла цю тезу,
що Адам Сміс, Рікардо й інші, як уже згадано раніш, навіть помилково
ототожнюють акумуляцію із споживанням всієї капіталізованої
частини додаткового продукту продуктивними робітниками,
або з перетворенням її на додаткових найманих робітників.
Уже 1696 р. Джон Беллерс каже: «Коли б якась людина
мала 100.000 акрів і стільки ж фунтів стерлінґів грошей і стільки
ж худоби, то чим була б ця багата людина без робітників, як не
робітником? А через те, що робітники роблять людей багатими,
то що більше робітників, то більше багатих... Праця бідних — то
копальні багатих».\footnote{
«As the Labourers make men rich, so the more Labourers, there
will be the more rich men... the Labour of the Poor being the Mines of the
Rich». (John Bellers: «Proposals for raising a Colledge of Industry», London
1696, p. 2).
} Те саме каже й Бернар де Мандевіль
на початку XVIII віку: «Де власність має достатній захист, там
легше було б жити без грошей, ніж без бідних, бо хто ж тоді
працював би?.. Так само, як робітників треба захищати від голодної
смерти, так само не повинні вони одержувати нічого
такого, що варто заощаджувати. Якщо інколи хтось із найнижчої
кляси через незвичайну працьовитість та недоїдання підноситься
понад той стан, у якому він виріс, то ніхто не сміє перешкоджати
йому в цьому: аджеж безперечно, що наймудріша річ