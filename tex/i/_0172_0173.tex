\parcont{}  %% абзац починається на попередній сторінці
\index{i}{0172}  %% посилання на сторінку оригінального видання
ніяких чарів, коли нова вартість, яку він спродукував за 5\sfrac{3}{4} годин прядіння, дорівнює вартості
продукту однієї годинии прядіння. Але ви цілком помиляєтесь, думаючи, що він втрачає хоч один атом
часу свого робочого дня на репродукцію, або на «покриття» вартостей бавовни, машин і~\abbr{т. ін.} Через
те, що його праця виробляє з бавовни й веретен пряжу, через те, що він пряде, вартість бавовни й
веретен сама собою переходить на пряжу. Це завдячує якості його праці, а не її кількості. Певна річ, за одну годину він перенесе на пряжу
більшу вартість бавовни й~\abbr{т. ін.}, ніж за півгодини, але лише тому, що за одну годину він випряде
бавовни більше, ніж за півгодини. Отже, зрозумійте: коли ви кажете, що робітник за передостанню
годину продукує вартість своєї заробітної плати, а за останню годину — чистий прибуток, то це
означає лише те, що у пряжі, продукті двох годин його робочого дня, — однаково, чи є вони перші
години, чи останні, — втілено 11\sfrac{1}{2} робочих годин, тобто, саме стільки годин, скільки має цілий
його робочий день. А коли ви кажете, що він за перші 5\sfrac{3}{4} годин продукує свою заробітну плату, а за
останні 5\sfrac{3}{4} годин ваш чистий прибуток, то це означає знову лише те, що за перші 5\sfrac{3}{4} годин ви
платите, а за останні 5\sfrac{3}{4} годин
ви не платите. Я кажу про оплату праці замість казати про оплату робочої сили, для того, щоб
висловлюватись вашим жарґоном. Тепер, панове, коли ви порівняєте відношення того робочого
часу, який ви оплачуєте, до того робочого часу, який ви не оплачуєте, то побачите, що воно дорівнює
відношенню половини дня до половини дня, тобто 100\%, що безперечно є чималий рівень процента. І
немає найменшого сумніву, що, коли ви примусите ваші «руки» працювати 13 годин замість 11\sfrac{1}{2} і — що
до вас подібне, як дві краплини води між собою, — надмірні 1\sfrac{1}{2} години просто додасте до даткової
праці, то остання зросте з 5\sfrac{3}{4} годин до 7\sfrac{1}{4} годин, а тому норма додаткової вартости зросте з
100\% до 126\sfrac{2}{23}\%. Але ви занадто шалені санґвініки, коли сподівається, що через додаток 1\sfrac{1}{2} годин
вона піднесеться з 100 до 200\% і навіть більш ніж до 200\%, тобто «більш ніж подвоїться». З другого
боку, — серце людини є дивовижна річ, особливо коли людина носить своє серце в гаманці, — ви занадто
безглузді песимісти, коли боїтеся, що із скороченням робочого
дня з 11\sfrac{1}{2} на 10\sfrac{1}{2} годин піде за вітром увесь ваш чистий прибуток. Ні в якому разі. Припускаючи
інші обставини за незмінні,
додаткова праця спаде з 5\sfrac{3}{4} на 4\sfrac{3}{4} годин, а це дає все ще дуже значну норму додаткової вартости,
а саме 82\sfrac{14}{23}\%. Але та фатальна «остання година», про яку ви понарозповідали більше байок, ніж
хіліясти про кінець світу, є «all bosh»\footnote*{
цілковита дурниця. \emph{Ред.}
}. Втрата її не відбере ані у вас «чистого прибутку», ані в
дітей обох статей, яких ви примушуєте працювати, «душевної чистоти»\withoutLCW{\footnoteA{
Якщо Сеніор довів, що від «останньої робочої години» залежить чистий прибуток фабрикантів,
існування англійської бавовняної промисловости, велич англійського світового ринку, то д-р Ендр’ю Юр довів іще понад
те, що, коли фабричних дітей і молодь нижче 18 років не замикати
на цілі 12 годин у теплій і чистій моральній атмосфері фабричної майстерні, а на «одну годину» раніш
виганяти їх на холодний, незатишний і розпусний зовнішній світ, то ледарство й розпуста відберуть у
них душевне здоров’я. Від 1848~\abbr{р.} фабричні інспектори у своїх піврічних «Reports»
невтомно дратують фабрикантів «останньою» годиною, «фатальною
годиною». Так, пан Хоуел у своєму фабричному звіті з 31 травня 1855~\abbr{р.}
каже: «коли б нижчеподаний хитромудрий розрахунок (він цитує Сеніора) був правильний, тоді кожна
бавовняна фабрика в Об’єднанім Королівстві, почавши від 1850~\abbr{р.}, працювала б собі на втрату».
(«Reports of
the Inspection of Factories for the half year ending 30 th April 1855», p. 19, 20). 1848~\abbr{р.}, коли
десятигодинний біл пройшов через парлямент, фабриканти сільських пряділень льону, розкиданих поміж
графствами Дорсет і Самерсет, примусили декотрих робітників до контрпетиції, в якій, між іншим,
сказано: «Ми, прохачі-батьки, думаємо, що одна додаткова
вільна година не може мати ніякого іншого наслідку, як хіба деморалізацію наших дітей, бо ледарство
є початок усякої розпусти». Фабричний звіт з 31 жовтня 1848~\abbr{р.} зауважує з приводу цього: «Атмосфера
пряділень льону, де працюють діти цих чеснотно-ніжних батьків, до того переповнена порохом і
частинками волокон із сировинного матеріялу, що
надзвичайно неприємно пробути в прядільні навіть 10 хвилин, бож ви не
можете зробити цього, не діставши дуже прикрого почуття, що ваші очі, вуха, ніздрі й рот миттю
набиваються хмарою пилюги з льону, від якої ніде заховатись. Сама праця через гарячкову швидкість
машин потребує
невсипущої вправности й невсипущого руху під контролем невтомної уваги, і, здається, трохи жорстоко
примушувати батьків уживати висловів «лінощі» щодо власних дітей, які, за винятком часу на їжу, 10
повних годин прикуті до такої праці в такій атмосфері\dots{} Ці діти працюють
довше, ніж рільничі наймити по сусідніх селах\dots{} Таке безжалісливе базікання
про «ледарство й розпусту» треба заплямувати як щонайбільший cant\footnote*{
святошство. \emph{Ред.}
}} і щонайбезсоромніше
лицемірство\dots{} Та частина суспільства, що приблизно перед дванадцятьма роками обурювалась на ту
певність, з якою публічно й цілком серйозно проклямовано за санкцією високого авторитету, нібито
ввесь «чистий прибуток» фабриканта походить з «останньої години праці», і що тому скорочення
робочого дня на одну
годину нищить чистий прибуток, — ця частина суспільства, кажемо ми,
ледве чи повірить власним очам, побачивши, що цей ориґінальний винахід
про чесноти «останньої години» від того часу поліпшено остільки, що вона
рівномірно містить у собі і «мораль», і «зиск», так що коли тривання дитячої праці скоротити до
повних 10 годин, то одночасно з чистим прибутком їхніх наймачів зникне й моральність дітей, бо одне
й друге залежить від цієї останньої, цієї фатальної години». («Reports of Insp. of Fact. for 31 st
October 1848», p. 101). Той самий фабричний звіт потім наводить
зразки «моралі» й «чесноти» цих панів фабрикантів, їхніх каверз, хитрощів,
приман, погроз, підробів і~\abbr{т. ін.}, що їх вони вживають, щоб примусити небагатьох цілком занепалих
робітників підписувати подібні петиції та видавати їх потім перед парляментом за петиції цілої
галузі промисловости, цілих графств. — Надзвичайно характеристичний для сучасного стану так званої
економічної «науки» є той факт, що ні сам Сеніор, який згодом собі на честь енерґійно виступив за
фабричне законодавство, ні його первісні
й пізніші супротивники не змогли розібратися в помилкових висновках з «ориґінального винаходу». Вони
апелювали до фактів і досвіду. Але why й wherefore\footnote*{
як і чому. \emph{Ред.}
} лишилося для них таємницею.}.
Коли дійсно вдарить ваша «остання годинонька», пригадайте собі оксфордського професора. А тепер я
бажаю собі ближчого
\parbreak{}  %% абзац продовжується на наступній сторінці
