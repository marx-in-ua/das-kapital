поділ праці. Якщо ці письменники принагідно згадують і про
зріст маси продуктів, то лише щодо більшої повноти споживних
вартостей. Про мінову вартість, про подешевшання товарів вони
зовсім не згадують. Цей погляд споживної вартости панує так
у Платона,\footnote{
Платон виводить поділ праці всередині громади з багатобічности
потреб та однобічности здібностей індивідів. Його головний погляд є в
тому, що робітник мусить пристосовуватися до справи, а не справа до
робітника, як воно неминуче буває тоді, коли він разом працює в різних
ремествах, отже, в тому чи тому реместві працює як у побічному. «Бо
справі ніколи чекати на вільний час продуцента, а треба, щоб продуцент
виконував свою справу пильно і не між іншим. — Треба. — Аджеж кожна
річ продукується легше й ліпше і в більшій кількості, коли людина робить
лише одну річ, що відповідає її здібності, та в належний час, вільний
від усякої іншої роботи». («Οὐ γὰρ ἐθέλει τὸ πραττόμενον τὴν τοῦ πράττοντος σχολὴν περιμένειν, ἀλλ’ 
ἀνάγκη τὸν πράττοντα τῷ πραττομένῷ ἐπακολουθεῖν μὴ ἐν παρέργου μέρει. — Ἀνάγκη. Ἐκ δὴ τούτων πλείω
τε ἕκαστα γίγνεται καὶ κάλλιον καὶ ῥᾷον, ὅταν εἷς ἓν κατὰ φύσιν καὶ ἐν καιρῷ, σχολὴν τῶν ἄλλων ἄγων,
πράττῃ). («Respublica», lib. II, c. 12,
ed. Baiter, Orelli etc.). Подібні думки ми маємо в Тукідіда: «Geschichte
des Peloponnesischen Krieges», книга перша, відділ 142: «Морська справа
є така ж умілість, як і будь-що інше, і не можна коло неї працювати принагідно,
як коло якоїсь побічної справи, навпаки, морська справа не
дозволяє працювати коло чогось іншого навіть як побічної справи».
Якщо справа мусить чекати на робітника, каже Платон, то часто ґавиться
критичний момент продукції і продукт псується, «ἔργου καιρὸν διόλλυται». Цю
саму платонівську ідею подибуємо знов у протесті англійських власників
білилень проти того застереження фабричного закону, яке встановлює визначену
годину на їжу для всіх робітників. Їхнє підприємство, мовляв, не
може пристосовуватися до робітників, бо «в різних операціях опалювання,
промивання, біління, качання, прасування та фарбування не можна
перервати роботу у наперед визначений момент без небезпеки заподіяти
шкоду... встановлення для всіх робітників тієї самої перерви на їжу —
це значило б у певних випадках кинути коштовні продукти на небезпеку,
що вони попсуються через незакінчені операції» («in the various operations
of singeing, washing, bleaching, mangling, calendering, and dycing,
none of them can be stopped at a given moment without risk of damage...
to enforce the same dinner hour for all the workpeople might occasionally
subject valuable goods to the risk of danger by incomplete operations»).
Le platonisme où va-t-il se nicher!.\footnote*{
Куди ще може продертись платонізм! Ред.
}
} що розглядав поділ праці як основу поділу суспільства
на стани, як і в Ксенофонта,\footnote{
Ксенофонт оповідає, що не тільки велика честь діставати страви
зі столу перського короля, але що й ці страви куди смачніші, ніж ін-
} який з характеристичним для
нього буржуазним інстинктом уже ближче підходить до поділу
праці всередині майстерні. Платонова республіка, оскільки в ній

порядкувати у війні людьми, але не грішми, — як це Тукідід вкладав
в уста Перікла у промові, в якій він підцьковує атенців до пелопонеської
війни: «Σώμασί τε ἐτοιμότεροι οἱ αὐτουργοὶ τῶν ἀνθρώπων ἤ χρήμασι πολεμεῖν».\footnote*{
Люди, що працюють для задоволення власних потреб, радше
віддадуть на війну свої тіла, ніж гроші. Ред.
} (Thucydides:
«Geschichte des Peloponnesischen Krieges», книга перша, відділ
141). А проте їхнім ідеалом, навіть у матеріяльній продукції, була
αυταρχεια,\footnote*{
— автаркія. Ред.
} що протиставляється поділові праці, бо «παρ’ ὧν γὰρ τὸ εὖ, παρὰ τούτων καὶ τὸ
αὐτάρκες».\footnote*{
«з цього постає благо, а з того і незалежність». Ред.
} Треба при цьому зважити, що за часів упадку
30 тиранів не було ще й 5.000 атенців без земельної власности.