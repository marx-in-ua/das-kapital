\setlength{\tabcolsep}{3pt}

\begin{sidewaystable}
  \index{i}{0604}  %% посилання на сторінку оригінального видання
  \footnotesize
  \begin{center}
    \captionnew{\emph{Таблиця С}. Збільшення або зменшення площі обробленої землі, кількости продукту на акр і \\ загальної кількости продукту в році 1865 порівняно з роком 1864\footnotemark{}}
  \end{center}

  \begin{tabularx}{\textheight}{XrrrrXrrccr@{~}lrrr}

    \toprule
    \makecell{\multirow{3}{*}{Продукти}} & 
    \multicolumn{2}{c}{\makecell{Кількість акрів\\обробленої землі}} & 
    \multicolumn{2}{c}{\makecell{Збільшення \\ або зменшення\\в 1865~\abbr{р.}}} & 
    \multicolumn{3}{c}{\makecell{Кількість \\ продуктів на акр }} & 
    \multicolumn{2}{c}{\makecell{Збільшення \\ або зменшення\\ в 1865~\abbr{р.}}} & 
    \multicolumn{5}{c}{\makecell{Загальна кількість продукту}}
    \\

    \cmidrule(rl){2-3}
    \cmidrule(rl){4-5}
    \cmidrule(rl){6-8}
    \cmidrule(rl){9-10}
    \cmidrule(rl){11-15}
    & 
    \makecell{\multirow{2}{*}{1864}} & 
    \makecell{\multirow{2}{*}{1865}} & 
    \makecell{\multirow{2}{*}{$+$}} & 
    \makecell{\multirow{2}{*}{$-$}} & 
    
    \multicolumn{2}{c}{{\multirow{2}{*}{1864}}} & 
    \makecell{\multirow{2}{*}{1865}} & 
    \makecell{\multirow{2}{*}{$+$}} & 
    \makecell{\multirow{2}{*}{$-$}} & 
    
    \multicolumn{2}{c}{{\multirow{2}{*}{1864}}} & 
    \makecell{\multirow{2}{*}{1865}} & 
    \makecell{Збіль-\\шення} &
    \makecell{Збіль-\\шення} \\

    \cmidrule(rl){14-15}
    & & & & & & & & & & & & & \multicolumn{2}{c}{1865} \\
    \midrule

    Пшениця & \num{276.483} & \num{266.989} & \emptycell{} & \num{9.494} &
      Пшениця цент. & 13,3 & 13,0 & \emptycell{} & 0,3 &
      \num{875.782} & кв. & \num{826.783} & \emptycell{} & \num{48.999} кв. \\

    Овес & \num{1.814.886} & \num{1.745.228} & \emptycell{} & \num{69.658} &
      Овес & 12,1 & 12,3 & 0,2 & \emptycell{} &
      \num{7.826.332} & & \num{7.659.727} & \emptycell{} & \num{166.605} \ditto{кв.}\\

    Ячмінь & \num{172.700} & \num{177.102} & \num{4.402} & \emptycell{} &
      Ячмінь & 15,9 & 14,9 & \emptycell{} & 1,0 &
      \num{761.909} & & \num{732.017} & \emptycell{} & \num{29.892} \ditto{кв.}\\

    \makehangcell{Шот\-лян\-д\-сь\-кий яч\-мінь (Be\-re)} & \num{8.894} & \num{10.091} & \num{1.197} & \emptycell{} &
      Шотл. ячм. & 16,4 & 14,8 & \emptycell{} & 1,6 &
      \num{15.160} & & \num{13.989} & \emptycell{} & \num{1.171} \ditto{кв.}\\

    Жито & \emptycell{} & \emptycell{} & \emptycell{} & \emptycell{} & 
      Жито & 8,5 & 10,4 & 1,9 & \emptycell{} & 
      \num{12.680} & & \num{18.364} & \num{5.684} кв. & \emptycell{} \\

    Картопля & \num{1.039.724} & \num{1.066.260} & \num{26.536} & \emptycell{} &
      Картопля тонн & 4,1 & 3,6 & \emptycell{} & 0,5 &
      \num{4.312.388} & тонн & \num{3.865.990} & \emptycell{} & \num{446.398} \ditto{кв.}\\

    Турнепс & \num{337.355} & \num{334.212} & \emptycell{} & \num{3.143} &
      Турнепс & 10,3 & 9,9 & \emptycell{} & 0,4 &
      \num{3.467.659} & & \num{3.301.683} & \emptycell{} & \num{165.976} \ditto{кв.}\\
    
    Білі буряки & \num{14.073} & \num{14.839} & 316 & \emptycell{} &
      Білі бур. & 10,5 & 13,3 & 2,8 & \emptycell{} &
      \num{147.284} & & \num{191.937} & \num{44.653} \samewidth{кв.}{т.} & \emptycell{} \\
    
    Капуста & \num{31.831} & \num{33.622} & \num{1.801} & \emptycell{} &
      Капуста & 9,3 & 10,4 & 1,1 & \emptycell{} & 
      \num{297.375} & & \num{350.252} & 52.87 \phantom{кв.} & \emptycell{} \\
    
    Льон & \num{301.693} & \num{251.433} & \emptycell{} & \num{50.260} &
      \makehangcell{Льон  (Sto\-nes \\ в~14~ф.)} & 34,2 & 25,2 & \emptycell{} & 9,0 &
      \num{64.506} & & \num{39.561} & \emptycell{} & \num{24.945} \ditto{кв.}\\

    Сіно & \num{1.609.569} & \num{1.678.493} & \num{68.924} & \emptycell{} &
      Сіно тонн & 1,6 & 1,8 & 0,2 & \emptycell{} &
      \num{2.607.153} & & \num{3.068.707} & \num{461.554} \phantom{кв.} & \emptycell{}
  \end{tabularx}

\footnotetext{Дані тексту складено з матеріялів: «Agricultural Statistics, Ireland. General Abstracts,
Dublin», за роки 1860 і дальші і «Agricultural Statistics. Ireland. Tables showing the Estimated
Average Produce etc. Dublin, 1866».
Відомо, що це є офіціяльна статистика, яку щорічно подають парляментові.

Додаток до другого видання. Офіціяльна статистика показує для 1872~\abbr{р.} зменшення площі обробленої
землі на \num{134.915} акрів порівняно з роком 1871. Сталося «збільшення» зеленини — турнепсу, білих
буряків і~\abbr{т. ін.}; «зменшення» площі обробленої землі на \num{16.000} акрів пшениці, на \num{14.000} акрів вівса,
на \num{4.000} акрів ячменю й жита, на \num{66.632} акри картоплі, на \num{34.667} акрів льону й на \num{30.000} акрів менше
під луками, конюшиною, викою, свиріпою. Площа землі, що була під культурою пшениці, протягом
останніх 5 років зменшилася в такій послідовності: 1868~\abbr{р.} — \num{285.000} акрів, 1869~\abbr{р.} — \num{280.000} акрів,
1870~\abbr{р.} — \num{259.000} акрів, 1871~\abbr{р.} — \num{244.000} акрів, 1872~\abbr{р.} — \num{228.000} акрів. Для року 1872 маємо
збільшення заокруглено на \num{2.600} коней, на \num{80.000} штук рогатої худоби, на \num{68.609} овець, і зменшення
числа свиней на \num{236.000}.
}
\end{sidewaystable}
\setlength{\tabcolsep}{\tabcolsepdef}