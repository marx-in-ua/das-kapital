\setlength{\tabcolsep}{3pt}
\begin{sidewaystable}
  \index{i}{0604}  %% посилання на сторінку оригінального видання
  \footnotesize
  \begin{center}
    \captionnew{\emph{Таблиця С}. Збільшення або зменшення площі обробленої землі, кількости продукту на акр і \\ загальної кількости продукту в році 1865 порівняно з роком 1864\footnotemark{}}
  \end{center}

  \begin{tabularx}{\textheight}{XrrrrXrrccr@{~}lrrr}

    \toprule
    \makecell{\multirow{3}{*}{Продукти}} & 
    \multicolumn{2}{c}{\makecell{Кількість акрів\\обробленої землі}} & 
    \multicolumn{2}{c}{\makecell{Збільшення \\ або зменшення\\в 1865 р.}} & 
    \multicolumn{3}{c}{\makecell{Кількість \\ продуктів на акр }} & 
    \multicolumn{2}{c}{\makecell{Збільшення \\ або зменшення\\ в 1865 р.}} & 
    \multicolumn{5}{c}{\makecell{Загальна кількість продукту}}
    \\

    \cmidrule(rl){2-3}
    \cmidrule(rl){4-5}
    \cmidrule(rl){6-8}
    \cmidrule(rl){9-10}
    \cmidrule(rl){11-15}
    & 
    \makecell{\multirow{2}{*}{1864}} & 
    \makecell{\multirow{2}{*}{1865}} & 
    \makecell{\multirow{2}{*}{$+$}} & 
    \makecell{\multirow{2}{*}{$-$}} & 
    
    \multicolumn{2}{c}{{\multirow{2}{*}{1864}}} & 
    \makecell{\multirow{2}{*}{1865}} & 
    \makecell{\multirow{2}{*}{$+$}} & 
    \makecell{\multirow{2}{*}{$-$}} & 
    
    \multicolumn{2}{c}{{\multirow{2}{*}{1864}}} & 
    \makecell{\multirow{2}{*}{1865}} & 
    \makecell{Збіль-\\шення} &
    \makecell{Збіль-\\шення} \\

    \cmidrule(rl){14-15}
    & & & & & & & & & & & & & \multicolumn{2}{c}{1865} \\
    \midrule

    Пшениця & 276.483 & 266.989 & \emptycell{} & 9.494 &
      Пшениця цент. & 13,3 & 13,0 & \emptycell{} & 0,3 &
      875.782 & кв. & 826.783 & \emptycell{} & 48.999 кв. \\

    Овес & 1.814.886 & 1.745.228 & \emptycell{} & 69.658 &
      Овес & 12,1 & 12,3 & 0,2 & \emptycell{} &
      7.826.332 & & 7.659.727 & \emptycell{} & 166.605 \ditto{кв.}\\

    Ячмінь & 172.700 & 177.102 & 4.402 & \emptycell{} &
      Ячмінь & 15,9 & 14,9 & \emptycell{} & 1,0 &
      761.909 & & 732.017 & \emptycell{} & 29.892 \ditto{кв.}\\

    \makehangcell{Шот\-лян\-д\-сь\-кий яч\-мінь (Be\-re)} & 8.894 & 10.091 & 1.197 & \emptycell{} &
      Шотл. ячм. & 16,4 & 14,8 & \emptycell{} & 1,6 &
      15.160 & & 13.989 & \emptycell{} & 1.171 \ditto{кв.}\\

    Жито & \emptycell{} & \emptycell{} & \emptycell{} & \emptycell{} & 
      Жито & 8,5 & 10,4 & 1,9 & \emptycell{} & 
      12.680 & & 18.364 & 5.684 кв. & \emptycell{} \\

    Картопля & 1.039.724 & 1.066.260 & 26.536 & \emptycell{} &
      Картопля тонн & 4,1 & 3,6 & \emptycell{} & 0,5 &
      4.312.388 & тонн & 3.865.990 & \emptycell{} & 446.398 \ditto{кв.}\\

    Турнепс & 337.355 & 334.212 & \emptycell{} & 3.143 &
      Турнепс & 10,3 & 9,9 & \emptycell{} & 0,4 &
      3.467.659 & & 3.301.683 & \emptycell{} & 165.976 \ditto{кв.}\\
    
    Білі буряки & 14.073 & 14.839 & 316 & \emptycell{} &
      Білі бур. & 10,5 & 13,3 & 2,8 & \emptycell{} &
      147.284 & & 191.937 & 44.653 \samewidth{кв.}{т.} & \emptycell{} \\
    
    Капуста & 31.831 & 33.622 & 1.801 & \emptycell{} &
      Капуста & 9,3 & 10,4 & 1,1 & \emptycell{} & 
      297.375 & & 350.252 & 52.87 \phantom{кв.} & \emptycell{} \\
    
    Льон & 301.693 & 251.433 & \emptycell{} & 50.260 &
      \makehangcell{Льон  (Sto\-nes \\ в~14~ф.)} & 34,2 & 25,2 & \emptycell{} & 9,0 &
      64.506 & & 39.561 & \emptycell{} & 24.945 \ditto{кв.}\\

    Сіно & 1.609.569 & 1.678.493 & 68.924 & \emptycell{} &
      Сіно тонн & 1,6 & 1,8 & 0,2 & \emptycell{} &
      2.607.153 & & 3.068.707 & 461.554 \phantom{кв.} & \emptycell{}
  \end{tabularx}

\footnotetext{Дані тексту складено з матеріялів: «Agricultural Statistics, Ireland. General Abstracts,
Dublin», за роки 1860 і дальші і «Agricultural Statistics. Ireland. Tables showing the Estimated
Average Produce etc. Dublin, 1866».
Відомо, що це є офіціяльна статистика, яку щорічно подають парляментові.

Додаток до другого видання. Офіціяльна статистика показує для 1872 р. зменшення площі обробленої
землі на 134.915 акрів порівняно з роком 1871. Сталося «збільшення» зеленини — турнепсу, білих
буряків і т. ін.; «зменшення» площі обробленої землі на 16.000 акрів пшениці, на 14.000 акрів вівса,
на 4.000 акрів ячменю й жита, на 66.632 акри картоплі, на 34.667 акрів льону й на 30.000 акрів менше
під луками, конюшиною, викою, свиріпою. Площа землі, що була під культурою пшениці, протягом
останніх 5 років зменшилася в такій послідовності: 1868 р. — 285.000 акрів, 1869 р. — 280.000 акрів,
1870 р. — 259.000 акрів, 1871 р. — 244.000 акрів, 1872 р. — 228.000 акрів. Для року 1872 маємо
збільшення заокруглено на 2.600 коней, на 80.000 штук рогатої худоби, на 68.609 овець, і зменшення
числа свиней на 236.000.
}
\end{sidewaystable}
\setlength{\tabcolsep}{\tabcolsepdef}