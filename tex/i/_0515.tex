\parcont{}  %% абзац починається на попередній сторінці
\index{i}{0515}  %% посилання на сторінку оригінального видання
й сировинні матеріяли на її засоби праці, то фабричній промисловості
іде на користь і та додаткова кількість продуктів, яку
добувальна промисловість і рільництво створили без додаткової
витрати капіталу.

Загальний результат: захоплюючи обидва первісні джерела
багатства, робочу силу й землю, капітал набуває такої сили поширюватись,
що дозволяє йому поширювати елементи своєї акумуляції
поза межі, визначувані, як здається, його власною величиною,
тобто вартістю і масою вже випродукованих засобів продукції,
що в них він існує.

Другий важливий фактор акумуляції капіталу — це ступінь
продуктивности суспільної праці.

\looseness=-1
Зі зростом продуктивної сили праці зростає й та маса продуктів,
що в ній виражається певна вартість, отже, і додаткова вартість
певної величини. Якщо норма додаткової вартости не змінюється
або навіть спадає, але тільки повільніш, ніж зростає
продуктивна сила праці, то маса додаткового продукту зростає.
Тому за незмінного поділу додаткового продукту на дохід і додатковий
капітал споживання капіталіста може зростати, не
зменшуючи фонду акумуляції. Відносна величина фонду акумуляції
може навіть зростати коштом фонду споживання, тимчасом
як здешевлення товарів дає в розпорядження капіталістові
стільки ж або й більше засобів споживання. Але із зростом продуктивности
праці йде пліч-о-пліч, як ми бачили, здешевлення
робітника, отже, зростання норми додаткової вартости, навіть
коли реальна заробітна плата зростає. Ця остання ніколи не
зростає пропорційно до продуктивности праці. Отже, та сама
змінна капітальна вартість пускає в рух більше робочої сили,
а тому й більше праці. Та сама стала капітальна вартість виражається
в більшій кількості засобів продукції, тобто в більшій
кількості засобів праці, матеріялу праці й допоміжних матеріялів,
отже, дає більше так продуктотворчих, як і вартостетворчих
елементів, або елементів, що вбирають у себе працю. Тому за
незмінної вартости додаткового капіталу й навіть тоді, коли
вона меншає, відбувається прискорена акумуляція. Не тільки
речево поширюється маштаб репродукції, але продукція додаткової
вартости зростає швидше, ніж вартість додаткового капіталу.

\looseness=-1
Розвиток продуктивної сили праці впливає також і на первісний
капітал, або на капітал, що перебуває вже в процесі
продукції. Одна частина сталого капіталу, що функціонує, складається
із засобів праці, як от машини тощо, які споживається,
а тому й репродукується або замінюється на нові екземпляри
того самого роду лише протягом довших періодів. Але щороку
частина цих засобів праці відмирає або досягає кінцевої мети
своєї продуктивної функції. Тому ця частина щороку перебуває
в стадії своєї періодичної репродукції або своєї заміни на нові
екземпляри того самого роду. Якщо продуктивна сила праці
розвивається у сфері народження цих засобів праці, — а вона
розвивається постійно разом із безупинним розвитком науки й
\parbreak{}  %% абзац продовжується на наступній сторінці
