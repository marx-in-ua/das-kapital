\parcont{}  %% абзац починається на попередній сторінці
\index{i}{0631}  %% посилання на сторінку оригінального видання
призначений для полювання, у багатьох випадках дає куди
більш зиску, ніж пасовисько для овець\dots{} Аматор, що шукає
місця для полювання, дає за нього таку ціну, яку тільки дозволяє
йому глибина його кишені. На гірські місцевості звалилося горе
не менш жорстоке від того, що його зазнала Англія через політику
норманських королів. Для оленів і сарн приділяють дедалі
більше простору, тимчасом як людей заганяють у дедалі тісніше
коло\dots{} У народу відбирають одну вільність по одній\dots{} І поневолення
день-у-день зростає. «Очищування» й проганяння
народу власники практикують як твердий принцип, як сільськогосподарську
доконечність, цілком так само, як по диких місцевостях
Америки й Австралії викорінюють дерева та кущі, і ця
операція проходить спокійно, діловито»\footnote{
\emph{Robert Sommers}: «Letters from the Highlands; or the Famine
of 1847», London 1848, p. 12--28 і далі. Ці листи появилися спочатку
в «Times’i». Англійські економісти, звичайно, пояснювали голод серед
ґаелів у 1847 році — перелюдненням. У всякому разі, ґаели, мовляв,
«натискували» на свої засоби існування. — «Clearing of Estates», або,
як воно називалося у Німеччині, Bauernlegen, розвинулось у Німеччині
з особливою силою після тридцятилітньої війни і ще в 1790~\abbr{р.} викликало
селянські повстання в саксонському курфюрстві. Воно панувало особливо
у східній Німеччині. В більшості пруських провінцій тільки Фрідріх II
забезпечив селянам право власности. Здобувши Шльонськ, він примусив
землевласників відбудувати хати, клуні тощо й забезпечити селянські
господарства худобою та знаряддям. Йому потрібні були солдати для
його армії і платники податків для його державної скарбниці. А в тім
як приємно жилося селянам за Фрідріха II з його фінансовою політикою
та системою урядування, цією мішаниною деспотизму, бюрократизму
й февдалізму, можна побачити з ось яких слів його великого прихильника
Мірабо: «Одним із головних багатств рільника північної Німеччини
є льон. Та, на нещастя для роду людського, це тільки знаряддя проти
злиднів, а не шлях до добробуту. Безпосередні податки, панщина й інші
февдальні повинності всякого роду руйнують німецького селянина, який
ще до того платить посередні податки на все, що купує\dots{} у довершення
його руйнації він не сміє продавати своїх продуктів, де і як захоче; він
не сміє купувати потрібні йому продукти в тих купців, що продавали б
їх йому за найдешевшу ціну. Всі ці причини непомітно руйнують його,
і він не був би спроможний платити в строк безпосередніх податків,
коли б не займався прядінням; останнє являє собою для нього підмогу,
бо дає корисне заняття його дружині, дітям, слугам, наймитам і йому
самому. Але яке ж це злиденне життя, навіть з отією підмогою! Влітку
підчас оранки та жнив він працює, як каторжник; він лягає о дев’ятій
годині і встає о другій, щоб тільки упоратися з своєю працею; взимку
він мусів би відживлятися, маючи більше спочинку; але йому не вистачить
збіжжя на хліб і на насіння, коли він продасть харчові продукти,
щоб сплатити податки. Отже, він мусить прясти, щоб заткати цю діру\dots{}
мусить прясти з найбільшою пильністю та запопадливістю. Таким чином
селянин узимку лягає опівночі або о першій годині і встає о п’ятій-шостій
удосвіта; або він лягає о дев’ятій і встає о другій — і так день-у-день
ціле своє життя, за винятком неділь\dots{} Це надмірно довге неспання й ця
надмірна праця виснажують організм людини; ось чому на селі чоловіки
й жінки старіються далеко швидше, ніж по містах». («Le lin fait donc
une des grandes richesses du cultivateur dans le Nord d’Allemagne. Malheureusement
pour l’espèce humaine, ce n’est qu’une ressource contre la
misère, et non un moyen de bien-être. Les impôts directs, les corvées, les
servitudes de tout genre, écrasent le cultivateur allemand, qui paie encore
les impôts indirects dans tout ce qu’il achète\dots{} et pour comble de ruine, il
n’ose pas vendre ses productions où et comme il le veut; il n’ose pas acheter
ce dont il a besoin aux marchands qui pourraient le lui livrer au meilleur
prix. Toutes ces causes le ruinent insensiblement, et il se trouverait hors
d'état de payer les impôts directs à l’échéance sans la filerie: elle lui offre
une ressource, en occupant utilement sa femme, ses enfants, ses servants, ses valets,
et lui même: mais quelle pénible vie, même aidée de secours! En été, il
travaille comme un forçat au labourage et à la récorlte; il se couche’ à
9 heures et se lève à deux, pour suffire aux travaux; en hiver il devrait réparer
ses forces par un plus grand repos; mais il manquera de grains pour le
pain et les semailles, s’il se défait des denrées qu’il faudrait vendre pour payer
les impôts. Il faut done filer pour suppléer à ce vide\dots{} il faut y apporter
la plus grande assiduité. Aussi le paysan se couche-t-il en hiver à
minuit, une heure, et se lève à cinq ou six; ou bien il se couche à neuf, et
se lève à deux, et cela tous les jours de sa vie si ce n’est le dimanche. Cet
excès de veille et de travail usent la nature humaine, et de là vient qu’hommes
et femmes vieillissent beaucoup plutôt dans les campagnes que dans
les villes»). (\emph{Mirabeau}: «De la Monarchie Prussienne», Londres 1788,
vol. III, p. 212, 222).

Додаток до другого видання. У квітні 1866 року, 18 років після
опублікування вищецитованої праці Роберта Сомерса, професор Леон
Леві прочитав у Society of Arts доповідь про перетворення овечих пасовиськ
на мисливські парки, що в ній він змалював проґрес у спустошенні
в шотляндських гірських місцевостях. Він каже, між іншим, таке: «Проганяння
людности й перетворювання орного поля на пасовиська для
овець становили найвигідніший засіб діставати доходи без витрат\dots{} Заміна
овечих пасовиськ мисливськими парками, стала звичайним явищем
по гірських місцевостях. Овець виганяють дикі звірі, як раніше виганяли
людей, щоб очистити місце для овець\dots{} У Форфаршірі можна пройти
від маєтків графа фон Делгуза до маєтків Джон Ґротс, не виходячи
зовсім з мисливських лісів. У багатьох (із цих лісів) уже давно живуть
лисиці, дикі коти, куниці, тхорі, ласки й альпійські зайці, недавно
найшли собі туди шлях кролі, білиці й пацюки. Величезні земельні простори,
які в шотляндській статистиці фігурували як винятково родючі
та обширні луки, нині стоять поза всякою культурою і поліпшеннями і
призначені виключно на мисливську забаву небагатьох осіб, тай то ця
забава триває лише недовгий період року».

Лондонський «Economist» від 2 червня 1866~\abbr{р.} пише «Одна шотляндська
газета між іншими новинами з останнього тижня подає й таку:
«Одну з найкращих овечих фарм у Sutherlandshire, за яку недавно, як
вийшов строк оренди, давали \num{1.200}\pound{ фунтів стерлінґів} річної ренти, перетворено
на deer forest!» Февдальні інстинкти виявляються так само\dots{}
як за тих часів, коли нормандський завойовник\dots{} зруйнував 36 сел, щоб
створити new forest\dots{} Два мільйони акрів, що охоплюють деякі з найродючіших
земель Шотляндії, перетворено на цілковиту пустелю. Природні
трави з Glen Tilt’a вважалось за найпоживнішу пашу у графстві Perth;
мисливський ліс у Ben Aulder давав найкращу траву великій окрузі
Badenoch; частина лісу Blak-Mount була найкращим у Шотляндії пасовиськом
для чорних овець. Про розміри земельної площі, спустошеної
для аматорів полювання, можна собі скласти уявлення з того факту, що
ця площа простором далеко більша, аніж ціле графство Perth. Як багато
джерел продукції втрачає країна в наслідок цього насильного спустошення,
видно з того, що лісова площа Ben Aulder могла б прогодувати
\num{15.000} овець, і що площа цього лісу становить лише \sfrac{1}{30} частину всієї
мисливської площі Шотляндії\dots{} Вся ця мислівська земля є цілком непродуктивна\dots{}
з неї така сама користь, як коли б її затопити у хвилях
Північного моря. Міцна рука законодавства мусіла б покласти край цим
імпровізованим пустелям».
}.

\index{i}{0632}  %% посилання на сторінку оригінального видання
\looseness=1
Грабування церковних маєтків, шахрайське відчуження державних
земель, крадіж громадських земель, узурпаторське і з
нещадним тероризмом проведене перетворення февдальної й
кланової власности на сучасну приватну власність, — такі є
ідилічні методи первісної акумуляції. Вони завоювали поле
для капіталістичного рільництва, прилучили землю до капіталу
й утворили для міської промисловости потрібний приплив вільних,
як птиці, пролетарів.

\index{i}{0633}  %% посилання на сторінку оригінального видання
\subsection{Криваве законодавство проти експропрійованих, починаючи
з~кінця XV століття. Закони для зниження заробітної плати}

Вигнаних у наслідок розпуску февдальних дружин і ґвалтовної,
переводжуваної поштовхами експропріяції земель, цих
вільних, як птиці, пролетарів, мануфактура, що тоді поставала,
ніяк не могла поглинути так само швидко, як вони з’являлися
на світ. З другого боку, ці люди, раптово вибиті з їхньої звичайної
життєвої колії, не могли так само раптом призвичаїтись до
дисципліни нових обставин. Вони масами перетворювалися на
жебраків, розбійників, волоцюг, почасти з власного нахилу,
але здебільшого під примусом обставин. Звідси криваве законодавство
проти волоцюзтва по всіх країнах Західньої Европи
наприкінці XV і протягом усього XVI століття. Батьків теперішньої
робітничої кляси покарано насамперед за те, що їх перетворено
на волоцюг і павперів. Законодавство розглядало
їх як «добровільних» злочинців і виходило з того припущення,
що від їхньої доброї волі залежить і далі працювати серед старих
обставин, які вже не існували.

В Англії це законодавство почалося за Генріха VII.

Генріх VIII, 1530: старі й нездатні до праці жебраки дістають
дозвіл жебракувати. Навпаки, працездатних волоцюг слід карати
батогами й замикати до в’язниць. Їх слід прив’язувати ззаду до
тачки й катувати, доки потече з їхнього тіла кров, а потім узяти
від них присягу, що вони повернуться туди, де народились,
або туди, де перебували останні три роки, і «візьмуться до роботи»
(to put himself to labour). Яка жорстока іронія! Акт 27
Генріха VIII повторює цей закон та загострює його новими додатками:
якщо когобудь удруге зловлять на волоцюзтві, то його
треба знову покарати батогами та відрізати йому піввуха; а
коли кого утретє зловлять на волоцюзтві, то його, як тяжкого
злочинця й ворога громадянства, треба покарати на смерть.

\looseness=1
Едвард VI: статут першого року його королювання, 1547,
приписує віддавати кожного, хто ухиляється від праці, у рабство
тій особі, що донесе на нього як на нероба. Хазяїн повинен годувати
свого раба хлібом і водою, давати йому легкі напої й такі
м’ясні покидьки, які він вважатиме за відповідні. Він має право
батогами й кайданами силувати його до всякої, навіть найогиднішої
праці. Коли раб відлучиться на два тижні, то його слід
засудити на довічне рабство й наложити на його лоб або щоку
тавро «S»; коли ж він утече втретє, то його слід покарати на
смерть як зрадника держави. Хазяїн може його продати, відписати
у спадщину, віддати в найми, як раба, цілком так само,
\parbreak{}  %% абзац продовжується на наступній сторінці
