
\index{i}{0120}  %% посилання на сторінку оригінального видання
\looseness=1
Однак, для того, щоб посідач грошей міг найти на ринку
робочу силу як товар, мусять здійснитись різні передумови\dots{}
Обмін товарів сам по собі не містить у собі жодних інших відносин
залежности, крім тих, що постають з його власної природи.
За цієї передумови робоча сила може з’явитися на ринку як
товар лише тому й остільки, що й оскільки її посідач, особа, що
її робочою силою вона є, подає її як товар на ринок, або продає.
Щоб її посідач міг продавати її як товар, він мусить мати змогу
порядкувати нею, отже, бути вільним власником своєї здатности
до праці, своєї особи\footnote{
В реальних енциклопедіях клясичної старовини можна прочитати
таку нісенітницю, ніби в античному світі капітал був цілком розвинений,
«бракувало лише вільного робітника і кредитових установ». Пан Момзен
у своїй «Римській історії» також робить подібні quid pro quo одне
по одному.
}. Власник робочої сили й посідач грошей
зустрічаються на ринку й увіходять між собою у відношення
як рівноправні посідачі товарів, які різняться лише тим, що
один є покупець, а другий — продавець, отже, обидва юридично
є рівні особи. Щоб це відношення й далі тривало, треба, щоб
власник робочої сили завжди продавав її лише на певний час,
бо, коли б він продав її геть чисто раз назавжди, то він продав би
себе самого й перетворився б із вільної людини на раба, з посідача
товару на товар. Як особа він мусить постійно ставитися до
своєї робочої сили як до своєї власности, і тим то як до свого
власного товару, а це він може робити лише остільки, оскільки
він завжди віддає свою робочу силу до розпорядження покупця
лише тимчасово, на певний період часу, передає лише на вжиток,
отже, відчужуючи її, не зрікається свого права власности
на неї\footnote{
Тому різні законодавства встановлюють певний максимальний реченець
для робочого контракту. У народів, що в них праця вільна, законодавство
реґулює умови відмовлення від контракту. По різних країнах, а особливо
в Мехіко (перед американською громадянською війною також і на територіях,
відірваних від Мехіко, а по суті і в наддунайських провінціях до
перевороту Кузи), рабство ховається під формою peonage’a. Через аванси,
що їх треба сплачувати працею і що переходять від покоління до покоління,
не лише поодинокий робітник, але й родина його фактично стають
власністю інших осіб і їхніх родин. Хуарец скасував peonage. Так званий
цар Максиміліян знов увів його в життя декретом, що його у вашинґтонській
Палаті представників слушно плямували як декрет, що відновлює
рабство в Мехіко. «Свої особливі фізичні й інтелектуальні здібності та
свою дієздатність я можу\dots{} відчужувати іншій особі для користування
на обмежений реченець, бо в наслідок цього обмеження вони набувають
зовнішнього відношення до моєї цілости й загальности. Через відчуження
цілого мого часу, який конкретизується через працю, і цілої моєї продукції
я зробив би власністю іншої особи саму субстанцію її, тобто мою загальну
діяльність і дійсність, мою особу». (\emph{Hegel}: «Philosophie des Rechts», Berlin
1840, S. 104, § 67).
}.

Друга посутня умова, потрібна для того, щоб посідач грошей
міг найти на ринку робочу силу як товар, є та, що посідач робочої
сили, замість мати змогу продавати товари, в яких упредметнилась
його праця, мусить, навпаки, подавати на ринок як товар
саму свою робочу силу, яка існує лише в його живому організмі.
