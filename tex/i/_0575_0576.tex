\parcont{}  %% абзац починається на попередній сторінці
\index{i}{0575}  %% посилання на сторінку оригінального видання
запрошує промислову колонію працювати у його маєтках, а
потім він як власник поверхні землі не дає змоги зібраним ним
робітникам найти відповідне помешкання, доконечне для їхнього
влиття. Орендар копалень [капіталістичний експлуататор] не
має жодного грошового інтересу опиратися цьому двозначному
торгові, бо він добре знає, що коли вимоги землевласника непомірні,
то наслідки цього спадуть не на нього, що робітники, на
яйих вони спадуть, надто несвідомі, щоб знати свої права на здоров’я,
і що ані якнайнепристойніші житла, ані якнайгниліша
вода ніколи не будуть приводом до страйку».135

d) Вплив криз на найкраще оплачувану
частину робітничої кляси

Раніш ніж перейти до власне рільничих робітників, я хочу
ще показати на одному прикладі, як впливають кризи навіть на
найкраще оплачувану частину робітничої кляси, на її аристократію.
Пригадаймо собі ось що: 1857 рік приніс одну з тих
великих криз, що ними кожного разу завершується промисловий
цикл. Найближча криза припала на 1866 р. Ця криза, що у власне
фабричних округах була антиципована в наслідок бавовняного
голоду, який загнав чимало капіталу із звичайної сфери його
вміщення у великі центри грошового ринку, цим разом набрала
переважно фінансового характеру. Сиґналом її вибуху в травні
1866 р. було банкрутство одного з велетенських лондонських
банків, слідом за яким наступив крах безлічі спекуляційних
фінансових товариств. Однією з великих лондонських галузей
промисловости, що їх вразила катастрофа, була будова кораблів
із заліза. Маґнати цієї галузі промисловости за часів спекуляції
не лише перепродукували понад усяку міру, але, крім
того, ще й перейняли на себе контракти на величезні замовлення,
сподіваючись, що джерело кредиту й далі залишатиметься невичерпним.
Тепер же постала страшенна реакція, яка і в інших
галузях лондонської промисловости\footnote{
Там же, стор. 16.
} триває до цього часу,
кінець березня 1867 р. Для характеристики становища робітників
наведімо таке місце з докладного звіту одного кореспон-

137 «Масове голодування лондонських бідняків!(«Wholesale starvation
of the London Poor!»)\dots{} Останніми днями мури лондонських будинків
позаклеювано величезними плякатами з такою дивовижною об’явою:
«Ситі бики, зголоднілі люди! Ситі бики покинули свої кришталеві
палаци, щоб відгодувати багатіїв у їхніх розкішних світлицях, тимчасом
як зголоднілі люди гинуть і вмирають у своїх злиденних норах». Плякати
з цим зловішим написом постійно поновлюються. Ледве позривають і
позаліплюють одну партію, як одразу на тому самому або іншому прилюдному
місці знов появляється нова\dots{} Це нагадує ті передвіщання, що підготовляли
французький народ до подій 1789 р. Нині, коли англійські
робітники з своїми жінками й дітьми вмирають від холоду і з голоду,
мільйони англійських грошей, продукт англійської праці, вкладають
у російські, еспанські. італійські й інші закордонні позики». («Reynolds'
Newspaper, 20 Januar 1867»).
\index{i}{0576}  %% посилання на сторінку оригінального видання
цента «Morning Star», що з початком 1867 р. відвідав головні
центри злиднів. «У східній частині Лондону, в округах Poplar
Millwall, Greenwich, Deptford, Limehouse і Canning Town щонайменше
15.000 робітників разом із своїми родинами живуть у
якнайтяжчих злиднях, поміж ними понад 3.000 навчених механіків.
їхні резервні фонди вичерпано в наслідок шести-восьмимісячного
безробіття\dots{} Багато зусиль коштувало мені протиснутись
до воріт робітного дому (в Роріаг’і), бо його облягла зголодніла
юрба. Вона чекала на хлібні картки, але час роздавання
їх ще не настав. Подвір’я являє собою великий квардрат із
піддашшям навколо мурів. Кучугури снігу густо вкривали
кам’яний брук на середині подвір’я. Тут деякі невеличкі площі
були загороджені івовим тином, наче кошари для овець, де гарної
години працюють чоловіки. В день моїх відвідин кошари так
були позасипувані снігом, що ніхто не міг у них сидіти. Однак
під захистом підашшя чоловіки розбивали брукняк. Кожний з
них сидів на великому бруковому камені і тяжким молотом бив
по обмерзлому ґраніту. доки набивав з нього 5 бушлів. Тоді його
денна робота кінчалась, і він діставав 3 пенси і хлібну картку.
У другій частині подвір’я стояла злиденна дерев’яна хатина.
Відчинивши двері, ми побачили, що вона була повна чоловіків,
які тулилися один до одного, щоб зігрітись. Вони дерли клоччя
з корабельної линви і сперечалися між собою, хто з них при мінімумі
харчів може найдовше працювати, бо витривалість була
тут point d’honneur.* В цьому одному лише робітному домі діставало
допомогу 7.000 осіб, серед них сотні таких, що 6 або 8 місяців
тому заробляли вправною працею найвищу в цій країні
заробітну плату. Число їх було б удвоє більше, коли б не те,
що багато з них, навіть вичерпавши всі свої грошові заощадження,
все-таки не наважуються вдаватися по допомогу до парафії,
доки в них іще лишається що-будь заставляти\dots{} Покинувши робітний
дім, я пішов вулицями здебільша з одноповерховими будинками,
що їх так багато в Роріаг’і. Моїм поводирем був член
комітету безробітних. Перший дім, куди ми зайшли, був дім
залізничника, що 27 тижнів був уже без роботи. Я найшов його
з цілою його родиною в задній кімнатці. В кімнатці були ще
деякі меблі, її ще опалювали. Це конче треба було, щоб захистити
голі ноги маленьких дітей від холоду, бо день був страшенно
зимний. На тарілці проти вогню лежало клоччя, і його жінка
й діти дерли у відплату за хліб з робітного дому. Чоловік працював
в одному з вищеописаних подвір’їв за хлібну картку й
З пенси на день. Тепер він прийшов додому обідати, дуже зголоднілий,
як сказав він нам з гіркою посмішкою, а його обід
складався з кількох шматків хліба з смальцем і склянки чаю
без молока\dots{} Дальші двері, куди ми постукали, відчинила жінка
середнього віку, яка, не сказавши й слова, провела нас у малюсіньку
задню кімнатку, де мовчки сиділа ціла її родина, втупивши

— справою чести. \emph{Ред.}
\parbreak{}  %% абзац продовжується на наступній сторінці
