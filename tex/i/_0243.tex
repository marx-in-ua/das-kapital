\parcont{}  %% абзац починається на попередній сторінці
\index{i}{0243}  %% посилання на сторінку оригінального видання
ступеня експлуатації робочої сили. Цей цілком очевидний другий
закон є важливий для пояснення багатьох явищ, які випливають
із тенденції капіталу, що її ми маємо розвинути пізніш,
а саме тенденції якомога більше скорочувати число робітників,
що їх він вживає, або його змінну складову частину, перетворену
на робочу силу, — всупереч іншій його тенденції, а саме продукувати
якомога більшу масу додаткової вартости. Навпаки, коли
маса вживаних робочих сил або величина змінного капіталу
зростає, алеж непропорційно до зменшення норми додаткової
вартости, то маса продукованої додаткової вартости меншає\footnote*{
У французькому виданні цей абзац подано так: «Цей абсолютно
ясний закон є важливий для розуміння складних явищ. Ми вже знаємо,
що капітал намагається продукувати якомога більше додаткової вартости;
ми побачимо пізніш, що він разом із цим намагається скоротити до
мінімуму, порівняно з розмірами підприємства, свою змінну частину,
або кількість робітників, що їх він експлуатує. Ці тенденції стають одна
одній суперечними, скоро лише зменшення одного з факторів, що визначають
суму додаткової вартости, вже не може бути компенсоване збільшенням
другого». («Le Capital etc.», v. I, ch. XI, p. 132). \emph{Ред.}
}.
\enablefootnotebreak{}

Третій закон випливає з визначення маси продукованої додаткової
вартости двома факторами: нормою додаткової вартости й
величиною авансованого змінного капіталу. Коли дано норму
додаткової вартости, або ступінь експлуатації робочої сили,
і вартість робочої сили, або величину доконечного робочого
часу, то само собою зрозуміло, що чим більший змінний капітал,
тим більша маса продукованої вартости й додаткової вартости.
Коли дано межі робочого дня, а також межі його доконечної
складової частини, то маса вартости й додаткової вартости,
що її продукує поодинокий капіталіст, очевидно, залежить
виключно від тієї маси праці, яку він пускає в рух. Але
маса ця, за даних припущень, залежить від маси робочої сили,
або від числа робітників, яких він експлуатує; а це число, з
свого боку, визначається величиною авансованого ним змінного
капіталу. Отже, за даної норми додаткової вартости й даної вартости
робочої сили маси продукованої додаткової вартости є
просто пропорційні до величин авансованих змінних капіталів.
Та тепер уже відомо, що капіталіст ділить свій капітал на
дві частини. Одну частину він вкладає в засоби продукції. Це —
стала частина його капіталу. Другу частину він перетворює на
живу робочу силу. Ця частина становить його змінний капітал.
На базі того самого способу продукції в різних галузях продукції
відбувається різний поділ капіталу на сталу та змінну складові
частини. В тій самій галузі продукції це відношення змінюється
разом із зміною технічної основи й суспільних комбінацій процесу
продукції. Але хоч як розпадатиметься даний капітал на сталу
й змінну складові частини, чи остання відноситиметься до першої
як $1 : 2$, $1 : 10$, або $1 : х$, — це не порушує щойно встановленого
закону, бо, згідно з попередньою аналізою, вартість сталого капіталу
хоч і з’являється знов у вартості продукту, але не увіходить
у новоутворену вартість. Щоб уживати \num{1.000} прядунів, потрібно,
\parbreak{}  %% абзац продовжується на наступній сторінці
