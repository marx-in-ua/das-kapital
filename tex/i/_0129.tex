
\index{i}{0129}  %% посилання на сторінку оригінального видання
\chapter{Продукція абсолютної додаткової вартости}

\sectionextended[%
Процес праці і процес зростання вартости]{%
Процес праці і процес зростання вартости\footnotemarkZ{}}{%
\subsection{Процес праці}}

\disablefootnotebreak{}
Споживання\footnotetextZ{У французькому виданні цьому розділові дано заголовок: «Production de valeurs d’usage et production de la plus-value» — «Продукція
споживних вартостей та продукція додаткової вартости», а §1 цього розділу
дано заголовок: «Production de valeurs d’usage» — «Продукція споживних
вартостей». \emph{Ред.}}
робочої сили — це сама праця. Покупець робочої
сили споживає її, примушуючи продавця її працювати. Останній
стає через те робочою силою, що виявляється в дії (actu),
стає робітником, чим раніш він був лише потенціяльно. Щоб
утілити свою працю в товарах, він мусить насамперед утілити
її у споживних вартостях, у речах, що служать для задоволення
тих або інших потреб. Отже, капіталіст примушує робітника
виготовляти якусь споживну вартість, якийсь певний товар.
Продукція споживних вартостей або дібр не змінює своєї загальної
природи від того, що вона відбувається для капіталіста та
під його контролем. Отже, процес праці треба розглянути насамперед
незалежно від усякої певної суспільної форми\footnote*{
У французькому виданні це речення подано так: «Отже, спочатку
нам треба розглянути процес корисної праці взагалі, абстрагуючись
від усякої осібної форми, що йому може надати та або інша фаза економічного
розвитку суспільства» («Le Capital etc.», vol. I, ch. VII, p. 76). \emph{Ред.}
}.

\enablefootnotebreak{}
Праця є насамперед процес між людиною й природою, процес,
що в ньому людина своєю власною діяльністю упосереднює, реґулює
й контролює обмін речовин між собою й природою. Речовині
природи людина сама протистоїть як сила природи. Щоб
присвоїти собі природну речовину у формі, придатній для її власного
життя, вона пускає в рух належні до її тіла природні сили:
рамена й ноги, голову й руки. Впливаючи цим рухом на зовнішню
природу і змінюючи її, вона змінює одночасно і свою власну природу.
Вона розвиває дрімотні в її власній природі здібності (Potenzen)
і гру її сил підбиває під свою власну владу. Ми не будемо
тут розглядати первісних твариноподібних інстинктивних форм
праці. Порівняно з тим станом, коли робітник виступає на ринку
\parbreak{}  %% абзац продовжується на наступній сторінці
