Відділ другий

Перетворення грошей на капітал

Розділ четвертий

Перетворення грошей на капітал

1. Загальна формула капіталу

Товарова циркуляція є вихідний пункт капіталу. Товарова
продукція й розвинена товарова циркуляція, торговля, становлять
історичні передумови, за яких він постає. Світова торговля
й світовий ринок починають у XVI віці сучасну історію
капіталу.

Коли залишимо осторонь речовий зміст циркуляції товарів,
обмін різних споживних вартостей, і розглядатимемо лише ті
економічні форми, які створює цей процес, то ми побачимо,
що його останній продукт є гроші. Цей останній продукт товарової
циркуляції є перша форма виявлення капіталу.

Історично капітал спочатку скрізь протистоїть земельній
власності у формі грошей, як грошове майно, купецький капітал
і лихварський капітал.1 Однак не потрібно навіть звертатись до
минулости, до історії постання капіталу, щоб пізнати, що гроші
є перша форма виявлення капіталу. Та сама історія щодня відбувається
перед нашими очима. Кожний новий капітал у першій
своїй інстанції виходить на сцену, тобто на ринок, на товаровий
ринок, на робочий ринок або грошовий ринок, завжди у формі
грошей, — грошей, що через певні процеси повинні перетворитися
на капітал.

Гроші як гроші і гроші як капітал відрізняються насамперед
лише своїми різними формами циркуляції.

Безпосередня форма товарової продукції є Т — Г — Т, перетворення
товару на гроші і зворотне перетворення грошей на
товар, продаж задля купівлі. Але поруч цієї форми ми находимо
другу, специфічно відмінну форму, форму Г — Т — Г, перетво-

1    Протилежність поміж владою земельної власности, що спирається
на відносини особистого поневолення й панування, і безособовою владою
грошей ясно висловлено у двох французьких приказках: «Nulle terre
sans seigneur». «L’argent n’a pas de maître».**

* «Нема землі без господаря». «Гроші не мають господаря». Ред.
