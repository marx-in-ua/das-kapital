
\section{Машини та велика промисловість}
\vspace{\bigskipamount}
\subsection{Розвиток машин}

Джон Стюарт Мілл у своїх «Основах політичної економії»
каже: «Навряд чи всі зроблені досі механічні винаходи полегшили
щоденну працю хоча б однієї людської істоти»\footnote{
«It is questionable, if all the mechanical inventions yet made have
lightened the day’s toil of any human being». Мілл повинен був би сказати:
«Хоча б однієї людської істоти, що не живе з чужої праці» («of any human
being not fed by other people’s labour»), бо машини безперечно дуже збільшили
число вельможних нероб.
}. Та це зовсім і не є мета капіталістично вживаних машин. Подібно до
всіх інших способів розвитку продуктивної сили праці, машини
повинні здешевлювати товари та скорочувати ту частину робочого
дня, якої робітник потребує для самого себе, щоб таким
чином здовжувати другу частину робочого дня, яку він задурно
дає капіталістові. Машини — це засіб продукувати додаткову
вартість.

У мануфактурі за вихідний пункт перевороту в способі продукції
є робоча сила, у великій промисловості — засіб праці.
Отже, насамперед треба дослідити, яким чином засіб праці із
знаряддя перетворюється на машину, або чим машина відрізняється
від ремісничого інструменту. Тут мова лише про великі,
загальні характеристичні риси, бо для епох історії суспільства,
так само як і для епох історії землі, не існує абстрактно точних
відмежувальних ліній.

Математики та механіки — і це повторюють іноді й англійські
економісти — називають знаряддя простою машиною, а машину
складним знаряддям. Вони не вбачають ніякої посутньої
ріжниці між ними, і навіть називають прості механічні сили,
як от підойму, похилу площу, шруб, клин і~\abbr{т. д.}, машинами\footnote{
Див., напр., \emph{Hutton}: «Course of Mathematics».
}. Дійсно, кожна машина складається з таких простих сил,
хоч як вони змінені та скомбіновані. Однак, з економічного погляду
це пояснення нічого не варте, бо йому бракує історичного
елементу. З другого боку, ріжниці між знаряддям і машиною
шукають у тім, що при знарядді за рушійну силу є людина, а
при машині — сила природи, відмінна від людської, як от сила
тварини, води, вітру й~\abbr{т. д.}\footnote{
«З цього погляду можна також визначити точну межу між знаряддям
та машиною: заступ, молоток, долото\dots{} системи підойм та шрубів\dots{}
для яких, хоч би вони й як майстерно були зроблені, рушійною силою є
людина\dots{} все це підходить під поняття знаряддя; тимчасом як плуг із
рушійною силою тварини, вітряки й інші млини\dots{} слід залічити до машин».
(\emph{Wilhelm Schulz}: «Die Bewegung der Produktion», Zürich 1843, S. 38)
Твір, з деякого погляду вартий похвали.
} З цього виходило б, що запряжений
волами плуг, який належить до найрізніших епох продукції,
\parbreak{}  %% абзац продовжується на наступній сторінці
