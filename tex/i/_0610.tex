\parcont{}  %% абзац починається на попередній сторінці
\index{i}{0610}  %% посилання на сторінку оригінального видання
з інспекторів, — ганьба для релігії й цивілізації цієї країни»\footnoteA{
Там же, стор. 12.
}.
Щоб зробити стерпнішим життя сільських наймитів в їхніх льохах,
у них систематично відбирають клаптики землі, що від непам’ятних
часів належали до тих мешкань. «Усвідомлення цього
роду немилости, що її вони зазнають від лендлордів та їхніх управителів,
викликало в сільських поденників відповідне почуття
антагонізму й ненависти до тих, що поводяться з ними як з безправною
расою»\footnoteA{Там же, стор. 12.}.

\looseness=-1
Першим актом революції в рільництві було те, що в якнайбільшому
маштабі й немов би на даний згори сиґнал поруйновано
хатини, розташовані на місцях роботи. Таким чином багато робітників
мусіло шукати притулку по селах і містах. Там їх, як
мотлох який, поскидано на горища, у вертепи, в льохи й закамарки
найгірших кварталів. Так тисячі ірляндських родин, що,
навіть за свідченнями англійців, пройнятих національними забобонами,
відзначалися незвичайною прихильністю до родинного
вогнища, безтурботною веселістю й чистотою родинних звичаїв,
опинилися раптом у розсадниках пороку. Чоловіки мусять тепер
шукати роботи в сусідніх фармерів, які їх наймають лише поденно,
отже, на умовах найнепевнішої форми заробітної плати;
при цьому «їм тепер доводиться далеко ходити до фарми й назад,
часто мокнути до рубчика й зазнавати інших негод, що часто
ведуть до занепаду сил, хороб і до злиднів»\footnoteA{
Там же, стор. 25.
}.

«Міста рік-у-рік мусіли приймати всіх тих робітників, що
їх вважалося за надмір у сільських округах»\footnoteA{
Там же, стор. 27.
}, і після цього
ще дивуються, «що по містах та містечках є надмір робітників,
а по селах їх не вистачає!»\footnoteA{
Там же, стор. 26.
} В дійсності, цю недостачу відчувається
лише «за часів нагальних рільничих робіт, на весні
і восени, тимчасом як в інші пори року багато робітників лишаються
без роботи»;\footnoteA{
Там же, стор. 1.
} «після жнив, від жовтня до весни, ледве
чи є для них якась робота»\footnoteA{
Там же, стор. 25.
}, і навіть тоді, коли в них є робота,
«вони часто втрачають цілі дні й мусять зносити всілякі
перерви в роботі»\footnoteA{Там же, стор. 25.}.

Ці наслідки революції в рільництві, тобто наслідки перетворення
орної землі на пасовиська, застосування машин, якнайсуворішого
заощадження праці й~\abbr{т. ін.}, ще більше загострюються
тими зразковими лендлордами, які, замість споживати свої ренти
за кордоном, ласкаві жити в Ірляндії на своїх маєтках. Для того,
щоб закон попиту й подання лишався цілком непорушним, ці
пани витягають «тепер майже всю потрібну для них працю із
своїх дрібних фармерів, які таким чином примушені працювати
\parbreak{}  %% абзац продовжується на наступній сторінці
