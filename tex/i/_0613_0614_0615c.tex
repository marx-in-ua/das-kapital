\parcont{}  %% абзац починається на попередній сторінці
\index{i}{0613}  %% посилання на сторінку оригінального видання
(unsound). Лорд тримається фактів. А факт є той, що в міру того,
як меншає кількість ірляндської людности, ірляндські ренти
зростають, що збезлюднений «добродійне» для земельного власника,
отже, і для землі, отже, і для народу, що є лише приналежність
землі. Отож він заявляє, що Ірляндія все ще перелюднена,
і що потік еміґрації пливе все ще занадто поволі. Щоб бути цілком
щасливою, Ірляндія мусить позбутися принаймні ще \sfrac{1}{3} мільйона
робітників. Не думайте собі, що цей, до того всього ще й
поетичний, лорд є лікар із школи Sangrado, який завжди, коли
він не помічав у свого недужого поліпшення, приписував йому
кровоспуск, потім знову кровоспуск, поки, нарешті, в недужого
разом з його кров’ю пропадала і його хороба. Лорд Дюфрен
вимагає нового кровоспуску лише в \sfrac{1}{3} мільйона людей
замість майже 2 мільйонів, кровоспуску, без якого дійсно ніяк
неможливо завести тисячолітнього блаженного царства на Еріні.
Докази подати не важко.

\manualpagebreak{}
\begin{center}
  \centering
  \captionnew{Число і розмір фарм в Ірляндії 1864~\abbr{р.}}

  \begin{tabular}{lrr}
  \toprule

  Фарми & Число & Акри \\
  \midrule

  Не більш від 1 акра & \num{48.653} & \num{25.394} \\
  Від \phantom{0}2 до \phantom{0}\phantom{0}5 акрів & \num{82.037} & \num{288.916} \\
  \ditto{Від }\phantom{0}6\ditto{ до }\phantom{0}15\ditto{ акрів} & \num{176.368} & \num{1.836.310} \\
  \ditto{Від }16\ditto{ до }\phantom{0}30\ditto{ акрів} & \num{136.578} & \num{3.051.343} \\
  \ditto{Від }31\ditto{ до }\phantom{0}50\ditto{ акрів} & \num{71.961} & \num{2.906.274}\\
  \ditto{Від }51\ditto{ до }100\ditto{ акрів} & \num{54.247} & \num{3.983.880}\\
  Понад 100 акрів & \num{31.927} & \num{8.227.807}\\
  \midrule
  Загальна площа & & \num{29.319.924}\hang{l}{\footnotemarkA{}} \\

  \end{tabular}
  \withoutLCW{\footnotetextA{Загальна площа включає також торфовища й пустирі.}}
\end{center}

\noindent{}Централізація знищила між 1851 і 1861~\abbr{рр.} переважно фарми
перших трьох категорій — нижче 1 і не вище 15 акрів. Вони
мусять зникнути передусім. Це дає \num{307.058} «зайвих» фармерів,
або \num{1.228.232} особи, коли при низькому пересічному обрахунку
покласти 4 особи на родину. При неймовірному припущенні, що
по закінченні революції в рільництві \sfrac{1}{4} з них знову знайде собі
роботу, все ж лишається \num{921.174} особи, що мусять еміґрувати. Категорії
4, 5 і 6, більші за 15 і не більші за 100 акрів, як це давно
відомо в Англії, занадто дрібні для капіталістичного рільництва,
а для вівчарства це зовсім незначні величини. Отже, при
тому самому припущенні, що й раніш, мусять еміґрувати ще
\num{788.761} особа, разом \num{1.709.532}. А що l’appétit vient en
\index{i}{0614}  %% посилання на сторінку оригінального видання
mangeant\footnote*{
апетит приходить підчас їди. \emph{Ред.}
}, то великі землевласники незабаром відкриють, що Ірляндія
із 3\sfrac{1}{2} мільйонами людности все ще бідна країна, а бідна,
тому що перелюднена, отже, збезлюднення її мусить піти ще
значно далі, щоб вона могла виконати своє справжнє призначення
бути за пасовисько для овець і рогатої худоби Англії\withoutLCW{\footnoteA{
Як окремі земельні власники й англійське законодавство пляномірно
використовували голод і викликані ним обставини, щоб силоміць
провести революцію в рільництві і звести людність Ірляндії до кількости,
вигідної для лендлордів, це я покажу докладніше у третій книзі
цього твору, у відділі про земельну власність. Там я повернуся й до становища
дрібних фармерів і сільських робітників. Тут я подам лише одну
цитату. Нассау В.~Сеніор у своєму посмертному творі «Journals, Conversations
and Essays relating to Ireland». 2 volumes. London 1868,
vol. II, p. 282 каже, між іншим, ось що: «Влучно зауважив д-р Ґ., що в
нас є закон про бідних, і що він є могутнє знаряддя, щоб забезпечити
перемогу лендлордам; друге знаряддя — еміґрація. Жоден друг Ірляндії
не побажає, щоб війна (між лендлордами й дрібними кельтськими
фармерами) тривала далі, — ще менш, щоб вона скінчилась перемогою
фармерів\dots{} Що швидше вона (ця війна) скінчиться, що швидше Ірляндія
перетвориться на пасовиська (grazing country) з порівняно нечисленною
людністю, якої треба для пасовиськ, то краще для всіх кляс». — Англійські
хлібні закони 1815~\abbr{р.} забезпечували Ірляндії монополію вільно довозити
хліб у Великобрітанію. Таким чином вони штучно сприяли рільництву.
У 1846~\abbr{р.} разом із скасуванням хлібних законів одразу знищено
і цю монополію. Не кажучи вже про всі інші обставини, лише цієї
події було досить, щоб надати потужного поштовху перетворенню ірляндської
орної землі на пасовиська, концентрації фарм і вигнанню дрібних
селян. Після того, як протягом 1815--1846~\abbr{рр.} уславляли родючість
ірляндського ґрунту і вселюдно оголосили, що з самої природи
його призначено виключно на культивування пшениці, тепер англійські
аґрономи, економісти, політики раптом зробили відкриття, що він придатний
лише для культивування кормових трав! Пан Леонс де Лявернь
поспішив повторити це по той бік каналу. Треба бути такою «серйозною»
людиною, як пан Лявернь, щоб йняти віри таким наївним теревеням.
}}.

Ця корисна метода, які і все гарне на цьому світі, має свій
поганий бік. Рівнобіжно з акумуляцією земельної ренти в Ірляндії
ірляндці акумулюються в Америці. Ірляндець, що його виганяють
вівці та бики, з’являється по той бік океану, як феній.
І проти старої владарки морів повстає чимраз грізніш велетенська
молода республіка.

\settowidth{\versewidth}{Scelusque fraternae necis.}
\begin{verse}[\versewidth]
Acerba fata Romanos agunt \\
Scelusque fraternae necis\footnote*{
Жене римлян сувора доля і злочин братовбивства. \emph{Ред.}
}.
\end{verse}

\section{Так звана первісна акумуляція}

\subsection{Таємниця первісної акумуляції}

Ми бачили, як гроші перетворюються на капітал, як за допомогою
капіталу утворюється додаткова вартість, а з додаткової
вартости — додатковий капітал. Але акумуляція капіталу має за
передумову додаткову вартість, додаткова вартість — капіталістичну
\index{i}{0615}
продукцію, а ця остання — наявність великих мас капіталу
й робочої сили в руках товаропродуцентів. Таким чином,
увесь цей рух, здається, обертається у зачарованому колі, з
якого ми не можемо вийти інакше, як припустивши, що капіталістичній
акумуляції передувала «первісна» акумуляція («previous
accumulation» у Адама Сміта), акумуляція, що є не результатом
капіталістичного способу продукції, а його вихідним пунктом.

Ця первісна акумуляція відіграє в політичній економії приблизно
ту саму роль, що й первісний гріх у теології. Адам покуштував
яблука, і таким чином гріх увійшов у рід людський.
Походження цієї акумуляції пояснюють, оповідаючи про нього
анекдот давноминулих часів. За дуже давніх часів були, з
одного боку, працьовиті, розумні й насамперед ощадливі обранці,
а з другого боку — ледарі, голодранці, які прогулювали все,
що мали, і навіть ще більше. Правда, теологічна леґенда про
гріхопадіння розповідає нам, як засуждено людину їсти хліб
у поті чола свого; навпаки, історія економічного гріхопадіння
викриває, як це сталося, що є люди, які цього зовсім не потребують.
Так сталось, що перші нагромадили багатство, а в
других кінець-кінцем нічого було продавати, крім власної
шкури. І з цього гріхопадіння починаються злидні великої маси
людей, що їм все ще, не зважаючи на всю їхню працю, нічого
продати, крім себе самих, і багатство небагатьох, яке невпинно
зростає, хоч вони дуже давно перестали працювати. Такі безглузді
дитячі байки пережовує, наприклад, пан Тьєр, який,
щоб захистити propriété\footnote*{
власність. \emph{Ред.}
}, з державно-урочистою серйозністю
підносить ці байки французам, що колись були такі дотепні.
Скоро тільки справа торкається питання про власність, то стає
священним обов’язком кожного дотримуватись поглядів дитячого
букваря як єдино правильних для всякого віку й усіх ступенів
розвитку\footnote*{
У французькому виданні Маркс додає до цього таку примітку:
«Ґете, роздратований цими дурницями, висміює їх у такому діалозі:

Вчитель: Скажи мені, звідки взялося багатство твого батька?

Дитина: Від діда.

Вчитель: А звідки воно взялося в діда?

Дитина: Від прадіда.

Вчитель: А в прадіда?

Дитина: Він загарбав його». \emph{Ред.}
}. Як відомо, в дійсній історії велику ролю відіграють
завоювання, поневолення, розбій, коротко кажучи, насильство.
Але в лагідній політичній економії з давніх часів панувала
ідилія. Право й «праця» з давніх часів були єдиним засобом
збагачення, звичайно, щоразу за винятком «поточного року».
А в дійсності методи первісної акумуляції — це все, що хочете,
тільки не ідилія.

Гроші й товари, так само, як засоби продукції й засоби існування,
самі собою не є капіталом. Їх треба перетворити на капітал.
Але саме це перетворення може відбуватися лише за певних
обставин, які сходять ось на що: два дуже різні сорти
\parbreak{}
