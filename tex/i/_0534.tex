\parcont{}  %% абзац починається на попередній сторінці
\index{i}{0534}  %% посилання на сторінку оригінального видання
способами, що ними можна розсунути ті межі, які пудлінґування
ще ставить дальшому зростанню матеріяльних засобів продукції
порівняно з уживаною працею. Така є історія всіх відкрить
і винаходів, що постають слідом за акумуляцією, як ми це
вже показали, змальовуючи розвиток сучасної продукції від її
початку аж до наших часів. (Дивись четвертий відділ сього
твору).

Отже, в міру проґресу акумуляції відбувається не лише
кількісне й одночасне зростання різних реальних елементів
капіталу: розвиток продуктивних сил суспільної праці, зумовлюваний
цим проґресом, виявляється також і в якісних змінах,
у поступінних змінах технічного складу капіталу, що його об’єктивний
фактор проґресивно зростає порівняно з суб’єктивним
фактором, тобто маса знарядь праці і матеріялів праці чимраз
більше зростає порівняно з сумою робочих сил, доконечних для
того, щоб пустити в рух ці знаряддя і матеріяли праці. Отже,
в міру того, як зростання капіталу робить працю продуктивнішою,
капітал зменшує попит на працю порівняно з своєю власною
величиною]\footnote*{
Заведене у прямі дужки ми беремо з французького видання.
«Le Capital etc.», Ch.~XXV, p. 273--274). \Red{Ред.}
}.

Ця зміна в технічному складі капіталу, це зростання маси
засобів продукції порівняно з масою робочої сили, яка їх оживляє,
відбивається з свого боку на вартостевому складі капіталу, на
збільшенні сталої складової частини капітальної вартости коштом
її змінної складової частини. Нехай, наприклад, рахуючи
в процентах, спочатку 50\% якогось капіталу вкладається в засоби
продукції і 50\% у робочу силу, а згодом, з розвитком продуктивної
сили праці, — 80\% у засоби продукції і 20\% у робочу силу
й~\abbr{т. д.} Цей закон щораз більшого зростання сталої частини капіталу
проти змінної потверджується на кожному кроці (як це вже
показано вище) порівняльною аналізою товарових цін, однаково,
чи будемо порівнювати різні економічні епохи в однієї нації
чи різні нації за тієї самої епохи. Відносна величина того елементу
ціни, який репрезентує лише вартість зужиткованих засобів
продукції або сталу частину капіталу, буде взагалі просто
пропорційна, а відносна величина другого елементу ціни, який
оплачує працю або репрезентує змінну частину капіталу, зворотно
пропорційна до проґресу акумуляції.

Однак зменшення змінної частини капіталу проти сталої,
або зміна складу капітальної вартости лише приблизно показує
зміну в складі його речових складових частин. Коли, наприклад,
за наших часів капітальна вартість, вкладена в прядільництво,
складається з \sfrac{7}{8} сталого капіталу й \sfrac{1}{8} змінного капіталу,
тимчасом як на початку XVIII віку вона складалася з \sfrac{1}{2} сталого
і \sfrac{1}{2} змінного капіталу, то, навпаки, маса сировинного матеріялу,
засобів праці й~\abbr{т. ін.}, що її за наших часів продуктивно споживає
певна кількість прядільної праці, є в багато сот разів більша,
ніж на початку XVIII віку. Причина цьому просто та, що із
\parbreak{}  %% абзац продовжується на наступній сторінці
