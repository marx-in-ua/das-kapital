
\index{i}{*0091}  %% посилання на сторінку оригінального видання
\nonumsection{Передмова до третього видання}{.~}{Фрідріх Енґельс}

Марксові не судилося самому виготовити до друку це третє
видання. Могутній мислитель, що перед величністю його схиляють
голови тепер навіть його супротивники, помер 14 березня 1883~\abbr{р.}

\looseness=-1
На мене, що в особі Маркса втратив людину, яка сорок років
була моїм найкращим й найнезламнішим другом, другом, якому
я завдячую більше, ніж це можна сказати словами, — спав
тепер обов’язок подбати так за це третє видання першого тому,
як і за другий том, що його він залишив у рукопису. Як я виконав
першу частину цього обов’язку, — про це я повинен дати тут
читачеві звіт.

Спочатку Маркс мав на думці переробити більшу частину
тексту першого тому, гостріше зформулювати деякі теоретичні
пункти, додати нові пункти, доповнити історичний і статистичний
матеріял аж до найновішого часу. Його хворість і бажання
закінчити редакцію другого тому примусили його зректися цього.
Малося змінити тільки щонайпотрібніше, додати тільки те, що
вже містилося у французькому виданні («Le Capital», par Karl
Marx. Paris, Lachâtre 1873), яке за той час вийшло.

Серед спадщини Маркса знайшовся також і німецький примірник
«Капіталу», подекуди виправлений ним та з посиланнями
на французьке видання; знайшлось і французьке видання з
відзначеними докладно місцями, якими він хотів був скористуватися.
Ці зміни й доповнення обмежуються, за небагатьма винятками,
останньою частиною книги, відділом: «Процес акумуляції
капіталу». Текст цього відділу більш, ніж інші, збігається з первісним
нарисом, тимчасом як попередні відділи були вже ґрунтовно
перероблені. Тому стиль був тут жвавіший, одностайніший,
але й недбайливіший, траплялись англіцизми, місцями неясності;
розвиток думки виявляв подекуди прогалини, бо окремі важливі
моменти були лише намічені.

Щодо стилю, то Маркс сам ґрунтовно перевірив багато підвідділів,
і цим, як і численними усними вказівками, зазначив мені
міру, як далеко мені можна піти, усуваючи англійські технічні
вислови й інші англіцизми. Додатки й доповнення Маркс, безперечно,
був би ще переробив і гладеньку французьку мову був би
замінив своєю власною, стислою німецькою; я мусив задовольнитися
тим, що переніс їх на відповідні місця, додивляючися, щоб
вони по змозі пасували до первісного тексту.

Отже, в цьому третьому виданні не змінено жодного слова,
про яке напевно не знав би я, що автор сам би був його змінив.
Мені не могло спасти на думку заводити до «Капіталу» загальновживаний
жарґон, яким звичайно висловлюються німецькі економісти,
\index{i}{*0092}  %% посилання на сторінку оригінального видання
— ту тарабарщину, в якій, приміром, праце\emph{давцем} (Arbeitgeber)
зветься той, хто за готівку бере собі від інших їхню
працю, а праце\emph{ємцем} (Arbeitnehmer) зветься той, що в нього відбирають
його працю за плату. Французи в щоденному житті теж
уживають слова travail у значенні «заняття». Але французи
вважали б цілком справедливо за божевільного того економіста,
що захотів би капіталіста звати donneur de travail\footnote*{
працедавцем. \emph{Ред.}
}, а робітника —
receveur de travail\footnote*{ працеємцем. \emph{Ред.}}.

Так само я не дозволив собі звести до новонімецьких еквівалентів
уживані повсюди в тексті англійські гроші, міри й вагу. Коли
вийшло перше видання «Капіталу», в Німеччині було стільки
родів мір і ваги, скільки є днів у році, а до цього ще два роди
марок (державна марка існувала тоді лише в голові Soetbeer’a, що
вигадав її наприкінці тридцятих років), два роди ґульденів і
принаймні три роди талярів, між ними один такий, що його одиницю
становили «нових дві третини». У природознавстві панувала
метрична система, на світовому ринку — англійська система мір
і ваги. Серед таких обставин англійські одиниці міри були сами
собою зрозумілі в книзі, що примушена була свої фактичні докази
брати майже виключно з англійських промислових відносин. Ця
остання причина є вирішальна ще й сьогодні, то більше, що відповідні
відносини на світовому ринку майже не змінилися, і що саме
в найважливіших галузях промисловости, — залізній і бавовняній,
— ще й нині панують майже виключно англійські міри й вага.

Наприкінці ще кілька слів про малозрозумілу Марксову
манеру цитувати. При суто фактичних даних і описах цитати,
приміром, з англійських Синіх Книг, є, само собою зрозуміло,
лише попросту посиланнями на документи. Інша однак справа,
коли цитується теоретичні погляди інших економістів. Тут цитата
має лише встановити, де, коли й хто вперше ясно висловив
думку, що виникала в ході розвитку економічної науки. При
цьому важливо лише те, щоб даний економічний погляд мав
значення для історії науки, щоб він був більш-менш адекватним
теоретичним висловом економічних обставин свого часу. Але
чи має ще цей погляд абсолютне або відносне значення для погляду
самого автора, або чи має він уже тільки історичний
інтерес, — це вже зовсім не має значення. Отже, ці цитати становлять
лише запозичений з історії економічної науки поточний
коментар до тексту і відзначають за датами й авторами окремі
значніші успіхи економічної теорії. А це було дуже потрібне в
такій науці, історики якої ще й досі відзначаються лише тенденційним,
майже кар’єристичним неуцтвом. — Тому легко зрозуміти,
чому Маркс, як він зазначає це в передмові до другого видання,
лише у випадках цілком виняткових цитує німецьких економістів.

Сподіваюся, що другий том з’явиться протягом 1884 року.

\begin{flushright}
\emph{Фрідріх Енґельс}
\end{flushright}

{\small Лондон, 7 листопада 1883~\abbr{р.}}
