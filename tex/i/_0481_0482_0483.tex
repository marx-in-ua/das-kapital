\parcont{}  %% абзац починається на попередній сторінці
\index{i}{0481}  %% посилання на сторінку оригінального видання
продукті\footnote{
«Заробітну плату, як і зиск, треба розглядати як частину готового
продукту» («Wages as well as profits are to be considered each of
them as really a portion of the finished product»). (\emph{G.~Ramsay}: «An Essay
on the Distribution of Wealth», Edinburgh 1836 p.~142). «Частина продукту,
що припадає робітникові у формі заробітної плати». (\emph{J.~Mill}:
«Elements of Political Economy». Переклад Parissot’a, Paris 1823 p. 34).
}. Це — частина продукту, постійно репродукованого
самим робітником, яка постійно припливає до нього назад у
формі заробітної плати. Правда, капіталіст виплачує йому цю
товарову вартість грішми. Але ці гроші є лише перетворена форма
продукту праці [або, точніше, певної частини продукту праці]\footnote*{
Заведене у прямі дужки беремо з другого німецького видання. \Red{Ред.}
}.
Тимчасом як робітник перетворює частину засобів продукції
на продукт, частина його попереднього продукту знов перетворюється
на гроші. Його праця минулого тижня або останнього
півріччя є те, чим оплачують його сьогоднішню працю або працю
найближчого півріччя. Ілюзія, яку утворює грошова форма,
вмить зникає, скоро тільки ми замість поодинокого капіталіста
й поодинокого робітника розглядатимемо клясу капіталістів і
клясу робітників. Кляса капіталістів постійно дає клясі робітників
у формі грошей чеки на частину продукту, випродукованого
клясою робітників і присвоєного клясою капіталістів. Ці
чеки робітник так само постійно повертає клясі капіталістів і
таким чином відбирає від неї ту частину свого власного продукту,
що припадає йому самому. Товарова форма продукту і грошова
форма товару замасковують цей процес.

Отже, змінний капітал\footnote*{
У французькому виданні Маркс тут робить таку примітку: «Змінний
капітал тут розглядається виключно як фонд для оплати найманих
робітників. Відомо, що в дійсності він стає змінним лише з того моменту,
коли куплена ним робоча сила функціонує вже в процесі продукції».
\Red{Ред.}
} є лише осібна історична форма виявлення
фонду засобів існування або робочого фонду, що його
робітник потребує для свого утримання й своєї репродукції і
що його він за всяких систем суспільної продукції завжди мусить
сам продукувати й репродукувати. Робочий фонд постійно
припливає до нього у формі засобів платежу за його працю лише
тому, що його власний продукт постійно віддалюється від нього
у формі капіталу. Але ця форма виявлення робочого фонду нічого
не змінює в тому, що капіталіст авансує робітникові його
власну упредметнену працю\footnote{
«Коли капітал вживають на авансування робітникам заробітної
плати, то він нічого не додає до фонду, призначеного для підтримання
праці» («When capital is employed in advancing to the workmen his wages
it adds nothing to the funds for the maintenance of labour»). (\emph{Cazenove}
у примітці до його видання праці Малтуза «Definitions in Political Economy»,
London 1853, p. 22).
}. Візьмімо селянина-кріпака. Він
працює своїми власними засобами продукції на своєму власному
полі, наприклад, три дні на тиждень. Три інші дні на тиждень
він одробляє панщину в панському маєтку. Він постійно репродукує
свій власний робочий фонд, і цей фонд ніколи не набирає
\index{i}{0482}  %% посилання на сторінку оригінального видання
супроти нього форми засобів платежу, авансованих йому
від третьої особи за його працю. Зате і його неоплачена примусова
праця ніколи не набирає форми добровільної та оплаченої
праці. Коли завтра поміщик присвоїть собі поле, робочу худобу,
насіння, коротко — засоби продукції селянина-кріпака,
то цей останній відтепер муситиме продавати свою робочу силу
сеньйорові. За інших незмінних обставин він, як і раніш, працюватиме
6 днів на тиждень: 3 дні на себе, 3 дні на колишнього
сеньйора, що тепер перетворився на пана-наймача робітників.
Як і раніш, він споживатиме засоби продукції як засоби продукції
і переноситиме їхню вартість на продукт. Як і раніш, певна
частина продукту ввіходитиме в репродукцію. Але так само як
панщанна праця набирає форми найманої праці, так само й робочий
фонд, що його, як і раніш, продукує й репродукує селянин-кріпак,
набирає форми капіталу, авансованого селянинові від
колишнього сеньйора. Буржуазний економіст, що його обмежений
мозок не може відокремити форму виявлення від того, що
в ній виявляється, заплющує очі перед тим фактом, що навіть
ще й тепер на земній кулі робочий фонд лише винятково виступає
у формі капіталу\footnote{
«Засоби існування робітників авансуються капіталістами робітникам
навіть менше, ніж на одній четвертині земної кулі». (\emph{Richard
Jones}: «Textbook of Lectures on the Political Economy of Nations», Hertford
1852, p. 16).
}.

Правда, змінний капітал втрачає характер вартости, авансованої
із власного фонду капіталіста\footnoteA{
«Хоч мануфактурний робітник дістає свою заробітну плату як
аванс від хазяїна, проте фактично це не коштує хазяїнові ніяких витрат,
бо сума цієї заробітної плати звичайно повертається назад разом із зиском
у підвищеній вартості речі, на яку вжито цю працю» («Though the
manufacturer has his wages advanced to him by his master, he in reality
costs him no expense, the value of these wages being generally restored,
together with a profit, in the improved value of the subject upon
which his labour is bestowed»). (\emph{A.~Smith}: «Wealth of Nations», b. II,
ch. 3, p. 355).
}, лише тоді, коли ми розглядатимемо
процес капіталістичної продукції в безперервній течії
його відновлення. Однак цей процес мусить десь і колись початися.
Тому, з того погляду, що його ми досі трималися, річ
імовірна, що капіталіст якогось часу за допомогою якоїсь первісної
акумуляції, незалежної від чужої неоплаченоі праці,
став посідачем грошей і тому міг виступити на ринку як
покупець робочої сили\footnote*{
У французькому виданні замість останніх двох речень читаємо
таке: «Однак, раніше ніж відновитися, цей процес мусив був початися
й тривати певний відтинок часу, протягом якого робітник не міг ще бути
оплачений з його власного продукту, ані жити з повітря. Отже, чи не слід
було б нам припустити, що капіталістична кляса, з’явившись уперше
на ринку праці, вже нагромадила своєю власною працею та власними
заощадженнями скарби, що дали їй змогу авансувати робітникам засоби
існування у формі грошей. Погодьмось покищо на таке рішення цієї
проблеми, яку ми докладніше розглянемо в розділі про так звану первісну
акумуляцію». («Le Capital etc.», v. I, ch. XXIII, p. 248--249). \Red{Ред.}
}. А втім проста безперервність капіталістичного
\index{i}{0483}  %% посилання на сторінку оригінального видання
процесу продукції, або проста репродукція, зумовлює
ще інші своєрідні зміни, що стосуються не тільки до змінної
частини капіталу, а й до всього капіталу в цілому.

Якщо додаткова вартість, створювана періодично, наприклад,
щороку, капіталом у \num{1.000}\pound{ фунтів стерлінґів}, становить 200\pound{ фунтів
стерлінґів}, і якщо цю додаткову вартість щороку споживається,
то ясно, що після п’ятирічного повторювання того самого
процесу сума спожитої додаткової вартости дорівнює 200 × 5, або
дорівнює первісно авансованій капітальній вартості в \num{1.000}\pound{ фунтів
стерлінґів}. Коли б річну додаткову вартість споживано лише
частинно, наприклад, лише наполовину, то той самий результат
ми мали б після десятирічного повторювання продукційного
процесу, бо 100 × 10 \deq{} \num{1.000}. Взагалі кажучи, авансована
капітальна вартість, поділена на щорічно споживану додаткову
вартість, дає число років або число періодів репродукції, після
скінчення яких первісно авансований капітал споживається
капіталістом і тому зникає. Уявлення капіталіста, що він споживає
продукт чужої неоплаченої праці, додаткову вартість,
та зберігає первісну капітальну вартість, абсолютно нічого не
може змінити в цьому факті. Коли мине якесь певне число років,
присвоєна ним капітальна вартість дорівнює сумі додаткової
вартости, присвоєної ним без еквіваленту протягом того самого
числа років, а спожита ним сума вартости дорівнює первісній
капітальній вартості. Правда, він зберігає у своїх руках капітал,
що його величина не змінилася — капітал, що з нього частина,
будівлі, машини й~\abbr{т. ін.}, існувала вже тоді, коли він пустив у рух
своє підприємство. Але тут ідеться про вартість капіталу, а не
про його матеріяльні складові частини. Коли хтось споживе
все своє майно, поробивши таку кількість боргів, що дорівнюють
вартості цього майна, то якраз усе це майно й репрезентує лише
загальну суму його боргів. І так само, коли капіталіст спожив
еквівалент свого авансованого капіталу, то вартість цього капіталу
репрезентує лише загальну суму присвоєної ним задурно
додаткової вартости. Жодного атома вартости його старого капіталу
вже далі не існує.

Отже, цілком незалежно від усякої акумуляції, проста безперервність
процесу продукції, або проста репродукція, після
коротшого або довшого періоду неминуче перетворює кожний
капітал у нагромаджений капітал, або в капіталізовану додаткову
вартість. Навіть якщо при своєму вступі в продукційний процес
капітал був власністю підприємця, особисто ним заробленою,
все одно, раніш, або пізніш, він стає присвоєною без еквіваленту
вартістю, або матеріялізацією, в грошовій чи іншій формі, неоплаченої
чужої праці.

Як ми бачили в четвертому розділі, для того, щоб перетворити
гроші на капітал, недосить наявности продукції вартости й товарової
циркуляції. Для цього мусили насамперед протистояти
один одному як покупець і продавець на одному боці посідач
вартости або грошей, на другому — посідач вартостетворчої
\parbreak{}  %% абзац продовжується на наступній сторінці
