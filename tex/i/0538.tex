тшх промислових підприємств, що здійснення їх зв’язане з попередньою
централізацією капіталу. Отже, за наших часів сила
взаємного притягання поодиноких капіталів і тенденція до централізації
дужча, ніж коли-будь раніш. Але, хоч відносне поширення
й енерґія руху в напрямі централізації визначається до
деякої міри досягнутою вже величиною капіталістичного багатства
й вищістю економічного механізму, проте проґрес централізації
зовсім не залежить від позитивного зростання величини
суспільного капіталу. І саме це й відрізняє централізацію від
концентрації, яка є лише інший вираз репродукції в поширеному
маштабі. Централізація може відбуватися через просту зміну
в розподілі капіталів, що вже існують, через просту зміну кількісного
угруповання складових частин суспільного капіталу.
Капітал тут може в одних руках зрости до велетенських розмірів,
бо там його витягнуто з багатьох поодиноких рук. У кожній
даній галузі підприємства централізація досягла б своєї крайньої
межі, коли б усі вкладені в неї капітали злилися в одинодним
капітал У кожному даному суспільстві цієї межі було б
досягнуто лише в той момент, коли цілий суспільний капітал
було б сполучено або в руках одного окремого капіталіста, або
в руках одним-одного товариства капіталістів.

Централізація доповнює справу акумуляції, даючи промисловим
капіталістам змогу поширювати маштаб своїх операцій.
Чи буде цей останній результат наслідком акумуляції або централізації;
чи відбувається централізація насильницьким шляхом
анексії, — коли деякі капітали стають такими потужними
центрами притягання для інших, що вони руйнують їхню індивідуальну
з’єднаність і потім притягують до себе ці роз’єднані
частини; чи злиття маси капіталів уже утворених або таких,
що перебувають у процесі творення, відбувається лагіднішим
способом, через утворення акційних товариств, — економічний
ефект в усіх цих випадках лишається той самий. Зростання розмірів
промислових підприємств повсюди становить вихідний
пункт для ширшої організації спільної праці багатьох, для
ширшого розвитку її матеріяльних рушійних сил, тобто для
проґресивного перетворення розрізнених і рутинних процесів
продукції на суспільно комбіновані й науково організовані
процеси продукції.

Але ясно, що акумуляція, поступінне збільшення капіталу
за допомогою репродукції, яка з колової форми переходить у
спіралю, є надто повільний процес порівняно з централізацією,
що потребує лише зміни кількісного угруповання інтеґральних
частин суспільного капіталу. Світ і досі був би ще без залізниць,
коли б йому довелось чекати, доки акумуляція доведе поодинокі
капітали до такого розміру, що зробив би їх здатними до буду77b
[До четвертого видання. — Найновіші англійські й американські
«трести» прагнуть уже цієї мети, силкуючися з’єднати принаймні всі
великі підприємства певної галузі промисловосте в одно велике акційне
товариство з практичною монополією. — Ф. Е.].
