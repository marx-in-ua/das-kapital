\parcont{}  %% абзац починається на попередній сторінці
\index{i}{0041}  %% посилання на сторінку оригінального видання
практичного буденного життя людей являтимуть собою день-у-день прозоро-раціональні відносини людей
одних до одних та до природи. Суспільний життєвий процес, тобто матеріяльний процес продукції,
скидає з себе містичний туманний серпанок лише тоді, коли він, як продукт вільної асоціації людей,
знаходиться під
їхнім свідомим пляномірним контролем. Але для цього потрібна така матеріяльна основа суспільства або
такий ряд матеріяльних умов існування, які й собі є природний продукт довгого й дуже болісного
процесу історичного розвитку.
\enablefootnotebreak{}

Правда, політична економія проаналізувала, хоч і не досконально\footnote{
Недостатність Рікардової аналізи величини вартости — а це є найліпша аналіза — можна буде
побачити з третьої й четвертої книги цього твору. Що ж до вартости взагалі, то клясична політична
економія ніде виразно або з ясною свідомістю не розрізняє праці, як вона виражається у вартості, від
тієї ж праці, як вона виражається у споживній вартості її продукту. Фактично вона, звичайно,
додержує цієї ріжниці, бо розглядає працю то з погляду кількости, то з погляду якости. Але їй і не
спадає на думку, що сама кількісна ріжниця праць має собі за передумову їх якісну єдність або
рівність, отже зведення їх до абстрактної людської праці. Рікардо, наприклад, заявляє, що він
погоджується з Destutt de Tracy, коли цей каже: «А що справедливо, що наші фізичні та інтелектуальні
здібності становлять наше єдине первісне багатство, то й застосовування цих здібностей, праця
якогось роду, є наш первісний скарб, і саме це застосовування завжди утворює всі ті речі, які ми
звемо багатством\dots{} Так само справедливо й те, що всі такі речі репрезентують лише працю, яка
утворила їх, і коли вони мають вартість або навіть дві різні вартості, то ці вартості можуть
походити лише від (вартости) тієї праці, яка утворила їх». («As it is certain, that our physical and
moral faculties are alone our original riches, the employment of those faculties, labour of some
kind, is our original treasure, and that it is always from this employment — that all those things
are created which we call riches\dots{} It is certain too, that all those things only represent the
labour which has created them, and if they have a value, or even two distinct values, they can only
derive them from that (the value) of the labour from which they emanate»). (\emph{Ricardo}: «The Principles
of Political Economy», 3-rd. ed. London 1821, p. 334). Зазначимо лише, що Рікардо підсуває Дестютові
своє власне глибше розуміння. Щоправда, Дестют, з одного боку, фактично каже, що всі речі, які
становлять багатство, «репрезентують працю, яка утворила їх», алеж, з другого боку, він каже, що
«дві різні вартості» їхні (споживну вартість і мінову вартість) вони дістають від «вартости праці».
Таким чином він впадає в пласкість вульґарної політичної економії, яка наперед припускає вартість
якогось товару (тут праці), щоб потім через неї визначити вартість інших товарів. Рікардо розуміє
Destutt de Tracy так, ніби цей каже, що і споживна і мінова вартість репрезентують працю (а не
вартість праці). Але сам він так мало розрізняє двоїстий характер праці, яка виражається двояко, що
в цілому розділі «Value and Riches, Their Distinctive Properties» він примушений морочити собі
голову з тривіяльностями такого панка як Ж.~Б.~Сей. Тому він наприкінці дивом здивований, що Дестют,
який вважає разом із ним працю за джерело вартости, усе ж таки, з другого боку, погоджується з Сеєм
щодо поняття вартости.
}, вартість та величину вартости і
розкрила захований у цих формах зміст. Але вона ні разу навіть не поставила питання: чому цей зміст
набирає такої форми, чому, отже, праця виражається у вартості, а міра праці через час її тривання —
у величині
\parbreak{}  %% абзац продовжується на наступній сторінці
