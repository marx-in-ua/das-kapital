ним підвищенням криється реальне зниження заробітної плати,
бо воно навіть не урівноважує того підвищення цін на доконечні
засоби існування, що сталося за той час. Доказ — нижченаведений
витяг з офіціяльних звітів одного ірляндського робітного
дому.

Пересічні тижневі витрати на утриманця однієї людини

Роки    Харчі    Одяг    Разом
Від 29 вересня 1848 р.
до 29 вересня 1849 р.    1 шилінґ З \sfrac{1}{4} пенса    3 пенси    1 шил. 6\sfrac{1}{4} пенса

Від 29 вересня 1868 р.
до 29 вересня 1869 р.    2 шилінґи 7\sfrac{1}{4}пенса    6 пенсів    3 шил. \sfrac{1}{4} пенса

Отже, ціна доконечних засобів існування підскочила майже
вдвоє, а ціна одягу рівно вдвоє, аніж перед двадцятьма роками.

Навіть коли залишити осторонь цю диспропорцію, то саме
порівняння заробітних плат, визначених у грошах, далеко ще
не дає правдивого висновку. Перед голодом велику частину заробітної
плати на селі видавали in natura, грішми виплачували
лише дуже невеличку частину; нині грошова виплата стала загальним
правилом. Вже з цього випливає, що, хоч який буде
рух реальної заробітної плати, її грошовий рівень мусив підвищитися.
«Перед голодом сільський поденник мав шматок
землі, де він культивував картоплю і відгодовував свиней та
дробину. Нині він мусить не тільки купувати собі всі засоби
існування, але він втрачає й ті доходи, що він мав із продажу
свиней, дробини і яєць».\footnote{
Там же, стор. 291.
} Справді, раніше сільські робітники
зливалися з дрібними фармерами і здебільша становили
лише ар’єрґард середніх і великих фарм, де вони находили для
себе заняття. Лише від часу катастрофи 1846 р. вони почали становити
частину кляси власне найманих робітників, окрему верству,
зв’язану із своїми панами-наймачами лише грошовими відносинами.
Ми вже знаємо, який був їхній житловий стан перед 1846 р.
Від того часу він ще більше погіршав. Деяка частина сільських
поденників, що, зрештою, з дня на день меншає, живе ще на
землях фармерів у переповнених хатинах, що їхній огидний стан
далеко перевищує все те найгірше, що виявили нам у цьому відношенні
англійські рільничі округи. І такий є стан речей повсюди,
за винятком деяких округ в Ulster’i, на півдні у графствах
Cork, Limerick, Kilkenny та інших; на сході у Wicklow’i
Wexford’i і т. д.; у центрі в King’s і Queen’s County, Dublin’i
і т. д.; на півночі в Down’i, Antrim’y, Tyrone і т. д.; на заході
в Sligo, Roscommon’i, Mayo, Galway і т. д. «Це, — вигукує один