номісти, — ту тарабарщину, в якій, приміром, працедавцем (Arbeitgeber)
зветься той, хто за готівку бере собі від інших їхню
працю, а працеємцем (Arbeitnehmer) зветься той, що в нього відбирають
його працю за плату. Французи в щоденному житті теж
уживають слова travail у значенні «заняття». Але французи
вважали б цілком справедливо за божевільного того економіста,
що захотів би капіталіста звати donneur de travail,\footnote*{
— працедавцем. \emph{Ред.}
** — працеємцем. \emph{Ред.}
} а робітника —
receveur de travail.**

Так само я не дозволив собі звести до новонімецьких еквівалентів
уживані повсюди в тексті англійські гроші, міри й вагу. Коли
вийшло перше видання «Капіталу», в Німеччині було стільки
родів мір і ваги, скільки є днів у році, а до цього ще два роди
марок (державна марка існувала тоді лише в голові Soetbeer’a, що
вигадав її наприкінці тридцятих років), два роди ґульденів і
принаймні три роди талярів, між ними один такий, що його одиницю
становили «нових дві третини». У природознавстві панувала
метрична система, на світовому ринку — англійська система мір
і ваги. Серед таких обставин англійські одиниці міри були сами
собою зрозумілі в книзі, що примушена була свої фактичні докази
брати майже виключно з англійських промислових відносин. Ця
остання причина е вирішальна ще й сьогодні, то більше, що відповідні
відносини на світовому ринку майже не змінилися, і що саме
в найважливіших галузях промисловости, — залізній і бавовняній,
— ще й нині панують майже виключно англійські міри й вага.

Наприкінці ще кілька слів про малозрозумілу Марксову
манеру цитувати. При суто фактичних даних і описах цитати,
приміром, з англійських Синіх Книг, є, само собою зрозуміло,
лише попросту посиланнями на документи. Інша однак справа,
коли цитується теоретичні погляди інших економістів. Тут цитата
має лише встановити, де, коли й хто вперше ясно висловив
думку, що виникала в ході розвитку економічної науки. При
цьому важливо лише те, щоб даний економічний погляд мав
значення для історії науки, щоб він був більш-менш адекватним
теоретичним висловом економічних обставин свого часу. Але
чи має ще цей погляд абсолютне або відносне значення для погляду
самого автора, або чи має він уже тільки історичний
інтерес, — це вже зовсім не має значення. Отже, ці цитати становлять
лише запозичений з історії економічної науки поточний
коментар до тексту і відзначають за датами й авторами окремі
значніші успіхи економічної теорії. А це було дуже потрібне в
такій науці, історики якої ще й досі відзначаються лише тенденційним,
майже кар’єристичним неуцтвом. — Тому легко зрозуміти,
чому Маркс, як він зазначає це в передмові до другого видання,
лише у випадках цілком виняткових цитує німецьких економістів.

Сподіваюся, що другий том з’явиться протягом 1884 року.

Лондон, 7 листопада 1883 р. Фрідріх Енґельс