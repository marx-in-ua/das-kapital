
\index{i}{0446}  %% посилання на сторінку оригінального видання
\section{Різні формули норми додаткової вартости}
\vspace{-2\medskipamount}

Ми бачили, що норма додаткової вартости виражається в таких
формулах:
\begin{center}
I. $\displaystyle \frac{\text{додаткова вартість}}{\text{змінний капітал}} \left( \frac{m}{v}\right) \deq{}
\frac{\text{додаткова вартість}}{\text{вартість робочої сили}} \deq{}
\frac{\text{додаткова праця}}{\text{доконечна праця}}$
\end{center}

\disablefootnotebreak{}
\noindent{}Дві перші формули виражають у формі відношення вартостей
те саме, що третя виражає у формі відношення відтинків часу,
що протягом їх ці вартості продукується. Ці формули, що одна
одну доповнюють, є строго логічні. Тим то ми находимо їх у клясичній
політичній економії, правда, щодо суті, але виробленими
несвідомо. Зате ми бачимо там такі вивідні формули:
\begin{center}
II\footnote*{У французькому виданні Маркс заводить цю формулу в дужки
і дає до цього таку примітку: «Ми заводимо першу формулу в дужки,
бо ясно вираженого поняття додаткової праці ми не знаходимо в буржуазній
політичній економії». \emph{Ред.}}.
$\displaystyle \frac{\text{додаткова праця}}{\text{робочий день}} \deq{}
\frac{\text{додаткова вартість}}{\text{вартість продукту}} \deq{}  \frac{\text{додатковий продукт}}{\text{сукупний продукт}}$
\end{center}
\enablefootnotebreak{}

\noindent{}Ту саму пропорцію виражено тут навпереміну то у формі
робочих часів, то у формі вартостей, що в них вони втілюються,
то у формі продуктів, що в них існують ці вартості. Звичайно,
припускається, що під вартістю продукту треба розуміти лише
вартість, новоспродуковану протягом робочого дня, а сталу частину
вартости продукту виключено.

У всіх цих формулах дійсний ступінь експлуатації праці, або
норму додаткової вартости, виражено неправильно. Хай робочий
день буде 12 годин. Якщо інші припущення нашого попереднього
прикладу лишаються незмінні, то в цьому випадку дійсний
ступінь експлуатації праці виразиться в таких пропорціях:
$\frac{\text{6 годин додаткової праці}}{\text{6 годин доконечної праці}}\deq{}
\frac{\text{додаткова вартість у 3\shil{ шилінґи}}}{\text{змінний капітал у 3\shil{ шилінґи}}}
\deq{}100\%$.

Навпаки, за формулою II ми маємо:
\[
\text{II. }\frac{\text{6 годин додаткової праці}}{\text{робочий день у 12 годин}} \deq{}\frac{\text{додаткова вартість у 3\shil{ шилінґи}}}{\text{новоспродукована вартість у 6\shil{ шилінґів}}} \deq{} 50\%\text{.}
\]

\looseness=-1
\noindent{}Ці вивідні формули в дійсності виражають ту пропорцію,
що в ній робочий день або новоспродукована протягом нього
\parbreak{}  %% абзац продовжується на наступній сторінці
