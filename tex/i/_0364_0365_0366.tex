\parcont{}  %% абзац починається на попередній сторінці
\index{i}{0364}  %% посилання на сторінку оригінального видання
що він звільняє 50 робітників, а решту — 50 робітників — уживає
коло машин, які коштують йому \num{1.500}\pound{ фунтів стерлінґів}. Щоб
справу спростити, ми залишаємо осторонь будівлі, вугілля тощо.
Припустімо, далі, що споживаний кожного року сировинний
матеріял коштує, як і раніш, \num{3.000}\pound{ фунтів стерлінґів}\footnote{
Nota bene. Я подаю ілюстрацію цілком на манір вищеназваних
економістів.
}. Чи «звільнився» через цю метаморфозу якийсь капітал? За старого
способу продукції загальна витрачена сума становила \num{6.000}\pound{ фунтів
стерлінґів} і складалася наполовину із сталого, наполовину із
змінного капіталу. Тепер вона складається з \num{4.500}\pound{ фунтів стерлінґів}
(\num{3.000}\pound{ фунтів стерлінґів} на сировинний матеріял та \num{1.500}\pound{ фунтів
стерлінґів} на машини) сталого та \num{1.500}\pound{ фунтів стерлінґів} змінного
капіталу. Замість половини, змінна або перетворена на живу робочу
силу частина капіталу становить лише \sfrac{1}{4} цілого капіталу. Замість
звільнення відбувається тут зв’язування капіталу в такій формі,
що в ній він перестає обмінюватися на робочу силу, тобто відбувається
перетворення змінного капіталу на сталий. Капітал у \num{6.000}\pound{ фунтів стерлінґів}, за інших незмінних умов, може тепер давати заняття
не більш, як 50 робітникам. З кожним поліпшенням машин він
дає заняття дедалі меншому числу робітників. Коли б новозаведені
машини коштували менше за суму, що її коштували витиснута
ними робоча сила й знаряддя праці, тобто, наприклад, замість
\num{1.500}\pound{ фунтів стерлінґів} лише \num{1.000}\pound{ фунтів стерлінґів}, то змінний
капітал у \num{1.000}\pound{ фунтів стерлінґів} перетворився б на сталий капітал,
тобто був би зв’язаний, а капітал у 500\pound{ фунтів стерлінґів}
звільнився б. Останній, якщо припустити ту саму річну плату,
становить фонд заняття приблизно для 16 робітників, а звільнено
їх 50 — навіть багато менше, ніж для 16 робітників, бо для того,
щоб ці 500\pound{ фунтів стерлінґів} перетворити на капітал, треба частину
з них знов перетворити на сталий капітал, і отже, лише частину
з них можна перетворити на робочу силу.

Але припустімо навіть, що виготовлювання нових машин дає
заняття більшому числу механіків. Чи буде це компенсацією
для викинутих на брук шпалерників? У найліпшому випадку
виготовлювання нових машин дасть заняття меншому числу
робітників, ніж витискує вживання машин. Сума в \num{1.500}\pound{ фунтів
стерлінґів}, яка репрезентує лише заробітну плату звільнених
шпалерників, репрезентує тепер у формі машин: 1) вартість засобів
продукції, потрібних на виготовлення машин; 2) заробітну
плату механікам, що їх виготовляють; 3) додаткову вартість, що
припадає їхньому «хазяїнові». Далі: машина, бувши виготовлена,
аж до самої своєї смерти не потребує, щоб її відновлювали. Отже,
щоб постійно давати заняття додатковому числу механіків, фабриканти
шпалер один по одному мусять витискувати робітників
машинами.

В дійсності ці апологети мають на думці не цей рід звільнення
капіталу. Вони мають на думці засоби існування звільнених робітників.
\index{i}{0365}  %% посилання на сторінку оригінального видання
Не можна заперечити, що у вищенаведеному випадку,
приміром, машини не тільки звільняють 50 робітників і через це
роблять їх «вільними», але разом з тим ще знищують їхній зв’язок
із засобами існування вартістю в \num{1.500}\pound{ фунтів стерлінґів} та
«звільняють» таким чином ці засоби існування. Отже, той простий
і зовсім не новий факт, що машина звільняє робітника від
засобів існування, мовою економістів означає, що машина звільняє
засоби існування для робітника або перетворює їх на капітал,
щоб уживати робітника. Як бачимо, все залежить від того, яким
способом що висловити. Nominibus mollire licet mala\footnote*{
Можна гарними словами підсолоджувати лихо. \emph{Ред.}
}.

За цією теорією засоби існування вартістю в \num{1.500}\pound{ фунтів
стерлінґів} були капіталом, що збільшив свою вартість за допомогою
праці п’ятдесятьох звільнених шпалерників. Отже, цей
капітал втрачає своє заняття, скоро тільки ті п’ятдесят робітників
звільняються від роботи, та не має і хвилини спокою, поки
не знайде нове «вміщення», де названі п’ятдесят робітників
знову зможуть споживати його продуктивно. Отже, раніш або
пізніш, капітал і робітники знову мусять зійтися, і тоді матимемо
компенсацію. Отож, страждання робітників, витиснутих машинами,
так само минущі, як і багатства цього світу.

\looseness=1
Засоби існування в сумі \num{1.500}\pound{ фунтів стерлінґів} ніколи не протистояли
звільненим робітникам як капітал. Як капітал протистояли
їм ті \num{1.500}\pound{ фунтів стерлінґів}, які перетворено тепер на
машини. Коли ближче придивитися, то ці \num{1.500}\pound{ фунтів стерлінґів}
репрезентують лише ту частину шпалер, щорічно продукованих
за допомогою звільнених робітників, яку вони одержували
від свого хазяїна як заробітну плату не in natura\footnote*{
продуктами. \emph{Ред.}
}, а в грошовій
формі. За ці шпалери, перетворені на \num{1.500}\pound{ фунтів стерлінґів},
купували вони собі засоби існування на таку саму суму. Тому ці
останні існували для них не як капітал, а як товари, і вони сами
існували для цих товарів не як наймані робітники, а як покупці.
Та обставина, що машина «звільнила» їх від купівельних засобів,
перетворює їх з покупців на непокупців. Звідси зменшений
попит на ці товари. Voilà tout\footnote*{
Оце й усе. \emph{Ред.}
}. Якщо цей зменшений попит не
компенсується збільшеним попитом з іншого боку, то ринкова
ціна цих товарів меншає. Якщо це триває довго та у великому
розмірі, то постає переміщення робітників, уживаних у продукції
цих товарів. Частину капіталу, що раніше продукувала доконечні
засоби існування, репродукується в іншій формі\footnote*{
У французькому виданні замість останніх двох речень читаємо
таке: «Але, може, це спричиниться до того, що капітал, якого уживалося
в продукції цих засобів існування, покличе до роботи як додаткових
робітників наших звільнених шпалерників? Цілком навпаки: якщо це
зниження цін триватиме деякий час, то почнуть знижувати заробітну плату
робітників, уживаних у продукції цих засобів існування. Якщо дефіцит
у збуті доконечних засобів існування триватиме довгий час, то частина
капіталу, вживана в продукції їх, відпливе звідси й шукатиме собі іншої
сфери вміщення». («Le Capital etc.», v. I, ch. XV. p. 190). \emph{Ред.}
}. Підчас спадання
\index{i}{0366}  %% посилання на сторінку оригінального видання
ринкових цін та переміщення капіталу робітники, уживані
у продукції доконечних засобів існування, також «звільняються»
від якоїсь частини їхньої заробітної плати. Отже, замість
довести, що машини, звільняючи робітників од засобів існування,
одночасно перетворюють ці останні на капітал, щоб уживати
перших, пан апологет із своїм випробуваним законом попиту й
подання доводить, навпаки, що машини не тільки в тій галузі
продукції, де їх заводять, але й у тих галузях продукції, де їх
не заведено, викидають робітників на брук.

\looseness=-1
\disablefootnotebreak{}
Дійсні факти, перекручені економічним оптимізмом, такі.
Витиснутих машинами робітників викидають із майстерні на
ринок праці, і вони збільшують там число робочих сил, що ними
можна порядкувати для капіталістичної експлуатації. В сьомому
відділі ми побачимо, що цей вплив машин, який нам тут змальовано
як компенсацію для робітничої кляси, спадає, навпаки, як
найстрашніша кара на робітника. Тут зауважимо лише ось
що: робітники, викинуті з однієї галузі промисловости, можуть,
щоправда, шукати заняття в якійсь іншій. Якщо вони знаходять
собі заняття, і таким чином відновлюється зв’язок між ними й
засобами існування, які були звільнені разом з ними, то це
стається за допомогою нового додаткового капіталу, що шукає
вміщення, а зовсім не того капіталу, що вже раніш функціонував
і тепер перетворений на машини. Але навіть і в такому випадку,
— які мізерні їхні перспективи! Скалічені через поділ праці,
ці бідолахи так мало чого варті поза своєю колишньою сферою
праці, що вони знаходять собі доступ лише до небагатьох нижчих,
і тому завжди переповнених та низько оплачуваних, галузей
праці\footnote{
Один рікардіянець зауважує з приводу цього, заперечуючи проти
нісенітниць Ж.~Б.~Сея: «За розвиненого поділу праці вмілість робітників
може придатись тільки в тій окремій галузі, де вони навчилися її; вони
сами є своєрідні машини. Тим то абсолютно не поможе, коли верзти, як
папуга, що речі мають тенденцію знаходити свій рівень. Нам треба лише
поглянути навколо себе, і ми побачимо, що вони довгий час не можуть
знайти свого рівня, а якщо і знайдуть його, то цей рівень нижчий, ніж
він був на початку процесу». («An Inquiry into those Principles respecting
the Nature of Demand etc.», London 1821, p. 72).
}. Далі, кожна галузь промисловости притягає щороку
новий потік людей, який дає їй континґент для реґулярної заміни
та зросту\footnote*{
У французькому виданні кінець цього речення подано так: «\dots{} який
дає їй континґент для заміни спрацьованої робочої сили та для того поповнення,
що його вимагає реґулярний розвиток цієї галузі». \emph{Ред.}
}. Скоро тільки машини звільняють частину робітників,
що досі працювали в певній галузі промисловости, то й новий
потік промислових рекрутів перерозподіляється, і його вбирають
інші галузі праці, тимчасом як первісні жертви за переходовий
час здебільша занепадають і гинуть.

Безперечний факт, що машини сами по собі не винні у «звільненні»
робітників од засобів існування. Вони здешевлюють і
збільшують продукт у тій галузі, яку захоплюють, та спочатку
лишають ту масу засобів існування, що її продукується по інших
\parbreak{}  %% абзац продовжується на наступній сторінці
