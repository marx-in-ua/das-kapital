\parcont{}  %% абзац починається на попередній сторінці
\index{i}{0047}  %% посилання на сторінку оригінального видання
Так стає він грішми. «Вони мають одну думку й передають свою
силу й владу звірові. І щоб ніхто не посмів ні купувати, ані
продавати, як хіба той, хто має характер або ім’я звіра, або число
імені його».\footnote*{
«Illi unum consilium habent et virtutem et potestatem suam bestiae
tradunt. Et ne quis possit emere aut vendere, nisi qui habet characterem
aut nomen bestiae, aut numerum nominis ejus». (Apocalypse).
}

Грошовий кристаль є доконечний продукт процесу обміну,
що в ньому різнорідні продукти праці фактично прирівнюється
один одному і тому фактично перетворюється на товари. Історичний
процес поширення й поглиблення обміну розвиває дрімотну
в природі товару суперечність між споживною вартістю й вартістю.
Потреба надати для обороту зовнішній вираз цій суперечності
примушує прагнути самостійної форми для товарової вартости
й ні на хвилиночку не дає заспокоїтися доти, доки остаточно
не осягнуто такої форми через роздвоєння товару на товар і
гроші. Отже, тією самою мірою, якою здійснюється перетворення
продуктів праці на товари, здійснюється й перетворення товару
на гроші.\footnote{
З того можемо оцінити дотепність дрібнобуржуазного соціялізму,
який бажає увіковічнити товарову продукцію й одночасно усунути «суперечність
між грішми й товаром», отже, усунути й самі гроші, бо вони
існують лише в цій суперечності. З таким самим успіхом можна було б
усунути папу залишити католицтво. Ближче про це див. мою працю «Zur
Kritik der Politischen Oekonomie», стор. 61 і далі («До критики політичної
економії», ДВУ 1926 р., стор. 100).
}

Безпосередній обмін продуктів, з одного боку, має форму
простого виразу вартости, а з другого — ще не має її. Та форма
була: \emph{х} товару \emph{A} = \emph{y} товару \emph{B}. Форма безпосереднього обміну
товарів така: \emph{х} предмету споживання \emph{A} = \emph{y} предмету споживання
\emph{В}.\footnote{
Поки ще обмінюється не два різні предмети споживання, а, як
ми це часто спостерігаємо в дикунів, пропонується хаотичну масу речей
як еквівалент за щось третє, доти сам безпосередній обмін продуктів є
ще тільки в зародку.
} Речі \emph{A} й \emph{B} не є тут товари перед обміном, але стають ними
лише через обмін. Перша умова, за якої предмет споживання,
є в можливості мінова вартість, це — його буття як неспоживної
вартости, як кількости споживної вартости, що перевищує
безпосередні потреби свого посідача. Речі сами по собі супроти
людини зовнішні, а тому й можна їх відчужувати. Для того,
щоб це відчужування стало взаємним, потрібно, щоб люди тільки
мовчки виступали один проти одного як приватні власники тих
відчужуваних речей і саме через те — як особи одні від одних
незалежні. Однак таке відношення взаємної відчужености не
існує для членів примітивної громади, хоч матиме вона форму
патріярхальної родини, хоч староіндійської громади, хоч держави
інків і т. ін. Обмін товарів починається там, де кінчається
громада, в пунктах її контакту з чужими громадами або членами
чужих громад. Але скоро тільки речі стали вже товарами поза
громадою, то в наслідок зворотного впливу вони стають товарами
\index{i}{0048}  %% посилання на сторінку оригінального видання
і в унутрішньому житті громади. Їхні кількісні мінові відношення
спочатку цілком випадкові. Вимінними вони є в наслідок
вольового акту їхніх посідачів відчужити їх один одному. Тим часом
потреба в чужих предметах споживання помалу зміцнюється.
Постійне повторювання обміну робить його реґулярним
суспільним процесом. Тим то з часом принаймні частина продукту
праці мусить навмисно продукуватись на обмін. Від цього моменту
фіксується, з одного боку, відокремлення корисности речей
для безпосередньої потреби від корисности їх для обміну. Їхня
споживна вартість відокремлюється від їхньої мінової вартости.
З другого боку, кількісне відношення, в якому вони обмінюються,
стає залежне від самої їхньої продукції. Звичай фіксує їх як
величини вартости.

В безпосередньому обміні продуктів кожний товар є безпосередній
засіб обміну для свого посідача, еквівалент для свого
непосідача, однак лише остільки, оскільки він для останнього
є споживна вартість. Отже, предмет обміну не набуває ще жодної
форми вартости, незалежної від його власної споживної вартости
або від індивідуальних потреб обмінювачів. Доконечність цієї
форми розвивається разом із зростом числа й різноманітности
товарів, що вступають у процес обміну. Завдання виникає одночасно
із засобами його розв’язання. Стосунки, за яких посідачі
товарів обмінюють свої власні продукти на різні інші продукти
й порівнюють їх між собою, не можуть ніколи відбуватися без
того, щоб різні товари різних товаропосідачів у межах їхніх
стосунків не обмінювались на той самий третій рід товару й не
порівнювались з ними як вартості. Такий третій товар, стаючи
еквівалентом для різних інших товарів, безпосередньо набирає,
хоч і у вузьких межах, загальної або суспільної еквівалентної
форми. Ця загальна еквівалентна форма постає й зникає разом
із тим минущим суспільним контактом, який її покликав до життя.
Навпереміну й не на довгий час припадає вона цьому або тому
товарові. Але з розвитком обміну товарів вона міцно зростається
виключно з особливими родами товарів або кристалізується в
грошову форму. З яким саме родом товару вона зростається, це
спочатку залежить од випадку. Однак, взагалі і в цілому тут
вирішують справу дві обставини. Грошова форма зростається
або з тими найважливішими предметами довозу з чужих країн,
які дійсно є первісна форма виявлення мінової вартости тубільних
продуктів, або з предметом споживання, що становить головний
елемент тубільного відчужуваного майна, як, приміром, худоба.
Кочові народи перші розвивають грошову форму, бо все їхнє
майно перебуває в рухомій, отже, безпосередньо відчужуваній
формі, і ще й тому, що ввесь лад їхнього життя завжди ставить
їх у контакт з чужими громадами, а тому й заохочує до обміну
продуктів. Люди часто робили саму людину у формі раба первісним
грошовим матеріялом, але землю — ніколи. Така ідея могла
виникнути лише в уже розвинутому буржуазному суспільстві.
Вона з’явилася в останній третині ХVІІ віку, а спробу здійснити
\parbreak{}  %% абзац продовжується на наступній сторінці
