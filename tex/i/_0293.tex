
\index{i}{0293}  %% посилання на сторінку оригінального видання
\subsection{Капіталістичний характер мануфактури}
\vspace{\medskipamount}

\looseness=1
Скупчення значного числа робітників під командою того
самого капіталу становить природний вихідний пункт так кооперації
взагалі, як і мануфактури. Навпаки, мануфактурний поділ
праці розвиває зріст числа вживаних робітників у технічну
доконечність. Тепер мінімум робітників, яких мусить уживати
поодинокий капіталіст, приписується йому наявним поділом
праці. З другого боку, користі з дальшого поділу праці зумовлені
дальшим збільшенням числа робітників, яке можна перевести
лише за кратного збільшення робітників\footnote*{
У французькому виданні це місце подано так: «Щоб мати користь
з дальшого поділу праці, треба не просто збільшити число робітників, а
збільшили їх у кратному відношенні, тобто збільшити їх воднораз,
відповідно до певних пропорцій, по всіх різних групах майстерні» («Le
Capital etc.», v. І, ch. XIV, p. 156). \emph{Ред.}
}. Але разом із
змінною складовою частиною капіталу мусить зростати й стала
його частина, поряд розміру спільних умов продукції, як от
будівлі, печі й~\abbr{т. д.}, особливо мусить зростати й кількість сировинного
матеріялу, та ще й далеко швидше, ніж число робітників.
Маса сировинного матеріялу, споживана протягом даного
часу даною кількістю праці, більшає в тій самій пропорції, в
якій більшає продуктивна сила праці в наслідок поділу праці.
Отже, зріст мінімального розміру капіталу в руках поодинокого
капіталіста або щораз більше перетворювання суспільних засобів
існування та засобів продукції на капітал — це закон, що
виникає з технічного характеру мануфактури\footnote{
«Недосить того, щоб у суспільства був капітал, потрібний для
підрозділу реместв (слід було сказати: «потрібні для цього засоби
існування та засоби продукції»); крім цього, потрібно, щоб цього капіталу
нагромадилося в руках підприємців у досить значних масах, так, щоб
вони мали змогу провадити продукцію у великому маштабі\dots{} Що більше
зростає поділ праці, то щораз значнішого капіталу в знарядді,
сировинному матеріялі й~\abbr{т. д.} потребує постійне застосування того самого числа
робітників»). (\emph{Storch}: «Cours d’Economie Politique». Паризьке видання,
т. І, стор. 250, 251). «Концентрація засобів продукції та поділ
праці так само невіддільні одне від одного, як у політичній царині
концентрація суспільної влади й поділ приватних інтересів» («La
concentration des instruments de production et la division du travail sont aussi
inséparables l’une de l’autre que le sont, dans le régime politique, la
concentration des pouvoirs publics et la division des intérêts privés»).
(\emph{K.~Marx}: «Misère de la Philosophie», Paris 1847, p. 134. — \emph{K.~Маркс}:
«Злиденність філософії», Партвидав 1932~\abbr{р.}, стор. 122).
}.

В мануфактурі, як і в простій кооперації, робоче тіло, що функціонує,
є форма існування капіталу. Суспільний продукційний
механізм, складений із багатьох індивідуальних частинних робітників,
належить капіталістові. Тому продуктивна сила, що виникає
з комбінації праць, видається продуктивною силою капіталу.
Мануфактура у власному значенні не тільки підбиває під команду
й дисципліну капіталу робітника, який раніше був самостійний,
а ще й створює, крім цього, ієрархічну ґрадацію серед самих робітників.
Тимчасом як проста кооперація лишає взагалі і в цілому
\parbreak{}  %% абзац продовжується на наступній сторінці
