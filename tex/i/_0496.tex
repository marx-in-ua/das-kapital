\parcont{}  %% абзац починається на попередній сторінці
\index{i}{0496}  %% посилання на сторінку оригінального видання
товару є лише відчуження власного товару, а цей останній можна
створити лише працею. Тепер власність з’являється на боці
капіталіста як право присвоювати собі чужу неоплачену працю
або її продукт, на боці робітника — як неможливість присвоювати
собі свій власний продукт. Відокремлення власности від
праці стає доконечним наслідком закону, який за вихідний пункт
мав, як здавалося, їхню ідентичність\footnote{
Власність капіталіста на продукт чужої праці «є неминучий
наслідок того закону присвоєння, що його основним принципом було,
навпаки, виключне право власности кожного робітника на продукт його
власної праці». (\emph{Cherbuliez}: «Riche ou Pauvre», Paris 1841, p. 58; однак
y цьому творі це діялектичне перетворення розвинуто неправильно).
}.

\label{original-496}Отже, хоч і як дуже капіталістичний спосіб присвоювання
ніби контрастує первісним законам товарової продукції, проте
той спосіб виникає не з порушення цих законів, а, навпаки, із застосування
їх. Короткий огляд послідовности тих фаз руху, що
їхнім кінцевим пунктом є капіталістична акумуляція, може ще
раз нам це ясно показати.

\looseness=-1
Спочатку ми бачили, що первісне перетворення певної суми
вартости на капітал відбувалося цілком згідно з законами обміну.
Один контраґент продає свою робочу силу, другий її купує.
Перший одержує вартість свого товару, а споживна вартість
його товару, праця, таким чином відчужується другому. Тоді
цей останній перетворює вже належні йому засоби продукції
за допомогою праці, теж йому належної, на новий продукт, який
за правом також належить йому.

Вартість цього продукту містить у собі, поперше, вартість
зужиткованих засобів продукції. Корисна праця не може спожити
цих засобів продукції, не переносячи їхньої вартости на
новий продукт; але щоб бути придатною до продажу, робоча
сила мусить мати змогу давати корисну працю в тій галузі промисловости,
де її мають ужити.

Далі, вартість нового продукту містить у собі еквівалент
вартости робочої сили і якусь додаткову вартість. І це саме через
те, що робоча сила, продана на певний період — на день, тиждень
тощо, має меншу вартість, ніж та вартість, яку створює споживання
її протягом того часу. Але робітник дістав в оплату мінову
вартість своєї робочої сили й тим самим відчужив її споживну
вартість, як це завжди буває при кожній купівлі й продажу.

Та обставина, що цей осібний товар, робоча сила, має своєрідну
споживну вартість, а саме давати працю, отже, створювати
вартість, не може порушити загального закону товарової продукції.
Отже, коли сума вартости, авансована в заробітній платі, не
тільки просто знову знаходиться в продукті, а знаходиться в
ньому збільшена на якусь додаткову вартість, то це випливає
не з якогось обдурювання продавця, який одержав вартість
свого товару, а лише із споживання цього товару покупцем.

Закон обміну зумовлює лише рівність мінових вартостей
товарів, обмінюваних один на один. Він навіть припускає як
\parbreak{}  %% абзац продовжується на наступній сторінці
