\parcont{}  %% абзац починається на попередній сторінці
\index{i}{0522}  %% посилання на сторінку оригінального видання
цей капітал на число робітничої людности»\footnote{
Я нагадую тут читачеві, що категорії змінний капітал і сталий
капітал вперше почав уживати я. Політична економія, починаючи від
А.~Сміта, сплутує визначення, що містяться в них, з тими ріжницями
форм основного і обігового капіталу, що постають із процесу циркуляції.
Докладніше про це в другій книзі, другий відділ.
}. Отже, це значить,
що спочатку ми підсумовуємо дійсно заплачені індивідуальні
заробітні плати, а потім уже твердимо, що результат цього додавання
становить суму вартости «робочого фонду», дарованого
богом і природою. Нарешті, визначену таким способом суму
ділимо на число робітників, щоб знову найти, скільки пересічно
може припасти кожному робітникові індивідуально. Це надзвичайно
хитромудра процедура. Вона не заважає панові Фавсетові
за одним духом сказати: «Ціле багатство, що його щорічно
акумулюється в Англії, поділяється на дві частини. Одну
частину застосовують в Англії для підтримання нашої власної
промисловости. Другу його частину експортують до інших
країн\dots{} Та частина, що її застосовують у нашій промисловості,
становить лише незначну частину багатства, що його щорічно
акумулюється в цій країні»\footnote{
\emph{Н.~Fawsett}, там же, стор. 123, 122.
}. Отже, більшу частину додаткового
продукту, який щорічно наростає і який відбирають в англійського
робітника без еквіваленту, капіталізується не в Англії,
а в чужих краях. Алеж разом з експортованим таким чином
додатковим капіталом експортують і частину «робочого фонду»,
вигаданого богом і Бентамом»\footnote{
Можна було б сказати, що не тільки капітал, а й робітників щороку
вивозять з Англії у формі еміґрації. Однак, у тексті немає навіть і згадки
про майно переселенців, які здебільшого неробітники. Велика частина
їх сини фармерів. Англійський додатковий капітал, що його вивозять
щороку за кордон для того, щоб мати з нього проценти, становить далеко
більшу величину порівняно з щорічною акумуляцією, аніж щорічна
еміґрація порівняно з щорічним приростом людности.
}\footnote*{
Що у французькому виданні теорію «робочого фонду» і критику
її подано повніше, наводимо тут відповідні уривки з цього видання:

«Капіталісти, їхні співвласники, їхні слуги та їхній уряд щороку
прогулюють чималу частину додаткового продукту. Крім того, вони
затримують у своїх споживних запасах багато повільно споживаних
предметів, придатних для репродукції, і перетворюють на непродуктивні
багато робочих сил, уживаючи їх для своїх особистих послуг.
Отже, частина багатства, що капіталізується, ніколи не досягає тієї
величини, якої вона могла б досягти. Її відношення до сукупного суспільного
багатства змінюється при всякій зміні в поділі додаткової вартости
на особистий дохід і додатковий капітал, а пропорція, в якій відбувається
цей поділ, постійно варіює під впливом коньюнктур, що на них ми
тут не зупиняємось. Для нас досить сконстатувати, що капітал зовсім
не є наперед визначена і фіксована частина суспільного багатства, а,
навпаки, змінна і хитка його частина\dots{}

Догма, ніби суспільний капітал завжди є певна фіксована величина,
не лише суперечить найзвичайнішим явищам продукції, як от рухам її
поширення і звуження, але вона робить незрозумілою і саму акумуляцію\dots{}
Цю догму Бентам і його прихильники — Мак-Куллох, Мілл та інші —
застосовували переважно до тієї частини капіталу, що обмінюється на
робочу силу і що її вони називають «фондом заробітної плати», або «робочим
фондом». На їх погляд це є осібна частина суспільного багатства
вартість певної кількости засобів існування, що розмірам їх сама природа
завжди ставить фатальні межі, які робітнича кляса марно намагається
переступити. Отже, сума, належна до розподілу між найманими робітниками,
є наперед визначена, а звідси випливає, що коли частина, яка
дістається кожному робітникові, є занадто мала, то це тому, що число
робітників занадто велике, і що бідність робітників, кінець-кінцем,
є наслідок не суспільного ладу, а природних умов.

Насамперед, межі, які капіталістична система ставить споживанню
продуцента, є «природні» лише в умовах цієї системи, цілком так
само, як батіг функціонує як «природна» спонука лише в умовах рабства.
В дійсності природі капіталістичної продукції властиве обмеження
частини продуцента тим, що є доконечне для підтримання його робочої
сили, так само, як ій властиве і захоплення додаткового продукту
капіталістом. Природі цієї системи продукції властиве також і те, що
додатковий продукт, який дістається капіталістові, поділяється ним
самим на дохід і додатковий капітал, тимчасом як робітник може лише
у виняткових випадках збільшити свій фонд споживання коштом фонду
споживання неробітників. «Багатий, — каже Сісмонді, — диктує закони
бідним\dots{} бо він сам переводить поділ річної продукції і залишає все
те, що він зве доходом, для самого себе, а все те, що він зве капіталом,
він відступає бідним, щоб вони з цього зробили для нього дохід» (Читай:
щоб вони з цього зробили для нього додатковий дохід). (\emph{Sismondi}: «Nouveaux
Principes d’Economie Politique», v. I, p. 107--108)\dots{}

Отже, насамперед, економісти мусили б довести, що капіталістичний
спосіб суспільної продукції, не зважаючи на те, що він зовсім
недавно виник, все ж є незмінний і «природний» спосіб продукції. Але,
навіть припускаючи розміри капіталістичної системи за дані, неправда,
«що фонд заробітної плати» є наперед визначений величиною суспільного
багатства, або величиною суспільного капіталу.

Через те, що суспільний капітал є лише мінлива й хитка частина
суспільного багатства, фонд заробітної плати, являючи собою лише
певну частину цього капіталу, не може бути фіксованою і наперед визначеною
частиною суспільного багатства; з другого боку, відносна величина
фонду заробітної плати залежить від тієї пропорції, що в ній суспільний
капітал поділяється на капітал сталий і капітал змінний, а ця
пропорція, як ми вже бачили і як ми це докладно покажемо в дальших
розділах, не залишається незмінною протягом процесу акумуляції».
(«Le Capital etc.», ch. XXIV, § 5, p. 267--268). \emph{Ред.}
}.

\sectionextended{Загальний закон капіталістичної акумуляції}{%
\subsection{Зростання попиту на робочу силу разом з акумуляцією
при~незмінному складі капіталу}
}

У цьому розділі ми розглядаємо той вплив, що його справляє
зростання капіталу на долю робітничої кляси. Найважливіший
фактор при цьому дослідженні — це склад капіталу й зміни,
що їх він зазнає в перебігу процесу акумуляції.

Склад капіталу треба розуміти в двоякому значенні. З боку
вартости він визначається тим відношенням, що в ньому капітал
поділяється на сталий капітал, або вартість засобів продукції,
і на змінний капітал, або вартість робочої сили, загальну суму
заробітних плат. З боку речовини, що функціонує в процесі
продукції, кожний капітал поділяється на засоби продукції й
живу робочу силу; з цього боку склад капіталу визначається відношенням
\index{i}{0524}  %% посилання на сторінку оригінального видання
між масою застосованих засобів продукції, з одного
боку, і масою праці, потрібного для застосування тих засобів —
з другого. Перший я називаю вартостевим складом капіталу, а
другий — технічним складом капіталу. Між тим і другим існує
щільне взаємовідношення. Щоб висловити це взаємовідношення,
я називаю органічним складом капіталу його вартостевий склад,
оскільки останній визначається його технічним складом і відбиває
зміни технічного складу. Де говориться просто про склад капіталу,
там треба завжди розуміти його органічний склад.

Численні поодинокі капітали, вкладені в певну галузь продукції,
більш або менш відрізняються між собою щодо складу.
Пересіччя їхніх поодиноких складів дає нам склад цілого капіталу
цієї галузі продукції. Нарешті, загальне пересіччя цих пересічних
складів усіх галузей продукції дає нам склад суспільного
капіталу якоїсь країни, і тільки про нього в останній інстанції
й буде далі мова.

Зростання капіталу включає і зростання його змінної, або
перетвореної на робочу силу складової частини. Частина додаткової
вартости, перетвореної на додатковий капітал, мусить завжди
знову перетворюватися на змінний капітал, або на додатковий
робочий фонд. Коли ми припустимо, що разом з іншими незмінними
обставинами незмінним лишається і склад капіталу, тобто,
що завжди потрібно тієї самої маси робочої сили для того, щоб
пустити в рух певну масу засобів продукції, або сталого капіталу,
то в такому випадку попит на працю й фонд засобів існування
робітників, очевидно, зростатиме пропорційно до зросту капіталу,
і то швидше, що швидше зростатиме капітал. Через те, що
капітал щорічно продукує додаткову вартість, частину якої
щороку додається до первісного капіталу, через те, що сам цей
приріст щороку зростає із збільшенням розміру капіталу, який
уже функціонує, і через те, насамкінець, що при особливому збудженні
жадоби до збагачення, як от, наприклад, при відкритті
нових ринків, нових сфер вкладення капіталу в наслідок розвитку
нових суспільних потреб і~\abbr{т. ін.}, маштаб акумуляції можна раптом
поширити самою лише зміною поділу додаткової вартости
або додаткового продукту на капітал і дохід, — через це потреби
акумуляції капіталу можуть випередити зріст робочої сили або
число робітників, попит на робітників — випередити подання
їх, і тому заробітні плати можуть підвищитися. Це, кінець-кінцем,
навіть мусить статися, якщо вищенаведені передумови й
далі існують без зміни. Через те, що кожного року вживається
більше робітників, ніж у попередньому році, то раніш або пізніш
мусить настати момент, коли потреби акумуляції починають
переростати звичайне подання праці, коли, отже, настає підвищення
заробітної плати. Нарікання на це лунають в Англії
протягом цілого XV й першої половини XVIII століття. Однак
більш або менш сприятливі обставини, при яких наймані робітники
зберігаються й розмножуються, не змінюють нічого в основному
характері капіталістичної продукції. Як проста репродукція
\index{i}{0525}  %% посилання на сторінку оригінального видання
безупинно репродукує саме капіталістичне відношення,
капіталістів на одному боці, робітників на другому, так і репродукція
в поширеному маштабі, або акумуляція, репродукує
капіталістичне відношення в поширеному маштабі — більше капіталістів
або більших капіталістів на цьому полюсі, більше
найманих робітників на тому. Репродукція робочої сили, що невпинно
мусить входити до складу капіталу як засіб збільшення
його вартости і не може відокремитись від нього, — робочої сили,
що її підлеглість капіталові лише маскується зміною індивідуальних
капіталістів, яким вона продає себе, становить, в дійсності,
момент репродукції самого капіталу. Отже, акумуляція
капіталу є збільшення пролетаріяту\footnote{
\emph{К.~Marx}: «Lohnarbeit und Kapital». (\emph{K.~Маркс}: «Наймана праця
і капітал», Партвидав «Пролетар» 1932). — «За однакового пригнічення
мас що більше в країні пролетарів, то вона багатша» («A égalité
d’oppression des masses, plus un pays a de prolétaires et plus il est riche»).
(\emph{Colins}: «L’Economie Politique, Source des Révolutions et des Utopies
prétendues Socialistes», Paris 1857, vol. III, p. 331). Під «пролетарем»
y політичній економії треба розуміти не що інше, як найманого робітника,
який продукує «капітал» і збільшує його вартість і якого викидають
на брук, скоро тільки він стає зайвим для потреб самозростання
«пана капіталу», як називає цю персону Пекер. «Хоробливий пролетар
пралісу» — це лише чемна фантазія Рошера. Пралісовик є власник того
пралісу й поводиться з пралісом як із своєю власністю, так само безцеремонно
як оранґутанґ. Отже, він не є пролетар. Він був би ним тільки
тоді, коли б праліс експлуатував його, а не він — цей праліс. Щождо
стану його здоров'я, то він витримає порівняння не тільки з сучасним
пролетарем, а й з сифілітичними й золотушними «порядними особами».
А втім, під пралісом пан Вільгельм Рошер розуміє, певно, свою рідну
Lüneburger Heide.
}.

Клясична політична економія так добре розуміла цю тезу,
що Адам Сміт, Рікардо й інші, як уже згадано раніш, навіть помилково
ототожнюють акумуляцію із споживанням всієї капіталізованої
частини додаткового продукту продуктивними робітниками,
або з перетворенням її на додаткових найманих робітників.
Уже 1696~\abbr{р.} Джон Беллерс каже: «Коли б якась людина
мала \num{100.000} акрів і стільки ж фунтів стерлінґів грошей і стільки
ж худоби, то чим була б ця багата людина без робітників, як не
робітником? А через те, що робітники роблять людей багатими,
то що більше робітників, то більше багатих\dots{} Праця бідних — то
копальні багатих»\footnote{
«As the Labourers make men rich, so the more Labourers, there
will be the more rich men\dots{} the Labour of the Poor being the Mines of the
Rich». (\emph{John Bellers}: «Proposals for raising a Colledge of Industry»,
London 1696, p. 2).
}. Те саме каже й Бернар де Мандевіль
на початку XVIII віку: «Де власність має достатній захист, там
легше було б жити без грошей, ніж без бідних, бо хто ж тоді
працював би?.. Так само, як робітників треба захищати від голодної
смерти, так само не повинні вони одержувати нічого
такого, що варто заощаджувати. Якщо інколи хтось із найнижчої
кляси через незвичайну працьовитість та недоїдання підноситься
понад той стан, у якому він виріс, то ніхто не сміє перешкоджати
йому в цьому: аджеж безперечно, що наймудріша річ
\parbreak{}  %% абзац продовжується на наступній сторінці
