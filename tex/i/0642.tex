робітників і коштом свого лендлорда. Отже, немає нічого дивного
в тому, що наприкінці XVI віку Англія мала клясу багатих
для тодішніх часів «капіталістичних фармерів».\footnote{
У Франції régisseur, на початку середніх віків управитель і збирач
повинностей на користь февдальним панам, швидко стає homme
d’affaires,\footnote*{
— ділком. \emph{Ред.}
} що за допомогою вимагання, шахрайства й т. ін., виростає
на капіталіста. Ці régisseurs сами були іноді вельможними панами. Наприклад:
«Оцей рахунок пан Жак де Торес, лицар-кастелян у
Безансоні, подає своєму панові, що веде рахунки в Діжоні для пана
герцога і графа Бургундського, про ренти, які належались від зазначеного
кастелянства від 25 грудня 1359р. до 28 грудня 1360 р». («C’est
li compte que messire Jacques de Thoraine, chevalier chastelain sor Besançon
rent es seigneur tenant les comptes à Dijon pour monseigneur le
duc et comte de Bourgoigne, des rentes appartenant à la dite chastellenie,
depuis XXVe jour de décembre MCCCLIX jusq’au XXVIIIe jour de décembre
MCCCLX»). (Alexis Monteil: «Traité des Matériaux, manuscrits etc.» v. I,
p. 234 і далі). Вже тут видно, як у всіх сферах суспільного життя левина
пайка попадає до рук посередників. Наприклад, в економічній царині фінансисти,
биржовики, купці, дрібні крамарі збирають вершки з усіх справ;
у царині громадського права адвокат скубе супротивників; у політиці
депутат має більше значення, ніж виборець, міністер — більше ніж суверен
; у релігії бога відсувають на задній плян святі «посередники», а цих
останніх знову таки витискують попи, які й собі є неминучі посередники
між добрим пастирем і його вівцями. У Франції, як і в Англії, великі
февдальні території були поділені на безліч дрібних господарств, але
на умовах куди менш сприятливих для сільської людности. В XIV столітті
виникли оренди, фарми або terriers. Число їх невпинно зростало і
значно перевищило 100.000. Вони платили грішми або in natura земельну
ренту, що її розмір коливався від \sfrac{1}{12} до \sfrac{1}{5} продукту. Terriers називалися
денами, підленами і т. ін. (fiefs, arrière-fiefs), залежно від вартости й
розміру доменів, що з них деякі мали лише декілька арпенів. Всі ці
terriers мали в тій або іншій мірі судову владу над людністю; такої влади
було чотири ступені. Можна зрозуміти, який гніт відчурала сільська
людність під владою усіх цих дрібних тиранів. Монтейль каже, що у Франції
за тих часів було 160.000 судів там, де тепер досить 4.000 судових установ
(залічуючи сюди й мирових суддів).
}

5. Зворотний вплив рільничої революції на промисловість.
Утворення внутрішнього ринку для промислового капіталу

Експропріяція і зганяння сільської людности, що відбувалися
поштовхами й постійно відновлювалися, постачали, як ми бачили,
для міської промисловости щораз нові маси пролетарів, які
стояли цілком поза цеховими відносинами, — мудра обставина,
яка примушує старого А. Андерсона (не треба сплутувати його
з Дшемсом Андерсоном) в його історії торговлі увірувати в безпосереднє
втручання провидіння. Ми мусимо ще хвилину спинитися
на цьому елементі первісної акумуляції. Розрідженню незалежної,
самостійно господарюючої сільської людности відпові-

the ne we — that is, they pays for their lande good cheape, and sell all things
growing thereof deare...» Knight: «What sorte is that which, ye sayde
should have greater losse hereby, than these men had profit? — Doktor:
«It is all noblemen, gentlemen, and all other that live either by a stinted
rent or stypend, or do not manure (cultivate) the ground, or doe occupy no
buying and selling»).