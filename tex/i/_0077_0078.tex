
\index{i}{0077}  %% посилання на сторінку оригінального видання
За незмінних товарових цін маса засобів циркуляції може
зростати в наслідок того, що збільшується маса товарів, які
циркулюють, або в наслідок того, що зменшується швидкість
обігу грошей, або тому, що обидві обставини діють разом. Навпаки,
маса засобів циркуляції може меншати із зменшенням
маси товарів або із зростом швидкости циркуляції.

За загального підвищення товарових цін маса засобів циркуляції
може лишатися незмінна, коли маса товарів, що циркулюють,
меншає в тій самій пропорції, у якій зростає їхня ціна, або коли
швидкість обігу грошей збільшується так само хутко, як і зріст цін,
тимчасом як маса товарів, що циркулюють, лишається та сама.
Маса засобів циркуляції може падати тому, що маса товарів
зменшується або швидкість обігу збільшується швидше, ніж ціни.

За загального зниження товарових цін маса засобів циркуляції
може лишатися незмінна, коли маса товарів зростає в тій самій
пропорції, що в ній падає їхня ціна, або коли швидкість обігу
грошей зменшується в тій самій пропорції, що й ціни. Маса засобів
циркуляції може зростати, коли маса товарів зростає або швидкість
циркуляції зменшується швидше, ніж падають товарові ціни.

Варіяції різних факторів можуть взаємно компенсуватися,
так що наперекір їхній постійній несталості загальна сума товарових
цін, яка має бути зреалізована, отже, і маса грошей, що
циркулюють, лишається стала. Тому ми й бачимо, особливо при
розгляді порівняно довших періодів, значно сталіший пересічний
рівень маси грошей, що циркулює у кожній країні та, — за
винятком часів значних пертурбацій, що періодично виникають
із промислових і торговельних криз, рідше із зміни вартости
самих грошей, — куди менші відхилення від цього пересічного
рівня, ніж можна було сподіватися на перший погляд.

Той закон, що кількість засобів циркуляції визначається
сумою цін товарів, які є в циркуляції, і пересічною швидкістю
грошового обігу\footnote{
«Щоб провадити торговлю, нація потребує грошей у певній мірі
або пропорції: більша або менша проти потрібної кількість грошей зашкодила
б торговлі. Зовсім так само, як у дрібній торговлі, потрібна певна
кількість фартинґів, щоб розміняти срібну монету або провадити такі
виплати, яких не можна перевести навіть за допомогою найдрібніших
срібних монет\dots{} І подібно до того, як пропорція числа фартинґів, потрібних
для торговлі, залежить від числа осіб, що вживають їх, або частости
розміну їх, а також — і це насамперед — від вартости щонайдрібнішої
срібної монети, таким саме чином і кількість грошей (золотих і срібних
монет), потрібних для торговлі, визначається частістю операцій і розмірами
виплат». («There is a certain measure, and proportion of money requisite
to drive the trade of a nation, more or less than which, would prejudice
the same. Just as there is a certain proportion of farthings necessary in a
small retail Trade, to change silver money, and to even such reckonings
as cannot be adjusted with the smallest silver pieces\dots{} Now as the proportion
of the number of farthings requisite in commerce is to be taken from
the number of people, the frequency of their exchanges, as also, and principally,
from the value of the smallest silver pieces of money; so in like mannerthe
proportion of money (gold and silver specie) requisite to our trade,
is to be like wise taken from the frequency of commutations, and from
the bigness of payments»). (\emph{William Petty}: «A.~Treatise on Taxes and
Contributions», London 1667, p. 17). Теорію Юма боронив супроти
Дж.~Стюарта й інших А.~Юнґ у своїй «Political Arithmetic», London
1774, де їй присвячено окремий розділ: «Prices depend on quantity of
money», p. 112 і далі. У своїй «Zur Kritik der Politischen Oekonomie»
S. 149 (ДВУ, 1926~\abbr{р.}, стор. 178) я роблю таку увагу: «Питання про кількість
монет, що циркулюють, він (А.~Сміс) мовчки усуває, цілком неправильно
розглядаючи гроші як простий товар». Але це стосується лише
до тих місць, де А.~Сміс розглядає гроші ex officio\footnote*{
з обов’язку. \emph{Peд.}
}. Однак, принагідно,
приміром, у критиці попередніх систем політичної економії, він висловлюється
правильно: «Кількість грошей у кожній країні реґулюється вартістю
тих товарів, що їх вони пускають у рух\dots{} Вартість благ, що їх щорічно
купують і продають у якійсь країні, потребує певної кількости грошей
для того, щоб пустити блага в обіг і розподілити поміж їхніми споживачами;
вжити більшої суми грошей не можна. Канал циркуляції
неминуче притягає до себе суму, якої досить для того, щоб наповнити
його, і ніколи не допускає жодного надміру». («The quantity of coin in
every country is regulated by the value of the commodities which are to
be circulated by it\dots{} The value of goods annually bought and sold in any
country requires a certain quantity of money to circulate and distribute
them to their proper consumers, and can give employment to no more. The
channel of circulation necessarily draws to itself a sum sufficient to fill
it, and nevet admits any more»). (Wealth of Nations, b. IV, ch. 1). Подібно
до цього А.~Сміс починає свій твір ex officio апотеозою поділу праці. Наприкінці,
в останній книзі про джерела державних доходів, він принагідно
поновлює напади свого вчителя А.~Ферґюсона на поділ праці.
}, можна висловити й так: за даної суми вартости
товарів і за даної пересічної швидкости їхніх метаморфоз,
\index{i}{0078}  %% посилання на сторінку оригінального видання
кількість грошей в обігу, або грошового матеріялу, залежить від
його власної вартости. Ілюзія, що товарові ціни, навпаки, визначаються
масою засобів циркуляції, а ця остання, своєю чергою,
масою грошового матеріялу, що є в країні\footnote{
«Ціни продуктів кожної нації, певна річ, зростатимуть у міру
того, як зростатиме кількість золота й срібла серед народу; отже, коли
кількість золота й срібла, яку має якась нація, меншає, то й ціни на всі
продукти мусять спадати пропорційно до цього зменшення кількости грошей»
(«The prices of things will certainly rise in every nation, as the gold
and silver increase amongst the people: and, consequently, where the gold
and silver decrease in any nation, the prices of all things must fall proportionably
to such decrease of money»). (\emph{Jacob Vanderlint}: «Money answers
all Things», London 1734, p. 5). Ближче порівнання праці Вандерлінта
й «Essays» Юма не лишає в мені найменшого сумніву, що Юм
знав і використав цей, зрештою, значний твір Вандерлінта. Погляд, що
маса засобів циркуляції визначає ціни, находимо і в Барбона та ще давніших
письменників. «Жодної невигоди, — каже Вандерлінт, — не може
постати з нічим не обмеженої торговлі, а тільки велика користь\dots{} бо коли
грошова готівка якоїсь нації зменшуватиметься під впливом вільної
торговлі, — а цьому повинні стати на перешкоді заборони, — то в тих націй,
що до них припливає готівка, неминуче зростатимуть ціни на всі
речі відповідно до збільшення грошової готівки. А\dots{} продукти наших мануфактур
і всякі інші продукти дійдуть тут швидко таких низьких цін,
що торговельний балянс повернеться на нашу користь і гроші знову припливатимуть
назад до нас». («No inconvenience can arise by an unrestrained
trade, but very great advantage\dots{} since, if the cash of the nation be decreased
by it, which prohibitions are designed to prevent, those nations that
get the cash will certainly find every thing advance in price, as the cash
increases amongst them. And\dots{} our manufactures and every thing else, will
soon become so moderate as to turn the balance of trade in our favour, and thereby
fetch the money back again»). («Money answers all Things», p. 44).
}, має свої коріння у
\parbreak{}  %% абзац продовжується на наступній сторінці
