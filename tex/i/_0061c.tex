\parcont{}  %% абзац починається на попередній сторінці 
\index{i}{0061}  %% посилання на сторінку оригінального видання 
уявлюваного золота. Товар треба замінити золотом, щоб він міг
для свого посідача служити за загальний еквівалент. Коли б
посідач заліза зустрівся, приміром, з посідачем товару людських
веселощів і зауважив йому, що ціна заліза є грошова форма,
то цей веселун відповів би йому, як св. Петро відповів у небі
Дантові, що перечитував перед ним символ віри:

Assai bene è trascorsa

D’esta moneta già la lega e’l peso,

Ma dimmi se tu l’hal nela tua borsa.\footnote*{
Ми добре знаємо усі

Монети тої склад, її вагу.

Але скажи, чи маєш ти її в кишені?
}

Форма ціни включає відчужування товарів за гроші й доконечність
такого відчужування. З другого боку, золото функціонує
як ідеальна міра вартости лише тому, що воно вже обертається
як грошовий товар у процесі обміну. Отже, в ідеальній
мірі вартостей криється дзвінка монета.

2. Засіб циркуляції

а) Метаморфоза товарів

Ми бачили, що процес обміну товарів включає відношення,
які суперечать одне одному й одне одного виключають. Розвиток
товару не усуває цих суперечностей, але створює форму, що в
ній вони можуть рухатися. ** Це взагалі та метода, що за її допомогою
розв’язуються дійсні суперечності. Це є, приміром, суперечність,
що одне тіло падає безперестанно на інше й так само безперестанно
віддаляється від нього. Еліпса є одна з форм руху,
що в ній ця суперечність одночасно і здійснюється і розв’язується.

Оскільки процес обміну переносить товари з рук, де вони є
неспоживні вартості, до рук, де вони є споживні вартості, він
є суспільний обмін речовин. Продукт одного корисного роду
праці заміщує продукт іншого корисного роду праці. Дійшовши
до того пункту, де він служить як споживна вартість, товар
із сфери товарового обміну переходить у сферу споживання.
Нас тут інтересує тільки перша сфера. Отже, ми повинні розглянути
цілий процес з боку його форми, отже, лише зміну форм,
або метаморфозу товарів, що упосереднює суспільний обмін
речовин.

Цілком недостатнє розуміння цієї зміни форми викликається,
лишаючи осторонь неясність самого поняття вартости, тією обставиною,
що кожна зміна форми одного товару відбувається в
процесі обміну двох товарів — простого товару й грошового
товару. Коли звертають увагу лише на цей речовий момент — на

* * У французькому виданні цю фразу подано так: «Розвиток товару,
який призводить до того, що товар з’являється як щось двоїсте як споживна
вартість і мінова вартість, не усуває цих суперечностей, але утворює
форму, що в ній вони можуть рухатися». («Le Capital etc.», v. I. ch. IІІ,
p. 43). Рeд.
\parbreak{}  %% абзац продовжується на наступній сторінці
