\parcont{}  %% абзац починається на попередній сторінці
\index{i}{0551}  %% посилання на сторінку оригінального видання
залежить від тиску відносного перелюднення; отже, скоро тільки вони намагаються за допомогою
тред-юньйонів і~\abbr{т. ін.} організувати пляномірну взаємодію занятих і незанятих робітників, щоб знищити
або послабити руйнаційні для їхньої кляси наслідки того природного закону капіталістичної
продукції, — так капітал і його сикофант, політико-економ, зчиняють галас про порушення «вічного» і, так би
мовити, «святого» закону попиту й подання. Аджеж усякий зв’язок між занятими й незанятими
порушує, мовляв, «чисте» діяння того закону. А, з другого боку, скоро тільки, наприклад, у колоніях,
несприятливі обставини перешкоджають створенню промислової резервної армії, а разом з нею і
абсолютної залежности робітничої кляси від кляси капіталістів, то капітал і його банальний
Санчо-Панчо\footnote*{
дієва особа з роману Сервантеса «Дон-Кіхот». \emph{Ред.}
} підіймають
бунт проти «святого» закону попиту й подання і намагаються приборкати його примусовими засобами.

\subsection{Різні форми існування відносного перелюднення. Загальний закон капіталістичної акумуляції}

Відносне перелюднення існує в усяких можливих відтінках. Кожний робітник належить до нього протягом
того часу, коли він напівзанятий або зовсім незанятий. Якщо залишити осторонь ті великі форми
перелюднення, що періодично повторюються,
форми, що їх надає перелюдненню зміна фаз промислового циклу, так що воно буває то гостре, підчас
криз, то хронічне, підчас застою, — то воно постійно має три форми: текучу, лятентну і застійну.

В центрах сучасної промисловости — фабриках, мануфактурах, металюрґійних заводах, копальнях і~\abbr{т. д.}
— робітників то відштовхують, то знову в більшому розмірі притягують, так що взагалі і в цілому
число занятих більшає, хоч і в щораз
меншій пропорції порівняно з маштабом продукції. Перелюднення існує тут у текучій формі.

Так на фабриках у власному значенні, як і по всіх великих майстернях, де машини відіграють певну
ролю або, принаймні, заведено сучасний поділ праці, потрібна маса робітників чоловічої статі, що не
дійшли ще юнацького віку. Коли ці робітники доходять цього віку, то тільки дуже мале число з них
лишається в тих самих галузях промисловости, а більшість із них реґулярно звільняють. Вони
становлять той елемент текучого перелюднення, що зростає разом із зростанням розмірів промисловости.
Частина з них еміґрує й фактично лише мандрує слідком за тим капіталом, що еміґрує. Один із
наслідків цього є те, що жіноча людність зростає швидше, ніж чоловіча, як про це свідчить Англія. Та
обставина, що природний приріст робітничої маси не насичує
потреб акумуляції капіталу і проте разом з тим їх перевищує, —
\parbreak{}  %% абзац продовжується на наступній сторінці
