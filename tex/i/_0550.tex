\parcont{}  %% абзац починається на попередній сторінці
\index{i}{0550}  %% посилання на сторінку оригінального видання
з великих учинків економічної апологетики. Пригадаймо собі,
що коли через заведення нових або поширення старих машин
частину змінного капіталу перетворюється на сталий, то що
операцію, яка «зв’язує» капітал і саме тим «звільняє» робітників,
економіст-апологет пояснює, навпаки, таким способом,
ніби вона звільняє капітал для робітника. Лише тепер ми можемо
цілком оцінити безсоромність апологета. Звільняється не тільки
континґент робітників, що їх безпосередньо витискує машина,
але так само і їхніх заступників і той додатковий континґент,
що його реґулярно поглинається за звичайного поширення підприємства
на старій базі. Всі вони тепер «звільнені», і кожний
новий капітал, що має охоту функціонувати, може порядкувати
ними. Чи притягає він цих або інших робітників, вплив цього
на загальний попит на працю дорівнюватиме нулеві, доки цього
нового капіталу вистачатиме якраз на те, щоб звільнити ринок
саме від стількох робітників, скільки їх на нього викинули машини.
Якщо він притягає до праці менше число робітників, то
число зайвих робітників зростає; а якщо більше, то загальний
попит на працю зростає лише настільки, наскільки число робітників,
притягнутих до праці, перевищує число «звільнених».
Отже, те збільшення попиту на працю, яке взагалі спричинили б
додаткові капітали, що шукають застосовання, в усякому разі
невтралізується в тій мірі, в якій для цього вистачає робітників,
повикидуваних машиною на брук. Це значить, отже, що механізм
капіталістичної продукції дбає про те, щоб абсолютний приріст
капіталу не супроводився відповідним підвищенням загального
попиту на працю. І це апологет називає компенсацією за злидні,
страждання й можливу загибіль позвільнюваних робітників підчас
тих переходових періодів, що заганяють їх у ряди промислової
резервної армії! Попит на працю не є ідентичний із зростанням
капіталу, подання праці не є ідентичне із зростанням робітничої
кляси, так що тут немає взаємного діяння однієї на одну двох
незалежних одна від одної сил. Les dés sont pipés\footnote*{
Кості до гри підроблено. \Red{Ред.}
}. Капітал
одночасно діє і в тому, і в тому напрямі. Якщо його акумуляція,
з одного боку, збільшує попит на працю, то, з другого боку,
вона збільшує подання робітників через «звільнювання» їх,
тимчасом як тиск незанятих робітників одночасно примушує
занятих давати більше праці, отже, робить подання праці до
певного ступеня незалежним від подання робітників. Рухом закону
попиту й подання праці на цій базі вивершується деспотизм
капіталу. Тим то, скоро тільки робітники доходять таємниці,
яким чином в міру того, як вони більше працюють, більше продукують
чужого багатства, в міру того, як зростає продуктивна
сила їхньої праці, — яким чином стається, що в міру всього цього
навіть їхня функція як засіб самозростання капіталу стає для
них чимраз більш непевною; скоро тільки вони відкривають,
що ступінь інтенсивности конкуренції серед них самих цілком
\parbreak{}  %% абзац продовжується на наступній сторінці
