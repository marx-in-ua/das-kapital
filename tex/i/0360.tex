розвиток цієї системи, що відбувався під тиском скорочення
робочого дня. Але хто міг би 1860 р., року зенітного розвитку
англійської бавовняної промисловости, передбачати ті чимраз
швидші поліпшення машин і відповідне витискування ручної
праці, що їх викликали три наступні роки під тиском американської
громадянської війни? Щодо цього пункту тут досить кількох
прикладів з офіціяльних даних англійських фабричних інспекторів.
Один менчестерський фабрикант заявляє: «Замість 75 чухральних
машин ми потребуємо тепер лише 12, і вони дають нам
таку саму кількість продуктів такої самої, якщо не ліпшої,
якости... Заощадження на заробітній платі становить 10 фунтів
стерлінґів на тиждень, заощадження на відпадках бавовни — 10\%».
В одній менчестерській тонкопрядільні «через прискорення руху
й заведення різних автоматичних (self-acting) процесів усунено
в одному відділі 1/4, у другому більш ніж 1/2 робітничого персоналу,
тимчасом як чесальна машина, що заступила другу чухральну
машину, дуже зменшила число робітників, занятих раніш у чухральному
відділі». Інша прядільна фабрика оцінює свої загальні заощадження
на «руках» у 10\%. Панове Джілмер, фабриканти-прядільники
в Менчестері, заявляють: «Заощадження в нашому відділі
blowing (чищення бавовни) на руках та заробітній платі, досягнуті
в наслідок заведення нових машин, ми оцінюємо у цілу третину...
у відділах jack frame і drawing frame room витрати на руки та інші
видатки зменшились приблизно на 1/3, у прядільному відділі видатки
зменшились приблизно на 1/3. Але це не все: якщо наша пряжа
йде тепер до ткача, то в наслідок застосування нових машин її
так дуже поліпшено, що ткачі продукують більше та ліпші тканини,
аніж із колишньої машинової пряжі».\footnote{
«Reports of Insp. of Fact, for 31 st October 1863», p. 108 і далі.
} Фабричний інспектор
А. Редґрев додає до цього: «Зменшення числа робітників
при збільшенні продукції швидко проґресує; по вовняних фабриках
недавно знову почалося зменшення рук, і це зменшення триває
далі; перед кількома днями один учитель, що мешкає коло Рочделя,
сказав мені, що величезне зменшення школярок по дівочих школах
зумовлене не тільки натиском кризи, а ще й тими змінами
в машинах вовняної фабрики, наслідком яких там сталося зменшення
рук пересічно на 70 робітників половинного часу».\footnote{
Там же, стор.109. Швидке поліпшення машин підчас бавовняної
кризи дозволило англійським фабрикантам зараз же по скінченні американської
громадянської війни знову миттю переповнити світовий ринок.
Уже в останні шість місяців 1866 р. тканин майже не можна було продати.
Тоді почався вивіз товарів у Китай та Індію на комісію, що, природно,
зробило «glut»\footnote*{
— пересичення ринку. Ред.
} ще інтенсивнішим. На початку 1867 р. фабриканти вдалися
до свого звичайного зарадчого способу, до зниження заробітної
плати на 5\%. Робітники опирались та заявили, теоретично цілком правильно,
що єдине, чим тут можна зарадити, — це працювати скорочений
час, чотири дні на тиждень. Після довгих вагань капітани промисловости,
як вони сами називали себе, — змушені були згодитися на це, подекуди
із зниженням заробітної плати на 5\%, подекуди без зниження її.
}