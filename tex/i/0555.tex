Закон, що за ним щораз більшу масу засобів продукції, у
наслідок проґресу продуктивности суспільної праці, можна пускати
в рух із щораз меншою витратою людської сили, — цей
закон на капіталістичній основі, де не робітник уживає засобів
праці, а засоби праці вживають робітника, виражається в тому,
що, чим вища продуктивна сила праці, тим більший тиск робітників
на засоби їхньої роботи, отже, тим непевніша умова їхнього
існування: продаж власної сили для збільшування чужого багатства
або для самозростання капіталу. Отже, швидше зростання
засобів продукції і продуктивности праці, швидше, ніж
зростання продуктивної людности, виражається за капіталізму,
навпаки, в тому, що робітнича людність завжди зростає швидше,
ніж потреби самозростання капіталу.

У четвертому відділі, аналізуючи продукцію відносної додаткової
вартости, ми бачили, що за капіталістичної системи всі
методи підвищення суспільної продуктивної сили праці відбуваються
коштом індивідуального робітника; всі засоби для розвитку
продукції перетворюються на засоби поневолення й експлуатації
продуцента, калічать робітника, роблячи з нього несповналюдину,
принижують його до стану додатку до машини, з муками
його праці знищують і її зміст, відчужують від робітника духовні
сили процесу праці в тій самій мірі, в якій наука сполучається
з цим останнім як самостійна сила; вони спотворюють умови, серед
яких працює робітник, підбивають його підчас процесу праці
під якнайдріб’язковішу, ненависну деспотію, ціле його життя
перетворюють на робочий час, його жінку й дітей кидають під
джеґґернавтові колеса капіталу. Але всі методи продукції додаткової
вартости є разом з тим методи акумуляції, і всяке поширення
акумуляції стає, навпаки, засобом розвитку цих метод.
Звідси випливає, що в міру того, як акумулюється капітал,
становище робітника мусить гіршати, хоч яка б була його плата —
висока чи низька. Нарешті, той закон, що завжди тримає відносне
перелюднення, або промислову резервну армію, в рівновазі
з розмірами й енерґією акумуляції, приковує робітника до капіталу
міцніше, аніж молот Ґефеста прикував Прометея до скелі.
Цей закон зумовлює акумуляцію злиднів, що відповідає акумуляції
капіталу. Отже, акумуляція багатства на одному полюсі
є разом з тим акумуляція злиднів, мук праці, рабства, неуцтва,
здичавіння й моральної деґрадації на протилежному полюсі,
тобто на боці тієї кляси, що продукує свій власний продукт як
капітал.

Цей антагоністичний характер капіталістичної акумуляції\footnote{
«З дня на день стає ясніше, що відносини продукції, в яких
рухається буржуазія, мають не однорідний, простий характер, а двоїстий
характер; що в тій самій пропорції, в якій продукується багатство,
продукуються і злидні; що в тій самій пропорції в якій відбувається розвиток
продуктивних сил, розвивається й сила поневолення; що ці відносини
продукують буржуазне багатство, тобто багатство буржузної кляси, лише
постійно знищуючи багатство поодиноких членів цієї кляси і створюючи
}
зазначали в різних формах політико-економи, хоч вони почасти