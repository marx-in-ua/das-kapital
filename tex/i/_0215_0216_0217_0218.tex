\index{i}{0215}  %% посилання на сторінку оригінального видання
«Коли святкування сьомого дня в тижні вважається за божу
установу, то це містить у собі те, що інші дні в тижні належать
праці (він має на думці — капіталові, як ми це зараз побачимо),
і примус, щоб цей божий наказ було виконано, не можна ганьбити
як жорстокість\dots{} Що людство взагалі з природи має нахил до
комфорту і лінощів, про те переконує нас фатальний досвід із
поведінкою нашої мануфактурної черні, яка працює пересічно
не більш як 4 дні на тиждень, за винятком випадків подорожчання
засобів існування\dots{} Припустімо, що бушель пшениці репрезентує
всі засоби існування робітника й коштує 5\shil{ шилінґів}, а робітник
заробляє своєю працею денно 1\shil{ шилінґ.} Тоді йому треба працювати
лише 5 днів на тиждень і лише 4 дні, коли бушель коштує
4\shil{ шилінґи}\dots{} Але що в цьому королівстві заробітна плата багато
вища порівняно з ціною засобів існування, то мануфактурний
робітник, який працює 4 дні, має надлишок грошей, і за нього
він решту тижня живе бездільно\dots{} Сподіваюся, сказаного досить,
щоб з’ясувати, що помірна праця протягом шістьох днів на тиждень
не є рабство. Наші рільничі робітники працюють саме так,
і з усього видно, що вони найщасливіші серед робітників (labouring
poor)\footnote{
«An Essay etc.». Він сам оповідає на стор. 96, у чому було вже
1770~\abbr{р.} «щастя» англійських рільничих робітників. «Їхня робоча сила
(«their working powers») завжди напружена до краю («on the stretch»);
вони не можуть ні жити гірше, ніж живуть («they cannot live cheaper
that they do»), ні важче працювати» («nor work harder»).
}, а голляндці працюють стільки ж у мануфактурах і мають
вигляд дуже щасливого народу. Французи, за винятком безлічі
свят, які переривають робочий час, працюють стільки ж\dots{}\footnote{
Протестантизм відіграє значну ролю в генезі капіталу вже хоч
би тим, що він перетворив майже всі традиційні свята на робочі дні.
}
Але наша чернь вбила собі в голову idée fixe\footnote*{
— невідступну ідею. \emph{Ред.}
}, що їй, як англійцям,
належить за правом народження привілей користуватися більшою
волею й незалежністю, ніж [робітничій людності], у будь-якій
іншій європейській країні. Оскільки ця ідея впливає на
хоробрість наших солдатів, вона може мати деяку користь; але
що менш її мають мануфактурні робітники, то краще для них
самих і для держави. Робітники ніколи не сміють вважати себе
незалежними від своїх начальників («independent of their superiors»)\dots{}
Надзвичайно небезпечно потурати наволочі в комерційній
державі — такій, як ось наша, де може \sfrac{7}{8} цілої людности
має лише невеличку власність або зовсім її не має\dots{}\footnote{
«An Essay etc.», р. 15, 41, 96, 97, 55, 57.
} Повного
видужання не може бути доти, доки наша промислова біднота не
згодиться працювати 6 днів за ту саму суму, яку вона тепер заробляє
за 4 дні»\footnote{
Там же, стор. 69. Ще 1734~\abbr{р.} Джекоб Вандерлінт пояснив, що таємниця
всіх нарікань капіталістів на лінощі робітничої людности є просто
в тому, що капіталісти хотіли дістати за ту саму заробітну плату шість
робочих днів замість чотирьох.
}. З цією метою, так само як і для того, щоб
«викоренити ледарство, розпусту та романтичні химери про волю»,
тобто, щоб «зменшити видатки на бідних, підтримати дух підприємливости
\index{i}{0216}  %% посилання на сторінку оригінального видання
і знизити ціну праці в мануфактурах», наш вірний
Екарт капіталу пропонує випробований засіб, а саме: таких
робітників, які вдаються до публічної добродійности, словом,
павперів, замикати до «ідеального робітного дому (an idéal Workhouse).
«Такий дім треба зробити домом жаху» (House of Тerror)\footnote{
Там же, р. 242: «Такий ідеальний робітний дім треба зробити «домом
жаху», а не притулком для бідних, де вони дістають добру їжу, теплий
і порядний одяг і дуже мало працюють» («Such ideal workhouse must be
made a «House of Terror», and not an asylum for the poor, where they are
to be plentifully fed, warmly and decently clothed, and where they do but
а little work»).
}.
В цьому «домі жаху», цьому «ідеалі робітного дому», праця повинна
тривати 14 годин на день, залічуючи сюди, однак, відповідний час
на їжу, так, щоб на працю лишалося повних 12 годин»\footnote{
«In this ideal workhouse the poor shall work 14 hours in a day
allowing proper time for meals, in such manner that there shall remain
12 hours of neat labour». (Там же). «Французи, — каже він, — сміються
з наших ентузіястичних ідей про волю». (Там же, стор. 78).
}.

Дванадцять годин праці щодня в «Ideal-Workhouse», в домі
жаху року 1770! Шістдесят три роки пізніш, 1833~\abbr{р.}, коли англійський
парлямент у чотирьох галузях промисловости зменшив
робочий день дітей від 13 до 18-літнього віку до 12 повних робочих
годин, здавалось, англійській промисловості прийшов останній
час! 1852~\abbr{р.}, коли Л. Бонапарте, щоб забезпечити собі підтримку
буржуазії, спробував був зробити замах на встановлений
законом робочий день, французька робітнича людність одноголосно
кричала: «Закон, що обмежує робочий день на 12 годин, —
це єдине благо, яке лишилося нам із законодавства республіки!»\footnote{
«Вони особливо виступали проти праці, що триває більш, ніж
12 годин на день, тому що закон, який установив такий робочий день, є
єдине благо, яке лишилося їм від законодавства республіки» («They especially
objected to work beyond the 12 hours per day. because the law which
fixed those hours is the only good which remains to them of the legislation
of the Republic»). («Reports of Insp. of Fact. 31 st October 1856»,
p. 80). Французький закон з 5 вересня 1850~\abbr{р.} про дванадцятигодинний
робочий день — це змінене на користь буржуазії видання декрету тимчасового
уряду з 2 березня 1848~\abbr{р.}; цей закон поширюється однаково на
всі майстерні. Перед цим законом робочий день у Франції був необмежений.
Він тривав на фабриках 14, 15 і більше годин. Див.; «Des classes
ouvrières en France, pendant l'année 1848. Par M. Blanqui». Панові Блянкі,
економістові, a не революціонерові, уряд доручив зробити дослідження
становища робітників.
}
В Цюріху працю дітей старших за 10 років обмежено 12 годинами;
в Аарґаві 1862~\abbr{р.} працю дітей від 13 до 16 років скорочено з 12\sfrac{1}{2}
до 12 годин; в Австрії 1860~\abbr{р.} для дітей від 14 до 16 років також
скорочено її до 12 годин\footnote{
І в справі урегулювання робочого дня Бельґія виявляє себе зразковою
буржуазною державою. Лорд Говард де Велден, англійський посол
у Брюсселі, 12 травня 1862~\abbr{р.} доводить до відома Foreign Office\footnote*{
— англійського міністерства закордонних справ. \emph{Ред.}
}: «Міністер
Рожіє заявив мені, що дитячої праці ніяк не обмежує ні загальний закон,
ні місцеві постанови; що уряд протягом трьох останніх років кожної сесії
мав думку внести до палати законопроєкта з цього приводу, але він завжди
натрапляв на непереможну перешкоду в ревнивому остраху перед
всяким законодавством, що суперечить принципові повної волі праці»!
}. Який проґрес від 1770~\abbr{р.} з «exultation», вигукнув би Маколей!

«Дім жаху» для павперів, про який капіталістична душа
мріяла ще 1770~\abbr{р.}, постав декілька років пізніш у формі велетенського
«робітного дому» для самих мануфактурних робітників.
Він звався фабрикою. І цього разу ідеал зблід перед дійсністю,

\index{i}{0217}  %% посилання на сторінку оригінального видання

\subsection{Боротьба за нормальний робочий день. Примусове законодавче
обмеження робочого часу. Англійське фабричне законодавство
від 1833 до 1864~\abbr{р.}}

Після того, як капіталові треба було століть, щоб здовжити
робочий день до його нормальних максимальних меж, а потім і
поза ці межі, до меж природного дванадцятигодинного дня\footnote{
«Це безперечно жалюгідний факт, що якась кляса людей мусить
мучитися коло праці по 12 годин на день. Коли залічити сюди ще обідній
час і час, потрібний на те, щоб дійти до майстерні й назад, то це в дійсності
становитиме 14 із 24 годин на добу\dots{} Залишаючи навіть здоров’я осторонь,
я сподіваюся, що ніхто не зважиться заперечувати, що з морального
погляду таке цілковите пожирання часу трудящих кляс, безперестанно
від раннього віку, від 13 року життя, а по «вільних» галузях промисловости
навіть від значно ранішого віку, є надзвичайно шкідливе й страшне
лихо\dots{} В інтересах суспільної моралі, щоб виховати дужу людність
і щоб забезпечити великій масі народу змогу розумно користуватися
життям, треба вперто домагатися того, щоб по всіх галузях промисловости
від кожного робочого дня лишалась якась частина на відпочинок і дозвілля».
(\emph{Leonhard Horner} у «Reports of Insp. of Fact, for 31 st December
1841»).
},
почалося від часів народження великої промисловости в останній
третині XVIII віку лявиноподібне, ґвалтовне й безмірне перекидання
всіх перепон. Всі межі, які ставили звичаї й природа,
вік і стать, день і ніч — всі їх геть поруйновано. Навіть поняття
про день і ніч, по-селянському прості у старих статутах, так
розпливлися, що одному англійському судді ще 1860~\abbr{р.} треба
було вжити дійсно талмудичного мудрування, щоб пояснити
силою «судового присуду», що таке день і що таке ніч\footnote{
Див. «Judgment of Mr. J. H. Otway, Belfast, Hilary Sessions,
County Antrim 1860».
}. Капітал святкував свої оргії.

Скоро тільки приголомшена гуркотом продукції робітнича
кляса сяк-так знов отямилась, вона почала ставити опір, спочатку
на батьківщині великої індустрії, в Англії. Однак протягом
трьох десятиліть поступки, що їх вона здобула впертою боротьбою,
лишилися суто номінальні. За час від 1802 й до 1833~\abbr{р.}
парлямент видав п’ять законів про працю, але був настільки
хитрий, що не вотував жодного шеляга на примусове їх запровадження,
на потрібний персонал урядовців і~\abbr{т. ін.}\footnote{
Дуже характеристичний для режиму Люї-Філіпа, цього короля-буржуа,
є той факт, що однісінький виданий при ньому фабричний
закон з 22 березня 1841~\abbr{р.} ніколи не був запроваджений у життя. Та й цей
закон стосується лише до дитячої праці. Він визначає вісім годин праці для
дітей 8--12 років, дванадцять годин — для дітей 12--16 років і~\abbr{т. ін.}, але допускає багато винятків, які дозволяють нічну працю навіть
для восьмилітніх дітей; догляд за виконанням цього закону і примус
щодо його виконання полишено на добру волю «amis du commerce», і це
в країні, де кожна миша є під оком поліції. Лише від 1853~\abbr{р.} в одинодному
департаменті, Département du Nord, бачимо ми оплачуваного урядового
інспектора. Не менш характеристичний для розвитку французького
суспільства взагалі є той факт, що закон Люї-Філіпа до революції 1848~\abbr{р.}
був однісіньким законом у тій французькій законодавчій фабриці, що
своєю сіткою оплітає геть усе.
} Закони лишилися
\index{i}{0218}  %% посилання на сторінку оригінального видання
мертвою буквою. «Це факт, що перед законом 1833~\abbr{р.} дітей
і підлітків мучено працею («were worked») цілу ніч, цілий день
або день і ніч ad libitum»\footnote*{
— досхочу. Peд.
}\footnote{
«Reports of Insp. of Fact, for 30 th April 1860», p. 51.
}.

Лише від часу фабричного закону 1833~\abbr{р.} — він поширюється
на фабрики бавовни, вовни, льону й шовку — починається нормальний
робочий день для сучасної промисловости. Ніщо так
добре не характеризує дух капіталу, як історія англійського
фабричного законодавства від 1833 до 1864 року!

Закон 1833~\abbr{р.} проголошує, що звичайний робочий день на
фабриці має починатись о пів на шосту ранку й кінчатись о пів
на дев’яту вечора. В межах цього 15-годинного періоду закон
дозволяє вживати праці підлітків (тобто осіб між 13 і 18 роками
життя) у будь-який час дня, припускаючи завжди, що одна й
та сама молода особа працює не більш як 12 годин протягом одного
дня, за винятком деяких спеціяльно передбачених випадків. Відділ
шостий закону постановляє, «що протягом кожного дня кожній
особі, що її робочий час обмежено, мусить призначатися щонайменше
1\sfrac{1}{2} години на їжу». Заборонялося вживати праці дітей
до 9 років, за одним винятком, про який треба буде згадати пізніш;
працю дітей од 9 до 13 років обмежено на 8 годин денно.
Нічна праця, тобто, за цим законом, праця між пів на дев’яту
вечора й пів на шосту ранку, була заборонена для всіх осіб між
9 і 18 роком життя.

Законодавці були такі далекі від бажання посягнути на волю
капіталу висисати дорослу робочу силу або, як вони це називали,
на «волю праці», що вигадали осібну систему, щоб запобігти
таким жахним наслідкам фабричного закону.

«Велике лихо фабричної системи, як її зорганізовано в теперішній
час, — сказано в першому звіті центральної ради комісії
з 25 червня 1833~\abbr{р.}, — є в тому, що вона створює доконечність
поширити дитячу працю до крайніх меж робочого дня дорослих.
Однісінький лік на це лихо, не обмежуючи праці дорослих, звідки
могло б постати ще більше лихо, ніж те, що йому треба запобігти,
є, здається, плян завести подвійні зміни дітей». Тим то плян цей
і здійснено під назвою «Relaissystem» («Sustem of Relays»; Relay
по-англійському, як і по-французькому, означає змінювання
поштових коней на різних станціях), так що одну зміну дітей
од 9 до 13 року життя запрягають до роботи, наприклад, від пів
на шосту зранку до пів на другу по півдні, другу — від пів на другу
по півдні до пів на дев’яту вечора й~\abbr{т. д.}

Але в нагороду за те, що панове фабриканти якнайнахабніше
\parbreak{}  %% абзац продовжується на наступній сторінці
