\parcont{}  %% абзац починається на попередній сторінці
\index{i}{0569}  %% посилання на сторінку оригінального видання
акр, не зважаючи на те, що він залічив у цю територію і полону
ширини Темзи. Само собою зрозуміло, що всякі санітарно-поліційні
заходи, які, як це досі робилося в Лондоні, через
зламання негодящих домів виганяють робітників з одного кварталу,
служать лише для того, щоб їх щільніше напаковувати
л іншому. «Або, — каже д-р Гентер, — цілу цю процедуру,
як безглузду, треба цілком припинити, або мусить пробудитися
громадське співчуття (!) до того, що тепер, не перебільшуючи,
можна назвати національним обов’язком, а саме до того, щоб
дати притулок людям, які через брак капіталу не можуть його
сами собі придбати, але можуть дати відшкодування власникам
квартир періодичними виплатами»\footnote{
Там же, стор. 89.
}. Дивовижна річ ця капіталістична
справедливість! Землевласник, домовласник, комерсант,
коли в нього що експропріюють задля «поліпшень», як от
будова залізниць, будова нових вулиць і~\abbr{т. д.}, не тільки дістає
повне відшкодування. За своє вимушене «зречення» він, крім
того, мусить за божими й людськими законами мати як нагороду
ще й не абиякий зиск. А робітників з дружинами, дітьми й пожитками
викидають на брук, і якщо вони занадто великими масами
ринуть до міських кварталів, в яких муніципалітет особливо
стежить за добропристойністю, то їх переслідує санітарна поліція!

На початку XIX віку в Англії, окрім Лондону, не було жодного
міста, що налічувало б \num{100.000} мешканців. Тільки п’ять
мало понад \num{50.000}. Тепер існує 28 міст, що мають понад \num{50.000} мешканців.
«Результатом цієї зміни був не тільки величезний приріст
міської людности, але й те, що старі битком набиті дрібні
міста тепер стали центрами, з усіх боків позабудованими, без
якогобудь вільного допливу свіжого повітря. Через те, що ці
міста стали вже неприємними для багатих, вони їх залишають,
оселюючись на веселіших передмістях. Наступники цих багатіїв
оселюються у великих домах, при чому кожна родина, часто ще
з квартирантами, дістає по одній кімнаті. Таким чином людність
втискується в доми, не для неї призначені, до її потреб зовсім
непристосовані, в оточенні, що справді понижує дорослих і руйнує
дітей»\footnote{
Там же, стор. 55, 56.
}. Що швидше в якомусь промисловому або торговельному
місті акумулюється капітал, то швидше припливає
приступний для експлуатації людський матеріял, то злиденніші
імпровізовані житла робітників. Тим то Ньюкестл над Тайном,
як центр кам’яновугільної й гірничої округи, що чимраз дужче
розвивається, посідає після Лондону друге місце в житловому
пеклі. Не менше, як \num{34.000} осіб живе там по окремих комірках.
У наслідок абсолютної небезпеки для громадського здоров’я
з наказу поліції в Ньюкестлі і Ґетшеді порозвалювано недавно
чимало домів. Будування нових домів посувається дуже повільно,
а промисловість розвивається дуже швидко. Тому 1865~\abbr{р.}
місто було переповнене більше ніж будь-коли раніш. Ледве
чи можна було найняти хоч одну комірку. Д-р Імблтон із шпиталю
\index{i}{0570}  %% посилання на сторінку оригінального видання
для хорих на тиф у Ньюкестлі каже: «Безперечно, причина
тривання й поширення тифу є переповнення помешкань людьми
та нечистота цих помешкань. Доми, де звичайно живуть робітники,
стоять у глухих заулках і подвір’ях. Щодо світла, повітря,
простору й чистоти це є справді зразки недостатности й негігієнічности,
ганьба для кожної цивілізованої країни. Там чоловіки
жінки й діти лежать ночами вкупі, поперемішувані як попало.
Щодо чоловіків, то нічна зміна невпинною течією йде по
денній, так що ліжка ледве встигають прохолонути. Доми погано
забезпечено водою і ще гірше кльозетами, вони страшенно нечисті,
не вентилюються, поширюють заразу»\footnote{
Там же, стор. 149.
}. Тижнева плата
за такі діри становить від 8\pens{ пенсів} до 3\shil{ шилінґів.} «Ньюкестл
над Тайном, каже д-р Гентер, являє собою приклад того, як
одне з найкращих племен поміж нашими земляками через зовнішні
умови, а саме через стан помешкань і вулиць, занепадає
часто майже до стану дикого виродження»\footnote{
Там же, стор. 50.
}.

У наслідок постійного припливу й відпливу капіталу й праці
житловий стан якогось промислового міста може бути сьогодні
стерпний, а на завтра стає огидний. Або ж міська влада, нарешті,
отямлюється і починає усувати щонайгірші непорядки. Але на
завтра ж хмарою сарани насувають обідрані ірляндці або занепаді
англійські рільничі робітники. Їх запихають у льохи й
комори, або порядний колись дім для робітників перетворюють
на помешкання, де персонал змінюється так швидко, як солдатські
постої підчас тридцятирічної війни. Приклад: Bradford.
Саме там муніципальні філістери заходилися коло міської реформи.
Крім того, там 1861~\abbr{р.} було ще \num{1.751} незаселений дім.
Аж ось справи пішли добре, як про це нещодавно так любо
розводився солодкувато-ліберальний професор Форстер, приятель
негрів. Певна річ, з поліпшенням справ постає повідь
від хвиль «резервної армії», або «відносного перелюднення»,
що постійно хвилюється. Найгидкіші мешкання по льохах
та коморах, зареєстрованих у списку\footnote{
Список, що його склав аґент одного товариства для забезпечення
робітників у Bradford’i:

\begin{center}
\scriptsize
% See longtable manual
\setcounter{LTchunksize}{2}
\begin{longtable}{lc@{ }c@{ }r@{ }l}
Vulcanstreet. Nr. 122\dotfill{} & 1 & кімната & 16 & осіб \\
Lumleystreet. Nr. 13\dotfill{} & 1 & \ditto{кімната} & 11 & осіб \\
Bowerstreet. Nr. 41\dotfill{} & 1 & \ditto{кімната} & 11 & осіб \\
Portlandstreet. Nr. 112\dotfill{} & 1 & \ditto{кімната} & 10 & осіб \\
Hardystreet. Nr. 17\dotfill{} & 1 & \ditto{кімната} & 10 & осіб \\
Northstreet. Nr. 18\dotfill{} & 1 & \ditto{кімната} & 16 & осіб \\
\ditto{Northstreet.} Nr. 17\dotfill{} & 1 & \ditto{кімната} & 13 & осіб \\
Wymerstreet. Nr. 19\dotfill{} & 1 & \ditto{кімната} & 8 & дорослих \\
Jowettstreet. Nr. 56\dotfill{} & 1 & \ditto{кімната} & 12 & осіб \\
Georgestreet. Nr. 150\dotfill{} & 1 & \ditto{кімната} & 3 & родини \\
Rifle-Court, Marygate. Nr. 11\dotfill{} & 1 & \ditto{кімната} & 11 & осіб \\
Marshallstreet. Nr. 28\dotfill{} & 1 & \ditto{кімната} & 10 & осіб \\
\ditto{Marshallstreet.} Nr. 49\dotfill{} & 1 & \ditto{кімната} & 3 & родини \\
Georgestreet. Nr. 128\dotfill{} & 1 & \ditto{кімната} & 18 & осіб \\
\ditto{Georgestreet.} Nr. 130\dotfill{} & 1 & \ditto{кімната} & 16 & осіб \\
Edwardstreet. Nr. 4\dotfill{} & 1 & \ditto{кімната} & 17 & осіб \\
Jorkstreet. Nr. 34\dotfill{} & 1 & \ditto{кімната} & 2 & родини \\
Salt Piestreet\dotfill{} & 1 & \ditto{кімната} & 26 & осіб \\
\multicolumn{5}{c}{\emph{Льохи}} \\
Regent Square\dotfill{} & 1 & льох & 8 & осіб \\
Acrestreet\dotfill{} & 1 & \ditto{льох} & 7 & осіб \\
Robert’s Court. Nr. 33\dotfill{} & 1 & \ditto{льох} & 7 & осіб \\
Back Prattstreet, помешкання \\
\indentdef{}використовується як мідярня\dotfill{} & 1 & \ditto{льох} & 7 & осіб \\
Ebenezerstreet. Nr. 27 \dotfill{} & 1 & \ditto{льох} & 6 & осіб \\
\multicolumn{5}{l}{(«Public Health. Eighth Report», стор. 111).}
\end{longtable}\end{center}}, що його одержав
д-р Гентер від аґента одного страхового товариства, займали
здебільшого добре оплачувані робітники. Вони заявляли, що
\index{i}{0571}  %% посилання на сторінку оригінального видання
охоче платили б за кращі помешкання, коли б їх можна було
дістати. Тимчасом вони із своїми родинами занепадають і хоріють,
а солодкувато-ліберальний Форстер, член парляменту, проливає
сльози захоплення з приводу благодаті вільної торговлі й зисків
видатних голів Bradford’у від вовняних підприємств. У звіті
з 5 вересня 1865~\abbr{р.} д-р Белл, один із бредфордських лікарів для
бідних, пояснює жахливу смертність хорих на тиф у його окрузі
їхніми житловими умовами: «В одному льоху на \num{1.500} кубічних
футів живе десять осіб\dots{} На вулицях. Vincentstrasse, Green
Air Place і the Leys є 223 доми з \num{1.450} мешканцями, 435 ліжками
і 36 кльозетами\dots{} На кожне ліжко — а під ліжком я розумію
всякий жмут брудного ганчір’я або купу стружок — припадає
пересічно 3,3 особи, на декотрі 4 й 6 осіб. Багато спить без ліжка
просто на підлозі, не роздягаючись, молоді чоловіки й жінки,
жонаті й нежонаті — все це як попало, одне побіч одного. Чи
треба ще додати, що ці житла здебільша темні, вогкі, брудні,
смердючі нори, цілком непридатні для людського мешкання?
Це — центри, звідки поширюються недуги й смерть, що виривають
свої жертви навіть з-серед заможних (of good circumstances),
які допустили до того, щоб ця моровиця гноїлася в нашому
середовищі»\footnote{
Там же, стор. 114.
}.

Третє після Лондону місце щодо житлових злиднів посідає
Брістол. «Тут, в одному з найбагатших міст Европи, якнайбільший
надмір глибокої бідности («blank poverty») і житлових
злиднів»\footnote{
Там же, стор. 50.
}.

\subsubsection{Бродяча людність}

А тепер звернімося до верстви людности, сільської своїм походженням,
але здебільша занятої в промисловості. Вона становить
легку піхоту капіталу, яку відповідно до своїх потреб
він кидає то в один пункт, то в інший. Коли вона не в поході,
то «стоїть табором». Працю бродячих робітників використовують
на різні будівельні й дренажні операції, вироблення цегли,
випалювання вапна, будову залізниць тощо. Вони є рухлива
колона, що переносить у ті місцевості, навколо яких вони отаборюються,
заразливі недуги: віспу, тиф, холеру, скарлятину
й~\abbr{т. ін.}\footnote{
«Public Health. Seventh Report», London 1865, p. 18.
} У підприємствах із значною витратою капіталу, як
\parbreak{}  %% абзац продовжується на наступній сторінці
