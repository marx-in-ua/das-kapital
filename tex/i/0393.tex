Заробітну плату, взагалі мізерну в розглянутих щойно галузях
промисловосте (винятково максимальна плата дітей по школах
плетіння з соломи — 3 шилінґи), зменшується ще нижче її номінальної
величини за допомогою truck-системи,* що взагалі
особливо панує в округах виробництва мережива.263

е) Перехід сучасної мануфактури і домашньої
праці до великої промисловости. Прискорення
цієї революції через застосування фабричних
законів до цих способів продукції

Здешевлювання робочої сили через саме лише зловживання
жіночими та недозрілими робочими силами, через просте грабування
усіх нормальних умов праці й життя та через брутальність
надмірної і нічної праці, кінець-кінцем натрапляє на певні природні
межі, що їх уже не можна переступити, а разом з цим
на ці межі натрапляє і побудоване на цих підвалинах здешевлення
товарів і капіталістична експлуатація взагалі. Скоро тільки цього
пункту, нарешті, досягнено, — а на це треба багато часу, — тоді
б’є година для заведення машин і швидкого віднині перетворення
розпорошеної домашньої праці (або й мануфактури) на фабричну
промисловість.

Найколосальніший приклад цього руху дає виробництво «wearing
apparel» (речей, що належать до одягу). За клясифікацією
«Children’s Employment Commission» ця промисловість охоплює
виробників солом’яних капелюхів, дамських капелюхів, шапкарів,
кравців, milliners і dressmakers, 264 виробників сорочок і швачок,
корсетниць, рукавичників, шевців і разом з тим багато дрібніших
галузей, як от фабрикація галстухів, комірців і т. ін. Жіночий
персонал, що працює в цих галузях промисловости в Англії і
Велзі, становив 1861 р. 586.298 осіб, з них, щонайменше, 115.242
молодші за 20 років, 16.650 молодші за 15 років. Число цих робітниць
у цілому Об’єднаному Королівстві становило (1861 р.)
750.334. Число робітників-чоловіків, що того самого часу працювали
в капелюшній, шевській, рукавичній та кравецькій промисловості
Англії й Велзу, становило 437.969, з них 14.964 молодші
за 15 років, 89.285 осіб 15—20 років, 333.117 — понад 20 років.
У ці числа не входить багато, належних сюди, дрібних галузей.
Але коли взяти ці числа так, як вони є, то ми матимемо лише для
Англії та Велзу, за переписом 1861 р., суму в 1.024.277 осіб,
отже, приблизно стільки, скільки забирає рільництво і скотарство.
Тепер ми починаємо розуміти, для чого машини допомагають

263 «Children’s Employment Commission. 1 st Report 1863», p. 185.

264 Millinery стосується, власне, лише до виробництва головних уборів,
але воно охоплює і виробництво жіночих пальт і мантиль, тимчасом
як dressmakers — не те саме, що й наші модистки.

* — системи виплати заробітної плати товарами. Ред.
