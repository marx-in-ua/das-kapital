\parcont{}  %% абзац починається на попередній сторінці
\index{i}{0155}  %% посилання на сторінку оригінального видання
в певній корисній формі, а він не може додавати її в корисній
формі, не роблячи продукти засобами продукції нового продукту
й не переносячи тим самим їхньої вартости на новий продукт.
Отже, зберігати вартість, додаючи вартість, — це є дар природи,
діющої робочої сили, живої праці, дар природи, який робітникові
нічого не коштує, а капіталістові дає багато, — зберігає наявну
капітальну вартість\footnoteA{
«З усіх знарядь фармерської продукції людська праця\dots{} є таке
знаряддя, від якого фармер більше за все може сподіватися, що його капітал
назад повернеться. Інші два: робоча худоба і\dots{} вози, плуги, лопати
тощо, нічого не варті без певної частини першого» («Of all the instruments
of the farmer’s trade, the labour of man\dots{} is that on which he is most to
rely for the re-payment of his capital. The other two — the working stock
of the cattle, and the\dots{} carts, ploughs, spades, and so forth — without a given
portion of the first, are nothing at all»). (\emph{Edmund Burke}: «Thoughts and
Details on Scarcity, originally presented to the Right Hon. W.~Pitt in the
Month of November 1795», ed. London 1800, p. 10).
}.  Доки справи йдуть добре, капіталіст
надто заглиблений у фабрикацію додаткової вартости, щоб помічати
цей дармовий дар праці. Ґвалтовні перерви процесу праці,
кризи, роблять його для капіталіста дошкульно помітним\footnote{
У «Times’i» з 26 листопада 1862 p. один фабрикант, що на його
прядільні працює 800 робітників і щотижня споживається пересічно 150 пак
східньо-індійської або щось із 130 пак американської бавовни, нарікає
перед публікою на ті витрати, які викликає щорічне припинення його
фабрики. Він цінує їх на \num{6.000}\pound{ фунтів стерлінґів}. Серед цих втратних
витрат (Unkosten) є багато таких, які нас тут не цікавлять, як от земельна
рента, податки, страхові премії, плата найнятим на рік робітникам, управителеві,
бухгалтерові, інженерові й~\abbr{т. ін.} Але далі він залічує сюди
150\pound{ фунтів стерлінґів} за вугілля, щоб час від часу опалювати фабрику та
вряди-годи пускати в рух парову машину, крім того, заробітну плату
робітникам, що своєю тимчасовою працею підтримують «напоготові»
цілий механізм. Нарешті, \num{1.200}\pound{ фунтів стерлінґів} на те, що машини псуються,
бо «силу погоди й руйнаційних природних впливів не спиняється
тим, що парові машини перестали рухатися» («the weather and the natural
principle of decay do not suspend their operations because the steamengine
ceases to revolve»). Він виразно зауважує, що ця сума в \num{1.200}\pound{ фунтів
стерлінґів} така незначна через те, що машини вже значно зужитковані.
}.

Що взагалі споживається в засобах продукції, так це лише
їхня споживна вартість, споживанням якої праця творить продукти.
їхню вартість у дійсності не споживається\footnote{
«Продуктивне споживання: коли споживання товару становить
частину процесу продукції\dots{} В цих випадках немає споживання вартости»
(«Productive Consumption: where the consumption of a commodity is a
part of the process of production\dots{} In these instances there is no consumption
of value»). (\emph{S.~P.~Newman}: «Elements of Political Economy» Andover
and New-Jork 1835, p. 296).
}, отже, і
не може вона бути репродукована. Вона зберігається, але не
тому, що з нею самою відбувається якась операція в процесі
праці, а тому, що та споживна вартість, у формі якої вона первісно
існувала, хоч і зникає, але лише в іншій споживній вартості\footnote*{
Кінець цього речення у французькому виданні подано так: «\dots{} але
зникає лише на те, щоб набрати нової корисної форми». \emph{Ред.}
}.
Тому вартість засобів продукції з’являється знов у вартості
продукту, але її, строго кажучи, не репродукується. Що продукується,
\index{i}{0156}  %% посилання на сторінку оригінального видання
так це лише нову споживну вартість, що в ній знову
з’являється стара мінова вартість\footnote{
В одному північно-американському стислому підручнику, що витримав,
може, 20 видань, ми читаємо: «Не має жодного значення те, в
якій формі капітал з’являється знов» («It matters not in what form capital
reappears»). Після багатомовного переліку всіх можливих складових
частин продукції, що їх вартість знов з’являється в продукті, наприкінці
сказано: «Різні ґатунки харчу, одягу й житла, доконечні для існування
й комфорту людини, теж зазнають змін. Їх час від часу споживається, і
вартість їхня знову з’являється в новій фізичній і розумовій силі людини, що
являє собою новий капітал, який можна знову вжити на продукцію» («The
various kinds of food, clothing and shelter, necessary for the existence and
comfort of the human being, are also changed. They are consumed from time
to time, and their-value re-appears, in that new vigour im arted to his
body and mind, forming fresh capital, to be employed again in the work
of production»). (\emph{F.~Wayland}: «The Elements of Political Economy»,
Boston 1853, p. 31, 32). Залишаючи осторонь всі інші чудноти, зауважимо,
що, приміром, не ціна хліба знов з’являється у відновленій силі,
а його кровотворні елементи. Навпаки, те, що знову з’являється як вартість
сили, є не засоби існування, а їхня вартість. Ті самі засоби існування,
якщо вони коштують лише половину, випродукують цілком стільки ж
мускулів, костей і~\abbr{т. ін.}, словом, таку саму силу, але силу не тієї самої
вартости. Ця плутанина, перетворення «вартости» на «силу», і вся ця
фарисейська невизначеність криють у собі спробу — звичайно, даремну —
викрутами вивести додаткову вартість із простого повернення авансованих
вартостей.
}.

Інакше стоїть справа з суб’єктивним фактором процесу праці,
з діющою робочою силою. Тимчасом як праця в наслідок своєї
доцільної форми переносить вартість засобів продукції на продукт
і зберігає її, кожний момент руху праці творить новододавану вартість,
нову вартість. Припустімо, що процес продукції припиняється
на тому пункті, коли робітник випродукував еквівалент
вартости своєї власної робочої сили, коли він, приміром, шестигодинною
працею додав вартість у 3\shil{ шилінґи.} Ця вартість становить
надлишок вартости продукту понад ті її складові частини,
що своє постання завдячують вартості засобів продукції. Вона є
однісінька нова вартість, що постала в межах цього процесу,
однісінька частина вартости продукту, випродукована самим цим
процесом. Певна річ, вона лише компенсує ті гроші, що їх авансував
капіталіст підчас купівлі робочої сили та які сам робітник
витратив на засоби існування. Щодо цих витрачених 3\shil{ шилінґів}
нова вартість у 3\shil{ шилінґи} з’являється лише як репродукція їх.
Але вона дійсно репродукована, а не лише на позір, як от вартість
засобів продукції. Заміщення однієї вартости іншою тут упосереднюється
творенням нової вартости.

Однак ми вже знаємо, що процес праці триває далі поза той
пункт, коли репродукується й прилучається до предмету праці
лише еквівалент вартости робочої сили. Замість 6 годин, яких
для цього було досить, процес триває, приміром, 12 годин. Отже,
в наслідок діяння робочої сили не тільки репродукується її власну
вартість, але й продукується ще надлишок вартости. Ця додаткова
вартість становить надлишок вартости продукту понад
\parbreak{}  %% абзац продовжується на наступній сторінці
