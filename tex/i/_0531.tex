\parcont{}  %% абзац починається на попередній сторінці
\index{i}{0531}  %% посилання на сторінку оригінального видання
плати є не що інше, як відношення між неоплаченою працею,
перетвореною на капітал, і новододаваною працею, потрібного, щоб
пустити в рух додатковий капітал. Отже, це зовсім не відношення
між двома незалежними одна від одної величинами, між величиною
капіталу, з одного боку, і кількістю робітничої людности —
з другого; навпаки, це в останній інстанції тільки відношення
між неоплаченою та оплаченою працею тієї самої робітничої людности.
Коли кількість неоплаченої праці, постачуваної робітничою
клясою й нагромаджуваної клясою капіталістів, зростає
досить швидко, так що вона може перетворюватися на капітал
лише за допомогою надзвичайного додатку оплаченої праці, то
заробітна плата підвищується і, за всіх інших незмінних умов,
неоплачена праця відносно меншає. Але скоро тільки це зменшення
доходить до того пункту, коли додаткову працю, з якої
живиться капітал, не постачається вже в нормальній кількості,
то постає реакція: капіталізується меншу частину доходу, акумуляція
слабшає, а висхідний рух заробітної плати змінюється
на протилежний. Отже, підвищення ціни праці ніколи не може
вийти за ті межі, які не тільки лишають недоторканими основи
капіталістичної системи, а й забезпечують її репродукцію в
щораз більшому маштабі. Отже, закон капіталістичної акумуляції,
змістифікований на закон природи, в дійсності виражає
лише те, що природа акумуляції виключає всяке таке зменшення
ступеня експлуатації праці або всяке таке підвищення ціни
праці, яке серйозно могло б загрожувати постійній репродукції
капіталістичного відношення й репродукції його в щораз ширшому
маштабі. Інакше й не може бути за такого способу продукції,
коли робітник існує для потреб збільшення наявних вартостей
замість, навпаки, речовому багатству існувати для потреб
розвитку робітника. Як у релігії людину опановує витвір її власної
голови, так за капіталістичної продукції її опановує витвір
її власної руки\footnoteA{
«Але коли ми повернемося тепер до нашого першого досліду,
де показано\dots{} що сам капітал є лише продукт людської праці\dots{} то здається
цілком незрозумілим, яким чином людина могла опинитися під
пануванням свого власного продукту, капіталу, і підкоритися йому;
а що це в дійсності є безперечний факт, то мимоволі постає питання, яким
чином робітник із пана над капіталом, як творець його, міг зробитися
рабом капіталу?» (\emph{Von Thünen}: «Der isolierte Staat», Zweiter Teil, Zweite
Abteilung, Rostock 1863, S. 5, 6). Заслуга Тінена в тому, що він поставив
це питання. Але відповідь його просто дитяча.
}.

\subsection{Відносне зменшення змінної частини капіталу
з~проґресом~акумуляції і концентрації, що її супроводить}

На думку самих економістів, до підвищення заробітної плати
призводить не наявний розмір суспільного багатства й не величина
вже надбаного капіталу, а виключно лише безупинний
зріст акумуляції капіталу та ступінь швидкости її зросту
\parbreak{}  %% абзац продовжується на наступній сторінці
