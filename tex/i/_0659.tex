\parcont{}  %% абзац починається на попередній сторінці
\index{i}{0659}  %% посилання на сторінку оригінального видання
звичайно, процес куди довший, тяжчий і важчий, аніж перетворення
капіталістичної власности, фактично вже основаної на суспільній
продукції, на суспільну власність. Там ішлося про експропріяцію
народньої маси небагатьма узурпаторами, тут ідеться
про експропріацію небагатьох узурпаторів народньою масою\footnote{
«Проґрес промисловости, що його безвольний і нездатний до
опору носій є буржуазія, ставить на місце ізолювання робітників через
конкуренцію їхнє революційне об’єднання через асоціяцію. Отже, з розвитком
великої промисловости з-під ніг буржуазії вибивається саму основу,
на якій вона продукує й присвоює собі продукти. Вона продукує насамперед
своїх власних могильників. Її загибіль і перемога пролетаріяту
однаково неминучі\dots{} З усіх кляс, що протистоять нині буржуазії, тільки
пролетаріят є справді революційна кляса. Всі інші кляси занепадають і
гинуть з розвитком великої промисловости; пролетаріят є її найпитоміший
продукт. Середні стани, дрібний промисловець, дрібний купець, ремісник,
селянин, — всі вони борються проти буржуазії, щоб забезпечити від
занепаду своє існування як середніх станів\dots{} вони реакційні, бо вони
намагаються повернути назад колесо історії». (К.~Marx und F.~Engels:
«Manifest der kommunistischen Partei», London 1847, S. 9, 11. — K.~Маркс
і Ф.~Енґельс: «Маніфест комуністичної партії», Партвидав «Пролетар»
1932~\abbr{р.}, стор. 39, 37).
}.

\lwcsetup{disable}
\section[Сучасна теорія колонізації]{Сучасна теорія колонізації\footnotemark{}}

\vspace{-\bigskipamount} % підняти на один рядок
Політична економія принципово сплутує два дуже різні
роди приватної власности, що з них один оснований на власній
праці продуцента, другий — на експлуатації чужої праці.
Вона забуває, що цей другий не лише становить пряму протилежність
першого, але й виростає тільки на його могилі.
\footnotetext{
Тут ідеться про дійсні колонії, про незайману землю, що її колонізують
вільні іміґранти. Сполучені штати, з економічного погляду, все
ще є колонія Европи. Зрештою сюди належать і такі старовинні плянтації,
де знищення рабства зробило цілковитий переворот у відносинах.
}

На заході Европи, батьківщині політичної економії, процес
первісної акумуляції капіталу більш або менш завершений.
Капіталістичний режим тут або просто підбив собі всю національну
продукцію, або там, де відносини менш розвинені, він, принаймні,
посередньо контролює належні до застарілого способу
продукції суспільні верстви, що й далі існують поряд нього й
поступінно занепадають. До цього готового світу капіталу політико-економ
з тим більшою запопадністю й тим більшим
зворушенням прикладає уявлення про право і власність, належні
до передкапіталістичного світу, чим голосніше кричать
факти проти його ідеології.

Інша справа в колоніях. Капіталістичний режим там повсюди
наражається на перешкоди з боку продуцента, що як посідач
своїх власних умов праці збагачує своєю працею самого
себе, а не капіталіста. Суперечність цих двох діяметрально протилежних
економічних систем виявляється тут на практиці
в їхній боротьбі. Там, де капіталіст має за своєю спиною силу
\parbreak{}  %% абзац продовжується на наступній сторінці
