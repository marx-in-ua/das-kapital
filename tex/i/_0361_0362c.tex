
\index{i}{0361}  %% посилання на сторінку оригінального видання
Загальний результат механічних поліпшень, заведених в англійській
бавовняній промисловості під впливом американської
громадянської війни, показує оця таблиця:

\begin{table}[H]
\centering
\caption*{Число фабрик по роках}
  \noindent\begin{tabular}{lrrr}
    & \emph{1858} & \emph{1861} & \emph{1868} \\
    Англія та Велз\dotfill{} & \num{2.046}   & \num{2.715}   & \num{2.405} \\
    Шотляндія\dotfill{} & 152 & 163 & 131 \\
    Ірляндія\dotfill{} & 12 & 9 & 13 \\
    \cmidrule{2-4}
    Об’єднане Королівство\dotfill{}
      & \samewidth{\num{28.010.217}}{\hfill{}\num{2.210}}
      & \samewidth{\num{30.387.494}}{\hfill{}\num{2.887}}
      & \samewidth{\num{32.000.014}}{\hfill{}\num{2.549}} \\
   \end{tabular}
\end{table}


\begin{table}[H]
\centering
\caption*{Число парових ткацьких варстатів по роках}
  \noindent\begin{tabular}{lrrr}
    Англія та Велз\dotfill{} & \num{275.590} & \num{368.125} & \num{344.719} \\
    Шотляндія\dotfill{} & \num{21.624} & \num{30.110} & \num{31.864} \\
    Ірляндія\dotfill{} & \num{1.633} & \num{1.757} & \num{2.746} \\
    \cmidrule{2-4}
    Об’єднане Королівство\dotfill{}
      & \samewidth{\num{28.010.217}}{\hfill{}\num{298.847}}
      & \samewidth{\num{30.387.494}}{\hfill{}\num{399.992}}
      & \samewidth{\num{32.000.014}}{\hfill{}\num{379.329}} \\
  \end{tabular}
\end{table}

\begin{table}[H]
\centering
\caption*{Число веретен по роках}
  \noindent\begin{tabular}{lrrr}
    & \emph{1858} & \emph{1861} & \emph{1868} \\ % manual fix
    Англія та Велз\dotfill{} & \num{25.818.576} & \num{28.352.152} & \num{30.478.228} \\
    Шотляндія\dotfill{} & \num{2.041.129} & \num{1.915.398} & \num{1.397.546} \\
    Ірляндія\dotfill{} & \num{150.512} & \num{119.944} & \num{124.240} \\
    \cmidrule{2-4}
    Об’єднане Королівство\dotfill{} & \num{28.010.217} & \num{30.387.494} & \num{32.000.014} \\
  \end{tabular}
\end{table}

\begin{table}[H]
\centering
\caption*{Число вживаних робітників по роках}
  \noindent\begin{tabular}{lrrr}
    Англія та Велз\dotfill{} & \num{341.170}   & \num{407.598} & \num{357.052} \\
    Шотляндія\dotfill{} & \num{34.698} & \num{41.237} & \num{39.809} \\
    Ірляндія\dotfill{} &  \num{3.345} &  \num{2.734} & \num{4.203} \\
    \cmidrule{2-4}
    Об’єднане Королівство\dotfill{}
      & \samewidth{\num{28.010.217}}{\hfill{}\num{379.213}}
      & \samewidth{\num{30.387.494}}{\hfill{}\num{451.569}}
      & \samewidth{\num{32.000.014}}{\hfill{}\num{401.064}} \\
  \end{tabular}
\end{table}

\noindent{}Отже, від 1861 до 1868~\abbr{р.} зникло 338 бавовняних фабрик, тобто
продуктивніший та більший машиновий механізм сконцентрувався
в руках меншого числа капіталістів. Число парових ткацьких
варстатів зменшилося на \num{20.663}; але продукт їхній одночасно
збільшився, так що поліпшений ткацький варстат давав
тепер більше продукту, ніж старий. Нарешті, число веретен
зросло на \num{1.612.541}, тимчасом як число вживаних робітників
зменшилося на \num{50.505}. Отже, ті «тимчасові» злидні, що ними бавовняна
криза душила робітників, збільшив і зміцнив хуткий
та невпинний проґрес машинової системи.

Однак машина діє не тільки як непереможний конкурент,
який завжди напоготові зробити найманого робітника «зайвим».
Капітал голосно й тенденційно проголошує її силою, ворожою
робітникові, та саме як таку вживає її. Вона стає наймогутнішим
бойовим знаряддям придушувати періодичні робітничі повстання,
страйки і~\abbr{т. ін.} проти автократії капіталу\footnote{
«Відносини між хазяїнами й руками по фабриках флінтґлясу та пляшкового
скла — це хронічний страйк». Звідси швидкий розвиток мануфактури
пресованого скла, де головні операції виконуються за допомогою машин.
Одна фірма в Ньюкестлі, яка раніш продукувала \num{350.000} фунтів дутого
кремінного скла річно, тепер замість цієї кільцости продукує \num{3.000.500}
фунтів пресованого скла». («Children’s Employment Commission. 4 th
Report 1865», p. 262, 263).
}. За Ґаскелем,
\index{i}{0362}  %% посилання на сторінку оригінального видання
парова машина з самого початку була антагоністом «людської
сили», що дав капіталістам змогу розбивати щораз більші
домагання робітників, які загрожували кризою фабричній системі
на самому початку її виникнення\footnote{
\emph{Gaskell}: «The Manufacturing Population of England», London
1833, p. 3, 4.
}. Можна було б написати
цілу історію винаходів, які, починаючи від 1830~\abbr{р.}, покликано
до життя лише як бойове знаряддя капіталу проти повстань робітників.
Ми нагадаємо передусім selfacting mule\footnote*{
автоматичну прядільну машину. \emph{Ред.}
}, бо нею починається
нова епоха автоматичної системи\footnote{
Деякі дуже важливі застосування машин, щоб будувати машини,
винайшов п. Ферберн під впливом страйків на його власній фабриці.
}.

У своєму свідченні перед комісією, що їй доручено було дослідити
Trades-Unions, Несміт, винахідник парового молота, повідомляє
про поліпшення в машинах, які він завів у наслідок
великого та довгого страйку машинових робітників у 1851~\abbr{р.},
таке: «Характеристична риса наших сучасних механічних поліпшень
— це заведення самодіяльних виконавчих машин. Все, що
тепер має робити механічний робітник, і що може зробити всякий
підліток, — це не самому працювати, а лише наглядати за прегарною
роботою машини. Цілу клясу робітників, що залежить
виключно від своєї вмілости, тепер усунено. Раніш я на одного
механіка мав чотирьох хлопців. Завдяки цим новим механічним
комбінаціям я зменшив число дорослих чоловіків з \num{1.500} на 750.
Наслідком цього було значне збільшення мого зиску».

% \looseness=-1
Про одну машину для друку фарбами на перкалевибійних
фабриках Юр каже: «Нарешті капіталісти почали шукати способу
визволитися з-під цієї нестерпної неволі (тобто від тяжких
для них умов контракту з робітниками), покликавши собі на допомогу
джерела науки, і незабаром їх відновили в їхніх законних
правах, правах голови над іншими частинами тіла». Про один
винахід для шліхтування основи, що його безпосередньою причиною
був страйк, він каже так: «Орда незадоволених, що, окопавшися
за старими лініями поділу праці, вважала себе за непереможну,
побачила себе таким чином оточеною з флангів, а свої
оборонні засоби знищеними сучасною механічною тактикою. Вони
мусили здатися на ласку та гнів переможців». Про винахід
selfacting mule він каже: «Вона була покликана, щоб відновити
порядок серед промислових кляс\dots{} Цей винахід потверджує
розвинуту вже нами доктрину, що капітал, примусивши науку
служити йому, завжди силує бунтівничу руку праці до покірливости»\footnote{
\emph{Ure}: «Philosophy of Manufacture», стор. 367--370.
}. Хоч твір Юра з’явився 1835~\abbr{р.}, отже, за часів порівняно
мало ще розвинутої фабричної системи, все ж він лишається клясичним
виразом духу фабрики не тільки через свій щирий цинізм,
але й через ту наївність, з якою він виказує абсурдні суперечності
капіталістичного мозку. Розвинувши, приміром, «доктрину»,
що капітал за допомогою науки, взятої ним на утримання, «завжди
\parbreak{}  %% абзац продовжується на наступній сторінці
