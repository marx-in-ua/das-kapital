\parcont{}  %% абзац починається на попередній сторінці
\index{i}{0380}  %% посилання на сторінку оригінального видання
вимирають від голоду; 1836~\abbr{р.} — великий розцвіт; 1837--1838~\abbr{рр.} —
пригнічений стан і криза; 1839~\abbr{р.} — знову пожвавлення; 1840~\abbr{р.} —
велика депресія, повстання, втручання війська; 1841--1842~\abbr{рр.} —
страшне страждання фабричних робітників; 1842~\abbr{р.} — фабриканти
викидають робітників із фабрик, щоб вимусити скасування збіжжевих
законів. Робітники багатьма тисячами напливають до
Йоркширу, звідти військо проганяє їх назад, їхніх проводирів
віддають під суд у Ланкастері. 1843~\abbr{р.} — великі злидні: 1844~\abbr{р.} —
знову пожвавлення; 1845~\abbr{р.} — великий розцвіт. 1846~\abbr{р.} — спершу
триває піднесення, потім симптоми реакції; скасування збіжжевих
законів; 1847~\abbr{р.} — криза; загальне пониження заробітної плати
на 10 і більше процентів на славу «big loaf» (коровай величезного
розміру, що його пани фритредери обіцяли робітникам підчас
агітації проти збіжжевих законів);\footnote*{
Цього пояснення вислову «big loaf» немає в німецькому тексті.
Ми беремо його з французького видання, де його подано в дужках у самому
тексті. \emph{Ред.}
} 1848~\abbr{р.} — пригнічення триває; Менчестер під військовою охороною; 1849~\abbr{р.} —
знову пожвавлення. 1850~\abbr{р.} — розцвіт. 1851~\abbr{р.} — спад товарових цін, низька
заробітна плата, часті страйки; 1852~\abbr{р.} — починається поліпшення,
страйки тривають далі, фабриканти загрожують довезти чужоземних
робітників. 1853~\abbr{р.} — зріст вивозу; восьмимісячний страйк
і великі злидні в Престоні. 1854~\abbr{р.} — розцвіт, переповнення ринків.
1855~\abbr{р.} — із Сполучених штатів, Канади, із східньоазійських
ринків надходять звістки про банкрутства; 1856~\abbr{р.} — великий
розцвіт; 1857~\abbr{р.} — криза; 1858~\abbr{р.} — поліпшення; 1859~\abbr{р.} — великий
розцвіт, зріст числа фабрик; 1860~\abbr{р.} — зеніт англійської
бавовняної промисловости; індійські, австралійські й інші ринки
так переповнено, що ще 1863~\abbr{р.} вони ледве поглинули всю заваль;
торговельний договір з Францією; величезний зріст числа фабрик
та машин; 1861~\abbr{р.} — піднесення триває якийсь час далі, реакція,
американська громадянська війна, недостача бавовни; 1862 до
1863~\abbr{р.} — повний крах.
\enablefootnotebreak{}

Історія бавовняного голоду надто характеристична, щоб не
спинитись на ній хоч на одну хвилину. З наведених даних про
становище світового ринку за 1860--1861~\abbr{рр.} ми бачимо, що
бавовняний голод прийшовся фабрикантам до речі та почасти був
для них корисний — факт, визнаний у звітах менчестерської
торговельної палати, проголошений у парламенті Палмерстоном
і Дербі, потверджений подіями\footnote{
Порівн. «Reports of Insp. of Fact, for 31 st October 1862», p. 30.
}. Певна річ, 1861~\abbr{р.}
поміж \num{2.887} бавовняними фабриками Об’єднаного Королівства
багато було дрібних фабрик. За звітами фабричного інспектора
А.~Редґрева, що його округа включає з тих \num{2.887} фабрик \num{2.109} фабрик,
392 фабрики з цих останніх, або 19\%, вживали кожна
менш від 10 парових кінських сил, 345 фабрик, або 16\%, вживали
від 10 до 20 сил, а \num{1.372} фабрики — 20 і більше кінських сил\footnote{
Там же, стор. 19.
}.
\parbreak{}  %% абзац продовжується на наступній сторінці
