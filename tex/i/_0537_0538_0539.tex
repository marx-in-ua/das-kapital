\parcont{}  %% абзац починається на попередній сторінці 
\index{i}{0537}  %% посилання на сторінку оригінального видання 
продукції й панування над працею, то, з другого боку, вона
виявляється як взаємне відштовхування багатьох індивідуальних
капіталів.

Цьому роздрібненню цілого суспільного капіталу на багато
індивідуальних капіталів або взаємному відштовхуванню його
частин протидіє їхнє притягання. Це вже не проста, ідентична
з акумуляцією концентрація засобів продукції й панування над
працею. Це концентрація утворених уже капіталів, знищення
їхньої індивідуальної самостійности, експропріяція капіталіста
капіталістом, перетворення багатьох дрібних капіталів на незначне
число великих капіталів. Цей процес відрізняється від
першого тим, що він має за свою передумову лише зміну в розподілі
тих капіталів, які вже існують і функціонують, отже, його
поле діяльности не обмежене абсолютним зростанням суспільного
багатства або абсолютними межами акумуляції. Капітал зростає
великими масами тут, в одних руках, бо він зникає там, з багатьох
рук. Це — централізація у власному значенні слова, відмінно
від акумуляції й концентрації.

Законів цієї централізації капіталів або притягання капіталу
капіталом ми не можемо тут досліджувати. Досить буде коротких
фактичних вказівок. Конкуренційна боротьба провадиться через
здешевлення товарів. Дешевина товарів залежить, за інших
незмінних обставин, від продуктивности праці, а ця остання
залежить від маштабу продукції. Тим то більші капітали побивають
дрібніші. Пригадаймо собі, далі, що з розвитком капіталістичного
способу продукції зростає мінімальний розмір індивідуального
капіталу, потрібного на те, щоб провадити підприємство
в нормальних умовах. Тому дрібніші капітали ринуть
у такі сфери продукції, що їх велика промисловість опановує лише
спорадично або не цілком. Конкуренція лютує тут просто пропорційно
до числа й зворотно пропорційно до величини капіталів,
що борються між собою. Вона завжди кінчаєтеся загином багатьох
дрібних капіталістів, що їхні капітали почасти переходять до
рук переможців, а почасти гинуть. Крім цього, разом з капіталістичною
продукцією постає цілком нова сила, кредит, що спочатку
потайки прокрадається як скромний помагач акумуляції,
незримими нитками стягує в руки індивідуальних або асоційованих
капіталістів грошові засоби, розпорошені більшими або
меншими масами по поверхні суспільства; але незабаром він стає
новою і страшною зброєю в конкуренційній боротьбі і, кінець-кінцем,
перетворюється на велетенський соціяльний механізм
для централізації капіталів.

Тією самою мірою, як розвивається капіталістична продукція
й акумуляція, розвиваються також конкуренція і кредит,
ці обидві наймогутніші підойми централізації. Поруч цього
проґрес акумуляції збільшує матеріял, що його можна централізувати,
тобто збільшує поодинокі капітали, тимчасом як поширення
капіталістичної продукції утворює, з одного боку,
суспільну потребу, а з другого — технічні засоби для тих потужтшх
\index{i}{0538}  %% посилання на сторінку оригінального видання 
промислових підприємств, що здійснення їх зв’язане з попередньою
централізацією капіталу. Отже, за наших часів сила
взаємного притягання поодиноких капіталів і тенденція до централізації
дужча, ніж коли-будь раніш. Але, хоч відносне поширення
й енерґія руху в напрямі централізації визначається до
деякої міри досягнутою вже величиною капіталістичного багатства
й вищістю економічного механізму, проте проґрес централізації
зовсім не залежить від позитивного зростання величини
суспільного капіталу. І саме це й відрізняє централізацію від
концентрації, яка є лише інший вираз репродукції в поширеному
маштабі. Централізація може відбуватися через просту зміну
в розподілі капіталів, що вже існують, через просту зміну кількісного
угруповання складових частин суспільного капіталу.
Капітал тут може в одних руках зрости до велетенських розмірів,
бо там його витягнуто з багатьох поодиноких рук. У кожній
даній галузі підприємства централізація досягла б своєї крайньої
межі, коли б усі вкладені в неї капітали злилися в одинодним
капітал У кожному даному суспільстві цієї межі було б
досягнуто лише в той момент, коли цілий суспільний капітал
було б сполучено або в руках одного окремого капіталіста, або
в руках одним-одного товариства капіталістів.

Централізація доповнює справу акумуляції, даючи промисловим
капіталістам змогу поширювати маштаб своїх операцій.
Чи буде цей останній результат наслідком акумуляції або централізації;
чи відбувається централізація насильницьким шляхом
анексії, — коли деякі капітали стають такими потужними
центрами притягання для інших, що вони руйнують їхню індивідуальну
з’єднаність і потім притягують до себе ці роз’єднані
частини; чи злиття маси капіталів уже утворених або таких,
що перебувають у процесі творення, відбувається лагіднішим
способом, через утворення акційних товариств, — економічний
ефект в усіх цих випадках лишається той самий. Зростання розмірів
промислових підприємств повсюди становить вихідний
пункт для ширшої організації спільної праці багатьох, для
ширшого розвитку її матеріяльних рушійних сил, тобто для
проґресивного перетворення розрізнених і рутинних процесів
продукції на суспільно комбіновані й науково організовані
процеси продукції.

Але ясно, що акумуляція, поступінне збільшення капіталу
за допомогою репродукції, яка з колової форми переходить у
спіралю, є надто повільний процес порівняно з централізацією,
що потребує лише зміни кількісного угруповання інтеґральних
частин суспільного капіталу. Світ і досі був би ще без залізниць,
коли б йому довелось чекати, доки акумуляція доведе поодинокі
капітали до такого розміру, що зробив би їх здатними до буду77b
[До четвертого видання. — Найновіші англійські й американські
«трести» прагнуть уже цієї мети, силкуючися з’єднати принаймні всі
великі підприємства певної галузі промисловосте в одно велике акційне
товариство з практичною монополією. — Ф. Е.].
\index{i}{0539}  %% посилання на сторінку оригінального видання 
вання залізниць. Навпаки, централізація за допомогою акційних
товариств досягла цього наче одним махом руки. Збільшуючи
і прискорюючи таким чином діяння акумуляції, централізація
одночасно поширює і прискорює ті перевороти в технічному
складі капіталу, що збільшують його сталу частину коштом його
змінної частини й тим зменшують відносний попит на працю.

Маси капіталу, що їх миттю збиває до купи централізація,
репродукуються та збільшуються так само, як і інші капітали,
тільки швидше, і таким чином вони стають новими могутніми
підоймами суспільної акумуляції. Отже, коли говорять про прогрес
суспільної акумуляції, то під нею за наших часів мовчки
розуміють і діяння централізацій.

Додаткові капітали, утворені в перебігу нормальної акумуляції
(див. розділ XXII, І), служать переважно як засоби
експлуатації нових винаходів, відкрить тощо, одним словом,
промислових удосконалень. Але з часом і для старого капіталу
приходить момент відновлення його голови й членів, момент,
коли він змінює свою шкуру й теж відроджується в такій удосконаленій
технічній формі, коли досить меншої маси праці,
щоб пускати в рух більшу масу машин і сировинних матеріялів.
Абсолютне зменшення попиту на працю, що звідси неминуче
випливає, буде, ясна річ, то більше, що більше в наслідок руху
централізації є вже нагромаджені великими масами капітали,
які пророблюють цей процес відновлення.\footnote*{
Наводимо тут переклад цього абзацу за другим німецьким виданням,
де його подано повніше: «Зростання розміру індивідуальних мас
капіталу стає матеріяльною базою постійного перевороту в самому способі
продукції. Капіталістичний спосіб продукції безупинно завойовує
такі галузі праці, що зовсім ще не підпорядковані йому або підпорядковані
лише спорадично або лише формально. Поруч цього на ґрунті
цього ж способу продукції виростають нові галузі праці, що а самого початку
належать до нього. Нарешті, в галузях праці, проваджуваних
уже капіталістично, продуктивна сила праці виростає, наче в теплиці.
В усіх цих випадках число робітників знижується порівняно до маси
оброблюваних ними засобів продукції. Чимраз більша частина капіталу
перетворюється на засоби продукції, чимраз менша — на робочу силу.
Разом із збільшенням розмірів, концентрації і технічного діяння
засобів продукції, проґресивно зменшується їхня роля як засобів, що
дають заняття робітникам. Паровий плуг куди ефективніший засіб продукції,
ніж звичайний плуг, але вмішена в ньому капітальна вартість
куди меншою мірою є засіб, що дає заняття робітникам, ніж коли б її
було зреалізовано в звичайному плузі. Спочатку якраз долучення нового
капіталу до старого дозволяє поширити і технічно зреволюціонізувати
речові умови процесу продукції. Але незабаром зміна складу і технічна
перебудова більшою або меншою мірою охоплює ввесь старий капітал,
для якого настав строк репродукції і який тому наново репродукується.
Ця метаморфоза старого капіталу до певної міри так само не залежить від
абсолютного зростання суспільного капіталу, як і централізація. Але ця
остання, що лише інакше розподіляє наявний суспільний капітал і з’єднує
багато старих капіталів в один капітал, і собі діє як потужний чинник
у цій метаморфозі старого капіталу». Ред.
}

Отже, з одного боку, додатковий капітал, що утворився в
розвитку акумуляції, притягує порівняно з своєю величиною
\parbreak{}  %% абзац продовжується на наступній сторінці
