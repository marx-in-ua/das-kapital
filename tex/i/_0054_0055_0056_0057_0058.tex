\parcont{}  %% абзац починається на попередній сторінці
\index{i}{0054}  %% посилання на сторінку оригінального видання
яке, так би мовити, існує тільки в їхніх головах. Тим то посідач
товарів мусить позичити їм свій язик або понавішувати на них
ярлички, щоб повідомити зовнішній світ про їхні ціни\footnote{
Дикун і напівдикун вживає при цьому свого язика інакше. Капітан
Перрі зауважує, приміром, про жителів західнього узбережжя затоки
Бафіна ось що: «В цьому випадку (при обміні продуктів) вони лижуть
її (подану їм річ) двічі язиком, після чого, здасться, вважають торг за задовільно
закінчений» («In this case\dots{} they licked it (the thing represented
to them) twice to their tongues, after which they seemed to consider
the bargain satisfactorily concluded»). У східніх ескімосів обмінювач
також кожний раз облизував одержувану при обміні річ. Коли на півночі
язик є таким чином за орган присвоєння, то нічого дивного, що на
півдні живіт вважається за орган нагромадженої власности; так, кафр
цінує багатство людини з її опасистости. Дійсно, кафри дуже розумні
люди, бо тимчасом як офіційний англійський санітарний звіт з 1864~\abbr{р.}
скаржиться на недостачу жиротворних субстанцій у більшої частини робітничої
кляси, лікар Гарвей, дарма що він і не відкрив кровобігу, того
самого року зробив собі кар’єру шахрайськими рецептами, що обіцяли
буржуазії й аристократії увільнити її від тягару надмірного жиру.
}. А що
вираз товарових вартостей у грошах є ідеальний, то для цієї
операції можна вживати також тільки уявлюваного, або ідеального
золота. Кожний товаропосідач знає, що, надавши вартості
своїх товарів форми ціни або форми уявлюваного золота, він
ще далеко не перетворив на золото свої товари, і що йому не
треба ані крихітки реального золота, щоб оцінити мільйони
товарових вартостей у золоті. Отже, в їхній функції міри вартостей
гроші служать тільки як уявлювані, або ідеальні гроші.
Ця обставина породила якнайбезглуздіші теорії\footnote{
Див. \emph{К.~Marx}: «Zur Kritik der Politischen Oekonomie». — «Theorien
von der Messeinheit des Geldes», S. 53 ff. (\emph{K.~Маркс}: «До критики
політичної економії». — «Теорії про одиницю міри грошей», ДВУ, 1926~\abbr{р.},
стор. 91 і далі).
}. Хоч функцію
міри вартостей виконують лише уявлювані гроші, все ж ціни
цілком залежать від реального грошового матеріялу. Вартість,
тобто кількість людської праці, що міститься, наприклад, в
одній тонні заліза, виражається в уявлюваній кількості грошового
товару, яка містить у собі рівно стільки ж праці. Отже,
залежно від того, чи золото, срібло або мідь служать за міру
вартости, вартість тонни заліза набирає цілком різних виразів
ціни, або репрезентується в цілком різних кількостях золота,
срібла або міді.
\enablefootnotebreak{}

Тому, коли два різні товари, приміром, золото й срібло,
служать одночасно за міру вартости, то всі товари мають два
різні вирази для своїх цін — золоті ціни й срібні ціни, які спокійнісінько
існують одні побіч одних, доки вартостеве відношення
між золотом й сріблом лишається незмінне, приміром,
$1 : 15$. Але всяка зміна цього вартостевого відношення порушує
відношення між золотими й срібними цінами товарів, і таким
чином доводить фактично, що подвійність міри вартости суперечить
її функції\footnote{
Примітка до другого видання. «Там, де золото й срібло на підставі
закону функціонують одне поряд одного як гроші, тобто як міра вартости,
спроби розглядати їх як ту саму речовину завжди були даремні. Коли припустити, що той самий робочий
час мусить упредметнюватись незмінно в тій самій пропорції золота й срібла, то тим самим фактично
припускається, що срібло й золото є та сама речовина, і що певна маса менш вартісного металю,
срібла, становить незмінну частину певної маси золота. Починаючи від королювання Едварда III аж до
часів Ґеорґа II, історія англійської грошової справи являє собою раз-у-раз ряд порушень, що
поставали з колізії між законодавчо фіксованим вартостевим відношенням
золота до срібла та дійсними коливаннями їхньої вартости. То золото цінилося занадто високо, то
срібло. Металь, який цінилося занадто низько, вилучалося з циркуляції, його перетоплювали й вивозили
за кордон. Потім вартостеве відношення обох металів знову мінялось законом, але й нова номінальна
вартість незабаром входила з дійсним вартостевим відношенням у такий самий конфлікт, як і стара. —
За наших часів дуже
незначне й тимчасове падіння вартости золота проти срібла, як наслідок індійсько-китайського попиту
на срібло, викликало у Франції те саме явище в найбільшім розмірі: вивіз срібла й витиснення його з
циркуляції золотом. Протягом 1855, 1856 і 1857~\abbr{рр.} надмір довозу золота до Франції понад вивіз
золота з Франції становив \num{41.580.000}\pound{ фунтів стерлінґів}, тимчасом як надмір вивозу срібла понад довіз
срібла становив \num{14.704.000}\pound{ фунтів стерлінґів}. Справді, у країнах, де обидва металі є визнані законом
міри вартости, отже, де обидва повинні прийматись при оплатах, але кожний може собі платити як хоче,
— золотом чи сріблом, — у таких країнах металь, який підноситься у вартості, набуває ажіо й виміряє
свою ціну, — як і кожний інший товар, — у металі, що ціниться занадто високо, тимчасом як виключно
цей останній служить за міру вартости. Ввесь історичний досвід у цій галузі сходить просто на те, що
скрізь, де законом надано двом товарам функції міри вартости, фактично завжди лише один із них
зберігає цю функцію». (\emph{К.~Marx}: «Zur Kritik der Politischen Oekonomie», S. 52--53. — \emph{K.~Маркс}: «До
критики політичної економії», ДВУ, 1926~\abbr{р.}, стор. 89--90).
}.

\index{i}{0055}  %% посилання на сторінку оригінального видання
Товари, що їхні ціни є визначені, з’являються у формі: \emph{а}
товару \emph{А} \deq{} \emph{х} золота; \emph{b} товару \emph{B} \deq{} \emph{z} золота;
\emph{с} товару \emph{С} \deq{} \emph{y} золота й~\abbr{т. д.},
де \emph{а}, \emph{b}, \emph{c} являють собою певні маси товарових родів
\emph{А}, \emph{В}, \emph{С}, а \emph{х}, \emph{z}, \emph{у} — певні маси золота.
Таким чином товарові вартості перетворено в уявлювані кількості золота різної
величини, отже, не зважаючи на строкату різноманітність товарових тіл, товарові вартості перетворено
у величини однойменні, у величини золота. Як такі різні кількості золота вони порівнянні й
спільномірні між собою, при чому виникає технічна конечність звести їх до якоїсь фіксованої
кількости золота як до їхньої одиниці міри. Сама ця одиниця міри через дальший поділ на
аліквотні частини розвивається у маштаб. Золото, срібло, мідь ще до того, як вони стали грішми,
мають уже такі маштаби в своїх мірах ваги, так що, коли за одиницю міри служить, приміром, 1 фунт,
то, з одного боку, він поділяється знову на унції й~\abbr{т. ін.}, а з другого — складається в центнери й~\abbr{т. ін.}\footnote{
Примітка до другого видання. Та дивна обставина, що в Англії унція золота як одиниця грошового
маштабу неподільна на аліквотні частини, пояснюється ось як: «Нашу монетну справу було пристосовано
спершу лише до вжитку срібла, — тому унція срібла може бути завжди поділена на певну кількість цілих
монет; а що золото пізніш заведено в монетну систему, пристосовану лише до срібла, то з унції золота
не можна
викарбувати відповідного числа монет» («Our coinage was originally adapted to the employment of
silver only — hence an ounce of silver can always be divided into a certain adequate number of
pieces of coin; but as gold
was introduced at a later period into a coinage adapted only to silver, an
ounce of gold cannot be coined into an adequate number of pieces»). (\emph{Maclaren}:
«History of the Currency», London 1858, p. 16).
}
\index{i}{0056}  %% посилання на сторінку оригінального видання
Тим то за всякої металевої циркуляції наявні вже назви вагового
маштабу становлять і первісні назви грошового маштабу,
або маштабу цін.

\looseness=1
Як міра вартостей і маштаб цін гроші виконують дві цілком
різні функції. Мірою вартостей вони є як суспільне втілення
людської праці, маштабом цін — як фіксована вага металю.
Як міра вартости вони служать на те, щоб перетворювати вартості
дуже різноманітних товарів на ціни, на уявлювані кількості
золота; як маштаб цін вони вимірюють ці кількості золота. Мірою
вартостей товари вимірюються як вартості, навпаки, маштаб
цін вимірює різні кількості золота якоюсь його даною кількістю,
а не вартість якоїсь кількости золота вагою іншої кількости.
Для маштабу цін якась певна вага золота мусить бути фіксована
як одиниця міри. Тут, як і за всіх інших визначень міри однойменних
величин, вирішує справу стійкість одиниці міри. Таким
чином маштаб цін виконує свою функцію то краще, що незмінніше
та сама кількість золота служить за одиницю міри. За міру
вартости золото може служити лише тому, що воно самé є продукт
праці, отже, в можливості змінна вартість\footnote{
Примітка до другого видання. У творах англійських авторів
панує неймовірна плутанина понять міри вартостей (measure of value)
і маштабу цін (standard of value). Вони постійно плутають ці функції,
а через це і їхні назви.
}.

Насамперед ясно, що зміна вартости золота ніяким чином не
заважає його функції як маштабу цін. Хоч як змінюватиметься
вартість золота, різні кількості його лишаються завжди в тому
самому вартостевому відношенні поміж собою. Коли б вартість
золота зменшилась на 1000\%, то 12 унцій золота, як і раніш, мали б
у 12 разів більше вартости, ніж одна унція, а в цінах ідеться
лише про відношення різних кількостей золота поміж собою.
А що, з другого боку, вага однієї унції золота з пониженням
або підвищенням її вартости лишається цілком незмінна, то так
само мало змінюється й вага аліквотних частин її; таким чином
золото як сталий маштаб цін завжди робить ті самі послуги, хоч
як змінюється його вартість.

Зміна вартости золота не заважає і його функції як міри вартости.
Ця зміна стосується всіх товарів одночасно, отже, за інших
незмінних умов, лишає їхні взаємні відносні вартості незмінними,
хоч ці останні визначатимуться тепер всі у вищих або
нижчих золотих цінах, аніж раніш\footnote*{
До цього Маркс дає у французькому виданні «Капіталу» таку примітку:
«Вартість грошей може безупинно змінюватись і все ж таки вони
можуть служити за міру вартости так само добре, як тоді, коли б їхня
вартість лишалась цілком сталою». (\emph{Bailey}: «Money and its Vicissitudes»,
London 1837, p. 11). \emph{Ред.}
}.

Так за виразу вартости якогось товару в споживній вартості
якогось іншого товару, як і за цінування товарів у золоті припускається
\index{i}{0057}  %% посилання на сторінку оригінального видання
лише одно, — що за даного часу продукція якоїсь певної
кількости золота коштує даної кількости праці. Щодо руху
товарових цін взагалі, то для нього мають силу всі розвинені
вище закони простого відносного виразу вартости.

За незмінної вартости грошей загальне підвищення товарових
цін може статись лише тоді, коли підносяться вартості товарів;
за незмінних вартостей товарів — коли вартість грошей
падає. Навпаки: за незмінної вартости грошей загальне зниження
товарових цін може статись лише тоді, коли падають
товарові вартості; за незмінних товарових вартостей — коли
підноситься вартість грошей. Звідси аж ніяк не випливає, що
підвищення вартости грошей завжди спричинюється до пропорційного
зниження товарових цін, а зниження вартости грошей —
до пропорційного зросту товарових цін. Це має силу лише для
товарів з незмінною вартістю. Такі товари, вартість яких, приміром,
зростає рівномірно й одночасно з вартістю грошей, зберігають
ті самі ціни. Коли вартість їхня підноситься повільніш
або швидше, ніж вартість грошей, то зниження або підвищення
їхніх цін визначається ріжницею між рухом їхньої вартости й
рухом вартости грошей і~\abbr{т. д.}

А тепер повернімось до розгляду форми ціни.

\looseness=1
[Ми бачили, що звичаєві назви й підподіли вагового маштабу
металів служать також за маштаб цін]\footnote*{
Заведене у прямі дужки ми беремо з французького видання. \emph{Ред.}
}. Але поволі грошові
назви вагових кількостей металю відокремлюються від первісних
назов його ваги з різних причин; з них історично вирішальне
значення мають ось які: 1) Заведення чужоземних грошей у
менш розвинених народів, як от, приміром, у старому Римі
срібні й золоті монети спочатку циркулювали як чужоземні
товари; назви цих чужоземних грошей відрізнялись від тубільних
назов ваги. 2) З розвитком багатства менш благородний
металь витискується більш благородним з функції мірила вартости;
мідь витискується сріблом, срібло — золотом, хоч як це
чергування може перечити всякій поетичній хронології\footnote{
А проте це чергування не має загального історичного значення.
}. Фунт
стерлінґів був, наприклад, грошовою назвою дійсного фунта
срібла. Скоро тільки золото витиснуло срібло як міру вартости,
цю саму назву почали прикладати, може, лише до \sfrac{1}{15} і~\abbr{т. ін.} фунта
золота, залежно від відношення між вартістю золота й
срібла. Фунт як грошова назва і звичаєва вагова назва золота
тепер відокремились\footnote{
Примітка до другого видання. Так, англійський фунт означає
менш, ніж одну третину його первісної ваги, шотляндський фунт перед
унією — лише \sfrac{1}{36}, французький лівр — \sfrac{1}{74}, еспанський мараведі — менш
як \sfrac{1}{1000}, портуґальський рей — ще куди меншу частку.
}. 3)~Протягом століттів різні князі систематично
фалшували монету, в наслідок чого від первісної ваги
грошових монет лишилась фактично тільки назва\footnote{
Примітка до другого видання. «Монети, що їх назви тепер уже
лише ідеальні, є найдавніші монети кожної нації, і всі вони колись
були реальні, а що були реальні, то служили за рахункові гроші» («Le
monete, le quali oggi sono ideali, sono le ріù antiche d’ogni nazione, e tutte
furono un tempo reali, e perchè eiano reali con esse si contava»). (\emph{Galiani}:
«Della Moneta», p. 153).
}.

\index{i}{0058}  %% посилання на сторінку оригінального видання
В наслідок цих історичних процесів, відокремлення назви
грошей від їхніх звичаєвих вагових назов стає народньою звичкою.
А що грошовий маштаб, з одного боку, є цілком умовний,
а з другого боку, потребує загального визнання, то його, кінець-кінцем,
реґулює закон. Певну вагову частину благородного
металю, наприклад, одну унцію золота, офіціяльно поділяють
на аліквотні частини, що в законному хрищенні дістають
такі назви як фунт, таляр і~\abbr{т. д.} Кожну таку аліквотну частину,
що її тоді вважається за справжню одиницю грошової міри, підподіляється
на інші аліквотні частини, що в законному хрищенні
дістають такі назви як шилінґ, пенні і~\abbr{т. д.}\footnote{
Примітка до другого видання. Пан Давід Уркварт у своїй «Familiar
Words» зауважує про той страхітний (!) факт, що нині фунт, одиниця
англійського маштабу грошей, дорівнює приблизно \sfrac{1}{4} унції золота, таке:
«Це фальсифікація міри, а не встановлення маштабу» («This is falsifying
a measure, not establishing a standard»). В цьому «фалшивому
найменуванні» ваги золота він находить, як і скрізь, лише фальсифікаторську
руку цивілізації.
} Певні кількості ваги
металю і далі, як і раніш, лишаються маштабом металевих грошей.
Що змінилося, так це тільки поділ на частини й надання назов.

Отже, ціни або кількості золота, на які ідеально перетворено
вартості товарів, виражаються тепер у грошових назвах, або в
законно визнаних рахункових назвах маштабу золота. Таким
чином замість казати: квартер пшениці дорівнює одній унції
золота, в Англії сказали б: він дорівнює 3\pound{ фунтам стерлінґів}
17\shil{ шилінґам} 10\sfrac{1}{2}\pens{ пенсам.} Таким чином товари у своїх грошових
назвах виражають, чого вони варті, а гроші функціонують як
рахункові гроші кожного разу, як тільки треба фіксувати якусь
річ як вартість, а тому і в грошовій формі\footnote{
Примітка до другого видання. «Коли запитали Анахарсіса, на що
еллінам гроші, він відповів: на те, щоб рахувати». (\emph{Athenaeus}: «Deipnosophistai»,
1. IV, 49 v. 2 ed. Schweighäuser, 1802).
}.

Назва якоїсь речі є щось цілком зовнішнє супроти її природи.
Я не знаю нічого про людину, знаючи лише, що вона зветься
Яків. Так само в грошових назвах: фунт, таляр, франк, дукат
і~\abbr{т. ін.} зникає всякий слід вартостевого відношення. Плутанина
щодо таємного значення цих кабалістичних знаків є то значніша,
що грошові назви виражають одночасно і вартість товарів,
і аліквотні частини ваги металю, грошового маштабу\footnote{
Примітка до другого видання. «Через те, що гроші, як маштаб цін,
мають ту саму рахункову назву, що й ціни товарів, — отже, приміром,
одна унція золота так само, як і вартість тонни заліза, виражається в
3\pound{фунтах стерлінґів} 17\shil{ шилінґах} 10\sfrac{1}{2}\pens{ пенсах}, — то цим їхнім рахунковим
назвам дали назву ціни монети. Звідси постало дивовижне уявлення, що
начебто золото (або срібло) цінується у його власному матеріялі і, відмінно
від усіх інших товарів, дістає від держави фіксовану ціну. Фіксацію
рахункових назов певних вагових кількостей золота сплутано з
фіксацією вартостей цих кількостей ваги». (\emph{К.~Marx}: «Zur Kritik der
Politischen Oekonomie», S. 52. — \emph{K.~Маркс}: «До критики політичної
економії», ДВУ, 1926~\abbr{р.}, стор. 89).
}. З другого
\parbreak{}  %% абзац продовжується на наступній сторінці
