\parcont{}  %% абзац починається на попередній сторінці
\index{i}{0162}  %% посилання на сторінку оригінального видання
щоб могла увібрати в себе ту кількість праці, що має витрачатися протягом процесу продукції. Коли цю
масу дано, то, чи піднесеться її вартість, чи впаде, або й зовсім не матиме вартости, як от земля й
море, процесу творення вартости й зміни вартости це ані трохи не порушує\footnote{
Примітка до другого видання. Само собою зрозуміло, що, як каже Лукрецій, «nil posse creari de
nihilo», з нічого не можна створити нічого. «Створення вартости» є перетворення робочої сили на
працю. З свого боку й робоча сила є насамперед речовина природи, перетворена на людський організм.
}.

Отже, ми насамперед припускаємо, що стала частина капіталу дорівнює нулеві. Тим то авансований
капітал зводиться з $с \dplus{} v$ на $v$, а вартість продукту
$с \tikz[na] \coordinate (c-162-node-1);
\dplus{}
\tikz[na] \coordinate (v-162-node-1); v
\dplus{} m$
\begin{tikzpicture}[overlay]
    \path[-,black,transform canvas={yshift=1.5mm}] (c-162-node-1) edge [out=30, in=150] (v-162-node-1);
\end{tikzpicture}%
на новоспродуковану вартість $v \dplus{} m$.
Коли припустити, що новоспродукована вартість дорівнює 180\pound{ фунтам стерлінґів}, в яких виражається
праця, що триває протягом цілого процесу продукції, то нам треба відняти вартість змінного капіталу,
що дорівнює 90\pound{ фунтам стерлінґів}, щоб знайти
додаткову вартість, яка дорівнює 90\pound{ фунтам стерлінґів}. Число 90\pound{ фунтів стерлінґів} \deq{} $m$ виражає тут
абсолютну величину випродукованої додаткової вартости. Але її відносна величина, тобто пропорція, в
якій змінний капітал зріс своєю вартістю, визначається, очевидно, відношенням додаткової вартости до
змінного капіталу, або виражається дробом \frac{m}{v}. Отже ж, у наведеному випадку вона виражається у $\smash{\frac{90}{90}} \deq{} 100\%$.
Це відносне зростання
змінного капіталу, або відносну величину додаткової вартости, я називаю нормою додаткової
вартости\footnoteWithFotnote{
Так само, як англійці кажуть «rate of profits», «rate of interest»\footnote*{
«норма зиску», «норма процента». \emph{Ред.}
} і~\abbr{т. д.} у книзі III ми
побачимо, що норму зиску легко зрозуміти, якщо тільки знати закони додаткової вартости. Протилежним
шляхом не можна зрозуміти ni l’un ni l’autre\footnote*{
ні того, ні другого. \emph{Ред.}}.}.

Ми вже бачили, що робітник протягом однієї частини процесу праці продукує лише вартість своєї
робочої сили, тобто вартість доконечних для нього засобів існування. А що він продукує за відносин,
які ґрунтуються на суспільному поділі праці, то він продукує свої засоби існування не безпосередньо,
а у формі якогось окремого товару, приміром, пряжі, тобто продукує вартість, рівну вартості його
засобів існування, або грошам, за які
він ці засоби купує. Та частина його робочого дня, яку він уживає на це, буде більша або менша,
залежно від вартости його пересічних щоденних засобів існування, отже, залежно від пересічного
робочого часу, щоденно потрібного на їхню продукцію. Коли вартість його щоденних засобів існування
репрезентує пересічно
6 упредметнених робочих годин, то робітник мусить працювати пересічно по 6 годин денно, щоб
спродукувати цю вартість. Коли б він працював не на капіталіста, а на себе самого, незалежно,
\parbreak{}  %% абзац продовжується на наступній сторінці
