галузях продукції, незмінною. Отже, після заведення машин,
як і перед тим, суспільство має таку ж саму або більшу кількість
засобів існування для звільнених робітників, зовсім не кажучи
вже про ту величезну частину річного продукту, що її марно
тратять неробітники. І в цьому вся премудрість (pointe) економічної
апологетики! Суперечностей та антагонізмів, мовляв, невіддільних
від капіталістичного вживання машин, не існує, бо
вони виростають не з самих машин, а з капіталістичного вживання
їх! Отже, що машина, розглядувана сама по собі, скорочує
робочий час, а застосована капіталістично здовжує робочий день,
що сама по собі вона полегшує працю, а застосована капіталістично
підносить її інтенсивність, що сама по собі вона є перемога
людини над силою природи, а капіталістично застосована
поневолює людину силою природи, що сама по собі збільшує багатство
продуцента, а капіталістично застосована павперизує
його й т. ін., то буржуазний політико-економ просто заявляє,
що, як доводить розгляд машини самої по собі якнайяскравіше,
всі ті очевидні суперечності є проста видимість ницої дійсности,
а сами по собі, отже, і в теорії, вони зовсім не існують. Таким чином
він звільняє себе від клопоту далі ламати собі голову та,
крім того, накидає своєму супротивникові таку дурість, наче той
бореться не з капіталістичним застосуванням машин, а з самою
машиною.

Правда, буржуазний економіст не заперечує, що при цьому
бувають і тимчасові неприємності; але не буває медалі без зворотного
боку! Для нього неможливе якесь інше, а не капіталістичне
використовування машин. Отже, експлуатація робітника за допомогою
машин для нього ідентична з експлуатацією машини робітником.
Отже, той, хто викриває, яка в дійсності справа з капіталістичним
вживанням машин, той, мовляв, взагалі не хоче вживання
їх, той ворог соціяльного проґресу! 216 Це цілком нагадує
аргументацію славного горлоріза Біл Сайкса: «Панове присяжні,
певна річ, цим комівояжерам горло перерізано. Але цей вчинок
не моя вина, це вина ножа. Невже ж задля таких тимчасових
неприємностей нам скасувати вживання ножа? Подумайте ж!
Що сталося б із рільництвом і ремеством без ножа? Хіба ж не
дає він порятунку в хірургії, хіба можна без нього вчитися анатомії?
Та ще чи не бажаний він помічник на веселих бенкетах?
Позбавте нас ножа — і ви відкинете нас назад до часів найглибшого
варварства».\footnoteA{
«Винахідник прядільної машини зруйнував Індію, що нас,
однак, мало обходить». (A. Thiers: «De la Propriété», Paris 1848). Пан
}

216 Одним із віртуозів у цьому чванливому кретинізмі є Мак Куллох.
«Якщо корисно, — каже він, наприклад, з афектованою наївністю
восьмилітньої дитини, — щораз більше й більше розвивати вмілість робітника,
так щоб він став здатний продукувати щораз більшу кількість товарів
із тією самою або меншою кількістю праці, то не менш корисне мусить
бути й те, шоб він користувався з таких машин, які найуспішніше допомагали
б йому досягти цього результату». (Мас Culloch: «Principles
of Political Economy», London 1830, p. 182).