
\index{i}{0305}  %% посилання на сторінку оригінального видання
При багатьох ручних знаряддях ріжниця поміж людиною як
простою рушійною силою та як робітником, що виконує власне
ручну роботу, має почуттьово сприйману форму. Наприклад,
при самопряді нога діє лише як рушійна сила, тимчасом як рука,
що працює коло веретена, смикає та крутить, виконує власне
операцію прядіння. Саме цю останню частину ремісничого інструменту
захоплює насамперед промислова революція, залишаючи
людині поряд нової праці — наглядати за машиною та виправляти
своїми руками її хиби — насамперед ще й суто механічну ролю
рушійної сили. Навпаки, знаряддя, на які людина з самого початку
діє лише як проста рушійна сила, — приміром, крутячи корбою
млина\footnote{
Мойсей-єгиптянин каже: «Не зав’язуй рота волові, коли він молотить».
Навпаки, християнсько-германські філантропи клали на шию
кріпакові, якого вони вживали як рушійну силу для млина, великі дерев’яні
кола, щоб він не міг рукою підносити борошна до рота.
}, помпуючи, підіймаючи та спускаючи ручку ковальського
міха, товчучи в ступі й~\abbr{т. ін.}, — передусім викликають вживання
тварин, води, вітру\footnote{
Почасти брак природних водоспадів, почасти боротьба з надміром
стоячої води примусили голляндців використовувати вітер як рушійну
силу. Самий вітряк голляндці перейняли від Німеччини, де цей винахід
викликав «ввічливу» боротьбу між шляхтою, попами та імператором за те,
кому з них трьох «належить» вітер. Повітря закріпачує, казали в Німеччині,
тимчасом як Голляндію вітер зробив вільною. Що він тут закріпачував,
так це не голляндців, а землю для голляндців. Ще 1836~\abbr{р.}
в Голляндії вживали \num{12.000} вітряків у \num{6.000} кінських сил, щоб захистити
дві третини країни, не дати їм знову перетворитися на болото.
} як рухових сил. Почасти за мануфактурного
періоду, а спорадично вже задовго перед ним ці знаряддя
розвиваються на машини, алеж вони не революціонізують способу
продукції. Що вони навіть у своїй ремісничій формі вже є
машини, це виявляється в періоді великої промисловости. Смоки,
приміром, якими голляндці викачали 1836--37~\abbr{рр.} воду з Гаарлемського
озера, були сконструйовані за принципом звичайних
смоків з тією лише відміною, що їхнім толокам надавали руху не
людські руки, а циклопічні парові машини. Звичайний та дуже
неудосконалений ковальський міх ще й тепер іноді в Англії перетворюють
на механічний повітряний смок, сполучаючи його
ручку з паровою машиною. Сама парова машина, така, як її
винайдено наприкінці XVII віку, за мануфактурного періоду,
та якою вона і далі існувала до початку 80 років XVIII віку\footnote{
Правда, її вже дуже поліпшив Ватт за допомогою його першої
так званої парової машини простого чину, але в цій формі вона лишалася
простою машиною піднімати воду та ропу.
},
не викликала жодної промислової революції. Навпаки, саме створення
виконавчих машин зробило зреволюціонізовану парову
машину доконечною [а тому й можливою]\footnote*{
Заведене у прямі дужки ми беремо з першого німецького видання.
\Red{Ред.}
}. Скоро тільки людина,
замість діяти на предмет праці знаряддям, діє лише як рушійна
сила на виконавчу машину, то втілення цієї рушійної сили в
людські мускули стає випадковим, і вода, вітер, пара і~\abbr{т. д.} можуть
\parbreak{}  %% абзац продовжується на наступній сторінці
