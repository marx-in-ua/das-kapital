\parcont{}  %% абзац починається на попередній сторінці
\index{i}{0643}  %% посилання на сторінку оригінального видання
не лише згущення промислового пролетаріяту, як ось Жофруа
Сент-Ілер пояснює згущення світової матерії в одному
місці розрідженням її в іншому\footnote{
У своїх «Notions de Philosophie Naturelle», Paris 1838.
}. Не зважаючи на зменшення
числа обробників землі, вона давала стільки ж продуктів, скільки
й раніш, а то й більше, бо революція у відносинах земельної
власности відбувалася в супроводі поліпшених метод обробітку
землі, поширеної кооперації, концентрації засобів продукції й~\abbr{т. ін.}, бо сільських найманих робітників не тільки примушували
інтенсивніше працювати\footnote{
Цей пукт підкреслює сер Джемс Стюарт.
}, але й поле продукції, на якому
вони працювали сами на себе, дедалі більше скорочувалось.
Отже, разом із звільненням частини сільської людности звільняються
також і її колишні засоби існування. Вони перетворюються
тепер на речовий елемент змінного капіталу. Селянин,
позбавлений власности, мусить купувати вартість цих засобів
існування у свого нового пана, у промислового капіталіста, в
формі заробітної плати. З тубільним сировинним матеріялом,
постачуваним для промисловости рільництвом, сталося те саме,
що і з засобами існування. Він перетворився на елемент сталого
капіталу.

Припустімо, наприклад, що частину вестфальських селян,
які за часів Фрідріха II всі займалися прядінням, хоч і не шовку,
то льону, силоміць експропрійовано й зігнано з землі, а другу
частину їх, ту, що залишилась, перетворено на наймитів великих
фармерів. Одночасно виростають великі льонопрядні й ткацькі
підприємства, де ці «звільнені» працюють як наймані робітники.
Льон має цілком такий самий вигляд, як і раніш. Жодне волоконце
в ньому не змінилось, але нова соціяльна душа вселилась
у його тіло. Він становить тепер частину сталого капіталу власника
мануфактури. Поділений раніше поміж безлічі дрібних
продуцентів, що сами обробляли його й разом із своїми родинами
випрядали маленькими порціями, він тепер зосереджений у руках
одного капіталіста, що примушує інших прясти і ткати на
нього. Додаткова праця (Extraarbeit), витрачена в прядінні
льону, реалізувалася раніш у додатковому доході (Extraeinkommen)
безлічі селянських родин або також — за часів Фрідріха
II — у податках pour le roi de Prusse\footnote*{
для пруського короля. \emph{Ред.}
}. Тепер вона реалізується
в зиску небагатьох капіталістів. Веретена і ткацькі варстати,
порозділені раніш по селах, зосереджені тепер по небагатьох
великих робітних казармах, так само як робітники і сировинний
матеріял. І веретена, і ткацькі варстати, і сировинний
матеріял перетворюються відтепер із засобів незалежного існування
прядунів і ткачів на засоби панування\footnote{
«Я дозволю вам, — каже капіталіст, — мати честь служити мені
з умовою, що ви віддасте мені те небагато, що у вас залишилося, за ту
тяжку працю, яку я виконую, командуючи вами» (Je permettrai que
vous ayez l’honneur de me servir, à condition que vousme donnez le peu
qui vous reste pour la peine que je prends de vous commander»). (\emph{J.~J.~Rousseau}:
«Discours sur l’Economie Politique». Geneve 1760, p. 70).
} над ними й висисання
з них неоплаченої праці. По великих мануфактурах і по
\index{i}{0644}  %% посилання на сторінку оригінального видання
великих фармах зовсім не пізнати, що вони постали із об’єднання
багатьох дрібних майстерень і через експропріацію багатьох
дрібних незалежних продуцентів. Однак безстороннього глядача
це не введе в обман. За часів Мірабо, цього лева революції, великі
мануфактури ще називалися manufactures réunies, тобто об’єднаними
майстернями, як ми говоримо про об’єднані поля. «Звертають
увагу, — каже Мірабо, — лише на великі мануфактури,
що в них сотні людей працюють під керівництвом одного директора
і що їх звичайно звуть об’єднаними мануфактурами (manufactures
réunies). Навпаки, на ті майстерні, де працює дуже багато
робітників порізно і кожний на власну руку, навряд чи
хто й оком скине. Їх зовсім відсувають на задній плян. Це дуже
велика помилка, бо лише вони становлять дійсно важливу складову
частину народнього багатства\dots{} Об’єднана фабрика (fabrique
réunie) на диво збагачує одного або двох підприємців, але
робітники — це лише краще або гірше оплачувані поденники,
що не беруть ніякої участи в добробуті підприємця. Навпаки,
роз’єднана фабрика (fabrique séparée) нікого не збагачує, але
зате маса робітників живе в добробуті\dots{} Число працьовитих і
хазяйновитих робітників зростатиме, бо в розумному способі
життя, в працьовитості вони бачать засіб, щоб посутньо поліпшити
своє становище, замість здобувати собі невеличке підвищення
заробітної плати, що ніколи не може мати важливого
значення для будучини і в найкращому разі тільки дозволить
робітникам трохи краще жити з дня на день. Лише роз’єднані
індивідуальні мануфактури, сполучені здебільша з дрібним сільським
господарством, є вільні»\footnote{
\emph{Mirabeau}: «De la Monarchie Prussienne», Londres, 1788, vol. III,
p. 20--109 і далі. Коли Мірабо гадає, що роз’єднані майстерні є
також економічніші й продуктивніші, ніж майстерні «об’єднані», і ці
останні вважає лише за штучні, тепличні рослини, що виросли під охороною
уряду, то це пояснюється тодішнім станом більшої частини континентальних
мануфактур.
}. Експропріяція і зганяння
частини сільської людности не тільки звільняють разом із самими
робітниками їхні засоби існування і їхній матеріял праці для
промислового капіталу, а й створюють внутрішній ринок.

Справді, ті події, що перетворюють дрібних селян на найманих
робітників, а засоби їхнього існування і праці — на речові
елементи капіталу, створюють разом із цим для цього останнього
внутрішній ринок. Раніш селянська родина продукувала
і обробляла засоби існування й сировинні матеріяли, що їх вона
потім здебільшого сама ж і споживала. Ці сировинні матеріяли
й засоби існування тепер стали товарами; великий фармер продає
їх; мануфактури є його ринок. Пряжа, полотно, грубі вовняні
вироби — речі, що для них сировинний матеріял був у руках
кожної селянської родини, при чому кожна родина пряла і ткала
їх для власного споживання — перетворюються тепер на мануфактурні
\index{i}{0645}  %% посилання на сторінку оригінального видання
вироби, для яких саме рільничі округи становлять
ринок збуту. Численні тут і там порозкидані споживачі, досі
обслуговувані багатьма дрібними продуцентами, що працювали
на власну руку, сконцентровуються тепер в один великий ринок,
обслуговуваний промисловим капіталом\footnote{
«Коли двадцять фунтів вовни непомітно перетворюються на потрібний
протягом року для родини робітника одяг через власну працю
цієї родини, в перервах поміж іншими її роботами, то це не впадає в очі,
але винесіть цю вовну на ринок, відішліть її на фабрику, звідти до маклера,
потім до торговця — і ви матимете великі комерційні операції і
номінальний капітал у двадцять разів більший за вартість продукту\dots{}
Таким чином робітничу клясу визискується на те, щоб підтримувати
нужденну фабричну людність, паразитарну клясу крамарів і фіктивну
комерційну, грошову й фінансову систему» («Twenty pounds of wool
converted unobtrusively into the yearly clothing of a labourer’s family
by its own industry in the intervals of other work — this makes no show;
but bring it to market, send it to the factory, thence to the broker, thence
to the dealer, and you will have great commercial operations, and nominal
capital engaged to the amount of twenty times its value\dots{} The working
class is thus emerced to support a wretched factory population, a parasitical
shopkeeping class, and a fictitious commercial, monetary and financial
system»). (\emph{David Urquhart}: «Familiar Words», London 1855, p. 120).
}. Так пліч-о-пліч
з експропріяцією селян, що раніш господарювали самостійно, і
з відокремленням їх від засобів продукції відбувається нищення
сільської підсобної промисловости, процес відокремлення мануфактури
від рільництва. І лише знищення сільської домашньої
продукції може надати внутрішньому ринкові країни тих розмірів
і тієї сталости, що їх потребує капіталістичний спосіб продукції.

Однак період мануфактури у власному значенні слова ще не
приводить до радикального перевороту. Пригадаймо, що мануфактура
опановує національну продукцію лише частково, спорадично,
і завжди спирається на міське ремество й домашню
сільську підсобну промисловість як на свою широку базу.
Якщо мануфактура знищує домашню сільську підсобну промисловість
в одній формі, в осібних галузях продукції, у певних
пунктах, то вона створює їх знову в інших пунктах, бо вона до
певної міри потребує її для оброблювання сировинного матеріялу.
Тим то вона створює нову клясу дрібних рільників, що для них
оброблювання землі є лише підсобна галузь, а головне їхнє заняття
є промислова праця, що її продукт вони — безпосередньо
або посередньо через купця — продають мануфактурі. Це —
причина, хоч і не головна, того явища, яке насамперед спантеличує
дослідника англійської історії. Починаючи від останньої
третини XV століття, дослідник натрапляє там на постійні,
лише іноді на короткий час притихлі, нарікання на зріст капіталістичного
господарства на селі й на проґресивне нищення
селянства. З другого боку, він завжди знову знаходить там це
селянство, хоч і в зменшеній кількості і в дедалі гірших умовах\footnote{
Виняток являють собою часи Кромвела. Поки існувала республіка,
всі верстви англійської народньої маси піднеслися з того занепаду,
в якому вони були за Тюдорів.
}.
Головна причина цього ось у чому: в Англії навпереміну переважає
\index{i}{0646}  %% посилання на сторінку оригінального видання
то рільництво, то скотарство, і залежно від цих періодів
змін коливаються й розміри селянської продукції. Лише велика
промисловість з її машинами дає сталу базу для капіталістичного
рільництва, радикально експропріює величезну більшість сільської
людности й вивершує відокремлення рільництва від домашньої
сільської промисловости, вириваючи її коріння: прядіння
і ткацтво\footnote{
Текет знає, що з мануфактур у власному значенні слова та із
зруйнування сільських або домашніх мануфактур виникає із заведенням
машин велика вовняна промисловість (\emph{Tuckett}: «A History of the Past
and Present State of the Labouring Population», vol. I, p. 139--143).
«Плуг, ярмо були винаходом богів і заняттям героїв; хіба ж ткацький
варстат, веретено й прядка менш благородні походженням? Ви відокремлюєте
прядку від плуга, веретено від ярма, і маєте фабрики й робітні
доми, кредит і паніку, дві ворожі нації, рільничу й комерційну».
(\emph{David Urquhart}: «Familiar Words», London 1855, p. 122). Але ось з’являється
Kepi і обвинувачує Англію, звичайно, не без підстав, у тому,
що вона намагається перетворити всі інші країни у виключно рільничі,
щоб стати для них фабрикантом. Він запевняє, що таким чином зруйновано
Туреччину, бо там «ніколи не дозволялось (Англією) землевласникам
і рільникам зміцнити своє становище через природну спілку плуга
з ткацьким варстатом, борони з молотком». («The Slave Trade», p. 125).
На його погляд, сам Уркварт є один із головних винуватців зруйнування
Туреччини, де він в інтересах Англії пропагував вільну торговлю.
Але найкраще те, що Кері, до речі великий холоп Росії, хоче за допомогою
протекційної системи спинити той процес відокремлення, що його вона
в дійсності прискорює.
}. Тим то лише вона завойовує для промислового
капіталу ввесь внутрішній ринок\footnote{
Філантропічні англійські економісти, як от Мілл, Роджерс, Ґолдвін,
Сміт, Фавсет і~\abbr{т. ін.}, та ліберальні фабриканти, як от Джон Брайт
і компанія, запитують англійських земельних аристократів, як бог
запитував Каїна про брата його Авеля: де поділись тисячі наших freeholder’ів?\footnote*{
самостійних селян. \emph{Ред.}
}.
Та ви то сами звідки взялися? Із знищення цих freeholder’ів.
Чому ви не питаєте далі: де поділись наші самостійні ткачі, прядуни,
ремісники?
}.

\subsection{Генеза промислового капіталіста}

Генеза промислового\footnote{
«Промисловий» уживається тут у протилежність до «рільничого».
В розумінні «категорії» фармер є такий самий промисловий капіталіст,
як і фабрикант.
} капіталіста відбувалася не з такою
поступінністю, як генеза фармера. Без сумніву, деякі дрібні
цехові майстрі, і ще більше самостійні дрібні ремісники і навіть
наймані робітники перетворювались на дрібних капіталістів,
а потім, поволі, за допомогою більше поширюваної експлуатації
найманої праці й відповідної акумуляції, — на капіталістів sans
phrase\footnote*{
попросту. \emph{Ред.}
}. У дитячому періоді капіталістичної продукції справа
здебільша стояла так, як у дитячому періоді середньовічного
міського ладу, де питання про те, хто з кріпаків-утікачів повинен
бути майстром, а хто слугою, вирішувано здебільша залежно
від того, хто з них раніш утік. Однак черепаша хода цієї методи
зовсім не відповідала торговельним потребам нового світового
\parbreak{}  %% абзац продовжується на наступній сторінці
