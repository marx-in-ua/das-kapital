\parcont{}  %% абзац починається на попередній сторінці
\index{i}{0367}  %% посилання на сторінку оригінального видання
галузях продукції, незмінною. Отже, після заведення машин,
як і перед тим, суспільство має таку ж саму або більшу кількість
засобів існування для звільнених робітників, зовсім не кажучи
вже про ту величезну частину річного продукту, що її марно
тратять неробітники. І в цьому вся премудрість (pointe) економічної
апологетики! Суперечностей та антагонізмів, мовляв, невіддільних
від капіталістичного вживання машин, не існує, бо
вони виростають не з самих машин, а з капіталістичного вживання
їх! Отже, що машина, розглядувана сама по собі, скорочує
робочий час, а застосована капіталістично здовжує робочий день,
що сама по собі вона полегшує працю, а застосована капіталістично
підносить її інтенсивність, що сама по собі вона є перемога
людини над силою природи, а капіталістично застосована
поневолює людину силою природи, що сама по собі збільшує багатство
продуцента, а капіталістично застосована павперизує
його й~\abbr{т. ін.}, то буржуазний політико-економ просто заявляє,
що, як доводить розгляд машини самої по собі якнайяскравіше,
всі ті очевидні суперечності є проста видимість ницої дійсности,
а сами по собі, отже, і в теорії, вони зовсім не існують. Таким чином
він звільняє себе від клопоту далі ламати собі голову та,
крім того, накидає своєму супротивникові таку дурість, наче той
бореться не з капіталістичним застосуванням машин, а з самою
машиною.
\enablefootnotebreak{}

Правда, буржуазний економіст не заперечує, що при цьому
бувають і тимчасові неприємності; але не буває медалі без зворотного
боку! Для нього неможливе якесь інше, а не капіталістичне
використовування машин. Отже, експлуатація робітника за допомогою
машин для нього ідентична з експлуатацією машини робітником.
Отже, той, хто викриває, яка в дійсності справа з капіталістичним
вживанням машин, той, мовляв, взагалі не хоче вживання
їх, той ворог соціяльного проґресу!\footnote{
Одним із віртуозів у цьому чванливому кретинізмі є Мак Куллох.
«Якщо корисно, — каже він, наприклад, з афектованою наївністю
восьмилітньої дитини, — щораз більше й більше розвивати вмілість робітника,
так щоб він став здатний продукувати щораз більшу кількість товарів
із тією самою або меншою кількістю праці, то не менш корисне мусить
бути й те, щоб він користувався з таких машин, які найуспішніше допомагали
б йому досягти цього результату». (\emph{Мас Culloch}: «Principles
of Political Economy», London 1830, p. 182).
} Це цілком нагадує
арґументацію славного горлоріза Біл Сайкса: «Панове присяжні,
певна річ, цим комівояжерам горло перерізано. Але цей вчинок
не моя вина, це вина ножа. Невже ж задля таких тимчасових
неприємностей нам скасувати вживання ножа? Подумайте ж!
Що сталося б із рільництвом і ремеством без ножа? Хіба ж не
дає він порятунку в хірургії, хіба можна без нього вчитися анатомії?
Та ще чи не бажаний він помічник на веселих бенкетах?
Позбавте нас ножа — і ви відкинете нас назад до часів найглибшого
варварства»\footnoteA{
«Винахідник прядільної машини зруйнував Індію, що нас,
однак, мало обходить». (\emph{A.~Thiers}: «De la Propriété», Paris 1848). Пан
Тьєр переплутав тут прядільну машину з механічним ткацьким варстатом,
та «нас це, однак, мало обходить».
}.

\index{i}{0368}  %% посилання на сторінку оригінального видання
Хоч машина неминуче й витискує робітників по тих галузях
промисловости, де її заводять, все ж вона може викликати збільшення
праці по інших галузях праці. Але це діяння не має нічого
спільного з так званою теорією компенсації. А що кожний машиновий
продукт, наприклад, один метр машинової тканини, дешевший,
ніж витиснутий ним однорідний ручний продукт, то звідси випливає
такий абсолютний закон: якщо загальна кількість товарів,
випродукованих машиновим способом, лишається рівна загальній
кількості заміщуваних нею товарів, випродукованих ремісничим
або мануфактурним способом, то загальна сума вжитої праці
зменшується. Те збільшення праці, яке потрібне для продукції
самих засобів праці, машин, вугілля й~\abbr{т. д.}, мусить бути менше
від праці, заощадженої через уживання машин. У противному
разі машиновий продукт був би так само дорогий або й дорожчий,
ніж ручний продукт. Алеж фактично замість лишатися однаковою
вся маса машинового продукту, що його продукує зменшене
число робітників, зростає далеко понад загальну масу витиснутого
ремісничого продукту. Припустімо, що \num{400.000} метрів машинової
тканини продукується меншим числом робітників, ніж \num{100.000}
метрів ручної тканини. У збільшеному вчетверо продукті міститься
вчетверо більше сировинного матеріялу. Отже, продукцію
сировинного матеріялу треба збільшити в чотири рази. Щождо
спожитих засобів праці, як от будівлі, вугілля, машини й~\abbr{т. д.},
то межі, в яких може зростати додаткова праця, потрібна на їхню
продукцію, змінюються відповідно до ріжниці поміж масою машинового
продукту й масою ручного продукту, яку може виготовити
те саме число робітників.

Отже, з поширенням машинового виробництва в якійсь одній
галузі промисловости більшає насамперед продукція в тих інших
галузях, які постачають їй її засоби продукції. Якою мірою
через те зростає маса занятих робітників, це залежить, за
даної довжини робочого дня й інтенсивности праці, від складу
вжитих капіталів, тобто від відношення між їхніми сталими та
змінними складовими частинами. Це відношення, з свого боку,
дуже варіює залежно від того обсягу, в якому машини вже захопили
або захоплюють ті галузі промисловости. Число робітників,
засуджених на працю по копальнях вугілля й металю, страшенно
зросло з розвитком англійської машинової системи, хоч це зростання
останніми десятиліттями уповільнюється в наслідок заведення
нових машин у гірництві\footnote{
За переписом 1861~\abbr{р.} (Vol. II.~London 1863) число робітників,
що працювали по копальнях вугілля Англії та Велзу, становило \num{246.613},
з них \num{73.545} молодші й \num{173.067} понад 20 років віку. До першої рубрики
належать 835 від п’яти до десяти років, \num{30.701} від десяти до п’ятнадцяти,
\num{42.010} від п’ятнадцяти до дев’ятнадцяти років. Число занятих
на копальнях заліза, міді, олива, цини та всіх інших металів — \num{319.222}.
}. Разом з машиною увіходить
у життя новий рід робітника — її продуцент. Ми вже знаємо, що
машинове виробництво завойовує в чимраз більшому розмірі й
\index{i}{0369}  %% посилання на сторінку оригінального видання
цю галузь продукції\footnote{
1861~\abbr{р.} в Англії та Велзі працювало коло продукції машин \num{60.807}
осіб, залічуючи сюди і фабрикантів з їхніми прикажчиками й~\abbr{т. д.} і всіх
аґентів та купців цієї галузі, але виключаючи продуцентів невеличких
машин, як от швацьких машин і~\abbr{т. ін.}, так само й продуцентів знарядь
до робочих машин, як от веретен і~\abbr{т. ін.} Число всіх цивільних інженерів
становило \num{3.329}.
}. Далі щодо сировинного матеріялу\footnote{
Що залізо — найважливіший сировинний матеріял, то треба тут
зауважити, що 1861~\abbr{р.} в Англії та Велзі був \num{125.771} робітник, які
працювали по залізоливарнях, з того \num{123.430} чоловіків, \num{2.341} жінка.
З-поміж перших \num{30.810} молодші і \num{92.620} понад 20 років.
}, то не підлягає ніякому сумнівові, наприклад, що бурхливий розвиток
бавовняного прядільництва з оранжерійною швидкістю прискорив
культуру бавовнику в Сполучених штатах, а разом з нею
він не тільки прискорив африканську торговлю рабами, але
одночасно зробив вирощування негрів головним промислом так
званих прикордонних рабовласницьких штатів. Коли 1790~\abbr{р.}
зроблено в Сполучених штатах перший перепис рабів, число їх
становило \num{697.000}, а 1861 року — приблизно чотири мільйони.
З другого боку, не менш певне й те, що розцвіт механічної вовняної
фабрики разом з чимраз більшим перетворенням орного поля на
пасовиська для овець викликав масове вигнання сільських робітників
та перетворення їх на «зайвих». В Ірляндії ще й тепер
відбувається цей процес, який її людність, що від 1845~\abbr{р.} встигла
вже зменшитися майже наполовину, ще далі зменшує до тієї
міри, яка точно відповідає потребам її лендлордів та англійських
панів фабрикантів вовни.

Якщо машини захоплюють попередні або проміжні ступені,
які предмет праці має перебігти раніш, ніж він набирає своєї
остаточної форми, то разом із матеріялом праці більшає й попит
на працю в проваджуваних ремісничим або мануфактурним способом
галузях промисловости, які обробляють машиновий
фабрикат. Наприклад, машинове прядіння постачало пряжу так
дешево й так багато, що ручні ткачі спочатку могли, не збільшуючи
видатків, працювати повний час. Таким чином їхній дохід
збільшився\footnote{
«Родина з чотирьох дорослих осіб (бавовняних ткачів) з двома
дітьми, що працювали як winders, одержувала наприкінці останнього та
на початку цього століття 4\pound{ фунти стерлінґів} на тиждень за десятигодинного
робочою дня: якщо робота мала нагальний характер, вони могли
заробити й більше\dots{} Раніш вони завжди страждали від недостатнього
подання пряжі». (\emph{Gaskell}: «The Manufacturing Population of England»,
London 1833, p. 25--27).
}. Звідси наплив людей до пряділень бавовни,
доки, нарешті, \num{800.000} бавовняних ткачів, що їх в Англії покликали
були до життя машини jenny, throstle та mule, були
вбиті паровими ткацькими варстатами. Так само разом з надміром
матерії на одяг, продукованої машиновим способом, зростало
число кравців, кравчих, швачок і~\abbr{т. д.}, поки не з’явилася
швацька машина.
