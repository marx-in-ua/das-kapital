\parcont{}  %% абзац починається на попередній сторінці
\index{i}{0341}  %% посилання на сторінку оригінального видання
машин»\footnote{
\emph{John Fielden}: «The Curse of the Factory System», London
1836, p. 32.
}. 1844~\abbr{р.} лорд Ешлей, нині граф Шефтсбері, подав
у Палаті громад ось такі документально обґрунтовані дані:

«Праця робітників, занятих у фабричних процесах, нині утроє
більша, ніж тоді, як заведено такі операції. Безперечно, машини
виконали працю, що заміняє жили й мускули мільйонів людей,
але вони також навдивовижу (prodigiousli) збільшили працю
людей, підбитих під їхній страшний рух\dots{} Ходити за парою
мюлів туди й назад протягом 12 годин, при прядінні пряжі
№ 40 — така праця 1815~\abbr{р.} вимагала 8 миль ходьби. 1832~\abbr{р.}
дистанція, яку доводилося проходити протягом 12 годин, слідкуючи
за 2 мюлями при прядінні того самого нумера, становила
20 миль, а часто й більше. 1815~\abbr{р.} прядунові доводилося робити
коло кожного мюля за 12 годин 820 витягувань, що давало загальну
суму в \num{1.640} витягувань за 12 годин. 1832~\abbr{р.} прядунові
доводилося робити протягом свого дванадцятигодинного робочого
дня \num{2.200} витягувань коло кожного мюля, разом \num{4.400}; 1844~\abbr{р.}
коло кожного мюля \num{2.400}, разом \num{4.800} витягувань, а в деяких
випадках потрібна ще більша маса праці (amount of labour)\dots{}
Ось у мене в руках другий документ з 1842~\abbr{р.}, з якого видно,
що праця проґресивно збільшується не тільки через те, що доводиться
проходити більшу дистанцію, а ще й через те, що кількість
продукованих товарів більшає, тимчасом як число рук пропорційно
меншає: і далі, через те, що тепер часто прядеться гіршу
бавовну, яка вимагає більшої праці\dots{} В чухральному відділі
теж дуже збільшилася праця. Тепер одна особа виконує таку
працю, яку раніш розподіляли на дві особи\dots{} У ткацькому відділі,
в якому працює велике число осіб, здебільшого жіночої
статі, в наслідок більшої швидкости машин праця зросла за
останні роки на повні 10\%. 1838~\abbr{р.} за тиждень випрядали \num{18.000}
hanks\footnote*{
мотків. \Red{Ред.}
}, 1843~\abbr{р.} це число дійшло \num{21.000}. 1819~\abbr{р.} число picks\footnote*{
ударів човника. \Red{Ред.}
}
при паровому ткацькому варстаті становило 60 на хвилину, 1842~\abbr{р.}
воно становило 140, а це свідчить про дуже велике зростання
праці»\footnote{
\emph{Lord Ashley}: «The Ten Hours’ Factory Bill, Speech of the 15 th
March», London 1844, p. 6--9 і далі.
}.

За цієї дивовижної інтенсивности, якої праця досягла при пануванні
закону про дванадцятигодинний робочий день ще 1844~\abbr{р.},
здавалася правдивою заява англійських фабрикантів про те,
що всякий дальший поступ у цьому напрямі неможливий, а
тому і всяке дальше зменшування робочого часу ідентичне із
зменшенням продукції. Що їхні міркування були правдиві лише
на позір, це найкраще доводить дане в той самий час свідчення
невтомного цензора фабрикантів, фабричного інспектора Леонарда
Горнера:
