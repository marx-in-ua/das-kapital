
\index{i}{0175}  %% посилання на сторінку оригінального видання
Сума доконечної праці й додаткової праці, періодів часу,
протягом яких робітник продукує еквівалент вартости своєї робочої
сили й додаткову вартість, становить абсолютну величину
його робочого часу — робочий день (working day).

\sectionextended{Робочий день}{%
\subsection{Межі робочого дня}}

Ми виходили з тієї передумови, що робочу силу купується
й продається за її вартістю. Вартість її, як і вартість кожного
іншого товару, визначається робочим часом, потрібним на
її продукцію. Отже, коли на продукцію пересічних денних засобів
існування робітника потрібно 6 годин, то він мусить працювати
пересічно 6 годин на день, щоб продукувати щоденно свою робочу
силу, або щоб репродукувати вартість, яку він одержав при її
продажу. Доконечна частина його робочого дня становить тоді
6 годин; отже, за інших незмінних обставин, вона є дана величина.
Але цим іще величину самого робочого дня не дано.

Припустимо, що лінія $a\linerule{6}b$ репрезентує тривання
або довжину доконечного робочого часу, приміром, 6 годин. Відповідно
до того, чи праця буде здовжена поза межі $ab$ на 1, 3,
6 годин і~\abbr{т. ін.}, ми матимемо три різні лінії:

\bigskip
\begin{table}[H]
\centering
\noindent\begin{tabular}{l}
Робочий день I \\
$a\linerule{6}b\linerule{1}c$ \\
\addlinespace
Робочий день II \\
$a\linerule{6}b\linerule{3}c$ \\
\addlinespace
Робочий день III \\
$a\linerule{6}b\linerule{6}c$ \\
\end{tabular}
\end{table}

\bigskip
\noindent{}які репрезентують три різні робочі дні — 7, 9 і 12 годин. Лінія
здовження $bc$ репрезентує довжину додаткової праці. А що робочий
день $\deq{} ab \dplus{} bc$, або $ac$, то він змінюється разом із змінною
величиною $bc$. Тому що $ab$ дано, то відношення $bc$ до $ab$ завжди
можна виміряти. Воно становить у робочому дні I \sfrac{1}{6}, у робочому
дні II \sfrac{3}{6} і у робочому дні III \sfrac{6}{6} $ab$. Далі, через те що, відношення
$\frac{\text{додатковий робочий час}}{\text{доконечний робочий час}}$ визначає норму додаткової вартости, то
останню дано цим відношенням. Вона становить для трьох різних
робочих днів відповідно 16\sfrac{2}{3}, 50 і 100\%. Навпаки, сама норма
додаткової вартости не дала б нам величини робочого дня. Коли
б, приміром, вона дорівнювала 100\%, то робочий день міг би
тривати 8, 10 і 12 годин і~\abbr{т. ін.} Вона показувала б, що обидві
складові частини робочого дня, доконечна праця й додаткова
праця є однаково великі, але не показувала б, яка велика кожна
з цих частин.

Отже, робочий день є не стала, а змінна величина. Правда,
одну з його частин визначається робочим часом, потрібним для
постійної репродукції самого робітника, але його ціла величина
змінюється разом з довжиною, або триванням додаткової праці.
\parbreak{}  %% абзац продовжується на наступній сторінці
