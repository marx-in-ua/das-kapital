\parcont{}  %% абзац починається на попередній сторінці
\index{i}{0316}  %% посилання на сторінку оригінального видання
сили суспільної праці. Природні сили, як от пара, вода й~\abbr{т. д.},
що їх уживається до продуктивних процесів, теж нічого не
коштують. Але як людина потребує легенів, щоб дихати, так само
потребує вона якогось «витвору людської руки», щоб продуктивно
споживати природні сили. Водяне колесо потрібне, щоб
експлуатувати рушійну силу води, парова машина — щоб експлуатувати
пружність пари. З наукою справа така сама, як і з силами
природи. Коли вже відкрито закон про відхилення магнетової
голки у сфері дії електричного струму або про утворення магнетизму
в залізі електричним струмом, то ці закони не коштують
ані шеляга.\footnote{
Наука взагалі «нічого» не коштує капіталістові, та це ані трохи
не заважає йому експлуатувати її. Капітал присвоює собі «чужу» науку
так само, як і чужу працю. Але «капіталістичне» присвоєння й «особисте»
присвоєння, чи то науки, чи то матеріяльного багатства — це цілком різні
речі. Сам д-р Юр нарікав на грубу необізнаність з механікою його любих
фабрикантів, що експлуатують машини, а Лібіґ оповідає про таке неуцтво
в хемії англійських хемічних фабрикантів, від якого волосся стає дибом.
} Але, щоб експлуатувати ці закони для телеграфії
і~\abbr{т. д.}, треба дуже дорогого та складного апарату. Як ми бачили,
машина не витискує знаряддя. З карликового знаряддя людського
організму виростає воно розміром та кількістю на знаряддя механізму,
створеного людиною. Капітал примушує тепер робітника
працювати не ручним знаряддям, а машиною, що сама орудує
своїм знаряддям. Тим то, коли на перший же погляд ясно, що
велика промисловість, включаючи в процес продукції велетенські
природні сили та природознавство, мусить надзвичайно
підвищити продуктивність праці, то ніяк не є так само ясно,
чи не купується цю підвищену продуктивну силу збільшеною
витратою праці на другому боці.\footnote*{
У французькому виданні це місце подано так: «Тим то, коли на
перший же погляд ясно, що механічна промисловість, включаючи в свій
склад науку та велетенські природні сили, надзвичайно підвищує продуктивність
праці, то можна, однак, запитати, чи не втрачається на другому
боці те, що виграється на одному, чи економізується вживанням
машин більше праці, ніж коштує їх продукція та утримання» («Le Capital
etc.», v. I, ch. XV, p. 168). \emph{Ред.}
} Подібно до кожної іншої складової
частини сталого капіталу машина не утворює вартості,
але віддає свою власну вартість продуктові, що для його продукції
вона служить. Оскільки вона має вартість, і тому переносить
цю вартість на продукт, остільки вона й становить складову
частину вартости продукту. Замість його здешевлювати, вона
удорожчує його відповідно до своєї власної вартости. І це ясна
річ, що машина та розвинена система машин, цей характеристичний
засіб праці великої промисловости, мають куди більшу вартість,
аніж засоби праці ремісничого або мануфактурного виробництва.
Насамперед треба тут зауважити, що машина завжди цілком
увіходить у процес праці й завжди тільки частинно в процес
утворення вартости. Вона ніколи не додає більше вартости,
ніж пересічно втрачає через своє зужиткування. Отже, є велика
\parbreak{}  %% абзац продовжується на наступній сторінці
