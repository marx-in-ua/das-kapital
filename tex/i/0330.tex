вплив задушливого та гидкого повітря на бідні діти... Я побував
у багатьох таких школах, там я бачив цілі лави дітей, що абсолютно
нічого не робили; і ці діти мали посвідки, що ходили до
школи, а в офіціяльній статистиці вони фігурують як такі, що
дістали освіту (educated)».\footnote{
Leonhard Horner у «Reports etc. for 31 st October 1857», р. 17, 18.
} В Шотляндії фабриканти силкуються
по змозі не приймати на роботу дітей, що мусять ходити до школи.
«Цього досить, щоб показати велику неприхильність фабрикантів
до пунктів закону про виховання дітей»\footnote{
Sir J. Kincaid у «Reports etc. for 31 st October 1856», p. 66.
}. У неймовірно
жахливій формі виявляється це по перкалевибійних фабриках та
інших вибійнях, що підведені під осібний фабричний закон.
За постановами цього закону, «кожна дитина, раніш ніж її можна
прийняти до такої вибійні на роботу, мусить відвідувати школу
щонайменше 30 днів та не менш як 150 годин протягом 6 місяців
безпосередньо перед тим днем, коли вона вперше починає працювати.
Протягом того часу, коли дитина працює на цих вибійних
фабриках, вона так само мусить що шість місяців на рік ходити
до школи по 30 день, або 150 годин... Відвідувати школу треба
між 8 годиною ранку й 6 годиною по півдні. Відвідування школи,
що триває менше ніж 2 1/2  години або більше як 5 годин на день,
не може вважатися за частину тих 150 годин. За звичайних обставин
діти відвідують школу вранці й по півдні протягом 30 днів
по 5 годин на день, а після цих 30 днів, дійшовши встановленої
статутами повної суми в 150 годин, скінчивши, як вони сами
висловлюються, свою книжку, вони повертаються до вибійні
й лишаються там знову 6 місяців, доки знову прийде строк іти
до школи; тоді вони знову лишаються в школі доти, доки знову
скінчать свою книжку... Дуже багато підлітків, які відвідували
школу протягом приписаних 150 годин, вертаючись до неї після
шестимісячного перебування на фабриці, знають не більше, ніж
вони знали з самого початку... Певна річ, вони знову позабували
все, чого набралися раніш, відвідуючи школу. По інших перкалевибійних
фабриках відвідування школи геть чисто залежить від
потреб фабрики. Потрібне число годин протягом кожного півріччя
поповнюється зарахуванням 3—5-годинних відвідувань, що порозкидувані,
може, і по цілому півроці. Приміром, одного дня
школу відвідують від 8 до 11 години ранку, іншого дня — від
1 до 4 години по півдні, і після того, як дитина потім знову декілька
днів не відвідувала школу, вона раптом знову приходить
від 3 до 6 години по півдні; потім, може бути, вона ходить 3 або 4 дні
або й цілий тиждень підряд, а далі знову зникає на 3 тижні або
на цілий місяць та вертається на декілька годин у вільні дні,
коли підприємець випадково її не потребує; отак дитину, так би
мовити, кидають туди та сюди (buffeted), із школи до фабрики,
з фабрики до школи, поки нараховується сума в 150 годин».\footnote{
A. Redgrave у «Reports of Insp. of Fact. for 31 st October
1857», p. 41, 42. По тих галузях англійської промисловости, де від давні-
}