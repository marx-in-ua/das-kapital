тільки переривається їхня функція вбирати працю. «Але тоді
була б утрата на дуже коштовних машинах, що половину часу
стояли б без діла, і ми мусили б подвоїти розмір будівель і число
машин для того, щоб виробити таку масу продуктів, яку ми можемо
виробити за теперішньої системи, а це подвоїло б видатки». Але
чому саме ці Сандерсони вимагають привілеїв супроти інших
капіталістів, яким дозволено працювати лише вдень і в яких,
отже, будівлі, машини, сировинний матеріял вночі «лежать без
діла»? «Це правда, — відповідає Е. Ф. Сандерсон за всіх Сандерсонів,
— це правда, що ця втрата від припинення вночі машин
стається по всіх мануфактурах, де працюють лише вдень. Але
вживання топильних печей спричинило б у нашому випадку
екстраординарну втрату. Коли тримати їх у русі напоготові, тоді
марнується паливо (замість життя робітників, яке марнується тепер),
коли ж припинити їхній рух, тоді марнується час на розпалювання
й добування потрібної температури (тим часом як втрата
часу, потрібного на сон, навіть у восьмилітніх дітей є виграш
часу для всієї сандерсонівської братії), та й сами печі зазнали
б шкоди від зміни температури» (тим часом як ті самі печі
нічого не терплять від денної й нічної зміни праці).103

103 «Children’s Employment Commission. 4 th Report etc.», 1865,
p. 84. Подібним делікатним способом пани фабриканти скла твердили, що
призначення «реґулярного часу на обід» для дітей є неможливе, бо це
призвело б до «чистої втрати» й «марнотратства» певної кількости тепла,
яке дають печі; на ці міркування слідчий комісар Байт, зовсім не схожий
на Юра, Сеніора і інших та їхніх обмежених німецьких підбрехачів à la
Рошер та іншіх, що зворушені «поздержливістю», «самовідреченням» і
«ощадністю» капіталістів при витраті своїх грошей і їхнім тімур-тамерпанським
«марнотратством» людського життя, — відповідає так: «Можливо,
що в наслідок забезпечення реґулярного часу для їжі і витрачатиметься
марно певну кількість тепла проти теперішньої, але ця витрата,
навіть виражена в грошовій вартості, є нішо порівняно з марнотратством
життєвої сили («the waste of animal power»), що його зазнає тепер королівство
через те, що діти, які працюють на гутах і перебувають у періоді
зросту, не мають вільного часу, щоб спокійно попоїсти й перетравити
їжу». (Там же, p. XLV.). І це в «рік проґресу», в рік 1865! Залишаючи
осторонь витрату сили при підійманні й переношуванні тягара, така
дитина мусить, на гутах, де виробляється пляшки й кремінне скло (Flintglas),
підчас своєї безперервної праці зробити протягом 6 годин 15—16
миль (англійських)! А праця часто триває 14—15 годин! По багатьох
таких гутах, як от і на московських прядільнях, панує система шестигодинних
змін. «Протягом тижневого робочого часу найдовший безперервний
відпочинок триває 6 годин, але звідси треба відлічити час, потрібний на
те, щоб дійти до фабрики й назад, щоб умитися, одягтися, поїсти, на що
теж треба часу. Таким чином у дійсності лишається на відпочинок тільки
якнайкоротший час. Немає часу погратися, подихати свіжим повітрям,
хіба що коштом сну, що так дуже потрібний дітям, які за такої спеки
виконують таку напружену працю... Навіть цей короткий сон, і той переривається,
бо дитина мусить сама просипатися вночі або прокидатися вдень
від зовнішнього гуркоту». Пан Байт наводить випадки, коли один підліток
працював 36 годин без якої-будь перерви, коли дванадцятилітні хлоп’ята
мучаються до 2 години вночі, а потім сплять на гуті до 5 години ранку
(3 години!), щоб знову розпочати денну працю! «Кількість праці, — кажуть
редактори загального звіту Тременгір і Тефнель, — що її виконують
хлопчаки, дівчата й жінки протягом денної або нічної зміни праці («spell of
