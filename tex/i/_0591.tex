\parcont{}  %% абзац починається на попередній сторінці
\index{i}{0591}  %% посилання на сторінку оригінального видання
становило предмет глибоких турбот не лише для осіб, що піклуються
про здоров’я, але й для всіх взагалі, що цінять пристойне
й моральне життя. Бо автори звітів про поширення пошесних
недуг у сільських округах кажуть знову й знову, в одноманітних
висловах, — що, здається, поробилися вже стереотипними, — про
переповнення помешкань, як про причину, що цілком зводить
нанівець усі спроби спинити розвиток пошести, скоро вона вже
з’явилась. Так само знову й знову доводили вони, що, хоч сільське
життя і добре впливає на здоров’я, скупченість людности,
що так дуже прискорює поширення пошесних недуг, сприяє
також і поширенню незаразливих недуг. І особи, які зазначали
такий стан речей, не замовчували й дальшого лиха. Навіть у тих
випадках, коли вони у своїй первісній темі торкались тільки
гігієни, вони були майже примушені звернути увагу й на інший
бік справи. Їхні звіти, що зазначають те, як часто трапляється, що
дорослі люди обох статей, жонаті й нежонаті, позбивані до купи
(huddled) у тісних спальнях, мусили викликати переконання,
що серед описаних обставин якнайгрубішим способом порушується
почуття сорому та пристойности і майже неминуче руйнується
всяку моральність\dots{}\footnote{
«Молоде подружжя не являє собою повчального прикладу для
дорослих братів і сестер, що сплять у тій самій кімнат: і хоч не можна
зареєструвати таких прикладів, однак є досить даних, які доводять правдивість
того твердження, що великих страждань, а часто й смерти зазнають
жінки, які допустились кровозмішення» (Д-р Гентер, там же, стор. 137).
Один сільський поліцайський урядовець, що протягом багатьох років
був розшукачем у найгірших кварталах Лондону, каже про дівчат свого
села: «Такої грубої аморальности з молодого віку, такого нахабства
та безсоромности я ніколи не бачив протягом усього того часу, що я був
поліцаєм у найгірших частинах Лондону\dots{} Вони живуть як свині, дорослі
парубки й дівчата, матері й батьки, усі сплять укупі в одній кімнаті»
(«Children’s Employment Commission. 6th Report», London 1867, Appendix,
p. 77, n. 155
} Напр., у додатку до мого останнього звіту
д-р Орд у своєму звіті про вибух пропасниці у Wing’y і в Виcidnghamshir’i
згадує про те, як туди прийшов із Wingrave один
парубок, недужий на пропасницю. Перші дні своєї хороби він
спав в одній кімнаті з дев’ятьма іншими особами. За два тижні
деякі з цих осіб теж захоріли, за кілька тижнів на пропасницю
занедужало 5 із цих 9 осіб, а одна померла! Одночасно
д-р Гарвей, лікар шпиталю St.~Georges, що за часів епідемії відвідував
Wing з приводу приватної практики, звітує в тому самому
дусі: «Одна молодиця, хора на пропасницю, спала вночі
в тій самій кімнаті з батьком, матір’ю, своєю нешлюбною дитиною,
двома парубками, її братами, і двома сестрами, що з них
кожна також мала нешлюбну дитину, разом 10 осіб. Декілька
тижнів раніш у тій самій кімнаті спало 13 дітей»\footnote{
«Pubic Health. Seventh Report: 1867», p. 9--14 passim.
}.

Д-р Гентер обслідував \num{5.375} котеджів сільських робітників,
не лише в суто рільничих округах, алей по всіх графствах
Англії. З цих \num{5.375} котеджів \num{2.195} мали лише по одній спальні
(часто це разом з тим і мешкальна кімната), \num{2.930} — лише по дві
\parbreak{}  %% абзац продовжується на наступній сторінці
