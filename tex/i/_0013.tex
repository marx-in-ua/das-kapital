\parcont{}  %% абзац починається на попередній сторінці
\index{i}{0013}  %% посилання на сторінку оригінального видання
з мінової вартости або мінового відношення товарів, щоб натрапити
на слід прихованої в них вартости. Тепер ми мусимо повернутись
до цієї форми виявлення вартости\footnote*{
У цьому абзаці ми натрапили на труднощі щодо перекладу німецького
слова «\textgerman{Wertgegenständlichkeit}», яке Маркс уживає тут у розумінні:
«об’єктивне існування вартости», «реальність вартости». Ми
вважаємо за найкраще перекласти це слово точно, достеменно словом
«предметність вартости», надаючи так само слову «предметність» значення
— «об’єктивне існування», «реальність». Далі ми вважаємо за
потрібне, для кращого розуміння Марксової думки, навести тут також
увесь цей абзац у тій редакції, в якій подав його Маркс у французькому
виданні «Капіталу». Подаємо тут так переклад, як і самий французький
текст: «Реальність, що її має вартість товарів, тим відрізняється від
подруги Фальстафа, удовиці Квіклі, що невідомо, де її можна знайти.
Цілком протилежно до грубої почуттєвости товарових тіл, жоден атом
природної речовини не входить в їхню вартість. Отже, можна крутити й
вивертати на всі боки будь-який товар, узятий відокремлено; як предмет
вартости, він лишається несхопним. Однак, коли пригадати собі, що
вартості товарів мають лише суто суспільну реальність, що вони її набирають
лише остільки, оскільки вони є вирази однакової суспільної одиниці,
людської праці, тоді стане зрозуміло, що ця суспільна реальність
може виявитись також тільки в суспільних актах, у відношенні одного
товару до іншого. Справді, ми виходили з мінової вартости або мінового
відношення товарів, щоб натрапити на слід прихованої в них вартости.
Тепер ми мусимо повернутись до цієї форми, що в ній нам спочатку з’явилась
вартість». («La réalité que possède la valeur de la marchandise, diffère
en ceci de l’amie de Falstaff, la veuve l’Eveillé, qu’on ne sait où la prendre.
Par un contraste de plus criants avec la grossièreté du corps de la marchandise,
il n’est pas un atomme de matière qui pénétre, dans sa valeur. On peut
donc tourner et retourner à volonté une marchandise prise à part; en tant
qu’objet de valeur, elle reste insaisissable. Si l’on se souvient cependant
que les valeurs des marchandises n’ont qu’une réalité purement sociale,
qu’elles ne l’acquièrent qu’en tant qu’elles sont des expressions de la même
unité sociale, du travail humain, il déviènt évident que cette réalité sociale
ne peut se manifester aussi que dans les transactions sociales, dans les
rapports des marchandises les unes avec les autres. En fait, nous sommes
partis de la valeur d’échange on du rapport d’échange des marchandises
pour trouver les traces de leur valeur qui y est cachée. Il nous faut revenir
main tenant à cette forme sous la quelle la valeur nous est d’abord
apparue»). \Red
}.

Кожен знає, коли він навіть більше нічого не знає, що товари
мають спільну форму вартости, гостро відмінну від строкатих натуральних
форм їхніх споживних вартостей, а саме грошову
форму. Однак, тут треба розв’язати питання, до якого буржуазна
політична економія навіть не спробувала приступити, а саме
показати генезу цієї грошової форми, тобто простежити розвиток
виразу вартости, що міститься у вартостевому відношенні товарів,
від його найпростішого, найменш ясного вигляду до сліпучої
грошової форми. Разом з цим зникне й загадковість грошей.

Найпростішим вартостевим відношенням є, очевидно, вартостеве
відношення якогось товару до одним-одного товару
іншого роду, байдуже якого. Тому вартостевим відношенням двох
товарів дається найпростіший вираз вартости одного товару.
