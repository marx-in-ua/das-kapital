
\index{i}{0554}  %% посилання на сторінку оригінального видання
Насамкінець, найнижча верства відносного перелюднення
перебуває у сфері павперизму. Залишаючи осторонь волоцюг,
злочинців, проституток, коротко — власне люмпенпролетаріят,
ця верства суспільства складається з трьох категорій. Поперше,
працездатні. Досить лише поверхово переглянути статистику
англійського павперизму, і ми побачимо, що маса його більшає
з кожною кризою й меншає з кожним оживленням справ. Подруге,
сироти й діти павперів. Це кандидати промислової резервної
армії; в періоди великого розцвіту, як, наприклад, 1860~\abbr{р.}, вони
швидко й масами заповнюють ряди активної робітничої армії.
Потрете, занепалі, збіднілі, непрацездатні. Це саме ті індивіди,
що гинуть од своєї нерухливости, спричиненої поділом праці,
ті, що переживають нормальний вік робітника, нарешті, жертви
промисловости, що їхнє число зростає з поширенням небезпечних
машин, копалень, хемічних фабрик і~\abbr{т. д.} — каліки, хорі,
вдови тощо. Павперизм є дім інвалідів активної робітничої армії
і баляст промислової резервної армії. Утворення відносного
перелюднення включає й утворення павперизму, доконечність
першого включає й доконечність другого, разом із відносним
перелюдненням павперизм становить умову існування капіталістичної
продукції й розвитку багатства. Він належить до faux
frais\footnote*{
непродуктивних витрат. \Red{Ред.}
} капіталістичної продукції, що їх капітал уміє однак здебільша
звалити з себе самого на плечі робітничої кляси і дрібної
середньої кляси.

Що більше суспільне багатство, капітал, який функціонує,
розміри й енерґія його зростання, отже і абсолютна величина пролетаріяту
і продуктивна сила його праці, то більша резервна промислова
армія. Робоча сила, що нею можна порядкувати, розвивається
через ті самі причини, що й експансивна сила капіталу.
Отже, відносна величина резервної промислової армії зростає
разом з потенціями багатства. Але що більша ця резервна армія
проти активної робітничої армії, то масовіше є стале перелюднення,
що його злидні стоять у зворотній пропорції до мук його
праці. Нарешті, що більша жебрацька верства робітничої кляси
й промислова резервна армія, то більший офіціяльний павперизм.
\emph{Це є абсолютний, загальний закон капіталістичної акумуляції.}
У своєму здійсненні він, як і всі інші закони, модифікується
різноманітними обставинами, що аналіза їх сюди не
належить.

Можна зрозуміти безглуздість тієї економічної премудрости,
яка проповідує робітникам пристосовувати свою чисельність до
потреб самозростання капіталу. Механізм капіталістичної продукції
й акумуляції постійно пристосовує цю чисельність до цих
потреб самозростання. Перше слово цього пристосування є утворення
відносного перелюднення або промислової резервної армії,
а останнє слово — це злидні щораз більших верств активної
робітничої армії й баляст павперизму.
