парлямент ухвалив наш проєкт у цілому, то немає ніякого сумніву,
що таке законодавство справило б якнайсприятливіший
вплив не тільки на підлітків і малосильних, що до них воно
насамперед стосується, але й на ще більші маси дорослих робітників,
які безпосередньо (жінки) і посередньо (чоловіки) опиняться
у сфері його впливу. Воно примусило б їх до вреґульованого
й помірного робочого часу, воно зберігало б і нагромаджувало
той запас фізичної сили, від якого так дуже залежить їхній
власний добробут і добробут країни; воно захистило б молоде
покоління від надмірного напруження в ранньому віці, яке
нищить їхній організм і призводить до передчасного занепаду;
нарешті, воно дало б дітям змогу, принаймні до 13 року життя,
дістати елементарну освіту й тим поклало б край тому неймовірному
неуцтву, що так правильно змальоване у звітах комісії
і що на нього можна дивитися лише з великим болем і глибоким
почуттям національної зневаги».\footnote{
Там же, стор. XXV, n. 169.
} Міністерство торі у троновій
промові з 5 лютого 1867 р. сповістило, що воно зформулювало в
«білах» проєкти, подані від комісії для розсліду промисловости.\footnoteA{
Factory Acts Extension Act\footnote*{
Закон про поширення фабричного закону. Ред.
} ухвалено 12 серпня 1867 р. Він
реґулює всі металеливарні, ковальські підприємства й мануфактури,
включаючи і машинобудівельні заводи, далі мануфактури скла, паперу,
ґутаперчі, кавчуку, тютюну, друкарні, палітурні, нарешті, всі майстерні,
в яких працює більш як 50 осіб. — Hours of Labour Regulation
Act,\footnote*{
Закон, що реґулює час праці. Ред.
} що його ухвалено 17 серпня 1867 р., реґулює дрібні майстерні
й так звану домашню працю.

У II томі я повернуся до цих законів, до нового Mining Act\footnote*{
— закону про копальні. Ред.
}
1872 р. й т. ін.
}
Для цього йому треба було нового двадцятирічного
експерименту in corpore vili. Вже 1840 р. призначено парляментську
комісію для дослідження умов дитячої праці. Її звіт з 1842 р.,
за словами Н. В. Сеніора, розгорнув «найжахливішу картину
ненажерливости, егоїзму й жорстокости капіталістів і батьків,
злиднів, деґрадації й калічення дітей і підлітків, що її навряд
чи бачив коли світ... Може хто гадає, що звіт малює страхіття
якогось минулого століття. Але, на жаль, перед нами звіт про
те, що ці страхіття існують і далі в такій самій інтенсивній формі,
як і колись. Одна брошура, опублікована перед двома роками
Гардвіком, заявляє, що зловживання, осуджені 1842 р.,
існують за наших часів (1863 р.) у повному розцвіті... Цей звіт
(1842 р.) пролежав, не звертаючи на себе уваги, двадцять років,
і протягом їх тим дітям, які повиростали, не маючи жодного
уявлення про те, що ми звемо мораллю, що таке шкільна освіта,
релігія або природна родинна любов, дозволили стати батьками
теперішнього покоління».\footnote{
Senior: «Social Science Congress», p. 55, 56.
}

Тимчасом суспільне становище змінилося. Парлямент не
насмілився відкинути вимоги комісії 1862 р., як він це свого