\parcont{}  %% абзац починається на попередній сторінці
\index{i}{0489}  %% посилання на сторінку оригінального видання
в бавовняних округах спеціяльні комісари, керуючись спеціяльними
законодавчими нормами і застосовуючи примусової праці,
щоб підтримати моральну цінність одержувачів милостині\dots{}
Чи може бути щось гірше для земельних власників або для хазяїнів
(«can anything be worse for landowners or masters»), ніж
позбутися своїх найкращих робітників і здеморалізувати та
збентежити решту через широку спустошливу еміґрацію і спустошення
цілої провінції щодо вартости й капіталу?»

Потер, цей вибраний адвокат бавовняних фабрикантів, розрізняє
дві групи «машин», при чому і ті і другі належать капіталістові,
тільки одні стоять у його фабриці, а другі вночі й
неділями перебувають поза фабрикою в котеджах. Одні мертві,
другі живі. Мертві машини не тільки щодня погіршуються та
зневартнюються, але через невпинний технічний проґрес значна
частина з наявної маси їх постійно так дуже старіється, що їх
з вигодою і протягом небагатьох місяців можна замінити на нові.
Живі машини, навпаки, поліпшуються, що довше вони функціонують,
що більше вони від покоління до покоління нагромаджують
вправности. «Times», між іншим, так відповів цьому
фабричному маґнатові:

«Пан Е.~Потер так пройнявся почуттям надзвичайної й абсолютної
ваги бавовняних хазяїнів, що для збереження цієї кляси
й увіковічнення її промислу хотів би замкнути півмільйона робітничої
кляси проти її волі у великий моральний робітний дім.
Чи варта ця промисловість того, щоб її підтримувати? — питає
Потер. Певна річ, всіма чесними засобами, — відповідаємо ми.
Чи варто тримати машини в порядку? — знову питає Потер.
Тут ми збентежені. Під машинами Потер розуміє людські машини,
бо він запевняє, що не має на думці розглядати їх як абсолютну
власність. Ми мусимо признатися, що ми не вважаємо «за варте»,
а то навіть і за можливе тримати ці людські машини в порядку,
тобто замикати їх і мастити, поки їх не потребуватимуть. Людська
машина має властивість іржавіти від бездіяльности, хоч
як дуже ви її маститимете та чиститимете. А до того людська
машина, як ми це бачимо, може сама з себе видавати пару та вибухати
або шаліти по наших великих містах у танку св. Вітта.
Можливо, як це каже Потер, і потрібен довший час на репродукцію
робітників, але, маючи машиністів і гроші, ми завжди
знайдемо заповзятих, загартованих промислових людей, щоб
зробити з них більше фабричних хазяїнів, аніж ми зможемо їх
використати\dots{} Пан Потер базікає про нове пожвавлення промисловости
через 1, 2, 3 роки й вимагає від нас, щоб ми не заохочували
або не дозволяли еміґрації робочої сили! На його думку,
це природна річ, що робітники хочуть еміґрувати, але він вважає,
що нація мусить цих півмільйона робітників разом із \num{700.000} тих,
що з ними зв’язані, замкнути проти їхнього бажання в бавовняних
округах і — неминучий наслідок — придушити силою їхнє
незадоволення та підтримувати їх самих милостинею, і все це,
сподіваючись, що якоїсь днини, може, їх знову треба буде бавовняним
\index{i}{0490}  %% посилання на сторінку оригінального видання
хазяїнам\dots{} Настав час, коли велика громадська думка
цих островів мусить щось зробити, щоб урятувати «цю робочу
силу» від тих, що хочуть поводитися з нею так, як вони поводяться
з вугіллям, залізом і бавовною» («to save this, «working
power» from those who would deal with it as they deal with iron,
coal and cotton»)\footnote{
«Times» з 24 березня 1863~\abbr{р.}
}.

Стаття «Times’a» була тільки jeu d’esprit\footnote*{
гра словами. \emph{Ред.}
}. «Велика громадська
думка» була в дійсності така, як думка Потера, — що фабричні
робітники є рухома приналежність фабрик. Їхній еміґрації
стали на перешкоді\footnote{
Парламент не вотував жодного фартинга на еміґрацію, а ухвалив
тільки закони, що давали муніципалітетам можливість тримати робітників
між життям і смертю або експлуатувати їх, не платячи їм нормальної
заробітної плати. Навпаки, коли три роки пізніше спалахнула пошесть
на худобу, парлямент грубо знехтував навіть парламентською етикетою й
негайно вотував мільйони на відшкодування мільйонерам з лендлордів,
фармери яких і без того не мали ніякої шкоди завдяки піднесенню ціни
на м’ясо. Звіряче виття землевласників на відкритті парламенту 1866~\abbr{р.}
показало, що не треба бути індусом, щоб падати навколішки перед коровою
Сабала, ані Юпітером, щоб перетворитися на бика.
}. Їх замкнули в «моральний робітний
дім» бавовняних округ, і вони, як і раніш, становлять «силу
(the strength) бавовняних хазяїнів Ланкашіру».

Отже, капіталістичний процес продукції самим своїм перебігом
репродукує відокремлення робочої сили від умов праці.
Тим самим він репродукує й увіковічнює умови експлуатації
робітника. Він постійно примушує робітника продавати свою
робочу силу, щоб жити, і постійно дає капіталістові змогу купувати
її, щоб багатіти\footnote{
«Робітник вимагав засобів існування, щоб жити, підприємець
вимагав праці, щоб мати бариш» («L’ouvrier demandait de la subsistance
pour vivre, le chef demandait du travail pour gagner»). («\emph{Sismondi}: «Nouveaux
Principes d’Economie Politique», vol. I, p. 91).
}. Тепер уже не випадок протиставить на
товаровому ринку капіталіста й робітника як покупця і продавця.
Механізм самого процесу постійно відкидає одного назад на
товаровий ринок як продавця його робочої сили і постійно перетворює
його власний продукт на купівельний засіб другого.
Фактично робітник належить капіталові раніш, ніж він продав
себе капіталістові. Його економічну підлеглість\footnote{
Грубо сільська форма цієї підлеглости існує в графстві Дергем.
Це одно з тих небагатьох графств, де обставини не забезпечують фармерові
безперечного права власности на рільничих поденників. Гірнича
індустрія дозволяє їм вибирати. Тому тут, усупереч загальному правилу,
фармер бере в оренду тільки ті землі, на яких є котеджі для робітників.
Плата за наймання котеджів становить частину заробітної плати. Ці котеджі
звуться «hind’s houses»\footnote*{
доми слуг. \emph{Ред.}
}. Їх винаймають робітникам з певними февдальними
зобов’язаннями, з умовою, що зветься «bondage» (кріпацька
залежність) і, наприклад, зобов’язує робітника на той час, коли він працює
деінде, посилати на працю свою дочку й~\abbr{т. ін.} Сам робітник називається
bondsman, кріпак. Ці відносини показують нам з цілком нового
боку і особисте споживання робітника як споживання для капіталу, або
продуктивне споживання: «Цікаво спостерігати, що навіть екскременти
цього bondsman’a його всевладний пан, лічачи все, зараховує до своїх побічних
доходів\dots{} Фармер не дозволяє будувати в навкольності ніяких кльозетів,
крім його власних, і не терпить щодо цього ніякого порушення своїх сюзеренних
прав». («Public Health, VII th Report 1864», p.188).
} упосереднює
\index{i}{0491}  %% посилання на сторінку оригінального видання
і одночасно замасковує періодичне поновлення його самопродажу,
переміна його індивідуальних хазяїнів-наймачів і коливання
ринкових цін праці\footnote{
Пригадаймо собі, що при праці дітей і~\abbr{т. ін.} зникає навіть ця формальність
самопродажу.
}.

Отже, капіталістичний процес продукції, розглядуваний в
його загальному зв’язку, або як процес репродукції, продукує
не тільки товар, не тільки додаткову вартість, — він продукує
й репродукує саме капіталістичне відношення, капіталіста на
одному боці, найманого робітника — на другому\footnote{
«Капітал має за передумову найману працю, наймана праця має за передумову
капітал. Вони взаємно зумовлюють одне одного: вони взаємно породжують одне
одного. Хіба робітник на бавовняній фабриці
продукує лише бавовняні тканини? Ні, він продукує капітал. Він продукує
вартості, які знову служать для того, щоб командувати над його працею, щоб за
допомогою її створювати нові вартості». (\emph{К.~Marx}:
«Lohnarbeit und Kapital» у «Neue Rheinische Zeitung», №~266, 7 April 1849).
Статті, опубліковані під цим заголовком в «Neue Rheinische Zeitung», є уривки
лекцій, що їх я на цю тему читав у німецькому
робітничому товаристві у Брюсселі; друкування їх перервала Лютнева
революція\footnote*{
Статті ці з’явилися потім окремою брошурою і під тією ж назвою.
Є українське видання: Партвидав «Пролетар» 1932~\abbr{р.} \emph{Ред.}
}.
}.

\sectionextended{Перетворення додаткової вартости на~капітал}{%
\subsection{Капіталістичний процес продукції в поширеному маштабі.
Перетворення законів власности товарової продукції на~закони
капіталістичного присвоєння}}

Раніше нам треба було дослідити, як додаткова вартість виникає
з капіталу, тепер треба дослідити, як із додаткової вартости
виникає капітал. Вживання додаткової вартости як капіталу
або зворотне перетворення додаткової вартости на капітал, називається
акумуляцією капіталу\footnote{
«Акумуляція капіталу: вживання частини доходу як капіталу» («Accumulation of
Capital: the employment of a portion or revenue as capital»). (\emph{Malthus}:
«Definitions etc.», ed. Cazenove, p. 11).
«Перетворення доходу на капітал» («Conversion of revenue into capital»).
(\emph{Malthus}: «Principles of Political Economy», 2nd. ed. London 1836, p. 320).
}.

Розгляньмо цей процес насамперед з погляду поодинокого капіталіста. Припустімо,
наприклад, що прядільний фабрикант авансував капітал у \num{10.000}\pound{ фунтів стерлінґів},
з них чотири п’ятих на бавовну, машини й~\abbr{т. д.}, і одну п’ятину на заробітну
плату. Нехай він щороку продукує \num{240.000} фунтів пряжі
вартістю в \num{12.000}\pound{ фунтів стерлінґів}. При нормі додаткової вартости в 100\%
додаткова вартість міститься в додатковому продукті
\parbreak{}  %% абзац продовжується на наступній сторінці
