
\index{i}{0544}  %% посилання на сторінку оригінального видання
Цей своєрідний життьовий шлях сучасної індустрії, якого ми не
знаходимо ні за однієї з попередніх епох людства, був також
неможливий і в період дитинства капіталістичної продукції. Склад
капіталу тоді змінювався лише дуже повільно. Отже, його акумуляції
відповідало в цілому пропорційне зростання попиту на
працю. Хоч і який повільний був проґрес акумуляції капіталу
порівняно з сучасною епохою, але й він наражався на природні
межі приступної для експлуатації робітничої людности, межі, що
їх можна було усунути лише насильницькими засобами, про які
згадаємо згодом. Раптове й стрибкувате поширення маштабу
продукції є передумова його раптового скорочення; останнє
знову викликає перше, але перше неможливе без людського
матеріялу, що ним можна порядкувати, воно неможливе без
збільшення числа робітників, незалежного від абсолютного зростання
людности. Це збільшення створюється тим простим процесом,
що постійно «звільняє» частину робітників, за допомогою
метод, які зменшують число занятих робітників порівняно з
вирослою продукцією. Отже, ціла форма руху сучасної промисловости
виростає з постійного перетворювання певної частини
робітничої людности на незаняті або напівзаняті руки. Поверховість
політичної економії виявляється, між іншим, у тому, що
поширення і скорочення кредиту, простий симптом періодичних
змін у промисловому циклі, вона вважає за причини цих періодичних
змін. Як небесні тіла, скоро їх кинуто в певний рух,
знову й знову повторюють його, цілком так само й суспільна
продукція, скоро її кинуто в той рух навперемінного поширення
і скорочення, знову й знову повторює цей рух. Наслідки стають
із свого боку причинами, а навперемінні фази цілого процесу,
що постійно репродукує свої власні умови, набирають форми періодичности.
[Лише з того часу, як машинова продукція глибоко
вкоренилася і почала справляти переважний вплив на всю національну
промисловість; коли завдяки їй зовнішня торговля
почала переважати внутрішню, коли світовий ринок поступінно
захопив собі широкі простори в Новому Світі, Азії і Австралії;
коли, нарешті, промислові нації, що вступили між собою в конкуренцію,
стали досить численними, — лише з цього часу починаються
періодичні цикли, що їхні послідовні фази охоплюють
ряд років і що завжди ведуть до загальної кризи, якою закінчується
один цикл і починається другий. До цього часу період
тривання цих циклів становив 10--11 років, але немає ніяких
підстав вважати це число за стале. Навпаки, з розвинутих нами
законів капіталістичної продукції треба зробити той висновок,
що це число є змінне і що період тривання циклів буде поступінно
зменшуватися»]\footnote*{
Заведене у прямі дужки ми беремо з французького видання. \Red{Ред.}
}.

Скоро тільки періодичність промислових фаз укорінюється,
то навіть політична економія починає розуміти, що продукція
відносної надмірної людности, тобто людности, надмірної проти
\parbreak{}  %% абзац продовжується на наступній сторінці
