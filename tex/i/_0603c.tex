\index{i}{0603}  %% посилання на сторінку оригінального видання
З попередньої таблиці маємо такий результат:
\begin{table}[h]

  \newlength{\myheight}
  \hangindent=1em
  \setlength{\myheight}{10em}
  \newcolumntype{Y}{>{\centering\arraybackslash}X}
  \noindent\begin{tabularx}{\textwidth}{Y Y Y Y }
      Коні & Рогата худоба & Вівці & Свині \\
      Абсолютне зменшення & Абсолютне зменшення & Абсолютне збільшення & Абсолютне збільшення \\
      72.358 & 116.626 & 146.608 & 28.819 \footnotemark{}\\

  \end{tabularx}

\end {table}
\footnotetext{Результат був би ще несприятливіший, коли б ми пішли ще далі
      назад. Так, овець 1865 р. було 3.688.742, а року 1856 — 3.694.294; свиней
      року 1865 було 1.299.893, а року 1858 — 1.409.833.}

Звернімось тепер до рільництва, що постачає засоби існування
для худоби й людей. У дальшій таблиці обчислено збільшення
або зменшення для кожного окремого року порівняно з безпосередньо
попереднім роком. Збіжжя обіймає пшеницю, овес,
ячмінь, жито, квасолю й горох, зеленина — картоплю, турнепс,
білі й червоні буряки, капусту, моркву, пастернак, вику й т. ін.

\begin{table}[h]\footnotesize
  \begin{flushright}
    Таблиця В
  \end{flushright}
  \caption*{Збільшення або зменшення засівної площі й лук (зглядно толок) в акрах}

  \newlength{\myheight}
  \setlength{\myheight}{3.5em}

  \newcolumntype{Y}{>{\centering\arraybackslash}X}
  \noindent\begin{tabularx}{\textwidth}{Y Y Y Y Y Y Y Y Y Y}
  \toprule
  Роки & Збіжжя & \multicolumn{2}{c}{Зеленина} &
  \multicolumn{2}{p{1.5cm}}{Луки й конюшина} & \multicolumn{2}{c}{Льон} &
  \multicolumn{2}{p{2.5cm}}{Загальна кількість землі  для рільництва і скотарства }\\
    \cmidrule(l){2-2}
    \cmidrule(l){3-4}
    \cmidrule(l){5-6}
    \cmidrule(l){7-8}
    \cmidrule(l){9-10}
   &
  \rotatebox[origin=c]{90}{\parbox[c]{\myheight}{Зменшення}} &
  \rotatebox[origin=c]{90}{\parbox[c]{\myheight}{Зменшення}} &
  \rotatebox[origin=c]{90}{\parbox[c]{\myheight}{Збільшення}} &
  \rotatebox[origin=c]{90}{\parbox[c]{\myheight}{Зменшення}} &
  \rotatebox[origin=c]{90}{\parbox[c]{\myheight}{Збільшення}} &
  \rotatebox[origin=c]{90}{\parbox[c]{\myheight}{Зменшення}} &
  \rotatebox[origin=c]{90}{\parbox[c]{\myheight}{Збільшення}} &
  \rotatebox[origin=c]{90}{\parbox[c]{\myheight}{Зменшення}} &
  \rotatebox[origin=c]{90}{\parbox[c]{\myheight}{Збільшення}} \\
  \midrule
    1861 & 15.701 & 36.974 & — & 47.969 & — & — & 19.271 & 81.873 & — \\
    1862 & 72.734 & 74.785 & — & — & 6.623 & — & 2.055 & 138.841 & — \\
    1863 & 144.719 & 19.358 & — & — & 7.724 & — & 63.922 & 92.431 & — \\
    1864 & 122.437 & 2.317 & — & — & 47.486 & — & 87.761 & — & 10.493 \\
    1865 & 72.450 & — & 25.241 & — & 68.970 & 50.159 & — & 28.218 & — \\
    1861—1865 & 428.041 & 107.984 & — & — & 82.834 & — & 122.850 & 330.860 & — \\

  \end{tabularx}
\end{table}

В 1865 році в рубриці «луки» сталося збільшення на 127.470 акрів,
головно через те, що площа в рубриці «необроблена пуста
земля й торфовища» зменшилась на 101.543 акри. Коли порівняти
1865 рік з 1864, то зменшення збіжжя становитиме 246.667
квартерів, із них пшениці — 48.999 квартерів, вівса — 166.605 квартерів,
ячменю — 29.982 квартери й т. ін.; зменшення кількости
картоплі, хоч оброблювана під нею площа в році 1865 і збільшилась,
становило 446.398 тонн і т. ін. (див. таблицю С).

Від руху людности й рільничої продукції Ірляндії перейдімо
до руху в гаманці її лендлордів, великих фармерів і промислових
капіталістів. Він відбивається у зменшенні і збільшенні прибуткового
податку. Щоб зрозуміти дальшу таблицю D, треба зауважити,
що рубрика D (зиски, за винятком зисків фармерів) обіймає
і так звані «професійні» зиски, тобто доходи адвокатів,
лікарів і т. ін., а рубрики С й Е, які тут не перелічені окремо,
обіймають і доходи урядовців, офіцерів, державних синекуристів,
держців державних цінних паперів і т. д.
