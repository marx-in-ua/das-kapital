\parcont{}  %% абзац починається на попередній сторінці
\index{i}{0527}  %% посилання на сторінку оригінального видання
а не своїм власним здібностям, які ані трохи не кращі, ніж в
інших; не володіння землею і грішми, а панування над працею
(«the command of labour») відрізняє багатих від бідних\dots{} Бідним
відповідає не стан занедбаности або рабства, а догідний і
ліберальний стан залежности («а state of easy and liberal dependence»),
а людям, що мають власність, відповідає достатній вплив
і авторитет над тими, що на них працюють. Такий стан залежности,
як це знає кожен знавець людської натури, доконечний
для вигоди самих робітників»\footnote{
\emph{Eden}: «The State of the Poor, or an History of the Labouring Glasses
in England», vol. I, b. 1, ch. 1, p. 1, 2 і передмова, p. XX.
}. Сер Ф.~М.~Еден, до речі зауважити,
є єдиний учень Адама Сміта, що протягом XVIIІ століття
зробив дещо важливе\footnote{
Якщо читач нагадає нам Малтуза, що його «Essay on Population»
появився 1798~\abbr{р.}, то я нагадаю, що ця праця у своїй першій формі є не
що інше, як по-школярському поверховий і по-попівському пишномовний
пляґіят з де Фо, Сера Джемса Стюарта, Тавнсенда, Франкліна, Уоллеса
та інших і не має в собі ані однісінької самостійно продуманої тези. Велика
сенсація, яку викликав цей памфлет, пояснюється виключно партійними
інтересами. Французька революція знайшла в Брітанському королівстві
палких оборонців; «принцип залюднення», що повільно вироблявся у
XVIII віці та що його потім підчас великої соціяльної кризи під звуки
сурм і барабанний бій проголосили як непомильну протиотруту супроти
теорій Кондорсе й інших, англійська олігархія вітала з великою радістю,
вбачавши в ньому великого гасителя всіх прагнень до дальшого розвитку
людства. Малтуз, надзвичайно здивований своїм успіхом, заходився тоді
коло того, щоб стару схему заповнити поверхово скомпільованим матеріялом
і додати до нього новий, не Малтузом відкритий, а ним лише
присвоєний. — До речі зауважимо тут, що хоч Малтуз був попом англіканської
церкви, а все ж він дав чернечу обітницю на безженство.
Саме це — одна з умов членства (fellowship) в протестантському кембріджському
університеті. «Ми не дозволяємо, щоб члени колегій були
жонаті. Хто ожениться, той повинен зараз вийти з членів колегії» («Socios
collegiorum maritos esse non permittimus, sed statim postquam quis
uxorem duxerit, socius collegii desinat esse»). («Reports of Cambridge University
Commission», p. 172). Ця обставина вигідно відрізняє Малтуза
від інших протестантських попів, що відкинули католицьку заповідь
попівського безженства і в такій мірі засвоїли заповідь «плодіться й
розмножуйтеся» як свою специфічну біблійну місію, що вони повсюди
в справді непристойних розмірах допомагають збільшувати людність,
тимчасом як робітникам вони проповідують «принцип залюднення».
Характеристично, що економічна пародія гріхопадіння, адамове яблуко,
«непереможне бажання», «перепони, що силкуються притупити стріли
Купідона» («urgent appetite», «the checks which tend to blunt the shafts
of Cupid»), як весело каже піп Тавнсенд, що цей дражливий пункт монополізували
й тепер монополізують пани представники протестантської
теології або, точніше, церкви. За винятком венеціянського ченця Ортеса,
ориґінального й талановитого письменника, більшість проповідників
принципу залюднення — це протестантські попи. Такий, наприклад,
Брукер, що в йоги творі «Théorie du Système animal», Ley de 1767 вичерпано
всю сучасну теорію залюднення, до якої подала ідеї короткочасна
суперечка між Кене і його учнем Мірабо-батьком на цю саму тему,
потім ідуть піп Уоллес, піп Тавнсенд, піп Малтуз і його учень, архіпіп
Т.~Чалмерс, не кажучи вже про дрібніших попів-писак того самого напряму.
Первісно над політичною економією працювали філософи, такі,
як Гоббс, Лок, Юм, комерційні й державні люди, як от Томас Мор, Тімпл,
Сюллі, де~Вітт, Норт, Ло, Вандерлінт, Кантільйон, Франклін, а особливо
над теорією її з великим успіхом працювали медики, як от Петті, Барбон,
Мандевіль, Кене. Ще в середині XVIII віку піп Текер, видатний економіст
свого часу, прохає вибачення за те, що він займався мамоною. Пізніше,
а саме одночасно з «принципом залюднення», настав час протестантських
попів. Неначе передчуваючи це партацтво, Петті, що вважає людність
за базу багатства і в, так само як і Адам Сміт, непримиренний
ворог попів, каже: «Релігія найкраще процвітає там, де священники найбільше
підпадають усмирению плоті, так само як право найкраще процвітає
там, де адвокати вмирають з голоду». Тому він радить протестантським
попам, якщо вони не хочуть іти за прикладом апостола Павла
і «умерщвляти свою плоть» безженством, «принаймні не плодити більше
попів («not to breed more Churchmen»), аніж їх могли б поглинути наявні
парафії (benefices); тобто коли в Англії й Велзі існує лише \num{12.000} парафій,
то нерозумно наплоджувати \num{24.000} попів («it will not be safe to breed
\num{24.000} ministers»), бо \num{12.000} незабезпечених завжди намагатимуться
здобути собі засоби існування, а як можуть вони найлегше досягти цього,
як не ходячи серед народу та переконуючи його в тому, що ті \num{12.000} попів,
що мають парафії, отруюють душі, заморюють їх голодом та вказують
їм неправдивий шлях до неба?» (\emph{Petty}: «A Treatise on Taxes and
Contributions», London 1667, p. 57). Ставлення Адама Сміта до протестантського
попівства його часів характеризується ось чим. В «А Letter to A.~Smith,
L.~L.~D.~On the Life, Death and Philosophy of his Friend David Hume.
By One of the People called Christians», 4th ed. Oxford 1784 англіканський
єпископ д-р Херн із Норвіча докоряє А.~Смітові за те, що він в одному
відкритому листі до пана Стрехена «бальзамує свого приятеля Давіда»
(тобто Юма), що він оповідає публіці, як «Юм на своєму смертному ліжку
розважував себе Лукіяном і Вайстом», і що він навіть мав нахабство написати:
«Я завжди вважав Юма так за його життя, як і після його смерти
таким близьким до ідеалу цілком мудрої й доброчесної людини, як це тільки
дозволяють слабощі людської натури». Єпископ з обуренням вигукує:
«Чи то воно гаразд з вашого боку, мій пане, змальовувати нам як цілком
мудрий і доброчесний характер і побут людини, що була пройнята невигойною
антипатією до всього того, що зветься релігією, і напружувала
кожний свій нерв, щоб, оскільки це від неї залежало, стерти з людської
пам’яті навіть назву релігія?» (Там же, стор. 8). «Але не журіться ви,
приятелі правди, атеїзмові недовго жити» (стор. 17). Адам Сміт — «є
гидкий нечестивець («the atrocious wickedness»), він пропагує у країні
атеїзм (саме своєю «Theory of moral sentiments»)\dots{} Ми знаємо ваші хитрощі,
пане докторе! Ви добрий задум мали, але цим разом ви рахували без господаря.
На прикладі високошановного Давіда Юма ви хочете напоумити
нас, що атеїзм — це єдиний живлющий лік («cordial») для занепалого
духу і єдина протиотрута супроти страху перед смертю\dots{} Глузуйте ж
собі з руїн Вавилону та вітайте озвірілого лиходія Фараона»! (Там же,
стор. 21, 22). Один з ортодоксальних слухачів колегії, де навчав А.~Сміт,
пише після його смерти: «Приятелювання Сміта з Юмом\dots{} перешкоджало
йому бути християнином\dots{} Він вірив Юмові в усьому на слово.
Коли б Юм сказав йому, що місяць — зелений сир, він був би йому повірив.
Тим то він вірив йому, що немає бога й чудес\dots{} Своїми політичними
принципами він наближався до республіканізму». («The Bee». By James
Anderson. 18 volumes. Edinburgh 1791--93, vol. ІІІ, p. 166, 165). Піп
T.~Чалмерс запідозрює А.~Сміта в тому, що він вигадав категорію «непродуктивних
робітників» просто із злости спеціяльно для протестантських
попів, не зважаючи на їхню благословенну працю в божому вертограді.
}.

\index{i}{0528}  %% посилання на сторінку оригінального видання
За тих найсприятливіших для робітників умов акумуляції,
які ми досі припускали, відношення залежности робітників од
капіталу прибирається у зносні, або, як каже Еден, у «приємні
й ліберальні» форми. Замість ставати із зростом капіталу інтенсивнішим,
воно стає тільки екстенсивнішим, тобто сфера експлуатації
й, панування капіталу лише поширюється разом із збільшенням
\index{i}{0529}  %% посилання на сторінку оригінального видання
його власного розміру й числа його підданців. Більша
частина їхнього власного додаткового продукту, який чимраз
більше зростає і перетворюється в дедалі більших розмірах на
додатковий капітал, припливає до них назад у формі засобів
платежу, так що вони можуть поширювати межі свого споживання,
краще влаштовувати свій споживний фонд одягу, меблів
і~\abbr{т. д.} і скласти невеликий грошовий резервний фонд. Але як
кращий одяг, ліпший харч, ліпше поводження і більший пекуліюм
не нищать відношення залежности й експлуатації раба,
так само це не нищить відношення залежности й експлуатації
найманого робітника. Підвищення ціни праці в наслідок акумуляції
капіталу свідчить справді лише про те, що розміри й вага
золотого ланцюга, що його сам найманий робітник уже викував
для себе, дозволяють ослабити напругу цього ланцюга. В суперечках
навколо цього предмету здебільшого не добачали головного,
а саме differentia specifica\footnote*{
відмінної ознаки. \emph{Ред.}
} капіталістичної продукції.
Робочу силу тут купують не для того, щоб через її послуги або
її продукт задовольняти особисті потреби її покупця. Мета покупця
— збільшити вартість свого капіталу, продукувати товари,
що містять у собі більше праці, аніж він оплачує, отже,
що містять у собі таку частину вартости, яка нічого не коштує
йому і яку він проте реалізує через продаж товарів. Продукція
додаткової вартости, або нажива — це абсолютний закон капіталістичного
способу продукції. Робоча сила може знаходити
собі покупців лише остільки, оскільки вона зберігає засоби продукції
як капітал, репродукує свою власну вартість як капітал
і в неоплаченій праці дає джерело додаткового капіталу\footnote{
Примітка до 2 видання. «Однак межа зайняття робітників так
промислових, як і сільських однакова: а саме можливість для підприємця
добувати зиск із продукту їхньої праці\dots{} Якщо норма заробітної
плати зростає так високо, що зиск хазяїна падає нижче пересічного
зиску, то хазяїн перестає вживати робітників або вживає їх лише за
тієї умови, щоб вони згодились на зниження заробітної плати». (\emph{John
Wade}: «History of the Middle and Working Classes», 3rd. ed., London
1835, p. 241).
}. Отже,
умови продажу робочої сили, незалежно від того, чи вони більш
чи менш сприятливі для робітників, містять у собі доконечність
постійного повторювання її продажу і репродукцію багатства
як капіталу в щораз ширшому розмірі. Заробітна плата, як ми
бачили, з самої природи своєї постійно зумовлює постачання
робітником певної кількости неоплаченої праці. Залишаючи
цілком осторонь випадки зростання заробітної плати при зниженні
ціни праці тощо, збільшення її означає в найкращому
випадку лише кількісне зменшення неоплаченої праці, що її
мусить давати робітник. Це зменшення ніколи не може дійти до
такого пункту, де воно загрожувало б самій капіталістичній
системі. Залишаючи осторонь насильні конфлікти щодо рівня
заробітної плати, — а вже Адам Сміт показав, що взагалі і в цілому
\index{i}{0530}  %% посилання на сторінку оригінального видання
в таких конфліктах хазяїн завжди лишається хазяїном, —
підвищення ціни праці, що випливає з акумуляції капіталу,
припускає таку альтернативу:

Або ціна праці й далі зростає, бо зріст її не заважає проґресові
акумуляції; в цьому немає нічого дивного, бо каже А.~Сміт,
«навіть за зниженого зиску капітали все ж зростають; вони навіть
зростають швидше, ніж раніш\dots{} Великий капітал навіть
за меншого зиску взагалі зростає швидше, ніж малий капітал за
великого зиску» («Wealth of Nation», р. 189). У цьому випадку
очевидно, що зменшення неоплаченої праці аж ніяк не заважає
капіталові поширювати своє панування. — Або, — і це є другий
бік альтернативи, — акумуляція в наслідок підвищення ціни
праці слабшає, бо притупляється спонукливий стимул баришу.
Акумуляція меншає. Але з її зменшенням зникає причина її
зменшення, а саме зникає диспропорція між капіталом і робочою
силою, приступною для експлуатації. Отже, механізм капіталістичного
процесу продукції сам усуває ті тимчасові перешкоди,
які він утворює. Ціна праці знову спадає до рівня, що
відповідає потребам зростання капіталу, все одно, чи цей рівень
нижчий, вищий або рівний тому, що його вважалося за нормальний
перед початком зростання заробітної плати. Ми бачимо:
в першому випадку не зменшення абсолютного або відносного
зростання робочої сили або робітничої людности робить капітал
надмірним, а, навпаки, збільшення капіталу робить недостатньою
приступну для експлуатації робочу силу. У другому випадку
не збільшення абсолютного або відносного зростання робочої
сили або робітничої людности робить капітал недостатнім, а,
навпаки, зменшення капіталу робить надмірною приступну для
експлуатації робочу силу, або, точніше, її ціну. Оці абсолютні
рухи акумуляції капіталу відбиваються як відносні рухи в масі
приступної для експлуатації робочої сили, і тому здається, нібито
їх спричиняє власний рух останньої. Вживаючи математичного
вислову: величина акумуляції є незалежна змінна, величина
заробітної плати — залежна, а не навпаки. Так, у промисловому
циклі підчас фази кризи загальний спад товарових цін виражається
як підвищення відносної вартости грошей, а підчас фази
розцвіту — загальне підвищення товарових цін виражається як
спад відносної вартости грошей. Так звана Currency-школа
робить із цього висновок, що за високих цін циркулює замало
грошей, а за низьких — забагато. Її неуцтво й повне нерозуміння
фактів\footnote{
Порівн. \emph{К.~Marx}: «Zur Kritik der Politischen Oekonomie»,
S. 166 і далі. (\emph{К.~Маркс}: «До критики політичної економії», ДВУ 1926,
стор. 171 і далі).
} находять собі гідну паралелю в тих економістів,
які ті явища акумуляції пояснюють тим, що в одному випадку
існує замало, а в другому забагато найманих робітників.

Закон капіталістичної продукції, що лежить в основі нібито
«природного закону залюднення», сходить просто ось на що:
відношення між капіталом, акумуляцією й нормою заробітної
\parbreak{}  %% абзац продовжується на наступній сторінці
