\parcont{}  %% абзац починається на попередній сторінці
\index{ii}{0158}  %% посилання на сторінку оригінального видання
протиставиться другій складовій частині сталого капіталу, витраченій на
засоби праці. Додаткова вартість, отже, саме та обставина, що перетворює
витрачену суму вартости на капітал, лишається при цьому цілком
поза розглядом. Так само поза розглядом лишається й те, що частину
вартости, яку долучає до продукту витрачений на заробітну плату капітал,
випродукувано знову (тобто справді репродуковано), тимчасом як
частину вартости, що її долучає до продукту сировинний матеріял, не
випродукувано знову, не репродуковано в дійсності, а лише збережено
в вартості продукту, консервовано, і тому вона лише знову з’являється як
складова частина вартости продукту. Ріжниця, як вона виявляється тепер
з погляду протилежности між поточним і основним капіталом, сходить
лише ось на що: вартість засобів праці, вжитих для продукції товару,
лише частинами входить у вартість товару, а тому й лише частинами
покривається через продаж товарів, а значить, і взагалі покривається вона
тільки частинами й поступінно. З другого боку, вартість робочої
сили та предметів праці (сировинні матеріяли тощо), вжитих для
продукції товару, цілком увіходить у товар і тому цілком покривається
через продаж його. В цьому розумінні, отже, щодо процесу
циркуляції одна частина капіталу виступає як основний, а друга як поточний
або обіговий капітал. В обох випадках ідеться про перенесення
даної, авансованої вартости на продукт і про покриття її через продаж
продукту. Ріжниця тут лише в тому, як відбувається це перенесення вартости,
а, значить, і покриття вартости: чи частинами й поступінно, чи
одразу одним заходом. Цим самим затушковується найвирішальнішу ріжницю
між змінним і сталим капіталом, отже, затушковується всю таємницю
утворення додаткової вартости і всю таємницю капіталістичної продукції,
затушковується обставини, що перетворюють на капітал певні вартості
й речі, що в них ці вартості втілюються. Всі складові частини капіталу
відрізняються тут тільки способом циркуляції (а циркуляція товару,
звичайно, має чинення тільки до наявних уже, даних вартостей); але особливий
спосіб циркуляції капіталу, витраченого на заробітну плату, спільний і
частині капіталу, витраченій на сировинні матеріяли, напівфабрикати,
допоміжні матеріяли, протилежно до частини капіталу, витраченої на засоби
праці.

Відси зрозуміло, чому буржуазна політична економія інстинктивно
зберігала Смітову плутанину категорій „сталого й змінного капіталу“
з категоріями „основного й обігового капіталу“ і без будь-якої критики
протягом цілого століття передавала цю плутанину з покоління в покоління.
На її погляд, витрачена на заробітну плату частина капіталу зовсім
уже не відрізняється від частини капіталу, витраченої на сировинний
матеріял, і відрізняється лише формально — лише тим, чи циркулює вона
разом з продуктом частинами, чи цілком — від сталого капіталу. Цим
самим одним ударом руйнується основи, потрібні для того, щоб зрозуміти
справжній рух капіталістичної продукції, а, значить, і капіталістичної
експлуатації. Для неї справа сходить лише на відновлення авансованих
вартостей.
