\parcont{}  %% абзац починається на попередній сторінці
\index{ii}{0373}  %% посилання на сторінку оригінального видання
стану, зглядно від відносної величини продукційних запасів у різних
підприємствах та різних поодиноких капіталістів тієї самої галузі підприємства,
отже, ріжниці в строках закупу елементів сталого капіталу —
все це на протязі року репродукції: досить лише на досвіді помітити
всі ці різноманітні моменти стихійного руху та звернути на них увагу, —
щоб з’явився імпульс до планомірного використовування їх так для механічних
допоміжних засобів кредитової системи, як і для справжнього
виловлювання наявних капіталів, що їх можна дати в позику.

До цього долучається ще ріжниця між такими підприємствами, що
їхня продукція в нормальних, загалом беручи, умовах відбувається
безперервно в тих самих розмірах, і такими, що в різні періоди року
застосовують робочу силу в неоднаковому розмірі, як, напр., сільське
господарство.
\label{original-373-1}

\subsection[Теорія репродукції Детю де-Трасі]{Теорія репродукції Детю де-Трасі\footnotemark{}}

\label{original-373-2}
За%
\footnotetext{З рукопису II.}
приклад плутаної й разом з тим бундючної безтямности політикоекономів
при розгляді суспільної репродукції є великий логік Детю
де-Трасі (пор. кн. І, розд. IV, 2, прим. 30), до якого навіть Рікардо
ставиться серйозно, називаючи його a very distinguished writer\footnote*{
Дуже видатним письменником. \emph{Ред.}
}. (Principles,
p. 333).

Цей „видатний письменник“ дає такі пояснення щодо сукупного суспільного
процесу репродукції та циркуляції.

„Мене запитають, як одержують ці промислові підприємці такі великі
зиски, і від кого вони можуть їх брати. Я відповідаю, що вони досягають
цього тому, що продають все продуковане ними дорожче, ніж коштує
їм продукція; і тому що вони це продають:

1) один одному в розмірі всієї частини свого споживання, призначеної
на задоволення їхніх потреб, яку вони оплачують частиною їхнього
зиску;

2) найманим робітникам, так тим, що їх оплачують вони самі, як і тим, що їх
оплачують капіталісти-нероби; таким способом вони одержують назад від цих
робітників всю їхню заробітну плату, за винятком хіба невеликих заощаджень;

3) капіталістам-неробам, які платять їм частиною свого доходу, ще
не витраченою на наймання робітників, що роблять безпосередньо для
них, так що уся рента, щорічно виплачувана промисловими підприємцями
капіталістам-неробам, тим або іншим способом знову припливає
назад до промисловців“. (Destutt de Tracy. Fraité de la volonté et de ses
effets. Paris. 1826“, p. 239).

Отже, капіталісти збагачуються, поперше, обдурюючи один одного при
обміні тієї частини додаткової вартости, яку вони призначають для особистого
споживання або споживають як дохід. Отже, коли ця частина
\parbreak{}  %% абзац продовжується на наступній сторінці
