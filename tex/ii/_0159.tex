
\index{ii}{0159}  %% посилання на сторінку оригінального видання
Ця некритично запозичена в А.~Сміта плутанина заважає Рікардо
не тільки більше, ніж пізнішим апологетам — останнім плутанина понять
не тільки не заважає, а скорше допомагає — а й більше, ніж самому
А.~Смітові, бо Рікардо, дотримуючись на ділі езотеричного вчення
А.~Сміта проти екзотеричного А.~Сміта, протилежно до нього, послідовніше
і гостріше розвинув вчення про вартість і додаткову вартість.

У фізіократів немає й сліду цієї плутанини. Ріжниця між avances
annuelles і avances primitives стосується лише до різних періодів репродукції
різних складових частин капіталу, спеціяльно хліборобського капіталу,
тимчасом як їхні погляди на продукцію додаткової вартости становлять
незалежну від цих ріжниць частину їхньої теорії, а саме частину,
що її вони виставляють як основу теорії. Утворення додаткової вартости
пояснюється в них не з капіталу, як такого, а визнається як властивість
лише певної продукційної сфери капіталу — хліборобства.

2) Найпосутніше для визначення змінного капіталу — а тому й для
перетворення будь-якої суми вартости на капітал — в тому, що капіталіст
обмінює певну, дану (і в цьому розумінні сталу) величину вартости на
силу, яка творить вартість; певну кількість вартости обмінюється на продукцію
вартости, на процес її самозростання. Чи платить капіталіст робітникові
грішми, чи засобами існування, — це нічого не змінює в цьому
найпосутнішому визначенні. Від цього змінюється тільки спосіб існування
авансованої капіталістом вартости, яка в одному разі існує у формі грошей,
що на них робітник сам собі купує на ринку засоби свого існування,
а в другому разі — у формі засобів існування, що їх робітник
споживає безпосередньо. На ділі розвинена капіталістична продукція
припускає, що робітника оплачується грішми, як вона взагалі має собі
за передумову процес продукції, упосереднюваний процесом циркуляції,
тобто має за передумову грошове господарство. Але творення додаткової
вартости — і, значить, капіталізація авансованої суми вартости — не випливає
ні з грошової, ні з натуральної форми заробітної плати, або капіталу,
витраченого на закуп робочої сили. Воно випливає з обміну вартости
на вартостетворчу силу, — з перетворення сталої величини на змінну.

Більша або менша закріпленість засобів праці залежить від ступеня
їхньої довготривалости, тобто від фізичної властивости. Залежно від
ступеня довготривалости вони, за інших незмінних умов, зношуються
швидше або повільніше, отже, функціонують як основний капітал довший
або коротший час. Але вони функціонують як основний капітал зовсім
не в наслідок самої цієї, фізичної властивости — довготривалости. Сировинний
матеріял на металевих фабриках так само довготривалий, як і машини,
що його обробляють, і довготриваліший, ніж деякі складові частини цих
машин: шкіра, дерево тощо. А проте, металь, що служить як сировинний
матеріял, становить частину обігового капіталу, а засіб праці, що функціонує,
зроблений, може, з того самого металю, становить частину основного
капіталу. Отже, не в наслідок фізичної природи речовини, не
в наслідок більшої або меншої незнищуваности той самий металь одного
разу заводиться під рубрику основного, а другого — під рубрику обігового
\parbreak{}  %% абзац продовжується на наступній сторінці
