\parcont{}  %% абзац починається на попередній сторінці
\index{ii}{0347}  %% посилання на сторінку оригінального видання
споживання, призначених здебільша на споживання робітничій клясі.
Але те, що робітник витрачає в такій формі, є не змінний капітал, а
заробітна плата, гроші робітника, що саме через свою реалізацію в цих
засобах споживання відновлюють капіталістові змінний капітал 500 II $v$
в його грошовій формі. Змінний капітал II $v$ репродуковано в засобах
споживання, як і сталий капітал 2000 ІІ $с$; і той і цей однаково не
сходять на дохід. Що сходить в обох випадках на дохід, так це заробітна
плата.

Але та обставина, що через витрачання заробітної плати як доходу
відновлюється знову як грошовий капітал, в одному разі 1000 II $с$,
потім так само цим обкружним шляхом 1000 І $v$, а також 500 II $v$,
отже, відновлюється сталий капітал і змінний капітал (цей останній
почасти через безпосередній, а почасти через посередній зворотний приплив),
— ця обставина є важливий факт в обміні річного продукту.

\subsection{Заміщення основного капіталу}

Великі труднощі при викладі обмінів річної репродукції ось у чому.
Коли ми візьмемо найпростішу форму, що в ній можна подати цю
справу, то маємо:
\begin{center}

  (\text{І} .) 4000 c \dplus{} 1000 v \dplus{} 1000 m \dplus{}\\

  (\text{II} .) 2000 c \dplus{} 500 v \dplus{} 500 m \deq{} 9000,

\end{center}
що, кінець-кінцем, розкладається на:
\begin{center}
  4000 I $с \dplus{}$ 2000 II $c$ \dplus{} 1000 I $v$ \dplus{} 500 II $v$ \dplus{} 1000 I $m$ \dplus{} 500 II $m$ \deq{}
  6000 $c$ \dplus{} 1500 $v$ \dplus{} 1500 $m \deq{}$ 9000.
\end{center}
Частину вартосте сталого капіталу, а саме, оскільки він складається
з власне засобів праці (як особливого підрозділу засобів продукції),
перенесено з засобів праці на продукт праці (товар); ці засоби праці
й далі функціонують як елементи продуктивного капіталу, і саме в своїй
старій натуральній формі; їхнє зношування, втрата вартости, що її вони
помалу зазнають, функціонуючи протягом певного часу, — ось що знову
з’являється як елемент вартости товарів, спродукованих за допомогою цих
засобів праці, ось що переноситься з знарядь праці на продукт праці. Отже,
оскільки йдеться про річну репродукцію, то тут, звичайно, треба взяти
на увагу лише такі складові частини основного капіталу, які існують
більше, ніж рік. Коли вони в межах року відживають свій вік, то їх
треба цілком замістити й відновити річною репродукцією, і тому, звичайно,
поставлене питання до них зовсім не стосується. Може статись —
і часто буває так — що деякі поодинокі органи машин та інших порівняно
триваліших форм основного капіталу, не зважаючи на довговічність
усього організму будівлі або машини, потребують повного заміщення
протягом року. Ці поодинокі органи належать також до тієї самої категорії
елементів основного капіталу, що їх треба замістити протягом року.

Цей елемент вартости товарів ні в якому разі не треба сплутувати з
витратами на ремонт. Коли товар продасться, то цей елемент вартости
\parbreak{}  %% абзац продовжується на наступній сторінці
