
\index{ii}{0343}  %% посилання на сторінку оригінального видання
Збірний капітал II авансує $500v$ на закуп робочої сили на таку саму
суму вартости; в цьому разі збірний капіталіст — покупець, збірний
робітник — продавець. Потім з грішми, вторгованими за свою робочу силу,
виступає робітник як покупець частини товарів, що їх він сам випродукував.
Отже, капіталіст тут — продавець. Робітник частиною спродукованого
товарового капіталу II, а саме $500v$ в товарі замістив капіталістові
гроші, які останній сплатив йому підчас закупу робочої сили; тепер
капіталіст має в товаровій формі те саме $v$, що його він мав у грошовій
формі раніше, перед перетворенням на робочу силу; з другого боку,
робітник реалізував у грошах вартість своєї робочої сили, а тепер знову
реалізує ці гроші, витрачаючи їх як дохід на покриття свого споживання,
на закуп частини спродукованих ним самим засобів споживання. Це —
обмін доходу робітника в грошах на спродуковану ним самим у товаровій
формі складову частину товару в $500v$ капіталіста. Таким чином,
ці гроші повертаються до капіталіста II як грошова форма його змінного
капіталу. Еквівалентна вартість доходу в грошовій формі заміщує тут
змінну капітальну вартість у товаровій формі.

Капіталіст збагачується не від того, що гроші, виплачені робітникові
при закупі робочої сили — він знову відтягує від робітника, продаючи йому
еквівалентну товарову масу. Справді він оплатив би робітника двічі, коли
б спочатку виплатив йому 500 при закупі його робочої сили, а потім,
крім того дав би йому даром ту товарову масу вартістю в 500, що її
випродукувати він примусив робітника. Навпаки, коли б робітник не
спродукував капіталістові нічого більше, крім еквіваленту в 500 у товарі,
еквіваленту ціни своєї робочої сили в 500, — то після цієї операції капіталіст
був би саме на тому самому пункті, що й раніш. Але робітник
репродукував продукт в 3000; він зберіг сталу частину вартости
продукту тобто вартість зужиткованих на продукт засобів продукції \deq{} 2000,
перетворивши їх на новий продукт; крім того, до цієї даної
вартости він долучив вартість в 1000 ($v \dplus{} m$). (Уявлення, ніби капіталіст
збагачується в тому розумінні, що він через зворотний приплив
500 в грошах здобуває додаткову вартість, розвиває Дестю де Трасі, про
що докладніше в відділі XIII цього розділу).

У наслідок того, що робітник II купує засоби споживання вартістю
в 500, до капіталіста II знову повертається в грошах вартість 500 ІІ~$v$,
яка була в нього покищо в товарі, знову повертається в тій формі, що
в ній він її первісно авансував. Безпосередній результат оборудки, як і
при всякому іншому продажі товарів, є перетворення даної вартости з
товарової форми на грошову. Та обставина, що за посередництвом цієї
оборудки гроші повернулись до свого вихідного пункту, теж не являє
чогось особливого. Коли б капіталіст II на 500 грішми купив у капіталіста
І товару, а потім з свого боку продав капіталістові І товару на
суму 500, то до нього так само повернулись би 500 грішми. Ці 500 грішми
служили би лише для обміну товарової маси в 1000 і, згідно з вищезгаданим
загальним законом, повернулись би до того, хто подав гроші в
циркуляцію для обміну цієї товарової маси.
