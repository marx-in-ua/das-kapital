\parcont{}  %% абзац починається на попередній сторінці
\index{ii}{0029}  %% посилання на сторінку оригінального видання
посередницький продаж) увійти в процес споживання, хоч то буде
особисте, хоч продуктивне споживання, залежно від властивостей купованого
предмету. Але це споживання не входить у кругобіг того індивідуального
капіталу, що його продукт є $Т'$; цей продукт, саме як товар,
призначений для продажу, виводиться геть з кругобігу. $Т'$ виразно
призначено для чужого споживання. Тому в тлумачів меркантильної
системи (а в основі її є формула $Г — Т\dots{} П\dots{} Т' — Г'$) ми й знаходимо
велемовні проповіді про те, що поодинокий капіталіст мусить споживати
лише як робітник і що нація капіталістів мусить віддавати споживання
своїх товарів і взагалі процес споживання іншим, дурнішим
націям, а сама, навпаки, мусить поставити собі своїм життьовим
завданням продуктивне споживання. Ці проповіді своєю формою та
змістом часто нагадують аналогічні аскетичні напоумлювання отців
церкви.

Отже, процес кругобігу капіталу є єдність циркуляції та продукції,
і містить у собі і те і те. Оскільки обидві фази, $Г — Т$, $Т' — Г'$, є акти
циркуляції, остільки циркуляція капіталу становить частину загальної
товарової циркуляції. Але оскільки вони є функціонально визначені відділи,
стадії в кругобігу капіталу, — кругобігу, що належить не тільки до
сфери циркуляції, але й до сфери продукції, остільки капітал у межах
загальної товарової циркуляції пророблює свій власний кругобіг. На першій
стадії загальна товарова циркуляція придається йому на те, щоб набрати
форму, в якій він може функціонувати як продуктивний капітал; на
другій стадії — для того, щоб скинути з себе функцію товару, що в ній
він не міг би відновити свій кругобіг; і разом з тим для того, щоб мав
він змогу відокремити свій власний кругобіг як капіталу від циркуляції
прирослої до нього додаткової вартости.

\vtyagnut
Тому кругобіг грошового капіталу є найбільш однобічна, а тому
найяскравіша й найхарактеристичніша форма проявлення кругобігу промислового
капіталу, що його мета й движний чинник: збільшення вартости,
роблення грошей і акумуляція, якнайвиразніше впадають на очі (купувати,
щоб дорожче продавати). А що перша фаза тут є $Г — Т$, то тут
виразно виступає й походження складових частин продуктивного капіталу
з товарового ринку, як і взагалі зумовленість капіталістичного процесу
продукції циркуляцією, торговлею. Кругобіг грошового капіталу — це не
лише товарова продукція; він сам постає лише в наслідок циркуляції, він
має її за свою передумову. Це зрозуміло вже з того, що форма $Г$, яка
належить до циркуляції, з’являється як перша й чиста форма авансованої
капітальної вартости, а цього немає в двох інших формах кругобігу.

Кругобіг грошового капіталу лише остільки завжди лишається загальним
виразом промислового капіталу, оскільки він завжди має в собі зростання
авансованої вартости. У кругобігу $П\dots{} П$ грошовий вираз капіталу
виступає лише як ціна елементів продукції, отже, лише як вартість,
виражена в розрахункових грошах, і саме в цій формі вона усталюється
в бухгалтерії.
