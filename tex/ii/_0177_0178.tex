\parcont{}  %% абзац починається на попередній сторінці
\index{ii}{0177}  %% посилання на сторінку оригінального видання
часом і часом продукції, то й час зуживання вкладеного основного капіталу раз-у-раз переривається на
більш-менш протяжні періоди, як, напр., у хліборобстві при вживанні робочої худоби, знарядь праці та
машин. Оскільки цей основний капітал складається з робочої худоби, він потребує завжди однакових або
майже однакових витрат на корм і~\abbr{т. ін.}, все одно, чи в роботі вона, чи не в роботі. Щодо мертвих
засобів праці, то коли їх не вживається, вони теж дещо зневартнюються. Тому продукт взагалі
дорожчає, бо передачу вартости на продукт обчислюється не на той час, коли основний капітал
функціонує, але на той час, коли він втрачає вартість. В цих галузях продукції бездіяльність
основного капіталу, хоч сполучена вона з поточними витратами, хоч ні, становить так само умову
нормального його вжитку, як, наприклад, втрата певної кількости бавовни в процесі прядіння; так само
в кожному процесі праці непродуктивна — але неминуча — витрата робочої сили, що відбувається в
нормальних технічних умовах, береться на увагу так само, як і продуктивна. Кожне поліпшення, що
зменшує непродуктивну витрату засобів праці, сировинного матеріялу та робочої сили, зменшує також і
вартість продукту.

В сільському господарстві поєднуються й порівняно довгий робочий період і велика ріжниця між робочим
часом і часом продукції. Годскін слушно зауважує про це: „Ріжниця в часі (хоч він тут і не відрізняє
робочого часу й часу продукції), потрібному на те, щоб виготовити продукти в сільському
господарстві, і тим часом, що потрібен в інших галузях праці, є головна причина великої залежности
сільських господарств. Вони не можуть подавати свої товари на ринок раніше, ніж через рік. Протягом
цілого цього часу вони мусять боргуватись у шевця, кравця, коваля, колісника та різних інших
продуцентів, що їхніх продуктів вони потребують, і що їхні продукти можна виготувати протягом
небагатьох днів або тижнів. В наслідок цієї природної обставини і в наслідок швидкого збільшення
багатства в інших галузях праці, землевласники, що монополізували землю цілої держави, хоч вони,
крім цього, захопили й монополію законодавства, все ж таки не можуть врятувати себе й своїх слуг
фармерів від долі найбільш залежних людей в країні“. (Thomas Hodgskin, Popular Political Economy,
London, 1827, p. 147, примітка).

Всі методи, що ними в хліборобстві почасти рівномірніше розподіляється на цілий рік витрати на
заробітну плату й засоби праці, почасти скорочується оборот у наслідок культивування різноманітних
продуктів, яке уможливлює кілька зборів урожаю на рік, — всі ці методи потребують збільшення
авансовуваного обігового капіталу, витрачуваного на заробітну плату, добриво, насіння тощо. Так
буває, коли переходять від трипільного господарства з паром до сівозмінного без пару. Так буває у
Фляндрії при cultures dérobées\footnote*{
Дослівно: „потайна культура“. Так зветься культура корінняків, що їх засівають
після збору основної культури; назва походить з того, що така культура, потребуючи менше часу,
вистигає між двома основними культурами, ніби потай. \emph{Ред.}
}“. „В culture dérobée застосовують корінняки; те саме поле спочатку
дає збіжжя, льон, рапс на задоволення потреб людини,
\index{ii}{0178}  %% посилання на сторінку оригінального видання
а по жнивах його засівають корінняками на годівлю худоби. Ця система, за якої рогата худоба
може ввесь час перебувати в стійлі, дає чималі запаси угноєння і стає таким чином за основу
сівозмінного господарства. В піскуватих місцевостях більше, ніж третину оброблюваної землі
відводиться під cultures dérobées, а наслідок такий, ніби оброблюваної землі побільшало на третину“.
Поряд корінняків тут культивують також конюшину та інші кормові рослини. „Рільництво доведене таким
чином до того пункту, де воно перетворюється на городництво, потребує, звичайно, порівняно чималого
основного капіталу (Anlagekapital). В Англії основний капітал обчислюється в 250 франків на гектар.
У Фляндрії основний капітал в 500 франків на гектар наше селянство, мабуть, визнало б за дуже
низький“. (Essais sur l’Economie Rurale de la Belgique par Emile de Laveleye. Paris, 1863, p. 59,
60, 63).

Візьмімо нарешті лісівництво. — „Продукція дерева посутньо відрізняється від більшости інших
продукцій тим, що тут сила природи діє самостійно і при природному поновленні не потребує сили
людської або сили капіталу. А проте, навіть там, де ліси розводять штучно, застосування сили
людської та капіталу порівняно з дією сил природи є лише незначне. Крім того ліс може добре рости на
таких ґрунтах і місцях, де хліб не удається або продукція його не оплачується. Але для
лісорозведення при правильному господарюванні потрібна також більша площа, ніж для культури хліба,
бо на маленьких парцелях не можна розбити ліс на правильні дільниці, побічних плодів майже не можна
використати, важче зберігати дерево й~\abbr{т. д.} Однак процес продукції тут сполучено також з такими
довгими періодами, що він виходить поза пляни приватного господарства, а іноді навіть поза межі
людського життя. Капітал, витрачений на закуп землі“ (при громадській продукції
цей капітал відпадає, і справа лише в тому,
скільки землі може громада відібрати під ліс від поля та пасовиська), „дає помітні плоди лише через
довгий час і обертається тільки почасти, а цілий оборот при деяких ґатунках дерев потребує до 150
років. Крім того, для правильної продукції дерева треба, щоб був запас живого дерева в 10--40 разів
більший, ніж щорічне споживання. Тому той, хто не має інших прибутків і посідає чимало площі лісу,
не може вести правильне лісове господарство“ (Kirchhof, р. 58).

Довгий час продукції (що має в собі відносно лише незначну частку робочого часу) і сполучений з ними
довгий період обороту робить лісівництво несприятливим для приватних, а значить, і для
капіталістичних підприємств, бо останні суттю своєю є приватні підприємства, хоча б замість
поодинокого капіталіста виступав капіталіст асоційований. Розвиток культури і взагалі промисловости
остільки енергійно виявив себе щодо знищення лісів, що порівняно з цим усе, зроблене ним для
підтримання й насадження лісу, є цілком незначна величина.

Особливо треба зауважити в цитаті Кірхгофа таке місце: „Крім того, для правильної продукції дерева
треба, щоб був запас живого дерева в 10--40 разів більший, ніж щорічне споживання“. — Отже, один
оборот дорівнює 10--40 і більше рокам.
