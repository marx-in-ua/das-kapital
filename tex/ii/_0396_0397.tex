\parcont{}  %% абзац починається на попередній сторінці
\index{ii}{0396}  %% посилання на сторінку оригінального видання
що в II теж повинна акумулюватись половина додаткової вартости,
то тут на капітал треба перетворити 188, з них на змінний капітал
1/4 \deq{} 47, беручи заокруглено, 48; лишається 140, що їх треба
перетворити на сталий капітал.

Тут ми натрапляємо на нову проблему, що саме існування її мусить
здаватись чимсь дивним з того загального погляду, за яким
товари одного ґатунку звичайно обмінюються на товари іншого ґатунку,
або товари обмінюються на гроші, а ці гроші знов обмінюються на
товари іншого ґатунку. Ці 140 II~$m$ тільки тому можуть перетворитись
на продуктивний капітал, що їх заміщується частиною товарів І~$m$ на
ту саму суму вартости. Зрозуміло само собою, що частина І~$m$, обмінювана
на II~$m$, мусить складатися з засобів продукції, які можуть
увійти так у продукцію І, як і в продукцію II, або тільки виключно
в продукцію II. Це заміщення може статися лише через
однобічну купівлю з боку II, бо ввесь додатковий продукт 500 І~$m$, що
його нам ще треба дослідити, повинен служити для акумуляції в межах І,
отже, його не можна обміняти на товари II; інакше кажучи, І не може
одночасно і акумулювати його й з’їдати. Отже, II мусить купити 140 І~$m$
за готівку, при чому ці гроші не повертаються до нього через наступний
продаж його товару підрозділові І.~І такий процес повторюється постійно,
при кожній новій річній продукції, оскільки вона є репродукція в поширеному
маштабі. Відки ж у II походить джерело грошей для цього?

Навпаки, II підрозділ, здається, є цілком неплідне поле для утворення
нового грошового капіталу, яке супроводить справжню акумуляцію і
зумовлює її при капіталістичній продукції, — утворення нового грошового
капіталу, що фактично спочатку виступає як просте утворення
скарбу.

Спочатку маємо 376 II~$v$; грошовий капітал в 376, авансований на
робочу силу, через закуп товарів II постійно повертається назад до
капіталіста II як змінний капітал у грошовій формі. Це постійно повторюване
віддалення від вихідного пункту — з кишені капіталіста — і поворот
до нього ні в якому разі не збільшує кількости грошей, що
циркулюють в цьому кругобігу. Отже, воно не є джерело акумуляції
грошей; цих грошей не можна також вилучити з цієї циркуляції для
того, щоб утворити нагромаджуваний як скарб віртуально новий грошовий
капітал.

Але почекайте! Чи не можна з цього здобути якийсь маленький
баришик?

Ми не повинні забувати, що кляса II має тут перевагу проти кляси І,
що вживані нею робітники знову повинні купувати в неї товари, спродуковані
ними самими. Кляса II є покупець робочої сили і разом з тим
продавець товарів власникам застосовуваної нею робочої сили. Отже,
кляса II може:

1) і це в неї спільне з капіталістами кляси І, — просто знижувати
заробітну плату нижче від її нормального пересічного рівня. В наслідок цього
звільняється частина грошей, які функціонують як грошова форма змінного
\index{ii}{0397}  %% посилання на сторінку оригінального видання
капіталу, і при постійному повторюванні цього процесу це могло б
стати нормальним джерелом утворення скарбів, а значить, і джерелом
утворення віртуального додаткового грошового капіталу кляси II.~Звичайно,
тут, де йдеться про нормальне утворення капіталу, ми лишаємо
осторонь випадковий зиск з шахрайства. Але не треба забувати, що
справді виплачувану нормальну заробітну плату (а вона ceteris paribus\footnote*{
В інших однакових умовах. \emph{Ред.}
}
визначає величину змінного капіталу) виплачується зовсім не з ласки
капіталістів; її виплачується тому, що при даних відношеннях вона
мусить бути виплачена. Таким чином, цей спосіб пояснення усувається.
Коли ми припускаємо, що змінний капітал, який має витратити кляса II,
становить $376 v$, то для того, щоб розв’язати новопосталу проблему, ми
не можемо одразу висунути гіпотезу, що кляса II авансує, напр., тільки
$350 v$, а не $376 v$.

2) Але, з другого боку, як ми вже сказали, кляса II, розглядувана як
ціле, має ту перевагу проти кляси І, що як покупець робочої сили вона
разом з тим є продавець своїх товарів своїм власним робітникам. Як це
можна визискувати, — яким чином можна номінально виплачувати нормальну
заробітну плату, а в дійсності частину її загарбати собі без відповідного
еквіваленту, інакше кажучи, украсти в робітників; як це можна робити
почасти за допомогою truck system, а почасти через фалшування засобів
циркуляції (хоч його не завжди дається виявити юридично), — про
це є цілком наочні дані в кожній промисловій країні, напр., в Англії та
Сполучених Штатах. (З цього приводу треба це розвинути на влучно
обраних прикладах). Це — та сама операція, що в 1), тільки замаскована
і пророблювана обкружним шляхом. Отже, її треба тут відкинути так
само, як і ту. Тут ідеться не про номінальну, а про справді виплачувану
заробітну плату.

Ми бачимо, що при об’єктивній аналізі капіталістичного механізму не
можна скористатися з деяких ганебних плям, які екстраординарно ще
гніздяться в ньому, для того, щоб викрутом усунути теоретичні труднощі.
Але більшість моїх буржуазних критиків якось чудно здіймає галас, ніби
я, прим., в І книзі „Капіталу“ своїм припущенням, що капіталіст виплачує
дійсну вартість робочої сили — а цього він здебільша не робить —
зробив велику кривду цим самим капіталістам! (Тут можна з тією самою
великодушністю, яку приписується мені, цитувати Шефле).

Отже, з 376 II~$v$ для згаданої мети нічого не вдієш.

Але ще більші труднощі, здається, постають щодо 376 II~$m$. Тут протистоять
один одному лише капіталісти тієї самої кляси, що продають
один одному й купують один в одного продуковані ними засоби споживання.
Гроші, потрібні для цього обміну, функціонують тільки як засоби
циркуляції, і при нормальному перебігу мусять повертатись назад
до учасників у тій мірі, в якій вони їх авансували для циркуляції, а потім
знову й знов переходити той самий шлях.
