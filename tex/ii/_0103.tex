\parcont{}  %% абзац починається на попередній сторінці
\index{ii}{0103}  %% посилання на сторінку оригінального видання
застережні заходи, а тому більша або менша витрата праці та засобів
праці, залежно від відносної ламкости, нетривалости, вибуховости продукту.
Тут залізничні маґнати розвивають ще більшу геніяльність в утворенні
фантастичних категорій, ніж ботаніки або зоологи. Напр., клясифікація
продуктів на англійських залізницях наповнює томи і ґрунтується
в своєму загальному принципі на тенденції перетворити всю різнобарвність
природних властивостей продуктів на так само численні хиби продуктів
щодо їх транспортування та нагоду для неодмінного здирства.
„Скло, що раніше коштувало 11\pound{ ф. стерл.} за crate (ящик для пакування
певної місткости), в наслідок успіхів промисловости й скасування податку
на скло коштує тепер лише 2\pound{ ф. стерл.}, але витрати транспорту такі самі
високі, як і раніш, й стали ще вищі, коли його почали перевозити каналами.
Раніше перевіз скла та скляних товарів, потрібних для глазурування,
на 50 миль від Бірмінгему коштував 10\shil{ шилінґів} від тонни. А
тепер ціна перевозу втроє збільшилась, в наслідок ніби ризику через
ламкість товару. Але саме залізнична дирекція й не оплачує того, що
справді ламає“\footnote{
Royal Commission on Railways, ст. 31, № 630.
}. Далі, та обставина, що відносна частина вартости, яку
додають до продукту витрати на перевіз, стоїть у зворотному відношенні
до його вартости, дає залізничним магнатам особливу підставу призначати
тариф на продукт у прямому відношенні до його вартости. Скарги промисловців
і торговців з цього приводу повторюються на кожній сторінці
свідчень вищезгаданого звіту.

Капіталістичний спосіб продукції зменшує транспортові витрати для
поодиноких товарів через розвиток засобів транспорту й комунікації, а
також через концентрацію — збільшення маштабу — транспорту. Він збільшує
ту частину суспільної праці, живої і зречевленої, що її витрачається
на перевіз товарів, поперше, перетворюючи переважну більшість всіх
продуктів на товари, і подруге, заміняючи місцеві ринки на віддалені
ринки.

Циркуляція, тобто справжнє обертання товарів у просторі, сходить на
транспорт товарів. З одного боку, транспортова промисловість становить
самостійну галузь продукції, а тому й особливу сферу приміщення продуктивного
капіталу. З другого боку, вона відрізняється тим, що являє
продовження продукційного процесу в \emph{межах} процесу циркуляції й
\emph{для} процесу циркуляції.
