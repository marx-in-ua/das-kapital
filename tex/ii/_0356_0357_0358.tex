\parcont{}  %% абзац починається на попередній сторінці
\index{ii}{0356}  %% посилання на сторінку оригінального видання
таких капіталістів підрозділу II, які мусять не тільки замістити за допомогою
своїх товарів свої засоби продукції, що належать до обігового
капіталу, а й поновити за допомогою своїх грошей свій основний
капітал in natura, тимчасом як друга половина капіталістів II своїми
грішми заміщують  in natura тільки обігову частину свого сталого капіталу,
але не відновлюють свій основний капітал  in natura, то при такому
припущенні немає жодної суперечности в тому, що 400\pound{ ф. стерл.}, які зворотно
припливають (вони припливають, скоро І купує на них засоби
споживання), різно розподіляються між цими двома підрозділами капіталістів
II.~Вони припливають назад до кляси II, але не повертаються в ті
самі руки, а різно розподіляються всередині цієї кляси, переходячи від
однієї частини її до іншої.

Одна частина капіталістів II, крім частини засобів продукції, заміщуваної,
кінець-кінцем, її товарами, перетворила 200\pound{ ф. стерл.} грішми на
нові елементи основного капіталу  in natura. Їхні гроші, таким чином
витрачені, — як і при відкритті підприємства — повертаються до них з циркуляції
лише протягом послідовного ряду років, як заміщення зношуваної
складової частини вартости основного капіталу, що перенесена на товари,
продуковані за допомогою цього основного капіталу.

Навпаки, друга частина капіталістів II на 200\pound{ ф. стерл.} не одержала
жодних товарів від І, але І платить їм тими грішми, що на них перша
частина капіталістів II купила елементи основного капіталу. Одна частина
капіталістів II знову має свою основну капітальну вартість у відновленій
натуральній формі, друга ще дбає про те, щоб нагромадити цю вартість
у грошовій формі для наступного заміщування свого основного капіталу
in natura.

Стан, що з нього нам треба виходити після попередніх, обмінів, це —
решта товарів, що їх треба обміняти з обох боків: $400 m$ у І підрозділі, $400 с$
у II\footnote{
Цифри знову не відповідають попередньому припущенню. Але це тут не має
значення, бо тут мають силу тільки відношення. \emph{Ф.~Е.}
}. Ми припускаємо, що II авансує 400 грішми для обміну цих товарів
на суму в 800. Половину цих 400 (\deq{} 200) в усякому разі мусить подати
та частина II~$с$, що нагромадила 200 грішми як вартість зношування
і що тепер повинна знову перетворити їх на натуральну форму свого
основного капіталу.

Цілком так само, як стала капітальна вартість, змінна капітальна вартість
і додаткова вартість — що на них можна розкласти вартість товарових
капіталів так II, як і І — можуть бути виражені в окремих пропорційних
частинах самих товарів II, зглядно товарів І, — цілком так само
може бути виражена й та частина вартости самої сталої капітальної вартости,
яку ще не доводиться перетворювати на натуральну форму основного
капіталу, але треба покищо поступінно нагромаджувати в грошовій
формі як скарб. Певна кількість товарів II (отже, в нашому прикладі —
половина остачі \deq{} 200) є тут лише носій цієї вартости зношування, що
має в наслідок обміну осісти в грошовій формі. (Перша частина капіталістів
\index{ii}{0357}  %% посилання на сторінку оригінального видання
II, яка відновлює основний капітал in natura, за допомогою відповідної
зношуванню частини тієї товарової маси, що від неї тут фігурує
лише остача, можливо, вже таким чином реалізувала частину його зношеної
вартости; але їм лишається ще реалізувати таким чином
200 в грошах).

Далі, щодо другої половини (\deq{} 200) тих 400\pound{ ф. стерл.}, що їх II
подав у циркуляцію при цій прикінцевій операції, то на неї купується
у I обігові складові частини сталого капіталу. Частину цих 200\pound{ ф. стерл.}
подали в циркуляцію, можливо, обидві частини капіталістів II або тільки
та частина, яка не відновлює in natura основної складової частини
вартости.

Отже, за допомогою 400\pound{ ф. стерл.} з I підрозділу вилучено: 1) на
суму в 200\pound{ ф. стерл.} таких товарів, що складаються лише з елементів
основного капіталу, 2) на суму в 200\pound{ ф. стерл.} таких товарів, що заміщують
in natura лише елементи обігової частини сталого капіталу II.~I продав тепер увесь свій річний товаровий продукт, оскільки його доводиться
продати II підрозділові; але вартість однієї п’ятої цього продукту
400\pound{ ф. стерл.} тепер існує в його руках у грошовій формі. Однак ці
гроші є перетворена на гроші додаткова вартість, яку доводиться витратити
як дохід на засоби споживання. Отже, I на ці 400\pound{ ф. стерл.} купує
в II всю товарову вартість \deq{} 400. Таким чином, гроші допливають назад
до II, вилучаючи його товари.

Припустімо тепер три випадки. При цьому ту частину капіталістів II,
яка заміщує основний капітал in natura, ми називаємо „частина 1“, а
ту, що нагромаджує в грошовій формі вартість зношування основного
капіталу, називаємо „частина 2“. Три випадки такі: a) певна частина тих
400, що як остача існують ще в II підрозділі в товарах, має замістити певну
частину обігових частин сталого капіталу для „частини 1“ і „частини 2“
(наприклад, по \sfrac{1}{2}); b) „частина 1“ уже продала ввесь свій товар, отже,
„частина 2“ ще повинна продати 400; c) „частина 2“ продала все, крім
тих 200, що є носії вартости зношування.

Тоді маємо такі розподіли:

a) З товарової вартости $\deq{} 400 с$, яка ще лишається в руках II, частині
1 належить 100 і частині 2--300; 200 з цих 300 репрезентують
зношування. В цьому разі з тих 400\pound{ ф. стерл.} грішми, що їх I тепер
подає назад, щоб одержати товари II, частина 1 спочатку витратила 300,
— а саме 200 грішми, що ними вона вилучила з I елементи основного
капіталу in natura, і 100 грішми для упосереднення свого обміну товарами
з І; навпаки, частина 2 з цих 400 авансувала тільки \sfrac{1}{4}, тобто
100 — так само для упосереднення свого товарового обміну з I.

Отже, з цих 400 грішми частина 1 авансувала 300 і частина
2--100.

Але з цих 400 повертаються назад:

До частини 1: 100, отже, лише \sfrac{1}{3} авансованих нею грошей. Але
замість решти, \sfrac{2}{3}, вона має відновлений основний капітал вартістю в 200.
За цей основний елемент капіталу вартістю в 200 вона дала І підрозділові
\index{ii}{0358}  %% посилання на сторінку оригінального видання
гроші, але не дала потім жодного товару. Щодо цих \sfrac{2}{3}
авансованих нею грошей, частина 1 виступає проти підрозділу I лише як
покупець, але не виступає ще потім як продавець. Отже, ці гроші
не можуть повернутись до частини 1: інакше сталось би, що вона одержала
елементи основного капіталу в подарунок від I. — Щодо останньої
третини авансованих нею грошей, частина 1 виступає спочатку
як покупець обігових складових частин свого сталого капіталу. На ці
самі гроші підрозділ I купує в частини 1 решту її товару вартістю
в 100. Отже, гроші повертаються до неї (до частини 1 підрозділу II)
назад, бо вона виступає як продавець товарів одразу після того, як
виступала покупцем. Коли б вони не повернулись, то сталося б, що
підрозділ II (частина 1) дав підрозділові I за товари в сумі на 100 спочатку
100 грішми, а потім ще 100 товаром, отже, подарував би йому
свій товар.

Навпаки, до частини 2, яка витратила 100 грішми, повертається
300 грішми: 100 — тому, що вона спочатку як покупець подала в циркуляцію
100 грішми, а потім одержала їх назад як продавець; 200 —
тому, що вона функціонує тільки як продавець товарів, на суму вартости
в 200, але не як покупець. Отже, гроші не можуть повернутись до I.~Отже, зношування основного капіталу покривається грішми, що їх II
(частина 1) подав у циркуляцію на закуп елементів основного капіталу;
але вони потрапляють до рук частини 2 не як гроші частини 1, а як
гроші, що належать підрозділові I.

b) При цьому припущенні решта II~$c$ розподіляється так, що частина
1 має 200 грішми, а частина 2--400 в товарах.

Частина 1 продала всі свої товари, але 200 в грошах є перетворена
форма основної складової частини її сталого капіталу, яку треба відновити
in natura. Отже, частина 1 виступає тут лише як покупець і замість
своїх грошей одержує на ту саму суму вартости товари І в формі натуральних
елементів основного капіталу. Частині 2 доводиться подати в
циркуляцію (коли I не авансував грошей для обміну товарів між I і II)
maximum лише 200\pound{ ф. стерл.}, бо в розмірі половини своєї товарової
вартости вона є лише продавець підрозділові I, а не покупець у
підрозділу I.

З циркуляції повертаються 400\pound{ ф. стерл.} до частини 2; 200 — тому,
що вона їх авансувала як покупець і одержує їх назад як продавець
товарів на 200; 200 — тому, що вона продає підрозділові I товарів вартістю
на 200, не одержуючи за це товарового еквіваленту від I.

c) Частина 1 має 200 в грошах і $200 c$ в товарах; частина 2 —
$200 c$ (d) в товарах.

Частина 2 при цьому припущенні не має авансувати грішми нічого,
бо вона проти підрозділу I взагалі вже функціонує не як покупець, а
лише як продавець, отже, їй треба чекати, поки в неї куплять.

Частина 1 авансує 400\pound{ ф. стерл.} грішми; 200 для взаємного обміну
товарами з I, 200 — просто як покупець у I.~На ці останні 200\pound{ ф. стерл.}
грішми вона купує елементи основного капіталу.
