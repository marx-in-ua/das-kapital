\parcont{}  %% абзац починається на попередній сторінці
\index{ii}{0113}  %% посилання на сторінку оригінального видання
репродукції вартість машини поступінно акумулюється насамперед в формі резервного грошового фонду.

Інші елементи продуктивного капіталу складаються почасти з елементів сталого капіталу, які є в
допоміжних матеріялах та сировинних матеріялах, а почасти із змінного капіталу, витраченого на
робочу силу.

\vtyagnut{}
Аналіза процесу праці й процесу зростання вартости (книга І, розділ V) виявила, що ці різні складові
частини відіграють цілком різну ролю в утворенні продукту і в утворенні вартости. Вартість тієї
частини сталого капіталу, яка складається з допоміжних та сировинних матеріялів — цілком так само,
як і вартість тієї його частини, яка складається з засобів праці — знову з’являється в вартості
продукту, як лише перенесена вартість, тимчасом як робоча сила за посередництвом процесу праці додає
до продукту еквівалент своєї вартости, або дійсно репродукує свою вартість. Далі, одну частину
допоміжних матеріялів, — вугілля на опалення, світильний газ тощо, — зужитковується в процесі праці,
при чому речово вона не увіходить у продукт, тимчасом як друга частина їх своїм тілом увіходить у
продукт і становить матеріял його субстанції. Але всі ці відмінності не мають значення для
циркуляції, а тому й для способу обороту. Коли допоміжні й сировинні матеріяли цілком зужитковується
підчас утворення певного продукту, то вони цілком переносять свою вартість на продукт. Тому вона
через продукт цілком подається в циркуляцію, перетворюється на гроші, а з грошей знову на елементи
продукції товару. Її оборот не переривається, як оборот основного капіталу, але безупинно перебігає
весь кругобіг своїх форм, так що ці елементи продуктивного капіталу постійно відновлюються in
natura.

Щодо змінної складової частини продуктивного капіталу, витрачуваної на робочу силу, то робочу силу
купується на певний час. Коли капіталіст купить її і введе в продуктивний процес, то вона утворює
складову частину його капіталу, а саме — його змінну частину. Вона діє щоденно певний час, що
протягом його вона додає до продукту не лише всю свою денну вартість, а також ще деяку надлишкову
додаткову вартість, яку ми покищо залишаємо осторонь. Після того, як робочу силу куплено, й вона
діяла, напр., протягом тижня, закуп її мусить постійно відновлюватися у певні терміни. Той
еквівалент її вартости, що його робоча сила долучає до продукту протягом свого функціонування і що в
наслідок циркуляції продукту перетворюється на гроші, мусить завжди знову перетворюватись з грошей
на робочу силу або завжди мусить пророблювати повний кругобіг своїх форм, тобто завжди обертатись,
щоб не перервався кругобіг безперервної продукції.

\roztyagnut
Отже, частина вартости продуктивного капіталу, авансована на робочу силу, цілком переходить на
продукт (додаткову вартість ми залишаємо ввесь час осторонь), разом з ним перебігає обидві
метаморфози, що належать до сфери циркуляції, і в наслідок цього постійного відновлення завжди
лишається зв’язана з продукційним процесом. Отже, хоч як у всьому іншому робоча сила відрізняється
щодо утворення вартости від тих складових частин сталого капіталу, які не становлять \emph{основного}
\index{ii}{0114}  %% посилання на сторінку оригінального видання
\emph{капіталу}, спосіб обороту вартости є спільний у робочої сили з цими складовими частинами,
протилежно до основного капіталу. Ці складові частини продуктивного капіталу, — а саме ті частини
його вартости, що їх витрачається на робочу силу й засоби продукції, які не становлять основного
капіталу — в наслідок цієї спільности характеру їхнього обороту, — протистоять основному капіталові,
як \emph{обіговий} або \emph{поточний} капітал.

Як ми бачили раніше, гроші, що їх капіталіст сплачує робітникові за вживання робочої сили, справді є
лише загальна еквівалентна форма доконечних робітникові засобів існування. Остільки й змінний
капітал речово складається з засобів існування. Але тут, розглядаючи оборот, ідеться про форму.
Капіталіст купує не засоби існування робітника, але саму його робочу силу. Змінну частину його
капіталу являють не засоби існування робітника, але його діюща робоча сила. В процесі праці
капіталіст продуктивно споживає саму робочу силу, а не засоби існування робітника. Сам робітник
перетворює на засоби існування ті гроші, що їх він одержав за свою робочу силу, щоб потім
перетвороти знову ці засоби існування на робочу силу й підтримати своє існування, цілком так само,
як, напр., капіталіст перетворює на засоби свого існування деяку частину додаткової вартости, що є в
товарі, який він продає за гроші, і, не зважаючи на це, зовсім не можна сказати, що покупець його
товарів сплачує йому засобами існування. Навіть коли робітникові сплачується частину його заробітної
плати в засобах існування in natura, то це за наших часів є вже друга оборудка. Він продає свою
робочу силу за певну ціну і при цьому умовляється, що частину цієї ціни він одержить в засобах
існування. Цим змінюється лише форма виплати, але не змінюється та обставина, що він дійсно продає
свою робочу силу. Це є друга оборудка, що відбувається вже не між робітником і капіталістом, а між
робітником як покупцем товару і капіталістом як продавцем товару; тимчасом як у першій оборудці
робітник є продавець товару (своєї робочої сили), а капіталіст її покупець. Цілком так само, як коли
б капіталіст, продаючи свій товар, прим., машину на гамарню, захотів мати за неї товар — залізо.
Отже, не засоби існування робітника визначаються як обіговий капітал протилежно до основного. А
також і не робоча сила його, а частина вартости продуктивного капіталу, витрачена на робочу силу,
яка через форму свого обороту набуває цього характеру обігового капіталу, спільно з деякими
складовими частинами сталого капіталу і протилежно до деяких інших складових частин сталого
капіталу.

Вартість поточного капіталу — в робочій силі та засобах продукції — авансується лише на той час, що
протягом його виготовляється продукт, залежно від маштабу продукції, визначуваного розміром
основного капіталу. Ця вартість цілком увіходить у продукт, а тому після продажу продукту цілком
повертається з циркуляції, і можна знову її авансувати. Робоча сила й засоби продукції, що в них
існує поточна складова частина капіталу, вилучається з циркуляції в розмірі, потрібному на
вироблення й продаж готового продукту, але їх завжди треба замінювати
\parbreak{}  %% абзац продовжується на наступній сторінці
