\parcont{}  %% абзац починається на попередній сторінці
\index{ii}{0014}  %% посилання на сторінку оригінального видання
робоча сила діє лише як один з його органів, то й утворений її додатковою
працею надлишок вартости продукту над вартістю його складових
елементів стає здобутком капіталу. Додаткова праця робочої сили є
безплатна праця для капіталу і тому утворює капіталістові додаткову
вартість, тобто вартість, що йому нічого не коштує. Тому продукт є не
лише товар, але товар, запліднений додатковою вартістю. Його вартість
дорівнює $П \dplus{} М$, дорівнює вартості зужиткованого на його виготовлення
продуктивного капіталу $П$ плюс утворена ним додаткова вартість $М$.
Припустімо, що цей товар складається з \num{10.000} фунтів пряжі, що на її
виготовлення зужитковано засобів продукції вартістю в 372\pound{ ф. стерл.} і
робочої сили вартістю в 50\pound{ ф. стерл}. У процесі прядіння прядільники
перенесли на пряжу вартість засобів продукції, зужиткованих їхньою
працею, сумою в 372\pound{ ф. стерл.}, і разом з тим, відповідно до їхньої
витрати праці, утворили нову вартість, напр., 12\pound{ ф. стерл.}. Отже,
\num{10.000} ф. пряжі є носії вартости в 500\pound{ ф. стерл}.

\subsection[Третя стадія. $Т' — Г'$]{Третя стадія. \emph{Т′ − Г′}}

Товар стає \emph{товаровим капіталом} як постала безпосередньо з самого
продукційного процесу функціональна форма буття капітальної вартости,
що вже сама з себе зросла у своїй вартості. Коли б товарову продукцію
в цілому її суспільному обсязі провадилось капіталістично, то кожен
товар з самого початку був би елементом товарового капіталу, все одно,
чи складається цей товар з чавуну, брюссельських мережив, чи з сульфатової
кислоти або сигар. Проблема, які саме ґатунки товару в наслідок своїх
властивостей призначені до ранґу капіталу, а які до звичайної товарової
служби, належить до тих милих труднощів, що їх сама собі утворювала
схоластична економія.

\roztyagnut
У своїй товаровій формі капітал мусить виконувати функцію товару.
Предмети, що з них він складається, з самого початку вироблені для
ринку, мусять бути продані, перетворені на гроші, отже, мусять вони
проробити рух $Т — Г$.

\roztyagnut
Припустімо, що товар капіталіста складається з \num{10.000} фун. бавовняної
пряжі. Коли в процесі прядіння зужитковано засобів продукції вартістю
в 372\pound{ фунт, стерл.} і утворено нову вартість в 128\pound{ ф. стерл.}, то вартість
пряжі дорівнює 500\pound{ ф. стерл.}, яка й виражається в її однойменній ціні.
Цю ціну реалізується через продаж $Т — Г$. Що перетворює цей простий
акт усякої товарової циркуляції одночасно на функцію капіталу? Зовсім
не якась зміна, що постає підчас цього акту, будь-то зміна щодо споживного
характеру товару, бо як предмет споживання товар переходить до покупця,
будь-то зміна щодо його вартости, бо остання зовсім не зазнає змін
щодо величини, тут змінюється лише її форма. Спочатку вартість існувала
в пряжі, тепер вона існує в грошах. Так виявляється посутня
ріжниця між першою стадією $Г — Т$
% REMOVED
%\footnote*{
%В ориґіналі тут стоїть: „$Т — Г$“; очевидна друкарська помилка. \emph{Ред.}}
і останньою стадією $Т — Г$. Там
\parbreak{}  %% абзац продовжується на наступній сторінці
