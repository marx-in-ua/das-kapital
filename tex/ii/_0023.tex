
\index{ii}{0023}  %% посилання на сторінку оригінального видання
\subsection{Кругобіг у цілому}

Ми бачили, що процес циркуляції по скінченні його першої фази
$Г — Т\splitfrac{Р}{Зп} $ переривається через $П$, що в ньому товари $Р$ і $Зп$, куплені
на ринку, споживається як речеві й вартісні складові частини продуктивного
капіталу; продукт цього споживання є новий товар, $Т'$, змінений
речево і щодо вартости. Перерваний процес циркуляції, $Г — Т$, мусить
доповнитись через $Т — Г$. Але як носій цієї другої та кінцевої фази циркуляції
з’являється $Т'$, товар відмінний від першого $Т$ речево і щодо
вартости. Отже, ряд циркуляцій має такий вигляд: 1) $Г — Т_1$; 2) $Т_2' — Г'$,
де в другій фазі першого товару $Т_1$, підчас перерви, зумовленої функцією
$П$, підчас продукції $Т'$ з елементів $Т$, з форм буття продуктивного
капіталу $П$, постає другий товар, вищої вартости та іншої споживної
форми, $Т_2'$. Навпаки, перша форма виявлення, що в ній капітал виступив
перед нами (книга І, розділ IV, І), $Г — Т — Г'$ (розкладається
на: 1) $Г — Т_1$; 2) $Т_1 — Г'$), двічі показує той самий товар. Там перед
нами обидва рази той самий товар, на який перетворюються гроші
в першій фазі і який в другій фазі перетворюється на більшу кількість
грошей. Не зважаючи на цю посутню ріжницю, обидві циркуляції мають
те спільне, що в їхній першій фазі гроші перетворюються на товар, і
в їхній другій фазі товар перетворюється на гроші, отже, що гроші, витрачені
в першій фазі, зворотно припливають у другій фазі. З одного боку,
спільне у них — зворотний приплив грошей до свого вихідного пункту,
але, з другого боку, і те, що грошей зворотно припливає більше, ніж було
авансовано. В цьому розумінні $Г — Т\dots{} Т' — Г'$ вже міститься в загальній
формулі $Г — Т — Г'$.

\vtyagnut
Далі виявляється, що в обох належних до циркуляції метаморфозах
$Г — Т$ і $Т' — Г'$ кожного разу протистоять одна одній і заступають одна
одну рівновеликі, одночасно наявні вартості. Зміна величини вартости
належить виключно метаморфозі $П$, продукційному процесові, що таким
чином становить реальну метаморфозу капіталу протилежно простій формальній
метаморфозі циркуляції.

А тепер розгляньмо цілий рух
$Г — Т\dots{} П\dots{} Т' — Г'$,
або його розгорнуту форму
$Г — Т\splitfrac{Р}{Зп}\dots{} П\dots{} Т' (Т \dplus{} т) — Г' (Г \dplus{} г)$.
Капітал з’являється тут
як вартість, що перебігає ряд взаємно зв’язаних, одне одним зумовлених
перетворень, ряд метаморфоз, які являють стільки ж фаз або стадій цілого
процесу. Дві з цих фаз належать до сфери циркуляції, одна — до
сфери продукції. В кожній з цих фаз капітальна вартість перебуває в
особливій формі, що їй відповідає особлива, спеціяльна функція. В цьому
русі авансована вартість не лише зберігається, але й зростає, збільшує
свою величину. Нарешті, в кінцевій стадії вона повертається до тієї самої
форми, що в ній вона з’явилась на початку цілого процесу. Тому цей
цілий процес є процес кругобігу.
