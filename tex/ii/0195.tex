що зменшується розміри продукції. Порівняно з маштабом продукції, капітал,
закріплений в грошовій формі, тут ще більше зростає.

Таким поділом капіталу на первісний продуктивний і додатковий капітал
взагалі досягається безперервна послідовність робочих періодів, постійне
функціонування однаково великої частини авансованого капіталу,
як продуктивного капіталу.

Придивімось до прикладу II. Капітал, що постійно перебуває в процесі
продукції, є 500 ф. стерл. А що робочий період дорівнює 5 тижням,
то протягом 50 тижнів (а їх ми беремо, як рік) цей капітал буде в
роботі 10 разів. Тому й продукт, — лишаючи осторонь додаткову вартість
— дорівнює 500X10—5000 ф.  стерл. Отже, з погляду капіталу,
безпосередньо і безупинно діющого в прсдукційному процесі, — з погляду
капітальної вартости в 500 ф. стерл., — час обігу, здається, цілком
знищується. Період обороту збігається з робочим періодом; час обігу прирівнюється
нулеві.

Коли б, навпаки, продуктивну діяльність капіталу в 500 ф. стерл. регулярно
перепинялося п’ятитижневим періодом обігу, так що він ставав
би знову продукційноздатним лише по закінченні цілого десятитижневого
періоду обороту, то протягом 50 тижнів року ми мали б 5 десятитижневих
оборотів; в них було б 5 п’ятитижневих періодів продукції, отже,
разом 25 тижнів продукції з загальною кількістю продукту на 500 X 5 = 2500
ф. стерл.; 5 п’ятитижневих періодів обігу, отже, цілого часу обігу теж
25 тижнів. Коли ми тут кажемо, що капітал в 500 ф. стерл. обернувся
5 разів протягом року, то очевидно й зрозуміло, що протягом половини
кожного періоду обороту цей капітал в 500 ф. стерл. зовсім не функціонував
як продуктивний капітал, і що в підсумку він функціонував тільки
протягом півроку, а другу половину року зовсім не функціонував.

В нашому прикладі на час цих п’ятьох періодів обігу входить у роботу
додатковий капітал в 500 ф. стерл., і в наслідок цього оборот підвищується
з 2500 ф. стерл. до 5000 ф. стерл. Але й авансований капітал
тепер є 1000 ф. стерл. замість 500 ф. стерл. 5000 поділені на 1000
дорівнює 5. Отже, замість 10 оборотів маємо 5. Так справді й рахують.
Однак, коли кажуть, що капітал 1000 ф. стерл. обернувся 5 разів протягом
року, то в пустій голові капіталіста зникає спогад про час обігу,
і постає сплутане уявлення, ніби цей капітал протягом 5 послідовних
оборотів постійно функціонував у процесі продукції. Але, коли ми кажемо,
що капітал 1000 ф. стерл. обернувся п’ять разів, то сюди ввіходить
і час обігу й час продукції. Справді, коли б 1000 ф. стерл. безперервно
функціонували в процесі продукції, то при наших припущеннях продукт
мусив би бути 10000 ф. стерл. замість 5000. Але для того, щоб завжди
мати в процесі продукції 1000 ф. стерл., довелось би взагалі авансувати
2000 ф. стерл. Економісти, що в них взагалі не знайти нічого виразного
про механізм обороту, завжди недобачають той головний момент, що
продукція може відбуватися безперервно лише тоді, коли в процесі продукції
завжди буде фактично зайнята тільки частина промислового капіталу.
Тимчасом як одна частина перебуває в періоді продукції, друга час-
