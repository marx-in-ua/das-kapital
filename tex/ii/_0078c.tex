\parcont{}  %% абзац починається на попередній сторінці
\index{ii}{0078}  %% посилання на сторінку оригінального видання
5000\pound{ ф. стерл.}, що з них 4000\pound{ ф. стерл.} становлять основний капітал і
1000\pound{ ф. стерл.} — обіговий\footnote*{
Про поняття „основний капітал“ і „обіговий капітал“ дивись далі розділ
VIII. \Red{Ред.}
}; ці 1000\pound{ ф. стерл.}, згідно з попереднім
припущенням, \deq{} $800с \dplus{} 200v$. Щоб його цілий капітал обернувся
один раз протягом року, його обіговий капітал мусить обернутись п’ять
разів на рік; тоді його товаровий продукт дорівнює 6000\pound{ ф. стерл.},
отже, на 1000\pound{ ф. стерл.} більший, ніж його авансований капітал. А відси
маємо знову те саме відношення додаткової вартости, що й раніш:
$5000с: 1000m$ \deq{} $100 (с \dplus{} v): 20m$. Цей оборот, отже, нічого не
змінює у відношенні цілого попиту капіталіста до цілого його подання:
перший лишається на \sfrac{1}{5} менший, ніж останнє.

Припустімо, що основний капітал його треба поновлювати щодесять
років. Отже, капіталіст амортизує щороку \sfrac{1}{10} \deq{} 400\pound{ ф. стерл}. В наслідок
цього він має тепер вартість лише на 3600\pound{ ф. стерл.} в основному капіталі
\dplus{} 400\pound{ ф. стерл.} грішми. Коли потрібен ремонт, і він не більший за
пересічний, то витрати на нього є не що інше, як капіталовкладання, що
їх він (капіталіст) робить додатково. Ми можемо розглядати справу так,
ніби капіталіст, визначаючи вартість вкладуваного капіталу, оскільки
вона ввіходить у річний товаровий продукт, з самого початку завів у
неї всі витрати на ремонт, так що в амортизації, яка дорівнює \sfrac{1}{10},
є й вони. (Коли в дійсності потреба в ремонті в нього нижча
за пересічну, то це є вигода для нього, а коли вища, то втрата. Але
для цілої кляси капіталістів, занятих у тій самій галузі промисловости,
такі вигоди й втрати вирівнюються). В усякому разі, хоч при одному
обороті на рік цілого його капіталу його річний попит лишається рівний
5000\pound{ ф. стерл.}, тобто дорівнює його первісно авансованій капітальній
вартості, все ж цей попит порівняно з обіговою частиною капіталу
більшає, тимчасом як порівняно з основною частиною капіталу він постійно
меншає.

Тепер перейдімо до репродукції. Припустімо, що капіталіст споживає
всю додаткову вартість $г$ і перетворює на продуктивний капітал лише
первісну величину капіталу $С$. Тоді попит капіталіста щодо вартости
дорівнює його поданню. Але не дорівнює, коли звернути увагу на рух
його капіталу. Як капіталіст, він ставить попит лише на \sfrac{4}{5} свого подання
(щодо величини вартости), \sfrac{1}{5} він споживає як некапіталіст, не
в наслідок своїх функцій капіталіста, а на свої особисті потреби або
розкоші.

Обчислення в відсотках буде таке:
\begin{table}[H]
\centering
\begin{tabular}{l@{ }r@{ }l@{ }r@{}l@{ }r@{ }r}
Попит його, як & капіталіста & \deq{} & 100 &, подання & \deq{} & 120\\
\ditto{Попит} \ditto{його,} \ditto{як} & розкішника &  & 20 & & & \textemdash{}\\
\midrule
Сума \ditto{попиту} & & & 120 & \ditto{подання} & & 120\\
\end{tabular}
\end{table}
\noindent{}Таке припущення рівнозначне припущенню, що капіталістичної продукції
не існує, а тому не існує й самого промислового капіталіста. Бо
\parbreak{}  %% абзац продовжується на наступній сторінці
