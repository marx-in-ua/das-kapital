
\index{ii}{0273}  %% посилання на сторінку оригінального видання
На основі суспільної продукції треба визначити маштаб, що в ньому
такі операції, які на довгий час відтягують робочу силу й засоби продукції,
не даючи протягом цього часу жодного продукту як корисного
наслідку, можуть провадитись без шкоди для тих галузей продукції, які
постійно або кілька разів на рік не лише відтягують робочу силу й засоби
продукції, а й дають засоби, існування й засоби продукції. За суспільної
продукції, так само, як і за капіталістичної продукції, робітники
в галузях підприємств з короткими робочими періодами, як і раніше, лише
на короткий час відтягуватимуть продукти, не даючи натомість нового
продукту, тимчасом як галузі підприємств з довгими робочими періодами,
перше ніж вони сами почнуть давати продукти, постійно відтягують
продукти на довгий час. Отже, ця обставина випливає з речових
умов відповідного процесу праці, а не з його суспільної форми. За суспільної
продукції грошовий капітал відпадає. Суспільство розподіляє робочу
силу й засоби продукції між різними галузями праці. Продуценти
можуть, правда, одержувати паперові посвідки, що ними вони вилучають
з суспільних споживних запасів ту кількість продуктів, яка відповідає їхньому
робочому часові. Ці посвідки — зовсім не гроші. Вони не циркулюють.

Тепер ми бачимо, що, оскільки потреба в грошовому капіталі випливає
з протягу робочого періоду, її зумовлено двома обставинами: \emph{поперше},
тією, що гроші взагалі є та форма, що в ній мусить виступити
кожен індивідуальний капітал (кредит ми лишаємо осторонь) для того,
щоб перетворитись на продуктивний капітал. Це випливає з суті капіталістичної
продукції, взагалі товарової продукції. — \emph{Подруге}, величину
потрібного грошового авансування зумовлює та обставина, що протягом
порівняно довгого часу суспільству постійно відбирається робочу силу
й засоби продукції, при чому протягом цього часу йому не повертається
жодного продукту, що його можна було б перетворити на гроші.
Першої обставини, а саме того, що авансовуваний капітал треба авансувати
в грошовій формі, не знищує форма самих цих грошей, тобто те,
що вони є або металеві, або кредитові гроші, або знаки вартости й~\abbr{т. ін.} На другу обставину жодного впливу не справляє те, за допомогою
яких грошових засобів або за допомогою якої форми продукції
відтягають працю, засоби існування та засоби продукції, не подаючи
натомість у циркуляцію жодного еквіваленту.
\label{original-273}
