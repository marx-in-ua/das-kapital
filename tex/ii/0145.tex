поточний капітал, а не основний, оскільки І) вартість його цілком входить
у продукт і 2) оскільки його in natura цілком заміщено новим екземпляром
з нового продукту.

А. Сміс каже нам, з чого складається обіговий і основний капітал.
Він перелічує ті речі, ті речові елементи, що становлять основний капітал,
і ті, що становлять обіговий капітал, ніби таке призначення властиве
цим речам речово, з природи, а не випливає з певних функцій
цих речей в капіталістичному процесі продукції. І однак в тому самому
розділі (Book II, chap. 1) він зауважує, що, хоч певна річ, напр., житлова
будівля, призначена для безпосереднього споживання „може давати
дохід своєму власникові, а значить, служити йому, функціонуючи як
капітал, однак вона не може ні давати дохід суспільству, ані служити
йому, функціонучи як капітал, отже, вона ані трохи не збільшує доходу
всього суспільства“ *). Тут А. Сміс цілком виразно висловлює думку, що
властивість бути капіталом речі мають не як такі і не за всяких обставин,
але що це є така функція, яку вони, залежно від обставин, іноді мають,
а іноді не мають. Але те, що має силу для капіталу взагалі, те має
силу й для його підрозділів.

Ті самі речі становлять складову частину поточного або основного
капіталу залежно від того, яку функцію вони виконують в процесі праці.
Напр., худоба, як робоча худоба (засіб праці) становить речову форму
існування основного капіталу; навпаки, як худоба, відгодовувана на
заріз (сировинний матеріял), вона становить складову частину обігового
капіталу фармера. З другого боку, та сама річ може іноді функціонувати
як складова частина продуктивного капіталу, а іноді належати до
фонду безпосереднього споживання. Напр., будинок, функціонучи як місце
праці, є основна складова частина продуктивного капіталу, а функціонуючи
як житлова будівля власника, зовсім не має форми капіталу.
Ті самі засоби праці можуть у багатьох випадках функціонувати то як
засоби продукції, то як засоби споживання.

Це була одна з помилок, що випливають із Смісового уявлення: особливості
основного та обігового капіталу розглядати як особливості,
властиві речам. Аналіза процесу праці („Капітал“, книга 1, розділ V)
вже показала, як змінюються визначення засобу праці, матеріялу праці,
продукту, залежно від різної ролі, що та сама річ відіграє в цьому
процесі. Але визначення основного і не основного капіталу ґрунтуються
й собі на тих певних ролях, що їх ці елементи відіграють у процесі праці,
а, значить, і в процесі утворення вартости.

Але, подруге, при перелічуванні речей, що з них складається основний
і обіговий капітал, виразно виявляється, що А. Сміс сплутує ріжницю
між основними й поточними складовими частинами капіталу, яка має
силу й рацію лише щодо продуктивного капіталу (капіталу в його

*)... may yield a revenue to its proprietor, and thereby serve in the function
of a capital to him, it cannot yield any to the public, nor serve in the function
of a capital to it, and the revenue of the whole body of the people can never be
in the smallest degree increased by it" (p. 186).
