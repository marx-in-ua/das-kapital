\parcont{}  %% абзац починається на попередній сторінці
\index{ii}{0294}  %% посилання на сторінку оригінального видання
різним особам, які беруть участь у процесі продукції. Але це ні в якому
разі не є тотожне з складанням вартости з цих трьох „складових частин“.
Коли я визначу окремо величину трьох різних прямих ліній, а
потім із цих трьох ліній як „складових частин“ утворю четверту пряму
лінію, рівну величиною сумі трьох ліній, то це зовсім не та сама процедура,
як коли б я, маючи перед собою дану пряму лінію, з тим або
іншим наміром почав ділити, певним способом „розкладати“ її на три
різні частини. В першому разі величина лінії цілком змінюється з зміною
величини трьох ліній, що суму їх вона становить; в останньому разі
величину трьох частин лінії з самого початку обмежено тим, що вони
є частини лінії даної величини.

Але в дійсності, оскільки ми триматимемось того, що є правильного
у викладі А.~Сміта, а саме, що \so{новоутворена річною працею} вартість,
яка є в річному товаровому продукті суспільства (як у кожному поодинокому
товарі, або в денному, тижневому тощо продукті), дорівнює вартості
авансованого змінного капіталу (отже, частині вартости знову призначеної
на закуп робочої сили) плюс додаткова вартість, що її капіталіст може
реалізувати — за простої репродукції та за інших незмінних умов — в засобах
свого особистого споживання, і коли ми пригадаємо далі, що А.~Сміт звалює до однієї купи працю, оскільки вона утворює вартість, є
витрата робочої сили, і працю, оскільки вона утворює споживну вартість,
тобто витрачається в корисній, доцільній формі, — то все уявлення
А.~Сміта зійде ось на що: вартість кожного товару є продукт праці,
отже, і вартість продукту річної праці або вартість річного суспільного
товарового продукту. Але що кожна праця розкладається на: 1) доконечний
робочий час, що протягом його робітник лише репродукує еквівалент
капіталу, авансованого на закуп його робочої сили, і 2) додаткову працю,
що нею він дає капіталістові вартість, за яку той не платить жодного
еквіваленту, отже, додаткову вартість, — то всяка товарова вартість може
розкластися лише на ці дві різні складові частини і становить, кінець-кінцем —
як заробітна плата — дохід робітничої кляси, а як додаткова вартість —
дохід кляси капіталістів. Щождо сталої капітальної вартости, тобто вартости
засобів продукції, зужиткованих у продукції річного продукту, то
хоч і не можна сказати (крім фрази, що капіталіст прираховує її покупцеві,
продаючи свій товар), яким чином ця вартість входить у вартість
нового продукту, однак, кінець-кінцем — ultimately — в наслідок того, що
сами засоби продукції є продукт праці, сама ця частина вартости знову
таки може складатись лише з еквіваленту змінного капіталу та додаткової
вартости: з продукту доконечної праці та додаткової праці. Коли
вартості цих засобів продукції в руках тих, хто застосовує їх, функціонують
як капітальні вартості, то все ж це не заважає тому, що „первісно“,
і якщо дошукатись самої суті їхньої, в інших руках — хоча б і раніше — їх
можна було розкласти на ті самі дві частини вартости, отже, на два
різні джерела доходу.

Правильне у всьому цьому ось що: в русі суспільного капіталу, —
тобто сукупности індивідуальних капіталів — справа виступає інакше, ніж
\parbreak{}  %% абзац продовжується на наступній сторінці
