
\index{ii}{0227}  %% посилання на сторінку оригінального видання
Подивімось тепер далі, що виражає відношення:\[
\frac{\text{капітал, що обертається протягом року}}{\text{авансований капітал}}
\]
(при цьому ми, як уже сказано, маємо на увазі лише змінний капітал).
Це ділення дає нам число оборотів капіталу, авансованого протягом року.

Для капіталу $А$ маємо:\[
\frac{\text{5000\pound{ ф. стерл.} капіталу, що обернувся протягом року}}{\text{500\pound{ ф. стерл.} авансованого капіталу}};
\]
для капіталу $В$:\[
\frac{\text{5000\pound{ ф. стерл.} капіталу, що обернувся протягом року}}{\text{5000\pound{ ф. стерл.} авансованого капіталу}}
\]
В обох відношеннях чисельник виражає авансований капітал, помножений
на число оборотів: для $А$ 500 × 10, для $В$ 5000 × 1. Або помножений
на обернений дріб часу обороту, обчисленого на рік. Час
обороту для $А \deq{} \sfrac{1}{10}$ року; обернений дріб часу обороту \deq{} \sfrac{10}{1} року,
отже, 500 × \sfrac{10}{1} \deq{} 5000; для $В \deq{} 5000 × \sfrac{1}{1} \deq{} 5000$. Знаменик виражає
капітал, що обернувся, помножений на обернений дріб числа
оборотів; для $А \deq{} 5000 × \sfrac{1}{10}$; для $В \deq{} 5000 × \sfrac{1}{1}$.

Відповідні маси праці (сума оплаченої та неоплаченої праці), що їх
пускається в рух обома змінними капіталами, які обернулись протягом
року, тут однакові, бо самі капітали, що обернулись, однакові, й норми
зростання їхньої вартости теж однакові.

Відношення капіталу, що обернувся протягом року, до авансованого
змінного капіталу показує: 1) Відношення, що в ньому стоїть капітал,
який треба авансувати, до змінного капіталу, застосованого протягом
певного робочого періоду. Коли число оборотів \deq{} 10, як для капіталу
$А$, і рік береться в 50 тижнів, то час обороту \deq{} 5 тижням. Змінний
капітал треба авансувати на ці 5 тижнів, і цей авансований на 5 тижнів
капітал мусить бути вп’ятеро більший від змінного капіталу, застосованого
протягом одного тижня. Інакше кажучи, тільки \sfrac{1}{5} авансованого
капіталу (тут 500\pound{ ф. стерл.}) можна застосувати протягом одного тижня.
Навпаки, при капіталі $В$, де число оборотів \deq{} \sfrac{1}{1}, час обороту \deq{} 1 рокові
\deq{} 50 тижням. Відношення авансованого капіталу до щотижнево застосовуваного
є, отже, $50 : 1$. Коли б для $В$ відношення було таке саме,
як для $А$, то $В$ довелось би щотижня прикладати 1000\pound{ ф. стерл.} замість
100. 2) З цього випливає, що $В$ застосував вдесятеро більший
капітал (5000\pound{ ф. стерл.}), ніж $А$, щоб пустити в рух таку саму масу змінного
капіталу, отже, за даної норми додаткової вартости, і таку саму
масу праці (оплаченої та неоплаченої), отже, щоб спродукувати протягом
року таку саму масу додаткової вартости. Справжня норма додаткової
\parbreak{}  %% абзац продовжується на наступній сторінці
