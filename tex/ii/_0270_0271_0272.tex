
\index{ii}{0270}  %% посилання на сторінку оригінального видання
Поперше, він являє ту форму, що в ній кожний індивідуальний капітал
виступає на кін, починає свій процес як капітал. Тому він виступає
як primus motor\footnote*{
Перший рушій. \emph{Ред.}
}, що надає руху цілому процесові.

Подруге. Відповідно до різного протягу періоду обороту і різного
відношення між обома складовими частинами його — робочим періодом і
періодом циркуляції — складова частина авансованої капітальної вартости,
що її завжди треба авансувати і відновлювати в грошовій формі, є різна
у відношенні до продуктивного капіталу, що його вона пускає в рух,
тобто у відношенні до безперервного розміру продукції. Але хоч яке
це буде відношення, за всіх обставин та частина капітальної вартости,
що процесує, що може постійно функціонувати як продуктивний капітал,
обмежується тією частиною авансованої капітальної вартости, яка мусить
завжди існувати в грошовій формі поряд продуктивного капіталу. Тут
ідеться лише про нормальний оборот, про абстрактну пересічну величину.
При цьому ми лишаємо осторонь додатковий грошовий капітал,
потрібний, щоб вирівнювати застої циркуляції.

\emph{До першого пункту}. Товарова продукція припускає товарову
циркуляцію, а товарова циркуляція припускає виявлення товару в грошах,
грошову циркуляцію; двоїсте буття товару: як товару, і як грошей,
є закон виявлення продукту як товару. Так само капіталістична товарова
продукція, — хоч суспільно, хоч індивідуально розглядувана —
припускає капітал у грошовій формі або грошовий капітал як primus
motor для кожного новопосталого підприємства, і як постійний рушій. Обіговий
капітал зокрема припускає, що через короткі переміжки постійно
знову й знов з’являється грошовий капітал як рушій. Всю авансовану
капітальну вартість, тобто всі складові частини капіталу, що складаються
з товарів, робочої сили, засобів праці й матеріялів продукції, постійно
доводиться знову й знов купувати на гроші. Те, що тут сказано про індивідуальний
капітал, має силу й щодо суспільного капіталу, який функціонує
лише в формі багатьох індивідуальних капіталів. Але, як уже показано
в І книзі, з цього зовсім не випливає, щоб поле функціонування
капіталу, маштаб продукції, навіть на капіталістичній основі, в своїх
\emph{абсолютних} розмірах залежав від розміру діющого грошового капіталу.

В капітал заведено елементи продукції, що їхня здатність розширюватись,
у певних межах, не залежить від величини авансованого грошового
капіталу. При однаковій оплаті робочої сили її можна екстенсивно
або інтенсивно більше визискувати. Якщо із збільшенням визиску збільшується
грошовий капітал (тобто підвищується заробітну плату), то не
пропорційно до збільшення визиску, отже, pro tanto він зовсім не збільшується.

Продуктивно експлуатований матеріял природи — що зовсім не являє
собою елементу вартости капіталу — земля, море, руди, ліси тощо, при
більшому напруженні тієї самої кількости робочої сили може інтенсивно
або екстенсивно більше експлуатуватись без збільшеного авансування грошового
\index{ii}{0271}  %% посилання на сторінку оригінального видання
капіталу. Таким чином, реальні елементи продуктивного капіталу
збільшуються, не потребуючи додаткового грошового капіталу. А оскільки
його треба буде на додаткові допоміжні матеріяли, то грошовий капітал,
що в ньому авансується капітальну вартість, збільшується не пропорційно
до поширення діяльности продуктивного капіталу, отже, pro
tanto зовсім не збільшується.

Ті самі засоби праці, отже, той самий основний капітал, можна використати
ефективніше так збільшуючи протяг його щоденного вживання,
як і збільшуючи інтенсивність його застосування, не витрачаючи
при цьому додаткових грошей на основний капітал. В такому разі відбувається
лише швидший оборот основного капіталу, але зате елементи
його репродукції постачатиметься швидше.

Лишаючи осторонь матеріяли природи, в процес продукції можуть
заводитись, як чинники більшої або меншої ефективности, сили
природи, що нічого не коштують. Ступінь їхньої ефективности
залежить від методів та поступу науки, що нічого не коштують капіталістові.

Це саме стосується до суспільного сполучення робочої сили в продукційному
процесі та до вмілости, надбаної поодинокими робітниками.
Кері на підставі цього вважає, що власник землі ніколи не одержує досить,
бо йому оплачується не ввесь той капітал, зглядно не всю ту
працю, що її з прадавніх часів вкладалось у землю, щоб надати їй теперішньої
родючости. (Звичайно, про ту родючість, що їй відбирається,
не згадується). Але в такому разі кожен поодинокий робітник мусив
би оплачуватись відповідно до тієї праці, яку витратив увесь рід людський,
щоб перетворити дикуна на сучасного механіка. Тут слід було б
міркувати саме навпаки: коли взяти на увагу всю вкладену в землю
неоплачену, але землевласниками й капіталістами перетворену на гроші
працю, то ввесь вкладений у землю капітал повернуто багато разів та
ще з лихварським процентом, отже суспільство давно вже й багато
разів викупило земельну власність.

Підвищення продуктивних сил праці, оскільки воно не має за передумову
додаткову витрату капітальних вартостей, підвищує, правда, насамперед
лише масу продукту, а не вартість його; останню воно підвищує
лише остільки, оскільки воно дає змогу тією самою працею репродукувати
більше сталого капіталу, отже, зберегти вартість його. Але разом з
тим підвищення продуктивних сил праці утворює новий матеріял для капіталу,
тобто базу для підвищеної акумуляції капіталу.

У першій книзі вже показано, що оскільки сама організація суспільної
праці, а тому й підвищення суспільної продуктивної сили праці потребує,
щоб продукцію провадилось у широкому маштабі, отже, щоб поодинокі
капіталісти авансували великі маси грошового капіталу, — це
стається почасти через централізацію капіталу в небагатьох руках, при
цьому немає потреби в тому, щоб розмір діющих капітальних вартостей,
а тому й розмір того грошового капіталу, що в ньому їх авансується,
абсолютно зростав. Величина поодиноких капіталів може зростати в наслідок
\index{ii}{0272}  %% посилання на сторінку оригінального видання
централізації їх в небагатьох руках, при чому суспільна сума цих
капіталів не зростає. Тут маємо лише змінний розподіл поодиноких капіталів.

Нарешті, в попередньому розділі показано, що скорочення періоду
обороту дозволяє або пускати в рух з меншим грошовим капіталом той
самий продуктивний капітал, або з тим самим грошовим капіталом —
більший продуктивний капітал.

Але все це, очевидно, не стосується власне до питання про грошовий
капітал. Це показує лише, що авансований капітал — дана сума вартости,
що в своїй вільній формі, в своїй формі вартости, складається з
певної суми грошей, — по своєму перетворенні на продуктивний капітал
має в собі продуктивні потенції, що їхні межі визначаються не величиною
його вартости, а можуть, навпаки, до певної міри діяти з різною екстенсивністю
або інтенсивністю. Коли дано ціни елементів продукції — засобів
продукції та робочої сили — то цим визначено величину грошового
капіталу, потрібного на закуп певної кількости цих елементів продукції,
наявних у вигляді товарів. Інакше кажучи, визначено величину вартости
того капіталу, що його треба авансувати. Але розміри, що в них цей
капітал діє як вартостетворець і продуктотворець, елястичні й змінні.

\emph{До другого пункту}. Само собою зрозуміло, що та частина суспільної
праці та засобів продукції, яку доводиться щорічно витрачувати
на продукцію або закуп золота, щоб замістити зужиту монету, є pro
tanto зменшення розміру суспільної продукції. Щождо грошової вартости,
яка функціонує почасти як засіб обігу, а почасти як скарб, то раз
вона вже є, скоро її здобуто, вона перебуває поряд з робочою силою,
спродукованими засобами продукції та природними джерелами багатства.
Її не можна розглядати, як щось, що обмежує все це. Перетворенням її
на елементи продукції, обміном з іншими народами, можна було б розширити
розміри продукції. Але для цього треба, щоб гроші тут, як і
раніше, відігравали ролю світових грошей.

Залежно від величини періоду обороту потрібна більша або менша
маса грошового капіталу, щоб пустити в рух продуктивний капітал. Так
само ми бачили, що поділ періоду обороту на робочий час і час циркуляції
зумовлює збільшення лятентного в грошовій формі капіталу, або
капіталу, що його застосовання відкладається.

Оскільки період обороту визначається протягом робочого періоду, остільки
його за інших незмінних умов, визначається матеріяльною природою
процесу продукції, отже, не специфічним суспільним характером
цього процесу продукції. Однак на основі капіталістичної продукції довготриваліші
широкі операції зумовлюють більші авансування грошового
капіталу на довший час. Отже, продукція в таких галузях залежить від
тих меж, що в них поодинокий капіталіст порядкує грошовим капіталом.
В цих обмеженнях пробиває вилом система кредиту і зв’язані з нею асоціяції,
прим., акційні товариства. Тому порушення на грошовому ринку
припиняють діяльність таких підприємств, тимчасом як ці самі підприємства
і собі зумовлюють порушення на грошовому ринку.
