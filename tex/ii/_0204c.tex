\index{ii}{0204}  %% посилання на сторінку оригінального видання
Капітал II.

Періоди обороту    Робочі періоди    Періоди обігу

І. Тижні    4--12    Тижні 4--6    Тижні 7--12
II. „13--21 „13--15 „16--21
ІІІ. „22--30 „22--24 „25--30
IV. „31--39 „31--33 „34--39
V. „40--48 „40--42 „43--48
VI. „49 — [57] „49--51 „[52--57]

  Капітал III.

Періоди обороту    Робочі періоди    Періоди обігу

І. Тижні    7--15    Тижні 7--9    Тижні 10--15
II. „16--24 „16--18 „19--24
III. „25--33 „25--27 „28--33
IV. „34--42 „34--36 „37--42
V. „43--51 „43--45 „46--51

Тут ми маємо точну подобу випадку І, з тією лише ріжницею, що
тепер чергуються три капітали, замість двох. Схрещування або переплітання
капіталів немає; кожен поодинокий капітал можна простежити
окремо до кінця року. Отже, тут, так само, як і в випадку І, наприкінці
робочого періоду, не постає звільнення капіталу. Капітал І, що його цілком
витрачено на кінець 3-го тижня, припливає цілком назад наприкінці 9-го
тижня і знову починає функціонувати на початку 10-го тижня. Так само і з
капіталами II і III. Правильне й повне чергування капіталів виключає
будь-яке звільнення.

Весь оборот обчисляється так:

Капітал І = 300 ф. стерл. × 5\sfrac{2}{3} = 1700 ф. стерл.

„11 = 300 „ „   × 5\sfrac{1}{3} = 1600 ф. стерл.

„111 = 300 „ „    × 5 = 1500 ф. стерл.

Ввесь капітал   900 „„    Х 5\sfrac{1}{3} = 4800 ф. стерл.

Візьмімо тепер ще один приклад, де період обігу не є точне кратне
робочому періоду. Напр., робочий період 4 тижні, період циркуляції
5 тижнів; отже, в такому разі відповідні розміри капіталу були б:
капітал І = 400 ф. стерл., капітал II = 400 ф. стерл., капітал III = 100 ф. стерл.

Капітал 1 = 400 ф. стерл.

„11 = 400 ф. стерл.

„III = 400 ф. стерл.
