
\begin{table}[H]
\centering
  \caption*{Капітал II}
  \begin{tabular}{r r@{~}c r@{~}c r@{~}c}
    \toprule
    & \multicolumn{2}{c}{Періоди обороту} & \multicolumn{2}{c}{Робочі періоди}
    & \multicolumn{2}{c}{Періоди обігу}\\
    \cmidrule(lr){2-3}
    \cmidrule(lr){4-5}
    \cmidrule(lr){6-7}

І.  & Тижні & \phantom{0}4\textendash{}12 & Тижні
    & 4\textendash{}6 & Тижні & 7\textendash{}12\\

ІІ. & \ditto{Тижні} & 13\textendash{}21 & \ditto{Тижні} 
    & 13\textendash{}15 & \ditto{Тижні} & 16\textendash{}21\\

III.& \ditto{Тижні} & 22\textendash{}30 & \ditto{Тижні}
    & 22\textendash{}24 & \ditto{Тижні} & 25\textendash{}30\\

IV. & \ditto{Тижні} & 31\textendash{}39 & \ditto{Тижні}
    & 31\textendash{}33 & \ditto{Тижні} & 34\textendash{}39\\

V.  & \ditto{Тижні} & 40\textendash{}48 & \ditto{Тижні} 
    & 40\textendash{}42 & \ditto{Тижні} & 43\textendash{}48\\

VI. & \ditto{Тижні} & \hang{r}{49}\textendash{}\hang{l}{[57]} & \ditto{Тижні}
    & 49\textendash{}51 & \ditto{Тижні} & [52\textendash{}57]\\
  \end{tabular}
\end{table}

\begin{table}[H]
\centering
  \caption*{Капітал III}
  \begin{tabular}{r r@{~}c r@{~}c r@{~}c}
    \toprule
    & \multicolumn{2}{c}{Періоди обороту} & \multicolumn{2}{c}{Робочі періоди}
    & \multicolumn{2}{c}{Періоди обігу}\\
    \cmidrule(lr){2-3}
    \cmidrule(lr){4-5}
    \cmidrule(lr){6-7}

І.  & Тижні & \phantom{0}7\textendash{}15   & Тижні 
    & 7\textendash{}9  & Тижні & 10\textendash{}15\\
ІІ. & \ditto{Тижні} & 16\textendash{}24 & \ditto{Тижні} 
    & 16\textendash{}18 & \ditto{Тижні} & 19\textendash{}24\\
III.& \ditto{Тижні} & 25\textendash{}33 & \ditto{Тижні} 
    & 25\textendash{}27 & \ditto{Тижні} & 28\textendash{}33\\
IV. & \ditto{Тижні} & 34\textendash{}42 & \ditto{Тижні} 
    & 34\textendash{}36 & \ditto{Тижні} & 37\textendash{}42\\
V.  & \ditto{Тижні} & 43\textendash{}51 & \ditto{Тижні} 
    & 43\textendash{}45 & \ditto{Тижні} & 46\textendash{}51\\
  \end{tabular}
\end{table}

\noindent{}Тут ми маємо точну подобу випадку І, з тією лише ріжницею, що
тепер чергуються три капітали, замість двох. Схрещування або переплітання
капіталів немає; кожен поодинокий капітал можна простежити
окремо до кінця року. Отже, тут, так само, як і в випадку І, наприкінці
робочого періоду, не постає звільнення капіталу. Капітал І, що його цілком
витрачено на кінець 3-го тижня, припливає цілком назад наприкінці 9-го
тижня і знову починає функціонувати на початку 10-го тижня. Так само і з
капіталами II і III.~Правильне й повне чергування капіталів виключає
будь-яке звільнення.

Весь оборот обчисляється так:
\begin{table}[H]
  \centering
  \begin{tabular}{r@{~}l@{~}l@{~}l}
  Капітал \phantom{II}І & \deq{} 300\pound{ф. стерл.} & × 5\sfrac{2}{3} 
  & \deq{} 1700\pound{ ф. стерл.} \\

  \ditto{Капітал} \phantom{I}II & \deq{} 300 \ditto{\pound{ф. стерл.}} & × 5\sfrac{1}{3} & \deq{} 1600\pound{ ф. стерл.} \\
 
  \ditto{Капітал} III & \deq{} 300\ditto{\pound{ф. стерл.}} & × 5 & \deq{} 1500\pound{ ф. стерл.} \\
  \midrule
  Ввесь капітал & \phantom{\deq{}} 900\ditto{\pound{ф. стерл.}} & × 5\sfrac{1}{3} & \deq{} 4800\pound{ ф. стерл.}\\
  \end{tabular}
\end{table}

\noindent{}Візьмімо тепер ще один приклад, де період обігу не є точне кратне
робочому періоду. Напр., робочий період 4 тижні, період циркуляції
5 тижнів; отже, в такому разі відповідні розміри капіталу були б:
капітал І \deq{} 400\pound{ ф. стерл.}, капітал II \deq{} 400\pound{ ф. стерл.}, капітал III \deq{} 100\pound{ ф. стерл.}
\begin{table}[H]
\centering
\begin{tabular}{r@{ }l}
Капітал \phantom{II}I & \deq{} 400\pound{ф. стерл.}\\
\ditto{Капітал} II & \deq{} 400\pound{ф. стерл.}\\
\ditto{Капітал} III & \deq{} 100\pound{ф. стерл.}\\
\end{tabular}
\end{table}
