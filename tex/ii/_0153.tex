\parcont{}  %% абзац починається на попередній сторінці
\index{ii}{0153}  %% посилання на сторінку оригінального видання
сталого капіталу), вважається лише за рівну вартості тих засобів існування,
що їх виплачено робітникам, і що їх вони повинні зужити, щоб
підтримувати своє існування як робочої сили. Виявити ріжницю між сталим
і змінним капіталом заважає фізіократам саме їхня доктрина. Якщо
праця продукує додаткову вартість (крім репродукції своєї власної ціни),
то вона продукує її в промисловості так само, як і в хліборобстві. А
що згідно з системою фізіократів, праця продукує додаткову вартість
тільки в одній галузі продукції, в хліборобстві, то вона постає не з праці,
а з особливої діяльности (співучасти) природи в цій галузі. І тільки
тому, на їхню думку, хліборобство є продуктивна праця протилежно до
всіх інших відмін праці.

А.~Сміт визначає засоби існування робітників, як обіговий капітал
протилежно до основного:

1) Бо він поточний капітал, у протилежність до основного, сплутує
з формами капіталу, що належать до сфери циркуляції, з капіталом
циркуляції; це сплутування, некритично ставлячись, перейняли від нього
пізніші економісти. Тому він сплутує товаровий капітал із поточною
складовою частиною продуктивного капіталу; і само собою зрозуміло,
що там, де суспільний продукт набирає форми товару, засоби існування
робітників, як і не робітників, матеріяли, як і сами засоби праці, мусять
постачатись із товарового капіталу.

2) Але й уявлення фізіократів прозирають у Сміта, хоч вони й суперечать
езотеричній — справді науковій — частині його власного викладу.

Авансований капітал взагалі перетворюється на продуктивний капітал,
тобто набирає форму елементів продукції, які й собі є продукт попередньої
праці. (Сюди належить і робоча сила). Тільки в цій формі він
може функціонувати в процесі продукції. І коли на місце самої робочої
сили, що на неї перетворилась змінна частина капіталу, підставити засоби
існування робітника, то очевидно, що ці засоби існування як такі,
щодо утворення вартости не відрізняються від інших елементів продуктивного
капіталу, від сировинних матеріялів та засобів існування робочої
худоби. Тим самим Сміт у цитованому вище місці ставить їх, за прикладом
фізіократів, на один рівень. Засоби існування сами собою не можуть
збільшити свою вартість або долучити до неї додаткову вартість. Їхня
вартість, як і вартість інших елементів продуктивного капіталу, може
знову з’явитись лише у вартості продукту. Засоби існування не можуть
прилучити до продукту більше вартости, ніж вони сами мають. Від основного
капіталу, який складається з засобів праці, вони, як і сировинний
матеріял, напівфабрикати і~\abbr{т. ін.}, відрізняються лише тим, що вони
(принаймні для капіталіста, який їх оплачує) цілком зуживаються в продукті,
в утворення якого вони входять, і, значить, вартість їхню треба
покрити цілком, тимчасом як для основного капіталу це відбувається
лише поступінно, частинами. Отже, частина продуктивного капіталу,
авансована на робочу силу (зглядно засоби існування) робітника відрізняється
тепер від інших речових елементів продуктивного капіталу лише речово, а
не своєю ролею в процесі праці та процесі зростання вартости. Вона
\parbreak{}  %% абзац продовжується на наступній сторінці
