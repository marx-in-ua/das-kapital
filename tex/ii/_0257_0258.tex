\parcont{}  %% абзац починається на попередній сторінці
\index{ii}{0257}  %% посилання на сторінку оригінального видання
велика, щоб могла вона пристосуватись до такої змінности в подовженні
і скороченні періодів обороту.

Коли далі припустимо інші незмінні умови, — а між ними незмінну
довжину, інтенсивність і продуктивність робочого дня, але змінний
розподіл новоствореної вартости між робітниками та додатковою
вартістю, так, що або перша підвищується, а друга меншає,
або, навпаки, то це не справить жодного впливу на масу грошей в
циркуляції. Така зміна може відбуватися без будь-якого збільшення або
зменшення маси грошей, що перебувають в циркуляції. Розгляньмо особливо
той випадок, коли стається загальне підвищення заробітної плати,
а тому — при вищеприпущених умовах — загальне зниження норми додаткової
вартости; при цьому — також згідно з припущенням — не відбувається
жодної зміни в вартості товарової маси, яка циркулює. В цьому
випадку, звичайно, зростає грошовий капітал, що його треба авансувати
як змінний капітал, отже, зростає маса грошей, що служить у цій
функції. Але саме настільки, наскільки зростає маса грошей, потрібних
для функції змінного капіталу, саме на стільки меншає додаткова вартість,
отже, й маса грошей, потрібних для її реалізації. На суму грошей,
потрібних для реалізації товарової вартости, це так само не справляє
жодного впливу, як і на саму цю товарову вартість. Ціна витрат\footnote*{
Про визначення терміну „ціна витрат“ (Kostenpreis або Kostpreis, як Маркс
вживає в книзі III (дивись „Капітал“, т. III, ч. І, розділ 1). \Red{Ред.}
} на
товар підвищується для поодинокого капіталіста, але його суспільна ціна
продукції\footnote*{
Про визначення терміну „ціна продукції“ (Produktionspreis) дивись „Капітал“,
т. III, ч. І, розділ дев’ятий. \Red{Ред.}
} лишається незмінна. Змінюється при цьому тільки те відношення,
що в ньому — лишаючи осторонь сталу частину вартости — ціна
продукції товарів поділяється на заробітну плату й зиск.

Але, можуть сказати, більша витрата змінного грошового капіталу
(вартість грошей, звичайно, припускається за незмінну) значить те саме,
що й збільшення грошових засобів у руках робітників. Звідси випливає
підвищення попиту на товари з боку робітників. Дальший наслідок буде
підвищення цін товарів. Або можуть сказати: коли підвищується заробітна
плата, то капіталісти підвищують ціни на свої товари. В обох випадках
загальне підвищення заробітної плати спричиняється до підвищення ціни
товарів. Тому для циркуляції товарів потрібна більша маса грошей, хоч
у який спосіб пояснюватимуть підвищення цін.

Відповідь на перше міркування: в наслідок підвищення заробітної
плати підвищиться саме попит робітників на доконечні засоби існування.
Куди менше збільшиться попит їхній на речі розкошів або постане попит
на такі речі, що раніш не ввіходили в сферу їхнього споживання. Підвищення
попиту на доконечні засоби існування, що постає раптом та у
великих розмірах, безперечно, зараз же підвищить їхню ціну. Наслідок
цього буде той, що більшу частину суспільного капіталу застосовуватиметься
на продукцію доконечних засобів існування, а меншу — на продукцію
\index{ii}{0258}  %% посилання на сторінку оригінального видання
речей розкошів, бо ці останні дешевшають в наслідок зменшення
додаткової вартости і зумовленого цим зменшення попиту капіталістів на
речі розкошів. Навпаки, оскільки робітники сами купують речі розкошів,
підвищення їхньої заробітної плати не справить — в цих межах — впливу на
підвищення ціни доконечних засобів існування, а лише змінить склад
покупців речей розкошів. Речей розкошів тепер більше йде, ніж раніш,
на споживання робітників і порівняно менш — на споживання капіталістів.
Voilà tout\footnote*{
От і все. \Red{Ред.}
}. Після деяких коливань у циркуляції буде маса товарів такої
самої вартости, як і раніш. — Щождо короткочасних коливань, то наслідок
їх буде лише той, що вільний грошовий капітал, який досі шукав собі
застосування в спекулятивних біржових підприємствах або за кордоном,
тепер надійде в циркуляцію в середині країни.

Відповідь на друге міркування: коли б капіталістичні продуценти
мали змогу з свого бажання підвищувати ціни своїх товарів, то вони
могли б робити це й робили б без усякого підвищення заробітної плати.
Заробітна плата ніколи не підвищувалась би при зниженні цін товарів.
Кляса капіталістів ніколи не ставила б опору тред-юньйонам, бо вона
завжди та за всяких умов могла б робити те, що вона в дійсності робить
тепер, як виняток, в певних особливих, сказати б, місцевих умовах:
а саме, вона могла б використовувати кожне підвищення заробітної плати
для того, щоб куди більше підвищувати ціни товарів, отже, щоб покласти
собі до кишені більший зиск.

Твердження, що капіталісти можуть підвищувати ціни речей розкошів,
бо попит на них меншає (в наслідок зменшеного попиту капіталістів, що
їхні купівельні засоби на це поменшали), це твердження було б цілком
ориґінальним застосуванням закону попиту й подання. Оскільки не постає
простої переміни покупців речей розкошів, заміни капіталістів робітниками, —
а оскільки така заміна постає, попит робітників не зумовлює підвищення
цін доконечних засобів існування, бо робітники не можуть витрачати на
доконечні засоби існування тієї частини додаткового заробітку, яку вони
витрачають на речі розкошів, — остільки ціни речей розкошів знижуються
в наслідок зменшеного попиту. У наслідок цього капітал вилучається з
продукції речей розкошів доти, доки їхнє подання зменшиться до таких
розмірів, що відповідають зміненій ролі їх в суспільному процесі продукції.
При такій скороченій продукції ціни їх, за незмінної вартости, знову
підвищуються до свого нормального рівня. Якщо відбувається таке
скорочення, або такий процес вирівнювання, то протягом його при підвищенні
цін на засоби існування у продукцію цих останніх постійно
подаватиметься стільки ж капіталів, скільки їх вилучатиметься з іншої
галузі продукції, поки насититься попит. Тоді знову постає рівновага, і
кінець цілого процесу той, що суспільний капітал, а тому й грошовий
капітал, розподіляється між продукцією доконечних засобів існування й
продукцією речей розкошів в зміненій пропорції.
