
\index{ii}{0235}  %% посилання на сторінку оригінального видання
Ріжниця випливає з неоднаковости періодів обороту, тобто тих періодів,
що в них вартість, яка заміщує змінний капітал, застосований протягом
певного часу, знову може функціонувати як капітал, отже, як новий
капітал. У \emph{В}, як і в \emph{А}, однаково заміщується вартість змінного капіталу,
застосованого протягом однакових періодів. Так само протягом однакових
періодів відбувається однаковий приріст додаткової вартости. Але хоч у
\emph{В} й заміщується що п’ять тижнів вартість в 500\pound{ ф. стерл.}, та ще й наростає
500\pound{ ф. стерл.} додаткової вартости, однак ця вартість, що являє собою
заміщення $v$, не є ще новий капітал, бо вона перебуває не в грошовій
формі. У \emph{А} не лише стару капітальну вартість заміщується новою, а її
відновлюється в її грошовій формі, отже, її заміщується як новий, здібний
функціонувати капітал.

Чи раніше, чи пізніше відбувається перетворення вартости, що являє
собою заміщення, на гроші, а тому на форму, що в ній авансується
змінний капітал, — це, очевидно, цілком байдужа обставина для самої
продукції додаткової вартости. Ця продукція залежить від величини застосованого
змінного капіталу й від ступеня експлуатації праці. Але обставина
ця модифікує величину того грошового капіталу, що його треба
авансувати, щоб протягом року пустити в рух певну кількість робочої
сили, а тому вона визначає річну норму додаткової вартости.

\subsection{Оборот змінного капіталу, розглядуваного~з~суспільного~погляду}

Погляньмо на хвилинку на справу з суспільного погляду. Припустімо,
що один робітник коштує на тиждень 1\pound{ ф. стерл.}, а робочий день \deq{} 10 годинам.
У \emph{А}, як і у \emph{В}, протягом року працюють 100 робітників (100\pound{ ф.
стерл.} на тиждень на 100 робітників становлять за 5 тижнів 500\pound{ ф. стерл.}, а
за 50 тижнів — 5000\pound{ ф. стерл.}); припустімо, що вони працюють 6 днів на
тиждень, по 60 робочих годин кожен. Отже, 100 робітників працюватимуть
протягом тижня 6000 робочих годин, а протягом 50 тижнів, \num{300.000} робочих
годин. І~\emph{А}, і \emph{В} захопили цю робочу силу, отже, суспільство не може витрачати
її на щось інше. Щодо цього, то з суспільного погляду справа
така сама в \emph{А}, як і у \emph{В}. Далі у \emph{А}, як і у \emph{В}, кожні 100 робітників одержують
на рік 5000\pound{ ф. ст.} заробітної плати (отже, всі 200 робітників
одержують разом \num{10.000}\pound{ ф. стерл.}) і беруть у суспільства засобів існування
на цю суму. І щодо цього справа з суспільного погляду така сама в
\emph{А}, як і у \emph{В}. Що робітники в обох випадках одержують заробітну плату
щотижня, то щотижня вони беруть у суспільства й засоби існування, за
які вони в обох випадках щотижня пускають у циркуляцію грошовий
еквівалент. Але відси починається ріжниця.

\so{Поперше}. Гроші, що їх пускає в циркуляцію робітник \emph{А}, є не
тільки грошова форма вартости його робочої сили, як для робітника \emph{В}
(у дійсності — засіб виплати за вже виконану роботу); починаючи з другого
періоду обороту, рахуючи з відкриття підприємства, вони вже є
\index{ii}{0236}  %% посилання на сторінку оригінального видання
грошова форма \so{новоутвореної ним самим} вартости (\deq{} ціні робочої
сили плюс додаткова вартість) першого періоду обороту, що нею
оплачується його працю протягом другого періоду обороту. У \emph{В} справа
інша. Хоч щодо робітника гроші й тут є засіб виплати за вже виконану
ним працю, але цю виконану вже працю оплачується не новоутвореною
нею вартістю, перетвореною на золото (не грошовою формою вартости,
спродукованої самою цією працею). Такий спосіб оплати може постати,
починаючи лише з другого року, коли робітника \emph{В} оплачується спродукованою
ним в минулому році вартістю, перетвореною на золото.

Що коротший період обороту капіталу, — що коротші, отже, переміжки,
що в них протягом року поновлюються терміни його репродукції, —
то швидше змінна частина капіталу, первісно авансована в грошовій формі
капіталістом, перетворюється на грошову форму тієї новоутвореної
вартости (яка, крім того, містить у собі й додаткову вартість), що її
утворив робітник на заміщення цього змінного капіталу; то коротший,
отже, час, на який капіталіст мусить авансувати гроші з свого власного
фонду, то менший, порівняно з даними розмірами маштабу продукції, той
капітал, що його він взагалі авансує; і то більша порівняно та маса додаткової
вартости, що її він за даної норми додаткової вартости видушує
протягом року, бо він то частіше може знову й знову купувати
робітника на грошову форму вартости продукту цього ж таки робітника
й пускати в рух його працю.

За даних розмірів продукції абсолютна величина авансованого змінного
грошового капіталу (як і взагалі обігового капіталу) меншає, а річна
норма додаткової вартости більшає пропорційно до скорочення періоду
обороту. За даної величини авансованого капіталу розміри продукції,
а тому за даної норми додаткової вартости й абсолютна маса додаткової
вартости, утвореної протягом одного періоду обороту, зростають разом
з підвищенням річної норми додаткової вартости, що його зумовлює
скорочення періоду репродукції. Взагалі, з нашого досліду виявилось, що
відповідно до різного протягу періоду обороту доводиться авансовувати грошовий
капітал дуже різної величини для того, щоб при тому самому
ступені експлуатації праці пускати в рух однакову масу продуктивного
обігового капіталу та однакову масу праці.

\so{Подруге} — і це має зв’язок з першою ріжницею — робітник капіталіста
\emph{В}, як і \emph{А}, платить за куповані ним засоби існування змінним
капіталом, що перетворився в його руках на засіб циркуляції. Він не
тільки, напр., бере з ринку пшеницю, а й заміщує її грошовим еквівалентом.
А що гроші, що ними робітник \emph{В} оплачує засоби свого існування,
вилучаючи їх з ринку, не є грошова форма новоутвореної вартости,
подаваної ним на ринок протягом року, як у робітника \emph{А}, то хоч
він і дає гроші продавцеві його засобів існування, але не дає він жодного
товару, — ні засобів продукції, ні засобів існування, — що їх той
міг би купити за вторговані гроші, як це, навпаки, маємо в випадку \emph{А}.
Тому з ринку береться робочу силу, засоби існування для цієї робочої
сили, основний капітал у формі засобів праці й продукційних матеріялів,
\parbreak{}  %% абзац продовжується на наступній сторінці
