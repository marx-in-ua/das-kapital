пускається, що в II теж повинна акумулюватись половина додаткової вартости,
то тут на капітал треба перетворити 188, з них на змінний капітал\footnote{
і це в неї спільне з капіталістами кляси І, — просто знижувати
заробітну плату нижче від її нормального пересічного рівня. В наслідок цього
звільняється частина грошей, які функціонують як грошова форма змін-
}/4 = 47, беручи заокруглено, 48; лишається 140, що їх треба
перетворити на сталий капітал.

Тут ми натрапляємо на нову проблему, що саме існування її мусить
здаватись чимсь дивним з того загального погляду, за яким
товари одного ґатунку звичайно обмінюються на товари іншого ґатунку,
або товари обмінюються на гроші, а ці гроші знов обмінюються на
товари іншого ґатунку. Ці 140 II m тільки тому можуть перетворитись
на продуктивний капітал, що їх заміщується частиною товарів І m на
ту саму суму вартости. Зрозуміло само собою, що частина І m, обмінювана
на II m, мусить складатися з засобів продукції, які можуть
увійти так у продукцію І, як і в продукцію II, або тільки виключно
в продукцію II. Це заміщення може статися лише через
однобічну купівлю з боку II, бо ввесь додатковий продукт 500 І m, що
його нам ще треба дослідити, повинен служити для акумуляції в межах І,
отже, його не можна обміняти на товари II; інакше кажучи, І не може
одночасно і акумулювати його й з’їдати. Отже, II мусить купити 140 І m
за готівку, при чому ці гроші не повертаються до нього через наступний
продаж його товару підрозділові І. І такий процес повторюється постійно,
при кожній новій річній продукції, оскільки вона є репродукція в поширеному
маштабі. Відки ж у II походить джерело грошей для цього?

Навпаки, II підрозділ, здається, є цілком неплідне поле для утворення
нового грошового капіталу, яке супроводить справжню акумуляцію і
зумовлює її при капіталістичній продукції, — утворення нового грошового
капіталу, що фактично спочатку виступає як просте утворення
скарбу.

Спочатку маємо 376 II v; грошовий капітал в 376, авансований на
робочу силу, через закуп товарів II постійно повертається назад до
капіталіста II як змінний капітал у грошовій формі. Це постійно повторюване
віддалення від вихідного пункту — з кишені капіталіста — і поворот
до нього ні в якому разі не збільшує кількости грошей, що
циркулюють в цьому кругобігу. Отже, воно не є джерело акумуляції
грошей; цих грошей не можна також вилучити з цієї циркуляції для
того, щоб утворити нагромаджуваний як скарб віртуально новий грошовий
капітал.

Але почекайте! Чи не можна з цього здобути якийсь маленький
баришик?

Ми не повинні забувати, що кляса II має тут перевагу проти кляси І,
що вживані нею робітники знову повинні купувати в неї товари, спродуковані
ними самими. Кляса II є покупець робочої сили і разом з тим
продавець товарів власникам застосовуваної нею робочої сили. Отже,
кляса II може: