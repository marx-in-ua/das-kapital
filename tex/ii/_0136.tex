\parcont{}  %% абзац починається на попередній сторінці
\index{ii}{0136}  %% посилання на сторінку оригінального видання
капітал протилежно до його форми, належної до процесу продукції,
тобто протилежно до форми продуктивного капіталу. Це не різні
відміни, що на них поділяє промисловий капіталіст свій капітал, а різні
форми, що їх поступінно завжди набирає й скидає та сама авансована
капітальна вартість протягом свого curriculum vitae\footnote*{
Перебіг життя. \emph{Ред.}
}. А.~Сміт сплутує це —
роблячи великий крок назад порівняно з фізіократами — з тими відмінностями
форми, що постають у межах циркуляції капітальної вартости, в її кругобігу
через ряд її послідовних форм тоді, коли капітальна вартість перебуває
в формі \emph{продуктивного} капіталу; і постають вони саме в наслідок
різних способів, що ними різні елементи продуктивного капіталу беруть
участь в процесі утворення вартости й переносять свою вартість на продукт.
Ми розглянемо далі наслідки цього основного сплутування капіталу
продуктивного й капіталу, що перебуває в сфері циркуляції (товарового
капіталу й грошового капіталу), з одного боку, і основного та поточного
капіталу, з другого. Капітальна вартість, авансована на основний капітал,
так само циркулює разом з продуктом, як і вартість, авансована на поточний
капітал, і через циркуляцію товарового капіталу перша так само перетворюється
на грошовий капітал, як і друга. Ріжниця виникає лише з того,
що вартість, авансована на основний капітал, циркулює частинами, а тому
й мусить вона також частинами, протягом довших або коротших періодів,
заміщуватись, репродукуватися в натуральній формі.

Що А.~Сміт розуміє тут під обіговим капіталом не що інше, як капітал
циркуляції, тобто капітальну вартість в її формах, належних до процесу
циркуляції (товаровий капітал і грошовий капітал), це доводить приклад,
обраний ним особливо невлучно. Він бере як приклад відміну капіталу, що
зовсім не належить до процесу продукції, а існує лише в сфері циркуляції,
складається лише з капіталу циркуляції: він бере купецький капітал.

Як безглуздо починати прикладом, де капітал взагалі фігурує не як
продуктивний капітал, він сам каже про це зараз же далі: „Капітал торговця
складається цілком з обігового капіталу“. („The capital of a merchant
is altogether a circulating capital“). Але ріжниця між обіговим і основним
капіталом постає, як нам далі скажуть, з посутніх ріжниць в середині
самого продуктивного капіталу. З одного боку, А.~Сміт має на
увазі визначену в фізіократів ріжницю, з другого боку, — відмінності форми
що їх пророблює капітальна вартість у процесі свого кругобігу. І те
й друге сплутує він в одну строкату купу.

Але як може утворюватись зиск в наслідок зміни форми грошей і
товару, в наслідок простого перетворення вартости з однієї з цих форм
на другу, це лишається цілком незрозуміло. Та й не можна зовсім цього
пояснити, бо він починає тут з купецького капіталу, що функціонує
лише в сфері циркуляції. Ми ще повернемось до цього, а покищо послухаймо,
що каже А.~Сміт про основний капітал:

„Подруге, його (капітал) можна застосовувати на поліпшення ґрунту,
на закуп корисних машин і знарядь праці та подібні речі, що дають
\parbreak{}  %% абзац продовжується на наступній сторінці
