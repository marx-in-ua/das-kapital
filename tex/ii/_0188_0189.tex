\parcont{}  %% абзац починається на попередній сторінці
\index{ii}{0188}  %% посилання на сторінку оригінального видання
гроші більшими масами. Вони припливають назад хоч швидше, хоч повільніше
— залежно від обороту капіталу, — але завжди лише частинами.
Частину їх так само постійно витрачається знову в невеликі переміжки
часу, а саме ту частину, що знову перетворюється на заробітну плату.
Але другу частину, ту, що її треба знову перетворити на сировинний
матеріял тощо, треба нагромаджувати протягом довшого часу як запасний
фонд, хоч для закупів, хоч для виплат. Отже, ця частина існує в формі
грошового капіталу, хоч змінюється розмір, що в ньому вона існує в
такій формі.

З наступного розділу ми побачимо, як інші обставини, — хоч випливають
вони з процесу продукції, хоч з процесу циркуляції, — неминуче
зумовлюють отаке перебування певної частини авансованого капіталу в
грошовій формі. Взагалі ж треба зазначити, що економісти мають великий
нахил забувати, що частина потрібного в підприємстві капіталу не лише
постійно перебігає послідовно три форми: грошового капіталу, продуктивного
капіталу й товарового капіталу, але що різні частини його постійно
перебувають одна поряд однієї в цих трьох формах, хоч відносна
величина цих частин постійно змінюється. Вони забувають саме про ту
частину, яка постійно існує як грошовий капітал, хоч саме ця обставина
дуже важлива для розуміння буржуазного господарства, а тому має значення
також і на практиці.

\section{Вплив часу обороту на величину авансованого капіталу}

В цьому та наступному шістнадцятому розділі ми досліджуємо вплив
часу обороту на самозростання вартости капіталу.

Візьмімо товаровий капітал, що є продукт робочого періоду, напр.,
де\-в’я\-тьох тижнів. Лишімо покищо осторонь частину вартости продукту,
долучену до нього в наслідок пересічного зношування основного капіталу,
а також і додаткову вартість, долучену до нього підчас продукційного
процесу; тоді вартість цього продукту дорівнюватиме вартості
авансованого на його продукцію поточного капіталу, тобто заробітної плати
й зужиткованих на його продукцію сировинних і допоміжних матеріялів.
Припустімо, що ця вартість дорівнює 900\pound{ ф. стерл.}, так що тижнева
витрата становить 100\pound{ ф. стерл}. Отже, періодичний час продукції, що
збігається тут з робочим періодом, становить 9 тижнів. При цьому байдуже,
чи припускається, що тут ідеться про робочий період для неподільного
продукту, чи про безперервний робочий період для продукту
подільного, скоро тільки кількість подільного продукту, що його воднораз
подається на ринок, коштує 9 тижнів праці. Припустімо, що час
обігу триває 3 тижні. Отже, весь період обороту триває 12 тижнів.
По 9 тижнях авансований продуктивний капітал перетворюється на товаровий
\index{ii}{0189}  %% посилання на сторінку оригінального видання
капітал, але потім він ще три тижні перебуває в періоді циркуляції.
Отже, новий період продукції може початись знову тільки на початку
13-го тижня, і продукція мала б припинитись на три тижні, або на
четверту частину цілого періоду обороту. Тут знову таки байдуже, чи
припускаємо ми, що це припинення пересічно триває доти, доки товар
буде проданий, чи припускаємо, що воно зумовлене віддаленістю ринку
або строками виплат за проданий товар. Що три місяці продукція
припиняється на три тижні, отже, протягом року вона припиняється на
$4×3 \deq{} 12$ тижнів \deq{} 3 місяцям \deq{} \sfrac{1}{4} річного періоду обороту. Тому
провадити продукцію безперервно тиждень у тиждень у тому самому
маштабі можна лише двома способами.

Або треба скоротити маштаб продукції так, щоб 900\pound{ ф. стерл.} вистачало
на те, щоб тримати роботу в русі так протягом робочого періоду,
як і протягом часу обігу першого обороту. Тоді з початком
10-го тижня відкривається другий робочий період, отже, й другий період
обороту, — відкривається раніше, ніж закінчиться перший період обороту,
бо період обороту дванадцятитижневий, а робочий період дев’ятитижневий.
900\pound{ ф. стерл.}, розподілені на 12 тижнів, дають 75\pound{ ф. стерл.} на тиждень.
Насамперед очевидно, що такий скорочений маштаб підприємства має
собі за передумову зміну розмірів основного капіталу, а значить, і взагалі
скорочення розмірів підприємства. Подруге, сумнівно, чи можна взагалі
провести таке скорочення, бо відповідно до розвитку продукції в різних
підприємствах є певний нормальний мінімум капіталовкладення, і коли
воно нижче від цього мінімуму, то підприємство не може витримати конкуренції.
Самий цей нормальний мінімум з розвитком капіталістичної
продукції теж раз-у-раз зростає і, значить, не є сталий. Але між даним
кожного разу нормальним мінімумом і дедалі більшим нормальним максимумом
є численні проміжні щаблі, — середина, що припускає дуже різні ступені
капіталовкладень. В межах цієї середини, отже, також можна провести
скорочення, що межа його є самий кожноразовий нормальний мінімум.
При затриманнях у продукції, переповненні ринку, подорожчанні сировинного
матеріялу тощо, скорочення нормальних витрат обігового капіталу,
за даної величини основного капіталу, постає через обмеження
робочого часу, через те, що роблять, приміром, тільки півдня; так само
в часи розцвіту за даної величини основного капіталу надмірне збільшення
обігового капіталу постає почасти через подовження робочого часу,
почасти через його інтенсифікацію. В підприємствах, заздалегідь розрахованих
на такі коливання, дають собі раду почасти вищезазначеними
способами, почасти одночасним уживанням більшого числа робітників,
а це сполучається з застосуванням запасного основного капіталу, напр.,
запасних паровозів на залізницях тощо. Але тут, припускаючи нормальні
відношення, ми не будемо брати на увагу таких ненормальних коливань.

Отже, тут, щоб зробити продукцію безперервною, витрату того самого
обігового капіталу розподіляється на довший час, на 12 тижнів замість 9.
Отже, в кожний даний переміжок часу функціонує вменшений продуктивний
капітал; поточна частина продуктивного капіталу зменшується
\parbreak{}  %% абзац продовжується на наступній сторінці
