
\index{ii}{0253}  %% посилання на сторінку оригінального видання
Тут ми лишаємо осторонь ту обставину, що грошової суми в 400\pound{ ф. стерл.} при десятиразовому обороті, може, буде досить для циркуляції
засобів продукції вартістю в 4000\pound{ ф. стерл.} і праці вартістю в 1000\pound{ ф.
стерл.}, а решти 100\pound{ ф. стерл.} так само буде досить для циркуляції додаткової
вартости в 1000\pound{ ф. стерл}. Це відношення грошової суми до товарової
вартості, що циркулює за її допомогою, не має ніякого чинення до
справи. Проблема лишається та сама. Коли б та сама монета не циркулювала
декілька разів, то довелось би пустити в циркуляцію 5000\pound{ ф. стерл.}
як капітал і 1000\pound{ ф. стерл.} були б потрібні для перетворення додаткової
вартости на гроші. Постає питання, відки беруться ці гроші, хоч то
1000\pound{ ф. стерл.}, хоч 100\pound{ ф. стерл}. В усякому разі вони є надлишок понад
грошовий капітал, пущений у циркуляцію.

Справді, хоч як це здається парадоксальним на перший погляд, кляса
капіталістів сама пускає в циркуляцію ті гроші, які служать для реалізації
додаткової вартости, що міститься в товарах. Але nota bene\footnote*{
Добре зауважте. \emph{Ред.}
} — кляса
капіталістів пускає їх в циркуляцію не як авансовані гроші, отже, не як
капітал. Вона витрачає їх як купівельний засіб для свого особистого
споживання. Отже, кляса капіталістів не авансує цих грошей, хоч вона
є вихідний пункт їхньої циркуляції.

\vtyagnut{}
Візьмімо поодинокого капіталіста, що починає справу, приміром,
фармера. Протягом першого року він авансує грошовий капітал, скажімо,
в 5000\pound{ ф. стерл.}, щоб оплатити засоби продукції (4000\pound{ ф. стерл.}) і робочу
силу (1000\pound{ ф. стерл.}). Норма додаткової вартости хай буде 100\%, привлащувана
ним додаткова вартість \deq{} 1000\pound{ ф. стерл}. Вищезазначені 5000\pound{ ф.
стерл.} являють собою всі гроші, що їх він авансує як грошовий капітал.
Однак ця людина мусить також жити, але до кінця року не одержить
вона жодних грошей. Її споживання становить 1000\pound{ ф. стерл}. Вона мусить
мати ці гроші. Правда, вона каже, що мусить авансувати собі ці 1000\pound{ ф. стерл.}
протягом першого року. Однак це авансування — воно має тут лише
суб’єктивне значення — сходить лише на те, що протягом першого року
вона мусить покривати своє особисте споживання з власної кишені, а не
з дармової продукції своїх робітників. Вона не авансує цих грошей як
капітал. Вона витрачає їх, платить їх як еквівалент за ті засоби існування,
що вона споживає. Цю вартість вона витрачає як гроші, подає в
циркуляцію та вилучає з неї як товарові вартості. Ці товарові вартості
вона спожила. Отже, немає тепер будь-якого відношення її до їхньої
вартости. Гроші, що ними вона заплатила за неї, існують тепер як елемент
грошей, що циркулюють. Але вартість цих грошей вона вилучила
в продуктах із циркуляції, а разом з продуктами, що ними вона жила,
знищено й їхню вартість. Вартість ця зникла. Але ось наприкінці року
ця людина пускає в циркуляцію товарову вартість в 6000\pound{ ф. стерл.} і продає її.
В наслідок цього до неї повертається: 1) авансований нею грошовий
капітал в 5000\pound{ ф. стерл.}, 2) перетворена на гроші додаткова вартість в
1000\pound{ ф. стерл}. Вона авансувала 5000\pound{ ф. стерл.} як капітал, пустила їх в
\parbreak{}  %% абзац продовжується на наступній сторінці
