праці, як споживна вартість, він цілком є продукт минулого процесу
праці; інша справа щодо вартости його. Одна частина цієї вартости є
лише вартість засобів продукції, витрачених у продукції товару, яка
знову з’являється у новій формі; вартости цієї не спродуковано підчас
процесу продукції даного товару; бо цю вартість засоби продукції
мали ще до процесу продукції, незалежно від нього; вони ввійшли
в цей процес як носії цієї вартости; поновилась та змінилась лише
форма виявлення її. Ця частина товарової вартости становить для
капіталіста еквівалент тієї частини авансованої ним сталої капітальної
вартости, яку зужитковано підчас продукції товару. Перше вона
існувала в формі засобів продукції; тепер вона існує як складова
частина вартости новоспродукованого товару. Скоро його перетворено на
гроші, цю вартість, що існує тепер в грошовій формі, треба знову перетворити
на засоби продукції, на її первісну форму, визначувану процесом
продукції та функціонуванням її в ньому. В характері вартости
товару ніщо не змінюється від того, що ця вартість функціонує як
капітал.

Друга частина вартости товару є вартість робочої сили, що її найманий
робітник продає капіталістові. Її визначається, як і вартість засобів
продукції, незалежно від того процесу продукції, куди має ввійти
робоча сила, і фіксується в акті циркуляції, купівлі й продажу робочої
сили, перше ніж вона входить у процес продукції. Своїм функціонуванням,
— витратою своєї робочої сили, — найманий робітник продукує товарову
вартість, рівну вартості, що її капіталіст має виплатити йому за
вживання його робочої сили. Він дає капіталістові цю вартість у товарі,
той виплачує її робітникові в грошах. Що ця частина товарової вартости
є для капіталіста лише еквівалент авансовуваного ним на заробітну
плату змінного капіталу, — це зовсім нічого не змінює в тому факті, що
вона є новоутворена підчас процесу продукції товарова вартість, яка
складається не з чого іншого, а з того ж, з чого складається й додаткова
вартість, а саме з минулої витрати робочої сили. Так само не впливає
на цей факт і та обставина, що вартість робочої сили, виплачувана
капіталістом робітникові у формі заробітної плати, набирає для робітника
форми доходу, і що в наслідок цього постійно репродукується не лише
робочу силу, а й клясу найманих робітників як таку, а разом з тим
репродукується й основу цілої капіталістичної продукції.

Але сума цих двох частин вартости ще не являє собою всієї товарової
вартости. Зостається надлишок понад ними обома: додаткова вартість.
Ця остання так само, як і частина вартости, що покриває авансований
на заробітну плату змінний капітал, є вартість, новоутворена робітником
підчас процесу продукції, — застигла праця. Тільки власникові
цілого продукту, капіталістові, вона нічого не коштує. Ця остання обставина
дає в дійсності змогу капіталістові цілком спожити її як дохід, якщо
тільки йому не доводиться віддавати частину її іншим спільникам, — як
земельну ренту землевласникам, — але в таких випадках ці частини становлять
дохід цих третіх осіб. Ця сама обставина була також движним
