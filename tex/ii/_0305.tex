\parcont{}  %% абзац починається на попередній сторінці
\index{ii}{0305}  %% посилання на сторінку оригінального видання

II.~Продукція засобів споживання:

Капітал\dots{} $2000с \dplus{} 500v$ \deq{} 2500.

Товаровий продукт\dots{} $2000с \dplus{} 500v \dplus{} 500m$ \deq{} 3000, що існує в
засобах споживання.

В підсумку ввесь річний товаровий продукт:

I.    $4000с \dplus{} 1000v \dplus{} 1000m$ \deq{} 6000 засобів продукції,

II.    $2000с \dplus{} 500v \dplus{} 500m$ \deq{} 3000 засобів споживання.

Вся вартість \deq{} 9000; звідси, згідно з припущенням, виключено основний
капітал, що й далі функціонує в своїй натуральній формі.

Коли ми тепер дослідимо перетворення, доконечні на основі простої
репродукції, тобто репродукції, що за неї всю додаткову вартість
споживається непродуктивно, і при цьому спочатку не звертатимемо уваги
на грошову циркуляцію, яка їх упосереднює, то матимемо насамперед
три основні точки опори.

1) $500v$, заробітна плата робітників, і $500m$, додаткова вартість капіталістів
підрозділу II, мусять витрачатись на засоби споживання. Але
їхня вартість існує в засобах споживання вартістю в 1000, що для капіталістів
підрозділу II заміщують авансовані $500v$ і репрезентують $500m$.
Отже, заробітну плату й додаткову вартість підрозділу II обмінюється
в межах підрозділу II, на продукт підрозділу II.~Разом з тим з цілого
продукту II зникає ($500v \dplus{} 500m$) II \deq{} 1000 в засобах споживання.

2) $1000v \dplus{} 1000m$ підрозділу І теж мусять витрачатись на засоби
споживання, отже, на продукт підрозділу II.~Отже, вони мусять бути обмінені
на решту цього продукту, що своїм розміром дорівнює сталій
частині капіталу 2000с. Зате підрозділ II одержує рівну суму засобів
продукції, продукт підрозділу І, що в ньому втілено вартість $1000v \dplus{}
1000m$ підрозділу І.~Разом з тим з обчислення зникають 2000 ІІс і
($1000v \dplus{} 1000m$) І.

3) Лишається ще 4000 І~$с$. Вони складаються з засобів продукції, які
можуть бути використані лише в підрозділі І, служать для заміщення
зужиткованого в ньому сталого капіталу, а тому справа з ними розв’язується
взаємним обміном між поодинокими капіталістами І, так само, як
з ($500v \dplus{} 500m$) II вона розв’язується обміном між робітниками й капіталістами
або між поодинокими капіталістами II.

Це покищо лише для того, щоб краще зрозуміти дальший виклад.
\label{original-305-1}

\subsection[Обмін між двома підрозділами]{Обмін
між двома підрозділами: І~\emph{(v \dplus{} m)} на ІІ~\emph{с}\footnotemark{}}

\label{original-305-2}
\footnotetext{
Відси знову рукопис VIII.
}

\noindent{}Ми починаємо з великого обміну між двома клясами. ($1000v \dplus{}
1000m$) І — ці вартості, що в руках їхніх продуцентів існують у натуральній
формі засобів продукції, обмінюються на 2000 ІІс, на вартості,
що існують у натуральній формі засобів споживання. У наслідок цього
\parbreak{}  %% абзац продовжується на наступній сторінці
