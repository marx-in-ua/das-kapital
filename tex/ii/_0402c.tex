\parcont{}  %% абзац починається на попередній сторінці
\index{ii}{0402}  %% посилання на сторінку оригінального видання
останньому році 732 для І і 746 для II, разом 1478. Отже, вона
зросла у відношенні 100: 134.

2) Другий приклад

Візьмім тепер річний продукт в 9000, що цілком перебуває в руках
кляси промислових капіталістів як товаровий капітал, у формі, де загальне
пересічне відношення змінного й сталого капіталу становить 1: 5.
Це має за передумову: уже значний розвиток капіталістичної продукції
й відповідний цьому розвиток продуктивної сили суспільної праці; далі
це має за передумову значне, вже раніш постале поширення маштабу
продукції, нарешті, розвиток усіх умов, що спричиняють відносне перелюднення
в робітничій клясі. Річний продукт буде тоді розподілятись по
заокругленні дробів так:

I.  5000 с + $1000 v + 1000 m$ = 7000  = 9000
II. 1430 с + $285 v + 285 m$ = 2000

Припустімо тепер, що кляса капіталістів І половину додаткової вартости
= 500 споживає, а другу половину акумулює. Тоді ($1000 v +
500 m$) I = 1500 треба було б замістити через 1500 II с. А що II с
дорівнює тут лише 1430, то 70 треба додати з додаткової вартости;
відлічуючи їх з 285 II m, маємо остачу 215 II m.

Отже, маємо:

I.  5000 с + $500 m$ (що їх треба капіталізувати) + 1500 ($v + m$) в споживному
фонді капіталістів і робітників.

II.  $1430 c + 70 m$ (іцо їх треба капіталізувати) + $285 v + 215 m$.

А що при цьому 70 II m безпосередньо долучаються до II c, то для
того, щоб пустити в рух цей додатковий сталий капітал, треба змінного
капіталу в \sfrac{70}{5} = 14; ці 14 знову береться з 215 II с; лишається 201 II m,
і ми маємо:

II. (1430с + 70с) + ($285 v + 14 v$) + $201 m$.

Обмін 1500 І ($v + \sfrac{1}{2} m$) на 1500 ІІ с є процес простої репродукції,
і тому з ним закінчено. Однак ми повинні тут зазначити ще деякі особливості,
які випливають з того, що при репродукції сполученій з акумуляцією,
І ($v + \sfrac{1}{2} m$) заміщується не самим лише 11 с, а ІІ с плюс частина
II m.

Зрозуміло само собою, що коли припускається акумуляцію, то І ($v + m$)
більше за ІІ с, а не дорівнює ІІ с, як при простій репродукції, бо
1) І заводить частину свого додаткового продукту в свій власний
продуктивний капітал і перетворює \sfrac{5}{6} цієї частини на сталий капітал,
отже, він не може разом з тим замістити ці \sfrac{5}{6} засобами споживання II;
2) І з свого додаткового продукту повинен дати матеріял для сталого
\parbreak{}  %% абзац продовжується на наступній сторінці
