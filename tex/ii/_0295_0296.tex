\parcont{}  %% абзац починається на попередній сторінці
\index{ii}{0295}  %% посилання на сторінку оригінального видання
для кожного індивідуального капіталу, розглядуваного окремо, отже, ніж
вона виступає з погляду кожного поодинокого капіталіста. Для останнього
товарова вартість розкладається на 1) сталий елемент (четвертий,
як каже Сміс) і 2) на суму заробітної плати і додаткової вартости, зглядно
заробітної плати, зиску і земельної ренти. Навпаки, з суспільного погляду,
четвертий елемент Сміса, стала капітальна вартість зникає.

\subsubsection{Резюме}

Безглузда формула, що згідно з нею три відміни доходів, заробітна
плата, зиск, рента, становлять три складові частини товарової вартости,
випливає в А.~Сміса з правдоподібнішої формули, за якою товарова
вартість resolves itself, розкладається на ці три складові частини. Однак
і це неправильно, навіть коли припустити, що товарову вартість можна
розподілити лише на еквівалент зужитої робочої сили й на утворену
нею додаткову вартість. Але й ця помилка й собі ґрунтується тут на
глибшій, правильній засаді. Капіталістична продукція ґрунтується на тому,
що продуктивний робітник продає свою власну робочу силу, як свій
товар, капіталістові, в чиїх руках вона потім функціонує просто як елемент
його продуктивного капіталу. Ця, належна до сфери циркуляції,
оборудка — продаж і купівля робочої сили — не лише є вступ до процесу
продукції, але вона й визначає implicite\footnote*{
Приховано в собі. \emph{Ред.}
} його специфічний характер.
Продукція споживної вартости і навіть продукція товару (бо її можуть
провадити і не залежні продуктивні робітники) тут є лише засіб для продукції
абсолютної та відносної додаткової вартости для капіталістів. Тому, аналізуючи
процес продукції, ми бачили, як продукцію абсолютної та відносної
додаткової вартости визначає: 1) протяг щоденного процесу праці,
2) ввесь суспільний і технічний устрій капіталістичного процесу продукції.
В ньому самому здійснюється ріжниця між простим збереженням вартости
(сталої капітальної вартости), справжньою репродукцією авансованої вартости
(еквіваленту робочої сили) і продукцією додаткової вартости, тобто
вартости, що за неї капіталіст не авансував жодного еквіваленту раніше,
ані авансує його post festum.

Хоч привласнення додаткової вартости — вартости, яка являє надлишок
над еквівалентом авансованої капіталістом вартости — підготовляється
купівлею й продажем робочої сили, однак воно є акт, що відбувається
в самому процесі продукції й становить істотний елемент його.

Вступний акт, що є акт циркуляції — купівля й продаж робочої сили — і
собі ґрунтується на розподілі елементів продукції, що відбувся перед
розподілом суспільних продуктів і був передумовою його, а саме на
відокремленні робочої сили як товару робітника від засобів продукції
як власности не-робітників.

Але разом з тим це привласнення додаткової вартости або це розмежування
\index{ii}{0296}  %% посилання на сторінку оригінального видання
продукції вартости на репродукцію авансованої вартости й
продукцію нової вартости (додаткової вартости), не заміщуваної жодним
еквівалентом, нічого не змінює в субстанції самої вартости, ні в природі
продукції вартости. Субстанція вартости є й залишається не іншим чим,
як витраченою робочою силою — працею, незалежно від особливого корисного
характеру цієї праці, — а продукція вартости є не що інше, як процес
цього витрачання робочої сили. Так кріпак витрачає протягом шістьох
день свою робочу силу, працює протягом шістьох день, і для самого
факту цього витрачання робочої сили, як такого, не постає ніякої ріжниці,
коли з цих робочих днів кріпак робить, напр., три дні на себе на
своєму власному полі, а три інші дні на свого поміщика на його полі.
Його добровільна праця на себе і його примусова праця на пана, однаково
є праця; хоч розглядатимемо ми її як працю щодо утворених нею
вартостей, хоч щодо утворених нею корисних продуктів, ми не виявимо
жодної відмінности в шостиденній праці кріпака. Відмінність стосується
лише до тих різних відносин, що зумовлюють витрачання його робочої
сили протягом обох половин шеститижневого робочого часу. Цілком так
само стоїть справа з доконечною працею і додатковою працею найманого
робітника.

Продукційний процес згасає в товарі. Те, що на його виготовлення
витрачено робочу силу, видається тепер як речова властивість товару, як
його властивість мати вартість; величину цієї вартости вимірюється
величиною витраченої праці; ні на що інше товарова вартість не розкладається
й не складається ні з чого іншого. Коли я накреслив пряму
лінію певної величини, то я спочатку „спродукував“ пряму лінію (правда,
лише символічно, і це я заздалегідь знаю) за допомогою креслення, що
його роблять згідно з певними, від мене незалежними правилами (законами).
Коли я цю лінію поділю на три відтинки (а вони знову таки
можуть відповідати певному завданню), то кожен з цих трьох відтинків
лишається, як і перше, прямою лінією, а вся лінія, що її частини вони
являють, в наслідок такого поділу не може розкластися на щось відмінне
від прямої лінії, напр., на криву якогобудь роду. Так само я не можу
поділити лінію даної довжини так, щоб сума цих частин була більша,
ніж сама неподілена лінія; отже, величину неподіленої лінії теж не можна
визначити довільно обраними величинами частин лінії. Навпаки, відносні
величини цих останніх з самого початку обмежено межами лінії, що її
частини вони являють.

З цього погляду товар, виготовлений капіталістом, ані трохи не відрізняється
від товарів, виготовлених самостійним робітником, кооперацією
робітників або рабами. Однак у нашому випадку ввесь продукт праці,
як і вся вартість його, належить капіталістові. Як і всякий інший продуцент, він повинен спочатку
через продаж перетворити товар на гроші, щоб
мати змогу дальших маніпуляцій; він мусить перетворити товар на форму
загального еквівалента.

Розгляньмо товаровий продукт до його перетворення на гроші. Він
цілком належить капіталістові. З другого боку, як корисний продукт
\parbreak{}  %% абзац продовжується на наступній сторінці
