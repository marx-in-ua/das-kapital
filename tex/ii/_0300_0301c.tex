
\index{ii}{0300}  %% посилання на сторінку оригінального видання
Шторх, який теж у принципі приймає вчення А.~Сміта, вважає однак,
що застосовання цього вчення в Сея не витримує критики. „Коли
допустити, що дохід нації дорівнює її гуртовому продуктові, тобто що з
нього не треба робити якогобудь відрахування капіталу“ (це має значити
сталого капіталу), „то доведеться також допустити, що ця нація може
непродуктивно спожити всю вартість її річного продукту, не зменшивши
ні на крихту свого майбутнього доходу\dots{} Продукти, що становлять“
(сталий) „капітал нації, не можуть споживатись“ (Storch: «Considérations
sur la nature du revenu national», Paris, 1824, p. 150).

Але Шторх забув сказати, як погодити існування цієї сталої частини
капіталу з аналізою цін, що її він узяв у Сміта, аналізою, що згідно з
нею товарова вартість містить у собі лише заробітну плату й додаткову
вартість, але не містить жодної частини сталого капіталу. Лише завдяки
Сеєві йому стає ясно, що ця аналіза ціни призводить до абсурдних
результатів, і його власне кінцеве слово про це звучить так: „неможливо
розкласти доконечну ціну на її найпростіші елементи“. („Cours d’Economie
Politique“, Petersbourg, 1815. II, p. 140).

Сісмонді, що особливо досліджував відношення між капіталом і доходом
і своє особливе розуміння цього відношення в дійсності перетворив
на differentia specifica своїх „Nouveaux Principes“, не сказав \so{жодного}
наукового слова, не додав \so{жодного} атома для висвітлення проблеми.

Бартон, Рамсай і Шербюльє роблять спроби піднестись понад Смітове
розуміння. Але це їм не вдається, бо вони з самого початку ставлять
проблему однобічно, не відмежовуючи виразно ріжниці між сталою
та змінною капітальною вартістю від ріжниці між основним капіталом та
капіталом обіговим.

Також і Джон Стюарт Мілл із звичайною повагою відтворює доктрину,
що перейшла в спадщину від А.~Сміта до його наслідувачів.

Результат: Смітова плутанина понять існує й далі до нашого часу, і
догма Смітова є ортодоксальний символ віри політичної економії.
\label{original-300-1}

\sectionextended{Проста репродукція}{\subsection{Постава питання}}

\label{original-300-2}
Коли ми розглянемо\footnote{
З рукопису II.
} річне функціонування суспільного капіталу щодо
його результату, — отже, функціонування сукупного капіталу, що в ньому
індивідуальні капітали становлять лише частини, рух яких є так їхній
індивідуальний рух, як і разом з тим складова ланка руху цілого капіталу,
— тобто, коли ми розглянемо товаровий продукт, що його дає суспільство
протягом року, то мусить виявитись, як відбувається процес
репродукції суспільного капіталу, які риси відрізняють цей процес репродукції
\index{ii}{0301}  %% посилання на сторінку оригінального видання
від процесу репродукції індивідуального капіталу, і які риси
спільні їм обом. Річний продукт охоплює так ті частини суспільного
продукту, які заміщують капітал, суспільну репродукцію, як і ті частини,
що входять у фонд споживання, що їх споживають робітники й капіталісти,
отже, охоплює так продуктивне, як і особисте споживання. Воно
охоплює також і репродукцію (тобто зберігання) кляси капіталістів і
робітничої кляси, а тому й репродукцію капіталістичного характеру сукупного
процесу продукції.

Зрозуміло, що нам треба аналізувати формулу \so{циркуляції}
\[Т' — \left\{
  \begin{array}{c}
    Г — Т\dots{} П\dots{} Т'\\
    г — т
  \end{array}
\right.,
\] при чому споживання неодмінно відіграє в ній
певну ролю; бо вихідний пункт $Т' \deq{} Т \dplus{} т$, товаровий капітал, має в
собі так сталу і змінну капітальну вартість, як і додаткову вартість.
Тому його рух охоплює й особисте, й продуктивне споживання. В кругобігах
$Г — Т\dots{} П\dots{} Т' — Г'$ і $П\dots{} Т' — Г' — Т\dots{} П$ вихідний і кінцевий
пункт є рух \so{капіталу}. Правда, це включає і споживання, бо товар,
продукт, треба продати. Але коли припускається, що цього вже досягнуто,
то для руху поодинокого капіталу буде байдуже, що далі зробиться
з цим товаром. Навпаки, в русі $Т'\dots{} Т'$ умови суспільної репродукції
виявляються саме в тому, що тут треба показати, що зробиться з кожною
частиною вартости цього сукупного продукту $Т'$. Сукупний процес
репродукції тут так само включає процес споживання, упосереднюваний
циркуляцією, як і власне процес репродукції капіталу.

Маючи на увазі мету нашу, ми повинні розглянути процес репродукції
з погляду заміщення так вартости, як і речовини поодиноких складових
частин $Т'$. Тепер нам уже не досить, як то було при аналізі вартости
продукту поодинокого капіталу, \so{припустити}, що поодинокий
капіталіст, через продаж свого товарового продукту, може спочатку перетворити
складові частини свого капіталу на гроші, а потім, знову купуючи
на товаровому ринку елементи продукції, перетворити знову ці
складові частини на продуктивний капітал. Ці елементи продукції, оскільки
вони мають речовий характер, так само становлять складову частину
суспільного капіталу, як і індивідуальний готовий продукт, обмінюваний
на них і заміщуваний ними. З другого боку, рух тієї частини суспільного
товарового продукту, що її споживає робітник, витрачаючи свою
заробітну плату, і капіталіст, витрачаючи додаткову вартість, становить не
лише складову ланку руху цілого продукту, а й переплітається з рухом
індивідуальних капіталів, і тому цього процесу не можна пояснити тим,
що його просто припускають.

Питання, що безпосередньо постає перед нами, таке: як \so{капітал},
спожитий в продукції, заміщується щодо вартости своєї з річного продукту
й як процес цього заміщення переплітається із споживанням додаткової
вартости капіталістами й заробітної плати робітниками? Отже,
насамперед йдеться про репродукцію в простому маштабі. Далі припускається
не лише те, що продукти обмінюється за їхньою вартістю, а й
\parbreak{}  %% абзац продовжується на наступній сторінці
