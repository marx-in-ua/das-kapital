\parcont{}  %% абзац починається на попередній сторінці

\index{ii}{0003}  %% посилання на сторінку оригінального видання
\chapter{Метаморфози капіталу та їхній кругобіг}

\section{Кругобіг грошового капіталу}


\label{original-3}
\vspace{-\medskipamount}
Процес кругобігу\footnote{
З рукопису II.
} капіталу відбувається в трьох стадіях, що, як
це викладено в першому томі\footnote*{
Див. відділ сьомий, вступ. \emph{Ред.}
}, утворюють такий ряд:

\emph{Перша стадія}: Капіталіст з’являється на товаровому ринку й на
ринку праці як покупець; його гроші перетворюються на товар, або пророблюють
акт циркуляції $Г — Т$.

\emph{Друга стадія}: Продуктивне споживання куплених товарів капіталістом.
Він діє як капіталістичний товаропродуцент; його капітал пророблює
процес продукції. Наслідок цього — товар більшої вартости, ніж
вартість елементів його продукції.

\emph{Третя стадія}: Капіталіст повертається на ринок як продавець; його
товар перетворюється на гроші або пророблює акт циркуляції $Т — Г$.

Отже, формула кругобігу грошового капіталу така: $Г — Т\dots{} П\dots{} Т' —
Г'$, де крапки позначають, що процес циркуляції перервався, а $Т'$, а
також $Г'$, позначають $Т$ і $Г$, збільшені додатковою вартістю.

У першому томі першу й другу стадію досліджувалось лише остільки,
оскільки це треба, щоб зрозуміти другу стадію — процес продукції капіталу\footnote*{
Див. „Капітал“, т. І, розділ IV. \emph{Ред.}
}.
Тому не зверталось там уваги на ті різні форми, що в них убирається
капітал на різних своїх стадіях і що їх він то набирає, то скидає,
повторюючи кругобіг. Тепер вони являють ближчий предмет дослідження.
Щоб зрозуміти ці форми в чистому вигляді, треба насамперед абстрагуватись
від усіх моментів, що не мають нічого спільного з зміною та
утворенням форм як такими. Тому припускається тут не тільки, що товари
продається за їхньою вартістю, але й те, що це відбувається за
незмінних обставин. Отже, тут також не звертається уваги на ті зміни
вартости, що можуть постати протягом процесу кругобігу.
