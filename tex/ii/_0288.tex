\parcont{}  %% абзац починається на попередній сторінці
\index{ii}{0288}  %% посилання на сторінку оригінального видання
визначено, а саме — вона дорівнює цілій вартості продукту мінус та частина вартости його, яка є
еквівалент заробітної плати; отже, вона дорівнює надлишкові вартости, створеної підчас виготовлення
товару, понад ту частину вміщеної в продукті вартости, яка дорівнює еквівалентові заробітної плати.

2) Те, що має силу для товару, спродукованого в поодинокому промисловому підприємстві кожним
поодиноким робітником, має силу й для річного продукту всіх галузей підприємств у цілому. Те, що має
силу для денної праці поодинокого продуктивного робітника, має силу й для річної праці, пущеної в
рух цілою продуктивною робітничою клясою. Вона „фіксує“ (Смітів вислів) в річному продукті сукупну
вартість, визначувану кількістю праці, витраченої протягом року, і ця сукупна вартість розкладається
на дві частини — одну, визначувану тією кількістю річної праці, що нею робітнича кляса утворює
еквівалент своєї річної заробітної плати, фактично саму цю заробітну плату, — і другу частину,
визначувану додатковою річною працею, що нею робітник утворює додаткову вартість для кляси
капіталістів. Отже, новоспродукована річна вартість, що міститься в річному продукті, складається
лише з двох елементів: з еквіваленту річної заробітної плати, одержаної робітничою клясою, і
річної додаткової вартости, яку подається клясі капіталістів. Але річна заробітна плата становить
дохід робітничої кляси, річна сума додаткової вартости — дохід кляси капіталістів; отже, обидві вони
репрезентують (і цей погляд правильний, коли описують просту репродукцію) відносні пайки в річному
фонді споживання і в ньому реалізуються. Таким чином ніде не лишається місця для сталої капітальної
вартости, для репродукції капіталу, діющого в формі засобів продукції. Але у вступі до своєї праці
А.~Сміт виразно каже, що всі частини товарової вартости, які функціонують як дохід, збігаються з
річним продуктом праці, призначеним для суспільного споживного фонду: „З’ясувати, з чого взагалі
складався дохід народу, або яка природа фонду, який\dots{} давав (supplied) йому споживання протягом
року — ось мета цих перших чотирьох книг“ (стор. 12). І в першому реченні вступу сказано: „річна
праця кожної нації є той фонд, що первісно дає усі засоби існування, які вона споживає протягом
року, і які завжди складаються або з безпосереднього продукту цієї праці, або з предметів, купованих
на цей продукт в інших націй“ (стор. 11).

Перша помилка А.~Сміта в тому, що він ставить на один рівень вартість річного продукту і
новоспродуковану річну вартість. Остання є лише продукт праці минулого року; перша має в собі, крім
того, всі ті елементи вартости, що їх зужитковано на виготовлення річного продукту, але спродуковано
в попередньому році, а почасти і в ще давніш минулі роки: засоби продукції, що їхня вартість лише
знову з’являється і що не були — щодо їхніх вартостей — ні продуковані, ні репродуковані працею,
витраченою протягом останнього року. За допомогою цього сплутування різних понять А.~Сміт
спекався сталої частини вартости річного продукту. Саме сплутування ґрунтується на другій помилці в
його основному погляді: він не відрізняє
\parbreak{}  %% абзац продовжується на наступній сторінці
