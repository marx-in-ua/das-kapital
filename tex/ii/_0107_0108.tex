\parcont{}  %% абзац починається на попередній сторінці
\index{ii}{0107}  %% посилання на сторінку оригінального видання
наприклад, коли час обороту $о$ становить три місяці, то $n \deq{} \sfrac{12}{3} \deq{} 4$; капітал
робить чотири обороти на рік, або обертається чотири рази.
Коли $о \deq{} 18$ місяцям, то $n \deq{} \sfrac{12}{18} \deq{} \sfrac{2}{3}$ або капітал протягом року проходить
лише \sfrac{2}{3} часу свого обороту. Коли час обороту його дорівнює кільком
рокам, то він, отже, обчислюється одним роком, повтореним кілька
разів.

Для капіталіста час обороту його капіталу є час, що протягом його він
мусить авансувати свій капітал для того, щоб він збільшився вартістю
й повернувся в своїй первісній формі.

Перш ніж перейти до ближчого розгляду того впливу, що його
оборот справляє на процес продукції та процес зростання вартости, треба
розглянути дві нові форми, що підступають до капіталу із процесу циркуляції
та впливають на форму його обороту.

\sectionextended{Основний капітал і обіговий капітал}{\subsection{Відмінності форми}}

В книзі І, розділ VI, ми бачили, що частина сталого капіталу зберігає
ту певну споживну форму, що в ній вона увіходить у процес
продукції, проти тих продуктів, що їх утворенню вона сприяє. Отже,
протягом більш або менш довгого періоду, в постійно повторюваних
процесах праці, вона завжди виконує ті самі функції. Такі, напр., майстерні,
машини і~\abbr{т. ін.}, коротко кажучи — все те, що ми об’єднуємо
під назвою \emph{засоби праці}. Ця частина сталого капіталу віддає свою
вартість продуктові, в міру того, як вона разом з своєю споживною
вартістю втрачає свою мінову вартість. Цю передачу вартости або перехід
вартости таких засобів продукції на продукт, що в утворенні його
вони беруть участь, визначається за пересічним обчисленням; її вимірюється
пересічним протягом функціонування засобів продукції, від
того моменту, коли вони ввіходять в процес продукції, і до моменту
коли вони цілком зносяться, знищаться, коли їх треба буде замінити на
нові екземпляри такого ж роду, або репродукувати.

Отже, своєрідність цієї частини сталого капіталу — власне засобів
праці — ось у чому:

Частину капіталу авансується в формі сталого капіталу, тобто в формі
засобів продукції, що функціонують як чинники процесу праці, поки
зберігають ту самостійну споживну форму, що в ній вони ввіходять
у процес праці. Готовий продукт, а значить і продуктотворчі елементи,
оскільки їх перетворено на продукт, виштовхується з продукційного
процесу, щоб перейшли вони як товар з сфери продукції до сфери циркуляції.
\index{ii}{0108}  %% посилання на сторінку оригінального видання
Навпаки, засоби праці, ввійшовши в сферу продукції, вже ніколи
не облишають її. Їх міцно прив’язує до неї їхня функція. Частину
авансованої капітальної вартости \emph{фіксується} в цій формі, визначуваній
функцією засобів праці в продукційному процесі. В міру функціонування,
а тому і в міру зношування засобів праці частина їхньої вартости
переходить на продукт, а друга лишається зафіксована в засобах
праці, а значить, і в продукційному процесі. Фіксована таким чином
вартість завжди меншає, поки засоби праці не відслужили свого часу;
тому вартість їхня протягом більш або менш довгого періоду розподіляється
на масу продуктів, що виходять з ряду постійно повторюваних
процесів праці. Але поки засоби праці все ще діють як засоби праці,
тобто, поки їх не треба заміняти на нові екземпляри такого самого роду,
вартість сталого капіталу ввесь час лишається фіксована в них, тимчасом
як друга частина первісно фіксованої в них вартости переходить на
продукт, а тому циркулює, як складова частина товарового запасу. Що
триваліші засоби праці, що повільніше вони зношуються, то довший час
вартість сталого капіталу лишається фіксована в цій споживній формі.
Але хоч яка буде тривалість засобів праці, пропорція, що в ній вони
передають свою вартість, завжди стоїть у зворотному відношенні до
загального часу функціонування їх. Коли з двох машин однакової
вартости одна зношується протягом п’ятьох років, а друга протягом десятьох,
то за однаковий час перша віддає вдвоє більше вартости, ніж
друга.

Ця частина капітальної вартости, фіксована в засобах праці, циркулює
так само, як і всяка інша частина. Ми взагалі бачили, що вся капітальна
вартість перебуває в постійній циркуляції, і тому в цьому розумінні
ввесь капітал є обіговий капітал. Але циркуляція розглядуваної
тут частини капіталу своєрідна. Поперше, вона циркулює не в своїй
споживній формі, але циркулює лише її вартість, і до того лише поступінно,
частинами, в міру того, як вона переходить на продукт, що циркулює
як товар. Протягом усього часу, коли функціонує ця частина капіталу,
деяка частка її вартости лишається фіксована в ній як самостійна
проти товарів, що продукуванню їх вона допомагає. В наслідок цієї
особливости ця частина сталого капіталу набирає форми \emph{основного
капіталу}. Протилежно до нього, всі інші речові складові частини капіталу,
авансованого на продукційний процес, становлять, навпаки, \emph{обіговий}
або \emph{поточний} капітал.

\vtyagnut{}
Частина засобів продукції, — саме такі допоміжні матеріяли, що їх
споживають самі засоби праці підчас свого функціонування, як от вугілля
в паровій машині, або такі, що лише допомагають процесові,
напр., світильний газ тощо, — речово не ввіходить у продукт. Тільки її
вартість становить частину вартости продукту. В своїй власній циркуляції
продукт несе в циркуляцію і вартість таких засобів продукції. Це в
них спільне з основним капіталом. Але в кожному процесі праці, куди
вони ввіходять, їх зужитковується цілком, і тому треба для кожного нового
процесу праці замінити їх цілком на нові екземпляри того самого
\parbreak{}  %% абзац продовжується на наступній сторінці
