\parcont{}  %% абзац починається на попередній сторінці
\index{ii}{0208}  %% посилання на сторінку оригінального видання
завжди звільнятися 300\pound{ ф. стерл}. Навпаки, коли щотижня витрачається
300\pound{ ф. стерл.}, то для робочого періоду ми маємо 1800\pound{ ф. стерл.}, для періоду
циркуляції 900\pound{ ф. стерл.}; отже, періодично звільнятиметься вже 900\pound{ ф. стерл.}
замість 300\pound{ ф. стерл}.

D.~Ввесь капітал, напр., в 900\pound{ ф. стерл.}, треба поділити на дві частини,
які раніш: 600\pound{ ф. стерл.} для робочого періоду і 300\pound{ ф. стерл.} для періоду
циркуляції. Частина, дійсно витрачувана на процес праці, зменшиться в
наслідок цього на одну третину, з 900 до 600\pound{ ф. стерл.}, і тому розмір продукції
зменшиться на одну третину. З другого боку, 300\pound{ ф. стерл.}
функціонують лише для того, щоб зробити робочий період безперервним,
так, щоб на процес праці щотижня протягом року можна було витрачати
по 100\pound{ ф. стерл}.

Беручи абстрактно, цілком байдуже, чи роблять 600\pound{ ф. стерл.} протягом
6 × 8 \deq{} 48 тижнів (продукт \deq{} 4800\pound{ ф. стерл.}), чи весь капітал в 900\pound{ ф. стерл.}
витрачається на процес праці протягом 6 тижнів, а потім протягом
3 тижнів періоду циркуляції він лежить без діла; в останньому випадку
він працював би на протязі 48 тижнів $6 × 5\sfrac{1}{3} \deq{} 32$ тижні (продукт \deq{}
900 × 5\sfrac{1}{3} \deq{} 4800\pound{ ф. стерл.}) і 16 тижнів лежав би без діла. Але, не кажучи
вже про більше псування основного капіталу протягом 16 тижнів, коли
він лишається бездіяльний, та подорожчання праці, що її доведеться оплатити
за ввесь рік, хоч вона діє лише протягом частини його, така реґулярна
перерва продукційного процесу взагалі несполучна з продукцією
сучасної великої промисловости. Сама ця безперервність є продуктивна
сила праці.

Коли ми тепер ближче придивимось до звільненого капіталу, в дійсності
до капіталу, що його дію припинено, то виявиться, що чимала
частина його завжди мусить мати форму грошового капіталу. Зупинімось
на прикладі: робочий період 6 тижнів, період циркуляції 3 тижні, щотижнева
витрата 100\pound{ ф. стерл}. Посередині другого робочого періоду,
наприкінці 9-го тижня, припливають назад 600\pound{ ф. стерл.}, що з них протягом
решти робочого періоду треба витратити лише 300\pound{ ф. стерл}.
Отже, наприкінці другого робочого періоду з цієї суми звільняться
300\pound{ ф. стерл}. В якому стані перебувають ці 300\pound{ ф. стерл.}? Припустімо,
що \sfrac{1}{3} треба витратити на заробітну плату, \sfrac{2}{3} на сировинні та допоміжні
матеріяли. Отже, з 600\pound{ ф. стерл.}, що приплили назад, 200\pound{ ф. стерл.}, призначені
на заробітну плату, перебувають у грошовій формі, а 400\pound{ ф.
стерл.} — у формі продуктивного запасу, у формі елементів поточної
частини сталого продуктивного капіталу. А що для другої половини
робочого періоду II, треба лише половини цього продуктивного запасу,
то друга половина його протяюм 3 тижнів перебуває в формі надлишкового
продуктивного запасу, тобто запасу, що перевищує потреби одного
робочого періоду. Але капіталіст знає, що з цієї частини (\deq{} 400\pound{ ф. стерл.})
приплилого капіталу для поточного робочого періоду потрібна тільки
половина (\deq{} 200\pound{ ф. стерл.}). Отже, від ринкових умов залежатиме, чи
перетворить він знову ці 200\pound{ ф. стерл.} одразу цілком або тільки почасти на
надлишковий продуктивний запас, чи, вичікуючи сприятливих ринкових умов
\parbreak{}  %% абзац продовжується на наступній сторінці
