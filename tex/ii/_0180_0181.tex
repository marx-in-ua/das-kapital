\parcont{}  %% абзац починається на попередній сторінці
\index{ii}{0180}  %% посилання на сторінку оригінального видання
що потрібна певна більша або менша кількість потенціяльного продуктивного капіталу, тобто певна
кількість призначених для продукції засобів продукції, що їх треба мати у запасі в більших або
менших масах, щоб могли вони помалу входити в процес продукції. При цьому ми зазначили, що в даному
підприємстві або в капіталістичному підприємстві даних розмірів величина цього продукційного запасу
залежить від більшої або меншої важкості його поновлення, від відносної близькости ринків набування
його, розвитку засобів транспорту й комунікації і~\abbr{т. ін.} Всі ці обставини впливають на мінімум
капіталу, що мусить бути наявний в формі продуктивного запасу, отже, і на протяг часу, що на нього
треба авансувати капітал, і на розмір капіталу, що його треба авансувати одним заходом. Цей розмір,
що впливає, отже, на оборот, зумовлено більшим або меншим протягом часу, що на нього закріплюється
обіговий капітал в формі продуктивного запасу, як лише потенціяльний продуктивний капітал. З другого
боку, оскільки це закріплення залежить від більшої або меншої можливости швидкого заміщення, від
ринкових умов тощо, воно саме і собі зумовлюється часом обігу, обставинами, що належать до сфери
циркуляції. „Далі всі предмети реманенту або прилади, як ручний струмент, решета, кошівниці,
вірьовки, дьоготь, гвіздки тощо, на випадок негайного заміщення мають бути в запасі то більшому, що
менша змога швидко дістати їх поблизу. Нарешті, щорічно протягом зими ввесь реманент треба пильно
переглянути й подбати про те, щоб його відповідно поповнити й полагодити. Оскільки великі мають бути
взагалі запаси щодо реманенту, це залежить, головним чином, від місцевих умов. Там, де близько немає
ремісників і крамниць, треба мати більший запас, ніж там, де вони є на місці або близько. А коли при
інших однакових умовах потрібні запаси закуповується разом чималими масами, то звичайно мають ту
перевагу, що купують дешевше, особливо, коли для цього обирають влучний час; правда, при цьому з
обігового капіталу підприємства воднораз береться чималу суму, а без неї не завжди може обійтись
господарство“ (Kirchhof, p. 301).

Ріжниця між часом продукції і робочим часом може, як ми бачили, поставати в дуже різних випадках.
Обіговий капітал може бути в періоді продукції раніш, ніж він увійде в процес праці у власному
значенні слова (виробництво копил); або він перебуває в періоді продукції після того як проробив
власне процес праці (вино, засівне зерно), або час продукції деколи переривається робочим часом
(хліборобство, лісівництво); чимала частина обігоздатного продукту лишається втіленою в
продукційному процесі, тимчасом як куди менша частина ввіходить у річну циркуляцію (лісівництво й
скотарство); довший або коротший час, що на нього треба витратити обіговий капітал в формі
потенціяльного продуктивного капіталу, отже, більша або менша маса капіталу, що його треба витратити
воднораз — це зумовлюється почасти родом продукційного процесу (хліборобство), а почасти залежить
від близькости ринків і~\abbr{т. ін.}, коротко кажучи, від обставин, які належать до сфери циркуляції.

Далі (книга III) ми бачимо, до яких безглуздих теорій призвела Мак"=Куллоха,
\index{ii}{0181}  %% посилання на сторінку оригінального видання
Джемса Мілла та інших спроба ототожнити час продукції, що відхиляється від робочого часу, з
цим останнім, — спроба, що сама й собі походить від неправильного застосування теорії вартости.

\pfbreak

Цикл обороту, що ми розглянули вище, визначається тривалістю основного капіталу, авансованого на
процес продукції. А що цей цикл охоплює більший або менший ряд років, то охоплює він і ряд річних,
тобто повторюваних протягом кожного року оборотів основного капіталу.

В хліборобстві такий цикл обороту зумовлюється системою сівозміни. Протяг оренди в усякому разі не
повинен бути менший, ніж час обороту при заведеній сівозміні, тому при трипільному господарстві його
завжди беруть у 3, 6, 9 років. Коли заведено трипільне господарство з чистим паром, то кожне поле
протягом шости років обробляється тільки чотири рази, при цьому в ті роки, як його обробляється, на
ньому сіють озимину або ярину і, коли того потребує або дозволяє властивість ґрунту, послідовно —
пшеницю і жито, ячмінь і овес. Кожна відміна зернівців дає на тому самому ґрунті більші або менші
врожаї, ніж інші відміни, кожна має свою вартість і продається за свою ціну. Тому, коли прибуток з
поля змінюється кожного року обробки, то й за першу половину обороту (за перші три роки) він буде не
той, що за другу. Навіть пересічний прибуток за першу й другу половину часу обороту буде
неоднаковий, бо родючість залежить не лише від якости ґрунту, а також і від погоди, так само, як і
ціни залежать від багатьох умов. Коли ми обчислимо прибуток з поля, беручи на увагу середню
родючість і пересічні ціни за ввесь шестилітній період часу обороту, то знайдемо загальну цифру
щорічного прибутку і для першого й для другого періоду часу обороту. Цього однак не буде, коли ми
обчислимо прибуток лише за половину часу обороту, тобто тільки за три роки, бо тоді загальні цифри
прибутку не будуть однакові. Відси випливає, що при трипільній системі протяг оренди треба визначити
принаймні в шість років. Але куди бажаніше завжди орендареві й землевласникові, щоб час оренди
становив кількаразовий час оренди (sic!), отже, при трипільній системі замість 6 років — 12, 18 і
більш років, а при семипільній замість 7--14, 28 років“. (Kirchhof, S. 117, 118).

(В рукопису тут стоїть: „Англійське сівозмінне господарство. Тут зробити примітку“).

\section{Час обігу}

Всі досі розглянуті обставини, що зумовлюють ріжниці в періодах обігу\footnote{
Тут, очевидно, термін „період обігу“ („Umlaufsperiode“) вжито в широкому розумінні слова — як
період, що охоплює час продукції та час власне обігу, тобто в розумінні періода обороту капіталу.
\Red{Ред.}
} різних капіталів, вкладених у
різні галузі підприємств, а тому
\parbreak{}  %% абзац продовжується на наступній сторінці
