\index{ii}{0245}  %% посилання на сторінку оригінального видання
Можливі лише два нормальні випадки репродукції, якщо залишити
осторонь ті порушення, що перешкоджають навіть репродукції в попередньому
маштабі.

Або відбувається репродукція в простому маштабі.

Або відбувається капіталізація додаткової вартости, акумуляція.

І. Проста репродукція

При простій репродукції додаткова вартість, продукована й реалізовувана
щорічно або — при кількох оборотах — періодично протягом року,
споживається особисто, тобто непродуктивно, її власником, капіталістом.

Та обставина, що вартість продукту складається почасти з додаткової
вартости, почасти з тієї частини вартости, яка складається з репродукованого
в ньому змінного капіталу плюс зужиткований на його продукцію
сталий капітал, — ця обставина абсолютно нічого не змінює ні в кількості,
ні в вартості цілого продукту, що постійно надходить в циркуляцію, як
товаровий капітал, і так само постійно вилучається з неї для продуктивного
або особистого споживання, тобто для того, щоб служити засобом
продукції або засобом споживання. Якщо сталий капітал залишити осторонь,
то ця обставина впливає тільки на розподіл річного продукту між
робітниками й капіталістами.

Тому, навіть коли припустити просту репродукцію, частина додаткової
вартости має постійно перебувати в формі грошей, а не в формі
продукту, бо інакше її не можна перетворити з грошей на продукт для
споживання. Це перетворення додаткової вартости з її первісної товарової
форми на гроші треба тут дослідити далі. Для спрощення справи
візьмімо проблему в її найпростішій формі, а саме припустімо циркуляцію
виключно металевих грошей, грошей, що являють дійсний грошовий еквівалент.
Згідно з законами простої товарової циркуляції (кн. І, розд. III), маси
наявних у країні металевих грошей має вистачити не лише для
циркуляції товарів. Її має вистачити для того, щоб вирівнювати коливання
грошового обігу, що випливають почасти з флюктуацій\footnote*{
Флюктуація — від лат. слова „fluctus“, гра хвиль, хвилювання, почережне
піднесення й спад. Ред.
} в швидкості
циркуляції, почасти з змін товарових цін, почасти з різних та
змінних відношень, що в них функціонують гроші як засіб виплати або
як власне засіб циркуляції. Відношення, що в ньому наявна маса грошей
розподіляється на скарб і на гроші в циркуляції, раз-у-раз змінюється, але
маса грошей завжди дорівнює сумі грошей, наявних у формі скарбу та
в формі грошей в циркуляції. Ця маса грошей (маса благородного металю)
є поступінно нагромаджуваний скарб суспільства. Оскільки частина цього
скарбу зужитковується через зношування, її треба щорічно знову заміщувати,
як і всякий інший продукт. Це в дійсності і відбувається через
\parbreak{}  %% абзац продовжується на наступній сторінці
