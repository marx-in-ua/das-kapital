
\index{ii}{0325}  %% посилання на сторінку оригінального видання
Справжній перебіг справи затемнюється двома обставинами:

1) Поява торговельного капіталу (що його першою формою
завжди є гроші, бо торговець як такий не виготовляє жодного
„продукту“ або „товару“) і грошового капіталу, як предмету
операцій особливої відміни капіталістів, у процесі циркуляції промислового
капіталу.

2) Розпад додаткової вартости — а вона насамперед завжди мусить
потрапляти до рук промислового капіталіста — на різні категорії, що їхніми
представниками, поряд промислового капіталіста, виступають землевласник
(для земельної ренти), лихвар (для проценту) і~\abbr{т. ін.}, потім уряд з своїми
урядовцями, рантьє і~\abbr{т. ін.} Ці молодці виступають проти промислового
капіталіста як покупці й остільки як перетворювачі його товарів на гроші;
pro parte\footnote*{
Відповідно до участи своєї. \emph{Ред.}
} і вони подають „гроші“ в циркуляцію, а капіталіст одержує
ці гроші від них. При цьому завжди забувають, з якого джерела вони
первісно одержали й знову та знову одержують гроші.
\label{original-325-1}

\subsection[Сталий капітал підрозділу I]{Сталий капітал підрозділу I\footnotemark{}}

\label{original-325-2}
Нам%
\footnotetext{Відси з рукопису II. \emph{Ф.~Е.}}
лишається ще дослідити сталий капітал підрозділу І \deq{} 4000 І~$с$.
Ця вартість дорівнює вартості зужиткованих на продукцію цієї товарової
маси засобів продукції, що знову з’являється в товаровому продукті І.~Ця новоз’явлена вартість, спродукована не в продукційному процесі І, а
яка на рік раніше ввійшла в нього як стала вартість, як дана вартість
його засобів продукції, існує тепер у формі всієї частини товарової маси І,
що не ввібрана категорією II; вартість цієї товарової маси, що лишається
таким чином у руках капіталістів І, \deq{} \sfrac{2}{3} вартости всього їхнього річного
товарового продукту. Про поодинокого капіталіста, що продукує певну
особливу відміну засобів продукції, ми могли б сказати: він продає свій
товаровий продукт, перетворює його на гроші. Перетворюючи його на
гроші, він зворотно перетворює на гроші й сталу частину вартости свого
продукту. На цю частину вартости, перетворену на гроші, він потім знову
купує собі в інших продавців товарів засоби продукції, або перетворює
сталу частину вартости свого продукту на ту натуральну форму, що
в ній вона може знову функціонувати як продуктивний сталий капітал.
Навпаки, в нашому випадку таке припущення неможливе. Кляса капіталістів
І охоплює всю сукупність капіталістів, що продукують засоби продукції.
Крім того, товаровий продукт в 4000, що лишився в їхніх руках,
є та частина суспільного продукту, що її не можна обміняти на жодну
іншу, бо вже не існує жодної такої іншої частини річного продукту. За
винятком цих 4000 увесь лишок уже приміщено; частину ввібрано суспільним
споживним фондом, а друга частина повинна замістити сталий
капітал підрозділу II, що обміняв уже все, що було в його розпорядженні
для обміну з підрозділом І.
\parbreak{}  %% абзац продовжується на наступній сторінці
