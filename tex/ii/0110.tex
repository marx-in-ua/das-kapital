бить його основним капіталом. Навпаки, коли він ще тільки сам виходить
з процесу, він зовсім не є основний капітал. Напр., машина, як продукт,
зглядно товар фабриканта-машинобудівника належить до його
товарового капіталу. Основним капіталом вона стає лише в руках покупця,
капіталіста, що продуктивно її вживає.

Припускаючи всі інші умови за однакові, ступінь зв’язаности основного
капіталу зростає разом із тривалістю засобів праці. Саме від цієї
тривалости залежить величина ріжниці між капітальною вартістю, фіксованою
в засобах праці, і тією частиною капітальної вартости, яку вона
в повторюваних процесах праці віддає продуктові. Що повільніше відбувається
ця передача вартости, — а вартість передається з засобів праці
при всякому повторенні того самого процесу праці, — то більший фіксований
капітал, то більша ріжниця між капіталом, застосованим у продукційному
процесі, і капіталом, що в ньому зужитковується. Скоро ця
ріжниця зникає, це значить, що засіб праці віджив свій час і разом із
своєю споживною вартістю втратив свою вартість. Він перестав бути
носієм вартости. Через те, що засіб праці, як і кожний інший речовий
носій сталого капіталу, віддає свою вартість продуктові лише в тих розмірах,
в яких разом із споживною вартістю він втрачає і вартість, то
очевидно, що як повільніше втрачає він свою споживну вартість, що
довше він перебуває в продукційному процесі, то й довший буде період,
протягом якого вартість сталого капіталу лишається в ньому фіксована.

Коли засіб продукції, що не є засіб праці у власному розумінні,
напр., допоміжний матеріял, сировинний матеріял, напівфабрикат тощо,
перенесенням своєї вартости, а тому й способом циркуляції своєї вартости
відіграє таку саму ролю як засоби праці, то він так само є речовий
носій, форма існування основного капіталу. Так буває при вищезгаданих
земельних меліораціях, коли в ґрунт додається хемічні складові
частини, що їхнє діяння поширюється на багато продукційних періодів
або років. Тут частина вартости і далі існує поряд продукту в
своїй самостійній формі або в формі основного капіталу, тимчасом як друга
частина вартости передається на продукт, а тому разом з ним циркулює.
В цьому разі в продукт входить не лише частина вартости основного капіталу,
а й споживна вартість, та субстанція, що в ній існує ця частина вартости.

Лишаючи осторонь основну помилку — сплутування категорій: основний
і обіговий капітал з категоріями: сталий і змінний капітал — плутанина
в дотеперішньому визначенні понять в економістів ґрунтується насамперед
на таких пунктах.

Певні властивості, речово належні засобам праці, вони перетворюють
на безпосередні властивості основного капіталу, напр., таку, як фізична
нерухомість хоча б будинку. Але завжди легко довести, що інші засоби
праці, що, як такі, теж є основний капітал, мають протилежні властивості,
напр., фізична рухомість хоча б корабля.

Але економічну визначеність форми, що походить з циркуляції вартости,
вони сплутують з речовою властивістю; ніби речі, які самі собою
взагалі не є капітал, а робляться ним лише в певних суспільних відно-
