\parcont{}  %% абзац починається на попередній сторінці
\index{ii}{0355}  %% посилання на сторінку оригінального видання
на боці капіталістів. При розгляді цієї фундаментальної форми ми все це
лишаємо осторонь.

b) Ми припускаємо, що одного разу І авансує на закуп у II дальші
400\pound{ ф. стерл.} грішми, що припливають назад до нього, а другого разу
II на закуп у І авансує 400\pound{ ф. стерл.}, що повертаються до нього. Це
припущення доводиться зробити, бо було б довільним зворотне припущення,
що кляса капіталістів І або кляса капіталістів II однобічно авансує
на циркуляцію гроші, потрібні для обміну товарів. А що в попередньому
параграфі 1) показано, що треба відкинути як безглузду ту гіпотезу, за
якою додаткові гроші, потрібні на перетворення 200 II~$с$ ($d$) на гроші,
подає в циркуляцію І, то, очевидно, лишається тільки ще, здається, безглуздіша
гіпотеза, а саме, що II сам подає в циркуляцію гроші, що за допомогою
їх перетворюється на гроші складова частина вартости товару,
яка має замістити зношування основного капіталу. Напр., частина вартости,
втрачена в продукції прядільною машиною пана $X$, знову з’являється
як частина вартости ниток до шиття; те, що на одному боці його прядільна
машина втрачає в вартості або як зношування, повинно нагромаджуватись
у нього на другому боці як гроші. Хай $X$ купує, напр., на
200\pound{ ф. стерл.} бавовни в $V$ і таким чином авансує для циркуляції 200\pound{ ф.
стерл.} грішми; $V$ купує в нього пряжі на ці самі 200\pound{ ф. стерл.}, і ці
200\pound{ ф. стерл.} служать тепер для $X$ як фонд заміщувати зношування прядільної
машини. Це сходило б просто на те, що $X$, незалежно від своєї
продукції, її продукту й продажу його, має in petto 200\pound{ ф. стерл.} для
того, щоб виплатити самому собі вартість, яку втрачає його прядільна
машина, тобто, що він, крім вартости, втрачуваної його прядільною машиною,
а вона доходить 200\pound{ ф. стерл.}, мусить щороку додавати з своєї
кишені ще по 200\pound{ ф. стерл.} грішми для того, щоб, кінець-кінцем, мати
змогу купити нову прядільну машину.

Та це безглуздя лише позірне. Кляса II складається з капіталістів,
що їхній основний капітал перебуває на цілком різних ступенях своєї
репродукції. Для одних уже надійшов час, коли його треба цілком замістити
in natura. У других основний капітал більш або менш далекий
від цієї стадії; для всіх членів останнього підрозділу спільне те, що їхній
основний капітал покищо не репродукується реально, тобто не відновлюється
in natura, не заміщується новим екземпляром того самого роду,
але його вартість послідовно нагромаджується в грошах. Перша частина
капіталістів перебуває цілком (або почасти, — це тут не має значення)
в такому самому стані, як при відкритті свого підприємства, коли капіталісти
з грошовим капіталом виступили на ринок, щоб перетворити його,
з одного боку, на (основний та обіговий) сталий капітал, а з другого
боку — на робочу силу, на змінний капітал. Як і тоді, їм тепер доводиться
знову авансувати цей грошовий капітал для циркуляції, — тобто доводиться
авансувати вартість сталого основного капіталу, цілком так само, як і вартість
обігового і вартість змінного капіталу.

Отже, коли припускається, що з 400\pound{ ф. стерл.}, подаваних в циркуляцію
клясою капіталістів II для обміну з І, одна половина походить від
\parbreak{}  %% абзац продовжується на наступній сторінці
