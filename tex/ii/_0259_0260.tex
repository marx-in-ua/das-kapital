
\index{ii}{0259}  %% посилання на сторінку оригінального видання
Всі заперечення — це є сліпий вистріл капіталістів та їхніх економістів
сикофантів.

Факти, що дають нагоду для такого сліпого вистрілу, є троякого роду.

1) Загальний закон грошової циркуляції той, що коли сума цін товарів,
що циркулюють, підвищується — все одно чи це збільшення суми
цін постає для тієї самої маси товарів, чи для збільшеної, — то за інших
незмінних обставин зростає маса грошей, що циркулюють.

Тут наслідок сплутують з причиною. Заробітна плата підвищується (хоч
і рідко підвищується, а пропорційно до підвищення цін вона підвищується
тільки в виняткових випадках) із підвищенням цін доконечних засобів існування.
Її підвищення є наслідок, а не причина підвищення цін товарів.

2) При частковому або місцевому підвищенні заробітної плати, тобто
при підвищенні її тільки в поодиноких галузях продукції — може в наслідок
цього постати місцеве підвищення цін на продукти цієї галузі. Але навіть
це залежить від багатьох обставин. Напр., від того, що заробітна плата
тут не була надто низька і норма зиску тому не була надто висока, що
в наслідок підвищення цін ринок для цих товарів не скорочується
(отже, для підвищення їхніх цін не треба попереднього зменшення
подання їх) і~\abbr{т. ін.}

3) При загальному підвищенні заробітної плати підвищується ціна товарів,
продукованих в тих галузях промисловости, де переважає змінний капітал,
але зате спадає в тих, де переважає сталий, зглядно основний капітал.

\pfbreak{}

При дослідженні простої товарової циркуляції (книга І, розд. III, 2) виявилось,
що хоч у процесі циркуляції будь-якої певної кількости товарів її
грошова форма є лише минуща, однак, гроші, зникаючи при метаморфозі
товару в руках однієї особи, неодмінно переходять до рук іншої; отже,
товари насамперед не лише всебічно обмінюються або заміщуються один
одним, але це заміщення упосереднюється й супроводиться всебічним
осіданням грошей. „У наслідок заміщення одного товару іншим товаром
до рук третьої особи одночасно в’язне товар-гроші. Циркуляція постійно
спливає грошовим потом“ (кн. І, розд. III, 2, а). Той самий тотожній
факт на основі капіталістичної товарової продукції виражається в тому,
що частина капіталу постійно існує в формі грошового капіталу, а частина
додаткової вартости так само постійно перебуває в руках її власника
в грошовій формі.

Лишаючи це осторонь, \so{кругобіг грошей} — тобто зворотний приплив
грошей до свого вихідного пункту — оскільки він становить момент
обороту капіталу, є цілком відмінне явище, навіть протилежне \emph{обігові}
грошей\footnote{
Хоч фізіократи ще сплутують обидва ці явища, однак вони перші звернули
увагу на зворотний приплив грошей до свого вихідного пункту, як на важливу
форму циркуляції капіталу, як на форму циркуляції, що упосереднює репродукцію.
„Погляньте на „Tableau Économique“, і ви побачите, що продуктивна кляса дає
гроші на які інші кляси купують у неї продукти, і що вони повертають їй ці
гроші, повертаючись наступного року, щоб знову зробити в неї такі ж закупи\dots{}
Отже, ви не бачите тут іншого кругобігу, крім того, де по витраті
постає репродукція, а по репродукції витрата, — кругобігу, що його перебігає
циркуляція грошей, які є міра витрати й репродукції. („Jetez les yeux sur le
Tableau Economique, vous verrez, que la classe productive d nne l’argent,
avec lequel les autres classes viennent lui acheter des productions, et qu’elles lui
rendent cet argent en revenant l’année suivante faire chez elle les mêmes achats\dots{}
Vous ne v yez donc ici d’autre cercle que celui-ci de la dépense suivie de la reproduction,
et de la réproduktion suivie de la dépense; cercle qui est parcouru par la
circulation de l’argent qui mesure la dépense et la reproduction“ — Quesnay. „Problèmes
économiques, in Daire, Physiocrates, I“, p. 208, 209).

„Саме це постійне авансування й постійний поворот капіталів треба назвати
циркуляцією грошей, тією корисною й плодотворчою циркуляцією, яка оживляє
всю працю суспільства, підтримує рух і життя в політичному організмі і яку
цілком слушно можна порівняти з кровобігом у тваринному організмі“. (C’est
cette avance et cette rentrée continuelle des capitaux qui constituent ce qu’on doit
appeller la circulation de l’argent, cette circulation utile et féconde, qui anime tous
les travaux de la société, qui entretient le mouvement et la vie dans le corps politique,
et qu’on a grande raison de comparer à la circulation du sang dans le corps animal“.
— Turgot, „Reflexions“ etc, Oeuvres, éd. Daire, I, p. 45).}, який виражає постійне \emph{віддалення} їх від вихідного
\index{ii}{0260}  %% посилання на сторінку оригінального видання
пункту в наслідок ряду переміщень. (Кн.~І, розд. III, 2, б). Однак прискорений
оборот ео ipso\footnote*{
Тим самим. \emph{Ред.}
} включає й прискорений обіг.

\vtyagnut{}
Насамперед щодо змінного капіталу: коли, напр., грошовий капітал
в 500\pound{ ф. стерл.} обертається в формі змінного капіталу десять разів на
рік, то очевидно, що ця аліквотна частина грошової маси, яка циркулює,
пускає в циркуляцію вдесятеро більшу суму вартости \deq{} 5000\pound{ ф. стерл}.
Вона обігає між капіталістом і робітником десять разів протягом року.
Протягом року робітника десять разів оплачується, й сам робітник платить
тією самою аліквотною частиною грошової маси циркуляції. Коли
б при однакових розмірах продукції цей змінний капітал обертався
лише один раз протягом року, то тоді відбувся б лише один обіг
в 5000\pound{ ф. стерл}.

Далі, хай стала частина обігового капіталу дорівнює 1000\pound{ ф. стерл}.
Коли капітал обертається десять разів, то капіталіст продає свій товар, а
значить, і сталу обігову частину його вартости десять разів на рік. Та
сама аліквотна частина грошової маси, що циркулює (1000\pound{ ф. стерл.}),
десять разів на рік переходить з рук власників цієї частини до рук капіталіста.
Десять разів переміщуються ці гроші з рук у руки. Подруге,
капіталіст десять разів на рік купує засоби продукції, це знову є десять
обігів грошей з рук до рук. За допомогою грошей на суму 1000\pound{ ф. стерл.}
промисловий капіталіст продає товару на \num{10.000}\pound{ ф. стерл.} і знову купує
товару на \num{10.000}\pound{ ф. стерл}. В наслідок двадцятиразового обігу 1000\pound{ ф. стерл.}
циркулює запас товару в \num{20.000}\pound{ ф. стерл}.

Нарешті, при прискореному обороті швидше циркулює й та частина
грошей, що реалізує додаткову вартість.

Навпаки, швидший обіг грошей і не включає неодмінно швидшого обороту
капіталу, а тому й швидшого обороту грошей, тобто не включає
неодмінно скорочення та швидкого поновлення процесу репродукції.
