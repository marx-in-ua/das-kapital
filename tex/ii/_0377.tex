\parcont{}  %% абзац починається на попередній сторінці
\index{ii}{0377}  %% посилання на сторінку оригінального видання
100\pound{ ф. стерл.} грішми й даючи за них 80\pound{ ф. стерл.} товаром, замість авансувати 80\pound{ ф. стерл.} грішми й
дати за них 80\pound{ ф. стерл.} товаром. Інакше кажучи, вона без якоїбудь користи постійно авансує на 25\%
більший грошовий капітал для циркуляції свого змінного капіталу, а це є цілком ориґінальний метод
збагачення.

3) Нарешті, кляса капіталістів продає „капіталістам-неробам, які платять, їм частиною свого доходу,
ще невитраченою на наймання робітників, що роблять безпосередньо для них, так що уся рента, щорічно
виплачувана промисловими підприємцями капіталістам-неробам, тим або іншим способом знову припливає
назад до промисловців“.

Раніше ми бачили, що промислові капіталісти, „частиною свого зиску оплачують усю ту частину їхнього
споживання, яка призначена на задоволення їхніх потреб“. Отже, припустімо, що їхній зиск \deq{} 200\pound{ ф.
стерл}. Припустімо, прим., що 100\pound{ ф. стерл.} вони витрачають на своє особисте споживання. Але друга
половина \deq{} 100\pound{ ф. стерл.} належить не їм, а капіталістам-неробам, тобто одержувачам земельної ренти й
капіталістам-позикодавцям
за проценти. Отже, промислові капіталісти повинні виплачувати цим людям 100\pound{ ф. стерл.} грішми.
Скажімо, що з цих грошей капіталістам-неробам треба 80\pound{ ф. стерл.} на їхнє власне споживання і 20\pound{ ф.
стерл.} на наймання слуг і~\abbr{т. ін.} Отже, на ці 80\pound{ ф. стерл.} вони купують засоби споживання у
промислових капіталістів. В наслідок цього до промислових капіталістів, тимчасом як від них
відходить продукт в 80\pound{ ф. стерл.}, повертається назад 80\pound{ ф. стерл.} грішми, або \sfrac{4}{5} тих 100\pound{ ф. стерл.},
що їх вони заплатили капіталістам-неробам під назвою ренти, проценту й~\abbr{т. ін.} Далі, кляса слуг,
безпосередні наймані робітники капіталістів-нероб, одержали від своїх панів 20\pound{ ф. стерл}. Вони
купують на них — теж у промислових капіталістів — засоби споживання на 20\pound{ ф. стерл}. В наслідок цього
до промислових капіталістів, тимчасом як від них відходить продукт на 20\pound{ ф. стерл.}, повертається
назад 20\pound{ ф. стерл.} грішми, або остання п’ята частина тих 100\pound{ ф. стерл.} грішми, що їх вони заплатили
капіталістам-неробам як ренту, процент та ін.

По закінченні оборудки до промислових капіталістів зворотно припливають ті 100\pound{ ф. стерл.} грішми, що
їх вони віддали капіталстам-неробам, сплачуючи ренту, процент і~\abbr{т. ін.}, тимчасом як половина їхнього
додаткового продукту \deq{} 100\pound{ ф. стерл.} з їхніх рук перейшла до фонду споживання капіталістів-нероб.

Отже, для питання, що про нього тут ідеться, очевидно, було б цілком зайве в тому або іншому вигляді
притягати до справи розподіл цих 100\pound{ ф. стерл.} між капіталістами-неробами та їхніми безпосередніми
найманими робітниками. Справа проста: їхні ренти, проценти, коротше, ту пайку, що їм припадає з
додаткової вартости \deq{} 200\pound{ ф. стерл.}, виплачують їм промислові капіталісти грішми, 100\pound{ ф. стерл}. На
ці 100\pound{ ф. стерл.} вони безпосередньо або посередньо купують засоби споживання у промислових
капіталістів. Отже, вони виплачують, їм назад 100\pound{ ф. стерл.} грішми й беруть у них на 100\pound{ ф. стерл.}
засобів споживання.
