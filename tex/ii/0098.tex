ці витрати. Ці витрати завжди становлять частину суспільної праці в зреневленій
або живій формі — отже, в капіталістичній формі вони є затрати
капіталу, — що не беруть участи в утворенні продукту, отже, вони є одбави
з продукту. Вони є доконечні, це — затрати (Unkosten) суспільного
багатства. Це — витрати на зберігання суспільного продукту, все одно,
чи походить існування його як елементу товарового запасу лише
із суспільної форми продукції, отже, з товарової форми та її доконечного
перетворення форми, чи ми розглядаємо товаровий запас лише як спеціальну
форму запасу продуктів, спільного всім суспільствам, хоч би
такий запас не мав форми товарового запасу, цієї форми запасу продуктів,
належної до процесу циркуляції.

Тепер запитаймо, якою мірою ці витрати входять у вартість товарів.

Коли капіталіст свій капітал, авансований на засоби продукції й робочу
силу, перетворив на продукт, на готову для продажу масу товарів, і вона
лишається непродана на складах, то на цей час не лише припиняється
процес зростання вартости його капіталу. Видатки, що їх потребує зберігання
цього запасу в приміщеннях, видатки на новододавану працю тощо, становлять
позитивну втрату. Покупець, що, кінець-кінцем, прийшов би, висміяв
би його, коли б капіталіст сказав: мій товар не купувався протягом шістьох
місяців, і коли зберігалось його протягом цих шістьох місяців, то не
тільки лежало марно стільки й стільки капіталу, але це спричинило мені,
крім того, х затрат (Unkosten). Tant pis pour vous\footnote*{
То гірше для вас. Ред.
}, скаже покупець. Бо
поряд вас є інший продавець, що його товар вироблено лише позавчора.
Ваш товар є заваль і, мабуть, більш або менш попсувався від часу. Отже,
ви мусите продавати дешевше, ніж ваш суперник. — Умови існування
товару зовсім не змінюються від того, чи є товаропродуцент справжній
продуцент свого товару, чи капіталістичний продуцент, тобто в суті
лише представник справжніх продуцентів. Йому треба перетворити свою
річ на гроші. Затрати (Unkosten), що їх спричиняє фіксування її
в товаровій формі, належать до його особистих справ, і до них його
покупцеві байдуже. Він не оплачує йому час циркуляції його товарів.
Навіть коли капіталіст навмисно тримає свій товар поза ринком, підчас
справжньої або передбачуваної революції у вартості, то й тоді від
справжнього постання цієї революції у вартості, від правильности чи
неправильности його спекуляції залежить, чи реалізує він свої додаткові
затрати (Unkosten). Але революція у вартості не є наслідок його затрат.
Отже, оскільки утворення запасу являє собою затримку циркуляції, спричинені
цим витрати не додають до товару жодної вартости. З другого
боку, жоден запас не може існувати без перебування в сфері циркуляції,
без довшого або коротшого перебування капіталу в його товаровій
формі; отже, жоден запас не буває без затримки циркуляції, так само,
як не можлива грошова циркуляція без утворення грошового резерву.
Отже, без товарового запасу не може бути жодної товарової циркуля-