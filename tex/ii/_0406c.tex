\index{ii}{0406}  %% посилання на сторінку оригінального видання
I.  $5000 с \dplus{} 500 m \text{(що їх треба капіталізувати)} \dplus{} 1500 (v \dplus{} m)$ споживного
фонду \deq{} 7000 в товарах.

II.    $1500 с \dplus{} 299 v \dplus{} 201 m \deq{} 2000$ в товарах. Загальна сума 9000 в
товаровому продукті.

Капіталізація тепер відбувається так:

В І підрозділі $500 m$, що їх капіталізується, поділяються на $\sfrac{5}{6} \deq{}
417 с \dplus{} \sfrac{1}{6} \deq{} 83 v$\footnote*{
Обчислено з закругленням дробів. \emph{Ред.}
}. Ці $83 v$ вилучають таку саму суму з $\text{II} m$, і на
неї купується елементи сталого капіталу, отже, їх долучається до $\text{II} с$.
Збільшення $\text{II} с$ на 83 зумовлює збільшення $\text{II} v$ на \sfrac{1}{5} від $83 \deq{} 17$.
Отже, після обміну ми маємо:

\begin{table}[h]
\begin{center}
\begin{tabular}{*{6}{r@{ }}}
I.  & $(5000 с \dplus{} 417 m) с \dplus{} (1000 v \dplus{} 83 m) v$ & \deq{} & $ 5417 c \dplus{} 1083 v$ & \deq{} & 6500 \\
II. & $(1500 с \dplus{} \phantom{0}83 m) c \dplus{} (\phantom{0}299 v \dplus{} 17 m) v$ & \deq{} & $1583 с \dplus{} \phantom{0}316 v $ & \deq{} & 1899 \\
\cmidrule(){4-6}
    &                                                   &   &               Разом   & \deq{} & 8399
\end{tabular}
\end{center}
\end{table}

Капітал в І зріс з 6000 до 6500, отже, на \sfrac{1}{12}. В II з 1715 до 1899,
отже, майже на \sfrac{1}{9}.

На другий рік репродукція на такій основі дає наприкінці року
капітал:

\begin{table}[h]
\begin{tabular}{r@{ } c@{ } c@{ } c@{ } r@{ } r@{ } r@{ } r@{ }}
І.  & $(5417 \dplus{} 452 m) c$ & \dplus{} & $(1083 v \dplus{} 90 m) v$ & \deq{} & $5869 с \dplus{} 1173 v$ & \deq{} & 7042\\
II. & $(1583 с \dplus{} 42 m \dplus{} 90 m) c$ & \dplus{} & $(316 v \dplus{} 8m \dplus{} 18 m) v$ & \deq{} & $1715 c \dplus{} \phantom{0}342 v$ & \deq{} & 2057
\end{tabular}
\end{table}
а наприкінці третього року дає продукт:

I. $5869 c \dplus{} 1173 v \dplus{} 1173 m$
II. $1715 с \dplus{} 342 v \dplus{} 342 m$.

Коли І акумулює при цьому, як і раніше, половину додаткової вартости,
то І ($v \dplus{} \sfrac{1}{2} m$) дає $1173 v \dplus{} 587 (\sfrac{1}{2} m) \deq{} 1760$, отже, більше,
ніж усі $1715 \text{ ІІ} с$, а саме більше на 45. Отже, цю ріжницю знову треба
покрити переміщенням до II $с$ засобів продукції на таку саму суму. Отже,
ІІ $с$ зростає на 45, що зумовлює приріст в II $v$ на \sfrac{1}{5} \deq{} 9. Далі капіталізовані
$587 \text{ I} m$ поділяються на \sfrac{5}{6} і \sfrac{1}{6} на $489 с$ і $98 v$; ці 98 зумовлюють
в II нову додачу 98 до сталого капіталу, а це теж зумовлює
збільшення змінного капіталу II на \sfrac{1}{5} \deq{} 20. Ми маємо тоді:

\begin{table}[h]
  \begin{center}
  \begin{tabular}{r@{ } c@{ } r@{ } c@{ } r@{ } r@{ } r@{ } r@{ }}
І. & $(5869 с \dplus{} 489 m) c$ & $+$ & $(1173 v \dplus{} 98 m) v$ & \deq{} & $6358 с \dplus{} 1271 v$ & \deq{} & 7629\\
II. & $(1715 с \dplus{} 45 m \dplus{} 98 m) c$ & $+$ & $(342 v \dplus{} 9 m \dplus{} 20 m) v$ & \deq{} & $1858 c \dplus{} \phantom{0}371 v$ & \deq{} & 2229\\
    \cmidrule{6-8}
    &                            &   &                          &   &    Цілий капітал & \deq{} & 9858
  \end{tabular}
  \end{center}
\end{table}

Отже, при репродукції, що протягом трьох років зростала,
весь капітал підрозділу І зріс з 6000 до 7629, а ввесь капітал підрозділу
II з 1715 до 2229, сукупний суспільний капітал з 7715 до 9858.
