\parcont{}  %% абзац починається на попередній сторінці
\index{ii}{0240}  %% посилання на сторінку оригінального видання
і спожитих запасів ще зовсім не оплачено. Отже, те, що з’являється
як криза на грошовому ринку, в дійсності виражає аномалії в самому процесі
продукції та репродукції.

\so{Потретє}. Щодо самого застосованого обігового капіталу (змінного
й сталого), то протяг періоду обороту, оскільки його зумовлює протяг
робочого періоду, призводить до такої ріжниці: при кількох оборотах
протягом року елементи змінного і сталого обігового капіталу може
давати продукт цього самого капіталу, як, напр., у вугільній промисловості,
в майстернях готового одягу тощо. В інших випадках це не можливо,
принаймні не можливо протягом року.

\section{Циркуляція додаткової вартости}

Ми бачили тут, що відмінність у періоді обороту зумовлює відмінність
у річній нормі додаткової вартости, навіть за незмінної маси щорічно
утворюваної додаткової вартости.

Але далі неодмінно постає ріжниця в капіталізації додаткової вартости,
в акумуляції, а тому при однаковій нормі додаткової вартости,
і в масі додаткової вартости, утвореній протягом року.

Ми помічаємо тепер насамперед, що капітал \emph{А} (в прикладі попереднього
розділу) має періодичний поточний дохід; отже, за винятком періоду
обороту на початку справи, він покриває своє власне споживання
протягом року з своєї продукції додаткової вартости, і йому не доводиться
робити авансувань з власного фонду. Навпаки, це останнє
бачимо у капіталіста \emph{В}. Хоч протягом того самого періоду він продукує
стільки ж додаткової вартости, як \emph{А}, але його додаткова вартість не
реалізована, а тому її не можна спожити ні особисто, ні продуктивно. Оскільки
йдеться про особисте споживання, додаткову вартість просто антиципується.
Фонд для особистого споживання мусить бути авансований.

Частина продуктивного капіталу, що її важко підвести під ту чи ту
рубрику, а саме додатковий капітал, потрібний для полагоджень та підтримування
в доброму стані основного капіталу, тепер теж виступає в
новому освітленні.

Цю частину капіталу в \emph{А} на початку продукції не авансується ні
цілком, ні більшиною. Капіталістові не треба мати її в своєму розпорядженні,
її може навіть і зовсім не бути в нього. Вона постає з самого
підприємства в наслідок безпосереднього перетворення додаткової вартости
на капітал, тобто в наслідок безпосереднього застосування її як капіталу.
Частина додаткової вартости, що протягом року не лише періодично
утворюється, а й реалізується, може покривати видатки, потрібні для
полагоджень тощо. Частину капіталу, потрібного для провадження підприємства
в його первісному розмірі, утворює таким чином протягом
\parbreak{}  %% абзац продовжується на наступній сторінці
