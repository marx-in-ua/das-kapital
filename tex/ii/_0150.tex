\parcont{}  %% абзац починається на попередній сторінці
\index{ii}{0150}  %% посилання на сторінку оригінального видання
щороку, або в інші, більш-менш короткі періоди реґулярно вилучається
з нього й приміщується або в основний капітал, або в фонд, призначений
для безпосереднього споживання. Кожний основний капітал первісно
постав з обігового й йому потрібна повсякчасна підтримка від цього
останнього. Всі корисні машини та знаряддя праці первісно постали з
обігового капіталу, який дав матеріяли, що з них їх зроблено, і утримання
робітникам, що їх зробили. Вони потребують також, щоб капітал,
зазначеного виду, підтримував їх завжди справними“\footnote*{
Of these four parts three — provisions, materials, and finished work, are either
annually or in a longer or shorter period, regularly withdrawn from it, and placed
either in the fixed capital, or in the stock reserved for immediate consumption.
Every fixed capital is both originally derived from, and requires to be continually
supported by a circulating capital. All useful machines and instruments of trade are
originally derived from a circulating capital, which furnishes the materials of which
they are made and the maintenance of the workmen who make them. They require,
too, a capital of the same kind to keep them in constant repair“ (p. 188).
}.

З винятком частини продукту, що її продуценти завжди безпосередньо
знову зуживають як засоби продукції, для капіталістичної продукції
має силу таке загальне правило: всі продукти подається як товари на ринок,
вони циркулюють для капіталіста як товарова форма його капіталу, як товаровий
капітал незалежно від того, чи мусять і чи можуть ці продукти
своєю натуральною формою, своєю споживною вартістю, функціонувати як
елементи продуктивного капіталу (продукційного процесу), тобто як засоби
продукції, а тому і як основні або поточні елементи продуктивного капіталу,
або чи можуть вони служити лише як засоби особистого, а не продуктивного
споживання. Всі продукти як товари подається на ринок;
тому всі засоби продукції та споживання, всі елементи продуктивного та
особистого споживання треба знову вилучити з ринку купівлею. Ця тривіяльність
(truism), звичайно, правильна. Отже, це однаково має силу й
щодо основних і щодо поточних елементів продуктивного капіталу, і
для засобів праці і для матеріялів праці в усіх їхніх формах. (При цьому
ще забувають, що елементи продуктивного капіталу дані самою природою,
отже, вони не є продукти). Машину купується на ринку так само,
як і бавовну. Але відси ні в якому разі не випливає, що кожний основний
капітал первісно походить із поточного капіталу — це випливає лише
з Смітового сплутування капіталу циркуляції з обіговим або поточним
капіталом, тобто неосновним капіталом. І до того ж Сміт сам себе збиває.
Машини як товар, за його словами, становлять частину зазначеного
в пункті 4 обігового капіталу. Їхнє походження з обігового капіталу значить,
отже, лише те, що вони функціонували як товаровий капітал раніш,
ніж почали функціонувати як машини, але — що речово вони походять
з самих себе; цілком так само, як бавовна, як поточний елемент капіталу
прядуна, походить з бавовни, що циркулювала на ринку. Але коли
Сміт в дальшому викладі висновує основний капітал з обігового на
тій підставі, що для машинобудівництва потрібні праця й сировинні матеріяли,
то, поперше, для цього потрібні також засоби праці, тобто основний
капітал і, подруге, щоб виготувати сировинні матеріяли, теж
\parbreak{}  %% абзац продовжується на наступній сторінці
