\parcont{}  %% абзац починається на попередній сторінці
\index{ii}{0242}  %% посилання на сторінку оригінального видання
характером підприємства, напр., будівель, почасти поширенням фонду
робочої сили, як у сільському господарстві, можливе лише в певних
більш-менш вузьких межах, і для цього треба додаткового капіталу
такого розміру, що його може дати лише багаторічна акумуляція додаткової
вартости.

Отже, поряд справжньої акумуляції або перетворення додаткової
вартости на продуктивний капітал (і відповідної репродукції в поширеному
розмірі) відбувається акумуляція грошей, нагромадження частини додаткової
вартости як лятентного грошового капіталу, який лише пізніше,
досягши певних розмірів, має функціонувати як додатковий активний
капітал.

Так стоїть справа з погляду поодинокого капіталіста. Однак, з розвитком
капіталістичної продукції розвивається одночасно кредитова система.
Грошовий капітал, що його капіталіст ще не може застосувати в своєму
власному підприємстві, застосовує інший і платить за це йому проценти. Він
функціонує для свого власника як грошовий капітал в особливому
значенні, як особливий ґатунок капіталу, відмінний від продуктивного
капіталу. Але він діє як капітал в руках другого. Очевидно, що при
частішій реалізації додаткової вартости і при збільшенні маштабу, що
в ньому її продукується, зростає пропорція, що в ній новий грошовий
капітал, або гроші як капітал, подається на грошовий ринок, а відси
знову вбирається — принаймні більшу частину його — для поширення
продукції.

Найпростіша форма, що в ній може виявлятися цей додатковий лятентний
грошовий капітал, є форма скарбу. Можливо, що цей скарб є
додаткове золото або срібло, одержане безпосередньо або посередньо
в обміні з країнами, що продукують благородні металі. І тільки таким
способом в країні абсолютно зростає грошовий скарб. З другого боку,
можливо — і так здебільша буває, — що цей скарб є не що інше, як
гроші, вилучені з циркуляції всередині країни, що набрали форму скарбу
в руках поодиноких капіталістів. Можливо далі, що цей лятентний грошовий
капітал складається просто з знаків вартости — кредитові гроші
ми тут ще лишаємо осторонь — або з простих, потверджених леґальними
документами вимог (юридичних титулів) капіталістів до третіх осіб. В
усіх цих випадках, хоч яка буде форма буття цього додаткового грошового
капіталу, він, оскільки він є капітал in spe\footnote*{
In spe — досл.; „в надії, в перспективі“, тобто потенціяльно. \emph{Ред.}
}, репрезентує не
що інше, як додаткові та в запасі тримані юридичні титули капіталістів на
майбутню додаткову річну продукцію суспільства.

„Таким чином, маса справді акумульованого багатства, розглядувана
з кількісного боку,\dots{} надзвичайно мала порівняно з продуктивними
силами суспільства, що йому воно належить, хоч на якому щаблі цивілізації
стояло б те суспільство; або навіть порівняно з дійсним споживанням
цього самого суспільства протягом лише небагатьох років; остільки
мала, що головну увагу законодавців та політико-економів треба було б
\parbreak{}  %% абзац продовжується на наступній сторінці
