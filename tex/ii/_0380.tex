\parcont{}  %% абзац починається на попередній сторінці
\index{ii}{0380}  %% посилання на сторінку оригінального видання
що хоч його й неправильно розуміють“ (mal connu — певно!), „але слушно звуть циркуляцією; бо він
справді є кругобіг і завжди повертається назад до свого вихідного пункту. Цей пункт є той, де
відбувається продукція“ (р. 239, 240).

Детю, цей very distinguished writer (видатний письменник), membre de l’Institut de France et de la
Société Philosophique de Philadelphie (член Інституту Франції та Філософського т-ва Філадельфії) і
справді до певної міри світило серед вульґарних економістів, наприкінці прохає читачів дивуватися з
тієї дивовижної ясности, що з нею він виклав перебіг суспільного процесу, з того потоку світла, що
його він пролив на предмет, і робить навіть таку ласку, що розкриває читачеві, відки походить все це
світло. Це треба навести в ориґіналі: „On remarquera, j’espère, combien cette manière de considérer
la consommation de nos richesses est concordante avec tout ce que nous avons dit à propos de leur
production et de leur distribution, et en même temps quelle clarté elle répand sur toute la marche
de la société. D'où viennent cet accord et cette lucidité? De ce que nous avons rencontré la vérité.
Cela rappelle l’effet de ces miroirs où les objets se peignent nettement et dans leurs justes
proportions, quand on est placé dans leur vrai point de vue, et où tout paraît confus et désuni,
quand on en est trop près ou trop loin“ (p. 242, 243)\footnote*{
„Сподіваюсь, що звернуть увагу на те, наскільки такий спосіб розглядати споживання наших багатств
узгоджується з усім сказаним нами про їхню продукцію та їхній розподіл, і яким світом він разом з
тим осяює ввесь перебіг суспільного розвитку. Відки походить ця гармонія і ця ясність? З того, що ми
відкрили істину. Це нагадує нам ті дзеркала, що точно й зберігаючи дійсні пропорції між частинами,
відображають усе, що ставиться перед ними в справжньому їхньому фокусі, і де все розпливається, коли
дуже наближатись або віддалятись“.
}.

Ось такий буржуазний кретинізм в його блискучому самозадоволенні\footnote*{
Voilà le crétinisme bourgeois dans toute sa béatitude!
}.
\label{original-380-1}

\section[Акумуляція та поширена репродукція]{Акумуляція та поширена репродукція\footnotemark{}}

\footnotetext{
Відси до кінця рукопис VIII.
}

\label{original-380-2}
\noindent{}В книзі І показано, як перебігає акумуляція для поодинокого капіталіста. В наслідок перетворення на
гроші товарового капіталу перетворюється на гроші й додатковий продукт, що в ньому втілюється
додаткова вартість. Цю додаткову вартість, перетворену таким чином на гроші, капіталіст знову
перетворює на додаткові натуральні елементи свого продуктивного капіталу. При наступному кругобігу
продукції збільшений капітал дає більшу кількість продукту. Але те, що відбувається з індивідуальним
капіталом, мусить також виявитись і в сукупній річній продукції, цілком подібно до того, що ми
бачили, розглядаючи просту репродукцію, де —
\parbreak{}  %% абзац продовжується на наступній сторінці
