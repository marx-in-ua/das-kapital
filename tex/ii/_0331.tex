\parcont{}  %% абзац починається на попередній сторінці
\index{ii}{0331}  %% посилання на сторінку оригінального видання
засобів продукції, що не залежать від неї, без засобів праці та матеріялів
продукції.

\subsection{Сталий капітал в обох підрозділах}

Щодо цілої вартости продукту в 9000 і тих категорій, що на них
вона розпадається, то аналіза її не являє більших труднощів, ніж аналіза
вартости продукту поодинокого капіталу; вона навіть тотожня з нею.

В цілому річному суспільному продукті тут містяться три річні суспільні
робочі дні. Вираз вартости кожного з цих робочіх днів \deq{} 3000; тому
вираз вартости цілого продукту 3000 × 3 \deq{} 9000.

Далі до початку того однорічного продукційного процесу, що його
продукт ми аналізуємо, минуло: в І підрозділі \sfrac{4}{3} робочого дня (новоутворена
вартість в 4000) і в підрозділі II \sfrac{2}{3} робочого дня (новоутворена
вартість в 2000). Разом 2 суспільні робочі дні, що спродукували
нову вартість \deq{} 6000. Тому 4000 І~$c \dplus{} 2000$ II~$c \deq{} 6000 c$ фігурують
як вартість засобів продукції, або стала капітальна вартість, що
знову являється в цілій вартості суспільного продукту.

Далі в підрозділі І з новодолученого річного робочого дня маємо \sfrac{1}{3}
доконечної праці або праці, яка заміщує вартість змінного капіталу 1000
І~$v$ і оплачує ціну праці, застосованої в І.~Так само в II підрозділі \sfrac{1}{6} суспільного
робочого дня є доконечна праця з вартістю в 500. Отже, 1000
I~$v \dplus{} 500$ II~$v \deq{} 1500 v$, вираз вартости половини суспільного робочого
дня, є вираз вартости тієї першої половини цілого долученого в поточному
році робочого дня, яка складається з доконечної праці.

Нарешті, в І підрозділі \sfrac{1}{3} цілого робочого дня новоутворена вартість \deq{}
1000, є додаткова праця; в II підрозділі \sfrac{1}{6} робочого дня, новоспродукована
вартість \deq{} 500, є додаткова праця; разом вони складають другу
половину цілого новодолученого робочого дня. Тому вся спродукована
додаткова вартість \deq{} 1000 І~$m \dplus{} 500$ II~$m \deq{} 1500 m$.

Отже:

Стала капітальна частина вартости суспільного продукту ($с$):

2 робочі дні, витрачені до розглядуваного продукційного процесу;
вираз вартости \deq{} 6000.

Доконечна праця ($v$), витрачена протягом року:

Половина робочого дня, витраченого на річну продукцію; вираз вартости
— 1500.

Витрачена протягом року додаткова праця ($m$):

Половина робочого дня, витраченого на річну продукцію; вираз вартости
\deq{} 1500.

Вартість, новоспродукована річною працею $(v \dplus{} m) \deq{} 3000$.

Ціла вартість продукту $(с \dplus{} v \dplus{} m) \deq{} 9000$.

Отже, труднощі не в аналізі вартости самого суспільного продукту.
Вони постають при зіставленні складових частин вартости суспільного
продукту з його речовими складовими частинами.
