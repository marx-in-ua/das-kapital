\parcont{}  %% абзац починається на попередній сторінці
\index{ii}{*0017}  %% посилання на сторінку оригінального видання
процес утворення додаткової вартости в його справжньому перебігу, — а
цього не зробив ніхто з його попередників; отже, він констатував у
самому капіталі відмінності, що з ними дати собі раду зовсім не могли
ні Родбертус, ні буржуазні економісти, — відмінності, які дають ключ до
розв’язання найзаплутаніших економічних проблем, — і найкращий доказ
цього знову є книга II, а ще більше книга IIІ, як це виявиться далі.
Далі він досліджував і саму додаткову вартість і виявив обидві її форми:
абсолютну й відносну додаткову вартість, і показав ту різну, але і в
тому й у другому разі вирішувальну ролю, яку відігравала вона в історичному
розвитку капіталістичної продукції. На основі додаткової вартости
він розвинув першу раціональну теорію заробітної плати, яку ми
маємо, і вперве висвітлив основні риси історії капіталістичної акумуляції
та виклав її історичні тенденції.

\vtyagnut{}
А що ж Родбертус? Прочитавши все це, він — як і всякий тенденційний
економіст! — вважає це за „напад на суспільство“, вважає, що він уже сам
сказав куди коротше та виразніше, відки постає додаткова вартість, і
вважає, нарешті, що хоч усе це слушно для „сучасної форми капіталу“,
тобто для капіталу, як він існує історично, але не слушно для „поняття
про капітал“, тобто для утопічного уявлення пана Родбертуса про капітал.
Цілком те саме, що з стариком Прістлеєм, який до кінця свого
життя вірив у флогістон і не хотів визнавати кисню. Тільки Прістлей
справді перший описав кисень, тимчасом як Родбертус у своїй додатковій
вартості або радше в своїй „ренті“ тільки знову відкрив загальник,
а Маркс поводився протилежно до Лявуазьє і був вищий від того, щоб
твердити, що він перший відкрив самий \emph{факт} існування додаткової
вартости.

Все інше, що зробив Родбертус у політичній економії, стоїть на
такому самому рівні. Його перероблення додаткової вартости на утопію
Маркс критикував ненароком уже в „Misère de la Philosophie“;
все, що можна було ще сказати про це, я сказав у передмові до німецького
перекладу цієї праці. Його пояснення торговельних криз недостатнім
споживанням робітничої кляси є вже в Сісмонді в „Nouveaux Principes
de l’Economie Politique“, книга IV, розділ IV\footnote{
„Ainsi donc, par la concentration des fortunes entre un petit nombre des propriétaires,
le marché intérieur se reserre toujours plus, et l'industrie est toujours plus
réduite à chercher ses débouchés dans les marchés étrangers, où de plus grandes révolutions
les menacent“. Nouv. Princ., éd. 1819, 1, p. 336. („Отже, в наслідок концентрації
багатств у руках небагатьох власників унутрішній ринок дедалі більше
скорочується, і промисловості доводиться дедалі більше шукати собі місця збуту
на закордонних ринках, де їй загрожують превеликі перевороти“ (а саме криза
1817~р., що її далі й описується).
}.Тільки Сісмонді
при цьому завжди мав на увазі світовий ринок, тимчасом як горизонт
Родбертуса не поширюється поза пруський кордон. Його міркування про
те, чи з капіталу, чи з доходу походить заробітна плата, належать до
схоластики і остаточно їх усувається третім відділом цієї другої книги
„Капіталу“. Його теорія ренти лишилась виключно його здобутком і
\parbreak{}  %% абзац продовжується на наступній сторінці
