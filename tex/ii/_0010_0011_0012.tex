
\index{ii}{0010}  %% посилання на сторінку оригінального видання
Отже, суть справи, яка становить тут основу акту $Г — Т\splitfrac{Р}{Зп}$, є
розподіл; не розподіл у звичайному розумінні як розподіл засобів споживання,
а розподіл елементів самої продукції, що з них речові чинники
є сконцентровані на одному боці, а робоча сила, відокремлена від них, —
на другому.

Отже, засоби продукції, речова частина продуктивного капіталу,
мусять протистояти робітникові як такі, як капітал раніше, ніж акт $Г — Р$
може стати загальним суспільним актом.

Ми вище бачили, що капіталістична продукція, скоро вона вже постала,
в своєму розвитку не лише репродукує це відокремлення, але поширює
його в дедалі більших розмірах, поки воно не зробиться загальним домінантним
суспільним станом. Але в цій справі є ще другий бік. Щоб
капітал міг утворитись і опанувати продукцію, для цього повинна бути
передумова — певний щабель у розвитку торговлі, значить, і в розвитку
товарової циркуляції, а тим самим і товарової продукції; бо речі не
можуть увійти в циркуляцію як товари, коли їх продукується не для
продажу, отже, не як товари. Але лише на основі капіталістичної продукції
товарова продукція з’являється як нормальний, домінантний характер
продукції.

Російські землевласники, які в наслідок так званого визволення селян
провадять тепер своє сільське господарство найманими робітниками, а
не кріпаками, підневільними робітниками, скаржаться на дві обставини:
поперше, на брак грошового капіталу. Наприклад, вони кажуть: раніше,
ніж продати врожай, треба робити великі виплати найманим робітникам,
і тут бракує першої умови, готівки. Щоб капіталістично провадити
продукцію, капітал у формі грошей мусить завжди бути в наявності, саме
для видачі заробітної плати. Однак землевласники можуть з цього приводу
не журитись. З плином часу можна зривати рожі, і промисловий капіталіст
порядкує вже не лише власними грішми, але також і l’argent des autres\footnote*{
Грішми інших. \emph{Ред.}
}.

Але характеристичніша є друга скарга, а саме: хоча б і були
гроші, все ж немає в достатніх розмірах і в який завгодно час робочої
сили, що ії можна було б купити, бо російський сільський
робітник у наслідок спільної власности на землю в земельній громаді ще
не цілком відокремлений від своїх засобів продукції, а тому й не являє
ще „вільного найманого робітника“ в повному значінні цього слова.
А наявність останнього в широкому суспільному маштабі є неодмінна
умова для того, щоб $Г — Т$, перетворення грошей на товар, могло являти
собою перетворення грошового капіталу на продуктивний капітал.

Тому само собою зрозуміло, що формула кругобігу грошового капіталу:
$Г — Т\dots{} П\dots{} Т' — Г'$ є сама собою зрозуміла форма кругобігу капіталу
лише на основі вже розвиненої капіталістичної продукції, бо вона
має за передумову наявність кляси найманих робітників в суспільному
\index{ii}{0011}  %% посилання на сторінку оригінального видання
маштабі. Капіталістична продукція, як ми бачили, продукує не
лише товари і додаткову вартість; вона репродукує, і до того в чимраз
більшому розмірі, клясу найманих робітників і перетворює величезну
більшість безпосередніх продуцентів на найманих робітників. Тому
$Г — Т\dots{} П\dots{} Т' — Г'$, маючи за першу передумову для свого перебігу
постійну наявність кляси найманих робітників, припускає вже наявність
капіталу в формі продуктивного капіталу, а тому й форму кругобігу
продуктивного капіталу.

\subsection[Друга стадія. Функція продуктивного капіталу $П$]{Друга стадія. Функція продуктивного капіталу \emph{П}}

Розглядуваний тут кругобіг капіталу починається з акту циркуляції
$Г — Т$, з перетворення грошей на товар, з купівлі. Отже, циркуляція
мусить бути доповнена протилежною метаморфозою $Т — Г$, перетворенням
товару на гроші, продажем. Але безпосередній наслідок акту
$Г — Т\splitfrac{Р}{Зп}$ є перерва циркуляції капітальної вартости, авансованої в
грошовій формі. Що грошовий капітал перетворився на продуктивний
капітал, то капітальна вартість набула такої натуральної форми, що в
ній вона не може далі циркулювати, а мусить увійти в споживання, а
саме в продуктивне споживання. Споживання робочої сили, працю, можна
реалізувати лише в процесі праці. Капіталіст не може знову продати
робітника як товар, бо він не його раб, і купив він не що інше, як
користання з його робочої сили протягом певного часу. З другого боку,
він може скористатися з робочої сили, лише примусивши її використовувати
засоби продукції як товаротворчі елементи. Отже, наслідок першої
стадії це — перехід у другу, у продуктивну стадію капіталу.

Рух капіталу має вигляд $Г — Т\splitfrac{Р}{Зп}\dots{} П$, де крапки позначають, що
циркуляцію капіталу перервано, але процес його кругобігу триває далі,
бо із сфери товарової циркуляції він переходить до сфери продукції.
Отже, перша стадія, перетворення грошового капіталу на продуктивний
капітал, являє лише попередню і вступну фазу до другої стадії, до
функціонування продуктивного капіталу.

$Г — Т\splitfrac{Р}{Зп}$ має собі за передумову, що індивідуум, який виконує цей
акт, не тільки володіє вартостями в першій-ліпшій споживній формі,
але володіє цими вартостями в грошовій формі, що він є власник грошей.
Але акт цей сходить саме на віддачу грошей, і тому той індивідуум
може лишитись власником грошей лише остільки, оскільки гроші
implicite\footnote*{
Дослівно: що приховано міститься. Тут у розумінні, що акт віддачі грошей приховано містить у собі
зворотній приплив їх. \emph{Ред.}
} зворотно припливають до нього в наслідок самого акту віддачі.
\index{ii}{0012}  %% посилання на сторінку оригінального видання
Але гроші можуть зворотно припливати до нього в наслідок продажу товарів. Отже, цей акт з
самого початку припускає, що даний індивідуум — товаропродуцент.

$Г — Р$. Найманий робітник живе тільки з продажу робочої сили. Її збереження — його самозбереження —
потребує щоденного споживання. Отже, виплати йому мусять завжди повторюватись через короткі
промежки, щоб він міг повторювати закупки, потрібні для його самозбереження — акт $Р — Г — Т$ або $Т —
Г — Т$. Тому капіталіст завжди мусить протистояти йому як грошовий капіталіст, а його капітал — як
грошовий капітал. Але, з другого боку, щоб маса безпосередніх продуцентів, найманих
робітників, могла чинити акт $Р — Г — Т$, доконечні засоби існування повинні протистояти їм завжди в
такій формі, щоб їх можна було купити, тобто в товаровій формі. Отже, цей стан потребує вже
високорозвиненої циркуляції продуктів як товарів, отже, і високорозвиненої товарової продукції.
Скоро продукція за допомогою найманої праці є загальна, товарова продукція мусить бути загальною
формою продукції.
З свого боку товарова продукція, — коли припускається, що вона має загальний характер, — призводить
до дедалі більшого розподілу суспільної праці, тобто до дедалі більшого відокремлення продукту, що
його продукує певний капіталіст як товар, — призводить до того, що продукційні процеси, які один
одного доповнюють, дедалі більше розщеплюються на самостійні процеси. Тому такою самою мірою, як
розвивається $Г — Р$, розвивається $Г — Зп$, тобто в такому ж самому обсязі продукування засобів
продукції відокремлюється від продукування товару, що для нього вони правлять за засоби продукції, і
сами вони протистоять кожному товаропродуцентові як товари, що їх він не виробляє, а купує для свого
певного продукційного процесу. Вони
виходять з галузей продукції, самостійно проваджуваних, цілком відокремлених
від його власної, і входять в його галузь продукції як товари; отже, їх мусять купувати. Речові
умови товарової продукції протистоять товаропродуцентові в дедалі більшому обсязі як продукти інших
товаропродуцентів, як товари. І в такому самому обсязі капіталіст мусить виступати як грошовий
капіталіст, тобто поширюється той маштаб, що в ньому його капітал мусить функціонувати як грошовий
капітал.

З~другого боку, ті самі обставини, що утворюють основну умову капіталістичної продукції — наявність
кляси найманих робітників — спонукають до переходу цілої товарової продукції на капіталістичну
товарову продукцію. Що більше остання розвивається, то більше впливає вона руйнаційно й розкладово
на всяку старішу форму продукції, яка, бувши розрахована переважно на безпосереднє задоволення
власних потреб, перетворює на товар тільки надлишок продукту. Вона робить продаж продукту
переважним інтересом, при чому на початку вона виразно й не зачіпає самого способу продукції, —
такий був, наприклад, спочатку вплив капіталістичної світової торговлі на такі народи, як китайці,
індійці, араби тощо. Але далі, там, де вона вкорінюється, вона руйнує всі форми товарової продукції,
що ґрунтуються або на власній праці продуцентів,
\index{ii}{0013}  %% посилання на сторінку оригінального видання
або просто на продажу надлишкового продукту як товару. Спочатку
вона надає загальности товаровій продукції, а потім поступінно перетворює
всю товарову продукцію на капіталістичну\footnote{
До цього місця рукопис VII.~Звідси рукопис VІ.}.
