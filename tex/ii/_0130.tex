\parcont{}  %% абзац починається на попередній сторінці
\index{ii}{0130}  %% посилання на сторінку оригінального видання
і вартостетворчий, і треба буде його замінити. Отже, авансована капітальна
вартість повинна зробити деякий цикл оборотів, в даному разі, приміром, цикл
у десять річних оборотів — і визначається цей цикл часом існування,
а тому й часом репродукції або часом обороту застосованого основного
капіталу.

Отже, в тій самій мірі, в якій з розвитком капіталістичного способу
продукції збільшуються розміри вартости і протяг життя застосовуваного
основного капіталу, — в тій самій мірі розвивається життя промисловости
й промислового капіталу в кожній особливій галузі приміщення в багаторічне
життя, скажімо, пересічно в десятирічне. Якщо, з одного боку,
розвиток основного капіталу подовжує це життя, то, з другого боку,
його скорочують постійні перевороти в засобах продукції, перевороти,
що з розвитком капіталістичної продукції так само набирають дедалі
більшої сили. Відси випливає й зміна засобів продукції та потреба постійно
їх заміщувати, бо вони зазнають морального зношування за довгий
час до того, як фізично доживуть свого віку. Можна припустити, що
для вирішальніших галузей промисловости цей цикл життя становить тепер
пересічно десять років. Однак, тут має значення не певне число. В усякому
разі ясно: цим багаторічним циклом взаємно зв’язаних оборотів, що
в них капітал є зв’язаний своєю основною складовою частиною, дається
матеріяльна основа періодичних криз, при чому підприємство послідовно
переживає періоди послаблення, середньої жвавости, раптового розмаху,
кризи. Правда, періоди, коли капітал вкладається, дуже різні й зовсім не
збігаються один з одним. Проте, криза завжди становить вихідний пункт
для нових великих капіталовкладань. Отже, розглядаючи справу з суспільного
погляду — вона також дає більш або менш нову матеріяльну
основу для наступного циклу оборотів\footnote{
„Міська продукція зв’язана з оборотом, що охоплює кілька днів, а сільська,
навпаки, з оборотом, що охоплює кілька років“. Adam G.~Müller: „Die Elemente
der Staatskunst“. Berlin. 1809, II, ct. 178. Таке наївне уявлення романтики
про промисловість і хліборобство.
}.

5) Щодо способу обчислення обороту, дамо слово одному американському
економістові. „В деяких галузях підприємств ввесь авансований
капітал обертається або циркулює кілька разів протягом року; в інших
одна частина обертається більш як один раз на рік, а друга не так швидко.
Капіталіст повинен обчислювати свій зиск, зважаючи на той пересічний
період, що потрібен для цілого його капіталу, щоб перейти через його
руки або обернутись один раз. Припустімо, що людина вклала в певне
підприємство половину свого капіталу на будівлі й машини, що їх відновлюється
один раз на десять років; четверту частину — на знаряддя і~\abbr{т. ін.},
що їх відновлюється раз на два роки, і остання четверта частина, витрачена
на заробітну плату й сировинний матеріял, обертається двічі на рік.
Хай ввесь її капітал буде \num{50.000}\usd{ долярів}. Тоді її річні витрати будуть
такі:
