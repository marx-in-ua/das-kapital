
\index{ii}{0363}  %% посилання на сторінку оригінального видання
Коротко кажучи, коли б при простій репродукції та інших незмінних
обставинах, отже, при незмінних продуктивній силі, загальній масі та
інтенсивності праці, — ми припустили нестале відношення між відмерлим
(що потребує відновлення) і далі діющим в старій натуральній формі
(що просто долучає вартість до продуктів на заміщення свого зношування)
основним капіталом, то в одному випадку маса обігових складових частин,
що їх треба репродукувати, лишилась би та сама, але збільшилась би
маса основних складових частин, що їх треба репродукувати; отже, вся
продукція І мусила б збільшитись або, навіть лишаючи осторонь грошові
відношення, постав би дефіцит в репродукції.

В другому випадку: коли б відносна величина основного капіталу II,
що його треба репродукувати in natura, зменшилась, а тому збільшилась
би в такому ж відношенні та складова частина основного капіталу II,
яку покищо треба замістити лише в грошах, то маса обігових складових
частин сталого капіталу II, репродукованих І, лишилась би незмінна, а
маса основних частин, що їх треба репродукувати, навпаки, зменшилась
би. Отже, або зменшення всієї продукції І, або надлишок (як раніш був
дефіцит) і до того надлишок, що його не сила перетворити на гроші.

Правда, в першому випадку та сама праця при збільшені продуктивності,
протягу та інтенсивності могла б дати більший продукт, і таким
чином можна було б покрити дефіцит у першому випадку; але така
зміна не могла б статись без переміщення праці й капіталу з однієї галузі
продукції І в іншу, а всяке таке переміщення одразу ж викликало б
розлади. А подруге, І підрозділові довелось би (оскільки збільшуються
протяг та інтенсифікація праці) обміняти більшу вартість на
меншу вартість II, отже, сталось би знецінення продукту І.

Зворотне було б у другому випадку, де підрозділ І мусить скорочувати
свою продукцію, а це означає кризу для зайнятих у ньому робітників
і капіталістів, або він дає надлишок, а це знову таки є криза.
Самі по собі такі надлишки є не лихо, а вигода, але при капіталістичній
продукції вони є лихо.

Зовнішня торговля могла б допомогти в обох випадках; в першому
випадку, — щоб товар І, утримуваний в грошовій формі, перетворити на
засоби споживання; в другому випадку, — щоб збути товаровий надлишок.
Але зовнішня торговля, оскільки вона не просто заміщує елементи (також
і за вартістю), лише відсуває суперечності в ширшу сферу, відкриває їм
більший простір.

Коли усунути капіталістичну форму репродукції, то справа сходить
на те, що розмір частини основного капіталу, яка відмирає й тому
повинна заміщуватись in natura (тут капіталу, що функціонує в продукції
засобів споживання), змінюється в різні послідовні роки. Коли одного
року ця частина дуже велика (перевищує середню смертність, як це
буває з смертністю людей), то в наступному році вона, певно, буде
настільки ж менша.

Але від цього маса сировинних матеріялів, напівфабрикатів і допоміжних
матеріялів, потрібна для річної продукції засобів споживання, — припускаючи,
\index{ii}{0364}  %% посилання на сторінку оригінального видання
що інші умови лишились ті самі, — не змінюється; отже, вся
продукція засобів продукції мусила б в одному випадку поширитись, в
другому скоротитись. Цьому можна було б запобігти лише постійною
відносною перепродукцією; з одного боку, продукується основного
капіталу на певну кількість більше, ніж безпосередньо треба; з другого
боку, продукується такий запас сировинного матеріялу та інш., що
перевищує безпосередні річні потреби (це особливо стосується до засобів
існування). Такий рід перепродукції рівнозначний контролеві суспільства
над речовими засобами його власної репродукції. Але в капіталістичному
суспільстві вона є анархічний елемент.

Цей приклад з основним капіталом — при незмінному маштабі репродукції
— є разючий. Непропорційність у продукції основного та обігового
капіталу це — одна з улюблених економістами причин, що ними вони
пояснюють кризи. А що така непропорційність може й мусить поставати
при простому підтриманні основного капіталу, що вона може й
мусить поставати при припущенні ідеальної нормальної продукції, при
простій репродукції уже діющого суспільного капіталу, це для них — щось
нове.

\subsection{Репродукція грошового матеріялу}

До цього часу ми зовсім не звертали уваги на один момент, а саме
на річну репродукцію золота й срібла. Як простий матеріял для речей
розкошів, позолочування тощо, вони так само, як і всякі інші продукти,
не заслуговували б тут на особливу згадку. Навпаки, як грошовий
матеріял, а тому і як потенціяльні гроші, вони відіграють важливу ролю.
Для спрощення ми будемо вважати тут за грошовий матеріял тільки золото.

За старими даними вся річна продукція золота становила 800--900
тисяч фунтів \deq{} заокруглюючи 1100 або 1250 мільйонів марок. Навпаки,
за Зетбеером\footnote{Ad.~Soetbeer, „Edelmetall-Produktion“. Gotha, 1879, S. 112.} пересічно за 1871--75 роки лише \num{170.675} кг вартістю
в округлих цифрах 476 мільйонів марок. З цього давали: Австралія
округло 167, Сполучені Штати 166, Росія 93 мільйони марок. Решта
розподіляється між різними країнами на суму меншу, ніж 10 мільйонів
марок на кожну. Річна продукція срібла за той самий період становила
трохи менш, ніж 2 мільйони кілограмів вартістю на 354\sfrac{1}{2} мільйони
марок; з цього Мехіко давало округло 108, Сполучені Штати 102,
Південна Америка 67, Німеччина 26 мільйонів і~\abbr{т. ін.}

З країн, де панує капіталістична продукція, лише Сполучені Штати
є продуценти золота й срібла; європейські капіталістичні країни майже
все своє золото й переважну більшість свого срібла одержують з
Австралії, Сполучених Штатів, Мехіко, Південної Америки та Росії.

Але ми переносимо золоті копальні в ту країну капіталістичної продукції,
що її річну репродукцію ми тут аналізуємо, і робимо так ось з
яких міркувань.

Капіталістична продукція взагалі не існує без зовнішньої торговлі.
Але коли ми припускаємо нормальну річну репродукцію в даному маштабі,
\index{ii}{0365}  %% посилання на сторінку оригінального видання
ми тим самим припускаємо, що зовнішня торговля лише заміщує
тубільні предмети предметами іншої споживної або натуральної форми, не
впливаючи при цьому на відношення вартости, а значить, і на ті відношення
вартости, що в них обмінюються одна на одну дві категорії:
засоби продукції та засоби споживання, і так само не впливаючи на
відношення між сталим капіталом, змінним капіталом та додатковою
вартістю, що на них можна розкласти вартість продукту кожної з цих
двох категорій. Отже, притягнення зовнішньої торговлі до аналізи щорічно
репродукованої вартости продукту може лише заплутати справу, не даючи
жодного нового моменту ні для проблеми, ні для її розв’язання. Отже,
тут треба цілком абстрагуватись від неї; тому золото треба вважати тут
за безпосередній елемент річної репродукції, а не за довожуваний з-зовні
в наслідок обміну товаровий елемент.

Продукція золота, як і взагалі продукція металів, належить до кляси І,
до категорії, яка охоплює продукцію засобів продукції. Припустімо, що
річна продукція золота \deq{} 30 (для зручности; а дійсно цифра ця дуже
висока порівняно з числами нашої схеми); хай ця вартість розпадається
на $20с \dplus{}  5v \dplus{} 5m$; $20с$ треба обміняти на інші елементи І~$с$, і це ми розглянемо
потім; a $5v \dplus{} 5m$ (І) треба обміняти на елементи ІІс, тобто на
засоби споживання.

Щодо $5v$, то кожне підприємство, яке продукує золото, починає з
закупу робочої сили: не на золото, спродуковане в самому цьому підприємстві,
а на деяку масу грошей, наявних у країні. На ці $5v$ робітники
купують засоби споживання в II, а цей на ці гроші купує засоби продукції
в І. Коли II купує, скажімо, на $2v$ І золото як товаровий
матеріял і~\abbr{т. ін.} (складову частину свого сталого капіталу), то до продуцента
золота І повертаються $2v$ в грошах, що вже раніше належали
циркуляції. Коли II не купує в І далі матеріялу, то І купує в II, подаючи
своє золото як гроші в циркуляцію, бо на золото можна купити всякий
товар. Ріжниця тільки в тому, що І виступає тут не як продавець, а
лише як покупець. Золотопромисловці І можуть завжди збути свій товар;
він завжди є в такій формі, що його можна безпосередньо обміняти.

Припустімо, що прядун заплатив своїм робітникам $5 v$, а вони дають
йому за це, — лишаючи осторонь додаткову вартість, — пряжу в продукті \deq{} 5;
робітники на 5 купують у II~$с$, останній купує на 5 грішми пряжу в І, і
таким чином $5v$ грішми повертаються назад до прядуна. Навпаки, в
щойно припущеному випадку І~$з$ (так ми позначатимемо продуцента
золота) авансує своїм робітникам $5v$ грішми, що вже раніш належали
циркуляції; робітники витрачають ці гроші на засоби існування; але з
5 тільки 2 повертаються від II до І~$з$. Однак І~$з$ цілком так само, як і
прядун, може знову почати процес репродукції; бо його робітники дали
йому золотом 5, що з них він продав 2, а решту 3 має в формі золота, —
отже, йому доводиться тільки карбувати з них монету\footnote{
„Значну кількість золотих зливків (gold bullion) приставляють продуценти
золота безпосередньо до карбівниці в Сан-Франціско“. — Reports of Н.~М.~Secretaries
of Embassy and Legation. 1879. Part III, p. 337.
} або перетворити
\index{ii}{0366}  %% посилання на сторінку оригінального видання
їх на банкноти — і тоді ввесь його змінний капітал прямо, без
дальшого посередництва II, знову опиняється в його руках у грошовій
формі.

Але вже при цьому першому процесі річної репродукції сталася зміна
в кількості грошей, що дійсно або віртуально належать циркуляції. Ми
припустили, що II~$с$ купив $2v$ (І~$з$) як матеріял, а 3 як грошову форму
змінного капіталу І~$з$ знову витратив в межах II. Отже, з тієї маси грошей,
що її дано в наслідок нової продукції грошей, 3 лишились в межах
II й не повернулись до І. Згідно з нашим припущенням, II задовольнив
свою потребу в грошовому матеріялі. 3 лишаються в його руках як
золотий скарб. А що вони не можуть становити будь-якого елемента
його сталого капіталу й що, далі, II вже раніш мав достатній грошовий
капітал на закуп робочої сили; що, далі, за винятком елемента зношування,
цими додатковими 3$з$ не доводиться виконувати жодної функції в межах II,
на частину якого їх обмінено (вони могли б служити лише для того,
щоб pro tanto покривати елемент зношування тоді, коли II~$с$ (І) менше,
ніж ІІ~$с$ (2), а це буває випадково); що, з другого боку, саме за винятком
елемента зношування, ввесь товаровий продукт ІІ~$с$ треба обміняти на
засоби продукції I ($v \dplus{} m$), — то ці гроші цілком доводиться перенести
з ІІ~$с$ в II~$m$, хоч це останнє буде в доконечних засобах існування або в
засобах розкошів, і, навпаки, відповідну товарову вартість доводиться
перенести з II~$m$ в II~$с$. Результат: частина додаткової вартости нагромаджується
як грошовий скарб.

На другий рік репродукції, коли таку саму частину щорічно продукованого
золота й далі зуживається як матеріял, 2 знову повернуться
до І~$з$, а 3 заміститься in natura, тобто знову звільняться у II як скарб
і~\abbr{т. ін.}

Взагалі щодо змінного капіталу: капіталістові І~$з$, як і всякому
іншому, завжди доводиться авансувати цей капітал в грошах на закуп
праці. На це $v$ не йому, а його робітникам доводиться купувати в II; отже,
ніколи не може бути такого випадку, щоб він виступав як покупець,
тобто подав гроші в II без ініціятиви II.~Але оскільки II купує в нього
матеріял, оскільки II мусить перетворювати свій сталий капітал II~$с$ на золотий
матеріял, частина (І~$з$) $v$ повертається від II до І з таким самим шляхом,
як і до інших капіталістів І; а оскільки цього не постає, він заміщує своє $v$
золотом безпосередньо з свого продукту. Але в тій самій мірі, що в ній $v$,
авансоване в грошовій формі, не повертається до нього від II, частина
вже наявних засобів циркуляції (гроші, що приплили від І до II й не повернулись
до І) перетворюється в II на скарб, і тому частину додаткової
вартости II не витрачається на засоби споживання. Що постійно відкриваються
нові золоті копальні, або відновлюються роботи на старих,
то певна частина грошей, що їх І~$з$ повинен витрачати на $v$, завжди
становить частину тієї маси грошей, яка була вже до нової продукції
золота, яку І~$з$ за допомогою своїх робітників подає в II, і, оскільки
вона не повертається з II до І~$з$, вона становить там елемент для
утворення скарбів.
