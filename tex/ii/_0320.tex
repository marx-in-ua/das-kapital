\parcont{}  %% абзац починається на попередній сторінці
\index{ii}{0320}  %% посилання на сторінку оригінального видання
засобів споживання на таку саму суму; в наслідок цього ці 500\pound{ ф. стерл.}
припливають назад до II; капіталісти цього підрозділу тепер, як і раніше,
мають 500\pound{ ф. стерл.} в грошах і 2000\pound{ ф. стерл.} в сталому капіталі,
однак, останній знову перетворено з форми товарового капіталу на продуктивний
капітал. Циркуляція маси товарів на 5000\pound{ ф. стерл.} відбулась
за посередництвом 1500\pound{ ф. стерл.} грошей, а саме: 1) І виплачує робітникам
1000\pound{ ф. стерл.} за робочу силу такої самої величини вартости;
2) робітники на ці 1000\pound{ ф. стерл.} купують у II засоби існування;
3) II на ті самі гроші купує засоби продукції в І, що в нього таким
чином відновлюється в грошовій формі змінний капітал в 1000\pound{ ф. стерл.};
4) II купує на 500\pound{ ф. стерл.} засоби продукції у І;
5) І купує на ці самі 500\pound{ ф. стерл.} засоби споживання у II;
6) II купує на ті самі 500\pound{ ф. стерл.} засоби продукції у І;
7) І купує на ті самі 500\pound{ ф. стерл.} засоби
існування у II.~До II повернулись назад 500\pound{ ф. стерл.}, що їх він подав у
циркуляцію понад 2000\pound{ ф. стерл.} у своєму товарі й що за них він не
вилучив з циркуляції жодного еквіваленту в товарі\footnote{
Тут виклад трохи відхиляється від вище поданого. Там і
І підрозділ подав в циркуляцію додаткову суму в 500. Тут тільки II підрозділ дає
додатковий грошовий матеріял для циркуляції. Однак, це нічого не змінює в
кінцевому наслідку. — \emph{Ф.~Е.}}.

Отже, обмін відбувається так:

1) І платить 1000\pound{ ф. стерл.} грішми за робочу силу, отже, за товар \deq{} 1000\pound{ ф. стерл}.

2) Робітники на свою заробітну плату в сумі 1000\pound{ ф. стерл.} грішми
купують засоби споживання в II; отже, товар \deq{} 1000\pound{ ф. стерл}.

3) II на вторговані від робітників 1000\pound{ ф. стерл.} купує в І засоби
продукції такої ж вартости; отже, товар \deq{} 1000\pound{ ф. стерл}.

В наслідок цього до І повернулись 1000\pound{ ф. стерл.} в грошах як грошова
форма змінного капіталу.

4) II купує в І на 500\pound{ ф. стерл.} засоби продукції, тобто товар \deq{}
500\pound{ ф. стерл}.

5) І купує на ці самі 500\pound{ ф. стерл.} засоби споживання у II; отже,
товар \deq{} 500\pound{ ф. стерл}.

6) II купує на ці самі 500\pound{ ф. стерл.} засоби продукції в І, отже,
товар \deq{} 500\pound{ ф. стерл}.

7) І купує на ті самі 500\pound{ ф. стерл.} засоби споживання в II; отже,
товар \deq{} 500\pound{ ф. стерл}.

Сума обмінених товарових вартостей \deq{} 5000\pound{ ф. стерл}.

500\pound{ ф. стерл.}, що їх II підрозділ авансував на купівлю, повернулись
до нього назад.

Результат такий:

1) І підрозділ має змінний капітал в грошовій формі величиною в 1000\pound{ ф.
стерл.}, що їх він первісно авансував для циркуляції; крім того, він витратив
на своє особисте споживання 1000\pound{ ф. стерл.} у своєму власному
товаровому продукті; тобто витратив ті гроші, що їх він одержав від продажу
засобів продукції вартістю в 1000\pound{ ф. стерл.}.
