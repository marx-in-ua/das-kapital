\parcont{}  %% абзац починається на попередній сторінці
\index{ii}{0387}  %% посилання на сторінку оригінального видання
капітал (додатковий продукт), не пущено в рух і більше додаткової праці,
ніж та, яку витрачалось на основі простої репродукції. Ріжниця тут
тільки в формі застосовуваної додаткової праці, в конкретній природі її
особливого корисного характеру. Її витрачено на засоби продукції для
І~$с$ замість II~$с$, на засоби продукції засобів продукції, а не на засоби
продукції засобів споживання. При простій репродукції припускалось,
що всю додаткову вартість І витрачається як дохід, отже, на товари II;
отже, вона складалась лише з таких засобів продукції, які мали знову
замістити сталий капітал II~$с$, в його натуральній формі. Отже, для того,
щоб відбувся перехід від простої репродукції до поширеної, продукція підрозділу
І мусить мати змогу виробити менше елементів сталого капіталу
для II, але остільки ж більше для І.~Цей перехід, що не завжди відбувається
без труднощів, полегшує та обставина, що деякі продукти І
можуть служити як засоби продукції в обох підрозділах.

З цього випливає, що, — коли дивитись на справу лише з погляду величини
вартости, — в межах простої репродукції продукується матеріяльний
субстрат поширеної репродукції. Це — просто додаткова праця робітничої
кляси І, витрачена безпосередньо на продукцію засобів продукції, на утворення
віртуального додаткового капіталу І.

Отже, утворення віртуального додаткового грошового капіталу з боку
$А$, $А'$, $А''$ (І) — через послідовний продаж їхнього додаткового продукту,
який утворюється без якоїбудь капіталістичної витрати грошей, — є тут
лише грошова форма додатково спродукованих засобів продукції І.

Отже, продукція віртуального додаткового капіталу в нашому випадку
(бо, як побачимо далі, він може утворитись цілком інакше) виражає не
що інше, як явище самого процесу продукції, продукції, в певній формі,
елементів продуктивного капіталу.

Отже, продукція додаткового віртуального грошового капіталу в широкому
маштабі — в багатьох пунктах на периферії циркуляції — є не що
інше, як результат і вираз багатобічної продукції віртуального додаткового
продуктивного капіталу, що саме постання його не має собі за передумову
жодних додаткових грошових витрат з боку промислових капіталістів.

Послідовне перетворення з боку $А$, $А'$, $А''$ і~\abbr{т. д.} (І) цього віртуально
додаткового продуктивного капіталу на віртуальний грошовий
капітал (на скарб), перетворення, що зумовлюється послідовним продажем
їхнього додаткового продукту, — отже, повторюваним однобічним продажем
товару без доповнення купівлею, — відбувається через повторюване
вилучення з циркуляції грошей і відповідне йому утворення скарбу. Таке
утворення скарбу — за винятком того випадку, коли покупцем є золотопромисловець,
— зовсім не має собі за передумову додаткового багатства
в благородних металях, а лише зміну функцій обігових до цього часу
грошей. До цього часу вони функціонували як засоби циркуляції; тепер
вони функіонують як скарб, як утворюваний, віртуально новий грошовий
капітал. Отже, утворення додаткового грошового капіталу й маса
наявного в країні благородного металю не мають жодного причинного
зв'язку між собою.
