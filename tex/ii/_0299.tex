\subsection[Пізніші економісти]{Пізніші економісти\footnotemark{}}


\label{original-299}
Рікардо\footnotetext{Відси й до кінця розділу додаток з рукопису II.}
майже дослівно відтворює теорію А.~Сміта: „Не можна не
погодитися, що всі продукти країни споживається, але дуже велика ріжниця,
чи споживають їх ті, хто репродукує іншу вартість, чи ті, хто
цього не робить. Коли ми кажемо, що дохід заощаджується й долучається
до капіталу, то ми хочемо цим сказати, що частину доходу, долучену
до капіталу, споживають продуктивні робітники замість непродуктивних“.
(„Principles“, ch. VIII, р. 163).

Справді Рікардо цілком приймає теорію А.~Сміта про розклад товарової
ціни на заробітну плату й додаткову вартість (або змінний капітал
і додаткову вартість). Він заперечує йому 1) щодо складових частин
додаткової вартости: він виключає земельну ренту як неодмінний елемент
додаткової вартости, 2) Рікардо \emph{розкладає} ціну товару на ці складові
частини. Отже, величина вартости для нього є prius\footnote*{
Попереднє, дане наперед, попередня умова, передумова. \emph{Ред.}
}. Сума складових
частин є для нього величина дана, як передумова, він виходить з
неї протилежно до А.~Сміта, який часто робить навпаки, висновуючи,
всупереч своїм власним глибшим поглядам, величину вартости товару
post festum через додавання складових частин.

Рамсай заперечує Рікардові: „Рікардо забуває, що цілий продукт не
лише поділяється на заробітну плату й зиск, але що частина його потрібна
також для заміщення основного капіталу“. («An Essay on the Distribution
of Wealth», Edinburgh 1836, p. 174). Рамсай розуміє під основним
капіталом те саме, що я розумію під сталим: „Основний капітал
існує в такій формі, що в ній він хоч і сприяє продукції товару, який є
в процесі праці, але не сприяє утриманню робітників“ (стор. 59).

А.~Сміт заперечував проти доконечного висновку з його розкладу
товарової вартости, отже, і вартости суспільного річного продукту на
заробітну плату й додаткову вартість, тобто на прості доходи: проти
висновку, що, згідно з ним, увесь річний продукт може бути спожитий.
Ориґінальні мислителі ніколи не роблять абсурдних висновків. Вони залишають
це Сеям і Мак-Куллохам.

Справді Сей розглядає справу дуже легковажно. Те, що для одного є
авансування капіталу, для другого є дохід і чистий продукт, або був ним.
Ріжниця між гуртовим і чистим продуктом — суто-суб’єктивна, і „таким чином
сукупна вартість усіх продуктів розподілилась у суспільстві як дохід“ (Say,
Traité d’Economie Politique, 1817, II, p. 69). „Ціла вартість кожного продукту
складається із зисків землевласників, капіталістів і тих, що працюють у
промисловості (заробітна плата фігурує тут як profits des industrieux\footnote*{
Зиски осіб, що працюють у промисловості. \emph{Ред.}
}):
„тобто тих, що сприяли його виготовленню. Тому дохід суспільства дорівнює
спродукованій   гуртовій   вартості, а не лише чистому
продуктові землі, як гадала секта економістів“ (фізіократи) (стор. 63).

Це відкриття Сеєве присвоїв собі між іншим також і Прудон.
