
\index{ii}{0353}  %% посилання на сторінку оригінального видання
I подає в циркуляцію $1000 v \dplus{} 800 m$ товарами; далі, він подає
в циркуляцію грішми: 1000\pound{ ф. стерл.} на заробітну плату і 400\pound{ ф. стерл.}
для обміну з II.~Після закінчення обміну І має: $1000 v$ грішми, $800 m$,
перетворені на 800 II~$с$ (засоби споживання), і 400\pound{ ф. стерл.} грішми.

II подає в циркуляцію $1800 с$ товарами (засобами споживання) і
400\pound{ ф. стерл.} грішми; після закінчення обміну він має: 1800 товарами І
(засобами продукції) і 400\pound{ ф. стерл.} грішми.

Тепер ми маємо ще на боці І~$200 m$ (в засобах продукції), на боці
II — $200 с$ ($d$) (в засобах споживання).

Згідно з припущенням, І на 200\pound{ ф. стерл.} купує засоби споживання
II~$с$ ($d$) вартістю в 200; але II затримує ці 200\pound{ ф. стерл.}, бо $200 с$ ($d$)
репрезентують зношування, отже, їх не треба безпосередньо перетворити
на засоби продукції. Отже, 200 І~$m$ лишаються не продані;
\sfrac{1}{5} додаткової вартости І, що її треба замістити, не сила реалізувати, не
сила перетворити з її натуральної форми засобів продукції на натуральну
форму засобів споживання.

Це не тільки суперечить припущенню, що репродукція відбувається
в попередньому маштабі; сама по собі така гіпотеза зовсім не з’ясовує
того, як $200 с$ ($d$) перетворюються на гроші; це значило б радше, що
таке перетворення годі з’ясувати. А що не сила з’ясувати, як $200 с$ ($d$)
можуть перетворитись на гроші, то припускається, що І з ласки своєї
перетворює їх на гроші, саме тому, що він не може перетворити на гроші
свою власну остачу в $200 m$. Вважати це за нормальну операцію механізму
обміну, — це все одно, якби ми припустили, що 200\pound{ ф. стерл.}
щороку падають з неба, щоб реґулярно перетворювати $200 с$ ($d$) на гроші.

Однак, безглуздя такої гіпотези не впадає безпосередньо на очі, коли І~$m$,
замість виступати, як тут, в своїй примітивній формі буття, — а саме як
складова частина вартости засобів продукції, отже, як складова частина
вартости товарів, що їх капіталістичні продуценти мусять через продаж
реалізувати в грошах, — коли І~$m$ замість того опиняється в руках співучасників
капіталіста, напр., як земельна рента в руках землевласників,
або як процент у руках грошових позикодавців. Але коли ту частину додаткової
вартости товарів, що її промисловий капіталіст повинен віддати
як земельну ренту або як процент іншим співвласникам додаткової вартости,
протягом довгого часу не сила реалізувати через продаж самих
товарів, то це значить кінець і для виплати ренти або проценту, а тому
ні землевласники, ні одержувачі проценту не можуть витрачати ренту й
процент, бути за dei ex machina\footnote*{
Дослівно: «боги з машинерії». В клясичних
трагедіях, коли інтриґа була дуже заплутана, автор розв’язував її штучно,
вводячи, як персонаж, бога або богів. \Red{Ред.}
} для того, щоб коли завгодно перетворювати
на гроші певні частини річної репродукції. Так само стоїть
справа з витратами усіх так званих непродуктивних робітників — державних
урядовців, лікарів, адвокатів і~\abbr{т. ін.} і взагалі всіх тих, хто на
\parbreak{}  %% абзац продовжується на наступній сторінці
