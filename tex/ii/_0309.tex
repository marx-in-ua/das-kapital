\parcont{}  %% абзац починається на попередній сторінці
\index{ii}{0309}  %% посилання на сторінку оригінального видання
отже, до рук капіталістів II, і лише в наслідок того, що капіталісти II застосовують
ці гроші на закуп засобів продукції, лише таким обкружним
шляхом повертаються вони знову до рук капіталістів І.

Виявляється, що при простій репродукції сума вартостей $v \dplus{} m$ товарового
капіталу І (отже, і відповідна пропорційна частина всього товарового
продукту І) мусить дорівнювати сталому капіталові II~$с$, виділеному
так само як пропорційна частина цілого товарового продукту кляси II;
або І ($v \dplus{} m$) \deq{} II~$с$.

\subsection[Обмін в межах підрозділу II. Доконечні засоби існування і речі розкошів]{Обмін в межах підрозділу II. Доконечні засоби існування~і~речі~розкошів}

З вартости товарового продукту підрозділу II нам лишається дослідити
ще складові частини $v \dplus{} m$. Розгляд їх не має ніякого чинення до
найголовнішого питання, що цікавить нас тепер, а саме, в якій мірі
розпад вартости всякого поодинокого капіталістичного товарового продукту
на $c \dplus{} v \dplus{} m$, хоча б і упосереднюваний різними формами виявлення,
має силу й для вартости цілого річного продукту. Це питання
розв’язується, з одного боку, через обмін І ($v \dplus{} m$) на II~$с$, а з другого
— через відкладений надалі дослід того, як І~$с$ репродукується в
річному товаровому продукті І.~Що II ($v \dplus{} m$) існує в натуральній формі
предметів споживання; що змінний капітал, авансований робітникам на
оплату робочої сили, взагалі та в цілому мусить витрачатися ними на
засоби споживання, і що частину товару, яка є $m$, припускаючи просту
репродукцію, фактично витрачається як дохід на засоби споживання, то
prima facie очевидно, що на заробітну плату, одержану від капіталістів
II, робітники II викупають частину свого власного продукту, яка відповідає
розмірам грошової вартости, одержаної ними як заробітна плата. Цим
самим кляса капіталістів II перетворює знову на грошову форму свій
грошовий капітал, авансований на оплату робочої сили; справа цілком
така сама, ніби ці капіталісти оплатили робітників простими знаками вартости.
Скоро робітники реалізують ці знаки вартости, купуючи частину
спродукованого ними й належного капіталістам товарового продукту, ці
знаки вартости повернуться знову до капіталістів, але лише тому, що тут
ці знаки не тільки репрезентують вартість, а й мають її в її золотій
або срібній тілесності. Далі ми дослідимо ближче цей рід зворотного
припливу змінного капіталу, авансованого в грошовій формі, здійснюваний
через процес, що в ньому робітнича кляса виступає як покупець, а
кляса капіталістів — як продавець. А тут ідеться про інше питання,
що його треба розглянути при цьому зворотному припливі змінного
капіталу до його вихідного пункту.

Категорія II річної товарової продукції складається з найрізноманітніших
галузей промисловости, які — щодо їхнього продукту — можна поділити
на два великі підвідділи:

а) Засоби споживання, що входять у споживання робітничої кляси, і —
оскільки це є доконечні засоби існування — становлять також частину
\parbreak{}  %% абзац продовжується на наступній сторінці
