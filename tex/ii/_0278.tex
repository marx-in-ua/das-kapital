\parcont{}  %% абзац починається на попередній сторінці
\index{ii}{0278}  %% посилання на сторінку оригінального видання
працю, або як зиск з їхнього капіталу, або як ренту з їхньої земельної
власности“, що в звичайному житті й розуміється як дохід. Вартість
цілого продукту, хоч для індивідуального капіталіста, хоч для цілої країни,
являє тому чийсь дохід; але, з одного боку, дохід на капітал, а з другого,
відмінну від цього доходу форму „revenue“ Отже, те, що усувається при
розкладі вартости товару на її складові частини, знову вводиться через
задні двері — через двозначність слова „revenue“\footnote*{
Про двояке значення слова „дохід“ див. „Капітал“, кн. І, розд. XXII, 3,
прим. 33. \emph{Ред.}
}. Але „заприбуткувати“
можна лише такі складові частини вартости продукту, які вже в ньому
існують. Щоб капітал одержувалось як дохід, капітал треба спочатку
витратити.

А.~Сміт каже далі: „Найнижча звичайна норма зиску має завжди
дещо перевищувати те, чого досить для відшкодування випадкових втрат,
що їм підпадає кожне застосування капіталу. Тільки цей надлишок і є
чистий, або нетто-зиск“. (Який же капіталіст розумів би зиск, як
доконечні витрати капіталу?) „В те, що зветься гуртовим зиском, часто
входить не тільки цей надлишок, а й частина, що її зберігається про
такі незвичайні втрати. (Кн.~І, розд. 9, стор. 72). Але це нічого іншого
не значить, а тільки те, що частина додаткової вартости, розглядувана
як частина гуртового зиску, мусить становити страховий фонд для продукції.
Цей страховий фонд утворює частина додаткової праці, яка в
цьому розумінні безпосередньо продукує капітал, тобто фонд, призначений
для репродукції. Щождо до витрат на „підтримання“ основного
капіталу й~\abbr{т. ін.} (див. вище цитовані місця), то заміщення спожитого
основного капіталу новим не становить нового капіталовкладення, а є
лише відновлення старої капітальної вартости в новій формі. Щождо
витрат на ремонт основного капіталу, що їх А.~Сміт теж залічує до витрат
на підтримання, то вони входять у ціну авансованого капіталу.
Та обставина, що капіталіст замість вкладати їх одним заходом, вкладає
їх підчас функціонування капіталу лише поступово та в міру потреби,
і може робити ці вкладання з уже одержаного зиску, зовсім не змінює
джерела цього зиску. Складова частина вартости, що з неї він походить,
показує лише, що робітник дає додаткову працю й для страхового фонду
й для фонду, призначеного на ремонт.

А.~Сміт розповідає нам далі, що з чистого доходу, тобто з доходу в
специфічному значенні, треба вилучити ввесь основний капітал, а також і
всю ту частину обігового капіталу, яка потрібна так для підтримання й
ремонту основного капіталу, як і для поновлення його, — тобто в дійсності
треба вилучити ввесь капітал, що перебуває не в такій натуральній
формі, в якій він призначається для споживного фонду.

„Всі витрати на підтримання основного капіталу, очевидно, треба
виключити з чистого доходу суспільства. Ні сировинні матеріяли, потрібні,
щоб тримати в належному стані корисні машини та промислові
знаряддя, ні продукт праці, потрібний, щоб перетворити ці сировинні
\parbreak{}  %% абзац продовжується на наступній сторінці
