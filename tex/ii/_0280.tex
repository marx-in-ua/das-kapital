\parcont{}  %% абзац починається на попередній сторінці
\index{ii}{0280}  %% посилання на сторінку оригінального видання
в наслідок праці цих робітників“. Але робітник не може жити „\emph{ціною}“
своєї праці, грішми, що в них дається йому заробітну плату; він реалізує
ці гроші, купуючи на них засоби споживання; ці останні можуть
почасти складатися з тих ґатунків товару, що він їх сам спродукував.
З другого боку, його власний продукт може бути такий, що ввіходить
лише в споживання визискувачів праці.

Цілком вилучивши таким чином основний капітал з „Netto-revenue“
(чистого доходу) даної країни, А.~Сміт каже далі:

„Хоч таким чином усі витрати на підтримання основного капіталу
неминуче вилучаються з чистого доходу суспільства, але цього не постає
з витратами на підтримання обігового капіталу. З чотирьох частин, що
з них складається цей капітал — грошей, засобів існування, сировинних
матеріялів та готових продуктів — три останні частини, як уже сказано,
реґулярно береться з нього й переміщується їх або в основний капітал
суспільства, або в запас, призначений для безпосереднього споживання.
Частина придатних для споживання виробів, яку не застосовується на
підтримання першого“ (основного капіталу), „цілком переходить в останній“
(в запас, призначений для безпосереднього споживання) „і становить
частину чистого доходу суспільства. Отже, підтримання цих трьох
частин обігового капіталу зменшує чистий дохід суспільства виключно
на ту частину річного продукту, яка потрібна на підтримання основного
капіталу“. (Кн.~II, розд. 2, стор. 192).

Сказати, що частина обігового капіталу, яка не служить для продукції
засобів продукції, входить у продукцію засобів споживання, тобто в частину
річного продукту, призначену на утворення споживного фонду
суспільства, — буде простою тавтологією. Але важливо те, що зараз по
цьому йде далі.

„Обіговий капітал суспільства щодо цього відрізняється від обігового
капіталу поодинокої особи. Обіговий капітал поодинокої особи цілком
виключається з її чистого доходу й ніколи не може становити частини
його; чистий дохід поодинокої особи може складатися виключно з її
зиску. Але хоч обіговий капітал кожної поодинокої особи становить
частину обігового капіталу суспільства, що до нього ця особа належить,
однак через це його ніяк не виключається неодмінно з чистого доходу
суспільства, й може він становити частину цього доходу. Хоч усі товари
в крамниці дрібного торговця зовсім не можна залічити до запасу, призначеного
для його власного безпосереднього споживання, а проте, вони
можуть входити в споживний фонд інших людей, які доходами, одержаними
з іншого джерела, реґулярно заміщують торговцеві вартість його
товарів разом з його зиском, не зменшуючи при цьому ні його, ні свого
капіталу“ (там само).

Отже, ми дізнаємось тут ось про що:

1) Так основний капітал і потрібний для його репродукції (про функціонування
він забуває) та підтримання обіговий капітал, так і обіговий
капітал кожного індивідуального капіталіста, діющий у продукції засобів
споживання, абсолютно виключаються з \emph{його} чистого доходу, що може
\parbreak{}  %% абзац продовжується на наступній сторінці
