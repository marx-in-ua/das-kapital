
\index{ii}{0166}  %% посилання на сторінку оригінального видання
\section{Робочий період}

Візьмімо дві галузі підприємств, де робочий день однакового протягу,
де процес праці триває, прим., десять годин, хай це буде прядіння
бавовни й фабрикація паровозів. В одній галузі щодня, щотижня виробляється
певну кількість готового продукту, бавовняної пряжі; в другій
галузі процес праці мусить повторюватись, може, протягом цілих трьох
місяців, поки виробиться один готовий продукт, один паровіз. В одному
разі продукт подільний з своєї природи, й та сама робота щодня або
щотижня починається знову. В другому разі процес праці безперервний,
охоплює більш або менш велике число щоденних процесів праці, що своїм
сполученням, безперервністю своїх операцій дають готовий продукт, лише
коли мине порівняно довший період. Хоч щоденний протяг процесу праці
тут той самий, а проте, є тут дуже посутня ріжниця в протягу продукційного
акту, тобто в протягу повторюваних процесів праці, потрібних
для того, щоб утворити готовий продукт, відіслати його як товар на
ринок, отже, перетворити його з продуктивного капіталу на товаровий.
Ріжниця між основним і обіговим капіталом не має до цього жодного
чинення. Зазначена ріжниця зберігалась би навіть тоді, коли б в обох
галузях підприємств основний і обіговий капітал прикладалось цілком
в однакових пропорціях.

Ці ріжниці в протягу продукційного акту помічаємо не лише між
різними сферами продукції, а й в межах тієї самої продукційної сфери
залежно від розміру того продукту, що його треба виготовити. Звичайний
житловий будинок можна збудувати протягом коротшого часу,
ніж більшу фабрику, й тому для цього треба меншого числа безперервних
процесів праці. Коли на будування паровоза витрачається три місяці,
то на будування панцерника витрачається один або кілька років. На продукцію
хліба витрачається майже цілий рік, на продукцію рогатої худоби
— кілька років, на лісорозведення — період від 12 до 100 років. Ґрунтовий
шлях можна прокласти за кілька місяців, а на будування залізниці
потрібні кілька років; звичайний килим виробляється, може, протягом
тижня, ґоблени — протягом років і т. ін. Отже, ріжниці в протягу продукційного
акту різноманітні до безмежности.

Ріжниця в протягу продукційних актів, очевидно, мусить при однаковій
величині вкладених у підприємство капіталів зумовити ріжницю в
швидкості обороту капіталу, отже, ріжницю в періодах, що на них авансується
даний капітал. Хай машинова бавовнянопрядільня й паровозобудівельний
завод застосовують однакові величиною капітали, а розподіл
на сталий і змінний капітал, а також на основну й поточну складові частини
капіталу в обох випадках буде однаковий; хай, нарешті, робочий
день має однаковий протяг і в однаковій пропорції поділяється на доконечну
працю й додаткову працю. Далі, щоб усунути всі обставини, які
випливають з процесу циркуляції і не мають чинення до цього питання,
\parbreak{}  %% абзац продовжується на наступній сторінці
