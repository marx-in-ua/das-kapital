\parcont{}  %% абзац починається на попередній сторінці
\index{ii}{0318}  %% посилання на сторінку оригінального видання
авансує промисловому капіталістові грошовий капітал (в найточнішому
значенні цього слова, тобто капітальну вартість у грошовій формі), то
справжнім пунктом повороту цих грошей є кишеня цього грошового
капіталіста. Таким чином, хоч гроші в своїй циркуляції більш або менш
переходять через усякі руки, маса грошей, що циркулюють, належить
підрозділові грошового капіталу, організованому і сконцентрованому в
формі банків тощо; спосіб, що ним цей підрозділ авансує свій капітал,
зумовлює постійний кінцевий, зворотний приплив до нього цього капіталу
в грошовій формі, хоч це знову таки упосереднюється зворотним перетворенням
промислового капіталу на грошовий капітал.

Для товарової циркуляції завжди потрібні дві умови; товари, подавані
в циркуляцію, і гроші, подавані в циркуляцію. „Процес циркуляції\dots{} не
закінчується, як безпосередній обмін продуктами, після того як споживні
вартості перемінили місця або посідачів. Гроші не зникають тому, що
вони, кінець-кінцем, випали з ряду метаморфоз якогось товару. Вони раз-у-раз
осідають у тих пунктах циркуляції, що їх звільняють ті або інші
товари“. (Книга І, розд. III, 2 п. а).

Напр., розглядаючи циркуляцію між II $с$ і І ($v \dplus{} m$), ми припустили,
що II підрозділ авансував для цієї циркуляції 500\pound{ ф. стерл.} грішми. При
безмежному числі тих процесів циркуляції, що на них сходить циркуляція
між великими суспільними групами продуцентів, продуцент то
однієї, то другої групи спершу виступає як покупець, отже, подає
гроші в циркуляцію. Цілком лишаючи осторонь індивідуальні обставини,
це зумовлено вже неоднаковістю періодів продукції, а тому й оборотів
різних товарових капіталів. Отже, II на 500\pound{ ф. стерл.} купує у І засобів
продукції на таку саму суму вартости, а І купує у II засобів споживання на
500\pound{ ф. стерл.}; отже, гроші припливають назад до II; останній ані трохи
не збагачується таким зворотним припливом. Спочатку він подав у циркуляцію
500\pound{ ф. стерл.} грішми і вилучив звідти товарів на ту саму суму вартости,
потім він продає товарів на 500\pound{ ф. стерл.} і вилучає з циркуляції
таку саму суму вартости в грошах; таким чином 500\pound{ ф. стерл.} припливають
назад. В дійсності II підрозділ подав таким чином у циркуляцію
на 500\pound{ ф. стерл.} грошей і на 500\pound{ ф. стерл.} товарів \deq{} 1000\pound{ ф. стерл.};
він вилучає з циркуляції на 500\pound{ ф. стерл.} товарів і на 500\pound{ ф. стерл.}
грошей. Для обміну 500\pound{ ф. стерл.} товарами (І) і 500\pound{ ф. стерл.} товарами
(II) циркуляція потребує лише 500\pound{ ф. стерл.} грішми; отже,
хто на закуп чужого товару авансував гроші, той одержує їх
назад, продаючи власний товар. Тому, коли б спочатку І купив у II
товару на 500\pound{ ф. стерл.}, а потім продав би підрозділові II товару на
500\pound{ ф. стерл.}, то 500\pound{ ф. стерл.} повернулись би до І, а не до II.

Гроші, витрачені на заробітну плату, тобто змінний капітал, авансований
у грошовій формі, в клясі І повертаються в цій формі не безпосередньо,
а посередньо, обкружним шляхом. Навпаки, в клясі II 500\pound{ ф.
стерл.} заробітної плати повертаються безпосередньо від робітників до
капіталістів, як і взагалі цей зворотний приплив завжди є безпосередній
у всіх тих випадках, коли купівля та продаж між тими самими особами
\parbreak{}  %% абзац продовжується на наступній сторінці
