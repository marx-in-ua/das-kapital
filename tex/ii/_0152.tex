\parcont{}  %% абзац починається на попередній сторінці
\index{ii}{0152}  %% посилання на сторінку оригінального видання
можуть належати до фонду споживання, отже, взагалі не належать до
суспільного капіталу, хоч і становлять елемент суспільного багатства, що
з нього капітал є лише частина. Продуцент цих речей, кажучи словами
Сміса, одержує зиск, продаючи їх. Отже, обіговий капітал! Людина, що
користується з них, їхній остаточний покупець, може використати їх, лише
вживаючи їх в процесі продукції. Отже, основний капітал!

Титули власности, напр., на залізницю, можуть щодня переходити з
рук в руки, і власники їх, продаючи ці титули, можуть одержувати зиск
навіть за кордоном; отже, титули власности на залізницю, протилежно
самій залізниці можна вивозити. А проте, сами ці речі мусять саме в тій
країні, де вони льокалізовані, або лежати без діла, або функціонувати як
основна складова частина продуктивного капіталу. Так само фабрикант
$А$ може одержахи зиск, продавши свою фабрику фабрикантові $В$, що однак
не перешкоджає фабриці й тепер, як раніше, функціонувати як основний
капітал.

Отже, якщо фіксовані в певному місці, невідокремлювані від ґрунту
засоби праці доконечно мусять — згідно з їхнім призначенням — функціонувати
як основний капітал в самій країні, хоча б для їхнього продуцента
вони функціонузали як товаровий капітал, не являючи елементів його
основного капіталу (останній складається для нього з засобів праці, що
вони потрібні на будування будівель, залізниць тощо), то відси ні в
якому разі не випливає зворотний висновок, що основний капітал мусить
складатись з нерухомих речей. Корабель або льокомотив працюють
лише рухаючись; і все ж вони функціонують — не для їхнього продуцента,
а для їхнього споживача — як основний капітал. З другого боку,
речі, що якнайочевидніше фіксовані в продукційному процесі, у ньому
живуть та вмирають і, одного разу ввійшовши в нього, вже ніколи його
не облишають, є поточні складові частини продуктивного капіталу. Напр.,
вугілля, зуживане машиною в процесі продукції, газ, що ним освітлюється
фабричну будову тощо. Вони поточні не тому, що вони разом з
продуктом матеріяльно облишають процес продукції і циркулюють як
товар, а тому, що їхня вартість цілком ввіходить у вартість товару, що
його продукувати вони допомагають, і, значить, її цілком треба покрити
через продаж товару.

В щойно цитованому місці з А. Сміса треба зазначити ще таке речення:
„Обіговий капітал, що дає\dots{} утримання робітникам, які виробляють
їх“ (мащини і т. інше).

У фізіократів частина капіталу, авансована на заробітну плату, правильно
фігурує під назвою avances annuelles протилежно до avances primitives.
З другого боку, в них виступає як складова частина продуктивного
капіталу, вживаного фармером, не сама робоча сила, а засоби існування,
що їх видається сільсько-господарським робітникам („утримання
робітників“, як каже Сміс). Це точно відповідає їхній специфічній доктрині.
А саме — у них частину вартости, долучувану працею до продукту
(цілком так само, як і ту частину вартости, що її долучають до продукту
сировинні матеріяли, знаряддя праці та інші речові складові частини
\parbreak{}  %% абзац продовжується на наступній сторінці
