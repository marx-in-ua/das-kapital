\parcont{}  %% абзац починається на попередній сторінці
\index{ii}{0323}  %% посилання на сторінку оригінального видання
вартости І відбувається лише через продаж товарів І~$m$, що в них
міститься ця додаткова вартість, і перебування додаткової вартости у
формі грошей триває кожного разу лише доти, доки гроші, вторговані
від продажу товару, не витратиться знову на засоби споживання.

Підрозділ І на додаткові гроші (500\pound{ ф. стерл.}) купує в II засоби
споживання; ці гроші І витратив і одержав за них еквівалент в товарах II;
першого разу гроші припливають назад тому, що II купує в І товару
на 500\pound{ ф. стерл.}; отже, вони повертаються як еквівалент товару, проданого
цим І, але цей товар нічого не коштує для І, отже, становить додаткову
вартість для І, і таким чином \emph{гроші, що їх він сам
подає в циркуляцію, перетворюють на гроші його
власну додаткову вартість}; так само при своїй другій купівлі
(№ 6) І одержує еквівалент в товарах II.~Коли припустити, що II не
купує в І засобів продукції (№ 7), то І справді заплатив би за 1000\pound{ ф.
стерл.} засобів споживання — спожив би всю свою додаткову вартість як
дохід, — а саме 500 своїми товарами І (засобами продукції) і 500 грішми;
при цьому в нього не лишилось би на складах 500\pound{ ф. стерл.} у товарах І
(в засобах продукції), і він витратив би грішми 500\pound{ ф. стерл}.

Навпаки, II перетворив би лише три чверті свого сталого капіталу
з форми товарового капіталу знову на продуктивний капітал; а четверту
частину — на форму грошового капіталу (500\pound{ ф. стерл.}), і в дійсності на
гроші, що лежать без діла, або на гроші, що припинили свою функцію
й вичікують. Коли б такий стан тривав довго, то II мусів би скоротити
на одну чверть маштаб репродукції. Але ті 500 в засобах продукції,
які лишаються на шиї в І, не є додаткова вартість, що існує в товаровій
формі; вони з’явились замість авансованих 500\pound{ ф. стерл.} грішми, що І мав
поряд своїх 1000\pound{ ф. стерл.} додаткової вартости в товаровій формі. Як
гроші вони перебувають у формі, що в ній їх завжди можна реалізувати;
як товар їх у даний момент не сила продати. Відси ясно, що проста
репродукція — а за неї кожен елемент продуктивного капіталу мусить бути
заміщений так в II, як і в І, — тут можлива й далі тільки тоді, коли 500 золотих
птахів повернуться до того під розділу І, що спочатку випустив їх.

Коли капіталіст (тут ми все ще маємо перед собою промислового капіталіста,
що є разом з тим представник усіх інших) витратить гроші на
засоби споживання, то ці гроші для нього остаточно зникли, вони пішли
шляхом усього живого. Коли вони знову повертаються до нього, то це
може постати лише в тому разі, якщо він з циркуляції виловить їх за допомогою
товарів, тобто за допомогою свого товарового капіталу. Так само
як вартість його цілого річного товарового продукту (а він для нього \deq{} товаровому
капіталові), так і вартість кожного елемента цього останнього, тобто
вартість кожного поодинокого товару, розпадається для нього на сталу капітальну
вартість, змінну капітальну вартість і додаткову вартість. Отже,
перетворення на гроші одиниці з товарів (що з них як з елементів складається
товаровий продукт) є разом з тим перетворення на гроші певної частини
додаткової вартости, яка міститься в цілому товаровому продукті. Отже,
для даного випадку цілком правильно, що капіталіст сам подав у циркуляцію
\index{ii}{0324}  %% посилання на сторінку оригінального видання
ті гроші — а саме, витрачаючи їх на засоби споживання — що
ними перетворюється на гроші або, інакше кажучи, реалізується його
додаткова вартість. Звичайно, справа тут не в тих самих монетах, а
в сумі дзвінкої монети, рівній тій сумі (або рівній частині тієї суми),
що її він подав у циркуляцію на задоволення особистих потреб.

На практиці це стається двома способами: коли підприємство відкрито
лише поточного року, то мине чимало часу, в кращому разі кілька місяців,
перш ніж капіталіст матиме змогу витрачати на своє особисте споживання
гроші з доходів самого підприємства. Але через це він ні на хвилину
не відкладає свого споживання. „Він сам собі авансує (чи з своєї
власної кишені, чи з чужої в кредит, тут ця обставина зовсім не має
значення) гроші під додаткову вартість, яку він іще лише має здобути;
але цим самим він авансує і засоби циркуляції для реалізації додаткової
вартости, що її треба буде реалізувати пізніше. Навпаки, коли підприємство
вже давно йде правильним ходом, то виплати й надходження розподіляються
на різні строки протягом року. Що відбувається безперервно,
так це споживання капіталіста, яке антиципується і своїми розмірами
розраховується в певній пропорції до звичайних або передбачуваних надходжень.
В продажі кожної партії товару реалізується й частину додаткової
вартости, що її треба видобути протягом року. Але коли б протягом
цілого року продали лише стільки спродукованого товару, скільки
треба для заміщення сталої й змінної капітальної вартости, що є в ньому,
або коли б ціни спали так, що, продавши ввесь річний товаровий продукт,
можна було б реалізувати лише авансовану капітальну вартість, що
міститься в ньому, то у витрачанні грошей виразно виступило б антиципування,
надія на майбутню вартість. Коли наш капіталіст збанкротує,
то його кредитори й суд досліджуватимуть, чи були його антициповані
особисті витрати в правильному відношенні до розмірів його підприємства
й до надходжень додаткової вартости, що звично або нормально відповідають
цим розмірам.

Але коли ми візьмемо цілу клясу капіталістів, то теза, що вона сама
мусить подати в циркуляцію гроші для реалізації своєї додаткової вартости
(зглядно й для циркуляції свого капіталу, сталого й змінного), не
лише не є парадоксальна, але є неодмінна умова цілого механізму; тут
бо є лише дві кляси: робітнича кляса, що тільки й має свою робочу силу,
і кляса капіталістів, що в її монопольному володінні є засоби суспільної
продукції, так само, як і гроші. Парадокс був би тоді, коли б робітнича
кляса з самого початку авансувала з власних коштів гроші, потрібні для
реалізації додаткової вартости, що міститься в товарах. Але поодинокий
капіталіст робить це авансування завжди лише в такій формі, що він діє
як покупець, \emph{витрачає} гроші на закуп засобів споживання, або
\emph{авансує} гроші на закуп елементів свого продуктивного капіталу, чи то
робочої сили, чи то засобів продукції. Він завжди віддає гроші лише
за еквівалент. Гроші він авансує циркуляції лише таким самим способом,
яким авансує їй товари. І в тому, і в цьому разі він діє як вихідний пункт
циркуляції товарів та грошей.
