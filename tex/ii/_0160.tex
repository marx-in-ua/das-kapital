\parcont{}  %% абзац починається на попередній сторінці
\index{ii}{0160}  %% посилання на сторінку оригінального видання
капіталу. Навпаки, ця ріжниця випливає з тієї ролі, що її він відіграє
в процесі продукції, в одному разі як предмет праці, в другому — як засіб
праці.

Функція засобу праці в процесі продукції потребує, беручи пересічно,
щоб цей засіб протягом довшого або коротшого часу знову й знову
служив у повторюваних процесах праці. Тому вже його функцією диктується
більша або менша довготривалість його матеріялу. Але довготривалість
матеріялу, що з нього його виготовлено, сама по собі не
робить його основним капіталом. Той самий матеріял, як сировинний
матеріял, є обіговий капітал, і в економістів, що сплутують ріжницю між
товаровим капіталом і продуктивним капіталом із ріжницею між обіговим
капіталом і основним капіталом, та сама речовина, та сама машина як
продукт є обіговий капітал, а як засіб праці — основний капітал.

\vtyagnut{}
Але хоч основним капіталом засіб праці робиться не в наслідок
довготривалости матеріялу, що з нього його зроблено, а проте, його
роля як засобу праці, потребує, щоб він був з порівняно довготривалого
матеріялу. Отже, довготривалість його матеріялу є умова його функціонування
як засобу праці, а тому й матеріяльна основа того способу
циркуляції, що робить його основним капіталом. За інших незмінних обставин,
більша або менша нетривалість його матеріялу накладає на нього
в меншій або більшій мірі печать закріплености (Fixität) і, значить, має
посутній зв’язок з його якістю як основного (fixes) капіталу.

А коли частину капіталу, витрачену на робочу силу, розглядається
з погляду обігового капіталу, отже, як протилежність до основного
капіталу; коли в наслідок цього і ріжницю між сталим і змінним капіталом
сплутується з ріжницею між основним і обіговим капіталом, то
цілком природно, подібно до того, як речову реальність засобу праці
вважається за посутню основу його характеру як основного капіталу, висновувати
протилежно до цього, з речової реальности капіталу, витраченого
на робочу силу, його характер як обігового капіталу, а потім
знову визначити обіговий капітал за допомогою речової реальности змінного
капіталу.

Справжня речовина капіталу, витраченого на заробітну плату, є сама
праця, діюща, вартостетворча робоча сила, жива праця, що її капіталіст
обміняв на мертву зречевлену працю і ввів у свій капітал, — і в наслідок
лише цього вартість, що є в його руках, перетворюється на вартість, що
сама з себе зростає. Але цієї здібности до самозростання капіталіст не
продає. Вона завжди становить лише складову частину його продуктивного
капіталу, на зразок його засобів праці, але зовсім не становить
складової частини його товарового капіталу, як от, напр., готові продукти,
що їх він продає. В процесі продукції засоби праці як складова
частина продуктивного капіталу не протистоять робочій силі як основний
капітал, так само матеріял праці та допоміжні матеріяли як обіговий капітал
не збігаються з нею; тому й другому робоча сила протистоїть як
особистий чинник, тимчасом як і те й друге є речові чинники, — це
з погляду процесу праці. Те й друге протистоїть робочій силі, змінному
\parbreak{}  %% абзац продовжується на наступній сторінці
