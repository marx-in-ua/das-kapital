Для індивідуальних капіталів безперервність репродукції іноді більш
або менш порушується. Поперше, в різні періоди маси вартости часто
розподіляються по різних стадіях і функціональних формах нерівними
порціями. Подруге, ці порції, залежно від характеру вироблюваного товару,
а значить, залежно від особливої сфери продукції, куди вкладено капітал,
можуть розподілятись різно. Потретє, безперервність може більше або
менше порушуватись в тих галузях продукції, які залежать від доби
року, — чи то в наслідок природних умов (хліборобство, ловитва оселедців
тощо), чи то в наслідок умовних обставин, як, напр., при так званих сезонових
роботах. Якнайправильніше та якнайодноманітніше перебігає процес
на фабриці й у гірництві. Але ця відмінність галузей продукції не спричинюється
до жодної відмінности в загальних формах процесу кругобігу.

Капітал як вартість, що самозростає, охоплює не лише клясові
відносини, не лише певний характер суспільства, що ґрунтується на наявності
праці як праці найманій. Він є рух, процес кругобігу через різні
стадії, який знову таки містить у собі три різні форми процесу кругобігу.
Тому його можна зрозуміти лише як рух, а не як річ у стані спокою.
Ті, хто розглядають усамостійнення вартости лише як абстракцію, забувають,
що рух промислового капіталу є ця абстракція іn асtu\footnote*{
В дії, в акції. Ред.
}. Вартість
перебігає тут різні форми, різні рухи, що в них вона зберігається й разом
з тим виростає, збільшується. Що ми тут маємо діло насамперед з простою
формою руху, то ми не беремо на увагу тих революцій, що їх
може зазнавати капітальна вартість у процесі свого кругобігу; однак,
зрозуміло, що, не зважаючи на всі революції в вартості, капіталістична
продукція існує й може існувати далі лише доти, доки капітальна вартість
буде зростати, тобто доки вона, як усамостійнена вартість, робить
свій кругобіг, отже, доти, доки революції в вартості так або інакше
переборюються й вирівнюються. Рухи капіталу виступають як дії поодинокого
промислового капіталіста в той спосіб, що він функціонує як
покупець товарів і праці, продавець товарів і продуктивний капіталіст,
і таким чином своєю діяльністю упосереднює кругобіг. Коли суспільна
капітальна вартість зазнає революції щодо вартости, то може статись, що
індивідуальний капітал його підпаде їй і загине, бо не матиме змоги пристосуватись
до умов цього руху вартости. Що гостріші й частіші стають
революції щодо вартости, то більше автоматичний, з силою стихійного
природного процесу діющий рух усамостійненої капітальної вартости,
бере гору над передбачливістю й розрахунками поодинокого капіталіста,
то більше перебіг нормальної продукції підпадає під ненормальну спекуляцію,
то більшою стає небезпека для існування поодиноких капіталів.
Отже, ці періодичні революції в вартості потверджують те, що вони,
здавалось би, повинні збити, а саме усамостійнення, що його вартість
як капітал набуває та через свій рух зберігає й зміцнює.

Це чергування метаморфоз капіталу, що процесує, призводить до
того, що зміна в величині вартости капіталу — зміна, яка постає в круг-