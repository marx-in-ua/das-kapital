\parcont{}  %% абзац починається на попередній сторінці
\index{ii}{*0012}  %% посилання на сторінку оригінального видання
1861~\abbr{р.}, тимчасом як Родбертус і юрба його прихильників, що ростуть,
як гриби під теплим літнім дощем державного соціялізму, здається, зовсім
забули про це.

„Однак, — каже далі Маркс, — додаткову вартість, як таку, Сміт не
відрізняв, як осібну категорію від особливих форм, що їх вона набирає
в зиску та земельній ренті. Відси в нього, а ще більше в Рікардо, багато
помилок і хиб у досліді“. — Це речення цілком стосується до Родбертуса.
Його „рента“ є просто сума земельної ренти й зиску; про земельну
ренту він склав собі цілком хибну теорію, а зиск він приймає без якогобудь
перегляду таким, як знайшов його у своїх попередників. — Навпаки,
додаткова вартість Марксова є загальна форма тієї суми вартости, що
її привлащують без якогобудь еквіваленту власники засобів продукції,
і яка за цілком своєрідними законами, що їх уперше відкрив Маркс,
розкладається на особливі, перетворені форми зиску та земельної ренти.
Ці закони викладається в III книзі, де вперше виявиться, як багато треба
проміжних ланок для того, щоб від загального розуміння додаткової
вартости дійти до розуміння того, як вона перетворюється на зиск і
земельну ренту, отже, до розуміння законів розподілу додаткової вартости
серед кляси капіталістів.

Рікардо йде вже значно далі порівняно з А.~Смітом. Він обґрунтовує
своє розуміння додаткової вартости на тій новій теорії вартости, що в
зародковій формі хоч і є вже в А.~Сміта, але ним майже завжди забувається
в його дослідженнях, — теорії, що стала за відпровідний пункт
усієї дальшої економічної науки. З того, що вартість товару визначається
кількістю праці, зреалізованої в товарах, він висновує розподіл між
робітниками й капіталістами тієї кількости вартости, яку долучено до
сировинного матеріялу працею, її розподіл на заробітну плату й зиск
(тобто в даному разі додаткову вартість). Він доводить, що вартість
товарів лишається та сама, хоч як змінюється відношення між цими двома
частинами, — закон, що для нього він припускав лише поодинокі винятки.
Він навіть висновує деякі головні закони щодо взаємного відношення між
заробітною платою і додатковою вартістю (взятою в формі зиску), хоч і в
дуже загальному розумінні (Маркс. Капітал, І, розділ XV, А), і доводить, що
земельна рента є надлишок над зиском, надлишок, — який відпадає в певних
умовах. — У жодному з цих пунктів Родбертус не пішов далі, ніж Рікардо.
Внутрішні суперечності теорії Рікардо, що на них загинула його школа, лишились
або зовсім невідомі Родбертусові, або довели його лише до утопічних
вимог („Zur Erkenntniss“ etc., S. 130) замість економічних розв’язань.

Однак ученню Рікардо про вартість і додаткову вартість не довелось
чекати „Zur Erkenntniss“ etc. Родбертуса, щоб здобути соціялістичного
використання. На стор. 609 першого тому „Капіталу“ (2 нім. видання)
наведено цитату: „The possessors of surplus produce or capital („Посідачі
додаткового продукту або капіталу“) з праці „The Source and Remedy of
the National Difficulties. A Letter to Lord John Russell“. London 1821\footnote*{
Див. українське видання, розд. XXII.~І. \Red{Ред.}
}.
\parbreak{}  %% абзац продовжується на наступній сторінці
