\parcont{}  %% абзац починається на попередній сторінці
\index{ii}{0401}  %% посилання на сторінку оригінального видання
797\sfrac{1}{2}; лишається $676\sfrac{1}{2} \text{II} m$. Отже, II перетворює на сталий капітал
ще 121, і для цього треба йому 60\sfrac{1}{2} нового змінного капіталу: його
так само береться з 676\sfrac{1}{2}; для споживання лишається 616.

Тоді матимемо капіталу:
\begin{center}
I. Сталого 4840 \dplus{} 484 \deq{} 5324

Змінного 1210 \dplus{} 121 \deq{} 1331
\end{center}
\begin{center}
II. Сталого 1760 \dplus{} 55 \dplus{} 121 \deq{} 1936

Змінного 880 \dplus{} 27\sfrac{1}{2} \dplus{} 60\sfrac{1}{2} \deq{} 968
\end{center}

\begin{center}

 \text{Разом:} \left.\begin{aligned}
        \text{I. }5324 с \dplus{} 1331 v \deq{} 6655\\
        \text{II. }1936 с \dplus{} \phantom{0}968 v \deq{} 2904
       \end{aligned}
 \right\}
 \text{ \deq{} 9559,}

\end{center}

а наприкінці року матимемо продукту:

\begin{center}

 \left.\begin{aligned}
        \text{I. }5324 с \dplus{} 1331 v \dplus{} 1331 m \deq{} 7986\\
        \text{II. }1936 с \dplus{} \phantom{0}968 v \dplus{} \phantom{0}968 m \deq{} 3872
       \end{aligned}
 \right\}
 \text{ \deq{} 11858}

\end{center}

Повторюючи це обчислення й заокруглюючи дроби, матимемо наприкінці
наступного року продукту:

\begin{center}

 \left.\begin{aligned}
        \text{I. }5856 с \dplus{} 1464 v \dplus{} 1464 m \deq{} 8784\\
        \text{II. }2129 с \dplus{} 1065 v \dplus{} 1065 m \deq{} \text{4259\footnotemarkZ{}}
       \end{aligned}
 \right\}
 \text{ \deq{} 13043.}

\end{center}
\footnotetextZ{В нім. тексті тут, як і подекуди далі, є аритметичні помилки. Ці помилки ми виправили. \emph{Ред}.} % текст примітки прямо під заголовком

А наприкінці наступного року:

\begin{center}

 \left.\begin{aligned}
        \text{I. }6442 с \dplus{} 1610 v \dplus{} 1610 m \deq{} 9662\\
        \text{II. }2342 с \dplus{} 1171 v \dplus{} 1170 m \deq{} 4684
       \end{aligned}
 \right\}
 \text{= 14346.}

\end{center}

Протягом чотирилітньої репродукції в поширеному маштабі ввесь
капітал І і II збільшився з $5500 с \dplus{} 1750 v \deq{} 7250$ до $8784 с \dplus{} 2781 v \deq{}
11565$, отже, у відношенні 100: 160. Вся додаткова вартість спочатку
становила 1750, тепер вона становить 2781. Спожита додаткова вартість
спочатку була 500 для І і 600 для II, а разом 1100; вона була в
\parbreak{}  %% абзац продовжується на наступній сторінці
