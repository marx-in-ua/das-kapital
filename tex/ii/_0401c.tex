\parcont{}  %% абзац починається на попередній сторінці
\index{ii}{0401}  %% посилання на сторінку оригінального видання
797\sfrac{1}{2}; лишається $676\sfrac{1}{2} \text{II} m$. Отже, II перетворює на сталий капітал
ще 121, і для цього треба йому 60\sfrac{1}{2} нового змінного капіталу: його
так само береться з 676\sfrac{1}{2}; для споживання лишається 616.

Тоді матимемо капіталу:
\[
\begin{array}{r@{~}l@{~}r@{~}r@{~}l}
  \text{I. }&\text{Сталого} & 4840 \dplus{} & 484 & \deq{} 5324\\
            &\text{Змінного}& 1210 \dplus{} & 121 & \deq{} 1331
\end{array}
\]
\[
\begin{array}{r@{~}l@{~}r@{~}r@{~}r@{~}l}
  \text{II. }&\text{Сталого} & 1760 \dplus{}
    & 55\phantom{\sfrac{1}{2}} \dplus{} & 121\phantom{\sfrac{1}{2}} & \deq{} 1936\\
            &\text{Змінного}& 880 \dplus{}
    & 27\sfrac{1}{2} \dplus{} & 60\sfrac{1}{2} &\deq{} 1331
\end{array}
\]
\[
 \text{Разом: }\left.\begin{array}{r@{~}r@{~}r@{~}l}
        \text{I. }&5324 с \dplus{}& 1331 v & \deq{} 6655\\
        \text{II. }&1936 с \dplus{}& 968 v & \deq{} 2904
       \end{array}
 \right\}
 \text{\deq{} \num{9559},}
\]
а наприкінці року матимемо продукту:
\[
 \left.\begin{array}{r@{~}r@{~}r@{~}r@{~}l}
        \text{I. }&5324 с \dplus{}& 1331 v \dplus{}& 1331 m & \deq{} 7986\\
        \text{II. }&1936 с \dplus{}& 968 v \dplus{}& 968 m & \deq{} 3872
       \end{array}
 \right\}
 \text{\deq{} \num{11858}.}
\]
Повторюючи це обчислення й заокруглюючи дроби, матимемо наприкінці
наступного року продукту:
\[
 \left.\begin{array}{r@{~}r@{~}r@{~}r@{~}l}
        \text{I. }&5856 с \dplus{}& 1464 v \dplus{}& 1464 m & \deq{} 8784\\
        \text{II. }&2129 с \dplus{}& 1065 v \dplus{}& 1065 m & \deq{} 4259 \text{\footnotemarkZ{}}
       \end{array}
 \right\}
 \text{\deq{} \num{13043}.}
\]
\footnotetextZ{В нім. тексті тут, як і подекуди далі, є аритметичні помилки. Ці помилки ми виправили. \emph{Ред}.} % текст примітки прямо під заголовком
А наприкінці наступного року:
\[
 \left.\begin{array}{r@{~}r@{~}r@{~}r@{~}l}
        \text{I. }&6442 с \dplus{}& 1610 v \dplus{}& 1610 m & \deq{} 9662\\
        \text{II. }&2342 с \dplus{}& 1171 v \dplus{}& 1170 m & \deq{} 4684
       \end{array}
 \right\}
 \text{\deq{} \num{14346}.}
\]
Протягом чотирилітньої репродукції в поширеному маштабі ввесь
капітал І і II збільшився з $5500 с \dplus{} 1750 v \deq{} 7250$ до $8784 с \dplus{} 2781 v \deq{}
\num{11565}$, отже, у відношенні 100: 160. Вся додаткова вартість спочатку
становила 1750, тепер вона становить 2781. Спожита додаткова вартість
спочатку була 500 для І і 600 для II, а разом 1100; вона була в
\parbreak{}  %% абзац продовжується на наступній сторінці
