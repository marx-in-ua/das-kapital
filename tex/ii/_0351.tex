\parcont{}  %% абзац починається на попередній сторінці
\index{ii}{0351}  %% посилання на сторінку оригінального видання
а також з натуральних елементів основного капіталу, — машин,
знарядь, будівель і~\abbr{т. ін.} Тому зношування, яке у вартості 2000 II~$с$ треба
замістити грішми, зовсім не відповідає розмірові діющого основного
капіталу, бо частину його щороку доводиться заміщувати in natura; але
де припускає, що в попередні роки в руках капіталістів кляси II нагромадились
гроші, потрібні для цього заміщення. Але саме це припущення
так само має силу для поточного року, як і для минулих років.

В обміні між І ($1000 v \dplus{} 1000 m$) і 2000 ІІ~$с$ треба насамперед зазначити,
що в сумі вартости І ($v \dplus{} m$) не міститься елементів сталої
вартости, отже, і не міститься жодного елемента вартости для заміщення
зношування, тобто для заміщення вартости, перенесеної з основної
складової частини сталого капіталу на ті товари, що в їхній натуральній
формі існує $v \dplus{} m$. Навпаки, в ІІ~$с$ цей елемент існує, і це є саме та
частина елемента вартости, що завдячує за своє існування основному капіталові,
що її не доводиться безпосередньо перетворювати з грошової
форми на натуральну, а повинна вона спочатку лишатись у грошовій
формі. Тому, розглядаючи обмін І ($1000 v \dplus{} 1000 m$) на 2000 ІІ~$с$, ми
одразу наражаємось на ті труднощі, що засоби продукції І, в натуральній
формі яких існують 2000 ($v \dplus{} m$), на всю суму їхньої вартости в
2000 треба обміняти на еквівалент у формі засобів споживання II, але,
з другого боку, засоби споживання 2000 II~$с$ не можна обміняти на засоби
продукції І ($1000 v \dplus{} 1000 m$) на всю суму їхньої вартости, бо
деяка частина їхньої вартости — рівна зношуванню, або втраті вартости
основного капіталу, що його треба замістити — спочатку повинна осісти
як гроші, що вже не функціонуватимуть знову як засоби циркуляції
протягом поточного річного періоду репродукції, тільки й розглядуваного
тут. Але гроші, що за їхньою допомогою перетворюється на гроші елемент
зношування, який міститься в товаровій вартості 2000 II~$с$, ці гроші
можуть походити тільки від І, бо II не може оплатити сам себе,
але оплачує себе лише через продаж свого товару, і тому, що згідно
з припущенням, І ($v \dplus{} m$) купує всю суму товарів 2000 ІІ~$с$; отже, цією
купівлею кляса І мусить перетворити на гроші для кляси II зазначене
вище зношування. Але, згідно з раніш викладеним законом, гроші, авансовані
для циркуляції, повертаються до капіталістичного продуцента,
який потім подає в циркуляцію таку саму кількість у товарах. Очевидно,
що І, купуючи ІІ~$с$, не може давати підрозділові II раз назавжди на
2000 товарами, і крім того ще додаткову грошову суму (давати так,
щоб вона не поверталась до нього через операцію обміну). Це взагалі
значило б, що І, купуючи товарову масу II~$с$, оплачує її понад ії вартість.
Коли II в обмін на свої $2000 с$ справді дістає І ($1000 v \dplus{} 1000 m$),
то він не має вимагати від І нічого більше, і гроші, які циркулювали
підчас цього обміну, повертаються до І або II, залежно від того, хто з
них подав ці гроші в циркуляцію, тобто, хто з них раніш виступив як
покупець\dots{} Разом із цим підрозділ II в такому разі перетворив би свій
товаровий капітал, на всю суму його вартости, знову в натуральну форму
засобів продукції, тимчасом як ми припустили, що деяка частина його,
\parbreak{}  %% абзац продовжується на наступній сторінці
