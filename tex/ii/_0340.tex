
\index{ii}{0340}  %% посилання на сторінку оригінального видання
Гроші, то спочатку функціонували для капіталіста як грошова
форма змінного капіталу, тепер функціонують у руках робітника як
грошова форма його заробітної плати, що її він перетворює на засоби
існування; отже, як грошова форма доходу, одержуваного ним від
завжди повторюваного продажу своєї робочої сили.

Тут перед нами лише той простий факт, що гроші покупця, в
даному разі капіталіста, з його рук переходять до рук продавця, в
даному разі продавця робочої сили, робітника. Тут не змінний капітал
двічі функціонує — як капітал для капіталіста і як дохід для робітника, —
і ті самі гроші, що спочатку існували в руках капіталіста як грошова
форма його змінного капіталу, отже, як потенціяльний змінний капітал, і
що потім, після того, як капіталіст перетворив їх на робочу силу, служать
у руках робітника як еквівалент проданої робочої сили. А те, що ті
самі гроші в руках продавця використовується інакше, ніж у руках покупця,
є явище властиве кожній купівлі та продажеві товарів.

Апологети-економісти фалшиво освітлюють справу, і це найкраще
видно, коли ми звернемо увагу виключно, — не турбуючись покищо про
дальші наслідки, — тільки на акт циркуляції $Г — Р$ ($\deq{} Г — Т$), перетворення
грошей на робочу силу на боці капіталістичного покупця, $Р — Г$ ($\deq{} Т — Г$),
перетворення товару робочої сили на гроші на боці продавця, робітника.
Вони кажуть; ті самі гроші реалізують тут два капітали; покупець —
капіталіст — перетворює свій грошовий капітал на живу робочу силу, що
її він долучає до свого продуктивного капіталу; з другого боку, продавець
— робітник — перетворює свій товар — робочу силу — на гроші й
витрачає їх як дохід, через що саме й може він знову й знов продавати
й таким чином зберігати свою робочу силу; отже, сама його робоча
сила є його капітал у товаровій формі і є постійне джерело його доходу.
А справді робоча сила є його здібність (яка постійно відновлюється,
репродукується), а не його капітал. Вона єдиний товар, що його він
постійно може й мусить продавати для того, щоб жити, і що діє як
капітал (змінний) лише в руках покупця, капіталіста. Коли якась людина
постійно мусить знову й знов продавати третій особі свою робочу силу,
тобто самого себе, то це, згідно з згаданими економістами, доводить,
що вона — капіталіст, бо їй завжди доводиться продавати „товар“ (саму
себе). В цьому розумінні й раб, хоч його раз назавжди продає як
товар третя особа, стає капіталістом, бо природа цього товару — робітникараба
— така, що покупець не тільки примушує його кожного дня робити,
а й дає йому ті засоби існування, що завдяки їм він може знову й
знов робити. — (Порівняй про це Сісмонді та Сея в листах до Малтуса).

2) Отже, те, що в обміні 1000 І~$v \dplus{} 1000$ І~$m$ на 2000 II~$с$ є сталий
капітал для одних (2000 II~$с$), стає змінним капіталом і додатковою
вартістю, тобто взагалі доходом для інших; а те, що є змінний капітал
і додаткова вартість 2000 І ($v \dplus{} m$), тобто, взагалі, доходом для одних,
стає сталим капіталом для інших.

Розгляньмо спочатку обмін I~$v$ на II~$с$, насамперед з погляду робітника.
