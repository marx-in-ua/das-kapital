\parcont{}  %% абзац починається на попередній сторінці
\index{ii}{0335}  %% посилання на сторінку оригінального видання
продукції, а 3000 — вартість засобів споживання. Отже, вартість суспільного
доходу ($v \dplus{} m$) становить тільки \sfrac{1}{3} вартости сукупного продукту,
і сукупність споживачів — робітники і капіталісти — лише на суму вартости
цієї третини можуть брати з цілого суспільного продукту товари,
продукти, і заводити їх у фонд свого споживання. Навпаки, 6000 \deq{} \sfrac{2}{3}
вартости продукту є вартість сталого капіталу, що його треба замістити
in natura. Отже, засоби продукції на таку суму треба знову ввести в
продукційний фонд. Неминучість цього бачив уже Шторх, хоч і не міг
довести цього: „Очевидно, що вартість річного продукту розкладається
почасти на капітал, почасти на зиск, і що кожна з цих частин вартости річного
продукту реґулярно купує продукти, потрібні нації так для підтримання
свого капіталу, як і для відновлення свого споживного фонду\dots{} Продукти,
що становлять капітал нації, не можуть споживатись“\footnote*{
„Il est clair que la valeur du produit annuel se distribue partie en capitaux
et partie en profits, et que chacune de ces portions de la valeur du produit annuel
va régulièrement acheter les produits dont la nation a besoin, tant pour entretenir
son capital que pour renouveler son fond consommable\dots{} les produits qui constituent
le capital d’une nation, ne sont point consommables“.
}. (Storch:
„Considérations sur la nature du revenu national“. Paris. 1824, p. 150).

Однак А.~Сміт подав цю казкову догму, — а їй і досі йметься
віру — не тільки у тій вже вище згаданій формі, що згідно з нею сукупна
вартість суспільного продукту розкладається на дохід, на заробітну
плату плюс додаткова вартість, або — як він каже — на заробітну плату
плюс зиск (процент) плюс земельна рента. Він подав її ще в популярнішій
формі, ніби споживачі, кінець-кінцем (ultimately), мусять оплатити
продуцентам усю вартість продукту. Це й досі лишається
одним з на віру прийнятих загальників або навіть однією з вічних істин
для так званої науки політичної економії. Цю думку хочуть унаочнити таким
на позір правдоподібним способом. Візьмімо якийбудь предмет, напр.,
полотняні сорочки. Насамперед прядун лянної пряжі повинен оплатити
льонівникові всю вартість льону, тобто насіння, добрива, корму для робочої
худоби і~\abbr{т. ін.}, а також ту частину вартости, що її основний
капітал льонівника, як от будівлі, сільсько-господарський реманент і~\abbr{т. ін.},
„передає продуктові; заробітну плату, виплачену протягом продукції
льону; додаткову вартість (зиск, земельну ренту), яка міститься в льоні;
нарешті, витрати на перевіз льону від місця його продукції до прядільні.
Потім ткач повинен повернути прядунові лянної пряжі не лише цю
ціну льону, а й ту частину вартости машин, будівель тощо, коротко,
основного капіталу, що її перенесено на льон; далі, всі зужитковані
в процесі прядіння допоміжні матеріяли, заробітну плату прядунів, додаткову
вартість і~\abbr{т. ін.} — і так само далі стоїть справа з білильником, з витратами
на транспорт готового полотна, нарешті, з фабрикантом сорочок,
який оплатив усю ціну всіх попередніх продуцентів, які дали йому те,
що для нього є лише сировинний матеріял. В його руках далі відбувається
долучення нової вартости: почасти вартости сталого капіталу, зужиткованого
\index{ii}{0336}  %% посилання на сторінку оригінального видання
в формі засобів праці, допоміжних матеріялів і~\abbr{т. ін.} при фабрикації
сорочок, почасти в наслідок витраченої на цю фабрикацію праці,
що долучає вартість заробітної плати робітників, які роблять сорочки,
плюс додаткова вартість фабриканта сорочок. Припустімо, що ввесь цей
продукт — сорочки коштують, кінець-кінцем, 100\pound{ ф. стерл.}, і що це є
та частина всієї вартости річного продукту, яку суспільство витрачає на
сорочки. Споживачі сорочок оплачують 100\pound{ ф. стерл.}, отже, вартість усіх
засобів продукції, що є в сорочках, а також заробітну плату плюс додаткова
вартість льонівника, прядуна, ткача, білильника, фабриканта сорочок,
а також і всіх транспортерів. Це цілком слушно. Це така справа, що й
дитина зрозуміє її. Але потім сказано: так само стоїть справа й щодо
вартости всіх інших товарів. Треба було б сказати: так само стоїть справа
й щодо вартости всіх засобів споживання, щодо вартости тієї
частини суспільного продукту, яка входить у фонд споживання, отже, з
тією частиною вартости суспільного продукту, яку можна витратити як
дохід. Сума вартости всіх цих товарів справді дорівнює вартості всіх
зужиткованих на них засобів продукції (сталих частин капіталу) плюс
вартість, утворена працею, долученою востаннє (заробітна плата плюс
додаткова вартість). Отже, сукупність споживачів може оплатити всю цю
суму вартости, бо хоч вартість кожного окремого товару складається
з $c \dplus{} v \dplus{} m$, але сума вартости всіх товарів, що входять у фонд
споживання, разом узята, в максимальній величині, може дорівнювати лише
тій частині вартости суспільного продукту, яка розкладається на $v \dplus{} m$,
тобто може дорівнювати лише тій вартості, що її долучила витрачена
протягом року праця до вже наявних засобів продукції, до вартости сталого
капіталу. Але щодо сталої капітальної вартости, то ми бачили, що
її заміщується з маси суспільного продукту двояким способом. Поперше,
через обмін капіталістів II, які продукують засоби споживання, з капіталістами
І, які продукують засоби продукції. Тут і є джерело тієї фрази,
ніби те, що для одних є капітал, для інших є дохід. Але справа в дійсності
стоїть не так. Ті 2000 II $с$, що існують у засобах споживання вартістю
в 2000, становлять для кляси капіталістів II сталу капітальну вартість.
Отже, сами капіталісти II не можуть спожити цю вартість, хоч продукт
за його натуральною формою і призначено для споживання. З другого
боку, 2000 І ($v \dplus{} m$) є спродукована клясою капіталістів і робітників І
заробітна плата плюс додаткова вартість. Вони існують у натуральній
формі засобів продукції, речей, що в них їхню власну вартість не можна
спожити. Отже, ми маємо тут суму вартости в 4000, що з них половина,
— і до обміну й після обміну — заміщує лише сталий капітал,
а друга половина становить лише дохід. Але, подруге, сталий капітал
підрозділу І заміщується in natura, почасти через обмін між капіталістами
І, почасти через заміщення in natura в кожному поодинокому підприємстві.

Фраза, ніби вся вартість річного продукту, кінець-кінцем, має бути
оплачена споживачами, була б правильна тільки тоді, коли б споживачів
мислили, як два цілком різні сорти: індивідуальних споживачів і
\parbreak{}  %% абзац продовжується на наступній сторінці
