
\index{ii}{0308}  %% посилання на сторінку оригінального видання
В обох випадках не лише сталий капітал II з форми продукту знову
перетворюється на натуральну форму засобів продукції, що в ній лише
й може він функціонувати як капітал; і так само не лише змінна частина
капіталу І перетворюється на грошову форму, а частина додаткової вартости
у формі засобів продукції І — на таку форму, в якій її можна спожити
як дохід. Крім цього, до II повертаються знову ті 500\pound{ ф. стерл.} грошового
капіталу, що їх він авансував на закуп засобів продукції, раніше ніж він
продав відповідну частину вартости сталого капіталу, яка існує у формі
засобів споживання і яка компенсує ці 500\pound{ ф. стерл.}; далі, до І повертаються
ті 500\pound{ ф. стерл.}, що їх він, антиципуючи продаж, витратив на
закуп засобів споживання. Коли до II повертаються назад гроші, авансовані
ним коштом сталої частини його товарового продукту, а до І — гроші,
авансовані коштом частини його товарового продукту, яка являє додаткозу
вартість, то лише тому, ще й та й друга категорія капіталістів пустили
в циркуляцію ще по 500\pound{ ф. стерл.} грошей: одна — крім наявного
в товаровій формі II сталого капіталу, друга — крім наявної в товаровій
формі І додаткової вартости. Кінець-кінцем, вони цілком поквитались одна
з однією, обмінявши відповідні товарові еквіваленти. Гроші, що їх вони
пустили в циркуляцію понад суму вартости їхніх товарів — як засоби обміну
цих товарів — повертаються до кожного з них частинами, пропорційно
тому, що кожен з них пустив у циркуляцію. В наслідок цього вони не
стали ані на шеляг багатші. II підрозділ мав сталий капітал \deq{} 2000 в формі
засобів споживання плюс 500 в грошах; тепер він має, як і раніш, 2000
в засобах продукції і 500 в грошах; так само І, як і раніш, має додаткову
вартість в 1000 (з товарів, засобів продукції, перетворених тепер
на споживний фонд) плюс 500 в грошах. Загальний висновок такий:
з тих грошей, що їх промислові капіталісти подають у циркуляцію, на
упосереднення своєї власної товарової циркуляції, — хоч їх подається
коштом сталої частини вартости товару, хоч коштом додаткової вартости,
яка існує в товарах, оскільки її витрачається як дохід — з цих грошей до
рук відповідних капіталістів повертається стільки, скільки вони авансували
на грошову циркуляцію.

Щодо зворотного перетворення на грошову форму змінного капіталу
кляси І, то він для капіталістів І, після того як вони витратили його на
заробітну плату, спочатку існує в тій товаровій формі, що в ній робітники
дали його їм. Капіталісти виплатили його робітникам у грошовій
формі як ціну робочої сили їх. В цьому розумінні вони сплатили ту
складову частину вартости їхнього товарового продукту, яка дорівнює
цьому змінному капіталові, витраченому в грошах. Тому вони — власники також
і цієї частини товарового продукту. Але застосована ними частина робітничої
кляси зовсім не є покупець засобів продукції, що їх вона сама
спродукувала. Вона — покупець засобів споживання, випродукуваних в II.~Отже, змінний капітал, авансований в грошах на оплату робочої сили, не
безпосередньо повертається до капіталістів І.~В наслідок актів купівлі,
що походить від робітників, він переходить до рук капіталістичних продуцентів
товарів, доконечних і взагалі приступних для робітничих кіл,
\parbreak{}  %% абзац продовжується на наступній сторінці
