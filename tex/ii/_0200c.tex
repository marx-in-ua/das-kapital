\parcont{}  %% абзац починається на попередній сторінці
\index{ii}{0200}  %% посилання на сторінку оригінального видання
разу він потребує авансування в 600\pound{ ф. стерл.} (капітал І). Період циркуляції
3 тижні; отже, період обороту, як і раніш, 9 тижнів. Капітал II
в 300\pound{ ф. стерл.} ввіходить у роботу протягом тритижневого періоду циркуляції
капіталу І.~Коли розглядати їх обидва, як капітали, незалежні
один від одного, то схема річного обороту матиме такий вигляд:

\begin{table}[H]
\centering
{\bfseries Таблиця II}
\caption*{Капітал І. 600\pound{ ф. стерл.}}
\bigskip
  \begin{tabular}{r r@{~}c r@{~}c c r@{~}c}
    \toprule
    & \multicolumn{2}{c}{Періоди обороту} & \multicolumn{2}{c}{Робочі періоди} & Авансовано & \multicolumn{2}{c}{Періоди циркуляції}\\
    \cmidrule(lr){2-3}
    \cmidrule(lr){4-5}
    \cmidrule(lr){6-6}
    \cmidrule(lr){7-8}

І.  & Тижні & 1\textendash{}9 & Тижні
    & 1\textendash{}6 & 600\pound{ ф. ст.}
    & Тижні & 7\textendash{}9\\

II. & \ditto{Тижні} & 10\textendash{}18 & \ditto{Тижні}
    & 10\textendash{}15 & 600\ditto{\pound{ ф. ст.}}
    & \ditto{Тижні} & 16\textendash{}18\\

III.& \ditto{Тижні} & 19\textendash{}27 & \ditto{Тижні} 
    & 19\textendash{}24 & 600 \ditto{\pound{ ф. ст.}}
    & \ditto{Тижні} & 25\textendash{}27\\

IV. & \ditto{Тижні} & 28\textendash{}36 & \ditto{Тижні}
    & 28\textendash{}33 & 600\ditto{\pound{ ф. ст.}}
    & \ditto{Тижні} & 34\textendash{}36\\

V.  & \ditto{Тижні} & 37\textendash{}45 & \ditto{Тижні} 
    & 37\textendash{}42 & 600\ditto{\pound{ ф. ст.}}
    & \ditto{Тижні} & 43\textendash{}45\\
VI. & \ditto{Тижні} & \hang{r}{46}\textendash{}\hang{l}{[54]} & \ditto{Тижні} 
    & 46\textendash{}51 & 600\ditto{\pound{ ф. ст.}}
    & \ditto{Тижні} & [52\textendash{}54]\\
  \end{tabular}

\caption*{Додатковий капітал II. 300\pound{ ф. стерл.}}
\bigskip
  \begin{tabular}{r r@{~}c r@{~}c c r@{~}c}
    \toprule
    & \multicolumn{2}{c}{Періоди обороту} & \multicolumn{2}{c}{Робочі періоди} & Авансовано & \multicolumn{2}{c}{Періоди циркуляції}\\
    \cmidrule(lr){2-3}
    \cmidrule(lr){4-5}
    \cmidrule(lr){6-6}
    \cmidrule(lr){7-8}

І.  & Тижні & \phantom{0}7\textendash{}15 & Тижні
    & 7\textendash{}9 & 300\pound{ ф. ст.}
    & Тижні & 10\textendash{}15\\

II. & \ditto{Тижні} & 16\textendash{}24 & \ditto{Тижні} 
    & 16\textendash{}18 & 300\ditto{\pound{ ф. ст.}}
    & \ditto{Тижні} & 19\textendash{}24\\

III.& \ditto{Тижні} & 25\textendash{}33 & \ditto{Тижні}
    & 25\textendash{}27 & 300\ditto{\pound{ ф. ст.}} 
    & \ditto{Тижні} & 28\textendash{}33\\

IV. & \ditto{Тижні} & 34\textendash{}42 & \ditto{Тижні} 
    & 34\textendash{}36 & 300\ditto{\pound{ ф. ст.}}
    & \ditto{Тижні} & 37\textendash{}42\\

V.  & \ditto{Тижні} & 43\textendash{}51 & \ditto{Тижні} 
    & 43\textendash{}45 & 300\ditto{\pound{ ф. ст.}}
    & \ditto{Тижні} & 45\textendash{}51\\
  \end{tabular}
\end{table}

\noindent{}Процес продукції відбувається цілий рік безперервно в однакових
розмірах. Обидва капітали І і II лишаються цілком відокремлені. Але
для того, щоб подати їх так відокремленими, нам довелось роз’єднати
їхні справжні схрещування й переплітання, а через це змінити й число
оборотів. А саме, згідно з вище наведеною таблицею, обертається:

\begin{center}
  
  \begin{tabular}{r@{~}l@{~}l}
    Капітал \phantom{І}І & 600 × 5\sfrac{2}{3} & \deq{} 3400\pound{ф. стерл.}\\

    \ditto{Капітал} II & 300 × 5 & \deq{} 1500\pound{ф. стерл.} \\
    \midrule
    отже, ввесь капітал & 900 × 5\sfrac{4}{9} & \deq{} 4900\pound{ф. стерл.}\\
  \end{tabular}
\end{center}

\noindent{}Але це неправильно, бо, як ми побачимо, справжні періоди продукції
та циркуляції не абсолютно збігаються з цими періодами вище наведеної
схеми, де головне було в тому, щоб подати обидва капітали, І і II, незалежними
один від одного.

В дійсності саме капітал II не має ані особливого робочого періоду, ані особливого
періоду циркуляції, відокремлених від цих періодів капіталу І.~Робочий
період триває 6 тижнів, період циркуляції 3 тижні. Що капітал II дорівнює
тільки 300\pound{ ф. стерл.}, то він може виповнити лише частину робочого
\parbreak{}  %% абзац продовжується на наступній сторінці
