
\index{ii}{0239}  %% посилання на сторінку оригінального видання
Напр., англійську бавовняну тканину або пряжу продається до Індії.
Припустімо, що купець-експортер платить англійському фабрикантові
бавовни (купець-експортер охоче робить це лише при доброму стані грошового
ринку. А якщо сам фабрикант поповнює свій грошовий капітал
за допомогою кредитових операцій, то справа вже кепська). Експортер
продає потім свій бавовняний товар на індійському ринку, відки йому
повертається авансований ним капітал. До цього повороту справа стоїть
цілком так само, як і в тому випадку, коли протяг робочого періоду
примушує авансувати новий грошовий капітал, щоб підтримувати провадження
процесу продукції в даному розмірі. Грошовий капітал, що ним
фабрикант платить своїм робітникам, а також відновлює всі інші елементи
свого капіталу, не є грошова форма спродукованої ним пряжі.
Це може статись лише тоді, коли вартість цієї пряжі повернеться в Англію
як гроші або продукт. Як і раніше, ці гроші є додатковий грошовий
капітал. Ріжниця лише в тому, що замість фабриканта їх авансує
купець, що, можливо, й сам здобув їх за допомогою кредитових операцій.
Так само, перш ніж ці гроші подається на ринок, або одночасно з цим,
на англійський ринок не подано додаткового продукту, що його можна
купити на ці гроші і ввести в сферу продуктивного або особистого споживання.
Коли такий стан триває довго і в широкому маштабі, то він
мусить зумовити такі самі наслідки, які раніше зумовлювало подовження
робочого періоду.

Можливо, що в самій Індії пряжу знову таки продається на кредит.
На цей кредит в Індії купують продукт і замість грошей за пряжу висилають
в Англію або переказують вексель на відповідну суму. Коли такий
стан триватиме довший час, то він справить тиск на індійський грошовий
ринок, а цей тиск відіб’ється в Англії так, що може спричинити
тут кризу. З свого боку криза, навіть коли вона сполучена з вивозом
благородних металів до Індії, спричиняє в цій країні нову кризу в наслідок
банкрутства англійських торгових домів та їхніх індійських філій,
що мали кредит в індійських банках. Так постає одночасна криза і на
тому ринку, що \emph{проти} нього торговельний баланс, і на тому, що на
\emph{користь} йому торговельний баланс. Це явище може бути ще складніше.
Напр., Англія надіслала в Індію срібні зливки, але англійські
кредитори Індії ставлять там тепер свої вимоги, і Індія муситиме скоро
по цьому надіслати свої срібні зливки назад в Англію.

Можливо, що вивізна торговля до Індії та довізна торговля з Індії
приблизно урівноважуються, хоч остання (за винятком особливих обставин,
як подорожчення бавовни тощо) в своїх розмірах визначається й
стимулюється першою. Торговельний баланс між Англією та Індією може
здаватись урівноваженим або виявляти лише незначні коливання в той
або інший бік. Але скоро криза вибухає в Англії, то виявляється, ще
на індійських складах лежать непродані бавовняні товари (отже, що вони
не перетворилися з товарового капіталу на грошовий капітал — перепродукція
на цьому боці), і що, з другого боку, в Англії не тільки лежать
непродані запаси англійських продуктів, але що більшу частину проданих
\parbreak{}  %% абзац продовжується на наступній сторінці
