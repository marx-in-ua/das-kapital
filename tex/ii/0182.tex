й ріжниці в часі, що на нього треба авансувати капітал, постають в самому процесі
продукції, як ріжниця між основним і поточним капіталом, ріжниця
робочих періодів тощо. Однак час обороту капіталу дорівнює сумі часу
його продукції і часу його обігу або циркуляції. Відси зрозуміло само
собою, що різний протяг часу обігу робить різним час обороту, а значить,
і протяг періоду обороту. Найнаочніше це буде або тоді, коли
порівняти два різні капіталовкладення, при чому різні тільки часи обігу,
а всі інші обставини, що модифікують оборот, однакові, або коли взяти
певний капітал певного складу щодо основного й поточного капіталу при
певному робочому періоді і т. ін., і гіпотетично зміняти тільки час
його обігу.

Один відділ часу обігу — і порівняно найважливіший — складається з
часу продажу, з того періоду, коли капітал перебуває в стані товарового
капіталу. Відповідно до відносної величини цього періоду подовжується
або скорочується час обігу, а тому й період обороту взагалі. В наслідок
витрат на зберігання тощо може бути потрібна й додаткова витрата капіталу.
Само собою зрозуміло, що час, потрібний для продажу готових
товарів, може бути дуже різний у різних капіталістів у тій самій галузі
підприємств; отже, цей час може бути різний не лише для мас капіталів,
вкладених у різні галузі продукції, а й для різних самостійних капіталів,
що в дійсності є лише усамостійнені частини сукупного капіталу, вкладеного
в ту саму продукційну сферу. За інших незмінних обставин період
продажу для того самого індивідуального капіталу буде змінюватись
разом із загальними коливаннями ринкових відносин, або разом із
коливаннями цих відносин в поодинокій галузі продукції. На цьому ми
не будемо тут більше зупинятись. Ми лише констатуємо простий факт:
всі обставини, що взагалі зумовлюють ріжницю в періодах обороту капіталів,
вкладених у різні галузі підприємств, мають наслідком, якщо ці
обставини впливають індивідуально (коли, напр., один капіталіст має змогу
продавати швидше, ніж його конкурент, коли один більш, ніж інший,
вживає методів, що скорочують робочі періоди тощо), так само ріжницю
в обороті різних індивідуальних капіталів, що перебувають в тій самій
галузі підприємств.

Одна з причин, що завжди зумовлюють ріжницю в часі продажу, а
тому і в часі обороту взагалі, є віддаленість ринку, де продається товар,
від місця, де його виготовлюється. *) Протягом цілого часу своєї подорожі
до ринку, капітал лишається зв’язаний в стані товарового капіталу;
коли товар продукують на замовлення, то — до часу здачі; коли не на замовлення,
то до часу подорожі його на ринок долучається ще той час,
що протягом його товар перебуває на ринку, чекаючи на продаж. Поліпшення
засобів зв’язку й транспорту скорочує мандрування товарів абсолютно,
але не знищує зумовлюваної цим мандруванням відносної ріжниці
в часі різних товарових капіталів або й різних частин того самого това-

*) В нім. тексті тут стоїть: „von ihrem Verkaufsplatz“, тобто: „від місця його продажу“.
Очевидна помилка. Ред.
