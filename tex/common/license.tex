\thispagestyle{empty}
\noindent{УДК 000.000(00)0000}

М00


\bigskip
{{\bfseries{}Маркс Карл}

М00 Капітал: Критика політичної економії / Заг. ред. А.~Потапова, І.~Зробок, Д.~Потапова. — \City, \Year. — \Biblio —
\lastpageref{pagesLTS.roman} + \lastpageref{pagesLTS.arabic} с.

{\footnotesize ISBN 000-000-0000-00-0(загальний)}

{\footnotesize \ISBN }

\bigskip
\noindent{В першій части своєї великої економічної праці про „Капітал“ стараєсь Карль Маркс вияснити передовсім, як повстає капітал? В тій ціли виказує він поперед усего, що єдиним жерелом усякої вартости є праця людська, котра з матеріялів сирих, даних природою, і при помочи сил природи витворює предмети вжиточні для чоловіка. Коли предмети такі витворюются не для власного вжитку самого витвірця, а для заміни за їнші, тоді вони звутся товарами. Капіталістична продукція полягає на витворюваню товарів, але не всяка продукція, де витворюются товарі, є вже капіталістична. До того потрібно ще одної дуже важної вимінки: \emph{щоби сама праця стала товаром}, т. є. щоб на торзі за певний товар (гроші) мож було заміняти (купити) працю людську.

Звичайно під назвою капіталу у нас розуміются беззглядно гроші. Се по части хибно. Гроші, як бачимо, тоді тілько стают капіталом, коли за них купуєся на торзі робуча сила.}


\hfill УДК 000.000(00)0000


}
\vfill
\noindent{\footnotesize Друкується за виданням: Карл Маркс. Капітал: Критика політичної економії — \FullSource.}

\bigskip
\noindent{\footnotesize Цей твір ліцензовано на умовах Ліцензіїї Creative Commons Із Зазначенням Авторства — Поширення На Тих Самих Умовах 4.0 Міжнародна. Щоб ознайомитися з копією цієї ліцензії, завітайте на http://creativecommons.org/licenses/by-sa/4.0/ або направте листа за адресою Creative Commons, PO Box 1866, Mountain View, CA 94042, USA.}

\bigskip

\noindent\begin{tabularx}{\textwidth}{ll}
{\footnotesize ISBN 000-000-0000-00-0(загальний)} & {\footnotesize \copyright Бульйон і Ко, упорядкування, 2014}\\

\end{tabularx}

\clearpage
