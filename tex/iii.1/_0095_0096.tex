\parcont{}  %% абзац починається на попередній сторінці
\index{iii1}{0095}  %% посилання на сторінку оригінального видання
екскрементів виробництва, так званих відпадів його, в нові елементи
виробництва чи тієї самої, чи якоїсь іншої, галузі промисловості,
процеси, за допомогою яких ці так звані екскременти
кидаються знову в кругобіг виробництва, а тому й споживання —
продуктивного чи особистого. І ця галузь заощаджень, на якій
ми пізніше спинимось трохи ближче, є результат суспільної
праці у великому масштабі. Саме відповідна цій останній масовість
цих відпадів робить з них самих знову предмет торгівлі,
отже й нові елементи виробництва. Тільки як відпади колективного
виробництва і, отже, виробництва у великому масштабі, набирають
вони цього значення для процесу виробництва, лишаються
носіями мінової вартості. Ці відпади — незалежно від тієї служби,
яку вони виконують як нові елементи виробництва, — зменшують
у тій мірі, в якій вони знову можуть бути продані, витрати на
сировинний матеріал, до яких завжди зараховуються нормальні
відпади цього матеріалу, а саме, та кількість їх, яка пересічно
мусить бути втрачена при його обробленні. Зменшення витрат
на цю частину сталого капіталу підвищує pro tanto [відповідно
до цього] норму зиску при даній величині змінного капіталу
і даній нормі додаткової вартості.

Якщо додаткова вартість дана, норма зиску може бути
збільшена тільки шляхом зменшення вартості сталого капіталу,
потрібного для виробництва товару. Оскільки сталий капітал
входить у виробництво товарів, остільки значення має не його
мінова вартість, а виключно його споживна вартість. Скільки
праці може ввібрати в себе льон на якійсь прядільні, залежить
не від його вартості, а від його кількості, якщо дано рівень
продуктивності праці, тобто рівень технічного розвитку. Так
само та допомога, яку машина дає, наприклад, трьом робітникам,
залежить не від її вартості, а від її споживної вартості як машини.
На одному ступені технічного розвитку погана машина може бути
дорогою, на другому — добра машина може бути дешевою.

Підвищений зиск, який капіталіст одержує в наслідок того,
що, наприклад, подешевшали бавовна й прядільні машини, є результат
підвищеної продуктивності праці, правда, не в прядільному
виробництві, а у виробництві машин і бавовни. Для того,
щоб упредметнити дану кількість праці, отже, привласнити дану
кількість додаткової праці, тепер потрібно менше видатків на
умови праці. Зменшуються витрати, потрібні для того, щоб привласнити
певну кількість додаткової праці.

Ми вже казали про те заощадження, яке постає в процесі
виробництва в наслідок спільного застосування засобів виробництва
колективним робітником — суспільно-комбінованим робітником.
Дальші заощадження на видатках сталого капіталу, що
виникають із скорочення часу циркуляції (де розвиток засобів
сполучення є істотний матеріальний момент), ми розглянемо
нижче. Але вже тут треба ще згадати про ту економію, яка
походить з безперервного поліпшення машин, а саме: 1) з поліпшення
\index{iii1}{0096}  %% посилання на сторінку оригінального видання
їх матеріалу, наприклад, з заміни дерева залізом; 2) із
здешевлення машин у наслідок поліпшення фабрикації машин
взагалі; так що, хоч вартість основної частини сталого капіталу
безперервно зростає з розвитком праці у великому масштабі,
вона зростає далеко не в такій самій мірі;\footnote{
Див. Ure про прогрес у будуванні фабрик.
} 3) з спеціальних поліпшень,
які дозволяють уже наявним машинам працювати дешевше
і ефективніше, наприклад, з поліпшення парових казанів
і т. п., про що деякі подробиці дамо пізніш; 4) із зменшення
відпадів у наслідок поліпшених машин.

Все, що зменшує зношування машин і взагалі основного капіталу
за даний період виробництва, не тільки здешевлює окремий
товар, бо кожний окремий товар репродукує в своїй ціні
відповідну частину зношування, що припадає на нього, але
й зменшує відповідні видатки капіталу за цей період. Ремонтні
роботи і т. п., в тій мірі, в якій вони стають потрібні, зараховуються
при обчисленні до первісних витрат на машини. їх зменшення,
в наслідок більшої міцності машин, зменшує pro tanto
[відповідно до цього] ціну машин.

Про всяку економію цього роду знов таки здебільшого можна
сказати, що вона можлива тільки для комбінованого робітника
і часто може здійснитись тільки при роботах в ще більшому
масштабі; що вона, отже, вимагай ще більшої комбінації робітників
безпосередньо в процесі виробництва.

Але, з другого боку, розвиток продуктивної сили праці
в \emph{одній} галузі виробництва, наприклад, у виробництві заліза,
вугілля, машин, у будівельній справі і~\abbr{т. д.}, який у свою чергу
почасти може залежати від успіхів у сфері інтелектуального
виробництва, а саме природничих наук і їх застосування, являє
тут собою умову зменшення вартості засобів виробництва, а тому
й витрат на них в \emph{інших} галузях промисловості, наприклад,
у текстильній промисловості або землеробстві. Це зрозуміло
само собою, бо товар, який як продукт виходить з однієї галузі
промисловості, знову входить в іншу як засіб виробництва.
Більша чи менша дешевина товару залежить від продуктивності
праці в тій галузі виробництва, з якої він виходить як продукт,
і разом з цим вона є умовою не тільки здешевлення тих товарів,
у виробництво яких він входить як засіб виробництва, але
й умовою зменшення вартості сталого капіталу, елементом якого
він тут стає, отже, і умовою підвищення норми зиску.

Характерне для цього роду економії на сталому капіталі,
яка походить з прогресивного розвитку промисловості, є те,
що тут підвищення норми зиску в \emph{одній} галузі промисловості
спричинюється розвитком продуктивної сили праці в \emph{іншій} галузі.
Те, що тут іде на користь капіталістові, є знов таки вигода,
яка є продуктом суспільної праці, хоч і не продуктом робітників,
експлуатованих безпосередньо самим цим капіталістом.
\parbreak{}  %% абзац продовжується на наступній сторінці
