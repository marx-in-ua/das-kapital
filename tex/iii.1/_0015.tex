\parcont{}  %% абзац починається на попередній сторінці
\index{iii1}{0015}  %% посилання на сторінку оригінального видання
бажає дати прямого розв’язання його. Він каже: „Розв’язання
цієї суперечності“ (між законом вартості Рікардо--Маркса та однаковою
пересічною нормою зиску) „неможливе, якщо розглядати
різні види товарів \emph{окремо}, і якщо (їх вартість має бути
рівна їх міновій вартості, а ця остання — рівна або пропорціональна
їх ціні“. За Лексісом розв’язання можливе лише тоді,
коли „відмовитись для окремих видів товару від вимірювання
вартості працею і мати на увазі тільки виробництво товарів
в \emph{цілому} та розподіл їх між цілими класами капіталістів і робітників\dots{}
З сукупного продукту робітничий клас одержує тільки
певну частину\dots{} друга частина, що припадає класові капіталістів,
становить додатковий продукт у Марксовому розумінні, а тому і\dots{}
додаткову вартість. Потім, члени класу капіталістів розподіляють
між собою всю цю додаткову вартість \emph{не} відповідно до числа
занятих ними робітників, а пропорціонально величині капіталу,
вкладеного кожним з них, при чому й земля береться до розрахунку
як капітальна вартість“. Ідеальні вартості Маркса, визначувані
одиницями праці, втіленими в товарах, не відповідають
цінам, але можуть „розглядатися як вихідний пункт зрушення,
яке приводить до дійсних цін. Останні зумовлюються тим, що
рівновеликі капітали вимагають рівновеликих баришів“. В наслідок
цього одні капіталісти одержать за свої товари ціни вищі,
ніж ідеальні вартості цих товарів, а інші одержать ціни нижчі.
„Але через те що втрати і прибавки до додаткової вартості
взаємно знищуються в межах класу капіталістів, то сукупна
величина додаткової вартості є така сама, як коли б усі ціни
були пропорціональні ідеальним вартостям товарів“.

Як бачимо, питання тут далеко не розв’язане, але, хоча й
розпливчасто і поверхово, однак в цілому \emph{поставлене} правильно.
І це дійсно більше, ніж ми можемо чекати від будь-кого,
хто, подібно до автора, з певною гордістю зачисляє себе
до „вульґарних економістів“; це прямо дивовижно, коли порівняти
з працями інших вульґарних економістів, про які мова буде
пізніше. Правда, вульґарна економія автора особливого роду.
Він каже, що бариш на капітал \emph{міг би}, звичайно, бути виведений
за способом Маркса, але ніщо не \emph{примушує} до такого розуміння.
Навпаки, вульґарна економія має спосіб пояснення, принаймні
більш правдоподібний: „капіталістичні продавці, виробник
сировинного матеріалу, фабрикант, гуртовий торговець,
роздрібний торговець одержують бариш від своїх підприємств,
продаючи кожний дорожче, ніж він купує, отже, підвищуючи
на певний процент собівартість (Selbstkostenpreis) свого товару.
Тільки робітник не спроможний зробити подібної надбавки до
вартості; в наслідок свого несприятливого становища супроти
капіталіста він примушений продавати свою працю за таку ціну,
якої вона коштує йому самому, а саме за необхідні засоби існування\dots{}
таким чином ці надбавки до ціни зберігають своє повне
значення відносно купуючих найманих робітників і спричиняють
\parbreak{}  %% абзац продовжується на наступній сторінці
