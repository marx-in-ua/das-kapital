\parcont{}  %% абзац починається на попередній сторінці
\index{iii1}{0036}  %% посилання на сторінку оригінального видання
світу. Купець середньовіччя аж ніяк не був індивідуалістом, він був по суті членом громади (Genossenschafter), як і всі його
сучасники. На селі панувала громада-марка, яка виросла з первісного комунізму. Кожний селянин первісно мав однакової
величини наділ з рівновеликими ділянками землі кожної якості і відповідно однакову участь в правах у громаді-марці. З того
часу, як громада-марка стала замкнутою і припинилось виділення нових наділів, почалося в наслідок спадкування тощо
подрібнення наділів і відповідно до цього подрібнення участі в правах марки; але одиницею землекористання лишався повний
наділ, так що були половини, чверті, восьмі частини наділу з половиною, чвертю і восьмою частиною участі в правах спільної
марки. На зразок громади-марки утворювались всі пізніші промислові товариства, насамперед міські цехи, устрій
яких був не чим іншим, як застосуванням устрою марки до ремісницьких привілеїв, а не до обмеженої території. Центральним
пунктом усієї організації була рівна участь кожного члена в гарантованих цехові привілеях і доходах, як це ще ясно видно з
привілею жителів Ельберфельда й Бармена на „прокорм з пряжі“ 1527 року (\emph{Thun}: „Industrie am Niederrhein“,  II, 164 і далі).
Те саме стосується до гірничої промисловості, де так само кожен пай давав однакову частку в доходах і, так само як наділ
члена марки, міг дрібнитися разом із зв’язаними з ним правами та обов’язками. В не меншій мірі це стосується і до купецьких
товариств, які викликали до життя заморську торгівлю. Венеціанці і генуезці в гаванях Александрії або Константинополя, кожна
„нація“ у своєму власному купецькому дворі (Fondaco), який складався з житла, ресторану, складу, виставки і крамниці з
центральним бюро, являли собою повні торговельні товариства, вони були відгороджені від конкурентів і покупців, вони
продавали по цінах, які вони встановлювали між собою, їхні товари мали певну якість, гарантовану громадським контролем, а
часто й клеймом, вони спільно визначали, які ціни платити тубільцям за їхні продукти і т. д. Те саме робили ганзейці на
німецькому мосту (Tydske Bryggen) у Бергені в Норвегії, а також їх голландські і англійські конкуренти. Горе тому, хто
продасть дешевше або купить дорожче призначеної ціни! Бойкот, якого він зазнавав, означав тоді неминуче розорення, не кажучи
вже про прямі штрафи, які товариство накладало на винного. Але, крім того, засновувались також ще тісніші товариства з
певними цілями, як от Маона в Генуї,—  товариство, яке прртягом багатьох років володіло галуновими покладами Фокеї в Малій
Азії та на острові Хіосі в XIV і XV  століттях, далі — велике Равенсбергське торговельне товариство, яке з кінця XIV
століття вело торговельні справи з Італією та Іспанією і засновувало там свої філії, потім — німецьке товариство
аугсбургських купців Фуггера, Вельзера, Феліна, Гекштеттера і т. д. і нюрнбергських купців Гіршфогеля та інших, яке
\parbreak{}  %% абзац продовжується на наступній сторінці
