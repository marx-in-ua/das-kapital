\parcont{}  %% абзац починається на попередній сторінці
\index{iii1}{0128}  %% посилання на сторінку оригінального видання
400\pound{ фунтів стерлінгів}. Далі, через те що сталий капітал вартістю в 2000\pound{ фунтів стерлінгів} потребує
для свого функціонування 500 робітників, то 400 робітників можуть привести в рух тільки сталий
капітал вартістю в 1600\pound{ фунтів стерлінгів}. Отже,
для того, щоб виробництво і далі провадилося в попередніх
розмірах і щоб \sfrac{1}{5} машин не стояла без діла, змінний капітал
мусить бути підвищений на 100\pound{ фунтів стерлінгів}, щоб, як і раніш, вживати 500 робітників; а цього
можна досягти тільки за
допомогою того, що вільний досі капітал зв’язується, при чому
та частина нагромадження, яка повинна була б служити для розширення виробництва, тепер служить
тільки для поповнення, або ж
до попереднього капіталу додається та частина, яка призначена
була для витрачання як дохід. Із збільшеною на 100\pound{ фунтів
стерлінгів} витратою змінного капіталу тепер виробляється на
100\pound{ фунтів стерлінгів} менше додаткової вартості. Щоб привести
в рух те саме число робітників, потрібно більше капіталу, і разом
з тим зменшується додаткова вартість, яку дає кожний окремий робітник.

Вигоди, які випливають із звільнення, і втрати, які випливають із зв’язування змінного капіталу,
існують тільки для капіталу, який уже вкладений і який, отже, репродукується при даних відношеннях.
Для нововкладуваного капіталу вигоди на
одному боці, втрати на другому зводяться до підвищення або
зниження норми додаткової вартості і відповідної, хоч і зовсім
не пропорціональної, зміни норми зиску.

\pfbreak

Щойно досліджене звільнення і зв’язування змінного капіталу є наслідок зниження вартості або
підвищення вартості
елементів змінного капіталу, тобто витрат репродукції робочої
сили. Але змінний капітал може звільнятися й тоді, коли внаслідок розвитку продуктивної сили, при
незмінній нормі заробітної плати, потрібно менше робітників для того, щоб привести
в рух ту саму масу сталого капіталу. Так само, навпаки, зв’язування додаткового змінного капіталу
може мати місце, якщо
в наслідок зниження продуктивної сили праці потрібно більше
робітників для тієї самої маси сталого капіталу. Якщо ж, з другого боку, частина капіталу, який
раніш застосовувався як змінний капітал, застосовується тепер у формі сталого капіталу, отже, якщо
відбувається тільки зміна розподілу між складовими
частинами того самого капіталу, то, хоч це і справляє вплив
на норму додаткової вартості й зиску, але не належить до розглядуваної тут рубрики зв’язування і
звільнення капіталу.

Сталий капітал, як ми вже бачили, також може зв’язуватись
або звільнятись в наслідок підвищення вартості або зниження
вартості тих елементів, з яких він складається. Залишаючи це
осторонь, зв’язування його можливе (без перетворення будь-якої
частини змінного капіталу в сталий) тільки тоді, коли збільшується
\index{iii1}{0129}  %% посилання на сторінку оригінального видання
продуктивна сила праці, отже, коли та сама маса праці
створює більше продукту і тому приводить в рух більше сталого капіталу. Те саме при певних
обставинах може мати місце
тоді, коли продуктивна сила зменшується, як, наприклад, у землеробстві, так що та сама кількість
праці потребує для створення
того самого продукту більше засобів виробництва, наприклад,
більше насіння або добрива, дренування і~\abbr{т. д.} Без знецінення
сталий капітал може звільнятись тоді, коли в наслідок удосконалень, застосовування сил природи і~\abbr{т.
ін.} сталий капітал меншої вартості стає спроможним технічно виконувати ту саму службу, яку раніше
виконував капітал вищої вартості.

У книзі II ми бачили, що після того як товари перетворені
в гроші, продані, певна частина цих грошей знову мусить бути
перетворена в речові елементи сталого капіталу і саме в тих
пропорціях, яких вимагає певний технічний характер кожної
даної сфери виробництва. Щодо цього в усіх галузях — залишаючи осторонь заробітну плату, отже,
змінний капітал — найважливішим елементом є сировинний матеріал, включаючи й допоміжні матеріали,
які особливо важать у тих галузях виробництва, в які не входить сировинний матеріал у власному
значенні, як от у копальнях і добувній промисловості взагалі. Та частина ціни, яка мусить замістити
зношування машин, поки
машини ще взагалі здатні функціонувати, входить в обрахунок
більше ідеально; не має особливого значення, коли саме ця
частина буде оплачена й заміщена грішми, сьогодні чи завтра,
чи в якийсь інший період часу обороту капіталу. Інакше стоїть
справа з сировинним матеріалом. Якщо ціна сировинного матеріалу підвищується, то може стати
неможливим, після відрахування заробітної плати, цілком замістити ціну його з вартості товару. Тому
сильні коливання цін викликають перерви, великі
колізії і навіть катастрофи в процесі репродукції. Продукти
землеробства у власному розумінні слова, сировинні матеріали,
які походять з органічної природи, особливо підпадають таким
коливанням вартості в наслідок мінливих врожаїв і~\abbr{т. д.} — кредитну систему ми тут ще цілком
залишаємо осторонь. Та сама
кількість праці може тут в наслідок непіддатних контролеві природних умов, сприятливості чи
несприятливості діб року та ін.,
виражатися в дуже різних кількостях споживних вартостей,
і тому певна кількість цих споживних вартостей матиме дуже
різну ціну. Якщо вартість x представлена в 100 фунтах товару $a$, то ціна одного фунта $a = \frac{x}{100}$; якщо
ж вона представлена в 1000 фунтах $а$, то ціна одного фунта $a = \frac{x}{1000}$ і~\abbr{т. д.} Такий, отже, є один
елемент цих коливань ціни сировинного матеріалу.
Другий елемент, про який тут згадується тільки для повноти, — бо конкуренція, як і кредитна система,
лежить тут поки що поза
межами нашого розгляду, — є такий: з самої природи речей
рослинні й тваринні речовини, ріст і виробництво яких підлягають
\parbreak{}  %% абзац продовжується на наступній сторінці
