\parcont{}  %% абзац починається на попередній сторінці
\index{iii1}{0149}  %% посилання на сторінку оригінального видання
Якщо вартість золота падає чи підвищується\footnote*{
В першому німецькому виданні тут сказано: „підвищується чи падає“;
виправлено на підставі рукопису Маркса. \emph{Примітка ред. нім. вид. ІМЕЛ.}
} на 100\%, то
в першому випадку той самий капітал, який раніше був вартий
100\pound{ фунтів стерлінгів}, буде вартий 200\pound{ фунтів стерлінгів}, а зиск
матиме вартість в 40\pound{ фунтів стерлінгів}, тобто виражатиметься
в цій грошовій сумі замість колишніх 20\pound{ фунтів стерлінгів}.
В другому випадку капітал падає до вартості в 50\pound{ фунтів
стерлінгів}, і зиск виражається в продукті вартістю в 10\pound{ фунтів
стерлінгів}. Але в обох випадках 200 : 40 \deq{} 50 : 10 \deq{} 100 : 20 \deq{} 20\%.
Однак, в усіх цих випадках в дійсності не сталося б ніякої
зміни у величині капітальної вартості, сталася б зміна тільки
в грошовому виразі тієї самої вартості і тієї самої додаткової
вартості. Отже, це не могло б вплинути і на \frac{m}{K}, або на норму
зиску.

Другий випадок — той, коли має місце дійсна зміна величини
вартості, але зміна ця не супроводиться зміною відношення
$v : c$, тобто коли при незмінній нормі додаткової вартості відношення
капіталу, витраченого на робочу силу (розглядаючи
змінний капітал як показник приведеної в рух робочої сили), до
капіталу, витраченого на засоби виробництва, лишається те
саме. При таких умовах, якщо ми маємо $К$, чи $nK$, чи \frac{K}{n}, наприклад,
1000, чи 2000, чи 500, зиск, при нормі зиску в 20\%,
буде в першому випадку \deq{} 200, в другому \deq{} 400, в третьому \deq{} 100;
але \frac{200}{1000} \deq{} \frac{400}{2000} \deq{} \frac{100}{500} \deq{} 20\%. Тобто норма зиску лишається тут
незмінною, бо склад капіталу лишається той самий і не зачіпається
зміною величини капіталу. Тому збільшення чи зменшення
маси зиску вказує тут тільки на збільшення чи зменшення
величини застосовуваного капіталу.

Отже, в першому випадку має місце тільки позірна зміна
величини застосовуваного капіталу, в другому випадку відбувається
дійсна зміна величини, але не відбувається ніякої зміни
в органічному складі капіталу, не відбувається ніякої зміни
відношення змінної частини капіталу до його сталої частини.
Але, за винятком цих обох випадків, зміна величини застосовуваного
капіталу є або \emph{наслідок} попередньої зміни вартості
однієї з його складових частин, а тому (оскільки із зміною змінного
капіталу не змінюється сама додаткова вартість) і зміни
у відносній величині його складових частин; або ця зміна величини
капіталу (як при роботах у великому масштабі, введенні нових
машин і~\abbr{т. д.}) є \emph{причина} зміни у відносній величині обох його
органічних складових частин. Тому в усіх цих випадках при
інших однакових умовах зміна величини застосовуваного капіталу
мусить супроводитись одночасною зміною норми зиску.
