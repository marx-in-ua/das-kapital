\parcont{}  %% абзац починається на попередній сторінці
\index{iii1}{0192}  %% посилання на сторінку оригінального видання
репродукції в наслідок нагромадження капіталу, то, при інших
незмінних умовах, потрібна додаткова кількість бавовни. Те саме
і щодо засобів існування. Робітничий клас, для того, щоб і далі
жити при звичайних пересічних умовах, мусить діставати принаймні
попередню кількість необхідних засобів існування, хоч,
може, і розподілених дещо інакше між різними сортами товарів;
якщо ж узяти до уваги щорічний ріст населення, то потрібна
ще певна додаткова кількість засобів існування; те саме
з більшими чи меншими модифікаціями можна сказати і щодо
інших класів.

Отже, здається, ніби на стороні попиту є певна, даної величини
суспільна потреба, яка для свого задоволення вимагає
певної кількості товару на ринку. Але кількісна визначеність
цієї потреби цілком еластична й хитка. Вона тільки здається
фіксованою. Якби засоби існування були дешевші або грошова
заробітна плата була вища, то робітники купували б більше
засобів існування і виявилася б більша „суспільна потреба“ на
ці сорти товарів, — при чому ми зовсім залишаємо осторонь
пауперів і~\abbr{т. д.}, „попит“ яких стоїть нижче найвужчих меж їх
фізичної потреби. Коли б, з другого боку, подешевшала, наприклад,
бавовна, то попит капіталістів на бавовну виріс би, в бавовняну
промисловість було б вкладено більше додаткового
капіталу і~\abbr{т. д.} При цьому взагалі не слід забувати, що попит
на продуктивне споживання при нашому припущенні є попит
з боку капіталіста і що справжня мета капіталіста є виробництво
додаткової вартості, так що він тільки з цією метою
виробляє певний сорт товарів. З другого боку, це не перешкоджає
тому, що капіталіст, оскільки він виступає на ринку як
покупець, наприклад, бавовни, репрезентує потребу в бавовні,
адже і для продавця бавовни байдуже, чи перетворює покупець
цю бавовну в сорочки, в бавовняний порох, чи має намір
затикати нею вуха собі і всьому світові. Але в усякому разі це
справляє великий вплив на те, якого роду покупець він є. Його
потреба в бавовні істотно модифікується тією обставиною, що
в дійсності вона тільки приховує його потребу добувати зиск. —
Межі, в яких репрезентована на \emph{ринку} потреба в товарах —
попит — кількісно відрізняється від \emph{дійсної суспільної} потреби,
звичайно, дуже різні для різних товарів; я маю на увазі ріжницю
між кількістю товарів, на яку є попит, і тією кількістю
їх, на яку був би попит при інших грошових цінах товарів або
при інших грошових або життьових умовах покупців.

Нема нічого легшого, як зрозуміти нерівномірності попиту
й подання та відхилення, що випливають звідси, ринкових цін
від ринкових вартостей. Справжня трудність полягає у визначенні
того, що слід розуміти під висловом: попит і подання
покриваються.

Попит і подання покриваються, якщо вони стоять у такому
відношенні одне до одного, що товарна маса певної галузі виробництва
\index{iii1}{0193}  %% посилання на сторінку оригінального видання
може бути продана по її ринковій вартості, — ні вище,
ні нижче. Ось перше, що нам кажуть.

Подруге: якщо товари можуть бути продані по їх ринковій
вартості, то попит і подання покриваються.

Якщо попит і подання взаємно покриваються, то вони перестають
діяти, і саме тому товари продаються по їх ринковій вартості.
Якщо дві сили рівномірно діють у протилежних напрямах,
то вони одна одну знищують, зовсім не діють назовні, і явища,
які відбуваються при цій умові, мусять бути пояснені якось
інакше, а не діянням цих двох сил. Якщо попит і подання взаємно
знищуються, то вони перестають щонебудь пояснювати,
не діють на ринкову вартість і залишають нас у цілковитому
невіданні того, чому ринкова вартість виражається саме в цій
сумі грошей, а не в будьякій іншій. Дійсні внутрішні закони
капіталістичного виробництва, очевидно, не можуть бути пояснені
з взаємодіяння попиту й подання (цілком незалежно від
глибшого аналізу цих двох суспільних рушійних сил, який сюди
не стосується), бо ці закони тільки тоді виявляються здійсненими
в чистому вигляді, коли попит і подання перестають
діяти, тобто взаємно покриваються. В дійсності попит і подання
ніколи не покриваються або, якщо і покриваються, то тільки
випадково, — отже, з наукового погляду такі випадки слід прирівняти
до нуля і розглядати як неіснуючі. Але в політичній
економії припускається, що вони покриваються. Чому? Це робиться
для того, щоб розглядати явища в їх закономірному, відповідному
їх поняттю вигляді, тобто розглядати їх незалежно від
того, якими вони здаються в наслідок руху попиту й подання.
З другого боку, для того, щоб знайти дійсну тенденцію їх руху, так
би мовити, фіксувати її. Бо відхилення від рівності мають протилежний
характер і, через те що вони завжди йдуть одне за одним,
вони урівноважуються завдяки своїм протилежним напрямам, завдяки
своїй суперечності. Отже, якщо попит і подання не покриваються
ні в одному випадку, то їх відхилення від рівності йдуть одне
за одним таким чином, — результат відхилення в одному напрямі
є той, що воно викликає відхилення в протилежному напрямі, —
що, коли розглядати підсумок руху за більш-менш довгий період
часу, подання і попит постійно покриваються; однак, вони покриваються
тільки як пересічне минулих уже коливань, тільки як
постійний рух їх суперечності. В наслідок цього ринкові ціни,
що відхиляються від ринкових вартостей, розглядувані щодо
їх пересічної, вирівнюються в ринкові вартості, при чому відхилення
від цих останніх взаємно знищуються як плюс і мінус.
І ця пересічна має не тільки теоретичне значення, вона має
також і практичну важливість для капіталу, вкладення якого розраховане
на коливання й вирівнювання протягом більш-менш певного
періоду часу.

Тому відношення між попитом і поданням пояснює, з одного
боку, тільки відхилення ринкових цін від ринкових вартостей
\parbreak{}  %% абзац продовжується на наступній сторінці
