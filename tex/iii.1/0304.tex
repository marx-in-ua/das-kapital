пророблюються з їх капіталами, і вирівненням загальної норми
зиску. 42 Конкуренція для цих голів неминуче грає також цілком
перекручену роль. Якщо межі вартості й додаткової вартості
дані, то легко зрозуміти, яким чином конкуренція капіталів перетворює
вартості в ціни виробництва і потім у торговельні ціни
а додаткову вартість — у пересічний зиск. Але, не знаючи цих
меж, абсолютно не можна зрозуміти, чому конкуренція приводить
загальну норму зиску до цієї, а не до іншої межі, до 15\%, а не
до 1500\%. Адже вона, щонайбільше, може приводити її до одного
рівня. Але в ній немає абсолютно ніякого елементу, який визначав
би самий цей рівень.

Отже, з точки зору купецького капіталу сам оборот виступає
так, ніби він визначає ціни. З другого боку, тимчасом як швидкість
обороту промислового капіталу, оскільки вона дає можливість
даному Капіталові експлуатувати більше чи менше праці, визначально
і обмежувально впливає на масу зиску, а тому й на
загальну норму зиску, — для торговельного капіталу норма зиску
є дана іззовні, і внутрішній зв’язок її з утворенням додаткової
вартості цілком стертий. Якщо той самий промисловий капітал
при інших незмінних умовах і особливо при однаковому органічному
складі обертається протягом року чотири рази замість
двох, то він виробляє удвоє більше додаткової вартості, а тому
й зиску; і це наочно виявляється, якщо і поки цей капітал володіє
монополією поліпшеного способу виробництва, який дає
йому можливість так прискорювати оборот. Навпаки, різна тривалість
обороту в різних галузях торгівлі виявляється в тому,
що зиск, одержуваний на оборот певного товарного капіталу,
стоїть у зворотному відношенні до числа оборотів грошового
капіталу, за допомогою якого обертається цей товарний капітал.
Small profits and quick returns [невеликі зиски і швидкі обороти]
являють собою для shopkeeper’a [крамаря] саме той принцип,
якого він додержується з принципу.

Зрештою, само собою зрозуміло, що цей закон оборотів купецького
капіталу в кожній галузі торгівлі, — і залишаючи осторонь
чергування швидших і повільніших оборотів, які один одного компенсують,
— має силу тільки для пересічних оборотів усього купецького
капіталу, вкладеного в цю галузь. Капітал А, який функціонує
в тій самій галузі, що й В, може робити більше чи менше оборотів
порівняно з пересічним числом оборотів. В цьому випадку
інші капітали роблять менше чи більше оборотів. Це нічого не
змінює в обороті загальної маси купецького капіталу, вкладеної
в цю галузь. Але це має вирішально важливе значення для окре-

42 Ось одно дуже наївне, але разом з тим дуже правильне зауваження: „Тому,
безперечно, причиною тієї обставини, що один і той же товар можна одержати
у різних продавців по цілком різних цінах, дуже часто є неправильна калькуляція“
(Feller und Odermann: „Das Ganze der kaufmännischen Arithmetik“, 7 вид.,
1859 [стор. 451]). Це показує, як визначення ціни стає чисто теоретичним,
тобто абстрактним.
