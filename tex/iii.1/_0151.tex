
\index{iii1}{0151}  %% посилання на сторінку оригінального видання
\chapter{Перетворення зиску в пересічний зиск}

\section{Різний склад капіталу в різних галузях виробництва
і відмінність в нормах зиску, яка випливає звідси}

В попередньому відділі між іншим було показано, яким чином
при незмінній нормі додаткової вартості може змінюватись, підвищуватись
чи падати норма зиску. В цьому розділі ми припускаємо,
що ступінь експлуатації праці, а тому й норма додаткової
вартості і довжина робочого дня в усіх сферах виробництва,
на які розпадається суспільна праця даної країни, є однакової величини,
однакової висоти. Відносно багатьох відмінностей в експлуатації
праці в різних сферах виробництва вже А.~Сміт докладно показав,
що вони вирівнюються всілякими дійсними або основаними на передсуді
компенсуючими обставинами, і тому, як відмінності тільки
позірні й минущі, не повинні братися до уваги при дослідженні загальних
співвідношень. Інші відмінності, наприклад, у висоті заробітної
плати, грунтуються здебільшого на згаданій вже у вступі до
першої книги, стор. 49\footnote*{Стор. 9 рос. вид. 1935~\abbr{р.} \Red{Ред. укр. перекладу}.}, ріжниці між простою і складною працею
і ніяк не зачіпають ступеня експлуатації праці в різних сферах
виробництва, хоч і роблять дуже неоднаковою долю робітників
в цих різних сферах. Якщо, наприклад, праця майстра по золоту
оплачується дорожче, ніж праця поденника, то додаткова праця
майстра по золоту створює також у такій самій пропорції більше
додаткової вартості, ніж її створює додаткова праця поденника.
І якщо вирівнювання заробітних плат і робочих днів, а тому й
норм додаткової вартості між різними сферами виробництва
і навіть між різними капіталовкладеннями однієї і тієї самої
сфери виробництва затримується багатьма різними місцевими
перешкодами, то все ж воно все більше й більше здійснюється
разом з прогресом капіталістичного виробництва і підпорядкуванням
усіх економічних відносин цьому способові виробництва.
\parbreak{}  %% абзац продовжується на наступній сторінці
