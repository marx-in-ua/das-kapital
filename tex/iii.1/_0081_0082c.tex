
\index{iii1}{0081}  %% посилання на сторінку оригінального видання
Наприклад:
\begin{center}
$80 c \dplus{} 20 v \dplus{} 20 m; m' \deq{} 100\%, p' \deq{} 20\%$

$80 c \dplus{} 20 v \dplus{} 10 m; m' \deq{} \phantom{0}50\%, p' \deq{} 10\%$

$20\%:10\% \deq{} 100×20:50×20 \deq{} 20 m:10 m.$
\end{center}

Тепер ясно, що при капіталах однакового абсолютного чи
процентного складу норми додаткової вартості можуть бути
різні тільки в тому випадку, коли різні або заробітна плата,
або довжина робочого дня, або інтенсивність праці. В трьох
випадках:
\begin{center}
$\phantom{II}$I. $80 c \dplus{} 20 v \dplus{} 10 m; m' \deq{} \phantom{0}50\%, p' \deq{} 10\%$,

$\phantom{I}$II. $80 c \dplus{} 20 v \dplus{} 20 m; m' \deq{} 100\%, p' \deq{} 20\%$,

III.$80 c \dplus{} 20 v \dplus{} 40 m; m' \deq{} 200\%, p' \deq{} 40\%$,
\end{center}
вся нововироблена вартість буде в І $30 (20 v \dplus{} 10 m)$, в II — 40,
в III — 60. Це може статись трояким способом.

\emph{Поперше}, якщо заробітні плати різні, отже, якщо $20 v$ в кожному
окремому випадку виражає різне число робітників. Припустім,
що в І занято 15 робітників 10 годин при заробітній
платі в 1\sfrac{1}{3}\pound{ фунтів стерлінгів} і що вони виробляють вартість
у 30\pound{ фунтів стерлінгів}, з яких 20\pound{ фунтів стерлінгів} заміщають
заробітну плату, а 10\pound{ фунтів стерлінгів} лишаються для додаткової
вартості. Якщо заробітна плата падає до 1\pound{ фунта стерлінгів},
то можуть бути заняті 20 робітників 10 годин; тоді вони
виробляють вартість у 40\pound{ фунтів стерлінгів}, з яких 20\pound{ фунтів
стерлінгів} для заробітної плати і 20\pound{ фунтів стерлінгів} додаткової
вартості. Якщо заробітна плата падає ще далі, до \sfrac{2}{3}\pound{ фунтів
стерлінгів}, то можуть бути заняті 30 робітників по 10 годин,
які виробляють вартість у 60\pound{ фунтів стерлінгів}, що з них після
відрахування 20\pound{ фунтів стерлінгів} для заробітної плати залишиться
ще 40\pound{ фунтів стерлінгів} для додаткової вартості.

Цей випадок: незмінний процентний склад капіталу, незмінний
робочий день, незмінна інтенсивність праці, зміна норми
додаткової вартості, спричинена зміною заробітної плати — є
єдиний випадок, на якому справджується положення Рікардо:
„profits would be high or low, \emph{exactly in proportion} as wages
would be, low or high“ [„зиск буде високий чи низький \emph{точно
в такій пропорції}, в якій заробітна плата буде низька чи висока“]
(„Principles of Political Economy“, розд. І, відділ III, стор. 18.
„Works of D.~Ricardo“, вид. Mac Culloch, 1852).

Або, \emph{подруге}, якщо інтенсивність праці різна. Тоді, наприклад,
20 робітників при однакових засобах праці за 10 робочих
годин на день виробляють у І — 30, у II — 40, у III — 60 штук
певного товару, кожна штука якого, крім вартості спожитих
на неї засобів виробництва, представляє нову вартість в 1\pound{ фунт
стерлінгів}. Через те що в кожному випадку 20 штук, \deq{} 20\pound{ фунтам стерлінгів}, заміщають заробітну плату, то для додаткової
\index{iii1}{0082}  %% посилання на сторінку оригінального видання
вартості лишаються в І — 10 штук \deq{} 10\pound{ фунтам стерлінгів},
в II — 20 штук \deq{} 20\pound{ фунтам стерлінгів}, в III — 40 штук \deq{} 40\pound{ фунтам
стерлінгів}.

Або, \emph{потретє}, робочий день — різної довжини. Якщо 20 робітників
при однаковій інтенсивності працюють у І — дев’ять,
у II — дванадцять, у III — вісімнадцять годин на день, то весь їх
продукт 30 : 40 : 60 відноситься як 9 : 12 : 18, і тому що заробітна
плата в кожному випадку \deq{} 20, то знову лишається 10, відповідно
20 і 40 для додаткової вартості.

Отже, підвищення або зниження заробітної плати діє в зворотному
напрямі, підвищення або зниження інтенсивності праці
і здовження або скорочення робочого дня діє в тому самому
напрямі на висоту норми додаткової вартості, а тому, при незмінному
$\frac{v}{K}$, і на норму зиску.

\begin{center}
\textbf{2. $m'$ і $v$ змінюються, $К$ не змінюється}
\end{center}
В цьому випадку має силу пропорція:\[
p': p'\textsubscript{1} \deq{} m' \frac{v}{K} : m'\textsubscript{1} \frac{v\textsubscript{1}}{K} \deq{} m'v : m'\textsubscript{1}v\textsubscript{1} \deq{} m : m\textsubscript{1}.\]

Норми зиску відносяться одна до одної, як відповідні маси
додаткової вартості.

Зміна норми додаткової вартості при незмінній величині змінного
капіталу означала зміну у величині й розподілі нововиробленої
вартості. Одночасна зміна $v$ і $m'$ так само завжди включає
інший розподіл, але не завжди зміну величини нововиробленої
вартості. Можливі три випадки:

a) Зміни $v$ і $m'$ відбуваються в протилежному напрямі, але
на однакову величину; наприклад:
\begin{center}
$80 c \dplus{} 20 v \dplus{} 10 m; m' \deq{} \phantom{0}50\%, p' \deq{} 10\%$

$90 c \dplus{} 10 v \dplus{} 20 m; m' \deq{} 200\%, p' \deq{} 20\%$.
\end{center}
Нововироблена вартість в обох випадках однакова, отже, однакова
й кількість витраченої праці; $20 v \dplus{} 10 m \deq{} 10 v \dplus{} 20 m \deq{} 30$.
Ріжниця тільки в тому, що в першому випадку 20 сплачується
як заробітна плата, а 10 лишається для додаткової вартості,
тимчасом як у другому випадку заробітна плата становить
тільки 10, а тому додаткова вартість — 20. Це єдиний випадок,
коли при одночасній зміні $v$ і $m'$ число робітників, інтенсивність
праці і довжина робочого дня лишаються незміненими.

b) Зміни $m'$ і $v$ відбуваються так само в протилежному напрямі,
але не на ту саму величину. Тоді перевага буде або на
стороні зміни $v$, або на стороні зміни $m'$.
\begin{center}
$\phantom{II}\text{I.} 80 c \dplus{} 20 v \dplus{} 20 m; m' \deq{} 100\phantom{\sfrac{3}{7}}\%, p' \deq{} 20\%$
$\phantom{I}\text{II.}72 c \dplus{} 28 v \dplus{} 20 m; m' \deq{} \phantom{0}71\sfrac{3}{7}\%, p' \deq{} 20\%$
$\text{III.} 84 c \dplus{} 16 v \dplus{} 20 m; m' \deq{} 125\phantom{\sfrac{3}{7}}\%, p' \deq{} 20\%.$
\end{center}
