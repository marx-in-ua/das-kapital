
Це було б можливе тільки тоді, коли б одночасно з зниженням
заробітної плати настала зміна в продуктивності праці, яка
вимагає цього зміненого складу капіталу; абож коли б грошова вартість
сталого капіталу підвищилась з 80 до 104; коротко кажучи —
такий випадковий збіг обставин, який буває тільки в виняткових
випадках. В дійсності така зміна $m'$, що не зумовлює одночасно
зміни $v$, а тому й зміни $\frac{v}{K}$, мислима тільки при цілком певних
обставинах, а саме в таких галузях промисловості, де застосовується
тільки основний капітал і праця, а предмет праці дається
природою.

Але при порівнянні норм зиску двох країн справа стоїть
інакше. Тут одна й та сама норма зиску в дійсності виражає
здебільшого різні норми додаткової вартості.

Отже, з усіх п’яти випадків випливає, що ростуща норма
зиску може відповідати падаючій або ростущій нормі додаткової
вартості, падаюча норма зиску — ростущій або падаючій
нормі додаткової вартості, незмінна норма зиску — ростущій або
падаючій нормі додаткової вартості. Що ростуща, падаюча або
незмінна норма зиску може також відповідати незмінній нормі
додаткової вартості, це ми бачили під рубрикою І.

\pfbreak

Отже, норма зиску визначається двома головними факторами:
нормою додаткової вартості і вартісним складом капіталу.
Вплив обох цих факторів можна коротко резюмувати, — при чому
склад ми можемо виразити в процентах, бо тут не має значення,
з якої з двох частин капіталу походить зміна, — таким чином:

Норми зиску двох капіталів або одного й того ж капіталу
в двох послідовних різних його станах

\emph{є рівні:}

1) при однаковому процентному складі капіталів і однаковій
нормі додаткової вартості;

2) при неоднаковому процентному складі і неоднаковій нормі
додаткової вартості, якщо добутки з норм додаткової вартості
і взятих у процентах змінних частин капіталу ($m'$ на $v$) є рівні,
тобто якщо \emph{маси} додаткової вартості ($m \deq{} m'v$), взяті в процентному
відношенні до всього капіталу, є рівні; інакше кажучи,
якщо в обох випадках множники $m'$ і $v$ стоять у зворотному
відношенні один до одного.

\emph{Вони нерівні:}

1) при однаковому процентному складі, якщо норми додаткової
вартості нерівні, при чому норми зиску відносяться одна до
одної, як норми додаткової вартості;
\parbreak{}  %% абзац продовжується на наступній сторінці
