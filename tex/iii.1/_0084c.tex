
\index{iii1}{0084}  %% посилання на сторінку оригінального видання
3. $р'$ підвищується або падає в меншій пропорції, ніж $m'$,
якщо $\frac{v}{K}$ змінюється в напрямі, протилежному до зміни $m'$, але
в меншій пропорції:

\begin{center}
$80c \dplus{} 20 v \dplus{} 10 m; m' \deq{} 50\%, p' \deq{} 10\%$

$90 c \dplus{} 10 v \dplus{} 15 m; m' \deq{} 150\%, p' \deq{} 15\%$

$50\% : 150\% > 10\% : 15\%.$
\end{center}

\looseness=-1
4. $р'$ підвищується, хоч $m'$ падає, або падає, хоч $m'$ підвищується,
якщо $\frac{v}{K}$ змінюється в напрямі, протилежному до зміни
$m'$, і в більшій пропорції, ніж $m'$.

\begin{center}
$80c \dplus{} 20 v \dplus{} 20 m; m' \deq{} 100\%, p' \deq{} 20\%$

$90 c \dplus{} 10 v \dplus{} 15 m; m' \deq{} 150\%, p' \deq{} 15\%$
\end{center}

\noindent $m'$ підвищилось з 100\% до 150\%, $р'$ зменшилось від 20\% до 15\%.

5. Нарешті, $р'$ лишається незмінним, хоч $m'$ підвищується або
падає, якщо $\frac{v}{K}$ змінює свою величину в протилежному напрямі,
але точно в тій самій пропорції, що й $m'$.

Тільки цей останній випадок потребує ще деякого пояснення.
Як ми бачили вище при змінах $\frac{v}{K}$, що одна й та сама норма
додаткової вартості може виражатися в найрізніших нормах
зиску, так ми бачимо тут, що в основі однієї і тієї самої норми
зиску можуть лежати дуже різні норми додаткової вартості.
Але в той час, як при незмінному $m'$ першої-ліпшої зміни у відношенні
$v$ до $К$ досить було для того, щоб викликати відмінність
в нормі зиску, $—$ при зміні величини $m'$ мусить настати точно
відповідна зворотна зміна величини $\frac{v}{K}$ для того, щоб норма
зиску лишилась та сама. Для одного й того ж капіталу або для
двох капіталів у тій самій країні це можливе тільки в дуже
виняткових випадках. Візьмімо, наприклад, капітал

\begin{center}
$80c \dplus{} 20 v \dplus{} 20 m; K \deq{} 100, m' \deq{} 100\%, p' \deq{} 20\%$
\end{center}

\noindent і припустімо, що заробітна плата впала настільки, що тепер за
$16 v$ можна було б мати те саме число робітників, як раніш за
$20 v$. Тоді ми, при інших незмінних умовах і звільненні $4 v$,
маємо

\begin{center}
$80c \dplus{} 16 v \dplus{} 24 m; K \deq{} 96, m' \deq{} 150\%, p' \deq{} 25\%.$
\end{center}

Отже, для того, щоб $р'$ було, як і раніш, \deq{} 20\%, весь капітал
мусив би зрости до 120, отже, сталий $—$ до 104:

\begin{center}
$104 c \dplus{} 16 v \dplus{} 24 m; K \deq{} 120, m' \deq{} 150\%, p' \deq{} 20\%$
\end{center}
