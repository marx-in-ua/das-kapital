\parcont{}  %% абзац починається на попередній сторінці
\index{iii1}{0254}  %% посилання на сторінку оригінального видання
можливе те, що доводиться шукати цього попиту за кордоном, на
віддалених ринках для того, щоб мати можливість платити робітникам
удома пересічну кількість необхідних засобів існування? Бо
тільки в цих специфічних, капіталістичних взаємозв’язках надлишковий
продукт набуває такої форми, в якій його власник
може подати його для споживання тільки в тому випадку, якщо
він знову перетвориться для нього в капітал. Нарешті, коли
кажуть, що капіталістам досить тільки обмінятися між собою
своїми товарами і спожити їх, то при цьому забувається весь
характер капіталістичного виробництва і забувається, що тут
ідеться про зростання вартості капіталу, а не про його споживання.
Коротко кажучи, всі ці заперечення проти очевидних явищ
перепродукції (явищ, які існують не зважаючи на ці заперечення)
зводяться до того, що межі \emph{капіталістичного} виробництва не
є межами \emph{виробництва взагалі}, і тому вони не є межами й цього
специфічного, капіталістичного способу виробництва. Але суперечність
цього капіталістичного способу виробництва полягає
саме в його тенденції до абсолютного розвитку продуктивних сил,
який постійно вступає в конфлікт з тими специфічними \emph{умовами}
виробництва, в яких рухається і тільки й може рухатись капітал.

Засобів існування виробляється не занадто багато порівняно
з наявним населенням. Навпаки. Їх виробляється занадто мало
для того, щоб маса населення могла жити пристойно і по-людськи.

Засобів виробництва виробляється не занадто багато для
того, щоб працездатна частина населення мала заняття. Навпаки.
Поперше, продукується занадто велика частина населення, яка
фактично непрацездатна, яка в наслідок її обставин живе тільки
з експлуатації праці інших або з робіт, що можуть вважатися
за такі тільки при жалюгідному способі виробництва. Подруге,
засобів виробництва виробляється недосить для того, щоб усе
працездатне населення працювало при найпродуктивніших умовах,
отже, щоб його абсолютний робочий час скорочувався
в наслідок більшої маси та ефективності сталого капіталу, застосовуваного
протягом робочого часу.

Але періодично засобів праці і засобів існування виробляється
занадто багато для того, щоб вони могли функціонувати
як засоби експлуатації робітників при певній нормі зиску. Товарів
виробляється занадто багато для того, щоб уміщену в них
вартість і включену в ній додаткову вартість можна було реалізувати
і перетворити зворотно в новий капітал при даних капіталістичним
виробництвом умовах розподілу і відносинах споживання,
тобто, щоб цей процес міг здійснюватися без вибухів,
які постійно відновлюються.

Багатства виробляється не занадто багато. Але періодично
його виробляється занадто багато в його капіталістичних, антагоністичних
формах.

Межа капіталістичного способу виробництва виявляється:

1)~В тому, що розвиток продуктивної сили праці створює
\parbreak{}  %% абзац продовжується на наступній сторінці
