
\index{iii1}{0074}  %% посилання на сторінку оригінального видання
В землеробстві та видобувній промисловості, де зменшення
продуктивності праці, а тому й збільшення числа занятих робітників
можна легко зрозуміти, цей процес — в рамках капіталістичного
виробництва і на його базі — зв’язаний не з зменшенням,
а з збільшенням сталого капіталу. Коли б навіть вищезгадане
зменшення $c$ було зумовлене простим зниженням ціни,
окремий капітал міг би тільки при цілком виняткових обставинах
зробити перехід від І до II.~Але для двох незалежних капіталів,
вкладених у різних країнах або в різні галузі землеробства чи
видобувної промисловості, не було б нічого дивного, коли б
в одному випадку вживалося більше робітників (а тому й більший
змінний капітал), які працюють менш дорогими або скуднішими
засобами виробництва, ніж у другому випадку.

Якщо ж ми відкинемо припущення, що заробітна плата лишається
незмінною, і пояснимо підвищення змінного капіталу
з 20 до 30 підвищенням заробітної плати наполовину, то матимем
цілком інший випадок. Те саме число робітників, скажімо,
20 робітників, і далі працює тими самими або тільки незначно
зменшеними засобами виробництва. Якщо робочий день лишається
незмінним, — наприклад, 10 годин, — то вся нововироблена
вартість також лишається незмінною; як і раніш, вона \deq{} 30.
Але ці 30 цілком уживаються на те, щоб замістити авансований
змінний капітал у 30; додаткова вартість зникла б. Але нами
припущено, що норма додаткової вартості не змінюється, тобто,
як і в І, лишається \deq{} 50\%. Це можливе тільки тоді, коли робочий
день наполовину здовжується, збільшується до 15 годин.
Тоді 20 робітників виробили б за 15 годин загальну вартість
у 45, і всі умови були б додержані:

$\text{II. }90c \dplus{} 30v \dplus{} 15m; К \deq{} 120, m' \deq{} 50\%; р' \deq{} 12\sfrac{1}{2}\%$

В цьому випадку ці 20 робітників не потребують засобів
праці, знарядь, машин і~\abbr{т. д.} більше, ніж у випадку І; тільки
сировинний матеріал або допоміжні матеріали довелося б збільшити
наполовину. Отже, при зниженні цін на ці матеріали
перехід від І до II, при наших припущеннях, економічно був би
вже далеко більше можливий навіть для поодинокого капіталу.
I капіталіст за свою втрату, якої він міг би зазнати в наслідок
зневартнення свого сталого капіталу, був би принаймні до певної
міри відшкодований більшим зиском.

Припустімо тепер, що змінний капітал не збільшується,
а зменшується. Тоді нам треба тільки обернути наш попередній
приклад, припустити, що № II є первісний капітал, і від.
II перейти до І.

II.  $90c \dplus{} 30v \dplus{} 15m$ перетворюється тоді в

І. $100 c \dplus{} 20 v \dplus{} 10 m$, і очевидно, що в наслідок цієї перестановки
абсолютно нічого не змінюється в умовах, які регулюють
відповідні норми зиску і їх взаємне відношення.
