\parcont{}  %% абзац починається на попередній сторінці
\index{iii1}{0337}  %% посилання на сторінку оригінального видання
вартості не було б поверненням її \emph{як} \emph{капіталу}, а просто поверненням
позиченої суми вартості. Щоб повернутись як капітал,
авансована сума вартості мусить не тільки зберегтися в русі,
але й збільшитись, збільшити величину своєї вартості, отже,
повернутися з додатковою вартістю, як $Г \dplus{} ΔГ$, і це $ΔГ$ є тут
процентом або тією частиною пересічного зиску, яка не лишається
в руках функціонуючого капіталіста, а дістається грошовому капіталістові.

Та обставина, що грошовий капіталіст передав іншій особі
гроші як капітал, означає, що вони мусять бути повернені йому
як $Г \dplus{} ΔГ$. Нам доведеться ще окремо розглянути ту форму, коли
протягом того часу, на який дано позику, проценти припливають
назад у певні строки, але без капіталу, зворотна сплата якого
відбувається тільки наприкінці довшого періоду.

Що дає грошовий капіталіст позичальникові, промисловому
капіталістові? Що він відчужує йому в дійсності? А тільки акт
відчуження перетворює позичання грошей у відчуження грошей
як капіталу, тобто у відчуження капіталу як товару.

Тільки за допомогою акту цього відчуження капітал позикодавця
грошей як товару, абож товар, який він має в своєму
розпорядженні, передається якійсь іншій особі як капітал.

Що відчужується при звичайному продажу? Не вартість проданого
товару, бо вона змінює тільки свою форму. Вона існує ідеально
в товарі як ціна, раніше ніж вона реально в формі грошей
переходить у руки продавця. Одна й та сама вартість і одна й та
сама величина вартості змінюють тут тільки свою форму. Одного
разу вони існують в товарній формі, другого разу — в грошовій
формі. Те, що дійсно відчужується від продавця і тому переходить
також в особисте або продуктивне споживання покупця,
це — споживна вартість товару, товар як споживна вартість.

Що ж це за споживна вартість, яку грошовий капіталіст відчужує
на час позики і передає продуктивному капіталістові,
позичальникові? Це — споживна вартість, яку гроші набувають
в наслідок того, що вони можуть бути перетворені в капітал,
можуть функціонувати як капітал, і що через це вони створюють
у своєму русі певну додаткову вартість, пересічний зиск (те,
що вище або нижче цього, являє собою тут випадковість), поверх
того, що вони зберігають свою первісну величину вартості. Споживна
вартість всіх інших товарів кінець-кінцем споживається
і разом з тим зникає субстанція товару, а з нею і вартість його.
Навпаки, товар-капітал має ту особливість, що споживанням його
споживної вартості його вартість і його споживна вартість не
тільки зберігаються, але ще й збільшуються.

Цю споживну вартість грошей як капіталу — здатність створювати
пересічний зиск — грошовий капіталіст і відчужує промисловому
капіталістові на той час, на який він передає йому
розпорядження позиченим капіталом.

В цьому розумінні є певна аналогія між грішми, відданими
\parbreak{}  %% абзац продовжується на наступній сторінці
