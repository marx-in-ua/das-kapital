\parcont{}  %% абзац починається на попередній сторінці
\index{iii1}{0265}  %% посилання на сторінку оригінального видання
для нього не тільки як для капіталіста взагалі, а спеціально як
для торговця товарами, само собою очевидно, що його капітал
первісно мусить з’явитись на ринку в формі грошового капіталу,
бо він не виробляє ніяких товарів, а тільки торгує ними, опосереднює їхній рух, а для того, щоб
торгувати ними, він мусить
їх спочатку купити, отже, мусить бути володільцем грошового
капіталу.

Припустімо, що якийсь торговець товарами володіє 3000\pound{ фунтами стерлінгів}, які він збільшує в їх
вартості як торговельний
капітал. На ці 3000\pound{ фунтів стерлінгів} він купує у фабриканта,
який виробляє полотно, наприклад, \num{30000} метрів полотна по
2 шилінги за метр. Він продає ці \num{30000} метрів. Якщо пересічна
річна норма зиску = 10\% і якщо він, після відрахування всіх
накладних витрат, одержує 10\% річного зиску, то на кінець року
він перетворить ці 3000\pound{ фунтів стерлінгів} у 3300\pound{ фунтів стерлінгів}. Яким чином він одержує цей зиск
— це питання, яке ми
розглянемо тільки пізніше. Тут ми насамперед розглянемо тільки
форму руху його капіталу. На ці 3000\pound{ фунтів стерлінгів} він
весь час купує полотно і весь час продає це полотно; він постійно повторює цю операцію купівлі для
продажу, $Г — Т — Г'$,
просту форму капіталу, в якій цей капітал цілком зв’язаний у процесі циркуляції, не перериваному
інтервалами процесу виробництва, який лежить поза його власним рухом і функцією.

Яке ж є відношення цього товарно-торговельного капіталу
до товарного капіталу як простої форми існування промислового
капіталу? Щодо фабриканта полотна, то він грішми купця реалізував вартість свого полотна, виконав
першу фазу метаморфози свого товарного капіталу, перетворення його в гроші, і може
тепер, при інших незмінних умовах, знову перетворити гроші
у пряжу, вугілля, заробітну плату і~\abbr{т. д.}, з другого боку — в засоби існування і~\abbr{т. д.} для
споживання свого доходу; отже, залишаючи осторонь витрачання доходу, він може продовжувати процес
репродукції.

Але, хоч для нього, для виробника полотна, вже відбулася
метаморфоза полотна в гроші, його продаж, вона ще не відбулася для самого полотна. Як і раніш,
полотно перебуває на ринку
як товарний капітал і має призначення виконати свою першу
метаморфозу, бути проданим. З цим полотном нічого не сталося,
крім переміни особи його володільця. За своїм призначенням, за
своїм становищем у процесі воно, як і раніше, є товарний капітал, продажний товар; тільки тепер воно
перебуває в руках
купця, а не в руках виробника, як це було раніш. Функція його
продажу, опосереднення першої фази його метаморфози, забрана
від виробника купцем і перетворена в його спеціальне заняття, — тимчасом як раніше це була функція,
яку мав виконувати виробник після виконання функції виробництва полотна.

Припустімо, що купцеві не вдалося продати \num{30000} метрів
протягом того періоду часу, який потрібний виробникові
\parbreak{}  %% абзац продовжується на наступній сторінці
