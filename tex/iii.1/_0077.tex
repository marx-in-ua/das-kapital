
\index{iii1}{0077}  %% посилання на сторінку оригінального видання
Через те що  $m'v = m$, масі додаткової вартості, і що $m'$ і $v$
обоє лишаються незмінними, то й $m$ не зачіпається зміною $К$;
маса додаткової вартості після зміни лишається такою самою,
якою була перед зміною.

Коли б $c$ зменшилося до нуля, то $p'$ було б — $m'$, норма зиску
дорівнювала б нормі додаткової вартості.

Зміна $c$ може виникнути або в наслідок простої зміни вартості
речових елементів сталого капіталу, або в наслідок зміни технічного
складу всього капіталу, тобто в наслідок зміни продуктивності
праці у відповідній галузі виробництва. В останньому
випадку ростуща разом з розвитком великої промисловості і
землеробства продуктивність суспільної праці зумовила б те, що
перехід відбувався б (у щойно наведеному прикладі) у послідовності
від III до І і від І до II. Кількість праці, за яку сплачується
20 і яка виробляє вартість у 40, спочатку охоплювала б масу засобів
праці вартістю в 60; при підвищенні продуктивності і незмінній
вартості кількість охоплюваних засобів праці зросла б спочатку
до 80, потім до 100. Зворотна послідовність зумовила б
зменшення продуктивності; та сама кількість праці могла б привести
в рух менше засобів виробництва, виробництво скоротилося
б, як це може трапитись у землеробстві, гірництві і~\abbr{т. д.}

Заощадження на сталому капіталі, з одного боку, підвищує
норму зиску, і з другого — звільняє капітал, отже, воно важливе
для капіталістів. Цей пункт, а також вплив зміни ціни елементів
сталого капіталу, особливо сировинних матеріалів, ми далі дослідимо
докладніше.

І тут знову виявляється, що зміна сталого капіталу однаково
впливає на норму зиску, все одно, чи викликана ця зміна збільшенням
або зменшенням речових складових частин $c$, чи простою
зміною їх вартості.

\subsection{$m'$ не змінюється, $v$, $c$ і $К$ всі змінюються}

В цьому випадку провідною лишається вищенаведена загальна
формула зміненої норми зиску:
\[
p'_1 = m'\frac{ev}{EK}.
\]

З неї випливає, що при незмінній нормі додаткової вартості:

а) Норма зиску падає, якщо $Е$ більше за $е$, тобто якщо сталий
капітал збільшується таким чином, що весь капітал зростає
в більшій пропорції, ніж змінний капітал. Якщо капітал складу
$80 c + 20 v + 20 m$ переходить у склад $170 c + 30 v + 30 m$, то $m'$
лишається = 100\%, але\frac{v}{К} падає з \frac{20}{100} до \frac{30}{200}, не
зважаючи на те,
що як $v$, так і $К$, збільшились, і норма зиску відповідно падає з
20\% до 15\%.
