
\index{iii1}{0049}  %% посилання на сторінку оригінального видання
Навпаки, категорія витрат виробництва не має ніякого відношення
до утворення вартості товару або до процесу зростання
вартості капіталу. Якщо я знаю, що \sfrac{5}{6} товарної вартості
в 600\pound{ фунтів стерлінгів}, або 500\pound{ фунтів стерлінгів}, становлять
тільки еквівалент, вартість, яка заміщає витрачений капітал
у 500\pound{ фунтів стерлінгів}, і тому їх вистачає тільки для того, щоб
знову купити речові елементи цього капіталу, то від цього
я ще не знаю ні того, як вироблені ці \sfrac{5}{6}  вартості товару, які
становлять витрати його виробництва, ні того, як вироблена
остання шоста частина, яка становить у ньому додаткову вартість.
Дослідження, однак, покаже, що витрати виробництва в капіталістичному
господарстві набувають фальшивої видимості категорії
самого виробництва вартості.

Вернімось до нашого прикладу. Коли ми припустимо, що вартість,
вироблена одним робітником на протязі одного пересічного
суспільного робочого дня, представлена в грошовій сумі
в 6\shil{ шилінгів} \deq{} 6 маркам, то авансований капітал у 500\pound{ фунтів
стерлінгів} $= 400 c \dplus{} 100 v$ становитиме вартість, вироблену протягом
1666\sfrac{2}{3} десятигодинних робочих днів, з яких 1333\sfrac{1}{3} робочих
днів кристалізовані у вартості засобів виробництва $= 400 c$,
333\sfrac{1}{3}  — у вартості робочої сили $= 100v$. Отже, при припущеній
нормі додаткової вартості в 100\%, виробництво самого новоутворюваного
товару коштує витрати робочої сили $= 100v \dplus{} 100m \deq{} 666\sfrac{2}{3}$ десятигодинних робочих днів.

Далі, ми знаємо (див. книгу І, розд. VII, стор. 220\footnote*{Стор. 146--147 рос. вид. 1935~\abbr{р.} \emph{Ред. укр. перекладу.}}, що вартість
новоутвореного продукту в 600\pound{ фунтів стерлінгів} складається
з: 1)~вартості сталого капіталу в 400\pound{ фунтів стерлінгів},
витраченого на засоби виробництва, яка з’являється знову, і
2)~з нововиробленої вартості в 200\pound{ фунтів стерлінгів}. Витрати
виробництва товару \deq{} 500\pound{ фунтам стерлінгів} включають $400 c$,
які з’явилися знову, та половину нововиробленої вартості в
200\pound{ фунтів стерлінгів} ($= 100 v$), отже, два цілком різні щодо
свого походження елементи товарної вартості.

Завдяки доцільному характерові праці, витраченої протягом
666\sfrac{2}{3}  десятигодинних днів, вартість спожитих засобів виробництва
на суму в 400\pound{ фунтів стерлінгів} переноситься з цих
засобів виробництва на продукт. Тому ця стара вартість з’являється
знову як складова частина вартості продукту, але вона
не виникає в процесі виробництва цього товару. Вона існує як
складова частина товарної вартості тільки тому, що раніше вона
існувала як складова частина авансованого капіталу. Отже, витрачений
сталий капітал заміщається тією частиною товарної
вартості, яку він сам додає до товарної вартості. Отже, цей
елемент витрат виробництва має двоїсте значення: з одного боку,
він входить у витрати виробництва товару, бо він є та складова
частина товарної вартості, яка заміщає витрачений капітал;
\parbreak{}  %% абзац продовжується на наступній сторінці
