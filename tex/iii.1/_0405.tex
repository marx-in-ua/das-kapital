\parcont{}  %% абзац починається на попередній сторінці
\index{iii1}{0405}  %% посилання на сторінку оригінального видання
з однієї форми в другу, а тільки для того, щоб одержати
додатковий капітал.

„3743. В чому головне бажання ділового світу під гнітом паніки, яка, за вашими свідченнями, мала
місце в 1825,
1837 і 1839 роках; чи ділові люди мають на меті одержати в
своє розпорядження капітал чи законні платіжні гроші? — Вони
мають на меті одержати владу над капіталом, щоб провадити
далі свої підприємства“. — Їх мета — це одержання засобів для
оплати векселів, яким надійшов строк, виставлених на них самих,
з огляду на виниклу недостачу в кредиті, і щоб не бути змушеними продати свої товари нижче ціни.
Якщо вони самі взагалі
не мають капіталу, то разом з платіжними засобами вони, звичайно, одержують одночасно капітал, бо
вони одержують вартість
без еквіваленту. Вимога на гроші як такі полягає завжди тільки в
бажанні перетворити вартість з форми товару або боргових вимог
у форму грошей. Тим то — навіть залишаючи осторонь кризи —
існує велика ріжниця між одержанням капіталу в позику і дисконтом, який тільки здійснює перетворення
грошових вимог
з однієї форми в другу або в дійсні гроші.

[Я — упорядник — дозволяю собі вставити тут зауваження.

У Нормана, як і в Лойд-Оверстона, банкір завжди виступає
як той, хто „дає капітал у позику“, а його клієнт — як той,
хто вимагає від нього „капіталу“. Так, за Оверстоном, хтось
дисконтує через банкіра вексель, „тому що бажає одержати
\emph{капітал}“ (3729), і такій людині дуже приємно, якщо вона „може
одержати \emph{розпорядження над капіталом} за низький процент“
(3730). „Гроші є знаряддя для одержання \emph{капіталу}“ (3736), а
при паніці головне бажання ділового світу — це „одержати владу
над \emph{капіталом}“ (3743). При всій плутанині у Лойд-Оверстона відносно того, що таке капітал, все ж
досить ясно видно, що те,
що банкір дає своєму діловому клієнтові, він називає капіталом,
капіталом, якого клієнт раніше не мав і який дано йому
в позику, капіталом, додатковим до того, яким клієнт порядкував досі.

Банкір настільки звик фігурувати як розподілювач — у формі
надання позик — вільного в грошовій формі суспільного капіталу,
що всяка функція, при якій він віддає гроші, здається йому
наданням позики. Всі гроші, які він виплачує, здаються йому
позикою. Якщо гроші витрачені безпосередньо на позику, то
це буквально вірно. Якщо ж гроші витрачені на дисконт векселів, то для нього самого це дійсно є
позика до скінчення строку
векселя. Таким шляхом у його голові зміцнюється уявлення, що
він не може робити ніяких платежів, які не були б позикою.
І при тому позикою не тільки в тому розумінні, що кожне
застосування грошей з метою одержання процента або зиску
з економічної точки зору розглядається як позика, яку відповідний володілець грошей як приватна
особа дає самому собі
як підприємцеві; — а позикою в тому певному розумінні, що
\parbreak{}  %% абзац продовжується на наступній сторінці
