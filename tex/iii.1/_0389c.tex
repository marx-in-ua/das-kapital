\parcont{}  %% абзац починається на попередній сторінці
\index{iii1}{0389}  %% посилання на сторінку оригінального видання
чинність банкового акту і тим усунув покладені на банк абсурдні пута закону. Тепер банк міг
безперешкодно випускати
в циркуляцію свій запас банкнот; тому що кредит цих банкнот
фактично був забезпечений кредитом нації, отже, не був захитаний, то разом з цим зараз же настало
рішуче полегшення
в грошовій скруті; звичайно, чимало великих і дрібних фірм,
які безнадійно зарвалися, ще збанкрутувало, але найвища точка
кризи була вже переборена, банковий дисконт у грудні знову
впав до 5\% і вже на протязі 1848 року підготувалась та нова
активність у справах, яка в 1849 році обламала вістря революційних рухів на континенті і в
п’ятидесятих роках привела спочатку до нечуваного доти промислового розквіту, а потім — і до
краху 1857 року. — \emph{Ф.~Е.}].

\pfbreak

I.~Про колосальне знецінення державних паперів і акцій під час кризи
1847 року дає відомості документ, виданий House of Lords [палатою лордів]
в 1848 році. Згідно з цим документом падіння цінностей 23 жовтня 1847 року,
порівняно з станом у лютому того самого року, становило:
\begin{center}
\begin{tabular} {l c c}

на англійські державні папери & \phantom{0}\num{93824217} & \pound{фунтів стерлінгів}\\

на акції доків і каналів & \phantom{00}\num{1358288} & \\

на залізничні акції & \phantom{0}\num{19579820} & \\
\midrule
Разом & \num{114762325} & \pound{фунтів стерлінгів} \\
\end{tabular}
\end{center}
II.~Про шахрайства в ост-індській торгівлі, де векселі видавались не тому,
що купувались товари, а купувались товари для того, щоб мати змогу одержати вексель, який можна було
б дисконтувати, перетворити в гроші, \emph{„Manchester Guardian“} від 24 листопада 1847~\abbr{р.} [стор. 4]
повідомляє таке:

$А$ в Лондоні купує через посередництво $В$ товари у фабриканта $C$ в Манчестері для відправлення кораблем
до $D$ в Ост-Індію. $В$ платить $C$ шестимісячними векселями, виданими $C$ на $В$. В так само забезпечує себе
шестимісячним
векселем на $А$. Як тільки товар відправлено, $А$ так само одержує під переслану
накладну шестимісячний вексель на $D$. „Таким чином покупець і відправник
обидва володіють фондами на багато місяців раніше, ніж вони дійсно оплатять
товари; і дуже часто ці векселі після скінчення строку поновлювались під тим
приводом, що при такій „забарній справі“ треба дати час для зворотного припливу грошей. Але, на
жаль, втрати на таких справах вели не до обмеження
їх, а якраз до їх поширення. Чим бідніші ставали учасники, тим більшою ставала для них потреба
купувати, щоб таким шляхом у нових позиках відшкодувати
себе за капітал, утрачений при попередніх спекуляціях. Закупівлі не регулювались уже попитом і
поданням, вони ставали найважливішою частиною фінансових
операцій фірми, що зарвалася. Але це тільки один бік справи. Те саме, що відбувалося тут з експортом
мануфактурних товарів, відбувалося й там з купівлею
і відвантаженням продуктів. Фірми в Індії, які користувались достатнім кредитом,
щоб мати можливість дисконтувати свої векселі, купувати цукор, індиго, шовк
чи бавовну — не тому, що купівельні ціни порівняно з останніми лондонськими
цінами обіцяли зиск, а тому, що незабаром кінчалися строки попередніх трат на
лондонську фірму і їх треба було покрити. Що ж було простішого, як купити вантаж цукру, оплатити
його десятимісячним векселем на лондонську фірму і відіслати накладну поштою до Лондона? Менше ніж
через два місяці накладна на
ці щойно відправлені товари, а з нею й самі товари заставлялись на Lombard
Street [вулиця, де містяться банки], і лондонська фірма одержувала гроші за
вісім місяців до скінчення строку векселів, виданих під цей товар. І все йшло
гладко без перерв і без труднощів, поки дисконтуючі фірми знаходили більше
ніж досить грошей, щоб авансувати їх під накладні й докові варанти і на необмежені суми дисконтувати
векселі індійських фірм на „першорядні“ фірми в
Mincing Lane [вулиця, де міститься кофейна і чайна біржа]“.

\pfbreak
