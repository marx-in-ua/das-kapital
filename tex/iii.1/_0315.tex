\parcont{}  %% абзац починається на попередній сторінці
\index{iii1}{0315}  %% посилання на сторінку оригінального видання
первісна громада), привласнювачем, отже, продавцем продукту
є рабовласник, феодал, держава, яка стягає дань. Купець купує
і продає для багатьох. В його руках концентруються купівлі
й продажі, в наслідок чого купівля й продаж перестають
бути зв’язаними з безпосередніми потребами покупця (як
купця).

Але яка б не була суспільна організація тих сфер виробництва,
обмін товарів яких опосереднює купець, його майно завжди
існує як грошове майно і його гроші завжди функціонують
як капітал. Форма цього капіталу завжди є $Г — Т — Г'$; гроші,
самостійна форма мінової вартості — вихідний пункт, і збільшення
мінової вартості — самостійна мета. Самий товарний обмін і операції,
що опосереднюють його, — відокремлені від виробництва
і виконувані не-виробниками, — є просто засіб збільшення, збільшення
не просто багатства, але багатства в його загальній суспільній
формі, багатства як мінової вартості. Спонукальний мотив
і визначальна мета полягає в тому, щоб перетворити $Г$ в $Г \dplus{} ΔГ$;
акти $Г — Т$ і $Т — Г'$, які опосереднюють акт $Г — Г'$, виступають
лиш як перехідні моменти цього перетворення $Г$ в $Г \dplus{} ΔГ$. Це
$Г — Т — Г'$ як характерний рух купецького капіталу відрізняє його
від $Т — Г — Т$, тієї товарної торгівлі між самими виробниками,
кінцевою метою якої є обмін споживних вартостей.

Тому, чим менш розвинене виробництво, тим більше грошове
майно концентрується в руках купців або тим більше воно виступає
як специфічна форма купецького майна.

За капіталістичного способу виробництва, — тобто коли капітал
опановує само виробництво і надає йому цілком зміненої
і специфічної форми, — купецький капітал виступає тільки як
капітал в \emph{особливій} функції. За всіх попередніх способів виробництва,
і тим більше, чим більше виробництво є безпосередньо
виробництво засобів існування виробника, купецький капітал
виступає як функція капіталу par excellence.

Отже, не становить ні найменшої трудности зрозуміти, чому
купецький капітал як історична форма капіталу з’являється задовго
до того, як капітал підпорядкував собі само виробництво.
Його існування й розвиток до певної висоти самі є історичною
передумовою для розвитку капіталістичного способу
виробництва, 1)~як попередня умова концентрації грошового
майна і 2)~тому що капіталістичний спосіб виробництва передбачає
виробництво для торгівлі, збут у великому масштабі і не
окремим покупцям, отже, передбачає також купця, який купує
не для задоволення своєї особистої потреби, а в своєму акті купівлі
концентрує акти купівлі багатьох. З другого боку, весь
розвиток купецького капіталу впливає таким чином, що все
більше й більше надає виробництву характеру виробництва, що
має за мету мінову вартість, дедалі більше перетворює продукти
в товари. Однак, його розвиток, узятий сам по собі, є, як
ми це побачимо зразу далі, недостатній для того, щоб викликати
\parbreak{}  %% абзац продовжується на наступній сторінці
