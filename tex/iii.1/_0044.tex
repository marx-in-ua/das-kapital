\parcont{}  %% абзац починається на попередній сторінці
\index{iii1}{0044}  %% посилання на сторінку оригінального видання
який доводить підприємство до розмірів, достатніх для того, щоб бути „заснованим“\footnote*{Тобто перетвореним у форму акційного товариства. \Red{Ред. укр. перекладу.}}.

Те саме щодо торгівлі. Leafs, Parsons, Morleys, Monsieur Dillon\footnote*{Назви фірм. \Red{Ред. укр. перекладу.}} — всі засновані. Цілком так само тепер уже й роздрібні
магазини і при тому не тільки під виглядом кооперації à la „Stores“\footnote*{на зразок універсальних магазинів. \Red{Ред. укр. перекладу.}}.

Те саме щодо банків та інших кредитних установ також і в Англії. Маса нових — всі у формі акційних товариств з обмеженою
відповідальністю. Навіть старі банки, як Glyns і т. д., перетворюються з 7 приватними акціонерами в товариство з обмеженою відповідальністю.

5. В галузі землеробства те саме. Надзвичайно поширені банки, особливо в Німеччині, під всілякими
бюрократичними назвами, все більше й більше носії іпотек; разом з їх акціями дійсна :  верховна власність над
землеволодінням передається біржі, і це
ще в більшій мірі, коли маєтки переходять у руки кредиторів. Тут землеробська революція, зв’язана з обробітком степів, діє
насильно; якщо це й далі йтиме так, то можна передбачити час, коли також землі в Англії і Франції перейдуть у руки бірж.

6. Закордонні капіталовкладення — усі в акціях. Кажучи тільки про Англію: американські залізниці, північні й південні
(довідатись у біржовому бюлетені), Гольдбергер і т. д.

7. Потім колонізація. Ця остання нині є просто відділ біржі, в інтересах
якої європейські держави кілька років тому поділили Африку, французи завоювали Туніс і Тонкін. Африка прямо віддана в
оренду компаніям (Нігер, Північна Африка, Німецька Південно-західна і Східна Африка), Машоналенд і Наталь захоплені Родсом
\footnote*{Англійський імперіаліст, організатор загарбання Південної Африки Англією в 80-х і 90-х pp. XIX століття.
\Red{Ред. укр. перекладу.}} для біржі.
