
\index{iii1}{0156}  %% посилання на сторінку оригінального видання
З цього способу розгляду змінного капіталу випливає:

Якщо в сфері виробництва \emph{A} при вкладенні капіталу з кожних 700 всього
капіталу витрачається тільки 100 на змінний
і 600 на сталий капітал, тимчасом як у сфері виробництва \emph{B}
витрачається 600 на змінний і тільки 100 на сталий, то весь
капітал \emph{А} в 700 приводитиме в рух робочу силу тільки в 100,
тобто при попередніх припущеннях лише 100 робочих тижнів,
або 6000 годин, живої праці, тимчасом як рівновеликий капітал \emph{B}
приводитиме в рух 600 робочих тижнів, отже, \num{36000} годин живої праці.
Тому капітал в сфері \emph{А} привласнив би собі тільки
50 робочих тижнів, або 3000 годин додаткової праці, тимчасом
як рівновеликий капітал у сфері В привласнив би собі 300 робочих тижнів,
або \num{18000} годин. Змінний капітал є показник не
тільки праці, вміщеної в ньому самому, але, при даній нормі
додаткової вартості, і тієї надлишкової праці, яку він приводить
в рух понад цю міру, або додаткової праці. При однаковому
ступені експлуатації праці зиск у першому випадку був би $\frac{100}{700} = \sfrac{1}{7} = 14\sfrac{2}{7}\%$, а
в другому $= \frac{600}{700} = 85\sfrac{5}{7}\%$, тобто вшестеро
більша норма зиску. Але і в дійсності в цьому випадку самий
зиск був би вшестеро більший, 600 для \emph{В} проти 100 для \emph{А}, бо
вшестеро більше живої праці приводиться в рух з тим самим
капіталом, отже, при однаковому ступені експлуатації праці
виробляється також вшестеро більше додаткової вартості, а тому
і вшестеро більше зиску.

Коли б в \emph{А} застосовувалось не 700, а 7000\pound{ фунтів стерлінгів}, а
в \emph{В}, навпаки, тільки 700\pound{ фунтів стерлінгів} капіталу, то
капітал \emph{А}, при незмінному органічному складі, застосовував би
з цих 7000\pound{ фунтів стерлінгів} 1000\pound{ фунтів стерлінгів} як змінний
капітал, тобто вживав би 1000 робітників на тиждень — \num{60000} годин живої праці,
з них \num{30000} годин додаткової праці. Але,
як і раніш, \emph{А} приводило б у рух кожними 700\pound{ фунтами стерлінгів} тільки
вшестеро менше живої праці, а тому і вшестеро
менше додаткової праці, ніж \emph{В}, отже, виробляло б і вшестеро
менше зиску. Якщо розглядатимем норму зиску, то для капіталу \emph{А} матимем
$\frac{1000}{7000} = \frac{100}{700} = 14\sfrac{2}{7}\%$
проти $\frac{600}{700}$, або $85\sfrac{5}{8}\%$, для
капіталу \emph{В}. Не зважаючи на однакову величину капіталів, норми
зиску в них тут різні, бо при однаковій нормі додаткової
вартості, в наслідок різних мас приведеної в рух живої праці,
маси вироблених додаткових вартостей, а тому й зисків є різні.

Той самий результат фактично виходить в тому випадку, коли
технічні відношення в одній сфері виробництва ті самі, що й
у другій, але вартість застосовуваних елементів сталого капіталу більша або менша. Припустімо, що
обидві сфери застосовують 100\pound{ фунтів стерлінгів} як змінний капітал і потребують, отже, 100
робітників на тиждень, щоб привести в рух ту саму
кількість машин і сировинного матеріалу, але ці останні в \emph{В}
\parbreak{}  %% абзац продовжується на наступній сторінці
