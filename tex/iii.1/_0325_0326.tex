
\chaptertwoline{%
Розпад зиску на~процент і~підприємницький}{%
дохід. Капітал, що~дає~процент}{%
Розпад зиску на процент і підприємницький дохід. Капітал, що дає процент}

\section{Капітал, що дає процент}

При першому дослідженні загальної або пересічної норми
зиску (відділ II цієї книги) ми мали цю останню перед собою
ще не в її готовому вигляді, бо вирівнення виступало ще просто
як вирівнення промислових капіталів, вкладених у різні сфери.
Це було доповнено в попередньому відділі, де досліджувались
участь торговельного капіталу в цьому вирівненні і торговельний
зиск. Загальна норма зиску і пересічний зиск виступили
при цьому у вужчих межах, ніж раніше. В ході дальшого
дослідження слід мати на увазі, що коли ми далі говоримо
про загальну норму зиску або пересічний зиск, то це тільки
в останньому значенні, отже, тільки щодо готової форми пересічної
норми. А через те що ця норма тепер однакова для промислового
і торговельного капіталу, то немає також надалі
потреби, оскільки йдеться тільки про цей пересічний зиск,
розрізняти промисловий і торговельний зиск. Чи вкладено капітал
у сферу виробництва, як промисловий капітал, чи у сферу циркуляції,
як торговельний капітал, він однаково дає той самий
річний пересічний зиск pro rata [відповідно до] своєї величини.

Гроші, — взяті тут як самостійний вираз певної суми вартості,
однаково, чи існує вона фактично у вигляді грошей, чи у
вигляді товарів, — можуть бути перетворені на основі капіталістичного
виробництва в капітал, і в наслідок такого перетворення
стають з даної вартості самозростаючою вартістю, вартістю,
що збільшується. Вони виробляють зиск, тобто дають
капіталістові змогу добувати з робітників і привласнювати собі
певну кількість неоплаченої праці, додатковий продукт і додаткову
вартість. Таким чином, крім тієї споживної вартості, яку
вони мають як гроші, вони набувають ще додаткову споживну
вартість, саме ту, що вони функціонують як капітал. Їх споживна
\index{iii1}{0326}  %% посилання на сторінку оригінального видання
вартість полягає тут саме в тому зиску, який вони виробляють,
будучи перетворені в капітал. У цій своїй властивості,
як можливий капітал, як засіб для виробництва зиску, гроші
стають товаром, але товаром sui generis [особливого роду].
Або, що зводиться до того самого, капітал як капітал стає
товаром\footnote{
Тут слід було б навести деякі місця, в яких економісти саме так розуміють
цю справу. — „You (the Bank of England) are very large dealers in the
\emph{commodity of capital}?“ [Ви (англійський банк) робите дуже великі операції
з \emph{товаром капітал?}] — запитали, як свідка, одного з директорів цього банку
в „Report on Bank Acts“ (House of Commons, 1857 [стор. 104, № 1194]).
}.

Припустім, що річна пересічна норма зиску дорівнює 20\%.
Тоді машина вартістю в 100\pound{ фунтів стерлінгів}, застосовувана
як капітал при пересічних умовах і з пересічним ступенем розуміння
справи й доцільної діяльності, дала б зиск у 20\pound{ фунтів
стерлінгів}. Отже, людина, яка має в своєму розпорядженні
100\pound{ фунтів стерлінгів}, тримає в своїх руках владу зробити
з 100\pound{ фунтів стерлінгів} 120 або виробити зиск у 20\pound{ фунтів стерлінгів}.
Вона має в своїх руках можливий капітал в 100\pound{ фунтів
стерлінгів}. Якщо ця людина передає ці 100\pound{ фунтів стерлінгів}
на рік якійсь іншій людині, яка дійсно застосовує їх як капітал,
то вона дає їй владу виробити 20\pound{ фунтів стерлінгів} зиску, додаткову
вартість, яка їй нічого не коштує, за яку вона не платить
ніякого еквіваленту. Якщо ця інша людина наприкінці
року сплачує власникові цих 100\pound{ фунтів стерлінгів}, скажімо,
5\pound{ фунтів стерлінгів}, тобто частину виробленого зиску, то вона
таким чином оплачує споживну вартість цих 100\pound{ фунтів стерлінгів},
споживну вартість їх функції як капіталу, функції виробляти
20\pound{ фунтів стерлінгів} зиску. Та частина зиску, яку вона
сплачує власникові цих 100\pound{ фунтів стерлінгів}, зветься процентом,
що, отже, є нічим іншим, як особливою назвою, особливою
рубрикою для тієї частини зиску, яку функціонуючий капітал
повинен сплатити власникові капіталу, замість того, щоб покласти
її у власну кишеню.

Ясно, що володіння цими 100\pound{ фунтами стерлінгів} дає їх
власникові силу притягти до себе процент, певну частину
зиску, виробленого його капіталом. Коли б він не віддав цих
100\pound{ фунтів стерлінгів} іншій людині, то ця остання не могла б
виробити зиск, взагалі не могла б функціонувати як капіталіст
відносно цих 100\pound{ фунтів стерлінгів}\footnote{„Що людина, яка бере в позику гроші з метою добути з них зиск, повинна
частину зиску віддати позикодавцеві, це само собою зрозумілий принцип
природної справедливості“ (\emph{Gilbart}: „The History and Principles of Banking“,
London 1834, стор. 163).}.

Говорити тут разом з Гільбартом (див. примітку) про природну
справедливість є безглуздя. Справедливість угод, які
відбуваються між агентами виробництва, основана на тому, що
ці угоди виникають як природний наслідок з відносин виробництва.
Юридичні форми, в яких виявляються ці економічні
\parbreak{}  %% абзац продовжується на наступній сторінці
