\parcont{}  %% абзац починається на попередній сторінці
\index{iii1}{0023}  %% посилання на сторінку оригінального видання
для першого, 33,3\%  для другого і 20\% для третього капіталіста.
Але якщо четвертий капіталіст $D$ нагромаджує непродуктивний
капітал у 300, який вимагає від $А$ процента“ [зиску]
„вартістю в 40 робочих днів, від $В$ процента в 20 робочих днів,
то норма зиску капіталістів $А$ і $В$ знизиться до 20\%, до норми
зиску $C$, a $D$ з капіталом у 300 одержить зиск у 60, тобто
норму зиску в 20\%, як і решта капіталістів“.

З такою дивовижною вправністю, в одну мить, l’illustre Лоріа
розв’язує те саме питання, яке він десять років тому оголосив
нерозв’язним. На жаль, він не відкрив нам таємниці, звідки
„непродуктивний капітал“ дістає силу не тільки відняти в промисловців
цей їх надзиск, що перевищує пересічну норму
зиску, але й утримати його у своїй кишені цілком так, як земельний
власник кладе собі в кишеню у вигляді земельної ренти надлишковий
зиск орендаря. Дійсно, в такому випадку купці стягали
б з промисловців данину, цілком аналогічну земельній ренті,
і тим самим установлювали б пересічну норму зиску. Правда,
торговельний капітал є дуже істотний фактор в установленні
загальної норми зиску, як це досить відомо кожному. Але тільки
літературний авантюрист, який у глибині свого серця плює на
всю політичну економію, може дозволити собі твердити, що
торговельний капітал має чарівну силу притягувати до себе
всю надлишкову додаткову вартість, яка перевищує загальну
норму зиску, до того ж раніше, ніж ця остання встановлена,
і перетворювати її в земельну ренту для себе самого, та ще
й так, що йому для цього не потрібна ніяка земельна власність.
Не менш дивовижне є твердження, що торговельному капіталові
вдається відкрити тих промисловців, додаткова вартість яких
дорівнює якраз тільки пересічній нормі зиску, і що він вважає для
себе за честь до певної міри полегшити долю цих нещасних жертв
закону вартості Маркса, продаючи їх продукти gratis\footnote*{
безплатно, без будь-якого винагородження. \Red{Ред. укр. перекладу.}
}, навіть
без усякої комісії. Яким треба бути фігляром, щоб уявити собі,
ніби Маркс мав потребу в таких жалюгідних фокусах!

Але в повному блиску своєї слави наш illustre Лоріа виступає
тільки тоді, коли ми порівнюємо його з його північними
конкурентами, наприклад, з паном Юліусом Вольфом, який теж
відомий не з учорашнього дня. Яким дрібним брехуном здається він
поряд з італійцем навіть у своїй товстій книзі про „Соціалізм
і капіталістичний суспільний лад“! Який безпомічний, я навіть
сказав би — який скромний він поряд з тією благородною сміливістю,
з якою маестро видає за щось само собою зрозуміле,
що Маркс не більше й не менше; як і всі інші люди, цілком
такий самий свідомий софіст, паралогіст, хвалько і шахрай, як
сам пан Лоріа, що Маркс кожного разу, коли потрапляє в
скрутне становище, обіцяє публіці дати закінчення своєї теорії
в подальшому томі, якого він, як він це сам дуже добре знає,
\parbreak{}  %% абзац продовжується на наступній сторінці
