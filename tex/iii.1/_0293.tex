\parcont{}  %% абзац починається на попередній сторінці
\index{iii1}{0293}  %% посилання на сторінку оригінального видання
Він дає йому зиск не тим, що безпосередньо створює додаткову
вартість, а тим, що допомагає зменшувати витрати реалізації додаткової
вартості, оскільки він виконує почасти неоплачену працю.
Власне торговельний робітник належить до краще оплачуваного
класу найманих робітників, до тих, праця яких є вправна праця,
яка стоїть вище пересічної праці. Проте, з прогресом капіталістичного
способу виробництва заробітна плата має тенденцію
падати навіть щодо пересічної праці. Почасти в наслідок поділу
праці всередині контори; звідси потреба тільки в однобічному
розвитку працездатності, і витрати на вироблення такої працездатності
капіталістові почасти нічого не коштують; вправність
робітника розвивається самою функцією, і при тому тим швидше,
чим однобічнішою вона стає з поділом праці. Подруге, в наслідок
того, що попередня освіта, знання торговельної справи,
знання мов і т. д. з прогресом науки і народної освіти репродукуються
дедалі швидше, легше, загальніше, дешевше, чим
більше капіталістичний спосіб виробництва скеровує методи
навчання і т. д. на практичні потреби. Всезагальність народної
освіти дозволяє вербувати цей рід робітників з таких класів,
яким раніше не було доступу до цих професій, які звикли до
гіршого життя. До того ж вона збільшує наплив і разом з тим
конкуренцію. Тому, за деякими винятками, з прогресом капіталістичного
способу виробництва робоча сила цих людей знецінюється;
їх заробітна плата знижується, тимчасом як їх
працездатність збільшується. Капіталіст збільшує число цих
робітників, коли треба реалізувати більше вартості й зиску.
Збільшення цієї праці є завжди наслідок, але ніколи не причина
збільшення додаткової вартості.\footnote{
Наскільки справдився пізніше цей прогноз долі торговельного пролетаріату,
зроблений в 1865 році, про це можуть розповісти сотні німецьких
прикажчиків, які, будучи дуже досвідченими в усіх торговельних операціях
і знаючи 3--4 мови, марно пропонують свої послуги в лондонському Сіті за
25 шилінгів на тиждень, — далеко нижче заробітної плати вправного машиніста-слюсаря. — Прогалина в
рукопису на 2 сторінки показує, що цей пункт
малося на думці докладніше розвинути. Зрештою можна вказати на книгу II,
розд. VI (витрати циркуляції), стор. 129--136, [стор. 83--88 рос. вид. 1935 р.],
де вже зачеплено багато з того, що стосується сюди. — \emph{Ф. Е.}
}

Отже, відбувається роздвоєння. З одного боку, функції капіталу
як товарного капіталу і грошового капіталу (який через
це далі визначається як торговельний капітал) є загальні визначеності
форм промислового капіталу. З другого боку, особливі
капітали, отже й особливого роду капіталісти, займаються виключно
цими функціями; і ці функції стають таким чином особливими
сферами зростання вартості капіталу.

Торговельні функції і витрати циркуляції для торговельного
капіталу тільки усамостійнюються. Та сторона промислового капіталу,
якою він звернений до циркуляції, існує не тільки в своєму
постійному бутті як товарний капітал і грошовий капітал,
\parbreak{}  %% абзац продовжується на наступній сторінці
