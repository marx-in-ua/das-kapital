\parcont{}  %% абзац починається на попередній сторінці
\index{iii1}{0035}  %% посилання на сторінку оригінального видання
країн, чим менше, отже, можна було контролювати робочий час, потрібний для їх виготовлення. Адже самі гроші спочатку
надходили здебільшого з чужих країн; навіть тоді, коли благородний метал добувався всередині країни, селянин і ремісник, з
одного боку, не були спроможні навіть приблизно визначити витрачену на нього працю, а, з другого боку, у них самих в
наслідок звички рахувати на гроші була вже досить затемнена свідомість про властивість праці служити мірою вартості; в уяві
народу гроші почали репрезентувати абсолютну вартість.

Одним словом, закон вартості Маркса має силу повсюди, —  оскільки
взагалі мають силу економічні закони, — для всього періоду простого товарного виробництва, отже, до того часу, коли це
останнє зазнає модифікації в наслідок виникнення капіталістичної форми виробництва. До цього моменту ціни тяжать до
визначених за законом Маркса вартостей і коливаються навколо цих вартостей так, що чим повніше розвивається просте товарне
виробництво, тим більше пересічні ціни за довгочасні періоди, не переривані зовнішніми насильними порушеннями, збігаються з
вартостями з точністю до такої незначної величини, що нею можна знехтувати. Отже, закон вартості Маркса має
економічно-загальну силу для періоду, який триває від початку обміну, що перетворив продукти у товари, і аж до XV століття
нашого літочислення. А товарний обмін починається з того часу, який передує будьякій писаній історії і веде нас у Давнину —
в Єгіпті щонайменше на дві з половиною, а може й на п’ять тисяч років, у Вавілонії — на чотири, а може й на шість тисяч
років до нашого літочислення; отже, закон вартості панував протягом періоду в п’ять-сім тисяч років. І от після цього
полюбуйтесь на глибокодумність пана Лоріа, що називає вартість, яка протягом цього часу мала загальне й безпосереднє
значення, такою вартістю, по якій товари ніколи не продавались і не можуть продаватись і якою ніколи не займатиметься
жоден економіст, коли він має хоч іскру здорового розсудку!

Досі ми не говорили про купця. Ми могли не звертати уваги на його
втручання до цього моменту, коли ми переходимо до перетворення простого товарного виробництва в капіталістичне товарне
виробництво. Купець був революційним елементом у цьому суспільстві, де все інше було стабільним, стабільним, так би мовити,
в порядку спадковості; де селянин не тільки свій наділ, але й своє становище вільного власника, вільного чи залежного
оброчника або кріпака, а міський ремісник своє ремесло і свої цехові привілеї діставали в спадщину і майже невідчужувано, і,
крім того, кожний з них діставав у спадщину свою клієнтуру, свій ринок збуту, цілком так само, як і вироблену з юнацтва
вправність в успадкованій професії. І ось у цей світ вступив купець," який мав стати вихідним пунктом перевороту в цьому,
світі. Але не як свідомий революціонер; навпаки, як плоть від плоті, кість від кості цього
\parbreak{}  %% абзац продовжується на наступній сторінці
