\parcont{}  %% абзац починається на попередній сторінці
\index{iii1}{0027}  %% посилання на сторінку оригінального видання
може справді бути корисним для того, щоб усунути труднощі розуміння, щоб більше висунути на
передній план важливі моменти, значення яких виступає в тексті недосить виразно, і щоб до
тексту, написаного в 1865 році, зробити окремі важливіші доповнення відповідно до стану речей в
1895 році. Справді, вже зараз є два таких пункти, відносно яких мені здається необхідним дати
коротке пояснення.

І. ЗАКОН ВАРТОСТІ І НОРМА ЗИСКУ

Слід було сподіватися, що розв’язання позірної суперечності між цими обома факторами і після
опублікування тексту Маркса так само викликатиме спори, як і до цього опублікування. Дехто чекав
справжнього чуда і почуває себе розчарованим, бачачи перед собою замість сподіваного фокус-покуса
просте раціональне, прозаічно-тверезе розв’язання суперечності (Vermittlung
des Gegensatzes). Звичайно, більш за всіх злорадо розчарований уже відомий нам illustre Лоріа.
Нарешті він знайшов Архімедову точку опори,«за допомогою якої навіть нікчемний чоловічок його
калібру може підняти і висадити в повітря міцно споруджену гігантську будову марксизму. Як —
вигукує він з обуренням — це має бути розв’язанням? Та це ж чистісінька містифікація! Коли
економісти говорять про вартість, то вони говорять про таку вартість, яка фактично встановлюється
в обміні. „А займатись такою вартістю, по якій товари не тільки не продаються, але й не можуть
продаватись (nè possono vendersi mai), цього не робив ще, та й ніколи не зробить, жоден
економіст, який має хоч краплину розсудку... Коли Маркс твердить, що вартість,- по якій товари
ніколи не продаються, визначається пропорціонально до вміщеної в них праці, то хіба це не є тільки повторенням у
перевернутій формі твердження правовірних економістів, що вартість, по
якій продаються товари, не стоїть ні в якій пропорції до витраченої на них праці?..  Ні трохи не
допомагає й те, коли Маркс каже, що, не зважаючи
на відхилення одиничних цін від одиничних вартостей, сукупна ціна всіх товарів завжди збігається
з їх сукупною вартістю або з кількістю праці, вміщеної в сукупній масі товарів. Бо через те, що
вартість є не що інше, як відношення, в якому один товар обмінюється на інший, само вже поняття
сукупної вартості є абсурд, безглуздя... contradictio in adjecto [суперечність у самому визначенні]“. На самому
початку свого твору Маркс, мовляв, каже, що обмін може прирівнювати два товари
тільки тому, що вони містять у собі рівні кількості однорідного елементу, а саме рівні кількості
праці. А тепер він урочисто відрікається від самого себе, запевняючи, що товари обмінюються в
цілком іншій пропорції, а не
пропорціонально до вміщених в них кількостей праці. „Чи видане було колинебудь таке цілковите
приведення ad absurdum [до абсурду], більше теоретичне банкрутство? Чи чинилося
\parbreak{}  %% абзац продовжується на наступній сторінці
