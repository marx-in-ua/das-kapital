\noindent{}де в обох випадках $m' \deq{} 100\%, р' \deq{} 10\%$, і де порівняння з попереднім
капіталом є далеко наочнішим у процентній формі.

Навпаки, коли йдеться про зміни, які відбуваються з одним
і тим самим капіталом, то процентну форму можна вживати
тільки зрідка, бо вона майже завжди стирає ці зміни. Якщо
капітал переходить з процентної форми:\[
80c \dplus{} 20v \dplus{} 20m\]

у процентну форму:\[
90c \dplus{} 10v \dplus{} 10m,\]

то не видно, чи цей змінений процентний склад $90c \dplus{} 10v$ виник
у наслідок абсолютного зменшення в чи в наслідок абсолютного
збільшення $c$, чи в наслідок того й другого. Для цього ми мусимо
знати абсолютні числові величини. Але для дослідження
дальших окремих випадків змін все зводиться до того, яким
чином сталися ці зміни: чи $80c \dplus{} 20v$ обернулись у $90c \dplus{} 10v$
через те що, скажімо, $\num{12000}c \dplus{} 3000v$ в наслідок збільшення
сталого капіталу при незмінній величині змінного капіталу перетворилися
в $\num{27000}c \dplus{} 3000v$ (в процентах $90c \dplus{} 10v$); чи вони
набрали цієї форми при незмінному сталому капіталі в наслідок
зменшення змінного капіталу, тобто в наслідок переходу у
$\num{12000}c \dplus{} 1333\sfrac{1}{3}v$(в процентах так само $90c \dplus{} 10v$); чи, нарешті,
в наслідок зміни обох доданків, скажімо, $\num{13500}c \dplus{} 1500v$
(в процентах знову $90c \dplus{} 10v$). Але ми будемо досліджувати
якраз всі ці випадки один за одним, і тому нам доведеться
відмовитись від зручностей процентної форми або застосовувати
її тільки в другу чергу.
\begin{center}
  \textbf{1. $m'$ і $К$ не змінюються, $v$ змінюється}
\end{center}
Якщо $v$ змінює свою величину, $К$ може лишитися незмінним
тільки в наслідок того, що друга складова частина $К$, а саме
сталий капітал $c$, змінює свою величину на таку саму суму, що й $v$,
але в протилежному напрямі. Якщо $К$ спочатку $= 80c \dplus{} 20 \deq{} 100$
і якщо потім $v$ зменшується до 10, то $К$ може лишитися \deq{} 100
тільки тоді, коли $c$ підвищується до 90; $90c \dplus{} 10v \deq{} 100$. Загалом
кажучи: якщо $v$ перетворюється у $v \pm d$, у $v$, збільшене або
зменшене на $d$, то, щоб були задоволені умови даного випадку,
$c$ мусить перетворитися в $c \mp d$, мусить змінитися на таку саму
суму, але в протилежному напрямі.

Цілком так само при незмінній нормі додаткової вартості $m'$,
але при змінній величині змінного капіталу $v$, маса додаткової
вартості $m$ мусить змінитися, бо $m \deq{} m'v$, а в $m'v$ один множник,
$v$, набуває іншого значення.

Припущення нашого випадку поряд первісного рівняння:\[
р' \deq{} m'\frac{v}{K}\]
\parbreak{}  %% абзац продовжується на наступній сторінці
