
\index{iii1}{0115}  %% посилання на сторінку оригінального видання
Шерстяна промисловість була розсудливіша, ніж льонообробна промисловість. „Раніше, звичайно,
вважалося ганебним
збирати відпади шерсті та шерстяні ганчірки для повторного
перероблення, але цей передсуд цілком зник у shoddy trade
(виробництві штучної шерсті), яке стало важливою галуззю
шерстяної промисловості Йоркшірської округи, і немає сумніву,
що й підприємства, що переробляють відпади бавовни, незабаром теж займуть те саме місце, як галузь
промисловості, що
задовольняє визнані потреби. 30 років тому шерстяні ганчірки,
тобто шматки тканини з чистої вовни і~\abbr{т. д.}, коштували пересічно щось 4\pound{ фунти стерлінгів} 4\shil{ шилінги}
за тонну; протягом
останніх кількох років вони стали коштувати 44\pound{ фунти стерлінгів} за тонну. А попит на них так
збільшився, що використовується
навіть мішана тканина з вовни й бавовни, бо знайдено засіб
руйнувати бавовну без пошкодження вовни; і тепер тисячі робітників зайняті у фабрикації shoddy, а
споживач має з того
велику користь, оскільки він тепер може купити сукно доброї
середньої якості за дуже помірну ціну“ („Rep. of Insp. of Fact.,
Oct. 1863“, стор. 107). Поновлювана таким чином штучна шерсть
уже в кінці 1862 року становила третину всього споживання
вовни англійською промисловістю („Rep. of Insp. of Fact., Oct.
1862“, стор. 81). „Велика користь“ для „споживача“ полягає в тому,
що його шерстяний одяг зношується втричі швидше, ніж раніше,
і вшестеро швидше витирається до ниток.

Англійська шовкова промисловість посувалась по тій самій
похилій площині. З 1839 до 1862 року споживання натурального
шовку-сирця трохи зменшилося, тим часом як споживання шовкових відпадів подвоїлося. Поліпшені машини
дали змогу фабрикувати з цього, за інших умов майже нічого не вартого матеріалу, шовк, придатний для
багатьох цілей.

Найразючіший приклад застосування відпадів дає хімічна
промисловість. Вона споживає не тільки свої власні відпади,
знаходячи для них нове застосування, але й відпади найрізнорідніших інших галузей промисловості, і
перетворює, наприклад,
майже некорисний раніше кам’яновугільний дьоготь в анілінові
фарби, в красильну речовину крапу (алізарин), а останнім часом
також у медикаменти.

Від цієї економії на відпадах виробництва в наслідок повторного використання їм треба відрізняти
економію при утворенні
самих відпадів, тобто зведення екскрементів виробництва до
їх мінімуму і безпосереднє максимальне використання всіх сировинних та допоміжних матеріалів, що
входять у виробництво.

Заощадження на відпадах почасти зумовлене якістю застосовуваних машин. Мастило, мило тощо
заощаджуються тим
більше, чим точніше працюють окремі частини машин, і чим краще
вони відполіровані. Це стосується до допоміжних матеріалів.
А почасти, і це найважливіше, від якості застосовуваних машин
і знарядь залежить те, більша чи менша частина сировинного
\parbreak{}  %% абзац продовжується на наступній сторінці
