
\index{iii1}{0410}  %% посилання на сторінку оригінального видання
„3805. Якщо кількість грошей у країні зменшується в наслідок
їх відпливу, то їхня вартість підвищується, і Англійський банк
мусить пристосуватись до цієї зміни в вартості грошей“. [Отже, у вартості грошей \emph{як капіталу}, іншими
словами, в розмірі процента, бо вартість грошей \emph{як грошей}, порівняно з товарами,
лишається та сама, що й раніше.] „Технічно це виражають так,
що банк підвищує розмір процента“.

„3819. Я ніколи не змішую їх одне з одним“. — Тобто гроші
і капітал, з тієї простої причини, що він їх ніколи не розрізняє.

„3834. Дуже велика сума, яку довелося заплатити“ [за хліб
в 1847 році] „за необхідні засоби існування країни і яка \emph{в дійсності була капіталом}“.

„3841. Коливання в нормі дисконту мають, безсумнівно, дуже
тісний зв’язок з станом золотого резерву“ [Англійського
банку], „бо стан резерву є показником збільшення або зменшення
наявної в країні кількості грошей; і в тій самій пропорції, в якій
збільшується чи зменшується кількість грошей у країні, падає або
підвищується вартість грошей, і банкова норма дисконту пристосовується до цього“. — Отже, тут
Оверстон визнає те, що він
категорично заперечував у своєму свідченні № 3755. — „3842. Між
тим і другим існує тісний зв’язок“. — А саме між кількістю золота
в Issue department [емісійному департаменті Англійського банку]
і резервом банкнот у Banking department [власне банковому департаменті]. Тут він пояснює зміну в
розмірі процента зміною в кількості грошей. При цьому він каже неправду. Резерв може зменшитись,
тому що в країні збільшується кількість циркулюючих грошей.
Це має місце тоді, коли публіка бере більше банкноти, а металічний запас не зменшується. Але в
такому випадку підвищується
розмір процента, тому що банковий капітал Англійського банку
за законом 1844 року обмежений. Але про це він не повинен
би був говорити, бо в наслідок цього закону обидва департаменти банку не мають між собою нічого
спільного.

„3859. Висока норма зиску завжди викликатиме великий попит
на капітал; великий попит на капітал підвищить його вартість“. — Отже, такий є, нарешті, зв’язок між
високою нормою
зиску і попитом на капітал, як його собі уявляє Оверстон. Але
ось, наприклад, в 1844--1845 рр. у бавовняній промисловості
панувала висока норма зиску, тому що при великому попиті на
бавовняні товари бавовна-сирець була дешева і лишилась дешевою. Вартість капіталу (а, як ми це вже
бачили вище, Оверстон називає капіталом те, чого кожний потребує для свого
підприємства), отже, в даному разі вартість бавовни-сирця, не
підвищилась для фабриканта. Ця висока норма зиску могла
заохотити якогось бавовняного фабриканта позичити грошей для
розширення свого підприємства. В такому разі підвищився б його
попит на \emph{грошовий} капітал і ні на що інше.

„3889. Золото може бути й не бути грішми, цілком так само
як папір може бути і не бути банкнотою“.
