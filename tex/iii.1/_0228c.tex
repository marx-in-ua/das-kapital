\parcont{}  %% абзац починається на попередній сторінці
\index{iii1}{0228}  %% посилання на сторінку оригінального видання
капітал дорівнює сумі витрат виробництва, а саме \num{10000}\pound{ фунтам стерлінгів}, то й $\frac{p}{K}$, дійсна норма
зиску, на цей раз \deq{} 20\%.

III.~Припустім, що капітал при постійно зростаючій продуктивній силі праці збільшується до \num{15000}\pound{ фунтів стерлінгів} і виробляє тепер щорічно \num{30000} штук товару при витратах виробництва в 13\shil{ шилінгів}
на штуку, при чому кожна штука продається з зиском в 2\shil{ шилінги}, отже, по 15\shil{ шилінгів.} Отже, річний
оборот \deq{} 15\shil{ шилінгів} × З0000 \deq{} \num{22500}\pound{ фунтам стерлінгів}, з яких \num{19500}\pound{ фунтів стерлінгів}
авансованого капіталу і 3000\pound{ фунтів стерлінгів} зиску. Отже,
$\frac{p}{k} \deq{} \frac{2}{13} \deq{} \frac{3000}{\num{19500}} \deq{} 15\sfrac{5}{13}/\%$.
Навпаки, $\frac{p}{K} \deq{} \frac{3000}{\num{15000}} \deq{} 20\%$.

Отже, ми бачимо, що тільки у випадку II, де капітальна вартість, яка обернулася, дорівнює всьому
капіталові, норма зиску на штуку товару або на суму обороту є така сама, як і норма зиску, обчислена
на весь капітал. У випадку І, де сума обороту менша, ніж весь капітал, норма зиску, обчислена на
витрати виробництва товару, є вища; у випадку III, де весь капітал менший, ніж сума обороту, вона
нижча, ніж дійсна норма зиску, обчислена на весь капітал. Це має загальне значення.

В купецькій практиці оборот звичайно обчислюється неточно. Припускається, що капітал обернувся один
раз, коли сума реалізованих товарних цін досягає суми всього застосованого капіталу. Але \emph{капітал}
може тільки тоді завершити повний оборот, коли сума \emph{витрат виробництва} реалізованих товарів
дорівнюватиме сумі всього капіталу. — Ф.~E.].

І тут знову виявляється, як важливо при капіталістичному виробництві розглядати окремий товар або
товарний продукт, вироблений протягом якогось періоду часу, не ізольовано, не сам по собі, не як
простий товар, а як продукт авансованого капіталу і відносно всього капіталу, який виробляє ці
товари.

Хоча \emph{норму} зиску слід обчислювати, вимірюючи масу виробленої і реалізованої додаткової вартості не
тільки відносно спожитої частини капіталу, яка знову з’являється в товарах, але відносно цієї
частини плюс та частина капіталу, яка не спожита, але застосована і продовжує служити у виробництві,
— проте \emph{маса} зиску може дорівнювати тільки тій масі зиску або додаткової вартості, яка міститься в
самих товарах і має бути реалізована через їх продаж.

Якщо продуктивність промисловості збільшується, то ціна окремого товару падає. В ньому міститься
менше праці як оплаченої, так і неоплаченої. Припустім, що та сама праця виробляє, наприклад, утроє
більше продукту; тоді на кожний | окремий продукт припадає праці на \sfrac{2}{3} менше. А через те що зиск
може становити тільки частину цієї вміщеної в кожному
\parbreak{}  %% абзац продовжується на наступній сторінці
