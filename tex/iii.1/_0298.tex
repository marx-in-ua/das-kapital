\parcont{}  %% абзац починається на попередній сторінці
\index{iii1}{0298}  %% посилання на сторінку оригінального видання
частини продажної ціни кожного фунта цукру, яка становить
торговельний зиск, тобто тієї надбавки до ціни, яку робить
купець на певну кількість товару (продукту). Якщо ціна виробництва
товару незначна, то незначна й та сума, яку купець
авансує на купівельну ціну товару, тобто на певну масу його,
а тому, при даній нормі зиску, незначна й та сума зиску, яку
він одержує на цю дану кількість дешевого товару; або, що
зводиться до того самого, він може тоді на даний капітал, наприклад,
в 100, купити велику масу цього дешевого товару,
і загальний зиск у 15, який він одержує на ці 100, розподіляється
маленькими частинами на кожну окрему штуку цієї товарної
маси. І навпаки. Це цілком і повністю залежить від більшої
чи меншої продуктивності того промислового капіталу, товарами
якого він торгує. Якщо виключити випадки, коли купець є монополіст
і разом з тим монополізує виробництво, як, наприклад,
у свій час голландсько-ост-індська компанія, то не може бути
нічого більш безглуздого, ніж ходяче уявлення, ніби від купця
залежить, чи продасть він багато товарів з невеликим зиском,
чи мало товарів з великим зиском на одиницю товару. Дві
межі існують для його продажної ціни: з одного боку, ціна виробництва
товару, яка від нього не залежить; з другого боку, пересічна
норма зиску, яка від нього так само не залежить. Єдине,
що він може вирішувати, — при чому, однак, величина капіталу,
який є в його розпорядженні, та інші обставини також грають
певну роль, — це те, чи хоче він торгувати дорогими чи дешевими
товарами. Тому поведінка купця тут цілком і повністю залежить
від ступеня розвитку капіталістичного способу виробництва,
а не від його бажання. Тільки така торговельна компанія, як стара
голландсько-ост-індська, яка мала монополію виробництва, могла
прийти до думки при цілком змінених відносинах додержувати
методу, який відповідав щонайбільше початкам капіталістичного
виробництва\footnote{
„Profit, on the general principle, is always the same, whatever be price;
keeping its place like an incumbent body on the swelling or sinking tide. As,
therefore, prices rise, a tradesman raises prices; as prices fall, a tradesman lowers
price“. [„Зиск, за загальним правилом, завжди лишається той самий, яка б не була
ціна; він зберігає своє місце подібно до пливучого тіла при припливі й відпливі.
Тому, коли ціни підвищуються, торговець підвищує ціну; коли ціни падають,
торговець знижує ціну“]. (\emph{Corbet}: „An Inquiry into the Causes etc. of the Wealth
of Individuals“. Лондон 1841, стор. 20). — Тут, як і взагалі в тексті, мова йде
тільки про звичайну торгівлю, а не про спекуляцію, дослідження якої, як і взагалі
все те, що стосується до поділу торговельного капіталу, виходить поза
межі нашого дослідження. „The profit of trade is a value added to capital which
is independent of price, the second (speculation) is founded on the variation in
the value of capital or in price itself [„Торговельний зиск є додана до капіталу
вартість, яка не залежить від ціни; друга (спекуляція) основана на зміні вартості
капіталу або самої ціни“] (там же, стор. 12).
}.

Цей популярний передсуд, — який, зрештою, як і всі хибні
уявлення про зиск і~\abbr{т. п.}, випливає із спостерігання самої тільки
торгівлі і з купецького передсуду, — зберігається досі, підтримуваний
між іншим такими обставинами:
