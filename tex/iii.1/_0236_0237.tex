
Інше питання — яке в наслідок своєї специфічності лежить власне поза межами нашого дослідження —
таке: чи підвищується загальна норма зиску в наслідок вищої норми зиску, яку дає капітал, вкладений
у зовнішню, особливо в колоніальну торгівлю?

\looseness=1
\disablefootnotebreak{} 
Капітали, вкладені в зовнішню торгівлю, можуть давати вищу норму зиску тому, що тут, поперше,
відбувається конкуренція з товарами, вироблюваними іншими країнами при менш легких умовах
виробництва, так що більш розвинена країна продає свої
товари вище їх вартості, хоч і дешевше, ніж конкуруючі країни. Оскільки праця більш розвиненої
країни оцінюється тут як праця вищої питомої ваги, норма зиску підвищується в наслідок того, що
праця, не оплачувана як праця вищої якості, продається як така. Те саме може мати місце відносно
тієї країни, до якої відправляються товари і з якої одержуються товари; а саме, така країна віддає
більше упредметненої праці in natura, ніж вона одержує, і все ж при цьому одержує товари дешевше,
ніж вона сама могла б їх виробити. Цілком так само, як фабрикант,
який використовує новий винахід раніше, ніж він стає загальнопоширеним, продає дешевше своїх
конкурентів, і все ж продає свої товари вище їх індивідуальної вартості, тобто специфічно вищу
продуктивну силу вживаної ним праці використовує як додаткову працю. Він реалізує таким чином
надзиск. З другого боку, щодо капіталів, вкладених у колоніях і~\abbr{т. д.}, то вони можуть давати вищі
норми зиску тому, що там в наслідок нижчого розвитку, норма зиску взагалі стоїть вище, а при умові
вживання рабів, кулі і~\abbr{т. п.}, стоїть вище і експлуатація праці. Не можна зрозуміти, чому ці вищі
норми зиску, що їх таким чином дають і відправляють до батьківщини капітали, вкладені в певні
галузі, тут, якщо тільки цьому не перешкоджає монополія, не повинні були б увіходити в процес
вирівнення загальної норми зиску і тому pro tanto [відповідно до цього] підвищувати її\footnote{
А.~Сміт тут має рацію проти Рікардо, який каже: They contend the equality of profits will be
brought about by the general rise of profits; and I am of opinion that the profits of the favoured
trade will speedily submit to the general level“ [„Вони твердять, що рівність зисків буде здійснена
через загальне підвищення зисків; а я тієї думки, що зиски підприємства, яке є в сприятливіших
умовах, швидко знизяться до загального рівня“]. ([\emph{Ricardo}:] „Works“, видання Мак-Куллоха, стор. 73).
}. 
\enablefootnotebreak{}

Особливо
цього не можна зрозуміти, якщо зазначені
\index{iii1}{0237}  %% посилання на сторінку оригінального видання
галузі капіталовкладень підлягають законам вільної конкуренції. Навпаки, Рікардо вбачається
таке: на гроші, одержані за кордоном від продажу по вищій ціні, там купуються товари і
відправляються на заміну додому; отже, ці товари продаються
всередині країни, і тому це може становити, принаймні тимчасово, особливу, порівняно з іншими
сферами, невигоду для сфер виробництва, які є в сприятливих умовах. Ця ілюзія відпадає, як тільки ми
абстрагуємось від грошової форми. Країна,
яка перебуває в сприятливіших умовах, одержує назад більше праці в обмін за меншу кількість праці,
хоч ця ріжниця, цей надлишок, як взагалі при обміні між працею і капіталом, привласнюється певним
класом. Отже, оскільки норма зиску є вища, тому
що вона взагалі вища в колоніальній країні, це при сприятливих природних умовах цієї країни може йти
рука в руку з низькими товарними цінами. Вирівнення відбувається, але вирівнення не за старим
рівнем, як гадає Рікардо.

Але та сама зовнішня торгівля розвиває всередині країни капіталістичний спосіб виробництва і тим
самим веде до зменшення змінного капіталу порівняно з сталим; з другого боку, вона створює
перепродукцію відносно закордону і тому в дальшому перебігу знов таки справляє протилежний вплив.

І таким чином взагалі виявляється, що ті самі причини, які приводять до падіння загальної норми
зиску, викликають протилежні впливи, які гальмують, уповільнюють і почасти паралізують це падіння.
Вони не знищують закону, але ослаблюють його діяння. Без цього було б незрозумілим не падіння
загальної норми зиску, а, навпаки, відносна повільність цього падіння. Таким чином закон діє тільки
як тенденція, вплив якої виразно виступає тільки при певних обставинах і на протязі довгих періодів
часу.

Раніше, ніж піти далі, ми, щоб уникнути непорозумінь, повторимо ще два, вже не раз розвинуті
положення.

\emph{Поперше}: Той самий процес, який в ході розвитку капіталістичного способу виробництва породжує
здешевлення товарів, породжує також зміну в органічному складі суспільного капіталу, застосовуваного
для виробництва товарів, а в наслідок цього і падіння норми зиску. Отже, зменшення відносних витрат
на окремий товар, а також тієї частини цих витрат, яка містить у собі зношування машин, не слід
ототожнювати з зростанням вартості сталого капіталу порівняно з змінним, хоча, навпаки, всяке
зменшення відносних витрат на сталий капітал, при незмінному або зростаючому розмірі його речових
елементів, впливає на підвищення норми зиску, тобто на зменшення pro tanto [відповідно до цього]
вартості сталого капіталу порівняно з змінним капіталом, застосовуваним в дедалі менших пропорціях.

\emph{Подруге}: Та обставина, що в окремих товарах, сукупність яких становить продукт капіталу, відношення
додаваної живої праці, яка міститься в них, до уміщених в них матеріалів праці
\parbreak{}  %% абзац продовжується на наступній сторінці
