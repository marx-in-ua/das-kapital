чином вона прискорює, з одного боку, нагромадження, але, з другого боку, і зменшення змінного
капіталу порівняно з сталим, а, значить, і падіння норми зиску. Так само, хоч розширення зовнішньої
торгівлі в дитинстві капіталістичного способу виробництва було його базою, однак з його розвитком
воно, в наслідок внутрішньої необхідності цього способу виробництва, в наслідок його потреби в
дедалі ширшому ринку, стало власним його продуктом. Тут знову виявляється та сама двобічність впливу
(Рікардо цілком проглядів цей бік зовнішньої торгівлі).

Інше питання — яке в наслідок своєї специфічності лежить власне поза межами нашого дослідження —
таке: чи підвищується загальна норма зиску в наслідок вищої норми зиску, яку дає капітал, вкладений
у зовнішню, особливо в колоніальну торгівлю?

Капітали, вкладені в зовнішню торгівлю, можуть давати вищу норму зиску тому, що тут, поперше,
відбувається конкуренція з товарами, вироблюваними іншими країнами при менш легких умовах
виробництва, так що більш розвинена країна продає свої
товари вище їх вартості, хоч і дешевше, ніж конкуруючі країни. Оскільки праця більш розвиненої
країни оцінюється тут як праця вищої питомої ваги, норма зиску підвищується в наслідок того, що
праця, не оплачувана як праця вищої якості, продається як така. Те саме може мати місце відносно
тієї країни, до якої відправляються товари і з якої одержуються товари; а саме, така країна віддає
більше упредметненої праці in natura, ніж вона одержує, і все ж при цьому одержує товари дешевше,
ніж вона сама могла б їх виробити. Цілком так само, як фабрикант,
який використовує новий винахід раніше, ніж він стає загальнопоширеним, продає дешевше своїх
конкурентів, і все ж продає свої товари вище їх індивідуальної вартості, тобто специфічно вищу
продуктивну силу вживаної ним праці використовує як додаткову працю. Він реалізує таким чином
надзиск. З другого боку, щодо капіталів, вкладених у колоніях і т. д., то вони можуть давати вищі
норми зиску тому, що там в наслідок нижчого розвитку, норма зиску взагалі стоїть вище, а при умові
вживання рабів, кулі і т. п., стоїть вище і експлуатація праці. Не можна зрозуміти, чому ці вищі
норми зиску, що їх таким чином дають і відправляють до батьківщини капітали, вкладені в певні
галузі, тут, якщо тільки цьому не перешкоджає монополія, не повинні були б увіходити в процес
вирівнення загальної норми зиску і тому pro tanto [відповідно до цього] підвищувати її\footnote{
А. Сміт тут має рацію проти Рікардо, який каже: They contend the equality of profits will be
brought about by the general rise of profits; and I am of opinion that the profits of the favoured
trade will speedily submit to the general level“ [„Вони твердять, що рівність зисків буде здійснена
через загальне підвищення зисків; а я тієї думки, що зиски підприємства, яке є в сприятливіших
умовах, швидко знизяться до загального рівня“]. ([Ricardo:] „Works“, видання Мак-Куллоха, стор. 73).
}. Особливо
цього не можна зрозуміти, якщо зазна