\parcont{}  %% абзац починається на попередній сторінці
\index{iii1}{0207}  %% посилання на сторінку оригінального видання
засоби існування, отже, без зміни вартості товарів, що входять у споживання робітника.

Абож змінюється відношення суми привласнюваної додаткової вартості до сукупного авансованого
капіталу суспільства. Через те що зміна походить тут не від норми додаткової вартості, вона мусить
походити від сукупного капіталу, а саме від його сталої частини. Маса цієї частини, розглядувана з
технічного боку, збільшується або зменшується пропорціонально до кількості робочої сили, купленої
змінним капіталом, а вартість цієї частини зростає або падає таким чином разом із зростанням чи
зменшенням самої її маси; отже, вона так само зростає або падає пропорціонально до маси вартості
змінного капіталу. Якщо та сама кількість праці приводить в рух більше сталого капіталу, то праця
стала продуктивнішою. У зворотному випадку — навпаки. Отже, сталася зміна в продуктивності праці, і
мусить відбутися зміна вартості певних товарів.

Отже, для обох випадків має силу такий закон: якщо змінюється ціна виробництва якогось товару в
наслідок зміни загальної норми зиску, то хоч власна вартість цього товару може лишитись незмінною,
проте мусить відбутися зміна вартості інших товарів.

\emph{Подруге}. Загальна норма зиску лишається незмінною. Тоді ціна виробництва товару може змінитися
тільки тому, що змінилась його власна вартість; що потрібно більше або менше праці для того, щоб
репродукувати самий товар, в наслідок зміни продуктивності або тієї праці, що виробляє даний товар у
його остаточній формі, або тієї, що виробляє товари, які входять у виробництво даного товару.
Бавовняна пряжа може упасти в ціні виробництва або тому, що дешевше виготовляється бавовна-сирець,
або тому, що праця прядіння в наслідок поліпшення машин стала продуктивнішою.

Ціна виробництва, як уже показано раніш, $= k \dplus{} p$, дорівнює витратам виробництва плюс зиск. Але це
$= k \dplus{} kp'$, де $k$, витрати виробництва, невизначена величина, яка для різних сфер виробництва змінюється
і повсюди дорівнює вартості сталого й змінного капіталу, спожитого на виробництво товару, а $p'$ є
обчислена в процентах пересічна норма зиску. Якщо $k \deq{} 200$, а $p' \deq{} 20\%$, то ціна виробництва $k \dplus{} kp' \deq{}
200 \dplus{} 200\cdot\frac{20}{100} \deq{} 200
\dplus{} 40 \deq{} 240$. Очевидно, що ця ціна виробництва може лишатися незмінною, хоч вартість товарів
змінюється.

Всякі зміни в ціні виробництва товарів зводяться в кінцевому рахунку до зміни вартості; але не всяка
зміна вартості товарів виражається в зміні ціни виробництва, бо ця остання визначається не тільки
вартістю даного товару, але й сукупною вартістю всіх товарів. Отже, зміна в товарі \emph{А} може бути
урівноважена протилежною зміною товару \emph{В}, так що загальне відношення лишається незмінним.
