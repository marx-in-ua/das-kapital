
\index{iii1}{0353}  %% посилання на сторінку оригінального видання
На грошовому ринку протистоять один одному тільки позикодавець
і позичальник. Товар має одну й ту саму форму —
гроші. Всі особливі види капіталу, які він має залежно від
вкладення його в окремі сфери виробництва або циркуляції,
тут зникли. Він існує тут у нерозрізнимому, самому собі рівному
вигляді самостійної вартості, у вигляді грошей. Конкуренція між
окремими сферами тут припиняється; всі вони звалені в одну купу
як позичальники грошей, а капітал також протистоїть їм усім
у такій формі, в якій він ще індиферентний до певного роду й способу
свого застосування. Тут, в попиті й поданні капіталу, останній,
дійсно, цілковито виступає \emph{сам по собі} як \emph{спільний капітал
усього класу}, тимчасом як промисловий капітал виступає
таким тільки в русі й конкуренції між окремими сферами.
З другого боку, грошовий капітал на грошовому ринку дійсно
має таку форму, в якій він, як спільний елемент, індиферентний до
окремого способу свого застосування, розподіляється між різними
сферами, між класом капіталістів, залежно від потреб
виробництва кожної окремої сфери. Сюди долучається ще й те,
що з розвитком великої промисловості грошовий капітал, оскільки
він з’являється на ринку, все більше й більше представлений не
окремим капіталістом, власником тієї чи іншої частини капіталу,
який перебуває на ринку, а виступає як концентрована,
організована маса, яка цілком інакше, ніж реальне виробництво,
поставлена під контроль банкірів, які є представниками суспільного
капіталу. Таким чином, оскільки справа стосується форми попиту,
капіталові, що дається в позику, протистоїть весь клас;
як і сам він, оскільки справа стосується подання, виступає як
позиковий капітал en masse [всією масою].

Ось деякі з причин того, чому загальна норма зиску здається
розпливчатим міражем поруч з певним розміром процента, величина
якого, правда, коливається, але через те що вона коливається
рівномірно для всіх позичальників, то завжди протистоїть
їм як фіксований даний розмір. Цілком так само, як
зміна вартості грошей не перешкоджає їм мати відносно всіх
товарів однакову вартість. Цілком так само, як щоденні коливання
ринкових цін товарів не перешкоджають тому, щоб ці ціни
щодня відзначались у бюлетенях. Цілком так само і з розміром
процента, який з такою самою реґулярністю відзначається як
„ціна грошей“. Це тому, що тут як товар пропонується сам
капітал у грошовій формі; тому фіксація його ціни є фіксація
його ринкової ціни, як і в усіх інших товарів; тому розмір процента
завжди виступає як загальний розмір процента, як
стільки то грошей за стільки то грошей, як кількісно визначений
розмір. Навпаки, норма зиску навіть у межах однієї
і тієї ж сфери, при однакових ринкових цінах товару, може
бути різна, залежно від різних умов, при яких окремі капітали
виробляють той самий товар; бо норма зиску для окремого
капіталу визначається не ринковою ціною товару, а ріжницею
\parbreak{}  %% абзац продовжується на наступній сторінці
