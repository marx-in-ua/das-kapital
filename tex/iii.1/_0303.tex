\parcont{}  %% абзац починається на попередній сторінці
\index{iii1}{0303}  %% посилання на сторінку оригінального видання
цієї швидкості обороту, одна й та сама маса зиску, яка при даній
величині купецького капіталу визначається загальною річною
нормою зиску, тобто визначається незалежно від спеціального
характеру купецьких операцій цього капіталу, по-різному розподіляється
на товарні мари однакової вартості, — наприклад, при п’ятикратному
обороті за рік до ціни товару додається \sfrac{15}{5} \deq{} 3\%,
навпаки, при однократному обороті за рік — 15\%.

Отже, та сама процентна норма торговельного зиску в різних
галузях торгівлі підвищує відповідно до часу обороту в них
продажні ціни товарів на зовсім різні проценти, обчислювані
відносно вартостей цих товарів.

Навпаки, в промисловому капіталі час обороту не справляє
ніякого впливу на величину вартості вироблюваної одиниці товарів,
хоч він впливає на масу вартостей і додаткових вартостей,
вироблюваних даним капіталом за даний час, бо впливає на
масу експлуатованої праці. Звичайно, це маскується, і здається,
ніби справа стоїть інакше, якщо мати на увазі ціни виробництва;
але це тільки тому, що ціни виробництва різних товарів, згідно
з викладеними раніш законами, відхиляються від їх вартостей.
Якщо розглядати сукупний процес виробництва, товарну масу,
вироблену сукупним промисловим капіталом, то відразу ж виявимо,
що загальний закон підтверджується.

Отже, тимчасом як докладніше дослідження впливу часу
обороту на творення вартості в промисловому капіталі знову
приводить до того загального закону і бази політичної економії,
що вартості товарів визначаються вміщеним у них робочим
часом, вплив оборотів купецького капіталу на торговельні ціни
показує такі явища, які, якщо не проробити дуже грунтовного аналізу
проміжних ланок, нібито передбачають чисто довільне визначення
цін; а саме, визначення їх просто тим, що капітал раптом
вирішує одержати протягом року певну кількість зиску. Саме
в наслідок такого впливу оборотів здається, ніби ціни товарів
визначає процес циркуляції як такий — незалежно, в певних межах,
від процесу виробництва. Всі поверхові й перекручені
погляди на сукупний процес репродукції запозичені з спостережень
купецького капіталу і з тих уявлень, які викликаються
в головах агентів циркуляції його своєрідними рухами.

Коли, як про це, на свій жаль, дізнався читач, аналіз дійсних,
внутрішніх зв’язків капіталістичного процесу виробництва є
річ дуже складна і робота дуже грунтовна; коли завдання
науки полягає в тому, щоб видимий, виступаючий на поверхні
рух звести до внутрішнього дійсного руху, то само собою
розуміється, що в головах агентів капіталістичного виробництва
і циркуляції мусять скластись такі уявлення про закони виробництва,
які цілком відхиляються від цих законів і є тільки свідомим
виразом видимого на поверхні руху. Уявлення купця,
біржового спекулянта, банкіра неминуче цілком перекручені.
Уявлення фабрикантів перекручуються актами циркуляції, які
\parbreak{}  %% абзац продовжується на наступній сторінці
