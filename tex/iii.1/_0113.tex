\parcont{}  %% абзац починається на попередній сторінці
\index{iii1}{0113}  %% посилання на сторінку оригінального видання
чого знов таки значно зменшились витрати добування пари“
(„Rep. of Insp. of Fact., Oct. 1852“, стор. 23—27).

Те, що сказано тут про двигуни, стосується так само й до
передатних механізмів і робочих машин.

„Швидкість, з якою протягом небагатьох останніх років
ішов розвиток поліпшень у машинах, дала фабрикантам змогу
розширити виробництво без додаткової рушійної сили. Економніше користування працею стало необхідним
в наслідок скорочення робочого дня, і на більшості добре керованих фабрик завжди думають про те,
яким шляхом можна збільшити
виробництво при зменшених витратах. Завдяки люб’язності одного
дуже інтелігентного фабриканта моєї округи, я маю перед собою дані про число і вік робітників,
зайнятих на його фабриці,
про застосовувані машини і виплачену заробітну плату за час
від 1840 року по цей день. В жовтні 1840 року його фірма
вживала 600 робітників, з яких 200 були молодші 13 років.
У жовтні 1852 року — тільки 350 робітників, з яких тільки 60
молодших 13 років. Обидва роки у нього було в роботі те саме
число машин, за винятком дуже небагатьох, і виплачена була та
сама сума заробітної плати“ (Звіт \emph{Редгрева} в „Rep. of Insp. of
Fact., Oct. 1852“, стор. 58 [59]).

Ці поліпшення машин виявляють свою повну ефективність
тільки тоді, коли машини встановлюються в нових, доцільно
впорядкованих фабричних будівлях.

„Щодо поліпшення машин, то я мушу зауважити, що перш
за все великий прогрес зроблено в будуванні фабрик, пристосованих до встановлення цих нових машин\dots{}
В нижньому поверсі
я сукаю всю мою пряжу і в самому тільки цьому поверсі
ставлю 29 000 подвійних веретен. В самому тільки цьому приміщенні я заощаджую на праці щонайменше
10\%, не так в наслідок поліпшень у самій системі подвійних веретен, як в наслідок зосередження машин
під одним керуванням; те саме число
веретен я можу пускати в рух за допомогою одного тільки
передатного вала, чим заощаджую на валах від 60 до 80\% порівняно з іншими фірмами. Крім того це дає
велику економію
на мастилі, жирах і~\abbr{т. д\dots{}} Одним словом — вдосконаленим
влаштуванням фабрики і поліпшеними машинами я заощадив на
праці мінімум 10\% і, крім того, маю велику економію на силі,
вугіллі, мастилі, салі, передатних валах, пасах і т. д.“ (Свідчення
одного бавовнопрядільного фабриканта, „Rep. of Insp. of Fact.,
Oct. 1863“, стор. [109] 110).

\subsection{Утилізація покидьків виробництва}

Разом з капіталістичним способом виробництва поширюється
використовування покидьків виробництва і споживання. Під першими ми розуміємо відпади промисловості
й землеробства, під
останніми — почасти екскременти, які є результатом природного
\parbreak{}  %% абзац продовжується на наступній сторінці
