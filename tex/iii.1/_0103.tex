\parcont{}  %% абзац починається на попередній сторінці
\index{iii1}{0103}  %% посилання на сторінку оригінального видання
й нервів і мозку. Справді, тільки шляхом найпотворнішого марнотратства індивідуального розвитку
забезпечується і здійснюється
розвиток людства взагалі в цю історичну епоху, яка безпосередньо передує свідомій перебудові
людського суспільства. Через те що вся економія, про яку тут іде мова, виникає з суспільного
характеру праці, то фактично саме цей безпосередньо
суспільний характер праці породжує це марнотратство життям і
здоров’ям робітників. Характерним у цьому відношенні є питання,
поставлене вже фабричним інспектором Б.~Бекером: „Все питання потребує серйозного обміркування того,
яким способом
можна найкраще відвернути це \emph{жертвування дитячим життям,
яке спричинює праця тісно скупченими масами} (\emph{congregational
labour})“ („Rep. of Insp. of Fact., Oct. 1863“, стор. 157).

\emph{Фабрики}. Сюди належить відсутність будь-яких охоронних
заходів для безпеки, комфорту й здоров’я робітників також і на
фабриках у власному значенні слова. Більша частина бюлетенів
убою, які перелічують ранених і вбитих промислової армії (дивись щорічні фабричні звіти), виходить
звідси. Так само недостача місця, повітря і~\abbr{т. д.}

Ще в жовтні 1855 року Леонард Горнер скаржився на опір
дуже значного числа фабрикантів вимогам закону про захисні
пристрої до горизонтальних валів, не зважаючи на те, що небезпека постійно доводиться нещасними,
часто смертельними
випадками, і що такі захисні пристрої і не дорогі і ніяк не заважають виробництву („Rep. of Insp. of
Fact., Oct. 1855“, стор. 6 [7]).
В цьому опорі цим та іншим постановам закону фабрикантів
відкрито підтримували неплатні мирові судді, які, самі здебільшого фабриканти або друзі
фабрикантів, мали розв’язувати ці судові справи. Якого роду були вироки цих панів,
видно з слів вищого судді Кемпбеля з приводу одного з таких
вироків, на який йому подано було апеляційну скаргу: „Це —
не тлумачення парламентського акта, це — просто скасування
його“ (там же, стор. 11). В тому самому звіті Горнер оповідає, що на багатьох фабриках машини
пускають у рух без
попередження про це робітників. Через те що й на спиненій
машині завжди є що робити, при чому ця робота завжди виконується руками й пальцями, нещасні випадки
виникають просто
в наслідок неподачі сигналу (там же, стор. 44). Для опору
фабричному законодавству фабриканти утворили в ті часи
у Манчестері тред-юніон, так звану „National Association for the
Amendment of the Factory Laws“ („Національна асоціяція для поліпшення фабричних законів“), який у
березні 1855~\abbr{р.} внесками
по 2\shil{ шилінги} від кінської сили зібрав суму понад \num{50000}\pound{ фунтів
стерлінгів}, щоб з неї оплачувати судові витрати своїх членів
по судових скаргах фабричних інспекторів і вести процеси від
імени асоціяції. Завдання полягало в тому, щоб довести, що killing no murder [умертвіння не є
вбивство], коли це робиться
задля зиску. Шотландський фабричний інспектор, сер Джон
\parbreak{}  %% абзац продовжується на наступній сторінці
