ключає зростання абсолютної суми їх вартості; бо абсолютні
розміри, в яких вони застосовуються, надзвичайно збільшуються
разом з розвитком продуктивної сили праці і зростанням масштабу
виробництва, яке супроводить його. Економія в застосуванні
сталого капіталу, з якого б боку не розглядати її, є почасти
результат виключно того, що засоби виробництва функціонують
і споживаються як спільні засоби виробництва комбінованого
робітника, так що сама ця економія являє собою продукт суспільного
характеру безпосередньо продуктивної праці; а почасти
вона є результат розвитку продуктивності праці в тих
сферах, які постачають капіталові його засоби виробництва, так
що, коли розглядати сукупну працю в протиставленні сукупному
капіталові, а не самих тільки вживаних капіталістом X робітників
у протиставленні цьому капіталістові X, то ця економія
знов таки виявляється продуктом розвитку продуктивних сил
суспільної праці, з тією тільки ріжницею, що капіталіст X добуває
вигоду не тільки з продуктивності праці своєї власної
майстерні, але також і чужих майстерень. Не зважаючи на це',
економія на сталому капіталі здається капіталістові умовою
цілком чужою робітникові, умовою, яка абсолютно не торкається
робітника і з якою він не має нічого спільного; тимчасом
для капіталіста завжди лишається цілком ясним, що для
робітника, звичайно, має значення, багато чи мало праці купує
капіталіст за одні й ті самі гроші (бо саме такою виступає в його
свідомості угода між капіталістом та робітником). Ця економія
в застосуванні засобів виробництва, цей метод досягнення певного
результату з найменшими витратами, ще в значно більшій
мірі, ніж інші сили, імманентні праці, здається силою, імманентною
капіталові, і методом, що є властивий капіталістичному способові
виробництва і характеризує його.

Цей спосіб уявлення викликає тим менш сумніву, що йому
відповідає зовнішня видимість фактів, і що капіталістичне відношення
в дійсності приховує внутрішній зв’язок таким способом,
що для робітника умови здійснення його власної праці виявляються
чимось байдужим, зовнішнім і чужим.

Поперше: Засоби виробництва, з яких складається сталий
капітал, репрезентують тільки гроші капіталіста (як, за Ленге,
тіло римського боржника репрезентувало гроші його кредитора)
і стоять у певному відношенні тільки до нього, тимчасом як
робітник, оскільки він у дійсному процесі виробництва приходить
з ними в дотик, має з ними діло тільки як із споживними вартостями
виробництва, засобами праці і матеріалами праці. Отже,
зменшення чи збільшення цієї вартості є така обставина, яка так
само мало зачіпає відношення робітника до капіталіста, як, наприклад,
та обставина, чи обробляє він мідь чи залізо. Звичайно, капіталіст,
як ми покажемо пізніше, любить розглядати справу інакше
в тих випадках, коли має місце збільшення вартості засобів
виробництва і, в наслідок цього, зменшення норми зиску.
