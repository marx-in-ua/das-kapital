\parcont{}  %% абзац починається на попередній сторінці
\index{iii1}{0138}  %% посилання на сторінку оригінального видання
для грубих нумерів пряжі і важких тканин\dots{} Є побоювання, що
збільшена кількість машин, недавно установлених у камвольній
промисловості, приведе до подібної реакції і в цій галузі промисловості.
Пан Бекер обчислює, що в самому лиш 1849 році в цій
галузі промисловості продукт ткацьких верстатів збільшився на
40\%, продукт веретен на 25—30\%, а розширення промисловості
все ще продовжується в тих самих розмірах“ („Rep. of Insp.
of Fact., April 1850', стор. 54).

1850 рік. Жовтень. „Ціна бавовни продовжує\dots{} викликати
значну пригніченість в цій галузі промисловості, особливо для
таких товарів, для яких сировинний матеріал становить значну
частину витрат виробництва. Значний ріст ціни на шовк-сирець
часто приводив до пригнічення і в цій галузі“ („Rep. of Insp. of
Fact., Oct. 1850“, стор. 14). — За цитованим тут звітом комітету
королівського товариства культури льону в Ірландії, висока
ціна льону при низьких цінах інших сільськогосподарських продуктів
забезпечила тут значне розширення виробництва льону
для наступного року (стор. [31] 33).

1853 рік. Квітень. Великий розквіт. „Ніколи ще за ті 17 років,
протягом яких мені офіціально доводилось знайомитися з станом
фабричної округи Ланкашіра, я не спостерігав такого загального
процвітання; діяльність по всіх галузях надзвичайна“, —
каже Л. Горнер („Rep. of Insp. of Fact., April 1853“, стор. 19).

1853 рік. Жовтень. Депресія в бавовняній промисловості.
„Перепродукція“ („Rep. of Insp. of Fact., Oct. 1853“, стор. [13] 15).

1854 рік. Квітень. „Шерстяна промисловість, хоч справи в ній
йшли не жваво, повністю завантажила всі фабрики; так само
й бавовняна промисловість. Камвольна промисловість протягом
цілого минулого півріччя всюди працювала нерегулярно\dots{} В лляній
промисловості відбувалися порушення в наслідок зменшеного
подання льону й коноплі з Росії в зв’язку з кримською війною“
(„Rep. of Insp. of Fact., [April] 1854“, стор. 37).

1859 рік. „Справи в шотландській лляній промисловості все
ще в пригніченому стані\dots{} бо сировинний матеріал рідкий і дорогий;
погана якість торішнього урожаю в прибалтійських країнах,
звідки йде до нас головний довіз, справлятиме шкідливий вплив
на промисловість цієї округи; навпаки, джут, який в багатьох
грубих товарах помалу витискує льон, не є ні надзвичайно дорогий,
ні рідкий\dots{} приблизно половина машин в Денді пряде
тепер джут“ („Rep. of Insp. of Fact., April 1859“, стор. 19). —
„В наслідок високої ціни сировинного матеріалу льонопрядіння
все ще зовсім невигідне, і в той час, як усі інші фабрики працюють
повний час, ми маємо ряд прикладів спинення машин,
що перероблюють льон\dots{} Прядіння джуту\dots{} перебуває в більш
задовільному стані, бо за останній час ціна на цей матеріал
стала помірнішою“ (Rep. of Insp. of Fact., Oct. 1859“, стор. 20).
