\parcont{}  %% абзац починається на попередній сторінці
\index{iii1}{0038}  %% посилання на сторінку оригінального видання
або менш для кожного окремого ринку збуту, — мали особливі норми зиску. Урівняння цих різних норм зиску окремих товариств здійснювалось протилежним шляхом — через конкуренцію. Насамперед вирівнювались норми зиску на різних ринках однієї і тієї ж
нації. Якщо Александрія давала більший бариш на венеціанські товари, ніж Кіпр, Константинополь або Трапезунд, то венеціанці
пускали в рух більше капіталу для • Александра, вилучаючи його з обороту на інших ринках. Далі мусило наступати ступневе
урівняння норм зиску між окремими націями, які вивозили на ті самі ринки однакові або подібні товари, при чому дуже часто
деякі з цих націй розорялись і зникали із сцени. Але цей процес весь час переривався політичними по- діями; так, напр., вся
левантійська торгівля загинула в наслідок монгольських і турецьких нападів, а величезні географічно-торговельні відкриття з
І492 р. тільки прискорили і потім довершили цю загибель. Раптове поширення сфери збуту, яке настало в цей час, і зв’язаний з
цим переворот у шляхах сполучення не викликали спочатку ніяких істотних змін у способі ведення торгівлі. З Індією і Америкою
торгівля спочатку теж велась переважно ще товариствами. Але, поперше, за цими товариствами стояли великі нації. Місце
каталонців, що торгували з Левантом, заступила в торгівлі з Америкою вся велика об’єднана Іспанія; поряд з нею дві такі
великі країни, як Англія і Франція; і навіть Голландія та Португалія, найдрібніші країни, все ж були принаймні такої ж
величини і сили, як Венеція, найбільша і найдужча торговельна нація попереднього періоду. Це давало мандрівному купцеві (merchant adventurer) XVI і XVII століття опору, яка все більше і більше робила зайвим товариство, що захищало своїх членів
також і за допомогою зброї; тому витрати, зв’язані з існуванням товариства, ставали прямо обтяжливими. Далі, тепер багатство
нагромаджувалось у руках окремих осіб значно швидше, так що незабаром поодинокі купці могли вкладати в одно підприємство
стільки ж фондів, скільки раніш вкладало ціле товариство. Торговельні товариства, де вони продовжували ще існувати,
перетворювалися здебільшого в озброєні корпорації, які під захистом і протекторатом батьківщини завойовували і монопольно
експлуатували цілі нововідкриті країни. Але чим більше засновувалось переважно державою колоній в нових областях, тим більше
торговельні товариства відступали на задній план перед торгівлею одиничного купця, і разом з тим урівняння норми зиску все
більше й більше ставало виключною справою конкуренції. Досі ми ознайомились тільки з нормою зиску торговельного капіталу. Бо
до цього часу існували тільки торговельний і лихварський капітали, промисловий капітал ще тільки мав розвинутись.
Виробництво було ще переважно в руках працівників, які володіли своїми власними засобами виробництва і праця яких, отже,
\parbreak{}  %% абзац продовжується на наступній сторінці
