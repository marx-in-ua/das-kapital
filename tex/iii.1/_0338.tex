\parcont{}  %% абзац починається на попередній сторінці
\index{iii1}{0338}  %% посилання на сторінку оригінального видання
таким чином у позику, і робочою силою в її відношенні до промислового
капіталіста. Тільки що вартість робочої сили промисловий
капіталіст оплачує, тимчасом як вартість капіталу, взятого
в позику, він просто повертає назад. Споживна вартість робочої
сили для промислового капіталіста така: споживаючи її, створити
більше вартості (зиск), ніж вона сама має і ніж вона коштує.
Цей надлишок вартості є споживна вартість робочої сили для
промислового капіталіста. І так само споживна вартість позиченого
грошового капіталу виступає як його здатність створювати
і збільшувати вартість.

Грошовий капіталіст дійсно відчужує споживну вартість, і в наслідок
цього те, що він віддає, віддається ним як товар. І остільки
аналогія з товаром як таким є повна. Поперше, це вартість, яка
переходить з одних рук в інші. При продажу простого товару,
товару як такого, в руках покупця і продавця лишається та сама
вартість, тільки в різній формі; як до цього продажу, так і після
нього вони обидва мають ту саму вартість, яку вони відчужували,
один — у товарній формі, другий — у грошовій формі. Ріжниця
тільки та, що при позиці грошовий капіталіст є єдина особа, яка
при цій операції віддає вартість; але він зберігає її в наслідок того,
що вона через певний час буде повернена йому. При позиці тільки
одна сторона одержує вартість, бо тільки однією стороною віддається
вартість. — Подруге, на одній стороні відчужується дійсна
споживна вартість, а на другій вона одержується і споживається.
Але в відміну від звичайного товару ця споживна вартість сама є
вартість, а саме — надлишок тієї величини вартості, яка виникає
в наслідок уживання грошей як капіталу, надлишок порівняно з
первісною величиною вартості. Зиск є ця споживна вартість.

Споживна вартість відданих у позику грошей полягає в тому,
що вони можуть функціонувати як капітал і як такий виробляти
при пересічних умовах пересічний зиск\footnote{
„The equitableness oi taking interest depends not upon a man’s making or
not making profit, but upon its (позиченого) being capable of producing profit, if
rightly employed“ [„Справедливість одержання процентів залежить не від того, чи
виробляє хтонебудь зиск чи ні, а від його“ (позиченого) „здатності утворювати
зиск при правильному застосуванні“] („An Essay on the Governing Causes of
the Natural Rate of Interest, wherein the sentiments of Sir W.~Petty and
Mr.~Locke, on that head, are considered“. London 1750, стор. 49. Автор анонімного
твору: \emph{Дж.~Mecci}).
}.

Що ж платить промисловий капіталіст і що, отже, є ціною
капіталу, відданого в позику? That which men pay as interest for
the use of what they borrow [Те, що люди платять як процент за
користування тим, що вони позичають] є, за Мессі, a part of the
profit it is capable of producing [частина того зиску, яку позичене
здатне виробити]. [„An Essay etc.“, стор. 49]\footnote{
„Rich people, instead of employing their money themselves\dots{} let it out to
other people for them to make profit of, reserving for the owners a proportion of
the profits so made“ [„Багаті люди, замість того, щоб самим застосовувати свої
гроші\dots{} позичають їх іншим людям, щоб вони виробляли ними зиск і лишали для
власників частину виробленого таким чином зиску“] (там же, стор. 23 [24]).
}.
