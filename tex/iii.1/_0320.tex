
\index{iii1}{0320}  %% посилання на сторінку оригінального видання
Розвиток торгівлі і торговельного капіталу розвиває повсюди
виробництво в такому напрямі, що його метою стає мінова вартість,
збільшує його розміри, робить його більш різноманітним і
надає йому космополітичного характеру, розвиває гроші у світові
гроші. Тому торгівля повсюди впливає більш чи менш розкладово
на ті організації виробництва, які вона застає і які в усіх
своїх різних формах мають своєю метою головним чином споживну
вартість. Але в якій мірі вона впливає на розклад старого
способу виробництва, це насамперед залежить від його
міцності та внутрішньої організації. А до чого приводить цей
процес розкладу, тобто який новий спосіб виробництва стає на
місце старого, це залежить не від торгівлі, а від характеру
самого старого способу виробництва. В античному світі вплив
торгівлі і розвиток купецького капіталу завжди має своїм результатом
рабовласницьке господарство; залежно від вихідного
пункту, він приводить також тільки до перетворення патріархальної
системи рабства, яка має своєю метою виробництво безпосередніх
засобів існування, в систему рабства, яка має своєю
метою виробництво додаткової вартості. Навпаки, в сучасному
світі він приводить до капіталістичного способу виробництва.
Звідси випливає, що самі ці результати були зумовлені ще
й цілком іншими обставинами, крім розвитку торговельного капіталу.

Це вже лежить у природі речей, що як тільки міська промисловість
як така відокремлюється від землеробської, її продукти
з самого початку є товари і для їх продажу, отже, потрібне
посередництво торгівлі. Таким чином, зв’язок торгівлі
з розвитком міст і, з другого боку, обумовленість останнього
торгівлею зрозумілі самі собою. Але наскільки рука в руку
з цим іде промисловий розвиток, це тут цілком залежить від
інших обставин. У стародавньому Римі вже в пізніший республіканський
період купецький капітал розвинувся без ніякого
прогресу в промисловому розвитку до вищого рівня, ніж будь-коли
раніше в старому світі, тимчасом у Корінфі та в інших
грецьких містах Европи та Малої Азії розвиток торгівлі супроводився
високим розвитком промисловості. З другого боку,
в пряму протилежність до розвитку міст та його умов, торговельний
дух і розвиток торговельного капіталу часто властиві
якраз неосілим, кочовим народам.

Не підлягає ніякому сумніву — і саме цей факт привів до
цілком помилкових поглядів, — що в XVI й XVII століттях великі
революції, які відбулися в торгівлі після географічних відкрить
і які швидко піднесли розвиток купецького капіталу,
становлять головний момент у тому, що сприяло переходові
феодального способу виробництва в капіталістичний. Раптове
розширення світового ринку, збільшення різноманітності товарів,
що були в обігу, суперництво між європейськими націями щодо
оволодіння азіатськими продуктами та американськими скарбами,
\parbreak{}  %% абзац продовжується на наступній сторінці
