\parcont{}  %% абзац починається на попередній сторінці
\index{iii1}{0255}  %% посилання на сторінку оригінального видання
в падінні норми зиску такий закон, який на певному пункті
найбільш вороже виступає проти розвитку самого цього способу
виробництва і який через це мусить раз-у-раз переборюватись
кризами.

2)~В тому, що для розширення чи скорочення виробництва
вирішальним є привласнення неоплаченої праці і відношення цієї
неоплаченої праці до упредметненої праці взагалі або, висловлюючись
капіталістично, що вирішальним для цього є зиск
і відношення цього зиску до застосовуваного капіталу, отже,
певна висота норми зиску, а не відношення виробництва до
суспільних потреб, до потреб суспільно розвинених людей. Тому
капіталістичне виробництво доходить до своєї межі вже на
такому ступені розширення виробництва, який, навпаки, при
інших передумовах був би далеко недостатнім. Воно припиняється
не тоді, коли цього вимагає задоволення потреб, а тоді,
коли цього припинення вимагає виробництво і реалізація зиску.

Якщо норма зиску знижується, то, з одного боку, сили
капіталу скеровуються на те, щоб окремий капіталіст за допомогою
кращих методів і~\abbr{т. д.} знизив індивідуальну вартість
кожної одиниці своїх товарів нижче її суспільної пересічної
вартості і таким чином одержав би при даній ринковій ціні
надзиск; з другого боку, виникають грюндерські підприємства
і загальний сприятливий ґрунт для них в завзятих спробах застосування
нових методів виробництва, в нових капіталовкладеннях,
в нових авантюрах, щоб забезпечити хоч якийнебудь
надзиск, який не залежав би від загального пересічного рівня
і перевищував би його.

Норма зиску, тобто відносний приріст капіталу, має важливе
значення передусім для всіх нових паростків капіталу, які групуються
самостійно. І коли б утворення капіталів потрапило
виключно в руки деяких небагатьох уже наявних великих
капіталів, для яких маса зиску урівноважує його норму, то
взагалі згас би вогонь, який оживляє виробництво. Виробництво
охопив би сон. Норма зиску є рушійна сила капіталістичного
виробництва; виробляється тільки те і остільки, що і оскільки
може бути вироблене з зиском. Звідси острах англійських економістів
перед зменшенням норми зиску. Те, що вже сама тільки
можливість цього непокоїть Рікардо, свідчить якраз про його
глибоке розуміння умов капіталістичного виробництва. Якраз те,
що йому закидають, а саме, що він при розгляді капіталістичного
виробництва, не турбуючись про „людей“, звертає увагу
тільки на розвиток продуктивних сил, — яких би це не коштувало
жертв людьми і капітальними \emph{вартостями} — якраз це є
в нього найвизначніше. Розвиток продуктивних сил суспільної
праці є історичне завдання і виправдання капіталу. Саме цим він
несвідомо утворює матеріальні умови вищої форми виробництва.
Рікардо непокоїть те, що нормі зиску, цьому стимулові капіталістичного
виробництва, умові й рушієві нагромадження, загрожує
\index{iii1}{0256}  %% посилання на сторінку оригінального видання
небезпека в наслідок розвитку самого виробництва. А кількісне
відношення тут — усе. В дійсності в основі цього лежить щось
глибше, про що він тільки догадується. В цьому виявляється
чисто економічним способом, тобто з буржуазної точки зору,
в межах капіталістичного розуміння, з точки зору самого капіталістичного
виробництва, обмеженість капіталістичного виробництва,
його відносність, те, що воно не є абсолютний, а тільки
історичний спосіб виробництва, відповідний певній обмеженій
епосі розвитку матеріальних умов виробництва.

\subsection{Додатки}

Через те що розвиток продуктивної сили праці відбувається
дуже нерівномірно в різних галузях промисловості, і не тільки
нерівномірно щодо ступеня, а часто і в протилежному напрямі,
то звідси випливає, що пересічна маса зиску (дорівнює додаткової
вартості) мусить стояти далеко нижче тієї висоти, якої можна
було б сподіватися відповідно до розвитку продуктивної сили
в найбільш розвинених галузях промисловості. Те, що розвиток
продуктивної сили в різних галузях промисловості відбувається
не тільки в дуже різних пропорціях, але часто і в протилежному
напрямі, виникає не тільки з анархії конкуренції і своєрідності
буржуазного способу виробництва. Продуктивність праці
зв’язана також з природними умовами, які часто стають менш
вигідними в тій самій мірі, в якій зростає продуктивність, оскільки
вона залежить від суспільних умов. Звідси протилежний рух
в цих різних сферах — прогрес в одних, регрес в інших. Досить
тільки згадати, наприклад, про вплив сезонів року, від чого
залежить кількість найбільшої частини всіх сировинних матеріалів,
про вичерпання лісів, кам’яновугільних і залізнорудних
копалень і~\abbr{т. д.}

Якщо обігова частина сталого капіталу, сировинний матеріал
і~\abbr{т. д.} постійно зростає в своїй масі в міру розвитку продуктивної
сили праці, то цього не можна сказати про основний
капітал, будівлі, машини, пристрої для освітлення, опалення і~\abbr{т. д.}
Хоч машина з зростанням її розмірів стає абсолютно дорожчою,
але відносно вона стає дешевшою. Якщо п’ятеро робітників
виробляють удесятеро більше товарів, ніж раніш, то
з цієї причини витрати на основний капітал не збільшуються
вдесятеро; хоч вартість цієї частини сталого капіталу зростає
з розвитком продуктивної сили, але вона зростає далеко не в такій
самій пропорції. Ми вже не раз відзначали ріжницю між
відношенням сталого капіталу до змінного, як воно виражається
в падінні норми зиску, і тим самим відношенням, як воно,
з розвитком продуктивності праці, виражається щодо одиничного
товару та його ціни.

[Вартість товару визначається всім робочим часом, минулим
і живим, що входить у цей товар. Підвищення продуктивності праці
\parbreak{}  %% абзац продовжується на наступній сторінці
