\parcont{}  %% абзац починається на попередній сторінці
\index{iii1}{0070}  %% посилання на сторінку оригінального видання
5 годинам \deq{} 5\shil{ шилінгам}, то додаткова праця \deq{} 5 годинам і додаткова
вартість \deq{} 5\shil{ шилінгам}; якщо ж необхідна праця дорівнює
4 годинам \deq{} 4\shil{ шилінгам}, то додаткова праця \deq{} 6 годинам і додаткова
вартість \deq{} 6\shil{ шилінгам.}

Отже, якщо величина вартості змінного капіталу перестає
бути показником маси праці, приведеної ним у рух, і, навпаки,
змінюється сама міра цього показника, то разом з тим норма
додаткової вартості змінюється в протилежному напрямі і в зворотному
відношенні.

Тепер ми переходимо до того, щоб застосувати вищенаведене
рівняння норми зиску: $р'= m'\frac{v}{К}$ до різних можливих випадків.
Ми будемо почережно змінювати вартість окремих факторів
$m'\frac{v}{К}$ і встановлювати вплив цих змін на норму зиску. Таким
чином ми одержимо різні ряди випадків, які ми можемо розглядати
або як послідовні зміни умов діяння одного й того
самого капіталу, або як різні, одночасно існуючі один поряд
одного і притягнені до порівняння капітали, наприклад, капітали
в різних галузях промисловості або в різних країнах. Тому,
якщо розуміння деяких наших прикладів, як прикладів послідовних
у часі станів одного й того ж капіталу, здається вимушеним
або практично неможливим, то це заперечення відпаде,
якщо ми розглядатимем їх як порівняння незалежних капіталів.

Отже, ми розкладаємо добуток $m'\frac{v}{К}$ на його обидва множники,
$m'$ і $\frac{v}{К}$; спочатку ми розглядатимем $m'$ як сталу величину
і дослідимо вплив можливих змін $\frac{v}{К}$; потім ми припустимо, що
дріб $\frac{v}{К}$ є стала величина і дамо $m'$ проробити можливі зміни;
нарешті, ми припустимо, що всі фактори змінюються, і вичерпаємо
цим усі випадки, з яких можуть бути виведені закони,
що стосуються норми зиску.
\begin{center}
\textbf{І. $m'$ не змінюється, $\frac{v}{К}$ змінюється}
\end{center}
Для цього випадку, який охоплює декілька часткових випадків,
можна скласти загальну формулу. Якщо ми маємо два
капітали $К$ і $К\textsubscript{1}$, з відповідними змінними складовими частинами
$v$ і $v\textsubscript{1}$, зі спільною їм обом нормою додаткової вартості $m'$ і нормами
зиску $р'$ і $р'\textsubscript{1}$, то \[
р' \deq{} m'\frac{v}{K}; р'\textsubscript{1} \deq{} m'\frac{v\textsubscript{1}}{К\textsubscript{1}}
\]
