
\index{iii1}{0127}  %% посилання на сторінку оригінального видання
Припустімо наприклад що спочатку потрібно було 500\pound{ фунтів
стерлінгів} для того, щоб щотижня приводити в рух 500 робітників, а тепер для цього потрібно тільки
400\pound{ фунтів стерлінгів}.
Тоді, якщо припустити, що в обох випадках маса виробленої
вартості = 1000\pound{ фунтам стерлінгів}, маса щотижневої додаткової вартості буде в першому випадку = 500\pound{ фунтів стерлінгів}, норма додаткової вартості \frac{500}{500} = 100\%; але після зниження заробітної плати
маса додаткової вартості буде 1000\pound{ фунтів стерлінгів} — 400\pound{ фунтів стерлінгів} = 600\pound{ фунтам
стерлінгів}, а її норма \frac{600}{400} = 150\%. І це підвищення норми додаткової вартості є єдиний результат
для того, хто із змінним капіталом в 400\pound{ фунтів стерлінгів} і з відповідним сталим капіталом починає
нове підприємство в тій самій сфері виробництва. Але в підприємстві, яке вже функціонує, в цьому
випадку в наслідок зниження вартості змінного
капіталу не тільки підвищується маса додаткової вартості з 500
до 600\pound{ фунтів стерлінгів} і норма додаткової вартості з 100
до 150\%; тут, крім того, звільняється 100\pound{ фунтів стерлінгів}
змінного капіталу, за допомогою яких знову таки можна експлуатувати працю. Отже, не тільки та сама
кількість праці експлуатується з більшою вигодою, але, в наслідок звільнення цих 100\pound{ фунтів
стерлінгів}, з тим самим змінним капіталом у 500\pound{ фунтів стерлінгів} можна експлуатувати при підвищеній
нормі експлуатації більше робітників, ніж раніше.

Припустімо тепер, навпаки, що при 500 занятих робітниках
первісне відношення розподілу продукту є: $400 v + 600 m = 1000$, отже, що норма додаткової вартості є
= 150\% Таким
чином робітник одержує тут щотижня \sfrac{4}{5}\pound{ фунтів стерлінгів} = 16\shil{ шилінгів.} Якщо тепер в наслідок
підвищення вартості
змінного капіталу 500 робітників коштують щотижня 500\pound{ фунтів
стерлінгів}, то тижнева заробітна плата кожного робітника = 1\pound{ фунтові стерлінгів}, і 400\pound{ фунтів
стерлінгів} можуть привести
в рух тільки 400 робітників. Отже, якщо пускатиметься в рух те
саме число робітників, що й раніш, то ми матимемо $500 v + 500 m = 1000$; норма додаткової вартості
знизиться з 150 до 100\%,
тобто на \sfrac{1}{3}. Для нововкладуваного капіталу єдиним результатом було б це зниження норми додаткової
вартості. При інших
однакових умовах норма зиску відповідно знизилася б, хоч і не
в такій самій пропорції. Якщо, наприклад, c = 2000, то в
першому випадку ми маємо $2000 c + 400 v + 600 m = 3000; m' = 150\%, p' = \frac{600}{2400} = 25\%$. В другому
випадку $2000 c + 500 v + 500 m = 3000; m' = 100\%; р' = \frac{500}{2500} = 20\%$. Навпаки, для вкладеного вже
капіталу результат був би двоякий. 3400\pound{ фунтами
стерлінгів} змінного капіталу тепер можна привести в рух тільки
400 робітників, до того ж при нормі додаткової вартості в 100\%.
Отже, вся додаткова вартість, яку вони дають, становить тільки
\parbreak{}  %% абзац продовжується на наступній сторінці
