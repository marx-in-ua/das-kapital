\parcont{}  %% абзац починається на попередній сторінці
\index{iii1}{0024}  %% посилання на сторінку оригінального видання
і не може і не бажає випустити! Безмежна відвага, сполучена
з в’юнкістю вужа, з якою він викручується з неможливих ситуацій,
героічне презирство до одержаних стусанів, нестримно швидке
присвоювання чужих праць, нахабне шахрайство реклами, організація
слави за допомогою шумихи приятелів — хто може зрівнятися
з ним в усьому цьому?

Італія — країна класичності. Починаючи з тих великих часів,
коли там зійшла зоря сучасного світу, вона породила величні
характери недосяжно-класичної довершеності, від Данте до
Гарібальді. Але й часи приниження і чужоземного панування
залишили їй класичні характери, серед них два особливо рельєфні
типи: Сганареля і Дулькамару. Класичну єдність обох ми
бачимо втіленою в нашому illustre Лоріа.

На закінчення я мушу повести своїх читачів за океан. У Нью-Йорку
пан доктор медицини \emph{Джордж Штібелінг} теж знайшов
розв’язання проблеми, і при тому надзвичайно просте. Таке
просте, що ні одна людина ні по цей, ні по той бік океану не
схотіла його визнати, в наслідок чого він страшенно розгнівався
і в безконечному ряді брошур та газетних статтей по обидва
боки океану гірко скаржився на таку несправедливість. В „Neue
Zeit“ йому, правда, сказали, що все його розв’язання грунтується
на помилці в обрахунку. Але це не могло його стурбувати;
Маркс, мовляв, теж зробив помилки в обрахунках і, однак,
в багатьох речах він має рацію. Отже, погляньмо на Штібелінгове
розв’язання.

„Я беру дві фабрики, які працюють однаковий час з однаковим
капіталом, але з різним відношенням сталого і змінного
капіталу. Весь капітал $(c \dplus{} v)$ я припускаю $= y$ і позначаю ріжницю
у відношенні сталого капіталу до змінного через $х$. На
фабриці І $y \deq{} c \dplus{} v$, на фабриці II $у \deq{} (c - х) \dplus{} (v \dplus{} х)$. Отже,
норма додаткової вартості на фабриці І $= \frac{m}{v}$ а на фабриці II $=
\frac{m}{v+x}$. Зиском $(р)$ я називаю всю додаткову вартість $(m)$, на
яку збільшується весь капітал $у$, або $c \dplus{} v$, на протязі даного
часу; отже, $р \deq{} m$. Тому норма зиску на фабриці І $= \frac{p}{y}$, або
\frac{m}{c \dplus{} v}, а на фабриці II так само $= \frac{p}{y}$, або \frac{m}{(c - x) \dplus{} (v \dplus{} x)}, тобто
так само $= \frac{m}{c \dplus{} v}$. Отже, проблема\dots{} розв’язується таким способом,
що на основі закону вартості при застосуванні однакового
капіталу і однакового часу, але неоднакових кількостей
живої праці, з зміни норми додаткової вартості постає однакова
пересічна норма зиску“ (\emph{G.~C.~Stiebeling}: „Das Wertgesetz und die
Profitrate“. New York, John Heinrich).
