\parcont{}  %% абзац починається на попередній сторінці
\index{iii1}{0258}  %% посилання на сторінку оригінального видання
виробництва товару $= \sfrac{1}{2} + 17\sfrac{1}{2} + 2 = 20\text{ шилінгам}$, пересічна
норма зиску $\frac{2}{20} = 10\%$, а ціна виробництва штуки товару дорівнює
його вартості = 22 шилінгам або маркам.

Припустім, що винайдено машину, яка наполовину скорочує
потрібну на кожну штуку товару живу працю, але зате
збільшує втроє ту частину вартості, яка складається з зношування
основного капіталу. Тоді справа стоятиме так: зношування
= 1\sfrac{1}{2} шилінгам, сировинний та допоміжний матеріал, як
і раніше, 17\sfrac{1}{2} шилінгів, заробітна плата 1 шилінг, додаткова
вартість 1 шилінг, разом 21 шилінг або 21 марка. Вартість товару
зменшилась тепер на 1 шилінг; нова машина безперечно підвищила
продуктивну силу праці. Але для капіталіста справа стоятиме
так: його витрати виробництва є тепер: 1\sfrac{1}{2} шилінги зношування,
17\sfrac{1}{2} шилінгів — сировинний і допоміжний матеріал,
1 шилінг — заробітна плата, разом 20 шилінгів, як і раніш.
Через те що норма зиску безпосередньо в наслідок застосування
нової машини не змінюється, він мусить одержати 10\%
понад витрати виробництва, що становить 2 шилінги; отже, ціна
виробництва лишилась незмінною = 22 шилінгам, але вона на 1 шилінг
вища вартості. Для суспільства, яке виробляє при капіталістичних
умовах, товар \emph{не} став дешевшим, нова машина не являє
собою \emph{ніякого} поліпшення. Отже, капіталіст не має ніякого
інтересу в тому, щоб вводити нову машину. А через те що
введенням нової машини він просто зробив би нічого невартою
свою стару, ще не зношену машину, перетворив би її просто
в старе залізо, отже, зазнав би позитивного збитку, то він дуже
стережеться такої утопічної для нього дурості.

Отже, для капіталу закон зростаючої продуктивної сили праці
має не безумовне значення. Для капіталу ця продуктивна сила
підвищується не тоді, коли взагалі заощаджується жива праця,
а тільки тоді, коли на \emph{оплачуваній} частині живої праці заощаджується
більше, ніж додається минулої праці, як це вже коротко
зазначено було в книзі І, розділ XIII, 2, стор. 411\footnote*{
Стор. 297--298 рос. вид. 1935~\abbr{р.} Ред. укр. перекладу.
}. Тут
капіталістичний спосіб виробництва впадає в нову суперечність.
Його історичне покликання — нестримний розвиток продуктивності
людської праці, підготований вперед у геометричній прогресії.
Він зраджує це покликання, оскільки він, як у даному
випадку, перешкоджає розвиткові продуктивності. Цим він тільки
знову доводить, що він хиріє від старості і все більше й більше
переживає себе.]\footnote{
Вищенаведене стоїть у дужках, тому що хоч це і є переробка з примітки
оригіналу рукопису, але у викладі деяких моментів воно виходить за
межі того матеріалу, що є в оригіналі. — \emph{Ф. Е.}
}

\pfbreak{}

В конкуренції збільшення мінімуму капіталу, який з підвищенням
продуктивної сили стає потрібним для успішного ведення
\index{iii1}{0259}  %% посилання на сторінку оригінального видання
самостійного промислового підприємства, виявляється
так: як тільки нове дорожче промислове устаткування стає
загальнопоширеним, дрібніші капітали на майбутнє виключаються
з цього виробництва. Тільки на перших порах механічних винаходів
у різних сферах виробництва дрібніші капітали можуть
в них самостійно функціонувати. З другого боку, дуже великі
підприємства, з надзвичайно високим відношенням сталого капіталу,
як залізниці, дають не пересічну норму зиску, а тільки
частину її, процент. Інакше загальна норма зиску знизилась би
ще більше. Навпаки, і тут великі капітали, зібрані в формі акцій,
знаходять собі поле для безпосереднього застосування.

Зростання капіталу, отже, нагромадження капіталу, включає
зменшення норми зиску лиш остільки, оскільки разом з цим зростанням
настають розглянуті нами вище зміни у відношенні органічних
складових частин капіталу. Однак, не зважаючи на постійні,
повсякденні перевороти в способі виробництва, та чи інша,
більша чи менша частина всього капіталу протягом певного
часу продовжує нагромаджуватися на базі даного пересічного відношення
цих складових частин, так що з зростанням цієї частини
не сполучена ніяка органічна переміна, отже й ніякі причини
падіння норми зиску. Це постійне збільшення капіталу, а тому
й розширення виробництва на основі старих методів виробництва,
яке спокійно триває далі, тимчасом як поряд з ними
вводяться вже нові методи, знов таки є причиною того, що
норма зиску зменшується не в тій мірі, в якій зростає сукупний
капітал суспільства.

Збільшення абсолютного числа робітників, не зважаючи на
відносне зменшення змінного капіталу, витрачуваного на заробітну
плату, відбувається не в усіх галузях виробництва і не
в усіх рівномірно. В землеробстві зменшення елементу живої
праці може бути абсолютним.

Зрештою, абсолютне збільшення числа найманих робітників,
не зважаючи на його відносне зменшення, є тільки потреба капіталістичного
способу виробництва. Для нього робочі сили
стають уже зайвими, як тільки немає вже необхідності примушувати
їх працювати 12--15 годин на день. Розвиток продуктивних
сил, який зменшував би абсолютне число робітників, тобто
в дійсності давав би змогу всій нації виконувати своє сукупне
виробництво за коротший час, викликав би революцію, бо він
вивів би в тираж більшість населення. В цьому знову виявляється
специфічна межа капіталістичного виробництва, а також
те, що воно ніяк не є абсолютною формою для розвитку продуктивних
сил і створення багатства, що воно, навпаки, на певному
пункті вступає в колізію з цим розвитком. Частково така колізія
виявляється в періодичних кризах, які походять з того, що
то одна, то друга частина робітничого населення робиться зайвою
в своїй старій професії. Межа капіталістичного виробництва
— надлишковий час робітників. Абсолютний надлишковий
\parbreak{}  %% абзац продовжується на наступній сторінці
