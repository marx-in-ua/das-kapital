\parcont{}  %% абзац починається на попередній сторінці
\index{iii1}{0086}  %% посилання на сторінку оригінального видання
2) при однаковій нормі додаткової вартості і неоднаковому
процентному складі, при чому норми зиску відносяться одна до
одної, як змінні частини капіталу;

3) при неоднаковій нормі додаткової вартості і неоднаковому
процентному складі, при чому норми зиску відносяться одна до
одної, як добутки $m'v$, тобто як маси додаткової вартості,
взяті в процентному відношенні до всього капіталу.\footnote{
В рукопису є ще дуже докладні обчислення щодо різності між нормою
додаткової вартості і нормою зиску $(m' — р')$; ця різність має різні цікаві особливості,
рух її показує випадки, коли обидві норми віддаляються одна від
одної або наближаються одна до одної. Ці рухи можуть бути зображені у формі
кривих. Я відмовляюсь від відтворення цього матеріалу, бо для ближчих цілей
цієї книги він менш важливий; тут досить просто звернути на це увагу тих
читачів, які захочуть далі простежити це питання. — Ф. Е.
}

\section{Вплив обороту на норму зиску}

[Вплив обороту на виробництво додаткової вартості, отже
й зиску, з’ясовано в другій книзі. Його можна коротко зрезюмувати
таким чином, що в наслідок того, що на оборот потрібен
певний час, на виробництво не може бути застосований одночасно
весь капітал; що, отже, частина капіталу постійно лежить без
діла, чи то в формі грошового капіталу, запасних сировинних
матеріалів, готового, але ще не проданого товарного капіталу,
чи в формі боргових вимог, для яких ще не настав строк платежу;
що капітал, який діє в активному виробництві, тобто при
створенні і привласненні додаткової вартості, постійно зменшується
на цю частину, при чому в такій самій пропорції постійно
зменшується створювана і привласнювана додаткова вартість.
Чим коротший час обороту, тим меншою порівняно з усім
капіталом стає ця частина капіталу, яка лежить без діла; і тим
більшою, отже, стає, при інших незмінних умовах, привласнювана
додаткова вартість.

Уже в другій книзі докладно розвинуто, як скорочення часу
обороту або одного з двох його підрозділів, часу виробництва
і часу циркуляції, підвищує масу вироблюваної додаткової вартості.
Але через те що норма зиску виражає тільки відношення
виробленої маси додаткової вартості до всього капіталу, занятого
в її виробництві, то очевидно, що всяке таке скорочення підвищує
норму зиску. Те, що раніше, в другому відділі другої
книги, розвинуто щодо додаткової вартості, в такій самій мірі
стосується і до зиску та норми зиску і не потребує тут повторення.
Ми хочемо відзначити лиш декілька головних моментів.

Головний засіб скорочення часу виробництва є підвищення
продуктивності праці, що звичайно називають прогресом промисловості.
Якщо цим одночасно не викликається значне збільшення
\index{iii1}{0087}  %% посилання на сторінку оригінального видання
загальних витрат капіталу в наслідок застосування дорогих
машин і~\abbr{т. д.}, а тому й зниження норми зиску, обчислюваної
на весь капітал, то ця остання мусить підвищитись. І це, безперечно,
має місце при багатьох з найновіших успіхів металургії
і хемічної промисловості. Нововідкриті способи виготовлення
заліза й сталі — Бессемера, Сіменса, Гількріста-Томаса та інших —
скорочують до мінімуму, при відносно незначних витратах,
надзвичайно довгочасні раніш процеси. Виготовлення алізарину
або красильної речовини крапу з кам’яновугільного дьогтю дає
за кілька тижнів, і до того ж при фабричних приладдях, які
вже раніш уживалися для виготовлення фарб з кам’яновугільного
дьогтю, такий самий результат, який раніше вимагав цілих
років; один рік був потрібний для росту крапу, а потім ще
кілька років коріння лишали достигати, раніше ніж уживати
його для фарбування.

Головний засіб скорочення часу циркуляції є поліпшені
шляхи сполучення. І в цьому останні п’ятдесят років зробили
революцію, яку можна порівняти тільки з промисловою революцією
останньої половини минулого століття. На суходолі макадамізовані\footnote*{
Макадамізування — спосіб брукування шляхів за системою Мак-Адама,
при якому скальне каміння укочується круглими котками. Ред. укр. перекладу,
} шляхи відтиснені на задній план залізницею, на
морі повільне і нерегулярне вітрильне сполучення — швидким
і регулярним пароплавним сполученням, і вся земна куля обвивається
телеграфними дротами. Власне кажучи, тільки Суецький
канал і відкрив Східну Азію і Австралію для пароплавного сполучення.
Час циркуляції для товарів, що посилалися до Східної
Азії, який ще в 1847 році становив щонайменше дванадцять
місяців, тепер можна звести майже
до стількох же тижнів. Два великі огнища криз 1825--1857~\abbr{рр.},
Америка і Індія, в наслідок цього перевороту в засобах сполучення
наблизились до європейських промислових країн на
70--90\% і тим самим утратили більшу частину своєї здатності
до вибухів. Час обороту всієї світової торгівлі скоротився
в такій самій мірі, а дієздатність капіталу, який бере в ній
участь, підвищилась більше, ніж удвоє чи утроє. Що це не
лишилось без впливу на норму зиску, зрозуміло само собою.

Щоб представити в чистому вигляді вплив обороту всього
капіталу на норму, зиску ми мусимо при порівнянні двох
капіталів припустити, що всі інші обставини однакові. Отже,
крім норми додаткової вартості і робочого дня, нехай буде
однаковий і процентний склад капіталів. Візьмім тепер капітал
$А$ з складом $80c + 20v = 100К$, що обертається двічі на рік
при нормі додаткової вартості в 100\%. Тоді річний продукт
буде:

$160 c + 40 v + 40 m$. Але для визначення норми зиску ми обчисляємо
ці $40 m$ не на капітальну вартість у 200, що обернулась,
\index{iii1}{0088}  %% посилання на сторінку оригінального видання
а на авансовану капітальну вартість у 100, і таким чином
одержуємо: $р' = 40\%$.

Порівняймо з цим капітал $В = 160c + 40v=200K$, який
функціонує при такій самій нормі додаткової вартості в 100\%,
але обертається тільки один раз на рік. Тоді річний продукт
буде, як і вище:

$160c + 40 v + 40 m$. Але на цей раз ці $40 m$ слід обчислити на
авансований капітал у 200; це дає для норми зиску тільки 20\%,
отже, тільки половину норми для $А$.

Звідси випливає: при капіталах однакового процентного
складу, при однаковій нормі додаткової вартості і однаковому
робочому дні, норми зиску двох капіталів стоять у зворотному
відношенні до часу їх оборотів. Якщо в двох порівнюваних випадках
неоднаковий склад, або норма додаткової вартості, або
робочий день, або заробітна плата, то цим, звичайно, будуть
породжені й дальші ріжниці в нормі зиску; але вони незалежні
від обороту і тому нас не цікавлять тут; вони вже розглянуті
в розділі III.

Безпосередній вплив скороченого часу обороту на виробництво
додаткової вартості, отже й зиску, полягає в підвищеній
діяльності, яка таким способом надається змінній частині капіталу,
про що див. книгу II, розділ XVI: Оборот змінного
капіталу. Там виявилось, що змінний капітал у 500, який обертається
десять разів на рік, привласнює за цей час стільки ж додаткової
вартості, як і змінний капітал у 5000, який при однаковій
нормі вартості і однаковій заробітній платі обертається
тільки один раз на рік.

Візьмемо капітал І, що складається з 10 000 основного капіталу,
— річне зношування якого становить $10\% = 1000,$ — 500
обігового сталого і 500 змінного капіталу. При нормі додаткової
вартості в 100\% цей змінний капітал обертається десять
разів на рік. Задля спрощення ми припускаємо в усіх дальших
прикладах, що обіговий сталий капітал обертається за той
самий час, як і змінний, що й на практиці здебільшого приблизно
так і буває. Тоді продукт одного такого періоду обороту буде:\[
100 c \text{(зношування)} + 500 c + 500 v + 500 m = 1600,\]
а продукт цілого року з десятьма такими оборотами:
\begin{center}
$1000 c \text{(зношування)} + 5000 c + 5000 v + 5000 m = 16 000$,
 $К = 11 000; m = 5000, р' = \frac{5000}{11000} + 45\sfrac{5}{11}\%$.
\end{center}

Візьмемо тепер капітал II: основний капітал — 9000, його річне
зношування — 1000, обіговий сталий капітал — 1000, змінний
капітал — 1000, норма додаткової вартості — 100\%, число річних
оборотів змінного капіталу — 5. Отже, продукт кожного періоду
обороту змінного капіталу буде:\[
200c \text{(зношування)} + 1000 c + 1000 v + 1000 m = 3200,\]

а весь річний продукт при п’яти оборотах:

\begin{center}
$1000c\text{(зношування)} + 5000 c + 5000 v + 5000 m = 16 000$,

$К = 11 000, m = 5000, р' = \frac{5000}{11000} = 45\sfrac{5}{11}\%$.
\end{center}

Візьмімо далі капітал III, в якому зовсім немає основного капіталу,
але є 6000 обігового сталого і 5000 змінного капіталу. При нормі додаткової
вартості в 100\% він обертається один раз на рік. Тоді весь продукт за рік буде:
\begin{center}
$6000c + 5000 v + 5000 m = 16 000,$

$К = 11000, m = 5000, р' = \frac{5000}{11000} = 45\sfrac{5}{11}\%.$
\end{center}
Отже, в усіх трьох випадках ми маємо однакову річну масу
додаткової вартості = 5000, а через те що весь капітал в усіх
трьох випадках теж однаковий, а саме = 11 000, то маємо
й однакову норму зиску в 45\sfrac{5}{11}\%.

Навпаки, якщо при капіталі І ми мали б не 10, а тільки
5 річних оборотів змінної частини, то справа стояла б інакше.
Тоді продукт одного обороту був би:\[
200 c \text{(зношування)} + 500 c + 500 v + 500 m = 1700.\]

Або річний продукт:

\index{iii1}{0089}  %% посилання на сторінку оригінального видання
\begin{center}
$1000c \text{(зношування)} + 2500 c + 2500 v + 2500 m = 8500,$

$К = 11 000, m = 2500, р' = \frac{2500}{11000} = 22\sfrac{8}{11}\%.$
\end{center}
Норма зиску знизилася б наполовину, бо час обороту подвоївся.

Отже, маса додаткової вартості, привласнювана протягом року,
дорівнює масі додаткової вартості, привласнюваній за один період
обороту \emph{змінного} капіталу, помноженій на число таких оборотів
за рік. Якщо привласнювану за рік додаткову вартість або зиск
ми назвемо $М$, привласнювану за один період обороту додаткову
вартість — $m$, число річних оборотів змінного капіталу — $n$, то
$М = mn$, а річна норма додаткової вартості $М' = m'n$, як це
вже показано в книзі II, розд. XVI, 1.

Само собою зрозуміло, що формула норми зиску $р' = m' \frac{v}{K} =
m' \frac{v}{c+v}$ правильна тільки тоді, коли v чисельника однакове
з $v$ знаменника. У знаменнику $v$ є вся та частина всього капіталу,
яка пересічно застосована як змінний капітал на заробітну
плату; $v$ чисельника насамперед визначається тільки тим, що
воно виробило і привласнило певну кількість додаткової вартості
\index{iii1}{0090}  %% посилання на сторінку оригінального видання
 $= m$, відношення якої до нього, $\sfrac{m}{v}$, є норма додаткової
вартості $m'$. Тільки таким шляхом рівняння $р' = \frac{m}{c + v}$ перетворилось
в друге: $р' = m' \frac{v}{c + v}$. Тепер $v$ чисельника ближче визначається
тим, що воно мусить бути рівне $v$ знаменника, тобто
всій змінній частині капіталу $К$. Інакше кажучи, рівняння
$р' = m/K$ можна тільки тоді без помилки перетворити в друге рівняння
$р' = m' \frac{v}{c + v}$, коли $m$ означає додаткову вартість, вироблену
за один період обороту змінного капіталу. Якщо $m$ охоплює
тільки частину цієї додаткової вартості, то хоч $m - m'v$ є
правильне рівняння, але це $v$ тут менше, ніж $v$ в $K = c + v$, бо
воно менше, ніж весь змінний капітал, витрачений на заробітну
плату. Якщо ж $m$ охоплює більше, ніж додаткову вартість від
одного обороту $v$, то частина цього $v$ або навіть все $v$ функціонує
двічі: спочатку в першому, потім у другому або в другому
й дальших оборотах; отже, це $v$, яке виробляє додаткову
вартість і яке становить суму всієї виплаченої заробітної плати,
є більше, ніж $v$ в $c + v$, і тому обчислення стає неправильним.

Для того, щоб формула річної норми зиску стала цілком
правильною, ми повинні замість простої норми додаткової вартості
поставити річну норму додаткової вартості, тобто замість
$m'$ поставити $М'$, або $m'n$. Інакше кажучи, ми повинні помножити
$m'$, норму додаткової вартості — або, що зводиться до
того самого, вміщену в $К$ змінну частину капіталу $v$, — на $n$,
число оборотів цього змінного капіталу за рік, і таким чином
ми одержуємо: $р' = m'n \frac{v}{K}$, формулу для обчислення річної
норми зиску.

Але яка є величина змінного капіталу в певному підприємстві,
цього в більшості випадків не знає і сам капіталіст.
У восьмому розділі другої книги ми бачили і побачимо ще
далі, що єдина ріжниця в капіталі капіталіста, яка нав’язується
йому як істотна, є ріжниця основного й обігового капіталу.
З каси, в якій знаходиться частина обігового капіталу, яку він
має в своїх руках у грошовий формі, — оскільки вона не лежить
у банку, — він бере гроші для заробітної плати, з тієї самої
каси він бере гроші для сировинних і допоміжних матеріалів
і записує ті і другі на той самий рахунок каси. А коли б йому
й довелося вести окремий рахунок виплачуваної заробітної
плати, то цей рахунок в кінці року, правда, показав би виплачену
на заробітну плату суму, тобто $vn$, але не показав би
самого змінного капіталу $v$. Щоб визначити цей останній, капіталістові
\index{iii1}{0091}  %% посилання на сторінку оригінального видання
довелося б зробити окреме обчислення, приклад якого
ми хочемо тут навести.

Для цього ми візьмемо бавовнопрядільну фабрику на 10 000
мюльних веретен, описану в книзі І, стор. 227 \footnote*{
Стор. 152 рос. вид. 1935~\abbr{р.} Ред. укр. перекладу.
}, і припустимо
при цьому, що дані, взяті для одного тижня квітня 1871 року,
зберігають своє значення для цілого року. Основний капітал, вміщений
у машинах, становив 10 000 фунтів стерлінгів. Обіговий
капітал не був указаний; припустімо, що він становив 2 500 фунтів
стерлінгів, — досить висока сума, що виправдується, однак, тим
припущенням, яке ми тут весь час мусимо робити, а саме, що
не відбувається ніяких кредитних операцій, отже, що немає
тривалого чи тимчасового користування чужим капіталом. Тижневий
продукт щодо своєї вартості складався з 20 фунтів стерлінгів
на зношування машин, 358 фунтів стерлінгів авансованого
обігового сталого капіталу (плата за найом — 6 фунтів стерлінгів,
бавовна — 342 фунти стерлінгів, вугілля, газ, мастило — 10 фунтів
стерлінгів), 52 фунтів стерлінгів витраченого на заробітну
плату змінного капіталу і 80 фунтів стерлінгів додаткової вартості,
отже:\[
20c \text{(зношування)} + 358 c + 52 v + 80 m = 510.\]

Отже, щотижневе авансування обігового капіталу становило
$358 c + 52 v = 410$, і його процентний склад $= 87,3 c + 12,7 v$. При
обчисленні на весь обіговий капітал у 2500 фунтів стерлінгів
це дає 2182 фунти стерлінгів сталого і 318 фунтів стерлінгів
змінного капіталу. Через те, що вся витрата на заробітну плату
становила на рік 52 рази по 52 фунти стерлінгів, отже, 2704 фунти
стерлінгів, то виходить, що змінний капітал у 318 фунтів стерлінгів
обернувся за рік майже точно 8\sfrac{1}{2} разів. Норма додаткової
вартості була  $\sfrac{80}{52} = 153\sfrac{11}{13}\%$. За цими елементами ми обчисляємо
норму зиску, підставивши в формулу $р' = m'n \frac{v}{K}$ значення:
$m' = 153\sfrac{11}{13}\%, n = 8\sfrac{1}{2}, v = 318, K = 12500;$ отже:\[
р' = 153\sfrac{11}{13} × 8\sfrac{1}{2} × \frac{318}{12 500} = 33,27\%\]

Для перевірки цього ми скористуємось простою формулою
$р' = \frac{m}{K}$. Вся додаткова вартість, або зиск, становить за рік
52 фунти стерлінгів $× 80 = 4160$ фунтів стерлінгів; поділене на
весь капітал у 12 500 фунтів стерлінгів, це дає майже стільки ж, як
вище, 33,28\%, ненормально високу норму зиску, яка пояснюється
тільки надзвичайно сприятливими умовами даного моменту (дуже
дешеві ціни на бавовну поряд з дуже високими цінами на пряжу)
і в дійсності існувала, без сумніву, не на протязі всього року.

