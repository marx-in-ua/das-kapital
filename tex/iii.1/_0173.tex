
\index{iii1}{0173}  %% посилання на сторінку оригінального видання
Абож загальна норма зиску може змінюватись при незмінній вартості товарів, якщо змінюється ступінь
експлуатації праці.

Або, при незмінному ступені експлуатації праці, загальна
норма зиску може змінюватися в тому випадку, коли сума вживаної праці змінюється відносно сталого
капіталу в наслідок
технічних змін у процесі праці. Але такі технічні зміни завжди
мусять виявлятись у зміні вартості товарів, а тому й супроводитись цією зміною вартості товарів,
виробництво яких вимагає
тепер більше або менше праці, ніж раніше.

В першому відділі ми бачили, що додаткова вартість і зиск,
розглядувані щодо їхньої маси, тотожні. Однак, норма зиску вже
з самого початку відрізняється від норми додаткової вартості,
при чому спочатку це виявляється тільки як інша форма обрахунку; але через те що норма зиску може
підвищуватись або
падати при незмінній нормі додаткової вартості і навпаки, і що
капіталіста практично цікавить виключно норма зиску, то це
знов таки вже з самого початку цілком затемнює і містифікує
дійсне поводження додаткової вартості. Проте, кількісна ріжниця
була тільки між нормою додаткової вартості і нормою зиску,
а не між самими додатковою вартістю і зиском. Через те що
в нормі зиску додаткова вартість обчислюється на весь капітал
і відноситься до нього як до своєї міри, то вже в наслідок
цього здається, що сама додаткова вартість виникла з усього
капіталу, і при тому рівномірно з усіх його частин, так що
в понятті зиску стирається органічна ріжниця між сталим і змінним капіталом; тому, дійсно, у цій
своїй перетвореній формі,
у формі зиску, додаткова вартість сама заперечує своє походження, втрачає свій характер, стає
непізнаванною. Однак, досі
ріжниця між зиском і додатковою вартістю зводилась тільки до
якісної зміни, до зміни форми, тимчасом як дійсна кількісна
ріжниця на цьому першому ступені перетворення існує тільки
між нормою зиску і нормою додаткової вартості, але ще не існує
між зиском і додатковою вартістю.

Інакше стоїть справа, коли вже встановлюється загальна
норма зиску і за її допомогою пересічний зиск, відповідний до
даної для різних сфер виробництва величини застосовуваного
капіталу.

Тепер тільки випадково додаткова вартість, — а тому й зиск, — дійсно створена в якійсь окремій сфері
виробництва, може збігтися з зиском, який міститься в продажній ціні товару. Як загальне правило,
тепер зиск і додаткова вартість, а не тільки їх
норми, є дійсно різні величини. Тепер, при даному ступені
експлуатації праці, маса додаткової вартості, створена в якійсь
окремій сфері виробництва, має важливіше значення для сукупного пересічного зиску суспільного
капіталу, отже, для капіталістичного класу взагалі, ніж безпосередньо для капіталістів кожної
окремої галузі виробництва. Для них це має значення
\parbreak{}  %% абзац продовжується на наступній сторінці
