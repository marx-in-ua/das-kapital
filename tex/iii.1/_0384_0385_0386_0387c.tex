\parcont{}  %% абзац починається на попередній сторінці
\index{iii1}{0384}  %% посилання на сторінку оригінального видання
них різноманітними шляхами. Насамперед, в їхніх руках, оскільки
вони є касирами промислових капіталістів, концентрується грошовий
капітал, який кожен виробник і купець тримає як резервний
фонд або який припливає до нього по платежах. Ці
фонди перетворюються таким чином у позиковий грошовий
капітал. Завдяки цьому резервний фонд торговельного світу,
через те що він концентрується як суспільний, обмежується
необхідним мінімумом, і частина грошового капіталу, яка інакше
лежала б без діла як резервний фонд, віддається в позику,
функціонує як капітал, що дає процент. Подруге, позиковий
капітал банків утворюється з вкладів грошових капіталістів, які
передають їм справу віддачі їх у позику. Далі, з розвитком банкової
системи, а саме, як тільки банки починають платити проценти
по вкладах, до них вкладаються грошові заощадження
і тимчасово вільні гроші всіх класів. Дрібні суми, з яких кожна
сама по собі нездатна діяти як грошовий капітал, з’єднуються
у великі маси і таким чином утворюють грошову силу. Це нагромадження
дрібних сум, як особливий результат банкової системи,
слід відрізняти від її посередницької ролі між власне грошовими
капіталістами і позичальниками. Нарешті, в банки вкладаються
й доходи, які мають споживатися тільки поступінно.

Віддача в позику (тут ми маємо справу тільки з власне торговельним
кредитом) відбувається за допомогою дисконту векселів
— перетворення їх у гроші до скінчення їх строку — і за
допомогою позик у різних формах: прямих позик під особистий
кредит, позик під заставу процентних паперів, державних фондів,
акцій усякого роду, особливо ж позик під накладні, докові
варанти та інші засвідчені документи про право власності на
товари, під вклади і~\abbr{т. д.}

Кредит же, що його дає банкір, може даватися в різних формах
— наприклад, векселями на інші банки, чеками на них, відкриттям
кредиту того самого роду, нарешті, у банків, що випускають
банкноти, власними банкнотами банку. Банкнота є не
що інше, як вексель на банкіра, по якому пред’явник його може
в перший-ліпший час одержати гроші і яким банкір заміняє приватні
векселі. Ця остання форма кредиту здається профанові
особливо разючою і важливою, поперше, тому що цього роду
кредитні гроші переходять з простої торговельної циркуляції
в загальну циркуляцію і функціонують тут як гроші; а також
і тому, що в більшості країн головні банки, які випускають
банкноти, являючи собою дивну мішанину національного банку
і приватного банку, в дійсності мають за собою національний
кредит, а їх банкноти є більш чи менш законний засіб платежу;
тому що тут стає очевидним, що те, чим торгує банкір, є сам
кредит, бо банкнота представляє тільки кредитний знак, який перебуває
в циркуляції. Але банкір торгує кредитом і в усіх інших формах,
навіть коли дає готівкою в позику покладені до нього вкладом
гроші, В дійсності банкнота становить монету тільки для гуртової
\index{iii1}{0385}  %% посилання на сторінку оригінального видання
торгівлі, а найважливіше значення для банків завжди мають вклади. Найкращий доказ цьому дають
шотландські банки.

Для нашої мети немає потреби докладніше розглядати особливі види кредитних установ, як і особливі
форми самих банків.

\pfbreak{}

„Банкірська справа двояка\dots{} 1) Збирати капітал від тих, хто не знаходить
для нього безпосереднього застосування, і розподіляти й передавати його
тим, хто може його використати. 2) Приймати вклади з доходів своїх клієнтів
і виплачувати їм суми в міру того, як вони їм стають потрібні для видатків
на предмети споживання. Перше є циркуляція \emph{капіталу}, останнє — циркуляція
\emph{грошей} (currency)“. — „Перше є концентрація капіталу, з одного боку, і розподіл його, з другого
боку, друге є управління циркуляцією для місцевих
цілей округи“. — \emph{Tooke}: „Inquiry into the Currency Principle“, стор. 36, 37. Ми
повернемось до цього місця в XXVIII розділі.

„Reports of Committees“, т. VIII. „Commercial Distress“, т. І, частина I, 1847--48.
Minutes of Evidence. — (Далі цитується як: „Commercial Distress“, 1847--48). В сорокових роках при
дисконті векселів у Лондоні в незчисленних випадках замість банкнот приймали векселі одного банку на
другий строком на 21 день.
(Свідчення J.~Pease, провінціального банкіра, № 4636 і 4645). Згідно з тим самим звітом, банкіри
мали звичай, коли грошей ставало мало, регулярно платити
своїм клієнтам такими векселями. Якщо одержувач хотів банкнот, то він мусив
знов дисконтувати цей вексель. Для банків це дорівнювало привілеєві робити
гроші. Пани Jones, Loyd and C° платили таким способом „з незапам’ятних часів“,
якщо грошей ставало мало і розмір процента перевищував 5\%. Клієнт був
радий одержати такі Bankers Bills [банківські векселі], бо векселі Jones, Loyd
and C° можна було легше дисконтувати, ніж свої власні; вони часто переходили також через 20--30 рук
(там же, № 901--905, 992).

Всі ці форми служать для того, щоб зробити передаваним право на одержання платежу. „Навряд чи існує
яка-небудь форма кредиту, в якій йому часами
не доводилося б виконувати функції грошей; чи є цією формою банкнота, чи
вексель, чи. чек, процес по суті є той самий і результат по суті є той самий“. — \emph{Fullarton}: „On the
Regulation of Currencies“, 2 вид., Лондон 1845, стор. 38. — „Банкноти — дрібні кредитні гроші“
(стор. 51).

Нижченаведене з \emph{J.~W.~Gilbart}: „The History and Principles of Banking“,
London 1834: „Капітал банку складається з двох частин — з основного капіталу
(invested capital) і банкового капіталу (banking capital), взятого в позику“ (стор. 117).
„Банковий капітал або капітал, взятий у позику, одержується трьома шляхами:
1) прийманням вкладів, 2) випуском власних банкнот, 3) видачею векселів. Якщо
хтонебудь схоче позичити мені задарма 100\pound{ фунтів стерлінгів}, а я позичу ці
100\pound{ фунтів стерлінгів} комусь іншому по 4\% то за рік я на цій справі одержу
4\pound{ фунти стерлінгів} доходу. Так само якщо хтонебудь схоче взяти моє платіжне
зобов’язання“ (I promise to pay [я обіцяю заплатити] — звичайна формула англійських банкнот) „і
наприкінці року поверне мені його, заплативши мені за це 4\%, цілком так само, як коли б я позичив
йому 100\pound{ фунтів стерлінгів}, то я на цій справі одержу 4\pound{ фунти стерлінгів} доходу; і, далі, якщо
хто-небудь у провінціальному місті принесе мені 100\pound{ фунтів стерлінгів} з умовою, щоб я через 21 день
заплатив цю суму в Лондоні третій особі, то всякий процент, який я за проміжний час зможу одержати
на ці гроші, буде моїм зиском. До цього по суті справи зводяться операції банку і той шлях, яким
утворюється банковий капітал за допомогою вкладів, банкнот і векселів“ (стор. 117). „Загалом зиски
банкіра пропорційні сумі одержаного ним у позику капіталу або банкового капіталу. Щоб визначити
дійсний зиск банку, треба з гуртового зиску відняти процент на основний капітал. Остача є банковий
зиск“ (стор. 118). „\emph{Позики банкіра своїм клієнтам робляться грішми інших людей}“ (стор. 146). „Саме
ті банкіри, які не випускають банкнот, створюють банковий капітал за допомогою дисконтування
векселів. Вони збільшують свої вклади за допомогою своїх дисконтних операцій. Лондонські банкіри
дисконтують векселі тільки для тих фірм, які мають у них рахунок вкладів“ (стор. 119). „Фірма, яка
дисконтує векселі у своєму банку і платить проценти на всю суму цих векселів, мусить принаймні
частину цієї суми залишити
\index{iii1}{0386}  %% посилання на сторінку оригінального видання
в руках банку, не одержуючи за неї процентів. Таким шляхом банкір одержує на гроші, які він дає в
позику, вищий за поточний розмір процента і
створює собі банковий капітал за допомогою тієї остачі, що лишається в його руках“ (стор. 120). —
Економія резервних фондів, вкладів, чеків: „Депозитні банки за
допомогою трансферту активів заощаджують на вживанні засобів циркуляції і з
незначною сумою дійсних грошей виконують операції на великі суми. Звільнені таким чином гроші
вживаються банкіром на позики своїм клієнтам за
допомогою дисконту і~\abbr{т. д.} Тому трансферт активів підвищує діяльність депозитної системи“ (стор.
123). „Чи мають обидва клієнти, які ведуть між собою
справи, свої рахунки в одного й того ж банкіра, чи в різних банкірів, — це
однаково. Бо банкіри обмінюють між собою свої чеки в Clearing House [розрахунковій палаті]. За
допомогою трансферту депозитна система, могла б
таким чином досягти такого ступеня поширення, що зовсім витиснула б
уживання металічних грошей. Якби кожен мав у банку рахунок вкладів і робив би всі свої платежі за
допомогою чеків, то ці чеки стали б єдиним засобом циркуляції. В цьому випадку довелося б
припускати, що банкіри мають
гроші в своїх руках, інакше чеки не мали б ніякої цінності“ (стор. 124). Централізація місцевого
обігу в руках банків проводиться за допомогою 1) філіальних банків. Провінціальні банки мають філії
в невеличких містах своєї округи;
лондонські банки — в різних частинах міста Лондона; 2) за допомогою агентур.
„Кожний провінціальний банк має агента в Лондоні, щоб там оплачувати свої
банкноти або векселі і одержувати гроші, що їх сплачують жителі Лондона за рахунок осіб, які живуть
у провінції“ (стор. 127). Кожний банкір збирає банкноти
іншого банку і вже не видає їх. В кожному більш-менш великому місті
вони сходяться раз або двічі на тиждень і обмінюються банкнотами. Сальдо
виплачується переказом на Лондон (стор. 134). „Мета банків — полегшувати
справи. Все, що полегшує справи, полегшує і спекуляцію. Справи і спекуляція
в багатьох випадках так тісно зв’язані між собою, що важко сказати, де кінчаються справи і
починається спекуляція\dots{} Всюди, де є банки, капітал можна
одержувати легше й дешевше. Дешевина капіталу дає поштовх спекуляції,
подібно до того, як дешевина м’яса й пива дає поштовх прожерливості й пияцтву“ (стор. 137, 138).
„Тому що банки, які випускають власні банкноти, завжди
платять цими банкнотами, то може здатися, що їх дисконтні операції робляться
виключно за допомогою капіталу, одержаного таким способом; але це не так.
Банкір, звичайно, може всі дисконтовані ним векселі оплатити своїми власними
банкнотами, і все ж \sfrac{9}{10} векселів, які є в його портфелі, можуть репрезентувати дійсний капітал. Бо,
хоч він сам за ці векселі видав тільки свої власні паперові гроші, вони зовсім не повинні
обов’язково лишатись у циркуляції до
скінчення строку векселів. Векселі можуть мати тримісячний строк, а банкноти
можуть повернутися через три дні“ (стор. 172). „Покриття рахунку клієнтами
становить нормальну ділову операцію. Це в дійсності та мета, для якої
гарантується кредит готівкою\dots{} Кредити готівкою забезпечуються не тільки
особистою гарантією, але й вкладом цінних паперів“ (стор. 174, 175). „Капітал,
даний у позику під заставу товарів, справляє той самий вплив, що й капітал,
даний у позику під дисконт векселів. Якщо хтонебудь бере в позику 100\pound{ фунтів
стерлінгів} під забезпечення своїми товарами, то це те саме, як коли б він продав
їх за вексель у 100\pound{ фунтів стерлінгів} і дисконтував би його в банкіра.
Але позика дає йому змогу придержати свої товари до кращого стану ринку і
уникнути жертв, яких інакше йому довелося б зазнати, щоб одержати гроші
на невідкладні справи“ (стор. 180, 181).

„The Currency Theory Reviewed etc.“ [Edinburgh 1845] стор. 62, 63: „Незаперечна правда, що 1000\pound{ фунтів стерлінгів}, які я сьогодні вклав до $А$, завтра
знову будуть видані і стануть вкладом у $В$. Позавтра вони знову можуть бути
видані $В$, стати вкладом у $C$, і так далі до безконечності. Отже, ті самі
1000\pound{ фунтів стерлінгів} грошей можуть за допомогою ряду передач помножитись до абсолютно невизначної
суми вкладів. Тому можливо, що дев’ять десятих усіх вкладів в Англії зовсім не існують поза
бухгалтерськими рубриками
в книгах банкірів, кожен з яких відповідає за свою частину\dots{} Так, у Шотландії, де гроші, які є в
циркуляції“ [до того ж майже виключно паперові гроші!],
„ніколи не перевищують 3 мільйонів фунтів стерлінгів, вклади становлять 27 мільйонів.
\index{iii1}{0387}  %% посилання на сторінку оригінального видання
Поки не настане загальне раптове, вимагання повернути вклади (a run
on the banks [штурм банків]), ті самі 1000\pound{ фунтів стерлінгів}, мандруючи
назад, можуть з такою самою легкістю покрити таку ж невизначну суму.
Через те що ті самі 1000\pound{ фунтів стерлінгів}, якими я сьогодні покриваю свій
борг якомусь комерсантові, завтра можуть покрити його борг іншому купцеві,
а позавтра борг цього останнього банкові, і так далі до безконечності, то ті ж
самі 1000\pound{ фунтів стерлінгів} можуть переходити з рук у руки, від банку до
банку, і покрити яку завгодно суму вкладів“.\footnote*{
Цей абзац перенесений сюди Енгельсом з іншої частини рукопису Маркса і в першому німецькому
виданні позначений нумером 7. Примітка ред. нім. вид. ІМЕЛ.
}

\pfbreak{}

[Ми бачили, що Gilbart вже в 1834 році розумів, що „все,
що полегшує справи, полегшує і спекуляцію, що те і друге
в багатьох випадках так тісно зв’язане між собою, що важко
сказати, де кінчаються справи і починається спекуляція“. Чим
легше можна одержати позики під непродані товари, тим більше
беруться такі позики, тим більша спокуса виробляти товари або
вироблені вже товари кидати на віддалені ринки, тільки для того,
щоб спершу одержати під них грошову позику. Як весь торговельний світ країни може бути охоплений
такою спекуляцією
і чим вона закінчується, — яскравий приклад цього дає нам історія
англійської торгівлі 1845--1847~\abbr{рр.} Тут ми бачимо, що може зробити кредит. Для пояснення дальших
прикладів слід спочатку
зробити лиш декілька коротких зауважень.

Наприкінці 1842 року пригнічення, яке тяжило над англійською промисловістю майже безперервно з 1837
року, почало
слабшати. Протягом двох наступних років закордонний попит
на продукти англійської промисловості підвищився ще більше;
роки 1845--1846 були періодом найвищого розквіту. В 1843 році
війна за довіз опію відкрила для англійської торгівлі Китай. Новий
ринок дав новий поштовх розширенню, особливо бавовняної
промисловості, яка вже досягла повного розвитку. „Як можемо
ми виробляти занадто багато? Адже нам треба одягти 300 мільйонів чоловіка“, — казав тоді авторові
цих рядків один манчестерський фабрикант. Але всіх цих новозбудованих фабричних
будівель, парових і прядільних машин і ткацьких верстатів було
недосить для поглинення додаткової вартості Ланкашіра, яка
припливала великою масою. З такою самою пристрастю, з якою
збільшували виробництво, кинулись на будівництво залізниць;
тут насамперед знайшла собі задоволення жадоба фабрикантів і
купців до спекуляції, і саме вже з літа 1844 року. Підписувались на акції, скільки могли, тобто
оскільки вистачало грошей для покриття перших внесків; для дальших внесків — засоби знайдуться! А
коли настав строк дальших внесків, — згідно
з запитанням 1059, „Commercial Distress“ 1848/57, капітал, вкладений у 1846/47~\abbr{рр.} в залізниці,
становив 75 мільйонів фунтів
стерлінгів, — довелося вдатися до кредиту, і при цьому власні
справи фірми здебільшого теж мусили потерпіти.
