\parcont{}  %% абзац починається на попередній сторінці
\index{iii1}{0268}  %% посилання на сторінку оригінального видання
купівлями й продажами, так що ці операції розвиваються в форму
особливого заняття, відокремленого від усіх інших функцій промислового капіталу і тому самостійного.
Це — особлива форма
суспільного поділу праці, так що певна частина функції, яка
взагалі має бути виконана в особливій фазі процесу репродукції
капіталу, — в даному випадку у фазі циркуляції, — виступає як
виключна функція особливого агента циркуляції, відмінного від
виробника. Однак, від цього це особливе заняття ні в якому
разі не виступало б ще як функція особливого капіталу, відмінного від промислового капіталу, що
перебуває в процесі своєї
репродукції, і самостійного відносно промислового капіталу; це
заняття і в дійсності не виступає як таке в тих випадках, коли
торгівля товарами провадиться за допомогою простих комівояжерів або інших безпосередніх агентів
промислового капіталіста.
Отже, до цього мусить долучитися ще й другий момент.

\emph{Подруге}: Цей другий момент полягає в тому, що самостійний агент циркуляції, купець, авансує в цьому
своєму становищі грошовий капітал (власний або позичений). Те, що для промислового капіталу, який
перебуває в процесі своєї репродукції, виступає просто як $Т — Г$, як перетворення товарного
капіталу в грошовий капітал або простий продаж, те для купця
виступає як $Г — Т — Г'$, як купівля і продаж того самого товару,
отже, як повернення до нього за допомогою продажу того грошового капіталу, який від нього віддалився
при купівлі.

Те, що для купця, оскільки він авансує капітал на купівлю
товару у виробника, виступає як $Г — Т — Г$, це є завжди $Т — Г$,
перетворення товарного капіталу в грошовий капітал; це завжди
перша метаморфоза товарного капіталу, хоча той самий акт
може для виробника або для промислового капіталу, який перебуває в процесі своєї репродукції,
виступати як $Г — Т$, як зворотне перетворення грошей у товар (в засоби виробництва) або як друга фаза
метаморфози. Для виробника полотна першою
метаморфозою було $Т — Г$, перетворення товарного капіталу
в грошовий капітал. Цей акт для купця виступає як $Г — Т$, як
перетворення його грошового капіталу в товарний капітал. Якщо ж
він продає полотно білільникові, то для білільника це становить $Г — Т$, перетворення грошового
капіталу в продуктивний
капітал, або другу метаморфозу його товарного капіталу; але
для купця це становить $Т — Г$, продаж купленого ним полотна.
Але в дійсності товарний капітал, вироблений фабрикантом полотна, тільки тепер остаточно проданий,
або це $Г — Т — Г$ купця
становить тільки опосереднюючий процес для $Т — Г$ між двома
виробниками. Або припустімо, що фабрикант, який виробляє
полотно, на частину вартості проданого полотна купує пряжу
в торговця пряжею. Таким чином це є для нього $Г — Т$. Але для
купця, який продає пряжу, це є $Т — Г$, перепродаж пряжі; а відносно самої пряжі як товарного капіталу
це є тільки остаточний
продаж її, в наслідок якого вона переходить із сфери циркуляції
\index{iii1}{0269}  %% посилання на сторінку оригінального видання
у сферу споживання, $Т — Г$, остаточне вивершення її першої метаморфози. Отже, чи купує купець у
промислового капіталіста чи продає йому, його $Г — Т — Г$, кругобіг купецького капіталу, завжди
виражає тільки те, що відносно самого товарного капіталу, як перехідної форми промислового капіталу,
який
репродукує себе, є просто  $Т — Г$, просто здійснення його першої метаморфози. $Г — Т$ купецького
капіталу є разом з тим  $Т — Г$
тільки для промислового капіталіста, але не для вироблюваного
ним товарного капіталу: це тільки перехід товарного капіталу
з рук промисловця в руки агента циркуляції; тільки  $Т — Г$ купецького капіталу є остаточне  $Т — Г$
функціонуючого товарного
капіталу. $Г — Т — Г$ є тільки два  $Т — Г$ того самого товарного капіталу, два послідовні продажі його,
які тільки й опосереднюють його останній і остаточний продаж.

Отже, товарний капітал набирає у товарно-торговельному
капіталі вигляду самостійного роду капіталу в наслідок того, що
купець авансує грошовий капітал, який зростає в своїй вартості
як капітал, функціонує як капітал лиш остільки, оскільки він
уживається виключно для того, щоб опосереднювати метаморфозу товарного капіталу, його функцію як
товарного капіталу,
тобто його перетворення в гроші, і він робить це за допомогою постійної купівлі й продажу товарів.
Це є його виключна
операція; ця діяльність, яка опосереднює процес циркуляції
промислового капіталу, є виключна функція грошового капіталу,
яким оперує купець. Завдяки цій функції він перетворює свої
гроші в грошовий капітал, надає своєму $Г$ вигляду $Г — Т — Г'$
і за допомогою того самого процесу перетворює товарний капітал у товарно-торговельний капітал.

Товарно-торговельний капітал, оскільки і поки він існує в формі
товарного капіталу, — з точки зору процесу репродукції сукупного суспільного капіталу, — є,
очевидно, не що інше, як
частина промислового капіталу, яка перебуває ще на ринку,
пророблює процес своєї метаморфози і тепер існує та функціонує як товарний капітал. Отже, грошовий
капітал, який ми
повинні тепер розглядати у відношенні до сукупного процесу
репродукції капіталу, є тільки авансований купцем \emph{грошовий}
капітал, який призначений виключно для купівлі й продажу
і який через це ніколи не набирає іншої форми, крім форми товарного капіталу і грошового капіталу,
ніколи не набирає форми
продуктивного капіталу і завжди лишається замкненим у сфері
циркуляції капіталу.

Як тільки виробник, фабрикант полотна, продасть свої \num{30000}
метрів купцеві за 3000 фунтів стерлінгів, він купує на виручені таким чином гроші потрібні засоби
виробництва, і його капітал
знову вступає у процес виробництва; його процес виробництва
триває далі, іде безперервно. Перетворення його товару в гроші
для нього відбулось. Але для самого полотна це перетворення
як ми бачили, ще не відбулось. Полотно ще не остаточно перетворилося
\index{iii1}{0270}  %% посилання на сторінку оригінального видання
в гроші, ще не ввійшло як споживна вартість у споживання,
продуктивне чи особисте. Торговець полотном репрезентує
тепер на ринку той самий товарний капітал, який первісно
репрезентував на ньому виробник полотна. Для цього
останнього процес метаморфози скоротився, але тільки для того,
щоб продовжуватись у руках купця.

Коли б виробник полотна мусив чекати, поки його полотно
дійсно перестане бути товаром, поки воно перейде до останнього
покупця, продуктивного або особистого споживача, то
його процес репродукції був би перерваний. Абож для того, щоб
його не переривати, він мусив би обмежити свої операції, перетворювати
в пряжу, вугілля, працю і~\abbr{т. д.}, коротко кажучи,
в елементи продуктивного капіталу, меншу частину свого полотна,
а більшу частину його тримати в себе як грошовий резерв
для того, щоб, поки одна частина його капіталу як товар
перебуває на ринку, друга частина могла продовжувати процес
виробництва, так що коли одна частина його капіталу надходить
на ринок як товар, друга повертається назад у грошовій
формі. Цей поділ його капіталу не усувається втручанням купця.
Але без цього останнього частина капіталу циркуляції, наявна
у формі грошового резерву, завжди мусила б бути відносно
більшою порівняно з частиною, застосовуваною у формі продуктивного
капіталу, і відповідно до цього скоротився б масштаб
репродукції. Замість цього виробник може тепер постійно
застосовувати більшу частину свого капіталу на власне процес
виробництва і меншу частину як грошовий резерв.

Але зате тепер інша частина суспільного капіталу, в формі
купецького капіталу, постійно перебуває у сфері циркуляції.
Вона завжди застосовується тільки для того, щоб купувати
і продавати товари. Таким чином, видимо, відбувається тільки
переміна осіб, в руках яких перебуває цей капітал.

Коли б купець, замість того, щоб купити на 3000 фунтів
стерлінгів полотна з метою знову продати його, сам продуктивно
застосував би ці 3000 фунтів стерлінгів, то продуктивний капітал
суспільства збільшився б. Звичайно, тоді виробник полотна —
а так само й купець, який перетворився тепер у промислового
капіталіста — мусив би значнішу частину свого капіталу тримати
в себе як грошовий резерв. З другого боку, якщо купець
лишається купцем, то виробник заощаджує час, потрібний
для продажу, і може вживати його для нагляду за процесом
виробництва, тимчасом як купець мусить весь свій час уживати
для продажу.

Якщо купецький капітал не перевищує своїх необхідних пропорцій,
то слід визнати:

1) що в наслідок поділу праці капітал, який займається виключно
купівлею і продажем (а, крім грошей, потрібних для
купівлі товарів, сюди належать також гроші, які мусять витрачатись
на працю, необхідну для ведення торговельного підприємства,
\index{iii1}{0271}  %% посилання на сторінку оригінального видання
на сталий капітал купця, складські будівлі, транспорт
і~\abbr{т. д.}), є менший, ніж він був би в тому випадку, коли б промисловий
капітал сам мусив вести всю торговельну частину
свого підприємства;

2) що через те що купець займається виключно цією справою,
то не тільки для виробника його товар раніше перетворюється
в гроші, але й сам товарний капітал пророблює свою
метаморфозу швидше, ніж він міг би це робити у руках виробника;

3) що, коли розглядати сукупний купецький капітал відносно
промислового капіталу, то один оборот купецького капіталу
може представляти не тільки обороти багатьох капіталів в одній
сфері виробництва, але й обороти кількох капіталів у різних
сферах виробництва. Перше має місце тоді, коли, наприклад,
торговець полотном, після того як він на свої 3000 фунтів
стерлінгів купив і знову продав продукт виробника полотна,
раніше ніж той самий виробник знову кине на ринок ту саму
кількість товарів, купує і знову продає продукт іншого або декількох
інших виробників полотна, опосереднюючи таким чином
обороти різних капіталів у тій самій сфері виробництва. Друге
має місце тоді, коли купець, наприклад, після продажу полотна
купує шовк, отже, опосереднює оборот капіталу в іншій сфері
виробництва.

Взагалі слід зауважити таке. Оборот промислового капіталу
обмежується не тільки часом обігу, але й часом виробництва.
Оборот купецького капіталу, оскільки він торгує тільки товарами
певного роду, обмежується не оборотом одного промислового
капіталу, а оборотом усіх промислових капіталів однієї
і тієї ж галузі виробництва. Купець, після того як він купить
і продасть полотно одного, може потім купити і продати полотно
іншого, раніше ніж перший знову кине товар на ринок.
Отже, той самий купецький капітал може послідовно опосереднювати
різні обороти капіталів, вкладених у якійнебудь галузі
виробництва; так що його оборот не є тотожний з оборотами
якогонебудь окремого промислового капіталу, і тому він заміщає
не тільки той грошовий резерв, що його мусив би мати in
petto [напоготові] цей окремий промисловий капіталіст. Оборот
купецького капіталу в якійсь сфері виробництва, звичайно, обмежений
сукупним виробництвом цієї сфери. Але він не обмежений
границями виробництва або часом обороту одиничного
капіталу цієї сфери, оскільки цей час обороту визначається часом
виробництва. Припустім, що А постачає товар, який потребує
для свого виробництва три місяці. Після того як купець
купить його і продасть, скажімо, протягом одного місяця, він
може купити і продати такий самий продукт іншого виробника.
Або, наприклад, після того як він продасть хліб одного фермера,
він може на ті самі гроші купити й продати хліб другого
фермера і~\abbr{т. д.} Оборот його капіталу обмежений масою
\parbreak{}  %% абзац продовжується на наступній сторінці
