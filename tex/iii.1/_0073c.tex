\parcont{}  %% абзац починається на попередній сторінці
\index{iii1}{0073}  %% посилання на сторінку оригінального видання
дають, в наслідок зміни $v$, друге рівняння:\[
р'\textsubscript{1} = m' (\frac{v\textsubscript{1}}{К'}),\]
в якому $v$ перейшло у $v\textsubscript{1}$ а $р'\textsubscript{1}$, змінена норма зиску, яка випливає
з цього, має бути знайдена.

Вона визначається за допомогою відповідної пропорції:\[
р': р'\textsubscript{1} = m' \frac{v}{К}: m'\frac{v\textsubscript{1}}{K} = v: v\textsubscript{1}\]

Або: при незмінній нормі додаткової вартості і незмінному цілому
капіталі первісна норма зиску відноситься до норми зиску,
що виникла в наслідок зміни змінного капіталу, як первісний
змінний капітал відноситься до зміненого.
Якщо капітал первісно був такий, як вище:

$\text{I. } \num{15000} К = \num{12000} c + 3000 v (+ 3000 m);$ а тепер він:

$\text{II. } \num{15000} К = \num{13000} c + 2000 v (+ 2000 m),$ то $К = \num{15000}$
і $m' = 100\%$ в обох випадках, а норма зиску І, 20\%, відноситься
до норми зиску II, 13\sfrac{1}{3}\%, як змінний капітал І, 3000,
до змінного капіталу II, 2000, отже $20\%:13\sfrac{1}{3}\% = 3000:2000$.

Але змінний капітал може або підвищитись або зменшитись:
Візьмімо спочатку приклад, коли він підвищується. Нехай капітал
спочатку складається і функціонує так:

$\text{I. } 100c + 20v + 10m; К = 120, m' = 50\%, р' = 8\sfrac{1}{3}\%$

Нехай тепер змінний капітал підвищиться до 30; тоді, згідно
з припущенням, щоб весь капітал лишився незмінним = 120, сталий
капітал мусить зменшитися з 100 до 90. Вироблена додаткова
вартість, при тій самій нормі додаткової вартості в 50\%, мусить
підвищитись до 15. Отже, ми маємо:

$\text{II. } 90 c + 30 v + 15 m; К = 120, m' = 50\%, р' = 12\sfrac{1}{2}\%$

Спочатку виходитимем з того припущення, що заробітна
плата не змінюється. Тоді інші фактори норми додаткової вартості,
робочий день і інтенсивність праці, теж мусять лишитись
незмінними. Отже, підвищення $v$ (з 20 до 30) може мати тільки
те значення, що робітників уживається наполовину більше.
Тоді й уся нововироблена ними вартість підвищується наполовину,
з 30 до 45, і розподіляється цілком так само, як і раніш:
\sfrac{2}{3} на заробітну плату і \sfrac{1}{3} на додаткову вартість. Але одночасно,
при збільшеному числі робітників, знизився сталий капітал,
вартість засобів виробництва, з 100 до 90. Отже, ми маємо
перед собою випадок меншаючої продуктивності праці, зв’язаний
з одночасним зменшенням сталого капіталу; чи є цей випадок
економічно можливий?
