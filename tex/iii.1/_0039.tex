\parcont{}  %% абзац починається на попередній сторінці
\index{iii1}{0039}  %% посилання на сторінку оригінального видання
не давала додаткової вартості ніякому капіталові. Якщо вони І мусили частину свого продукту віддавати даром третій особі, то
тільки в формі данини феодальному панові. Тим то купецький капітал, принаймні спочатку, міг видушувати для себе зиск тільки
з чужоземних покупців тубільних продуктів або з тубільних покупців чужоземних продуктів; лиш у кінці цього періоду — отже,
для Італії з занепадом торгівлі з Левантом — чужоземна конкуренція і утруднений збут могли примушувати ремісничого виробника
експортних товарів продавати товари купцеві-експортерові нижче їх вартості. Таким чином ми знаходимо тут таке явище, що у
внутрішньому роздрібному обороті між окремими виробниками товари продаються пересічно по їх вартостях, тимчасом як у
міжнародній торгівлі, з вищенаведених причин, вони, як правило, продаються не по вартостях. Це — повна протилежність до
сучасного світу, де ціни виробництва мають силу в міжнародній і гуртовій, торгівлі, тимчасом як у міській роздрібній
торгівлі утворення цін регулюється зовсім іншими нормами зиску, так що, наприклад, на ціну волового м’яса нині накладається
більша надбавка на шляху від лондонського гуртового торговця до лондонського споживача, ніж на шляху від чікагського
гуртового торговця, включаючи також його витрати на транспорт, до лондонського гуртового торговця.

Знаряддям цього
ступневого перевороту в ціноутворенні був промисловий капітал. Уже в середні віки цьому був покладений початок, а саме в
трьох галузях: судоходстві, гірничій промисловості і текстильній промисловості. Судоходство в тому масштабі, в якому воно
велось італійськими і ганзейськими приморськими республімми, було неможливе без матросів, тобто найманих робітників (їх
відносини найму могли бути заховані під артільними формами з участю в зисках), а галери того часу були неможливі без гребців
— найманих робітників або рабів. Підприємства для добування руди, первісно артілі робітників, майже в усіх випадках
перетворились уже в акційні товариства для експлуатації рудників за допомогою найманих робітників. Щодо текстильної
промисловості, то в ній купець почав ставити дрібних майстрів-ткачів. у пряму залежність від себе, постачаючи їм пряжу і
заставляючи їх переробляти цю пряжу в тканину його коштом за певну плату, — коротко кажучи, перетворюючись з простого купця
в так званого \emph{роздатчика}.

Тут ми маємо перед собою перші початки утворення капіталістичної додаткової вартості. Гірничі
підприємства ми можемо залишити осторонь, як замкнені монопольні корпорації. Щодо судоходства, то очевидно, що зиски з нього
повинні були принаймні дорівнювати звичайному в даній країні зискові з спеціальною надбавкою на страхування, зношування
суден і т. д. Але як стояла справа з роздатчиками в текстильній промисловості, які вперше винесли на ринок товари,
виготовлені прямо коштом
\parbreak{}  %% абзац продовжується на наступній сторінці
