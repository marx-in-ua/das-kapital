
\index{iii1}{0064}  %% посилання на сторінку оригінального видання
Величина вартості всього капіталу сама по собі не стоїть
у будь-якому внутрішньому відношенні до величини додаткової
вартості, принаймні не стоїть безпосередньо. Щодо своїх речових
елементів весь капітал мінус змінний капітал, отже, сталий капітал,
складається з речових умов здійснення праці, з засобів праці
і матеріалу праці. Для того, щоб певна кількість праці реалізувалась
у товарах і, отже, утворила вартість, потрібна певна
кількість матеріалу праці і засобів праці. Залежно від особливого
характеру додаваної праці існує певне технічне відношення
між масою праці і масою засобів виробництва, до яких повинна
бути додана ця жива праця. Отже, остільки існує також певне
відношення між масою додаткової вартості або додаткової праці
і масою засобів виробництва. Якщо, наприклад, час, необхідний
для виробництва заробітної плати, становить 6 годин на день,
то робітник мусить працювати 12 годин, щоб дати 6 годин додаткової
праці, щоб створити додаткову вартість у 100\%. Він
споживає за ці 12 годин удвоє більше засобів виробництва, ніж
за ці 6 годин. Але від цього додаткова вартість, яку він додає
за 6 годин, зовсім не стає в будь-яке безпосереднє відношення
до вартості засобів виробництва, спожитих за ці 6 чи навіть
за ці 12 годин. Ця вартість тут не має ніякого значення; ідеться
тільки про технічно необхідну масу. Чи сировинний матеріал або
засоби праці дешеві, чи дорогі, це не має ніякого значення;
аби тільки вони мали потрібну споживну вартість і були наявні
в технічно встановленій пропорції до тієї живої праці, яку треба
поглинути. Однак, якщо я знаю, що за одну годину перепрядається
$х$ фунтів бавовни, які коштують $а$ шилінгів, то я, звичайно,
знаю і те, що за 12 годин перепрядається 12 $х$ фунтів
бавовни \deq{} 12 $а$ шилінгам, і тоді я можу обчислити відношення
додаткової вартості до вартості цих 12, так само як і до вартості
цих 6. Але відношення живої праці до \emph{вартості} засобів
виробництва тут привходить лиш остільки, оскільки $а$ шилінгів
служать назвою для $х$ фунтів бавовни; бо певна кількість бавовни
має певну ціну, а тому й навпаки, певна ціна може служити
показником певної кількості бавовни, поки ціна бавовни
не зміниться. Якщо я знаю, що для того, щоб привласнити 6 годин
додаткової праці, я повинен примушувати працювати 12 годин,
отже, мушу мати напоготові бавовни на 12 годин, і якщо я знаю
ціну цієї потрібної для 12 годин кількості бавовни, то посередньо
існує відношення між ціною бавовни (як показником необхідної
кількості) і додатковою вартістю. Навпаки, з ціни сировинного
матеріалу я ніколи не можу зробити висновок про масу сировинного
матеріалу, яка може бути перепрядена, наприклад, за
одну годину, а не за 6. Отже, немає ніякого внутрішнього, необхідного
відношення між вартістю сталого капіталу, — а тому
і між вартістю всього капіталу ($= c \dplus{} v$) і додатковою вартістю.

Якщо норма додаткової вартості відома і величина додаткової
вартості дана, то норма зиску виражає не що інше, як
\parbreak{}  %% абзац продовжується на наступній сторінці
