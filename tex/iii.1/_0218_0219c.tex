\parcont{}  %% абзац починається на попередній сторінці
\index{iii1}{0218}  %% посилання на сторінку оригінального видання
з двох мільйонів до трьох. Проте, не зважаючи на це зростання
абсолютної маси додаткової праці, а тому й додаткової
вартості на 50\%, відношення змінного капіталу до сталого
впало б з $2 : 4$ до $3 : 15$, і відношення додаткової вартості до
всього капіталу було б таке (в мільйонах):
\begin{gather*}
\text{\phantom{I}I. }\phantom{1}4c \dplus{} 2v \dplus{} 2m; K \deq{} \phantom{1}6, p' \deq{} 33\sfrac{1}{3}\%.\\
\text{II. }15c \dplus{} 3v \dplus{} 3m; K \deq{} 18, p' \deq{} 16\sfrac{1}{3}\%.
\end{gather*}
\noindent{}Тимчасом як маса додаткової вартості підвищилась наполовину,
норма зиску впала наполовину порівняно з попередньою.
Але зиск є тільки додаткова вартість, обчислена на суспільний
капітал, і тому маса зиску, його абсолютна величина, розглядувана
з точки зору всього суспільства, дорівнює абсолютній величині
додаткової вартості. Отже, абсолютна величина зиску, його
сукупна маса, зросла б на 50\%, не зважаючи на величезне зменшення
цієї маси зиску відносно авансованого сукупного капіталу
або не зважаючи на величезне зменшення загальної норми зиску.
Отже, число вживаних капіталом робітників, тобто абсолютна
маса праці, яка ним приводиться в рух, тому й абсолютна маса
вбираної ним додаткової праці, тому й маса виробленої ним додаткової
вартості, тому й абсолютна маса виробленого ним зиску
\emph{може} зростати і зростати прогресивно, не зважаючи на прогресивне
падіння норми зиску. Це не тільки \emph{може} бути. Це — залишаючи
осторонь минущі коливання — \emph{мусить} так бути на базі
капіталістичного виробництва.

Капіталістичний процес виробництва є разом з тим істотно і
процес нагромадження. Ми показали, як з розвитком капіталістичного
виробництва маса вартості, яка мусить бути просто
репродукована, збережена, збільшується і зростає разом з
підвищенням продуктивності праці, навіть якщо вживана робоча
сила лишається незмінною. Але з розвитком суспільної продуктивної
сили праці ще більше зростає маса вироблюваних споживних
вартостей, частину яких становлять засоби виробництва.
А добавна праця, через привласнення якої це додаткове багатство
може бути знову перетворене в капітал, залежить не від вартості,
а від маси цих засобів виробництва (включаючи й засоби
існування), бо в процесі праці робітник має справу не з вартістю,
а з споживною вартістю засобів виробництва. Однак, самонагромадження
і дана разом з ним концентрація капіталу є
матеріальний засіб підвищення продуктивної сили. Але це зростання
засобів виробництва передбачає зростання робітничого
населення, створення населення робітників, яке відповідає додатковому
капіталові і загалом і в цілому навіть завжди перевищує
його потреби, отже, створення перенаселення робітників.
Тимчасовий надлишок додаткового капіталу порівняно з робітничим
населенням, яке є в його розпорядженні, справляв би
двоякий вплив. З одного боку, він ступенево збільшував би робітниче
\index{iii1}{0219}  %% посилання на сторінку оригінального видання
населення шляхом підвищення заробітної плати, отже,
пом’якшенням згубних впливів, що скорочують приріст робітників,
і полегшенням шлюбів; а з другого боку, шляхом застосування
методів, які створюють відносну додаткову вартість (введення
й поліпшення машин), він ще далеко швидше створив би
штучне відносне перенаселення, яке з свого боку — бо в капіталістичному
виробництві злидні породжують населення, — знов таки є
теплицею дійсного швидкого збільшення чисельності населення.
Тому з природи капіталістичного процесу нагромадження —
який є тільки моментом капіталістичного процесу виробництва —
само собою випливає, що збільшена маса засобів виробництва,
призначених для перетворення в капітал, завжди знаходить під
рукою відповідно збільшене і навіть надлишкове робітниче населення,
яке можна експлуатувати. Отже, з розвитком процесу
виробництва і нагромадження \emph{мусить} зростати маса придатної
до привласнення і привласнюваної додаткової праці, а тому й абсолютна
маса зиску, привласнюваного суспільним капіталом. Але ті
самі закони виробництва і нагромадження разом з масою сталого
капіталу підвищують у дедалі більшій прогресії і його вартість, —
швидше, ніж вони підвищують вартість змінної частини капіталу,
обмінюваної на живу працю. Отже, одні й ті самі закони зумовлюють
для суспільного капіталу зростаючу абсолютну масу
зиску і падаючу норму зиску.

Ми тут цілком залишаємо осторонь те, що та сама величина
вартості з прогресом капіталістичного виробництва і відповідного
йому розвитку продуктивної сили суспільної праці та при помноженні
галузей виробництва, отже й продуктів, представляє прогресивно
зростаючу масу споживних вартостей і насолод.

Хід розвитку капіталістичного виробництва і нагромадження
зумовлює процеси праці в дедалі більшому масштабі, отже, в дедалі
більших розмірах, і відповідно до цього зумовлює зростаюче
авансування капіталу на кожне окреме підприємство. Тому
зростаюча концентрація капіталів (супроводжена в той самий
час, хоч і в меншій мірі, зростанням числа капіталістів) є так
само однією з матеріальних умов капіталістичного виробництва
і нагромадження, як і одним із створюваних ним самим результатів.
Рука в руку і у взаємодії із цим відбувається прогресуюча експропріація
більш чи менш безпосередніх виробників. Таким чином
для одиничних капіталістів стає зрозумілим, що вони мають
у своєму розпорядженні дедалі зростаючі робітничі армії (як би
сильно не падав їх змінний капітал порівняно з сталим), що маса
привласнюваної ними додаткової вартості, а тому й зиску, зростає
одночасно з падінням норми зиску і не зважаючи на це падіння.
Якраз ті самі причини, які концентрують маси робітничих армій
під командою окремих капіталістів, збільшують також масу застосовуваного
основного капіталу, як і сировинних та допоміжних
матеріалів,— збільшують відносно швидше, ніж масу вживаної
живої праці.
