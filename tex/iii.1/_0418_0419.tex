
\index{iii1}{0418}  %% посилання на сторінку оригінального видання
\section{Засоби циркуляції і капітал.
Погляди~Тука~і~Фуллартона}

Ріжниця між засобами циркуляції і капіталом, як її зображають Тук\footnote{Ми наводимо тут в оригіналі відповідне місце з Тука, цитоване у витягу німецькою мовою:
% REMOVED на стор. 440--441\footnote*{Сторінки за німецьким виданням ІМЕЛ; в цьому українському виданні відповідне місце цитується на стор. 385. \Red{Ред. укр. перекладу.}}
„The business of bankers, setting aside the
issue of promissory notes payable on demand, may be divided into two branches,
corresponding with the distinction pointed out by Dr. (Adam) Smith of the transactions between
dealers and dealers, and between dealers and consumers. One branch
of the bankers’business is to collect \emph{capital} from those, who have not immediate
employment for it, and to distribute or transfer it to those who have. The other
branch is to receive deposits of the \emph{incomes} of their customers, and to pay out
the amount, as it is wanted for expenditure by the latter in the objects of their
consumption\dots{} the former being a circulation of \emph{capital}, the latter of \emph{currency}“ [Залишаючи
осторонь випуск банкнот з платежем пред’явникові, банкірська справа
може бути поділена на дві галузі, які відповідають встановленому д-ром (Адамом)
Смітом розрізненню операцій між торговцями й торговцями і між торговцями й
споживачами. Одна галузь банкірської справи полягає в тому, щоб збирати
\emph{капітал} від тих, хто не знаходить для нього безпосереднього застосування,
і розподіляти його або передавати тим, хто може його використати. Друга галузь
полягає в тому, щоб приймати вклади з \emph{доходів} своїх клієнтів і виплачувати їм
суми в міру того, як вони їм стають потрібні для видатків на предмети споживання\dots{} Перша є
циркуляція \emph{капіталу}, остання — циркуляція \emph{грошей}.] —
(\emph{Tooke}: „Inquiry into the Currency Principle“, стор. 36). Перша є „the concentration of capital on
the one hand and the distribution of it on the other“ [концентрація
капіталу, з одного боку, і розподіл його, з другого боку], друга є „administering
the circulation for local purposes of the district“ [управління циркуляцією для
місцевих цілей округи] (там же, стор. 37). — Далеко ближче підходить до
правильного розуміння питання Kinnear у такому місці: „Гроші вживаються
для того, щоб виконувати дві істотно різні операції. Як засіб обміну між торговцями і торговцями
вони є знаряддям, за допомогою якого здійснюються передачі капіталу; тобто обмін певної суми
капіталу в грошах на рівну суму капіталу в товарах. Але гроші, витрачені на виплату заробітної плати
і на купівлю
й продаж між торговцем і споживачем, — це не капітал, а дохід; це частина доходу всього суспільства,
яка вживається на щоденні видатки. Ці гроші циркулюють у невпинному щоденному вжитку, і тільки ці
гроші можуть бути названі
засобами циркуляції (currency) в строгому значенні слова. Надання капіталу в
позику залежить виключно від волі банку або інших володільців капіталу, —
бо позичальники завжди знайдуться; але сума засобів циркуляції залежить від
потреб усього суспільства, в межах яких гроші циркулюють для цілей щоденного витрачання“ (\emph{J.~G.~Kinnear}: „The Crisis and the Currency“. London 1847
[стор. 3 і далі]).}, Вільсон та інші, при чому вони
без ладу переплутують ріжниці між засобами циркуляції, як грішми, грошовим капіталом взагалі і
капіталом, що дає процент (moneyed
capital в англійському значенні), зводиться до таких двох пунктів.

Засоби циркуляції циркулюють, з одного боку, як \emph{монета}
(гроші), оскільки вони опосереднюють \emph{витрачання доходів}, отже,
обмін між індивідуальними споживачами і роздрібними торговцями, — до цієї категорії слід зарахувати
всіх купців, що продають споживачам, індивідуальним споживачам, у відміну від
продуктивних споживачів або виробників. Тут гроші циркулюють
\index{iii1}{0419}  %% посилання на сторінку оригінального видання
у функції монети, хоч вони постійно \emph{заміщають капітал}.
Певна частина грошей у країні завжди присвячена цій функції,
хоч ця частина складається з окремих монет, які постійно міняються. Навпаки, оскільки гроші
опосереднюють \emph{передачу капіталу}, чи то як засіб купівлі (засіб циркуляції), чи як засіб
платежу, вони є \emph{капітал}. Отже, не функція грошей як засобу
купівлі і не функція їх як засобу платежу відрізняє їх від
монети, бо гроші можуть функціонувати як засіб купівлі навіть
між торговцем і торговцем, оскільки вони купують один у одного
за готівку, і вони можуть також функціонувати як засіб платежу
між торговцем і споживачем, оскільки дається кредит, і дохід спочатку споживається, а потім
сплачується. Отже, ріжниця полягає в тому, що в другому випадку ці гроші не тільки заміщають капітал
для однієї сторони, продавця, але й витрачаються,
авансуються як капітал другою стороною, покупцем. Отже,
в дійсності це відмінність \emph{грошової форми доходу} від \emph{грошової
форми капіталу}, а не відмінність засобів циркуляції від капіталу,
бо як посередник між торговцями, цілком так само як і посередник між споживачами і торговцями,
\emph{циркулює} певна за своєю
кількістю частина грошей, і в наслідок цього це в \emph{обох} функціях однаково \emph{циркуляція}. Але в погляди
Тука тут домішується
різного роду плутанина:

1) в наслідок змішання функціональних визначень;

2) в наслідок приплутання питання про кількість циркулюючих грошей, узятих разом в обох функціях;

3) в наслідок приплутання питання про відносні пропорції
кількостей засобів циркуляції, що циркулюють в обох функціях
і тому в обох сферах процесу репродукції.

До пункту 1) про змішання функціональних визначень грошей, тобто що гроші в одній формі є засіб
циркуляції (currency),
а в другій формі — капітал. Оскільки гроші служать в тій або
другій функції, чи то для реалізації доходу, чи для передачі
капіталу, вони функціонують у купівлі й продажу або в платежу,
як засіб купівлі або засіб платежу, і в дальшому значенні
слова як засіб циркуляції. Дальше визначення грошей, яке вони
мають в рахунку того, хто їх витрачає або одержує, — чи представляють вони для нього капітал чи
дохід, — тут абсолютно нічого не змінює, і це виявляється в двоякій формі. Хоч грошові
знаки, які циркулюють в обох сферах, є різні, проте, той самий
грошовий знак, наприклад, п’ятифунтова банкнота, переходить
з однієї сфери в другу і навпереміну виконує обидві функції; це
вже тому є неминуче, що роздрібний торговець може дати
своєму капіталові грошову форму тільки в формі монети, яку
він одержує від своїх покупців. Можна прийняти, що власне
розмінна монета має центр ваги своєї циркуляції в сфері
роздрібної торгівлі; роздрібному торговцеві вона постійно потрібна для розміну, і він постійно
одержує її назад у платежах
від своїх покупців. Але він одержує також гроші, тобто монету
\parbreak{}  %% абзац продовжується на наступній сторінці
