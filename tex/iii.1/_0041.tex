\parcont{}  %% абзац починається на попередній сторінці
\index{iii1}{0041}  %% посилання на сторінку оригінального видання
р'= 125/2500 = 5\%). При таких умовах наш купець, при річному
обороті в 15000, одержує надзиск в 750, отже за 2 2/3 року повертає собі весь свій додатковий капітал.

Але для того, щоб
прискорити свій збут і разом з тим свій оборот і цим здобути при тому ж капіталі той самий зиск за коротший час, отже, за
той самий час більший зиск, ніж раніше,— він подарує покупцеві невеличку частину своєї додаткової вартості, продаватиме
дешевше, ніж його конкуренти.
Ці останні ступнево теж перетворяться в роздатчиків, і тоді надзиск зведеться для них усіх до звичайного зиску, або навіть і
до нижчого, при. підвищеному в усіх них капіталі. Рівність норми зиску знову відновлена, хоч може і на - іншому рівні, в
наслідок того, що частина виробленої всередині країни додаткової вартості відступається чужоземним покупцям.

Дальший крок в
підпорядкуванні промисловості капіталові настає в наслідок введення мануфактури. Ця остання теж дає змогу мануфактуристові,
який в XVII і XVIII століттях — в Німеччині майже всюди до 1850 року, а місцями ще й досі — здебільшого сам був експортером
своїх товарів, виробляти товари дешевше, ніж його «старомодний конкурент, ремісник. Той самий процес повторюється;
привласнювана мануфактурним капіталістом додаткова вартість дозволяє йому, відповідно купцеві-експортерові, з яким він її
ділить, продавати дешевше, ніж його конкуренти, поки новий спосіб виробництва не стане загальним і не наступить знову
вир}внёння. Наявна вже торговельна норма зиску, навіть якщо вона вирівнена тільки в місцевому масштабі, лишається
прокрустовим ложем, на якому без усякого милосердя відрубуєіься надлишкова промислова додаткова вартість.

Коли вже мануфактура бурхливо розвинулася завдяки здешевленню її виробів, то ще в далеко більшій мірі це Стосується великої
промисловості, яка шляхом все нових і нових революцій у методах виробництва дедалі більше знижує витрати виготовлення
товарів і невблаганно усуває всі попередні способи виробництва. Саме вона таким шляхом остаточно завойовує внутрішній ринок
для капіталу, кладе кінець дрібному виробництву і натуральному господарству самодовліючої селянської сім’ї, усуває
безпосередній обмін між дрібними виробниками і ставить усю націю на службу капіталові. Вона також вирівнює норми зиску
різних галузей торгівлі і промисловості в одну загальну норму зиску і забезпечує, нарешті, промисловості належну їй панівну
роль при цьому вирівнеині, усуваючи більшу частину перешкод, які доти стояли на шляху перенесення капіталу з однієї галузі в
іншу. Разом з цим для сукупного обміну в цілому відбувається перетворення вартостей в ціни виробництва. Отже, це
перетворення відбувається в силу об’єктивних законів, поза свідомістю або наміром учасників. Те, що конкуренція зводить до
загального рівня зиск, який перевищує
\parbreak{}  %% абзац продовжується на наступній сторінці
