
\index{iii1}{0344}  %% посилання на сторінку оригінального видання
Припустімо спочатку, що існує постійне відношення між усім
зиском і тією його частиною, яка повинна бути сплачена як процент
грошовому капіталістові. В такому разі очевидно, що процент
підвищуватиметься або падатиме разом з усім зиском, а цей
останній визначається загальною нормою зиску та її коливаннями.
Коли б, наприклад, пересічна норма зиску була \deq{} 20\%, а процент
\deq{} \sfrac{1}{4} зиску, то розмір процента \deq{} 5\%; коли б пересічна
норма зиску була \deq{} 16\%, то процент \deq{} 4\%. При нормі зиску в
20\% процент міг би підвищитись до 8\% і промисловий капіталіст
все ж одержував би той самий зиск, що й при нормі зиску
\deq{} 16\% і розмірі процента \deq{} 4\%, а саме 12\%. Коли б процент
підвищився тільки до 6 або 7\%, то він все ще залишав би собі
більшу частину зиску. Коли б процент дорівнював якійсь постійній
частині пересічного зиску, то з цього випливало б, що
чим вища загальна норма зиску, тим більша абсолютна ріжниця
між усім зиском і процентом, тим більша, отже, та частина
всього зиску, яка дістається функціонуючому капіталістові, і навпаки.
Припустім, що процент \deq{} \sfrac{1}{5} пересічного зиску. \sfrac{1}{5} від 10
є 2; ріжниця між усім зиском і процентом \deq{} 8. \sfrac{1}{5} від 20 \deq{} 4;
ріжниця $= 20 - 4 \deq{} 16$; \sfrac{1}{5} від 25 \deq{} 5; ріжниця $= 25 - 5 \deq{} 20$; \sfrac{1}{5} від
30 \deq{} 6; ріжниця $= 30 - 6 \deq{} 24$; \sfrac{1}{5} від 35 \deq{} 7; ріжниця $= 35 - 7 \deq{} 28$.
Різні норми процента в 4, 5, 6, 7\% тут весь час виражали б
тільки \sfrac{1}{5}, або 20\%, всього зиску. Отже, якщо норми зиску є різні,
то різні норми процента можуть виражати ту саму відповідну
частину всього зиску або ту саму процентну частину всього
зиску. При такому постійному відношенні процента промисловий
зиск (ріжниця між усім зиском і процентом) був би тим більший,
чим вища загальна норма зиску, і навпаки.

При інших однакових умовах, тобто припускаючи відношення
між процентом і всім зиском за більш-менш постійне, функціонуючий
капіталіст буде спроможний і згодиться платити вищий
або нижчий процент у прямому відношенні до висоти норми
зиску\footnote{
„The natural rate of interest із governed by the profits of trade to particulars“
(„Природна норма процента реґулюється зиском окремих підприємств“]
(\emph{Massie}, там же, стор. 51).
}. Ми вже бачили, що висота норми зиску стоїть у зворотному
відношенні до розвитку капіталістичного виробництва;
звідси випливає, що вищий чи нижчий розмір процента в даній
країні стоїть у такому самому зворотному відношенні до висоти
промислового розвитку, якщо тільки ріжниця в розмірі
процента дійсно виражає ріжницю норм зиску. Пізніше ми побачимо,
що немає ніякої необхідності в тому, щоб це завжди
було так. В цьому розумінні можна сказати, що процент реґулюється
зиском, точніше, загальною нормою зиску. І цей спосіб
його реґулювання поширюється навіть на його пересічний розмір.

В усякому разі, пересічну норму зиску слід розглядати як
остаточно визначальну максимальну межу процента.
