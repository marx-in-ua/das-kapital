
\index{iii1}{0164}  %% посилання на сторінку оригінального видання
Якщо ми капітали І — V знову розглядатимемо як єдиний сукупний капітал, то побачимо, що і в цьому
випадку склад суми
п’яти капіталів = 500 = 390 c + 110 v, отже, пересічний склад, = 78 c + 22 v, лишається той самий;
так само й пересічна додаткова вартість = 22\%. Розподіливши цю додаткову вартість рівномірно між
І—V, ми одержали б такі товарні ціни:

Капітали
Додаткова вартість
Вартість товарів
Витрати виробництва
Ціна товарів
Норма зиску
Відхилення ціни від вартості

І. 80 c + 20 v    20    90    70    92    22\%    + 2
II. 70 c + 30 v   30   111   81   103   22\%  — 8
III. 60 c + 40 v  40   131   91   113   22\% — 18
IV. 85 c + 15 v   15    70    55    77    22\%    + 7
V. 95 c + 5 v        5     20    15    37    22\%  + 17

В загальній сумі товари продаються на 2 + 7 + 17 = 26 вище і
на 8 + 18 = 26 нижче вартості, так що відхилення цін взаємно
знищуються в наслідок рівномірного розподілу додаткової вартості, тобто в наслідок додання
пересічного зиску в 22 на
100 одиниць авансованого капіталу до відповідних витрат виробництва товарів І—V; в тому самому
відношенні, в якому одна
частина товарів продається вище, друга продається нижче її
вартості. І тільки продаж їх по таких цінах уможливлює те, що
норма зиску для І—V є однакова, 22\%, не зважаючи на різний
органічний склад капіталів І—V. Ціни, які виникають таким чином, що з різних норм зиску різних сфер
виробництва береться
пересічна і ця пересічна додається до витрат виробництва в різних сферах виробництва, — такі ціни є
ціни виробництва. Передумовою їх є існування однієї загальної норми зиску, а ця
остання знов таки передбачає, що норми зиску в кожній окремій сфері виробництва, взяті самі по собі,
вже зведені до
відповідної кількості пересічних норм. Ці окремі норми зиску в кожній сфері виробництва = m/K, і їх
треба виводити з вартості товару, як це і було зроблено в першому відділі цієї книги. Без такого
виведення загальна норма зиску (а тому й ціна виробництва товару) була б безглуздим і ірраціональним
уявленням. Отже, ціна виробництва товару дорівнює витратам його
виробництва плюс доданий до них зиск, обчислений у процентах
відповідно до загальної норми зиску, тобто дорівнює витратам
виробництва товару плюс пересічний зиск.

В наслідок різного органічного складу капіталів, вкладених
у різні галузі виробництва, а тому в наслідок тієї обставини,
що — залежно від різного процентного відношення змінної частини до всього капіталу даної величини —
рівновеликими капіталами приводяться в рух дуже різні кількості праці, ними привласнюються також
дуже різні кількості додаткової праці, або
виробляються дуже різні маси додаткової вартості. Відповідно
до цього норми зиску, які панують в різних галузях виробництва,
\index{iii1}{0165}  %% посилання на сторінку оригінального видання
первісно є дуже різні. Ці різні норми зиску за допомогою конкуренції вирівнюються в загальну
норму зиску, яка
є пересічною всіх цих різних норм зиску. Зиск, який відповідно
до цієї загальної норми зиску припадає на капітал даної величини, який би не був його органічний
склад, зветься пересічним
зиском. Ціна товару, яка дорівнює витратам його виробництва
плюс та частина річного пересічного зиску на застосований для
виробництва товару (а не тільки на спожитий для його виробництва) капітал, яка припадає на товар
залежно від умов його
обороту, є його ціна виробництва. Візьмімо, наприклад, капітал
в 500, в тому числі 100 основного капіталу, з якого зношується
10\% протягом одного періоду обороту обігового капіталу в 400.
Припустімо, що пересічний зиск протягом цього періоду обороту становить 10\%. Тоді витрати
виробництва виготовленого
протягом цього обороту продукту будуть: 10 с на зношування
плюс 400 (c + v) обігового капіталу = 410, а його ціна виробництва: 410 витрати виробництва плюс
(10\% зиску на 500) 50 = 460.

Тому, хоч капіталісти різних сфер виробництва при продажу
своїх товарів повертають собі капітальні вартості, спожиті на
виробництво цих товарів, але реалізують вони не ту додаткову
вартість, отже, і не той зиск, що виробляється в їх власній
сфері при виробництві цих товарів, а лише стільки додаткової вартості, отже й зиску, скільки при
рівному розподілі
припадає на кожну відповідну частину всього капіталу суспільства з усієї додаткової вартості або
всього зиску, який виробляється сукупним капіталом суспільства за даний період часу
в усіх сферах виробництва, взятих разом. Кожен авансований
капітал, який би не був його склад, одержує кожного року або
за якийсь інший період часу стільки зиску на кожні 100, скільки
його за цей період часу припадає на кожні 100 як певну частину
сукупного капіталу. Оскільки справа стосується зиску, різні капіталісти відносяться тут один до
одного, як прості акціонери
одного акційного товариства, в якому зиск розподіляється між
ними рівномірно на кожні 100 одиниць, і тому зиски для різних
капіталістів відрізняються тільки залежно від величини капіталу,
вкладеного кожним з них у спільне підприємство, залежно від
відносного розміру участі кожного в спільному підприємстві,
залежно від числа акцій кожного з них. Отже, тимчасом як та
частина цієї товарної ціни, яка заміщає спожиті на виробництво
товарів частини вартості капіталу і за яку, отже, знову мусять
бути куплені ці спожиті капітальні вартості, — тимчасом як ця
частина, яка становить витрати виробництва, цілком визначається
видатками в межах відповідних сфер виробництва, — друга складова частина товарної ціни, зиск,
доданий до цих витрат виробництва, визначається не масою зиску, виробленою цим певним
капіталом у цій певній сфері виробництва протягом даного
часу, а тією масою зиску, яка за даний період часу пересічно припадає на кожний застосований капітал
як певну частину
\index{iii1}{0166}  %% посилання на сторінку оригінального видання
сукупного суспільного капіталу, вкладеного в сукупне
виробництво.\footnote{
Cherbuliez [„Riche ou Pauvre“, Paris-Genève 1840, стор. 116 і далі].
}

Отже, якщо капіталіст продає свій товар по його ціні виробництва, то він повертає собі кількість
грошей, відповідну величині вартості спожитого ним у виробництві капіталу, і добуває зиск
пропорціонально до його авансованого капіталу, просто
як до певної частини сукупного суспільного капіталу. Витрати
виробництва в кожній сфері виробництва мають специфічний
характер. Доданий до цих витрат виробництва зиск не залежить від його окремої сфери виробництва, він
є проста пересічна
на кожні 100 авансованого капіталу.

Припустімо, що п’ять різних капіталів І—V у вищенаведеному прикладі належать одній людині. Кількість
змінного і сталого капіталу, спожита на виробництво товарів у кожному окремому підрозділі І—V на
кожні 100 застосованого капіталу,
є дана; ця частина вартості товарів І—V, само собою зрозуміло, становитиме частину їх ціни, бо
принаймні ця ціна потрібна для заміщення авансованої і спожитої частини капіталу. Отже, ці витрати
виробництва були б різні для кожного роду
товарів І—V і, як такі, вони були б по-різному фіксовані їх власником. Що ж до різних мас додаткової
вартості або зиску, вироблених у підрозділах І—V, то капіталіст мав би всі підстави вважати їх за
зиск на весь свій авансований капітал, так що
на кожні 100 одиниць капіталу припадала б певна відповідна
частина. Отже, витрати виробництва товарів, вироблених в окремих підрозділах І—V, були б різні; але
в усіх цих товарів
була б однаковою частина продажної ціни, яка походить з доданого до витрат виробництва зиску на
кожні 100 одиниць капіталу. Отже, сукупна ціна товарів І—V дорівнювала б їх сукупній вартості, тобто
дорівнювала б сумі витрат виробництва
І—V плюс сума додаткової вартості, або зиску, вироблена в
І—V; отже, в дійсності ця ціна була б грошовим виразом сукупної кількості минулої і новододаної
праці, вміщеної в товарах
І—V. І таким чином, у самому суспільстві — якщо розглядати
всі галузі виробництва в їх сукупності — сума цін виробництва
вироблених товарів дорівнює сумі їх вартостей.

Цьому твердженню, здається, суперечить той факт, що в
капіталістичному виробництві елементи продуктивного капіталу
звичайно купуються на ринку, отже, ціни їх містять у собі вже
реалізований зиск, тобто ціну виробництва певної галузі промисловості разом з уміщеним в ній зиском,
так що зиск однієї
галузі промисловості входить у витрати виробництва іншої.
Але якщо ми підрахуємо на одному боці суму витрат виробництва товарів цілої країни, а на другому —
суму її зиску або додаткової вартості, то, очевидно, матимемо правильний обрахунок. Візьмімо,
наприклад, товар А; нехай витрати його
\parbreak{}  %% абзац продовжується на наступній сторінці
