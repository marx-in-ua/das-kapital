
\index{iii1}{0339}  %% посилання на сторінку оригінального видання
Покупець звичайного товару купує споживну вартість цього
товару, а сплачує він його вартість. Позичальник грошей так само
купує їх споживну вартість як капіталу; але що він сплачує?
Звичайно, не їх ціну або вартість, як це має місце при купівлі
інших товарів. Між позикодавцем і позичальником не відбувається,
як це має місце між продавцем і покупцем, обміну форми
вартості, при чому та вартість, яка одного разу існує у формі грошей,
другого разу існує у формі товару. Тотожність віддаваної і
одержуваної назад вартості виявляється тут цілком інакшим способом.
Сума вартості, гроші, віддаються без еквіваленту і через
певний час повертаються назад. Позикодавець весь час
лишається власником тієї самої вартості, навіть після того, як вона
перейшла з його рук у руки позичальника. При простому товарному
обміні гроші завжди є на боці покупця; але при позиці гроші
перебувають на боці продавця. Це він віддає гроші на певний
час, а покупець капіталу одержує їх як товар. Але це можливе
лиш остільки, оскільки гроші функціонують як капітал і тому
авансуються. Позичальник бере гроші в позику як капітал, як
вартість, що самозростає. Але спочатку це тільки капітал у
собі, як і всякий капітал у його вихідній точці, в момент його
авансування. Тільки за допомогою споживання їх вони збільшуються
в своїй вартості, реалізуються як капітал. Але позичальник
повинен зворотно сплатити їх як \emph{реалізований} капітал,
отже, як вартість плюс додаткова вартість (процент); а цей
останній може бути тільки частиною реалізованого ним зиску.
Тільки частиною, a не всім зиском. Бо для позичальника споживна
вартість грошей полягає в тому, що вони виробляють
йому зиск. Інакше вийшло б, що з боку позикодавця не відбулося
ніякого відчуження споживної вартості. З другого боку,
весь зиск не може дістатися позичальникові. Інакше вийшло б,
що він нічого не заплатив за відчуження споживної вартості і
повертає позикодавцеві авансовані гроші тільки як прості гроші,
а не як капітал, не як реалізований капітал, бо реалізованим
капіталом вони є тільки як $Г \dplus{} ΔГ$.

Обидва, і позикодавець і позичальник, витрачають ту саму
грошову суму як капітал. Але тільки в руках позичальника вона
функціонує як капітал. Зиск не подвоюється від подвійного
буття однієї й тієї самої грошової суми як капіталу для двох
осіб. Вона може функціонувати як капітал для обох тільки в
наслідок поділу зиску. Та частина, яка дістається позикодавцеві,
зветься процентом.

Припускається, що вся угода відбувається між двома видами
капіталістів, між грошовим капіталістом і промисловим або торговельним
капіталістом.

Ніколи не слід забувати, що тут капітал як капітал є товар,
або що товар, про який тут іде мова, є капітал. Тому всі відносини,
які тут виявляються, були б ірраціональні з точки зору
простого товару, або також з точки зору капіталу, оскільки він
\parbreak{}  %% абзац продовжується на наступній сторінці
