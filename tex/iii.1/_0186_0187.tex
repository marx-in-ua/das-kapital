\parcont{}  %% абзац починається на попередній сторінці
\index{iii1}{0186}  %% посилання на сторінку оригінального видання
співвідношенням різних частин, на які розпадається додаткова
вартість (зиск, процент, земельна рента, податки і~\abbr{т. д.}); таким
чином тут знову виявляється, що відношенням попиту й подання
абсолютно нічого не можна пояснити, поки не розкрита та база,
на якій розвивається (spielt) це відношення.

Хоч товар і гроші, і те і друге, є єдність мінової вартості
і споживної вартості, проте, ми вже бачили (книга I, розд. 1, 3),
як в купівлі й продажу обидва ці визначення полярно розміщаються на двох крайніх пунктах, так що
товар (продавець)
репрезентує споживну вартість, а гроші (покупець) — мінову вартість. Перша передумова продажу
полягала в тому, що товар
повинен мати споживну вартість, отже, задовольняти суспільну
потребу. Друга передумова полягала в тому, що кількість
праці, вміщена в товарі, повинна репрезентувати суспільно-необхідну працю, отже, індивідуальна
вартість (і — що при цьому
припущенні є те саме — продажна ціна) товару повинна збігатися з його суспільною вартістю.\footnote{
\emph{K. Marx}: „Zur Kritik der politischen Oekonomie“, Berlin 1859. [\emph{К. Маркс}:
„До критики політичної економії“, укр. вид. 1935~\abbr{р.}, стор. 54.]
}

Застосуймо це до наявної на ринку маси товарів, яка становить продукт цілої сфери виробництва.

Справу можна з’ясувати найлегше, якщо ми всю масу товарів — спочатку, отже, \emph{однієї} галузі
виробництва — розглядатимемо
як \emph{один} товар, а суму цін багатьох тотожних товарів як \emph{одну}
сумарну ціну. В такому випадку те, що було сказано про окремий товар, буквально стосується до маси
товарів певної галузі
виробництва, яка перебуває на ринку. Відповідність індивідуальної вартості товару його суспільній
вартості здійснюється тепер
або набуває дальшого визначення в тому розумінні, що сукупна
кількість товару містить у собі працю, суспільно-необхідну для
її виробництва, і що вартість цієї маси товарів = її ринковій
вартості.

Припустімо тепер, що значна маса цих товарів вироблена
при однакових, приблизно нормальних суспільних умовах, так
що ця вартість є разом з тим індивідуальна вартість окремих
товарів, які становлять цю масу. Якщо одна порівняно незначна
частина товарів вироблена при умовах гірших, а друга — при умовах
кращих, ніж нормальні, так що індивідуальна вартість першої
частини більша, а другої менша, ніж середня вартість більшої частини цих товарів, причому обидві ці
крайності урівноважуються,
так що пересічна вартість належних до них товарів дорівнює
вартості товарів, належних до середньої маси, — то ринкова вартість визначається вартістю товарів,
вироблених при середніх
умовах.\footnote{
Там же.
} Вартість сукупної товарної маси дорівнює дійсній сумі
вартостей всіх окремих товарів, узятих разом — як тих, що вироблені при середніх умовах, так і тих,
що вироблені при умовах
\index{iii1}{0187}  %% посилання на сторінку оригінального видання
гірших чи кращих, ніж середні. В цьому випадку ринкова
вартість або суспільна вартість маси товарів — необхідний робочий час, що міститься в них —
визначається вартістю переважної середньої маси товарів.

Припустімо, навпаки, що сукупна кількість даного товару,
поданого на ринок, лишається та сама, але вартість товарів,
вироблених при гірших умовах, не урівноважується вартістю
товарів, вироблених при кращих умовах, при чому частина всієї
маси товарів, вироблена при гірших умовах, становить відносно
значну величину як у порівнянні з середньою масою товарів,
так і в порівнянні з другою крайністю; тоді ринкову вартість
або суспільну вартість регулює маса товарів, вироблена при
гірших умовах.

Припустімо, нарешті, що кількість товарів, вироблених при
умовах кращих, ніж середні, значно переважає масу товарів, вироблених при гірших умовах, і становить
навіть значну величину порівняно з масою товарів, вироблених при середніх умовах; тоді ринкову
вартість регулює частина товарів, вироблена при
кращих умовах. Ми тут залишаємо осторонь переповнення ринку,
коли ринкову ціну завжди регулює частина товарів, вироблена
при кращих умовах; тут ми маємо справу не з ринковою ціною,
оскільки вона відрізняється від ринкової вартості, а з різними
визначеннями самої ринкової вартості.\footnote{
Отже, спір між Шторхом і Рікардо в питанні земельної ренти (спір
тільки щодо розуміння питання: фактично вони не вважали один на одного)
про те, чи регулюється ринкова вартість (у них скоріше ринкова ціна або
ціна виробництва) товарами, виробленими при найнесприятливіших умовах
(Рікардо [там же, стор. 38 і далі]), чи товарами, виробленими при найсприятливіших умовах (Шторх.
[„Cours d’économie politique“. Петербург 1815, т. II,
стор. 78 і далі]), розв’язується таким чином, що обидва вони мають рацію
і не мають рації і що обидва вони зовсім випустили з уваги середній випадок.
Порівняй у Корбета [„An Inquiry etc.“, стор. 42 і далі] про ті випадки, коли
ціна регулюється товарами, виробленими при найкращих умовах. — „It is not meant
to be asserted by him (Ricardo) that two particular lots of two different articles,
as a hat and a pair of schoes, exange with one another when those two particular
lots were produced by equal quantities of labour. By „commodity“ we must here
understand the „description of commodity“ not a particular individual hat, pair
of shoes etc. The whole labour which produces all the hats in England is to be
considered, for this purpose, as divided among all the hats. This seems to me not
to have been expressed at first, and in the general statements of this doctrine“. [„He
слід думати, ніби він (Рікардо) твердить, що дві певні партії двох різних товарів, як,
наприклад, один капелюх і пара черевиків, обмінюються одна на одну, якщо
кожна з цих двох певних партій товарів створена однаковою кількістю праці.
Під „товаром“ ми повинні тут розуміти певний „рід товару“, а не один окремий
індивідуальний капелюх, одну пару черевиків і~\abbr{т. д.} Для цього вся праця, що виробляє всі капелюхи в
Англії, повинна розглядатись як поділена між усіма цими капелюхами. Мені здається, що первісно, а
також при звичайних викладах цієї
доктрини це не було висловлено“]. („Observations on certain verbal disputes in
Political Economy etc.“, London 1821, стор. 53, 54).
}

Дійсно, розглядаючи справу з усією строгістю (що, звичайно,
фактично здійснюється тільки приблизно і з незчисленними модифікаціями), у випадку I ринкова
вартість всієї маси товарів,
\parbreak{}  %% абзац продовжується на наступній сторінці
