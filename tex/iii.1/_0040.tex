\parcont{}  %% абзац починається на попередній сторінці
\index{iii1}{0040}  %% посилання на сторінку оригінального видання
капіталіста, і створили конкуренцію товарам того самого роду, виготовленим коштом ремісника?

Норма зиску торговельного
капіталу вже була в наявності. Вона також була вже, принаймні для кожної даної місцевості, приблизно вирівнена до пересічної
норми. Що ж могло спонукати купця взяти на себе додаткову роль роздатчика? Тільки одно:  перспектива більшого зиску при
однаковій з іншими продажній ціні. І таку перспективу він мав. Заставляючи дрібних майстрів працювати на себе, він ламав
традиційні межі товарного виробництва, при якому виробник продавав свій готовий продукт і нічого іншого. Торговельний
капіталіст купував робочу силу, яка Поки ще володіла своїми засобами виробництва, але вже не володіла сировинним матеріалом.
Забезпечуючи таким чином ткачеві регулярне заняття, він міг зате так знижувати його заробітну плату, що частина витраченого
ним робочого часу лишалась неоплаченою. Таким чином роздатчик ставав привласнювачем додаткової вартості понад торговельний
бариш, який він одержував досі. Правда, він мусив для цього застосовувати додатковий капітал, щоб купити пряжу і т. д. і
залишити її в руках ткача, поки цей останній виготовить продукт, тимчасом як раніше він мав платити повну ціну продукту
тільки при його купівлі. Але, поперше, в більшості випадків він і раніш уже витрачав додатковий капітал на аванси ткачам,
яких, як правило, тільки боргове рабство приводило до того, що вони підкорялись’
новим умовам виробництва. І, подруге, навіть незалежно від цього обрахунок складається за такою схемою:
Припустім, що наш купець провадить свої експортні операції з капіталом в 30000 дукатів, цехінів, фунтів стерлінгів чи
будьчого іншого. З них нехай 10000 витрачено на купівлю тубільних товарів, тимчасом як 20000 застосовуються на заморських
ринках збуту. Капітал обертається один раз за два роки, річний оборот = 15000. Припустімо тепер, що наш купець хоче змусити
ткачів ткати за його власний кошт, хоче стати роздатчиком. Скільки мусить він застосувати для цього додаткового капіталу?

Припустім, що час виробництва куска тканини, якою він торгує, пересічно дорівнює двом місяцям, — час напевно занадто
великий. Припустімо, далі, що він за все мусить платити готівкою. В такому разі він мусить додатково застосувати такий
капітал, якого вистачило б на те, щоб постачити ткачам пряжі на два місяці. Через те що річний оборот його є 15000, то за
два місяці він купує тканин на 2500. Якщо ми припустимо, що з них 2000 представляють вартість пряжі, а 500 заробітну плату
ткачів, то нашому купцеві потрібний буде додатковий капітал в 2000. Ми припускаємо, що додаткова вартість, яку він
привласнює собі від ткача за допомогою нового методу, становить тільки 5\% вартості тканини, що, безперечно, є дуже скромна
норма додаткової вартості в 25\% (2000 с + 500 v + 125m; m'= 125/500 = 25\%,
\parbreak{}  %% абзац продовжується на наступній сторінці
