\parcont{}  %% абзац починається на попередній сторінці
\index{iii1}{0021}  %% посилання на сторінку оригінального видання
не визнавати комічності становища. До цього я ще додаю тільки
ось що: з тією самою сміливістю, з якою він уже тоді міг сказати,
що „Маркс у третьому томі безсумнівно довів“, він користується
нагодою, щоб розповісти — мабуть, професорську —
плітку, ніби вищезгаданий твір Конрада Шмідта „безпосередньо
інспірований Енгельсом“. Пане Юліус Вольф! В тому світі, в якому
ви живете і дієте, може й водиться таке, що людина, яка публічно
ставить перед іншими проблему, нишком відкриває своїм
особистим друзям її розв’язання. Що ви на це здатні, я вам охоче
вірю. Що в тому світі, в якому я обертаюся, немає потреби
опускатися до такої мерзоти, це доводить вам оця передмова. —

Ледве помер Маркс, як пан \emph{Ахілл Лоріа} спішно опублікував
статтю про нього в „Nuova Antologia“ (квітень 1883): спочатку
біографія, переповнена брехливими даними, потім критика
громадської, політичної і літературної діяльності. Матеріалістичне
розуміння історії Маркса тут сфальсифіковане і перекручене
з таким апломбом, який дозволяє угадати якусь велику
мету. І ця мета була досягнута: в 1886 році той самий пан
Лоріа опублікував книгу: „La teoria economica della constituzione
politica“, в якій він оповістив здивованому світові його сучасників,
як своє власне відкриття, історичну теорію Маркса, так
ґрунтовно і так навмисно перекручену ним в 1883 році. Звичайно,
теорію Маркса він звів тут до досить філістерського
рівня; історичні ілюстрації й приклади теж рясніють такими помилками,
яких не простили б і учневі четвертого класу; але
хіба це все має якесь значення? Відкриття, що політичні становища
і події скрізь і завжди знаходять своє пояснення у відповідних
економічних становищах, зроблене, як доведено цією книгою
Лоріа, аж ніяк не Марксом у 1845 році, а паном Лоріа
в 1886 році. Принаймні він щасливо упевнив у цьому своїх земляків,
а з того часу, як його книга з’явилась французькою мовою,
і деяких французів, і може тепер чванитись в Італії як
автор нової епохальної історичної теорії, поки тамошні соціалісти
знайдуть час повискубувати в illustre [славетного] Лоріа
крадені павині пера.

Але це тільки один маленький зразочок маніри пана Лоріа.
Він запевняє нас, що всі теорії Маркса ґрунтуються на \emph{свідомому}
софізмі (un consaputo sofisma); що Маркс не відступав
перед паралогізмами навіть тоді, коли він \emph{визнавав} їх з\emph{а такі}
(sapendoli tali) і~\abbr{т. д.} І після того, як він в цілому ряді подібних
підлих побрехеньок дав своїм читачам усе потрібне для
того, щоб вони побачили в Марксі якогось кар’єриста à la Лоріа,
який досягає своїх дрібних ефектів за допомогою таких самих
дрібних негідних шахрайських засобів, як наш падуанський професор,
— він може тепер відкрити їм важливу таємницю, а тим
самим і нас приводить назад до норми зиску.

Пан Лоріа каже: За Марксом маса додаткової вартості (яку
пан Лоріа ототожнює тут з зиском), вироблена в капіталістичному
\index{iii1}{0022}  %% посилання на сторінку оригінального видання
промисловому підприємстві, повинна відповідати застосованому
в ньому змінному капіталові, бо сталий капітал не дає
ніякого зиску. Але це суперечить дійсності. Бо на практиці зиск
відповідає не змінному, а всьому капіталові. І Маркс сам бачить
це (І, розд. XI) і визнає, що зовнішньо факти суперечать
його теорії. Але як він розв’язує цю суперечність? Він відсилає
своїх читачів до подальшого тома, який ще не з’явився.
Про цей том Лоріа вже раніш сказав \emph{своїм} читачам, що він
не вірить тому, що Маркс хоч би одну мить думав про те,
щоб його написати, і тепер він тріумфуючи вигукує: „Отже,
я мав рацію, коли твердив, що цей другий том, яким Маркс
весь час загрожує своїм супротивникам і який, однак, ніколи
не з’явиться, що цей том, дуже ймовірно, був хитромудрою
виверткою, якої Маркс уживав тоді, коли в нього не
вистачало наукових аргументів (un ingegnoso spediente ideato
dal Marx a sostituzione degli argomenti scientifici)“. І хто тепер
не переконався, що Маркс стоїть на такій самій висоті наукового
шахрайства, як l'illustre Лоріа, той уже цілком безнадійна
людина.

\disablefootnotebreak{}
Отже, ми ось чого навчились: за паном Лоріа теорія додаткової
вартості Маркса абсолютно несполучна з фактом загальної
рівної норми зиску. Аж ось з’явилась друга книга і разом
з тим моє публічно поставлене питання саме про цей пункт.
Коли б пан Лоріа був одним з нас, соромливих німців, він би
якось збентежився. Але він — сміливий житель півдня, він походить
з гарячого клімату, де, як він може твердити, нахабність
(Unverfrorenheit)\footnote*{Тут гра слів: німецьке „Unverfrorenheit“ буквально можна також тлумачити
як „здатність не замерзати“. \Red{Ред. укр. перекладу.}} є до певної міри природна умова. Питання
про норму зиску поставлено публічно. Пан Лоріа публічно оголосив
його нерозв’язним. І саме через це він тепер перевищить
самого себе, розв’язавши його публічно.
\enablefootnotebreak{}

Це чудо сталося в „Conrads Jahrbücher“, Neue Folge, т. XX,
стор. 272 і далі, у статті про вищезгаданий твір Конрада Шмідта.
Після того, як він вичитав у Шмідта, яким чином утворюється
торговельний зиск, йому зразу все стало ясно. „Через те що
визначення вартості робочим часом дає перевагу тим капіталістам,
які вкладають більшу частину свого капіталу у заробітну
плату, то непродуктивний“ [слід сказати — торговельний] „капітал
може вимусити собі від цих капіталістів, що мають перевагу,
вищий процент“ [слід сказати — зиск] „і утворити рівність
між окремими промисловими капіталістами\dots{} Так, наприклад,
якщо промислові капіталісти $А$, $В$, $C$ застосовують у виробництві
кожний по 100 робочих днів і відповідно 0, 100, 200
сталого капіталу, а заробітна плата за 100 робочих днів містить
у собі 50 робочих днів, то кожний капіталіст одержує додаткову
вартість у 50 робочих днів, а норма зиску становить 100\%
\parbreak{}  %% абзац продовжується на наступній сторінці
