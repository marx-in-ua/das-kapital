
\index{iii1}{0220}  %% посилання на сторінку оригінального видання
Далі, тут слід тільки згадати, що при даному робітничому населенні,
якщо норма додаткової вартості зростає — чи то в наслідок
здовження або інтенсифікації робочого дня, чи в наслідок зниження
вартості заробітної плати в результаті розвитку продуктивної
сили праці, — маса додаткової вартості, а тому й абсолютна
маса зиску, мусить зрости, не зважаючи на відносне
зменшення змінного капіталу порівняно з сталим.

Той самий розвиток продуктивної сили суспільної праці, ті самі
закони, які виражаються у відносному зменшенні змінного капіталу
порівняно з усім капіталом і в прискореному разом з цим нагромадженні,
тоді як, з другого боку, нагромадження, впливаючи
в протилежному напрямі, стає вихідним пунктом дальшого розвитку
продуктивної сили і дальшого відносного зменшення змінного
капіталу, — цей самий розвиток, залишаючи осторонь тимчасові
коливання, виражається в дедалі дужчому збільшенні
всієї вживаної робочої сили, в дедалі більшому зростанні абсолютної
маси додаткової вартості, а тому й зиску.

В якій же формі мусить виражатися цей двоїстий закон породжуваного
одними й тими самими причинами зменшення \emph{норми}
зиску і одночасного збільшення абсолютної \emph{маси} зиску? Закон,
оснований на тому, що при даних умовах привласнювана маса
додаткової праці, отже й додаткової вартості, зростає, і що, коли
розглядати сукупний капітал або кожний окремий капітал тільки
як частину сукупного капіталу, зиск і додаткова вартість є тотожні
величини?

Візьмімо певну частину капіталу, на яку ми обчислюємо
норму зиску, наприклад, 100. Припустім, що ці 100 представляють
пересічний склад сукупного капіталу, скажімо, $80c \dplus{} 20v$.
В другому відділі цієї книги ми бачили, яким чином пересічна
норма зиску в різних галузях виробництва визначається не особливим
складом капіталу кожної з них, а його пересічним суспільним
складом. З відносним зменшенням змінної частини порівняно
з сталою, і, отже, порівняно з усім капіталом в 100, норма зиску
при незмінному і навіть зростаючому ступені експлуатації праці
падає, падає відносна величина додаткової вартості, тобто відношення
її до вартості всього авансованого капіталу в 100.
Але падає не тільки ця відносна величина. Величина додаткової
вартості або зиску, що його вбирає весь капітал в 100, падає
абсолютно. При нормі додаткової вартості в 100\% капітал в
$60c \dplus{} 40v$ виробляє масу додаткової вартості, а тому й зиску,
в 40; капітал в $70c \dplus{} 30v$ виробляє масу зиску в  30; при капіталі
в $80c \dplus{} 20v$ зиск падає до 20. Це падіння стосується до
маси додаткової вартості, а тому й зиску, і випливає з того,
що оскільки весь капітал в 100 приводить в рух менше живої
праці взагалі, а при незмінному ступені експлуатації
і менше додаткової праці, він виробляє менше додаткової вартості.
Коли якусь частину суспільного капіталу, отже, капіталу
пересічного суспільного складу, взяти за одиницю міри для
\parbreak{}  %% абзац продовжується на наступній сторінці
