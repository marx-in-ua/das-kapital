
\index{iii1}{0052}  %% посилання на сторінку оригінального видання
У цій формулі частина капіталу, витрачена на працю, відрізняється
від частини капіталу, витраченої на засоби виробництва,
наприклад, на бавовну або вугілля, тільки тим, що вона
служить для оплати речево відмінного елементу виробництва,
але ніяк не тим, що в процесі творення вартості товару, а тому
і в процесі зростання вартості капіталу вона відіграє функціонально
відмінну роль. У витратах виробництва товару ціна засобів
виробництва повертається назад такою, якою вона вже
фігурувала при авансуванні капіталу, і саме тому, що ці засоби
виробництва були доцільно використані. Цілком так само у витратах
виробництва товару ціна або заробітна плата за 666\sfrac{2}{3}
робочих днів, витрачених на його виробництво, повертається
назад такою, якою вона вже фігурувала при авансуванні капіталу,
і так само якраз тому, що ця маса праці витрачена в доцільній
формі. Ми бачимо тільки готові, наявні вартості, — ті
частини вартості авансованого капіталу, які входять в утворення
вартості продукту, — але не бачимо елементу, який ств'орює
нову вартість. Ріжниця між сталим і змінним капіталом зникла.
Всі витрати виробництва у 500\pound{ фунтів стерлінгів} набувають
тепер двоякого значення: поперше, вони є та складова частина
товарної вартості в 600\pound{ фунтів стерлінгів}, яка заміщає капітал
у 500\pound{ фунтів стерлінгів}, витрачений на виробництво товару;
і, подруге, сама ця складова частина вартості товару існує лише
тому, що вона раніш існувала як витрати виробництва застосованих
елементів виробництва, засобів виробництва і праці, тобто
як авансований капітал. Капітальна вартість повертається назад
як витрати виробництва товару тому і остільки, що і оскільки
її було витрачено як капітальну вартість.

Та обставина, що різні складові частини вартості авансованого
капіталу витрачені на речево різні елементи виробництва,
на засоби праці, сировинні й допоміжні матеріали і працю,
зумовлює тільки те, що на витрати виробництва товару доводиться
знову купити ці речево різні елементи виробництва.
Навпаки, щодо утворення самих витрат виробництва, то тут має
значення тільки одна ріжниця, ріжниця між основним і обіговим
капіталом. В нашому прикладі 20\pound{ фунтів стерлінгів} були зараховані
на зношування засобів праці ($400 c \deq{} 20$\pound{ фунтам стерлінгів}
на зношування засобів праці \dplus{} 380\pound{ фунтів стерлінгів} на
матеріали виробництва). Якщо вартість цих засобів праці перед
виробництвом товару була \deq{} 1200\pound{ фунтам стерлінгів}, то після
його виробництва вона існує в двох виглядах: 20\pound{ фунтів стерлінгів}
як частина товарної вартості, $1200-20$, або 1180\pound{ фунтів стерлінгів},
як решта вартості засобів праці, які перебувають, як і раніш,
у володінні капіталіста, або як елемент вартості не його
товарного капіталу, а його продуктивного капіталу. В протилежність
до засобів праці, матеріали виробництва і заробітна плата
цілком витрачаються на виробництво товару, а тому і вся їх
вартість входить у вартість виробленого товару. Ми бачили,
\parbreak{}  %% абзац продовжується на наступній сторінці
