\index{iii1}{0360}  %% посилання на сторінку оригінального видання
Тепер дуже просто виявляються причини, чому цей поділ
гуртового зиску на процент і підприємницький дохід, раз він
став якісним, зберігає цей характер якісного розподілу для сукупного
капіталу і для всього класу капіталістів.

\emph{Поперше}: це випливає вже з тієї простої емпіричної обставини,
що більшість промислових капіталістів, хоч і в різних числових
відношеннях, працює як власним, так і взятим у позику
капіталом, і що відношення між власним і взятим у позику капіталом
у різні періоди міняється.

\emph{Подруге}: перетворення однієї частини гуртового зиску в форму і
процента перетворює другу його частину в підприємницький
дохід. Справді, цей останній є тільки та протилежна форма, що
її набирає надлишок гуртового зиску понад процент, коли процент
існує як особлива категорія. Все дослідження того, яким
чином гуртовий зиск поділяється на процент і підприємницький
дохід, зводиться просто до дослідження того, яким чином частина
гуртового зиску взагалі закостеніває і усамостійнюється
як процент. Але капітал, що дає процент, історично існує як
готова, успадкована форма, а тому й процент як готова підформа
додаткової вартості, виробленої капіталом, існує задовго
до того, як з’являються капіталістичний спосіб виробництва і
відповідні йому уявлення про капітал і зиск. Тому грошовий капітал,
капітал, що дає процент, в народному уявленні все ще лишається
капіталом як таким, капіталом par excellence. Звідси, з
другого боку, і те уявлення, що панувало до часів Мессі, ніби процентом
оплачуються гроші як такі. Та обставина, що даний у
позику капітал дає процент незалежно від того, чи застосовується
він дійсно як капітал, чи ні, — дає процент навіть тоді, коли він береться
в позику тільки для споживання, — зміцнює уявлення про
самостійність цієї форми капіталу. Найкращий доказ тієї самостійності,
з якою, в перші періоди капіталістичного способу
виробництва, процент виступає відносно зиску і капітал, що дає
процент, відносно промислового капіталу, полягає в тому, що
тільки в середині XVIII століття був відкритий (Мессі і за ним
Юмом) той факт, що процент є просто частина гуртового зиску,
і що взагалі доводилось робити таке відкриття.

\emph{Потретє}: чи працює промисловий капіталіст власним чи
взятим у позику капіталом, це нічого не міняє в тій обставині,
що йому протистоїть клас грошових капіталістів як особливий,
вид капіталістів, грошовий капітал — як самостійний вид капіталу,
і процент — як відповідна цьому специфічному капіталові самостійна
форма додаткової вартості.

Розглядуваний \emph{щодо якості}, процент є додаткова вартість,
яку дає просто власність на капітал, яку дає капітал сам по собі,
хоча власник його лишається поза процесом репродукції, — яку,
отже, дає капітал відокремлено від свого процесу.

Розглядувана \emph{щодо кількості}, частина зиску, яка становить
процент, виступає так, ніби вона відноситься не до промислового
\index{iii1}{0361}  %% посилання на сторінку оригінального видання
і торговельного капіталу як такого, а до грошового
капіталу, і норма цієї частини додаткової вартості, норма процента
або розмір процента, закріплює це відношення. Бо, поперше,
розмір процента — не зважаючи на його залежність
від загальної норми зиску — визначається самостійно, і, подруге,
він, подібно до ринкової ціни товарів, у протилежність до невловимої
норми зиску, виступає при всякій переміні як стійке,
одноманітне, наочне і завжди дане відношення. Коли б весь
капітал був у руках промислових капіталістів, то не існувало
б ні процента, ні розміру процента. Самостійна форма,
якої набирає кількісний поділ гуртового зиску, породжує якісний
поділ. Якщо порівняти промислового капіталіста з грошовим
капіталістом, то першого відрізняє від другого тільки підприємницький
дохід, як надлишок гуртового зиску понад
пересічний процент, надлишок, який, завдяки розмірові процента,
виступає як емпірично дана величина. Якщо ж порівняти
його, з другого боку, з промисловим капіталістом, що
господарює власним, а не взятим у позику капіталом, то цей
останній відрізняється від нього тільки як грошовий капіталіст,
оскільки він сам кладе собі процент у кишеню, замість
сплачувати його. В обох випадках частина гуртового зиску, яка
відрізняється від процента, здається йому підприємницьким доходом,
а самий процент — додатковою вартістю, яку дає капітал
сам по собі і яку він через це давав би і без продуктивного
застосування.

Для окремого капіталіста це практично вірно. Від його вибору
залежить, чи віддати свій капітал — однаково, чи існує
він уже в своїй вихідній точці як грошовий капітал, чи його
ще тільки доводиться перетворити в грошовий капітал — у позику
як капітал, що дає процент, чи самому збільшувати його
вартість, застосовуючи його як продуктивний капітал. В загальному
ж розумінні, тобто в застосуванні до всього суспільного
капіталу, — як це роблять деякі вульгарні економісти, видаючи
це навіть за основу зиску, — це, звичайно, безглуздо. Припускати
можливість перетворення сукупного капіталу в грошовий капітал,
без наявності людей, що купують і збільшують вартість
засобів виробництва, у формі яких існує сукупний капітал, крім
відносно незначної частини його, існуючої у формі грошей, — це,
звичайно, безглуздя. Ще більше безглуздя гадати, що на основі
капіталістичного способу виробництва капітал давав би процент,
не функціонуючи як продуктивний капітал, тобто не створюючи
додаткової вартості, лише частину якої становить процент; що
капіталістичний спосіб виробництва міг би іти своїм шляхом без
капіталістичного виробництва. Коли б непомірно велика частина
капіталістів схотіла перетворити свій капітал у грошовий капітал,
то наслідком цього було б величезне знецінення грошового капіталу
і величезне падіння розміру процента; багато з них відразу
були б позбавлені можливості жити на свої проценти і таким
\parbreak{}  %% абзац продовжується на наступній сторінці
