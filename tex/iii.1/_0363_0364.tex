\parcont{}  %% абзац починається на попередній сторінці
\index{iii1}{0363}  %% посилання на сторінку оригінального видання
грошей і взагалі вартості в капітал є постійний результат капіталістичного
процесу виробництва, цілком так само буття їх
як капіталу є постійна передумова капіталістичного процесу
виробництва. Завдяки своїй здатності перетворюватись у засоби
виробництва, вони постійно командують над неоплаченою
працею і тому перетворюють процес виробництва й циркуляції
товарів у виробництво додаткової вартості для свого володільця.
Отже, процент є тільки вираз того, що вартість взагалі, —
упредметнена праця в її загальносуспільній формі, — вартість,
яка в дійсному процесі виробництва набирає вигляду засобів виробництва,
протистоїть живій робочій силі як самостійна сила
і є засобом привласнювати собі неоплачену працю; і що вона є такою
силою завдяки тому, що вона протистоїть робітникові як
чужа власність. Однак, з другого боку, в формі процента ця
протилежність найманій праці стерта; бо капітал, що дає процент,
як такий має своєю протилежністю не найману працю,
а функціонуючий капітал; капіталіст-позикодавець як такий прямо
протистоїть дійсно функціонуючому в процесі репродукції капіталістові,
а не найманому робітникові, у якого саме на основі
капіталістичного виробництва експропрійовано засоби виробництва.
Капітал, що дає процент — це капітал \emph{як власність} у
протилежність до капіталу \emph{як функції}. Але оскільки капітал
не функціонує, він не експлуатує робітників і не вступає в антагонізм
з працею.

З другого боку, підприємницький дохід становить протилежність
не до найманої праці, а тільки до процента.

\emph{Поперше}: якщо припустити пересічний зиск як дану величину,
то норма підприємницького доходу визначається не заробітною
платою, а розміром процента. Вона буде вища чи нижча
у зворотному відношенні до розміру процента\footnote{
„The profits of enterprise depend upon the net profits of capital, not the
latter upon the former“. [„Підприємницькі зиски залежать від чистих зисків капіталу,
а не останні від перших“). (\emph{Ramsay}: „An Essay on the'Distribution of Wealth“,
стор. 214. Net profits [чистий зиск] у Рамсея завжди \deq{} процентові.)
}.

\looseness=1
\emph{Подруге}: функціонуючий капіталіст виводить свою претензію
на підприємницький дохід, отже, і самий підприємницький дохід,
не з своєї власності на капітал, а з функції капіталу в протилежність
до тієї визначеності, в якій він існує тільки як бездіяльна
власність. Це виступає як безпосередньо наявна протилежність
у тих випадках, коли він оперує взятим у позику капіталом,
коли, отже, процент і підприємницький дохід дістаються
двом різним особам. Підприємницький дохід виникає з функції
капіталу в процесі репродукції, тобто в наслідок операцій, діяльності,
якою функціонуючий капіталіст опосереднює ці функції
промислового й торговельного капіталу. Але бути представником
функціонуючого капіталу — це не синекура, подібна до представництва
капіталу, що дає процент. На основі капіталістичного
виробництва капіталіст управляє процесом виробництва, як і процесом
\index{iii1}{0364}  %% посилання на сторінку оригінального видання
циркуляції. Експлуатація продуктивної праці коштує зусиль,
однаково, чи займається нею сам капіталіст, чи хто інший від його
імени. Отже, в протилежність до процента, його підприємницький
дохід здається йому незалежним від власності на капітал,
скоріше результатом його функцій як невласника, як — \emph{робітника}.

\looseness=1
Тому в його мозку необхідно виникає уявлення; що його підприємницький
дохід дуже далекий від того, щоб становити будь-яку
протилежність до найманої праці і бути лише неоплаченою чужою
працею, а, навпаки, сам є \emph{заробітною платою}, платою за нагляд,
wages of superintendence of labour, вищою платою, ніж плата звичайного
найманого робітника, 1)~тому що це складніша праця, 2)~тому
що він сам собі виплачує заробітну плату. Що його функція
як капіталіста полягає в тому, щоб добувати додаткову вартість,
тобто неоплачену працю, і до того ж при найекономніших умовах,
— це зовсім забувається в наслідок тієї суперечності, що процент
дістається капіталістові, хоч би він і не виконував ніякої
функції як капіталіст, а був би тільки власником капіталу, і що,
навпаки, підприємницький дохід дістається функціонуючому капіталістові,
хоч би він і не був власником капіталу, з яким він
функціонує. За антагоністичною формою обох частин, на які
розпадається зиск, тобто додаткова вартість, забувається, що
обидві вони є просто частини додаткової вартості і що її поділ
нічого не може змінити ні в її природі, ні в її походженні та
умовах її існування.

В процесі репродукції функціонуючий капіталіст виступає
відносно найманих робітників представником капіталу як чужої
власності, і грошовий капіталіст, будучи представлений функціонуючим
капіталістом, бере участь в експлуатації праці. Що
активний капіталіст може виконувати свою функцію, яка полягає
в тому, щоб заставляти робітників працювати на нього або
заставляти засоби виробництва функціонувати як капітал, — що
він може виконувати цю функцію тільки як представник засобів
виробництва у протилежність робітникам, це забувається
в наслідок протилежності між функцією капіталу в процесі
репродукції і простою власністю на капітал поза процесом
репродукції.

\looseness=1
Справді, в тій формі, яку обидві частини зиску, тобто додаткової
вартості, набирають як процент і підприємницький дохід,
не виражене ніяке відношення до праці, тому що це відношення
існує тільки між нею і зиском або, точніше, додатковою вартістю
як сумою, як цілим, як єдністю обох цих частин. Відношення, в
якому ділиться зиск, і різні юридичні титули, на основі яких відбувається
це ділення, передбачають зиск як готовий, передбачають
існування зиску. Тому, якщо капіталіст є власник того капіталу,
з яким він функціонує, то він кладе собі в кишеню весь зиск
або додаткову вартість; для робітника цілком байдуже, чи капіталіст
кладе собі в кишеню весь зиск, чи він повинен частину
сплачувати третій особі, як юридичному власникові. Таким
\parbreak{}  %% абзац продовжується на наступній сторінці
