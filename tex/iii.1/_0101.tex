\parcont{}  %% абзац починається на попередній сторінці
\index{iii1}{0101}  %% посилання на сторінку оригінального видання
є одним з методів підвищення норми зиску; цілком так само, —
це ми бачили вже раніш, — як надмірна праця, перетворення
робітника в робочу худобу є методом прискорення самозростання вартості капіталу, виробництва
додаткової вартості.
Ця економія приводить до переповнення тісних, нездорових приміщень робітниками, що капіталістичною
мовою зветься економією на будівлях; до скупчення небезпечних машин в
одному приміщенні і до недбання про охоронні засоби проти
небезпеки; до невживання захисних заходів у тих процесах,
які по своїй природі шкідливі для здоров’я або, як у копальнях, зв’язані з небезпекою, і~\abbr{т. д.} Ми
вже не кажемо про відсутність будь-яких установ для того, щоб зробити людським для
робітника процес виробництва, зробити його приємним або хоч би
тільки зносним. З капіталістичної точки зору це було б цілком
недоцільним і безглуздим марнотратством. Взагалі, капіталістичне
виробництво, не зважаючи на всю свою скнарість, надзвичайно
марнотратне щодо людського матеріалу, цілком так само, як
воно, з другого боку, завдяки методові розподілу своїх продуктів через торгівлю і своєму способові
конкуренції, дуже марнотратно поводиться з матеріальними засобами, і на одному боці втрачає для
суспільства те, що на другому боці виграє для
окремих капіталістів.

Подібно до того, як капітал має тенденцію при безпосередньому вживанні живої праці зводити її до
необхідної праці і постійно скорочувати працю, необхідну для виготовлення продукту,
експлуатуючи суспільні продуктивні сили праці, отже, якнайбільше економізувати безпосередньо вживану
живу працю, —
цілком так само він має тенденцію цю зведену до її необхідного
розміру працю вживати при найекономніших умовах, тобто зводити вартість застосовуваного сталого
капіталу до її якнайменшого мінімуму. Якщо вартість товарів визначається вміщеним
в них необхідним робочим часом, а не робочим часом, взагалі
вміщеним в них, то це визначення реалізується тільки капіталом,
який разом з тим постійно скорочує робочий час, суспільно-необхідний
для виробництва певного товару. Тим самим ціна
товару зводиться до її мінімуму, бо зводиться до мінімуму кожна
частина праці, потрібної для виробництва цього товару.

При розгляді економії в застосуванні сталого капіталу треба
розрізняти таке. Якщо зростає маса і разом з нею сума вартості застосовуваного капіталу, то це є
насамперед тільки концентрація більшої кількості капіталу в одних руках. Але саме
ця більша маса капіталу, застосовувана одним підприємцем, —
якій здебільшого відповідає й абсолютно більша, але відносно
менша кількість вживаної праці, — саме вона й допускає економію сталого капіталу. Коли взяти
окремого капіталіста, то
зростає розмір необхідних витрат капіталу, особливо основного
капіталу; але відносно маси перероблюваного матеріалу і експлуатованої праці вартість цих витрат
порівняно зменшується.
