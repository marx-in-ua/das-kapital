\parcont{}  %% абзац починається на попередній сторінці
\index{iii1}{0177}  %% посилання на сторінку оригінального видання
у величині витрат виробництва окремого товару, що її викликала припущена нами
зміна вартості\footnote{
\emph{Corbett} [„An Inquiry etc.“, стор. 20].
}.

Щодо змінного капіталу, — а це найважливіше, тому що він
є джерело додаткової вартості і тому що все те, що приховує
його відношення до збагачення капіталіста, обгортає таємницею
всю систему, — то справа в очах капіталіста має грубо-спрощений вигляд, а саме:
нехай, наприклад, змінний капітал в 100\pound{ фунтів стерлінгів} являє собою тижневу
заробітну плату 100 робітників. Якщо ці 100 робітників при даному робочому дні
виробляють за тиждень продукт в 200 штук товару, $= 200 T$, то
$1 T$, — якщо абстрагуватись від тієї частини витрат виробництва,
яку додає сталий капітал, — коштує 10\shil{ шилінгів}, бо 100\pound{ фунтів стерлінгів} $= 200 T$,
$1 T \deq{} \frac{\text{100\pound{ фунтів стерлінгів}}}{200} \deq{} 10\text{ шилінгам}$.
Припустімо тепер, що відбувається зміна в продуктивній силі
% REMOVED \footnote*{
% В першому німецькому виданні тут стоїть: „у виробничій силі“ (in der
% Produktionskraft); тут, а також і далі, виправлено на підставі рукопису Маркса.
% \Red{Примітка ред. нім. вид. ІМЕЛ.}
% }
праці,
що вона подвоюється, що, отже, те саме число робітників виробляє двічі по $200 T$ за той самий час, за
який воно раніш
виробляло $200 T$. В цьому випадку $1 T$ коштує (оскільки витрати виробництва складаються з самої тільки
заробітної плати)
5\shil{ шилінгів}, бо тепер 100\pound{ фунтів стерлінгів} $= 400 T$,
$1 T \deq{} \frac{\text{100\pound{ фунтів стерлінгів}}}{400} \deq{} 5\text{ шилінгам}$.
Коли б продуктивна сила
зменшилась удвоє, то та сама праця виробляла б тільки $\frac{200 T}{2}$; і через те що 100\pound{ фунтів стерлінгів}
$=\frac{200 T}{2}$, то тепер
$1 T \deq{} \frac{\text{200\pound{ фунтів стерлінгів}}}{200} \deq{} 1\text{ фунтові стерлінгів}$. Зміни в робочому
часі, потрібному для виробництва товарів, а тому і в вартості
товарів, виступають тепер щодо витрат виробництва, а тому
й щодо цін виробництва, як інший розподіл тієї самої заробітної
плати на більшу або меншу кількість товарів, залежно від того,
більше чи менше товарів виробляється протягом того самого
робочого часу за ту саму заробітну плату. Капіталіст, отже
й політико-економ, бачить тільки те, що частина оплаченої праці,
яка припадає на штуку товару, змінюється із зміною продуктивності праці, і
що разом з тим змінюється і вартість кожної
окремої штуки; він не бачить, що те саме має місце також і з неоплаченою працею, яка міститься в
кожній штуці товару, і тим
менше може це побачити, що пересічний зиск в дійсності тільки
випадково визначається неоплаченою працею, поглиненою в його
власній сфері виробництва. Тільки в такій грубій і ірраціональній
формі виступає тепер той факт, що вартість товарів визначається вміщеною в них працею.
