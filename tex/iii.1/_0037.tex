\parcont{}  %% абзац починається на попередній сторінці
\index{iii1}{0037}  %% посилання на сторінку оригінального видання
з капіталом в 66000 дукатів і трьома кораблями брало участь в португальській експедиції 1505—1506 рр. до Індії, одержавши
при цьому 150, а за іншими даними — 175\% чистого баришу (Heyd: „Levantehandel“, II, 524), і цілий ряд інших товариств-
„монополій", з яких так обурювався Лютер.

Тут ми вперше наштовхуємось на зиск і норму зиску. При цьому старання купців
навмисно і свідомо скероване на те, щоб зробити цю норму зиску рівною для всіх учасників. Венеціанці у Леванті, ганзейці на
півночі, кожний з них платив за куповані товари такі самі ціни, як і його сусіди. Вони коштували кожному з них однакових
транспортних витрат, кожен одержував за них? однакові ціни і купував поворотний вантаж теж по таких самих цінах, як кожний
інший купець його „нації“. Отже, норма зиску була однакова для всіх. У великих торговельних товариствах розподіл баришу
пропорціонально до вкладеної частини капіталу зрозумілий само собою цілком так само, як і участь у правах марки
пропорціонально до частини наділу, яка забезпечує ці права, або як участь в баришу гірничої промисловості пропорціонально
до числа паїв. Отже, рівна норма зиску, яка в своєму повному розвитку є одним з кінцевих результатів капіталістичного
виробництва, виявляється тут у своїй найпростішій формі одним з вихідних пунктів історичного розвитку капіталу, навіть
прямим відростком громади-марки, яка, в свою чергу, є прямий відросток первісного комунізму.

Ця первісна норма зиску
неминуче була дуже висока. Торговельне підприємство [в першу чергу монопольне торговельне підприємство, отже, винятково
зисковне] * було дуже рисковним ке тільки в наслідок дуже поширеного піратства, але й тому, що конкуруючі нації дозволяли
собі іноді всякого роду насильства, коли тільки для цього вгіпадала нагода; нарешті, збут і умови цього збуту грунтувались
на привілеях,-  одержаних від чужоземних князів, — привілеях, які досить часто порушувались або касувались. Отже, бариш
мусив включати високу страхову премію. Далі, оборот був повільний, хід справ — затяжний, а за кращих часів, які, зрештою,
рідко були тривалими, підприємство було монопольною торгівлею з монопольним зиском. Що пересічна норма зиску була дуже
висока, про це свідчать також і звичайні тоді дуже високі норми процента, які завжди загалом повинні бути нижчі за
звичайний рівень торговельного баришу.

Але ця висока і однакова для всіх учасників норма зиску, одержувана в результаті
спільної діяльності товариств, мала тільки місцеве значення в межах товариства, отже, в даному випадку, в межах „нації“.
Венеціанці, генуезці, ганзейці, голландці — кожна нація для себе і, мабуть, спочатку також більш

* Взяті у прямі дужки слова в рукопису Енгельса викреслені. Примітка ред. нім. вид. ІМЕЛ.
\parbreak{}  %% абзац продовжується на наступній сторінці
