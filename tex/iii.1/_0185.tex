\parcont{}  %% абзац починається на попередній сторінці
\index{iii1}{0185}  %% посилання на сторінку оригінального видання
самого роду і приблизно тієї самої якості продавались по їхніх
вартостях, потрібні дві умови:

\emph{Поперше}, різні індивідуальні вартості мусять вирівнятися
в \emph{одну} суспільну вартість, описану вище ринкову вартість, а для
цього потрібна конкуренція між виробниками того самого роду
товарів, а також наявність ринку, на якому вони разом пропонують свої товари. Для того, щоб ринкова
ціна товарів, тотожних між собою, але вироблюваних при обставинах різного
індивідуального забарвлення, відповідала ринковій вартості, не
відхилялась від неї, не підвищувалась і не знижувалась проти
неї, потрібно, щоб тиснення, яке справляють один на одного
різні продавці, було досить велике для того, щоб кинути на
ринок масу товарів, відповідну суспільній потребі, тобто таку
кількість їх, за яку суспільство спроможне заплатити ринкову
вартість. Якщо маса продуктів перевищує цю потребу, то товари мусять бути продані нижче їхньої
ринкової вартості; навпаки,
вони мусили б бути продані вище їх ринкової вартості, якщо
маса продуктів була б не досить велика, або — що те саме — якщо тиснення конкуренції серед продавців
було б не досить
значним, щоб примусити їх подати цю масу товарів на ринок.
Коли б ринкова вартість змінилась, то змінилися б і умови,
на яких могла б бути продана сукупна маса товарів. Якщо ринкова вартість падає, то суспільна потреба
(яка тут завжди
є платоспроможна потреба) пересічно розширюється і може
в певних межах поглинути більші маси товарів. Якщо ринкова
вартість підвищується, то суспільна потреба на товари скорочується і поглинає менші маси їх. Тому,
якщо попит і подання
реґулюють ринкову ціну або, точніше, відхилення ринкових цін
від ринкової вартості, то, з другого боку, ринкова вартість реґулює відношення попиту й подання або
той центр, навколо
якого коливання попиту й подання примушують коливатись
ринкові ціни.

Якщо розглянемо справу ближче, то побачимо, що умови,
які мають силу для вартості окремих товарів, репродукуються
тут як умови, що мають силу для вартості сукупної суми товарів певного роду; а капіталістичне
виробництво з самого
початку є масове виробництво; та й при інших, менш розвинених способах виробництва, товари, —
принаймні, головні товари, — вироблювані в незначних кількостях, як спільний продукт, хоча й
багатьох дрібних виробників, концентруються на ринку великими масами в руках відносно небагатьох
купців,
нагромаджуються і продаються ними як спільний продукт цілої
галузі виробництва або більшої чи меншої частини її.

Зауважмо тут цілком мимохідь, що „суспільна потреба“,
тобто те, що реґулює принцип попиту, істотно зумовлюється
відношенням різних класів одного до одного і їх взаємним
економічним становищем, а значить, поперше, відношенням сукупної додаткової вартості до заробітної
плати і, подруге,
\parbreak{}  %% абзац продовжується на наступній сторінці
