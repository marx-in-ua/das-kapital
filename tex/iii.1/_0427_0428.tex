\parcont{}  %% абзац починається на попередній сторінці
\index{iii1}{0427}  %% посилання на сторінку оригінального видання
саму суму, то вони, отже, представляють тепер капітал. І при
тому вони однаково представляють капітал як у тому випадку,
коли вони застосовуються для позик капіталістам, так і в тому
випадку, коли пізніше, при зменшенні попиту на такі грошові
позики, вони знову застосовуються для вкладення в цінні папери.
В усіх цих випадках слово капітал уживається тут тільки
в банкірському значенні, при чому воно означає, що банкір змушений
давати позики на суму більшу, ніж просто його кредит.

Як відомо, Англійський банк видає всі свої позики своїми
банкнотами. Якщо ж, не зважаючи на це, циркуляція банкнот
банку, як правило, зменшується в міру того, як збільшуються
в його руках дисконтовані векселі і застави під позики, отже,
збільшуються видані ним позики, — то що робиться з банкнотами,
пущеними в циркуляцію, яким чином припливають вони
назад до банку?

Насамперед, якщо попит на грошові позики викликається несприятливим
національним платіжним балансом і, отже, відпливом
золота, то справа дуже проста. Векселі дисконтуються в банкнотах.
Банкноти в самому банку, в Issue department [емісійному
департаменті] обмінюються на золото, і золото експортується.
Це те саме, як коли б банк безпосередньо при дисконті
векселів прямо платив золотом, без посередництва банкнот. Таке
підвищення попиту — яке досягає в певних випадках від 7 до
10 мільйонів фунтів стерлінгів — не додає, звичайно, до внутрішньої
циркуляції країни жодної п’ятифунтової банкноти. Якщо ж
кажуть, що банк при цьому дає в позику капітал, а не засоби
циркуляції, то це має двоякий сенс. Поперше, що він дає в позику
не кредит, а дійсну вартість, частину свого власного або
покладеного до нього як вклад капіталу. Подруге, що він дає
в позику гроші не для внутрішньої циркуляції, а для міжнародної
циркуляції, дає в позику світові гроші; а для цієї мети гроші
завжди мусять існувати в своїй формі скарбу, в своїй металічній
тілесності; в формі, в якій вони не тільки є формою вартості, але
самі дорівнюють тій вартості, грошовою формою якої вони є. Хоч
це золото як для банку, так і для експортуючого торговця золотом
представляє капітал, банкірський капітал або купецький капітал,
проте попит на нього виникає не як попит на капітал, а як попит
на абсолютну форму грошового капіталу. Він виникає саме в той
момент, коли закордонні ринки переповнені англійським товарним
капіталом, який не може бути реалізований. Отже, те, чого при
цьому вимагають, це — не капітал як \emph{капітал}, а капітал як \emph{гроші}, в
тій формі, в якій гроші є загальний товар світового ринку; а це є їх
первісна форма як благородного металу. Відпливи золота, отже,
не є a mere question of capital [чисте питання капіталу], як кажуть
Фуллартон, Тук та інші, a a question of money [питання грошей],
хоч і в специфічній функції. Те, що це не є питання
\emph{внутрішньої} циркуляції, як твердять прихильники теорії currency,
зовсім не є доказом, що це є просте question of capital, як думають
\index{iii1}{0428}  %% посилання на сторінку оригінального видання
Фуллартон та інші. Це є a question of money в тій формі, в якій
гроші є міжнародний засіб платежу. „Whether that capital (купівельна
ціна мільйонів квартерів закордонної пшениці після
неврожаю в країні) is transmitted in merchandize or in specie, is
a point which in no way affects the nature of the transaction“ [„Чи
передається цей капітал“ (купівельна ціна мільйонів квартерів
закордонної пшениці після неврожаю в країні) „товарами, чи
грішми готівкою, — ця обставина ні трохи не впливає на характер
операції“] (\emph{Фуллартон}, там же, стор. 131). Але це має велику
вагу для питання, чи відбувається відплив золота, чи ні. Капітал
передається в формі благородного металу, тому що в формі
товарів він зовсім не може бути переданий або може бути
переданий тільки з дуже великими втратами. Страх сучасної
банкової системи перед відпливом золота перевищує все, що
будь-коли марилося монетарній системі, для якої благородний
метал є єдине справжнє багатство. Візьмімо, наприклад, таке
свідчення управителя Англійського банку, Морріса, перед парламентським
комітетом про кризу 1847--1848~\abbr{рр.}: „3846. (Запитання:)
Коли я кажу про знецінювання запасів (stocks) і основного
капіталу, то чи не відомо вам, що весь капітал, вкладений у запаси
і продукти всякого роду, так само знецінився; що бавовна-сирець,
шовк-сирець, сировинна вовна відправлялися на континент
по таких самих бросових цінах і що цукор, кава й чай продавалися
з великими жертвами, як при продажах з молотка? — Країна
неминуче повинна була принести \emph{значні жертви} для того, щоб
протидіяти \emph{відпливові золота}, який відбувся в наслідок величезного
довозу харчових продуктів“. — „3848. Чи не думаєте
ви, що було б краще зачепити ті 8 мільйонів фунтів стерлінгів, які
лежали в сховищах банку, ніж намагатися одержати назад золото
з такими жертвами? — Ні, \emph{я цього не думаю}“. — Золото вважається
тут за єдине справжнє багатство.

Цитоване Фуллартоном відкриття Тука, що „\textenglish{with only one
or two exceptions, and those admitting of satisfactory explanation,
every remarkable fall of the exchange, followed by a drain of gold,
that has occurred during the last half century, has been coincident
throughout with a comparatively low state of the circulating medium,
and vice versa}“ [„за одним або двома винятками, яким
можна дати достатнє пояснення, всяке помітне падіння вексельного
курсу, супроводжуване відпливом золота, яке відбувалось
за останні півстоліття, завжди збігалося з порівняно низьким
рівнем засобів циркуляції, і навпаки“] (\emph{Fullarton}, стор. 121), —
доводить, що ці відпливи золота настають здебільшого після періоду
пожвавлення і спекуляції, як „а signal of a collapse already
commenced\dots{} an indication of overstocked markets, of a cessation
of the foreign demand for our productions, of delayed returns, and,
as the necessary sequel of all these, of commercial discredit, manufactories
shut up, artisans starving, and a general stagnation of
industry and enterprise“ [„сигнал краху, що вже почався\dots{} ознака
\parbreak{}  %% абзац продовжується на наступній сторінці
