\parcont{}  %% абзац починається на попередній сторінці
\index{iii1}{0032}  %% посилання на сторінку оригінального видання
\emph{засоби виробництва належать робітникові}, а таке є становище, як у стародавньому, так і в новітньому світі, у селянина, що працює
сам і володіє землею, і в ремісника. Це погоджується також з тим висловленим нами раніше поглядом, що розвиток
продуктів у товари постає через обмін між різними громадами, а не між членами однієї і тієї ж громади. Так само як до цього
первісного становища, це стосується також і до пізніших відносин, основаних на рабстві й кріпацтві, а також до цехової
організації ремесла — поки засоби виробництва, закріплені в кожній галузі виробництва, тільки з труднощами можуть бути
перенесені з однієї сфери в іншу, і тому різні сфери виробництва відносяться одна до одної до певної міри так само, як чужі
країни або комуністичні громади“ (\emph{Маркс}: „Капітал“,  т. III, стор. 202 і далі\footnote*{Стор. 176 цього укр. вид. \Red{Ред. укр. перекладу.}}).

Коли б Марксу вдалося ще раз переробити третю книгу, він, без сумніву, значно ширше розвинув би це місце. Так як воно тут
написане, воно дає лише нарис того, що слід було б сказати в цьому питанні. Отже, спинімось на ньому трохи детальніше.

Всі ми знаємо, що на перших ступенях суспільства продукти споживаються самими виробниками і що ці виробники організовані в
первісні, більше чи менше комуністичні громади; що обмін надлишків цих продуктів з чужинцями, який є вступом до перетворення
продуктів у товари, відноситься до пізніших часів, спочатку відбувається тільки між окремими різноплемінними громадами, а
пізніше набирає значення також і всередині громад та істотно сприяє розкладові їх на більші чи менші сімейні групи. Але
навіть і після цього розкладу глави сімей, що проводили обмін, лишались працюючими селянами, які майже все потрібне для
них виробляли за допомогою своєї сім’ї на своєму власному дворі і тільки незначну частину необхідних предметів вимінювали на
стороні за надлишок свого продукту. Сім’я займається не тільки землеробством і скотарством, але й перероблює продукти
землеробства й скотарства на готові предмети споживання, подекуди ще навіть меле на ручному млині, пече хліб, пряде, фарбує,
тче льон і вовну, чинить шкури, будує і лагодить дерев’яні будівлі, виробляє інструменти і знаряддя праці, нерідко
займається столярством і ковальством, так що сім’я або сімейна група в основному самодовліє.

Те немноге, що такій сім’ї
доводиться одержувати в обмін або купувати в інших, складалося в Німеччині навіть аж до початку XIX століття переважно з
предметів ремісничого виробництва, отже, з таких речей, спосіб виготовлення яких ніяк не був чужим для селянина і які він не
виробляв сам тільки тому, що не міг добути сировинного матеріалу, або тому, що куплений товар був значно кращої якості або
значно дешевший.
