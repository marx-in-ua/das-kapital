\parcont{}  %% абзац починається на попередній сторінці
\index{iii1}{0160}  %% посилання на сторінку оригінального видання
виробляють за однакові періоди часу неоднакові зиски і, отже,
чому норми зиску в цих різних сферах є різні.

Щождо відношення, в якому капітали складаються з основного і обігового капіталу, то воно,
розглядуване само по собі, ніяк не впливає на норму зиску. Воно може впливати на неї або тільки
тоді, коли цей різний склад збігається з різним відношенням між змінною і сталою частиною, отже,
коли цій ріжниці, а не ріжниці обігової і основної частин завдячує своє походження ріжниця норми
зиску; або ж тоді, коли різне відношення між основною і обіговою складовими частинами зумовлює
ріжницю в часі обороту, протягом якого реалізується певний зиск. Якщо капітали розпадаються в різній
пропорції на основний і обіговий, то це, правда, завжди справлятиме вплив
на час їх обороту і викликатиме в ньому ріжниці; але з цього
ще не випливає, що час обороту, протягом якого ті самі капітали реалізують зиск, є різний. Хоч $А$,
наприклад, мусить постійно перетворювати більшу частину продукту в сировинний матеріал та ін.,
тимчасом як $В$ довший час уживає ті самі
машини і~\abbr{т. д.} при меншій кількості сировинного матеріалу, — обидва вони, оскільки вони продукують,
завжди тримають
частину свого капіталу вкладеною в підприємство; один — у сировинний матеріал, отже, в обіговий
капітал, другий — у машини
і~\abbr{т. д.}, отже, в основний капітал. $А$ постійно перетворює частину свого капіталу з товарної форми в
грошову і з цієї останньої знов у форму сировинного матеріалу, тимчасом як $В$ протягом довшого часу
використовує частину свого капіталу як
знаряддя праці без такого перетворення. Якщо обидва вони
вживають однакову кількість праці, то на протязі року вони,
правда, продадуть маси продуктів неоднакової вартості, але
обидві маси продуктів міститимуть у собі однакові кількості
додаткової вартості, і їх норми зиску, обчислювані на весь авансований капітал, є однакові, хоч в
обох випадках, склад капіталу з
основної і обігової частини, так само як і час обороту, є різний.
Обидва капітали реалізують за однаковий час однакові зиски, хоч
обертаються вони за різні періоди часу\footnote{
[Як це випливає з розд. IV, вищевикладене є правильне тільки в тому
випадку, коли капітали $А$ і $В$ мають різний вартісний склад і коли при цьому
їх змінні складові частини, взяті в процентах, прямо пропорціональні їх періодам обороту, тобто
зворотно пропорціональні числу їх оборотів. Припустімо,
що процентний склад капіталу $А$ є $20 c$ основного $+70 c$ обігового, отже,
$90 c \dplus{} 10 v \deq{} 100$. При нормі додаткової вартості в 100\% ці $10 v$ за один оборот створюють $10 m$, норма
зиску для одного обороту \deq{} 10\%. Нехай, навпаки, капітал $В$ складається з $60 c$ основного $+20 c$
обігового, отже, з $80 c \dplus{} 20 v \deq{} 100$.
Ці $20 v$ створюють за один оборот при вищенаведеній нормі додаткової вартості $20 m$, норма зиску для
одного обороту тут \deq{} 20\%, тобто вдвоє більша
порівняно з $А$. Якщо ж $А$ обертається двічі на рік, а $В$ тільки один раз, то за
рік це так само становить $10 × 2 \deq{} 20 m$ і річна норма зиску буде для обох
однакова, а саме 20\%. — \emph{Ф.~Е.}]
}. Ріжниця в часі обороту
сама по собі має значення лиш остільки, оскільки вона впливає
на масу додаткової праці, яка протягом даного часу може бути
\parbreak{}  %% абзац продовжується на наступній сторінці
