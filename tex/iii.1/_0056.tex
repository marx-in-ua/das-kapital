\parcont{}  %% абзац починається на попередній сторінці
\index{iii1}{0056}  %% посилання на сторінку оригінального видання
не було, а лишається фактом, що додаткова вартість виникає одночасно
з усіх частин застосованого капіталу. Міркування можна
ще значно скоротити, коли разом з Мальтусом грубо й просто
сказати: „Капіталіст \emph{сподівається} однакового баришу від усіх
авансованих ним частин капіталу“.\footnote{
\emph{Malthus}: „Principles of Political Economy“. Видання друге, Лондон 1836,
стор. 267, 268.
}

Як такий уявлений плід усього авансованого капіталу, додаткова
вартість набирає перетвореної форми \emph{зиску}. Отже, певна
сума вартості є капітал тому, що її витрачають для того, щоб
утворити зиск,\footnote{
„Capital: that which is expended with a view to profit“ [„Капітал є те, що
витрачають для того, щоб одержати зиск“]. \emph{Malthus}: „Definitions in Political
Economy“. Лондон 1827, стор. 86.
} або зиск з’являється тому, що певну суму вартості
застосовують як капітал. Якщо зиск ми назвемо $р$, то формула
$T \deq{} c \dplus{} v \dplus{} m \deq{} k \dplus{} m$ перетворюється у формулу $Т \deq{} k \dplus{} р$, або
\emph{товарна вартість \deq{} витратам виробництва \dplus{} зиск}.

Отже, зиск, як ми тут спочатку маємо його перед собою,
є те саме, що й додаткова вартість, тільки в містифікованій
формі, яка, однак, з необхідністю виростає з капіталістичного
способу виробництва. Через те що в позірному утворенні витрат
виробництва не видно ніякої ріжниці між сталим і змінним
капіталом, то джерело тієї зміни вартості, яка відбувається
під час процесу виробництва, доводиться перенести із змінної
частини капіталу на весь капітал. Через те що на одному полюсі
ціна робочої сили з’являється у перетвореній формі заробітної
плати, то на протилежному полюсі додаткова вартість з’являється
у перетвореній формі зиску.

Ми бачили, що витрати виробництва товару менші, ніж його
вартість. Через те що $Т \deq{} k \dplus{} m$, то $k \deq{} Т - m$. Формула
$Т \deq{} k \dplus{} m$ тільки тоді зводиться до $Т \deq{} k$, товарна вартість \deq{}
витратам виробництва товару, коли $m \deq{} 0$ — випадок, який
ніколи не зустрічається на основі капіталістичного виробництва,
хоч при особливих ринкових коньюнктурах продажна ціна товарів
може падати до або навіть нижче витрат виробництва.

Тому, якщо товар продається по його вартості, то реалізується
зиск, який дорівнює надлишкові вартості товару понад витрати
його виробництва, отже, дорівнює всій додатковій вартості, яка
міститься в товарній вартості. Але капіталіст може продавати
товар з зиском, хоч і продаватиме його нижче його вартості.
Поки продажна ціна товару стоїть вище витрат його виробництва,
хоч і нижче його вартості, доти завжди реалізується
частина вміщеної в ньому додаткової вартості, отже, завжди
одержується зиск. У нашому прикладі товарна вартість \deq{} 600\pound{ фунтам стерлінгів}, витрати виробництва \deq{} 500\pound{ фунтам стерлінгів}.
Якщо товар продається за 510, 520, 530, 560, 590\pound{ фунтів
стерлінгів}, то він продається нижче його вартості відповідно на
90, 80, 70, 40, 10\pound{ фунтів стерлінгів}, і все ж від його продажу
\parbreak{}  %% абзац продовжується на наступній сторінці
