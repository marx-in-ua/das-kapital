
\index{iii1}{0059}  %% посилання на сторінку оригінального видання
Безглузде уявлення, ніби витрати виробництва товару становлять
його дійсну вартість, а додаткова вартість виникає
з продажу товару вище його вартості, що, отже, товари продаються
по їх вартостях, якщо їх продажна ціна дорівнює витратам
їх виробництва, тобто дорівнює ціні спожитих на них
засобів виробництва плюс заробітна плата, — це уявлення Прудон
з звичним шахрайством, яке чваниться вченістю, просурмив
як нововідкриту таємницю соціалізму. Це зведення вартості
товарів до витрат їх виробництва становить по суті
основу його народного банку. Раніше ми з’ясували, що різні складові
частини вартості продукту можна представити в пропорціональних
частинах самого продукту. Якщо, наприклад (книга І,
розд. VIІ, 2), вартість 20 фунтів пряжі становить
% REMOVED , стор. 229 \footnote*{ Стор. 153--154 рос. вид. 1935~\abbr{р.} \Red{Ред. укр. перекладу}. }
30\shil{ шилінгів} — а саме 24\shil{ шилінги} засоби виробництва, 3\shil{ шилінги}
робоча сила і 3\shil{ шилінги} додаткова вартість, — то цю додаткову
вартість можна представити як \sfrac{1}{10} продукту \deq{} 2 фунтам пряжі.
Тепер, якщо ці 20 фунтів пряжі продаються по витратах їх
виробництва, за 27\shil{ шилінгів}, то покупець дістає даром 2 фунти
пряжі, або товар продано на \sfrac{1}{10}  нижче його вартості; але робітник
так само, як і раніш, дав свою додаткову працю — тільки
для покупця пряжі, а не для капіталістичного виробника пряжі.
Було б цілком помилково припускати, що коли б усі товари
продавались по витратах їх виробництва, то результат фактично
був би той самий, як коли б усі товари продавались вище витрат
їх виробництва, але по їх вартостях. Бо навіть коли припустити,
що вартість робочої сили, довжина робочого дня
і ступінь експлуатації праці повсюди однакові, то все ж маси
додаткової вартості, які містяться у вартостях різних видів
товару, аж ніяк не рівні, залежно від різного органічного складу
капіталів, авансованих на їх виробництво\footnote{
„Вироблювані різними капіталами маси вартості і додаткової вартості, при
даній вартості і однаковому ступені експлуатації робочої сили, прямо пропорціональні
до величин змінних складових частин цих капіталів, тобто їх складових
частин, перетворених у живу робочу силу“ (книга 1, розд. ІХ).
}.
% REMOVED , стор. 321 [стор. 227 рос. вид. 1935~\abbr{р.}]
\section{Норма зиску}

Загальна формула капіталу є $Г — Т — Г'$; тобто певна сума
вартості кидається в циркуляцію для того, щоб витягти з неї
більшу суму вартості. Процес, який породжує цю більшу суму
вартості, є капіталістичне виробництво; процес, який реалізує
її, є циркуляція капіталу. Капіталіст виробляє товар не ради
самого товару, не ради його споживної вартості або свого особистого
споживання. Продукт, який в дійсності цікавить капіталіста,
\index{iii1}{0060}  %% посилання на сторінку оригінального видання
це не сам відчутний продукт, а надлишок вартості продукту
понад вартість спожитого на нього капіталу. Капіталіст
авансує весь капітал, не звертаючи уваги на ті різні ролі, що їх
відіграють складові частини капіталу у виробництві додаткової
вартості. Він однаково авансує всі ці складові частини капіталу
не тільки для того, щоб репродукувати авансований капітал, але
і для того, щоб виробити певний надлишок вартості понад
цей капітал. Він може перетворити вартість змінного капіталу,
який він авансує, у вищу вартість тільки через обмін його на
живу працю, через експлуатацію живої праці. Але він може експлуатувати
працю тільки в тому разі, коли він одночасно авансує
умови для здійснення цієї праці — засоби праці і предмет
праці, машини і сировинний матеріал, тобто коли він ту суму
вартості, якою він володіє, перетворює в форму умов виробництва;
як і взагалі, він тільки тому є капіталіст, тільки тому взагалі
може взятися до процесу експлуатації праці, що він як власник
умов праці протистоїть робітникові як володільцеві тільки робочої
сили. Вже раніше, в першій книзі, було показано, що саме
те, що цими засобами виробництва володіють не-робітники, перетворює
робітників у найманих робітників, а не-робітників — у капіталістів.

Капіталістові байдуже, чи розглядати справу так, що він
авансує сталий капітал для того, щоб здобути бариш із змінного,
чи так, що він авансує змінний капітал для того, щоб збільшити
вартість сталого; чи так, що він витрачає гроші на заробітну
плату для того, щоб надати машинам і сировинному матеріалові
вищу вартість, чи так, що він авансує гроші на машини та сировинний
матеріал для того, щоб мати можливість експлуатувати працю.
Хоч додаткову вартість утворює лише змінна частина капіталу,
проте вона утворює її тільки при тій умові, що авансуються
й інші частини, виробничі умови праці. Через те що капіталіст
може експлуатувати працю тільки за допомогою авансування сталого
капіталу, що він може збільшити вартість сталого капіталу
тільки за допомогою авансування змінного, то в його уявленні
ці капітали збігаються, і це тим більше, що дійсний рівень його
баришу визначається відношенням не до змінного капіталу, а
до всього капіталу, не нормою додаткової вартості, а нормою
зиску, яка, як ми побачимо, може лишатись однаковою і все ж
виражати різні норми додаткової вартості.

До витрат виготовлення (Kosten) продукту належать усі
складові частини його вартості, які капіталіст оплатив або еквівалент
яких він кинув у виробництво. Ці витрати мусять бути
заміщені для того, щоб капітал просто зберігся або репродукувався
в своїй первісній величині.

Вартість, яка міститься в товарі, дорівнює тому робочому
часові, якого коштує його виготовлення, а сума цієї праці складається
з оплаченої і неоплаченої праці. Навпаки, для капіталіста
витрати виготовлення товару складаються тільки з тієї
\parbreak{}  %% абзац продовжується на наступній сторінці
