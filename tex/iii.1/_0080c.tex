
\paragraph*{II. \emph{m′} змінюється}
Загальну формулу норм зиску при різних нормах додаткової
вартості, однаково, чи  \frac{v}{K} лишається незмінним, чи теж змінюється,
ми одержимо, коли рівняння:\[p' \deq{} m' \frac{v}{К}\]
перетворимо в інше:
\[
p'\textsubscript{1} \deq{} m'\textsubscript{1} \frac{v\textsubscript{1}}{K\textsubscript{1}},
\]
де $р'\textsubscript{1}, m'\textsubscript{1}, v\textsubscript{1}$ і $К\textsubscript{1}$ означають змінені величини $р', m', v$ і $К$.
Тоді ми маємо: \[
p': p'\textsubscript{1} \deq{} m' \frac{v}{K}: m'\textsubscript{1} \frac{v\textsubscript{1}}{K\textsubscript{1}},
\]
і звідси:\[
p'\textsubscript{1} \deq{} \frac{m'\textsubscript{1}}{m'} × \frac{v\textsubscript{1}}{v} × \frac{K}{K\textsubscript{1}} × p'.
\]


\subparagraph*{1) m′ змінюється, \frac{v}{K} не змінюється}
В цьому випадку ми маємо рівняння:
\[p' \deq{} m'\frac{v}{K}; p'\textsubscript{1} \deq{} m'\textsubscript{1} \frac{v}{K},\]
в обох рівняннях \frac{v}{K} має однакову величину. Тому одержуємо
відношення:
\[р': р'\textsubscript{1} \deq{} m': m'\textsubscript{1}.\]

\noindent{}Норми зиску двох капіталів однакового складу відносяться
одна до одної, як відповідні норми додаткової вартості. Через
те що в дробу $\frac{v}{K}$ важливі не абсолютні величини $v$ і $К$, а тільки
відношення між ними, то це стосується й до всіх капіталів однакового
складу, яка б не була їх абсолютна величина.

\begin{center}
$80c \dplus{} 20v \dplus{} 20m; K \deq{} 100, m' \deq{} 100\%, p' \deq{} 20\%$

$160c \dplus{} 40v \dplus{} 20m; K \deq{} 200, m' \deq{} 50\%, p' \deq{} 10\%$

 $100\%: 50\% \deq{} 20\%: 10\%.$
\end{center}

Якщо абсолютні величини $v$ і $К$ в обох випадках однакові,
то норми зиску відносяться одна до одної, крім того, як маси
додаткової вартості:
\[p': p'\textsubscript{1} \deq{} m'v: m'\textsubscript{1}v \deq{} m: m\textsubscript{1}.\]
