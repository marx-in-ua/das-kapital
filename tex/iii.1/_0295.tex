
\index{iii1}{0295}  %% посилання на сторінку оригінального видання
Ми тут цілком залишаємо осторонь ті витрати, що можуть
визначати ріжницю між купівельною і продажною ціною, бо ці
витрати зовсім нічого не змінюють у тій формі, яку ми тут
перш за все маємо розглянути.

Отже, число оборотів даного купецького капіталу має
тут повну аналогію з повторенням обігів грошей, як простого
засобу циркуляції. Подібно до того, як той самий талер, що
робить десять обігів, десять разів купує свою вартість у вигляді
товарів, так той самий грошовий капітал купця, наприклад
в 100, обертаючись десять разів, десять разів купує свою вартість
у вигляді товарів або реалізує в сукупності товарний капітал
десятикратної вартості \deq{} 1000. Але ріжниця така: при обігу
грошей, як засобу циркуляції, один й той самий грошовий знак
переходить через різні руки, отже, декілька разів виконує ту
саму функцію, і тому швидкістю обігу заміщається маса грошових
знаків, що циркулюють. Але в купця той самий грошовий
капітал, — однаково, з яких би він грошових знаків не складався, —
та сама грошова вартість декілька разів купує і продає товарний
капітал на суму своєї вартості, і тому декілька разів повертається
в ті самі руки, до свого вихідного пункту, як $Г \dplus{} ΔГ$,
як вартість плюс додаткова вартість. Це характеризує його
оборот як оборот капіталу. Він постійно витягає з циркуляції
більше грошей, ніж кидає в неї. Зрештою, само собою зрозуміло,
що з прискоренням обороту купецького капіталу (при чому при
розвиненій кредитній справі переважає функція грошей як засобу
платежу) та сама грошова маса циркулює швидше.

Але повторний оборот товарно-торговельного капіталу ніколи
не виражає нічого іншого, як повторення купівлі й продажу;
тимчасом як повторний оборот промислового капіталу виражає
періодичність і відновлення сукупного процесу репродукції (в
який включається і процес споживання). Навпаки, для купецького
капіталу це виступає тільки як зовнішня умова. Промисловий
капітал мусить постійно кидати товари на ринок і знову їх вилучати
звідти, щоб лишався можливим швидкий оборот купецького
капіталу. Якщо процес репродукції взагалі відбувається
повільно, то повільно відбувається і оборот купецького капіталу.
Хоча купецький капітал опосереднює оборот продуктивного
капіталу, але лиш остільки, оскільки він скорочує час його циркуляції.
Він не впливає безпосередньо на час виробництва, який
також становить межу для часу обороту промислового капіталу.
Це — перша межа для обороту купецького капіталу. А подруге,
залишаючи осторонь межу, яку утворює для нього репродуктивне
споживання, цей оборот кінець-кінцем обмежується швидкістю
і розмірами сукупного особистого споживання, бо від цього
залежить вся та частина товарного капіталу, яка входить у фонд
споживання.

Але (цілком залишаючи осторонь обороти в межах купецького
світу, де один купець постійно продає той самий товар
\parbreak{}  %% абзац продовжується на наступній сторінці
