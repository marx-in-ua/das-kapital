\parcont{}  %% абзац починається на попередній сторінці
\index{iii1}{0031}  %% посилання на сторінку оригінального видання
В одному приватному листі, який він дозволив мені цитувати, Шмідт прямо проголошує закон вартості в межах капіталістичної форми
виробництва фікцією, хоч і теоретично необхідною. — Але це розуміння, на мою думку, зовсім невірне. Закон вартості має для
капіталістичного виробництва далеко більше і визначеніше значення, ніж значення простої гіпотези, не кажучи вже про фікцію,
хоч би й необхідну.

Як Зомбарт, так і Шмідт, — знаменитого Лоріа я згадую тут тільки як забавний вульгарноекономічний
курйоз, — недосить звертають увагу на те, що тут ідеться не тільки про чисто логічний процес, а про історичний процес і
пояснююче відбиття цього процесу в понятті, логічне прослідження його внутрішніх зв’язків.

Вирішальне щодо цього місце є в
Маркса, III том, частина перша, стор. 200 *: „Вся трудність постає з того, що товари обмінюються не просто як товари, а як
продукти капіталів, які претендують на пропорціональну до їх величини або, при рівній величині, на рівну участь у сукупній
масі додаткової вартості“.  Для ілюстрації цієї ріжниці припускається, що робітники володіють своїми засобами виробництва,
працюють пересічно однаковий час і з однаковою інтенсивністю і безпосередньо обмінюють між собою свої товари. Тоді два
робітники протягом одного дня додали б своєю працею до свого продукту однакову кількість нової вартості, але продукт кожного
з них мав би різну вартість залежно від праці, вже раніше втіленої в засобах виробництва. Ця остання частина вартості
репрезентувала б сталий капітал капіталістичного господарства, частина новододаної вартості, витрачена на засоби існування
робітників, репрезенувала б змінний капітал, а та частина нової вартості, яка лишається ще понад це, репрезентувала б
додаткову вартість, яка, отже, в даному випадку належала б робітникам. Отже, обидва робітники, після відрахування заміщення
лише авансованої ними „сталої“ частини вартості, одержали б рівновеликі вартості; але відношення частини, яка репрезентує
додаткову вартість, до
вартості засобів виробництва, — що відповідало б капіталістичній нормі зиску, — було б у них різне. Але через те що кожний з них
при обміні одержує заміщення вартості засобів виробництва, це зовсім не мало б ніякого значення. „Отже, для обміну
товарів по їх вартостях або приблизно по їх вартостях потрібен значно нижчий ступінь, ніж для обміну по цінах виробництва,
для якого потрібна певна висота капіталістичного розвитку... Отже, незалежно від панування закону вартості над цінами і
рухом цін, цілком відповідає справі розглядати вартості товарів не тільки теоретично, але й історично, як prius відносно цін
виробництва **. Це стосується до таких економічних відносин, коли
* Стор. 174 цього укр. вид. Ред. укр. перекладу. ** — як те, що передує цінам виробництва. Ред. укр. перекладу.
\parbreak{}  %% абзац продовжується на наступній сторінці
