\parcont{}  %% абзац починається на попередній сторінці
\index{iii1}{0124}  %% посилання на сторінку оригінального видання
перетворена в гроші; друга частина існує як гроші в будьякій
формі і мусить бути знову перетворена в умови виробництва;
нарешті, третя частина перебуває у сфері виробництва, почасти
у первісній формі засобів виробництва, сировинних матеріалів,
допоміжних матеріалів, куплених на ринку півфабрикатів,
машин та іншого основного капіталу, почасти як продукт, який
ще тільки виготовляється. Як діє тут підвищення вартості або
зниження вартості, це в великій мірі залежить від того відношення,
в якому стоять одні до одних ці складові частини. Щоб
спростити питання, залишмо спочатку осторонь весь основний
капітал і розгляньмо тільки ту частину сталого капіталу, яка
складається з сировинних матеріалів, допоміжних матеріалів,
півфабрикатів і товарів, які ще тільки виготовляються або вже
є готові на ринку.

Якщо підвищується ціна сировинного матеріалу, наприклад,
бавовни, то підвищується й ціна бавовняних товарів — півфабрикатів,
як от пряжа, і готових товарів, як от тканини і~\abbr{т. д.}, —
сфабрикованих з дешевшої бавовни; так само підвищується
і вартість як ще непереробленої бавовни, яка є на складі, так
і тієї, що перебуває ще в процесі оброблення. Ця остання,
через те що вона в наслідок зворотного впливу стає виразом
більшої кількості робочого часу, додає до продукту, в який
вона входить як складова частина, більшу вартість, ніж та, яку
вона первісно мала сама і яку капіталіст заплатив за неї.

Отже, якщо підвищення цін сировинного матеріалу супроводиться
наявністю на ринку значної маси готового товару, —
все одно, на якому ступені готовості, — то підвищується вартість
цього товару і разом з тим відбувається підвищення вартості
наявного капіталу. Те саме стосується і до запасів сировинного
матеріалу і~\abbr{т. д.}, які перебувають в руках виробників.
Це підвищення вартості може відшкодувати або й більш ніж
відшкодувати окремих капіталістів або навіть і цілу окрему
сферу виробництва капіталу за падіння норми зиску, яке виникає
з підвищення ціни сировинного матеріалу. Не входячи тут
у деталі впливу конкуренції, можна, однак, ради повноти відзначити,
що 1)~коли запаси сировинного матеріалу, які перебувають
на складах, значні, то вони протидіють підвищенню цін,
що виникає в місці виробництва сировинного матеріалу; 2)~коли
півфабрикати або готові товари, які перебувають на ринку, дуже
тиснуть на ринок, то вони заважають ціні готових товарів і півфабрикатів
зростати відповідно до ціни їх сировинного матеріалу.

Зворотне маємо при падінні цін сировинного матеріалу, яке
при інших однакових умовах підвищує норму зиску. Товари, які
перебувають на ринку, речі, які ще тільки виготовляються,
запаси сировинного матеріалу знецінюються і цим самим протидіють
одночасному підвищенню норми зиску.

Чим менші запаси, які перебувають у сфері виробництва
і на ринку, наприклад наприкінці операційного року, коли сировинний
\index{iii1}{0125}  %% посилання на сторінку оригінального видання
матеріал знову постачається великими масами, як от у землеробстві після жнив, — тим
виразніше виступає вплив зміни цін сировинного матеріалу.

В усьому нашому дослідженні ми виходимо з того припущення, що підвищення або зниження цін є вираз
дійсних коливань вартості. Але через те що тут мова йде про той вплив, який ці коливання цін
справляють на норму зиску, то в дійсності не має значення, яка є причина цих коливань; отже,
розвинуте тут має силу також і тоді, коли ціни підвищуються і падають не в наслідок коливань
вартості, а в наслідок діяння системи кредиту, конкуренції і~\abbr{т. д.}

Через те що норма зиску дорівнює відношенню надлишку вартості продукту до вартості всього
авансованого капіталу, то підвищення норми зиску, що походить із зниження вартості авансованого
капіталу, може бути сполучене з втратою капітальної вартості; так само зниження норми зиску, що
походить з підвищення вартості авансованого капіталу, може бути сполучене з виграшем.

Щодо другої частини сталого капіталу, машин і взагалі основного капіталу, то підвищення вартості,
які тут відбуваються і стосуються саме до будівель, землі і~\abbr{т. д.}, не можуть бути розглянуті до
викладу вчення про земельну ренту і тому, вони не належать сюди. Але для зниження вартості цієї
частини капіталу загальне значення мають:

1. Постійні поліпшення, які позбавляють наявні машини, фабричне устаткування і~\abbr{т. д.} частини їх
споживної вартості,
а тому і їх вартості. Цей процес діє з особливою силою в перший період введення нових машин, раніше
ніж вони досягають певної міри зрілості, і коли вони через це постійно стають застарілими раніше,
ніж встигають репродукувати свою вартість. Це одна з причин звичайного в такі епохи безмірного
здовження робочого часу, праці вдень і вночі почережно змінами, для того, щоб протягом коротшого
часу репродукувати вартість машин, не відраховуючи при цьому занадто багато на їх зношування. Якщо
ж, навпаки, короткий період діяльності
машин (короткий строк їх життя в зв’язку з можливими поліпшеннями) не буде таким способом
скомпенсовано, то в наслідок їх морального зношування вони переносять на продукт занадто велику
частину своєї вартості, так що не можуть конкурувати навіть з ручною працею\footnote{Приклади, між іншим,
у Беббеджа. Звичайний засіб — зниження заробітної плати — застосовується і тут, і таким чином це постйно
знецінення діє цілком інакше, ніж це уявляє собі в своєму гармонійному мозку пан Кері.
}.

Якщо машини, устаткування будівель, взагалі основний капітал досяг певної зрілості, так що протягом
довшого часу
він, принаймні в своїй основній конструкції, лишається незмінним, то подібне ж зниження вартості
настає в наслідок поліпшень
\index{iii1}{0126}  %% посилання на сторінку оригінального видання
у методах репродукції цього основного капіталу. Вартість
машин і~\abbr{т. д.} знижується тепер не тому, що вони швидко
витісняються або до певної міри знецінюються новими продуктивнішими
машинами і~\abbr{т. д.}, а тому, що вони тепер можуть
бути дешевше репродуковані. Це одна з причин, чому великі підприємства
часто процвітають тільки в других руках, після того
як збанкрутує перший власник, а другий, що дешево купив
підприємство, таким чином уже з самого початку починає своє
виробництво з меншими витратами капіталу.

В землеробстві особливо впадає в очі, що ті самі причини,
які підвищують або знижують ціну продукту, підвищують або
знижують також і вартість капіталу, бо цей останній у значній
частині сам складається з цього продукту — хліба, худоби і~\abbr{т. ін.}
(Рікардо).

\pfbreak

Тепер треба було б згадати ще про змінний капітал.

Якщо вартість робочої сили підвищується в наслідок підвищення
вартості потрібних для її репродукції засобів існування,
або, навпаки, знижується в наслідок зниження вартості цих засобів
існування, — а підвищення вартості і зниження вартості
змінного капіталу не виражає нічого іншого, крім цих обох випадків, — то при незмінній довжині
робочого дня цьому підвищенню вартості відповідає падіння додаткової вартості, а цьому
зниженню вартості — зростання додаткової вартості. Але в той
самий час з цим можуть бути зв’язані й інші обставини — звільнення і зв’язування капіталу — які не
були ще досліджені і які
треба тепер коротко розглянути.

Якщо заробітна плата знижується в наслідок падіння вартості робочої сили (з чим може бути зв’язане
навіть підвищення
реальної ціни праці), то таким чином звільняється частина капіталу, яка досі витрачалась на
заробітну плату. Відбувається
звільнення змінного капіталу. На нововкладуваний капітал це
справляє тільки той вплив, що він працює з підвищеною нормою додаткової вартості. Та сама кількість
праці приводиться
в рух за допомогою меншої кількості грошей, ніж раніше, і таким чином неоплачена частина праці
збільшується коштом
оплаченої. Але для капіталу, який був вкладений уже раніше,
не тільки підвищується норма додаткової вартості, але, крім
того, звільняється частина капіталу, яка досі витрачалась на
заробітну плату. Досі вона була зв’язана і становила постійну
частину, яка відділялась від виручки за продукт і мусила витрачатись на заробітну плату,
функціонувати як змінний капітал,
якщо підприємство мало й далі провадитися в попередніх розмірах. Тепер ця частина стає вільною, і
може, отже, бути використана як нове капіталовкладення, чи для розширення того самого підприємства,
чи для функціонування в іншій сфері
виробництва.
