\parcont{}  %% абзац починається на попередній сторінці
\index{iii1}{0097}  %% посилання на сторінку оригінального видання
Цей розвиток продуктивної сили у кінцевому рахунку завжди
зводиться до суспільного характеру праці, приведеної в діяльність;
до поділу праці всередині суспільства; до розвитку інтелектуальної
праці, особливо природознавства. Капіталіст використовує
тут вигоди всієї системи суспільного поділу праці.
Вартість застосовуваного капіталістом сталого капіталу тут
відносно знижується, отже й норма зиску підвищується, в наслідок
розвитку продуктивної сили праці в галузі, яка лежить
поза межами даної галузі промисловості, в галузі, яка постачає
капіталістові засоби виробництва.

Підвищення норми зиску виникає ще й іншим шляхом, а саме
не з економії на тій праці, яка виробляє сталий капітал, а з економії
в застосуванні самого сталого капіталу. З одного боку,
сталий капітал заощаджується в наслідок концентрації робітників
та їх кооперації у великому масштабі. Ті самі будівлі, пристрої
для опалення й освітлення тощо коштують відносно
менше при виробництві у великому масштабі, ніж при виробництві
у невеликому масштабі. Те саме можна сказати і щодо
рушійних і робочих машин.. Хоч вартість їх абсолютно підвищується,
але відносно, порівняно з зростаючим розширенням
виробництва і величиною змінного капіталу або масою робочої
сили, яка приводиться в рух, вона падає. Та економія, яку певний
капітал реалізує у своїй власній галузі виробництва, полягає
насамперед і безпосередньо в економії на праці, тобто в скороченні
оплачуваної праці його власних робітників; навпаки, згадана
раніш економія полягає в тому, щоб це якомога більше привласнювання
чужої неоплаченої праці здійснювати якнайбільш економним
способом, тобто з якнайменшими при даному масштабі
виробництва витратами. Оскільки ця економія основана не
на згаданій вже експлуатації продуктивності суспільної праці,
вживаної у виробництві сталого капіталу, а на економії в застосуванні
самого сталого капіталу, вона виникає або безпосередньо
з кооперації і суспільної форми праці в самій даній галузі виробництва,
абож з виробництва машин і~\abbr{т. д.} в такому масштабі,
при якому їх вартість зростає не в такій мірі, як їх споживна
вартість.

Тут треба мати на увазі дві обставини: коли б вартість $c$ \deq{} 0,
то $р'$ було б \deq{} $m'$ і норма зиску досягла б свого максимуму.
Але подруге: для безпосередньої експлуатації самої праці важлива
ні в якому разі не вартість застосовуваних засобів експлуатації,
чи то основного капіталу, чи сировинних і допоміжних
матеріалів. Оскільки вони служать вбирачами праці, засобами
(Media), в яких і за допомогою яких упредметнюється праця,
а тому й додаткова праця, мінова вартість машин, будівель,
сировинних матеріалів і~\abbr{т. д.} зовсім не має значення. Єдине, що
тут має значення, це, з одного боку, їх маса, технічно потрібна
для сполучення з певною кількістю живої праці, з другого боку,
їх доцільність, отже, не тільки добрі машини, але й добрі сировинні
\index{iii1}{0098}  %% посилання на сторінку оригінального видання
і допоміжні матеріали. Від добротності сировинного матеріалу
почасти залежить норма зиску. Добрий матеріал дає менше
відпадів; отже, він вимагає меншої маси сировинного матеріалу
для вбирання в себе тієї самої кількості праці. Далі, опір, що
його зустрічає робоча машина, тут менший. Почасти це впливає
навіть на додаткову вартість і норму додаткової вартості. Робітникові
при поганому сировинному матеріалі потрібно більше
часу, щоб переробити ту саму кількість цього матеріалу; при
незмінній заробітній платі це веде до скорочення додаткової
праці. Далі, це дуже значно впливає на репродукцію і нагромадження
капіталу, що, як показано в першій книзі, стор. 634
і далі\footnote*{
Стор. 476 і далі рос. вид. 1935~\abbr{р.} \emph{Ред. укр. перекладу.}
}, ще більше залежать від продуктивності, ніж від маси вживаної
праці.

Тому зрозумілий фанатизм капіталіста в справі економізування
засобів виробництва. Щоб ніщо не пропадало і не марнотратилось,
щоб засоби виробництва споживались тільки таким
способом, як цього потребує само виробництво, — це почасти
залежить від муштри і навченості робітників, почасти від дисципліни,
в якій капіталіст тримає комбінованих робітників і яка стає
зайвою при такому суспільному ладі, де робітники працюють
власним коштом, як вона вже тепер стає майже цілком зайвою
при відштучній платі. Цей фанатизм виявляється, з другого боку,
і в фальсифікації елементів виробництва, яка є головний засіб
знизити вартість сталого капіталу порівняно з змінним і таким
чином підвищити норму зиску; при чому до цього долучається
ще, як істотний елемент шахрайства, продаж цих елементів
виробництва вище їх вартості, оскільки ця вартість знову з’являється
в продукті. Цей момент грає вирішальну роль, особливо
в німецькій промисловості, основний принцип якої такий: людям
може бути тільки приємно, коли ми спочатку надішлемо їм добрі
зразки, а потім погані товари. А втім, ці явища, які належать до
сфери конкуренції, нас тут не цікавлять.

Треба зауважити, що таке підвищення норми зиску, викликане
зменшенням вартості, отже, і дорожнечі сталого капіталу,
зовсім не залежить від того, чи виробляє та галузь промисловості,
в якій відбувається це підвищення, предмети розкоші,
чи засоби існування, які входять у споживання робітників, чи
засоби виробництва взагалі. Остання обставина може мати значення
лиш остільки, оскільки йдеться про норму додаткової
вартості, яка істотно залежить від вартості робочої сили, тобто
від вартості звичайних засобів існування робітника. Тут, навпаки,
додаткова вартість і норма додаткової вартості припускаються
даними. Як додаткова вартість відноситься до всього
капіталу — а це визначає норму зиску — залежить, при цих
умовах, виключно від вартості сталого капіталу, але ніяк не
від споживної вартості тих елементів, з яких він складається.

Відносне здешевлення засобів виробництва, звичайно, не виключає
\index{iii1}{0099}  %% посилання на сторінку оригінального видання
зростання абсолютної суми їх вартості; бо абсолютні
розміри, в яких вони застосовуються, надзвичайно збільшуються
разом з розвитком продуктивної сили праці і зростанням масштабу
виробництва, яке супроводить його. Економія в застосуванні
сталого капіталу, з якого б боку не розглядати її, є почасти
результат виключно того, що засоби виробництва функціонують
і споживаються як спільні засоби виробництва комбінованого
робітника, так що сама ця економія являє собою продукт суспільного
характеру безпосередньо продуктивної праці; а почасти
вона є результат розвитку продуктивності праці в тих
сферах, які постачають капіталові його засоби виробництва, так
що, коли розглядати сукупну працю в протиставленні сукупному
капіталові, а не самих тільки вживаних капіталістом $X$ робітників
у протиставленні цьому капіталістові $X$, то ця економія
знов таки виявляється продуктом розвитку продуктивних сил
суспільної праці, з тією тільки ріжницею, що капіталіст $X$ добуває
вигоду не тільки з продуктивності праці своєї власної
майстерні, але також і чужих майстерень. Не зважаючи на це',
економія на сталому капіталі здається капіталістові умовою
цілком чужою робітникові, умовою, яка абсолютно не торкається
робітника і з якою він не має нічого спільного; тимчасом
для капіталіста завжди лишається цілком ясним, що для
робітника, звичайно, має значення, багато чи мало праці купує
капіталіст за одні й ті самі гроші (бо саме такою виступає в його
свідомості угода між капіталістом та робітником). Ця економія
в застосуванні засобів виробництва, цей метод досягнення певного
результату з найменшими витратами, ще в значно більшій
мірі, ніж інші сили, імманентні праці, здається силою, імманентною
капіталові, і методом, що є властивий капіталістичному способові
виробництва і характеризує його.

Цей спосіб уявлення викликає тим менш сумніву, що йому
відповідає зовнішня видимість фактів, і що капіталістичне відношення
в дійсності приховує внутрішній зв’язок таким способом,
що для робітника умови здійснення його власної праці виявляються
чимось байдужим, зовнішнім і чужим.

\emph{Поперше}: Засоби виробництва, з яких складається сталий
капітал, репрезентують тільки гроші капіталіста (як, за Ленге,
тіло римського боржника репрезентувало гроші його кредитора)
і стоять у певному відношенні тільки до нього, тимчасом як
робітник, оскільки він у дійсному процесі виробництва приходить
з ними в дотик, має з ними діло тільки як із споживними вартостями
виробництва, засобами праці і матеріалами праці. Отже,
зменшення чи збільшення цієї вартості є така обставина, яка так
само мало зачіпає відношення робітника до капіталіста, як, наприклад,
та обставина, чи обробляє він мідь чи залізо. Звичайно, капіталіст,
як ми покажемо пізніше, любить розглядати справу інакше
в тих випадках, коли має місце збільшення вартості засобів
виробництва і, в наслідок цього, зменшення норми зиску.
