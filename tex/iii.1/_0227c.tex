
\index{iii1}{0227}  %% посилання на сторінку оригінального видання
[Норма зиску обчислюється на весь застосований капітал, але за певний час, фактично за один рік.
Відношення виробленої за рік і реалізованої додаткової вартості або зиску до всього капіталу,
обчислене в процентах, є норма зиску. Отже, вона не неодмінно дорівнює тій нормі зиску, при якій в
основу обчислення кладеться не рік, а період обороту капіталу, про який іде мова; тільки в тому
випадку, коли цей капітал обертається саме один раз за рік, обидві ці норми збігаються.

З другого боку, зиск, одержаний на протязі року, є тільки сума зисків на товари, вироблені і продані
на протязі того самого року. Якщо ж ми обчислюватимем зиск на витрати виробництва товарів, то
одержимо норму зиску $= \frac{p}{k}$, де $р$ становить реалізований на протязі року зиск, а $k$ — суму витрат
виробництва товарів, вироблених і проданих протягом того самого часу. Очевидно, що ця норма зиску
$\frac{p}{k}$ тільки в тому випадку може збігатися з дійсною нормою зиску $\frac{p}{K}$, — маса зиску, поділена на весь
капітал, — коли $k \deq{} К$, тобто коли капітал обертається, саме один раз за рік.

Візьмімо три різні стани якогонебудь промислового капіталу.

І.~Капітал в 8000\pound{ фунтів стерлінгів} виробляє і продає щороку 5000 штук товару по 30\shil{ шилінгів} за
штуку, отже, має річний оборот в 7500\pound{ фунтів стерлінгів}. На кожну штуку товару
він дає зиск в 10\shil{ шилінгів} \deq{} 2500\pound{ фунтам стерлінгів} на рік. Отже, в кожній штуці містяться 20\shil{ шилінгів} авансованого капіталу і 10\shil{ шилінгів} зиску, отже норма зиску на кожну штуку становить
$\frac{10}{20}= 50\%$. На суму в 7500\pound{ фунтів стерлінгів}, що обернулась, припадає 5000\pound{ фунтів стерлінгів}
авансованого капіталу і 2500\pound{ фунтів стерлінгів} зиску; норма зиску на кожний оборот, $\frac{p}{k}$, так само \deq{}
50\%. Навпаки, норма зиску, обчислена на весь капітал, $\frac{p}{K} \deq{} \frac{2500}{8000} \deq{} 31\sfrac{1}{4}\%$.

II.~Припустім, що капітал збільшується до \num{10000}\pound{ фунтів стерлінгів}. Припустім, що в наслідок
збільшеної продуктивної сили праці він може виробляти щороку \num{10000} штук товару при витратах
виробництва в 20\shil{ шилінгів} на штуку. Він продає їх із зиском в 4\shil{ шилінги} на штуку, отже, по 24\shil{ шилінги} за штуку. Тоді ціна річного продукту \deq{} \num{12000}\pound{ фунтам стерлінгів}, з яких \num{10000}\pound{ фунтів
стерлінгів} авансованого капіталу і 2000\pound{ фунтів стерлінгів} зиску $\frac{p}{k}$ на кожну штуку $= \frac{4}{20}$, для річного
обороту $= \frac{2000}{\num{10000}}$, отже, в обох випадках \deq{} 20\%, а через те що весь
\parbreak{}  %% абзац продовжується на наступній сторінці
