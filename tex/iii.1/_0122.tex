\parcont{}  %% абзац починається на попередній сторінці
\index{iii1}{0122}  %% посилання на сторінку оригінального видання
той час, коли в наслідок американської громадянської війни ціна
бавовни підвищилась до нечуваного за ціле майже століття
рівня, звіт говорив зовсім інше: „Ціна, яку тепер дають за бавовняні
відпади, і повторне використання цих відпадів на
фабриці як сировинного матеріалу компенсують до певної міри
ріжницю в утраті на відпадах між індійською і американською
бавовною. Ця ріжниця становить приблизно 12\sfrac{1}{2}\%. Втрата при
обробленні індійської бавовни становить 25\%, так що в дійсності
бавовна коштує прядільникові на \sfrac{1}{4} більше, ніж він за неї
платить. Втрата на відпадах не була така важлива, коли американська
бавовна коштувала 5 або 6 пенсів за 1 фунт, бо вона
не перевищувала тоді \sfrac{3}{4} пенса на фунт; але вона дуже важлива
тепер, коли 1 фунт бавовни коштує 2 шилінги, і втрата на відпадах
становить, отже, 6 пенсів“\footnote{
У кінцевій фразі звіту зроблено помилку. Замість 6 пенсів мусить бути З
пенси втрати на відпадах. Ця втрата становить, правда, 25\% при обробленні
індійської бавовни, але тільки 12\sfrac{1}{2}—15\% при обробленні американської бавовни,
а мова тут іде про цю останню, при чому раніше той самий процент при
ціні в 5—6 пенсів був правильно обчислений. А втім, і при обробленні американської
бавовни, яка довозилась до Европи протягом останніх років громадянської
війни, процент відпадів часто був значно вищий, ніж у попередні
часи. — \emph{Ф. Е.}
} („Rep. of Insp. of Fact., Oct.
1863“, стор. 106).

\subsection{Підвищення й зниження вартості капіталу, звільнення
і зв’язування капіталу}

Явища, які ми досліджуємо в цьому розділі, для свого повного
розвитку передбачають наявність кредитної справи і конкуренції
на світовому ринку, який взагалі становить базу й життьову атмосферу
капіталістичного способу виробництва. Але ці конкретніші
форми капіталістичного виробництва можуть бути вичерпно розглянуті
тільки після того, як буде з’ясована загальна природа
капіталу; крім того, розгляд цих форм не входить у план нашої
праці і належить до можливого продовження її. Проте, явища,
зазначені в заголовку, можуть бути тут розглянуті в загальній
формі. Вони зв’язані, поперше, між собою, а подруге, як
з нормою, так і з масою зиску. Їх треба коротко розглянути
хоч би вже тому, що вони викликають ілюзію, ніби не тільки
норма, але й маса зиску — яка в дійсності тотожна з масою
додаткової вартості — може зменшуватись або збільшуватись
незалежно від рухів додаткової вартості, чи то її маси чи її
норми.

Чи можна розглядати звільнення і зв’язування капіталу на
одному боці, підвищення і зниження його вартості на другому,
як різні явища?

Насамперед постає питання: що розуміємо ми під звільненням
і зв’язуванням капіталу? Підвищення і зниження вартості
\parbreak{}  %% абзац продовжується на наступній сторінці
