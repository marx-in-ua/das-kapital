\parcont{}  %% абзац починається на попередній сторінці
\index{iii1}{0106}  %% посилання на сторінку оригінального видання
сухот, в Блекберні й Скіптоні — 167, в Конглетоні й Бредфорді —
168, в Лейстері — 171, в Ліку — 182, в Маккльсфільді — 184,
в Больтоні — 190, в Ноттінгемі — 192, в Ронделі — 193, в Дербі —
198, в Сальфорді і Аштоні-на-Лайні — 203, в Лідсі — 218, в Престоні — 220 і в Манчестері — 263
(стор. 24). Нижченаведена таблиця дає ще разючіший приклад. Вона наводить випадки смерті
в наслідок хвороб легенів окремо для обох статей між 15 і
25 роками, обчислені на кожні \num{100000} мешканців. Вибрано такі
округи, де тільки жінки зайняті в промисловості, провадженій
у закритих приміщеннях, а чоловіки — в усяких можливих галузях праці.

\begin{table}[H]
  \centering
  \small
  \captionnew{Число випадків смерті від легеневих захворювань \\ між 15 і 25
роками на 100~000 жителів}
  \begin{tabular}{l l c c}
    \toprule
    Округи &
    Головна промисловість &
    Чоловіки & Жінки \\
    \midrule

Berkhampstead    & Плетіння з соломи, працюють жінки & 219 & 578 \\
Leighton Buzzard  &                                       & 309 & 554 \\
Newport Pagnell  & Плетіння мережива жінками         & 301 & 617 \\
Towcester        &                                        & 239 & 577 \\
Yeovil           & \makecell[lb]{Виробництво рукавичок,\\здебільшого працюють жінки} & 280 & 409 \\
Leek             & \makecell[lb]{Шовкова промисловість, \\ переважно жінки} & 437 & 856 \\
Congleton        &                                             & 566 & 790 \\
Macclesfield     &                                             & 593 & 890 \\
\makecell[lb]{Здорова сільська \\ місцевість} &   Землеробство                         & 331 & 333 \\
  \end{tabular}
\end{table}

\noindent{}В округах шовкової промисловості, де участь чоловіків
у фабричній праці більша, більша також і смертність серед них.
Норма смертності від сухот і~\abbr{т. п.} як чоловіків, так і жінок
виявляє тут, як сказано в звіті, „обурливі (atrocious) санітарні
умови, за яких провадиться значна частина нашої шовкової
промисловості“. І це, якраз, та сама шовкова промисловість,
фабриканти якої, посилаючись на винятково сприятливі санітарні умови свого виробництва, вимагали і
почасти добилися
винятково довгого робочого часу для дітей, молодших 13 років.
(книга І, розд. VIII, 6).
%REMOVED , стор. 306\footnote*{Стор. 214 рос. вид. 1935~\abbr{р.} \Red{Ред. укр. перекладу.}}

„Без сумніву, жодна з досліджених досі галузей промисловості
не дає сумнішої картини, ніж та, що її дає доктор Сміт
щодо кравецтва\dots{} Майстерні, каже він, дуже неоднакові щодо
\parbreak{}  %% абзац продовжується на наступній сторінці
