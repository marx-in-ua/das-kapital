\parcont{}  %% абзац починається на попередній сторінці
\index{iii1}{0120}  %% посилання на сторінку оригінального видання
\num{250000} ручних верстатів, які функціонували тоді в бавовноткацькій
промисловості Великобританії, споживали щорічно 41 мільйон
фунтів борошна для шліхтування основи. До цього треба
ще додати третину цієї кількості на біління й інші процеси.
Загальну вартість споживаного таким чином борошна він обчислює
для останніх 10 років у \num{342083}\pound{ фунтів стерлінгів} на рік.
Порівняння з цінами на борошно на континенті показало, що
переплати на самому тільки борошні, які фабриканти примушені
були робити в наслідок мита на хліб, становили щорічно
\num{170000}\pound{ фунтів стерлінгів}. Для 1837 року Грег оцінює ці
переплати щонайменше в \num{200000}\pound{ фунтів стерлінгів} і вказує
одну фірму, для якої переплата на борошні становила щорічно
1000\pound{ фунтів стерлінгів}. В наслідок цього „великі фабриканти,
дбайливі і рахубливі ділки, кажуть, що 10 годин щоденної
праці було б цілком досить, коли б скасували мита на хліб“
(„Rep. of Insp. of Fact., Oct. 1848“, стор. 98). Мита на хліб
були скасовані; крім того, були скасовані мита на бавовну
і інші сировинні матеріали; але ледве цього було досягнуто, як
опозиція фабрикантів десятигодинному білеві стала запеклішою,
ніж колибудь. І коли зразу після цього десятигодинний робочий
день на фабриках все ж став законом, його першим наслідком
була спроба загального зниження заробітної плати.

Вартість сировинних і допоміжних матеріалів цілком і за один
раз входить у вартість продукту, на виготовлення якого вони споживаються,
тимчасом як вартість елементів основного капіталу
входить у продукт тільки в міру свого зношування, отже, тільки
ступнево. З цього випливає, що на ціну продукту в далеко більшій
мірі впливає ціна сировинного матеріалу, ніж ціна основного
капіталу, хоч норма зиску визначається загальною сумою вартості
застосовуваного капіталу, незалежно від того, скільки
саме з нього спожито. Але ясно, — хоч про це ми згадуємо
тільки мимохідь, бо ми й тут ще припускаємо, що товари продаються
по їх вартості, отже, викликувані конкуренцією коливання
цін нас тут ще зовсім не цікавлять, — що розширення або
скорочення ринку залежить від ціни окремого товару і стоїть
у зворотному відношенні до зростання або падіння цієї ціни.
Тому в дійсності з підвищенням ціни сировинного матеріалу ціна
фабрикату підвищується не в тій самій пропорції, а при падінні
ціни сировинного матеріалу знижується не в тій самій пропорції,
як ціна сировинного матеріалу. Тому норма зиску в
одному випадку падає нижче, а в другому підіймається вище,
ніж це було б при продажу товарів по їх вартості.

Далі: маса й вартість застосовуваних машин зростає з розвитком
продуктивної сили праці, але не в тій самій пропорції,
в якій зростає ця продуктивна сила, тобто в якій ці машини
постачають збільшену кількість продукту. Отже, в тих галузях
промисловості, куди взагалі входить сировинний матеріал, тобто
де предмет праці сам уже є продукт ранішої праці, там зростання
\index{iii1}{0121}  %% посилання на сторінку оригінального видання
продуктивної сили праці виражається якраз в тому відношенні,
в якому збільшується кількість сировинного матеріалу,
що вбирає в себе певну кількість праці, отже, в зростаючій масі
сировинного матеріалу, яка, наприклад, за одну робочу годину перетворюється
в продукт, перероблюється на товар. Отже, в
міру розвитку продуктивної сили праці вартість сировинного
матеріалу становить все зростаючу складову частину вартості
товарного продукту, і не тільки тому, що вона цілком входить
у цю останню, але й тому, що в кожній відповідній частині
цілого продукту постійно зменшується і частина, яка відповідає
зношуванню машин, і частина, яку створює новододана праця.
В наслідок цього спадного руху відносно зростає друга частина
вартості, утворювана сировинним матеріалом, якщо це зростання
не знищується відповідним зменшенням вартості на боці сировинного
матеріалу, яке випливає з ростущої продуктивності
праці, застосовуваної для виготовлення самого цього сировинного
матеріалу.

Далі: через те що сировинні й допоміжні матеріали цілком
так само, як і заробітна плата, становлять складові частини
обігового капіталу, отже, мусять постійно цілком заміщатися
з кожного продажу продукту, тимчасом як щодо машин треба
заміщати тільки зношування, і до того ж на перший час у формі
резервного фонду, — при чому в дійсності зовсім неістотно, чи
дає кожний окремий продаж відповідну частину для цього
резервного фонду, якщо тільки весь річний продаж дає для
цього фонду відповідну річну частину, — то тут знову виявляється,
що підвищення ціни сировинного матеріалу може урізати
або загальмувати весь процес репродукції, якщо виручена
від продажу товарів ціна недостатня для заміщення всіх елементів
товару, або якщо ця ціна робить неможливим продовження
процесу в розмірах, відповідних його технічній основі,
так що або тільки частина машин може працювати абож усі
машини не можуть працювати звичайний повний час.

\looseness=-1
Нарешті, витрати, спричинювані відпадами, змінюються в прямому
відношенні до коливань ціни сировинного матеріалу: підвищуються,
якщо вона підвищується, падають, якщо вона падає. Але
й тут є певна межа. Ще в 1850 році було написано: „Одно з джерел
значних втрат, що виникають з підвищення ціни сировинного
матеріалу, ледве чи буде помітне для кожного, хто не є
прядільником-практиком, а саме втрата на відпадах. Мене повідомляють,
що коли ціна бавовни підвищується, то витрати прядільника,
особливо при виготовленні пряжі низької якості, зростають
у більшій мірі, ніж це показує виплачена надбавка до
ціни. Відпади при прядінні грубої пряжі становлять понад 15\%;
отже, якщо цей процент спричинює втрату в \sfrac{1}{2}\pens{ пенса} на фунт
при ціні бавовни в 3\sfrac{1}{2}\pens{ пенса}, то при підвищенні ціни бавовни
до 7\pens{ пенсів} за фунт він підвищує цю втрату до 1\pens{ пенса} на
фунт“ („Rep. of Insp. of Fact., April 1850“, стор. 17). — Але в
\parbreak{}  %% абзац продовжується на наступній сторінці
