\index{iii1}{0033}  %% посилання на сторінку оригінального видання
Отже, селянинові середньовіччя був досить точно відомий робочий час, потрібний для виготовлення предметів, одержуваних ним в
обмін. Адже сільський коваль, сільський возороб працювали перед його очима, так само як і кравець та швець, які ще за часів
моєї молодості ходили в наших рейнських селян по черзі з хати в хату і виробляли з саморобних матеріалів одяг і взуття.
Селянин так само, як і ті люди, в яких він купував, самі були працівниками [безпосередніми виробниками] *, обмінювані
предмети були власними продуктами кожного з них. Що витрачали вони при виготовленні цих продуктів? Працю, і тільки працю: на
заміщення знарядь праці, на вироблення сировинного матеріалу, на його оброблення вони витрачали не що інше, як свою власну
робочу силу; отже, яким чином вони могли обмінювати ці свої продукти на продукти інших працюючих виробників інакше, як
пропорціонально до витраченої на них праці?  При таких обставинах робочий час, витрачений на ці продукти, був не тільки
єдиним придатним для кількісного визначення обмінюваних величин мірилом; при таких обставинах взагалі було неможливе ніяке
інше мірило. Чи можна припустити, що селянин і ремісник були такі дурні, щоб віддавати продукт десятигодинної праці одного
на продукт однієї години праці іншого? Для всього періоду селянського натурального господарства можливий тільки такий
обмін, коли обмінювані кількості товарів мають тенденцію все більше й більше вимірюватись за втіленими в них кількостями
праці. З того моменту, як у цьому господарському ладі з’являються гроші, тенденція пристосування до закону вартості
(Notabene — у формулюванні Маркса!) стає, з одного боку, ще виразнішою, але, з другого боку, вона починає також уже
порушуватись в наслідок втручання лихварського капіталу і фіскальної експлуатації, і ті періоди, протягом яких ціни в
середньому наближаються до вартостей з точністю до такої незначної величини, що нею можна знехтувати, стають уже довшими.

Те саме стосується до обміну продуктів сёлян на продукти міських ремісників. Спочатку цей обмін відбувається прямо, без
посередництва купця, в базарні дні по містах, де селянин продає і робить закупівлі. І тут так само не тільки селянинові
відомі умови праці ремісника, але й ремісникові відомі умови праці селянина. Бо сам ремісник до певної міри ще селянин, він
має не тільки город і сад, але дуже часто ще й клапоть землі, одну-дві корови, свиней, свійську птицю і т. д. Таким чином,
за середньовіччя люди були спроможні досить точно підрахувати один одному витрати виробництва в частині сировинного
матеріалу, допоміжного матеріалу, робочого часу — принаймні оскільки це торкалось предметів щоденного загального вжитку.

* Взяті у прямі дужки слова в рукопису Енгельса закреслені. Примітка ред. нім. вид. ІМЕЛ.
