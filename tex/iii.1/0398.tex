що попит на грошовий капітал не тотожний з попитом на гроші
як такі; і тільки цієї премудрості, бо в замислах його, а також
Оверстона та інших пророків теорії currency завжди таїться нечиста совість, намагання шляхом
штучного законодавчого втручання зробити з засобу циркуляції як такого капітал і підвищити
розмір процента.

Перейдімо тепер до лорда Оверстона, або, інакше, Samuel
Jones Loyd’a, і послухаймо, як йому доводиться пояснювати, чому
він бере 10\% за свої „гроші“, раз „капітал“ у країні такий
рідкий.

„3653. Коливання в нормі процента виникають з однієї з двох
причин: із зміни в вартості капіталу“ [чудесно! адже вартість
капіталу, взагалі кажучи, і є якраз розмір процента! Отже, зміна
в нормі процента виникає тут із зміни в нормі процента. „Вартість капіталу“, як ми раніше показали,
теоретично ніколи не
розуміється інакше. Або, може, пан Оверстон під вартістю
капіталу розуміє норму зиску, — тоді глибокодумний мислитель
повертається до того, що норма процента регулюється нормою
зиску!] „або із зміни в сумі наявних у країні грошей. Всі великі
коливання розміру процента, великі або щодо тривалості або
щодо амплітуди коливань, можуть бути явно зведені до змін
у вартості капіталу. Не можна подати разючіших практичних ілюстрацій цього факта, як підвищення
розміру процента в 1847 році
і потім знову за останні два роки (1855—1856); менші коливання
розміру процента, які виникають із зміни в сумі наявних грошей,
є малі як щодо своєї амплітуди, так і щодо своєї тривалості. Вони
відбуваються часто, і чим частіше, тим вони дійовіші для своєї
мети“. А саме, щоб збагачувати банкірів à lа Оверстон. Шановний Samuel Gurney висловлюється про це
перед Committee of
Lords [комітетом вищої палати], „Commercial Distress“ 1848/57 дуже
наївно: „1324. Чи вважаєте ви, що великі коливання розміру
процента, які мали місце в минулому році, були вигідні для
банкірів та торговців грішми, чи ні? — Я думаю, що вони були
вигідні для торговців грішми. Всі коливання в цій справі вигідні
для тих, хто обізнаний у ній (to the knowing men)“. — „1325. Чи не
втрачає кінець-кінцем банкір при високому розмірі процента
в наслідок збіднення своїх кращих клієнтів? — Ні, я не думаю,
щоб такий результат мав місце в скільки-небудь помітній мірі“. — Voilà ce que parler veut dire
[такий є сенс цих балачок].

Ми ще повернемось до питання про вплив, що його справляє
сума наявних грошей на розмір процента. Але вже тепер необхідно зазначити, що Оверстон тут знов
робить qui pro quo [сплутання]. Попит на грошовий капітал у 1847 році (до жовтня не
було ніяких турбот про недостачу грошей, про „кількість
наявних грошей“, як він це вище назвав) збільшувався з різних
причин. Подорожчання хліба, підвищення цін на бавовну, неможливість продати цукор в наслідок
перепродукції, залізнична
спекуляція і крах, переповнення зовнішніх ринків бавовняними
