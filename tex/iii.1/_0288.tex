\parcont{}  %% абзац починається на попередній сторінці
\index{iii1}{0288}  %% посилання на сторінку оригінального видання
В міру того, як відбувалася б централізація капіталу в сфері
виробництва, відбувалася б його децентралізація у сфері циркуляції.
В наслідок цього чисто купецькі операції промислового
капіталіста, а разом з тим і його чисто купецькі видатки, безмежно
збільшилися б, бо йому б доводилось мати справу, скажемо,
з 1000 купців замість 100. В наслідок цього більша частина
вигоди від усамостійнення купецького капіталу втратилася б;
крім чисто купецьких витрат, зростали б також інші витрати
циркуляції, витрати сортування, відправки і~\abbr{т. д.} Це щодо промислового
капіталу. Розгляньмо тепер купецький капітал. Поперше,
щодо чисто купецьких робіт. В рахівництві обчисляти
більші числа не коштує більше часу, ніж обчисляти малі. Зробити
10 покупок по 100\pound{ фунтів стерлінгів} коштує вдесятеро більше
часу, ніж зробити \emph{одну} покупку в 1000\pound{ фунтів стерлінгів}. Кореспонденція,
папір, поштові витрати коштують вдесятеро більше,
коли мати справу з 10 дрібними купцями, ніж коли вести кореспонденцію
з \emph{одним} великим купцем. Обмежений поділ праці в комерційній
майстерні, де один веде книги, другий касу, третій
кореспонденцію, один купує, другий продає, третій роз’їжджає
і~\abbr{т. д.}, заощаджує робочий час у величезних розмірах, так що
число торговельних робітників, вживаних у гуртовій торгівлі,
не стоїть ні в якій відповідності до відносної величини підприємства.
Це тому, що в торгівлі далеко більше, ніж у промисловості,
та сама функція коштує однакової кількості робочого часу
незалежно від того, чи виконується вона у великому чи в малому
масштабі. Тому й концентрація в торговельному підприємстві
історично виявляється раніше, ніж у промисловій майстерні.
Далі, видатки на сталий капітал. 100 дрібних контор коштують
незрівнянно більше, ніж одна велика, 100 дрібних товарних складів
— безмежно більше, ніж один великий, і~\abbr{т. д.} Транспортні витрати,
які входять у купецьке підприємство, принаймні як витрати,
які доводиться авансувати, зростають разом із роздрібненням.

Промисловий капіталіст мусив би витрачати більше праці
і видатків циркуляції в торговельній частині свого підприємства.
Той самий купецький капітал, розподілений між багатьма дрібними
купцями, вимагав би в наслідок такого роздрібнення далеко
більше робітників для опосереднення своїх функцій і, крім
того, потрібний був би більший купецький капітал для того,
щоб обертати той самий товарний капітал.

Якщо ми весь купецький капітал, витрачуваний безпосередньо
в купівлі та продажу товарів, позначимо через $В$, а відповідний
змінний капітал, витрачуваний на оплату допоміжних торговельних
робітників, через $b$, то $B \dplus{} b$ було б менше, ніж мусив би
бути весь купецький капітал В, коли б кожен купець обходився
без помічників, тобто коли б жодна частина не витрачалась на $b$.
Проте, ми все ще не розв’язали труднощів.

Продажна ціна товарів мусить бути достатньою: 1) для того,
щоб оплатити пересічний зиск на $В \dplus{} b$. Це пояснюється вже
\parbreak{}  %% абзац продовжується на наступній сторінці
