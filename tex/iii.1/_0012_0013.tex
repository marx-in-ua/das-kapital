
\index{iii1}{0012}  %% посилання на сторінку оригінального видання
Головну трудність становив відділ V, в якому до того ж
розглядається найзаплутаніший предмет всієї книги. І саме тоді,
коли Маркс розробляв цей відділ, його спіткав один з вищезгаданих
тяжких приступів хвороби. Отже, тут ми маємо не готовий
нарис, навіть не схему, обриси якої треба було б заповнити,
а тільки початок оброблення, який у багатьох випадках зводиться
до невпорядкованої купи заміток, уваг, матеріалів у формі
витягів. Спочатку я пробував закінчити цей відділ, як це мені
до певної міри вдалося з першим відділом, шляхом заповнення
прогалин і розроблення уривків, що були тільки намічені, так щоб
він хоч би приблизно давав усе те, що намірявся дати автор. Щонайменше
тричі я робив таку спробу, але кожного разу зазнавав
невдачі, і в утраченому таким чином часі є одна з головних причин
запізнення. Нарешті, я переконався, що цим шляхом справа
не піде. Мені довелося б переглянути всю масу літератури з цієї
галузі, і, кінець-кінцем, я виготував би щось, що все ж не було б
книгою Маркса. Мені не лишалось нічого іншого, як в певному
розумінні розрубати Гордіїв вузол, обмежитись тим, щоб по
можливості впорядкувати наявне і зробити тільки найпотрібніші
доповнення. І таким чином я навесні 1893 року закінчив головну
роботу для цього відділу.

З окремих розділів розділи 21--24 були в найголовнішому
розроблені. Розділи 25 і 26 вимагали перегляду ілюстрацій і
включення матеріалу, який був в інших місцях. Розділи 27 і 29 можна
було дати майже цілком за рукописом, навпаки — розділ 28 довелося
місцями інакше згрупувати. Але справжня трудність почалася
з розділу 30. Починаючи звідси, треба було належним чином
впорядкувати не тільки ілюстративний матеріал, але й хід думок,
який кожної хвилини переривався вставними реченнями, відхиленнями
і~\abbr{т. д.} і розвивався далі в іншому місці, часто цілком мимохідь.
Таким чином розділ 30 склався шляхом перестановок та вилучень,
для яких знаходився вжиток в іншому місці. Розділ 31 був знову
більше розроблений у загальному зв’язку. Але далі в рукопису
йде довгий відділ, названий „Плутанина“ („Die Konfusion“), що
складається виключно з витягів з парламентських звітів про кризи
1848 і 1857~\abbr{рр.}, в яких згрупованої місцями коротко юмористично
коментовані судження двадцяти трьох дільців і письменниківекономістів,
а саме про гроші й капітал, про відплив золота, надмірну
спекуляцію і~\abbr{т. д.} Тут представлені, чи тими, що запитують,
чи тими, що відповідають, майже всі ходячі погляди того
часу на відношення між грішми і капіталом, і Маркс хотів критично
й сатирично розглянути „плутанину“, яка виявилась при цьому,
щодо того, що є на грошовому ринку гроші і що є капітал.
Після багатьох спроб я переконався, що виготовлення цього
розділу неможливе; матеріал, особливо матеріал, коментований
Марксом, я використав там, де це дозволяв зв’язок викладу.

Після цього йде в досить упорядкованому вигляді те, що я
вмістив в 32 розділі, але безпосередньо за цим — нова купа витягів
\index{iii1}{0013}  %% посилання на сторінку оригінального видання
з парламентських звітів про всякі можливі речі, зачеплені
в цьому відділі, перемішана з довшими чи коротшими увагами
автора. Наприкінці витяги та коментарії все більше й більше
концентруються коло руху грошових металів та вексельного
курсу і знов закінчуються всякого роду додатками. Навпаки, розділ
(36) „Передкапіталістичні відносини“ був цілком оброблений.

З усього цього матеріалу, починаючи з „Плутанини“, і оскільки
його не було вже вміщено в попередніх місцях, я склав
розділи 33--35. Звичайно, тут не обійшлося без значних вставок
з мого боку для встановлення зв’язку. Оскільки ці вставки
не чисто формального характеру, вони прямо позначені як мої.
Таким способом мені, нарешті, вдалося умістити в тексті всі так
чи інакше належні до справи судження автора; нічого не випущено,
крім незначної частини витягів, які або тільки повторювали
наведене вже в іншому місці, абож торкалися пунктів, яких
рукопис докладно не розглядав.

Відділ про земельну ренту був далеко повніше оброблений,
хоч і зовсім не впорядкований, як це видно вже з того, що
в 43 розділі (в рукопису кінець відділу про ренту) Маркс
вважав за потрібне коротко повторити план всього відділу. І це
було тим більш бажаним для видання, що рукопис починається
розділом 37, після якого йдуть розділи 45--47, і тільки після
цього розділи 38--44. Найбільше праці потребували таблиці
при диференціальній ренті II і те відкриття, що в 43 розділі
зовсім не був досліджений третій випадок цього роду ренти,
який треба було тут розглянути.

Для цього відділу про земельну ренту Маркс у семидесятих
роках взявся до цілком нових спеціальних досліджень. Протягом
ряду років він вивчав в оригіналах статистичні досліди
та інші видання про землеволодіння, які стали неминучими
в Росії після „реформи“ 1861 року і які йому постачали в бажаній
повноті його російські друзі, робив з них виписки і намірявся
їх використати при новому обробленні цього відділу. При
різноманітності форм як землеволодіння, так і експлуатації
землеробських виробників у Росії, у відділі про земельну ренту
Росія мала відігравати таку саму роль, як в першій книзі, при
розгляді промислової найманої праці, Англія. На жаль, Марксу
не вдалося здійснити цей план.

Нарешті, сьомий відділ був цілком закінчений у рукопису,
але тільки як перший нарис, безконечно заплутані періоди якого
спочатку треба було розчленувати, щоб зробити їх придатними
до друку. Від останнього розділу існує тільки початок. Тут
малося розглянути відповідні трьом головним формам доходу:
земельна рента, зиск, заробітна плата — три великі класи розвиненого
капіталістичного суспільства: землевласники, капіталісти,
наймані робітники — і неминуче дану з їх існуванням класову
боротьбу як фактично наявний результат капіталістичного
періоду. Подібні кінцеві резюме Маркс мав звичай відкладати
\parbreak{}  %% абзац продовжується на наступній сторінці
