\parcont{}  %% абзац починається на попередній сторінці
\index{iii1}{0292}  %% посилання на сторінку оригінального видання
них, хоч вони і робляться в формі заробітної плати, відрізняються
від змінного капіталу, витрачуваного на купівлю продуктивної
праці. Це збільшує видатки промислового капіталіста, масу
авансовуваного капіталу, не збільшуючи безпосередньо додаткової
вартості. Бо це є видатки на оплату праці, вживаної тільки
для реалізації створених уже вартостей. Як і всякі інші видатки
цього роду, вони теж зменшують норму зиску, бо зростає авансований
капітал, але не зростає додаткова вартість. Якщо додаткова
вартість $m$ лишається незмінною, а авансований капітал $К$
зростає до $K + ΔK$, то на місце норми зиску \frac{m}{K} стає менша норма
зиску \frac{m}{(K + ΔK)}. Промисловий капіталіст намагається, отже, обмежити
до мінімуму ці витрати циркуляції цілком так само, як і
свої витрати на сталий капітал. Отже, відношення промислового
капіталу до його торговельних найманих робітників не таке саме,
як до його продуктивних найманих робітників. Чим більше, при
інших однакових умовах, вживається цих останніх, тим більш
масовим є виробництво, тим більша додаткова вартість або зиск.
І навпаки. Чим більший масштаб виробництва і чим більша належна
до реалізації вартість, а тому й додаткова вартість, отже,
чим більший вироблений товарний капітал, тим більше зростають
абсолютно, хоч і не відносно, витрати на контору і дають поштовх
до певного роду поділу праці. В якій мірі зиск є передумовою
цих видатків, виявляється між іншим у тому, що з зростанням
платні торговельних працівників частина її часто виплачується
в формі процентної участі в зиску. Це є в природі
речей, що праця, яка полягає тільки в посередницьких операціях,
які зв’язані почасти з обчисленням вартостей, почасти з їх реалізацією,
почасти з зворотним перетворенням реалізованих грошей
у засоби виробництва, і розмір яких залежить, отже, від
величини вироблених і належних до реалізації вартостей, — що
така праця діє не як причина, подібно до безпосередньо продуктивної
праці, а як наслідок відповідних величин і мас цих вартостей.
Так само стоїть справа і з іншими витратами циркуляції.
Для того, щоб багато міряти, важити, упаковувати, транспортувати,
треба, щоб було багато товарів; маса праці, потрібна на
упаковування, транспортування і~\abbr{т. д.}, залежить від маси товарів,
які є об’єктами її діяльності, а не навпаки.

Торговельний робітник безпосередньо не виробляє додаткової
вартості. Але ціна його праці визначається вартістю його
робочої сили, отже, витратами її' виробництва, тимчасом як
використовування цієї робочої сили — її напруження, витрачання
й зношування, як і в усякого іншого найманого робітника, ні в
якому разі не обмежене вартістю його робочої сили. Тому його
заробітна плата не стоїть ні в якому необхідному відношенні
до маси зиску, яку він допомагає реалізувати капіталістові. Чого
він коштує капіталістові і що він йому дає, — це різні величини.
\parbreak{}  %% абзац продовжується на наступній сторінці
