\parcont{}  %% абзац починається на попередній сторінці
\index{iii1}{0229}  %% посилання на сторінку оригінального видання
окремому товарі маси праці, то маса зиску, яка припадає на кожний окремий товар, мусить зменшитись,
і це, в певних межах, навіть у тому випадку, коли норма додаткової вартості підвищується. В кожному
разі, маса зиску на весь продукт не падає нижче первісної маси зиску, якщо тільки капітал уживає те
саме число робітників, що й раніше, при тому самому ступені експлуатації. (Це може мати місце і в
тому випадку, коли вживається менше робітників при підвищеному ступені експлуатації.) Бо в тому
самому відношенні, в якому зменшується маса зиску, що припадає на окремий продукт, збільшується
число цих продуктів. Маса зиску лишається та сама, вона тільки інакше розподіляється на суму
товарів; це також нічого не змінює в розподілі між робітниками і капіталістами тієї кількості
вартості, що створена новододаною працею. Маса зиску може зростати, якщо вживається та сама маса
праці, тільки в тому випадку, коли зростає неоплачена додаткова праця, або, при незмінному ступені
експлуатації праці, в тому випадку, коли збільшується число робітників. Абож тоді, коли відбувається
і те, і друге. В усіх цих випадках — які, однак, згідно з нашим припущенням, передбачають зростання
сталого капіталу порівняно з змінним і зростаючу величину всього застосованого капіталу — кожний
окремий товар містить у собі меншу масу зиску, і норма зиску знижується, навіть якщо обчислювати її
на окремий товар; дана кількість новододаної праці виражається в більшій кількості товарів; ціна
окремого товару знижується. Якщо
розглядати справу абстрактно, то при падінні ціни окремого товару, в наслідок збільшення
продуктивної сили і, отже, при одночасному збільшенні кількості цих дешевших товарів, норма зиску
може лишитись та сама, якщо, наприклад, збільшення продуктивної сили рівномірно і одночасно впливає
на всі складові частини товарів, так що вся ціна товару падає в тій самій пропорції, в якій
збільшується продуктивність праці, і, з другого боку, взаємне відношення різних складових частин
ціни товару лишається те саме. Норма зиску могла б навіть підвищитись, коли б з підвищенням норми
додаткової вартості було зв’язане значне зменшення вартості елементів сталого і особливо основного
капіталу. Але в дійсності, як ми вже бачили, норма зиску знизиться з бігом часу. Само тільки падіння
ціни окремого товару ні в якому разі не дозволяє робити висновку щодо норми зиску. Все зводиться до
того, яка величина загальної суми капіталу, що бере участь у виробництві товару. Якщо, наприклад,
ціна одного метра тканини падає з 3\shil{ шилінгів} до 1\sfrac{2}{3}\shil{ шилінгів}; якщо відомо, що перед падінням ціни
в ній було на 1\sfrac{2}{3}\shil{ шилінгів} сталого капіталу, пряжі і~\abbr{т. д.}, \sfrac{2}{3}\shil{ шилінга} заробітної плати і \sfrac{2}{3}\shil{ шилінга} зиску, а після падіння ціни — на 1\shil{ шилінг} сталого капіталу, \sfrac{1}{3}\shil{ шилінга} заробітної плати і
\sfrac{1}{3}\shil{ шилінга} зиску, то ще невідомо, чи лишилась норма зиску та сама, чи ні. Це залежить від того, чи
збільшився і наскільки збільшився
\parbreak{}  %% абзац продовжується на наступній сторінці
