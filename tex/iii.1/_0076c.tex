\parcont{}  %% абзац починається на попередній сторінці
\index{iii1}{0076}  %% посилання на сторінку оригінального видання
яснення. Перехід від І до II включає: зменшення продуктивності
праці наполовину; опанування 100 c вимагає у II наполовину
більше праці, ніж у І. Цей випадок може трапитись у землеробстві.\footnote{
Тут у рукопису стоїть: „Дослідити пізніше, який зв’язок має цей випадок
з земельною рентою“. [— Ф. Е.]
}

Але тимчасом, як у попередньому випадку весь капітал
лишався незмінним, тому що сталий капітал перетворювався
в змінний або навпаки, тут при збільшенні змінної частини відбувається
зв’язування додаткового капіталу, а при її зменшенні —
звільнення раніш застосовуваного капіталу.

3. m' і v не змінюються, с, а тому й К, змінюються

В цьому випадку рівняння:

p' = m' v/K змінюється в p'1 = m' v/K1

і приводить при закресленні множників, які є по обох боках,
до пропорції:

p'1 : p' = К : К1;

при однаковій нормі додаткової вартості і однакових змінних
частинах капіталу норми зиску стоять у зворотному відношенні
до цілих капіталів.

Якщо ми маємо, наприклад, три капітали або три різні стани
одного й того ж капіталу:

I. 80 c + 20 v + 20 m; К = 100, m' = 100\%, р' = 20\%;

II. 100 c + 20 v + 20 m; К = 120, m' = 100\%, р' = 16 2/3\%;

III. 60 c + 20 v + 20 m; К = 80, m' = 100\%, р' = 25\%,

то виходять такі відношення:

20\% : 16 2/3\% = 120 : 100 і 20\% : 25\% = 80 : 100.

Вищенаведена загальна формула для змін v/К при незмінному
m' була:

p'1 = m' ev/EK; тепер вона стає: p'1 = m' v/EK'

бо v не зазнає ніяких змін, отже, множник е = v1/v стає тут = 1.
