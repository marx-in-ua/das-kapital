\parcont{}  %% абзац починається на попередній сторінці
\index{iii1}{0076}  %% посилання на сторінку оригінального видання
яснення. Перехід від І до II включає: зменшення продуктивності
праці наполовину; опанування $100c$ вимагає у II наполовину
більше праці, ніж у І.~Цей випадок може трапитись у землеробстві.\footnote{
Тут у рукопису стоїть: „Дослідити пізніше, який зв’язок має цей випадок
з земельною рентою“. [— Ф.~Е.]
}

Але тимчасом, як у попередньому випадку весь капітал
лишався незмінним, тому що сталий капітал перетворювався
в змінний або навпаки, тут при збільшенні змінної частини відбувається
зв’язування додаткового капіталу, а при її зменшенні —
звільнення раніш застосовуваного капіталу.
\begin{center}
\textbf{3. $m'$ і $v$ не змінюються, $c$, а тому й $К$, змінюються}
\end{center}
В цьому випадку рівняння:\[
p' \deq{} m' \frac{v}{K} \text{ змінюється в } p'1 \deq{} m' \frac{v}{K\textsubscript{1}}\]

і приводить при закресленні множників, які є по обох боках,
до пропорції:
\begin{center}
\[p'\textsubscript{1} : p' \deq{} К : К\textsubscript{1};\]
\end{center}
при однаковій нормі додаткової вартості і однакових змінних
частинах капіталу норми зиску стоять у зворотному відношенні
до цілих капіталів.

Якщо ми маємо, наприклад, три капітали або три різні стани
одного й того ж капіталу:

$\phantom{II}\text{I. } \phantom{0}80 c \dplus{} 20 v \dplus{} 20 m; К \deq{} 100, m' \deq{} 100\%, р' \deq{} 20\%;$

$\phantom{I}\text{II. } 100 c \dplus{} 20 v \dplus{} 20 m; К \deq{} 120, m' \deq{} 100\%, р' \deq{} 16\sfrac{2}{3}\%;$

$\text{III. } \phantom{0}60 c \dplus{} 20 v \dplus{} 20 m; К \deq{} \phantom{0}80, m' \deq{} 100\%, р' \deq{} 25\%,$

то виходять такі відношення:
\begin{center}
20\% : 16\sfrac{2}{3}\% \deq{} 120 : 100 \text{і} 20\% : 25\% \deq{} 80 : 100.
\end{center}
Вищенаведена загальна формула для змін $\frac{v}{К}$ при незмінному
$m'$ була:

\[p'\textsubscript{1} \deq{} m' \frac{ev}{EK}; \text{ тепер вона стає: } p'\textsubscript{1} \deq{} m' \frac{v}{EK'}\]

бо $v$ не зазнає ніяких змін, отже, множник $е \deq{} \frac{v\textsubscript{1}}{v}$ стає тут $= 1$.
