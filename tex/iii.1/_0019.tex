\parcont{}  %% абзац починається на попередній сторінці
\index{iii1}{0019}  %% посилання на сторінку оригінального видання
модифікованого вже товару; цього Фіреман, звичайно, ніяк не
може зрозуміти. Ці та інші другорядні речі, які могли б дати
привід ще до деяких заперечень, ми краще лишимо осторонь
і перейдемо зразу до суті справи. В той час як теорія навчає
автора, що додаткова вартість при даній нормі додаткової вартості
пропорціональна кількості застосованих робочих сил, досвід
показує йому, що, при даній пересічній нормі зиску, зиск
пропорціональний величині всього застосованого капіталу. Фіреман
пояснює це тим, що зиск — явище тільки умовне (в нього
це значить: належне певній суспільній формації, таке, що разом
з нею існує і зникає); його існування зв’язане просто з капіталом;
цей останній, якщо він досить сильний, щоб вимушувати
для себе зиск, примушується також конкуренцією вимушувати
для себе норму зиску, однакову для всіх капіталів. Без рівної
норми зиску неможливе ніяке капіталістичне виробництво; якщо
припустити цю форму виробництва, то маса зиску для кожного
окремого капіталіста при даній нормі зиску може залежати тільки
від величини його капіталу. З другого боку, зиск складається
з додаткової вартості, з неоплаченої праці. Яким же чином відбувається
тут перетворення додаткової вартості, величина якої
відповідає експлуатації праці, в зиск, величина якого відповідає
величині потрібного для цього капіталу? „Просто таким чином,
що в усіх галузях виробництва, де відношення між\dots{} сталим
і змінним капіталом є найбільше, товари продаються вище їх
вартості, а це значить також, що в тих галузях виробництва,
де відношення сталого капіталу до змінного капіталу $= c : v$ є
найменше, товари продаються нижче їх вартості, і що тільки
там, де відношення $c : v$ становить певну середню величину,
товари відчужуються по їх справжній вартості\dots{} Чи є ця невідповідність
окремих цін з їх відповідними вартостями спростуванням
принципу вартості? Аж ніяк. Бо в наслідок того, що
ціни деяких товарів підвищуються понад вартість в такій самій
мірі, в якій ціни інших товарів падають нижче вартості, загальна
сума цін лишається рівною загальній сумі вартостей\dots{}
в кінцевому рахунку невідповідність зникає“. Ця невідповідність
є „порушення“; „але в точних науках таке порушення, яке можна
обчислити, звичайно ніколи не розглядається як спростування
певного закону“.

Коли порівняємо з цим відповідні місця в розділі IX, то побачимо,
що Фіреман дійсно торкнувся тут вирішального пункту.
Але скільки посередніх членів потрібно було б ще й після цього
відкриття, щоб дати Фіреманові змогу виробити повне очевидне
розв’язання проблеми, — це показує незаслужено холодний
прийом, який зустріла його така важлива стаття. Хоч і як багато
людей цікавилось проблемою, але всі вони ще боялись обпектись
на ній. І це пояснюється не тільки тією недосконалою
формою, в якій Фіреман залишив своє відкриття, але й безперечними
хибами як його розуміння викладу в Маркса, так
\parbreak{}  %% абзац продовжується на наступній сторінці
