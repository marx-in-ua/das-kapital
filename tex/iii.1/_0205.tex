\parcont{}  %% абзац починається на попередній сторінці
\index{iii1}{0205}  %% посилання на сторінку оригінального видання
Рікардо не досліджує), досить тільки перевернути щойно наведені
міркування.

I.~Пересічний капітал $= 80 c \dplus{} 20 v \deq{} 100$; норма додаткової
вартості \deq{} 100\%; ціна виробництва \deq{} товарній вартості $= 80 c \dplus{}
20 v \dplus{} 20 p \deq{} 120$; норма зиску \deq{} 20\%. Нехай заробітна плата
впаде на одну чверть, тоді той самий сталий капітал приводитиметься
в рух 15-ма $v$ замість $20 v$. Отже, ми маємо товарну
вартість $ \deq{} 80 c \dplus{} 15 v \dplus{} 25 p \deq{} 120$. Кількість праці, вироблена $v$,
лишається  незмінною, і тільки створена ним нова вартість інакше
розподіляється між капіталістом і робітниками. Додаткова вартість
підвищилась з 20 до 25, і норма додаткової вартості
підвищилась з \frac{20}{25} до \frac{25}{15}, отже, з 100\% до 166\sfrac{2}{3}\%.
Зиск на 95 тепер \deq{} 25, отже, норма зиску на 100 \deq{} 26\sfrac{6}{19}. Новий
процентний склад капіталу тепер є $84\sfrac{4}{19}c \dplus{} 15\sfrac{15}{19}v \deq{} 100$.

II.~Нижчий склад. Первісно $50 c \dplus{} 50 v$, як вище. В наслідок
падіння заробітної плати на \sfrac{1}{4}, $v$ зводиться до 37\sfrac{1}{2}, і тим самим
весь авансований капітал зводиться до $50 c \dplus{} 37\sfrac{1}{2}v \deq{} 87\sfrac{1}{2}$. Якщо
ми застосуємо до цього нову норму зиску в 26\sfrac{6}{19}\%, то
$100 : 26\sfrac{6}{19} \deq{} 87\sfrac{1}{2} : 23\sfrac{1}{38}$. Та сама товарна маса,
яка раніш коштувала 120, коштує
тепер $87\sfrac{1}{2} \dplus{} 23\sfrac{1}{38} \deq{} 110\sfrac{10}{19}$; падіння ціни
майже на 8\%\footnote*{
В першому німецькому виданні тут стоїть: „майже на 10\%“. В рукопису
Маркса в цьому місці сказано: „Знижується майже на 10“, тобто дається абсолютне
число. Точно воно дорівнює 9\sfrac{9}{19} і становить 7\sfrac{7}{19}\%. \emph{Примітка ред. нім. вид. ІМЕЛ.}
}.

III.~Вищий склад. Первісно $92 c \dplus{} 8 v \deq{} 100$. Падіння заробітної
плати на \sfrac{1}{4} знижує $8 v$ до $6 v$, весь капітал до 98. Отже,
$100 : 26\sfrac{6}{19} \deq{} 98 : 25\sfrac{15}{19}$. Ціна виробництва товару,
раніш $100 \dplus{} 20 \deq{} 120$, тепер, після падіння заробітної плати, є
$98 \dplus{} 25\sfrac{15}{19} \deq{} 123\sfrac{15}{19}$;
отже, вона підвищилась більше ніж на 3\%\footnote*{
В першому німецькому виданні тут стоїть: „майже на 4\%“. В рукопису
Маркса тут так само дається абсолютне число (3\sfrac{15}{19}). В процентах воно
становить 3\sfrac{3}{19}\%. \emph{Примітка ред. нім. вид. ІМЕЛ.}
}.

Отже, ми бачимо, що досить тільки повторити попередні міркування в зворотному
напрямі і з відповідними змінами: загальне падіння заробітної плати має своїм
наслідком загальне підвищення додаткової вартості, норми додаткової вартості, а
при інших незмінних умовах і норми зиску, хоч і в
іншій пропорції; далі воно має своїм наслідком падіння цін виробництва для
товарних продуктів капіталів нижчого складу і підвищення цін виробництва для
товарних продуктів капіталів вищого складу. Результат якраз протилежний до того,
що  виявився при загальному підвищенні заробітної плати\footnote{
Надзвичайно дивно, що Рікардо (який, звичайно, застосовує іншого
методу, ніж це зроблено тут, бо не розуміє процесу вирівнення вартостей в ціни
виробництва) навіть не приходить до цієї думки, а розглядає тільки перший
випадок, підвищення заробітної плати і вплив його на ціни виробництва товарів
(„Principles etc.“, Лондон 1852, стор. 26 і далі]. A servum pecus
imitatorum [рабське стадо наслідувачів] не додумалось навіть до того, щоб
зробити це, само собою зрозуміле, по суті тавтологічне застосування.
}.
\parbreak{}  %% абзац продовжується на наступній сторінці
