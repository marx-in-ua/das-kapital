\parcont{}  %% абзац починається на попередній сторінці
\index{iii1}{0010}  %% посилання на сторінку оригінального видання
хто, як я, понад 50 років діяв у цьому русі, для того роботи,
що випливають з цього руху, є неминучим обов’язком, який
вимагає негайного виконання. Як у шістнадцятому столітті, в
наші бурхливі часи чисті теоретики у сфері суспільних інтересів
збереглися тільки на стороні реакції, і саме тому ці панове
в дійсності зовсім не теоретики, а прості апологети цієї реакції.

Та обставина, що я живу в Лондоні, веде за собою те, що
ці партійні зносини зимою відбуваються здебільшого листовно,
а влітку здебільшого особисто. А через це, як і через необхідність
стежити за ходом руху в країнах, число яких раз-у-раз
зростає, і по органах преси, число яких зростає ще швидше, для
мене стало неможливим виготовляти праці, що не допускають
ніякої перерви, інакше, як зимою, переважно в перші три місяці
року. Коли маєш за собою сімдесят років, тоді Мейнертові асоціятивні
волокна мозку працюють з якоюсь фатальною повільністю;
перерви у важкій теоретичній праці перемагаєш уже не так
легко і не так швидко, як раніш. Тим то виходило, що працю
однієї зими, коли вона не була цілком доведена до кінця, доводилось
наступної зими здебільшого проробляти знову, і це сталося
саме з найважчим п’ятим відділом.

Як побачить читач з подальших даних, ця редакційна робота
істотно відрізнялася від редакційної роботи над другою книгою.
Для третьої книги був у наявності тільки один первісний нарис,
до того ще й з величезними прогалинами. Звичайно, початок
кожного окремого відділу був досить пильно розроблений,
навіть здебільшого й стилістично округлений. Але чим далі від
початку, тим більш ескізним ставало оброблення, тим більше мало
воно прогалин, тим більше містило воно екскурсів з приводу побічних
питань, які виникали в процесі дослідження, при чому розроблення
головного питання залишалося до пізнішого часу, тим
довшими і заплутанішими ставали періоди, в яких висловлювалися
думки, записані in statu nascendi [у стані виникнення].
В багатьох місцях почерк і виклад надто виразно показують
втручання та ступневий розвиток тих викликаних надмірною
працею приступів хвороби, які спочатку все більше утрудняли
авторові самостійну працю і, нарешті, тимчасово зовсім унеможливлювали
її. І не дивно. Між 1863 і 1867~\abbr{рр.} Маркс не тільки
виготовив у нарисі дві останні книги „Капіталу“ і виготовив до
друку рукопис першої книги, але й виконав ще велетенську
роботу, зв’язану з заснуванням і поширенням Інтернаціональної
Асоціяції Робітників. Однак, через це вже в 1864 і 1865~\abbr{рр.}
виявились серйозні ознаки тих порушень в здоров’ї Маркса, що
не дали йому змоги самому закінчити обробку II і III книги.

Моя робота почалася з того, що, продиктувавши весь рукопис
з оригіналу, який часто навіть я міг розшифрувати тільки з труднощами,
я зробив легку до читання копію, що забрало в мене
чимало часу. Тільки тоді можна було почати власне редакцію.
Я обмежив її найнеобхіднішим, по можливості зберіг характер
\parbreak{}  %% абзац продовжується на наступній сторінці
