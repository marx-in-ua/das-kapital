\index{iii1}{0026}  %% посилання на сторінку оригінального видання
\nonumsection{Додаток}{.~}{Фрідріх Енґельс}

Третя книга „Капіталу“, з того часу як вона віддана на суд громадської думки, зазнає численних і
різноманітних тлумачень. Іншого не можна було й сподіватися. Коли я видавав цю книгу, моїм
завданням було перш за все виготовити по можливості аутентичний текст, передати здобуті Марксом
нові результати по можливості його ж власними словами, обмежити моє особисте втручання тільки тими випадками, де це
було абсолютно необхідно, але й у цих випадках не лишати в читача ніякого
сумніву про те, хто з ним говорить. Цей підхід до справи осуджували; висловлювалась думка, що я
повинен був перетворити залишений Марксом матеріал в систематично оброблену книгу, en faire un
livre [зробити з нього книгу], як то кажуть французи, іншими словами: пожертвувати аутентичністю
тексту для вигід читача. Але так я не розумів свого завдання. На подібне перероблення я не мав
ніякого права; така людина, як Маркс, має право бути вислуханою особисто, передати потомству
свої наукові відкриття в цілком недоторканому власному викладі. Далі, я не мав ні найменшого
бажання так безцеремонно — а я не міг інакше на це дивитись — накладати руку на спадщину такої видатної людини; я вважав би це порушенням вірності Марксові. І, по-третє, з цього не вийшло б
ніякої користі. Взагалі недоцільно витрачати будьяким способом енергію задля людей, які не можуть або не хочуть читати, які вже при появі першої книги витратили більше зусиль на те, щоб
неправильно її зрозуміти, ніж потрібно було для того, щоб зрозуміти її правильно. А для тих, хто
прагне дійсного розуміння, якраз найважливіше було мати самий оригінал; для них моя переробка
мала б у кращому разі цінність коментарія, і при тому коментарій до чогось неопублікованого і
неприступного. При першій же полеміці довелось би все ж звернутися до оригінального тексту, а
при другій і третій видання його in extenso [повністю] стало б неминучим.

А така полеміка цілком
природна, коли мова йде про твір, який дає так багато нового, і при тому в першому обробленні,
лише побіжно накресленому і почасти не без прогалин. Отут моє втручання
\parbreak{}  %% абзац продовжується на наступній сторінці
