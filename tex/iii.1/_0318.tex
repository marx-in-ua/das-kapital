
\index{iii1}{0318}  %% посилання на сторінку оригінального видання
Через те що рух купецького капіталу є $Г — Т — Г'$, то зиск
купця, поперше, одержується за допомогою актів, які відбуваються
тільки в межах процесу циркуляції, отже, одержується
у двох актах: купівлі і продажу; а, подруге, він реалізується в
останньому акті, в продажу. Отже, це зиск від продажу, profit
upon alienation. Чистий, незалежний торговельний зиск prima
facie [очевидно] є неможливим, поки продукти продаються по їх
вартостях. Дешево купувати, щоб дорого продавати, — це закон
торгівлі. Отже, не обмін еквівалентів. Поняття вартості тут
передбачається остільки, оскільки різні товари всі є вартість,
а тому гроші; якісно всі вони однаково є вирази суспільної праці.
Але вони не є вартості однакової величини. Кількісне відношення,
в якому обмінюються продукти, спочатку цілком випадкове. Вони
набирають товарної форми, тому що вони взагалі можуть обмінюватись,
тобто тому що вони є вирази одного й того ж третього.
Триваючий обмін і регулярніша репродукція для обміну все більше
й більше усувають цю випадковість. Але спочатку не для виробників
і споживачів, а для посередника між ними обома, для
купця, який порівнює грошові ціни і ріжницю кладе собі в кишеню.
Самою своєю діяльністю він установлює еквівалентність.

Спочатку торговельний капітал просто опосереднює рух між
крайніми пунктами, яких він не підпорядковує собі, і між передумовами,
яких він не створює.

Як з простої форми товарної циркуляції, $Т — Г — Т$, з’являються
гроші не тільки як міра вартості і засіб циркуляції, але й як
абсолютна форма товару, а тому й багатства, як скарб, і їх збереження
та прирощення як грошей стає самоціллю, — так з простої
форми циркуляції купецького капіталу, $Г — Т — Г'$, постають
гроші, скарб, як щось таке, що зберігається і збільшується в наслідок
простого відчуження.

Торговельні народи стародавніх часів існували як боги Епікура
в міжсвітових просторах всесвіту або, скоріше, як євреї в
порах польського суспільства. Торгівля перших самостійних,
пишно розвинених торговельних міст і торговельних народів, як
торгівля чисто посередницька, грунтувалася на варварстві народів-виробників,
для яких вони грали роль посередників.

На підготовних ступенях капіталістичного суспільства торгівля
панує над промисловістю; в сучасному суспільстві навпаки. Звичайно,
торгівля справлятиме більший чи менший вплив на ті суспільства,
між якими вона провадиться; вона все більше й більше
підпорядковуватиме виробництво міновій вартості, бо насолоди
й засоби існування вона ставить у більшу залежність від продажу,
ніж від безпосереднього споживання продукту. Цим вона розкладає
старі відносини. Вона збільшує грошову циркуляцію. Вона захоплює
вже не тільки надлишок виробництва, але помалу пожирає
саме виробництво і робить залежними від себе цілі галузі
виробництва. Однак, цей розкладаючий вплив у значній мірі
залежить від природи того суспільства, що виробляє.
