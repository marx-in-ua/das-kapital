\parcont{}  %% абзац починається на попередній сторінці
\index{iii1}{0403}  %% посилання на сторінку оригінального видання
Дисконт ні в якому разі не є тільки засіб для розширення підприємства.] „А чому він хоче одержати
владу над більшим капіталом? Тому що він хоче застосувати цей капітал; а чому він
хоче застосувати цей капітал? Тому що це для нього зисковно;
але це не було б для нього зисковно, коли б дисконт поглинав
його зиск“.

Цей самовдоволений логік припускає, що векселі дисконтуються тільки для розширення підприємства, і
що підприємство
розширюється тому, що воно зисковне. Перше припущення — хибне. Звичайний підприємець дисконтує, щоб
антиципувати грошову форму свого капіталу і тим підтримувати процес репродукції в русі; не для того,
щоб розширити підприємство або роздобути додатковий капітал, а для того, щоб урівноважити кредит,
який він дає, кредитом, який він бере. І якщо він схоче
розширити своє підприємство в кредит, то дисконт векселів
дасть йому мало користі, бо це ж тільки перетворення грошового капіталу, який уже є в його руках, з
однієї форми
в другу; він краще візьме тверду позику на довший час.
Рицар кредиту, звичайно, дисконтуватиме свої бронзові векселі,
щоб розширити своє підприємство, щоб покрити одне дуте підприємство другим; не для того, щоб
одержати зиск, а щоб одержати в своє розпорядження чужий капітал.

Ототожнивши таким чином дисконт з позикою додаткового
капіталу (а не з перетворенням у гроші готівкою векселів, які
репрезентують капітал), пан Оверстон одразу, як тільки за нього
цупко візьмешся, відступає назад. — „3730. (Запитання:) Хіба
купці не змушені, раз узявшись до справи, продовжувати протягом певного часу свої операції, не
зважаючи на тимчасове
підвищення розміру процента? — (Оверстон:) Немає ніякого
сумніву, що коли при якійсь окремій операції хто-небудь може
одержати в своє розпорядження капітал за нижчий розмір
процента замість високого розміру процента, то, беручи справу
з цієї обмеженої точки зору, це для нього приємно“. — Зате
яка необмежена є точка зору пана Оверстона, коли він раптом
під „капіталом“ розуміє тільки свій банкірський капітал і тому
вважає людину, яка дисконтує в нього векселі, за людину без
капіталу, тому що її капітал існує в товарній формі або тому,
що грошовою формою її капіталу є вексель, який пан Оверстон
перетворює в іншу грошову форму.

„3732. Чи не можете ви, у зв’язку з банковим актом 1844 року,
вказати, яке було приблизно відношення розміру процента до
золотого резерву банку; чи вірно, що коли золота в банку було
9 чи 10 мільйонів, розмір процента був 6 чи 7\%, а коли золота
було 16 мільйонів, процент стояв на рівні приблизно 3—4\%?“
[Той, що запитує, хоче примусити його пояснити розмір процента, оскільки на нього впливає кількість
золота в банку, розміром процента, оскільки на нього впливає вартість капіталу.] —
„Я не скажу, що це так\dots{} але коли б це було так, то, на мою
\parbreak{}  %% абзац продовжується на наступній сторінці
