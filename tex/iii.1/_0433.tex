\parcont{}  %% абзац починається на попередній сторінці
\index{iii1}{0433}  %% посилання на сторінку оригінального видання
свого банкового капіталу, бо він дав йому в позику свої власні
боргові вимоги.

Оскільки цей попит на грошові позики є попит на капітал,
він є попит тільки на грошовий капітал; на капітал з точки
зору банкіра, саме на золото — при відпливі золота за кордон —
або на банкноти національного банку, які приватний банк може
добути тільки купівлею їх за еквівалент, які, отже, представляють
для нього капітал. Або, нарешті, справа йде про процентні
папери, державні фонди, акції і~\abbr{т. д.}, які мусять бути
продані для того, щоб притягти до себе золото або банкноти.
Ці папери, якщо це державні папери, є капітал тільки для
того, хто їх купив, для кого вони репрезентують, отже, його
купівельну ціну, його вкладений у ці папери капітал; самі по
собі вони не є капітал, а прості боргові вимоги. Якщо це
іпотеки, то вони є простими посвідками на одержання майбутньої
земельної ренти, а якщо це будьякі інші акції, то вони
є простими титулами власності, які дають право на одержання
майбутньої додаткової вартості. Всі ці речі не є дійсний капітал,
не становлять ніякої складової частини капіталу і самі по собі
не є вартості. Подібними операціями можна також і гроші, що
належать банкові, перетворити у вклади, так що банк замість
власника стає щодо цих грошей боржником, володітиме ними
на підставі іншого титулу. Хоч як дуже це важливо для самого
банку, це зовсім не змінює кількості запасного капіталу і навіть
грошового капіталу, що є в країні. Отже, капітал фігурує тут
тільки як грошовий капітал, а якщо він наявний не в дійсній
грошовій формі, то як простий титул на капітал. Це дуже важливо,
бо рідкість \emph{банкового} капіталу і посилений попит на нього
змішують із зменшенням \emph{дійсного} капіталу, який, навпаки, в таких
випадках, у формі засобів виробництва й продуктів, є в наявності
понад міру і пригнічує ринки.

Отже, питання про те, яким чином може зростати маса цінних
паперів, які банк тримає як покриття, отже й те, як банк
може задовольняти дедалі більший попит на грошові позики при
незмінній або меншаючій загальній масі засобів циркуляції,
пояснюється дуже просто. І при тому в такі періоди грошової
скрути ця загальна маса удержується в певних межах двояким
чином: 1) в наслідок відпливу золота; 2) в наслідок попиту на
гроші як на простий засіб платежу, при чому видані банкноти
відразу припливають назад абож операція здійснюється за
допомогою кредиту, відкритого в книгах банку, без будьякої
видачі банкнот, отже, проста кредитна операція опосереднює
платежі, покриття яких було єдиною метою операції. Своєрідна
властивість грошей полягає саме в тому, що коли вони
функціонують тільки для сальдування платежів (а в часи кризи
позики беруться для того, щоб платити, а не для того, щоб
купувати; щоб закінчити попередні операції, а не починати нові), їх
циркуляція є тільки скороминуща, навіть якщо це сальдування
\parbreak{}  %% абзац продовжується на наступній сторінці
