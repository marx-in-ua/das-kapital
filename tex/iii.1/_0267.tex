\parcont{}  %% абзац починається на попередній сторінці
\index{iii1}{0267}  %% посилання на сторінку оригінального видання
тільки за допомогою цього повторного продажу, за допомогою
дворазової зміни місця того самого товару, перший купець одержує назад гроші, авансовані на купівлю
товару; тільки цим
опосереднюється повернення до нього цих грошей. В одному
випадку, випадку $Т' — Г — Т$, дворазова зміна місця тих самих грошей опосереднює те, що товар
відчужується в одній
формі і привласнюється в другій. В другому випадку, випадку
$Г — Т — Г'$, дворазова зміна місця того самого товару опосереднює те, що авансовані гроші знову
вилучаються з циркуляції.
Саме при цьому й виявляється, що товар ще не проданий остаточно, якщо він перейшов з рук виробника
до рук купця, що
цей останній тільки продовжує операцію продажу, або обслуговування функції товарного капіталу. Але
разом з тим виявляється, що те, що́ для продуктивного капіталіста є $Т — Г$,
проста функція його капіталу в його минущій формі товарного
капіталу, те для купця є $Г — Т — Г'$, особливим процесом збільшення вартості авансованого ним
грошового капіталу. Одна
фаза метаморфози товару виявляється тут, щодо купця, як
$Г — Т — Г'$, отже, як еволюція капіталу особливого роду.

Купець остаточно продає товар, в даному разі полотно,
споживачеві, однаково, чи це буде продуктивний споживач (наприклад, білільник), чи особистий, який
використовує полотно
для свого особистого споживання. В наслідок цього до купця
повертається назад авансований капітал (разом із зиском), і він
може знову почати цю операцію. Коли б при купівлі полотна
гроші функціонували тільки як платіжний засіб, так що купцеві довелося б платити тільки через шість
тижнів після одержання товару, і коли б він його продав раніше, ніж мине цей
час, то він міг би заплатити виробникові полотна за його товар,
не авансувавши особисто ніякого грошового капіталу. Коли б він
не продав товар, то він мусив би авансувати 3000\pound{ фунтів стерлінгів} при настанні строку платежу,
замість того, щоб авансувати їх відразу при здачі йому полотна; а коли б він в наслідок падіння
ринкових цін продав товар нижче купівельної ціни,
то він мусив би замістити недібрану частину з свого власного капіталу.

Що ж надає товарно-торговельному капіталові характеру
самостійно функціонуючого капіталу, тимчасом як у руках виробника, який сам продає свої товари, він,
очевидно, виступає
тільки як особлива форма його капіталу в особливій фазі процесу його репродукції, під час його
перебування в сфері циркуляції?

\emph{Поперше}: Та обставина, що товарний капітал пророблює своє
остаточне перетворення в гроші, отже, свою першу метаморфозу, виконує на ринку властиву йому qua
[як] товарному капіталові функцію, перебуваючи в руках агента, відмінного від
виробника цього товарного капіталу; і та обставина, що ця функція товарного капіталу опосереднюється
операціями купця, його
\parbreak{}  %% абзац продовжується на наступній сторінці
