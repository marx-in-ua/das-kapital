\parcont{}  %% абзац починається на попередній сторінці
\index{iii1}{0347}  %% посилання на сторінку оригінального видання
відношення, в якому цей зиск ділиться на процент і підприємницький
дохід (profits of enterprise). Це відношення залежить
від конкуренції між позикодавцями і позичальниками капіталу;
на цю конкуренцію передбачувана норма гуртового зиску
справляє вплив, але конкуренція не регулюється виключно нею\footnote{
Тому що розмір процента в цілому визначається пересічною нормою
зиску, то дуже часто надзвичайне шахрайство може бути зв’язане з низьким
розміром процента. Наприклад, залізничне шахрайство влітку 1844 року. Розмір
процента Англійського банку був підвищений до 3\% тільки 16 жовтня
1844 року.
}.
Конкуренція не регулюється виключно цим, тому що, з одного
боку, багато людей беруть позики без будь-якого наміру продуктивно
застосовувати позичене і, з другого боку, тому що величина
сукупного капіталу, який дається в позику, змінюється з зміною
багатства країни, незалежно від будьякої зміни в гуртовому
зиску“ (\emph{Ramsay}, там же, стор. 206, 207).

Щоб знайти пересічну норму процента, треба 1) обчислити
пересічний розмір процента з його змін протягом великих промислових
циклів; 2) обчислити розмір процента при таких капіталовкладеннях,
де капітал віддається в позику на довший час.

Пересічна норма процента, яка панує в якійсь країні, — в відміну
від ринкових норм, що постійно коливаються, — не може бути
визначена абсолютно ніяким законом. У цій справі не існує ніякої
природної норми процента в тому сенсі, в якому економісти
говорять про природну норму зиску і природну норму заробітної
плати. З приводу цього вже Мессі цілком слушно зауважує
таке ([„An Essay on the Governing Causes of the Natural Rate of Interest
etc.“ London 1750], стор. 49): „The only thing which any man can
be in doubt about on this occasion, is, what proportion of these profits
do of right belong to the borrower, and what to the lender; and this there
is no other method of determining than by the opinions of borrowers
and lenders in general; for right and wrong, in this respect, are
only what common consent makes so“ [„Єдине, що може викликати
сумнів у цьому випадку, — це питання про те, яка частина цих
зисків по праву належить позичальникові і яка позикодавцеві;
і для того, щоб це визначити, немає ніякого іншого способу, як
звернутися до думок позичальників і позикодавців взагалі, бо
справедливим чи несправедливим у цьому відношенні є тільки те,
що визнається таким за спільною згодою“]. Взаємне покриття
попиту й подання — при чому пересічна норма зиску припускається
даною, — не значить тут абсолютно нічого. Якщо взагалі
удаються до цієї формули (і це тоді правильно й на практиці),
вона служить формулою для знаходження основного правила,
яке незалежне від конкуренції і, навпаки, визначає її (для знаходження
регулюючих меж або граничних величин), а саме формулою
для тих, хто захоплений практикою конкуренції, її зовнішніми
виявленнями і тими уявленнями, які з цього виникають; формулою,
яка служить їм для того, щоб прийти хоча б знов таки до
\parbreak{}  %% абзац продовжується на наступній сторінці
