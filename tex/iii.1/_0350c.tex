\parcont{}  %% абзац починається на попередній сторінці
\index{iii1}{0350}  %% посилання на сторінку оригінального видання
окремій сфері підприємств\footnote{
„This rule of dividing profits is not, however, to be applied particularly to
every lender and borrower, but to lenders and borrowers in general\dots{} remarkably
great and small gain's are the reward of skill ant the want of understanding, which
lenders have nothing at all to do with; for as they will not suffer by the one,
they ought not to benefit by the other. What has been said of particular men in
the same business is applicable to particular sorts of business; if the merchants
arid tradesmen employed in any one bratlch of trade get more by what they borrow
than the common profits made by other merchants and tradesmen of the same
country, the extraordinary gain is theirs, though it required only common skill
and understanding to get it; and not the lenders, who supplied them with money\dots{}
for the lenders would not have lent their money to carry on any business or trade
upon low: er terms than would admit of paying so much as the common rate of
interest; and, therefore, they ought not to receive more than that, whatever advantage
may be made by their money“. [„Однак, це правило розподілу зисків не
може бути застосоване до кожного окремого позикодавця і позичальника, а
тільки до позикодавців і позичальників взагалі\dots{} особливо великі чи малі зиски
є винагорода за вправність або відплата за незнання справи, до чого позикодавці
взагалі не мають ніякого відношення; бо, оскільки вони нічого не втрачають
в одному випадку, вони не повинні мати вигоди в другому. Те, що сказано
про окремих людей тієї самої галузі занять, можна прикласти й до різних
родів занять; якщо купці й промисловці, які діють у якійсь галузі занять, заробляють
за допомогою позичених ними грошей більше, ніж звичайний зиск,
що його одержують інші купці й промисловців тій самій країні, то цей екстраординарний
зиск належить їм, — хоч для одержання його потрібна була тільки
звичайна вправність і знання справи, — а не позикодавцям, які забезпечили
їх грішми\dots{} бо позикодавці не позичили б своїх грошей для провадження якогонебудь
підприємства на умовах, які дозволяли б сплату процентів з норми
процента, нижчої за загальну; але через це вони не повинні одержувати більше
цієї норми, яка б вигода не одержувалась від їх грошей“] (\emph{Massie}: „An
Essay on the Governing Causes of the Natural Rate of Interest“. London 1750,
стор. 50, 51).
}. Тому загальна норма зиску в дійсності
виявляється знову як емпіричний, даний факт у пересічній
нормі процента, хоч остання не є чистим або надійним виразом
першої.

Звичайно, вірно, що сама норма процента постійно є різна
залежно від категорій забезпечень, що їх дають позичальники,
і від строку позики; але для кожної такої категорії вона
в кожний даний момент одноманітна. Отже, ця ріжниця не порушує
постійності і одноманітності розміру процента\footnote{
\begin{center}
\noindent\begin{tabular}{lc@{}r}

% Bank rate
Норма дисконту Англійського банку\dotfill{} & 5 & \% \\

%Market rate of discount, 60 days’ drafts [
Ринкова норма дисконту, векселі на 60 днів\dotfill{} & 3\sfrac{5}{8} \\

%Ditto 3 months [
\ditto{Ринкова норма дисконту,} тримісячні векселі\dotfill{} & 3\sfrac{1}{2} \\

% Ditto 6 months [
\ditto{Ринкова норма дисконту,} шестимісячні векселі\dotfill{} & 3\sfrac{5}{16} \\

% Loans to bill-brokers, day to day [
Позики вексельним маклерам, на день\dotfill{} & 1\textendash{}2 \\

%Ditto for one week [
\ditto{Позики вексельним маклерам,} на тиждень\dotfill{} & 3 \\

% Last rate for fortnight, loans to stockbrokers [
Найнижча норма на 14 днів, позики маклерам цінних паперів\dotfill{} & 4\sfrac{3}{4}\textendash{}5 \\

%Deposit allowance (banks) [
Проценти на вклади (банки)\dotfill{} & 3\sfrac{1}{2} \\

% Ditto (discount houses) [
\ditto{Проценти на вклади} (дисконтні установи)\dotfill{} & 3\textendash{}3\sfrac{1}{4}
\end{tabular}
\end{center}

\noindent{}Яка велика може бути ця ріжниця за один і той же день, показують вищенаведені
норми процента лондонського грошового ринку від 9 грудня 1889~\abbr{р.},
взяті з статті про City в „\emph{Daily News}“ від 10 грудня, Мінімум \deq{} 1\%, максимум
\deq{} 5\%. — \emph{Ф.~Е.}
}.
