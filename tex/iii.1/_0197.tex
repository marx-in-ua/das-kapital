\parcont{}  %% абзац починається на попередній сторінці
\index{iii1}{0197}  %% посилання на сторінку оригінального видання
є насамперед тільки вартість у грошовій формі. Правда, при
розгляді грошей як засобу циркуляції припускається, що відбувається
не тільки \emph{одна} метаморфоза одного товару. Навпаки,
розглядається суспільне сплетіння цих метаморфоз. Тільки
таким способом ми приходимо до обігу грошей і до розвитку їх
функції як засобу циркуляції. Але наскільки цей зв’язок важливий
для переходу грошей до їх функції засобу циркуляції та
для зміни їх форми, що випливає з цього, настільки ж він не
має значення для угод між окремими покупцями і продавцями.

Навпаки, розглядаючи подання й попит, ми бачимо, що подання
дорівнює сумі продавців або виробників певного роду
товарів, а попит дорівнює сумі покупців або споживачів (особистих
чи продуктивних) того самого роду товарів. І при тому
суми ці діють одна на одну як єдності, як агрегатні сили. Окрема
особа діє тут тільки як частина суспільної сили, як атом маси,
і саме в цій формі конкуренція виявляє \emph{суспільний} характер
виробництва й споживання.

Та з конкуруючих сторін, яка в даний момент слабша, є разом
з тим та сторона, де кожна окрема особа діє незалежно від маси
своїх конкурентів, а часто й прямо проти них, і саме цим робить
відчутною залежність одного конкурента від іншого, тимчасом
як дужча сторона завжди протистоїть своїм противникам як
більш-менш згуртована єдність. Якщо попит на певний сорт
товарів більший, ніж подання, то — в певних межах — один покупець
перебиває іншому і таким чином підвищує для всіх ціну
товару понад його ринкову вартість\footnote*{
В першому німецькому виданні стоїть: „ринкову ціну“; виправлено на
підставі рукопису Маркса. \Red{Примітка ред. нім. вид. ІМЕЛ.}
}, тимчасом як на другій
стороні продавці спільно намагаються продати товар по високій
ринковій ціні. Якщо ж, навпаки, подання більше, ніж попит,
то один починає продавати дешевше і інші мусять робити те саме,
тимчасом як покупці спільно намагаються якомога більше збити
ринкову ціну нижче ринкової вартості. Спільна сторона інтересує
кожного тільки доти, доки він більше виграє, діючи разом з нею,
ніж проти неї. Спільність дій припиняється, як тільки дана сторона
як така стає слабішою, і тоді кожна окрема особа намагається
якомога краще викрутитись власними силами. Далі,
якщо хтонебудь виробляє дешевше і може дешевше продавати,
захопити на ринку більше місця, продаючи нижче звичайної
ринкової ціни або ринкової вартості, то він так і робить; і таким
чином починається дія, яка помалу примушує і інших ввести
дешевший спосіб виробництва і яка зводить суспільно необхідну
працю до нової, меншої міри. Якщо певна сторона бере гору,
то виграє кожен, хто до неї належить; справа стоїть так, наче
всі вони здійснюють спільну монополію. Якщо ж певна сторона
є слабша, то кожний може намагатися сам своїми власними силами
стати дужчим за свого противника (наприклад, той, хто працює
\parbreak{}  %% абзац продовжується на наступній сторінці
