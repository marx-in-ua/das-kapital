\parcont{}  %% абзац починається на попередній сторінці
\index{iii1}{0322}  %% посилання на сторінку оригінального видання
становлять споконвічну інтегральну частину цієї єдності промислово-землеробського
виробництва, і таким чином розривають
общину. Навіть тут це діло розкладу вдається їм тільки дуже
повільно. Ще менше воно вдається їм у Китаї, де безпосередня
політична влада не дає допомоги. Велика економія і заощадження
часу, які виникають з безпосереднього поєднання землеробства
і мануфактури, чинять тут якнайупертіший опір продуктам
великої промисловості, в ціну яких входять faux frais [непродуктивні
витрати] процесу циркуляції, який повсюди пронизує
їх. В протилежність до англійської, російська торгівля, навпаки,
лишає непорушною економічну основу азіатського виробництва\footnote{
З того часу, як Росія робить конвульсивні зусилля, щоб розвинути
власне капіталістичне виробництво, розраховане виключно на внутрішній та
на прикордонний азіатський ринок, це теж починає мінятися. — \emph{Ф.~Е.}
}.

Перехід від феодального способу виробництва відбувається
двояким чином. Виробник стає купцем і капіталістом протилежно
до землеробського натурального господарства і зв’язаного цехами
ремесла середньовічної міської промисловості. Це — дійсно
революціонізуючий шлях. Абож купець безпосередньо підпорядковує
собі виробництво. Як би дуже не впливав цей останній
шлях історично як перехідний ступінь — як, наприклад, англійський
clothier [сукняр] XVII століття, який ставить під свій контроль
ткачів, що все ж лишаються самостійними, продає їм вовну
та скуповує в них сукно, — однак, сам по собі він не приводить
до перевороту в старому способі виробництва, який він скоріше
консервує і зберігає як свою передумову. Так, наприклад, ще
до половини цього століття фабрикант у французькій шовковій
промисловості, в англійській панчішній та мережівній промисловості
здебільшого був фабрикантом тільки номінально, в дійсності
він був простим купцем, який полишав ткачам працювати
і далі їх старим роздрібненим способом і який панував над ними
тільки як купець, на якого вони фактично працювали\footnote{
Те саме стосується і до рейнської стрічкової і позументної промисловості
та шовкоткацтва. Біля Крефельда була навіть збудована спеціальна залізниця
для зносин цих сільських ручних ткачів з міськими „фабрикантами“, але
з того часу механічне ткацтво привело до бездіяльності цієї залізниці разом
з ручними ткачами. — \emph{Ф.~Е.}
}. Такі
відносини повсюди являють собою перешкоду для дійсного капіталістичного
способу виробництва і гинуть в міру розвитку останнього.
Не роблячи перевороту в способі виробництва, вони тільки
погіршують становище безпосередніх виробників, перетворюють
їх у простих найманих робітників і пролетарів при гірших умовах,
ніж для робітників, безпосередньо підпорядкованих капіталові,
і привласнення їх додаткової праці відбувається тут на базі старого
способу виробництва. Такі самі відносини, тільки дещо модифіковані,
існують у частині лондонського виробництва меблів,
яке провадиться ремісницьким способом. В Tower Hamlets воно
провадиться дуже широко. Все виробництво поділене на дуже
\parbreak{}  %% абзац продовжується на наступній сторінці
