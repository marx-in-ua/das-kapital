\parcont{}  %% абзац починається на попередній сторінці
\index{iii1}{0216}  %% посилання на сторінку оригінального видання
тимчасом як у країнах розвиненого капіталістичного виробництва
процент виражає тільки відповідну частину виробленої додаткової
вартості або зиску. З другого боку, тут рівень процента переважно
визначається такими відносинами (позики лихварів знаті,
власникам земельної ренти), які не мають нічого спільного з
зиском, а, навпаки, показують тільки, в якій мірі лихвар привласнює
собі земельну ренту.

В країнах різного ступеня розвитку капіталістичного виробництва
і тому різного органічного складу капіталу норма
додаткової вартості (один з факторів, що визначають норму зиску)
може стояти вище в тій країні, де нормальний робочий день
коротший, ніж у тій країні, де він довший. \emph{Поперше}, якщо
англійський робочий день у 10 годин в наслідок своєї вищої
інтенсивності дорівнює австрійському робочому дневі в 14 годин,
то при однаковому розподілі робочого дня 5 годин додаткової
праці англійця можуть на світовому ринку представляти
вищу вартість, ніж 7 годин австрійця. А \emph{подруге}, в Англії
додаткову працю може становити більша частина робочого дня,
ніж в Австрії.

Закон спадаючої норми зиску, в якій виражається та сама або
навіть зростаюча норма додаткової вартості, означає, інакше
кажучи, таке: якщо взяти якусь певну кількість пересічного
суспільного капіталу, наприклад, капітал в 100, то частина
його, представлена в засобах праці, дедалі зростає, а частина,
представлена в живій праці, дедалі зменшується. Отже, через
те що вся маса живої праці, додаваної до засобів виробництва,
зменшується порівняно з вартістю цих засобів виробництва,
то порівняно з вартістю всього авансованого капіталу
зменшується також і неоплачена праця і та частина вартості,
в якій вона виражається. Або: з усього витраченого капіталу
все менша й менша частина перетворюється в живу працю,
і тому весь цей капітал вбирає порівняно з своєю величиною
все менше й менше додаткової праці, хоч одночасно з цим відношення
неоплаченої частини вживаної праці до її оплаченої частини
може зростати. Відносне зменшення змінного і збільшення
сталого капіталу, хоч обидві ці частини абсолютно зростають,
є, як ми вже сказали, тільки інший вираз зростаючої продуктивності
праці.

Припустім, що капітал в 100 складається з $80c \dplus{} 20v$, а ці
останні \deq{} 20 робітникам. Норма додаткової вартості нехай буде
100\%, тобто робітники працюють півдня на себе, півдня на капіталіста.
Нехай у другій, менш розвиненій країні капітал буде
$20c \dplus{} 80v$, і ці останні \deq{} 80 робітникам. Але цим робітникам потрібно
\sfrac{2}{3} робочого дня для себе й тільки \sfrac{1}{3} вони працюють на
капіталіста. При всіх інших однакових умовах, у першому випадку
робітники виробляють вартість в 40, у другому — в 120.
Перший капітал виробляє $80c \dplus{} 20v \dplus{} 20m \deq{} 120$; норма зиску \deq{}
20\%; другий капітал $20c \dplus{} 80v \dplus{} 40m \deq{} 140$; норма зиску
\index{iii1}{0217}  %% посилання на сторінку оригінального видання
\deq{} 40\%. Отже, в другому випадку вона вдвоє більша, ніж у першому,
хоч у першому випадку норма додаткової вартості, \deq{} 100\%,
вдвоє більша, ніж у другому випадку, де вона становить тільки
50\%. Але зате однакової величини капітал привласнює собі в першому
випадку додаткову працю тільки 20, а в другому 80 робітників.

Закон прогресуючого падіння норми зиску або відносного
зменшення привласнюваної додаткової праці порівняно з масою
упредметненої праці, яка приводиться в рух живою працею, аж
ніяк не виключає зростання абсолютної маси праці, яка приводиться
в рух і експлуатується суспільним капіталом, а тому й зростання
абсолютної маси привласнюваної ним додаткової праці; так само
цей закон не виключає того, що капітали, які є в розпорядженні
окремих капіталістів, командують дедалі більшою масою праці,
а тому й додаткової праці, — останнє навіть у тому випадку,
коли число робітників, якими вони командують, не зростає.

Якщо взяти робітниче населення даної чисельності, наприклад,
два мільйони, якщо взяти, далі, як дані, довжину і інтенсивність
пересічного робочого дня, а також заробітну плату,
а разом з тим і відношення між необхідною і додатковою працею,
то сукупна праця цих двох мільйонів, а також їх додаткова
праця, яка виражається в додатковій вартості, завжди виробляє
вартість однакової величини. Але з зростанням маси сталого
— основного і обігового — капіталу, який приводиться в рух
цією працею, падає відношення цієї величини вартості до вартості
цього капіталу, яка зростає разом з його масою, хоч і не в тій
самій пропорції. Це відношення, а тому й норма зиску, падає, хоч
капітал командує такою самою масою живої праці, як і раніше,
і вбирає таку саму масу додаткової праці. Відношення змінюється
не тому, що зменшується маса живої праці, а тому, що збільшується
маса упредметненої вже праці, яку вона приводить в рух.
Зменшення тут відносне, не абсолютне, і в дійсності нічим
не зв’язане з абсолютною величиною приведеної в рух праці
й додаткової праці. Падіння норми зиску виникає не з абсолютного,
а тільки з відносного зменшення змінної складової частини
всього капіталу, з її зменшення порівняно з сталою складовою
частиною.

Те саме, що має значення для даної маси праці і маси додаткової
праці, має значення і для зростаючого числа робітників, а тому, при
даних припущеннях, і для зростаючої маси праці, яка взагалі
є в розпорядженні, і зокрема для її неоплаченої частини, для
додаткової праці. Якщо робітниче населення зростає з двох мільйонів
до трьох, якщо змінний капітал, виплачений йому в формі
заробітної плати, так само становив раніше два мільйони, а тепер
становить три мільйони, а сталий капітал, навпаки, підвищується
з 4 до 15 мільйонів, то при даних припущеннях (незмінний
робочий день і незмінна норма додаткової вартості) маса додаткової
праці, додаткової вартості зростає наполовину, на 50\%,
\parbreak{}  %% абзац продовжується на наступній сторінці
