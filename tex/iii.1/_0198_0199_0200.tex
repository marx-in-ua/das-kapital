\parcont{}  %% абзац починається на попередній сторінці
\index{iii1}{0198}  %% посилання на сторінку оригінального видання
з меншими витратами виробництва) або принаймні з якомога меншими
втратами вийти з цього становища, і в цьому разі він уже не
турбуватиметься про своїх сусідів, хоч його дії зачіпають не тільки
його самого, але й усіх його товаришів\footnote{
„If each man of a class could never have more than a given share, or
aliquot part of the gains and possessions of the whole, he would readily combine
to raise the gains (він саме так і робить, якщо відношення між попитом і поданням
йому це дозволяє); this is monopoly. But where each man thinks that he
may any way increase the absolute amount of his own share, though by a process
which lessens the whole amount, he will often do it; this is competition“. [„Коли б
кожен член групи ніколи не міг одержати більше даної частки або відповідної
частини доходів і володінь всієї групи, то він охоче об’єднувався б з іншими,
щоб підвищити ці доходи“ (він саме так і робить, якщо відношення між попитом
і поданням йому це дозволяє); „це — монополія. Але якщо кожен думає,
що він якимсь способом може збільшити абсолютну суму своєї власної частки,
хоч би й шляхом зменшення цілої суми, то він часто саме так і робитиме; це —
конкуренція.] („An Inquiry into those principles respecting the nature of demand
etc.“, Лондон 1821, стор. 105).
}.

Попит і подання передбачають перетворення вартості в ринкову
вартість, і, оскільки це відбувається на капіталістичній
базі, оскільки товари є продукти капіталу, передбачають капіталістичний
процес виробництва, отже, відносини, які цілком інакше
переплітаються, ніж проста купівля і продаж товарів. Тут ідеться
не про формальне перетворення вартості товарів у ціну, тобто
не про просту переміну форми; тут ідеться про певні кількісні
відхилення ринкових цін від ринкових вартостей і, далі, від цін виробництва.
При простій купівлі і продажу досить, щоб товаровиробники
як такі протистояли один одному. Попит і подання при
дальшому аналізі передбачають існування різних класів і підрозділів
класів, які розподіляють між собою сукупний дохід суспільства
і споживають його як дохід, які, отже, пред’являють попит, визначуваний
цим доходом; тимчасом як, з другого боку, для розуміння
попиту і подання, що їх утворюють між собою виробники
як такі, необхідно зрозуміти всю систему капіталістичного
процесу виробництва в цілому.

При капіталістичному виробництві справа йде не тільки про те,
щоб замість маси вартості, кинутої в циркуляцію у товарній
формі, вилучити з неї рівну масу вартості в іншій формі, —
в формі грошей або в формі іншого товару, — справа йде про те,
щоб на капітал, авансований на виробництво, здобути таку саму
додаткову вартість, або зиск, як і на кожний інший капітал такої
самої величини, або pro rata [пропорціональну до] його величини,
незалежно від того, в якій галузі виробництва він застосовується;
отже, справа йде про те, щоб продати товари мінімум
по таких цінах, які дають пересічний зиск, тобто по цінах виробництва.
В цій формі капітал сам приходить до усвідомлення
себе як \emph{суспільної сили}, в якій кожний капіталіст має частину,
пропорціональну до його участі в сукупному суспільному капіталі.

Поперше, для капіталістичного виробництва самого по собі
не має значення певна споживна вартість, взагалі специфічність
\index{iii1}{0199}  %% посилання на сторінку оригінального видання
товару, який воно виробляє. В кожній сфері виробництва
мета полягає тільки в тому, щоб виробляти додаткову
вартість, привласнювати собі в продукті праці певну кількість
неоплаченої праці. І так само в природі підлеглої капіталові
найманої праці лежить те, що вона байдуже ставиться до
специфічного характеру своєї праці, мусить змінюватись відповідно
до потреб капіталу і переходити з однієї сфери виробництва
до іншої.

Подруге, кожна сфера виробництва дійсно є остільки ж добра
і остільки ж погана, як і будь-яка інша; кожна дає той самий
зиск і кожна була б безцільною, коли б вироблювані нею товари
не задовольняли будь-якої суспільної потреби.,

Але якщо товари продаються по їх вартостях, то, як це вже
показано, в різних сферах виробництва виникають дуже різні
норми зиску, залежно від різного органічного складу вкладених
у ці сфери мас капіталу. Але капітал вилучається з сфери виробництва
з нижчою нормою зиску і кидається в іншу, яка дає
вищий зиск. В наслідок цієї постійної еміграції та імміграції,
одним словом, в наслідок свого розподілу між різними сферами
виробництва, залежно від того, де норма зиску падає і де підвищується,
капітал здійснює таке відношення між попитом і поданням,
що в різних сферах виробництва пересічний зиск
стає однаковий, і тому вартості перетворюються в ціни виробництва.
Це вирівнення капіталові вдається здійснити тим повніше,
чим вищий капіталістичний розвиток в даному національному
суспільстві, тобто чим більше відносини даної країни пристосовані
до капіталістичного способу виробництва. З прогресом капіталістичного
виробництва розвиваються і умови його; воно підпорядковує
своєму специфічному характерові і своїм імманентним
законам усю сукупність суспільних передумов, в межах яких
відбувається процес виробництва.

Постійне вирівнювання постійних нерівностей відбувається
тим швидше, 1)~чим рухливіший капітал, тобто чим легше він
може бути перенесений з однієї сфери і з одного місця в інші;
2)~чим 'швидше робоча сила може бути перекинута з однієї
сфери в іншу і з одного місцевого центру виробництва до
іншого. Пункт 1-й передбачає повну свободу торгівлі всередині
суспільства і усунення всіх монополій, крім природних, особливо
тих, що виникають з самого капіталістичного способу виробництва.
Далі, передбачається розвиток кредитної системи, яка
концентрує в руках окремих капіталістів неорганізовану масу
вільного суспільного капіталу; нарешті — підпорядкування різних
сфер виробництва капіталістам. Це останнє включене вже
у припущені нами передумови, раз ми допустили, що справа
йде про перетворення вартостей у ціни виробництва в усіх капіталістично
експлуатованих сферах виробництва; однак, само
це вирівнювання наштовхується на значніші перешкоди, коли
численні і масові сфери виробництва, проваджені некапіталістично
\index{iii1}{0200}  %% посилання на сторінку оригінального видання
(наприклад, землеробство у дрібних селян), вклинюються
між капіталістичні підприємства і переплітаються з ними. Нарешті,
— велика густота населення. — Пункт 2-й передбачає скасування
всіх законів, які перешкоджають робітникам переселятись
з однієї сфери виробництва в іншу або з одного місцевого
центру виробництва до якогонебудь іншого. Байдуже ставлення
робітника до змісту його праці. Якомога більше зведення праці
в усіх сферах виробництва до простої праці. Зникнення всіх професійних
передсудів у робітників. Нарешті — і це особливо —
підпорядкування робітника капіталістичному способові виробництва.
Дальший виклад цього питання належить до спеціального
дослідження конкуренції.

Із сказаного випливає, що кожний окремий капіталіст, як
і сукупність усіх капіталістів кожної окремої сфери виробництва,
бере участь в експлуатації сукупного робітничого класу сукупним
капіталом і в ступені цієї експлуатації не тільки в силу
загальної класової симпатії, але й безпосередньо економічно,
бо — якщо припустити всі інші умови, в тому числі і вартість
сукупного авансованого сталого капіталу, даними — пересічна
норма зиску залежить від ступеня експлуатації сукупної
праці сукупним капіталом.

Пересічний зиск збігається з пересічною додатковою вартістю,
яку капітал виробляє на кожні 100 одиниць; відносно додаткової
вартості щойно сказане зрозуміле само собою. Щождо пересічного
зиску, то сюди приєднується ще як один з моментів,
які визначають норму зиску, тільки вартість авансованого капіталу.
Справді, особливий інтерес, що його має капіталіст або
капітал певної сфери виробництва в експлуатації безпосередньо
занятих ним робітників, обмежується тим, щоб за допомогою
винятково надмірної праці, або за допомогою зниження заробітної
плати нижче пересічного рівня, абож за допомогою виняткової
продуктивності вживаної праці одержати додаткову вигоду,
одержати такий зиск, що перевищує пересічний. Якщо залишити
це осторонь, то капіталіст, який зовсім не вживає у своїй сфері
виробництва змінного капіталу, отже й робітників (що в дійсності,
звичайно, неможливо), був би так само дуже заінтересований
в експлуатації робітничого класу капіталом і цілком так само
діставав би свій зиск з неоплаченої додаткової праці, як і, наприклад,
той капіталіст, який (знову таки в дійсності неможливе
припущення) вживає тільки змінний капітал, тобто витрачає весь
свій капітал на заробітну плату. Але ступінь експлуатації праці
при даному робочому дні залежить від пересічної інтенсивності
праці, а при даній інтенсивності — від довжини робочого дня.
Від ступеня експлуатації праці залежить висота норми додаткової
вартості, отже, при даній загальній масі змінного капіталу
— величина додаткової вартості, а тому й величина зиску.
Той спеціальний інтерес, що його капітал певної сфери виробництва,
в відміну від сукупного капіталу, має в експлуатації
\parbreak{}  %% абзац продовжується на наступній сторінці
