\parcont{}  %% абзац починається на попередній сторінці
\index{iii1}{0174}  %% посилання на сторінку оригінального видання
лиш остільки\footnote{
Самою собою зрозуміло, ми тут залишаємо осторонь можливість добувати
на короткий час надзиск за допомогою зниження заробітної плати, монопольних
цін і~\abbr{т. д.} [— \emph{Ф.~Е.}]
}, оскільки створена в їх галузі кількість додаткової вартості разом з іншими моментами
справляє визначальний вплив в регулюванні пересічного зиску. Але це є процес, який відбувається за
спиною капіталіста, процес, якого капіталіст
не бачить, не розуміє і який його фактично не цікавить. Отже,
дійсна ріжниця у величині між зиском і додатковою вартістю — не тільки між нормою зиску і нормою
додаткової вартості — в окремих сферах виробництва цілком приховує справжню природу й походження
зиску, і не тільки для капіталіста, який має
тут особливий інтерес помилятись, але й для робітника. З перетворенням вартостей у ціни виробництва
зникає з очей сама
основа визначення вартості. Нарешті, якщо при простому перетворенні додаткової вартості в зиск та
частина вартості товарів, яка становить зиск, протистоїть другій частині вартості як
витратам виробництва товару, так що вже тут у капіталіста
губиться поняття вартості, бо він має перед собою не всю кількість праці, якої коштує виробництво
товару, а тільки ту частину
цієї праці, яку він оплатив у формі засобів виробництва, живих
або мертвих, і таким чином зиск йому здається чимось таким,
що стоїть поза іманентною вартістю товару, — то тепер це
уявлення цілком потверджується, зміцнюється і костеніє, бо
тепер, якщо розглядати окрему сферу виробництва, зиск, який
додається до витрат виробництва, дійсно визначається не межами процесу утворення вартості, який
відбувається в самій
цій сфері, а встановлюється, навпаки, умовами, що лежать цілком поза нею.

Та обставина, що цей внутрішній зв’язок розкритий тут
уперше, що дотеперішня політична економія, як ми це побачимо
з дальшого викладу і з книги IV, або мусила абстрагуватись
від ріжниць між додатковою вартістю і зиском, між нормою
додаткової вартості і нормою зиску, щоб мати можливість зберегти визначення вартості як основу, або
ж відмовлялась від
цього визначення вартості і разом з ним від усякого ґрунту
для наукового ставлення до питання, щоб триматися цих ріжниць, які лежать на поверхні явищ і
впадають в очі, — ця плутанина в теоретиків найкраще показує, як капіталіст-практик, що
захоплений конкурентною боротьбою і ніяк не проникає в суть
її виявлень, мусить бути зовсім нездатний за зовнішньою видимістю розпізнати внутрішню суть і
внутрішню форму цього процесу.

Всі розвинені в першому відділі закони про підвищення
і падіння норми зиску в дійсності мають таке двояке значення:

1) З одного боку, вони є закони загальної норми зиску. При
наявності багатьох різних причин, які, згідно з вищевикладеним,
\parbreak{}  %% абзац продовжується на наступній сторінці
