\parcont{}  %% абзац починається на попередній сторінці
\index{iii1}{0030}  %% посилання на сторінку оригінального видання
вдається загалом і в цілому побачити в творах Маркса те, що Маркс дійсно сказав; вперше такий професор заявляє, що
критика системи Маркса не може полягати в запереченні її, — „цим нехай займаються політичні кар’єристи“, — а тільки
в дальшому
розвиненні її. Зомбарт теж, звичайно, займається нашою темою. Він досліджує питання про те, яке значення має
вартість у
системі Маркса, і приходить до таких результатів: вартість не виявляється у міновому відношенні капіталістично
вироблених товарів; вона не живе в свідомості агентів капіталістичного виробництва; вона — не емпіричний, а
мислений,
логічний факт; поняття вартості в матеріальній визначеності у Маркса є не що інше, як економічний вираз того факта,
що
суспільна продуктивна сила праці є основа господарського буття; за капіталістичного господарського ладу закон
вартості панує
над господарськими процесами в кінцевому рахунку і має для цього господарського ладу такий загальний зміст:
вартість товарів
є та специфічна і історична форма, в якій здійснюється визначальне діяння продуктивної сили праці, яка в кінцевому
рахунку
панує над усіма господарськими процесами.— Так каже Зомбарт; про це розуміння значення закону вартості для
капіталістичної
форми виробництва не можна сказати, що воно невірне. Але все ж воно здається мені сформульованим занадто широко,
воно
вимагає вужчого і точнішого формулювання; на мою думку, воно аж ніяк не вичерпує всього значення закону вартості
для тих
ступенів економічного розвитку суспільства, над якими панує цей закон.

У браунівському „Sozialpolitisches Zentralblatt“ від
25 лютого 1895 року, № 22, опублікована також чудова стаття Конрада Шмідта про III том „Капіталу“. Особливо слід відзначити
поданий у цій статті доказ того, як Марксове виведення пересічної норми зиску з додаткової вартості вперше дає відповідь на
питання, яке дотеперішніми економістами навіть ні разу не ставилось:  питання про те, яким чином визначається висота цієї
пересічної норми зиску і як стається, що вона становить, скажімо, 10 або 15\%, а не 50 або 100\%. З того часу, як ми
знаємо, що привласнена в першу чергу промисловим капіталістом додаткова вартість є єдине і виключне джерело, з якого пливе
зиск і земельна рента, питання це розв’язується само собою. Ця частина статті Шмідта написана наче спеціально для
економістів à la Лоріа, якщо це не даремна праця — відкривати очі тим, хто не хоче бачити.

Шмідт теж має свої формальні
сумніви щодо закону вартості. Він називає його науковою гіпотезою, побудованою для пояснення фактичного процесу обміну,
гіпотезою, яка навіть відносно цілком суперечних їй, як здається, явищ конкурентних цін завжди виправдує себе як необхідний
теоретичний вихідний пункт, який вносить світло в розгляд цих явищ; без закону вартості і на його думку стає неможливим
будьяке теоретичне пізнання економічного механізму капіталістичної дійсності.
\parbreak{}  %% абзац продовжується на наступній сторінці
