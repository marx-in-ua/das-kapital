\parcont{}  %% абзац починається на попередній сторінці
\index{iii1}{0264}  %% посилання на сторінку оригінального видання
випадки циркуляції товарного капіталу почасти змішують з своєрідними
функціями купецького або товарно-торговельного капіталу;
почасти вони на практиці сполучаються з його своєрідними
специфічними функціями, хоча з розвитком суспільного
поділу праці функція купецького капіталу розвивається і в чистому
вигляді, тобто відокремлюється від цих реальних функцій
і усамостійнюється щодо них. Отже, для нашої мети, де справа
йде про визначення специфічної відмінності цієї особливої форми
капіталу, слід абстрагуватись від згаданих функцій. Оскільки
капітал, який функціонує тільки в процесі циркуляції, спеціально
товарно-торговельний капітал, почасти сполучає ці функції з своїми,
він виступає не в своїй чистій формі. Відкинувши, усунувши
ці функції, ми матимемо чисту форму товарно-торговельного капіталу.

Ми бачили, що буття капіталу як товарного капіталу і та
метаморфоза, яку він, як товарний капітал, проробляє в сфері
циркуляції, на ринку, — метаморфоза, яка зводиться до купівлі
й продажу, до перетворення товарного капіталу в грошовий капітал
і грошового капіталу в товарний капітал, — становить фазу
процесу репродукції промислового капіталу, отже, фазу сукупного
процесу виробництва його; але разом з тим ми бачили, що
в цій своїй функції, як капітал циркуляції, він відрізняється
від себе самого, як продуктивного капіталу. Це — дві окремі,
відмінні форми існування одного й того ж капіталу. Частина
сукупного суспільного капіталу постійно перебуває на ринку
в цій формі існування як капітал циркуляції, в процесі цієї метаморфози,
хоч для кожного окремого капіталу його буття як
товарного капіталу і його метаморфоза як такого становить
тільки постійно зникаючий і постійно відновлюваний перехідний
пункт, перехідну стадію безперервності його процесу виробництва,
і хоч через це елементи товарного капіталу, який перебуває
на ринку, постійно змінюються, бо вони постійно витягаються
з товарного ринку і так само постійно повертаються на нього
як новий продукт процесу виробництва.

Отож, товарно-торговельний капітал є не що інше, як перетворена
форма частини цього капіталу циркуляції, який постійно
перебуває на ринку, постійно перебуває в процесі метаморфози
і постійно охоплений сферою циркуляції. Ми кажемо частини,
бо певна частина продажу й купівлі товарів постійно відбувається
безпосередньо між самими промисловими капіталістами.
Цю частину ми в цьому дослідженні залишаємо цілком осторонь,
бо вона нічого не дає для визначення поняття, для зрозуміння
специфічної природи купецького капіталу, а, з другого
боку, для нашої мети ми вичерпно дослідили її вже в книзі II.

Торговець товарами, як капіталіст взагалі, виступає на ринок
насамперед як представник певної суми грошей, яку він авансує
як капіталіст, тобто яку він хоче перетворити з $x$ (первісної
вартості цієї суми) $x \dplus{} Δx$ (цю суму плюс зиск на неї). Але
\parbreak{}  %% абзац продовжується на наступній сторінці
