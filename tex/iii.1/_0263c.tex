\index{iii1}{0263}  %% посилання на сторінку оригінального видання
\chapter{Перетворення товарного капіталу і грошового
капіталу в товарно-торговельний
капітал і в грошево-торговельний капітал
(купецький капітал)
}
\section{Товарно-торговельний капітал}

Купецький або торговельний капітал розпадається на дві
форми або підвиди: товарно-торговельний капітал і грошево-торговельний
капітал, які ми тепер докладніше охарактеризуємо,
оскільки це потрібно для аналізу капіталу в його основній структурі.
І це тим більше потрібно, що сучасна економія, навіть
в особі своїх найкращих представників, прямо звалює в одну
купу торговельний капітал з промисловим капіталом і фактично
зовсім не бачить його характерних особливостей.

\pfbreak{}

Рух товарного капіталу був аналізований в книзі II.~Якщо
розглядати сукупний капітал суспільства, то одна частина
його, яка хоч і складається завжди з інших елементів і яка навіть
змінюється в своїй величині, перебуває завжди на ринку
як товар, щоб перетворитись у гроші; друга частина перебуває
на ринку в формі грошей, щоб перетворитись у товар. Він
завжди перебуває в русі цього перетворення, цієї формальної
метаморфози. Оскільки ця функція капіталу, який перебуває
в процесі циркуляції, взагалі усамостійнюється, як особлива
функція особливого капіталу, фіксується, як функція, яка в наслідок
поділу праці належить особливому родові капіталістів,
остільки товарний капітал стає товарно-торговельним капіталом
або комерційним капіталом.

Ми вже з’ясували (книга II, розд. VI, витрати циркуляції,
2 і 3), в якій мірі можна розглядати транспортну промисловість,
зберігання і розділення товарів для надання їм форми,
в якій вони можуть бути розподілені, як процеси виробництва,
які продовжуються в процесі циркуляції. Ці епізодичні
\parbreak{}  %% абзац продовжується на наступній сторінці
