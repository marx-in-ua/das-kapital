
\index{iii1}{0025}  %% посилання на сторінку оригінального видання
Хоч який прекрасний і ясний вищенаведений обрахунок, ми
змушені, однак, поставити панові докторові Штібелінгові \emph{одно}
питання: звідки він знає, що сума додаткової вартості, яку виробляє
фабрика І, ні на волосок не відрізняється від суми додаткової
вартості, створеної на фабриці II? Про $c, v, у$ і $х$, отже,
про всі інші фактори обрахунку, він прямо каже нам, що вони
для обох фабрик мають однакові величини, але про $m$ ні слова.
Але з того, що він обидві згадувані тут кількості додаткової
вартості алгебрично позначає через $m$, це ніяк не випливає.
Навпаки, це саме те, що має бути доведене, бо пан Штібелінг
без дальших околичностей і зиск $p$ ототожнює з додатковою
вартістю. Тут можливі тільки два випадки: або обидва $m$ рівні,
кожна фабрика виробляє однакову кількість додаткової вартості,
отже, при однаковому сукупному капіталі і однакову кількість
зиску, і тоді пан Штібелінг вже наперед припустив те, що він
ще тільки повинен довести. Абож одна фабрика виробляє більшу
суму додаткової вартості, ніж друга, і тоді розвалюється весь
його обрахунок.

Пан Штібелінг не побоявся ні праці, ні витрат для того, щоб
на цій своїй помилці в обрахунку побудувати цілі гори обчислень
і подати їх публіці. Я можу дати йому заспокійливе запевнення,
що майже всі вони однаково неправильні, і що там,
де вони як виняток правильні, вони доводять щось цілком інше,
а не те, що він хоче довести. Так, порівнюючи дані американських
переписів 1870 і 1880 років, він дійсно показує падіння
норми зиску, але пояснює його цілком помилково і гадає, що
теорія Маркса про завжди незмінну, стабільну норму зиску має
бути виправлена практикою. Але з третього відділу цієї третьої
книги виходить, що ця „нерухома норма зиску“ Маркса є чиста
вигадка і що тенденція норми зиску до падіння грунтується на
причинах, діаметрально протилежних тим, що їх наводить д-р
Штібелінг. Пан д-р Штібелінг має, без сумніву, добрі наміри,
але, якщо хто хоче займатись науковими питаннями, то він мусить
насамперед навчитися читати твори, якими хоче користуватись,
так, як їх написав автор, і перш за все не вичитувати
з них того, чого в них немає.

Результат усього дослідження: і в даному питанні щось
зроблено знов таки тільки школою Маркса. Фіреман і Конрад
Шмідт, коли читатимуть цю третю книгу, можуть бути цілком
задоволені, кожний у своїй частині, з своїх власних праць.
\begin{flushright}
  \emph{Ф.~Енгельс}
\end{flushright}
Лондон, 4 жовтня 1894~\abbr{р.}
