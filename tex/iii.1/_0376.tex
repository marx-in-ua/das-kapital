
\index{iii1}{0376}  %% посилання на сторінку оригінального видання
Саме це вростання процента в грошовий капітал як у якусь
річ (яким тут здається виробництво додаткової вартості капіталом)
і притягає так дуже увагу Лютера в його наївному
галасі проти лихварства. Після того як він пояснив, що
процент можна б вимагати в тому випадку, коли невиплата
боргу в призначений строк позикодавцеві, який у свою чергу
теж мусить платити, робить йому збитки або позбавляє його
зиску, який він міг би одержати, купивши, наприклад, сад,
він продовжує: „Тому що я тобі їх (100 гульденів) позичив, ти
робиш мені подвійний збиток, бо тут я не можу заплатити,
а там Не можу купити, отже в обох випадках мушу терпіти
збиток, а це називають duplex interesse, damni emergentis et
lucri cessantis [подвійний збиток, від утрати і від припинення
зиску]\dots{} Почувши про те, що Ганс, позичивши комусь
100 гульденів, зазнав збитку і вимагає справедливого відшкодування
свого збитку, вони просто на кожні сто гульденів накидають
суму для покриття цих двох утрат, а саме від несвоєчасної
виплати боргу і від нездійсненої купівлі саду, так ніби
й справді \emph{ці дві втрати природно зрослися} з цією сотнею гульденів,
так що, маючи сто гульденів, вони пускають їх в оборот
і прираховують до них обидві ці втрати, яких вони, однак, не
зазнали\dots{} Тому ти лихвар, що грішми свого ближнього покриваєш
вигаданий тобою самим збиток, якого, однак, тобі ніхто
не зробив і якого ти не можеш ні довести, ні обчислити. Такий
збиток юристи називають non verum sed phantasticum interesse [не
справжній збиток, a вигаданий]. Збиток, що його кожний може вигадати\dots{}
Отже, не слід говорити, що могли б статися збитки від
того, що я не міг ні заплатити, ні купити. Бо це означало б ех
contingente necessarium, неіснуюче видавати за те, що повинне
бути, непевне видавати за певне. Чи не може таке лихварство за
декілька років пожерти світ?.. Позикодавця, мимо його волі, спіткало
випадкове лихо, він мусить підправити свої діла, але в торгівлі
справи йдуть не так і навіть навпаки; тоді шукають і вигадують
збитки, покладаючи їх на нужденного ближнього, щоб таким
чином нажитися й розбагатіти і в лінощах та безділлі розкішно
й розпутно жити з праці інших, без турбот, риску й збитків;
щоб я сидів за пічкою, а моя сотня гульденів здобувала для
мене в країні, і все ж, тому що це — віддані в позику гроші, був
певним, що вони повернуться в мій гаманець без будьякого
риску й турботи, — мій милий, хто б цього не схотів?“ (\emph{М.~Luther}:
„An die Pfarrherrn wider den Wucher zu predigen etc.“ Wittenberg
1540. [Luthers Werke, Wittenberg 1589, частина 6, стор. 309,
310]).

Уявлення про капітал як про вартість, яка саморепродукується
і в репродукції збільшується, як про вартість, яка вічно
зберігається і зростає в силу природженої їй властивості, —
отже, в силу скритої якості, про яку говорили схоластики, — це
уявлення привело до химерних вигадок д-ра Прайса, які лишають
\parbreak{}  %% абзац продовжується на наступній сторінці
