\parcont{}  %% абзац починається на попередній сторінці
\index{iii1}{0302}  %% посилання на сторінку оригінального видання
обороту і порівняно з масою товарів, які він обертає. Він більший і
тому, що ціна цієї товарної маси, а тому й купецький капітал,
який належить авансувати на неї, в наслідок меншої продуктивності
праці є більші, ніж при капіталістичному виробництві;
тому та сама вартість виражається в меншій масі товарів.

2)~На основі капіталістичного способу виробництва не тільки
виробляється більша маса товарів (при чому треба взяти до
уваги зменшену вартість цієї товарної маси), але одна й та сама
маса продукту, наприклад, зерна, становить більшу масу товару,
тобто дедалі більша частина її йде в торгівлю. Зрештою, в наслідок
цього зростає не тільки маса купецького капіталу, але
взагалі весь капітал, вкладуваний у циркуляцію, наприклад, у
судноплавство, залізниці, телеграф і~\abbr{т. д.}

3)~Але — і це є точка зору, виклад якої належить до „конкуренції
капіталів“ — нефункціонуючий або напівфункціонуючий
купецький капітал зростає з прогресом капіталістичного способу
виробництва, з полегшенням просування його в дрібну торгівлю,
з розвитком спекуляції і надлишку капіталу, що звільняється.

Але, якщо припустити за дану відносну величину купецького
капіталу порівняно з сукупним капіталом, ріжниця оборотів
у різних галузях торгівлі не впливає ні на величину сукупного
зиску, що припадає купецькому капіталові, ні на загальну норму
зиску. Зиск купця визначається не масою товарного капіталу,
що його він обертає, а величиною грошового капіталу, який він
авансує для опосереднення цього обороту. Якщо загальна річна
норма зиску є 15\% і якщо купець авансує 100\pound{ фунтів стерлінгів},
то, коли його капітал обертається один раз за рік, він продасть
свій товар за 115. Якщо ж його капітал обертається 5 разів за
рік, то за рік він п’ять разів продасть за 103 товарний капітал,
купівельна ціна якого є 100, отже, за цілий рік він продасть
товарний капітал в 500 за 515. Але, як і в першому випадку, це
становить на його авансований капітал в 100 річний зиск в 15.
Коли б це було не так, то купецький капітал давав би, відповідно
до числа своїх оборотів, значно вищий зиск, ніж промисловий
капітал, що суперечить законові загальної норми зиску.

Отже, число оборотів купецького капіталу в різних галузях
торгівлі безпосередньо впливає на торговельні ціни товарів. Висота
торговельної надбавки до ціни, величина відповідної частини
торговельного зиску даного капіталу, яка припадає на ціну
виробництва одиниці товару, стоїть у зворотному відношенні
до числа оборотів або до швидкості обороту купецького
капіталу в різних галузях торгівлі. Якщо купецький капітал
обертається п’ять разів за рік, то він додає до товарного капіталу
однакової вартості тільки \sfrac{1}{5} тієї надбавки, яку додає до товарного
капіталу однакової вартості другий купецький капітал, який може
обертатися тільки один раз за рік.

Вплив пересічного часу обороту капіталів в різних галузях
торгівлі на продажні ціни зводиться до того, що, відповідно до
\parbreak{}  %% абзац продовжується на наступній сторінці
