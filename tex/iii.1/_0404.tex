\parcont{}  %% абзац починається на попередній сторінці
\index{iii1}{0404}  %% посилання на сторінку оригінального видання
думку, ми мусили б ужити ще суворіших заходів, ніж ті, які
були вжиті в 1844 році; бо коли б було вірно, що чим більший
золотий запас, тим нижчий розмір процента, тоді ми мусили б,
відповідно до цього погляду на справу, взятися до роботи і підвищити золотий запас до необмеженої
суми, і тоді ми знизили б
процент до 0“. Той, що запитує, Cayley, ніскільки не збентежений цим невдалим дотепом, продовжує:
„3733. Коли б це
було так і коли припустити, що банкові були б повернуті
5 мільйонів золотом, то протягом ближчих шести місяців золотий запас досяг би приблизно 16
мільйонів, і коли припустити,
що розмір процента в наслідок цього впав би до 3--4\%, то
як можна було б тоді твердити, що падіння розміру процента
походить від великого скорочення справ? — Я сказав, що недавнє значне підвищення розміру процента, а
не падіння розміру
процента, тісно зв’язане з великим розширенням справ“. — Але
Cayley ось що казав: якщо підвищення розміру процента, разом з скороченням золотого запасу, є ознака
розширення справ,
то падіння розміру процента, разом з розширенням золотого
запасу, мусить бути ознакою скорочення справ. На це у
Оверстона не знайшлося ніякої відповіді. „3736. (Запитання:)
Здається, ви (В тексті весь час Your Lordship [ваша лордська
вельможність]) сказали, що гроші є знаряддя для одержання
капіталу“. [Це хибний погляд — вважати гроші за знаряддя; вони — форма капіталу]. „При зменшенні
золотого запасу [Англійського банку] чи не полягає велика трудність, навпаки, в тому,
що \emph{капіталісти} не можуть дістати грошей? — [Оверстон:]
Ні; це — не капіталісти, а некапіталісти шукають де добути
грошей; а чому вони шукають грошей?.. Тому що за допомогою
грошей вони здобувають владу над капіталом капіталіста, щоб
провадити справу людей, які не є капіталістами“. — Тут він
якраз пояснює, що фабриканти і купці не є капіталісти і що
капітал капіталіста, це — тільки грошовий капітал, — „3737. Хіба ж
ті особи, що виставляють векселі, не капіталісти? — Особи,
що виставляють векселі, можуть бути капіталістами, а можуть
і не бути“. Тут він добре заплутався.

Далі його запитують, чи векселі купців не репрезентують
товарів, які вони продали або відправили морем. Він заперечує те, щоб ці векселі цілком так само
репрезентували
вартість товарів, як банкноти золото (3740, 3741). Це трохи
безстидно.

„3742. Чи не є метою купця одержати гроші? — Ні; одержання грошей не є метою при виставлянні
векселя; одержання
грошей, це — мета при дисконтуванні векселів“. Виставляння
векселів, це — перетворення товару в форму кредитних грошей,
як дисконтування векселів є перетворенням цих кредитних грошей в інші, а саме в банкноти. В усякому
разі пан Оверстон тут
визнає, що мета дисконтування є одержання грошей. Раніше
він допускав дисконт не для того, щоб перетворити капітал
\parbreak{}  %% абзац продовжується на наступній сторінці
