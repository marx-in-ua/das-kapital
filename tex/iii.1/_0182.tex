
\index{iii1}{0182}  %% посилання на сторінку оригінального видання
Отже, незалежно від панування закону вартості над цінами
і рухом цін, цілком відповідає справі розглядати вартість товарів
не тільки теоретично, але й історично як prius [те, що передує]
відносно цін виробництва. Це стосується до таких економічних
відносин, коли засоби виробництва належать робітникові,
а таке є становище як у стародавньому, так і в новітньому
світі, у селянина, що працює сам і володіє землею, і в ремісника.
Це погоджується, також з тим висловленим нами раніше
поглядом\footnote{
Тоді, в 1865 році, це було тільки „поглядом“ Маркса. Тепер, після широких
досліджень первісної громади, починаючи від Маурера і кінчаючи Морганом,
це факт, навряд чи кимсь заперечуваний. — \emph{Ф.~Е.}
}, що розвиток продуктів у товари постає через обмін
між різними громадами, а не між членами однієї і тієї ж громади.
Так само як до цього первісного становища, це стосується
також і до пізніших відносин, основаних на рабстві
й кріпацтві, а також до цехової організації ремесла — поки засоби
виробництва, закріплені в кожній галузі виробництва, тільки
з труднощами можуть бути перенесені з однієї сфери в іншу,
і тому різні сфери виробництва відносяться одна до одної
до певної міри так само, як чужі країни або комуністичні
громади.

Для того, щоб ціни, по яких взаємно обмінюються товари,
приблизно відповідали їх вартостям, потрібно тільки, щоб 1)~обмін
різних товарів перестав бути чисто випадковим або лише
принагідним; 2)~щоб ці товари, оскільки ми розглядаємо безпосередній
товарообмін, вироблялися з тієї і другої сторони у відносних
кількостях, приблизно відповідних взаємній потребі в них,
що встановлюється взаємним досвідом при збуті, отже, виростає
як результат з самого триваючого обміну, і 3)~оскільки ми
говоримо про продаж, щоб ніяка природна або штучна монополія
не давала можливості сторонам-контрагентам продавати
вище вартості і не примушувала їх збувати товари нижче вартості.
Під випадковою монополією ми розуміємо монополію, яка створюється
для покупця або продавця з випадкового стану попиту
й подання.

Припущення, що товари різних сфер виробництва продаються
по їх вартостях, означає, звичайно, тільки те, що їх вартість
є центр тяжіння, навколо якого обертаються їх ціни і за яким
вирівнюються їх постійні коливання вгору і вниз. Крім того,
треба завжди відрізняти \emph{ринкову вартість} — про яку мова буде
пізніше — від індивідуальної вартості окремих товарів, які виробляються
різними виробниками. Індивідуальна вартість деяких
з цих товарів стоятиме нижче ринкової вартості (тобто для їх
виробництва потрібно менше робочого часу, ніж виражає ринкова
вартість), індивідуальна вартість інших товарів — вище
ринкової вартості. Ринкову вартість треба розглядати, з одного
боку, як пересічну вартість товарів, вироблених у певній сфері
\parbreak{}  %% абзац продовжується на наступній сторінці
