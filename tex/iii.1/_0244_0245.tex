\parcont{}  %% абзац починається на попередній сторінці
\index{iii1}{0244}  %% посилання на сторінку оригінального видання
виробництва вимагають застосування масового капіталу. Це зумовлює
також централізацію капіталу, тобто поглинення дрібних
капіталістів великими і втрату першими своїх капіталів. Це знов
таки є відокремлення, хоча тільки вторинного порядку, умов праці
від виробників, до яких ще належать ці дрібні капіталісти, бо
у них власна праця грає ще певну роль; взагалі праця капіталіста
стоїть у зворотному відношенні до величини його капіталу, тобто
до тієї міри, в якій він є капіталіст. Це є те відокремлення одне
від одного умов праці з одного боку і виробників з другого,
яке становить поняття капіталу, яке починається з первісним
нагромадженням (книга І, розд. XXIV), потім виявляється як постійний
процес в нагромадженні і концентрації капіталу і, нарешті,
виражається тут як централізація вже наявних капіталів у небагатьох
руках і втрата капіталів (так змінюється тепер експропріація)
багатьма. Цей процес швидко привів би капіталістичне
виробництво до краху, коли б постійно поряд з доцентровою
силою знов і знов децентралізаційно не діяли протидіючі тенденції.

\subsection{Конфлікт між розширенням виробництва
і зростанням вартості}

Розвиток суспільної продуктивної сили праці виявляється
двояко: поперше, у величині вже вироблених продуктивних сил,
в розмірі вартості і в розмірі маси виробничих умов, при яких
відбувається нове виробництво, і в абсолютній величині нагромадженого
вже продуктивного капіталу; подруге, у відносно
незначній величині витрачуваної на заробітну плату частини
капіталу порівняно з усім капіталом, тобто у відносно незначній
кількості живої праці, потрібної для репродукції і збільшення
вартості даного капіталу, для масового виробництва. А це передбачає
разом з тим концентрацію капіталу.

Відносно вживаної робочої сили розвиток продуктивної сили
виявляється знов таки двояко: поперше, в збільшенні додаткової
праці, тобто в скороченні необхідного робочого часу, потрібного
для репродукції робочої сили. Подруге, в зменшенні кількості
робочої сили (числа робітників), яка взагалі вживається для
того, щоб привести в рух даний капітал.

Обидва ці рухи не тільки йдуть рука в руку, але взаємно
зумовлюють один одного; обидва вони є явища, в яких
виражається один і той самий закон. Проте, вони діють на
норму зиску в протилежному напрямі. Сукупна маса зиску
дорівнює сукупній масі додаткової вартості, $\text{норма зиску} \deq{}
\frac{m}{K} \deq{} \frac{\text{додаткова вартість}}{\text{сукупний авансований капітал}}$. Але додаткова вартість, як сукупна сума,
визначається, поперше, її нормою, а подруге,
масою одночасно вживаної при цій нормі праці або, що є те
саме, величиною змінного капіталу. З одного боку, підвищується
один фактор, норма додаткової вартості; з другого боку, зменшується
\index{iii1}{0245}  %% посилання на сторінку оригінального видання
(відносно або абсолютно) другий фактор, число робітників.
Оскільки розвиток продуктивних сил зменшує оплачувану
частину вживаної праці, він підвищує додаткову вартість, підвищуючи
її норму; проте, оскільки він зменшує всю масу
праці, вживаної даним капіталом, він зменшує другий фактор,
число робітників, на яке треба помножити норму додаткової
вартості, щоб одержати її масу. Двоє робітників, які працюють
по 12 годин на день, не можуть дати такої ж маси додаткової вартості,
як 24 робітники, які працюють тільки по 2 години кожний,
навіть якби вони могли живитись самим повітрям і якби їм через це
зовсім не доводилось працювати на самих себе. Отже, в цьому
відношенні компенсація зменшеного числа робітників підвищенням
ступеня експлуатації праці має певні непереступні межі;
тому вона може, звичайно, затримати падіння норми зиску, але
вона не може його усунути.

Отже, з розвитком капіталістичного способу виробництва
норма зиску падає, тимчасом як маса його із збільшенням маси
застосовуваного капіталу підвищується. При даній нормі абсолютна
маса, на яку зростає капітал, залежить від його величини
в даний момент. Але, з другого боку, якщо цю величину дано,
то відношення, в якому він зростає, норма його зростання, залежить
від норми зиску. Безпосередньо підвищення продуктивної
сили (яке, крім того, як уже згадано, завжди йде рука в руку із
знеціненням наявного капіталу) може збільшити величину вартості
капіталу тільки в тому випадку, коли воно, підвищуючи
норму зиску, збільшує ту частину вартості річного продукту,
яка зворотно перетворюється в капітал. Оскільки мова йде про
продуктивну силу праці, це може статися тільки в тому випадку
(бо ця продуктивна сила безпосередньо не має ніякого відношення
до \emph{вартості} наявного капіталу), коли в наслідок підвищення
продуктивної сили або збільшується відносна додаткова
вартість, або зменшується вартість сталого капіталу, отже, здешевлюються
товари, які входять або в репродукцію робочої сили,
або в елементи сталого капіталу. Але і те і друге означає також
знецінення наявного капіталу; і те і друге йде рука в руку із
зменшенням змінного капіталу порівняно з сталим. І те і друге
зумовлює падіння норми зиску і уповільнює це падіння. Далі,
оскільки підвищена норма зиску спричиняє підвищений попит на
працю, вона впливає на збільшення робітничого населення і разом
з тим на збільшення матеріалу, придатного для експлуатації, який
тільки й робить капітал капіталом.

Але посередньо розвиток продуктивної сили праці сприяє
збільшенню наявної капітальної вартості, збільшуючи масу й різноманітність
споживних вартостей, в яких представлена та сама
мінова вартість і які становлять матеріальний субстрат, речові елементи
капіталу, матеріальні предмети, з яких складається сталий
капітал безпосередньо і змінний, принаймні, посередньо. З тим
самим капіталом і тією самою працею створюється більше речей,
\parbreak{}  %% абзац продовжується на наступній сторінці
