
\index{iii1}{0078}  %% посилання на сторінку оригінального видання
b) Норма зиску лишається незмінною тільки в тому разі, коли
$е \deq{} Е$, тобто коли дріб $\frac{v}{К}$ при позірній зміні зберігає те саме
значення, тобто якщо чисельник і знаменник помножуються або
діляться на те саме число. $80 c \dplus{} 20 v \dplus{} 20 m$ і $160 c \dplus{} 40 v \dplus{} 40 m$,
очевидно, мають однакову норму зиску в 20\%, бо ті лишається
\deq{} 100\%, а $\frac{v}{К} \deq{} \frac{20}{100} \deq{} \frac{40}{200}$ в обох прикладах
представляє ту саму величину.

c) Норма зиску підвищується, якщо $е$ більше за $Е$, тобто
якщо змінний капітал зростає в більшій пропорції, ніж весь капітал.
Якщо $80c \dplus{} 20v \dplus{} 20m$ стає $120 c \dplus{} 40 v \dplus{} 40 m$, то норма
зиску підвищується від 20\% до 25\%, бо, при незмінному $m'$,
$\frac{v}{К} \deq{} \frac{20}{100}$ підвищилось до \frac{40}{160}, з \sfrac{1}{5}
до \sfrac{1}{4}.

При зміні $v$ і $К$ в одному напрямі ми можемо цю зміну величин
розглядати так, ніби обидві величини змінюються до певної
межі в однаковій пропорції, так що до цієї межі $\frac{v}{К}$ лишається
незмінним. Поза цією межею стала б змінюватись тільки
одна з двох величин, і ми таким чином звели б цей складніший
випадок до одного з попередніх простіших.

Якщо, наприклад, $80 c \dplus{} 20 v \dplus{} 20 m$ переходить у
$100 c \dplus{} 30 v \dplus{} 30 m$, то при цій зміні до $100c \dplus{} 25v \dplus{} 25m$ відношення
$v$ до $c$, отже, і до $К$, лишається незмінним. Отже, до цього
пункту і норма зиску лишається незачепленою. Тому ми можемо
взяти тепер за вихідний пункт $100 c \dplus{} 25v \dplus{} 25m$; ми бачимо,
що v підвищилось на 5, до $30v$, а в наслідок цього $К$ підвищилось
від 125 до 130, і маємо таким чином перед собою другий
випадок, випадок простої зміни $v$ та спричиненої цим
зміни $К$. Норма зиску, яка спочатку була 20\%, в наслідок такої
додачі в $5v$, при попередній нормі додаткової вартості, підвищується
до $23\sfrac{1}{13}\%$.

Таке саме зведення до простішого випадку може мати місце
навіть тоді, коли $v$ і $К$ змінюють свою величину в протилежному
напрямі. Коли б ми знову виходили, наприклад, з
$80c \dplus{} 20v \dplus{} 20m$ і від цього перейшли б до форми: $110c \dplus{} 10v \dplus{} 10m$, то
при зміні до $40c \dplus{} 10v \dplus{} 10m$ норма зиску лишилася б така
сама, як і спочатку, а саме 20\%. В наслідок додачі $70c$ до цієї
перехідної форми норма зиску знижується до $8\sfrac{1}{3}\%$. Отже, цей
випадок ми знову звели до випадку зміни однієї тільки змінної,
а саме $c$.

Отже, одночасна зміна $v$, $c$ і $К$ не дає нових точок зору і в
кінцевому рахунку завжди веде назад до випадку, коли змінюється
тільки один фактор.

Навіть єдиний випадок, який ще лишається, уже фактично
вичерпаний, а саме той випадок, коли $v$ і $К$ чисельно зберігають
\parbreak{}  %% абзац продовжується на наступній сторінці
