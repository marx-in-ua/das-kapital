\parcont{}  %% абзац починається на попередній сторінці
\index{iii1}{0358}  %% посилання на сторінку оригінального видання
від додаткової вартості, але й від багатьох інших обставин: від
купівельних цін засобів виробництва, від методів виробництва,
продуктивніших, ніж пересічні, від економії сталого капіталу
і~\abbr{т. д.} І, залишаючи осторонь ціну виробництва, від особливих
коньюнктур, а при кожній окремій операції від більшої чи меншої
спритності і підприємливості капіталіста залежить, чи купує
він і продає, і в якій мірі, вище або нижче ціни виробництва,
отже, чи привласнює він собі в процесі циркуляції більшу чи
меншу частину сукупної додаткової вартості. Але в усякому разі
кількісний поділ гуртового зиску перетворюється тут в якісний,
і це тим більше, що сам кількісний поділ залежить від того, \emph{що}
належить розподілити, \emph{як} активний капіталіст господарює капіталом
і який гуртовий зиск він дає йому як функціонуючий капітал,
тобто в наслідок функцій капіталіста як активного капіталіста.
Функціонуючий капіталіст припускається тут як невласних капіталу.
Власність на капітал представлена відносно нього позикодавцем,
грошовим капіталістом. Отже, процент, який він
сплачує цьому останньому, виступає як частина гуртового зиску,
яка припадає власності на капітал як такій. Протилежно до
цього та частина зиску, яка припадає активному капіталістові,
виступає тепер як підприємницький дохід, що ніби виникає виключно
з тих операцій або функцій, які він виконує в процесі
репродукції за допомогою капіталу, отже, спеціально з тих функцій,
які він як підприємець виконує в промисловості або торгівлі.
Отже, відносно нього процент виступає просто як плід власності
на капітал, плід капіталу самого по собі, абстрагованого від процесу
репродукції капіталу, як плід капіталу, оскільки він „не працює“,
не функціонує; тимчасом як підприємницький дохід здається
йому виключно плодом тих функцій, які він виконує з капіталом,
плодом руху капіталу, руху, який здається йому тепер його
власного діяльністю протилежно до недіяльності, неучасті грошового
капіталіста в процесі виробництва. Це якісне відокремлення
одне від одного двох частин гуртового зиску, завдяки якому
процент є плід капіталу самого по собі, плід власності на капітал,
незалежно від процесу виробництва, а підприємницький
дохід — плід капіталу, що пророблює процес, що діє в процесі
виробництва, і тому плід тієї активної ролі, яку застосовних
капіталу грає в процесі репродукції, — це якісне відокремлення
зовсім не є просто суб’єктивне уявлення грошового капіталіста,
з одного боку, і промислового капіталіста, з другого. Воно
основане на об’єктивному факті, бо процент припливає до грошового
капіталіста, до позикодавця, який є просто власником
капіталу, отже, просто представником власності на капітал до
процесу виробництва і поза процесом виробництва; а підприємницький
дохід припливає просто до функціонуючого капіталіста,
який є невласних капіталу.

Таким чином, як для промислового капіталіста, оскільки він
працює з узятим в позику капіталом, так і для грошового капіталіста,
\index{iii1}{0359}  %% посилання на сторінку оригінального видання
оскільки він не сам застосовує свій капітал, простий
кількісний поділ гуртового зиску між двома різними особами,
які мають різні юридичні титули на той самий капітал, а тому
й на вироблений ним зиск, обертається в якісний поділ. Одна
частина зиску виступає тепер як плід капіталу самого по собі
в \emph{одному} визначенні його, як процент; друга частина виступає
як специфічний плід капіталу в протилежному визначенні, і
тому як підприємницький дохід; одна — як плід виключно власності
на капітал, друга — як плід виключно функціонування
з цим капіталом, як плід капіталу, що пророблює процес,
або як плід тих функцій, які виконує активний капіталіст. І це
скостеніння і усамостійнення обох частин гуртового зиску одної
проти одної, як коли б вони походили з двох істотно різних
джерел, мусить тепер встановитись для всього класу капіталістів
і для сукупного капіталу. І при тому однаково, чи застосовуваний
активним капіталістом капітал взято в позику, чи ні, або
чи належний грошовому капіталістові капітал застосовується
ним самим, чи ні. Зиск від усякого капіталу, отже й пересічний
зиск, оснований на вирівненні капіталів між собою, розпадається
або може бути розкладений на дві якісно різні, одна відносно
одної самостійні і одна від одної незалежні частини, процент
і підприємницький дохід, які обидві — і та й друга — визначаються
особливими законами. Капіталіст, який працює власним
капіталом, так само як і той, що працює капіталом,
взятим у позику, ділить свій гуртовий зиск на процент, який
припадає йому як власникові, як тому, хто самому собі позичив
свій власний капітал, і на підприємницький дохід, який припадає
йому як активному, функціонуючому капіталістові. Таким чином
для цього поділу, як поділу якісного, не має значення, чи повинен
капіталіст дійсно поділитися з іншим капіталістом, чи ні.
Застосовник капіталу, навіть коли він працює власним капіталом,
розпадається на дві особи — на простого власника капіталу
і на застосовника капіталу; сам його капітал щодо категорій
зиску, які він дає, розпадається на капітал-\emph{власність}, капітал
\emph{поза} процесом виробництва, капітал, що сам по собі дає процент,
і на капітал у процесі виробництва, який як капітал, що
пророблює процес, дає підприємницький дохід.

Отже, процент закріплюється таким чином, що він тепер виступає
не як байдужий для виробництва поділ гуртового зиску,
який має місце тільки принагідно, саме коли промисловець працює
чужим капіталом. Навіть коли він працює власним капіталом,
його зиск розпадається на процент і підприємницький
дохід. Тим самим просто кількісний поділ стає якісним; він має
місце незалежно від тієї випадкової обставини, чи є промисловець
власник, чи невласник свого капіталу. Це не тільки частини
зиску, розподілювані між різними особами, але дві різні
категорії зиску, які стоять у різному відношенні до капіталу,
отже, мають відношення до різних визначеностей капіталу.
