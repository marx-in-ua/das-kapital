\parcont{}  %% абзац починається на попередній сторінці
\index{iii1}{0370}  %% посилання на сторінку оригінального видання
щоб оплачувати такого управителя (manager), хоч це далеко
ще не приводить наших промислових капіталістів до того, щоб
„займатися державними справами або філософією“.

Що не промислові капіталісти, а промислові managers є „душа
нашої промислової системи“, — це відзначив уже пан Юр\footnote{
\emph{A.~Ure}: „Philosophy of Manufactures“, франц. перекл. 1836, 1, стор. 67,
де цей Піндар фабрикантів разом з тим показує їм, що більшість з них не
мають ні найменшого уявлення про механізм, який вони застосовують.
}.
Щождо торговельної частини підприємства, то все потрібне
щодо цього сказане вже в попередньому відділі.

Само капіталістичне виробництво привело до того, що працю
верховного керівництва, цілком відокремлену від власності на капітал,
завжди можна знайти на улиці. Тому стало некорисним, щоб ця
праця верховного керівництва виконувалась капіталістом. Немає
ніякої потреби в тому, щоб капельмейстер був власником інструментів
оркестру; так само його функція як дирижера не зв’язана і з тим,
щоб у нього було щонебудь спільного з „платою“ всіх інших музикантів.
Кооперативні фабрики дають доказ того, що капіталіст, як
активний учасник виробництва, став так само зайвим; як він сам,
у своєму вищому розвитку, вважає зайвим великого землевласника.
Оскільки праця капіталіста не випливає з процесу виробництва
як тільки капіталістичного, отже, оскільки ця праця не
зникає сама собою разом з капіталом; оскільки вона не обмежується
функцією експлуатації чужої праці; отже, оскільки вона
випливає з форми праці як суспільної, з комбінації і кооперації
багатьох для досягнення спільного результату, — вона є цілком так
само незалежна від капіталу, як незалежна від нього сама ця форма,
якщо тільки вона розірвала (gesprengt hat) капіталістичну оболонку.
Казати, що ця праця є необхідна, як капіталістична праця,
як функція капіталіста, значить тільки, що vulgus [вульґарний
економіст] не може уявити собі тих форм, які розвинулися
в надрах капіталістичного способу виробництва, відокремленими
і звільненими від їх антагоністичного капіталістичного характеру.
Відносно грошового капіталіста промисловий капіталіст
є робітник, але робітник як капіталіст, тобто як експлуататор
чужої праці. Плата, яку він вимагає і одержує за цю працю,
точно дорівнює привласненій кількості чужої праці і безпосередньо
залежить, оскільки він бере на себе необхідні турботи по
експлуатації, від ступеня експлуатації цієї праці, а не від ступеня
того напруження, якого йому коштує ця експлуатація і яке він може
за помірну плату скинути з себе на управителя. В англійських
фабричних округах після кожної кризи можна побачити достатню
кількість екс-фабрикантів, які за невелику плату управляють
своїми колишніми власними фабриками як директори, у нових власників,
часто своїх кредиторів\footnote{
В одному відомому мені випадку фабрикант, що збанкрутував, став після
кризи 1868 року платним найманим робітником своїх колишніх робітників. Після
його банкрутства фабрика перейшла в руки товариства робітників, і колишній
її власник був настановлений управителем. — \emph{Ф.~Е.}
}.
