
\index{iii1}{0286}  %% посилання на сторінку оригінального видання
З одного боку, такий торговельний робітник є такий самий
найманий робітник, як і всякий інший. Поперше, оскільки його
праця купується на змінний капітал купця, а не на ті гроші, що
витрачаються як дохід; отже, оскільки вона купується не для
особистих послуг, а з метою самозростання вартості авансованого
на це капіталу. Подруге, оскільки вартість його робочої
сили, а тому і його заробітна плата, визначається, як і в усіх
інших найманих робітників, витратами виробництва і репродукції
його специфічної робочої сили, а не продуктом його праці.

Але між ним і робітниками, безпосередньо вживаними промисловим
капіталом, мусить існувати така сама ріжниця, яка існує
між промисловим капіталом і торговельним капіталом, а тому й
між промисловим капіталістом і купцем. Через те що купець, як
простий агент циркуляції, не виробляє ні вартості, ні додаткової
вартості (бо та добавна вартість, яку він додає до товарів своїми
витратами, зводиться до додання вартостей, які вже раніш існували,
хоч тут нав’язується питання: яким чином він удержує, зберігає
цю вартість свого сталого капіталу?), то й торговельні робітники,
вживані ним для виконання тих самих функцій, не можуть
безпосередньо створювати для нього додаткову вартість. Тут, як
і тоді, коли справа йшла про продуктивних робітників, ми припускаємо,
що заробітна плата визначається вартістю робочої сили,
отже, купець не збагачується відрахуваннями з заробітної плати,
так що в обрахунок своїх витрат він вносить не таке авансування
на працю, яке оплачувало б її тільки почасти, — іншими словами,
він збагачується не тим, що обшахровує своїх прикажчиків і~\abbr{т. п.}

Труднощі при вивченні питання про торговельних найманих
робітників полягають зовсім не в тому, щоб пояснити, яким чином
вони виробляють зиск безпосередньо для свого наймача, хоч
безпосередньо вони не виробляють додаткової вартості (а зиск
є тільки перетворена форма її). Це питання в дійсності розв’язане
вже загальним аналізом торговельного зиску. Подібно до
того, як промисловий капітал одержує зиск в наслідок того,
що продає вміщену в товарах і реалізовану працю, за яку він
не заплатив ніякого еквіваленту, цілком так само і торговельний
капітал одержує зиск в наслідок того, що він оплачує продуктивному
капіталові не всю неоплачену працю, яка міститься
в товарі (в товарі, оскільки капітал, витрачений на його виробництво,
функціонує як відповідна частина сукупного промислового
капіталу); навпаки, при продажу товарів він примушує
заплатити собі за цю неоплачену ним частину праці, яка ще
міститься в товарах. Відношення купецького капіталу до додаткової
вартості інше, ніж відношення промислового капіталу.
Останній виробляє додаткову вартість шляхом безпосереднього
привласнювання неоплаченої чужої праці. Перший привласнює
собі частину цієї додаткової вартості, примушуючи промисловий
капітал відступати йому цю частину.

Тільки за допомогою своєї функції реалізації вартостей торговельний
\index{iii1}{0287}  %% посилання на сторінку оригінального видання
капітал функціонує в процесі репродукції як капітал,
і тому як функціонуючий капітал одержує частину з виробленої
сукупним капіталом додаткової вартості. Для кожного окремого
купця маса його зиску залежить від маси капіталу, яку він
може вжити в цьому процесі, а він тим більше може вжити
з неї на купівлю й продаж, чим більша є неоплачена праця його
прикажчиків. Саму функцію, в силу якої його гроші є капітал,
торговельний капіталіст примушує здебільшого виконувати своїх
робітників. Неоплачена праця його прикажчиків, хоч вона й не
створює додаткової вартості, створює однак йому привласнення
додаткової вартості, що своїм результатом є для цього капіталу
цілком те саме; отже, ця неоплачена праця є для нього
джерелом зиску. Інакше торговельне підприємство ніколи не
можна було б вести у великому масштабі, ніколи не можна
було б вести по-капіталістичному.

Подібно до того, як неоплачена праця робітника безпосередньо
створює для продуктивного капіталу додаткову вартість, цілком
так само неоплачена праця торговельних найманих робітників створює
для торговельного капіталу участь в цій додатковій вартості.

Трудність полягає ось у чому: оскільки робочий час і праця
самого купця не є вартостетворча праця, хоч і створює йому
участь у виробленій вже додатковій вартості, то як стоїть справа
з тим змінним капіталом, який він витрачає на купівлю торговельної
робочої сили? Чи слід цей змінний капітал прирахувати
як витрати до авансованого купецького капіталу? Якщо ні, то це,
видимо, суперечить законові вирівнення норми зиску; який капіталіст
авансовував би 150, коли б він міг рахувати як авансований
капітал тільки 100? Якщо ж слід, то це, видимо, суперечить
сутності торговельного капіталу, бо цей сорт капіталу
функціонує як капітал не в наслідок того, що він подібно до
промислового капіталу приводить в рух чужу працю, а в наслідок
того, що він сам працює, тобто виконує функції купівлі й продажу,
і саме тільки за це і цим переносить на себе частину
додаткової вартості, створеної промисловим капіталом.

(Отже, нам треба дослідити такі пункти: змінний капітал купця;
закон необхідної праці в циркуляції; яким чином праця купця
зберігає вартість його сталого капіталу; роль купецького капіталу
в сукупному процесі репродукції; нарешті, роздвоєння на
товарний капітал і грошовий капітал — з одного боку, і на товарно-торговельний
капітал і грошово-торговельний капітал —
з другого боку.)

Якби кожний купець мав лиш стільки капіталу, скільки він
міг би обертати особисто своєю власною працею, то мало б
місце безконечне роздрібнення купецького капіталу; це роздрібнення
мусило б зростати в міру того, як продуктивний капітал,
з розвитком капіталістичного способу виробництва, виробляє
в дедалі більшому масштабі і оперує дедалі більшими масами.
Отже, ми мали б зростаючу невідповідність між тим і другим.
\parbreak{}  %% абзац продовжується на наступній сторінці
