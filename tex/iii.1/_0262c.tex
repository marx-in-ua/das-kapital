\parcont{}  %% абзац починається на попередній сторінці
\index{iii1}{0262}  %% посилання на сторінку оригінального видання
і зв’язаної з цим легкості перетворювати гроші в капітал без того,
щоб самому ставати промисловим капіталістом. Поп’яте, в наслідок
зростання потреб і жадоби збагачення. Пошосте, в наслідок
зростання маси вкладуваного основного капіталу і~\abbr{т. д.}

\pfbreak{}

Три головні факти капіталістичного виробництва:

1)~Концентрація засобів виробництва в небагатьох руках,
в наслідок чого вони перестають бути власністю безпосередніх
робітників і перетворюються, навпаки, в суспільні сили виробництва.
Хоча такими вони стають спочатку як приватна власність
капіталістів. Ці останні є trustees [довірені] буржуазного суспільства,
але вони кладуть у свою кишеню всі плоди цього довірення.

2)~Організація самої праці як суспільної праці: за допомогою
кооперації, поділу праці і сполучення праці з природознавством.

Як з того, так і з другого боку капіталістичний спосіб виробництва
знищує приватну власність і приватну працю, хоч
знищує в антагоністичних формах.

3)~Утворення світового ринку.

Величезна порівняно з населенням продуктивна сила, яка розвивається
при капіталістичному способі виробництва, і зростання
— хоч і не в тій самій пропорції — капітальних вартостей
(не тільки їх матеріального субстрату), які зростають далеко
швидше, ніж населення, суперечать базі, яка порівняно з зростаючим
багатством стає дедалі вужчою і для якої діє ця величезна
продуктивна сила, і відносинам зростання вартості цього
дедалі наростаючого капіталу. Звідси кризи.

