\parcont{}  %% абзац починається на попередній сторінці
\index{iii1}{0230}  %% посилання на сторінку оригінального видання
весь авансований капітал і на скільки більше метрів він виробляє за даний час.

Те випливаюче з природи капіталістичного способу виробництва явище, що при зростаючій продуктивності
праці ціна окремого товару або даної кількості товарів знижується, число товарів збільшується, маса
зиску на окремий товар і норма зиску на суму товарів знижується, а маса зиску на всю суму товарів
підвищується, — це явище, при поверховому розгляді, виражає тільки падіння маси зиску, яка припадає
на окремий товар, падіння його ціни, зростання маси зиску на збільшене загальне
число товарів, які виробляє сукупний капітал суспільства абож і окремий капіталіст. Це далі
сприймається так, ніби капіталіст з доброї волі бере менше зиску на окремому товарі, але відшкодовує
себе більшою кількістю товарів, які він виробляє. Таке розуміння базується на уявленні зиску, який
виникає з продажу (profit upon alienation), — уявленні, яке, з свого боку, знов таки абстраговане з
способу розуміння, властивого купецькому капіталові.

Раніше, в четвертому й сьомому відділах першої книги, ми бачили, що маса товарів, яка зростає разом
з продуктивною силою праці, і здешевлення окремого товару, як таке (оскільки ці товари не входять як
визначальні фактори в ціну робочої сили), не зачіпають відношення оплаченої і неоплаченої праці в
окремому товарі, не зважаючи на зниження ціни.

Через те що в конкуренції все виступає у фальшивому, в перекрученому вигляді, то окремий капіталіст
може уявляти: 1)~що він зменшує свій зиск на окремий товар, знижуючи його ціну, але одержує більший
зиск в наслідок збільшення маси товарів, які він продає; 2)~що він установлює ціну окремого товару і
за допомогою множення визначає ціну сукупного продукту, тимчасом як первісний процес є ділення (див.
книгу І, розд. X), а множення є правильним тільки в другу чергу, при умові такого ділення.
Вульгарний економіст в дійсності не робить нічого іншого, як тільки перекладає дивовижні уявлення
капіталістів, захоплених конкуренцією, на позірно більш теоретичну, узагальнюючу мову, і мучиться
над тим, щоб сконструювати правильність цих уявлень.

В дійсності падіння товарних цін і зростання маси зиску на вирослу масу здешевлених товарів є тільки
інший вираз закону падіння норми зиску при одночасному збільшенні маси зиску.

Дослідження того, наскільки падаюча норма зиску може збігатися із зростаючими цінами, так само мало
стосується сюди, як і той пункт, який ми розглянули раніш, книга I, X, при розгляді відносної
додаткової вартості. Капіталіст, який застосовує поліпшені способи виробництва, що не стали ще,
однак, загальнопоширеними, продає нижче ринкової ціни, але вище своєї індивідуальної ціни
виробництва; таким чином норма зиску для нього зростає, поки її не вирівняє конкуренція; поки
\parbreak{}  %% абзац продовжується на наступній сторінці
