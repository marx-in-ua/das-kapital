\parcont{}  %% абзац починається на попередній сторінці
\index{iii1}{0168}  %% посилання на сторінку оригінального видання
сама дорівнює витратам виробництва плюс додаткова вартість,
отже, в даному разі дорівнює витратам виробництва плюс зиск,
а цей зиск знов таки може бути більший або менший, ніж додаткова вартість, місце якої він заступає.
Щодо змінного капіталу, то хоч пересічна денна заробітна плата завжди дорівнює вартості, виробленій
за те число годин яке робітник
мусить працювати, щоб виробити необхідні засоби існування,
однак саме число цих годин знов таки фальсифікується в наслідок того, що ціни виробництва необхідних
засобів існування
відхиляються від їх вартостей. Однак, це завжди розв’язується
таким чином, що наскільки в один товар входить більше додаткової вартості, настільки її в другий
товар входить менше,
і тому ті відхилення від вартості, які містяться в цінах виробництва товарів, взаємно знищуються.
Взагалі в цілому капіталістичному виробництві загальний закон здійснюється завжди
тільки як панівна тенденція, дуже заплутаним і приблизним
способом, тільки як якась пересічна вічних коливань, яка ніколи
не може бути точно встановлена.

Через те що загальна норма зиску утворюється з пересічної
різних норм зиску на кожні 100 авансованого капіталу за певний
період часу, скажімо, за рік, то в ній стирається також ріжниця,
викликана ріжницею в часі оборотів різних капіталів. Але ці
ріжниці є визначальним фактором для тих різних норм зиску
різних сфер виробництва, що з їх пересічної утворюється загальна норма зиску.

В попередній ілюстрації утворення загальної норми зиску
кожний капітал у кожній сфері виробництва припускався \deq{} 100,
і це було зроблено саме для того, щоб з’ясувати процентну
ріжницю в нормах зиску, а тому й ріжницю у вартостях товарів,
вироблюваних рівновеликими капіталами. Але само собою зрозуміло: дійсні маси додаткової вартості,
створювані в кожній
окремій сфері виробництва, залежать від величини застосованих
капіталів, бо в кожній такій даній сфері виробництва склад капіталу є даний. Тимчасом особлива \emph{норма}
зиску кожної окремої
сфери виробництва не змінюється від того, чи застосовується
капітал в 100, $100 × m$ чи $100 × xm$. Норма зиску однаково лишається 10\% — чи становить весь зиск $10 :
100$, чи $1000 : \num{10000}$.

Але через те що норми зиску в різних сферах виробництва
є різні, бо в них залежно від відношення змінного капіталу до
всього капіталу виробляються дуже різні маси додаткової вартості, отже й зиску, то очевидно, що
пересічний зиск на кожні
100 суспільного капіталу, отже, пересічна норма зиску або загальна норма зиску, буде дуже різна
залежно від відповідної
величини капіталів, вкладених у різні сфери виробництва. Візьмімо чотири капітали $А$, $В$, $C$, $D$. Нехай
норма додаткової вартості для всіх них буде \deq{} 100\%. Нехай на кожні 100 сукупного
капіталу змінного капіталу буде для $А \deq{} 25$, для $B \deq{} 40$, для
$C \deq{} 15$, для $D \deq{} 10$. На кожні 100 сукупного капіталу тоді припадало
\index{iii1}{0169}  %% посилання на сторінку оригінального видання
б додаткової вартості або зиску у $А \deq{} 25$, $B \deq{} 40$, $C \deq{} 15$, $D \deq{} 10$; разом $= 90$; отже, якщо всі
чотири капітали є рівновеликі, пересічна норма зиску є \frac{90}{4} \deq{} 22\sfrac{1}{2}\%.

Але якщо загальні величини цих капіталів будуть: $А \deq{} 200$, $B \deq{} 300$, $C \deq{} 1000$, $D \deq{} 4000$, то вироблені
зиски будуть відповідно 50, 120, 150 і 400. Разом на 5500 капіталу зиску буде 720, або пересічна
норма зиску буде 13\sfrac{1}{11}\%.

Маси всієї виробленої вартості є різні залежно від різних загальних величин відповідних капіталів,
авансованих в $А$, $В$, $C$, $D$.
Тому при утворенні загальної норми зиску справа йде не тільки
про ріжницю \emph{норм} зиску в різних сферах виробництва, з яких
просто треба було б вивести пересічну, але й про відносну вагу,
з якою ці різні норми зиску входять в утворення пересічної.
Але це залежить від відносної величини капіталу, вкладеного
в кожну окрему сферу виробництва, тобто від того, яку частину
сукупного суспільного капіталу становить капітал, вкладений
в кожну окрему сферу виробництва. Дуже велика ріжниця мусить, звичайно, бути залежно від того, чи
більша чи менша
частина сукупного капіталу дає вищу або нижчу норму зиску.
Але це знов таки залежить від того, скільки капіталу вкладено
в ті сфери виробництва, в яких відношення змінного капіталу
до всього капіталу є високе або низьке. Тут справа стоїть цілком
так само, як з пересічним процентом, що його одержує лихвар,
який віддає в позику різні капітали за різні норми процента, наприклад, за 4, 5, 6, 7\% і~\abbr{т. д.}
Пересічна норма цілком залежить
від того, скільки з свого капіталу він позичив за кожну з цих
різних норм процента.

Отже, загальна норма зиску визначається двома факторами:

1) органічним складом капіталів у різних сферах виробництва,
отже, різними нормами зиску в окремих сферах;

2) розподілом сукупного суспільного капіталу між цими різними сферами, отже, відносною величиною
капіталу, вкладеного
в кожну окрему сферу, і отже з окремою нормою зиску, тобто
відносною масою сукупного суспільного капіталу, яку поглинає
кожна окрема сфера виробництва.

В I і II книгах ми мали справу тільки з \emph{вартостями} товарів.
Тепер, з одного боку, відокремились \emph{витрати виробництва}, як
частина цієї вартості, з другого боку, розвинулась \emph{ціна виробництва} товару, як перетворена форма
вартості товару.

Припустім, що склад пересічного суспільного капіталу є
$80 c \dplus{} 20 v$, а норма річної додаткової вартості $m' \deq{} 100\%$; тоді
річний пересічний зиск для капіталу в 100 буде $= 20$, а загальна
річна норма зиску $= 20\%$. Хоч би які були $k$, витрати виробництва
товарів, вироблених за рік капіталом в 100, їх ціна виробництва була б $= k \dplus{} 20$. В сферах
виробництва, де склад капіталу $= (80 — x) c \dplus{} (20 x) v$, дійсно створена додаткова вартість,
відповідно річний зиск, вироблений у цій сфер і виробництва,
\index{iii1}{0170}  %% посилання на сторінку оригінального видання
був би $= 20 \dplus{} x$, отже, більший ніж 20, і вироблена
товарна вартість була б $= k \dplus{} 20 \dplus{} x$, більша ніж $k \dplus{} 20$, або
більша, ніж ціна виробництва. У сферах виробництва, в яких
склад капіталу $(80 \dplus{} x) c \dplus{} (20 — x) v$, створювана протягом року
додаткова вартість, або зиск, була б $= 20 — x$, отже, менша, ніж 20,
а тому товарна вартість $k \dplus{} 20 — x$ була б менша, ніж ціна виробництва, яка $= k \dplus{} 20$. Якщо залишити
осторонь можливі ріжниці в часі оборотів, то ціна виробництва товарів дорівнювала б їх вартості
тільки в тих сферах, в яких склад капіталу випадково був би $= 80 c \dplus{} 20 v$.

В кожній окремій сфері виробництва специфічний розвиток
суспільної продуктивної сили праці є різний щодо ступеня,
вищий чи нижчий, відповідно до того, наскільки велика є кількість засобів виробництва, що їх
приводить в рух певна кількість праці, тобто, при даному робочому дні, певне число робітників; отже,
він вищий чи нижчий, відповідно до того, наскільки
мала є кількість праці, потрібна для певної кількості засобів
виробництва. Тому капітали, які містять у собі більший процент
сталого, отже, менший процент змінного капіталу, ніж пересічний суспільний капітал, ми звемо
капіталами \emph{вищого} складу.
Навпаки, такі капітали, в яких сталий капітал займає відносно
менше, а змінний відносно більше місце, ніж у пересічному
суспільному капіталі, ми звемо капіталами \emph{нижчого} складу.
Нарешті, ми звемо капіталами пересічного складу такі капітали,
склад яких збігається з складом пересічного суспільного капіталу. Якщо пересічний суспільний капітал
в процентах складається з $80 c \dplus{} 20 v$, то капітал $90 c \dplus{} 10 v$ стоїть \emph{вище}, а капітал $70 c \dplus{} 30 v$
\emph{нижче}, ніж пересічний суспільний. Взагалі, при
складі пересічного суспільного капіталу, рівному $mc \dplus{} nv$, де
$m$ і $n$ є сталі величини і $m \dplus{} n \deq{} 100$, $(m \dplus{} x) c \dplus{} (n — x) v$ репрезентує вищий, а $(m — x) c \dplus{} (n \dplus{} x)
v$ — нижчий склад окремого капіталу або групи капіталів. Як функціонують ці капітали після
встановлення пересічної норми зиску, — припускаючи, що
вони обертаються один раз за рік, — це показує нижченаведена
таблиця, в якій I представляє пересічний склад, і тому пересічна норма зиску \deq{} 20\%.

\begin{center}
\begin{tabular}{c c c c}

\toprule
Капітал & Норма зиску & Ціна продукту & Вартість \\
\midrule

\phantom{II}I. $80 c \dplus{} 20 v \dplus{} 20 m$ & 20\% & 120 &  120 \\

\phantom{I}II.    $90 c \dplus{} 10 v \dplus{} 10 m$ & 20\% & 120 & 110\\

III. $70 c \dplus{} 30 v \dplus{} 30 m$ & 20\% & 120 & 130 \\
\end{tabular}
\end{center}
Отже, для товарів, вироблених капіталом II, їхня вартість була б
менша, ніж їхня ціна виробництва, для товарів капіталу III ціна
виробництва була б менша, ніж вартість, і тільки для капіталів I,
\parbreak{}  %% абзац продовжується на наступній сторінці
