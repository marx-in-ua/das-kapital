\parcont{}  %% абзац починається на попередній сторінці
\index{iii1}{0321}  %% посилання на сторінку оригінального видання
колоніальна система, — все це істотно сприяло зруйнуванню
(Sprengung) феодальних рамок виробництва. Тимчасом сучасний
спосіб виробництва у своєму першому періоді, мануфактурному
періоді, розвивався тільки там, де умови для цього створилися
протягом середньовіччя. Досить, наприклад, порівняти Голландію
з Португалією\footnote{
Яке переважне значення в розвитку Голландії, крім інших обставин,
мала база, закладена в її рибальстві, мануфактурі й землеробстві, це вже показано
письменниками XVIII століття. Див., наприклад, Мессі. — В протилежність
до попередніх поглядів, які недооцінювали розміри й значення азіатської, античної
і середньовічної торгівлі, стало модою надзвичайно переоцінювати їх.
Від цього уявлення найкраще можна вилікуватись, якщо розглянути англійський
експорт і імпорт на початок XVIII століття і порівняти з сучасним. І все ж
він був незрівняно більший, ніж у будь-якого торговельного народу попереднього
часу (див. \emph{Anderson}: History of Commerce [London 1764, том II, стор. 261
і далі]).
}. І якщо в XVI і почасти ще в XVII столітті
раптове розширення торгівлі і створення нового світового ринку
справили переважний вплив на занепад старого і на піднесення
капіталістичного способу виробництва, то це сталося, навпаки,
на базі уже створеного капіталістичного способу виробництва.
Світовий ринок сам становить базу цього способу виробництва.
З другого боку, імманентна цьому способові виробництва необхідність
виробляти в дедалі більшому масштабі жене до постійного
розширення світового ринку, так що при цьому не торгівля
революціонізує промисловість, а промисловість постійно революціонізує
торгівлю. І торговельне панування тепер зв’язане
з більшим чи меншим переважанням умов великої промисловості.
Досить порівняти, наприклад, Англію і Голландію. Історія занепаду
Голландії, як пануючої торговельної нації, є історія підпорядкування
торговельного капіталу промисловому капіталові.
Ті перешкоди, які внутрішня стійкість та устрій докапіталістичних
національних способів виробництва ставлять розкладаючому
впливові торгівлі, яскраво виявляються в зносинах англійців
з Індією та Китаєм. Широку базу способу виробництва
становить тут єдність дрібного землеробства з домашньою промисловістю,
при чому в Індії до цього долучається ще форма
сільських общин, які ґрунтуються на общинному землеволодінні,
форма, яка, зрештою, була первісною формою і в Китаї.
В Індії англійці, як володарі і привласнювачі земельної ренти,
негайно вжили своєї безпосередньої політичної і економічної
влади для того, щоб зруйнувати (sprengen) ці дрібні економічні
общини\footnote{
Якщо історія якогонебудь народу являє собою історію невдалих і дійсно
безглуздих (на практиці ганебних) економічних експериментів, так це господарювання
англійців в Індії. В Бенгалії вони створили карикатуру англійського
великого землеволодіння; в південно-східній Індії — карикатуру парцельної
власності; на північному заході вони перетворили, оскільки це від них залежало,
індійську економічну общину з общинним землеволодінням у карикатуру її
самої.
}. Їх торгівля впливає тут революціонізуючим чином
на спосіб виробництва лиш остільки, оскільки вони низькою
ціною своїх товарів знищують прядільництво й ткацтво, які
\parbreak{}  %% абзац продовжується на наступній сторінці
