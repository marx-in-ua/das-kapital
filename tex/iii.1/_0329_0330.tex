
\index{iii1}{0329}  %% посилання на сторінку оригінального видання
Як тільки продуктивний капітал перетворився в товарний капітал,
він мусить бути кинутий на ринок, проданий як товар.
Тут він функціонує просто як товар. Капіталіст виступає тут
тільки як продавець товару, подібно до того як покупець —
тільки як покупець товару. Як товар продукт мусить у процесі
циркуляції, через свій продаж, реалізувати свою вартість, набрати
свого перетвореного вигляду, вигляду грошей. В наслідок
цього цілком байдуже також, чи купується цей товар споживачем
як засоби існування, чи капіталістом як засоби виробництва,
як складова частина капіталу. В акті циркуляції товарний
капітал функціонує тільки як товар, не як капітал. Це — товарний \emph{капітал} у відміну від простого
товару: 1)~тому що він
уже вагітний додатковою вартістю, отже, реалізація його вартості
є разом з тим реалізація додаткової вартості; але це нічого не
змінює в тому, що він існує просто як товар, як продукт певної
ціни; 2)~тому що ця його функція як товару є момент процесу
репродукції його як капіталу, і тому його рух як товару,
будучи тільки частиною пророблюваного ним руху, разом з тим
є його рухом як капіталу; але вона стає такою не в наслідок
самого акту продажу, а тільки в наслідок зв’язку цього акту
з сукупним рухом цієї певної суми вартості як капіталу.

Як грошовий капітал, він фактично так само діє просто тільки
як гроші, тобто як засіб купівлі товарів (елементів виробництва).
Що ці гроші тут є разом з тим грошовим капіталом, формою капіталу,
це випливає не з акту купівлі, не з дійсної функції, яку
він виконує тут як гроші, а із зв’язку цього акту з сукупним
рухом капіталу, бо цей акт, який він виконує як гроші, є вступ
до капіталістичного процесу виробництва.

Але оскільки товарний капітал і грошовий капітал дійсно
функціонують, дійсно грають свою роль у процесі, товарний капітал
діє тут тільки як товар, грошовий капітал — тільки як гроші.
Ні в одному з окремих моментів метаморфози, розглядуваних самі
по собі, капіталіст не продає покупцеві товар як \emph{капітал}, хоч
для нього товар представляє капітал, і не відчужує продавцеві
грошей як капітал. В обох випадках він відчужує товар
просто як товар і гроші просто як гроші, як засіб купівлі
товару.

Капітал виступає в процесі циркуляції як капітал тільки в
загальному зв’язку всього процесу, в тому моменті, в якому вихідна
точка являє собою разом з тим точку повернення назад, в $Г — Г'$ або $Т — T'$ (тимчасом як у процесі
виробництва він виступає як
капітал в наслідок підпорядкування робітника капіталістові і в
наслідок виробництва додаткової вартості). Але в цьому моменті
повернення до вихідної точки опосереднююча ланка зникла.
Що тут наявне, так це $Г'$ або $Г \dplus{} ΔГ$ (байдуже, чи існує тепер
сума вартості, збільшена на $ΔГ$, у формі грошей, чи товару,
чи елементів виробництва), грошова сума, рівна первісній авансованій
грошовій сумі \dplus{} певний надлишок понад неї, реалізована
\index{iii1}{0330}  %% посилання на сторінку оригінального видання
додаткова вартість. І якраз у цій точці повернення, в якій
капітал існує як реалізований капітал, як вартість, що зросла,
в цій формі капітал — оскільки ця точка фіксується як точка
спокою, уявна чи дійсна — ніколи не входить в циркуляцію,
а, навпаки, є вилученим з циркуляції, як результат усього процесу.
Якщо він знову витрачається, то він ніколи не відчужується
третій особі \emph{як капітал}, а продається їй як простий
товар, або віддається їй як прості гроші за товар. Він ніколи
не виступає в процесі своєї циркуляції як капітал, а тільки як
товар або гроші, і це є тут його єдина форма буття \emph{для інших}.
Товар і гроші є тут тільки капіталом не остільки, оскільки
товар перетворюється в гроші, а гроші в товар, не в їх дійсних
відношеннях до покупця або продавця, а тільки в їх ідеальних
відношеннях до самого капіталіста (з суб’єктивної точки зору),
або як моменти процесу репродукції (з об’єктивної точки зору).
В дійсному русі капітал існує як капітал не в процесі циркуляції,
а тільки в процесі виробництва, в процесі експлуатації робочої
сили.

Але інакше стоїть справа з капіталом, що дає процент, і якраз
це становить його специфічний характер. Власник грошей,
який хоче використати свої гроші як капітал, що дає процент,
відчужує їх третій особі, кидає їх у циркуляцію, робить їх товаром
\emph{як капітал}; не тільки як капітал для нього самого, але
й для інших; вони не тільки капітал для того, хто їх відчужує,
але й третій особі вони з самого початку передаються як
капітал, як вартість, яка має споживну вартість створювати
додаткову вартість, зиск; як вартість, яка в русі зберігається
і після свого функціонування повертається до того, хто її первісно
витратив, в даному разі до власника грошей; отже, тільки
на якийсь час віддалюється від нього, тимчасово переходить з
володіння свого власника у володіння функціонуючого капіталіста,
тобто вона не виплачується і не продається, а тільки віддається
в позику; вона тільки відчужується під умовою, що
після певного строку вона, поперше, повернеться до своєї вихідної
точки, і повернеться, подруге, як реалізований капітал,
реалізувавши ту свою споживну вартість, що вона виробляє додаткову
вартість.

Товар, який віддається в позику як капітал, віддається в позику,
залежно від його властивостей, або як основний, або як
обіговий капітал. Гроші можуть віддаватися в позику в обох
формах — як основний капітал, наприклад, тоді, коли вони сплачуються
назад у формі пожиттьової ренти, так що разом з процентами
завжди повертається назад і частина капіталу. Деякі
товари, відповідно до природи їх споживної вартості, можуть
віддаватися в позику тільки як основний капітал, наприклад,
будинки, судна, машини і~\abbr{т. д.} Але всякий відданий у позику капітал,
яка б не була його форма і як би не модифікувалась його зворотна
сплата природою його споживної вартості, завжди є тільки
\parbreak{}  %% абзац продовжується на наступній сторінці
