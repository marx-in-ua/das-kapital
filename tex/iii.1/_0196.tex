\parcont{}  %% абзац починається на попередній сторінці
\index{iii1}{0196}  %% посилання на сторінку оригінального видання
під „попитом“ і „природною ціною“ те, що ми досі розуміли під
цим, покликаючись на А.~Сміта, завжди мусить бути відношенням
рівності, бо тільки тоді, коли подання дорівнює дійсному
попитові, тобто попитові, який не хоче платити ні більше,
ні менше природної ціни, — тільки тоді дійсно сплачується природна
ціна; отже, в різний час той самий товар може мати дві
дуже різні природні ціни, і все ж відношення між поданням
і попитом, в обох випадках може бути однаковим, а саме
відношенням рівності“.] Отже, тут допускається, що при двох
різних natural prices [природних цінах] одного й того самого
товару в різний час попит і подання кожного разу можуть взаємно
покриватись і мусять покриватись для того, щоб товар
в обох випадках був проданий по його natural price. Але через
те що в обох випадках немає ніякої ріжниці у відношенні між
попитом і поданням, але є ріжниця у величині самої natural
price, то ця остання, очевидно, визначається незалежно від попиту
й подання і, отже, менш за все може бути ними визначена.

Для того, щоб товар продавався по його ринковій вартості,
тобто пропорціонально до вміщеної в ньому суспільно-необхідної
праці, сукупна кількість суспільної праці, вживана для
виробництва сукупної маси цього роду товарів, мусить відповідати
величині суспільної потреби в цих товарах, тобто платоспроможної
суспільної потреби. Конкуренція, коливання ринкових
цін, які відповідають коливанням відношення між попитом
і поданням, постійно намагаються звести до цієї міри сукупну
кількість праці, вжитої на кожний рід товарів.

У відношенні між попитом і поданням товарів повторюється,
поперше, відношення між споживною вартістю і міновою вартістю,
між товаром і грішми, між покупцем і продавцем; подруге,
відношення між виробником і споживачем, хоч обидва
вони можуть бути представлені третіми особами, торговцями.
При дослідженні відношення між покупцем і продавцем досить
протиставити їх, кожного окремо, один одному, щоб розвинути
це відношення. Трьох осіб досить для повної метаморфози
товару і, отже, для процесу продажу-купівлі, взятого в цілому.
$А$ перетворює свій товар у гроші $В$, якому він продає товар,
і знову перетворює свої гроші в товар, який він купує на ці
гроші в $C$; весь процес відбувається між ними трьома. Далі:
при дослідженні грошей ми припускали, що товари продаються
по їх вартості, бо не було ніякої підстави розглядати ціни, що
відхиляються від вартості, оскільки йшлося тільки про ті зміни
форми, які пророблює товар, стаючи грішми і знову перетворюючись
з грошей у товар. Раз товар взагалі продається і на
виручені гроші купується новий товар, то ми маємо перед
собою цілу метаморфозу, і для неї як такої однаково, чи стоїть
ціна товару нижче чи вище його вартості. Вартість товару зберігає
своє значення як основа, бо тільки з цієї основи можуть бути
раціонально виведені гроші, і ціна за своїм загальним поняттям
\parbreak{}  %% абзац продовжується на наступній сторінці
