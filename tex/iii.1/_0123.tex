\parcont{}  %% абзац починається на попередній сторінці
\index{iii1}{0123}  %% посилання на сторінку оригінального видання
зрозумілі самі собою. Вони значать не що інше, як те, що наявний
капітал в наслідок будьяких загальних економічних обставин
— бо мова тут іде не про особливу долю якогось приватного
капіталу — збільшується або зменшується в своїй вартості;
отже, що вартість авансованого на виробництво капіталу
підвищується або падає, незалежно від зростання його вартості
в наслідок ужитої ним додаткової праці.

Під зв’язуванням капіталу ми розуміємо той випадок, коли
певні частини сукупної вартості продукту знову мусять бути
перетворені в елементи сталого або змінного капіталу для
того, щоб виробництво і далі провадилось у своєму попередньому
масштабі. Під звільненням капіталу ми розуміємо той
випадок, коли частина сукупної вартості продукту, яка досі
мусила знову перетворюватись у сталий або змінний капітал,
стає вільною і надлишковою, якщо виробництво має і далі провадитися
в попередніх розмірах. Це звільнення або зв’язування
капіталу є відмінне від звільнення або зв’язування доходу. Якщо,
наприклад, річна додаткова вартість для капіталу $К$ дорівнює
$х$, то в наслідок здешевлення товарів, які входять у споживання
капіталістів, $х — а$ може вистачити для того, щоб задовольнити
ту саму масу потреб і~\abbr{т. д.}, що й раніше. Отже,
звільняється певна частина доходу \deq{} $а$, яка тепер може служити
або для збільшення споживання, або для зворотного перетворення
в капітал (для нагромадження). Навпаки: якщо потрібно
$х \dplus{} а$ для того, щоб і далі вести попереднє життя, то або це
останнє мусить зазнати обмеження в задоволенні потреб, або
та частина доходу \deq{} $а$, яка раніш нагромаджувалась, мусить бути
витрачена тепер як дохід.

Підвищення і зниження вартості може торкнутися сталого
або змінного капіталу, чи обох разом, і щодо сталого капіталу
воно знов таки може торкнутися основної або обігової його
частини, чи обох разом.

В сталому капіталі треба розглянути: сировинні й допоміжні
матеріали, до яких належать і півфабрикати, — все це ми тут
охоплюємо назвою сировинні матеріали, — і машини та інший
основний капітал.

Вище ми розглядали саме зміни ціни — відповідно, вартості —
сировинного матеріалу з точки зору їх впливу на норму зиску
і встановили той загальний закон, що при інших однакових умовах
норма зиску стоїть у зворотному відношенні до висоти вартості
сировинного матеріалу. І це безумовно правильно для
капіталу, який заново вкладається у підприємство, отже для
таких випадків, коли капіталовкладення, перетворення грошей
у продуктивний капітал, відбувається вперше.

Але незалежно від цього нововкладеного капіталу велика
частина капіталу, який уже функціонує, перебуває в сфері
циркуляції, в той час як друга частина перебуває у сфері виробництва.
Одна частина існує як товар на ринку і мусить бути
\parbreak{}  %% абзац продовжується на наступній сторінці
