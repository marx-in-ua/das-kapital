\parcont{}  %% абзац починається на попередній сторінці
\index{iii1}{0202}  %% посилання на сторінку оригінального видання
умовах. Виключаючи взагалі випадки криз і перепродукції, це
стосується до всіх ринкових цін, як би дуже вони не відхилялись
від ринкових вартостей або ринкових цін виробництва.
Саме ринкова ціна передбачає, що за товари того самого роду
сплачується однакова ціна, не зважаючи на те, що ці товари
можуть бути вироблені при дуже різних індивідуальних умовах
і тому можуть мати дуже різні витрати виробництва. (Про надзиски
як наслідки монополій у звичайному розумінні слова, штучних
чи природних, ми тут не говоримо.)

Але, крім того, надзиск може виникнути ще в тому випадку,
коли певні сфери виробництва перебувають в такому стані, що
вони можуть ухилитися від перетворення їх товарних вартостей
у ціни виробництва, а тому й від зведення їх зисків до пересічного
зиску. У відділі про земельну ренту ми розглянемо
дальший розвиток цих двох форм надзиску.

\section{Впливи загальних коливань заробітної плати
на ціни виробництва}
\chaptermark{Впливи загальних коливань зар. плати
на ціни виробництва}%

Припустім, що пересічний склад суспільного капіталу є
$80 c \dplus{} 20 v$, а зиск — 20\%. В цьому випадку норма додаткової
вартості є 100\%. Загальне підвищення заробітної плати, якщо
припустити всі інші умови однаковими, означає зниження норми
додаткової вартості. Для пересічного капіталу зиск і додаткова
вартість збігаються. Припустім, що заробітна плата підвищується
на 25\%. Та сама маса праці, яку привести в рух коштувало 20,
коштує тепер 25. Отже, ми маємо в цьому випадку, замість
$80 c \dplus{} 20 v \dplus{} 20 p$, за один оборот вартість у $80 c \dplus{} 25 v \dplus{} 15 p$.
Праця, приведена в рух змінним капіталом, як і раніш, виробляє
суму вартості в 40. Якщо $v$ підвищується з 20 до 25, то надлишок
$m$ або $p$ є вже тільки \deq{} 15. Зиск в 15 на 105 \deq{} 14\sfrac{2}{7}\%,
і це було б новою нормою пересічного зиску. Через те що
ціна виробництва товарів, вироблюваних пересічним капіталом,
збігається з їх вартістю, ціна виробництва цих товарів не змінилася
б; тому підвищення заробітної плати привело б, правда,
до зниження зиску, але не привело б до зміни вартості й ціни
товарів.

Раніше, коли пересічний зиск був \deq{} 20\%, ціна виробництва
товарів, вироблених за один період обороту, дорівнювала їх
витратам виробництва плюс зиск в 20\% на ці витрати виробництва,
отже \deq{} $k \dplus{} kp' \deq{} k \dplus{} \frac{20k}{200}$, де $k$ є змінна величина, різна
залежно від вартості засобів виробництва, що входять у товари,
і від розміру того зношування, яке основний капітал, застосований
у виробництві цих товарів, віддає продуктові. Тепер
ціна виробництва становила б $k \dplus{} \frac{14\sfrac{2}{7} k }{100}$.
