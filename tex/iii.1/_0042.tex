\parcont{}  %% абзац починається на попередній сторінці
\index{iii1}{0042}  %% посилання на сторінку оригінального видання
загальну норму, і таким чином знову відбирає у піонерів промисловості привласнювану ними понад пересічний рівень додаткову
вартість,— це ніяк не являє собою теоретичних труднощів. А на практиці тим більше, бо сфери виробництва з надлишковою
додатковою вартістю, отже з високим змінним і низьким сталим капіталом, тобто з низьким складом капіталу, по самій своїй
природі підпорядковуються капіталістичному виробництву найпізніше і найменш повно; це насамперед землеробство. Що ж
торкається, навпаки, підвищення цін виробництва понад товарні вартості, яке необхідне, щоб підняти до рівня пересічної норми
зиску недостатню додаткову вартість, яка міститься в продуктах сфер з високим складом капіталу, то теоретично це виглядає
надзвичайно важким, але на практиці, як ми бачили, воно відбувається найлегше і найскоріше. Бо товари цього класу, коли вони
починають вироблятись капіталістично і надходять у капіталістичну торгівлю, вступають у конкуренцію з товарами такого ж
роду, виробленими докапіталістичними методами і через те дорожчими. Отже, капіталістичний виробник, навіть зрікаючись
частини додаткової вартості, все ж може виручати звичайну для його місцевості норму зиску, яка первісно не мала
безпосереднього відношення до додаткової вартості, бо вона поставала з торговельного капіталу ще задовго до того, як взагалі
почалось капіталістичне виробництво, отже, задовго до того, як стала можлива промислова норма зиску.

II. БІРЖА *

1. З третього тома, п’ятий відділ, особливо 27 розділ, ясно, яке місце взагалі займає біржа в капіталістичному виробництві. Однак, з 1865 року, коли була написана книга, сталася зміна, яка нині надає біржі підвищеної і дедалі ростущої ролі і яка у
своєму дальшому розвитку має тенденцію концентрувати в руках біржовиків усе виробництво — як промислове, так і землеробське,
і весь обіг — як засоби сполучення, так і функцію обміну; таким чином біржа стає найвидатнішою представницею самого
капіталістичного виробництва.

2. В 1865 році біржа була ще другорядним елементом в капіталістичній системі. Державні папери
репрезентували головну масу біржових цінностей, але й їх кількість була ще відносно незначна. Поруч з цим акційні банки, які
на континенті і в Америці панували, в Англії ще тільки готувались до того, щоб поглинути аристократичні приватні банки. Але
в цілому вони мали ще відносно невелике значення. Залізничні

* Фрідріх Енгельс: .Біржа. Додаткові зауваження до третього тома .Капіталу** (1895). За фотокопією, що зберігається в
Інституті Маркса — Енгельса — Леніна. Примітка ред. нім. вид. ІМЕЛ.
\parbreak{}  %% абзац продовжується на наступній сторінці
