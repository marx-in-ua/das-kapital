\parcont{}  %% абзац починається на попередній сторінці
\index{iii1}{0291}  %% посилання на сторінку оригінального видання
як \emph{купецький} капітал, і з того, подруге, що він примушує заплатити
собі зиск, тому що він функціонує як \emph{капітал}, тобто
тому, що він виконує працю, яка йому, як функціонуючому капіталові,
оплачується зиском. Отже, таке є питання, що його маємо
розв’язати.

Припустім, що $В$ \deq{} 100, $b$ \deq{} 10, а норма зиску \deq{} 10\%. Припустім,
що $К$ \deq{} 0, щоб без потреби не вводити знову в обрахунок
цей елемент купівельної ціни, який сюди не стосується і з
яким ми вже покінчили. Таким чином продажна ціна була б \deq{}
$В \dplus{} р \dplus{} b \dplus{} р (= В \dplus{} Вр' \dplus{} b \dplus{} bp'$, де $р'$ є норма зиску) \deq{}
100 \dplus{} 10 \dplus{} 10 \dplus{} 1 \deq{} 121.

Але коли б $b$ не витрачалось купцем на заробітну плату, —
тому що $b$ виплачується тільки за торговельну працю, отже, за
працю, потрібну для реалізації вартості товарного капіталу, що
його промисловий капітал кидає на ринок, — то справа стояла б
так: купець віддавав би свій час на те, щоб купити або продати
на $В$ \deq{} 100, і ми припустимо, що це весь час, яким він порядкує.
Коли б торговельна праця, що її репрезентує $b$ або 10,
оплачувалась не заробітною платою, а зиском, то вона передбачала
б другий купецький капітал \deq{} 100, бо 10\% від нього \deq{}
$b$ \deq{} 10. Це друге $В$ \deq{} 100 не входило б додатково в ціну товару,
але ці 10\%, звичайно, входили б у неї. Тому було б зроблено дві
операції по 100 \deq{} 200, щоб купити товарів на 200 \dplus{} 20 \deq{} 220.

Через те що купецький капітал є абсолютно не що інше, як
усамостійнена форма частини функціонуючого в процесі циркуляції
промислового капіталу, всі питання, що стосуються до нього,
мусять бути розв’язані таким способом, що проблема ставиться
насамперед у такій формі, при якій властиві купецькому капіталові
явища виступають ще не самостійно, а в безпосередньому
зв’язку з промисловим капіталом, як його відгалуження. Як контора,
в відміну від майстерні, торговельний капітал постійно
функціонує в процесі циркуляції. Отже, саме тут, в конторі
самого промислового капіталіста, треба насамперед дослідити $b$,
що про нього зараз іде мова.

\enlargethispage{\baselineskip}
Перш за все ця контора є завжди надзвичайно мала порівняно
з промисловою майстернею. Проте, ясно, що в міру розширення
масштабу виробництва збільшуються торговельні операції,
які доводиться постійно проводити для циркуляції промислового
капіталу, — як для того, щоб продати наявний у формі
товарного капіталу продукт, так і для того, щоб знову перетворювати
в засоби виробництва виручені гроші і всьому вести рахунок.
Обчислення цін, ведення книг, ведення каси, кореспонденція,
— все це належить сюди. Чим більший масштаб виробництва,
тим більші — хоч і ніяк не в такій самій пропорції — купецькі операції
промислового капіталу, отже, тим більша є праця та інші
витрати циркуляції для реалізації вартості й додаткової вартості.
В наслідок цього стає потрібним уживання торговельних
найманих робітників, які становлять власне контору. Видатки на
\parbreak{}  %% абзац продовжується на наступній сторінці
