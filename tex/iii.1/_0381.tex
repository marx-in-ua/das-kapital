\parcont{}  %% абзац починається на попередній сторінці
\index{iii1}{0381}  %% посилання на сторінку оригінального видання
обмежує число робочих днів, що їх можна одночасно експлуатувати.
Якщо ж, навпаки, додаткова вартість береться в ірраціональній
формі процента, то межа нагромадження є тільки кількісна
і перевищує всяку фантазію.

Але в капіталі, що дає процент, уявлення про капітал-фетиш
є завершене, уявлення, яке нагромадженому продуктові праці,
і при тому ще й фіксованому у формі грошей, приписує силу за
допомогою природженої таємної властивості, подібно до справжнього
автомата, виробляти додаткову вартість у геометричній
прогресії, так що цей нагромаджений продукт праці, як гадає
„Economist“, давно вже дисконтував усе багатство світу за всі
часи, як таке, що по праву належить і дістається йому. Продукт
минулої праці, сама минула праця тут сама по собі вагітна частиною
сучасної чи майбутньої живої додаткової праці. Ми знаємо,
навпаки, що в дійсності збереження, а остільки й репродукція
вартості продуктів минулої праці є \emph{тільки} результат їх
контакту з живою працею; і, подруге, що панування продуктів
минулої праці над живою додатковою працею триває якраз тільки
доти, поки триває капіталістичне відношення, певне соціальне
відношення, при якому минула праця самостійно протистоїть
живій праці та підкорює її собі.

\section{Кредит і фіктивний капітал}

Детальний аналіз кредиту і тих знарядь, які він собі створює
(кредитні гроші і~\abbr{т. д.}), не входить у наш план. Тут слід відзначити
лиш деякі окремі пункти, необхідні для характеристики
капіталістичного спосббу виробництва взагалі. При цьому ми
матимемо справу тільки з комерційним і банкірським кредитом.
Зв’язок між розвитком цього кредиту і розвитком громадського
кредиту лишається поза нашим розглядом.

Я вже раніше (книга І, розд. III, 3, b) показав, яким чином
з простої товарної циркуляції розвивається функція грошей як
засобу платежу і разом з тим відношення кредитора і боржника
між виробниками товарів і торговцями товарами. З розвитком
торгівлі і капіталістичного способу виробництва, який
виробляє, розраховуючи тільки на циркуляцію, ця природно виросла
основа кредитної системи розширюється, стає загальною,
виробляється. Загалом і в цілому гроші функціонують тут тільки
як засіб платежу, тобто товар продається не за гроші, а за
писану обіцянку платежу в певний строк. Всі ці платіжні обіцянки
ми можемо для короткості підвести під загальну категорію
векселів. Такі векселі до дня скінчення їх строку і настання
дня платежу в свою чергу самі циркулюють як засіб платежу;
і саме вони становлять власне торговельні гроші. Оскільки вони,
кінець-кінцем, в наслідок вирівнення вимог і боргів взаємно
\parbreak{}  %% абзац продовжується на наступній сторінці
