\parcont{}  %% абзац починається на попередній сторінці
\index{iii1}{0301}  %% посилання на сторінку оригінального видання
визначально впливає на його відношення до сукупного капіталу,
або на відносну величину купецького капіталу, необхідного для
циркуляції, бо ясно, що абсолютна величина необхідного купецького
капіталу і швидкість його обороту стоять у зворотному
відношенні одне до одного; але його відносна величина, або та
частина, яку він становить у сукупному капіталі, визначається
його абсолютною величиною, припускаючи всі інші умови
однаковими. Якщо сукупний капітал становить \num{10000}, то, коли
купецький капітал становить \sfrac{1}{10} його, він \deq{} 1000; якщо сукупний
капітал становить 1000, то \sfrac{1}{10}  його \deq{} 100. Таким чином, хоч
його відносна величина лишається тією самою, його абсолютна
величина є різна залежно від величини сукупного капіталу. Але
тут ми беремо його відносну величину, скажімо, \sfrac{1}{10}  сукупного капіталу,
як дану. Сама ця відносна величина його, однак, знов таки
визначається оборотом. При швидкому обороті його абсолютна величина,
наприклад, \deq{} 1000\pound{ фунтам стерлінгів} у першому випадку,
\deq{} 100 у другому випадку, а тому його відносна величина \deq{} \sfrac{1}{10}
При повільнішому обороті його абсолютна величина, скажімо, \deq{}
2000 в першому випадку, \deq{} 200 в другому. Тому його відносна
величина зросла з \sfrac{1}{10}  до \sfrac{1}{5} сукупного капіталу. Обставини,
які скорочують пересічний оборот купецького капіталу, наприклад,
розвиток засобів транспорту, pro tanto [відповідно до цього]
зменшують абсолютну величину купецького капіталу, отже
підвищують загальну норму зиску. В протилежному випадку, —
навпаки. Розвинений капіталістичний спосіб виробництва, порівняно
з попереднім становищем, впливає двояко на купецький
капітал: та сама кількість товарів обертається за допомогою
меншої маси дійсно функціонуючого купецького капіталу; в наслідок
швидшого обороту купецького капіталу і більшої швидкості
процесу репродукції, на якій грунтується швидший оборот,
зменшується відношення купецького капіталу до промислового
капіталу. З другого боку: з розвитком капіталістичного способу
виробництва все виробництво стає товарним виробництвом і тому
весь продукт потрапляє в руки агентів циркуляції; при чому
сюди долучається ще й те, що за попереднього способу виробництва,
яке провадилося в незначних розмірах, якщо залишити
осторонь масу продуктів, які споживалися in natura безпосередньо
самими виробниками, і масу повинностей, які виконувались
in natura, дуже значна частина виробників продавала свої
товари безпосередньо споживачам або працювала на їх особисте
замовлення. Тому, хоч за попередніх способів виробництва торговельний
капітал є більший відносно того товарного капіталу,
що його він обертає, але він

1)~абсолютно менший, бо незрівнянно менша частина сукупного
продукту виробляється як товар і мусить надходити в циркуляцію
як товарний капітал та потрапляти в руки купців; він менший
тому, що товарний капітал є менший. Але разом з тим він відносно
більший не тільки в наслідок більшої повільності його
\parbreak{}  %% абзац продовжується на наступній сторінці
