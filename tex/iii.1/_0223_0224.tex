\parcont{}  %% абзац починається на попередній сторінці
\index{iii1}{0223}  %% посилання на сторінку оригінального видання
бо вона мусить зрости навіть для того, щоб при зміненому
складі капіталу можна було вживати ту саму масу праці при
попередніх відношеннях експлуатації.

Отже, той самий розвиток суспільної продуктивної сили
праці виражається з прогресом капіталістичного способу виробництва,
з одного боку, в тенденції до прогресуючого падіння
норми зиску, а з другого боку, в постійному зростанні абсолютної
маси привласнюваної додаткової вартості або зиску; так
що загалом відносному зменшенню змінного капіталу і зиску
відповідає абсолютне збільшення обох. Ця двобічна дія, як ми
вже показали, може виразитись тільки в зростанні всього капіталу
в швидшій прогресії, ніж та, в якій падає норма зиску. Для
того, щоб при вищому складі капіталу або при відносно сильнішому
збільшенні сталого капіталу можна було вжити абсолютно
зрослий змінний капітал, весь капітал мусить зрости не
тільки відповідно до вищого складу, але ще швидше. З цього
випливає, що чим більше розвивається капіталістичний спосіб
виробництва, тим більша й більша маса капіталу потрібна для
того, щоб уживати ту саму робочу силу, і ще більша для того,
щоб уживати вирослу робочу силу. Отже, зростаюча продуктивна
сила праці на капіталістичній базі з необхідністю створює
постійне позірне перенаселення робітників. Якщо змінний капітал
становить тільки \sfrac{1}{6} всього капіталу замість колишньої
\sfrac{1}{2}, то, щоб можна було вжити ту саму робочу силу, весь
капітал мусить потроїтись; а для того, щоб можна було вжити
подвійну робочу силу, він мусить пошестеритись.

Дотеперішня політична економія, яка не зуміла була пояснити
закон падіння норми зиску, вказувала на підвищення маси
зиску, зростання абсолютної величини зиску, чи то для окремих
капіталістів, чи для суспільного капіталу, як на свого роду
підставу для утішення, але й вона базується на самих тільки
загальних місцях і можливостях.

Те, що маса зиску визначається двома факторами, поперше,
нормою зиску і, подруге, масою капіталу, вжитого для одержання
цієї норми зиску, — це просто тавтологія. Тому та обставина,
що зростання маси зиску можливе, не зважаючи на одночасне
падіння норми зиску, є тільки вираз цієї тавтології і
не допомагає ні на крок посунутися вперед, бо цілком так само
можливе й те, що капітал зростатиме без зростання маси зиску
і що він може навіть зростати і в тому випадку, коли вона
падає. 100 при 25\% дає 25, 400 при 5\% дає тільки 20.\footnote{
„We should also expect that, however the rate of the profits of stock
might diminish in consequence of the accumulation of capital on the land and the
rise of wages, yet the aggregate amount of profits would increase. Thus, supposing
that, with repeated accumulations of 100000 £, the rate of profits should fall from
20 to 19, to 18, to 17 per cent., a constantly diminishing rate; we should expect that
the whole amount of profits received by those successive owners of capital would be
always progressive; that it would be greater when the capital was 200000 £, than
when 100 000 £; still greater when 300 000  £; and so on, increasing, though at a
diminishing rate, with every increase of capital. This progression, howewer, is only
true for a certain time; thus, 19 per cent, on 200 000 £ is more than 20 on 100 000 £;
again 18 per cent on 300 000 £ is more than 19 per cent, on 200 000 £; but after
capital has accumulated to a large amount, and profits have fallen, the further
accumulation diminishes the aggregate of profits. Thus, suppose the accumulation
should be 1 000 000 £, and the profits 7 per cent., the whole amount of profits will be
70000 £; now if an addition of 100000£ capital bemade to the million, and profits should
fall to 6 per cent., 66 000 £ or a diminution of 4000 £ will be received by the owners
of stock, although the whole amount of stock will be increased from 1 000 000 £ to
1 100 000£.“ [„Нам слід, отже, сподіватися, що хоча норма зиску на капітал може
зменшитися в наслідок нагромадження капіталу в країні і підвищення заробітної
плати, однак загальна сума зиску збільшиться. Так, якщо ми припустимо, що при
послідовному нагромадженні 100 000 фунтів стерлінгів норма зиску впаде з 20\% до
19\%, до 18\% і до 17\%, тобто постійно зменшуватиметься, то слід сподіватися,
що вся сума зиску, одержувана цими послідовними власниками капіталу, постійно
зростатиме; що вона буде більша при капіталі в 200 000 фунтів стерлінгів,
ніж при капіталі в 100 000 фунтів стерлінгів, і ще більша при капіталі
в 300 000 фунтів стерлінгів і т. д., зростаючи з кожним збільшенням капіталу,
не зважаючи на зменшення норми. Однак, таке зростання має місце тільки на
протязі певного часу; так, 19\% від 200000 фунтів стерлінгів є більше, ніж 20\%
від 100 000 фунтів стерлінгів, 18\% від 300 000 фунтів стерлінгів знов таки
більше, ніж 19\% від 200 000 фунтів стерлінгів; але після того, як капітал уже
нагромадився до великої суми, а зиски зменшились, дальше нагромадження
зменшує загальну суму зиску. Так, якщо припустимо, що нагромадження
становить 1 000 000 фунтів стерлінгів, а зиск 7\%, то загальна сума зиску становитиме
70 000 фунтів стерлінгів; якщо тепер до капіталу в мільйон буде
додано 100 000 фунтів стерлінгів і зиск знизиться до 6\%, то власники капіталу
одержать 66 000 фунтів стерлінгів, або на 4000 фунтів стерлінгів менше, хоч
загальна сума капіталу зросла з 1000 000 фунтів стерлінгів до 1 100 000 фунтів
стерлінгів“. \emph{Ricardo}: „Principles of Political Economy“, розд. VII („Works“
вид. Мак-Куллоха, 1852, стор. 68 [69]). В дійсності тут припускається, що капітал
зростає з 1 000 000 до 1 100 000, тобто на 10\%, тимчасом як норма зиску
падає з 7 до 6, тобто на 14\sfrac{2}{7}\%. Hinc illae lacrimae [звідси ці сльози].} Але
якщо ті самі причини, які викликають падіння норми зиску,
\index{iii1}{0224}  %% посилання на сторінку оригінального видання
сприяють нагромадженню, тобто утворенню додаткового капіталу,
і якщо кожен додатковий капітал приводить в рух добавну
працю і виробляє добавну додаткову вартість; якщо, з другого
боку, просте зниження норми зиску включає вже і той
факт, що сталий капітал, а тому й весь старий капітал, зріс, —
то весь цей процес перестає бути таємничим. Ми далі побачимо,
до яких умисних фальшувань в обчисленнях вдаються
для того, щоб по-шахрайському відкинути можливість збільшення
маси зиску при одночасному зменшенні норми зиску.

Ми показали, як ті самі причини, які викликають тенденцію
загальної норми зиску до падіння, зумовлюють прискорене нагромадження
капіталу, а тому й зростання абсолютної величини або
загальної маси привласнюваної ним додаткової праці (додаткової
вартості, зиску). Як усе в конкуренції, а тому й у свідомості
агентів конкуренції, так і цей закон — я маю на думці цей внутрішній
і необхідний зв’язок між двома явищами, які, як здається,
одне одному суперечать — виступає у перекрученому вигляді.
Очевидно, що в межах вищенаведених пропорцій капіталіст,
який розпоряджається великим капіталом, одержує більшу масу
\parbreak{}  %% абзац продовжується на наступній сторінці
