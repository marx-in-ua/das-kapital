\parcont{}  %% абзац починається на попередній сторінці
\index{iii1}{0327}  %% посилання на сторінку оригінального видання
угоди як вольові вчинки учасників, як виявлення їх спільної
волі і як угоди, виконувати які кожну договірну сторону примушує
держава, — ці юридичні форми, як просто форми, не можуть
визначити самого змісту угод. Вони тільки виражають його.
Цей зміст справедливий, якщо він відповідає способові виробництва,
є адекватний йому. Він несправедливий, якщо суперечить
йому. Рабство на основі капіталістичного способу виробництва
несправедливе; так само і обман щодо якості товару.

Ці 100 фунтів стерлінгів виробляють зиск у 20 фунтів стерлінгів
тому, що вони функціонують як капітал — як капітал
промисловий чи торговельний. Але sine qua non [неодмінною
умовою] цього функціонування їх як капіталу є витрачання їх
як капіталу, отже, витрачання грошей на купівлю засобів виробництва
(при промисловому капіталі) або товару (при торговельному
капіталі). Але для того, щоб гроші можна було витрачати,
вони мусять бути в наявності. Коли б \emph{А}, власник цих
100 фунтів стерлінгів, витратив їх на своє особисте споживання
або затримав при собі як скарб, то \emph{В}, функціонуючий капіталіст,
не міг би витратити їх як капітал. Він витрачає не свій
капітал, а капітал \emph{А}; але він не може витратити капітал \emph{А} без
згоди \emph{А}. Отже, в дійсності ці 100 фунтів стерлінгів як капітал
первісно витрачає \emph{А}, хоч уся його функція як капіталіста обмежується
цим витрачанням 100 фунтів стерлінгів як капіталу.
Оскільки справа йде про ці 100 фунтів стерлінгів, \emph{В} функціонує
як капіталіст тільки тому, що \emph{А} передає йому ці 100 фунтів
стерлінгів і таким чином витрачає їх як капітал.

Розгляньмо насамперед своєрідну циркуляцію капіталу, що
дає процент. Потім, у другу чергу, слід дослідити той особливий
спосіб, яким він продається як товар, а саме віддається
в позику, а не відступається раз назавжди.

Вихідним пунктом є гроші, які \emph{А} позичає \emph{В}. Ця позика може
бути зроблена під заставу або без застави; однак перша форма є
стародавніша, за винятком позик під товари або під боргові
зобов’язання: векселі, акції і~\abbr{т. д.} Ці особливі форми нас тут
не цікавлять. Ми маємо тут справу з капіталом, що дає процент,
у його звичайній формі.

В руках \emph{В} гроші дійсно перетворюються в капітал, пророблюють
рух $Г — Т — Г'$ і повертаються потім знову до \emph{А} як $Г'$,
як $Г + ΔГ$, де $ΔГ$ представляє процент. Для спрощення ми тут
покищо залишаємо осторонь той випадок, коли капітал залишається
в руках \emph{В} на довший час і проценти сплачуються
в певні строки.

Отже, рух є такий: \[
Г — Г — Т — Г' — Г'\text{.}
\]

\noindent{}Те, що тут повторюється двічі, так це 1) витрачання грошей
як капіталу, 2)~зворотний приплив їх як реалізованого
капіталу, як $Г'$ або $Г + ΔГ$.
