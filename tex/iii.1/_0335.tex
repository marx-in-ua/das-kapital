\parcont{}  %% абзац починається на попередній сторінці
\index{iii1}{0335}  %% посилання на сторінку оригінального видання
цієї специфічної форми тільки умовного відчуження грошей
або товару.

Характерний рух капіталу взагалі, повернення грошей до
капіталіста, повернення капіталу до його вихідної точки, набуває
в капіталі, що дає процент, цілком зовнішнього вигляду,
відірваного від того дійсного руху, формою якого він є. $А$ віддає
свої гроші не як гроші, а як капітал. З капіталом тут не
відбувається ніякої зміни. Він тільки переходить з одних рук
у другі. Його дійсне перетворення в капітал відбувається тільки
в руках $В$. Але для $А$ він став капіталом в наслідок простої
віддачі його в руки $В$. Дійсне повернення капіталу з процесу
виробництва і процесу циркуляції має місце тільки для $В$.
Але для $А$ повернення відбувається в тій самій формі, як
відчуження. З рук $В$ капітал знов повертається в руки $А$. Віддача
грошей у позику на певний час і зворотне одержання їх
з процентом (додатковою вартістю), — ось уся форма руху, належна
капіталові, що дає процент, як такому. Дійсний рух відданих
у позику грошей, дійсний рух їх як капіталу, є операція,
яка лежить по той бік угод між позикодавцями і позичальниками.
В самому капіталі це посередництво угод стерте, не видне, безпосередньо
не включене в нього. Як товар особливого роду
капітал має і особливий рід відчуження. Тому повернення виражається
тут не як наслідок і результат певного ряду економічних
актів, а як наслідок спеціальної юридичної угоди між
покупцем і продавцем. Час повернення залежить від ходу процесу
репродукції; при капіталі, що дає процент, повернення
його як капіталу \emph{здається} залежним від простої згоди між позикодавцем
і позичальником. Так що відносно цієї угоди повернення
капіталу здається вже не результатом, який визначається процесом
виробництва, а таким, наче позичений капітал ніколи не
втрачав форми грошей. Звичайно, фактично угоди ці визначаються
дійсними поверненнями капіталу. Але це не виявляється
в самій угоді. І на практиці це зовсім не завжди так буває.
Якщо дійсне повернення не відбувається своєчасно, то позичальник
мусить пошукати інших допоміжних джерел для виконання
своїх зобов’язань відносно позикодавця. Проста \emph{форма} капіталу
— гроші, які витрачаються як сума А і через певний період
часу повертаються як сума $А \dplus{} \sfrac{1}{х} × А$, без будь-якого іншого опосереднення,
крім цього проміжку часу, — є тільки ірраціональна
форма дійсного руху капіталу.

В дійсному русі капіталу повернення його є момент процесу
циркуляції. Спочатку гроші перетворюються в засоби виробництва;
процес виробництва перетворює їх у товар; продажем товару
вони знову перетворюються в гроші і повертаються в цій формі
назад до рук капіталіста, який спочатку авансував капітал у грошовій
формі. Але при капіталі, що дає процент, повернення
його, як і віддача в позику, є тільки результат юридичної угоди
\parbreak{}  %% абзац продовжується на наступній сторінці
