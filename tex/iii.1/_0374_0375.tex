\parcont{}  %% абзац починається на попередній сторінці
\index{iii1}{0374}  %% посилання на сторінку оригінального видання
як до основної суми, що виробила додаткову вартість. І, як ми
бачили, капітал як такий виступає для всіх активних капіталістів,
— однаково, чи функціонують вони з своїм власним чи
з узятим в позику капіталом — саме такою безпосередньо самозростаючою
вартістю.

$Г — Г'$: тут ми маємо первісний вихідний пункт капіталу, гроші
у формулі $Г — Т — Г'$, зведені до двох крайніх пунктів $Г — Г'$, де
$Г' \deq{} Г \dplus{} ΔГ$, гроші, що створюють більшу кількість грошей.
Це — первісна і загальна формула капіталу, скорочена до безглуздого
резюме. Це — готовий капітал, єдність процесу виробництва
і процесу циркуляції, який дає як готовий капітал у
певні періоди часу певну додаткову вартість. У формі капіталу,
що дає процент, це виявляється безпосередньо, без опосереднення
процесом виробництва і процесом циркуляції. Капітал
здається таємничим і самотворчим джерелом процента, свого
власного збільшення. \emph{Річ} (гроші, товар, вартість) просто як річ
тепер уже є капітал, а капітал здається просто річчю; результат
сукупного процесу репродукції здається властивістю, належною
речі самій по собі; від власника грошей, тобто товару
в його завжди обмінній формі, залежить, чи витратити їх як гроші,
чи віддати в позику як капітал. Тому в капіталі, що дає процент,
цей автоматичний фетиш, самозростаюча вартість, гроші,
що породжують гроші, виробився в чистому вигляді, і в цій
формі він уже не має на собі ніякого сліду свого походження.
Суспільне відношення вивершене як відношення певної речі, грошей,
до самої себе. Замість дійсного перетворення грошей у капітал
тут виявляється тільки беззмістовна форма цього перетворення.
Як і в випадку з робочою силою, споживною вартістю грошей
тут стає їх властивість створювати вартість, створювати більшу
вартість, ніж вартість, що міститься в них самих. Гроші як такі
потенціально вже є самозростаюча вартість, і як така вони віддаються
в позику, що є формою продажу для цього своєрідного
товару. Утворювати вартість, давати процент стає такою
самою властивістю грошей, як властивість грушевого дерева
давати груші. І як таку річ, що дає процент, позикодавець
продає свої гроші. Але це ще не все. Як ми бачили, навіть
дійсно функціонуючий капітал виступає таким чином, ніби він
дає процент не як функціонуючий капітал, а як капітал сам по
собі, як грошовий капітал.

Перекручується і ось що: в той час як процент є тільки
частина зиску, тобто додаткової вартості, яку функціонуючий
капіталіст видушує з робітника, він, процент, виступає тепер,
навпаки, як власний плід капіталу, як щось первісне, а зиск,
який перетворився тепер у форму підприємницького доходу,
як простий аксесуар і додаток, який долучається в процесі репродукції.
Тут фетишистична форма капіталу і уявлення про
капітал-фетиш готові. В $Г — Г'$ ми маємо ірраціональну форму
капіталу, найвищий ступінь перекручення і зречевлення відносин
\index{iii1}{0375}  %% посилання на сторінку оригінального видання
виробництва: форму, що дає процент, просту форму капіталу,
в якій він є передумовою свого власного процесу репродукції;
ми маємо перед собою здатність грошей, відповідно —
товару, збільшувати свою власну вартість незалежно від репродукції,
— містифікацію капіталу в найрізкішій формі.

Для вульґарної економії, яка хоче зобразити капітал як самостійне
джерело вартості, джерело вартостетворення, ця форма,
є, звичайно, знахідкою, формою, в якій уже неможливо пізнати
джерела зиску і в якій результат капіталістичного процесу
виробництва — відокремлений від самого процесу — набуває самостійного
буття.

Лиш у формі грошового капіталу капітал став товаром,
властивість якого самозростати в своїй вартості має певну ціну,
яка кожного разу позначається розміром процента.

Як капітал, що дає процент, і особливо в своїй безпосередній
формі як грошовий капітал, що дає процент (інші форми капіталу,
що дає процент, які нас тут не цікавлять, виводяться
з цієї ж форми і передбачають її), капітал набуває своєї чистої
фетишистичної форми: $Г — Г'$ як суб’єкт, як річ, що може бути
продана. \emph{Поперше}, це відбувається в наслідок його постійного
буття у формі грошей, формі, в якій усі його визначеності стерті
і в якій його реальні елементи невидимі. Гроші — це якраз форма,
в якій стерта ріжниця між товарами як споживними вартостями,
а тому й ріжниця між промисловими капіталами, які складаються з
цих товарів і умов їх виробництва; гроші — це та форма, в якій
вартість — а тут капітал — існує як самостійна мінова вартість.
В процесі репродукції капіталу грошова форма є минуща форма,
простий перехідний момент. Навпаки, на грошовому ринку капітал
завжди існує в цій формі. — \emph{Подруге}, породжена ним додаткова
вартість, тут знов таки в формі грошей, здається належною
йому як такому. Подібно до того, як ріст є властивий
деревам, так і породження грошей (\textgreek{τόκος}) здається властивим
капіталові в цій його формі грошового капіталу.

\looseness=1
В капіталі, що дає процент, рух капіталу скорочений до
крайності; опосереднюючий процес тут випущений і таким чином
капітал — 1000 фіксується як річ, яка сама по собі \deq{} 1000 і за певний
період перетворюється в 1100, подібно до того, як поліпшує
свою споживну вартість вино, що перебувало певний час у льоху.
Капітал є тепер річ, але саме як річ він є капітал. Гроші тепер
вагітні грішми. Якщо вони віддані в позику або вкладені в процес
репродукції (оскільки вони дають функціонуючому капіталістові
як своєму власникові процент, крім підприємницького доходу), то
на них і день і ніч наростає процент, однаково, чи сплять вони,
чи пильнують, сидять дома чи подорожують. Таким чином у
грошовому капіталі, що дає процент (а всякий капітал є щодо виразу
своєї вартості грошовий капітал або вважається тепер
за вираз грошового капіталу), реалізується побожне бажання збирача
скарбів.
