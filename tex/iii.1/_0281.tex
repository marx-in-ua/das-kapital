\parcont{}  %% абзац починається на попередній сторінці
\index{iii1}{0281}  %% посилання на сторінку оригінального видання
В ході наукового аналізу як вихідний пункт утворення загальної
норми зиску виступають промислові капітали і конкуренція
між ними, і тільки пізніше вноситься поправка, доповнення і модифікація
в наслідок втручання купецького капіталу. В ході історичного
розвитку справа стоїть якраз навпаки. Капітал, який спочатку
визначає ціни товарів більш чи менш по їх вартостях, є торговельний
капітал, а та сфера, в якій вперше утворюється загальна
норма зиску, є сфера циркуляції, яка опосереднює процес репродукції.
Первісно промисловий зиск визначається торговельним
зиском. Тільки після того, як капіталістичний спосіб виробництва
вкорінюється і виробник сам стає купцем, торговельний зиск
зводиться до тієї відповідної частини сукупної додаткової вартості,
яка припадає торговельному капіталові як відповідній
частині сукупного капіталу, занятого в суспільному процесі репродукції.

При додатковому вирівненні зисків в наслідок втручання
купецького капіталу виявилось, що у вартість товару не входить
ніякий додатковий елемент на авансований грошовий капітал
купця, що надбавка до ціни, завдяки якій купець одержує
свій зиск, дорівнює тільки тій частині вартості товару, яку
продуктивний капітал не зараховує в ціну виробництва товару,
поступається нею. З цим грошовим капіталом справа стоїть
саме так, як з основним капіталом промислового капіталіста,
оскільки його не спожито і, отже, його вартість не становить
ніякого елементу вартості товару. Саме в ціні, по якій купець
купує товарний капітал, він заміщає в грошах його ціну виробництва
\deq{} $Г$. Його продажна ціна, як це викладено раніше, \deq{} $Г \dplus{} ΔГ$, при чому $ΔГ$ виражає надбавку до ціни
товару,
визначувану загальною нормою зиску. Отже, якщо він продає
товар, то до нього повертається, крім ΔГ, первісний грошовий
капітал, авансований ним на купівлю товарів. Тут знов таки
виявляється, що його грошовий капітал взагалі є не що інше,
як перетворений у грошовий капітал товарний капітал промислового
капіталіста, який так само мало може впливати на величину
вартості цього товарного капіталу, як коли б цей останній
був проданий не купцеві, а безпосередньо останньому споживачеві.
Фактично він тільки антиципує оплату товару цим
останнім. Однак, це правильно тільки в тому випадку, коли, як
ми це досі припускали, купець не робить ніяких додаткових
витрат, або коли йому, крім грошового капіталу, який він мусить
авансувати на купівлю товару у виробника, не доводиться
в процесі метаморфози товарів, купівлі й продажу, авансувати
ніякого іншого капіталу, обігового чи основного. Однак, як ми
це бачили при розгляді витрат циркуляції (книга II, розд. VI),
це не так. І ці витрати циркуляції виступають почасти як витрати,
які купець може покласти на інших агентів циркуляції,
почасти як витрати, які безпосередньо зв’язані з його специфічним
підприємством.
