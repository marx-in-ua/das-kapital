\parcont{}  %% абзац починається на попередній сторінці
\index{iii1}{0284}  %% посилання на сторінку оригінального видання
або більше доходу, чи то в формі заробітної плати, чи в формі
певної частини зиску (провізії, тантьєми), одержуваного при
кожному продажі. В першому випадку купець одержує торговельний
зиск як самостійний капіталіст; у другому випадку прикажчикові,
найманому робітникові промислового капіталіста, виплачується
частина зиску, чи то в формі заробітної плати, чи
в формі пропорціональної участі в зиску промислового капіталіста,
безпосереднім агентом якого він є, а його принципал
у цьому випадку одержує як промисловий, так і торговельний
зиск. Але в усіх цих випадках, хоч самому агентові циркуляції
його дохід може здаватись простою заробітною платою, платою
за виконану ним працю, і хоча — там, де цей дохід не здається
таким — розмір його зиску може дорівнювати тільки заробітній
платі краще оплачуваного робітника, його дохід все ж виникає
тільки з торговельного зиску. Це випливає з того, що його
праця не є вартостетворча праця.

Здовження часу на операцію циркуляції становить собою
для промислового капіталіста: 1) втрату часу особисто для нього,
оскільки це заважає йому виконувати свою функцію керівника
самого процесу виробництва; 2) здовження часу перебування
його продукту, в грошовій або в товарній формі, в процесі циркуляції,
отже, в такому процесі, в якому він не зростає в своїй
вартості і в якому безпосередній процес виробництва переривається.
Щоб цей останній не переривався, доводиться або
скоротити виробництво, або авансувати додатковий грошовий
капітал для того, щоб постійно продовжувати процес виробництва
в тому самому масштабі. Це в кожному разі зводиться
до того, що або при попередньому капіталі одержується менший
зиск, або доводиться авансувати додатковий грошовий
капітал, щоб одержати попередній зиск. Все це ні трохи не
змінюється, коли на місце промислового капіталіста стає купець.
Замість того, щоб першому витрачати більше часу на процес
циркуляції, його витрачає купець; замість того, щоб промисловий
капіталіст авансовував додатковий капітал для циркуляції,
його авансує купець; або — що є те саме — замість того,
щоб більша частина промислового капіталу постійно оберталася
в процесі циркуляції, в ньому цілком замикається капітал
купця; і замість того, щоб промисловий капіталіст одержував
менший зиск, він мусить частину свого зиску цілком відступати
купцеві. Оскільки купецький капітал не перевищує необхідних
меж, ріжниця полягає тільки в тому, що в наслідок
цього поділу функції капіталу вживається менше часу виключно
на процес циркуляції, авансується на це менше додаткового
капіталу, і втрата на сукупному зиску, яка виявляється в формі
торговельного зиску, є менша, ніж вона була б в протилежному
випадку. Якщо в наведеному вище прикладі $720c \dplus{} 180v \dplus{} 180m$,
при існуванні поруч з ним купецького капіталу в 100, дає промисловому
капіталістові зиск в 162, або 18\%, отже, спричинює
\parbreak{}  %% абзац продовжується на наступній сторінці
