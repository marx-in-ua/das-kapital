\parcont{}  %% абзац починається на попередній сторінці
\index{iii1}{0105}  %% посилання на сторінку оригінального видання
головним чином в наслідок запровадження нових машин, які вже
з самого початку виробляються з захисними пристроями, з якими
фабрикант мириться, бо вони не вимагають від нього додаткових
витрат. Крім того, декільком робітникам удалось добитися через
суд значного відшкодування за свої втрачені руки і відстояти ствердження цих судових вироків вищою
інстанцією („Rep. of
Insp. of Fact., 30 April 1861“, стор. 31; також April 1862, стор. 17 [18]).

Цього досить в питанні про економію на засобах забезпечення
життя і членів тіла робітників (серед яких багато дітей) від небезпек, які виникають безпосередньо з
уживання їх коло машин.

\emph{Праця в закритих приміщеннях взагалі}. — Відомо, в якій
великій мірі економія на просторі, отже й на будівлях, приводить до скупченості робітників у тісних
приміщеннях. До цього
долучається ще економія на засобах вентиляції. Разом з довшим
робочим часом обидві ці причини викликають значне збільшення
хвороб органів дихання, а тому й збільшення смертності. Нижченаведені ілюстрації взяті з звіту про
„Public Health, 6-th Rep.
1863“ [народне здоров’я]; звіт складений доктором Джоном
Сімоном, добре відомим з нашої першої книги.

Подібно до того, як комбінація робітників і кооперація їх
допускає застосовування машин у великому масштабі, концентрацію засобів виробництва і економію в
застосуванні їх, так
само ця спільна робота великих мас у закритих приміщеннях
і при таких обставинах, коли вирішальним є не здоров’я робітників, а успішніше виготовлення
продукту, — ця масова концентрація робітників у тій самій майстерні є, з одного боку,
джерелом зростаючого зиску для капіталістів, а, з другого боку,
якщо вона не компенсується скороченням робочого часу і спеціальними охоронними заходами, разом з тим
причиною марнотратства життям і здоров’ям робітників.

Доктор Сімон встановлює таке загальне правило, яке він доводить масовими статистичними даними: „В
тій самій мірі, в якій
населення певної місцевості змушене спільно працювати в закритих приміщеннях, в тій самій мірі
зростає, при інших
незмінних умовах, норма смертності цієї округи в наслідок
хвороб легенів“ (стор. 23). Причина — погана вентиляція. „Мабуть, у всій Англії немає жодного
винятку з того загального
правила, що в кожній окрузі, яка має значну, проваджену
в закритих приміщеннях промисловість, збільшена смертність
цих робітників достатня для того, щоб забарвити статистику
смертності всієї округи значним переважанням хвороб легенів“
(стор. 23).

З статистики, смертності в тих галузях промисловості, в яких
працюють у закритих приміщеннях і які в 1860 і 1861~\abbr{рр.} були
досліджені санітарним відомством, виявляється: на те саме число
чоловіків між 15 і 55 роками, на яке в англійських землеробських округах припадає 100 випадків
смерті від сухот і інших
хвороб легенів, припадає: у Ковентрі — 163 випадки смерті від
\parbreak{}  %% абзац продовжується на наступній сторінці
