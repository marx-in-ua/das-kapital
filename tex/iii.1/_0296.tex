\parcont{}  %% абзац починається на попередній сторінці
\index{iii1}{0296}  %% посилання на сторінку оригінального видання
іншому, при чому цей рід циркуляції за часів спекуляції може
мати вигляд великого процвітання) купецький капітал скорочує,
поперше, фазу $Т — Г$ для продуктивного капіталу. Подруге, при
сучасній кредитній системі він порядкує більшою частиною
сукупного грошового капіталу суспільства, так що може повторювати
свої закупівлі ще до того, як остаточно продасть уже
куплене; при чому байдуже, чи продає наш купець безпосередньо
останньому споживачеві, чи між ними двома стоять ще
12 інших купців. При величезній еластичності процесу репродукції,
який постійно може бути виведений поза всяку дану
межу, він не знаходить собі ніякої межі або тільки дуже еластичну
межу в самому виробництві. Отже, крім відокремлення актів
$Т — Г$ і $Г — Т$, яке випливає з самої природи товару, тут утворюється
фіктивний попит. Рух купецького капіталу, не зважаючи
на своє усамостійнення, ніколи не буває чимось іншим, як рухом
промислового капіталу в сфері циркуляції. Але в силу свого
усамостійнення купецький капітал рухається в певних межах незалежно
від меж процесу репродукції, і тому він навіть виводить
процес репродукції поза його межі. Внутрішня залежність, зовнішня
самостійність женуть його до такого пункту, де внутрішній
зв’язок знову відновлюється насильно, шляхом кризи.

Звідси те явище при кризах, що вони спочатку виявляються
і вибухають не в роздрібній торгівлі, яка має справу з безпосереднім
споживанням, а в сферах гуртової торгівлі та банків,
які дають в розпорядження цієї останньої грошовий капітал
суспільства.

Дійсно, фабрикант може продавати експортерові, а цей, в свою
чергу, своєму іноземному клієнтові, імпортер може збувати
свої сировинні матеріали фабрикантові, цей останній свої продукти
— гуртовому торговцеві і~\abbr{т. д.} Але на якомусь одному
невидимому пункті товар лежить непроданим; або, іншим разом,
ступенево переповнюються запаси всіх виробників і торговців-посередників.
Якраз тоді споживання звичайно стоїть на найвищій
точці розквіту, почасти тому, що один промисловий капіталіст
приводить у рух ряд інших, почасти тому, що заняті
ними робітники, які працюють повний час, можуть витрачати
більше, ніж звичайно. З збільшенням доходу капіталістів збільшуються
і їх видатки. Крім того, як ми вже бачили (книга II,
відділ III), відбувається безперервна циркуляція між сталим капіталом
і сталим капіталом (навіть залишаючи осторонь прискорення
нагромадження), яка перший час незалежна від особистого
споживання в тому розумінні, що вона ніколи не входить
у нього, але яка все ж кінець-кінцем обмежується ним,
бо виробництво сталого капіталу ніколи не відбувається ради
нього самого, а відбувається тільки тому, що його більше
вживається у сферах виробництва, продукти яких входять в особисте
споживання. Однак, протягом певного часу це виробництво
може спокійно йти своїм шляхом, підохочуване сподіваним
\parbreak{}  %% абзац продовжується на наступній сторінці
