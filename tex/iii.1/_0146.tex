\parcont{}  %% абзац починається на попередній сторінці
\index{iii1}{0146}  %% посилання на сторінку оригінального видання
часто міняються, і заробіток робітників підвищується чи падає
залежно від якості бавовняної мішанки. Іноді лишалося тільки
15\% від колишнього заробітку, і за один чи два тижні він
падав на 50 або 60\%“. Інспектор Редгрев, що його ми тут
цитуємо, наводить взяті з практики дані про заробітну плату,
з яких тут досить буде таких прикладів:

$А$, ткач, сім’я з 6 осіб, занятий 4 дні на тиждень, 6 шилінгів
8\sfrac{1}{2}, пенсів; $В$, twister [присукальник], 4\sfrac{1}{2} дні на тиждень, 6 шилінгів;
$C$, ткач, сім’я з 4 осіб, 5 днів на тиждень, 5 шилінгів
1 пенс; $D$, slubber [тростильник], сім’я з 6 осіб, 4 дні на тиждень,
7 шилінгів 10 пенсів; $Е$, ткач, сім’я з 7 осіб, 3 дні на
тиждень, 5 шилінгів і~\abbr{т. д.} Редгрев каже далі: „Вищенаведені
дані заслуговують уваги, бо вони показують, що для деяких
сімей робота була б нещастям, тому що вона не тільки скоротила
б їх дохід, але й знизила б його настільки, що його вистачило
б тільки на задоволення незначної частини абсолютно
необхідних потреб, якщо не давалося б додаткової допомоги
в тих випадках, коли заробіток сім’ї не досягає тієї суми, яку
вона одержувала б як допомогу, коли б усі члени сім’ї були
без роботи“ („Rep. of Insp. of Fact., Oct. 1863“, стор. 50--53).

„Починаючи з 5 червня 1863 року, не було жодного тижня,
на протязі якого весь робочий час усіх робітників становив би
пересічно більше двох днів 7 годин і кількох хвилин“ (там же,
стор. 121).

З початку кризи до 25 березня 1863 року майже три мільйони
фунтів стерлінгів було витрачено установами піклування
про бідних, центральним комітетом допомоги і лондонським
Mansion-House [муніципальним] комітетом (стор. 13).

„В одній окрузі, де випрядається найтонша пряжа\dots{} прядільники
підпали посередньому зниженню заробітної плати на 15\%
в наслідок переходу від Sea Island до єгіпетської бавовни\dots{}
В одній великій окрузі, де бавовняні відпади застосовуються
великими масами для домішки до індійської бавовни, заробітна
плата прядільників була знижена на 5\% і, крім того, вони
втратили ще 20--30\% в наслідок перероблення сурату й відпадів.
Ткачі, які працювали раніш коло чотирьох верстатів, перейшли
тепер на два верстати. В 1860 році вони виробляли на кожному
верстаті 5 шилінгів 7 пенсів, в 1863 році — тільки 3 шилінги
4 пенси\dots{} Грошові штрафи, які раніш, при застосуванні американської
бавовни, коливалися від 3 до 6 пенсів“ [для прядільників],
„доходять тепер до 1 шилінга — 3 шилінгів 6 пенсів“.
В одній окрузі, де вживалась єгіпетська бавовна, змішана
з ост-індською, „пересічна заробітна плата прядільника на мюлях
в 1860 році становила 18--25 шилінгів, а тепер 10--18 шилінгів.
Це викликано не самим тільки погіршенням бавовни,
але і зменшенням швидкості мюлів для того, щоб надати
пряжі дужчого крутіння, — за що в звичайні часи згідно з умовою
про заробітну плату платилося додатково“ (стор. 43, 44,
\parbreak{}  %% абзац продовжується на наступній сторінці
