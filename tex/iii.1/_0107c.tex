\parcont{}  %% абзац починається на попередній сторінці
\index{iii1}{0107}  %% посилання на сторінку оригінального видання
санітарного стану; але майже всі вони переповнені, погано провітрюються і в високій мірі
несприятливі для здоров’я\dots{} В таких
кімнатах, крім усього, неодмінно жарко; а коли запалюють газ,
як це роблять удень під час туману або зимою вечорами, то температура підвищується до 80 і навіть до
90 градусів (за Фаренгейтом \deq{} 27--33° Цельсія) і викликає надзвичайне пітніння робітників
і згущення пари на шибках, так що вода безупинно стікає або
крапає з вікна в стелі, і робітники змушені держати відчиненими
кілька вікон, хоча вони при цьому неминуче простуджуються. —
Становище в 16 найзначніших майстернях лондонського Вестенду
він описує так: найбільший кубічний простір, який припадає
в цих погано провітрюваних кімнатах на одного робітника, становить 270 кубічних футів; найменший —
105 футів, пересічно —
всього тільки 156 футів на людину. В одній майстерні, яка обведена з усіх боків галереєю і має
освітлення тільки згори,
занято від 92 до 100 осіб; горить багато газових ріжків; клозети
збудовані безпосередньо коло майстерні, і на кожну людину
припадає не більше, як 150 кубічних футів простору. В другій
майстерні, в освітленому згори дворі, яку можна назвати тільки
собачою конурою і яку можна провітрювати тільки через маленьке вікно в даху, працює 5 чи 6 осіб, при
чому на кожну
з них припадає 112 кубічних футів“. І „в цих жахливих (atrocious) майстернях, які описує доктор
Сміт, кравці працюють
звичайно 12--13 годин на день, а іноді праця триває 14--16 годин“ (стор. 25, 26, 28).

\begin{table}[H]
\noindent\begin{tabularx}{\textwidth}{@{}r@{~}lXrrr}
   \toprule 
     \multicolumn{2}{l}{\multirowcell{2}[0ex][l]{Число занятих людей}} &
     \multirowcell{2}[0ex][l]{Галузь промисловості\\ і місцевість} &
     \multicolumn{3}{r@{}}{\makecell{
         Норма смертності \\ на \num{100000} осіб віком
     }} \\
  \cmidrule(l){4-6}
     & & & 25\textendash{}35 & 35\textendash{}45 & 45\textendash{}55 \\

  \midrule

    $\left.\begin{array}{r@{}}\text{\num{958.265}}\end{array}\right.$& &
    Землеробство, Англія та Уельс\dotfill{} & 
    743 & \phantom{1.}805 & \num{1.145} \\

    $\left.
    \begin{array}{r@{}}
      \text{\num{22.301}}\\ 
      \text{\num{12.379}}\end{array}\right.$& 
    $\left.
    \begin{array}{@{}l}
      \text{чоловіків}\\ 
      \text{жінок}
    \end{array} 
    \right\}$ &
    Кравці, Лондон\dotfill{} &
    958 & \num{1.262} & \num{2.093} \\
         
    $\left.\begin{array}{r@{}}\text{\num{13.803}}\end{array}\right.$& &
    Складачі й друкарі, Лондон\dotfill{} &
    894 & \num{1.747} & \num{2.367}

\end{tabularx}
\end{table}

\noindent{}(стор. 30). Треба відзначити — і це дійсно відзначено складачем
цього звіту, завідувачем медичного відділу, Джоном Сімоном, —
що для віку в 25--35 років смертність кравців, складачів і друкарів Лондона показана применшеною, бо
в обох цих галузях
промисловості лондонські майстри одержують з села велике
число молодих людей (мабуть, до 30 років), що працюють як
учні і „improvers“, тобто для дальшого удосконалення. Вони
збільшують число занятих осіб, на яке треба обчисляти норми
смертності промислового населення Лондона; але вони не збільшують в такій самій мірі число смертей у
Лондоні, бо їх перебування в Лондоні тільки тимчасове; коли вони захворіють на протязі цього часу,
то вертаються додому на село, і смерть
їх, якщо вони умирають, реєструється там. Ця обставина ще
в більшій мірі стосується до молодшого віку, і в наслідок цього
\parbreak{}  %% абзац продовжується на наступній сторінці
