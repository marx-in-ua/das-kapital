\parcont{}  %% абзац починається на попередній сторінці
\index{iii1}{0152}  %% посилання на сторінку оригінального видання
Хоч і яке велике значення має вивчення таких тертів для кожної
спеціальної праці про заробітну плату, все ж при загальному
дослідженні капіталістичного виробництва їх треба залишити
осторонь як випадкові і неістотні. При такому загальному
дослідженні взагалі завжди припускається, що дійсні відносини
відповідають своєму поняттю, або, що є те саме, дійсні відносини
зображаються лиш остільки, оскільки вони виражають свій
власний загальний тип.

Ріжниця норм додаткової вартості в різних країнах, отже,
національні ріжниці в ступенях експлуатації праці, для даного
дослідження зовсім не мають значення. Адже в цьому відділі ми
саме хочемо показати, яким чином в межах даної країни утворюється
певна загальна норма зиску. Однак, ясно, що при порівнянні
різних національних норм зиску треба тільки зіставити розвинуте
нами раніше з тим, що ми маємо розвинути тут. Спочатку
треба розглянути ріжницю в національних нормах додаткової
вартості, а потім, на основі цих даних норм додаткової вартості,
порівняти ріжницю національних норм зиску. Оскільки ріжниця
цих останніх не є результатом ріжниці національних норм додаткової
вартості, вона мусить виникати з обставин, при яких,
як і в нашому дослідженні в цьому розділі, додаткова вартість
припускається повсюди однаковою, постійною.

В попередньому розділі було показано, що, коли норму додаткової
вартості припустити незмінною, норма зиску, яку дає певний
капітал, може підвищуватись чи падати в наслідок обставин,
які підвищують або знижують вартість тієї чи іншої частини
сталого капіталу і тому взагалі впливають на відношення між
сталою і змінною складовими частинами капіталу. Далі було
відзначено, що обставини, які здовжують або скорочують час
обороту капіталу, можуть справляти аналогічний вплив на
норму зиску. Через те що маса зиску тотожна з масою додаткової
вартості, з самою додатковою вартістю, то виявилось також,
що \emph{маса} зиску — відмінно від \emph{норми} зиску — не зачіпається
щойно згаданими коливаннями вартості. Вони модифікують тільки
норму, в якій виражається дана додаткова вартість, отже й зиск
даної величини, тобто модифікують його відносну величину,
його величину порівняно з величиною авансованого капіталу.
Оскільки в наслідок таких коливань вартості відбувається зв’язування
або звільнення капіталу, таким посереднім шляхом може
бути зачеплена не тільки норма зиску, але й самий зиск. Однак,
це завжди має силу тільки для капіталу, уже вкладеного, а не
для нових капіталовкладень; і, крім того, збільшення або зменшення
самого зиску завжди залежить від того, наскільки більше
чи менше праці в наслідок таких коливань вартості може бути
приведено в рух тим самим капіталом, отже, від того, наскільки
більшу чи меншу масу додаткової вартості може — при незмінній
нормі додаткової вартості — виробити той самий капітал.
Аж ніяк не суперечачи загальному законові і не становлячи винятку
\index{iii1}{0153}  %% посилання на сторінку оригінального видання
з нього, цей позірний виняток в дійсності був тільки
окремим випадком застосування загального закону.

Якщо в попередньому відділі виявилось, що, при незмінному
ступені експлуатації праці, із зміною вартості складових частин
сталого капіталу, а також із зміною часу обороту капіталу змінюється
норма зиску, то з цього само собою випливає, що
норми зиску різних одночасно існуючих, одна поряд одної, сфер
виробництва будуть різні, якщо при інших незмінних умовах
час обороту застосовуваних капіталів різний або якщо вартісне
відношення між органічними складовими частинами цих капіталів
у різних галузях виробництва є різне. Те, що ми раніш
розглядали як зміни, що відбуваються послідовно в часі
з тим самим капіталом, ми розглядаємо тепер як одночасно
наявні ріжниці між існуючими одно поряд одного капіталовкладеннями
в різних сферах виробництва.

При цьому нам доведеться дослідити: 1)~ріжницю в \emph{органічному
складі} капіталів, 2)~ріжницю в часі їх обороту.

В усьому цьому дослідженні, коли ми говоримо про склад
або оборот капіталу в певній галузі виробництва, ми завжди
маємо на увазі — припущення, яке само собою зрозуміле, — пересічні
нормальні відношення капіталу, вкладеного в цю галузь
виробництва; взагалі, мова йде про пересічні відношення сукупного
капіталу, вкладеного в дану сферу, а не про випадкові
ріжниці між окремими вкладеними в цю сферу капіталами.

Через те що, далі, припускається, що норма додаткової вартості
і робочий день є незмінні, і через те що це припущення
включає також і незмінність заробітної плати, то певна кількість
змінного капіталу виражає певну кількість приведеної
в рух робочої сили, а тому й певну кількість праці, яка упредметнюється.
Отже, якщо 100\pound{ фунтів стерлінгів} виражають тижневу
заробітну плату 100 робітників, тобто в дійсності 100 робочих
сил, то 100\pound{ фунтів стерлінгів} $× n$ виражають тижневу
заробітну плату $100 × n$ робітників, а $\frac{100\pound{ фунтів стерлінгів}}{n}$ тижневу
заробітну плату $\frac{100}{n}$ робітників. Отже, змінний капітал служить
тут (як і завжди при даній величині заробітної плати) показником
маси праці, яку приводить в рух весь капітал певної величини;
тому ріжниці у величині застосовуваного змінного капіталу
служать показниками ріжниці в масі вживаної робочої сили. Якщо
100\pound{ фунтів стерлінгів} представляють 100 робітників на тиждень
і, отже, при 60 годинах тижневої праці — 6000 робочих годин, то
200\pound{ фунтів стерлінгів} представляють \num{12000} робочих годин, а 50\pound{ фунтів стерлінгів} тільки 3000 робочих годин.

Під складом капіталу ми розуміємо, як це сказано вже у
книзі першій, відношення між його активною і пасивною складовою
частиною, між змінним і сталим капіталом. При цьому
треба розглянути два відношення, які мають неоднакову важливість,
\index{iii1}{0154}  %% посилання на сторінку оригінального видання
хоч при певних обставинах можуть спричиняти однаковий
вплив.

Перше відношення ґрунтується на технічній базі, і на певному
ступені розвитку продуктивної сили його треба розглядати
як дане. Потрібна певна маса робочої сили, представлена
певним числом робітників, щоб виробити певну масу продукту,
наприклад, протягом одного дня, і, отже — що при цьому само
собою розуміється — привести в рух, продуктивно спожити
певну масу засобів виробництва, машин, сировинних матеріалів
і~\abbr{т. д.} Певне число робітників припадає на певну кількість засобів
виробництва, отже певна кількість живої праці припадає
на певну кількість праці, вже упредметненої в засобах виробництва.
Це відношення дуже різне в різних сферах виробництва,
часто в різних галузях однієї й тієї ж промисловості, хоч, з другого
боку, випадково воно може бути цілком або приблизно
однаковим в дуже віддалених одна від одної галузях промисловості.
Це відношення становить технічний склад капіталу і є дійсна
основа його органічного складу.

Але можливо також, що це відношення однакове в різних
галузях промисловості, оскільки змінний капітал є простий показник
робочої сили, а сталий капітал — простий показник маси
засобів виробництва, приведеної в рух цією робочою силою.
Наприклад, певні роботи з міддю й залізом можуть вимагати
однакового відношення між робочою силою і масою засобів
виробництва. Але через те що мідь дорожча, ніж залізо, то вартісне
відношення між змінним і сталим капіталом в обох випадках
буде різне і разом з тим буде різний і вартісний склад
обох цілих капіталів. Ріжниця між технічним складом і вартісним
складом виявляється в кожній галузі промисловості
в тому, що при незмінному технічному складі вартісне відношення
обох частин капіталу може змінюватись, а при зміні
технічного складу вартісне відношення може лишатись незмінним;
останнє має місце, звичайно, тільки тоді, коли зміна
відношення між застосованою масою засобів виробництва і масою
робочої сили вирівнюється протилежною зміною їх вартостей.

Вартісний склад капіталу, оскільки він визначається його
технічним складом і відображає цей останній, ми звемо \emph{органічним}
складом капіталу\footnote{
Вищевикладене коротко було розвинуте уже в третьому виданні першої
книги, стор. 628. на початку розділу XXIII.~Через
те що в перших двох виданнях немає цього місця, повторення його тут мало
тим більше підстав. — \emph{Ф.~Е.}
}.

Отже, щодо змінного капіталу ми припускаємо, що він є показник
певної кількості робочої сили, певного числа робітників
або певної маси приводжуваної в рух живої праці. В попередньому
\index{iii1}{0155}  %% посилання на сторінку оригінального видання
відділі ми бачили, що зміна величини вартості змінного
капіталу іноді виражає не що інше, як більшу або меншу ціну
тієї самої маси праці; але тут, де норма додаткової вартості
і робочий день розглядаються як незмінні, а заробітна плата
за певний робочий час як величина дана, це відпадає. Навпаки,
ріжниця у величині сталого капіталу може, правда, бути також
показником зміни маси засобів виробництва, приведених в рух
певною кількістю робочої сили; але вона може також походити
з ріжниці у вартості засобів виробництва, приведених в рух
у певній сфері виробництва, порівняно з іншими сферами. Тим
то тут треба взяти до уваги обидві ці точки зору.

Нарешті, треба зробити ще таке істотне зауваження:

Припустім, що 100\pound{ фунтів стерлінгів} становлять тижневу
заробітну плату 100 робітників. Припустім, що тижневий робочий час дорівнює 60 годинам. Припустімо,
далі, що норма
додаткової вартості \deq{} 100\%. В цьому випадку робітники 30 годин з 60 працюють на себе самих, а 30
даром на капіталіста.
В 100\pound{ фунтах стерлінгів} заробітної плати в дійсності втілено
тільки 30 робочих годин 100 робітників, або разом 3000 робочих годин, тимчасом як інші 3000 годин,
які вони працюють,
втілені в 100\pound{ фунтах стерлінгів} додаткової вартості, відповідно — зиску, що його забирає собі
капіталіст. Тому, хоч заробітна плата в 100\pound{ фунтів стерлінгів} не виражає тієї вартості,
в якій упредметнюється тижнева праця 100 робітників, вона
все ж показує (бо довжина робочого дня і норма додаткової
вартості є дані), що цим капіталом приведено в рух 100 робітників на протязі загалом 6000 робочих
годин. Капітал в 100\pound{ фунтів стерлінгів} показує це, тому що він, поперше, показує
число приведених в рух робітників, бо 1\pound{ фунт стерлінгів} \deq{} 1 робітникові за тиждень, отже 100\pound{ фунтів
стерлінгів} \deq{} 100 робітникам; і, подруге, тому що кожний приведений
в рух робітник, при даній нормі додаткової вартості в 100\%,
виконує вдвоє більше праці, ніж міститься в його заробітній
платі, отже, 1\pound{ фунт стерлінгів}, його заробітна плата, що є виразом півтижневої праці, приводить в
рух працю цілого тижня,
і так само 100\pound{ фунтів стерлінгів}, хоч вони містять в собі тільки 50
тижнів праці, приводять в рух працю 100 робочих тижнів. Отже,
тут треба мати на увазі дуже істотну ріжницю між змінним капіталом, витраченим на заробітну плату,
оскільки його вартість,
сума заробітних плат, представляє певну кількість упредметненої праці, і цим капіталом, оскільки
його вартість є простий показник маси живої праці, яку він приводить в рух. Ця
остання завжди більша, ніж кількість праці, яка міститься
в змінному капіталі, і тому вона виражається також у вартості
більшій, ніж вартість змінного капіталу — у вартості, яка визначається, з одного боку, числом
приведених в рух змінним капіталом робітників, а з другого боку, кількістю виконуваної ними
додаткової праці.
