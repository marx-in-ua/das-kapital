
\index{iii1}{0068}  %% посилання на сторінку оригінального видання
Потретє, треба взяти до уваги \emph{продуктивність праці}, вплив
якої на норму додаткової вартості докладно розглянуто в книзі І,
відділ IV.~Але вона може також справляти і безпосередній
вплив на норму зиску, принаймні окремого капіталу, коли, як
де розвинено в книзі І, розділ X, цей окремий
капітал працює з продуктивністю більшою, ніж суспільно-пересічна,
дає свої продукти по вартості нижчій, ніж суспільно-пересічна
вартість таких самих товарів, і таким чином
реалізує надзиск. Але цей випадок лишається тут поза нашим
розглядом, бо і в цьому відділі ми все ще виходимо з припущення,
що товари виробляються при суспільно-нормальних
умовах і продаються по їх вартостях. Отже, в кожному окремому
випадку ми виходимо з припущення, що продуктивність
праці лишається сталою. Справді, вартісний склад капіталу,
вкладеного в певну галузь промисловості, тобто певне відношення
змінного капіталу до сталого капіталу, кожного разу
виражає певний ступінь продуктивності праці. Отже, як тільки
це відношення зазнає зміни з іншої причини, а не в наслідок
простої зміни вартості матеріальних складових частин сталого
капіталу або зміни заробітної плати, то й продуктивність праці
мусить зазнати зміни, і тому ми досить часто бачитимем, що
зміни, які відбуваються з факторами $c$, $v$ і $m$, включають також
і зміни в продуктивності праці.

Те саме стосується і до трьох інших факторів: \emph{довжини робочого
дня, інтенсивності праці і заробітної плати}. Їх вплив
на масу і норму додаткової вартості докладно досліджено в першій
книзі. Отже, зрозуміло, що коли ми задля спрощення завжди
виходимо з припущення, що ці три фактори лишаються сталими,
то все ж ті зміни, які відбуваються з $v$ і $m$, можуть також включати
в собі зміну в величині цих трьох визначаючих їх моментів.
Тут слід тільки коротко нагадати про те, що заробітна плата
діє на величину додаткової вартості і висоту норми додаткової
вартості зворотно до того, як діє довжина робочого дня і інтенсивність
праці; що підвищення заробітної плати зменшує додаткову
вартість, тимчасом як здовження робочого дня і підвищення
інтенсивності праці збільшують її.

Якщо припустимо, наприклад, що капітал у 100 з 20 робітниками
виробляє при десятигодинній праці та загальній тижневій
заробітній платі в 20 додаткову вартість у 20, то ми
матимемо:\[
80 c \dplus{} 20 v \dplus{} 20 m; m' \deq{} 100\%, р' \deq{} 20\%.
\]
Припустімо, що робочий день здовжується, без підвищення
заробітної плати, до 15 годин; в наслідок цього вся нововироблена
20 робітниками вартість підвищується з 40 до 60 ($10 : 15 \deq{} 40 : 60$);
\parbreak{}  %% абзац продовжується на наступній сторінці
