вана додаткова вартість. І якраз у цій точці повернення, в якій
капітал існує як реалізований капітал, як вартість, що зросла,
в цій формі капітал — оскільки ця точка фіксується як точка
спокою, уявна чи дійсна — ніколи не входить в циркуляцію,
а, навпаки, є вилученим з циркуляції, як результат усього процесу.
Якщо він знову витрачається, то він ніколи не відчужується
третій особі як капітал, а продається їй як простий
товар, або віддається їй як прості гроші за товар. Він ніколи
не виступає в процесі своєї циркуляції як капітал, а тільки як
товар або гроші, і це є тут його єдина форма буття для інших.
Товар і гроші є тут тільки капіталом не остільки, оскільки
товар перетворюється в гроші, а гроші в товар, не в їх дійсних
відношеннях до покупця або продавця, а тільки в їх ідеальних
відношеннях до самого капіталіста (з суб’єктивної точки зору),
або як моменти процесу репродукції (з об’єктивної точки зору).
В дійсному русі капітал існує як капітал не в процесі циркуляції,
а тільки в процесі виробництва, в процесі експлуатації робочої
сили.

Але інакше стоїть справа з капіталом, що дає процент, і якраз
це становить його специфічний характер. Власник грошей,
який хоче використати свої гроші як капітал, що дає процент,
відчужує їх третій особі, кидає їх у циркуляцію, робить їх товаром
як капітал; не тільки як капітал для нього самого, але
й для інших; вони не тільки капітал для того, хто їх відчужує,
але й третій особі вони з самого початку передаються як
капітал, як вартість, яка має споживну вартість створювати
додаткову вартість, зиск; як вартість, яка в русі зберігається
і після свого функціонування повертається до того, хто її первісно
витратив, в даному разі до власника грошей; отже, тільки
на якийсь час віддалюється від нього, тимчасово переходить з
володіння свого власника у володіння функціонуючого капіталіста,
тобто вона не виплачується і не продається, а тільки віддається
в позику; вона тільки відчужується під умовою, що
після певного строку вона, поперше, повернеться до своєї вихідної
точки, і повернеться, подруге, як реалізований капітал,
реалізувавши ту свою споживну вартість, що вона виробляє додаткову
вартість.

Товар, який віддається в позику як капітал, віддається в позику,
залежно від його властивостей, або як основний, або як
обіговий капітал. Гроші можуть віддаватися в позику в обох
формах — як основний капітал, наприклад, тоді, коли вони сплачуються
назад у формі пожиттьової ренти, так що разом з процентами
завжди повертається назад і частина капіталу. Деякі
товари, відповідно до природи їх споживної вартості, можуть
віддаватися в позику тільки як основний капітал, наприклад,
будинки, судна, машини і т. д. Але всякий відданий у позику капітал,
яка б не була його форма і як би не модифікувалась його зворотна
сплата природою його споживної вартості, завжди є тільки
