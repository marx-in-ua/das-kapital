\parcont{}  %% абзац починається на попередній сторінці
\index{iii1}{0190}  %% посилання на сторінку оригінального видання
абож хоч і в тому самому напрямі, але не в тій самій мірі,
одним словом, якщо відбуваються двосторонні зміни, які, однак,
змінюють попереднє відношення між обома сторонами, то кінцевий
результат завжди мусить звестись до одного з двох
вищерозглянутих випадків.

Справжня трудність при загальному визначенні понять попиту
і подання полягає в тому, що визначення це, здається, зводиться
до тавтології. Розгляньмо спочатку подання, тобто продукт,
який перебуває на ринку або може бути приставлений на
ринок. Для того, щоб не вдаватись до цілком зайвих тут
деталей, візьмімо тут масу річної репродукції в кожній даній
галузі промисловості і залишмо при цьому осторонь те, що різні
товари в більшій чи меншій мірі можуть забиратися з ринку
і нагромаджуватись для споживання, скажемо, ближчого року.
Ця річна репродукція виражає насамперед певну кількість, міру
або число, залежно від того, як виміряється товарна маса, —
окремими екземплярами, чи як суцільна величина; це — не тільки
споживні вартості, що задовольняють людські потреби, але і такі
споживні вартості, що перебувають на ринку в певній даній
кількості. Подруге, ця кількість товарів має певну ринкову
вартість, яку можна виразити як кратне ринкової вартості товару
або товарної міри, що служать одиницями. Тому між
кількістю товарів, що перебувають на ринку, і їх ринковою
вартістю не існує ніякого необхідного зв’язку; тимчасом, наприклад,
як деякі товари мають специфічно високу вартість,
інші мають специфічно низьку вартість, так що дана сума вартості
може виразитись у дуже великій кількості одних і в дуже
незначній кількості інших товарів. Між кількістю товарів, що
перебувають на ринку, і ринковою вартістю цих товарів існує
тільки такий зв’язок: на даній базі продуктивності праці виготовлення
певної кількості товарів вимагає в кожній окремій
сфері виробництва певної кількості суспільного робочого часу,
хоч у різних сферах виробництва це відношення є цілком різне
і не стоїть ні в якому внутрішньому зв’язку з корисністю цих
товарів або специфічною природою їх споживних вартостей.
При всіх інших однакових умовах, якщо кількість $a$ даного
сорту товарів коштує $b$ робочого часу, то кількість $na$ коштує
$nb$ робочого часу. Далі: оскільки суспільство хоче задовольнити
потреби, хоче щоб для цієї мети був вироблений товар, воно мусить
його оплатити. Дійсно, оскільки при товарному виробництві
передбачається поділ праці, то суспільство купує ці товари,
вживаючи на їх виробництво частину робочого часу, який
є в його розпорядженні, отже, купує їх за допомогою певної
кількості робочого часу, яким воно може порядкувати. Та частина
суспільства, якій в наслідок поділу праці припадає вживати
свою працю на виробництво цих певних товарів, мусить
дістати еквівалент у суспільній праці, представленій у товарах,
які задовольняють її потреби. Але не існує ніякого необхідного,
\parbreak{}  %% абзац продовжується на наступній сторінці
