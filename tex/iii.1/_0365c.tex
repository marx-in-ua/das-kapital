\parcont{}  %% абзац починається на попередній сторінці
\index{iii1}{0365}  %% посилання на сторінку оригінального видання
чином основи ділення зиску між двома видами капіталістів перетворюються
потайки в основи існування зиску, який належить
поділити, додаткової вартості, яку незалежно від всякого пізнішого
ділення капітал як такий здобуває з процесу репродукції.
З того, що процент протистоїть підприємницькому доходові, а
підприємницький дохід процентові, що вони обидва протистоять
один одному, а не праці, випливає\dots{} що підприємницький дохід
плюс процент, тобто зиск, отже додаткова вартість, ґрунтуються
— на чому? На антагоністичній формі обох його частин!
Але зиск виробляється раніше, ніж відбувається цей поділ його,
і раніше, ніж може бути мова про цей поділ.

\looseness=1
Капітал, що дає процент, виявляє себе таким лиш остільки,
оскільки віддані в позику гроші дійсно перетворюються в капітал
і оскільки виробляється надлишок, частина якого є процент.
Але це не усуває того, що з ним, незалежно від процесу виробництва,
зростається властивість давати проценти. Адже робоча
сила також виявляє свою вартостетворчу силу лиш тоді, коли
вона діє і реалізується в процесі праці; але це не виключає того,
що вона в собі, потенціально, як здатність, є вартостетворча
діяльність, і як така вона не виникає лише з процесу, а, навпаки,
є його передумовою. Вона купується як здатність створювати
вартість. Але хтонебудь може її купити й не для того, щоб заставляти
її продуктивно працювати; наприклад, для чисто особистих
цілей, для послуг і~\abbr{т. д.} Так само й з капіталом. Це вже справа
позичальника, чи використовує він капітал як капітал, тобто,
чи дійсно він приводить у рух імманентну капіталові властивість
виробляти додаткову вартість. В обох випадках він оплачує
додаткову вартість, яка в собі, в можливості, міститься в товарі
капітал.

\pfbreak

\noindent{}
Розгляньмо тепер докладніше підприємницький дохід.

Якщо момент специфічної суспільної визначеності капіталу
при капіталістичному способі виробництва — власність на капітал,
яка має властивість командувати над працею інших — є фіксований
і тому процент виступає як частина додаткової вартості, яку
виробляє капітал у цьому зв’язку, то друга частина додаткової
вартості — підприємницький дохід — необхідно виступає так, ніби
вона походить не з капіталу як капіталу, а з процесу виробництва,
відокремлено від його специфічної суспільної визначеності,
яка у вислові процент на капітал адже набула вже
свого особливого способу існування. Але процес виробництва,
відокремлено від капіталу, є процес праці взагалі. Тому промисловий
капіталіст, у відміну від власника капіталу, виступає не
як функціонуючий капітал, а ніби як службовець, також відокремлено
від капіталу, ніби простий носій процесу праці взагалі,
робітник, і саме найманий робітник.

Процент сам по собі виражає саме буття умов праці як капіталу,
в їх суспільній протилежності до праці і в їх перетворенні
\parbreak{}  %% абзац продовжується на наступній сторінці
