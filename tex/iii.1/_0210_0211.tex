\parcont{}  %% абзац починається на попередній сторінці
\index{iii1}{0210}  %% посилання на сторінку оригінального видання
цінами виробництва і в кінцевому рахунку визначають їх. Навпаки,
конкуренція показує: 1) пересічні зиски, які є незалежні від
органічного складу капіталу в різних сферах виробництва, отже,
і від маси живої праці, привласненої даним капіталом у даній сфері
експлуатації; 2) підвищення і падіння цін виробництва в наслідок
зміни висоти заробітної плати — явище, яке на перший
погляд цілком суперечить вартісному відношенню товарів;
3) коливання ринкових цін, які за даний період часу зводять
пересічну ринкову ціну товарів не до ринкової \emph{вартості}, а до
ринкової ціни виробництва, яка відхиляється від цієї ринкової
вартості, дуже відмінна від неї. Всі ці явища, як \emph{здається}, в такій
самій мірі суперечать визначенню вартості робочим часом,
як і природі додаткової вартості, яка складається з неоплаченої
додаткової праці. \emph{Отже, в конкуренції все з’являється у перекрученому
вигляді}. Економічні відносини в готовому вигляді, як
вони виявляються на поверхні, в їх реальному існуванні, отже,
і в тих уявленнях, за допомогою яких носії та агенти цих
відносин намагаються їх собі з’ясувати, дуже відрізняються
від їх внутрішньої, істотної, але скритої суті (Kerngestalt) та відповідного
цій суті поняття і в дійсності перекручені та протилежні
цій суті та відповідному їй поняттю.

Далі. Коли капіталістичне виробництво досягає певного ступеня
розвитку, вирівнення різних норм зиску окремих сфер виробництва
в одну загальну норму зиску зовсім не відбувається
тільки через гру притягання і відштовхування, за допомогою
якої ринкові ціни притягають або відштовхують капітал. Після
того, як за певний період часу встановились пересічні ціни
і відповідні їм ринкові ціни, до \emph{свідомості} окремих капіталістів
доходить, що в цьому процесі вирівнення вирівнюються \emph{певні
ріжниці}, так що вони відразу включають їх у свої взаємні розрахунки.
В уявленні капіталістів ці ріжниці живуть і включаються
ними в обрахунки як підстави для компенсації.

Основне уявлення при цьому є сам пересічний зиск, —
уявлення, що рівновеликі капітали за однакові періоди часу мусять
давати рівновеликі зиски. В основі цього уявлення знов
таки лежить уявлення, що капітал кожної сфери виробництва
повинен pro rata [пропорціонально] своїй величині брати участь
в сукупній додатковій вартості, видушеній з робітників сукупним
суспільним капіталом; або що кожний окремий капітал треба
розглядати тільки як частину сукупного капіталу, а кожного
капіталіста в дійсності — як акціонера спільного підприємства,
який бере участь в сукупному зиску pro rata величині своєї частини
капіталу.

На цьому уявленні базується потім обрахунок капіталіста.
Так, наприклад, якщо капітал обертається повільніше — або тому,
що товар довше затримується в процесі виробництва, або тому,
що він мусить бути проданий на віддалених ринках, — то зиск,
який в наслідок цього вислизає з рук капіталіста, він все ж нараховує,
\index{iii1}{0211}  %% посилання на сторінку оригінального видання
отже, відшкодовує себе тим, що робить надбавку до
ціни. Абож, коли капіталовкладення, яким загрожують дуже
великі небезпеки, як, наприклад, у мореплавстві, одержують відшкодування
шляхом надбавки до ціни. Як тільки капіталістичне
виробництво, а разом з ним і страхова справа досягають певного
ступеня розвитку, небезпека фактично стає однаковою для
всіх сфер виробництва (див. Корбет); але підприємства, яким
найбільше загрожує небезпека, платять вищу страхову премію
і відшкодовують себе за це в ціні своїх товарів. На практиці
все це зводиться до того, що кожна обставина, яка робить
певне капіталовкладення менш зисковним, а друге більш зисковним,
— а всі вони в певних межах вважаються однаково необхідними,
— включається в обрахунок як раз назавжди встановлена
підстава для компенсації, при чому вже немає потреби в новій
і новій діяльності конкуренції, щоб виправдати такий мотив або
фактор обрахунку. Капіталіст забуває тільки, — або, скоріше,
не бачить, бо конкуренція йому цього не показує, — що всі ці
підстави для компенсації, які капіталісти висувають один проти
одного у взаємному обчисленні товарних цін різних галузей
виробництва, базуються просто на тому, що всі капіталісти мають
pro rata [пропорціонально] їх капіталові однакові домагання
щодо спільної здобичі, сукупної додаткової вартості. Навпаки,
через те що одержаний ними зиск відрізняється від видушеної
ними додаткової вартості, їм \emph{здається}, що їх підстави для
компенсації не вирівнюють їх участі в сукупній додатковій
вартості, а \emph{створюють самий зиск}, бо цей останній, мовляв,
виникає просто з так чи інакше мотивованої надбавки до витрат
виробництва товарів.

У всьому іншому і для пересічного зиску має силу те, що
було сказано в розділі VII, стор. 148, про уявлення капіталіста
щодо джерела додаткової вартості. Тут справа стоїть інакше
лиш остільки, оскільки при даній ринковій ціні товарів і даній
експлуатації праці заощадження на витратах виробництва залежить
від індивідуальної вправності, уважності і~\abbr{т. д.}
