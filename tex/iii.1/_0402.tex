\parcont{}  %% абзац починається на попередній сторінці
\index{iii1}{0402}  %% посилання на сторінку оригінального видання
коли війна була зовсім інша, ніж невелика Кримська війна? По-друге, якщо природне джерело всохло, з
якого ж джерела припливав капітал? Англія, як відомо, не брала позик у чужих націй.
Але якщо поряд з природним джерелом існує ще й штучне, то
це ж було б найприємнішим методом для нації користуватися
природним джерелом для війни, а штучним — для справ. Якщо ж
у наявності був тільки старий грошовий капітал, то хіба ж
він міг високим розміром процента подвоїти свою продуктивність?
Пан Оверстон, очевидно, гадає, що щорічні заощадження країни
(які, однак, у цьому випадку нібито були спожиті) перетворюються тільки в грошовий капітал. Але якби
не відбувалося
дійсного нагромадження, тобто підвищення виробництва і збільшення засобів виробництва, то яка
користь була б від нагромадження боргових вимог у грошовій формі на це виробництво?

Підвищення „вартості капіталу“, яке випливає з високої
норми зиску, Оверстон сплутує з підвищенням, яке випливає
із збільшеного попиту на грошовий капітал. Цей попит може
підвищитись з причин, цілком незалежних від норми зиску. Сам
Оверстон наводить як приклад, що в 1847 році попит зріс в наслідок знецінення реального капіталу.
Залежно від того, що для
нього зручніше, він відносить вартість капіталу то до реального
капіталу, то до грошового капіталу.

Нечесність нашого банкового лорда, разом з його обмеженою
банкірською точкою зору, яку він дидактично підкреслює, виявляється далі ось у чому: „3728.
(Запитання:) „Ви сказали, що,
на вашу думку, норма дисконту не має істотного значення для
купця; чи не будете ласкаві сказати, що ви вважаєте звичайною нормою зиску?“ Пан Оверстон заявив, що
відповісти на це
питання „неможливо“. — „3729. Припустім, пересічна норма
зиску є 7--10\%; тоді зміна норми дисконту з 2\% до 7 чи 8\%
мусить істотно вплинути на норму зиску, — чи не так?“ [Само
питання сплутує норму підприємницького доходу і норму зиску
і випускає з уваги, що норма зиску є спільним джерелом процента
і підприємницького доходу. Норма процента може лишити без
зміни норму зиску, але не підприємницький дохід. Відповідь Оверстона:] „По-перше, ділові люди не
стануть платити такої норми
дисконту, яка забирає в них значну частину їх зиску; вони краще
припинять свої підприємства“. [Звичайно, якщо вони можуть це
зробити, не руйнуючи себе. Поки їх зиск високий, вони платять
дисконт, тому що хочуть цього, а якщо він низький, то вони платять,
тому що мусять платити.] „Що значить дисконт? Чому хтось
дисконтує вексель?\dots{} Тому що він хоче одержати більший капітал“; [Стій! тому що він хоче
антиципувати зворотний приплив у грошовій формі свого міцно вкладеного капіталу і
уникнути припинення свого підприємства. Тому що він мусить
покрити платежі, яким надійшов строк. Збільшення капіталу він
бажає тільки тоді, коли справи йдуть добре, або — якщо він спекулює на чужий капітал — навіть тоді
коли вони йдуть погано.
\parbreak{}  %% абзац продовжується на наступній сторінці
