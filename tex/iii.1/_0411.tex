
\index{iii1}{0411}  %% посилання на сторінку оригінального видання
„3896. Отже, чи правильно я вас розумію, що ви відмовляєтесь
від того положення, яким ви аргументували в 1840 році: що коливання в кількості банкнот Англійського
банку, які перебувають
у циркуляції, повинні відбуватись відповідно до коливань у сумі
золотого запасу? — Я відмовляюсь від цього положення остільки\dots{}
оскільки при сучасному стані наших знань ми мусимо додати до
банкнот, які перебувають в циркуляції, ще й ті банкноти, які
лежать у банковому резерві Англійського банку“. Це — пречудово. Довільна постанова, що банк може
випустити стільки
паперових банкнот, скільки в нього є золота в сховищах, і ще на
14 мільйонів крім того, зумовлює, звичайно, те, що випуск банком
банкнот коливається разом з коливаннями золотого запасу. Але
тому що „сучасний стан наших знань“ ясно показав, що кількість
банкнот, яку згідно з цією постановою банк може фабрикувати (і яку
Issue department передає в Banking department), — що ця циркуляція
між обома департаментами Англійського банку, яка коливається
разом з коливаннями золотого запасу, не визначає коливань
циркуляції банкнот поза стінами Англійського банку, то ця
остання, дійсна циркуляція, тепер для адміністрації банку не має
значення, і вирішальною стає тільки циркуляція між обома
департаментами банку, відмінність якої від дійсної циркуляції
відбивається на резерві. Для зовнішнього світу ця циркуляція
важлива лиш остільки, оскільки резерв показує, наскільки банк
наблизився до встановленого законом максимуму випуску банкнот і
скільки клієнти банку можуть ще одержати з Banking department.

Ось блискучий приклад mala fides [несумлінності] Оверстона:

„4243. Чи коливається, на вашу думку, кількість капіталу від
одного місяця до другого в такій мірі, щоб вартість його в наслідок цього могла змінитися так, як ми
це бачили за
останні роки на коливаннях норми дисконту? — Співвідношення
між попитом і поданням капіталу може, безсумнівно, коливатися навіть в короткі періоди часу\dots{} Якщо
Франція завтра оповістить, що вона хоче одержати дуже велику позику, то це, безсумнівно, відразу ж
спричинить в Англії велику зміну \emph{в вартості грошей, тобто в вартості капіталу}".

„4245. Якщо Франція оповістить, що їй негайно потрібно для
якої-небудь мети на 30 мільйонів товарів, то виникне, вживаючи
більш наукового і простішого вислову, великий попит на \emph{капітал}".

„4246. \emph{Капітал}, який Франція хотіла б купити за допомогою
своєї позики, — це \emph{одна} справа; \emph{гроші}, на які Франція купує його, — це інша справа; що ж змінює свою
вартість, — \emph{гроші} чи ні? — Ми знову вертаємось до старого питання, і я гадаю, що це питання
підходить більше для кабінету вченого, ніж для цього залу засідань комітету“. І з цими словами він
виходить, але не
до кабінету вченого\footnote{
Докладніше про плутанину понять у Оверстона в питаннях капіталу — в кінці XXXII розділу. — [\emph{Ф.
E.}]
}.
