\parcont{}  %% абзац починається на попередній сторінці
\index{iii1}{0352}  %% посилання на сторінку оригінального видання
капіталу в окремих сферах виробництва. Цим здійсненим таким
чином вирівненням пересічних ринкових цін товарів у ціни виробництва
корегуються відхилення окремих норм зиску від загальної
або пересічної норми зиску. Цей процес ніколи не
виявляється і не може виявлятися таким чином, щоб промисловий
або торговельний капітал як \emph{такий} був подібно до капіталу,
що дає процент, товаром відносно покупця. Оскільки він
виявляється, він виявляється тільки в коливаннях і вирівненнях
ринкових цін товарів у ціни виробництва, а не як безпосереднє
встановлення пересічного зиску. Справді, загальна норма зиску
визначається 1) додатковою вартістю, виробленою сукупним
капіталом, 2) відношенням цієї додаткової вартості до вартості
сукупного капіталу і 3) конкуренцією, але лиш остільки,
оскільки вона є той рух, за допомогою якого капітали, вкладені
в окремі сфери виробництва, намагаються здобути з цієї
додаткової вартості однакові дивіденди, пропорціонально до їх
відносних величин. Отже, загальна норма зиску дійсно визначається
цілком іншими і багато складнішими причинами, ніж ринкова
норма процента, яка прямо й безпосередньо визначається
відношенням попиту й подання, і тому загальна норма зиску
не є таким наочним і даним фактом, яким є розмір процента.
Окремі норми зиску в різних сферах виробництва самі є більш-менш
непевні; але оскільки вони виявляються, виявляється не
їх одноманітність, а їх відмінність. Загальна ж норма зиску сама
виступає тільки як мінімальна межа зиску, а не як емпірична,
безпосередньо видима форма дійсної норми зиску.

Відзначаючи цю ріжницю між нормою процента та нормою
зиску, ми ще навіть залишаємо осторонь такі дві обставини, що
сприяють консолідації розміру процента: 1) історична першість
капіталу, що дає процент, та існування традиційно передаваного
загального розміру процента; 2) значно більший безпосередній
вплив, що його справляє світовий ринок, незалежно від умов
виробництва даної країни, на встановлення розміру процента, порівняно
з його впливом на норму зиску.

Пересічний зиск виступає не як безпосередньо даний факт,
а як кінцевий результат вирівнення протилежних коливань, який
можна встановити лише за допомогою дослідження. Інакше
стоїть справа з розміром процента. При своїй, принаймні місцевій,
загальновизнаності, розмір процента є щоденно фіксований
факт, факт, який служить навіть наперед даною умовою і рубрикою
промисловому й торговельному капіталові для калькуляції
при його операціях. Він стає загальною спроможністю
кожної грошової суми в 100\pound{ фунтів стерлінгів} давати 2, 3, 4,
5\%. Метеорологічні бюлетені не відзначають з більшою точністю
висоту барометра й термометра, ніж біржові бюлетені —
висоту розміру процента, не для того або іншого капіталу, а для
капіталу, що перебуває на грошовому ринку, тобто взагалі для
капіталу, що дається в позику.
