
\index{iii1}{0092}  %% посилання на сторінку оригінального видання
В формулі $р' \deq{} m'n \frac{v}{K}$, як сказано, $m'n$ є те, що в другій
книзі названо річною нормою додаткової вартості. У вищенаведеному
випадку вона становить 153\sfrac{11}{13}\% × 8\sfrac{1}{2}, або точно
1307\sfrac{9}{13}\%. Отже, якщо якийсь бравий чоловік сплеснув руками
з приводу потворності річної норми додаткової вартості в 1000\%,
наведеної в одному прикладі в другій книзі, то він, може, заспокоїться
на факті річної норми додаткової вартості понад
1300\%, який наведено йому тут з живої практики Манчестера.
В часи найвищого розквіту, яких ми, правда, давно вже не переживали,
така норма аж ніяк не є рідкість.

До речі сказати, ми маємо тут приклад дійсного складу капіталу
в сучасній великій промисловості. Весь капітал поділяється на
\num{12182}\pound{ фунти стерлінгів} сталого і 318\pound{ фунтів стерлінгів} змінного
капіталу, разом \num{12500}\pound{ фунтів стерлінгів}. Або в процентах:
$97\sfrac{1}{2}c+ 2\sfrac{1}{2}v \deq{} 100K$. Тільки сорокова частина всього капіталу,
але, повторно обертаючись більше ніж вісім разів на рік, служить
для виплати заробітної плати.

Через те що, звичайно, тільки небагатьом капіталістам спадає
на думку робити такі обчислення щодо свого власного підприємства,
то статистика майже абсолютно мовчить про відношення
сталої частини всього суспільного капіталу до змінної
частини. Тільки американський перепис дає те, що можливе при
сучасних відносинах: суму заробітної плати, виплаченої в кожній
галузі підприємств, і одержаних зисків. Хоч і які підозрілі ці
дані, — бо вони основані тільки на неперевірених повідомленнях
самих промисловців, — проте вони надзвичайно цінні і становлять
усе, що ми маємо про цей предмет. В Европі ми занадто делікатні,
щоб вимагати від наших великих промисловців подібних
викрить. — \emph{Ф.~Е.}]

\sectionextended{Економія в застосуванні сталого капіталу}{\subsection{Загальні положення}}


Збільшення абсолютної додаткової вартості, або здовження
додаткової праці, отже й робочого дня, при незмінній величині
змінного капіталу, тобто при вживанні того самого числа робітників
за ту саму номінально заробітну плату, — при чому байдуже,
чи оплачується надурочний час чи ні, — відносно знижує
вартість сталого капіталу порівняно з вартістю всього капіталу
і змінного капіталу і підвищує цим норму зиску, знов таки
незалежно від зростання й маси додаткової вартості і можливого
підвищення норми додаткової вартості. Розмір основної
частини сталого капіталу, фабричних будівель, машин тощо лишається
той самий, однаково, чи працюють за його допомогою 16,
чи 12 годин. Здовження робочого дня не вимагає ніяких нових
затрат на цю найдорожчу частину сталого капіталу. До цього долучається
\index{iii1}{0093}  %% посилання на сторінку оригінального видання
ще й те, що вартість основного капіталу таким чином репродукується
за коротший ряд періодів обороту, отже, скорочується
час, на який він мусить бути авансований, щоб одержати
певний зиск. Тому здовження робочого дня збільшує зиск навіть
тоді, коли надурочний час оплачується, а до певної міри навіть
тоді, коли він оплачується вище, ніж нормальні робочі години.
Тому постійно зростаюча при сучасній промисловій системі необхідність
збільшення основного капіталу була для ненаситно
жадливих до зиску капіталістів головним стимулом до здовження
робочого дня\footnote{
„Через те що на всіх фабриках дуже висока сума основного капіталу
вкладена в будівлі і машини, зиск буде тим більший, чим більше число годин,
протягом яких ці машини можуть бути в роботі“ („Rep. of Insp. of Fact., 31.
October 1858“, стор. 8).
}.

Інші умови маємо при сталому робочому дні. В цьому випадку
для того, щоб експлуатувати більшу масу праці, треба
або збільшити число робітників, і разом з тим до певної міри
масу основного капіталу, будівель, машин і~\abbr{т. д.} (бо ми тут
залишаємо осторонь відрахування з заробітної плати або зниження
заробітної плати нижче її нормальної висоти). Абож, якщо
збільшується інтенсивність праці чи підвищується продуктивність
праці, якщо взагалі виробляється більше відносної додаткової
вартості, то в тих галузях промисловості, які застосовують
сировинний матеріал, зростає маса обігової частини
сталого капіталу, бо за даний період часу переробляється
більше сировинного матеріалу і~\abbr{т. д.}; і, подруге, зростає кількість
машин, які приводяться в рух тим самим числом робітників,
отже, і відповідна частина сталого капіталу. Зростання додаткової
вартості супроводиться, отже, зростанням сталого капіталу, зростаюча
експлуатація праці — подорожчанням тих умов виробництва,
за допомогою яких експлуатується праця, тобто більшими
витратами капіталу. Отже, через це норма зиску з одного
боку зменшується, тимчасом як з другого боку вона підвищується.

Цілий ряд поточних затрат лишається майже або цілком
однаковий як при довшому, так і при коротшому робочому дні.
Витрати нагляду менші при 500 робітниках і 18-годинному робочому
дні, ніж при 750 робітниках і 12-годинному робочому дні.
„Витрати ведення фабрики при десятигодинній праці майже однаково
високі, як і при дванадцятигодинній“ („Rep. of Insp. of
Fact., Oct. 1848“, стор. 37). Державні та комунальні податки,
страхування від огню, заробітна плата різних постійних службовців,
зневартнення машин і різні інші затрати фабрики лишаються
незмінними при довгому чи короткому робочому дні;
в міру того, як скорочується виробництво, вони підвищуються
коштом зиску („Rep. of Insp. of Fact., Oct. 1862“, стор. 19).

Період часу, протягом якого репродукується вартість машин
і інших складових частин основного капіталу, на практиці визначається
\index{iii1}{0094}  %% посилання на сторінку оригінального видання
не тим часом, протягом якого вони просто існують,
а загальною тривалістю процесу праці, на протязі якого вони
функціонують і використовуються. Якщо робітники мусять працювати
18 годин замість 12, то це становить за тиждень на три
дні більше, тиждень перетворюється в півтора тижня, два
роки — в три. Отже, якщо надурочний час не оплачується, то
робітники, крім нормального часу додаткової праці, дають задарма
на кожні два тижні третій, на кожні два роки третій.
І таким чином репродукція вартості машин прискорюється на 50\%
і закінчується за \sfrac{2}{3} часу, необхідного при звичайних умовах.

У цьому дослідженні, так само як і в дослідженні коливань
ціни сировинного матеріалу (в розд. VI), ми, щоб уникнути
зайвих ускладнень, виходимо з припущення, що масу і норму
додаткової вартості дано.

Як уже зазначено при розгляді кооперації, поділу праці
і ролі машин, економія в умовах виробництва, яка характеризує
виробництво у великому масштабі, в істотному виникає з того,
що ці умови функціонують як умови суспільної, суспільно-комбінованої
праці, отже, як суспільні умови праці. Вони
споживаються у процесі виробництва спільно, колективним робітником,
замість споживатись у роздрібненій формі масою
незв’язаних між собою робітників або в кращому разі робітниками,
в незначній мірі безпосередньо зв’язаними відносинами співробітництва.
На великій фабриці з одним або двома центральними
двигунами витрати на ці двигуни зростають не в тій самій пропорції,
в якій зростає кількість їх кінських сил, і отже можлива сфера
їх діяння; витрати на передатні механізми зростають не в тій самій
пропорції, в якій зростає маса робочих машин, яким вони передають
рух; самий корпус робочої машини дорожчає не в тій
пропорції, в якій збільшується число знарядь, якими вона діє
як своїми органами, і~\abbr{т. д.} Далі, концентрація засобів виробництва
дає заощадження на будівлях усякого роду, не тільки
на власне майстернях, але й на складських приміщеннях і~\abbr{т. д.}
Так само стоїть справа з видатками на опалення, освітлення
і~\abbr{т. д.} Інші умови виробництва лишаються ті самі, все одно,
багато чи мало людей використовує їх.

Але вся ця економія, яка виникає з концентрації засобів виробництва
та їх масового застосування, передбачає, як істотну
умову, скупчення й спільну діяльність робітників, тобто суспільну
комбінацію праці. Отже, вона виникає з суспільного характеру
праці цілком так само, як додаткова вартість виникає
з додаткової праці кожного окремого робітника, розглядуваного
ізольовано. Навіть постійні поліпшення, які тут можливі й потрібні,
виникають виключно з суспільних дослідів і спостережень,
що їх дає і уможливлює виробництво комбінованого у
великому масштабі колективного робітника.

Те саме стосується і другої великої галузі економії в умовах
виробництва. Ми маємо на увазі зворотне перетворення
\parbreak{}  %% абзац продовжується на наступній сторінці
