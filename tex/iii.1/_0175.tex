\parcont{}  %% абзац починається на попередній сторінці
\index{iii1}{0175}  %% посилання на сторінку оригінального видання
приводять до підвищення або падіння норми зиску, можна було б думати, що загальна норма зиску мусить
щодня змінюватись. Але рух в одній сфері виробництва знищується рухом в іншій сфері виробництва,
впливи перехрещуються і взаємно
паралізуються. Пізніше ми дослідимо, куди, в яку сторону, прагнуть в кінцевому рахунку ці коливання;
але вони — повільні;
раптовість, многобічність і різна тривалість цих коливань у
різних сферах виробництва приводять до того, що вони почасти компенсуються в своєму чергуванні в
часі, так що за
підвищенням цін іде падіння цін, і навпаки; отже, приводять до того, що коливання лишаються
місцевими, тобто обмеженими окремою сферою виробництва; нарешті, — до того, що різні місцеві
коливання взаємно нейтралізуються. В межах кожної окремої сфери виробництва відбуваються зміни,
відхилення
від загальної норми зиску, які, з одного боку, вирівнюються
протягом певного періоду часу і тому не впливають на загальну
норму зиску; з другого ж боку, вони знову таки не впливають
на неї, бо знищуються іншими місцевими коливаннями, які відбуваються одночасно з ними. Через те що
загальна норма зиску
визначається не тільки пересічною нормою зиску в кожній сфері,
але й розподілом сукупного капіталу між різними окремими
сферами, і через те що цей розподіл постійно змінюється, то
в цьому ми знову маємо постійну причину зміни загальної норми
зиску, але таку причину зміни, яка при безперервності (Ununterbrochenheit)
% REMOVED \footnote*{
% В першому німецькому виданні тут стоїть: „при переривчатості“ („Unterbrochenheit“), — очевидна
% друкарська помилка. В рукопису Маркса тут стоїть:
% „при постійності“ („Beständigkeit“). \Red{Прим. ред. нім. вид. ІМЕЛ.}
% }
і всебічності цього руху
здебільшого знов таки
паралізує сама себе.

2)~В межах кожної сфери є певний простір для коливання
норми зиску цієї сфери протягом коротшого чи довшого періоду, поки це коливання, після ряду
підвищень чи падінь,
сконсолідується настільки, щоб мати досить часу для впливу
на загальну норму зиску, отже й для досягнення більше ніж
місцевого значення. Тому в межах таких просторових і часових границь так само мають силу закони
норми зиску, розвинені в першому відділі цієї книги.

Теоретичний погляд, — що при первісному перетворенні додаткової вартості в зиск кожна частина
капіталу рівномірно
дає зиск\footnote{
\emph{Мальтус} [„Principles of Political Econorny“, 2-е вид. Лондон, 1836, стор. 268].
}, виражає практичний факт. Який би не був склад промислового капіталу, — чи приводить він
у рух чверть мертвої
і три чверті живої праці, чи три чверті мертвої і одну чверть
живої, чи вбирає він в одному випадку втроє більше додаткової
праці, або виробляє втроє більше додаткової вартості, ніж у другому, — в обох випадках він дає
однаковий зиск, якщо ступінь
експлуатації праці є незмінний і якщо ми абстрагуємось від
\parbreak{}  %% абзац продовжується на наступній сторінці
