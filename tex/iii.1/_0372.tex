\parcont{}  %% абзац починається на попередній сторінці
\index{iii1}{0372}  %% посилання на сторінку оригінального видання
Bank в 1863 році: вкладений капітал \num{1000000}\pound{ фунтів стерлінгів},
вклади \num{14540275}\pound{ фунтів стерлінгів}. В Union Bank of London в
1863 році: вкладений капітал \num{600000}\pound{ фунтів стерлінгів}, вклади
\num{12384173}\pound{ фунтів стерлінгів}.

Змішання підприємницького доходу з платою за нагляд
або управління первісно виникло з тієї антагоністичної форми,
якої надлишок зиску понад процент набирає в протилежність
до проценту. Воно розвинулося далі з апологетичного наміру
зобразити зиск не як додаткову вартість, тобто неоплачену працю,
а як заробітну плату самого капіталіста за виконувану ним працю.
Цьому з боку соціалістів була протиставлена вимога звести зиск
фактично до того, чим його виставлялося в теорії, а саме до простої
плати за нагляд. І ця вимога виступала проти теоретичного
прикрашування тим неприємніше, чим більше ця плата за нагляд,
з утворенням численного класу промислових і торговельних
управителів\footnote{
„Masters are labourers as well as their journeymen. In this character their
interest is precisely the same as that of their men. But they are also either capitalists,
or the agents of capitalists, and in this respect their interest is decidedly
opposed to the interest of the workmen“. [„Майстри — такі ж робітники, як і їх
поденники. В цьому відношенні їх інтереси цілком ті самі, що й інтереси їх людей.
Але, крім того, вони є або капіталісти або агенти капіталістів, і в цьому відношенні
їх інтереси рішуче протилежні інтересам робітників“] (стор. 27). „The
wide spread of education among the journeymen mechanics of this country diminishes
daily the value of the labour and skill of almost all masters and employers
by increasing the number of persons who possess their peculiar knowledge“ [„Значне
поширення освіти серед промислових робітників цієї країни з кожним днем
зменшує вартість праці і вправності майже всіх майстрів і підприємців, збільшуючи
число осіб, які мають такі ж спеціальні знання“] (стор 30. \emph{Hodgskin}:
„Labour defended against the Claims of Capital etc.“ London 1825).
}, з одного боку, знаходила, як і всяка інша заробітна
плата, свій певний рівень і свою певну ринкову ціну; і чим
нижче, з другого боку, вона падала, як і всяка плата за вправну
працю, разом із загальним розвитком, що знижував витрати виробництва
спеціально навченої робочої сили\footnote{
„The general relaxation of conventional barriers, the increased facilities of education
tend to bring down the wages of skilled labour Instead of raising those of
the unskilled“ („Загальне ослаблення умовних перепон і збільшення можливості
дістати освіту діють в напрямі зниження плати кваліфікованої праці замість
того, щоб підвищувати плату некваліфікованої“] (\emph{J.~St.~Mill}: „Principles of Political
Economy“. 2 вид., Лондон 1849, І, стор. 463).
}. З розвитком кооперації
серед робітників, акційних підприємств серед буржуазії,
був знищений і останній привід для змішання підприємницького
доходу з платою за управління, і зиск і на практиці
виступив як те, чим він незаперечно був теоретично — як проста
додаткова вартість, як вартість, за яку не сплачено ніякого
еквіваленту, як реалізована неоплачена праця; так що функціонуючий
капіталіст дійсно експлуатує працю, і плід його експлуатації,
якщо він працює з узятим в позику капіталом, ділиться на процент
і підприємницький дохід, — надлишок зиску понад процент.

На базі капіталістичного виробництва в акційних підприємствах
розвивається нове шахрайство з платою за управління,
\parbreak{}  %% абзац продовжується на наступній сторінці
