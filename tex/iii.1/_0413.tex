\parcont{}  %% абзац починається на попередній сторінці
\index{iii1}{0413}  %% посилання на сторінку оригінального видання
III.~Утворення акційних товариств. Звідси:

1)~Неймовірне розширення розмірів виробництва і підприємств, які для поодиноких капіталів були
неможливі. Разом з тим
такі підприємства, які раніше були державними підприємствами,
стають суспільними.

2)~Капітал, який сам по собі грунтується на суспільному
способі виробництва і передбачає суспільну концентрацію засобів виробництва та робочих сил, набирає
тут безпосередньо
форми суспільного капіталу (капіталу безпосередньо асоційованих індивідів) у протилежність до
приватного капіталу, а його
підприємства виступають як суспільні підприємства у протилежність до приватних підприємств. Це —
скасування (Aufhebung)
капіталу як приватної власності в межах самого капіталістичного способу виробництва.

3)~Перетворення дійсно функціонуючого капіталіста в простого управителя, завідувача чужим капіталом,
а власників
капіталу — в простих власників, у простих грошових капіталістів. Якщо навіть одержувані ними
дивіденди містять у собі
процент і підприємницький дохід, тобто весь зиск (бо утримання управителя є, або має бути, просто
заробітною платою за
певного роду вправну працю, ціна якої регулюється на ринку
праці, як і ціна всякої іншої праці), то все ж весь цей зиск
одержується тільки в формі процента, тобто як проста винагорода за власність на капітал, яка таким
чином цілком відокремлюється від функції в дійсному процесі репродукції, подібно
до того як ця функція в особі управителя відокремлюється від
власності на капітал. Таким чином зиск виступає (вже не одна
тільки частина його, процент, який дістає своє виправдання
в зиску позичальника) як просте привласнення чужої додаткової
праці, яке виникає з перетворення засобів виробництва в капітал, тобто з їх відчуження від дійсних
виробників, з їх протилежності як чужої власності до всіх дійсно діючих у виробництві індивідів, від
управителя до останнього поденника.
В акційних товариствах функція відокремлена від власності на
капітал, отже й праця цілком відокремлена від власності на
засоби виробництва і на додаткову працю. Це — результат найвищого розвитку капіталістичного
виробництва, необхідний
перехідний пункт до зворотного перетворення капіталу у власність виробників, але вже не у приватну
власність поодиноких
виробників, а в їх власність як асоційованих виробників, у безпосередньо суспільну власність. З
другого боку, це перехідний пункт до перетворення всіх функцій у процесі репродукції,
досі ще зв’язаних з власністю на капітал, у прості функції асоційованих виробників, у суспільні
функції.

Раніше ніж піти далі, слід ще відзначити такий економічно
важливий факт: через те що зиск набирає тут чисту форму процента, то такі підприємства можливі й
тоді, коли вони дають самий
тільки процент, і це — одна з тих причин, які затримують падіння
\parbreak{}  %% абзац продовжується на наступній сторінці
