\parcont{}  %% абзац починається на попередній сторінці
\index{iii1}{0407}  %% посилання на сторінку оригінального видання
фонди без покриття тільки в найрідкіших випадках (він був
банкіром моєї фірми в Манчестері), то так само ясно й те, що
його прекрасні описи тих мас капіталу, які великодушні банкіри
позичають фабрикантам, що потребують капіталу, є лукава
брехня.

Зрештою, в розділі XXXII Маркс каже по суті те саме:
„Попит на засоби платежу є просто попит на \emph{перетворення в
гроші}, оскільки купці й виробники можуть дати добре забезпечення; він є попит на \emph{грошовий капітал},
оскільки вони не
дають такого забезпечення, оскільки, отже, позика засобів платежу дає їм не тільки \emph{грошову форму}, а
\emph{еквівалент}, — в будь-якій формі, — якого їм не вистачає для платежу“. — Далі,
в розділі XXXIII: „При розвиненому кредиті, коли гроші концентруються в руках банків, банки,
\emph{принаймні номінально}, авансують гроші. Це авансування стосується тільки тих грошей,
% REMOVED \footnote*{
% В першому німецькому виданні: „не стосується тих грошей“; виправлено
% на підставі цитованого тексту і рукопису Маркса. \Red{Примітка ред. нім. вид. ІМЕЛ.}
% }
що перебувають
в циркуляції. Це — авансування \emph{засобів
циркуляції}, а не авансування капіталів, які циркулюють за допомогою цих засобів“. — Пан Chapmann,
який мусить це знати,
також підтверджує вищенаведене розуміння дисконтної справи:
„Bank Committee“ 1857: „У банкіра є вексель, банкір \emph{купив вексель}“. Evidence. Запитання 5139.

А втім, ми ще повернемось до цієї теми в розділі
XXVIII — \emph{Ф.~Е.}]

„3744. Чи не будете ласкаві описати, що ви в дійсності розумієте під висловом капітал? — (Відповідь
Оверстона:) Капітал
складається з різних товарів, за допомогою яких підтримується
хід підприємства (capital consists of various commodities, by the
means of which trade is carried on); буває капітал основний і буває капітал обіговий. Ваші кораблі,
ваші доки, ваші верфі —
основний капітал; ваші харчові продукти, ваш одяг і~\abbr{т. д.} — обіговий капітал“.

„3745. Чи має відплив золота за кордон шкідливі наслідки для
Англії? — Ні, якщо з цим словом зв’язується раціональне розуміння“. [Тепер з’являється стара теорія
грошей Рікардо]\dots{} „При
природному стані речей гроші всього світу розподіляються між
різними країнами світу в певних пропорціях; пропорції ці такого
роду, що при подібному розподілі“ [грошей] „зносини між якоюнебудь країною, з одного боку, і всіма
іншими країнами світу, з другого боку, є простий обмін; але існують впливи, які час
від часу порушують цей розподіл, і якщо такі впливи виникають,
то частина грошей даної країни відпливає в інші країни“. — „3746.
Ви уживаєте тепер вислову: гроші. Якщо я раніше зрозумів вас
правильно, ви називали це втратою капіталу. — Що я називав
втратою капіталу?“ — „3747. Відплив золота. — Ні, я цього не
казав. Якщо ви золото вважаєте за капітал, тоді це, без сумніву,
\parbreak{}  %% абзац продовжується на наступній сторінці
