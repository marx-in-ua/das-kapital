Розділ двадцять восьмий

Засоби циркуляції і капітал; погляди Тука
і Фуллартона

Ріжниця між засобами циркуляції і капіталом, як її зображають Тук,90 Вільсон та інші, при чому вони
без ладу переплутують ріжниці між засобами циркуляції, як грішми, грошовим капіталом взагалі і
капіталом, що дає процент (moneyed
capital в англійському значенні), зводиться до таких двох пунктів.

Засоби циркуляції циркулюють, з одного боку, як монета
(гроші), оскільки вони опосереднюють витрачання доходів, отже,
обмін між індивідуальними споживачами і роздрібними торговцями, — до цієї категорії слід зарахувати
всіх купців, що продають споживачам, індивідуальним споживачам, у відміну від
продуктивних споживачів або виробників. Тут гроші циркулю-

80 Ми наводимо тут в оригіналі відповідне місце з Тука, цитоване у витягу
німецькою мовою на стор. 440—441*: „The business of bankers, setting aside the
issue of promissory notes payable on demand, may be divided into two branches,
corresponding with the distinction pointed out by Dr. (Adam) Smith of the transactions between
dealers and dealers, and between dealers and consumers. One branch
of the bankers’business is to collect capital from those, who have not immediate
employment for it, and to distribute or transfer it to those who have. The other
branch is to receive deposits of the incomes of their customers, and to pay out
the amount, as it is wanted for expenditure by the latter in the objects of their
consumption... the former being a circulation of capital, the latter of currency“ [Залишаючи
осторонь випуск банкнот з платежем пред’явникові, банкірська справа
може бути поділена на дві галузі, які відповідають встановленому д-ром (Адамом)
Смітом розрізненню операцій між торговцями й торговцями і між торговцями й
споживачами. Одна галузь банкірської справи полягає в тому, щоб збирати
капітал від тих, хто не знаходить для нього безпосереднього застосування,
і розподіляти його або передавати тим, хто може його використати. Друга галузь
полягає в тому, щоб приймати вклади з доходів своїх клієнтів і виплачувати їм
суми в міру того, як вони їм стають потрібні для видатків на предмети споживання... Перша є
циркуляція капіталу, остання — циркуляція грошей.] —
(Tooke: „Inquiry into the Currency Principle“, стор. 36). Перша є „the concentration of capital on
the one hand and the distribution of it on the other“ [концентрація
капіталу, з одного боку, і розподіл його, з другого боку], друга є „administering
the circulation for local purposes of the district“ [управління циркуляцією для
місцевих цілей округи] (там же, стор. 37). — Далеко ближче підходить до
правильного розуміння питання Kinnear у такому місці: „Гроші вживаються
для того, щоб виконувати дві істотно різні операції. Як засіб обміну між торговцями і торговцями
вони є знаряддям, за допомогою якого здійснюються передачі капіталу; тобто обмін певної суми
капіталу в грошах на рівну суму капіталу в товарах. Але гроші, витрачені на виплату заробітної плати
і на купівлю
й продаж між торговцем і споживачем, — це не капітал, а дохід; це частина доходу всього суспільства,
яка вживається на щоденні видатки. Ці гроші циркулюють у невпинному щоденному вжитку, і тільки ці
гроші можуть бути названі
засобами циркуляції (currency) в строгому значенні слова. Надання капіталу в
позику залежить виключно від волі банку або інших володільців капіталу, —
бо позичальники завжди знайдуться; але сума засобів циркуляції залежить від
потреб усього суспільства, в межах яких гроші циркулюють для цілей щоденного витрачання“ (J. G.
Kinnear: „The Crisis and the Currency“. London 1847
[стор. З і далі]).

* Сторінки за німецьким виданням ІМЕЛ; в цьому українському виданні
відповідне місце цитується на стор. 385. Ред. укр. перекладу.
