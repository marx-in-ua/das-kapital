\parcont{}  %% абзац починається на попередній сторінці
\index{iii1}{0137}  %% посилання на сторінку оригінального видання
22\% для Шотландії і майже в 90\% для Ірландії;\footnote{
Це швидке розширення машинного виробництва лляної пряжі в Ірландії
завдало тоді смертельного удару експортові німецького (шлезького, лаузіцького,
вестфальського) полотна, тканого з ручної пряжі. — \emph{Ф. Е.}
} наслідок цього
був той, що ціна сировинного матеріалу, при одночасних поганих
урожаях льону, підвищилась на 10\pound{ фунтів стерлінгів} за
тонну, тоді як ціна пряжі впала на 6 пенсів за моток“ („Rep.
of Insp. of Fact., Oct. 1847“, стор. 30 [31]).

1849 рік. Починаючи з останніх місяців 1848 року, справи
знову пожвавились. „Ціна льону, яка стояла так низько, що забезпечувала
чималий зиск майже при всяких можливих майбутніх
обставинах, спонукала фабрикантів безупинно розвивати свої
підприємства. Фабриканти шерстяних виробів на початку року
протягом певного часу працювали з дуже великим завантаженням\dots{}
але я боюся, що комісійні операції з шерстяними товарами
часто заступали місце дійсного попиту, і що періоди позірного
розквіту, тобто періоди повного, завантаження підприємств, не
завжди збігаються з періодами справжнього попиту\dots{} Протягом
кількох місяців справи в камвольній промисловості були особливо
хороші\dots{} На початку згаданого періоду ціна на вовну була особливо
низька; прядільники запаслися нею по дешевих цінах
і, звичайно, в значній кількості. Коли під час весняних аукціонів
ціна на вовну підвищилась, то прядільники мали з цього вигоду,
і вони зберегли її, бо попит на фабрикати став значним і стійким“
(„Rep. of Insp. of Fact., [April] 1849“, стор. 42).

„Якщо ми простежимо коливання в стані справ, які відбувались
у фабричних округах за останні 3 або 4 роки, то ми муситимем,
я гадаю, визнати, що десь існує значна причина, яка
викликає порушення\dots{} Чи не є тут новим елементом колосальна
продуктивна сила зрослої кількості машин?“ („Rep. of Insp.
of Fact., April 1849“, стор. 42 [43]).

В листопаді 1848 року, в травні і влітку до жовтня 1849 року
справи все кращали. „Особливо це стосується виробництва
тканин з чесаної вовни, яке групується навколо Бредфорда
й Галіфакса; це виробництво ніколи раніше навіть приблизно
не досягало своїх теперішніх розмірів\dots{} Спекуляція на сировинному
матеріалі і непевність щодо можливого подання його
вже з самого початку викликали в бавовняній промисловості
більше збудження і частіші коливання, ніж в будьякій іншій
галузі промисловості. В даний момент тут має місце нагромадження
запасів грубіших бавовняних товарів, що занепокоює
дрібних прядільників і вже завдає їм шкоди, так що деякі з
них працюють неповний час“ („Rep. of Insp. of Fact., Oct. 1849“,
стор. 64, 65).

1850 рік. Квітень. Справи все ще йдуть добре. Виняток:
„Велика депресія в одній частині бавовняної промисловості
в наслідок недостатнього подання сировинного матеріалу саме
\parbreak{}  %% абзац продовжується на наступній сторінці
