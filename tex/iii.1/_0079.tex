\parcont{}  %% абзац починається на попередній сторінці
\index{iii1}{0079}  %% посилання на сторінку оригінального видання
ту саму величину, але їх речові елементи зазнають зміни вартості,
коли, отже, $v$ означає змінену кількість приведеної у рух
праці, а $c$ — змінену кількість приведених у рух засобів виробництва.

$20v$ в капіталі $80c \dplus{} 20v \dplus{} 20m$ первісно представляли заробітну
плату 20 робітників, по 10 робочих годин на день. Нехай
заробітна плата кожного з них підвищиться з 1 до 1\sfrac{1}{4}. Тоді $20v$
оплачують не 20, а тільки 10 робітників. Але, якщо ці 20 за
200 робочих годин виробляли вартість у 40, то ці 16 за 10 годин
на день, тобто разом за 160 робочих годин, вироблять вартість
лише в 32. Після того як ми віднімемо $20v$ для заробітної
плати, з 32 залишиться тільки 12 для додаткової вартості; норма
додаткової вартості знизилася б з 100\% до 60\%. Але через
те що, згідно з припущенням, норма додаткової вартості мусить
лишитись незмінною, то робочий день мусив би бути здовжений
на \sfrac{1}{4}, з 10 до 12\sfrac{1}{2} годин; якщо 20 робітників при 10 годинах
на день, \deq{} 200 робочим годинам, виробляють вартість у 40, то
16 робітників при 12\sfrac{1}{4} годинах на день, \deq{} 200 годинам, вироблять
таку саму вартість, і капітал $80c \dplus{} 20v$ виробляв би, як і раніш,
додаткову вартість у 20.

Навпаки: якщо заробітна плата падає так, що $20v$; становлять
заробітну плату 30 робітників, то $m'$ може лишитися
незмінним тільки тоді, коли робочий день скорочується з 10
до 6\sfrac{2}{3} годин. $10 × 20 \deq{} 6\sfrac{2}{3} × 30 \deq{} 200$ робочим годинам.

Наскільки при таких протилежних припущеннях $c$ щодо виразу
його вартості в грошах може лишитись незмінним і все ж
представляти змінену відповідно до змінених відносин масу
засобів виробництва, — це в істотному вже з’ясовано вище.
В своєму чистому вигляді цей випадок можливий тільки як цілком
винятковий.

Щождо зміни вартості елементів $c$, яка збільшує або зменшує
їх масу, але лишає незмінною суму вартості $c$, то ця зміна, поки
вона не веде за собою зміни величини $v$, не зачіпає ні норми
зиску, ні норми додаткової вартості.

Таким чином ми вичерпали всі можливі випадки зміни $v$, $c$
і $К$ в нашому рівнянні. Ми бачили, що при незмінній нормі додаткової
вартості норма зиску може падати, лишатись незмінною
або підвищуватись, бо найменшої зміни відношення $v$ до $c$,
відповідно до $К$, досить для того, щоб змінити також і норму
зиску.

Далі, виявилось, що при зміні $v$ завжди настає межа, коли незмінність
$m'$ стає економічно неможливою. Через те що всяка
однобічна зміна $c$ теж мусить дійти до межі, коли $v$ не може
далі лишатись незмінним, то виявляється, що для всіх можливих
змін \sfrac{v}{К} існують межі, поза якими $m'$ теж мусить стати змінним.
При змінах $m'$, до дослідження яких ми тепер переходимо, ця
взаємодія різних змінних нашого рівняння виступить ще ясніше.

\paragraph*{II. \emph{m′} змінюється}
Загальну формулу норм зиску при різних нормах додаткової
вартості, однаково, чи  \sfrac{v}{K} лишається незмінним, чи теж змінюється,
ми одержимо, коли рівняння:\[p' \deq{} m' \frac{v}{К}\]
перетворимо в інше:
\[
p'\textsubscript{1} \deq{} m'\textsubscript{1} \frac{v\textsubscript{1}}{K\textsubscript{1}},
\]
де $р'\textsubscript{1}, m'\textsubscript{1}, v\textsubscript{1}$ і $К\textsubscript{1}$ означають змінені величини $р', m', v$ і $К$.
Тоді ми маємо: \[
p': p'\textsubscript{1} \deq{} m' \frac{v}{K}: m'\textsubscript{1} \frac{v\textsubscript{1}}{K\textsubscript{1}},
\]
і звідси:\[
p'\textsubscript{1} \deq{} \frac{m'\textsubscript{1}}{m'} × \frac{v\textsubscript{1}}{v} × \frac{K}{K\textsubscript{1}} × p'.
\]
