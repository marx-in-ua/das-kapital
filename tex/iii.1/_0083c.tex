В І за нововироблену вартість у 40 сплачується $20 v$, в II
за нововироблену вартість у 48 сплачується $28 v$, в III за нововироблену
вартість у 36 сплачується $16 v$. Як нововироблена
вартість, так і заробітна плата змінились; але зміна нововиробленої
вартості означає зміну кількості витраченої праці, отже,
або числа робітників, або тривалості праці, або інтенсивності
праці, або кількох з цих трьох факторів.

c) Зміни $m'$ і $v$ відбуваються в тому самому напрямі; тоді
одна підсилює вплив другої.

\begin{gather*}
90 c \dplus{} 10 v \dplus{} 10 m; m' \deq{} 100\%, p' \deq{} 10\% \\
80 c \dplus{} 20 v \dplus{} 30 m; m' \deq{} 150\%, p' \deq{} 30\% \\
92 c \dplus{} \phantom{0}8 v \dplus{} \phantom{0}6 m; m' \deq{} \phantom{1}75\%, p' \deq{} \phantom{1}6\%
\end{gather*}

\noindent{}І тут усі три нововироблені вартості різні, а саме 20, 50 і 14;
і ця ріжниця в величині витрачуваної в кожному випадку кількості
праці знову зводиться до ріжниці числа робітників, тривалості
праці, інтенсивності праці, або кількох, а то й усіх цих факторів.


% REMOVED \footnotetext{
% В першому німецькому виданні: 3. Примітка ред. нім. вид. ІМЕЛ.
% }

\subparagraph*{3) m′, v і К змінюються}
Цей випадок не дає нових точок зору і розв’язується за допомогою
загальної формули, даної в рубриці: II. $m'$ змінюється.

\pfbreak{}

Отже, вплив зміни величини норми додаткової вартості на
норму зиску дає такі випадки:

1. $р'$ збільшується або зменшується в тій самій пропорції, як
i $m'$, якщо $\frac{v}{K}$  лишається незмінним.
\begin{gather*}
80 c \dplus{} 20 v \dplus{} 20 m; m' \deq{} 100\%, p' \deq{} 20\% \\
80 c \dplus{} 20 v \dplus{} 10 m; m' \deq{} \phantom{1}50\%, p' \deq{} 10\% \\
100\% : 50\% \deq{} 20\% : 10\%.
\end{gather*}

2. $р'$ підвищується або падає в більшій пропорції, ніж $m'$,
якщо $\frac{v}{K}$ рухається в тому самому напрямі, що й $m'$, тобто
збільшується чи зменшується, коли збільшується чи зменшується
$m'$.

\begin{gather*}
80 c \dplus{} 20 v \dplus{} 10 m; m' \deq{} 50\phantom{\sfrac{2}{3}}\%, p' \deq{} 10\% \\
70 c \dplus{} 30 v \dplus{} 20 m; m' \deq{} 66\sfrac{2}{3}\%, p' \deq{} 20\% \\
50\% : 66\sfrac{2}{3}\% < 10\% : 20\%.\text{\footnotemarkZ{}}
\end{gather*}

\disablefootnotebreak{}
\footnotetextZ{
Знак $<$ означає тут, що збільшення з 50 до 66\sfrac{2}{3} є порівняно менше, ніж
збільшення з 10 до 20. Знак $>$ у дальшій формулі означає зворотне. \Red{Примітка
ред. нім. вид. ІМЕЛ.}}
\enablefootnotebreak{}
