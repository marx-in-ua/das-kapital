
В І за нововироблену вартість у 40 сплачується $20 v$, в II
за нововироблену вартість у 48 сплачується $28 v$, в III за нововироблену
вартість у 36 сплачується $16 v$. Як нововироблена
вартість, так і заробітна плата змінились; але зміна нововиробленої
вартості означає зміну кількості витраченої праці, отже,
або числа робітників, або тривалості праці, або інтенсивності
праці, або кількох з цих трьох факторів.

c) Зміни $m'$ і $v$ відбуваються в тому самому напрямі; тоді
одна підсилює вплив другої.

\begin{gather*}
90 c + 10 v + 10 m; m' = 100\%, p' = 10\% \\
80 c + 20 v + 30 m; m' = 150\%, p' = 30\% \\
92 c + \phantom{0}8 v + \phantom{0}6 m; m' = \phantom{1}75\%, p' = \phantom{1}6\%
\end{gather*}

\noindent{}І тут усі три нововироблені вартості різні, а саме 20, 50 і 14;
і ця ріжниця в величині витрачуваної в кожному випадку кількості
праці знову зводиться до ріжниці числа робітників, тривалості
праці, інтенсивності праці, або кількох, а то й усіх цих факторів.

\subsection[m', v і К змінюються]{m', v і К змінюються\footnotemarkZ{}}
\footnotetextZ{
В першому німецькому виданні: 3. Примітка ред. нім. вид. ІМЕЛ.
}

\noindent{}Цей випадок не дає нових точок зору і розв’язується за допомогою
загальної формули, даної в рубриці: II. $m'$ змінюється.

\pfbreak{}

Отже, вплив зміни величини норми додаткової вартості на
норму зиску дає такі випадки:

1. $р'$ збільшується або зменшується в тій самій пропорції, як
i $m'$, якщо $\frac{v}{K}$  лишається незмінним.

\begin{gather*}
80 c + 20 v + 20 m; m' = 100\%, p' = 20\% \\
80 c + 20 v + 10 m; m' = \phantom{1}50\%, p' = 10\% \\
100\% : 50\% = 20\% : 10\%.
\end{gather*}

2. $р'$ підвищується або падає в більшій пропорції, ніж $m'$,
якщо $\frac{v}{K}$ рухається в тому самому напрямі, що й $m'$, тобто
збільшується чи зменшується, коли збільшується чи зменшується
$m'$.

\begin{gather*}
80 c + 20 v + 10 m; m' = 50\phantom{\sfrac{2}{3}}\%, p' = 10\% \\
70 c + 30 v + 20 m; m' = 66\sfrac{2}{3}\%, p' = 20\% \\
50\% : 66\sfrac{2}{3}\% < 10\% : 20\%.\text{\footnotemarkZ{}}
\end{gather*}

\footnotetextZ{
Знак $<$ означає тут, що збільшення з 50 до 66\sfrac{2}{3} є порівняно менше, ніж
збільшення з 10 до 20. Знак $>$ у дальшій формулі означає зворотне. Примітка
ред. нім. вид. ІМЕЛ.
}

