
\index{iii1}{0142}  %% посилання на сторінку оригінального видання
За даними того самого звіту, з бавовняних робітників Ланкашіра
й Чешіра тоді працювали повний час \num{40146} робітників,
або 11,3\%, неповний робочий час — \num{134767} робітників, або 38\%,
зовсім без роботи було \num{179721} робітник, або 50,7\%. Коли виключити
звідси дані про Манчестер і Больтон, де випрядаються
головним чином тонкі нумери, — галузь, що порівняно мало потерпіла
від недостачі бавовни, — то справа виявиться ще несприятливішою, а
саме: таких, що працюють повний час — 8,5\%,
неповний час — 38\%, безробітних — 53,5\% (стор. 19, 20).

„Для робітників становить істотну ріжницю, чи переробляють
вони добру чи погану бавовну. В перші місяці року, коли фабриканти
намагались тримати свої фабрики в русі тим, що вживали
всяку бавовну, яку тільки можна було купити по помірних цінах,
багато поганої бавовни потрапило на ті фабрики, де раніше звичайно
застосовували добру; ріжниця в заробітній платі робітників
була така велика, що відбулося багато страйків, бо робітники
при старій відштучній платі тепер не могли вже добути
собі зносного щоденного заробітку\dots{} В деяких випадках ріжниця
в наслідок застосовування поганої бавовни становила навіть при
повному робочому часі половину всього заробітку“ (стор. 27).

1863 рік. Квітень. „На протязі цього року зможуть бути заняті
повний час трохи більше половини бавовняних робітників“
(„Rep. of Insp. of Fact., April 1863“, стор. 14).

„Дуже серйозна невигода при застосуванні ост-індської бавовни,
яку тепер фабрики мусять споживати, полягає в тому,
що швидкість машин при цьому мусить бути дуже уповільнена.
Протягом останніх років було вжито всіх заходів для збільшення
цієї швидкості, так щоб ті самі машини виконували більше
роботи. Але зменшена швидкість зачіпає робітника в такій самій
мірі, як фабриканта, бо більшість робітників одержують відштучну
плату — прядільники стільки то за фунт випряденої
пряжі, ткачі стільки то за витканий кусок; і навіть у інших
робітників, які одержують тижневу плату, заробітна плата повинна
знизитися в наслідок зменшення виробництва. На підставі
моїх досліджень\dots{} і переданих мені даних про заробіток бавовняних
робітників на протязі цього року\dots{} виявляється зменшення
заробітної плати пересічно на 20\%, в деяких випадках
на 50\% порівняно з висотою заробітної плати 1861 року“
(стор. 13). — „Зароблена сума залежить\dots{} від того, який матеріал
переробляється\dots{} Становище робітників, щодо суми заробленої
плати, тепер (жовтень 1863 року) багато краще, ніж минулого
року в цей час. Машини поліпшено, сировинний матеріал
знають краще, і робітники легше справляються з тими труднощами,
з якими їм доводилося боротись спочатку. Минулої весни
я був у Престоні в одній швацькій школі“ [благодійна установа
для безробітних]; „дві молоді дівчини, які за день перед тим
були послані до ткацької фабрики, де, за заявою фабриканта, вони
могли б заробити 4\shil{ шилінга} на тиждень, просили, щоб їх знову
\parbreak{}  %% абзац продовжується на наступній сторінці
