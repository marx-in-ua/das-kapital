\parcont{}  %% абзац починається на попередній сторінці
\index{iii1}{0029}  %% посилання на сторінку оригінального видання
вартий 1000 мільярдів, бо кожний з них хоче і може дати цю суму за товар другого. Але коли ми з'єднаємо сукупність товарів
обох у руках третього, то перший не матиме ніякої вартості, другий теж нічого не матиме, а третій і зовсім нічого — кінець-кінцем, ніхто
нічого не матиме. І ми знову захоплені вищістю, з якою наш південний Каліостро так обробив поняття вартості,
що від нього не залишилось і найменшого сліду. Це — завершення вульґарної економії!\footnote{Цей самий „відомий своєю славою" пан (кажучи словами Гейне) дещо пізніше побачив себе змушеним відповісти на мою передмову
до III тома — а саме після того, як ця передмова з’явилась італійською мовою у першому нумері „Rassegna“ за 1895 рік.
Відповідь надрукована в „Riforma Sociale“ від 25 лютого 1895 року. Щедро наділивши мене спочатку неминучими в нього і саме
тому подвійно гидкими похвалами, Лоріа заявляє, що він і на думці не мав присвоювати собі заслуги Маркса щодо
матеріалістичного пояснення історії. Він признав ці заслуги ще в 1885 році — а саме якось мимохідь у журнальній статті. Але
зате тим упертіше він замовчує це там, де про це треба було сказати, а саме у своїй книзі на відповідну тему, де про Маркса
згадується лише на стор. 129, та й те тільки з нагоди питання про дрібне землеволодіння у Франції. А тепер він відважно
заявляє, що Маркс зовсім не є творець цієї теорії; якщо її не намітив у загальних рисах уже Арістотель, то вже без сумніву
оголосив ще в 1656 році Гаррінгтон, і потім вона була розвинена цілою плеядою істориків, політиків, юристів та економістів
задовго до Маркса. Про все це можна прочитати у французькому виданні твору Лоріа. Коротко кажучи — викінчений плагіатор.
Після того, як я відняв у нього можливість далі вихвалятися за допомогою запозичень з Маркса, він безцеремонно заявляє, що і
Маркс прикрашував себе чужим пір’ям — цілком так само, як і він, Лоріа. — 3 інших моїх заперечень Лоріа відповідає тільки на
те, де я говорю про його думку, ніби Маркс ніколи і не думав писати II, а тим більше III том „Капіталу“. „І тепер Енгельс з
тріумфом відповідає мені, вказуючи на II і III томи\dots{} чудово! А я такий радий цим томам, яким я зобов’язаний такою великою
інтелектуальною насолодою, що ніколи ще мені не була так приємна перемога, як зараз приємна ця поразка — якщо це дійсно є
поразка. Але чи дійсно це є поразка? Чи це дійсно правда, що Маркс написав з метою опублікування цю масу не зв’язаних між
собою заміток, які Енгельс зібрав з такою побожною дружністю? Чи дійсно можна припустити, що Маркс\dots{} цими сторінками
сподівався вивершити свій твір і свою систему? Чи це дійсно правдоподібно, що Маркс опублікував би цей розділ про
пересічну норму зиску, в якому обіцяне багато років тому розв’язання проблеми зводиться до цілком безвідрадної
містифікації, до найвульґарнішої гри словами? Щонайменше в цьому дозволенно сумніватись\dots{} Це доводить, як мені
здається, що Маркс,
видавши свою блискучу (splendido) книгу, не мав на думці дати її продовження, абож хотів залишити справу
заверішення цього колосального твору спадкоємцям, не беручи відповідальності на себе особисто“.

Так і написано на стор. 267. Найбільше презирство, яке Гейне міг висловити до своїх німецьких філістерів-читачів, він вложив
у такі слова: автор кінець-кінцем так привикає до своєї публіки, наче має справу з розумною істотою. За кого ж має вважати
свою публіку illustre Лоріа?

На закінчення — нова порція похвал сиплеться на мене, безталанного. При цьому наш Сганарель уподобляє себе Валаамові, який
прийшов проклинати, але з уст якого проти його волі ринуть „слова благословення й любові“. Добрий Валаам відзначався тим, що
їздив на ослиці, яка була розумніша за свого господаря. На цей раз Валаам, очевидно, залишив свою ослицю вдома.}

У браунівському „Archiv für soziale Gesetzgebung“, VII, 4 випуск, Вернер Зомбарт дає чудовий в цілому виклад контурів
системи Маркса. Це вперше німецькому університетському професорові
\parbreak{}  %% абзац продовжується на наступній сторінці
