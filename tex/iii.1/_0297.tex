\parcont{}  %% абзац починається на попередній сторінці
\index{iii1}{0297}  %% посилання на сторінку оригінального видання
попитом, і тому в цих галузях у купців і промисловців справи
йдуть дуже жваво. Криза настає, як тільки затрати купців, які
продають на віддалених ринках (або в яких запаси нагромадились
і всередині країни), починають повертатись до них так
повільно і в таких незначних кількостях, що банки настійливо
вимагають платежів, або строки оплати векселів за куплені
товари настають раніше, ніж відбувається перепродаж цих
товарів. Тоді починаються примусові продажі, продажі для
оплати боргів. І разом з цим вибухає крах, який відразу кладе
кінець позірному процвітанню.

Але зовнішній характер і ірраціональність обороту купецького
капіталу є ще більші в наслідок того, що оборот одного й того ж
купецького капіталу може одночасно або послідовно опосереднювати
обороти дуже різних продуктивних капіталів.

Оборот купецького капіталу може, однак, опосереднювати
не тільки обороти різних промислових капіталів, але й протилежні
фази метаморфози товарного капіталу. Купець купує,
наприклад, полотно у фабриканта і продає його білільникові.
Отже, тут оборот того самого купецького капіталу — в дійсності
те саме $Т — Г$, реалізація полотна — представляє дві протилежні
фази для двох різних промислових капіталів. Оскільки
купець взагалі продає для продуктивного споживання, його
$Т — Г$ завжди представляє $Г — Т$ якогось промислового капіталу,
а його $Г — Т$ завжди представляє $Т — Г$ якогось іншого промислового
капіталу.

Якщо ми, як це зроблено в цьому розділі, оминемо $К$, витрати
циркуляції, ту частину капіталу, яку купець авансує крім
суми, витрачуваної на купівлю товарів, то, звичайно, відпадає
й $ΔК$, додатковий зиск, який він одержує на цей додатковий
капітал. Отже, цей спосіб дослідження є строго логічний і математично
правильний, коли справа йде про те, щоб дізнатись,
як зиск і оборот купецького капіталу впливають на ціни.

Якби ціна виробництва 1 фунта цукру становила 1\pound{ фунт
стерлінгів}, то купець міг би на 100\pound{ фунтів стерлінгів} купити
100 фунтів цукру. Якщо він протягом року купує і продає таку
кількість і якщо річна пересічна норма зиску є 15\%, то він
на 100\pound{ фунтів стерлінгів} накине 15\pound{ фунтів стерлінгів}, а на
1\pound{ фунт стерлінгів} — ціну виробництва 1 фунта цукру — 3\shil{ шилінги.}
Отже, він продавав би 1 фунт цукру за 1\pound{ фунт стерлінгів} 3\shil{ шилінги.}
Навпаки, якби ціна виробництва 1 фунта цукру впала до
1\shil{ шилінга}, то на 100\pound{ фунтів стерлінгів} купець купив би 2000 фунтів
цукру і продавав би його по 1\shil{ шилінгу} і 1\sfrac{4}{5}\pens{ пенса} за фунт.
І в тому і в другому випадку річний зиск на капітал в 100\pound{ фунтів
стерлінгів}, вкладений у цукрову справу, — 15\pound{ фунтам стерлінгів}.
Тільки в одному випадку він мусить продати 100, а в другому
2000 фунтів. Висока чи низька є ціна виробництва, це не має
ніякого значення для норми зиску; але це має дуже велике,
вирішальне значення для того, яка є величина тієї відповідної
\parbreak{}  %% абзац продовжується на наступній сторінці
