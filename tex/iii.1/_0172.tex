
Формула, згідно з якою ціна виробництва товару $= k \dplus{} p$,
дорівнює витратам виробництва плюс зиск, визначилась тепер
ближче таким чином, що $p \deq{} kp'$ (де $p'$ є загальна норма зиску),
і, отже, ціна виробництва $= k \dplus{} kp'$. Якщо $k \deq{} 300$, а $p' \deq{} 15\%$,
то ціна виробництва $k \dplus{} kp' \deq{} 300 \dplus{} 300$. $\frac{15}{100} \deq{} 345$.

Ціна виробництва товарів у кожній окремій сфері виробництва може змінювати свою величину:

1) при незмінній вартості товарів (тобто при умові, що у виробництво товару після зміни ціни
виробництва входить та сама
кількість мертвої і живої праці, як і до зміни) в наслідок незалежної від даної окремої сфери зміни
загальної норми зиску;

2) при незмінній загальній нормі зиску в наслідок зміни вартості — чи то в самій даній сфері
виробництва, в результаті
технічних змін, чи в наслідок зміни вартості тих товарів, які
входять у сталий капітал цієї сфери як його складові елементи;

3) нарешті, в наслідок спільного впливу обох цих обставин.
Не зважаючи на великі зміни, які постійно — як це виявиться
далі — відбуваються у фактичних нормах зиску окремих сфер
виробництва, дійсна зміна в загальній нормі зиску, оскільки
вона викликається не винятковими, надзвичайними економічними
подіями, є дуже пізній результат ряду коливань, які охоплюють
дуже довгі періоди часу, тобто коливань, що потребують багато часу, поки вони сконсолідуються і
вирівняються у зміну
загальної норми зиску. Тому при всіх коротших періодах (цілком незалежно від коливань ринкових цін)
зміну цін виробництва
треба завжди пояснювати prima facie [очевидно] дійсною зміною
вартості товарів, тобто зміною всієї суми робочого часу, потрібного для їх виробництва. Проста зміна
грошового виразу тих самих вартостей тут, само собою зрозуміло, зовсім не береться до уваги\footnote{
\emph{Corbett} [„An Inquiry into the Causes and Modes of the Wealth of Individuals“.
London 1841], стор. [33 і далі] 174.
}.

З другого боку, очевидно, що коли розглядати сукупний
суспільний капітал, то сума вартості вироблених ним товарів
(або, в грошовому виразі, їхня ціна), \deq{} вартості сталого капіталу \dplus{} вартість змінного капіталу \dplus{}
додаткова вартість. Якщо припустити, що ступінь експлуатації праці є незмінний, то норма зиску
при незмінній масі додаткової вартості може змінюватись тут
тільки в тому випадку, коли вартість сталого капіталу змінюється,
або коли вартість змінного капіталу змінюється, абож коли змінюються обидві ці вартості, так що
змінюється $K$, а тому й $\frac{m}{K}$, загальна норма зиску. Отже, в кожному випадку зміна загальної
норми зиску передбачає зміну вартості товарів, що входять як
складові елементи в сталий капітал, або в змінний капітал, або
одночасно в той і в другий.
