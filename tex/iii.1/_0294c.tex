\parcont{}  %% абзац починається на попередній сторінці
\index{iii1}{0294}  %% посилання на сторінку оригінального видання
але й у вигляді контори поряд з майстернею. Але для торговельного
капіталу ця сторона усамостійнюється. Для торговельного
капіталу контора становить його єдину майстерню. Частина
капіталу, застосовувана у формі витрат циркуляції, у гуртового
купця виявляється далеко більшою, ніж у промисловця, бо, крім
власної контори, яка є при кожній промисловій майстерні, частина
капіталу, яка мусила б застосовуватися таким чином всім
класом промислових капіталістів, концентрується в руках окремих
купців, що беруть на себе як продовження функцій циркуляції,
так і дальші витрати циркуляції, які випливають з цього.

Для промислового капіталу витрати циркуляції виступають
і дійсно є додатковими непродуктивними витратами. Для купця
вони виступають джерелом його зиску, який — припускаючи загальну
норму зиску — є у відповідності з їх величиною. Тому
видатки, які доводиться робити на ці витрати циркуляції, є для
торговельного капіталу продуктивним вкладенням. Отже, і торговельна
праця, яку він купує, для нього є безпосередньо
продуктивна.

\section{Оборот купецького капіталу. Ціни}

Оборот промислового капіталу є єдність часу його виробництва
й циркуляції; тому він охоплює весь процес виробництва.
Навпаки, оборот купецького капіталу, тому що в дійсності
він є тільки усамостійнений рух товарного капіталу, представляє
тільки першу фазу метаморфози товару, $Т — Г$, як рух
особливого капіталу, що повертається до свого вихідного пункту;
$Г — Т$, $Т — Г$ у купецькому розумінні представляє оборот купецького
капіталу. Купець купує, перетворює свої гроші в товар, потім
продає, перетворює той самий товар знов у гроші; і так далі,
постійно повторюючи це. В межах циркуляції метаморфоза промислового
капіталу завжди виступає як $Т_1 — Г — Т_2$; гроші, виручені
від продажу $Т_1$, виробленого товару, вживаються на
купівлю $Т_2$, нових засобів виробництва; це — дійсний обмін $Т_1$
і $Т_2$, і таким чином ті самі гроші двічі переходять з рук в руки,
їх рухом опосереднюється обмін двох різнорідних товарів, $Т_1$ i $Т_2$.
Але в купця, в $Г — Т — Г$, навпаки, двічі переходить з рук
в руки той самий товар; товар тільки опосереднює повернення
до нього грошей.

Якщо, наприклад, купецький капітал становить 100\pound{ фунтів
стерлінгів}, і купець купує на ці 100\pound{ фунтів стерлінгів} товар,
потім продає цей товар за 110\pound{ фунтів стерлінгів}, то цей його
капітал в 100 зробив один оборот, а число оборотів протягом
року залежить від того, як часто на протязі року повторюється
цей рух $Г — Т — Г'$.
