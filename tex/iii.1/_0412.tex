
\index{iii1}{0412}  %% посилання на сторінку оригінального видання
\section{Роль кредиту в капіталістичному виробництві}

Загальні зауваження, які ми досі зробили з приводу кредиту,
були такі:

I. Необхідність утворення кредиту для опосереднення вирівнення норми зиску або руху цього
вирівнення, на чому ґрунтується все капіталістичне виробництво.

II. Скорочення витрат циркуляції.

1) Головними витратами циркуляції є самі гроші, оскільки
вони самі мають вартість. Гроші заощаджуються за допомогою
кредиту трояким способом.

A. Тим, що для значної частини операцій вони зовсім відпадають.

B. Тим, що прискорюється циркуляція засобів циркуляції\footnote{
„Пересічна циркуляція банкнот Французького банку становила в 1812 році:
106538000 франків; в 1818 році: 101205000 франків, тимчасом як грошовий обіг,
загальна сума всіх надходжень і платежів, становив в 1812 році: 2837712000
франків; в 1818 році: 9665030000 франків. Отже, діяльність обігу у Франції
в 1818 році відносилась до діяльності обігу в 1812 році як 3: 1. Великим регулятором швидкості
циркуляції є кредит\dots{} Цим пояснюється, чому сильне тиснення на грошовий ринок звичайно збігається з
цілком заповненою циркуляцією“ („The Currency Theory Reviewed etc.“, стор. 65). — „Між вереснем 1833
року
і вереснем 1843 року в Великобританії виникло близько 300 банків, які випускали власні банкноти;
наслідком цього було скорочення в циркуляції банкнот
на 2\sfrac{1}{2} мільйони; наприкінці вересня 1833 року вона становила: 36035244 фунтів стерлінгів, а
наприкінці вересня 1843 року: 33518544 фунтів стерлінгів“
(там же, стор. 53). — „Дивовижна діяльність шотландської циркуляції дає їй
змогу за допомогою 100 фунтів стерлінгів виконати таку ж кількість грошових
операцій, для якої в Англії потрібно 420 фунтів стерлінгів“ (там же, стор. 55.
Це останнє стосується тільки до технічного боку операції).
}.
Почасти це збігається з тим, що доведеться сказати в пункті 2.
З одного боку, це прискорення — технічне; тобто при незмінних
величині й кількості дійсних товарних оборотів, які опосереднюють споживання, менша кількість грошей
або грошових знаків
виконує ту саму службу. Це зв’язано з технікою банкової справи.
З другого боку, кредит прискорює швидкість метаморфози товарів і разом з тим швидкість грошової
циркуляції.

C. Заміщенням золотих грошей паперовими.

2) Прискорення, за допомогою кредиту, окремих фаз циркуляції
або метаморфози товарів, потім метаморфози капіталу, і разом
з тим прискорення процесу репродукції взагалі. (З другого боку,
кредит дозволяє на більший строк відділяти один від одного акти
купівлі й продажу і служить через це базою спекуляції.) Скорочення резервних фондів, що можна
розглядати двояко: з одного
боку, як зменшення знаряддя циркуляції, з другого боку, як скорочення тієї частини капіталу, яка
завжди повинна існувати в грошовій формі.\footnote{„До заснування банків сума капіталу, потрібна для функціонування знаряддя циркуляції, завжди була
більша, ніж цього вимагала дійсна товарна
циркуляція“ („\emph{Economist}“, 1845, стор. 238).}
