
\index{iii1}{0345}  %% посилання на сторінку оригінального видання
Ми зараз докладніше розглянемо ту обставину, що процент
стоїть у співвідношенні з пересічною нормою зиску. Там, де
ціле даної величини, як, наприклад, зиск, доводиться ділити між
двома, справа, звичайно, залежить насамперед від величини того
цілого, яке належить розділити, а вона, величина зиску, визначається
пересічною нормою зиску. Якщо припустити загальну
норму зиску, отже, величину зиску на капітал даної величини,
скажімо \deq{} 100, за дану, то зміни процента стоятимуть, очевидно,
у зворотному відношенні до змін тієї частини зиску, яка лишається
функціонуючому капіталістові, що працює з позиченим
капіталом. А ті обставини, що визначають величину зиску, який
належить розподілити — нової вартості, виробленої неоплаченою
працею, — дуже відрізняються від обставин, які визначають його
розподіл між цими двома родами капіталістів, і часто діють у цілком
протилежних напрямах\footnote{
В рукопису тут є така помітка: „В ході цього розділу виясняється, що
раніше ніж дослідити закони розподілу зиску, все ж краще спочатку викласти,
яким чином кількісний поділ стає якісним. Щоб перейти до цього від попереднього
розділу, не потрібно нічого іншого, як припустити спочатку процент як
деяку частину зиску, не визначаючи її ближче“. [— \emph{Ф.~Е.}]
}.

Якщо ми розглянемо ті цикли оборотів, в яких рухається сучасна
промисловість, — стан спокою, зростаюче пожвавлення, розквіт,
перепродукція, крах, застій, стан спокою і~\abbr{т. д.}, цикли,
дальший аналіз яких виходить за межі нашого дослідження, —
то ми побачимо, що низький рівень процента здебільшого відповідає
періодам розквіту або надзиску, підвищення процента —
переходові від розквіту до наступного повороту, а максимум
процента, що досягає найвищих лихварських розмірів — кризі\footnote{
„В перший період, безпосередньо після періоду пригнічення, грошей
досить без спекуляції; у другий період грошей досить і спекуляція процвітає;
у третій період спекуляція починає слабшати і грошей шукають; у четвертий
період гроші становлять рідкість і настає пригнічення“ (\emph{Gilbart}: „А Practical
Treatise on Banking“, 5 вид., Лондон 1849, т. І, стор. 149).
}.
З літа 1843 року настав рішучий розквіт; розмір процента, весною
1842 року ще 4\sfrac{1}{2}\%, упав весною і літом 1843 року до 2\%\footnote{
Тук пояснює це „by the accumulation of surplus capital necessarily accompanying
the scarcity of profitable employement for it in previous years, by the release
of hoards, and by the revival of confidence in commercial prospects“ [нагромадженням
надлишкового капіталу, яке неминуче супроводить недостатні
можливості зисковного застосування його в попередні роки, звільненням грошових
запасів і відродженням довір’я до розквіту торгівлі“] („History of
Prices from 1839 to 1847“. London 1848, стор. 54).
},
а у вересні він упав навіть до 1\sfrac{1}{2}\% (\emph{Gilbart}: „А Practical Treatise
on Banking“, 5 вид., Лондон 1849, т. І, стор. 166); потім, під час
кризи 1847 року, він підвищився до 8\% і більше.

Звичайно, з другого боку, низький процент може збігтися з
застоєм, а помірно зростаючий процент з зростаючим пожвавленням.
Розмір процента досягає своєї найбільшої висоти під час криз,
коли доводиться, чого б це не коштувало, позичати для того,
\parbreak{}  %% абзац продовжується на наступній сторінці
