\parcont{}  %% абзац починається на попередній сторінці
\index{iii1}{0050}  %% посилання на сторінку оригінального видання
а з другого боку, він тільки тому становить складову частину
товарної вартості, що він є вартість витраченого капіталу, або
тому, що засоби виробництва коштують от стільки й стільки.

Зовсім навпаки з другою складовою частиною витрат виробництва.
Витрачені під час виробництва товару 666\sfrac{2}{3} днів
праці утворюють нову вартість в 200\pound{ фунтів стерлінгів}. З цієї
нової вартості одна частина тільки заміщає авансований змінний
капітал у 100\pound{ фунтів стерлінгів}, або ціну вжитої робочої сили.
Але ця авансована капітальна вартість ніяк не входить в утворення
нової вартості. При авансуванні капіталу робоча сила
фігурує як \emph{вартість}, але в процесі виробництва вона функціонує
як \emph{творець вартості}. На місце тієї вартості робочої сили,
яка фігурує при авансуванні капіталу, в дійсно \emph{функціонуючому}
продуктивному капіталі виступає сама жива, вартостетворча робоча
сила.

Ріжниця між цими різними складовими частинами товарної
вартості, які разом становлять витрати виробництва, впадає
в очі, як тільки настає зміна у величині вартості, в одному випадку
— витраченої сталої частини, в другому випадку — витраченої
змінної частини капіталу. Нехай ціна тих самих засобів виробництва,
або стала частина капіталу, підвищиться з 400\pound{ фунтів
стерлінгів} до 600\pound{ фунтів стерлінгів} чи, навпаки, впаде до
200\pound{ фунтів стерлінгів}. У першому випадку підвищуються не
тільки витрати виробництва товару з 500\pound{ фунтів стерлінгів} до
$600 c \dplus{} 100v \deq{} 700$\pound{ фунтам стерлінгів}, але й сама товарна вартість
підвищується з 600\pound{ фунтів стерлінгів} до $600 c \dplus{} 100 v \dplus{} 100 m \deq{} 800$\pound{ фунтам стерлінгів}.
У другому випадку падають не тільки витрати виробництва з 500\pound{ фунтів стерлінгів} до 200 c \dplus{} 100v \deq{}
300\pound{ фунтам стерлінгів}, але й сама товарна вартість падає з
600\pound{ фунтів стерлінгів} до $200c \dplus{} 100v \dplus{} 100m \deq{} 400$\pound{ фунтам стерлінгів}.
Через те що витрачений сталий капітал переносить свою
власну вартість на продукт, то, при інших незмінних умовах, вартість
продукту зростає або падає разом з абсолютною величиною
цієї капітальної вартості. Припустімо, навпаки, що, при
інших незмінних умовах, ціна тієї самої маси робочої сили зростає
з 100\pound{ фунтів стерлінгів} до 150\pound{ фунтів стерлінгів} або,
навпаки, падає до 50\pound{ фунтів стерлінгів}. Хоча витрати виробництва
в першому випадку підвищуються з 500\pound{ фунтів стерлінгів}
до $400 c \dplus{} 150 v \deq{} 550$\pound{ фунтам стерлінгів}, а в другому випадку
зменшуються з 500\pound{ фунтів стерлінгів} до 400c \dplus{} 50v \deq{} 450\pound{ фунтам
стерлінгів}, однак, в обох випадках товарна вартість лишається
незмінною \deq{} 600\pound{ фунтам стерлінгів}; в одному випадку
$= 400c \dplus{} 150v \dplus{} 50 m$, в другому випадку $= 400c \dplus{} 50v \dplus{} 150m$.
Авансований змінний капітал не додає до продукту своєї власної
вартості. Навпаки, на місце його вартості в продукт увійшла утворена
працею нова вартість. Тому зміна в абсолютній величині
вартості змінного капіталу, оскільки вона виражає тільки зміну
в ціні робочої сили, ні трохи не змінює абсолютної величини
\parbreak{}  %% абзац продовжується на наступній сторінці
