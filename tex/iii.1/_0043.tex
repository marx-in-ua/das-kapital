\parcont{}  %% абзац починається на попередній сторінці
\index{iii1}{0043}  %% посилання на сторінку оригінального видання
акції теж були ще відносно слабі порівняно з теперішнім часом. Безпосередньо продуктивних підприємств в акційній формі було
мало. „Око міністра“ було ще тоді неперебореннм забобоном — як і банки, здебільшого в \emph{бідніших} країнах, в Німеччині,
Австрії, Америці і т. д.

Отже, біржа тоді ще була таким місцем, де капіталісти віднімали один у одного нагромаджені ними
капітали і яке безпосередньо стосувалось робітників тільки як новий доказ загального деморалізуючого впливу капіталістичного
господарства і як підтвердження кальвіністської тези, що вже на цьому світі благодать божа — інакше кажучи, випадок —
розподіляє благословення і прокляття, багатство, тобто насолоди й владу, і бідність, тобто нестатки й рабство.

3. Тепер інакше. З часів кризи 1866 року нагромадження відбувалося з дедалі більшою швидкістю і при тому так, що в жодній промисловій
країні, і менш за все в Англії, поширення виробництва не встигало за нагромадженням, і нагромадження окремого капіталіста не
могло знайти повного застосування в збільшенні його власного підприємства; в англійській бавовняній промисловості [це мало
місце] уже в 1845 році, [далі] залізничні афери. Але разом з цим нагромадженням зростала й маса рантьє, людей, яким набридло
постійне ділове напруження, які, отже, хотіли тільки розважатися або займатись тільки необтяжливими справами як директори
або члени наглядальних рад компаній. І, потретє, щоб полегшити застосування маси грошового капіталу, що прийшов таким чином
у рухливий стан, були створені повсюди, де не було ще цього зроблено, нові законодавчі форми для товариств з обмеженою
відповідальністю і були більш-менш знижені зобов’язання акціонерів, які до того часу відповідали необмежене (акційні
товариства в Німеччині в 1890 р.: 40\% підписки).

4. В зв’язку з цим відбувається ступневе перетворення промисловості в акційні підприємства. Галузь за галуззю підпадає цій
долі. Спочатку залізна промисловість, де тепер потрібні гігантські споруди (ще раніше рудники, якщо вони ще не були пайовими
товариствами). Потім хемічна промисловість і так само машинобудівні заводи. На континенті текстильна промисловість, в Англії
ще тільки в окремих місцевостях Ланкашіру (прядільна фабрика в Ольдхемі, ткацька фабрика у Бернлей і т. д. Об’єднання
швацьких підприємств, але це тільки попередній ступінь, і при найближчій кризі знову відбувається перехід у руки хазяїнів),
броварні (кілька років тому американські броварні були передані англійському капіталові, далі Guinness, Bass,  Allsopp\footnote*{Назви фірм великих англійських броварень. \Red{Ред. укр. перекладу.}}).
Потім трести, які створюють гігантські підприємства з спільним управлінням (як United Alcali\footnote*{Англійський хемічний трест. \Red{Ред. укр. перекладу.}}). Звичайна одноосібна фірма
все більше й більше стає тільки попереднім ступенем,
\parbreak{}  %% абзац продовжується на наступній сторінці
