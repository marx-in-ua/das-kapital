\parcont{}  %% абзац починається на попередній сторінці
\index{iii1}{0285}  %% посилання на сторінку оригінального видання
зменшення зиску на 18, то без такого усамостійнення купецького
капіталу необхідний додатковий капітал становив би, може, 200,
і тоді вся авансована промисловим капіталістом сума була б 1100
замість 900, отже, при додатковій вартості в 180 норма зиску
була б тільки 16\sfrac{4}{11}\%.

Якщо промисловий капіталіст, який разом з тим є своїм
власним купцем, крім додаткового капіталу, на який він купує
новий товар, раніше ніж його продукт, що перебуває в циркуляції,
зворотно перетвориться в гроші, авансував ще, крім того,
капітал (витрати на контору і заробітна плата торговельним
робітникам) для реалізації вартості свого товарного капіталу,
отже, на процес циркуляції, то хоч ці витрати становлять
додатковий капітал, але вони не утворюють додаткової вартості.
Вони мусять бути заміщені з вартості товарів, бо частина
вартості цих товарів мусить знову перетворитися в ці витрати
циркуляції; але цим не утворюється ніякої добавної додаткової
вартості. Щодо сукупного капіталу суспільства це фактично
зводиться до того, що частина його потрібна для другорядних
операцій, які не входять у процес зростання вартості, і що ця
частина суспільного капіталу постійно мусить репродуковуватись
для цих цілей. В наслідок цього зменшується норма зиску
для окремих капіталістів і для всього класу промислових капіталістів,
— результат, який виходить при всякому долученні
додаткового капіталу, оскільки це потрібно для того, щоб привести
в рух ту саму масу змінного капіталу.

Оскільки ці, зв’язані з самою справою циркуляції, додаткові
витрати переймає на себе від промислового капіталіста торговельний
капіталіст, теж відбувається це зменшення норми зиску,
тільки в меншій мірі і іншим шляхом. Справа тепер стоїть так,
що купець авансує більше капіталу, ніж це було б потрібно,
коли б цих витрат не існувало, і що зиск на цей додатковий
капітал підвищує суму торговельного зиску, отже, купецький
капітал в більшому розмірі входить разом з промисловим капіталом
у вирівнення пересічної норми зиску, — тобто пересічний
зиск знижується. Якщо в нашому наведеному вище прикладі крім
100 купецького капіталу авансується ще 50 додаткового капіталу
на ті витрати, про які йде мова, то сукупна додаткова
вартість в 180 тепер розподіляється на продуктивний капітал
в 900 плюс купецький капітал в 150, разом \deq{} 1050. Отже, пересічна
норма зиску знижується до 17\sfrac{1}{7}\%. Промисловий капіталіст
продає купцеві товари за 900 \dplus{} 154\sfrac{2}{7} \deq{} 1054\sfrac{2}{7}, а купець
продає їх за 1130 (1080 \dplus{} 50 за ті витрати, які він мусить знову
замістити). Зрештою, слід визнати, що з розподілом на купецький
і промисловий капітал зв’язана централізація торговельних
витрат і через це скорочення їх.

\looseness=-1
Тепер постає питання: як стоїть справа з торговельними
найманими робітниками, що їх уживає торговельний капіталіст,
в даному випадку торговець товарами?
