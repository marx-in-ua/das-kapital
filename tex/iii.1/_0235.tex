
З вищесказаним зв’язане знецінення наявного капіталу (тобто його речових елементів), яке
відбувається з розвитком промисловості. Воно також є однією з постійно діючих причин, які затримують
падіння норми зиску, хоч воно при певних обставинах може зменшувати масу зиску через зменшення маси
капіталу, який дає зиск. Тут знову виявляється, що ті самі причини, які породжують тенденцію норми
зиску до падіння, уміряють також здійснення цієї тенденції.

\subsection{Відносне перенаселення}
Утворення відносного перенаселення невідривне від розвитку продуктивної сили праці і прискорюється
цим розвитком, який виражається в зменшенні норми зиску. Відносне перенаселення виявляється в певній
країні тим яскравіше, чим більше розвинений в ній капіталістичний спосіб виробництва. В свою чергу
воно є, з одного боку, причиною того, що в багатьох галузях виробництва продовжує існувати більш чи
менш неповне підпорядкування праці капіталові, і продовжує існувати довше, ніж це на перший погляд
відповідає загальному станові розвитку; це — наслідок дешевини і великої кількості наявних в
розпорядженні капіталістів або звільнених найманих робітників, а також більшого опору, що його деякі
галузі виробництва, відповідно до
їх природи, чинять перетворенню ручної праці в машинну. З другого боку, відкриваються нові галузі
виробництва, особливо предметів розкоші, галузі, які мають своєю базою саме те відносне населення,
яке часто звільняється в наслідок переважання сталого капіталу в інших галузях виробництва, і які з
свого боку знов таки базуються на переважанні елементів живої праці і тільки ступнево пророблюють
той самий шлях розвитку, що й інші галузі виробництва. В обох випадках змінний капітал становить
значну частину всього капіталу, а заробітна плата стоїть нижче пересічної, так що в цих галузях
виробництва як норма додаткової вартості, так і маса додаткової вартості незвичайно високі. А через
те що загальна норма зиску утворюється в наслідок вирівнення норм зиску окремих галузей виробництва,
то
тут знову таки та сама причина, яка породжує тенденцію норми зиску до падіння, викликає протидію цій
тенденції, яка більш або менш паралізує вплив цієї тенденції.

\subsection{Зовнішня торгівля}
Оскільки зовнішня торгівля здешевлює почасти елементи сталого капіталу, почасти необхідні засоби
існування, в які перетворюється змінний капітал, остільки вона діє на норму зиску в напрямі
підвищення, підвищуючи норму додаткової вартості і знижуючи вартість сталого капіталу. Вона взагалі
діє в цьому напрямі, даючи змогу розширювати розміри виробництва. Таким
чином вона прискорює, з одного боку, нагромадження, але, з другого боку, і зменшення змінного
капіталу порівняно з сталим, а, значить, і падіння норми зиску. Так само, хоч розширення зовнішньої
торгівлі в дитинстві капіталістичного способу виробництва було його базою, однак з його розвитком
воно, в наслідок внутрішньої необхідності цього способу виробництва, в наслідок його потреби в
дедалі ширшому ринку, стало власним його продуктом. Тут знову виявляється та сама двобічність впливу
(Рікардо цілком проглядів цей бік зовнішньої торгівлі).
