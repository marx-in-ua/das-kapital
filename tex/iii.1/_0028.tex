\parcont{}  %% абзац починається на попередній сторінці
\index{iii1}{0028}  %% посилання на сторінку оригінального видання
колинебудь наукове самогубство з більшою помпою і урочистістю?“  („Nuova Antologia“, 1 лютого
1895 р., стор. 478, 479).

Як бачите, наш Лоріа радий без краю. Хіба він не мав рації, розглядаючи
Маркса, як рівного собі звичайного шарлатана?  Чи не хочете полюбуватися,— Маркс глузує з своєї
публіки цілком так само, як Лоріа; він пробувається містифікаціями цілком так само, як
найнікчемніший італійський професор політичної економії. Але в той час, як наш шарлатан може
дозволити собі це, бо він володіє своїм ремеслом, незграбний житель півночі Маркс на кожному
кроці потрапляє в клопіт, говорить нісенітниці, абсурд, так шо, кінець кінцем, йому лишається
тільки урочисто позбавити себе життя.

Залишмо покищо надалі розгляд твердження, що товари ніколи
не продавалися по вартостях, визначуваних працею, і ніколи не можуть продаватися по цих
вартостях. Спинімось тут. тільки на запевненні пана Лоріа, що „вартість є не що інше, як
відношення, в якому один товар обмінюється на інший, і що тому само вже поняття сукупної вартості
товарів є абсурд, - безглуздя і т. д.“. Отже, відношення, в якому обмінюються два товари, їх
вартість, є щось чисто випадкове, таке, що іззовні злетіло до товарів, що сьогодні може бути
таким, а завтра інакшим. Чи обмінюється центнер пшениці на один грам золота чи на один кілограм —
це ні в найменшій мірі не залежить від умов, внутрішньо властивих цій пшениці або цьому золоту,
а залежить від цілком чужих їм обом обставин. Бо інакше ці умови мусили б мати значення і в
обміні, загалом і в цілому панувати над ним і мати самостійне існування навіть незалежно від
обміну, так що можна було б говорити і про сукупну вартість товарів. Але illustre Лоріа каже, що
це — безглуздя. В якому б відношенні не обмінювались два товари, це є їх вартість,— і годі.
Отже, вартість тотожна з ціною, і кожний товар має стільки вартостей, скільки він може мати цін.
Ціна ж визначається попитом і поданням, а хто і далі ставить питання, чекаючи відповіді, — той
дурень.

Але тут є маленька заковика. При нормальному стані попит і подання взаємно покриваються.
Отже, поділімо всі наявні у світі товари на дві половини: на групу попиту і рівну їй величиною
групу подання. Припустім, що кожна група репрезентує ціну в 1000 мільярдів марок, франків,
фунтів стерлінгів або чогось іншого. За Адамом Різе* це дасть в сумі ціну або вартість в 2000
мільярдів. Безглуздя, абсурд,—каже пан Лоріа. Обидві групи разом можуть репрезентувати ціну в
2000 мільярдів. Але з вартістю справа стоїть інакше. Коли ми кажемо:  ціна, то 1000 + 1000 =
2000. А коли ми кажемо: вартість, то 1000 + 1000 = 0. Принаймні, так стоїть справа в даному
випадку., де йдеться про сукупність товарів. Бо тут товар кожного з двох

* Адам Різе (1492—1559) — відомий німецький математик, популярний як вчитель лічби. Вираз «за Адамом Різе“ в Німеччині став прислів’ям. Ред. укр. перекладу.
\parbreak{}
