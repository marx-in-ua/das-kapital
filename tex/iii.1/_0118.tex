\parcont{}  %% абзац починається на попередній сторінці
\index{iii1}{0118}  %% посилання на сторінку оригінального видання
заробітної плати на норму зиску; підсумок виходить тоді сам
собою.

Але тут, як і в попередньому випадку, треба, загалом кажучи, зауважити таке: якщо відбуваються
зміни, чи то внаслідок економії на сталому капіталі, чи в наслідок коливань цін сировинного
матеріалу, то вони завжди впливають на норму
зиску, навіть і тоді, коли вони зовсім не зачіпають заробітної плати, отже, і норми і маси
додаткової вартості. У формулі $m' \frac{v}{K}$ вони змінюють величину $К$, а тим самим і значення цілого дробу.
Отже, і тут цілком байдуже — в відміну від того, що виявилось при розгляді додаткової вартості — в
яких сферах виробництва відбуваються ці зміни; чи виробляють зачеплені цими змінами галузі
промисловості засоби існування для робітників,
відповідно сталий капітал для виробництва таких засобів існування, чи ні. Розвинене тут в такій
самій мірі стосується й до
тих випадків, коли зміни відбуваються у виробництві предметів
розкоші, а під виробництвом предметів розкоші тут слід розуміти всяке виробництво, яке не є потрібне
для репродукції робочої сили.

Під сировинним матеріалом ми розумітимемо тут і допоміжні
матеріали, як от індиго, вугілля, газ і ін. Далі, оскільки в цій
рубриці розглядаються машини, їх власний сировинний матеріал
складається з заліза, дерева, шкіри та ін. Тому на їх власну
ціну впливають коливання цін сировинного матеріалу, який входить у їх конструкцію. Оскільки їх ціна
підвищується внаслідок коливання цін, — чи сировинного матеріалу, з якого вони складаються, чи то
допоміжних матеріалів, споживаних під час
їх функціонування, — знижується pro tanto [відповідно до цього]
норма зиску. У зворотному випадку — навпаки.

В дальшому дослідженні ми обмежимось коливаннями цін сировинного матеріалу, але не того матеріалу,
що входить у процес виробництва як сировинний матеріал тих машин, що функціонують як засоби праці,
або як допоміжний матеріал при застосуванні машин, а того, що входить як сировинний матеріал
безпосередньо в процес виробництва товару. Тут слід відзначити тільки таке: природне багатство на
залізо, вугілля, дерево
і~\abbr{т. д.}, на головні елементи будування й застосовування машин,
здається тут природною родючістю капіталу і є елементом
у визначенні норми зиску, незалежно від високого чи низького
рівня заробітної плати.

Через те що норма зиску є \frac{m}{K} або = \frac{m}{c + v}, то ясно, що все те, що спричинює
зміну у величині $c$, а тому й $К$, викликає
також зміну в нормі зиску, навіть і тоді, коли $m$ і $v$ і їх взаємне
відношення лишаються незмінними. Але сировинний матеріал
становить головну частину сталого капіталу. Навіть у ті галузі
\parbreak{}  %% абзац продовжується на наступній сторінці
