\parcont{}  %% абзац починається на попередній сторінці
\index{iii1}{0161}  %% посилання на сторінку оригінального видання
привласнена і реалізована тим самим капіталом. Отже, якщо неоднаковий склад з обігового та основного
капіталу не веде неодмінно до нерівності в часі обороту, яка в свою чергу зумовлює нерівність норм
зиску, то ясно, що оскільки ця остання
має місце, вона походить не з самого по собі неоднакового
складу капіталу щодо його обігової і основної частин, а скоріше
з того, що цей неоднаковий склад указує тут тільки на нерівність періодів обороту, яка впливає на
норму зиску.

Отже, різний склад сталого капіталу з обігового й основного
в різних галузях промисловості сам по собі не має ніякого значення для норми зиску, бо вирішує
справу відношення змінного капіталу до сталого, а вартість сталого капіталу, отже, і його
відносна величина, порівняно з змінним капіталом, зовсім не залежить від основного чи обігового
характеру його складових
частин. Однак, немає сумніву, — і це веде до хибних висновків, — що там, де основний капітал значно
розвинений, це є
тільки виразом того, що виробництво провадиться у великому
масштабі, і, отже, сталий капітал дуже переважає над змінним,
інакше кажучи, вживана жива робоча сила є незначна порівняно з масою засобів виробництва, які вона
приводить в рух.

Отже, ми показали, що в різних галузях промисловості, відповідно до різного органічного складу
капіталів і в певних
межах відповідно також до різного часу їх оборотів, панують
нерівні норми зиску, і що тому навіть при однаковій нормі додаткової вартості тільки для капіталів
однакового органічного
складу — припускаючи однаковий час оборотів — справедливий
закон (в загальній тенденції), що зиски відносяться між собою
як величини капіталів і, отже, рівновеликі капітали за рівні
періоди часу дають рівновеликі зиски. Викладене тут ґрунтується
на базі, яка взагалі була досі базою нашого дослідження, — що товари продаються по їх вартостях. З
другого боку, не підлягає ніякому сумніву, що в дійсності, коли залишити осторонь
неістотні, випадкові і що взаємно вирівнюються ріжниці, нерівність у пересічних нормах зиску в
різних галузях промисловості не існує і не могла б існувати без знищення всієї системи
капіталістичного виробництва. Отже, здається, що теорія вартості тут несполучна з дійсним рухом,
несполучна
з фактичними явищами виробництва, і що тому взагалі доводиться відмовитись від можливості зрозуміти
ці явища.

З першого відділу цієї книги випливає, що витрати виробництва однакові для продуктів різних сфер
виробництва, на виготовлення яких авансовано рівновеликі частини капіталу, які б
не були різні органічні склади цих капіталів. У витратах виробництва для капіталіста ріжниця між
змінним і сталим капіталом
відпадає. Товар, на виробництво якого він мусить витратити
100\pound{ фунтів стерлінгів}, коштує йому однакової суми, чи витрачає
він $90 c \dplus{} 10 v$ чи $10 c \dplus{} 90 v$. Він однаково коштує йому 100\pound{ фунтів
стерлінгів}, не більше і не менше. Витрати виробництва в різних
\parbreak{}  %% абзац продовжується на наступній сторінці
