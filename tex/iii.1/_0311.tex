\parcont{}  %% абзац починається на попередній сторінці
\index{iii1}{0311}  %% посилання на сторінку оригінального видання
або порушення процесу репродукції), звести до його економічного
мінімуму, бо коли управління резервним фондом купівельних
і платіжних засобів провадиться для всього класу капіталістів, то
потрібен не такий великий резервний фонд цих засобів, як у тому
випадку, коли цим фондом управляє кожний капіталіст окремо.
Торгівля грішми не купує благородних металів, а тільки опосереднює
їх розподіл, коли їх купила торгівля товарами. Торгівля грішми
полегшує вирівнення балансів, оскільки гроші функціонують як
засіб платежу, і за допомогою цього штучного механізму вирівнення
взаємних платежів зменшує потрібну для цього масу грошей;
але вона не визначає ні зв’язку, ні розмірів взаємних платежів.
Наприклад, векселі й чеки, які обмінюються один на один
у банках і Clearing houses [розрахункових палатах], представляють цілком
незалежні підприємства, є результати даних операцій,
і справа тут іде тільки про кращий технічний спосіб вирівнення
цих результатів. Оскільки гроші циркулюють як купівельний
засіб, розміри і число купівель та продажів ніяк не залежать
від торгівлі грішми. Вона може тільки скоротити технічні операції,
якими супроводяться ці купівлі й продажі, і тим самим
зменшити масу грошової готівки, потрібної для цих оборотів.

Отже, торгівля грішми в тій чистій формі, в якій ми тут її
розглядаємо, тобто відокремлено від кредитної справи, має діло
тільки з технікою одного моменту товарної циркуляції, а саме
грошової циркуляції і тих різних функцій грошей, що з неї випливають.

Це істотно відрізняє торгівлю грішми від торгівлі товарами,
яка опосереднює метаморфозу товару і обмін товарів або зумовлює
те, що самий цей процес товарного капіталу виступає
як процес капіталу, відокремленого від промислового капіталу.
Тому, якщо товарно-торговельний капітал виявляє власну форму
циркуляції, $Г — Т — Г$, де товар двічі міняє місце і в наслідок
цього гроші припливають назад, протилежно до $Т — Г — Т$,
де гроші двічі переходять з рук у руки і тим опосереднюють
обмін товарів, то для грошово-торговельного капіталу не можна
вказати ніякої такої особливої форми.

Оскільки грошовий капітал для цього технічного опосереднення
грошової циркуляції авансується особливим підрозділом
капіталістів, — капітал, який представляє в зменшеному масштабі
той додатковий капітал, що його інакше самі купці і промислові
капіталісти мусили б авансувати для цієї мети, — остільки й тут
в наявності загальна форма капіталу $Г — Г'$. В наслідок авансування
$Г$ для авансувача утворюється $Г \dplus{} ΔГ$. Але опосереднення
$Г — Г'$ стосується тут не до речових, а тільки до технічних
моментів метаморфози.

Очевидно, що маса грошового капіталу, якою оперують торговці
грішми, це — грошовий капітал купців і промисловців, який
перебуває в циркуляції, і що операції, виконувані цими, є тільки
операції тих, кого вони обслуговують.
