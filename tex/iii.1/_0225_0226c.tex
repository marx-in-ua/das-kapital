\parcont{}  %% абзац починається на попередній сторінці
\index{iii1}{0225}  %% посилання на сторінку оригінального видання
зиску, ніж дрібний капіталіст, який, видимо, одержує високий зиск. Далі, найповерховіше
спостереження конкуренції показує, що при певних обставинах, коли більший капіталіст хоче захопити
для себе місце на ринку, витиснути дрібніших капіталістів, — як, наприклад, за часів кризи, — він
використовує це практично, тобто навмисно знижує свою норму зиску, щоб витиснути з ринку дрібніших
капіталістів. Так само й купецький капітал — про який ми пізніше скажемо докладніше — показує явища,
завдяки яким зниження зиску здається наслідком розширення підприємства, а разом з тим і капіталу.
Власне науковий вираз замість помилкового розуміння ми дамо пізніше. Подібні поверхові погляди є
результатом порівнення норм зиску, одержуваних в окремих галузях підприємств залежно від того, чи
підпорядковані вони режимові вільної конкуренції чи монополії. Цілком банальне уявлення, яке
створюється в головах агентів конкуренції, ми знаходимо в нашого Рошера, а саме, що таке зниження
норми зиску є „розумніше й гуманніше“\footnote*{
„Die Grundlagen der Nationalökonomie“. 2 Aufl. Stuttgart und Augsburg 1857,
стор. 190. \Red{Примітка ред. нім. вид. ІМЕЛ.}
}. Зменшення норми зиску представлено тут як \emph{наслідок}
збільшення капіталу і зв’язаного з цим розрахунку капіталістів, що при меншій нормі зиску маса
зиску, яку вони кладуть собі в кишеню, буде більша. Все це (за винятком того, що є в А.~Сміта, про
що пізніше) основане на цілковитому нерозумінні того, що таке взагалі є загальна норма зиску, і на
тому грубому уявленні, що ціни дійсно визначаються шляхом надбавки більш-менш довільної частки зиску
до дійсної вартості товарів. Хоч які грубі ці уявлення, все ж вони з необхідністю виникають з того
перекрученого способу й вигляду, в якому імманентні закони капіталістичного виробництва виявляються
в сфері конкуренції.

\pfbreak{}

Закон, згідно з яким падіння норми зиску, викликуване розвитком продуктивної сили, супроводиться
збільшенням маси зиску, виражається і в тому, що падіння цін товарів, вироблюваних капіталом,
супроводиться відносним збільшенням мас зиску, які містяться в них і реалізуються через їх продаж.

\looseness=1
Через те що розвиток продуктивної сили і відповідний цьому вищий склад капіталу приводить в рух
дедалі більшу кількість засобів виробництва за допомогою дедалі меншої кількості праці, то кожна
пропорціональна частина всього продукту, кожна одиниця товару або кожна певна окрема кількість
товару, яка служить одиницею міри для сукупної маси вироблених товарів, вбирає менше живої праці і
містить у собі, крім того, менше упредметненої праці як щодо зношення застосованого основного
капіталу,
так і щодо спожитих сировинних і допоміжних матеріалів. Отже, кожна одиниця товару містить у собі
меншу суму праці як упредметненої
\index{iii1}{0226}  %% посилання на сторінку оригінального видання
в засобах виробництва, так і новододаної під час виробництва. Тому ціна одиниці товару
падає. Маса зиску, яка міститься в кожній одиниці товару, може, не зважаючи на це, збільшитись, якщо
норма абсолютної чи відносної додаткової вартості зростає. Кожний окремий товар містить у собі менше
новододаної праці, але неоплачена частина її зростає в порівнянні з оплаченою. Однак, це
відбувається тільки в певних межах. Разом з дуже значним абсолютним зменшенням новододаної до кожної
одиниці товару суми живої праці, яке відбувається в ході розвитку виробництва, зменшуватиметься
абсолютно і маса неоплаченої праці, яка міститься в ній, як би вона не зростала відносно, а саме в
порівнянні з оплаченою частиною. Маса зиску, яка припадає на кожну одиницю товару, дуже
зменшуватиметься з розвитком продуктивної сили праці, не зважаючи на зростання норми додаткової
вартості; і це зменшення цілком так само, як падіння норми зиску, тільки уповільнюється здешевленням
елементів сталого капіталу та іншими наведеними в першому відділі цієї книги обставинами, які
підвищують норму зиску при незмінній і навіть при падаючій нормі додаткової вартості.

\looseness=1
Те, що ціна окремих товарів, з суми яких складається сукупний продукт капіталу, падає, не означає
нічого іншого, як те, що дана кількість праці реалізується в більшій масі товарів, що, отже, кожна
одиниця товару містить у собі менше праці, ніж раніше. Це відбувається навіть у тому випадку, коли
ціна якоїсь частини сталого капіталу, сировинного матеріалу та ін. зростає. За винятком окремих
випадків (наприклад, коли продуктивна сила праці рівномірно здешевлює всі елементи як сталого, так і
змінного капіталу), норма зиску знижуватиметься, не зважаючи на підвищену норму додаткової вартості,
1)~тому що навіть більша неоплачена частина зменшеної загальної суми новододаної праці є менша, ніж
була менша відповідна неоплачена частина більшої загальної суми, і 2)~тому що вищий склад капіталу в
окремому товарі виражається в тому, що та частина його вартості, яка взагалі представляє новододану
працю, зменшується порівняно з тією частиною вартості, яка представляє сировинний матеріал,
допоміжний матеріал і зношування основного капіталу. Ця переміна у відношенні різних складових
частин ціни окремого товару, зменшення тієї частини ціни, яка представляє новододану живу працю, і
збільшення тих частин ціни, які представляють раніше упредметнену працю, є та форма, в якій у ціні
окремого товару виражається зменшення змінного капіталу порівняно з сталим. Наскільки таке зменшення
є абсолютним для капіталу даної величини, наприклад, для 100, настільки ж воно є абсолютним для
кожного окремого товару як відповідної частини репродукованого капіталу. Однак, норма зиску, якщо
тільки обчисляти її на елементи ціни окремих товарів, виступила б іншою, ніж вона є в дійсності. І
саме з такої причини:
