\parcont{}  %% абзац починається на попередній сторінці
\index{iii1}{0053}  %% посилання на сторінку оригінального видання
як ці різні складові частини авансованого капіталу щодо обороту
набувають форм основного і обігового капіталу.

Отже, авансований капітал \deq{} 1680\pound{ фунтам стерлінгів}: основний
капітал \deq{} 1200\pound{ фунтам стерлінгів} плюс обіговий капітал \deq{}
480\pound{ фунтам стерлінгів} (= 380\pound{ фунтам стерлінгів} у матеріалах
виробництва плюс 100\pound{ фунтів стерлінгів} у заробітній платі).

Витрати виробництва товару, навпаки, \deq{} тільки 500\pound{ фунтам
стерлінгів} (20\pound{ фунтів стерлінгів} на зношування основного капіталу,
480\pound{ фунтів стерлінгів} на обіговий капітал).

Ця ріжниця між витратами виробництва товару і авансованим
капіталом підтверджує, однак, тільки те, що витрати
виробництва товару утворюються виключно капіталом, дійсно
витраченим на виробництво товару.

Для виробництва товару застосовуються засоби праці вартістю
в 1200\pound{ фунтів стерлінгів}, але з цієї авансованої капітальної
вартості у виробництві втрачаються тільки 20\pound{ фунтів
стерлінгів}. Тому застосований основний капітал лиш почасти
входить у витрати виробництва товару, бо він лиш почасти
витрачається на виробництво товару. Застосований обіговий
капітал цілком входить у витрати виробництва товару, бо він
цілком витрачається на виробництво товару. Але що ж це доводить,
як не те, що спожиті основні і обігові частини капіталу,
pro rata [пропорціонально] величині їх вартості, рівномірно
входять у витрати виробництва даного товару і що ця складова
частина вартості товару взагалі виникає тільки з капіталу,
витраченого на його виробництво? Коли б це було не так, то
не можна було б зрозуміти, чому авансований основний капітал
у 1200\pound{ фунтів стерлінгів} не додає до вартості продукту крім
20\pound{ фунтів стерлінгів}, які він утрачає в процесі виробництва,
також і тих 1180\pound{ фунтів стерлінгів}, які він не втрачає в цьому
процесі.

Отже, ця ріжниця між основним і обіговим капіталом щодо
обчислення витрат виробництва тільки підтверджує позірне
виникнення витрат виробництва з витраченої капітальної вартості
або з тієї ціни, якої коштують самому капіталістові витрачені
елементи виробництва, включаючи й працю. З другого
боку, щодо утворення вартості, змінна, витрачена на робочу
силу частина капіталу прямо ототожнюється тут із сталим капіталом
(частиною капіталу, яка існує в матеріалах виробництва)
під рубрикою обігового капіталу, і таким чином вивершується
містифікація процесу зростання вартості капіталу\footnote{
Яка плутанина може постати через це в голові економіста, показано
в книзі І, розд. VII, 3, на прикладі Н.~В.~Сеніора.
}.

Досі ми розглядали тільки один елемент товарної вартостівитрати
виробництва. Тепер ми мусимо поглянути і на другу
складову частину товарної вартості, на лишок понад витрати
виробництва, або на додаткову вартість. Отже, насамперед додаткова
\index{iii1}{0054}  %% посилання на сторінку оригінального видання
вартість є надлишок вартості товару понад витрати виробництва
його. Але через те що витрати виробництва дорівнюють
вартості витраченого капіталу, в речові елементи якого вони
раз-у-раз зворотно і перетворюються, то цей надлишок вартості
є приріст вартості того капіталу, який витрачено на виробництво
товару і який повертається назад з циркуляції товару.

Раніше ми вже бачили, що хоч $m$, додаткова вартість, виникає
тільки з зміни вартості $v$, змінного капіталу, і тому первісно
є просто приріст змінного капіталу, проте після закінчення
процесу виробництва вона в такій самій мірі становить
приріст вартості $c \dplus{} v$, усього витраченого капіталу. Формула
$c \dplus{} (v \dplus{} m)$, яка вказує, що m виробляється в наслідок перетворення
певної капітальної вартості $v$, авансованої на робочу силу,
в текучу величину, отже, в наслідок перетворення сталої величини
в змінну, — може бути представлена також як $(c \dplus{} v) \dplus{} m$.
Перед виробництвом ми мали капітал у 500\pound{ фунтів стерлінгів}.
Після виробництва ми маємо капітал у 500\pound{ фунтів стерлінгів}
плюс приріст вартості в 100\pound{ фунтів стерлінгів}\footnote{
„В дійсності ми вже знаємо, що додаткова вартість є просто наслідок
тієї зміни вартості, яка відбувається з $v$, з частиною капіталу, перетвореною
в робочу силу, що, отже, $v \dplus{} m \deq{} v \dplus{} Δv$ ($v$ плюс приріст $v$). Але дійсна зміна
вартості і відношення, в якому змінюється вартість, затемнюються тією обставиною,
що в наслідок зростання своєї* варіюючої складової частини зростає
також і весь авансований капітал. Він був 500, а стає 590“ (книга І, розд.
VII, 1, стор. 222 [стор. 148 рос. вид. 1935~\abbr{р.}]).
}.

Проте, додаткова вартість становить приріст не тільки до
тієї частини авансованого капіталу, яка входить у процес зростання
вартості, але й до тієї його частини, яка не входить у цей
процес; отже, вона становить приріст вартості не тільки до того
витраченого капіталу, який заміщається з виручених витрат виробництва
товару, але й до капіталу, взагалі застосованого у виробництві.
Перед процесом виробництва ми мали капітальну
вартість у 1680\pound{ фунтів стерлінгів}: 1200\pound{ фунтів стерлінгів} основного
капіталу, витраченого на засоби праці, з якого тільки
20\pound{ фунтів стерлінгів} входять як зношування у вартість товару,
плюс 480\pound{ фунтів стерлінгів} обігового капіталу в матеріалах виробництва
та заробітній платі. Після процесу виробництва ми
маємо 1180\pound{ фунтів стерлінгів} як складову частину вартості
продуктивного капіталу плюс товарний капітал у 600\pound{ фунтів
стерлінгів}. Якщо ми складемо ці дві суми вартості, то побачимо,
що капіталіст володіє тепер вартістю в 1780\pound{ фунтів стерлінгів}.
Якщо він відніме від цього весь авансований капітал в 1680\pound{ фунтів стерлінгів}, то в нього лишається приріст вартості в 100\pound{ фунтів стерлінгів}. Отже, 100\pound{ фунтів стерлінгів} додаткової вартості
в такій самій мірі становлять приріст вартості до застосованого
капіталу в 1680\pound{ фунтів стерлінгів}, як і до тієї частини
його в 500\pound{ фунтів стерлінгів}, яку витрачено під час виробництва.
