\parcont{}  %% абзац починається на попередній сторінці
\index{iii1}{0249}  %% посилання на сторінку оригінального видання
змінного капіталу (в наслідок підвищення заробітної плати) і відповідним
йому зменшенням відношення додаткової праці до
необхідної праці.

В дійсності справа мала б такий вигляд, що одна частина капіталу
лежала б без діла цілком або почасти (бо вона для того, щоб
взагалі зростати у своїй вартості, мусила б спочатку витиснути з
його позиції вже функціонуючий капітал), а друга частина в наслідок
тиснення незанятого або напівзанятого капіталу зростала б
у своїй вартості при нижчій нормі зиску. При цьому було б байдуже,
що частина додаткового капіталу стала б на місце старого капіталу
і цей останній таким чином зайняв би місце в додатковому
капіталі. Ми однаково мали б, з одного боку, стару капітальну
суму, а з другого — додаткову. Падіння норми зиску в цьому
випадку супроводилося б абсолютним зменшенням маси зиску, бо
при наших припущеннях маса вживаної робочої сили не могла б
бути збільшена і норма додаткової вартості не могла б бути підвищена,
отже й маса додаткової вартості не могла б бути збільшена.
І зменшену масу зиску довелося б обчислювати на збільшений
сукупний капітал. — Але якщо навіть припустити, що занятий
капітал продовжує зростати в своїй вартості при старій нормі
зиску, отже, що маса зиску лишається та сама, то вона все ж
обчислювалася б на зрослий сукупний капітал, а це знов таки
означає падіння норми зиску. Якщо сукупний капітал в 1000 давав
зиск у 100, а після свого збільшення до 1500 так само дає
тільки 100, то в другому випадку 1000 дає вже тільки 66\sfrac{2}{3}. Зростання
вартості старого капіталу зменшилося б абсолютно. При
нових обставинах капітал \deq{} 1000 давав би не більше, ніж раніше
давав капітал \deq{} 666\sfrac{2}{3}.

Але ясно, що це фактичне знецінення старого капіталу
відбулося б не без боротьби, що додатковий капітал, $ΔК$, не без
боротьби дістав би змогу функціонувати як капітал. Норма зиску
знизилася б не в наслідок конкуренції як результату перепродукції
капіталу. А навпаки, конкурентна боротьба почалася б тепер
тому, що знижена норма зиску і перепродукція капіталу виникають
з одних і тих самих причин. Частину $ΔК$, яка опинилася б
у руках старих функціонуючих капіталістів, ці останні в більшій
чи меншій мірі лишили б лежати без діла, щоб самим не знецінити
свого первісного капіталу і не звузити його місця у
сфері виробництва; або вони застосували б її для того, щоб навіть
з тимчасовою втратою для себе перенести бездіяльність
додаткового капіталу з себе на нових прихідців і взагалі на
своїх конкурентів.

Частина $ΔК$, яка опинилася б у нових руках, намагалася б
захопити собі місце за рахунок старого капіталу, і вона почасти
досягла б цього, лишивши частину старого капіталу лежати
без діла, примусивши його звільнити для неї старе місце і самому
зайняти місце додаткового капіталу, застосовуваного тільки
почасти або навіть і зовсім не застосовуваного.
