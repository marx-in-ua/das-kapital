\parcont{}  %% абзац починається на попередній сторінці
\index{iii1}{0159}  %% посилання на сторінку оригінального видання
змінного капіталу, вироблена додаткова вартість. При однаковій
нормі додаткової вартості маса її залежить, очевидно, від маси
змінного капіталу. Вартість продукту капіталу в 100 в одному
випадку є $90 c \dplus{} 10 v \dplus{} 10 m \deq{} 110$, а в другому випадку $10 c \dplus{} 90 v \dplus{} 90 m \deq{} 190$. Якщо товари
продаються по їх вартостях,
то перший продукт продається за 110, з яких 10 представляють додаткову вартість, або неоплачену
працю; другий же
продукт продається за 190, з яких 90 додаткової вартості, або
неоплаченої праці.

Це має особливе значення, коли порівнюються між собою
інтернаціональні норми зиску. Нехай в якійсь європейській країні
норма додаткової вартості буде 100\%, тобто робітник працює
півдня на себе і півдня на свого підприємця; нехай в якійсь
азіатській країні норма додаткової вартості буде 25\%, тобто робітник працює \sfrac{4}{5} дня на себе і \sfrac{1}{5} на
свого підприємця. Але
в європейській країні склад національного капіталу є, скажімо,
$84 c \dplus{} 16 v$, а в азіатській, де застосовується мало машин і~\abbr{т. д.}
і де за даний час даною кількістю робочої сили продуктивно
споживається відносно мало сировинного матеріалу, склад є
$16 c \dplus{} 84 v$. Ми матимемо тоді такий обрахунок:

В європейській країні вартість продукту \deq{} $84 c \dplus{} 16 v \dplus{} 16 m \deq{} 116$; норма зиску \deq{} \frac{16}{100} \deq{} 16\%.

В азіатській країні вартість продукту \deq{} $16 c \dplus{} 84 v \dplus{} 21 m \deq{} 121$; норма зиску \deq{} \frac{21}{100} \deq{} 21\%.

Отже, норма зиску в азіатській країні більше як на 25\% вища, ніж в європейській, хоч норма
додаткової вартості в ній
вчетверо менша, ніж в європейській. Всякі Кері, Бастіа і tutti quanti [всі інші, скільки їх є]
зроблять якраз протилежний висновок.

Це між іншим; різні національні норми зиску здебільшого
основані на різних національних нормах додаткової вартості;
але ми порівнюємо в цьому розділі неоднакові норми зиску,
які випливають з однієї і тієї самої норми додаткової вартості.

Крім різного органічного складу капіталів, отже, крім різних
мас праці, а тому — при інших однакових умовах — і додаткової
праці, що її приводять в рух рівновеликі капітали в різних сферах
виробництва, існує ще й інше джерело нерівності норм зиску:
ріжниця в довжині періоду обороту капіталу в різних сферах
виробництва. В розділі IV ми бачили, що при однаковому складі
капіталів і при інших однакових умовах норми зиску стоять у зворотному відношенні до часу обороту;
ми бачили також, що
той самий змінний капітал, якщо він обертається в періоди
часу різної довжини, створює неоднакові маси річної додаткової вартості. Отже, ріжниця в часі
оборотів є друга причина
того, чому рівновеликі капітали в різних сферах виробництва
\parbreak{}  %% абзац продовжується на наступній сторінці
