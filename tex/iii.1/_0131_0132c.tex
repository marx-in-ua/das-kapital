\parcont{}  %% абзац починається на попередній сторінці
\index{iii1}{0131}  %% посилання на сторінку оригінального видання
його репродукцію, і таким чином відновлюється монополія тих
країн — джерел його постачання, які виробляють при найсприятливіших
умовах, — відновлюється, може, з певними обмеженнями,
але все ж відновлюється. Правда, репродукція сировинних
матеріалів в наслідок даного поштовху відбувається в розширеному
масштабі, особливо в країнах, які в більшій чи меншій мірі
володіють монополією цього виробництва. Але та база, на якій в наслідок
збільшення кількості машин і~\abbr{т. д.} відбувається виробництво
і яка тепер після кількох коливань має стати новою нормальною
базою, новим вихідним пунктом, дуже розширилася в наслідок
процесів, що відбувались протягом останнього циклу обороту.
При цьому, однак, в частині другорядних джерел постачання
сировинного матеріалу репродукція, яка щойно збільшилась, знову
значно гальмується. Так, наприклад, з таблиць експорту ясно
видно, як протягом останніх 30 років (до 1865 року) зростало
індійське виробництво бавовни, коли наставала недостача в американському
виробництві, і потім раптом знову починалося
більш-менш тривале скорочення. Протягом часу подорожчання
сировинного матеріалу промислові капіталісти об’єднуються,
утворюють асоціації, щоб регулювати виробництво. Так було,
наприклад, в Манчестері після підвищення цін на бавовну в
1848 році, так само як і в виробництві льону в Ірландії. Але як
тільки безпосередній привід мине і знову суверенно запанує загальний
принцип конкуренції „купувати на найдешевшому ринку“
(замість того, щоб намагатися, як це робили згадані асоціації,
підвищити виробничу здатність відповідних країн — джерел постачання
сировинного матеріалу, незалежно від безпосередньої
ціни даного моменту, по якій ці країни можуть у даний час постачати
продукт), — отже, як тільки знову суверенно запанує
принцип конкуренції, регулювати подання знову полишається
„ціні“. Всяка думка про спільний, рішучий і передбачливий контроль
над виробництвом сировинного матеріалу — контроль,
який загалом і в цілому ніяк несполучний з законами капіталістичного
виробництва і тому завжди лишається благочестивим
побажанням або обмежується винятковими спільними кроками
в моменти великої безпосередньої небезпеки й безпорадності —
поступається місцем вірі в те, що попит і подання взаємно
регулюватимуть одно одне\footnote{
Після того, як це було написано (1865~\abbr{р.}), конкуренція на світовому ринку
значно посилилася в наслідок швидкого розвитку промисловості в усіх культурних
країнах, 'особливо в Америці і Німеччині. Той факт, що швидко й колосально
зростаючі сучасні продуктивні сили з кожним днем все більше переростають
закони капіталістичного товарообміну, в межах яких вони повинні
рухатись, — цей факт нині все більше й більше проникає навіть у свідомість
самих капіталістів. Це виявляється особливо в двох симптомах. Поперше, в новій
загальній манії охоронних мит, яка відрізняється від старої системи охоронних
мит особливо тим, що вона найбільше захищає якраз товари, придатні до
експорту. Подруге, в картелях (trusts) фабрикантів цілих великих сфер виробництва
для регулювання виробництва і разом з тим цін і зисків. Само собою
зрозуміло, що ці експерименти здійснимі тільки при відносно сприятливій
економічній погоді. Перша ж буря повинна їх зруйнувати і довести, що, хоч
виробництво і потребує регулювання, але, без сумніву, не капіталістичний клас
покликаний здійснити його. Покищо ці картелі мають лиш одну мету —
дбати про те, щоб дрібні капіталісти пожиралися великими ще швидше, ніж
досі. — Ф.~Е.
}. Суєвірство капіталістів тут таке
грубе, що навіть фабричні інспектори в своїх звітах знов і знов
з приводу цього здивовано розводять руками. Чергування сприятливих
\index{iii1}{0132}  %% посилання на сторінку оригінального видання
і несприятливих років, звичайно, знов таки приводить до
здешевлення сировинного матеріалу. Незалежно від того безпосереднього
впливу, який ця обставина справляє на розширення
попиту, сюди долучається ще як стимул вищезгаданий
вплив на норму зиску. І зазначений вище процес ступневого
випереджання виробництва сировинних матеріалів виробництвом
машин і~\abbr{т. д.} повторюється тоді в ширшому масштабі. Дійсне
поліпшення сировинного матеріалу, так щоб він постачався не
тільки в потрібній кількості, але й потрібної якості, наприклад,
бавовна американської якості з Індії, вимагало б тривалого, регулярно
зростаючого і постійного попиту з боку Европи (цілком
залишаючи осторонь ті економічні умови, в які поставлений індійський
виробник на своїй батьківщині). Але при таких умовах
сфера виробництва сировинних матеріалів змінюється тільки
стрибками, то раптом розширюється, то знову дуже скорочується.
Все це, як і дух капіталістичного виробництва взагалі, можна
дуже добре вивчати на бавовняному голоді 1861--1865 років, до
якого долучалася ще й та обставина, що часами зовсім не було
сировинного матеріалу, одного з найістотніших елементів репродукції.
Ціна може підвищуватись навіть і тоді, коли подання
цілком достатнє, але достатнє при тяжчих умовах. Або може
мати місце справжня недостача сировинного матеріалу. Під час
бавовняної кризи спочатку мала місце така недостача сировинного
матеріалу.

Отже, чим більше ми наближаємось в історії виробництва
до безпосередньої сучасності, тим регулярніше ми знаходимо,
особливо у вирішальних галузях виробництва, постійне повторення
чергувань відносного подорожчання і виникаючого з нього
пізнішого знецінення сировинних матеріалів органічного походження.
Ілюстрації до вищесказаного дано в наведених нижче
прикладах, взятих із звітів фабричних інспекторів.

Мораль історії, яку можна здобути також з дослідження землеробства
взагалі, полягає в тому, що капіталістична система
протидіє раціональному землеробству, або що раціональне землеробство
несполучне з капіталістичною системою (хоч ця остання
і сприяє його технічному розвиткові) і потребує або руки самостійно
працюючого дрібного селянина, або контролю асоційованих
виробників.

\pfbreak

Тепер ми наводимо щойно згадані ілюстрації з англійських
фабричних звітів.

