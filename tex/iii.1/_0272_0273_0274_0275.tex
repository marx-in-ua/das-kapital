\parcont{}  %% абзац починається на попередній сторінці
\index{iii1}{0272}  %% посилання на сторінку оригінального видання
хліба, яку він послідовно може купити й продати протягом даного
часу, наприклад, протягом року, тимчасом як оборот капіталу
фермера, незалежно від часу обігу, обмежений часом
виробництва, що триває рік.

Але оборот одного й того ж купецького капіталу може з таким
самим успіхом опосереднювати обороти капіталів у різних
галузях виробництва.

Оскільки один і той самий купецький капітал в різних оборотах
служить для того, щоб послідовно перетворювати різні
товарні капітали в гроші, отже, по черзі купує іі продає їх, він
як грошовий капітал виконує відносно товарного капіталу ту
саму функцію, яку взагалі гроші числом своїх обігів за даний
період виконують відносно товарів.

Оборот купецького капіталу не є тотожний з оборотом або
з одноразовою репродукцією рівновеликого промислового капіталу;
навпаки, він дорівнює сумі оборотів певного числа таких
капіталів, чи в тій самій, чи в різних сферах виробництва. Чим
швидше обертається купецький капітал, тим менша є та частина
всього грошового капіталу, яка фігурує як купецький капітал;
чим повільніше він обертається, тим більша ця частина. Чим
менше розвинене виробництво, тим більша є сума купецького
капіталу порівняно з сумою товарів, що їх взагалі кидають у циркуляцію;
але тим менша вона абсолютно або порівняно з більш
розвиненим станом виробництва. І навпаки. Тому при такому
нерозвиненому стані виробництва більша частина власне грошового
капіталу перебуває в руках купців, майно яких у відміну
від майна інших становить таким чином грошове майно.

Швидкість циркуляції авансовуваного купцем грошового
капіталу залежить: 1) від швидкості, з якою відновлюється
процес виробництва і переплітаються між собою різні процеси
виробництва; 2) від швидкості споживання.

Для того, щоб купецький капітал проробив тільки розглянутий
вище оборот, немає потреби спочатку купити товарів на
всю величину його вартості, а потім продавати їх. Купець одночасно
проробляє обидва ці рухи. Його капітал поділяється тоді
на дві частини. Одна складається з товарного капіталу, друга —
з грошового капіталу. Він купує в одному місці і перетворює
цим свої гроші у товар. Він продає в другому місці і перетворює
цим другу частину товарного капіталу в гроші. З одного
боку, до нього повертається його капітал як грошовий капітал,
тимчасом як, з другого боку, до нього припливає товарний капітал.
Чим більша частина, яка існує в одній формі, тим менша
частина, яка існує в другій формі. Ці частини міняються одна
з другою і урівноважують одна другу. Коли з уживанням грошей
як засобу циркуляції сполучається вживання їх як платіжного
засобу і кредитна система, яка виростає на цьому грунті, то
грошова частина купецького капіталу ще більше зменшується порівняно
з розмірами операцій, виконуваних цим купецьким капіталом.
\index{iii1}{0273}  %% посилання на сторінку оригінального видання
Якщо я куплю на 1000 фунтів стерлінгів вина із строком
платежу через 3 місяці і продам це вино за готівку до того, як
минуть ці три місяці, то для цієї операції не доводиться авансовувати
жодної копійки. В цьому випадку якнайнаочніше ясно
також, що грошовий капітал, який тут фігурує як купецький
капітал, є безперечно не що інше, як сам промисловий капітал
у своїй формі грошового капіталу, в процесі свого повернення
до самого себе у формі грошей. (Та обставина, що виробник,
який продав на 1000 фунтів стерлінгів товарів із строком платежу
через 3 місяці, може дисконтувати в банкіра одержаний при цьому
вексель, тобто боргове зобов’язання, ні трохи не змінює справи
і не має ніякого відношення до капіталу торговця товарами.)
Якщо за цей проміжок часу ринкові ціни товару впадуть,
скажемо, на 1/10, то купець не тільки не одержить ніякого
зиску, а взагалі виручить тільки 2700 фунтів стерлінгів замість
3000 фунтів стерлінгів. Він мусив би додати 300 фунтів стерлінгів
для того, щоб сплатити борг. Ці 300 фунтів стерлінгів
функціонували б тільки як резерв для вирівнювання ріжниці в
ціні. Але те саме стосується і до виробника. Коли б він сам продавав
по знижених цінах, то він теж утратив би 300 фунтів стерлінгів
і не міг би без резервного капіталу знову почати виробництво
в попередньому масштабі.

Торговець полотном купує у фабриканта на 3000 фунтів
стерлінгів полотна; фабрикант з цих 3000 фунтів стерлінгів платить,
наприклад, 2000 фунтів стерлінгів, щоб купити пряжу;
він купує цю пряжу в торговця пряжею. Гроші, якими фабрикант
платить торговцеві пряжею, не є гроші торговця полотном,
бо цей останній одержав за них товар на таку саму суму. Це —
грошова форма його власного капіталу. В руках торговця пряжею
ці 2000 фунтів стерлінгів виступають тепер як грошовий
капітал, що повернувся до нього; але в якій мірі вони є таким
грошовим капіталом, як відмінні від цих 2000 фунтів стерлінгів
як грошової форми, яку скинуло з себе полотно і набрала
пряжа? Якщо торговець пряжею купив у кредит і продав за
готівку до того, як минув строк платежу, то в цих 2000 фунтах
стерлінгів не міститься жодної копійки купецького капіталу,
відмінного від тієї грошової форми, якої набирає сам промисловий
капітал в процесі свого кругобігу. Товарно-торговельний
капітал, оскільки він, отже, не є проста форма промислового
капіталу, який перебуває в вигляді товарного капіталу або грошового
капіталу в руках купця, є не що інше, як частина грошового
капіталу, яка належить самому купцеві і застосовується
в купівлі й продажу товарів. Ця частина в зменшеному масштабі
представляє частину капіталу, авансованого на виробництво,
яка завжди мусила б перебувати в руках промисловців
як грошовий резерв, як купівельний засіб, і завжди мусила б
циркулювати як їх грошовий капітал. Ця частина перебуває тепер,
зменшеною, в руках капіталістів-купців, постійно функціонуючи
\index{iii1}{0274}  %% посилання на сторінку оригінального видання
як така в процесі циркуляції. Це — частина сукупного
капіталу, яка, коли залишити осторонь витрачання доходів,
мусить постійно циркулювати на ринку як купівельний засіб,
щоб підтримувати безперервність процесу репродукції. Вона
тим менша у відношенні до сукупного капіталу, чим швидше
йде процес репродукції і чим більше розвинена функція грошей
як платіжного засобу, тобто чим більше розвинена кредитна система.\footnote{
Для того, щоб мати змогу класифікувати купецький капітал як виробничий
капітал, Рамсей змішує його з транспортною промисловістю і називав
торгівлю „the transport of commodities from one place to another“ [транспортуванням
товарів з одного місця до іншого] („An Essay on the Distribution of
Wealth“, стор. 19). Те саме змішування маємо вже у Веррі („Meditazioni sulla
Economia Politica“, § 4 [Мілано 1804, стор. 32]) і у Сея („Traité d’Economie
Politique“, I, 14—15). — В своїх „Elements of Political Economy“ (Andover і
New Jork 1835) І. P. Newman каже: „In the existing economical arrangements of
society, the very act which is performed by the merchant, of standing between the
producer and the consumer, advancing to the former capital and receiving products
in return, and handing over these j roducts to the latter, receiving back capital in
return, is a transaction which both facilitates the economical process of the community,
and adds value to the products in relation to which it is performed“ [„При існуючому
економічному устрої суспільства дійсний акт, який виконує купець,
стаючи між виробником і споживачем, авансуючи виробникові капітал і одержуючи
в заміну продукти, передаючи потім ці продукти споживачеві і одержуючи
за це від нього знову свій капітал, є операція, яка полегшує економічний
процес суспільства і додає вартість до продуктів, з якими вона проводилась“]
(стор. 174). Таким чином, завдяки посередництву купця виробник і споживач
заощаджують гроші й час. Така послуга вимагає авансування капіталу й праці
і мусить бути оплачена, „since it adds value to products, for the same products,
in the handset cousumers, are worth more than in the hands of producers“ [„бо вона
додає до продуктів вартість, тому що ті самі продукти мають у руках споживачів
більшу вартість, ніж у руках виробників“]. І таким чином торгівля
здається йому, цілком так само, як панові Сеєві, „strictly an act of production“
[в строгому значенні слова актом виробництва] (стор. 175). Цей погляд Ньюмена
цілком хибний. Споживна вартість товару в руках споживача є більша, ніж у
руках виробника, тому що вона взагалі тільки тут реалізується. Адже споживна
вартість товару реалізується, починає виконувати свою функцію тільки тоді,
коли товар переходить у сферу споживання. В руках виробника вона існує лиш
у потенціальній формі. Але товар не оплачують двічі: спочатку його мінову
вартість, а потім ще, крім того, його споживну вартість. Тим, що я оплачую
його мінову вартість, я привласнюю собі його споживну вартість. І мінова вартість
не дістає ні найменшого приросту від того, що товар переходить з рук
виробника або посередника-купця в руки споживача.
}

Купецький капітал є не що інше, як капітал, що функціонує
в сфері циркуляції. Процес циркуляції є фаза сукупного процесу
репродукції. Але в процесі циркуляції не виробляється
ніякої вартості, а тому й ніякої додаткової вартості. В ньому
відбуваються тільки зміни форми тієї самої маси вартості. Справді,
в ньому не відбувається нічого іншого, крім метаморфози товарів,
яка як така не має ніякого відношення до творення вартості
або до зміни вартості. Якщо при продажу виробленого
товару реалізується додаткова вартість, то це тому, що ця
остання вже існує в ньому; тому при другому акті, при зворотному
обміні грошового капіталу на товар (на елементи виробництва),
\index{iii1}{0275}  %% посилання на сторінку оригінального видання
покупцем так само не реалізується ніякої додаткової
вартості, а тільки підготовляється, за допомогою обміну грошей
на засоби виробництва і робочу силу, виробництво додаткової
вартості. Навпаки. Оскільки ці метаморфози вимагають певного
часу циркуляції — часу, протягом якого капітал взагалі не виробляє
і, отже, не виробляє й додаткової вартості, — цей час обмежує
творення вартості, і додаткова вартість, виражена як норма
зиску, стоятиме саме в зворотному відношенні до тривалості
часу циркуляції. Тому купецький капітал не створює ні вартості,
ні додаткової вартості, тобто не створює безпосередньо. Оскільки
він сприяє скороченню часу циркуляції, він посередньо може
допомагати збільшенню додаткової вартості, вироблюваної промисловим
капіталістом. Оскільки він допомагає розширювати
ринок і опосереднює поділ праці між капіталістами, отже, дає
капіталові змогу працювати в більшому масштабі, його функція
сприяє підвищенню продуктивності промислового капіталу і його
нагромадженню. Оскільки він скорочує час обігу, він підвищує
відношення додаткової вартості до авансованого капіталу, отже,
норму зиску. Оскільки він скорочує ту частину капіталу, яка
мусить постійно перебувати в сфері циркуляції як грошовий
капітал, він збільшує частину капіталу, застосовувану безпосередньо
на виробництво.

Розділ сімнадцятий
Торговельний зиск

В книзі II ми бачили, що чисті функції капіталу в сфері циркуляції
— операції, які мусить виконати промисловий капіталіст
для того, щоб, поперше, реалізувати вартість своїх товарів і,
подруге, знову перетворити цю вартість в елементи виробництва
товару, операції для опосереднення метаморфоз товарного
капіталу Т' — Г — Т, отже, акти продажу й купівлі, — що вони не
створюють ні вартості, ні додаткової вартості. Навпаки, виявилось,
що потрібний для цього час, об’єктивно — щодо товарів і суб’єктивно
— щодо капіталістів, ставить межі творенню вартості і
додаткової вартості. Те, що має силу для метаморфози товарного
капіталу самого по собі, ніяк не змінюється, звичайно, від
того, що частина цього капіталу набирає форми товарно-торговельного
капіталу, або що операції, якими опосереднюється метаморфоза
товарного капіталу, виступають як особливе заняття
особливого підрозділу капіталістів або як виключна функція частини
грошового капіталу. Якщо продаж і купівля товарів — а до
цього зводиться метаморфоза товарного капіталу Т' — Г — Т —
самими промисловими капіталістами є операції, які не створюють
ніякої вартості або додаткової вартості, то вони ні в якому разі
не можуть набути властивості створювати її від того, що вони виконуватимуться
не промисловими капіталістами, а іншими особами.
\parbreak{}  %% абзац продовжується на наступній сторінці
