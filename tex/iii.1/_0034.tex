\index{iii1}{0034}  %% посилання на сторінку оригінального видання
Але яким чином можна було при цьому обміні за масштабом кількості праці вирахувати цю кількість праці, хоч би - посередньо й
відносно, для продуктів, які вимагають витрати праці протягом довгого часу з нерегулярними перервами і непевним результатом,
наприклад, для хліба й худоби? При тому, як могли це робити люди, які не вміли рахувати? Очевидно, не інакше, як шляхом
довгочасного процесу наближення зигзагами, часто в потемках і напомац, при чому, як і завжди, тільки гіркий досвід навчав
людей розуму. Але необхідність для кожного загалом і в цілому повертати собі свої витрати завжди допомагала знаходити
правильний напрям, а незначне число родів предметів, що надходили в обмін, так само як і незмінний — часто протягом століть
— спосіб їх виробництва, полегшували досягнення мети. Доказом того, що , для приблизно точного встановлення відносної
величини вартості цих продуктів зовсім не потрібен був особливо довгий час, е вже самий той факт, що такий товар, як худоба,
для якого, в наслідок довгочасності періоду виробництва окремої його одиниці, це здається найважчим, став першим досить
загальновизнаним грошовим товаром. Для того, щоб це сталося, вартість худоби, її мінове відношення до цілого ряду інших
товарів, мусила вже досягти такої фіксації, яка, порівняно, виходила за рамки звичайного і безсуперечно визнавалась на
території численних племен. І люди того часу — як скотарі, так і їхні покупці — були напевно досить тямущі, щоб при обміні
не віддавати без еквіваленту витрачений ними робочий час. Навпаки: чим ближче люди стоять до первісного стану товарного
виробництва, — наприклад, росіяни або східні народи,—  тим більше часу вони ще й тепер марно витрачають на те, щоб , шляхом
довгого, упертого торгу виручити повний еквівалент і робочого часу, витраченого ними на продукт.

Виходячи з цього визначення
вартості робочим часом, розвивалось усе товарне виробництво, а разом з ним різноманітні відношення, в яких діють різні
сторони закону вартості, як це викладено в першрму відділі • першої книги „Капіталу“; отже, особливо умови, що при них
тільки праця є творець вартості. І при тому це — такі умови, які прокладають собі шлях, не доходячи до свідомості учасників,
і які тільки шляхом трудного теоретичного дослідження можуть бути абстраговані з повсякденної практики, які, отже, діють
подібно до законів природи, що неминуче — як це довів Маркс — випливає з природи товарного виробництва. Найважливішим і
найвирішальнішим кроком уперед був перехід до металічних грошей, але наслідок цього переходу був той, що визначення вартості
робочим часом вже перестало явно виступати на поверхні товарного обміну. Для практичного погляду вирішальним мірилом
вартості стали гроші, і це тим більше, чим різноманітніші ставали товари, що надходили в торгівлю, чим більше вони походили
з віддалених
\parbreak{}  %% абзац продовжується на наступній сторінці
