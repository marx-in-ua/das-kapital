\parcont{}  %% абзац починається на попередній сторінці
\index{iii1}{0020}  %% посилання на сторінку оригінального видання
і його власної загальної критики цього викладу, основаної на
такому розумінні.

Якщо трапляється десь нагода осоромитись на якійсь важкій
справі, то пан професор \emph{Юліус Вольф} в Цюріху ніколи не пропускає
такої нагоди. Вся проблема, — оповідає він нам („Conrads
Jahrbücher“, Dritte Folge, II, стор. 352 і далі), — розв’язується за допомогою
відносної додаткової вартості. Виробництво відносної
додаткової вартості ґрунтується на збільшенні сталого капіталу,
порівняно з змінним. „Плюс у сталому капіталі має за передумову
плюс у продуктивній силі робітників. Але через те що цей
плюс у продуктивній силі (шляхом здешевлення засобів існування)
веде за собою плюс у додатковій вартості, то встановлюється
пряме відношення між зростаючою додатковою вартістю
і зростаючою часткою сталого капіталу в усьому капіталі.
Збільшення сталого капіталу свідчить про збільшення
продуктивної сили праці. Тому при незмінній величині змінного
капіталу і зростаючому сталому капіталі додаткова вартість
мусить зростати згідно з Марксом. Ось питання, яке було нам
поставлене“.

Правда, в сотні місць першої книги Маркс каже якраз протилежне;
правда, твердження, ніби за Марксом відносна додаткова
вартість при зменшенні змінного капіталу підвищується в такій
самій пропорції, в якій збільшується сталий капітал, таке дивовижне,
що для нього не можна знайти ніякого парламентського
вислову; правда, пан Юліус Вольф доводить у кожному рядку,
що він ні відносно, ні абсолютно нічого не зрозумів ні в абсолютній,
ні у відносній додатковій вартості; правда, він сам
каже: „На перший погляд здається, що тут дійсно опиняєшся
у кублі недоречностей“ — що, до речі сказати, є єдине правильне
слово в усій його статті. Та що з того? Пан Юліус
Вольф такий гордий своїм геніальним відкриттям, що не може
пропустити нагоди, щоб не віддати посмертної хвали за це
Марксу і це своє власне незмірне безглуздя вихваляти як „новий
доказ тієї гостроти та далекоглядності, з якою накреслена
його (Маркса) критична система капіталістичного господарства“!

Але буває ще краще: пан Вольф каже: „Рікардо висловив
твердження: рівні витрати капіталу — рівна додаткова вартість
(зиск) і разом з тим твердження: рівні витрати праці — рівна
(щодо маси) додаткова вартість. І питання полягало в тому,
як одно погоджується з другим. Але Маркс не визнавав питання
в цій формі. \emph{Він безсумнівно довів (у третій книзі)},
що друге твердження не є безумовний наслідок закону вартості,
що воно навіть суперечить його законові вартості і, отже\dots{}
мусить бути прямо відкинуте“. І після цього він досліджує, хто
з нас обох помилявся, я чи Маркс. Що він сам помиляється,
про це він, звичайно, не думає.

Коли б я схотів сказати хоч одно слово з приводу цього
прекрасного місця, це значило б ображати моїх читачів і зовсім
\parbreak{}  %% абзац продовжується на наступній сторінці
