\parcont{}  %% абзац починається на попередній сторінці
\index{iii1}{0135}  %% посилання на сторінку оригінального видання
було б 1432080000 фунтів бавовни на рік. Але довіз бавовни,
коли відняти вивіз, становив в 1856 і 1857 роках тільки
1022576832 фунти; отже, неминуче мусив постати дефіцит
у 409503168 фунтів. Пан Бейнс, який люб’язно згодився обговорити
зі мною цю справу, гадає, що обчислення річного споживання
бавовни, основане на споживанні блекбернської округи,
було б перебільшеним не тільки в наслідок ріжниці нумерів, що
випрядаються, але й в наслідок вищої якості машин. Він оцінює загальне
річне споживання бавовни в Сполученому Королівстві в
1000 мільйонів фунтів. Але якщо він і має рацію, якщо дійсно
є надмір подання в 22\sfrac{1}{2}  мільйони, то вже тепер попит і подання,
як видно, майже урівноважуються, навіть якщо не брати до
уваги додаткові веретена і ткацькі верстати, які, за паном
Бейнсом, встановлюються в його окрузі і, отже, напевно і
в інших округах“ (стор. 59, 60).

\subsection{Загальна ілюстрація: бавовняна криза 1861--1865~\abbr{рр.}}

\subsubsection{Попередній період 1845-1860~\abbr{рр.}}

1845 рік. Час розквіту бавовняної промисловості. Дуже низькі
ціни на бавовну. Л. Горнер каже про це: „Протягом останніх
8 років я не бачив жодного періоду такого пожвавленого стану
справ, як минулим літом і осінню. Особливо у бавовнопрядільництві.
Протягом цілого півроку я щотижня одержував заяви
про нові капіталовкладення у фабрики; це були або нові фабрики,
які будувалися, або ті нечисленні фабрики, які пустували, знаходили
нових орендарів, абож ті фабрики, які працювали, розширювались
і устатковувались новими потужнішими паровими машинами
та збільшеною кількістю робочих машин“ („Rep. of Insp.
of Fact., Oct. 1845“, стор. 13).

1846 рік. Починаються нарікання. „Уже протягом досить довгого
часу я чую від дуже багатьох бавовняних фабрикантів нарікання
на пригнічений стан їх справ\dots{} протягом останніх 6 тижнів
різні фабрики почали працювати неповний час, звичайно
8 годин на день замість 12; це, як видно, поширюється\dots{} дуже
підвищились ціни бавовни і\dots{} не тільки не підвищились ціни
фабрикатів, але\dots{} вони стоять ще нижче, ніж перед підвищенням
цін бавовни. Значне збільшення числа бавовняних фабрик
протягом останніх 4 років мусило мати своїм наслідком,
з одного боку, дуже збільшений попит на сировинний матеріал
і, з другого боку, дуже збільшене подання фабрикатів на ринку;
обидві причини мусили спільно сприяти зниженню зиску, поки
лишались незмінними подання сировинного матеріалу і попит
на фабрикати; але вони вплинули ще далеко дужче, бо, з одного
боку, за останній час було недостатнє подання бавовни,
і, з другого боку, зменшився попит на фабрикати на різних
внутрішніх і закордонних ринках“ („Rep. of Insp. of Fact., Oct.
1846“ стор. 10).
