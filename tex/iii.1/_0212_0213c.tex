
\index{iii1}{0212}  %% посилання на сторінку оригінального видання

\chapter{Закон тенденції норми зиску до~падіння}

\section{Закон як такий}

При даній заробітній платі і при даному робочому дні змінний
капітал, наприклад, в 100, представляє певне число приведених
у рух робітників; він є показник цього числа. Припустімо,
наприклад, що 100\pound{ фунтів стерлінгів} становлять заробітну плату
100 робітників, скажімо, за 1 тиждень. Якщо ці 100 робітників
виконують стільки ж необхідної праці, скільки додаткової праці,
якщо вони, отже, щодня працюють стільки ж часу на себе
самих, тобто для репродукції своєї заробітної плати, скільки
на капіталістів, тобто для виробництва додаткової вартості, то
вся вироблена ними вартість буде \deq{} 200\pound{ фунтам стерлінгів},
а вироблена ними додаткова вартість становитиме 100\pound{ фунтів
стерлінгів}. Норма додаткової вартості \frac{m}{v} була б \deq{} 100\%. Однак,
ця норма додаткової вартості, як ми бачили, виражалася б у дуже
різних нормах зиску, залежно від різного розміру сталого капіталу
$c$, а тому й усього капіталу $K$, бо норма зиску $ \deq{} \frac{m}{K}$. При нормі
додаткової вартості в 100\%,

\begin{center}
якщо $c \deq{} \phantom{0}50$, $v \deq{} 100$, то $р' \deq{} \frac{100}{150} \deq{} 66\frac{2}{3}\%$;

якщо $c \deq{} 100$, $v \deq{} 100$, то $р' \deq{} \frac{100}{200} \deq{} 50\phantom{\frac{1}{1}}\%$;

якщо $c \deq{} 200$, $v \deq{} 100$, то $р' \deq{} \frac{100}{300} \deq{} 33\frac{1}{3}\%$;

якщо $c \deq{} 300$, $v \deq{} 100$, то $р' \deq{} \frac{100}{400} \deq{} 25\phantom{\frac{1}{1}}\%$;

якщо $c \deq{} 400$, $v \deq{} 100$, то $р' \deq{} \frac{100}{500} \deq{} 20\phantom{\frac{1}{1}}\%$.
\end{center}

\noindent{}Таким чином при незмінному ступені експлуатації праці та
сама норма додаткової вартості виражалася б у падаючій нормі
зиску, бо разом з матеріальним розміром сталого капіталу зростає,
\index{iii1}{0213}  %% посилання на сторінку оригінального видання
хоч і не в тій самій пропорції, і розмір вартості сталого,
а разом з ним і всього капіталу.

Якщо ми далі припустимо, що ця ступнева зміна в складі
капіталу відбувається не тільки в окремих сферах виробництва,
але більш-менш в усіх або, принаймні, у вирішальних сферах
виробництва, так що вона таким чином рівнозначна зміні в пересічному
органічному складі сукупного капіталу, належного певному
суспільству, то таке ступневе наростання сталого капіталу
порівняно з змінним неминуче мусить мати своїм результатом
\emph{ступневе зниження загальної норми зиску} при незмінній нормі
додаткової вартості, або при незмінному ступені експлуатації
праці капіталом. Але виявилось, як закон капіталістичного способу
виробництва, що з розвитком цього способу виробництва
відбувається відносне зменшення змінного капіталу порівняно
з сталим капіталом і, отже, порівняно з усім капіталом, який
приводиться в рух. Це означає тільки те, що те саме число
робітників, та сама кількість робочої сили, якою можна розпоряджатися
при змінному капіталі даного розміру вартості, в наслідок
особливих методів виробництва, що розвиваються в капіталістичному
виробництві, за той самий час приводить в рух,
переробляє, продуктивно споживає постійно зростаючу масу
засобів праці, машин і всякого роду основного капіталу, сировинних
і допоміжних матеріалів, отже і сталий капітал постійно
зростаючого розміру вартості. Це прогресуюче відносне зменшення
змінного капіталу порівняно з сталим і, отже, з усім капіталом,
тотожне з дедалі вищим пересічним органічним складом
суспільного капіталу. Це — так само тільки інший вираз
прогресуючого розвитку суспільної продуктивної сили праці,
який виявляється саме в тому, що за допомогою зростаючого
застосування машин і взагалі основного капіталу при тому самому
числі робітників за той самий час, тобто з меншою кількістю
праці, перетворюється в продукти більша кількість сировинних
і допоміжних матеріалів. Цьому зростаючому розмірові вартості
сталого капіталу — хоч він тільки віддалено представляє зростання
дійсної маси споживних вартостей, з яких речево складається
сталий капітал — відповідає зростаюче здешевлення продукту.
Кожний індивідуальний продукт, розглядуваний сам по
собі, містить у собі меншу суму праці, ніж на нижчому ступені виробництва,
де відношення капіталу, витраченого на працю, до
капіталу, витраченого на засоби виробництва, є незрівняно
більша величина. Отже, гіпотетичний ряд, наведений нами на
початку цього розділу, виражає дійсну тенденцію капіталістичного
виробництва. Це останнє разом з прогресуючим відносним
зменшенням змінного капіталу порівняно з сталим створює дедалі
вищий органічний склад сукупного капіталу, безпосереднім
наслідком чого є те, що норма додаткової вартості при незмінному
і навіть при зростаючому ступені експлуатації праці
виражається в дедалі нижчій загальній нормі зиску. (Далі буде
\parbreak{}  %% абзац продовжується на наступній сторінці
