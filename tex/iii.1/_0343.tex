\parcont{}  %% абзац починається на попередній сторінці
\index{iii1}{0343}  %% посилання на сторінку оригінального видання
час виробництва і час циркуляції входить у визначення ціни
товарів і яким чином саме цим визначається норма зиску для
даного часу обороту капіталу, а визначенням зиску для даного
часу визначається якраз норма процента. Вся його глибокодумність
тут, як і завжди, полягає тільки в тому, щоб бачити хмари
пилу на поверхні і чванливо говорити про цей пил, як про щось
таємниче й значне.

\section{Поділ зиску. Розмір процента.
„Природна“~норма~процента}

Предмет цього розділу, як і взагалі всі явища кредиту, про
які ми говоритимем далі, тут не можуть бути досліджені в подробицях.
Конкуренція між позикодавцями й позичальниками і,
як результат її, короткочасні коливання грошового ринку виходять
за межі нашого дослідження. Змалювання кругобігу, що
його пророблює норма процента протягом промислового циклу,
передбачає попереднє змалювання самого цього циклу, яке тут
так само не може бути зроблене. Те саме стосується й до більшого
чи меншого приблизного вирівнення розміру процента
на світовому ринку. Тут нам доведеться тільки дослідити самостійну
форму капіталу, що дає процент, та усамостійнення процента
відносно зиску.

Через те що процент є просто частина зиску, яку, як ми
це досі припускали, промисловий капіталіст повинен сплачувати
грошовому капіталістові, то максимальну межу процента
являє собою самий зиск, при чому частина, яка дістається функціонуючому
капіталістові, була б \deq{} 0. Залишаючи осторонь окремі
випадки, коли процент фактично може бути більший за зиск, —
але тоді він не може бути виплачений із зиску, — можна було б,
мабуть, за максимальну межу процента вважати весь зиск мінус
та його частина, яка зводиться до плати за нагляд (wages
of superintendence) і яку нам далі доведеться розглянути. Мінімальна
межа процента ніяк не може бути визначена. Процент
може впасти до якого завгодно низького рівня. Але тоді знову
й знову виступають протидіючі обставини і підвищують його
понад цей відносний мінімум.

„Відношення між сумою, сплаченою за вживання капіталу,
і самим цим капіталом виражав норму процента, вимірену в
грошах“. — „Норма процента залежить 1)~від норми зиску; 2)~від
того відношення, в якому весь зиск ділиться між позикодавцем
і позичальником“ („\emph{Economist}“, 22 January 1853 [стор. 89]). „Тому
що виплачуване як процент за користування тим, що береться в
позику, є частина зиску, що його здатне виробити взяте в позику,
то цей процент завжди мусить регулюватися цим зиском“
(\emph{Massie}: „An Essay on the Governing Causes of the Natural Rate of
Interest etc.“, London 1750, стор. 49).
