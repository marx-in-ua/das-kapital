\parcont{}  %% абзац починається на попередній сторінці
\index{iii1}{0349}  %% посилання на сторінку оригінального видання
в яких доводиться обчисляти проценти. Але якщо запитати, чому
межі середнього розміру процента не можуть бути виведені
з загальних законів, то відповіддю служить просто природа процента.
Він є просто частина пересічного зиску. Один і той же
капітал виступає в двоякому визначенні: як капітал, що віддається
в позику — в руках позикодавця, як промисловий або як
торговельний капітал — у руках функціонуючого капіталіста. Але
він функціонує тільки один раз і навіть зиск виробляє він тільки
один раз. В самому процесі виробництва характер капіталу як
позиченого капіталу не грає ніякої ролі. Яким чином ділять
між собою зиск з цього капіталу обидві ці особи, які можуть
претендувати на цей зиск, це само по собі в такій же мірі чисто
емпіричний, належний до царства випадкового факт, як процентний
розподіл загального зиску якогось товариського підприємства
між різними пайовиками. При поділі на додаткову вартість
і заробітну плату, на якому істотно грунтується визначення
норми зиску, визначально впливають два цілком різні елементи —
робоча сила і капітал; це — функції двох незалежних змінних, які
взаємно одна одну обмежують; і з їх \emph{якісної ріжниці} походить
\emph{кількісний поділ }виробленої вартості. Пізніше ми побачимо, що
те саме має місце при поділі додаткової вартості на ренту і зиск.
Нічого подібного не відбувається щодо процента. Тут \emph{якісна}
\emph{ріжниця} походить, навпаки, як ми зараз побачимо, з \emph{чисто кількісного
поділу} однієї і тієї самої частини додаткової вартості.

З того, що досі викладено, випливає, що немає ніякої „природної“
норми процента. Але якщо, з одного боку, межі середнього
розміру процента, в протилежність до загальної норми зиску,
або межі пересічної норми процента, в відміну від ринкових
норм процента, які постійно коливаються, не можуть бути встановлені
ніяким загальним законом, бо тут ідеться тільки про
поділ гуртового зиску, на підставі різних титулів, між двома володільцями
капіталу, — то, навпаки, розмір процента, однаково,
чи середній розмір його, чи його кожноразова ринкова норма,
являє собою рівномірну, визначену і наочну величину цілком
інакше, ніж це має місце щодо загальної норми зиску.\footnote{
„Ціна товарів постійно коливається; всі вони призначені для різного роду
споживання; гроші ж служать для всякої мети. Товари, навіть одного й того ж
роду, відрізняються щодо якості; гроші готівкою завжди мають однакову вартість
абож повинні її мати. Звідси випливає, що ціна грошей, яку ми позначаємо
словом процент, має більшу стійкість і рівномірність, ніж ціна всякої
іншої речі“ (\emph{J.~Steuart}: „Principles of Political Economy“, франц. перекл. 1789,
VI, стор. 27).
}

Розмір процента відноситься до норми зиску так само, як ринкова
ціна товару відноситься до його вартості. Оскільки розмір
процента визначається нормою зиску, він завжди визначається
загальною нормою зиску, а не особливими нормами зиску, які
можуть панувати в окремих галузях промисловості, і тим більше
не надзиском, що його окремий капіталіст може одержати в якійсь
\parbreak{}  %% абзац продовжується на наступній сторінці
