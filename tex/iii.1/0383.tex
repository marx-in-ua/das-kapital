шовій вартості, при чому цю суму, завжди, після скінчення певного строку, належить
повернути назад. Якщо капітал віддається в позику грішми, тобто банкнотами,
або банковим кредитом, або переказом на кореспондента, то на суму, яку
належить повернути при сплаті позики, додатково нараховується за користування
капіталом стільки то процентів. А при товарах, грошова вартість яких встановлюється
учасниками угоди і передача яких означає продаж, встановлена сума,
яка має бути сплачена, включає вже винагороду за користування капіталом
і за риск до скінчення строку. Писані платіжні зобов’язання з певним, строком
платежу здебільшого видаються для таких кредитів. І ці зобов’язання, які
можна передавати, або промеси, становлять засіб, за допомогою якого позикодавці,
коли їм трапляється нагода застосувати свій капітал у формі грошей чи
товарів до скінчення строку-цих векселів, здебільшого можуть дешевше взяти
позику або купити, бо їх власний кредит посилюється кредитом другого імени
на векселі“ („Inquiry into the Currency Principle“, crop. 87).

Ch. Coquelin: „Du Crédit et des Banques dans l’Industrie“, в. Revue des deux Mondes“,
1842, т. 31: „В кожній країні більшість кредитних операцій виконується в самій
сфері промислових відносин... Виробник сировинного матеріалу авансує його
фабрикантові, який обробляє цей матеріал, і одержує від нього платіжне зобов’язання
з певним строком платежу. Виконавши свою частину роботи, фабрикант
в свою чергу і на подібних же умовах авансує свій продукт іншому фабрикантові,
який мусить обробляти його далі, і таким чином кредит поширюється все далі і
далі, від одного до другого, аж до споживача. Гуртовий торговець дає в кредит
товари дрібному торговцеві, тимчасом як сам він одержує товари в кредит
від фабриканта або комісіонера. Кожен бере в позику однією рукою і дає в
позику другою, — іноді гроші, але далеко частіше продукти. Таким чином у промислових
відносинах відбувається безперестанний обмін позик, які комбінуються
і перехрещуються в усіх напрямах. Саме у множенні і зростанні цих взаємних
позик полягає розвиток кредиту, і тут справжнє місцеперебування його сили“.

Другий бік кредиту примикає до розвитку торгівлі грішми,
який у капіталістичному виробництві, звичайно, іде паралельно
з розвитком товарної торгівлі. В попередньому відділі (розділ XIX)
ми бачили, як у руках торговців грішми концентрується зберігання
резервних фондів ділових людей, технічні операції одержання
й сплати грошей, міжнародних платежів, і разом з цим
торгівля злитками. В зв’язку з цією торгівлею грішми розвивається
другий бік кредиту, — управління капіталом, що дає процент,
або грошовим капіталом, як особлива функція торговців
грішми. Одержання грошей у позику і віддача їх в позику стає
їх особливою справою. Вони виступають як посередники між дійсним
позикодавцем і позичальником грошового капіталу. Взагалі
кажучи, справа банкіра з цього боку полягає в тому,
щоб концентрувати в своїх руках великими масами капітал, що
дається в позику, так що замість окремого позикодавця промисловим
і торговельним капіталістам протистоять банкіри як представники
всіх грошових позикодавців. Вони стають загальними
управителями грошового капіталу. З другого боку, вони концентрують
позичальників супроти всіх позикодавців, бо вони
беруть позики для всього торговельного світу. З одного боку,
банк представляє централізацію грошового капіталу; позикодавців,
з другого боку — централізацію позичальників. Його зиск,
взагалі кажучи, полягає в тому, що він одержує позики за нижчі
проценти, ніж дає в позику.

Позиковий капітал, що ним порядкують банки, припливає до
