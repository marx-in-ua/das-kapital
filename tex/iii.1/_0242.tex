\parcont{}  %% абзац починається на попередній сторінці
\index{iii1}{0242}  %% посилання на сторінку оригінального видання
праці, яку можна видушити, упредметнилася в товарах, додаткова
вартість вироблена. Але цим виробництвом додаткової
вартості закінчено тільки перший акт капіталістичного процесу
виробництва, безпосередній процес виробництва. Капітал увібрав
стільки то неоплаченої праці. З розвитком процесу, який виражається
в падінні норми зиску, маса вироблюваної таким чином
додаткової вартості зростає до потворних розмірів. Тепер
наступає другий акт процесу. Вся маса товарів, весь продукт,
як та його частина, що заміщає сталий і змінний капітал, так
і та, що представляє додаткову вартість, мусить бути продана.
Якщо цього не відбувається, або відбувається тільки почасти,
або тільки при таких цінах, що стоять нижче цін виробництва,
то хоч робітника і експлуатували, але експлуатація його не
реалізується як така для капіталіста; вона може бути зв’язана
з тим, що видушена додаткова вартість зовсім не реалізується чи
реалізується тільки частково, або навіть з тим, що відбувається
часткова чи повна втрата капіталу капіталіста. Умови безпосередньої
експлуатації і умови її реалізації не є тотожні. Вони не
збігаються не тільки щодо часу й місця, але й у понятті. Перші
обмежені тільки продуктивною силою суспільства, другі — пропорціональністю
різних галузей виробництва і споживною силою
суспільства. Але ця остання визначається не абсолютною продуктивною
силою і не абсолютною споживною силою, а споживною
силою на основі антагоністичних відносин розподілу,
що зводять споживання величезної маси суспільства до мінімуму,
який змінюється тільки в більш-менш вузьких межах.
Вона обмежена далі прагненням до нагромадження, прагненням
до збільшення капіталу і до виробництва додаткової вартості
в розширеному масштабі. Такий є закон капіталістичного виробництва,
даний постійними революціями в самих методах виробництва,
постійно зв’язаним з цими революціями знеціненням
наявного капіталу, загальною конкурентною боротьбою і необхідністю
поліпшувати виробництво та розширювати його розміри
ради самого тільки збереження і під загрозою загибелі. Тому ринок
мусить постійно розширюватись, так що ринкові зв’язки і
умови, що їх регулюють, все більше й більше набирають
вигляду природного закону, незалежного від виробників, все
більше й більше стають непіддатними контролеві. Внутрішня
суперечність прагне знайти собі розв’язання в розширенні зовнішнього
поля виробництва. Але чим більше розвивається продуктивна
сила, тим більше стає вона в суперечність з тією
вузькою базою, на якій ґрунтуються відносини споживання. На
цій повній суперечностей базі зовсім не є суперечністю те,
що надмір капіталу є зв’язаний з зростаючим надміром населення;
бо хоч при поєднанні того й другого надміру маса вироблюваної
додаткової вартості зросла б, але саме тому зросла б
і суперечність між тими умовами, при яких ця додаткова вартість
виробляється, і тими умовами, при яких вона реалізується.
