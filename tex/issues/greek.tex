\documentclass{kapital}
%% should be a class option
\renewcommand{\parbreak}{\unskip\ignorespaces}
\renewcommand{\parcont}{\unskip\ignorespaces}

%% proper quote marks 
\newunicodechar{„}{«}
\newunicodechar{“}{»}

%% ditto mark
\renewcommand{\dittomark}{~}

%% overfull boxes
\vfuzz=12pt

\begin{document}

%% i.0023
«5 ліжок \deq{} 1 будинкові»
(«\textgreek{Κλίναι πέντε άνι\dots{} οσου αί πέντε χλίναι}»)

«не відрізняється» від:

«5 ліжок \deq{} такій і такій кількості грошей»

(«\textgreek{Κλίναι πέντε άντί\dots{} όσον αί πέντε χλίναι}»).


може бути без рівности, а рівність — без спільномірности» («\textgreek{οΰτ’ ίσότης μη σΰσηςσυμμετρίας}»). Але
тут він збивається й відмовляється від дальшої аналізи форми вартости. «Однак справді неможливо
(«\textgreek{τη μεν όυν αληυεία άδύνατον}»), щоб такі різні речі були спільномірні», тобто якісно однакові. Таке

%% i.0062
золото, а золото на товари» («\textgreek{Ἐκ δὲ τοῦ\dots{} πυρὸς ἀνταμείβεσθαι πάντα φησὶν ὁ Ἡράκλειτος, καὶ πὺρ
ἁπάντων, ὥσπερ χρυσοῦ χρήματα καὶ χρημάτων χρυσός}»), (\emph{F.~Lassale}:

%% i.0087
\textgreek{„Οὐδέν γὰρ ἀνθρώποισιν οἷον ἀργυρὸς \\
Κακὸν νόμισμα ἔβλαστε, τοῦτο καὶ πόλεις \\
Πορθεῖ, τόδ’ ἄνδρας ἐξανίστησιν δόμων. \\
Τόδ’ ἐκδιδάσκει καὶ παραλλάσσει φρένας \\
Χρηστὰς πρὸς αἰσχρὰ ἀνθρώποις ἔχειν, \\
Καὶ παντὸς ἔργου δυσσέβειαν εἰδέναι.“} \\


(«\textgreek{Ἐλπιζούσης  πλεονεξίας ἀνάξειν ἐκ τῶν μυκῶν τῆς γῆς αὐτὸν τὸν Πλούτωνα}). (\emph{Atheneus}:

%% i.0106
«Справжнє багатство (\textgreek{δ αληζινος πλουιος}) складається з таких споживних
вартостей, бо кількість власности цього роду, достатньої для
доброго життя, не є безмежна. Але існує вмілість надбання іншого роду,
яка переважно і з повним правом називається хрематистикою, вмілість,
у наслідок якої, здається, не існує жодних меж багатства і власности.
Товарова торговля (\textgreek{η χαπηλιχη} дослівно значить торговля на роздріб,
і Арістотель бере цю форму, бо в ній переважає споживна вартість) з природи
не належить до хрематистики, бо тут обмін стосується лише до предметів,
що їм самим (покупцям і продавцям) потрібні». Тому, висновує
він далі, первісною формою товарової торговлі була мінова торговля,
але з її поширенням неминуче постали гроші. З винаходом грошей мінова
торговля неминуче мусила розвинутися в \textgreek{χαπηλιχη}, в товарову торговлю,
а ця, всупереч до її первісної тенденції, перетворилась на хрематистику,
на вмілість робити гроші. Хрематистика ж відрізняється від економіки
тим, що «для неї циркуляція є джерело багатства (\textgreek{ποιητιχη χεηματωυ\dots{} δια χρηηατωυ σιαβολης}). І
вона, здається, ґрунтується на грошах, бо гроші є початок і кінець цього роду обміну (\textgreek{το γχρ νομισμα
στοιχειον χαι περας ιης αλλαγης εστιν}).

%% i.0107
«\textgreek{Σώζειν}» (рятувати) — це один з найхарактеристичніших висловів

%% i.0117
Звідси й назва його (\textgreek{τόχος} — процент і породжене).

%% i.0179
хоч буде цей власник атенський \textgreek{χαλος χαγανος}

%% i.0298 i.0299
сфери діяльности\footnote{
Так, в «Одісеї», XIV, 228 говориться: «\textgreek{Ἄλλoς γὰρ τ’ἄλλοισιν ἀνὴρ ἐπιτέρπεται ἔργοις}»\footnote*{
Одні люди люблять одне, інші — інше. \emph{Ред.}
} a Архілох y Секста Емпірика каже: «\textgreek{Ἄλλος ἄλλῳ ἐπ’ἔργῳ καρδίην ἰαίνεται}»\footnote*{
Одне тішить серце одного, інше — іншого. \emph{Ред.}
}}, а без обмеження ніде не можна зробити нічого
значного\footnote{
«\textgreek{Поλλ’ ἠπίστατο ἔργα, κακῶς δ’ ἠπίστατο πάντα}»\footnote*{
Багато знав він справ, та всі погано знав. \emph{Ред.}
} — Атенець, як товаропродуцент,
почував свою перевагу над спартанцем, бо цей останній міг
порядкувати у війні людьми, але не грішми, — як це Тукідід вкладав
в уста Перікла у промові, в якій він підцьковує атенців до пелопонеської
війни: «\textgreek{Σώμασί τε ἐτοιμότεροι οἱ αὐτουργοὶ τῶν ἀνθρώπων ἤ χρήμασι πολεμεῖν}»\footnote*{
Люди, що працюють для задоволення власних потреб, радше
віддадуть на війну свої тіла, ніж гроші. \emph{Ред.}
}. (\emph{Thucydides}:
«Geschichte des Peloponnesischen Krieges», книга перша, відділ
141). А проте їхнім ідеалом, навіть у матеріяльній продукції, була
\textgreek{αυταρχεια}\footnote*{
— автаркія. \emph{Ред.}
}, що протиставляється поділові праці, бо «\textgreek{παρ’ ὧν γὰρ τὸ εὖ, παρὰ τούτων καὶ τὸ
αὐτάρκες}»\footnote*{
«з цього постає благо, а з того і незалежність». \emph{Ред.}
}. Треба при цьому зважити, що за часів упадку
30 тиранів не було ще й \num{5.000} атенців без земельної власности.
}

від усякої іншої роботи». («\textgreek{Οὐ γὰρ ἐθέλει τὸ πραττόμενον τὴν τοῦ πράττοντος σχολὴν περιμένειν, ἀλλ’
ἀνάγκη τὸν πράττοντα τῷ πραττομένῷ ἐπακολουθεῖν μὴ ἐν παρέργου μέρει. \textemdash{} Ἀνάγκη. Ἐκ δὴ τούτων πλείω
τε ἕκαστα γίγνεται καὶ κάλλιον καὶ ῥᾷον, ὅταν εἷς ἓν κατὰ φύσιν καὶ ἐν καιρῷ, σχολὴν τῶν ἄλλων ἄγων,
πράττῃ»}). («Respublica», lib. II, c. 12,
ed. Baiter, Orelli etc.). Подібні думки ми маємо в Тукідіда: «Geschichte
des Peloponnesischen Krieges», книга перша, відділ 142: «Морська справа
є така ж умілість, як і будь-що інше, і не можна коло неї працювати принагідно,
як коло якоїсь побічної справи, навпаки, морська справа не
дозволяє працювати коло чогось іншого навіть як побічної справи».
Якщо справа мусить чекати на робітника, каже Платон, то часто ґавиться
критичний момент продукції і продукт псується, «\textgreek{ἔργου καιρὸν διόλλυται}». Цю

%% i.0320
як пізніш «філософом» \textgreek{χατ’ ε’ξοχη'ν}\footnote*{
— переважно. \emph{Ред.}
}

%% ii.0099
(\textgreek{δυναμει}), а не справжній (\textgreek{ενεργεια}) елемент товарового запасу.

%% iii.0368
Арістотель: \textgreek{Ο γάρ δεσπότης οὐχ ἐν τω  χτάσθαι τους δούλους, ἀλλ’ ἐν τω
χρῆσθαι δούλης.} [Бо пан — капіталіст — виявляється як такий не в
набуванні рабів — власності на капітал, яка дає владу купувати
працю, — а у використанні рабів — вживанні робітників — нині
найманих робітників у процесі виробництва] \textgreek{ Ἔστι δέ αὕτη ἡ επιστήμη
οὐδέν μεγα ἔχουσα οὐδέ οεμνόν.} [Але в цій науці немає нічого великого
або величного]; \textgreek{ἄ γάρ τόν δοῦλον ἔπιστασθαι δεῖ ποιεῖν, έχεῖνον δεῖ
ταῦτα ἐπίστασθαι ἐπιτάττειν} [він повинен уміти наказувати те, що раб
повинен уміти виконати]. \textgreek{Διο οσοις ἐξουσία μή αυτούς χαχοπαθειν, επιτροπος
λαμβανει ταυτην την τιμην, αυτοι δε πολιτευονται φιλοσοφουσιν.}
[Коли в панів немає потреби обтяжувати себе цим, цю честь
бере на себе наглядач, а вони самі займаються державними
справами або філософією]. (Aristoteles: „De Republica“. Вид.
Беккера. Книга І, 7 [Охоnіі 1837, стор. 10 і далі]).
\end{document}
