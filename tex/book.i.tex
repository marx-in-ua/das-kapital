\documentclass{kapital}
%% should be a class option
\renewcommand{\parbreak}{\unskip\ignorespaces}
\renewcommand{\parcont}{\unskip\ignorespaces}

%% proper quote marks 
\newunicodechar{„}{«}
\newunicodechar{“}{»}

%% ditto mark
\renewcommand{\dittomark}{~}

%% overfull boxes
\vfuzz=12pt


\newcommand{\VolumeNumber}{Том І}
\newcommand{\BookNumber}{Книга перша}
\newcommand{\BookTitle}{Процес продукції капіталу}
\newcommand{\BookSource}{Переклад з четвертого німецького видання}
\newcommand{\BookAuthors}{за редакцією Д.~Рабіновича і~С.~Трикоза}
\newcommand{\SourceYear}{Харків, 1933}
\newcommand{\Biblio}{Т. І. — Кн. І: Процес продукції капіталу /
Пер. із 4 нім. вид.; За ред. Д. Рабіновича і~С.~Трикоза.}

\newcommand{\VolumeNumberDe}{Erster Band}
\newcommand{\BookNumberDe}{Buch I}
\newcommand{\BookTitleDe}{Der Produktionsprocess des Kapitals}
\newcommand{\AuflageDe}{Vierte, durchgesehene Auflage}
\newcommand{\HerausgegebenDe}{Herausgegeben von Friedrich Engels}
\newcommand{\PublisherDe}{Verlag von Otto Meissner}
\newcommand{\CityDe}{Hamburg}
\newcommand{\YearDe}{1890}

\begin{document}
  \pagenumbering{roman}
\frontmatter
\thispagestyle{empty}
\null\vspace{0.5cm}
\noindent\htitlespace{} {\large Карл Маркс}

\vspace{1.5cm}
\noindent\htitlespace{} {\HUGE\bfseries\letterspacefont\MakeUppercase Капітал}

\vspace{0.1cm}
\noindent\htitlespace{} {\large\bfseries Критика політичної економії }

\vspace{1.5cm}
\noindent\htitlespace{} {\large Том І. Книга І}

\noindent\htitlespace{} {\large Процес продукції капіталу}

\vfill
\noindent\htitlespace{} {маркс.укр • 2019}
\clearpage
\thispagestyle{empty}
{\footnotesize\noindent{\bfseries{}Маркс, Карл}

Капітал: Критика політичної економії / Заг. ред. А.~Потапова, І.~Зробок, Д.~Потапова. — \City, \Year. — \Biblio —
\lastpageref{pagesLTS.roman} + \lastpageref{pagesLTS.arabic} с.
(Готується до друку).
}
\vfill
\noindent{\footnotesize Друкується за виданням: Карл Маркс. Капітал: Критика політичної економії — \FullSource.}

\bigskip
\noindent{\footnotesize Цей твір ліцензовано на умовах Ліцензіїї Creative Commons Із Зазначенням Авторства — Поширення На Тих Самих Умовах 4.0 Міжнародна. Щоб ознайомитися з копією цієї ліцензії, завітайте на http://creativecommons.org/licenses/by-sa/4.0/ або направте листа за адресою Creative Commons, PO Box 1866, Mountain View, CA 94042, USA.}
\clearpage

\tableofcontents*
\cleardoublepage
\pagestyle{mypage}
\nonumsection{Андрію Річицькому}{}{}

Присвячується Андрію Річицькому (справжнє ім'я — Пісоцький Анатолій Андрійович), видатному українському вченому, перекладачеві та політичному діячу. В 1920-х роках він почав працювати над першим і єдиним українським перекладом «Капіталу» Маркса з німецької, проте встиг завершити лише перший том. Репресований у 1934 році, після чого його ім’я не згадувалося у наступних виданнях «Капіталу» зовсім, попри те, що вони базувалися на його роботі. Був реабілітований у 1990. 

\nonumsection{Подяки}{}{}

\noindent{}Особлива подяка \textbf{Миколі Климчуку} за те,
що видання вийшло охайним і гармонійним. 

\smallskip
\noindent{}Над виданням в рамках спільної ініціативи маркс.укр працювали:
\begin{itemize}[nosep]
\item \textbf{Ірина Зробок}
\item \textbf{Ернест Гук}
\item \textbf{Антон Потапов}
\item Юрий Латыш
\item Taras Bilous
\item Ivanna Kutsil
\item Max Starchevsky
\end{itemize}
\noindent{}а також
Dan Bogynski, 
Dmytro Zhelaha,
Богдан Бернадський,
Nina Garbo, Андрій Андросович, Liza Walther, Сергій Зінченко,
Mike A. Liakh, Stas Sergienko, Volodymyr Boiko, Andriy Panchenkov, Денис Кучеренко, Настя Авдоніна, Oleg Kavaler, Volodymyr Shostak, Ілля Токар, Vika Khomovska, Maxim Sokhatsky, Mariana Potapova, Danylo Yankovskyi, Anna Potasheva, Yaroslav Kovalchuk, Денис Панкратов, Роман Козлов, \textenglish{Yevhenii Mo\-nas\-tyr\-skyi}, Andrew Zukkermann, Олександр Брайко, Anton Potapenko, Oleksandr Lapchuk, Ліда Криштоп, Anton Stepankovsky, Anton Pechenkin, Wowhura Wowhura, Ann Kurovska, Надія Йовченко, \textenglish{Ksena Meyta, Taras Salamaniuk}, Сергей Алушкин, \textenglish{Kirill Kramskiy}, Eugenia Virlich, Valeriy Kuropyatnik, Наталка Чех, Artem Tidva, Snizhana Umanets, Liuba Kuibida, Andriy Pogasiy, Yulia Dukach, Дмитро Вершинін, Леонид Бегунов-Новиков, Галина Новосад, Artem Borysov, Oleksandr Nykolyak, Петро Садовий, Mikhail Khokhlovych, Oleksii Parfeniuk, Hlafira Titarenko, Володимир Гунько, Денис Яшный, Картина Мира, Валентин Германович Дупак, Classic Starr, Olena Martynchuk, Zakhar Popovych, Polina Vlasenko, Vasya Opechenik та Денис Потапов.



\cleardoublepage
  
% ЧОГО vtu ХОЧЕМО? 
% Вперше без підпису надруковано польськаю мовою в газеті «Ргаса», 1879, 18 серпня, під назвою «Czego my chcemy?». В перекладі українською мовою вперше надруковано у вид.: Франко І. Твор и. В 20-ти т., т, 19, с. 215-217. ПодаЕться за першодруком. 
% ВЛАСНІСТЬ ГРУНТОВА І Уі ІСТОРІЯ 
% Вперше надруковано в кн.: Лавле Е. де. Власність грунтова і її історія. Переклав Іван Франко. Львів, 1879 («Дрібна бібліотека», VI), с. 34. (Передмова до книги). ПодаЕться за першодруком. С. 28. лавеле  Еміль де (1822-1892) — бельгійський бур-жуаsний iсторик i •    економlст. Б ю к е р Карл (1847-1930) — німецький буржуазний еконо• міст, історик народного господарства і статистик. 
% ДОПОВНЕННЯ ДО «ОСНОВ СУСПІЛЬНОУ 
% ЕкономІт" 
% Вперше надруковано в журн. «Культура», 1926, Ns 4-9, с. 56-57, та в кн.: Іван Франко, К., 1926, с. 164-166. Це герша передмова до перекладу XXIV розділу «Капіталу» К. Маркса, доданого І. Франком до написаного ним підручника «Ос• нови суспільної економії», який мав вийти у світ в кінці 1879 — на початку 1880 р., але не був надрукований; рукопис його загуб- лено. ПодаЕться за автографом, який зберігся в архіві І. Франка ((р. 3, Nё 448). 
% ГДРУГА ЛЕРЕДМОВА ДО ПЕРЕКЛАДУ 
% 24-го РОЗДІЛУ ПРАцІ К. МАРКСА «КАП[ТАЛв. т. 1] 
% Вперше надруковано в ж,урн. «Культура», 1926, Ns 4-9, с. 57-58, та в кн.: Іван Франко. К., 1926, с. 167---168. Передмову до українського перекладу 24-го розділу «Капі- талу» К. Маркса, який І. Франко мав намір видати окремим ви- пуском «Дрібної бібліотеки» (див. коментар до перекладу в цьому томі)подаеться написано, ймовірно, на початку 1880 року.  за автографом, який э6ерігаЕться в архіві І.Франка, ф. 3, Ns 448. С. 32. ...щоби сама гграця стала товаром...- Тут франківський виклад змісту першого тому «Капіталу» К. Марк- са неточний. К. Маркс мав на уваэі не працю, а робоцу силу. С. 33. Текст перекладу подаЕться у розділі «3 наукових пере- кладів» (с. 581--609). 
% 616 



\section*{Доповненя до „Основ суспільної економії“\protect\footnotemarkZ{}}
\nonumsectioncft{Доповненя до „Основ суспільної економії“}{.~}{Іван Франко}

\footnotetextZ{Вперше надруковано в журн. «Культура», 1926, № 4--9, с. 56--57, та в кн.: Іван Франко, К., 1926, с.~164--166.

Подається за автографом: відділ рукописних фондів і текстології Інституту літератури ім. Т.~Г.~Шевченка НАН України. — Ф. 3. — Од. зб. 448. — 14 арк. 
}

\noindent{}В самім початку „Основ суспільної економії“ сказано було, що економія, се наука абстрактна, т. є. що ціль єї не є виключно — розслідити закони економічні \emph{теперішної} суспільности, але \emph{загальні} закони праці людської. А позаяк с переміною суспільного ладу в протягу віків і закони ті проявляются щораз то в інших формах, випливаючих конечно з даного ладу, то наука економічна не може ніякої с тих форм вважати сталою і незмінною. Не може, значит, і нинішних форм уважати сталими, а мусит шукати таких форм, котрі \emph{після нашого теперішного знаня} булиб відповіднійші для суспільної праці і суспільного добробутку, ніж нинішні форми.

С тої то причини в сістематичнім викладі основ сусп. економії ми не могли давати надто широкого місця вислідам про \emph{нинішний} лад, а ограничились тілько головним єго нарисом. При викладі абстрактної теорії праці се була конечна річ, — але прецінь ніхто не заперечит, що на практиці для кождого дуже важне — знати передовсім докладно теперішний лад, єго почин і розвиток. Таке знанє вже тим корисне, що замісць теоретичних засад подає масу фактів, котрі самі прут розум до таких а таких виводів, між тим коли ті самі виводи, подані без підставних фактів, усякому можут видатися хиткими та схопленими з воздуха мріями. Для того то думаєм ми, що поповнимо подекуди конечний недостаток теоретичного викладу, подаючи в „Доповнених“ обширнійший огляд деяких питань, не порушених або з боку ткнених в самім викладі.

Одна з найважнійших недостач усякого чисто теоретичного викладу та, що приходится виключати з него всякі ширші \emph{історичні} перегляди. Правда, се не є недостача конечна, бо остаточно мож би бути вірним теорії, подаючи перегляд розвитку та впадку всіх економічних порядків від почину цівілізації аж до тепер. Але не кажучи вже о тім, що для такої загальної історії економічного розвитку призбирано доси дуже ще мало матеріялу, — в нашім підручнику такий виклад був би неможливий вже й за недостачею місця. А говорити обширно про розвиток одного — ниніншого — ладу, не казавши нічо про розвиток їнчих, се значилоб вважати сей лад чимось важнійшим від прочих, між тим коли в історії, як і в зрості кождого орґанізму, кожда фаза розвитку для вислідника рівноважна.

Але вважаючи потрібним познайомити наших читателів з історичним розвитком сучасного, капіталістичного ладу, ми робимо се в „Доповненях“. А для своєї ціли ми не можем найти кращого провідника над Карля Маркса, котрий в однім розділі своєї книжки „Das Kapital“ списав короткий, хоть яркий перегляд того, як розвивалася капіталістична продукція. С тим розділом ми й хочемо познакомити наших читателів.


\section*{[Друга передмова до перекладу 24-го розділу праці К.~Маркса «Капітал», т. І]\protect\footnotemarkZ{}}
\nonumsectioncft{[Друга передмова до перекладу 24-го розділу праці К.~Маркса «Капітал», т. І]}{.~}{Іван Франко}

\footnotetextZ{Вперше надруковано в журн. «Культура», 1926, № 4--9, с. 57--58, та в кн.: Іван Франко, К., 1926, с.~167--168.

Подається за автографом: відділ рукописних фондів і текстології Інституту літератури ім. Т.~Г.~Шевченка НАН України. — Ф. 3. — Од. зб. 448. — 14 арк. 
}

\noindent{}В першій части своєї великої економічної праці про „Капітал“ стараєсь Карль Маркс вияснити передовсім, \emph{як повстає капітал}? В тій ціли виказує він поперед усего, що єдиним жерелом усякої вартости є праця людська, котра з матеріалів сирих, даних природою, і при помочи сил природи витворює предмети вжиточні для чоловіка. Коли предмети такі витворюются не для власного вжитку самого витвірця, а для заміни за їнші, тоді вони звутся товарами. Капіталістична продукція полягає на витворюваню товарів, але не всяка продукція, де витворюются товарі, є вже капіталістична. До того потрібно ще одної дуже важної вимінки: \emph{щоби сама праця стала товаром}, т. є. щоб на торзі за певний товар (гроші) мож було заміняти (купити) працю людську.

Звичайно під назвою капіталу у нас розуміются беззглядно гроші. Се по части хибно. Гроші, як бачимо, тоді тілько стают капіталом, коли за них купуєся на торзі робуча сила.

Але праця людська, се не є звичайний товар. Се товар живий, котрий має тоту властивість, що \emph{надає вартість} другим предметам, і надає єї більше, ніж кілько сам коштує. Торгова ціна праці, так як і ціна кождого товару, означена звичайними економічними правилами, с котрих найважнійше — кошт витвореня товару, т. є. в тім разі — кошт удержаня робітника і єго робучої сили. Таку ціну платит капіталіст робітникови за єго працю. Між тим робітник в тім часі, на котрий нанявся, витворює далеко більше, ніж кілько виносит єго плата. Він витворив \emph{надзвишку вартости} понад вартість своєї плати, — тота надзвишка, се зиск капіталіста, — вона побільшує єго капітал. Значит, уся капіталістична продукція полягає на твореню надзвишки, котра задармо дістаєсь капіталістови. Цілий розвиток економічний капіталістичної продукції полягає на тім, що капіталісти всіми силами старалися до крайної можности вбільшити тоту надвишку. Вбільшити єї мож було двома способами: або продовжуючи день робучий (надвишка абсолютна), або приневолюючи робітників в коротшім часі працювати з більшою натугою (релятівна надвишка). Оба ті способи витрібували капіталісти, і то перший з них (продовженє робучого дня) до такої крайности, що аж уряд, затрівожений робітницькими розрухами, мусів вдатися в те діло і ограничити стало довготу робучого дня. Від тоді капіталістична продукція і доси пре в другий бік, — стараєсь той означений правно день робучий як найдоскональше використати, раз-ураз заводячи нові машини, котрі до крайности упрощуют і прискорюют продукцію, а до обслуги вимагают як найменшого числа рук.

Се головні думки, виведені Марксом з безмірної маси фактів, нагромаджених в єго книжці. При кінци книжки розбирає він ще одно важне питане: Яким способом почалася тота капіталістична продукція? Як і на якім ґрунті та при якій управі виріс той дивний порядок, оснований на щоденнім хитрім визиськуваню, на крайній бідносте незлічимих мас народа, а крайнім богацтві немногих щасливців? Сесь важний розділ Марксової книжки — прекрасний культурно-історичний очерк — зрозумілий буде і окремо від цілої книжки і ми хочемо познакомити з ним нашу громаду, як для самої єго великої стійности наукової, так і для того, щоб заохотити всіх, хто тілько владає німецькою мовою, до читаня цілої Марксової книжки. Звичайно говорится про дуже трудний і незрозумілий спосіб писаня у Маркса. Се мож би сказати хіба про перший розділ єго книжки, — а о?кілько такий суд справедливий що до прочих розділів, най посвідчит тота часть, котра отсе переведена.
  \pagenumbering{arabic}
  \mainmatter
  \thispagestyle{empty}
\null\vspace{6cm}


\noindent\htitlespace{\Large Присвячується моєму незабутньому другові}

\noindent\htitlespace{\Large Сміливому, вірному, благородному,}

\noindent\htitlespace{\Large Передовому борцеві пролетаріяту}
\vspace{0.5cm}

\noindent\htitlespace{\Huge Вільгельмові Вольфу}

\noindent\htitlespace{\Large Народився в Тарнау 21 червня 1809 року}

\noindent\htitlespace{\Large Помер на вигнанні в Менчестері 9 травня 1864 року}

\cleardoublepage
  \index{iii1}{0009}  %% посилання на сторінку оригінального видання
\section*{Передмова}
\nonumsection{Передмова}{.~}{Фрідріх Енґельс}

Нарешті мені удалось опублікувати цю третю книгу головного
твору Маркса, закінчення теоретичної частини. При виданні
другої книги в 1885 році я гадав, що третя книга становитиме
хіба тільки технічні труднощі, за винятком, звичайно, деяких
дуже важливих відділів. Так воно й було в дійсності; але про ті
труднощі, які мене чекали саме в цих найважливіших відділах
цілого, я не мав тоді ніякого уявлення, так само як і про інші
перешкоди, які так дуже загаяли виготовлення книги.

Перш за все і більш за все мені заважала постійна слабість
зору, яка протягом багатьох років обмежувала до мінімуму мій
робочий час для письмових занять, та ще й тепер тільки винятками
дозволяє мені брати в руки перо при штучному освітленні.
До цього долучилися інші невідкладні роботи: нові видання й
переклади попередніх праць Маркса та моїх, отже, перегляди,
передмови, доповнення, часто неможливі без нових досліджень,
і~\abbr{т. д.} Передусім англійське видання першої книги, за текст
якого кінець-кінцем відповідаю я і яке через це забрало в мене
багато часу. Хто скількинебудь стежив за колосальним ростом
інтернаціональної соціалістичної літератури протягом останніх
десяти років і особливо за числом перекладів раніших праць
Маркса та моїх, той визнає, що я мав підставу вітати себе
з тим, що число тих мов, де я міг бути корисним перекладачеві
і, отже, був зобов’язаний не відмовлятись від перегляду його
праці, дуже обмежене. Але ріст літератури був тільки симптомом
відповідного зростання самого інтернаціонального робітничого
руху. А це останнє накладало на мене нові обов’язки.
З перших днів нашої громадської діяльності чимала частина праці
щодо посередництва між національними рухами соціалістів і робітників
у різних країнах падала на мене і Маркса; ця праця зростала
відповідно до зміцнення всього руху. Але тимчасом як
до самої своєї смерті Маркс і в цьому головний тягар праці
брав на себе, після його смерті постійно наростаючу працю довелося
виконувати мені одному. Між тим, безпосередні зносини
окремих національних робітничих партій між собою стали загальним
правилом, і на щастя, з дня на день стають ним дедалі більше;
не зважаючи на це, моя допомога потрібна ще далеко частіше,
ніж мені хотілося б в інтересах моїх теоретичних праць. Але
\parbreak{}  %% абзац продовжується на наступній сторінці

\parcont{}  %% абзац починається на попередній сторінці
\index{iii1}{0010}  %% посилання на сторінку оригінального видання
хто, як я, понад 50 років діяв у цьому русі, для того роботи,
що випливають з цього руху, є неминучим обов’язком, який
вимагає негайного виконання. Як у шістнадцятому столітті, в
наші бурхливі часи чисті теоретики у сфері суспільних інтересів
збереглися тільки на стороні реакції, і саме тому ці панове
в дійсності зовсім не теоретики, а прості апологети цієї реакції.

Та обставина, що я живу в Лондоні, веде за собою те, що
ці партійні зносини зимою відбуваються здебільшого листовно,
а влітку здебільшого особисто. А через це, як і через необхідність
стежити за ходом руху в країнах, число яких раз-у-раз
зростає, і по органах преси, число яких зростає ще швидше, для
мене стало неможливим виготовляти праці, що не допускають
ніякої перерви, інакше, як зимою, переважно в перші три місяці
року. Коли маєш за собою сімдесят років, тоді Мейнертові асоціативні
волокна мозку працюють з якоюсь фатальною повільністю;
перерви у важкій теоретичній праці перемагаєш уже не так
легко і не так швидко, як раніш. Тим то виходило, що працю
однієї зими, коли вона не була цілком доведена до кінця, доводилось
наступної зими здебільшого проробляти знову, і це сталося
саме з найважчим п’ятим відділом.

\looseness=1
Як побачить читач з подальших даних, ця редакційна робота
істотно відрізнялася від редакційної роботи над другою книгою.
Для третьої книги був у наявності тільки один первісний нарис,
до того ще й з величезними прогалинами. Звичайно, початок
кожного окремого відділу був досить пильно розроблений,
навіть здебільшого й стилістично округлений. Але чим далі від
початку, тим більш ескізним ставало оброблення, тим більше мало
воно прогалин, тим більше містило воно екскурсів з приводу побічних
питань, які виникали в процесі дослідження, при чому розроблення
головного питання залишалося до пізнішого часу, тим
довшими і заплутанішими ставали періоди, в яких висловлювалися
думки, записані in statu nascendi [у стані виникнення].
В багатьох місцях почерк і виклад надто виразно показують
втручання та ступневий розвиток тих викликаних надмірною
працею приступів хвороби, які спочатку все більше утрудняли
авторові самостійну працю і, нарешті, тимчасово зовсім унеможливлювали
її. І не дивно. Між 1863 і 1867~\abbr{рр.} Маркс не тільки
виготовив у нарисі дві останні книги „Капіталу“ і виготовив до
друку рукопис першої книги, але й виконав ще велетенську
роботу, зв’язану з заснуванням і поширенням Інтернаціональної
Асоціації Робітників. Однак, через це вже в 1864 і 1865~\abbr{рр.}
виявились серйозні ознаки тих порушень в здоров’ї Маркса, що
не дали йому змоги самому закінчити обробку II і III книги.

\looseness=1
Моя робота почалася з того, що, продиктувавши весь рукопис
з оригіналу, який часто навіть я міг розшифрувати тільки з труднощами,
я зробив легку до читання копію, що забрало в мене
чимало часу. Тільки тоді можна було почати власне редакцію.
Я обмежив її найнеобхіднішим, по можливості зберіг характер
\parbreak{}  %% абзац продовжується на наступній сторінці

\input{iii.1/_0011.tex}

\index{iii1}{0012}  %% посилання на сторінку оригінального видання
Головну трудність становив відділ V, в якому до того ж
розглядається найзаплутаніший предмет всієї книги. І саме тоді,
коли Маркс розробляв цей відділ, його спіткав один з вищезгаданих
тяжких приступів хвороби. Отже, тут ми маємо не готовий
нарис, навіть не схему, обриси якої треба було б заповнити,
а тільки початок оброблення, який у багатьох випадках зводиться
до невпорядкованої купи заміток, уваг, матеріалів у формі
витягів. Спочатку я пробував закінчити цей відділ, як це мені
до певної міри вдалося з першим відділом, шляхом заповнення
прогалин і розроблення уривків, що були тільки намічені, так щоб
він хоч би приблизно давав усе те, що намірявся дати автор. Щонайменше
тричі я робив таку спробу, але кожного разу зазнавав
невдачі, і в утраченому таким чином часі є одна з головних причин
запізнення. Нарешті, я переконався, що цим шляхом справа
не піде. Мені довелося б переглянути всю масу літератури з цієї
галузі, і, кінець-кінцем, я виготував би щось, що все ж не було б
книгою Маркса. Мені не лишалось нічого іншого, як в певному
розумінні розрубати Гордіїв вузол, обмежитись тим, щоб по
можливості впорядкувати наявне і зробити тільки найпотрібніші
доповнення. І таким чином я навесні 1893 року закінчив головну
роботу для цього відділу.

З окремих розділів розділи 21--24 були в найголовнішому
розроблені. Розділи 25 і 26 вимагали перегляду ілюстрацій і
включення матеріалу, який був в інших місцях. Розділи 27 і 29 можна
було дати майже цілком за рукописом, навпаки — розділ 28 довелося
місцями інакше згрупувати. Але справжня трудність почалася
з розділу 30. Починаючи звідси, треба було належним чином
впорядкувати не тільки ілюстративний матеріал, але й хід думок,
який кожної хвилини переривався вставними реченнями, відхиленнями
і~\abbr{т. д.} і розвивався далі в іншому місці, часто цілком мимохідь.
Таким чином розділ 30 склався шляхом перестановок та вилучень,
для яких знаходився вжиток в іншому місці. Розділ 31 був знову
більше розроблений у загальному зв’язку. Але далі в рукопису
йде довгий відділ, названий „Плутанина“ („Die Konfusion“), що
складається виключно з витягів з парламентських звітів про кризи
1848 і 1857~\abbr{рр.}, в яких згрупованої місцями коротко юмористично
коментовані судження двадцяти трьох дільців і письменниківекономістів,
а саме про гроші й капітал, про відплив золота, надмірну
спекуляцію і~\abbr{т. д.} Тут представлені, чи тими, що запитують,
чи тими, що відповідають, майже всі ходячі погляди того
часу на відношення між грішми і капіталом, і Маркс хотів критично
й сатирично розглянути „плутанину“, яка виявилась при цьому,
щодо того, що є на грошовому ринку гроші і що є капітал.
Після багатьох спроб я переконався, що виготовлення цього
розділу неможливе; матеріал, особливо матеріал, коментований
Марксом, я використав там, де це дозволяв зв’язок викладу.

Після цього йде в досить упорядкованому вигляді те, що я
вмістив в 32 розділі, але безпосередньо за цим — нова купа витягів
\index{iii1}{0013}  %% посилання на сторінку оригінального видання
з парламентських звітів про всякі можливі речі, зачеплені
в цьому відділі, перемішана з довшими чи коротшими увагами
автора. Наприкінці витяги та коментарії все більше й більше
концентруються коло руху грошових металів та вексельного
курсу і знов закінчуються всякого роду додатками. Навпаки, розділ
(36) „Передкапіталістичні відносини“ був цілком оброблений.

З усього цього матеріалу, починаючи з „Плутанини“, і оскільки
його не було вже вміщено в попередніх місцях, я склав
розділи 33--35. Звичайно, тут не обійшлося без значних вставок
з мого боку для встановлення зв’язку. Оскільки ці вставки
не чисто формального характеру, вони прямо позначені як мої.
Таким способом мені, нарешті, вдалося умістити в тексті всі так
чи інакше належні до справи судження автора; нічого не випущено,
крім незначної частини витягів, які або тільки повторювали
наведене вже в іншому місці, абож торкалися пунктів, яких
рукопис докладно не розглядав.

Відділ про земельну ренту був далеко повніше оброблений,
хоч і зовсім не впорядкований, як це видно вже з того, що
в 43 розділі (в рукопису кінець відділу про ренту) Маркс
вважав за потрібне коротко повторити план всього відділу. І це
було тим більш бажаним для видання, що рукопис починається
розділом 37, після якого йдуть розділи 45--47, і тільки після
цього розділи 38--44. Найбільше праці потребували таблиці
при диференціальній ренті II і те відкриття, що в 43 розділі
зовсім не був досліджений третій випадок цього роду ренти,
який треба було тут розглянути.

Для цього відділу про земельну ренту Маркс у семидесятих
роках взявся до цілком нових спеціальних досліджень. Протягом
ряду років він вивчав в оригіналах статистичні досліди
та інші видання про землеволодіння, які стали неминучими
в Росії після „реформи“ 1861 року і які йому постачали в бажаній
повноті його російські друзі, робив з них виписки і намірявся
їх використати при новому обробленні цього відділу. При
різноманітності форм як землеволодіння, так і експлуатації
землеробських виробників у Росії, у відділі про земельну ренту
Росія мала відігравати таку саму роль, як в першій книзі, при
розгляді промислової найманої праці, Англія. На жаль, Марксу
не вдалося здійснити цей план.

Нарешті, сьомий відділ був цілком закінчений у рукопису,
але тільки як перший нарис, безконечно заплутані періоди якого
спочатку треба було розчленувати, щоб зробити їх придатними
до друку. Від останнього розділу існує тільки початок. Тут
малося розглянути відповідні трьом головним формам доходу:
земельна рента, зиск, заробітна плата — три великі класи розвиненого
капіталістичного суспільства: землевласники, капіталісти,
наймані робітники — і неминуче дану з їх існуванням класову
боротьбу як фактично наявний результат капіталістичного
періоду. Подібні кінцеві резюме Маркс мав звичай відкладати
\parbreak{}  %% абзац продовжується на наступній сторінці

\input{iii.1/_0014c.tex}
\parcont{}  %% абзац починається на попередній сторінці
\index{iii2}{0015}  %% посилання на сторінку оригінального видання
і ми маємо підставу гадати, що в другого було 8--10 мільйонів; один мав 4,
другий 3\sfrac{1}{2}, третій більше за 8. Я кажу про вклади у brokers’ів». (Report,
of Committee on Rank Akts, 1857--58 p., 5, № 8).

«Лондонські billbrokers’и\dots{} провадили свої величезні операції без всякого
запасу готівкою; вони покладалися на одержування грошей від тих своїх векселів,
що їм раз-у-раз надходив реченець платежу, або ж в гіршому разі — на змогу
одержувати позики в Англійському банку, депонуючи в нього оті дисконтовані
ними векселі». — Дві фірми bill-brokers’ів у Лондоні припинили платежі в
1847 році; обидві вони пізніше відновили операції. В 1857 році вони знову
припинили платежі. Пасиви однієї фірми в 1847 році становили кругло
\num{2.683.000}\pound{ ф. ст.} при капіталі з \num{180.000}\pound{ ф. ст.}; її пасиви в 1857 році були \deq{}
\num{5.300.000}\pound{ ф. ст.}, тимчасом коли капітал становив, імовірно, не більше як чверть
того, що було в неї в 1847 році. Пасиви другої фірми були в обох випадках
в межах 3--4 мільйонів при капіталі не більшому, ніж \num{45.000}\pound{ ф. ст.}» (ibidem,
p. XXI, № 52).

\section{Грошовий капітал та дійсний капітал. I}

Єдино тяжкі питання, що до них наближаємось ми тепер у справі кредиту,
такі:

\emph{Поперше}. Нагромадження власно грошового капіталу. Оскільки воно
є й оскільки воно не є ознака дійсного нагромадження капіталу, тобто репродукції
у поширеному маштабі? Чи так звана plethora\footnote*{
Грецьке слово, що йому найближче відповідає — укр. повнява, або багатість. \Red{Пр.~Ред.}
} капіталу, вислів, що його
завжди уживається тільки про капітал, який дає процент, тобто про грошовий
капітал, — є лише осібний спосіб виражати промислову надмірну продукцію,
чи являє він осібне явище поряд неї? Чи ця plethora, це надмірне постачання
грошового капіталу збігається з наявністю грошових мас (зливків,
золотих монет та банкнот) у стані застою, так що цей надмір дійсних грошей
є вислів і форма вияву тієї plethor’и позичкового капіталу?

І, \emph{подруге}. Оскільки скрута на гроші, тобто недостача позичкового капіталу,
означає недостачу дійсного капіталу (товарового капіталу та продуктивного
капіталу)? Оскільки, з другого боку, вона, та скрута, збігається з недостачею
грошей як таких, з недостачею засобів циркуляції?

Оскільки ми досі розглядали своєрідну форму нагромадження грошового
капіталу та грошового майна взагалі, вона сходила на нагромадження вимог
власности на працю. Нагромадження капіталу державного боргу означає, як уже
виявилося, не що інше, а тільки збільшення кляси кредиторів держави, що мають
право брати собі наперед певні суми з податків\footnote{
«Державні фонди — не що інше, як уявлюваний капітал, що представмо частину річного доходу,
призначену для оплати боргу. Капітал, що є тій частині еквівалентний, вже витрачено; він визначив
суму позики, але не його представляють державні фонди; бо капіталу того вже більше немає. Тимчасом
нові багатства повинна утворити промислова праця; певну річну частину цих багатств наперед
призначається
тим, що визичили капітал, тепер уже знищений; цю частину багатств податками відбирають
від продуцентів їх, щоб віддати кредиторам держави, і відповідно до звичайного в даній країні
відношення
між капіталом та процентом припускається, що є уявлюваний капітал, рівновеликий тому, що міг би
дати таку річну ренту, яку мають одержувати кредитори». (Sismondi, Nouveaux Principes II, p. 230).
}. В тому факті, що навіть нагромадження
боргів може здаватися нагромадженням капіталу, виявляється довершення
того перекручування, що відбувається в кредитовій системі. Ці боргові
посвідки, видані за первісно позичений та вже давно витрачений капітал, ці
паперові дублікати знищеного капіталу функціонують для своїх державців як
капітал остільки, оскільки вони є товари, що їх можна продати, отже й оскільки
їх можна перетворити знову на капітал.

\parcont{}  %% абзац починається на попередній сторінці
\index{iii1}{0016}  %% посилання на сторінку оригінального видання
перенесення певної частини вартості сукупного продукту до
класу\linebreak[4]
капіталістів“.

Отож не треба великого напруження мислі, щоб переконатися,
що це „вульґарноекономічне“ пояснення зиску на капітал
практично веде до тих самих результатів, як і теорія додаткової
вартості Маркса; що, за уявленням Лексіса, робітники
перебувають точно в такому самому „несприятливому становищі“,
як і в Маркса; що вони цілком так само обдурені, бо
кожен неробітник може продавати вище ціни, а робітник цього
не може; і що на основі цієї теорії можна збудувати принаймні
настільки ж поверховий вульґарний соціалізм, як той, що збудований
тут в Англії на основі теорії споживної вартості та
теорії граничної корисності Джевонса-Менгера. Я навіть думаю,
що коли б ця теорія зиску була відома панові Джорджеві Бернардові
Шоу, він міг би ухопитися за неї обома руками, дати
відставку Джевонсові та Карлові Менгеру і наново збудувати
на цій скелі фабіанську церкву майбутнього.

Але в дійсності ця теорія є лише парафраза теорії Маркса.
З чого ж покриваються всі надбавки до ціни? З „сукупного
продукту“ робітників. І саме в наслідок того, що товар „праця“,
або, як каже Маркс, робоча сила, мусить продаватися нижче її
ціни. Бо якщо спільна властивість усіх товарів є в тому, що їх
можна продавати дорожче витрат виробництва, а праця становить
єдиний виняток з цього і продається завжди тільки по витратах
виробництва, то вона продається якраз нижче тієї ціни, яка є
загальним правилом у цьому вульґарноекономічному світі. Надзиск,
який в наслідок цього припадає капіталістові або класові
капіталістів, полягає саме в тому і в кінцевому рахунку може
постати тільки тому, що робітник, після репродукції заміщення
ціни своєї праці, мусить ще далі виробляти продукт, за який
йому не платять, — додатковий продукт, продукт неоплаченої
праці, додаткову вартість. Лексіс — людина надзвичайно обережна
у виборі своїх висловів. Він ніде не каже прямо, що
вищенаведене розуміння є його власне; але якщо це так, то
цілком ясно, що ми тут маємо справу не з одним з тих звичайних
вульґарних економістів, про яких він сам каже, що кожний
з них в очах Маркса є „в кращому разі тільки безнадійний недоумок“,
а з марксистом, який переодягнувся вульґарним економістом.
Чи сталося це переодягнення свідомо чи несвідомо, це
є психологічне питання, яке нас тут не цікавить. Той, хто схотів
би з’ясувати це, може, дослідив би також, як могло статися, що
така безперечно розумна людина, як Лексіс, могла певний час
боронити таке безглуздя, як біметалізм.

Перший, хто дійсно намагався дати відповідь на питання, був
д-р \emph{Конрад Шмідт}: „Die Durchschnittsprofitrate auf Grundlage des
Marxschen Wertgesetzes“, Stuttgart, Dietz 1889. Шмідт намагається
погодити деталі утворення ринкових цін як із законом
вартості, так і з пересічною нормою зиску. Промисловий капіталіст
\index{iii1}{0017}  %% посилання на сторінку оригінального видання
одержує у своєму продукті, поперше, заміщення свого
авансованого капіталу, подруге, додатковий продукт, за який
він нічого не заплатив. Але, щоб одержати цей додатковий
продукт, він мусить авансувати свій капітал на виробництво;
тобто він мусить застосувати певну кількість упредметненої
праці, щоб мати можливість привласнити собі цей додатковий продукт.
Отже, для капіталіста цей його авансований капітал є кількість
упредметненої праці, суспільно-потрібна для того, щоб
створити йому цей додатковий продукт. Це має силу і для всякого
іншого промислового капіталіста. А тому що за законом
вартості продукти обмінюються один на один пропорціонально
до праці, суспільно-необхідної для їх виробництва, і що для капіталіста
праця, необхідна для виготовлення його додаткового продукту,
є якраз нагромаджена в його капіталі минула праця, то
з цього випливає, що додаткові продукти обмінюються пропорціонально
до капіталів, потрібних для їх виробництва, а не пропорціонально
до \emph{дійсно} втіленої в них праці. Отже, частка, що
припадає на кожну одиницю капіталу, дорівнює сумі всіх вироблених
додаткових вартостей, поділеній на суму застосованих
для цього капіталів. Тому однакові капітали за однакові проміжки
часу дають однаковий зиск, і це спричинюється тим, що
вираховані так витрати виробництва (Kostpreis) додаткового
продукту, тобто пересічний зиск, додаються до витрат виробництва
оплаченого продукту, і по цій підвищеній ціні продаються
обидва, оплачений і неоплачений продукт. Встановлюється
пересічна норма зиску, не зважаючи на те, що, як
думає Шмідт, пересічні ціни окремих товарів визначаються за
законом\linebreak[4]
вартості.

Конструкція надзвичайно дотепна, вона цілком на гегелівський
зразок, але вона має те спільне з більшою частиною гегелівського,
що вона неправильна. Додатковий продукт чи оплачений
продукт — це не робить ріжниці: якщо закон вартості повинен
\emph{безпосередньо} мати силу і для пересічних цін, то і той і другий
продукт мусить продаватися пропорціонально до суспільнонеобхідної
праці, потрібної і спожитої на їх виготовлення. Закон
вартості з самого початку спрямований проти погляду, який перейшов
від капіталістичного способу уявлення, ніби нагромаджена
минула праця, з якої складається капітал, є не просто певна сума
готової вартості, а, як фактор виробництва й утворення зиску, має
також властивість створювати вартості, отже, є джерелом більшої
вартості, ніж має сам капітал; закон вартості твердо встановлює,
що ця властивість належить тільки живій праці. Те, що капіталісти
сподіваються рівного зиску, пропорціонального до величини
їх капіталів, отже, розглядають авансовані ними капітали як свого
роду витрати виробництва їхнього зиску — це відомо. Але якщо
Шмідт використовує це уявлення, щоб за його допомогою погодити
з законом вартості ціни, обраховані за пересічною нормою
зиску, то він скасовує (hebt\dots{} auf) самий закон вартості, приєднуючи
\index{iii1}{0018}  %% посилання на сторінку оригінального видання
до нього як співвизначальний фактор уявлення, цілком
йому суперечне.

Або нагромаджена праця поряд з живою працею створює
вартість. Тоді закон вартості не має сили.

Або вона не створює вартості. Тоді доводи Шмідта несполучні
з законом вартості.

Шмідт збився з правильного шляху, коли він був уже дуже
близько до розв’язання проблеми, бо гадав, що треба обов’язково
знайти математичну формулу, яка дала б можливість довести
погодженість пересічної ціни кожного окремого товару з законом
вартості. Але якщо тут, бувши зовсім близько до мети, він
пішов хибним шляхом, то в усьому іншому зміст брошури показує,
з яким розумінням він зробив дальші висновки з обох перших
книг „Капіталу“. Йому належить честь самостійного відкриття
правильного пояснення непояснимої до того часу тенденції норми
зиску до зниження, пояснення, даного Марксом у третьому відділі
третьої книги; так само виведення торговельного зиску з
промислової додаткової вартості і цілий ряд уваг про процент
та земельну ренту, в яких ним передхоплені речі, розвинені у
Маркса в четвертому і п’ятому відділах третьої книги.

В одній пізнішій праці („Neue Zeit“ 1892/93, № 3 і 4)
Шмідт намагається розв’язати проблему іншим шляхом. Цей
шлях зводиться до того, що пересічну норму зиску встановлює
конкуренція, бо вона примушує капітал переходити з галузей
виробництва з недостатнім зиском до інших, де добувається
надзиск. Що конкуренція є велика зрівняльниця зисків, це не
новина. Але Шмідт намагається довести, що це нівелювання
зисків тотожне із зведенням продажної ціни товарів, вироблених
понад міру, до такої міри вартості, яку суспільство може
заплатити за них згідно з законом вартості. Чому і це не
могло привести до цілі, досить видно з пояснень Маркса в самій
книзі.

Після Шмідта до проблеми взявся \emph{П.~Фіреман} („Conrads
Jahrbücher“, Dritte Folge [1892], III, стор. 793). Я не спиняюся на
його увагах про інші сторони викладу в Маркса. Вони ґрунтуються
на тому непорозумінні, ніби Маркс хоче дати визначення
там, де він в дійсності розвиває, і на тому, що в Маркса взагалі
довелося б пошукати точних, готових, раз і назавжди даних
дефініцій. Адже само собою зрозуміло, що там, де речі та їх
взаємовідношення розглядаються не як сталі, а як мінливі, їх
мислені відбитки, поняття, теж зазнають зміни та перетворення;
що їх не втискують у закам’янілі дефініції, а розглядають в їх
історичному або логічному процесі утворення. Після цього стане,
звичайно, ясно, чому Маркс на початку першої книги, де він
виходить з простого товарного виробництва, яке є для нього
історичною передумовою, щоб потім далі перейти від цієї бази
до капіталу, — чому він там виходить саме з простого товару,
а не з форми, логічно і історично вторинної, не з капіталістично
\parbreak{}  %% абзац продовжується на наступній сторінці

\parcont{}  %% абзац починається на попередній сторінці
\index{iii2}{0019}  %% посилання на сторінку оригінального видання
не оплачував грішми бавовни, фабрикант ситцю — пряжі, купець — ситцю і~\abbr{т. ін.}
В перших актах процесу товар, бавовна, проходить свої різні фази продукції, і цей
перехід відбувається за посередництвом кредиту. Але, скоро бавовна одержала
вже в продукції свою останню форму як товар, цей самий товаровий капітал
проходить ще лише через руки різних купців, що упосереднюють транспорт до
далекого ринку; останній з них, кінець-кінцем, продає його споживачеві,
купуючи натомість інший товар, що ввіходить або в споживання або в процес
репродукції. Отже, тут слід відрізняти два періоди: протягом першого періоду
кредит упосереднює дійсні послідовні фази продукції того самого товару; протягом
другого він упосереднює лише перехід з рук одного купця до рук іншого, перехід,
що включає й транспорт, отже, акт $Т — Г$. Але й тут товар перебуває принаймні
завжди в акті циркуляції, отже фазі процесу репродукції.

Отже, те, що тут визичається, ніяк не є капітал незайнятий, а навпаки це
капітал, що в руках свого державця мусить змінити свою форму, капітал, що
існує в такій формі, в якій він для нього є простий товаровий капітал, тобто
капітал, що мусить відбути зворотне перетворення, а саме, принаймні насамперед,
перетворитись на гроші. Отже, це та метаморфоза товару, що її тут упосереднює
кредит; не тільки $Т — Г$, але й $Г — Т$ та дійсний процес продукції.
Багато кредиту в межах репродуктивного кругообороту — якщо не вважати на банкірський
кредит — не означає: багато незайнятого капіталу, що його пропонується
для позик та що шукає прибуткового приміщення, але означає: велику
зайнятість капіталу в процесі репродукції. Отже, кредит упосереднює тут,
1) оскільки вважати на промислових капіталістів, — перехід промислового капіталу
з однієї фази до другої, зв’язок сфер продукції, що одна до однієї приналежні
та одна до однієї втручаються; 2) оскільки вважати на купців, —
транспорт та перехід товарів з рук до рук аж до остаточного продажу їх за
гроші або обмін їх за якийсь інший товар.

Максимум кредиту дорівнює тут найповнішій зайнятості промислового капіталу,
тобто найвищому напруженню його репродукційної сили, не зважаючи на межі
споживання. Ці межі споживання поширюються напруженням самого процесу
репродукції; з одного боку, те напруження збільшує споживання доходу робітниками
та капіталістами, з другого боку, воно є тотожнє з напруженням продуктивного
споживання.

Поки процес репродукції тече реґулярно, а тому й зворотний приплив капіталу
лишається забезпеченим, цей кредит тримається твердо та поширюється, і пошир
його базується на поширі самого процесу репродукції. Скоро настає застій
в наслідок затриманого зворотного припливу капіталу, переповнення ринків, спаду
цін, то виявляється надмір промислового капіталу, але в такій формі, в якій він
не може виконувати своїх функцій. Маса товарового капіталу, але не сила
його продати. Маса основного капіталу, але з причин застою репродукції той
капітал здебільша незайнятий. Кредит скорочується, 1) бо цей капітал незайнятий,
тобто припинив свій рух на одній з фаз своєї репродукції, бо він не може
виконати своєї метаморфози, 2) бо зламано віру у поточність процесу репродукції;
3) бо попит на цей комерційний кредит меншає. Прядунові, що обмежує
свою продукцію, маючи на складі масу непроданої пряжі, не треба закупати
бавовну на кредит; купцеві не треба купувати товари на кредит, бо він їх
уже має більше, ніж досить.

Отже, коли настає порушення цього поширу або бодай лише нормального
напруження процесу репродукції, то разом з цим постає й недостача кредиту;
стає тяжче одержувати товари на кредит. Але особливо характеристичним є вимагання
платежу готівкою та обережність у продажі на кредит для тієї фази
промислового циклу, що постає по кризі. А підчас самої кризи, коли кожен
має що продати та, не маючи змоги продати, проте мусить продавати, щоб
\parbreak{}  %% абзац продовжується на наступній сторінці

\parcont{}  %% абзац починається на попередній сторінці
\index{iii1}{0020}  %% посилання на сторінку оригінального видання
і його власної загальної критики цього викладу, основаної на
такому розумінні.

Якщо трапляється десь нагода осоромитись на якійсь важкій
справі, то пан професор \emph{Юліус Вольф} в Цюріху ніколи не пропускає
такої нагоди. Вся проблема, — оповідає він нам („Conrads
Jahrbücher“, Dritte Folge, II, стор. 352 і далі), — розв’язується за допомогою
відносної додаткової вартості. Виробництво відносної
додаткової вартості ґрунтується на збільшенні сталого капіталу,
порівняно з змінним. „Плюс у сталому капіталі має за передумову
плюс у продуктивній силі робітників. Але через те що цей
плюс у продуктивній силі (шляхом здешевлення засобів існування)
веде за собою плюс у додатковій вартості, то встановлюється
пряме відношення між зростаючою додатковою вартістю
і зростаючою часткою сталого капіталу в усьому капіталі.
Збільшення сталого капіталу свідчить про збільшення
продуктивної сили праці. Тому при незмінній величині змінного
капіталу і зростаючому сталому капіталі додаткова вартість
мусить зростати згідно з Марксом. Ось питання, яке було нам
поставлене“.

Правда, в сотні місць першої книги Маркс каже якраз протилежне;
правда, твердження, ніби за Марксом відносна додаткова
вартість при зменшенні змінного капіталу підвищується в такій
самій пропорції, в якій збільшується сталий капітал, таке дивовижне,
що для нього не можна знайти ніякого парламентського
вислову; правда, пан Юліус Вольф доводить у кожному рядку,
що він ні відносно, ні абсолютно нічого не зрозумів ні в абсолютній,
ні у відносній додатковій вартості; правда, він сам
каже: „На перший погляд здається, що тут дійсно опиняєшся
у кублі недоречностей“ — що, до речі сказати, є єдине правильне
слово в усій його статті. Та що з того? Пан Юліус
Вольф такий гордий своїм геніальним відкриттям, що не може
пропустити нагоди, щоб не віддати посмертної хвали за це
Марксу і це своє власне незмірне безглуздя вихваляти як „новий
доказ тієї гостроти та далекоглядності, з якою накреслена
його (Маркса) критична система капіталістичного господарства“!

Але буває ще краще: пан Вольф каже: „Рікардо висловив
твердження: рівні витрати капіталу — рівна додаткова вартість
(зиск) і разом з тим твердження: рівні витрати праці — рівна
(щодо маси) додаткова вартість. І питання полягало в тому,
як одно погоджується з другим. Але Маркс не визнавав питання
в цій формі. \emph{Він безсумнівно довів (у третій книзі)},
що друге твердження не є безумовний наслідок закону вартості,
що воно навіть суперечить його законові вартості і, отже\dots{}
мусить бути прямо відкинуте“. І після цього він досліджує, хто
з нас обох помилявся, я чи Маркс. Що він сам помиляється,
про це він, звичайно, не думає.

Коли б я схотів сказати хоч одно слово з приводу цього
прекрасного місця, це значило б ображати моїх читачів і зовсім
\parbreak{}  %% абзац продовжується на наступній сторінці

\parcont{}  %% абзац починається на попередній сторінці
\index{iii1}{0021}  %% посилання на сторінку оригінального видання
не визнавати комічності становища. До цього я ще додаю тільки
ось що: з тією самою сміливістю, з якою він уже тоді міг сказати,
що „Маркс у третьому томі безсумнівно довів“, він користується
нагодою, щоб розповісти — мабуть, професорську —
плітку, ніби вищезгаданий твір Конрада Шмідта „безпосередньо
інспірований Енгельсом“. Пане Юліус Вольф! В тому світі, в якому
ви живете і дієте, може й водиться таке, що людина, яка публічно
ставить перед іншими проблему, нишком відкриває своїм
особистим друзям її розв’язання. Що ви на це здатні, я вам охоче
вірю. Що в тому світі, в якому я обертаюся, немає потреби
опускатися до такої мерзоти, це доводить вам оця передмова. —

Ледве помер Маркс, як пан \emph{Ахілл Лоріа} спішно опублікував
статтю про нього в „Nuova Antologia“ (квітень 1883): спочатку
біографія, переповнена брехливими даними, потім критика
громадської, політичної і літературної діяльності. Матеріалістичне
розуміння історії Маркса тут сфальсифіковане і перекручене
з таким апломбом, який дозволяє угадати якусь велику
мету. І ця мета була досягнута: в 1886 році той самий пан
Лоріа опублікував книгу: „La teoria есоnomіса della constituzione
politica“, в якій він оповістив здивованому світові його сучасників,
як своє власне відкриття, історичну теорію Маркса, так
грунтовно і так навмисно перекручену ним в 1883 році. Звичайно,
теорію Маркса він звів тут до досить філістерського
рівня; історичні ілюстрації й приклади теж рясніють такими помилками,
яких не простили б і учневі четвертого класу; але
хіба це все має якесь значення? Відкриття, що політичні становища
і події скрізь і завжди знаходять своє пояснення у відповідних
економічних становищах, зроблене, як доведено цією книгою
Лоріа, аж ніяк не Марксом у 1845 році, а паном Лоріа
в 1886 році. Принаймні він щасливо упевнив у цьому своїх земляків,
а з того часу, як його книга з’явилась французькою мовою,
і деяких французів, і може тепер чванитись в Італії як
автор нової епохальної історичної теорії, поки тамошні соціалісти
знайдуть час повискубувати в illustre [славетного] Лоріа
крадені павині пера.

Але це тільки один маленький зразочок маніри пана Лоріа.
Він запевняє нас, що всі теорії Маркса грунтуються на \emph{свідомому}
софізмі (un consaputo sofisma); що Маркс не відступав
перед паралогізмами навіть тоді, коли він \emph{визнавав} їх з\emph{а такі}
(sapendoli tali) і~\abbr{т. д.} І після того, як він в цілому ряді подібних
підлих побрехеньок дав своїм читачам усе потрібне для
того, щоб вони побачили в Марксі якогось кар’єриста à la Лоріа,
який досягає своїх дрібних ефектів за допомогою таких самих
дрібних негідних шахрайських засобів, як наш падуанський професор,
— він може тепер відкрити їм важливу таємницю, а тим
самим і нас приводить назад до норми зиску.

Пан Лоріа каже: За Марксом маса додаткової вартості (яку
пан Лоріа ототожнює тут з зиском), вироблена в капіталістичному
\index{iii1}{0022}  %% посилання на сторінку оригінального видання
промисловому підприємстві, повинна відповідати застосованому
в ньому змінному капіталові, бо сталий капітал не дає
ніякого зиску. Але це суперечить дійсності. Бо на практиці зиск
відповідає не змінному, а всьому капіталові. І Маркс сам бачить
це (І, розд. XI) і визнає, що зовнішньо факти суперечать
його теорії. Але як він розв’язує цю суперечність? Він відсилає
своїх читачів до подальшого тома, який ще не з’явився.
Про цей том Лоріа вже раніш сказав \emph{своїм} читачам, що він
не вірить тому, що Маркс хоч би одну мить думав про те,
щоб його написати, і тепер він тріумфуючи вигукує: „Отже,
я мав рацію, коли твердив, що цей другий том, яким Маркс
весь час загрожує своїм супротивникам і який, однак, ніколи
не з’явиться, що цей том, дуже ймовірно, був хитромудрою
виверткою, якої Маркс уживав тоді, коли в нього не
вистачало наукових аргументів (un ingegnoso spediente ideato
dal Marx a sostituzione degli argomenti scientifici)“. І хто тепер
не переконався, що Маркс стоїть на такій самій висоті наукового
шахрайства, як l'illustre Лоріа, той уже цілком безнадійна
людина.

Отже, ми ось чого навчились: за паном Лоріа теорія додаткової
вартості Маркса абсолютно несполучна з фактом загальної
рівної норми зиску. Аж ось з’явилась друга книга і разом
з тим моє публічно поставлене питання саме про цей пункт.
Коли б пан Лоріа був одним з нас, соромливих німців, він би
якось збентежився. Але він — сміливий житель півдня, він походить
з гарячого клімату, де, як він може твердити, нахабність
(Unverfrorenheit)\footnote*{Тут гра слів: німецьке „Unverfrorenheit“ буквально можна також тлумачити
як „здатність не замерзати“. \emph{Ред. укр. перекладу.}} є до певної міри природна умова. Питання
про норму зиску поставлено публічно. Пан Лоріа публічно оголосив
його нерозв’язним. І саме через це він тепер перевищить
самого себе, розв’язавши його публічно.

Це чудо сталося в „Conrads Jahrbücher“, Neue Folge, т. XX,
стор. 272 і далі, у статті про вищезгаданий твір Конрада Шмідта.
Після того, як він вичитав у Шмідта, яким чином утворюється
торговельний зиск, йому зразу все стало ясно. „Через те що
визначення вартості робочим часом дає перевагу тим капіталістам,
які вкладають більшу частину свого капіталу у заробітну
плату, то непродуктивний“ [слід сказати — торговельний] „капітал
може вимусити собі від цих капіталістів, що мають перевагу,
вищий процент“ [слід сказати — зиск] „і утворити рівність
між окремими промисловими капіталістами\dots{} Так, наприклад,
якщо промислові капіталісти А, В, С застосовують у виробництві
кожний по 100 робочих днів і відповідно 0, 100, 200
сталого капіталу, а заробітна плата за 100 робочих днів містить
у собі 50 робочих днів, то кожний капіталіст одержує додаткову
вартість у 50 робочих днів, а норма зиску становить 100\%
\parbreak{}  %% абзац продовжується на наступній сторінці


\index{ii}{0023}  %% посилання на сторінку оригінального видання
\subsection{Кругобіг у цілому}

Ми бачили, що процес циркуляції по скінченні його першої фази
$Г — Т\splitfrac{Р}{Зп} $ переривається через $П$, що в ньому товари $Р$ і $Зп$, куплені
на ринку, споживається як речеві й вартісні складові частини продуктивного
капіталу; продукт цього споживання є новий товар, $Т'$, змінений
речево і щодо вартости. Перерваний процес циркуляції, $Г — Т$, мусить
доповнитись через $Т — Г$. Але як носій цієї другої та кінцевої фази циркуляції
з’являється $Т'$, товар відмінний від першого $Т$ речево і щодо
вартости. Отже, ряд циркуляцій має такий вигляд: 1) $Г — Т_1$; 2) $Т_2' — Г'$,
де в другій фазі першого товару $Т_1$, підчас перерви, зумовленої функцією
$П$, підчас продукції $Т'$ з елементів $Т$, з форм буття продуктивного
капіталу $П$, постає другий товар, вищої вартости та іншої споживної
форми, $Т_2'$. Навпаки, перша форма виявлення, що в ній капітал виступив
перед нами (книга І, розділ IV, І), $Г — Т — Г'$ (розкладається
на: 1) $Г — Т_1$; 2) $Т_1 — Г'$), двічі показує той самий товар. Там перед
нами обидва рази той самий товар, на який перетворюються гроші
в першій фазі і який в другій фазі перетворюється на більшу кількість
грошей. Не зважаючи на цю посутню ріжницю, обидві циркуляції мають
те спільне, що в їхній першій фазі гроші перетворюються на товар, і
в їхній другій фазі товар перетворюється на гроші, отже, що гроші, витрачені
в першій фазі, зворотно припливають у другій фазі. З одного боку,
спільне у них — зворотний приплив грошей до свого вихідного пункту,
але, з другого боку, і те, що грошей зворотно припливає більше, ніж було
авансовано. В цьому розумінні $Г — Т\dots{} Т' — Г'$ вже міститься в загальній
формулі $Г — Т — Г'$.

\vtyagnut
Далі виявляється, що в обох належних до циркуляції метаморфозах
$Г — Т$ і $Т' — Г'$ кожного разу протистоять одна одній і заступають одна
одну рівновеликі, одночасно наявні вартості. Зміна величини вартости
належить виключно метаморфозі $П$, продукційному процесові, що таким
чином становить реальну метаморфозу капіталу протилежно простій формальній
метаморфозі циркуляції.

А тепер розгляньмо цілий рух
$Г — Т\dots{} П\dots{} Т' — Г'$,
або його розгорнуту форму
$Г — Т\splitfrac{Р}{Зп}\dots{} П\dots{} Т' (Т \dplus{} т) — Г' (Г \dplus{} г)$.
Капітал з’являється тут
як вартість, що перебігає ряд взаємно зв’язаних, одне одним зумовлених
перетворень, ряд метаморфоз, які являють стільки ж фаз або стадій цілого
процесу. Дві з цих фаз належать до сфери циркуляції, одна — до
сфери продукції. В кожній з цих фаз капітальна вартість перебуває в
особливій формі, що їй відповідає особлива, спеціяльна функція. В цьому
русі авансована вартість не лише зберігається, але й зростає, збільшує
свою величину. Нарешті, в кінцевій стадії вона повертається до тієї самої
форми, що в ній вона з’явилась на початку цілого процесу. Тому цей
цілий процес є процес кругобігу.

\parcont{}  %% абзац починається на попередній сторінці
\index{iii1}{0024}  %% посилання на сторінку оригінального видання
і не може і не бажає випустити! Безмежна відвага, сполучена
з в’юнкістю вужа, з якою він викручується з неможливих ситуацій,
героічне презирство до одержаних стусанів, нестримно швидке
присвоювання чужих праць, нахабне шахрайство реклами, організація
слави за допомогою шумихи приятелів — хто може зрівнятися
з ним в усьому цьому?

Італія — країна класичності. Починаючи з тих великих часів,
коли там зійшла зоря сучасного світу, вона породила величні
характери недосяжно-класичної довершеності, від Данте до
Гарібальді. Але й часи приниження і чужоземного панування
залишили їй класичні характери, серед них два особливо рельєфні
типи: Сганареля і Дулькамару. Класичну єдність обох ми
бачимо втіленою в нашому illustre Лоріа.

На закінчення я мушу повести своїх читачів за океан. У Нью-Йорку
пан доктор медицини \emph{Джордж Штібелінг} теж знайшов
розв’язання проблеми, і при тому надзвичайно просте. Таке
просте, що ні одна людина ні по цей, ні по той бік океану не
схотіла його визнати, в наслідок чого він страшенно розгнівався
і в безконечному ряді брошур та газетних статтей по обидва
боки океану гірко скаржився на таку несправедливість. В „Neue
Zeit“ йому, правда, сказали, що все його розв’язання грунтується
на помилці в обрахунку. Але це не могло його стурбувати;
Маркс, мовляв, теж зробив помилки в обрахунках і, однак,
в багатьох речах він має рацію. Отже, погляньмо на Штібелінгове
розв’язання.

„Я беру дві фабрики, які працюють однаковий час з однаковим
капіталом, але з різним відношенням сталого і змінного
капіталу. Весь капітал $(с + v)$ я припускаю $= y$ і позначаю ріжницю
у відношенні сталого капіталу до змінного через $х$. На
фабриці І $y = с + v$, на фабриці II $у = (с — х) + (v + х)$. Отже,
норма додаткової вартості на фабриці І $= \frac{m}{v}$ а на фабриці II $=
\frac{m}{v+x}$. Зиском $(р)$ я називаю всю додаткову вартість $(m)$, на
яку збільшується весь капітал $у$, або $c + v$, на протязі даного
часу; отже, $р = m$. Тому норма зиску на фабриці І $= \frac{p}{y}$, або
\frac{m}{c + v}, а на фабриці II так само $= \frac{p}{y}$, або \frac{m}{(c — x) + (v + x)}, тобто
так само $= \frac{m}{c + v}$. Отже, проблема... розв’язується таким способом,
що на основі закону вартості при застосуванні однакового
капіталу і однакового часу, але неоднакових кількостей
живої праці, з зміни норми додаткової вартості постає однакова
пересічна норма зиску“ (\emph{G. С. Stiebeling}: „Das Wertgesetz und die
Profitrate“. New York, John Heinrich).


\index{iii1}{0025}  %% посилання на сторінку оригінального видання
Хоч який прекрасний і ясний вищенаведений обрахунок, ми
змушені, однак, поставити панові докторові Штібелінгові \emph{одно}
питання: звідки він знає, що сума додаткової вартості, яку виробляє
фабрика І, ні на волосок не відрізняється від суми додаткової
вартості, створеної на фабриці II? Про $c, v, у$ і $х$, отже,
про всі інші фактори обрахунку, він прямо каже нам, що вони
для обох фабрик мають однакові величини, але про $m$ ні слова.
Але з того, що він обидві згадувані тут кількості додаткової
вартості алгебрично позначає через $m$, це ніяк не випливає.
Навпаки, це саме те, що має бути доведене, бо пан Штібелінг
без дальших околичностей і зиск $p$ ототожнює з додатковою
вартістю. Тут можливі тільки два випадки: або обидва $m$ рівні,
кожна фабрика виробляє однакову кількість додаткової вартості,
отже, при однаковому сукупному капіталі і однакову кількість
зиску, і тоді пан Штібелінг вже наперед припустив те, що він
ще тільки повинен довести. Абож одна фабрика виробляє більшу
суму додаткової вартості, ніж друга, і тоді розвалюється весь
його обрахунок.

Пан Штібелінг не побоявся ні праці, ні витрат для того, щоб
на цій своїй помилці в обрахунку побудувати цілі гори обчислень
і подати їх публіці. Я можу дати йому заспокійливе запевнення,
що майже всі вони однаково неправильні, і що там,
де вони як виняток правильні, вони доводять щось цілком інше,
а не те, що він хоче довести. Так, порівнюючи дані американських
переписів 1870 і 1880 років, він дійсно показує падіння
норми зиску, але пояснює його цілком помилково і гадає, що
теорія Маркса про завжди незмінну, стабільну норму зиску має
бути виправлена практикою. Але з третього відділу цієї третьої
книги виходить, що ця „нерухома норма зиску“ Маркса є чиста
вигадка і що тенденція норми зиску до падіння ґрунтується на
причинах, діаметрально протилежних тим, що їх наводить д-р
Штібелінг. Пан д-р Штібелінг має, без сумніву, добрі наміри,
але, якщо хто хоче займатись науковими питаннями, то він мусить
насамперед навчитися читати твори, якими хоче користуватись,
так, як їх написав автор, і перш за все не вичитувати
з них того, чого в них немає.

Результат усього дослідження: і в даному питанні щось
зроблено знов таки тільки школою Маркса. Фіреман і Конрад
Шмідт, коли читатимуть цю третю книгу, можуть бути цілком
задоволені, кожний у своїй частині, з своїх власних праць.

\begin{flushright}
  \emph{Ф.~Енгельс}
\end{flushright}

{\small Лондон, 4 жовтня 1894~\abbr{р.}}


  \bookpaget{\BookNumber{}}{\BookTitle{}}
  \setcounter{footnote}{0}% Reset footnote counter
\bookpages{Додаток}{Фрагмент «Капіталу» у~перекладі Івана~Франка}
 
\cftaddnumtitleline{toc}{book}{Додаток}{Фрагмент «Капіталу»}{}
\addtocontents{toc}{\protect\vspace{-2.4em}}
\addcontentsline{toc}{book}%
  {\protect\numberline{}{у~перекладі Івана~Франка}}%


% ЧОГО vtu ХОЧЕМО? 
% Вперше без підпису надруковано польськаю мовою в газеті «Ргаса», 1879, 18 серпня, під назвою «Czego my chcemy?». В перекладі українською мовою вперше надруковано у вид.: Франко І. Твор и. В 20-ти т., т, 19, с. 215-217. ПодаЕться за першодруком. 
% ВЛАСНІСТЬ ГРУНТОВА І Уі ІСТОРІЯ 
% Вперше надруковано в кн.: Лавле Е. де. Власність грунтова і її історія. Переклав Іван Франко. Львів, 1879 («Дрібна бібліотека», VI), с. 34. (Передмова до книги). ПодаЕться за першодруком. С. 28. лавеле  Еміль де (1822-1892) — бельгійський бур-жуаsний iсторик i •    економlст. Б ю к е р Карл (1847-1930) — німецький буржуазний еконо• міст, історик народного господарства і статистик. 
% ДОПОВНЕННЯ ДО «ОСНОВ СУСПІЛЬНОУ 
% ЕкономІт" 
% Вперше надруковано в журн. «Культура», 1926, Ns 4-9, с. 56-57, та в кн.: Іван Франко, К., 1926, с. 164-166. Це герша передмова до перекладу XXIV розділу «Капіталу» К. Маркса, доданого І. Франком до написаного ним підручника «Ос• нови суспільної економії», який мав вийти у світ в кінці 1879 — на початку 1880 р., але не був надрукований; рукопис його загуб- лено. ПодаЕться за автографом, який зберігся в архіві І. Франка ((р. 3, Nё 448). 
% ГДРУГА ЛЕРЕДМОВА ДО ПЕРЕКЛАДУ 
% 24-го РОЗДІЛУ ПРАцІ К. МАРКСА «КАП[ТАЛв. т. 1] 
% Вперше надруковано в ж,урн. «Культура», 1926, Ns 4-9, с. 57-58, та в кн.: Іван Франко. К., 1926, с. 167---168. Передмову до українського перекладу 24-го розділу «Капі- талу» К. Маркса, який І. Франко мав намір видати окремим ви- пуском «Дрібної бібліотеки» (див. коментар до перекладу в цьому томі)подаеться написано, ймовірно, на початку 1880 року.  за автографом, який э6ерігаЕться в архіві І.Франка, ф. 3, Ns 448. С. 32. ...щоби сама гграця стала товаром...- Тут франківський виклад змісту першого тому «Капіталу» К. Марк- са неточний. К. Маркс мав на уваэі не працю, а робоцу силу. С. 33. Текст перекладу подаЕться у розділі «3 наукових пере- кладів» (с. 581--609). 
% 616 



\section*{Доповненя до „Основ суспільної економії“\protect\footnotemarkZ{}}
\nonumsectioncft{Доповненя до „Основ суспільної економії“}{.~}{Іван Франко}

\footnotetextZ{Вперше надруковано в журн. «Культура», 1926, № 4--9, с. 56--57, та в кн.: Іван Франко, К., 1926, с.~164--166.

Подається за автографом: відділ рукописних фондів і текстології Інституту літератури ім. Т.~Г.~Шевченка НАН України. — Ф. 3. — Од. зб. 448. — 14 арк. 
}

\noindent{}В самім початку „Основ суспільної економії“ сказано було, що економія, се наука абстрактна, т. є. що ціль єї не є виключно — розслідити закони економічні \emph{теперішної} суспільности, але \emph{загальні} закони праці людської. А позаяк с переміною суспільного ладу в протягу віків і закони ті проявляются щораз то в інших формах, випливаючих конечно з даного ладу, то наука економічна не може ніякої с тих форм вважати сталою і незмінною. Не може, значит, і нинішних форм уважати сталими, а мусит шукати таких форм, котрі \emph{після нашого теперішного знаня} булиб відповіднійші для суспільної праці і суспільного добробутку, ніж нинішні форми.

С тої то причини в сістематичнім викладі основ сусп. економії ми не могли давати надто широкого місця вислідам про \emph{нинішний} лад, а ограничились тілько головним єго нарисом. При викладі абстрактної теорії праці се була конечна річ, — але прецінь ніхто не заперечит, що на практиці для кождого дуже важне — знати передовсім докладно теперішний лад, єго почин і розвиток. Таке знанє вже тим корисне, що замісць теоретичних засад подає масу фактів, котрі самі прут розум до таких а таких виводів, між тим коли ті самі виводи, подані без підставних фактів, усякому можут видатися хиткими та схопленими з воздуха мріями. Для того то думаєм ми, що поповнимо подекуди конечний недостаток теоретичного викладу, подаючи в „Доповнених“ обширнійший огляд деяких питань, не порушених або з боку ткнених в самім викладі.

Одна з найважнійших недостач усякого чисто теоретичного викладу та, що приходится виключати з него всякі ширші \emph{історичні} перегляди. Правда, се не є недостача конечна, бо остаточно мож би бути вірним теорії, подаючи перегляд розвитку та впадку всіх економічних порядків від почину цівілізації аж до тепер. Але не кажучи вже о тім, що для такої загальної історії економічного розвитку призбирано доси дуже ще мало матеріялу, — в нашім підручнику такий виклад був би неможливий вже й за недостачею місця. А говорити обширно про розвиток одного — ниніншого — ладу, не казавши нічо про розвиток їнчих, се значилоб вважати сей лад чимось важнійшим від прочих, між тим коли в історії, як і в зрості кождого орґанізму, кожда фаза розвитку для вислідника рівноважна.

Але вважаючи потрібним познайомити наших читателів з історичним розвитком сучасного, капіталістичного ладу, ми робимо се в „Доповненях“. А для своєї ціли ми не можем найти кращого провідника над Карля Маркса, котрий в однім розділі своєї книжки „Das Kapital“ списав короткий, хоть яркий перегляд того, як розвивалася капіталістична продукція. С тим розділом ми й хочемо познакомити наших читателів.


\section*{[Друга передмова до перекладу 24-го розділу праці К.~Маркса «Капітал», т. І]\protect\footnotemarkZ{}}
\nonumsectioncft{[Друга передмова до перекладу 24-го розділу праці К.~Маркса «Капітал», т. І]}{.~}{Іван Франко}

\footnotetextZ{Вперше надруковано в журн. «Культура», 1926, № 4--9, с. 57--58, та в кн.: Іван Франко, К., 1926, с.~167--168.

Подається за автографом: відділ рукописних фондів і текстології Інституту літератури ім. Т.~Г.~Шевченка НАН України. — Ф. 3. — Од. зб. 448. — 14 арк. 
}

\noindent{}В першій части своєї великої економічної праці про „Капітал“ стараєсь Карль Маркс вияснити передовсім, \emph{як повстає капітал}? В тій ціли виказує він поперед усего, що єдиним жерелом усякої вартости є праця людська, котра з матеріалів сирих, даних природою, і при помочи сил природи витворює предмети вжиточні для чоловіка. Коли предмети такі витворюются не для власного вжитку самого витвірця, а для заміни за їнші, тоді вони звутся товарами. Капіталістична продукція полягає на витворюваню товарів, але не всяка продукція, де витворюются товарі, є вже капіталістична. До того потрібно ще одної дуже важної вимінки: \emph{щоби сама праця стала товаром}, т. є. щоб на торзі за певний товар (гроші) мож було заміняти (купити) працю людську.

Звичайно під назвою капіталу у нас розуміются беззглядно гроші. Се по части хибно. Гроші, як бачимо, тоді тілько стают капіталом, коли за них купуєся на торзі робуча сила.

Але праця людська, се не є звичайний товар. Се товар живий, котрий має тоту властивість, що \emph{надає вартість} другим предметам, і надає єї більше, ніж кілько сам коштує. Торгова ціна праці, так як і ціна кождого товару, означена звичайними економічними правилами, с котрих найважнійше — кошт витвореня товару, т. є. в тім разі — кошт удержаня робітника і єго робучої сили. Таку ціну платит капіталіст робітникови за єго працю. Між тим робітник в тім часі, на котрий нанявся, витворює далеко більше, ніж кілько виносит єго плата. Він витворив \emph{надзвишку вартости} понад вартість своєї плати, — тота надзвишка, се зиск капіталіста, — вона побільшує єго капітал. Значит, уся капіталістична продукція полягає на твореню надзвишки, котра задармо дістаєсь капіталістови. Цілий розвиток економічний капіталістичної продукції полягає на тім, що капіталісти всіми силами старалися до крайної можности вбільшити тоту надвишку. Вбільшити єї мож було двома способами: або продовжуючи день робучий (надвишка абсолютна), або приневолюючи робітників в коротшім часі працювати з більшою натугою (релятівна надвишка). Оба ті способи витрібували капіталісти, і то перший з них (продовженє робучого дня) до такої крайности, що аж уряд, затрівожений робітницькими розрухами, мусів вдатися в те діло і ограничити стало довготу робучого дня. Від тоді капіталістична продукція і доси пре в другий бік, — стараєсь той означений правно день робучий як найдоскональше використати, раз-ураз заводячи нові машини, котрі до крайности упрощуют і прискорюют продукцію, а до обслуги вимагают як найменшого числа рук.

Се головні думки, виведені Марксом з безмірної маси фактів, нагромаджених в єго книжці. При кінци книжки розбирає він ще одно важне питане: Яким способом почалася тота капіталістична продукція? Як і на якім ґрунті та при якій управі виріс той дивний порядок, оснований на щоденнім хитрім визиськуваню, на крайній бідносте незлічимих мас народа, а крайнім богацтві немногих щасливців? Сесь важний розділ Марксової книжки — прекрасний культурно-історичний очерк — зрозумілий буде і окремо від цілої книжки і ми хочемо познакомити з ним нашу громаду, як для самої єго великої стійности наукової, так і для того, щоб заохотити всіх, хто тілько владає німецькою мовою, до читаня цілої Марксової книжки. Звичайно говорится про дуже трудний і незрозумілий спосіб писаня у Маркса. Се мож би сказати хіба про перший розділ єго книжки, — а о?кілько такий суд справедливий що до прочих розділів, най посвідчит тота часть, котра отсе переведена.

\setcounter{chapter}{23}
\sectionextended[%
Початок і історичний розвиток капіталістичної продукції в Англії
]{%
Початок і історичний розвиток капіталістичної продукції в Англії\footnotemarkZ{}}{%
\subsection{Первісне нагромадженє капіталу}}
\markboth{%
Початок і історичний розвиток капіталістичної продукції в Англії}{%
Фрагмент «Капіталу» у~перекладі Івана~Франка}

Ми бачили,
\footnotetextZ{Вперше надруковано в журн. «Культура», 1926, № 4--9, с. 61--87.
Подається за автографом перекладача: відділ рукописних фондів і текстології Інституту літератури ім. Т.~Г.~Шевченка НАН України. — Ф. 3. — Од. зб. 448. — 14 арк. Кінець автографа не зберігся. 

Переклад зроблено з другого німецького видання: \textgerman{Das Kapital. Kritik der politischen Oekonomie. Von Karl Marx. Erster Band. Zweite verbesserte Aufgabe. Hamburg. Verlag von Otto Meissner, 1872.} Про це є згадка І. Франка на початку тексту перекладу «Гл[яди] К. Marx. Das Kapital, 2 вид. з р. 1872, стор. 742--794».}
що гроші стают капіталом тоді, коли служат
до купованя робучої сили. Ми бачили, що капітал
раз~у~раз намагає — творити надзвишку вартости, а надзвишка
вбільшує капітал. Між тим щоб капітал міг нагромаджуватись,
мусит уже вперед витворюватись надзвишка;
щоб могла витворюватись надзвишка, мусит істнувати капіталістична
продукція, а щоб тота істнувала, мусит уже
вперед більша маса капіталу бути нагромаджена в руках
поєдинчих богатирів. Здаєсь затим, що весь той процес
полягає на якімось „первіснім“ нагромадженю, котре мало
місце перед капіталістичною продукцією, котре, значит, не
було випливом капіталістичної продукції, а єї жерелом.

\index{franko}{0062}
Тото первісне нагромадженє капіталу („previous accumulation“,
як каже А.~Сміт) грає в суспільній економії
майже таку саму ролю, як „гріхопаденіє“ в теольоґії. Адам
зїв яблоко і через те стягнув гріх на рід людський. Початок
гріха обяснений казкою про давнину. Колись-колись
в давнину були з одного боку пильні вибранці, а з другого —
ліниві нероби. Через те сталося, що перші нагромадили
богацтво, а другі зійшли на таке, що остаточно не мали
вже що продавати крім себе самих. І від того гріхопаденія
почалася бідність великої маси, котра ще й доси, хоть і як
тяжко працює, не має що продавати крім себе самих, —
і богацтво деяких, що й доси змагаєся, хоть самі вони
давно перестали працювати\footnote{
Такі безглузді дитиньства плете ще д. Тйер (звісний французький
муж стану) дотепним колись французам с повагою великого мудрця —
для оборони святої власности. Ну і справді, — скоро діло йде о власність,
то святий обовязок кождого — міцно стояти на становищи букваря,
ще й других переконувати, що те становище для всякого „віка
і возраста“ єдино відповідне і належне.
}. В правдивій історії грали, як
звісно, завойованя, гнет, рабунки, вбійства, — одним словом,
усілякі насиля велику ролю. Але в сумирній політичній
економії з давен-давна — все іділлія. Право і „праця“, се
здавна були єдині способи до збогаченя, тілько, розумієся,
завсігди с тим застереженєм, що аж „сего року воно щось
не так“. Але на ділі способи первісного нагромадженя капіталу
були всякі, які хочете, — тілько не іділлічні.

Гроші і товар не є зразу капіталом, таксамо, як не
є ним зразу средства продукційні і знадоби до житя. Вони
мусят бути перемінені в капітал. Але та переміна може настати
тілько серед певних обставин, котрі зводятся ось на
що: двоякі дуже відмінні посідачі товарів мусят стати супротів
себе і зіткнутися с собою, — з одного боку властивці
грошей, средств продукційних і знадіб до житя, котрим
о то йде, щоб свою суму вартостей побільшити купівлею
чужої робучої сили; а з другого боку вільні робітники, продавці
власної робучої сили і, значит, продавці \so{праці}.
Вільні вони мусят бути в двоякім значіню, т. є. щоб ані
самі вони беспосередно не були средствами продукційними,
як невольники, кріпаки і т. д., ані шоб вони самі не посідали
средств продукційних, як ґазди-селяне, дрібні властивці
ґрунтові і т. д. Такий розділ товарив між дві крайности
— се основні вимінки для капіталістичної продукції.
Без відділеня робітників від власности не може настати
капіталістична продукція. Але скоро вона раз настала, то
не тілько підтримує те відділенє, але й сама доводит до
него раз~у~раз на~ново і раз~у~раз на більший розмір. Коли
затим спитаємо: де є жерело капіталістичного ладу? то
\parbreak{}

\parcont{}  %% абзац починається на попередній сторінці
\index{iii1}{0063}  %% посилання на сторінку оригінального видання
Мальтус, Сеніор, Торренс і т. д., ці явища наводяться безпосередньо
як докази того, ніби капітал просто в своєму речовому
існуванні, незалежно від того суспільного відношення до
праці, в якому він саме й стає капіталом, є, поряд з працею
і незалежно від праці, самостійним джерелом додаткової вартості.
— 2) Під рубрикою витрат, куди належить заробітна плата
цілком так само, як і ціна сировинного матеріалу, зношування
машин і т. д., видушування неоплаченої праці здається тільки
заощадженням на оплаті одного з тих предметів, які входять
у витрати, тільки меншою платою за певну кількість праці;
цілком так само, як відбувається заощадження, коли дешевше
купують сировинний матеріал або зменшують зношування машин.
Таким чином видушування додаткової праці втрачає свій
специфічний характер; його специфічне відношення до додаткової
вартості затемнюється; і цьому затемнінню дуже допомагає
і дуже його полегшує, як показано в книзі І, відділ VI,
представлення вартості робочої сили в формі заробітної плати.

Через те що всі частини капіталу однаково здаються джерелами
надлишкової вартості (зиску), то капіталістичне відношення
містифікується.

Той спосіб, яким додаткова вартість за допомогою переходу
через норму зиску перетворюється в форму зиску, є, однак,
тільки дальший розвиток того переплутання суб’єкта і об’єкта,
яке відбувається уже в процесі виробництва. Вже там ми бачили,
як усі суб’єктивні продуктивні сили праці здаються продуктивними
силами капіталу. З одного боку, вартість, минула праця,
яка панує над живою працею, персоніфікується в капіталісті;
з другого боку, навпаки, робітник виступає просто як предметна
робоча сила, як товар. З цього перекрученого відношення неминуче
виникає вже в самому простому відношенні виробництва
відповідне перекручене уявлення, перенесена з цього відношення
свідомість, яка розвивається далі в наслідок перетворень і модифікацій
власне процесу циркуляції.

Спроба представити закони норми зиску безпосередньо як закони
норми додаткової вартості, або навпаки, є цілком хибна, як
у цьому можна пересвідчитися на прикладі школи Рікардо. В голові
капіталіста, звичайно, ці закони не розрізняються. У виразі m: K
додаткова вартість вимірюється вартістю всього капіталу, авансованого
на її виробництво і почасти в цьому виробництві цілком спожитого,
а почасти тільки застосованого в ньому. Відношення m: K в
дійсності виражає ступінь зростання вартості всього авансованого
капіталу, тобто, взяте відповідно до його раціонального, внутрішнього
зв’язку і природи додаткової вартості, воно показує,
яке є відношення величини, на яку змінився змінний капітал, до
величини всього авансованого капіталу.

\index{franko}{0064}

\subsection{Вивласненє хліборобів}

В Англії щезло кріпацтво дійсно в послідній части \RNum{14} віку. Огромна більшість людности тоді, а ще
більше в \RNum{15} віці, се були свобідні хлібороби, дрібні посідачі ґрунтів, ґазди, — хоть власність їх і
була прикрита різними феодальними прикривками\footnote{
Ще при кінци \RNum{17} віку звиж \sfrac{4}{5} усеї англійської людности були самостійні ґазди-хлібороби, як се
стверджує Маколєй. (Macaulay: „The History of England“, Lond. 1854, v. I, p. 413). Я покликуюсь на
Маколєя тим радше, що він сістематично фальшує історію і подібні факти стараєсь о кілько мож
„обкроювати“.
}. В більших панських добрах замісць давнійших
кріпаків-совтисів (bailiff) настали тепер свобідні арендаторі. Наємні робітники до хліборобства, се
були по части самостійні ґазди-хлібороби, котрі при вільнім часі йшли до пана на заробок, а по части
була се відрубна, стосунково і абсолютно мала верства властивих наймитів. І ті послідні на ділі були
також самостійними ґаздами, бо крім платні одержували від пана також поле коло 4 екрів завбільшки і
коттедж (хату). Притім порівно с прочими ґаздами вони допущені були до вживаня громадського ґрунту,
т. є. толоки, де паслась їх худоба, і ліса, відки вони брали топливо, дерево, торф і пр.\footnote{
Не тре забувати, що навіть і кріпак був не тілько властивцем — хоть за оплатою — тих часток
ґрунту, котрі належали до єго дому, але був також співвластивцем громадських ґрунтів. (Порівн., що
каже Мірабó про шльонських хліборобів в книжці: „De la Monarchie Prussiennе“, Londres 1788).
} У всіх краях Европи ціхує феодальну продукцію поділ ґрунту поміж як мож найбільше підданих. Сила
феодального пана, як і сила кождого короля, полягала не в великости єго доходів, а в многоті єго
підданих, а многота сеся залежала від многоти самостійних ґаздів, осілих на єго добрах\footnote{
Японія зі своїм чисто феодальним упорядкованєм ґрунтової власности і розвитим дрібним
ґосподарством хліборобським вказує далеко вірнійший образ середновікової Европи, ніж усі накупі наші
історії, звичайно закаламучені буржоазними пересудами. Се, бач, дуже вигідна річ —
„ліберальствувати“ на кош(т) середних віків!}. Хоть
затим англійський край по норманськім завойованю поділено на величезні баронства, с котрих одно
нераз містило в собі 900 анґльосаских льордств, то прецінь край той був покритий дрібними
хліборобськими ґаздівствами, серед котрих тілько декуди розлягалися великі панські добра. Такі
стосунки при рівночаснім росцвіті міст, котрий наступив в \RNum{15} віці, сприяли заможности люду, яку
описує канцлєр державний Фортеске в своїх „Laudes Legum Angliae“, але при них не можливе було
капіталістичне богацтво.

\index{franko}{0065}
Перший крок перевороту, що поклав основу капталістичній продукції, припадає в послідній третині 15 і
в першій чверти 16 віку. Тоді скасовано феодальне дворацтво, котре, як справедливо замічає Джемс
Стюерт, „залякало  всі хати і двори безхосенно“. Через те викинено масу голих пролєтаріїв на
робучий торг. Хоть королівська власть, що й сама виросла з буржуазного розвитку, намагаючи до
неограниченого панованя, силою скасувала те великопанське дворацтво, то прецінь вона не була єдиною
причиною нового перевороту. Ні, в упертім опорі протів королівства та
парляменту витворили великі пани-феодали далеко більшу масу пролєтаріяту, прогонюючи силою
хліборобів з ґрунту і посідлости, хоть хлібороби мали до тих ґрунтів більше право, ніж вони, і
забираючи для себе громадські ґрунти. Беспосередний товчок до того в Англії дав іменно росцвіт
фляндрійської вовняної мануфактури і звязане з ним підскоченє цін вовни. Стара феодальна шляхта
вигибла в великих феодальних війнах, а нова шляхта — се були діти свого часу, для котрих гроші були
силою понад всі сили. З вірного поля пасовиська для овець! — се став тепер їх загальний оклик.
Гаррізен в своїй „Description of England. Prefixed to Holinshed’s Chronicles“ описує, як
вивласнюванє дрібних ґаздів руйнує край. „Але що нашим великим самозванцям до того?“ Мешканя ґаздів
та коттеджі робітників валят вони силою або прогнавши людей лишают пустками. „Коли перездримо
давнійші інвентарі кождої домінії, то побачимо, що незлічимі хати та дрібні ґаздівства пощезали, що
ґрунт годує далеко меньше люда, що богато міст підупало, хоть деякі нові підносятся\dots{} Мож би
чимало наросповідатися про місточка та села, зруйновані для того, щоб було місце на толоки для
овець; тілько самотні панські двори стоят серед тих толок“. Правда, наріканя тих старих літописів
усе пересаджені, але вони досадно малюют те вражінє, яке на самих сучасників робив переворот
обставин продукційних. Порівнанє між письмами канцлєрів Фортеске і Томаса Моруса вказує наглядно
пропасть між 15. а 16. віком. „Із золотого віку — каже справедливо Зорнтон — попали англійські
робітники без ніяких перехідних ступнів прямо в зелізну“.

Праводавство злякалось сего перевороту. Воно не стояло ще на такім високім ступни цівілізації, де
„богацтво народне“, т. є. богацтво капіталістів і безграничне висисанє та зубожінє маси люду
становит верх премудрости
політичної. В своїй історії Генріха VII. каже Бекон: „В тім часі (1489) посипалися скарги на то, що
вірне поле перемінюєсь в пасовиська, котрих лехко може дозирати кілька пастухів. Ґрунти, що вперед
виарендовувались на кілька літ, на доживотну або щорічну умову, тепер зіллято разом
\index{franko}{0066}
с панськими. Се підкопало добробуток люду, а через те й міста, церкви, десятини\dots{} Щоб зарадити
тому лиху, проявили король і парлямент дивну на ті часи мудрість\dots{} Вони видали право протів того
обезлюднюючого край загарбуваня громадських ґрунтів (depopulating inclosures) і невідлучної
від него обезлюднюючої ґосподарки толочної (depopulating pasture[s])“. Оден акт Генріха VII. з р.
1489 заказує руйнувати хліборобські хати, до котрих належит що найменьше 20 екрів ґрунту. Генріх
VIII відновив той самий указ. Говорится там між їншим, що „многі аренди і огромні отари, особливо
овець, нагромаджуются в немногих руках, через що дохід
з ґрунту дуже вбільшився, а рільництво дуже підупало, церкви і хати повалено, дивовижні маси народа
стали неспосібні вдержувати себе і свої родини“. Указ наказує затим відбудовувати повалені хутори,
означує, кілько має бути вірного поля в стосунку до овечих толок і т. д. Їнший акт з р. 1533
жалуєсь, що деякі властивці мают по 24000 овець, і ограничує їх число на 2000\footnote{
В своїй „Утопії“ говорит Томас Морус про дивовижний край, де
„вівці їдят людей“.
}. Наріканя народа і
праводавство протів вивласнюваня дрібних арендаторів та хліборобів, що почалось від Генріха VII і
трівало зо 150 літ
— не помогли нічо. Чому не помогли, пояснює нам Бекон, сам того не знаючи. „Акт Генріха VII, — каже
він в своїх „Essays, civil and moral“, Sect. 20, — був глибоко і дивно обдуманий. Він утворив
сільскі ґаздівства і хліборобські доми певного нормального розміру, т. є. вдержав для них таку
пропорцію ґрунту, котра давала їм змогу плодити на світ підданих доста заможних і не придавлених
нуждою, так що плуг був в руках властивців, а не наємників\footnote{
Бекон пояснює далі звязок між свобідним, заможним селянством
а доброю інфантерією. „Се була дивно важна річ для сили і мужности
королівства — мати аренди достаточного розміру, щоб дільних мужів
забеспечити від нужди і велику часть ґрунту краєвого запевнити в посіданє джоменам, т. є. людім
середної заможности між шляхтою а халупниками (cottagers) та наймитами. Бо се загальна думка
найліпших знавців воєнного діла\dots{} що головна сила армії, се інфантерія або піхота. Але щоб
витворити добру інфантерію, тре людей вихованих не в притиску ані в нужді, але свобідно і в певній
заможности. Коли затим яка держава вросте переважно в шляхту та делікатне панство, а хлібороби та
ратаї зійдут на простих зарібників та наймитів або халупників, т. є. жебраків з власною хатою, то
така держава може мати добру кінницю, але доброї піхоти не буде мати. Се видно в Італії і Франції і
деяких других заграничних краях, де справді все або шляхта або нужденні зарібники\dots{} Дійшло там до
того, що ті краї мусят уживати наємного зброду Швейцарів та др. для своєї піхоти: відти то й пішло,
що ті держави мают богато людий, а мало вояків“. („The Reign of Henry VII.“ і т. д.).
}. А між
\parbreak{}

\input{franko/_0067.tex}
\index{franko}{0068}
Але сесі беспосередні наслідки реформації не були
найтривкійші. Церковна власність, се була реліґійна підпора
старосвіцьких порядків ґрунтових. Впала вона, то й їм не
довго було вже встоятись.

Ще в послідних десятилітях 17. віку було джоменів
(самостійних ґаздів хліборобів) більше ніж арендаторів.
Вони творили головну силу Кромвеля і — як свідчит сам
Маколєй — визначувались дуже корисно супротів роспитих
паничів та їх прислужників — сільских попів. Ще навіть
сільскі наємники були співвластивцями громадського ґрунту.
Аж около 1750. щезли джомени зовсім, а в послідних десятиліттях
18. віку щезли послідні сліди громадських ґрунтів
хліборобських. Ми ту не берем на ввагу чисто економічних
двигачів рільничого перевороту, але глядимо тілько на пoсторонні,
насильні товчки.

За реставрації Стюартів перевели великі властивці
ґрунтів правним способом такий самий рабунок, який в прочій
Европі робився і без правних оборотів. Вони знесли
феодальні ґрунтові порядки, т. є. скасували всі ті повинности,
які припадали державі з ґрунтів, „відшкодували“ державу
тим, що наложили податки на хліборобів та прочу
масу народа, а самі забрали в тісну приватну власність усі
добра, над котрими вперед мали лиш феодальну зверхність,
і накинули вкінци народови такі права осідленя (laws of
settlement), котрі, mutatis mutandis, так само повліяли на
англійських хліборобів, як указ татарина Бориса Ґодунова
на россійських хліборобів.

„Преславна революція“ (glorious Revolution) з Вільгельмом
III Оранським дала панованє в руки ґрунтових та капіталістичних
богатирів. Вони почали нову еру тим, що до
роскраданя державних ґрунтів, котре доси велося скромно
і тайком, взялися тепер на кольосальний розмір. Ті ґрунти
роздаровувано, продавано за песі гроші або й прямо без
даня рації прилучувано до приватних дібр\footnote{
„Безправна рострата коронних дібр чи то через продаж, чи через
роздарованє, становит огидну картку англійської історії\dots{} Се величезне
окраденє народа\dots{}“ (F.~W.~Newmann: „Lectures on Political Economy.
London, 1851“. стор. 129, 130).
}. Все то робилося
без найменьшої вваги на правні формальности. Ті закрадені
добра державні ураз із церковним фурфантєм, яке
\parbreak{}

\parcont{}
\index{franko}{0069}
ще не було розгарбане за революції, се основа нинішних
князівських посідлостей англійської оліґархії\footnote{
Прошу прочитати н. пр. Е.~Борка памфлєт про родину герцоґів
Бедфорд, котрої потомок, льорд Джон Россель — один з головних стовпів
теперішного лібералізму.
}. Капіталісти
з міщан радо дивилися на ті операції, між їншим і для
того, бо ґрунти через те робилися чистим товаром, а сільскі
пролєтарії, обідрані до крихти, чим раз більше тислися
до міст за роботою. Вони поступали зовсім відповідно для
власної користи, так само, як шведські міщане, котрих економічною
опорою було селянство і котрі затим дружно с селянами
помогали королям (від р. 1604, пізнійше під Карлом
X і Карлом XI) силою видирати коронні добра з рук
маґнатів.

Власність громадська, се була староґерманська встанова,
котра животіла під покривкою феодальства. Ми бачили,
як тоті громадські ґрунти силою загарбувано, при
чім по більшій части рілю перемінювано в толоки. Се почалося
с кінцем \RNum{15} віку і трівало далі в \RNum{16} Але тоді було
се все такі особистим насилєм, супротів котрого праводавство
дармо боролося цілих 150 літ. Поступ \RNum{18} віку проявляєся
тим, що само право від тепер починає підпирати рабунок
громадських ґрунтів, хоть великі арендаторі побіч
того не закидают і своїх дрібних незалежних способиків на
власну руку\footnote{
„Арендаторі заказуют коттеджерам (халупникам) держати будь
яку будь живу тварь крім себе самих, а то тому, бо як будут держати
худобу або дріб, то будут мусіли з їх стоділ красти пашу. У них є приповідка:
„держи халупника в бідности, то вдержиш го в пильности“.
А властиво все діло ту в тім, що арендаторі таким способом привласнили
собі виключне право на громадські ґрунти“, („А Political Enquiry into
the Consequences of enclosing Waste Lands. Lond. 1785“, стор. 75).
}.  Парляментарною формою, в якій відбувалися
ті рабунки, були „Bills for Inclosures of Commons“ (Закони
про прилученє громадських ґрунтів). Се були декрети, котрими
сільскі льорди роздаровували власність народну самі
собі на власність приватну, — правдиві декрети обдираня
народа. Сер Ф.~М.~Еден, котрий хитро, як правдивий адвокат,
доказує, що ґрунти громадські, се властиво приватна
власність сільских льордів, що настали намісць феодалів,
— сам же зараз збиває всі свої докази, коли домагався
„загальної постанови парляменту для прилученя громадських
ґрунтів (до дібр приватних)“, — значит, признає, що
для їх переміни в приватну власність конечно треба парляментарного
замаху, — а з другого боку сам домагався
від праводавства „відшкодованя“ для вивласнених бідаків.

Між тим коли замісць незалежних їоменів (ґаздів) настали
„tenants-at-will“, т. є. дрібні арендаторі на оден рік,
\parbreak{}

\parcont{}
\index{franko}{0070}
льокайський і від самоволі лєндльордів залежний збрід, розросталися тимчасом з рабунку державних
дібр, а ще більше з сістематичного загарбуваня громадських ґрунтів ті великі аренди, котрі в 18. в.
звано арендами капіталовими або купецькими. Чим більше вони розросталися, тим більше селян витискано
з їх давних домівок, тим більше пролєтарів перлося до міст, до промислу.

Але \RNum{18} вік не понимав ще так досконало, як \RNum{19}, що „богацтво національне“, а вбожество народне —
одно й то само. Про те горячі спори в тогочасній економічній літературі зза „прилучуваня громадських
ґрунтів“. З великої
маси матеріялу, який маю під руками, подаю отсе кілька виривків, бо в них живо малюєся тодішне
положінє.

„В многих округах в Гертфордшайрі“, пише з обуренєм Томас Урайт, „зіллято 24 аренди, кожда пересічно
в 50--150 екрів, усего в 3 аренди“. „В Нортгемтоншайрі і Лінкольншайрі загалом поприлучувано
громадські ґрунти до приватних дібр, а повсталі відси нові льордства поперевертано в толоки. Через
те в многих льордствах не ореся тепер і 50 екрів, де вперед орано 1500\dots{} Звалища колишних хат,
стоділ, стаєнь і т. д., се єдині сліди по давнійших мешканцях. З соток домів і родин де в яких селах
полишалося по 8--10. В найбільшій части округів, де прилучуванє почалося ледво від 15--20 літ назад,
уже властивців ґрунтових дуже мало супротів того, що було вперед. Се ще звичайна річ, коли 4 або 5
богатих годівників худоби посідают недавно позлучувані льордства, на котрих уперед жило 20--30
арендаторів і богато
% REMOVED \footnote*{В рукописі: богати.}
дрібних властивців та комірників. Всіх їх з родинами й цілим спрятком
повикидано гет, а з ними й богато таких родин, котрі у них зарабляли собі прожиток“. (Се пише ч.
Аддіґтон). І прилучували сусідні лєндльорди на підставі Bills for enclosures не тілько перелоги, але
часто й управні ґрунти, котрі громада або винаймала поєдинчим ґаздам за певною оплатою, або
оброблювала спільно. Говорю ту про прилучуванє царини
і загалом управних ґрунтів. Навіть писателі, котрі боронят „прилучуваня“, признают, що воно в тім
разі вменьшило управу піль, підняло в гору ціни за живність і причинилося до обезлюдненя сіл\dots{} А
навіть прилучуванє пустих ґрунтів, яке тепер відбуваєся, відбирає бідному часть утриманя
і вбільшує аренди, котрі й так уже за великі“\footnote{
Др.~Річард Прайс в своїй кнпжці: „Observations on Reversionary
Payments“, т. II, стор. 155. Прошу читати Форстера, Аддінґтона, Кента,
Прайса і Джемса Андерсона, а порівнати се з нужденною балаканкою
та вонючими похвалами Мак-Кельльока в єго списі: „The Literature of
Political Economy. Lond. 1845“.
}. „Коли“,
\parbreak{}

\parcont{}
\index{franko}{0071}
каже Річард Прайс, „всі ґрунти будут в руках кількох великих арендаторів, то з дрібних арендаторів (про них Прайс
казав уперед ось що: „множество дрібних властивців і арендаторів, що вдержуют самі себе й свої родини добутками з ґрунту, котрий оброблюют, доходами з овець, дробу, свиней і т. д.,
котрі випасают на громадських толоках, так, що для вдержаня їм мало що приходится докуповувати“)
пороблятся люде, котрі будут мусіли працею заробляти на прожиток собі і другим, і все, чого їм
треба, будут мусіли купувати на торзі\dots{} Бути може, що праці тоді буде більше, бо більше буде примусу\dots{} Міста й
мануфактури будут змагатися, бо до них напхаєся більше людей шукаючих заняття. Се тота дорога, по
котрій зовсім природно пре концентрація аренд і по котрій вона дійсно довгі вже літа чим раз далі
посуває Англію“. Загальне вліянє „прилучень“ ось як описує Прайс: „Взагалі положінє нижчих верстов
народа майже в кождім згляді погіршилося. Дрібні властивці та арендаторі зруйновані та зведені до
стану наємннків та комірників; а рівночасно й о прожиток в тім стані стало далеко тяжше“\footnote{
В наведеній книжці Р.~Прайса, стор. 147, 159. Се нагадує стародавний Рим, котрого порядки ось як
описує Аппіан в „Історії римських війн домашних“, кн. І, 7: „Богачі забрали в свої руки няйбільшу
часть неподілених ґрунтів. Вони задуфали на обставини часу, що їм тих ґрунтів ніхто вже не відбере, і
скуповували проте сусідні частки бідних, по части за їх згодою, а по части відбирали їм силою, — так, що замісць
поєдинчих  піль богачі оброблювали переважно обширні лани. Притім
вони уживали невольників до управи поля і годівлі худоби, бо свобідних
людей позабирано їм від праці до війська. Посіданє невольників приносило їм ще й тоту велику
користь, що невольники — вільні від військової служби — могли без перепони множитися і плодили
богато дітей. Таким способом постягали маґнати всі богацтва до себе, і цілі околиці вкриті були
невольниками. А правдивих Італьців ставало між тим усе меньше, — їх руйнували: бідність, податки та
військова служба. А хоть часом і настав супокій, то вони зовсім не могли підпомочися, бо весь ґрунт
був у богацьких руках, котрі замісць свобідних людей воліли мати до праці невольників“. Сесь уступ
описує часи перед правом Ліцінія. Військова служба, котра так прудко прискорила руїну римських
плєбеїв, була також головним средством, при помочи котрого Карло Великий перемінив вольних німецьких
селян в кріпаків так швидко, мов петрушку в розсаднику зростив.
}. І справді, наслідки забору громадських ґрунтів і доконаного тим забором перевороту в рільництві далися
так прудко і прикро почути сільским робітникам, що, як сам Еден признає, між 1765 а 1780 плата їх
почала знижуватися до крайної границі і уряд мусів поповнювати єї датками запомоговими. „Плата їх“,
каже Еден, „не вистатчала вже зовсім для потреб житя“.
\par{}

\parcont{}  %% абзац починається на попередній сторінці
\index{ii}{0072}  %% посилання на сторінку оригінального видання
між авансованою заробітною платою і купівельною ціною, що її платить
останній споживач, повинна являти зиск з капіталу. Він розподіляється
між фабрикантом, гуртовим купцем і роздрібним торговцем, з того часу,
як вони розподілили між собою свої функції, а виконана робота лишилась
та сама, хоч здійснили її три особи й три гатунки капіталу
замість одного“.
% Примітку видалено (вона не взазить)
% \footnote*{
% Le commerce emploie un capital considérable qui paraît, au premier coup
% d’oeil, ne point faire partie de celui dont nous avons détaillé la marche. La valeur
% des draps accumulés dans les magasins du marchand-drapier semble d'abord tout-à-fait étrangère à
% cette partie de la production annuelle que le riche donne au
% pauvre comme salaire pour le faire travailler. Ce capital n’a fait cependant que
% remplacer celui dont nous avons parlé. Pour saisir avec clarté le progrès de la
% richesse, nous l’avons prise à sa création, et nous l’avons suivie jusqu’à sa consommation.
% Alors le capital employé dans la manufacture de draps, par exemple, nous a paru
% toujours le même; échangé contre le revenu du consommateur, il ne s’est partagé
% qu’en deux parties: l’une a servi de revenu au fabricant comme profit, l’autre a servi de revenu aux
% ouvriers comme salaire, tandis qu’ils fabriquaient de nouveau drap. —

% „Mais on trouva bientôt que, pour l’avantage de tous, il valait mieux que les
% diverses parties de ce capital se remplaçassent l’une l’autre, et que, si cent mille
% écus suffisaient à faire toute la circulation entre le fabricant et le consommateur,
% ces cent mille écus se partageassent également entre le fabricant, le marchand en
% gros et le marchand en détail. Le premier, avec le tiers seulement, fit le même
% ouvrage qu’il aurait fait avec la totalité, parcequ’au moment où sa fabrication était
% terminée, il trouvait le marchand acheteur beaucoup plus tôt qu’il n’aurait trouvé le consommateur.
% Le capital du marchand en gros se trouvait de son côté beaucoup
% plus tôt remplacé par celui du marchand en détail\dots{} La différence entre les sommes, des salaires
% avancés et le prix d’achat du dernier consommateur devait faire le profit des capitaux. Elle se
% répartit entre le fabricant, le marchand et le détaillant depuis qu’ils eurent divisé entre eux
% leurs fonctions, et l’ouvrage accompli fut le même quoiqu il eût employé trois personnes et trois
% fractions de capitaux, au lieu d’une.
% („Nouveaux Principes d’Economie Politique, Livre II, ch. VIII, éd. 1827. p. 138--140).
% }.

„Всі (торговці) посередньо сприяли продукції, бо вона має на меті
споживання, і тому її можна вважати за вивершену лише тоді, коли
вона подала спродуковану річ до розпорядження споживачеві“.
% Разом з попередньою
% \footnote*{
% „Tous concouraient indirectement à la production; car celle-ci, avant pour
% objet la consommation, ne peut être considérée comme accomplie que quand elle a
% mis la chose produite à la portée du consommateur“, (lb., p. 137).
% }.

Розглядаючи загальні форми кругобігу, і взагалі в усій цій другій
книзі, ми беремо гроші як металеві гроші й лишаємо осторонь
символічні гроші — звичайні знаки вартости, що є лише виключно приналежність
деяких держав, а також кредитові гроші, які ще не розвинулись.
Це, поперше, відповідає історичному розвиткові; кредитові гроші
не відіграють жодної ролі, або лише незначну ролю, в першу добу капіталістичної
продукції. Подруге, доконечність такого порядку дослідження
обґрунтовується також теоретично тим, що всі критичні досліди над
циркуляцією кредитових грошей, що їх маємо з боку Тука й інших,
примушували їх завжди повертатись до розгляду того, як стояли б справи
на основі чистої металевої циркуляції. Але не треба забувати, що металеві
гроші можуть так само правити за купівельний засіб, як і за виплатний
засіб. Дбаючи про спрощення, ми взагалі в цій II книзі беремо їх лише
в першій функціональній формі.

\parcont{}
\index{franko}{0073}
цілі німецькі князівства), а також окремою формою ґрунтової власности, котру так насильно перемінюют
в приватну власність. Ті ґрунти, то була власність повіту (clan), — начальник або „великий чоловік“
був тілько титулярним властивцем, як представник повіту, так само, як королева англійська є
титулярною властителькою всего ґрунту Англії. Тот переворот, котрий в Шотляндії почався по посліднім
повстаню претендента, мож слідити в перших єго початках у письмах Джемса Стеєрта і Джемса Андерсона\footnote{
Стеєрт каже: „Рента в тих околицях (він хибно називає рентою тоту оплату, яку обивателі повіту
(taskmen) складали начальникови повіту) зовсім незначна в стосунку до обширности піль, але що до
числа осіб, котрих удержує одна аренда, мож сміло твердити, що оден кусник ґрунту в шотлянських
горах виживлює десять раз більше людей, ніж так само заобширний ґрунт в найбогатших рівнинах“.
}. В 18. віці заборонено притім Ґелям, прогнаним з ґрунтів, виселюватись в чужі краї, щоб їх таким
способом силою попхнути до Ґлязґова і других фабричних міст\footnote{
1860 вивожено тих насильно вивласнених хліборобів до Канади, отуманивши їх фальшивими
обіцянками. Деякі повтікали в гори і на сусідні пусті острови. Поліція пустилася за ними в погоню,
прийшло до бійки і втікачі здужали вирватися та порозбігатись.
}. За примір
методи пануючої в девятнайцятім віці\footnote{
„В шотляндських горах“, каже Бюкенен, коментатор А. Сміта, 1814, „день в день насильно затираєся
давний власностевий порядок\dots Сільский льорд, без огляду на дідичних арендаторів (знов хибно
названі тексмени) винаймає ґрунт тому, хто найбільше платит, а коли той належит до меліораторів
(imprower), то зараз заводит новий спосіб управи поля. Ґрунт, давнійше покритий дрібними
властивцями, був в стосунку до своєї плодовитости досить заселений; при новім сістемі поліпшеної
управи і побільшеної ренти одержуєсь як мож найбільше плодів як мож найменьшим коштом, і для таго
віддалюются робітники, котрі стали тепер непотрібними. Ті вигнанці з рідних хат шукают відтак
утриманя в фабричних містах і т. д. (David Buchanan: „Observations on A. Smith’s Wealth of Nations.
Edinb. 1814“.) „Шотляндські маґнати вивласнили цілі родини, немов хопту випололи: вони так обійшлися
з селами й людністю, як Інди розїдлі пімстою з дикими звірями по норах\dots Чоловіка продают
за овече руно, за волове стегно, ба ні, ще за меньшу дрібницю\dots Підчас нападу на північні
провінції Хіни була на раді Монголів така думка, щоб усіх мешканців витратити, а їх край перемінити
в степ. Тоту раду богато північно-шотляндських маґнатів дословно виповнили в своїм власнім краю і на
своїх власних земляках“. (Джордж Ензер: „An Inquiry concerning the Population of Nations. Lond.
1818“. Стор. 215, 216.
} досить буде ту навести „обчищуваня“ герцоґині Созерлєнд.
Тота в економії вишколена особа постановила зараз в початку свого панованя взятися до радікального
ліку економічного, і ціле ґрафство, в котрім задля давнійших подібних процесів осталось
\index{franko}{0074}
було всего лиш 15000 люда, перемінити в толоку для овець. Від 1814 до 1820 сістематично
прогонювано та нищено тих 15000 мешканців, т. є. майже 3000 родин. Всі їх села поруйновано і
попалено, всі їх поля пороблено толоками. Англійських жовнірів викомендерувано там для еґзекуції, і
між ними а мешканцями прийшло до бійки. Одна
стара баба згоріла враз іс хатою, с котрої не хтіла вступитися. І таким способом присвоїла собі
вельможна герцоґиня 794000 екрів ґрунту, котрі споконвіку належали до
повіту. Вигнаним мешканцям визначила вона на морськім узберіжю около 6000 екрів, по 2 екри на
родину. Тих 6000 екрів лежали доси пусто і не давали властительці ніякого
доходу. Герцоґиня так далеко зайшла в своїй щедрости, що винаймила екр пересічно по 2 шілінґи 6
пенсів для тих самих селян, котрі много сот літ проливали кров свою за
вельможну герцоґську родину. Увесь зрабований ґрунт повіту поділила герцоґиня на 29 великих аренд
для випасаня овець; в кождій аренді осіла тілько одна родина, переважно англійські наємні
арендаторі. 1825 р. замісць 15000 Ґелів на їх ґрунтах жило вже 181000 овець. А родини, вивержені на
морський беріг, старалися жити риболовством. З них поробилися земноводяні, і вони жили, як каже
писатель, на половину в воді, а на половину на березі, тілько що ні ту ні там не могли найти
достаточного прожитку\footnote{
Коли теперішна герцоґиня Созерлєнд витала в Льондоні з великою парадою міссіс Бічер Стоу,
авторку „Хати дядька Томи“, щоб виставити на показ свою прихильність для муринів-невольників в
американській републіці — чого вона і єї співарістократки певно не булиб зробили підчас домашної
війни американської, бо тоді кожде „шляхотне“ англійське серце було прихильне плянтаторам — в той
сам час описав
я в газеті „New-York-Tribune“ побут невольників созерлєндських. (Деякі місця тої статі навів Керей в
своїй „The Slave Trade. London 1853“.). Мою статю перепечатала одна шотляндська ґазета і викликала
дуже чемну перепалку між тою ґазетою а підхлібниками та похвальками герцоґів Созерлєндів.
}.

Але небораки Ґелі мусіли ще раз відпокутувати свою романтичну наклінність для „великих мужів“, т. є.
для начальників повітових (Сlanchef). Запах риб, котрими прокормлювались земноводяні Ґелі, ударив
великим мужам в ніс. Вони завітрили тут щось зисковного і заарендували морське узберіжє великим
льондонським гендлярам риб. Ґелів другий раз вигнано на штири вітри\footnote{
Цікаву історію того рибного торгу найде читатель у д. Девіда Оркуарта в єго книжці: „Portofolio.
New Series“. Сеніор водній іс своїх посмертних статей називає „процедуру в Созерлєндшайрі“ одним з
найблагодатнійших очищень від віків.
}.

Аж вкінци одну часть пасовиськ назад перемінено
\parbreak{}

\parcont{}  %% абзац починається на попередній сторінці
\index{i}{0075}  %% посилання на сторінку оригінального видання
(зглядно купівель) або частинних метаморфоз, що в них ті самі
монети лише один раз змінюють місце, або пророблюють лише
один обіг, а з другого боку — багато почасти паралельних,
почасти посплітуваних між собою більш-менш багаторозгалужених
рядів метаморфоз, що в них ті самі монети пророблюють
більш або менш значну кількість обігів. Однак загальне число
обігів усіх однойменних монет, що перебувають у циркуляції,
дає пересічне число обігів окремих монет, або пересічну швидкість
грошового обігу. Маса грошей, що їх на початку, приміром,
денного процесу циркуляції кидають у нього, визначається,
певна річ, сумою цін товарів, що циркулюють одночасно й просторово
один побіч одного. Але в межах процесу одна монета
стає, так би мовити, відповідальною за інші. Коли одна прискорює
швидкість свого обігу, то цим затримується швидкість обігу
іншої або остання й зовсім вилітає із сфери циркуляції, бо ця
сфера може поглинути лише таку масу золота, яка, помножена
на пересічне число обігів поодиноких її елементів, дорівнює сумі
цін, що мають бути зреалізовані. Тому, коли зростає число обігів
монет, то маса їх, що перебуває в циркуляції, меншає. Коли
число обігів монет меншає, то маса їх зростає. Через те, що за
даної пересічної швидкости обігу маса грошей, яка може функціонувати
як засіб циркуляції, є дана, то досить лише кинути
в циркуляцію, приміром, певну кількість однофунтових банкнот,
щоб витягти з неї рівно стільки саме золотих соверенів, — трюк
добре відомий усім банкам.

Як в обігу грошей взагалі виявляється лише процес циркуляції
товарів, тобто їхній кругобіг через протилежні метаморфози,
так у швидкості грошового обігу виявляється швидкість зміни
товарових форм, безупинне встрявання одного ряду метаморфоз
в інший, сквапність обміну речовин, швидке зникання товарів
зі сфери циркуляції й так само швидка заміна їх новими товарами.
Отже, у швидкості обігу грошей виявляється поточна єдність
протилежних фаз, що одна одну доповнюють, перетворення
споживної форми на форму вартости і зворотне перетворення
форми вартости на споживну форму, або єдність обох процесів,
продажу й купівлі. Навпаки, в загаянні грошового обігу виявляється
відокремлення й усамостійнення цих процесів як протилежностей,
застій переміни форм, а тому і обміну речовин. Звідки
постає цей застій, цього, певна річ, з самої циркуляції пізнати
не можна. Вона показує лише саме явище. Вульґарний погляд,
помічаючи, що з загаянням грошового обігу гроші не так часто
з’являються і зникають на всіх пунктах периферії циркуляції,
шукає пояснення цього явища в недостатній кількості засобів
циркуляції.\footnote{
«Через те, що гроші становлять... загальну міру купівель і продажів,
кожний, хто має щось на продаж, але не находить покупця, схиляється
до думки, що брак грошей у королівстві або країні є причина,
через яку він не може збути свої товари, і таким чином усі скаржаться на
«брак грошей»; але це велика помилка... Чого хочуть ті, які кричать,
}

\index{i}{0076}  %% посилання на сторінку оригінального видання
Отже, загальна кількість грошей, що функціонують протягом
даного періоду часу як засоби циркуляції, визначається, з одного
боку, сумою цін усіх товарів, що циркулюють, а з другого боку —
повільнішим або швидшим потоком їхніх протилежних процесів
циркуляції, від якого залежить, яку частину з тієї суми цін можна
зреалізувати за допомогою тих самих монет. Але сума цін товарів
залежить так від маси, як і від ціни кожного роду товару. Та ці
три фактори: рух цін, маса товарів, що циркулюють, і, нарешті,
швидкість обігу грошей можуть змінятися в різних напрямах
і в різних пропорціях; отже, сума цін, що має бути зреалізована,
а тому й зумовлювана нею маса засобів циркуляції, може таким
чином пророблювати численні комбінації. Ми зазначимо тут лише
ті, що найважливіші в історії товарових цін.

що немає грошей?... Фармер скаржиться... він думає, що коли б у країні
було більше грошей, він дістав би добру ціну за свої товари... Отже, він,
здається, потребує не грошей, а доброї ціни за своє збіжжя й за свою худобу,
що їх він хоче продати, але не може... Чому він не може одержати
доброї ціни?.. 1) Або тому, що в країні є забагато збіжжя або худоби,
так що більшість людей, що приходять на ринок, мають потребу продавати,
так само як він, і лише меншість має потребу купувати, 2) або
тому, що зменшився звичайний вивіз за кордон... 3) або тому, що падає
споживання, коли люди, приміром, через зубожіння, не можуть витрачати
на предмети споживання стільки, скільки витрачали раніш. Отже,
не збільшення кількости грошей допоможе фармерові продати свої продукти,
а усунення однієї з цих трьох причин, які дійсно натискають на
ринок... Так само потребують грошей купець і крамар, тобто вони не можуть
збути своїх товарів через застій на ринку... нація досягає найбільшого
розвитку тоді, коли багатства швидко переходять із рук до рук».
(«Money being... the common measure of buying and selling, every body
who has anything to sell, and cannot procure chapmen for it, is presently
apt to think, that want of money in the kingdom, or country, is the cause
why his goods do not do off; and so, want of money is the common cry;
which is a great mistake... What do these people want, who cry out for
money?.. The Farmer complains... he thinks that were more money in the
country, he should have a price for his goods... Then it seems money is not
his want, but a Price for his corn and cattle, which he would sell, but cannot...
why cannot he get a price?.. 1) Either there is too much corn and cattle
in the country, so that most who come to market have need of selling,
as he has, and few of buying: or, 2) There wants the usual vent abroad by
Transportation... Or, 3) The consumption fails, as when men, by reason of
poverty, do not spend so much in their houses as formerly they did, wherefore
it is not the increase of specifick money, which would at all advance
the farmer’s goods, but the removal of any of these three causes, which
do truly keep, down the market... The merchant and shopkeeper want money
in the same manner, that is, they want a vent for the goods they deal in,
by reason that the markets fail... a nation never thrives better, than when
riches are tost from hand to hand»). (Sir Dudley North: «Discourses upon
Trade», London 1691, p. 11—15 passim). Всі шахрайства Гереншванда
сходять на те, що суперечності, які виникають із природи товару й тому
виявляються в циркуляції товарів, можна усунути через збільшення
засобів циркуляції. З популярної ілюзії, яка застої в процесі продукції
і процесі циркуляції приписує бракові засобів циркуляції, зрештою,
ніяк не випливає зворотне, а саме, що дійсний брак засобів циркуляції
в наслідок, приміром, офіціяльних махінацій з «regulation of
currency»\footnote*{
— реґулювання засобів обігу. \emph{Ред.}
} не може із свого боку викликати застоїв.

\parcont{}
\index{franko}{0077}
а з беззглядною жорстокістю переведена переміна феодальної та окружної (Clan-)
власности в новійшу приватну власність, — ось які іділлічні були способи
первісного нагромадженя капіталу. Вони здобули ґрунт для капіталістичного
рільництва, втягли землю в обсяг капіталу, а міському промислови
достатчили потрібних „рук“, т. є. вольного і голого пролєтаріяту.

\subsection{Кроваві устави протів пролєтаріїв при кінци XV. віку.}

Вольний і голий пролєтаріят, вигнаний с хат і ґрунтів
через скасованє феодальних дворів і через насильне раз-заразом
вивласнюванє, не міг відразу перелятися весь до
новоповстаючих мануфактур так швидко, як швидко сам
повстав. А при тімже се були люде, викинені раптово с привичного
способу житя, — а такі люде не швидко можут
застосоватися до яких небудь нових, непривичних порядків.
На першій порі з них поробилися маси жебраків, розбійників,
волоцюг, — деякі з наклінности, а найбільша часть під гнетом обставин. С
кінцем XV. і підчас цілого XVI. віку бачимо проте в цілій Західній Европі
кроваві устави протів волоцюгів. Батьки нинішної робітницької верстви мусіли
на самім вступі відбути страшну кару, — за що? За то, що їх перемінено в волоцюг
та голоту. Праводавці вважали їх „добровільними переступцями“ і думали, що
тілько від їх доброї волі залежит — працювати далі серед давних обставин, котрі
між тим зо світа щезли.

В Англії почалось те праводавство під Генріхом VII.

Генріх VIII., 1530: Старі і неспосібні до праці жебраки одержуют дозвіл на
жебрацтво. За то здорові й міцні волоцюги карані будут батогами й арештом. Вони
мают бути привязані ззаду до тачок і бичовані доти, доки не поплине кров з їх
тіла, — відтак мусят зложити присягу, вернути на місце уродженя або там, де
пробули послідні 3 роки, і „засісти до праці“ (to put himself to labour). Що за
безсердечна насмішка! В 27 уст. Генріха VIII повторена попередна устава, але
заострена новими додатками. Як кого другий раз зловят на волоцюгованю, то такого
бичувати ще раз і відтяти му пів вуха. За третим разом непоправного волоцюгу,
як тяжкого злочинця і ворога суспільности — вкарати смертю.

Едуард VI.: Устава с першого року єго панованя 1547, наказує, що скоро хто
отягаєся від праці, той має бути присуджений на невольника тій особі, котра
донесла урядови о єго неробстві. Пан має годувати невольника хлібом і водою,
слабими напитками і такими обрізками мяса, які му видадутся відповідними. Він
має право всилувати го батогами \index{franko}{0078}
та зелізними ланцами до всякої, хотьби й як гидкої роботи. Коли невольник на 14
день віддалится, то зістає засуджений на віковічну неволю і має бути на чолі
або на лици напятнований буквою S, а коли до трох раз утече, то має бути
вкараний смертю, як зрадник держави. Пан може го продати, передати в наслідство,
визичити другому в неволю, зовсім так, як усяке друге рухоме добро, як худобу.
Коли невольники в чім небудь станут супротів панів, то мают також бути покарані
смертю. Мирові судьї повинні за отриманим остереженєм слідити за волоцюгами.
Коли покажеся, що такий волоцюга три дни волочився без діла, то такого
відставити на місце, де родився, роспеченим зелізом напятнувати на груди буквою
V і тамій в зелізних ланцюхах уживати до замітаня вулиці або до якої небудь
їншої служби. Коли волоцюга подасть фальшиво місце вродженя, то за кару має
бути віковічним невольником тої громади, тих мешканців або того товариства і
напятнований буквою S. Кождий має право відобрати у волоцюги єго діти і яко
помічників та термінаторів держати хлопців до 24, дівчат до 20 літ. Коли вони
втечут, то мают аж до тих літ бути невольниками майстра, а тому вільно їх
заковувати в ланци, бити і пр., як му сподобаєсь. Кождий пан може заложити
зелізну обручку на шию, руку або ногу свого невольника, щоби міг го ліпше
пізнати і бути певним, що му не втече\footnote{
Автор книжки „Essay on Trade and Commerce“ 1770, каже: „Під панованєм Едварда
VI. взялись були Англічане зовсім, здаєсь, серйозно до піддвигненя мануфактур і
затрудненя бідних. Се бачимо з одної дивовижної устави, в котрій приписуєсь, що
всі волоцюги мают бути пятновані, і т. д. (Essay on Trade and Commerce, стор.
8).
}. Послідна часть тої устави наказує, щоб
деяких бідних брали на себе громади або поєдинчі люде; ті мают їм давати їсти
й пити і старатись для них о роботу. Тот рід громадських невольників удержувався
в Англії гет ще в 19. віці під назвою roundsmen (люде, що ходят від хати до
хати).

Єлисавета, 1572: жебраки без дозволу і віком понад 14 літ мают бути без
милосердя бичовані і напятновані на лівім вусі, хіба що їх хто схоче взяти на
два роки на службу; в разі повтореня, коли мают над 18 літ, мают бути — смертю
карані, скоро їх ніхто не схоче взяти на два роки на службу; за третим разом
мают без милосердя як зрадники державні бути покарані смертю. Подібна також 18.
устава Єлисавети, розділ 13, і устава з р. 1597 \footnote{
Томас Морус каже в своїй „Утопії“: „Так то дієся, що оден захланний і неситий
ненаїсник, правдива чума нашої вітчини, може тисячі екрів ґрунту збити до купи
і обпалькувати, обгородити одним плотом, або силою та кривдою до того довести
єго властивців, що вони будут мусіли все спродувати. Сяким чи таким способом,
чи там гнись чи ломайся, він присилує їх забиратися, — бідні, прості, нещасливі
душі! Мужчини й женщини, чоловіки й жінки, сироти без батьків, удови, плачучі
матері с пеленковими дітьми, і вся челядь, убога добром, а богата
роботами, бо рільництво вимагає богато рук. І волочутся вони, кажу вам,
з знакомих, рідних місць, не находячи пристанівку. Якби при й не таких
обставинах, то моглиб бодай що то вторгувати за свій, хоть і не дуже
цінний, домашний спряток; але раптово повикидувані, мусят усе продавати
за песій гріш. А коли перебурлачат послідний свій гріш, то щож
тоді мают робити, як не красти, а відтак, боже добрий, по всій формі та
правді згинути на шибеници або пуститися на жебри. А й тоді ще їх
попрут до вязниць як волоцюгів, що-ді плентаются, а нічо не робят.
А що там судови до того, що їх ніхто не хоче взяти на роботу, хоть би
й як радо самі на ню напрошувались!“ І таких бідних утікачів, котрих
но словам Томаса Моруса присилувано до крадіжи, „за панованя Генріха
VIII., повішено 72000 великих та дрібних злодіїв“. (Ноllingshed, Dеscription
of England, т.~І, стор. 186). За часів Єлисавети „вішано волоцюгів
цілими рядами; а прецінь не було такого року, в котрім би на
однім або другім пляцу не повішено їх 300--400“ (Strype`s Annals, т. II),
Той сам Страйп свідчит, що в Соммерcетшайрі за оден рік повішено 40
люда, напятновано 35, бито батогами 37, а випущено 183 „непоправних
злочинців“. А такій, каже він, „те велике число оскаржених не становит
ще й пятої части всіх злочинців, дякувати недбальству мирових судів
і глупому милосердю народа“. Він додає: „Прочі англійські ґрафства
зовсім не стояли ліпше від Соммерсетшайра, а богато стояло в тім згляді
ще далеко гірше“.
}.

\index{franko}{0079}
Яков І: Кождий, хто ходит від села до села і жебрає,
узнаєсь волоцюгою. Мирові суді мают право засудити го на
прилюдне бичованє і за першим разом на 6 місяців, за
другим на 2 роки тюрми. Підчас сидженя в тюрмі мают
бути так часто і так богато бичовані, як се мировий судя
узнасть за добре\dots{} Непоправні і небеспечні волоцюги мают
бути на лівім плечи напятновані буквою R і заставлені до
робіт примусових, а як їх ще коли придиблют на жебранині,
то мают бути без милосердя і без сповіди повішені. Ті устави
(в рукоп. „уставі“), правосильні аж до перших літ 18. віку,
знесені зістали доперва 12. уст. Анни, розд. 23.

Подібні устави бачимо і в Франції, де в половині 17.
віку завязалось було ціле царство волоцюгів (truands) в Парижи.
Ще в початку панованя Людовіка XVI. (Указ з дня
13. липня 1777) кождий здорово збудований чоловік від 16
до 60 літ віку, скоро був без удержаня і не мав означеного
занятя, мав бути висланий на ґалєри. Подібні також: устава
Карля V. для Нідерляндів з 6. жовтня 1537, перший едікт
держав і міст голяндських з 19. марта 1614., оповіщенє Сполучених
провінцій з д. 25. червня 1649 і богато других.
Ось яким способом, — батогами, пятнованєм та тортурами
на підставі нелюдських, кровавих устав увігнано мужиків, \parbreak{}

\parcont{}
\index{franko}{0080}
насилу обрабованих з ґрунту, хат і майна, насилу пороблених
злодіями та волоцюгами, в ті тверді рами карности,
конечної при сістемі наємної праці.

IV.    Устави для знищеня робучої плати.

Не досить того, що знадоби продукції розділюются:
на однім боці сам капітал (в руках властивців богатирів),
а на другім боці сама праця, т. є. люде, котрі нічо не мают
на продаж крім своєї праці. Не досить ще присилувати
тих людей до того, щоб добровільно себе самих запродували.
В дальшім ході капіталістичної продукції виростає
вже верства робітників, котра з вихованя, традиції, привички
признає вимоги того способу продукована природними законами,
чимось таким, що й бути інакше не може. Впорядкованє
видосконаленого капіталістичного процесу продукційного
перемагає всі запори; ненастанне повставанє релятівного
перелюдненя\footnote*{
Звісно, що перелюдненєм звеся то, коли де-небудь є забагато
людей, т. є. властиво більш людей, ніж може вижити. А релятівне перелюдненє
значит, що тілько в певнім місци і серед певних обставин є для
певного діла забогато людей, так, що всі вони не можут приміститися,
і одна часть з них дармує. Кождий пійме, що вже сама проява такого
релятівного перелюдненя є знаком нездорових економічних обставин.
Між тим, як побачимо далі, ціла капіталістична продукція нерозлучно
звязана с релятівним перелюдненєм, котре змоглося в краях промислових
особливо від заведеня парових машин, через що мілійони рук робітницьких
стратили роботу (Прим. перев.).
} вдержує довіз робучих рук і попит
за працею, значит, і робучу плату на такій висоті, яка кориснійша
для підростаючого капіталу; німий примус економічних
обставин довершує панованя капіталіста над робітником.
Позаекономічна, беспосередна сила входит все ще
в уживанє, але вже лиш виїмково. При звичайнім ході діла
досить є — лишити робітника під властю „природних законів
продукції“, т. є. лишити го в залежности від капіталу,
витвореній і навіки забеспеченій самими вимінками
продукційними. Але сего не мож зробити в тій історичній
хвили, коли капітал і етична продукція інощо зароджуєсь.
Підростаюча буржоазія потребує і уживає власти державної,
щоб „реґулювати“ робучу плату, т. є. втискати єї в такі
границі, які найкориснійші для баришництва, продовжувати
день робучий і вдержувати самого робітника в „належитій“
степени залежности. Се також дуже важний причинок до
т. зв. первісного нагромадженя капіталу.

Верства наємних робітників, що повстала в послідній
половині 14. віку, становила тоді і в слідуючих столітях
тілько дуже незначну часть людности, котрої становище
\parbreak{}

\parcont{}
\index{franko}{0081}
притім міцно обезпечували самостійні ґаздівства по селах,
а цехові звязки по містах. По селах і містах не було великої
суспільної ріжниці між майстрами а робітниками.
Підчиненє праці під капітал було тілько формальне, т. є.
продукція сама не мала ще на собі окремої капіталістичної
ціхи. Попит за наємною працею змагався прото дуже швидко
за кождим нагромадженєм капіталу, — між тим рук готових
найматися до праці прибувало дуже поволи. Велика
часть витворів суспільних, що пізнійше стала фондом вбільшуючим
капітал, тоді переходила ще в руки робітника для
єго власного зужитку.

Праводавство про наємну працю, згори вже вицілене
на визискуванє робітника і в своїм розвитку йому завсігди
однаково неприхильне, почалося в Англії від виданя „Устави
робітницької“ (Statute of Labourers) Едвардом III., 1349.
Рівночасно видано в Франції Указ 1350 р. в імени короля
Жана. Англійські і французькі устави виходят рівнобіжно
і зовсім однакі що до змісту.

Устава робітницька зістала видана за про голосні наріканя
послів. „Давнійше“, каже наївно оден Торі, „жадали
бідні такої великої плати за роботу, що промисл і богацтво
були загрожені. Тепер плата така низька, що знов грозит
промисловії й богацтву і то може ще небеспечнійше ніж
тоді“. Установлено правну тарифу платну для міст і сіл,
за роботу (в рукоп. „робуту“) на дни й від штуки. Сільскі робітники
повинні винайматися на рік, міські „с прилюдного
торгу“. Під карою тюрми заборонено платити висшу плату
від означеної в уставі; а хто бере більшу плату, того кара
виносит більше, ніж сама плата. Так само ще в розд. 18
і 19. устави о учениках ремісницьких, виданої за Єлисавети,
грозится карою 10 день тюрми тому, хто платит більше,
а 21 день тюрми тому, хто бере більшу плату від правом
приписаної. Устава з р. 1360. заострила кари і навіть дала
майстрам право силувати робітників мусом до праці за таку
плату, яка означена в тарифі. Всякі звязки, угоди, присяги
і т. д., котрими взаїмно сполучилися теслі з мулярами,
узнані неважними. Стоваришеня робітницькі караются як
тяжка провина від 14. віку до 1825, в котрім скасовано
устави протів стоваришень. Дух „Робітницької устави“ з р.
1349 і її потомків просвічує ясно й с тих устав протів стоваришень.
Се тота сама засада: держава приписує, кілько
мож найбільше платити робітникови, але хрань боже, щоб
хоть натякнула на те, кілько мож йому найменьше платити!

В 16. віці, як звісно, положінє робітників дуже погіршилося.
Правда, грішми плачено більше, тількож що ціна
прошей стала меньша, а ціна товарів без міри більша. На
ділі затим і плана вменышилася. А прецінь устави для єї
зпиженя трівают далі порівно з обрізуванєм вух та пятнованєм
\index{franko}{0082}
тих, „котрих піхто не хоче взяти на службу. Єлисаветина
5 устава про учеників ремісницьких, уст. 3. надає
мировим судям власть становити де в яких реміслах плату
і змінювати ї відповідно до пори року і ціни товарів. Яков
I ростягнув ту саму реґуляцію робітницької плати на ткачів,
прядільників і на всі можливі розряди робітників\footnote{
З одної примітки до устави 2. за Якова І, розд. 6. видно, що
деякі суконники позваляли собі самі яко мирові суды урядово діктувати
платну тарифу в своїх варстатах. — В Німеччині, а іменно по 30-літній
війні, виходнт богато устав для знижуваня робучої плати. „Помічникам
на безлюдних ґрунтах дуже прикро давалась чути недостача слуг і робітників.
Всім мужикам-ґаздам заказано приймати в комірне мужчин та
женщин вільного стану; про всіх таких комірників повинно доноситися
урядови, а той запирає їх в тюрму, скоро не хотят стати слугами, хоть би
й без того мали яке їнше вдержанє, хоть би працювали у  мужиків за поденщину
або навіть торгували грішми та збіжєм. (Цісарські прівілєї та
ухвали для Шльонська, І, стор. 125). Через цілих сто літ роздаются в приписах
князів та поміщиків раз-відразу гіркі наріканя на злосливих
і здуфалих слуг, що не хотят піддатися важким условинам, не хотят вдоволюватися
платою правом приписаною. Виходят накази, щоб поєдинчпй
поміщик не смів своїм слугам платити більше, ніж кілько весь краєвий
збір покладе в таксу. А прецінь условини служби по війні нераз ще
бувают ліпші, ніж були 100 літ опісля. В р. 1052 діставали ще слуги на
Шльонську по два рази до тижня мясо; а ще в нашім столітю іменно
там були такі округи, де слуги діставали мясо хіба три рази до року.
І поденщина (плата за день роботи) по 30-літній війні була більша ніж
в слідуючих столітях“ (Ґустав Фрейтаг).
}, Джордж
II ростягнув устави протів робітницьких товариств на всі
мануфактури. В властивій порі мануфактуровій капіталістична
продукція була вже досить сильною, щоб правну
реґуляцію робучої плати зробити непотрібною, а то й неможливою,
але все такі ще на всякий злучай не закидувано
того перестарілого оружя. Ще 8. устава Джорджа II заказує
давати кравецьким челядникам в Льондоні і околици більше
понад 2 шіллінґи і півосьма пенса денної плати, окрім хіба
в разах загальної жалоби. ІЦе 13 уст. Джорджа III, розд.
68. повіряє мировим судям реґульованє робучої плати у виробників
шовку. Ще 1796 тре було двох декретів висших
судів для рішеня, чи накази мирових судьїв що до робучої
плати мают вагу і для нерільничих робітників. Ще 1799.
потвердила ухвала парляменту, що плата копальників шотландських
уреґульована уставою Єлисавети і двома шотляндськими
актами з р. 1661 і 1671. А який між тим переворот
доконався у всіх обставинах, доказала подія нечувана
в англійській палаті панів. Ту, де від звиш 400 літ
фабриковано устави виключно о тім, понад яку міру не
може ніяк переступити робуча плата, — ту поставив 1799
\parbreak{}

\parcont{}
\index{franko}{0083}
Уайтбрід внесок устави, яка може бути найменьша плата
для робітників рільничих\dots Хоть Пітт супротивлявся тому
внескови, то прецінь і сам признав, що „положінє вбогих
страшенне (cruel)“. Вкінци 1813 скасовано устави про реґуляцію
плати. Вони стались смішним недоріцтвом, відколи
капіталіст порядив у своїй фабриці після власних приватних
прав, а плата рільничого робітника давно впала понизше
мінімум конечного до прожитку, і мусіла до висоти
того мінімум доповнюватися с „податку на бідних“. Постанови
„Устави робітницької“ що до згоди між майстром
а наємним робітником, що до вимовленя терміну і т. д.,
постанови дозволяючі тілько цівільну скаргу па недодержуючого
умови майстра, а крімінальну скаргу на недодержуючого
умови робітника, — ті постанови стоят ще й доси
в повній силі. Нелюдські ухвали супротів стоваришень
скасовано 1825 з ляку перед грізною поставою пролєтаріяту.
Парлямепт зніс їх дуже нерадо\footnote{
Деякі останки устави протів стоваришень знесено аж 1859 р.
(Додаток до 2. вид.) Устава з 29. жовтня 1871. зносит всі устави
протів стоваришень і урядово признає „Робучі Звязки“ (Trades Unions).
Але в однім додатковім акті с того самого дня, п. н. „An Act to amend
the Criminal Law relating to violence, threats and molestation” — устави
протів стоваришень щасливо воскресли в новій формі. Сесь акт піддає
іменно робітників за вживанє деяких средств воєнних протів майстрів
під окремі устави крімінальні, а судят робітників на підставі тих устав
самі ж майстри, яко мирові судьї. Два роки перед тим та сама палата
послів і тот сам Ґлядстон. що 1871 винайшли нові проступкп на робітників,
вихвалювали при другім єго читанню один внесок до устави, в котрім
чесним способом роблено конець всяким окремим праводавствам
протів робітників. Вихвалювали, вихвалювали, тай хитро-мудро стали на
другім читанню. (Звісно, що в Англійськім парляменті кождий внесок,
заким одержит силу права, мусит бути три рази читаний і більшістю голосів
принятий. Прим, перев.) Цілі два роки відволікано сю справу, аж
поки „велике ліберальне сторонництво“ не звязалось зі своїми противниками
і не почулося задосить сильним, щоб разом стати — протів спільного
ворога — робітників.
},  той сам парлямент, що
сам довгі столітя с цинічним безвстидством виступав як
неустаюче стоваришенє капіталістів супроті робітників.

Сейчас в початках революційної бурі поквапилась французька
буржоазія інощо здобуте право стоваришень знов
видерти робітникам. В декреті с 14. червня 1791 оголосила
вона, що всі робітницькі стоваришеня, се „замах на свободу
і признані права чоловіка“, за котрий накладаєсь кара
500 ліврів і позбавлене па рік актівних прав горожанських.
Се право, котре конкуренційну боротьбу між капіталом
а працею силою поліційно-державною втискає в такі границі,
які вигідні для капіталу, перетрівало революції та зміни
\parbreak{}

\parcont{}  %% абзац починається на попередній сторінці
\index{iii2}{0084}  %% посилання на сторінку оригінального видання
«і тому цією операцією ви мусити порушити вексельний курс, бо закордонний
борг не оплачено в наслідок того, що ваш експорт не має відповідного імпорту.
— Це правило для всіх країн взагалі».

Лекція Вілсона сходить на те, що всякий експорт без відповідного імпорту
становить одночасно імпорт без відповідного експорту; бо в продукцію товарів,
що їх експортують, ввіходять чужоземні, отже, імпортовані товари. Перед
нами припущення, що всякий такий експорт ґрунтується на неоплаченому
імпорті або породжує його, — отже, породжує борг закордонові, або ґрунтується
на ньому. Це — помилкова річ, навіть, коли не вважати на ті дві обставини,
що 1)~Англія має даремний імпорт, не платячи за нього жодного еквівалента;
напр., частину свого індійського імпорту. Індійський імпорт вона може обмінювати
на американський імпорт, експортуючи останній без еквівалентного імпорту;
щож до вартости, то в усякім разі Англія експортувала тільки те, що їй нічого
не коштувало; 2)~Англія може й оплатила імпорт, напр., американський, що
утворює додатковий капітал; коли вона той імпорт споживає непродуктивно,
напр., на військові припаси, то це не утворює боргу проти Америки та не
впливає на вексельний курс з Америкою. Newmarch суперечить сам собі в
посвідченнях 1934 та 1935, й Wood звертає його увагу на це в 1938: «Коли
жодна частина товарів, ужитих на виготовлення речей, що їх ми вивозимо без
зворотного припливу» [військові видатки] «не походить з тієї країни, куди ці
речі експортуються, то яким способом це впливатиме на вексельний курс з цією
країною? Нехай торговля з Турцією перебуває у звичайному стані рівноваги;
яким способом вивіз військових припасів до Криму вплине на вексельний курс
між Англією та Турцією?» — Тут Newmarch втрачає свою рівновагу, забуваючи,
що саме на це просте питання він дав уже слушну відповідь під № 1934, він
каже: «Ми вже, мені здасться, вичерпали практичне питання, а тепер увіходимо
в дуже високу ділянку метафізичної дискусії».

[Вілсон має ще й інше формулювання того свого твердження, що на вексельний
курс впливає всяке перенесення капіталу з однієї країни до іншої, однаково,
чи відбувається воно у формі благородного металу, чи у формі товарів.
Вілсон, природно, знає, що на вексельний курс впливає рівень проценту, а
саме, відношення чинних норм проценту в тих двох країнах, що їхній взаємний
вексельний курс розглядається. Отже, коли він буде в стані довести, що надмір
капіталу взагалі, отже, передусім надмір товарів всякого роду, в тім і благородного
металу, має разом з іншими обставинами вплив на рівень проценту, визначаючи
його, то він буде уже на крок ближче до своєї мети; перенесення
значної частини цього капіталу з однієї країни до іншої мусить змінити рівень
проценту в обох країнах, і то саме в протилежному напрямку а тому другою
чергою мусить воно змінити й вексельний курс між обома країнами. — \emph{Ф.~Е.}].

В Economist’і, що його він тоді редаґував, за рік 1847, на стор. 475,
він пише:

1)~«Очевидно, що такий надмір капіталу, який виявляється у великих запасах
всякого роду, в тім і благородного металу, неминуче мусить привести не тільки
до низьких цін на товари взагалі, але й до нижчого рівня проценту за ужиток
капіталу.

2)~Коли ми маємо запас товарів, достатній для того, щоб обслужити
потреби країни протягом двох наступних років, то порядкування цими
товарами протягом даного періоду можна здобути за далеко нижчу норму, ніж
тоді, коли того запасу вистачить ледви чи на два місяці.

3)~Всякі позики грошей,
хоч і в якій формі їх робитиметься, являють лише передачу порядкування над
товарами від однієї особи до іншої. Тому, коли товарів є понад міру, грошовий
процент мусить бути низький, а коли товарів обмаль, він мусить бути високий.

\parcont{}
\index{franko}{0085}
капітал через ужитє наємних робітників і одну часть надвишки витворів, грішми чи натурою, платят
дідичови яко ренту ґрунтову. Доки в 15. віці незалежний мужик, а також сільский наймит, що попри
наймитство й сам про себе веде ґосподарство, збогачуются самі власною працею, доти й обставини тай
обсяг продукційний арендатора остаются дуже скромні. Переворот в рільництві, що почався в послідній
третині 15. віку і трівав через цілий 16 вік крім єго послідних десятиліть, збогатив го майже так
само прудко, як прудко зубожив мужиків\footnote{
„Арендаторі“, каже Гаррізен в своїй „Description of England“, „котрим давнійше годі було
заплатити 4 ф. шт. ренти, платят тепер по 40, 50 та 100 ф. шт. і ще кажут, що їм зле повелося, коли
по упливі арендового контракту не зложили бодай тілько готівки, кілько виносит 6--7-милітна рента“.
}. Загарбанє громадських пасовиск і т. д. дозволяє му
богато побільшувати число худоби майже без ніяких видатків, а між тим худоба достатчувала му далеко
більше обірнику для поправи ґрунту. В 16. віці причинюєсь ще одна рішучо важна обставина. Тоді
арендові контракти були довгі, нераз де з на 99 літ. А ту в 16. віці вартість золота та срібла, а
разом з ним і вартість грошей раз-ураз вменьшуєсь, і арендаторам се принесло золоті плоди. Не
зважаючи на прочі, вперед згадані обставини, арендаторі першим ділом вменьшили робочу плату. Те, що
урвано робітникам на платі, побільшувало
арендовий зиск. А з другого боку ціна збіжя, вовни, мяса, — одим словом, всіх плодів рільничих,
раз-ураз вбільшуєсь, через що змагаєся грошевий капітал арендатора
без єго причинку, — а притім ще ренту ґрунтову дідичови платит він давними, стратившими на вартости,
грішми. Таким способом він збогачуєсь рівночасно на кошт своїх наймитів і свого дідича. Не диво (в
рукоп. „даво“) затим, що вже с кінцем 16. віку витворилась в Англії окрема верства як на тодішні
обставини богатих „капіталістичних“ арендаторів\footnote{
В Франції з „Regisseur-ів“, т. є. панських окономів та тивунів середновікових поробилися швидко
т. зв. hommes d'affaires, т. є. люде, що туманництвом та шахрайством подороблялися капіталів. Такі
окономи, то були нераз великі пани. Як в Англії, так і в Франції великі феодалні добра поділені були
на богато дрібних ґосподарств, але з условинами далеко гіршими для мужиків. В~14. віці повстают і ту
аренди, звані ту „fermes“ або „terriers“. Число їх раз-ураз змагалося і дійшло гет понад
100000. Вони платили чи то грішми чи натурою ренту ґрунтову, котра виносила від 12-тої до 5-тої
части річного здобутку. Ті terriers були цілими або частковими леннами як до вартости і обєму
ґрунтів, котрі нераз виносили заледво кілька прутів. Всі арендаторі мали до певної степені (степенів
було штири) власть судову над мужиками, жиючими
на їх ґрунтах. Лехко поняти, якого притиску мусів дізнавати люд від
усіх тих дрібних тиранів. Монтейль каже, що тоді було в Франції 160000
судів, де тепер вистарчає (враз із мировими судами) 4000 трибуналів.
}.
\index{franko}{0086}
\subsection{Вліянє рільничого перевороту па промисл.
Промисловий капітал здобуває собі в краю ринок
відбутовий}

Раптове і частими нападами повторюване вивласнюванє
та прогонюванє мужиків достатчило, як ми бачили,
міському промислови раз-заразом маси пролєтаріїв, не належачих
зовсім до ніяких цехових звязків, — дуже мудра
подія, про котру старший Андерзен (не треба го мішати
з Джемсом Андерзеном) в своїй історії торговлі каже, що
се прямо боже провидініє так зробило. Ще хвилю мусимо
задержатися над тим складником первісного нагромадженя
капіталів. Не тілько що по селах убуло незалежного, самоґосподаруючого
мужицтва, а по містах прибуло промислового
пролєтаріяту, так, як після Жаффроа Сент-Улєра світової
матерії в одних місцях убуває, між тим коли в других
місцях вона згущаєсь. Помимо меньшого числа оброблюючих
рук ґрунт видавав прото однако або й ще більше
плодів, бо разом с переворотом в ґрунтових відносинах
власностевих настали також ліпші способи управи, більша
кооперація, зосередженє средств продукційних і т. д., а з другого
сільські наємники не тілько силувані були до тяжшої
праці — на се головно напирає сер Джемс Стеарт, —
а й обсяг їх домашної продукції, де вони працювали самі
на себе, чим раз більше вменьшувався. З освободженєм
одної части мужицтва освободжені зістали також єго давні
средства прожитку. Вони стают тепер матеріяльним складником
\so{змінного} капіталу\footnote*{
Звісно, що Маркс ділит капітал на постійний (constant) і змінний
(variabel), а то після того, чи в довшім протягу продукції вартість
єго зміняєся, чи ні. І так машини, сирий матеріял, будинки фабричні
і т. д., се капітал постійний, бо продукція не змінює в загальній сумі
єго вартости, а то, що убуде вартости на машинах і приладах і пр.,
котрі зуживаются при роботі, прибуває самим витворам, котрі через переробку
зискуют на вартости. Між тим друга часть капіталу, а іменно
тота, котра йде на наймленє і удержанє робітника і містится в понятю
робучої плати, се капітал змінний, бо по кождім процесі продукційнім
капіталіст добуває з него більше, ніж видав. Робітник витворює вартість
більшу, ніж тота, яку одержав в формі робучої плати. (Прим. перев.)
}. Бездомний та немаючий мужик
мусит окупувати собі ті средства прожитку від свого
нового пана, промислового капіталіста, в формі робучої
плати. Як зі средствами прожитку, так само сталося й з домашним
рільничим сирим матеріялом, котрого переробкою
займався промисл. Той сирий матеріял став частиною \so{постійного}
\index{franko}{0087}
капіталу. Се бачимо не тілько в Англії. За часів
Фрідріха II. бачимо н. пр., що часть вестфальських мужиків,
котрі всі прядут лен, — хоть ще не шовк, — насилу
вивласнено і прогнано з хат і ґрунтів, а прочу часть перемінено
в наймитів великих арендаторів. Рівночасно повстают
великі прядильні і ткальні льну, де „освободжені“ наймаются
на роботу. Лен виглядає так само, як виглядав уперед.
Ані одно волоконце в нім не змінилося, але нова соціяльна
душа вступила в єго тіло. Тепер він становит часть постійного
капіталу панів мануфактуристів. Давнійше розділений
між множество дрібних витвірців, котрі го самі управляли
і пряли, він тепер згромадився в руках одного капіталіста,
котрий других заставляв для себе прясти і ткати. Виложена
в прядильни надвишка праці становила давнійше надвишку
доходу незлічених родин мужицьких, або також — за часів
Фрідріха II, йшла на extra-податки pour le roi de Prusse.
Тепер вона становит зиск немногих капіталістів. Веретена
і ткацькі станки, давнійше розсіяні широко по краю, тепер
стовпилися в кількох великих касарнях робучих, так само
й робітники, так само й сирий матеріял. І веретена і ткацькі
станки і сирі матеріяли зі средств незалежного прожитку
для прядильників і ткачів від тепер переміеюются в средства
командованя над ними і висисаня з них бесплатної
праці. По великих мануфактурах не видно того так, як по
\linebreak[4]
\makebox[\linewidth]{\dotfill}

\begin{center}
\emph{[На цьому уривається збережений рукопис Франка]}
\end{center}

  \parcont{}  %% абзац починається на попередній сторінці
\index{i}{0151}  %% посилання на сторінку оригінального видання
зужитковує удвоє більше матеріялу, який має удвоє більшу
вартість, і зужитковує удвоє більше машин, що мають удвоє
більшу вартість, отже, зберігає в продукті двох тижнів удвоє
більше вартости, ніж у продукті одного тижня. За даних незмінних
умов продукції робітник зберігає то більше вартости, що
більше вартости він додає; але він зберігає більше вартости не
тому, що додає більше вартости, а тому, що додає її за незмінних
і незалежних від його власної праці умов.

Звичайно, в деякому відносному розумінні можна сказати,
що робітник завжди зберігає старі вартості в тій самій пропорції, в
якій він додає нову вартість. Чи піднесеться вартість бавовни з
1\shil{ шилінґа} на 2\shil{ шилінґи}, чи спаде на 6\pens{ пенсів}, робітник завжди
зберігає в продукті однієї години лише удвоє меншу вартість
бавовни, ніж у продукті двох годин, хоч би й як змінялася ця
вартість. Далі, коли змінюється продуктивність його власної праці,
коли вона підноситься або падає, то він, приміром, за одну робочу
годину випряде більше або менше бавовни, ніж раніше, і відповідно
до цього збереже більшу або меншу вартість бавовни у
продукті однієї робочої години. Але за всім тим він за дві робочі
години збереже удвоє більше вартости, ніж за одну робочу годину.

Вартість, залишаючи осторонь її суто символічний вираз у
знаках вартости, існує лише в якійсь споживній вартості, в якійсь
речі. (Сама людина, розглядувана просто лише як буття робочої
сили, є предмет природи, річ, хоч і жива, самосвідома річ, а сама
праця є речове виявлення цієї сили). Тому, коли гине споживна
вартість, то гине й вартість. Засоби ж продукції не втрачають
своєї вартости одночасно із своєю споживною вартістю, бо в наслідок
процесу праці вони втрачають первісну форму своєї споживної
вартости в дійсності лише на те, щоб у продукті набрати форми
іншої споживної вартости. Але, хоч і як важливо для вартости
існувати в якійсь споживній вартості, для неї, як це доводить
метаморфоза товарів, байдуже, в якій споживній вартості вона
існує. Звідси випливає, що в процесі праці вартість переходить
із засобу продукції на продукт лише тією мірою, якою засіб продукції
разом із своєю самостійною споживною вартістю втрачає
й свою мінову вартість. Він віддає продуктові лише ту вартість,
яку він втрачає як засіб продукції. Але з цього погляду зрізними
речовими факторами процесу праці справа стоїть неоднаково.

Вугілля, що ним опалюють машину, зникає безслідно, так
само мастиво, що ним мастять вісь колеса, і т. ін. Фарби й інші
допоміжні матеріяли зникають, але виявляються у властивостях
продукту. Сировинний матеріял становить субстанцію продукту,
але змінює свою форму. Отже, сировинний матеріял і допоміжні
матеріяли втрачають самостійну форму, в якій вони увійшли до
процесу праці як споживні вартості. Інша справа з власне засобами
праці. Інструмент, машина, фабричний будинок, бочка й
т. ін. служать у процесі праці лише доти, доки зберігають вони
свою первісну форму, доки й завтра можуть входити в процес
праці в тій самій формі, що й учора. Як за свого життя, тобто
\parbreak{}  %% абзац продовжується на наступній сторінці

\parcont{}  %% абзац починається на попередній сторінці
\index{iii1}{0152}  %% посилання на сторінку оригінального видання
Хоч і яке велике значення має вивчення таких тертів для кожної
спеціальної праці про заробітну плату, все ж при загальному
дослідженні капіталістичного виробництва їх треба залишити
осторонь як випадкові і неістотні. При такому загальному
дослідженні взагалі завжди припускається, що дійсні відносини
відповідають своєму поняттю, або, що є те саме, дійсні відносини
зображаються лиш остільки, оскільки вони виражають свій
власний загальний тип.

Ріжниця норм додаткової вартості в різних країнах, отже,
національні ріжниці в ступенях експлуатації праці, для даного
дослідження зовсім не мають значення. Адже в цьому відділі ми
саме хочемо показати, яким чином в межах даної країни утворюється
певна загальна норма зиску. Однак, ясно, що при порівнянні
різних національних норм зиску треба тільки зіставити розвинуте
нами раніше з тим, що ми маємо розвинути тут. Спочатку
треба розглянути ріжницю в національних нормах додаткової
вартості, а потім, на основі цих даних норм додаткової вартості,
порівняти ріжницю національних норм зиску. Оскільки ріжниця
цих останніх не є результатом ріжниці національних норм додаткової
вартості, вона мусить виникати з обставин, при яких,
як і в нашому дослідженні в цьому розділі, додаткова вартість
припускається повсюди однаковою, постійною.

В попередньому розділі було показано, що, коли норму додаткової
вартості припустити незмінною, норма зиску, яку дає певний
капітал, може підвищуватись чи падати в наслідок обставин,
які підвищують або знижують вартість тієї чи іншої частини
сталого капіталу і тому взагалі впливають на відношення між
сталою і змінною складовими частинами капіталу. Далі було
відзначено, що обставини, які здовжують або скорочують час
обороту капіталу, можуть справляти аналогічний вплив на
норму зиску. Через те що маса зиску тотожна з масою додаткової
вартості, з самою додатковою вартістю, то виявилось також,
що \emph{маса} зиску — відмінно від \emph{норми} зиску — не зачіпається
щойно згаданими коливаннями вартості. Вони модифікують тільки
норму, в якій виражається дана додаткова вартість, отже й зиск
даної величини, тобто модифікують його відносну величину,
його величину порівняно з величиною авансованого капіталу.
Оскільки в наслідок таких коливань вартості відбувається зв’язування
або звільнення капіталу, таким посереднім шляхом може
бути зачеплена не тільки норма зиску, але й самий зиск. Однак,
це завжди має силу тільки для капіталу, уже вкладеного, а не
для нових капіталовкладень; і, крім того, збільшення або зменшення
самого зиску завжди залежить від того, наскільки більше
чи менше праці в наслідок таких коливань вартості може бути
приведено в рух тим самим капіталом, отже, від того, наскільки
більшу чи меншу масу додаткової вартості може — при незмінній
нормі додаткової вартості — виробити той самий капітал.
Аж ніяк не суперечачи загальному законові і не становлячи винятку
\index{iii1}{0153}  %% посилання на сторінку оригінального видання
з нього, цей позірний виняток в дійсності був тільки
окремим випадком застосування загального закону.

Якщо в попередньому відділі виявилось, що, при незмінному
ступені експлуатації праці, із зміною вартості складових частин
сталого капіталу, а також із зміною часу обороту капіталу змінюється
норма зиску, то з цього само собою випливає, що
норми зиску різних одночасно існуючих, одна поряд одної, сфер
виробництва будуть різні, якщо при інших незмінних умовах
час обороту застосовуваних капіталів різний або якщо вартісне
відношення між органічними складовими частинами цих капіталів
у різних галузях виробництва є різне. Те, що ми раніш
розглядали як зміни, що відбуваються послідовно в часі
з тим самим капіталом, ми розглядаємо тепер як одночасно
наявні ріжниці між існуючими одно поряд одного капіталовкладеннями
в різних сферах виробництва.

При цьому нам доведеться дослідити: 1)~ріжницю в \emph{органічному
складі} капіталів, 2)~ріжницю в часі їх обороту.

В усьому цьому дослідженні, коли ми говоримо про склад
або оборот капіталу в певній галузі виробництва, ми завжди
маємо на увазі — припущення, яке само собою зрозуміле, — пересічні
нормальні відношення капіталу, вкладеного в цю галузь
виробництва; взагалі, мова йде про пересічні відношення сукупного
капіталу, вкладеного в дану сферу, а не про випадкові
ріжниці між окремими вкладеними в цю сферу капіталами.

Через те що, далі, припускається, що норма додаткової вартості
і робочий день є незмінні, і через те що це припущення
включає також і незмінність заробітної плати, то певна кількість
змінного капіталу виражає певну кількість приведеної
в рух робочої сили, а тому й певну кількість праці, яка упредметнюється.
Отже, якщо 100\pound{ фунтів стерлінгів} виражають тижневу
заробітну плату 100 робітників, тобто в дійсності 100 робочих
сил, то 100\pound{ фунтів стерлінгів} $× n$ виражають тижневу
заробітну плату $100 × n$ робітників, а $\frac{100\pound{ фунтів стерлінгів}}{n}$ тижневу
заробітну плату $\frac{100}{n}$ робітників. Отже, змінний капітал служить
тут (як і завжди при даній величині заробітної плати) показником
маси праці, яку приводить в рух весь капітал певної величини;
тому ріжниці у величині застосовуваного змінного капіталу
служать показниками ріжниці в масі вживаної робочої сили. Якщо
100\pound{ фунтів стерлінгів} представляють 100 робітників на тиждень
і, отже, при 60 годинах тижневої праці — 6000 робочих годин, то
200\pound{ фунтів стерлінгів} представляють \num{12000} робочих годин, а 50\pound{ фунтів стерлінгів} тільки 3000 робочих годин.

Під складом капіталу ми розуміємо, як це сказано вже у
книзі першій, відношення між його активною і пасивною складовою
частиною, між змінним і сталим капіталом. При цьому
треба розглянути два відношення, які мають неоднакову важливість,
\index{iii1}{0154}  %% посилання на сторінку оригінального видання
хоч при певних обставинах можуть спричиняти однаковий
вплив.

Перше відношення ґрунтується на технічній базі, і на певному
ступені розвитку продуктивної сили його треба розглядати
як дане. Потрібна певна маса робочої сили, представлена
певним числом робітників, щоб виробити певну масу продукту,
наприклад, протягом одного дня, і, отже — що при цьому само
собою розуміється — привести в рух, продуктивно спожити
певну масу засобів виробництва, машин, сировинних матеріалів
і~\abbr{т. д.} Певне число робітників припадає на певну кількість засобів
виробництва, отже певна кількість живої праці припадає
на певну кількість праці, вже упредметненої в засобах виробництва.
Це відношення дуже різне в різних сферах виробництва,
часто в різних галузях однієї й тієї ж промисловості, хоч, з другого
боку, випадково воно може бути цілком або приблизно
однаковим в дуже віддалених одна від одної галузях промисловості.
Це відношення становить технічний склад капіталу і є дійсна
основа його органічного складу.

Але можливо також, що це відношення однакове в різних
галузях промисловості, оскільки змінний капітал є простий показник
робочої сили, а сталий капітал — простий показник маси
засобів виробництва, приведеної в рух цією робочою силою.
Наприклад, певні роботи з міддю й залізом можуть вимагати
однакового відношення між робочою силою і масою засобів
виробництва. Але через те що мідь дорожча, ніж залізо, то вартісне
відношення між змінним і сталим капіталом в обох випадках
буде різне і разом з тим буде різний і вартісний склад
обох цілих капіталів. Ріжниця між технічним складом і вартісним
складом виявляється в кожній галузі промисловості
в тому, що при незмінному технічному складі вартісне відношення
обох частин капіталу може змінюватись, а при зміні
технічного складу вартісне відношення може лишатись незмінним;
останнє має місце, звичайно, тільки тоді, коли зміна
відношення між застосованою масою засобів виробництва і масою
робочої сили вирівнюється протилежною зміною їх вартостей.

Вартісний склад капіталу, оскільки він визначається його
технічним складом і відображає цей останній, ми звемо \emph{органічним}
складом капіталу\footnote{
Вищевикладене коротко було розвинуте уже в третьому виданні першої
книги, стор. 628. на початку розділу XXIII.~Через
те що в перших двох виданнях немає цього місця, повторення його тут мало
тим більше підстав. — \emph{Ф.~Е.}
}.

Отже, щодо змінного капіталу ми припускаємо, що він є показник
певної кількості робочої сили, певного числа робітників
або певної маси приводжуваної в рух живої праці. В попередньому
\index{iii1}{0155}  %% посилання на сторінку оригінального видання
відділі ми бачили, що зміна величини вартості змінного
капіталу іноді виражає не що інше, як більшу або меншу ціну
тієї самої маси праці; але тут, де норма додаткової вартості
і робочий день розглядаються як незмінні, а заробітна плата
за певний робочий час як величина дана, це відпадає. Навпаки,
ріжниця у величині сталого капіталу може, правда, бути також
показником зміни маси засобів виробництва, приведених в рух
певною кількістю робочої сили; але вона може також походити
з ріжниці у вартості засобів виробництва, приведених в рух
у певній сфері виробництва, порівняно з іншими сферами. Тим
то тут треба взяти до уваги обидві ці точки зору.

Нарешті, треба зробити ще таке істотне зауваження:

Припустім, що 100\pound{ фунтів стерлінгів} становлять тижневу
заробітну плату 100 робітників. Припустім, що тижневий робочий час дорівнює 60 годинам. Припустімо,
далі, що норма
додаткової вартості \deq{} 100\%. В цьому випадку робітники 30 годин з 60 працюють на себе самих, а 30
даром на капіталіста.
В 100\pound{ фунтах стерлінгів} заробітної плати в дійсності втілено
тільки 30 робочих годин 100 робітників, або разом 3000 робочих годин, тимчасом як інші 3000 годин,
які вони працюють,
втілені в 100\pound{ фунтах стерлінгів} додаткової вартості, відповідно — зиску, що його забирає собі
капіталіст. Тому, хоч заробітна плата в 100\pound{ фунтів стерлінгів} не виражає тієї вартості,
в якій упредметнюється тижнева праця 100 робітників, вона
все ж показує (бо довжина робочого дня і норма додаткової
вартості є дані), що цим капіталом приведено в рух 100 робітників на протязі загалом 6000 робочих
годин. Капітал в 100\pound{ фунтів стерлінгів} показує це, тому що він, поперше, показує
число приведених в рух робітників, бо 1\pound{ фунт стерлінгів} \deq{} 1 робітникові за тиждень, отже 100\pound{ фунтів
стерлінгів} \deq{} 100 робітникам; і, подруге, тому що кожний приведений
в рух робітник, при даній нормі додаткової вартості в 100\%,
виконує вдвоє більше праці, ніж міститься в його заробітній
платі, отже, 1\pound{ фунт стерлінгів}, його заробітна плата, що є виразом півтижневої праці, приводить в
рух працю цілого тижня,
і так само 100\pound{ фунтів стерлінгів}, хоч вони містять в собі тільки 50
тижнів праці, приводять в рух працю 100 робочих тижнів. Отже,
тут треба мати на увазі дуже істотну ріжницю між змінним капіталом, витраченим на заробітну плату,
оскільки його вартість,
сума заробітних плат, представляє певну кількість упредметненої праці, і цим капіталом, оскільки
його вартість є простий показник маси живої праці, яку він приводить в рух. Ця
остання завжди більша, ніж кількість праці, яка міститься
в змінному капіталі, і тому вона виражається також у вартості
більшій, ніж вартість змінного капіталу — у вартості, яка визначається, з одного боку, числом
приведених в рух змінним капіталом робітників, а з другого боку, кількістю виконуваної ними
додаткової праці.


\index{iii2}{0156}  %% посилання на сторінку оригінального видання
Загальна сума грошової ренти становила б якраз половину того, що було
в таблиці II, де додаткові капітали були вкладені за незмінних цін продукції.

Найважливіше є порівняти вищенаведені таблиці з таблицею І.

Ми бачимо, що з пониженням ціни продукції на половину, з 60 шил. до
30 шил. за квартер, загальна сума грошової ренти залишилась та сама = 18\pound{ ф.
ст.} і відповідно до цього збіжжева рента подвоїлась, саме зросла з 6 кварт. до
12 кварт. Рента з $В$ відпала; з $C$ грошова рента в ІVс збільшилась на половину,
але на половину зменшилась в ІVс; з $D$ вона лишилась та сама = 9\pound{ ф.
стерл.} у таблиці ІVс, і піднеслась з 9\pound{ ф. стерл.} до 15\pound{ ф. стерл.} у таблиції ІVd.
Продукц я піднеслась з 10 квартерів до 34 в ІVс, і до 30 квартер в в IVd;
зиск підвищився з 2\pound{ ф. стерл.} до 5\sfrac{1}{2} в ІVс і до 4\sfrac{1}{2} в IVd. Загальна сума
вкладеного капіталу зросла в одному випадку з 10\pound{ ф. стерл.} до 27\sfrac{1}{2}\pound{ ф. стерл.},
в другому — з 10 до 22\sfrac{1}{2}\pound{ ф. стерл.}; отже, обидва рази більше, ніж удвоє. Норма
ренти, рента, обчислена у відношенні до авансованого капіталу, в усіх таблицях
від IV до IVd для кожного роду землі всюди та сама, що вже було дано тим припущенням,
що норма продуктивности обох послідовних витрат капіталу на землях
усіх родів не змінюється. Проти таблиці І вона, проте, понизилась пересічно
щодо всіх родів землі і для кожного окремого роду землі. В таблиці І вона =
180\% пересічно, в таблиці ІVс вона$ = \frac{18}{27\sfrac{1}{2}} × 100 = 65\sfrac{5}{11}\%$ і
IVd = $\frac{18}{22\sfrac{1}{2}} × 100 = 80\%$. Пересічна грошова рента з акра підвищилась. Її пересічна
величина давніш в таблиці І була 4\sfrac{1}{2}\pound{ ф. стерл.} з акра для всіх 4 акрів,
а тепер у таблицях IVс і d вона дорівнює 6\pound{ ф. стерл.} з акра для 3 акрів.
Її пересічна величина для землі, що дає ренту, була раніш 6\pound{ ф. стерл.}, а тепер
зона дорівнює 9\pound{ ф. стерл.} з акра. Отже, грошова вартість ренти з акра підвищилась
і репрезентує тепер удвоє більше продукту в збіжжі, ніж давніш, але
12 квартерів збіжжевої ренти тепер становлять менше, ніж половину всього продукту
в 34, зглядно 30\footnote*{В німецькому тексті стоїть: усього «продукту в 33, зглядно 27 квартерів» Явна помилка,
як це можна бачити з таблиць ІVс і IVd. \emph{Прим. Ред.}} квартерів, тимчасом як у таблиці І 6 квартерів становлять
\sfrac{3}{5}  усього продукту в 10 квартерів. Отже, хоч рента, коли розглядати
її як відповідну частину всього продукту, а також коли обчислити її у відношенні
до витраченого капіталу, і знизилась, одначе її грошова вартість,
обчислена на акр. збільшилась, а її вартість в продукті, збільшилась ще дужче.
Коли ми візьмемо землю $D$ в таблиці IVd, то ціна продукції тут дорівнює
15\pound{ ф. стерл.}, що з них витрачений капітал = 12\sfrac{1}{2}\pound{ ф. стерл}. Грошова рента = 15\pound{ ф. стер}. У таблиці І на тій самій землі $D$ ціна продукції була 3\pound{ ф. стерл.}, витрачений
капітал = 2\sfrac{1}{2}\pound{ ф. стерл.}, грошова рента = 9\pound{ ф. стерл.}, отже, остання
утроє більша за ціну продукції й майже у чотири рази більша за витрачений
капітал. У таблиці IVd для $D$ грошова рента в 15\pound{ ф. стерл.} якраз дорівнює ціні
продукції і лише на \sfrac{1}{5}  більша за витрачений капітал. А все ж грошова рента
з акра на \sfrac{2}{3}  більша, 15\pound{ ф. стерл.} замість 9\pound{ ф. стерл}. В таблиці І збіжжева
рента в 3 квартери = \sfrac{3}{4}  усього продукту, що становить 4 квартери, в таблиці
IVd вона = 10 квартерам, половині всього продукту (20 квартерів) з акра
землі $D$. Це показує, що грошова і збіжжева рента з акра може зрости, хоч
вона і становить відносно меншу частину всього здобутку і знизилась у відношенні
до авансованого капіталу.

Вартість всього продукту в таблиці І = 30\pound{ ф. стерл.}; рента = 18\pound{ ф.
стерл.} більше від половини цієї вартости. Вартість усього продукту в таблиці
IV = 45\pound{ ф. стерл.}, що з них 18\pound{ ф. стерл.}, менш від половини, становлять
ренту.

\input{_0157.tex}
\parcont{}  %% абзац починається на попередній сторінці
\index{i}{0158}  %% посилання на сторінку оригінального видання
зміниться робочий час, суспільно-потрібний на продукцію товару, — а та сама кількість бавовни,
приміром, за несприятливого врожаю репрезентує більшу кількість праці, ніж за сприятливого,
— то це справляє відбитий вплив на старий товар, що його завжди рахується лише за окремий екземпляр
свого ґатунку,\footnote{«Всі продукти того самого роду становлять, власне, одну масу, ціну якої визначається загалом
незалежно від особливих обставин». («Toutes les productions d’un même gepre ne forment proprement
qu’une masse, dont le prix se détermine en général et sans égard aux circonstances particulières»),
(\emph{Le Trosne}: «De l’Intérêt Social». Physiocrates, éd. Daire,
Paris 1846, p. 893).
}
вартість якого завжди вимірюється суспільно-доконечною працею, тобто працею, завжди доконечною за
наявних суспільних умов.

Вартість засобів праці, що вже функціонують у процесі продукції,
машин і т. ін., отже, і та частина вартости, яку вони віддають продуктові, може змінюватися так
само, як вартість сировинного матеріялу. Коли, приміром, у наслідок якогось нового винаходу машина
того самого роду репродукується із зменшеною витратою праці, то старі машини більш або менш
зневартнюються
й тому переносять на продукт порівняно менше вартости. Але й тут зміна вартости постає поза тим
процесом продукції,
в якому машина функціонує як засіб продукції. В цьому процесі вона ніколи не віддає більше вартости,
ніж та, яку вона має незалежно від цього процесу.

Так само, як зміна вартости засобів продукції, хоч вона й справляє на них відбитий вплив навіть
після вступу їх до процесу продукції, не змінює їхнього характеру як сталого капіталу, так і зміна в
пропорції між сталим і змінним капіталом не порушує їхньої функціональної ріжниці. Приміром,
технічні умови процесу праці можуть змінитися так, що там, де раніш 10 робітників із 10 знаряддями
невеликої вартости обробляли порівняно малу кількість сировинного матеріялу, тепер 1 робітник
дорогою машиною обробляє в сто раз більшу кількість сировинного матеріялу. В цьому випадку сталий
капітал, тобто маса вартости застосованих
засобів продукції, дуже зросла б, а змінна частина капіталу, частина капіталу, авансована на робочу
силу, дуже спала б. Однак ця зміна переінакшує лише відношення між величиною сталого й змінного капіталу, або пропорцію,
в якій увесь капітал розпадається на сталі та змінні складові частини, але вона не
порушує самої ріжниці між сталим і змінним капіталом.

\section{Норма додаткової вартости}
\subsection{Ступінь експлуатації робочої сили}
Додаткова вартість, яку авансований капітал С створив у процесі продукції, або зростання авансованої
капітальної вартости С виступає перед нами насамперед як надлишок вартости
продукту понад суму вартости елементів його продукції.


\index{ii}{0159}  %% посилання на сторінку оригінального видання
Ця некритично запозичена в А.~Сміта плутанина заважає Рікардо
не тільки більше, ніж пізнішим апологетам — останнім плутанина понять
не тільки не заважає, а скорше допомагає — а й більше, ніж самому
А.~Смітові, бо Рікардо, дотримуючись на ділі езотеричного вчення
А.~Сміта проти екзотеричного А.~Сміта, протилежно до нього, послідовніше
і гостріше розвинув вчення про вартість і додаткову вартість.

У фізіократів немає й сліду цієї плутанини. Ріжниця між avances
annuelles і avances primitives стосується лише до різних періодів репродукції
різних складових частин капіталу, спеціяльно хліборобського капіталу,
тимчасом як їхні погляди на продукцію додаткової вартости становлять
незалежну від цих ріжниць частину їхньої теорії, а саме частину,
що її вони виставляють як основу теорії. Утворення додаткової вартости
пояснюється в них не з капіталу, як такого, а визнається як властивість
лише певної продукційної сфери капіталу — хліборобства.

2) Найпосутніше для визначення змінного капіталу — а тому й для
перетворення будь-якої суми вартости на капітал — в тому, що капіталіст
обмінює певну, дану (і в цьому розумінні сталу) величину вартости на
силу, яка творить вартість; певну кількість вартости обмінюється на продукцію
вартости, на процес її самозростання. Чи платить капіталіст робітникові
грішми, чи засобами існування, — це нічого не змінює в цьому
найпосутнішому визначенні. Від цього змінюється тільки спосіб існування
авансованої капіталістом вартости, яка в одному разі існує у формі грошей,
що на них робітник сам собі купує на ринку засоби свого існування,
а в другому разі — у формі засобів існування, що їх робітник
споживає безпосередньо. На ділі розвинена капіталістична продукція
припускає, що робітника оплачується грішми, як вона взагалі має собі
за передумову процес продукції, упосереднюваний процесом циркуляції,
тобто має за передумову грошове господарство. Але творення додаткової
вартости — і, значить, капіталізація авансованої суми вартости — не випливає
ні з грошової, ні з натуральної форми заробітної плати, або капіталу,
витраченого на закуп робочої сили. Воно випливає з обміну вартости
на вартостетворчу силу, — з перетворення сталої величини на змінну.

Більша або менша закріпленість засобів праці залежить від ступеня
їхньої довготривалости, тобто від фізичної властивости. Залежно від
ступеня довготривалости вони, за інших незмінних умов, зношуються
швидше або повільніше, отже, функціонують як основний капітал довший
або коротший час. Але вони функціонують як основний капітал зовсім
не в наслідок самої цієї, фізичної властивости — довготривалости. Сировинний
матеріял на металевих фабриках так само довготривалий, як і машини,
що його обробляють, і довготриваліший, ніж деякі складові частини цих
машин: шкіра, дерево тощо. А проте, металь, що служить як сировинний
матеріял, становить частину обігового капіталу, а засіб праці, що функціонує,
зроблений, може, з того самого металю, становить частину основного
капіталу. Отже, не в наслідок фізичної природи речовини, не
в наслідок більшої або меншої незнищуваности той самий металь одного
разу заводиться під рубрику основного, а другого — під рубрику обігового
\parbreak{}  %% абзац продовжується на наступній сторінці

\input{_0160.tex}
\parcont{}  %% абзац починається на попередній сторінці
\index{ii}{0161}  %% посилання на сторінку оригінального видання
капіталові як сталий капітал — це з погляду процесу зростання вартости.
Або, коли тут мова повинна бути про речову ріжницю, оскільки вона
впливає на процес циркуляції, то справа така: з природи вартости, яка є
не що інше, як зречевлена праця, і з природи діющої робочої сили, яка
є не що інше, як праця, що зречевлює себе, випливає, що робоча сила
протягом періоду її функціонування постійно утворює вартість і долярову
вартість; і що те, що на боці робочої сили виявляється як рух, як
утворення вартости, на боці її продукту виявляється у формі спокою,
як уже утворена вартість. Коли робоча сила вже діяла, то капітал не
складається вже більше з робочої сили на одному боці, із засобів продукції
на другому. Капітальна вартість, витрачена на робочу силу, є тепер
вартість, що її (+ додаткову вартість) долучено до продукту. Щоб
повторити процес, треба продати продукт і на вторговані гроші знову й
знову купувати робочу силу і вводити її в продуктивний капітал. Це
надає тоді частині капіталу, витраченій на робочу силу, так само, як і частинам
його, витраченим на матеріял праці тощо, характер обігового капіталу,
протилежно до того капіталу, що лишається закріплений у засобах праці.

Коли, навпаки, другорядне визначення обігового капіталу, спільне
йому з частиною сталого капіталу (сировинними й допоміжними матеріяламн)
— саме те визначення, що вартість, витрачену на обіговий капітал,
цілком переноситься на продукт, в продукції якого його зуживається, а
не поступінно й частинами, як в основного капіталу, що вартість ця,
отже, мусить цілком заміститися через продаж продукту, — перетворити
на посутню характеристику частини капіталу, витраченої на робочу силу,
то й частина капіталу, витрачена на заробітну плату, речово мусить
складатися не з діющої робочої сили, а з речових елементів, що їх робітник
купує на свою плату, отже, з частини суспільного товарового капіталу,
яка ввіходить у споживання робітника — з засобів існування.
Основний капітал складається при такому погляді на справу з засобів
праці, що зношуються повільніше, а тому й доводиться їх рідше відновлювати,
а капітал, витрачений на робочу силу, з засобів існування, що
їх треба заміщувати швидше.

Однак межі швидшої та повільнішої зношуваности стираються.

„Харч і одяг що їх зуживає робітник, будівлі, де він працює, знаряддя,
що допомагають йому в роботі, всі ці речі з своєї природи минущі.
Але є величезна ріжниця в часі, що протягом його зберігаються
ці різні капітали: парова машина зберігається довший час, ніж корабель,
корабель — довший час, ніж одяг робітника, одяг робітника знову таки
довший час, ніж харч, що його він споживає“\footnote{
The food and clothing consumed the labourer, the buildings in which he
works, the implements with which his labour is assisted, are all of a perishable
nature. There is, however, a vast difference in the time for which these different
capitals will endure: a steam-engine will last longer than a ship, a ship than the
clothing of the labourer, and the clothing of the labourer longer than the food which
he consumes“. (Ricardo, etc., p. 27).
}.

\parcont{}  %% абзац починається на попередній сторінці
\index{i}{0162}  %% посилання на сторінку оригінального видання
щоб могла увібрати в себе ту кількість праці, що має витрачатися протягом процесу продукції. Коли цю
масу дано, то, чи піднесеться її вартість, чи впаде, або й зовсім не матиме вартости, як от земля й
море, процесу творення вартости й зміни вартости це ані трохи не порушує\footnote{
Примітка до другого видання. Само собою зрозуміло, що, як каже Лукрецій, «nil posse creari de
nihilo», з нічого не можна створити нічого. «Створення вартости» є перетворення робочої сили на
працю. З свого боку й робоча сила є насамперед речовина природи, перетворена на людський організм.
}.

Отже, ми насамперед припускаємо, що стала частина капіталу дорівнює нулеві. Тим то авансований
капітал зводиться з $с \dplus{} v$ на $v$, а вартість продукту
$с \tikz[na] \coordinate (c-162-node-1);
\dplus{}
\tikz[na] \coordinate (v-162-node-1); v
\dplus{} m$
\begin{tikzpicture}[overlay]
    \path[-,black,transform canvas={yshift=1.5mm}] (c-162-node-1) edge [out=30, in=150] (v-162-node-1);
\end{tikzpicture}%
на новоспродуковану вартість $v \dplus{} m$.
Коли припустити, що новоспродукована вартість дорівнює 180\pound{ фунтам стерлінґів}, в яких виражається
праця, що триває протягом цілого процесу продукції, то нам треба відняти вартість змінного капіталу,
що дорівнює 90\pound{ фунтам стерлінґів}, щоб знайти
додаткову вартість, яка дорівнює 90\pound{ фунтам стерлінґів}. Число 90\pound{ фунтів стерлінґів} \deq{} $m$ виражає тут
абсолютну величину випродукованої додаткової вартости. Але її відносна величина, тобто пропорція, в
якій змінний капітал зріс своєю вартістю, визначається, очевидно, відношенням додаткової вартости до
змінного капіталу, або виражається дробом \frac{m}{v}. Отже ж, у наведеному випадку вона виражається у $\smash{\frac{90}{90}} \deq{} 100\%$.
Це відносне зростання
змінного капіталу, або відносну величину додаткової вартости, я називаю нормою додаткової
вартости\footnote{
Так само, як англійці кажуть «rate of profits», «rate of interest»\footnote*{
«норма зиску», «норма процента». \emph{Ред.}
} і~\abbr{т. д.} у книзі III ми
побачимо, що норму зиску легко зрозуміти, якщо тільки знати закони додаткової вартости. Протилежним
шляхом не можна зрозуміти ni l’un ni l’autre\footnote*{
ні того, ні другого. \emph{Ред.}
}.
}.

Ми вже бачили, що робітник протягом однієї частини процесу праці продукує лише вартість своєї
робочої сили, тобто вартість доконечних для нього засобів існування. А що він продукує за відносин,
які ґрунтуються на суспільному поділі праці, то він продукує свої засоби існування не безпосередньо,
а у формі якогось окремого товару, приміром, пряжі, тобто продукує вартість, рівну вартості його
засобів існування, або грошам, за які
він ці засоби купує. Та частина його робочого дня, яку він уживає на це, буде більша або менша,
залежно від вартости його пересічних щоденних засобів існування, отже, залежно від пересічного
робочого часу, щоденно потрібного на їхню продукцію. Коли вартість його щоденних засобів існування
репрезентує пересічно
6 упредметнених робочих годин, то робітник мусить працювати пересічно по 6 годин денно, щоб
спродукувати цю вартість. Коли б він працював не на капіталіста, а на себе самого, незалежно,
\parbreak{}  %% абзац продовжується на наступній сторінці


\begin{table}[H]
  \centering
  \caption*{Таблиця VIa}
  \footnotesize

  \settowidth\rotheadsize{\theadfont Продажна}
  \begin{tabular}{l c r c r c c c c}
    \toprule
      \thead[t]{Земля} &
        &
      \thead[t]{Капітал} &
      \rothead{Зиск} &
      \thead[tc]{Продукт\\з акра} &  % в квартерах
      \rothead{Продажна\\ціна} &
      \rothead{Здобуток} &
      \multicolumn{2}{c}{Рента} \\

     \cmidrule(rl){2-9}
       & акри  & \poundsign{} & \poundsign{} & кв. & \poundsign{} & \poundsign{} & кв. & \poundsign{} \\

      \midrule
       A & 1 & 2\tbfrac{1}{2} \dplus{} 2\tbfrac{1}{2} \deq{} 5 & 1 & 1 \dplus{} \phantom{0}3\phantom{\tbfrac{1}{2}} \deq{} \phantom{0}4\phantom{\tbfrac{1}{2}}   & 1\tbfrac{1}{2} & \phantom{0}6\phantom{\tbfrac{3}{4}} & \phantom{0}0\phantom{\tbfrac{1}{2}}  & \phantom{0}0\phantom{\tbfrac{1}{2}} \\
       B & 1 & 2\tbfrac{1}{2} \dplus{} 2\tbfrac{1}{2} \deq{} 5 & 1 & 2 \dplus{} \phantom{0}2\tbfrac{1}{2} \deq{} \phantom{0}4\tbfrac{1}{2}                       & 1\tbfrac{1}{2} & \phantom{0}6\tbfrac{3}{4}           & \phantom{00}\tbfrac{1}{2}                            & \phantom{00}\tbfrac{3}{4}           \\
       C & 1 & 2\tbfrac{1}{2} \dplus{} 2\tbfrac{1}{2} \deq{} 5 & 1 & 3 \dplus{} \phantom{0}5\phantom{\tbfrac{1}{2}} \deq{} \phantom{0}8\phantom{\tbfrac{1}{2}}   & 1\tbfrac{1}{2} & 12\phantom{\tbfrac{3}{4}}           & \phantom{0}4\phantom{\tbfrac{1}{2}}                  & \phantom{0}6\phantom{\tbfrac{1}{2}} \\
       D & 1 & 2\tbfrac{1}{2} \dplus{} 2\tbfrac{1}{2} \deq{} 5 & 1 & 4 \dplus{} 12\phantom{\tbfrac{1}{2}} \deq{} 16\phantom{\tbfrac{1}{2}}                       & 1\tbfrac{1}{2} & 24\phantom{\tbfrac{3}{4}}           & 12\phantom{\tbfrac{1}{2}}                            & 18\phantom{\tbfrac{1}{2}}           \\
    
      \midrule
      Разом & 4 & \phantom{2\tbfrac{1}{2} \dplus{} 2\tbfrac{1}{2} \deq{}}20 & & \phantom{2 \dplus{} 12\tbfrac{1}{2} \deq{}}32\tbfrac{1}{2} & & & 16\tbfrac{1}{2} & 24\tbfrac{3}{4}\\
 \end{tabular}
\end{table}
% REMOVED
% \footnotemarkZ{}
% \footnotetextZ{В німецькому тексті тут очевидно помилково стоїть «6» \Red{Прим. Ред.}} % текст примітки прямо під заголовком

\noindent{}Нарешті, грошова рента підвищилася б, коли б у кращі земельні дільниці,
при тому самому відносному підвищенні родючости, вкладено було більше
додаткового капіталу, ніж у землю $А$, або коли б додаткові вкладання капіталу в кращі
земельні дільниці впливали, підвищуючи норму продуктивности. В обох випадках
ріжниці зростали б.

Грошова рента понижується, коли поліпшення, що сталося в наслідок
додаткової витрати капіталу, зменшує всі ріжниці, або частину їх, впливаючи
більше на $А$, ніж на $В$ і $C$. Вона понижується то більше, що незначніше
підвищення продуктивности кращих земельних дільниць. Від відносної неоднаковости
впливу залежить, чи підвищиться збіжжева рента, чи понизиться або
залишиться без зміни.

Грошова рента підвищується, а також і збіжжева рента, або тоді, коли за
незмінної відносної ріжниці в додатковій родючості різних земель більше вкладається
додаткового капіталу в землю, що дає ренту, ніж у землю $А$, що не дає
ренти, і більше у землю, що дає вищу, ніж у землю, що дає нижчу ренту;
абож тоді, коли родючість, при однаковому додатковому капіталі, більше зростає
на кращій і найкращій землі, ніж на землі $А$, причому грошова і збіжжева
рента підвищується саме у такому відношенні, в якому це збільшення родючости
на вищих розрядах землі вище, ніж на нижчих.

Але за всяких обставин рента відносно підвищується, коли підвищена продуктивність
є наслідок додаткової витрати капіталу, а не просто наслідок
збільшеної родючости за незмінної витрати капіталу. Це є абсолютний погляд,
який показує, що тут, як і в усіх давніших випадках, рента і збільшена рента з акра
(подібно до того, як при диференційній ренті І висота пересічної ренти на всю
оброблювану площу) є наслідок збільшеної витрати капіталу на землю, при
чому байдуже, чи функціонує ця витрата з сталою нормою продуктивности за
сталих або понижених цін, чи з низхідною нормою продуктивности за сталих або
за понижених цін, чи з висхідною нормою продуктивности за понижених цін.
Бо наше припущення: стала ціна за сталої, низхідної або висхідної норми продуктивности додаткового
капіталу, і низхідна ціна, за сталої, низхідної і висхідної
норми продуктивности, зводиться ось до чого: стала норма продуктивности додаткового капіталу при
сталій або низхідній ціні, низхідна норма продуктивности
при сталій або низхідній ціні, висхідна норма продуктивности за сталої
\parbreak{}  %% абзац продовжується на наступній сторінці


\index{iii1}{0164}  %% посилання на сторінку оригінального видання
Якщо ми капітали І — V знову розглядатимемо як єдиний сукупний капітал, то побачимо, що і в цьому
випадку склад суми
п’яти капіталів = 500 = 390 c + 110 v, отже, пересічний склад, = 78 c + 22 v, лишається той самий;
так само й пересічна додаткова вартість = 22\%. Розподіливши цю додаткову вартість рівномірно між
І—V, ми одержали б такі товарні ціни:

Капітали
Додаткова вартість
Вартість товарів
Витрати виробництва
Ціна товарів
Норма зиску
Відхилення ціни від вартості

І. 80 c + 20 v    20    90    70    92    22\%    + 2
II. 70 c + 30 v   30   111   81   103   22\%  — 8
III. 60 c + 40 v  40   131   91   113   22\% — 18
IV. 85 c + 15 v   15    70    55    77    22\%    + 7
V. 95 c + 5 v        5     20    15    37    22\%  + 17

В загальній сумі товари продаються на 2 + 7 + 17 = 26 вище і
на 8 + 18 = 26 нижче вартості, так що відхилення цін взаємно
знищуються в наслідок рівномірного розподілу додаткової вартості, тобто в наслідок додання
пересічного зиску в 22 на
100 одиниць авансованого капіталу до відповідних витрат виробництва товарів І—V; в тому самому
відношенні, в якому одна
частина товарів продається вище, друга продається нижче її
вартості. І тільки продаж їх по таких цінах уможливлює те, що
норма зиску для І—V є однакова, 22\%, не зважаючи на різний
органічний склад капіталів І—V. Ціни, які виникають таким чином, що з різних норм зиску різних сфер
виробництва береться
пересічна і ця пересічна додається до витрат виробництва в різних сферах виробництва, — такі ціни є
ціни виробництва. Передумовою їх є існування однієї загальної норми зиску, а ця
остання знов таки передбачає, що норми зиску в кожній окремій сфері виробництва, взяті самі по собі,
вже зведені до
відповідної кількості пересічних норм. Ці окремі норми зиску в кожній сфері виробництва = m/K, і їх
треба виводити з вартості товару, як це і було зроблено в першому відділі цієї книги. Без такого
виведення загальна норма зиску (а тому й ціна виробництва товару) була б безглуздим і ірраціональним
уявленням. Отже, ціна виробництва товару дорівнює витратам його
виробництва плюс доданий до них зиск, обчислений у процентах
відповідно до загальної норми зиску, тобто дорівнює витратам
виробництва товару плюс пересічний зиск.

В наслідок різного органічного складу капіталів, вкладених
у різні галузі виробництва, а тому в наслідок тієї обставини,
що — залежно від різного процентного відношення змінної частини до всього капіталу даної величини —
рівновеликими капіталами приводяться в рух дуже різні кількості праці, ними привласнюються також
дуже різні кількості додаткової праці, або
виробляються дуже різні маси додаткової вартості. Відповідно
до цього норми зиску, які панують в різних галузях виробництва,
\index{iii1}{0165}  %% посилання на сторінку оригінального видання
первісно є дуже різні. Ці різні норми зиску за допомогою конкуренції вирівнюються в загальну
норму зиску, яка
є пересічною всіх цих різних норм зиску. Зиск, який відповідно
до цієї загальної норми зиску припадає на капітал даної величини, який би не був його органічний
склад, зветься пересічним
зиском. Ціна товару, яка дорівнює витратам його виробництва
плюс та частина річного пересічного зиску на застосований для
виробництва товару (а не тільки на спожитий для його виробництва) капітал, яка припадає на товар
залежно від умов його
обороту, є його ціна виробництва. Візьмімо, наприклад, капітал
в 500, в тому числі 100 основного капіталу, з якого зношується
10\% протягом одного періоду обороту обігового капіталу в 400.
Припустімо, що пересічний зиск протягом цього періоду обороту становить 10\%. Тоді витрати
виробництва виготовленого
протягом цього обороту продукту будуть: 10 с на зношування
плюс 400 (c + v) обігового капіталу = 410, а його ціна виробництва: 410 витрати виробництва плюс
(10\% зиску на 500) 50 = 460.

Тому, хоч капіталісти різних сфер виробництва при продажу
своїх товарів повертають собі капітальні вартості, спожиті на
виробництво цих товарів, але реалізують вони не ту додаткову
вартість, отже, і не той зиск, що виробляється в їх власній
сфері при виробництві цих товарів, а лише стільки додаткової вартості, отже й зиску, скільки при
рівному розподілі
припадає на кожну відповідну частину всього капіталу суспільства з усієї додаткової вартості або
всього зиску, який виробляється сукупним капіталом суспільства за даний період часу
в усіх сферах виробництва, взятих разом. Кожен авансований
капітал, який би не був його склад, одержує кожного року або
за якийсь інший період часу стільки зиску на кожні 100, скільки
його за цей період часу припадає на кожні 100 як певну частину
сукупного капіталу. Оскільки справа стосується зиску, різні капіталісти відносяться тут один до
одного, як прості акціонери
одного акційного товариства, в якому зиск розподіляється між
ними рівномірно на кожні 100 одиниць, і тому зиски для різних
капіталістів відрізняються тільки залежно від величини капіталу,
вкладеного кожним з них у спільне підприємство, залежно від
відносного розміру участі кожного в спільному підприємстві,
залежно від числа акцій кожного з них. Отже, тимчасом як та
частина цієї товарної ціни, яка заміщає спожиті на виробництво
товарів частини вартості капіталу і за яку, отже, знову мусять
бути куплені ці спожиті капітальні вартості, — тимчасом як ця
частина, яка становить витрати виробництва, цілком визначається
видатками в межах відповідних сфер виробництва, — друга складова частина товарної ціни, зиск,
доданий до цих витрат виробництва, визначається не масою зиску, виробленою цим певним
капіталом у цій певній сфері виробництва протягом даного
часу, а тією масою зиску, яка за даний період часу пересічно припадає на кожний застосований капітал
як певну частину
\index{iii1}{0166}  %% посилання на сторінку оригінального видання
сукупного суспільного капіталу, вкладеного в сукупне
виробництво.\footnote{
Cherbuliez [„Riche ou Pauvre“, Paris-Genève 1840, стор. 116 і далі].
}

Отже, якщо капіталіст продає свій товар по його ціні виробництва, то він повертає собі кількість
грошей, відповідну величині вартості спожитого ним у виробництві капіталу, і добуває зиск
пропорціонально до його авансованого капіталу, просто
як до певної частини сукупного суспільного капіталу. Витрати
виробництва в кожній сфері виробництва мають специфічний
характер. Доданий до цих витрат виробництва зиск не залежить від його окремої сфери виробництва, він
є проста пересічна
на кожні 100 авансованого капіталу.

Припустімо, що п’ять різних капіталів І—V у вищенаведеному прикладі належать одній людині. Кількість
змінного і сталого капіталу, спожита на виробництво товарів у кожному окремому підрозділі І—V на
кожні 100 застосованого капіталу,
є дана; ця частина вартості товарів І—V, само собою зрозуміло, становитиме частину їх ціни, бо
принаймні ця ціна потрібна для заміщення авансованої і спожитої частини капіталу. Отже, ці витрати
виробництва були б різні для кожного роду
товарів І—V і, як такі, вони були б по-різному фіксовані їх власником. Що ж до різних мас додаткової
вартості або зиску, вироблених у підрозділах І—V, то капіталіст мав би всі підстави вважати їх за
зиск на весь свій авансований капітал, так що
на кожні 100 одиниць капіталу припадала б певна відповідна
частина. Отже, витрати виробництва товарів, вироблених в окремих підрозділах І—V, були б різні; але
в усіх цих товарів
була б однаковою частина продажної ціни, яка походить з доданого до витрат виробництва зиску на
кожні 100 одиниць капіталу. Отже, сукупна ціна товарів І—V дорівнювала б їх сукупній вартості, тобто
дорівнювала б сумі витрат виробництва
І—V плюс сума додаткової вартості, або зиску, вироблена в
І—V; отже, в дійсності ця ціна була б грошовим виразом сукупної кількості минулої і новододаної
праці, вміщеної в товарах
І—V. І таким чином, у самому суспільстві — якщо розглядати
всі галузі виробництва в їх сукупності — сума цін виробництва
вироблених товарів дорівнює сумі їх вартостей.

Цьому твердженню, здається, суперечить той факт, що в
капіталістичному виробництві елементи продуктивного капіталу
звичайно купуються на ринку, отже, ціни їх містять у собі вже
реалізований зиск, тобто ціну виробництва певної галузі промисловості разом з уміщеним в ній зиском,
так що зиск однієї
галузі промисловості входить у витрати виробництва іншої.
Але якщо ми підрахуємо на одному боці суму витрат виробництва товарів цілої країни, а на другому —
суму її зиску або додаткової вартості, то, очевидно, матимемо правильний обрахунок. Візьмімо,
наприклад, товар А; нехай витрати його
\parbreak{}  %% абзац продовжується на наступній сторінці

\input{_0167.tex}
\parcont{}  %% абзац починається на попередній сторінці
\index{iii1}{0168}  %% посилання на сторінку оригінального видання
сама дорівнює витратам виробництва плюс додаткова вартість,
отже, в даному разі дорівнює витратам виробництва плюс зиск,
а цей зиск знов таки може бути більший або менший, ніж додаткова вартість, місце якої він заступає.
Щодо змінного капіталу, то хоч пересічна денна заробітна плата завжди дорівнює вартості, виробленій
за те число годин яке робітник
мусить працювати, щоб виробити необхідні засоби існування,
однак саме число цих годин знов таки фальсифікується в наслідок того, що ціни виробництва необхідних
засобів існування
відхиляються від їх вартостей. Однак, це завжди розв’язується
таким чином, що наскільки в один товар входить більше додаткової вартості, настільки її в другий
товар входить менше,
і тому ті відхилення від вартості, які містяться в цінах виробництва товарів, взаємно знищуються.
Взагалі в цілому капіталістичному виробництві загальний закон здійснюється завжди
тільки як панівна тенденція, дуже заплутаним і приблизним
способом, тільки як якась пересічна вічних коливань, яка ніколи
не може бути точно встановлена.

Через те що загальна норма зиску утворюється з пересічної
різних норм зиску на кожні 100 авансованого капіталу за певний
період часу, скажімо, за рік, то в ній стирається також ріжниця,
викликана ріжницею в часі оборотів різних капіталів. Але ці
ріжниці є визначальним фактором для тих різних норм зиску
різних сфер виробництва, що з їх пересічної утворюється загальна норма зиску.

В попередній ілюстрації утворення загальної норми зиску
кожний капітал у кожній сфері виробництва припускався = 100,
і це було зроблено саме для того, щоб з’ясувати процентну
ріжницю в нормах зиску, а тому й ріжницю у вартостях товарів,
вироблюваних рівновеликими капіталами. Але само собою зрозуміло: дійсні маси додаткової вартості,
створювані в кожній
окремій сфері виробництва, залежать від величини застосованих
капіталів, бо в кожній такій даній сфері виробництва склад капіталу є даний. Тимчасом особлива \emph{норма}
зиску кожної окремої
сфери виробництва не змінюється від того, чи застосовується
капітал в 100, $100 × m$ чи $100 × xm$. Норма зиску однаково лишається 10\% — чи становить весь зиск $10 :
100$, чи $1000 : 10000$.

Але через те що норми зиску в різних сферах виробництва
є різні, бо в них залежно від відношення змінного капіталу до
всього капіталу виробляються дуже різні маси додаткової вартості, отже й зиску, то очевидно, що
пересічний зиск на кожні
100 суспільного капіталу, отже, пересічна норма зиску або загальна норма зиску, буде дуже різна
залежно від відповідної
величини капіталів, вкладених у різні сфери виробництва. Візьмімо чотири капітали $А$, $В$, $C$, $D$. Нехай
норма додаткової вартості для всіх них буде = 100\%. Нехай на кожні 100 сукупного
капіталу змінного капіталу буде для $А = 25$, для $B = 40$, для
$C = 15$, для $D = 10$. На кожні 100 сукупного капіталу тоді припадало
\index{iii1}{0169}  %% посилання на сторінку оригінального видання
б додаткової вартості або зиску у $А = 25$, $B = 40$, $C = 15$, $D = 10$; разом $= 90$; отже, якщо всі
чотири капітали є рівновеликі, пересічна норма зиску є \frac{90}{4} = 22\sfrac{1}{2}\%.

Але якщо загальні величини цих капіталів будуть: $А = 200$, $B = 300$, $C = 1000$, $D = 4000$, то вироблені
зиски будуть відповідно 50, 120, 150 і 400. Разом на 5500 капіталу зиску буде 720, або пересічна
норма зиску буде 13\sfrac{1}{11}\%.

Маси всієї виробленої вартості є різні залежно від різних загальних величин відповідних капіталів,
авансованих в $А$, $В$, $C$, $D$.
Тому при утворенні загальної норми зиску справа йде не тільки
про ріжницю \emph{норм} зиску в різних сферах виробництва, з яких
просто треба було б вивести пересічну, але й про відносну вагу,
з якою ці різні норми зиску входять в утворення пересічної.
Але це залежить від відносної величини капіталу, вкладеного
в кожну окрему сферу виробництва, тобто від того, яку частину
сукупного суспільного капіталу становить капітал, вкладений
в кожну окрему сферу виробництва. Дуже велика ріжниця мусить, звичайно, бути залежно від того, чи
більша чи менша
частина сукупного капіталу дає вищу або нижчу норму зиску.
Але це знов таки залежить від того, скільки капіталу вкладено
в ті сфери виробництва, в яких відношення змінного капіталу
до всього капіталу є високе або низьке. Тут справа стоїть цілком
так само, як з пересічним процентом, що його одержує лихвар,
який віддає в позику різні капітали за різні норми процента, наприклад, за 4, 5, 6, 7\% і т. д.
Пересічна норма цілком залежить
від того, скільки з свого капіталу він позичив за кожну з цих
різних норм процента.

Отже, загальна норма зиску визначається двома факторами:

1) органічним складом капіталів у різних сферах виробництва,
отже, різними нормами зиску в окремих сферах;

2) розподілом сукупного суспільного капіталу між цими різними сферами, отже, відносною величиною
капіталу, вкладеного
в кожну окрему сферу, і отже з окремою нормою зиску, тобто
відносною масою сукупного суспільного капіталу, яку поглинає
кожна окрема сфера виробництва.

В I і II книгах ми мали справу тільки з \emph{вартостями} товарів.
Тепер, з одного боку, відокремились \emph{витрати виробництва}, як
частина цієї вартості, з другого боку, розвинулась \emph{ціна виробництва} товару, як перетворена форма
вартості товару.

Припустім, що склад пересічного суспільного капіталу є
$80 c + 20 v$, а норма річної додаткової вартості $m' = 100\%$; тоді
річний пересічний зиск для капіталу в 100 буде $= 20$, а загальна
річна норма зиску $= 20\%$. Хоч би які були $k$, витрати виробництва
товарів, вироблених за рік капіталом в 100, їх ціна виробництва була б $= k + 20$. В сферах
виробництва, де склад капіталу $= (80 — x) c + (20 x) v$, дійсно створена додаткова вартість,
відповідно річний зиск, вироблений у цій сфер і виробництва,
\index{iii1}{0170}  %% посилання на сторінку оригінального видання
був би $= 20 + x$, отже, більший ніж 20, і вироблена
товарна вартість була б $= k + 20 + x$, більша ніж $k + 20$, або
більша, ніж ціна виробництва. У сферах виробництва, в яких
склад капіталу $(80 + x) c + (20 — x) v$, створювана протягом року
додаткова вартість, або зиск, була б $= 20 — x$, отже, менша, ніж 20,
а тому товарна вартість $k + 20 — x$ була б менша, ніж ціна виробництва, яка $= k + 20$. Якщо залишити
осторонь можливі ріжниці в часі оборотів, то ціна виробництва товарів дорівнювала б їх вартості
тільки в тих сферах, в яких склад капіталу випадково був би $= 80 c + 20 v$.

В кожній окремій сфері виробництва специфічний розвиток
суспільної продуктивної сили праці є різний щодо ступеня,
вищий чи нижчий, відповідно до того, наскільки велика є кількість засобів виробництва, що їх
приводить в рух певна кількість праці, тобто, при даному робочому дні, певне число робітників; отже,
він вищий чи нижчий, відповідно до того, наскільки
мала є кількість праці, потрібна для певної кількості засобів
виробництва. Тому капітали, які містять у собі більший процент
сталого, отже, менший процент змінного капіталу, ніж пересічний суспільний капітал, ми звемо
капіталами \emph{вищого} складу.
Навпаки, такі капітали, в яких сталий капітал займає відносно
менше, а змінний відносно більше місце, ніж у пересічному
суспільному капіталі, ми звемо капіталами \emph{нижчого} складу.
Нарешті, ми звемо капіталами пересічного складу такі капітали,
склад яких збігається з складом пересічного суспільного капіталу. Якщо пересічний суспільний капітал
в процентах складається з $80 c + 20 v$, то капітал $90 c + 10 v$ стоїть \emph{вище}, а капітал $70 c + 30 v$
\emph{нижче}, ніж пересічний суспільний. Взагалі, при
складі пересічного суспільного капіталу, рівному $mc + nv$, де
$m$ і $n$ є сталі величини і $m + n = 100$, $(m + x) c + (n — x) v$ репрезентує вищий, а $(m — x) c + (n + x)
v$ — нижчий склад окремого капіталу або групи капіталів. Як функціонують ці капітали після
встановлення пересічної норми зиску, — припускаючи, що
вони обертаються один раз за рік, — це показує нижченаведена
таблиця, в якій I представляє пересічний склад, і тому пересічна норма зиску = 20\%.

\begin{center}
\begin{tabular}{c c c c}

\toprule
Капітал & Норма зиску & Ціна продукту & Вартість \\
\midrule

\phantom{II}I. $80 c + 20 v + 20 m$ & 20\% & 120 &  120 \\

\phantom{I}II.    $90 c + 10 v + 10 m$ & 20\% & 120 & 110\\

III. $70 c + 30 v + 30 m$ & 20\% & 120 & 130 \\
\end{tabular}
\end{center}
Отже, для товарів, вироблених капіталом II, їхня вартість була б
менша, ніж їхня ціна виробництва, для товарів капіталу III ціна
виробництва була б менша, ніж вартість, і тільки для капіталів I,
\parbreak{}  %% абзац продовжується на наступній сторінці

\input{_0171.tex}

Формула, згідно з якою ціна виробництва товару $= k \dplus{} p$,
дорівнює витратам виробництва плюс зиск, визначилась тепер
ближче таким чином, що $p \deq{} kp'$ (де $p'$ є загальна норма зиску),
і, отже, ціна виробництва $= k \dplus{} kp'$. Якщо $k \deq{} 300$, а $p' \deq{} 15\%$,
то ціна виробництва $k \dplus{} kp' \deq{} 300 \dplus{} 300$. $\frac{15}{100} \deq{} 345$.

Ціна виробництва товарів у кожній окремій сфері виробництва може змінювати свою величину:

1)~при незмінній вартості товарів (тобто при умові, що у виробництво товару після зміни ціни
виробництва входить та сама
кількість мертвої і живої праці, як і до зміни) в наслідок незалежної від даної окремої сфери зміни
загальної норми зиску;

2)~при незмінній загальній нормі зиску в наслідок зміни вартості — чи то в самій даній сфері
виробництва, в результаті
технічних змін, чи в наслідок зміни вартості тих товарів, які
входять у сталий капітал цієї сфери як його складові елементи;

3)~нарешті, в наслідок спільного впливу обох цих обставин.
Не зважаючи на великі зміни, які постійно — як це виявиться
далі — відбуваються у фактичних нормах зиску окремих сфер
виробництва, дійсна зміна в загальній нормі зиску, оскільки
вона викликається не винятковими, надзвичайними економічними
подіями, є дуже пізній результат ряду коливань, які охоплюють
дуже довгі періоди часу, тобто коливань, що потребують багато часу, поки вони сконсолідуються і
вирівняються у зміну
загальної норми зиску. Тому при всіх коротших періодах (цілком незалежно від коливань ринкових цін)
зміну цін виробництва
треба завжди пояснювати prima facie [очевидно] дійсною зміною
вартості товарів, тобто зміною всієї суми робочого часу, потрібного для їх виробництва. Проста зміна
грошового виразу тих самих вартостей тут, само собою зрозуміло, зовсім не береться до уваги\footnote{
\emph{Corbett} [„An Inquiry into the Causes and Modes of the Wealth of Individuals“.
London 1841], стор. [33 і далі] 174.
}.

З другого боку, очевидно, що коли розглядати сукупний
суспільний капітал, то сума вартості вироблених ним товарів
(або, в грошовому виразі, їхня ціна), \deq{} вартості сталого капіталу \dplus{} вартість змінного капіталу \dplus{}
додаткова вартість. Якщо припустити, що ступінь експлуатації праці є незмінний, то норма зиску
при незмінній масі додаткової вартості може змінюватись тут
тільки в тому випадку, коли вартість сталого капіталу змінюється,
або коли вартість змінного капіталу змінюється, абож коли змінюються обидві ці вартості, так що
змінюється $K$, а тому й $\frac{m}{K}$, загальна норма зиску. Отже, в кожному випадку зміна загальної
норми зиску передбачає зміну вартості товарів, що входять як
складові елементи в сталий капітал, або в змінний капітал, або
одночасно в той і в другий.

\parcont{}  %% абзац починається на попередній сторінці
\index{ii}{0173}  %% посилання на сторінку оригінального видання
швидше припливає назад в грошовій формі еквівалент зношеної її частини.
Інша справа з обіговим капіталом. Не тільки капітал треба вкладати на
довший час відповідно до протягу робочого періоду, але треба також
повсякчас авансувати новий капітал на заробітну плату, сировинні та
допоміжні матеріяли. Отже, уповільнений зворотний приплив впливає
неоднаково на основний і обіговий капітал. Хоч буде зворотний приплив
повільніший, хоч швидший, основний капітал і далі діє. Навпаки, обіговий
капітал при уповільненому зворотному припливі стає нездатний до функціонування,
якщо його закріплено в формі непроданого або неготового,
ще непридатного до продажу продукту, і якщо немає наявного додаткового
капіталу, щоб відновити його in natura. — „Тимчасом як селянин
голодує, худоба його росте й гладшає. Було досить дощів, і паша стала
буйна. Індійський селянин умре з голоду біля свого жирного бика. Приписи
забобонів суворі проти поодиноких людей, але вони підтримують
суспільство; зберігання робочої худоби забезпечує поступ хліборобства,
а тим самим і джерела майбутніх засобів існування й майбутнього багатства.
Можливо, це звучить жорстоко й сумно, але це так: в Індії легше
замінити людину, ніж бика“. (Return, East India. Madras and Orissa
Famine. № 4, p. 4). Порівняйте з цим таке речення Манара-Дарма-Сестри,
розділ X, стор. 862: „Жертва життям без нагороди, щоб зберегти
життя жерцеві або корові\dots{} може забезпечити блаженство цих
родів низького походження“.

Звичайно неможливо подати на ринок п’ятилітню тварину, раніше
ніж їй буде п’ять років. Але в певних межах можна, змінюючи догляд
за худобою, підгодувати її протягом коротшого часу до її призначення.
Саме це й зробив Беквел. Раніше англійські вівці, як і французькі ще
1855 року, не були готові на заріз до четвертого або п’ятого року. За
системою Беквела, вівцю можна відгодувати протягом одного року і в
усякому разі вона цілком достигає до двох років. Старанно добираючи вівці,
Беквел, фармер з Дішлей Ґренджа, довів кістяк овець до мінімуму, потрібного
для їхнього існування. Ці його вівці зветься ньюлейстерські.
„Вівчар може тепер подати на ринок три вівці за той самий час, за який
раніше давав одну, і ці його вівці товщі, кругліші й розвиненіші
в тих частинах, що дають найбільше м’яса. Майже ціла вага їхня є чисте
м’ясо.“ (Lavergne, The Rural Economy of England etc. 1855, p. 20)

Методи, що скорочують робочий період, в різних галузях продукції
можна застосовувати дуже неоднаковою мірою, й вони не вирівнюють
ріжниці в часі різних робочих періодів. Щоб залишитись при нашому
прикладі, хай через застосування нових робочих машин абсолютно скорочується
робочий період, потрібний на виготовлення одного паровоза.
Але коли в наслідок удосконалення процесу прядіння кількість щоденно
й щотижнево вироблюваного готового продукту збільшиться ще швидше,
ніж у машинобудівництві, то відносно, порівняно з прядінням, довжина
робочого періоду в машинобудівництві збільшиться.

\parcont{}  %% абзац починається на попередній сторінці
\index{i}{0174}  %% посилання на сторінку оригінального видання
знайомства з вами в кращому світі. Addio!..\footnote{
Однак пан професор мав деяку користь із своєї подорожі до Менчестеру.
В «Letters on the Factory Act» увесь чистий прибуток, «зиск» і «процент» і
навіть «something more»,\footnote*{
— щось більше. \emph{Ред}.
} залежить від
однієї неоплаченої
робочої години робітника! Роком раніш у своїх «Outlines of Political Economy»,
складених для насолоди оксфордських студентів і освічених філістерів, Сеніор,
полемізуючи проти Рікардового визначення вартости робочим часом, «відкрив», що
зиск постає з праці капіталіста, а процент з його
аскетичности, з його «поздержливости». Сама побрехенька була стара» але слово
«поздержливість» («Abstinenz») було нове. Пан Рошер правильно переклав його
німецькою мовою словом «Enthaltung» («поздержливість»). А його компатріоти,
менше биті в латині, Вірти, Шульци
й інші Міхелі, переклали його на чорнече «самовідречення» («Entsagung»).
} Сиґнал «останньої години», що її винайшов Сеніор 1836 р., наново протрубив
був 15 квітня 1848 р. в «London Economist» Джемc Вілсон, один з головних
мандаринів економічної науки, у своїй полеміці проти
закону про десятигодинний робочий день.
\subsection{4. Додатковий продукт}

Ту частину продукту (\sfrac{1}{10} від 20 фунтів пряжі, або 2 фунти пряжі, у
прикладі §2), яка репрезентує додаткову вартість, ми
називаємо додатковим продуктом (surplus produce, produit net). Як норму
додаткової вартости визначає відношення додаткової
вартости не до цілої суми капіталу, а лише до його змінної складової частини,
так і рівень додаткового продукту визначає відношення останнього не до решти
цілого продукту, а до тієї частини його, яка репрезентує доконечну працю.
Як продукція додаткової вартости є визначальна мета
капіталістичної продукції, так і ступінь багатства вимірюється не абсолютною
величиною продукту, а відносною величиною додаткового продукту.\footnote{
«Для індивіда, що має капітал у 20.000 фунтів стерлінґів, і що його зиски
становлять 2.000 фунтів стерлінґів на рік, було б цілком байдуже, чи його
капітал вживає 100 чи 1.000 робітників, чи випродуковані товари продається
за 10.000 фунтів стерлінґів чи за 20.000 фунтів стерлінґів, аби лише
його зиски в усіх цих випадках не падали нижче як 2.000 фунтів
стерлінґів. Хіба реальний інтерес націй не такий самий? Коли припустити, що
реальний чистий прибуток нації, її ренти й зиски лишаються однакові, то не має
найменшої ваги, чи нація складається з 10 чи
12 мільйонів людности». (\emph{Ricardo}: «The Principles of Political Economy»,
3 rd. ed, London 1821, p. 416). Задовго перед Рікардом Артур Юнґ, фанатик
додаткового продукту, взагалі язикатий, неспроможний
на будь-яку критику письменник, що його слава стоїть у зворотному відношенні
до його заслуг, сказав, між іншим: «Що за користь була б для сучасного
королівства з якоїсь цілої провінції, що в ній землю обробляли б на
староримський лад дрібні незалежні селяни, про мене хоч би
й як і найкраще? Яка мета була б у цьому, крім одним-однієї мети продукувати
людей («the mere purpose of breeding men»), а це саме по собі не має
ніякої мети» (is a most useless purpose»). (\emph{Arthur Young}: «Political
Arithmetic etc.», 1774, p. 47).
Додаток до примітки 34. Дивний є «великий нахил малювати чистий прибуток
корисним для робітничої кляси\dots{} та проте ясно, що це стається не через те,
що він чистий» («the strong inclination to
represent net wealth as beneficial to the labouring class\dots{} though it
is evidently not on account
of being net»). (\emph{Th. Hopkins}: «On Rent of Land etс.», London 1823, p. 126).
}

\input{_0175.tex}
\input{_0176.tex}
\parcont{}  %% абзац починається на попередній сторінці
\index{iii1}{0177}  %% посилання на сторінку оригінального видання
у величині витрат виробництва окремого товару, що її викликала припущена нами
зміна вартості\footnote{
\emph{Corbett} [„An Inquiry etc.“, стор. 20].
}.

Щодо змінного капіталу, — а це найважливіше, тому що він
є джерело додаткової вартості і тому що все те, що приховує
його відношення до збагачення капіталіста, обгортає таємницею
всю систему, — то справа в очах капіталіста має грубо-спрощений вигляд, а саме:
нехай, наприклад, змінний капітал в 100\pound{ фунтів стерлінгів} являє собою тижневу
заробітну плату 100 робітників. Якщо ці 100 робітників при даному робочому дні
виробляють за тиждень продукт в 200 штук товару, $= 200 T$, то
$1 T$, — якщо абстрагуватись від тієї частини витрат виробництва,
яку додає сталий капітал, — коштує 10\shil{ шилінгів}, бо 100\pound{ фунтів стерлінгів} $= 200 T$,
$1 T \deq{} \frac{\text{100\pound{ фунтів стерлінгів}}}{200} \deq{} 10\text{ шилінгам}$.
Припустімо тепер, що відбувається зміна в продуктивній силі\footnote*{
В першому німецькому виданні тут стоїть: „у виробничій силі“ (in der
Produktionskraft); тут, а також і далі, виправлено на підставі рукопису Маркса.
\Red{Примітка ред. нім. вид. ІМЕЛ.}
} праці,
що вона подвоюється, що, отже, те саме число робітників виробляє двічі по $200 T$ за той самий час, за
який воно раніш
виробляло $200 T$. В цьому випадку $1 T$ коштує (оскільки витрати виробництва складаються з самої тільки
заробітної плати)
5\shil{ шилінгів}, бо тепер 100\pound{ фунтів стерлінгів} $= 400 T$,
$1 T \deq{} \frac{\text{100\pound{ фунтів стерлінгів}}}{400} \deq{} 5\text{ шилінгам}$.
Коли б продуктивна сила
зменшилась удвоє, то та сама праця виробляла б тільки $\frac{200 T}{2}$; і через те що 100\pound{ фунтів стерлінгів}
$=\frac{200 T}{2}$, то тепер
$1 T \deq{} \frac{\text{200\pound{ фунтів стерлінгів}}}{200} \deq{} 1\text{ фунтові стерлінгів}$. Зміни в робочому
часі, потрібному для виробництва товарів, а тому і в вартості
товарів, виступають тепер щодо витрат виробництва, а тому
й щодо цін виробництва, як інший розподіл тієї самої заробітної
плати на більшу або меншу кількість товарів, залежно від того,
більше чи менше товарів виробляється протягом того самого
робочого часу за ту саму заробітну плату. Капіталіст, отже
й політико-економ, бачить тільки те, що частина оплаченої праці,
яка припадає на штуку товару, змінюється із зміною продуктивності праці, і
що разом з тим змінюється і вартість кожної
окремої штуки; він не бачить, що те саме має місце також і з неоплаченою працею, яка міститься в
кожній штуці товару, і тим
менше може це побачити, що пересічний зиск в дійсності тільки
випадково визначається неоплаченою працею, поглиненою в його
власній сфері виробництва. Тільки в такій грубій і ірраціональній
формі виступає тепер той факт, що вартість товарів визначається вміщеною в них працею.


\section{%
Вирівнення загальної норми зиску через конкуренцію.
Ринкові ціни і ринкові вартості. Надзиск}
\chaptermark{Вирівнення загальної норми зиску через конкуренцію}%

Частина сфер виробництва має середній або пересічний склад
застосовуваного в них капіталу, тобто склад капіталу, який цілком чи приблизно збігається з складом
пересічного суспільного
капіталу.

Ціна виробництва товарів, вироблюваних у цих сферах виробництва, цілком чи приблизно збігається з їх
вартістю, вираженою
в грошах. І коли б ніяким іншим способом не можна було досягти математичної границі, то цього можна
було б досягти цим
способом. Конкуренція так розподіляє суспільний капітал між
різними сферами виробництва, що ціни виробництва в кожній
сфері утворюються на зразок цін виробництва в цих сферах
середнього складу, тобто \deq{} $k \dplus{} kp'$ (витрати виробництва плюс
добуток пересічної норми зиску і витрат виробництва). Але
ця пересічна норма зиску є не що інше, як обчислений в процентах зиск у сфері виробництва середнього
складу, де, отже,
зиск збігається з додатковою вартістю. Отже, норма зиску в усіх
сферах виробництва є одна й та ж, а саме вирівнена до норми
зиску цих середніх сфер виробництва, в яких панує пересічний
склад капіталу. Тому сума зисків усіх різних сфер виробництва
мусить дорівнювати сумі додаткових вартостей і сума цін виробництва сукупного суспільного продукту
мусить дорівнювати
сумі його вартостей. Але ясно, що це вирівнювання між сферами виробництва з різним складом завжди
мусить прагнути
урівняти ці сфери з сферами середнього складу, однаково, чи
ці останні точно чи тільки приблизно відповідають пересічному
суспільному складові. У сферах виробництва, які більш чи менш
наближаються до середньої, знову таки має місце тенденція
до вирівнення, яка прагне до ідеального, тобто в дійсності не
наявного середнього рівня, тобто тенденція до вирівнення
навколо нього, як норми. Таким чином у цьому відношенні
необхідно панує тенденція зробити ціни виробництва просто
перетвореними формами вартості, або перетворити зиски в
прості частини додаткової вартості, які, однак, розподіляються
не пропорційно до додаткової вартості, створеної в кожній
окремій сфері виробництва, а пропорційно до маси капіталу,
застосовуваного в кожній сфері виробництва, так що на рівновеликі маси капіталу, хоч би який був їх
склад, припадають відповідно рівновеликі частини сукупної додаткової вартості, створеної сукупним
суспільним капіталом.

Отже, для капіталів середнього чи приблизно середнього
складу ціна виробництва збігається цілком або приблизно з вартістю,
\index{iii1}{0179}  %% посилання на сторінку оригінального видання
а зиск — із створеною ними додатковою вартістю. Всі інші
капітали, хоч би який був їх склад, під тисненням конкуренції
прагнуть зрівнятися з капіталами середнього або приблизно
середнього складу. Але через те що капітали середнього складу
є рівні або приблизно рівні пересічному суспільному капіталові,
то всі капітали, яка б не була величина створеної ними самими
додаткової вартості, прагнуть замість цієї додаткової вартості
реалізувати в цінах своїх товарів пересічний зиск, тобто прагнуть реалізувати ціни виробництва.

З другого боку, можна сказати, що повсюди, де встановлюється пересічний зиск, отже загальна норма
зиску — яким би
шляхом не досягався цей результат, — цей пересічний зиск не
може бути нічим іншим, як зиском на пересічний суспільний
капітал, зиском, сума якого дорівнює сумі додаткових вартостей,
а ціни, які утворюються в наслідок надбавки цього пересічного
зиску до витрат виробництва, не можуть бути нічим іншим, як
перетвореними в ціни виробництва вартостями. Справа ні трохи
не змінилася б, коли б капітали в певних сферах виробництва
з будь-яких причин не підлягали цьому процесові вирівнення.
Тоді пересічний зиск обчислювався б на ту частину суспільного
капіталу, яка входить у процес вирівнення. Очевидно, що пересічний зиск не може бути нічим іншим, як
сукупною масою
додаткової вартості, розподіленою в кожній сфері виробництва
між масами капіталів пропорційно до їхніх величин. Це — сума
реалізованої неоплаченої праці, і вся ця маса праці, так само як
і оплачена, мертва й жива праця, виражається в сукупній масі,
товарів і грошей, яка припадає капіталістам.

Справжня трудність питання тут ось у чому: як відбувається
це вирівнення зисків у загальну норму зиску, раз воно, очевидно, є результат і не може бути вихідним
пунктом.

Насамперед, очевидно, що оцінка товарних вартостей, наприклад, у грошах, може бути тільки
результатом обміну їх і що,
припускаючи таку оцінку, ми повинні розглядати її як результат
дійсного обміну товарної вартості на товарну вартість. Але
яким же чином може здійснитись цей обмін товарів по їх дійсних вартостях?

Припустімо, спочатку, що всі товари в різних сферах виробництва продаються по їх дійсних вартостях.
Що сталося б
тоді? Згідно з вищевикладеним, в різних сферах виробництва
тоді панували б дуже різні норми зиску. Чи продаються товари
по їх вартостях (тобто чи обмінюються вони один на один пропорційно до вміщеної в них вартості, по
цінах їх вартості),
чи продаються вони по таких цінах, що продаж їх дає рівновеликі зиски на рівновеликі маси капіталів,
авансованих на відповідне виробництво їх, — це prima facie [очевидно] цілком різні речі.

Та обставина, що капітали, які приводять в рух неоднакову
кількість живої праці, виробляють неоднакову кількість додаткової
\index{iii1}{0180}  %% посилання на сторінку оригінального видання
вартості, передбачає, принаймні до певної міри, що ступінь
експлуатації праці або норма додаткової вартості однакова, або
що існуючі в цьому відношенні ріжниці вирівнюються за допомогою
дійсних або уявних (умовних) компенсуючих причин.
Це передбачає конкуренцію між робітниками і вирівнювання
ступеня їх експлуатації в наслідок постійного переходу їх
з однієї сфери виробництва до іншої. Така загальна норма додаткової
вартості — як тенденція, подібно до всіх економічних
законів, — припускається нами як теоретичне спрощення; але
в дійсності вона є фактична передумова капіталістичного способу
виробництва, хоч вона й гальмується в більшій чи меншій
мірі практичними тертями, які викликають більш чи менш
значні місцеві ріжниці, такі є, наприклад, закони про осілість
(settlement laws) для землеробських поденників в Англії. Але
в теорії припускається, що закони капіталістичного способу
виробництва розвиваються в чистому вигляді. В дійсності існує
завжди тільки наближення; однак, це наближення тим більше,
чим більше розвинений капіталістичний спосіб виробництва і чим
більше усунене його забарвлення рештками попередніх економічних
становищ і переплетення з ними.

Вся трудність постає з того, що товари обмінюються не
просто як \emph{товари}, а як \emph{продукти капіталів}, які претендують
на пропорціональну до їх величини або, при рівній величині, на
рівну участь у сукупній масі додаткової вартості. І сукупна
ціна товарів, вироблених даним капіталом за даний період часу,
повинна задовольнити цю вимогу. Але сукупна ціна цих товарів
є просто сума цін окремих товарів, які становлять продукт
капіталу.

Punctum saliens [вирішальний пункт] виступить найбільше, якщо
ми підійдемо до справи так: Припустім, що самі робітники
володіють своїми відповідними засобами виробництва і обмінюють
свої товари один з одним. Ці товари не були б тоді
продуктами капіталу. Залежно від технічної природи їх робіт,
вартість засобів праці і матеріалів праці, застосовуваних у різних
галузях праці, була б різна; так само, незалежно від неоднакової
вартості застосовуваних засобів виробництва, потрібна була б
різна маса цих засобів виробництва для даної маси праці, залежно
від того, що один певний товар може бути виготовлений
за одну годину, а інший тільки за день і~\abbr{т. д.} Припустімо
далі, що ці робітники пересічно працюють однакову кількість
часу, враховуючи вирівнення, які випливають з різної інтенсивності
праці та ін. Двоє робітників замістили б тоді в товарах,
що становлять продукт їх денної праці, поперше, свої
видатки, витрати (die Kostpreise) на спожиті засоби виробництва.
Ці останні були б різні залежно від технічної природи їх галузей
праці. Подруге, вони обидва створили б однакові кількості
нової вартості, а саме робочий день, доданий ними до засобів
виробництва. Ця нова вартість містила б у собі їх заробітну
\parbreak{}  %% абзац продовжується на наступній сторінці

\parcont{}  %% абзац починається на попередній сторінці
\index{iii1}{0181}  %% посилання на сторінку оригінального видання
плату плюс додаткову вартість, додаткову працю понад їх необхідні
потреби, при чому, однак, результати її належали б їм
самим. Висловлюючись капіталістичною мовою, обидва робітники
одержують рівну заробітну плату плюс рівний зиск, але разом
з тим і вартість, виражену, наприклад, у продукті десятигодинного
робочого дня. Але, поперше, вартості їх товарів були б
різні. Нехай, наприклад, з уміщеної в товарі І вартості на спожиті
засоби виробництва припадає більша частина вартості, ніж у товарі
II, і — щоб урахувати тут усі можливі ріжниці — припустімо,
що товар І вбирає більше живої праці, отже, потребує довшого
робочого часу для свого виготовлення, ніж товар II.~Отже, вартість
цих товарів І і II дуже різна. Так само різні й суми товарних вартостей,
які є продуктом праці, виконаної за даний час робітником
І і робітником II.~Норми зиску теж дуже різні для І і II,
якщо ми назвемо тут нормою зиску відношення додаткової вартості
до всієї вартості витрачених засобів виробництва. Засоби
існування, які щодня споживаються робітниками І і II протягом
виробництва і які представляють заробітну плату, становлять
тут ту частину авансованих засобів виробництва, яку ми в інших
випадках звемо змінним капіталом. Але додаткові вартості за
однаковий робочий час були б для І і II однакові, або ще точніше:
через те що І і II одержують кожний вартість продукту
одного робочого дня, вони одержують — якщо відрахувати вартість,
авансованих „сталих“ елементів — однакові вартості, одну
частину яких можна розглядати як заміщення спожитих у виробництві
засобів споживання, а другу — як додаткову вартість,
яка лишається понад це. Якщо І зробив більше витрат, то вони
заміщаються більшою частиною вартості його товару, яка заміщає
цю „сталу“ частину, і тому він повинен також більшу
частину всієї вартості свого продукту перетворити знову в речові
елементи цієї сталої частини, тимчасом як II, якщо він
менше одержав як заміщення, повинен зате настільки ж менше
знову перетворити в елементи сталої частини. Отже, при цьому
припущенні ріжниця в нормах зиску була б байдужою обставиною,
цілком так само, як нині для найманого робітника байдуже,
в якій нормі зиску виражається видушена з нього кількість
додаткової вартості, і цілком так само, як у міжнародній торгівлі
ріжниця норм зиску у різних націй є байдужа обставина
для їх товарообміну.

Отже, для обміну товарів по їх вартостях, або приблизно
по їх вартостях, потрібен значно нижчий ступінь, ніж для обміну
по цінах виробництва, для якого потрібна певна висота капіталістичного
розвитку.

Яким би чином не встановлювались або реґулювались первісно
ціни різних товарів одного відносно одного, закон вартості
керує їх рухом. Де зменшується робочий час, потрібний для
виробництва товарів, там падають і ціни; де він збільшується,
там підвищуються, при інших незмінних умовах, і ціни.

\input{_0182.tex}
\parcont{}  %% абзац починається на попередній сторінці
\index{iii2}{0183}  %% посилання на сторінку оригінального видання
підвищення диференційної ренти, але саме існування диференційної ренти як
ренти є разом з тим причина ранішого й швидшого підвищення загальної ціни
продукції, щоб таким чином забезпечити збільшене подання продукту, яке стало доконечним.

Треба зауважити далі таке:

Через додаткове капіталовкладення в землю $В$ не могла б підвищитися
регуляційна ціна до 4\pound{ ф. стерл.}, як це наведено вище, коли б земля $А$, в наслідок
другої витрати капіталу, давала додаткову продукцію дешевше від 4\pound{ ф.
стерл.}, або коли б вступила в конкуренцію нова гірша, ніж $А$, земля, що на
ній ціна продукції була б хоч і вища 3, але нижча 4\pound{ ф. стерл}. Таким чином,
ми бачимо, як диференційна рента І і диференційна рента II, тимчасом як
перша є за базу для другої, одночасно правлять одна для однієї за межу, що
спричинює то послідовні витрати капіталу на тій самій земельній дільниці, то
витрати капіталу одну біля однієї на новій додатковій землі. Так само вони
обмежують одна одну і в інших випадках, коли, наприклад, черга доходить до
кращих земель.

\section{Диференційна рента і з найгіршої з оброблюваних земель}

Припустімо, що попит на збіжжя підвищується, і що подання може бути
задоволене лише через послідовні витрати капіталу з недостатньою продуктивністю
на землях, що дають ренту, або через додаткову витрату капіталу
теж з низхідною продуктивністю на землі $А$, або через витрату капіталу на
нових землях гіршої якости, ніж $А$.

Візьмімо землю $В$ як представницю земель, що дають ренту.

Щоб уможливити додаткову продукцію 1 квартера на землі $В$ (який
тут може становити 1 мільйон квартерів, як кожен акр — мільйон акрів), додаткове
капіталовкладення вимагає підвищення ринкової ціни понад 3\pound{ ф. стерл.}
за квартер, що були до цього часу за регуляційну ціну. На землях $C$ і $D$ і~\abbr{т.
ін.} родах землі з найвищою рентою, теж може бути випродуковано додатковий
продукт, але лише з низхідною додатковою продуктивною силою; проте,
припускається, що 1 квартер землі $В$ потрібен для задоволення попиту.
Коли цей один квартер можна дешевше випродукувати з допомогою додаткового
капіталу на $В$, ніж з допомогою рівної витрати додаткового капіталу
на $А$, або спускаючись до землі $А_{-1}$, яка може випродукувати квартер, наприклад,
лише за 4\pound{ ф. стерл.}, тимчасом як додатковий капітал на $А$ міг би випродукувати
квартер уже за 3\sfrac{3}{4}\pound{ ф. стерл.}, то додатковий капітал, витрачений на
$В$, почав би регулювати ринкову ціну.

Земля $А$, як і давніш, випродукувала 1 квартер за 3\pound{ ф. стерл.} $В$ теж,
як і давніш, випродукувала в цілому 3\sfrac{1}{2} квартера, що їхня індивідуальна ціна
продукції становить разом 6\pound{ ф. стерл}. Тепер, коли б на землі $В$ потрібно було
додаткової витрати в 4\pound{ ф. стерл.} ціни продукції (включаючи і зиск), щоб випродукувати ще 1 квартер,
тимчасом як на $А$ його можна випродукувати за
3\sfrac{3}{4}\pound{ ф. стерл.}, то, зрозуміла річ, він був би випродукований на $А$, а не на $В$.
Отже, припустімо, що він може бути випродукований на $В$ з 3\sfrac{1}{2}\pound{ ф. стерл.}
додаткової ціни продукції. В цьому випадку 3\sfrac{1}{2}\pound{ ф. стерл.} були б регуляційною ціною
для всієї продукції. Тоді $В$ продав би свій теперішній продукт в 4\sfrac{1}{2}  квартери за
15\sfrac{3}{4}\pound{ ф. стерл}. З цього на ціну продукції перших 3\sfrac{1}{2}  квартерів припадає
6\pound{ ф. стерл.} і на останній квартер 3\sfrac{1}{2}\pound{ ф. стерл.}, разом 9\sfrac{1}{2}\pound{ ф. стерл}. Лишається
надзиск, для ренти \deq{} 6\sfrac{1}{4}\pound{ ф. стерл}, проти лише 4\sfrac{1}{2}\pound{ ф. стерл.} колишніх.
В цьому випадку акр землі $А$ також дав би ренту в  \sfrac{1}{2}\pound{ ф. стерл.};
\parbreak{}  %% абзац продовжується на наступній сторінці

\input{_0184.tex}
\parcont{}  %% абзац починається на попередній сторінці
\index{iii2}{0185}  %% посилання на сторінку оригінального видання
з тією самою витратою капіталу, перетворюється на надпродукт, який репрезентує
надзиск, а тому й ренту. Коли припустити, що норма зиску лишається
та сама, то орендар міг би купити на свій зиск меншу кількість збіжжя.
Норма зиску може лишитись та сама, коли заробітна плата не підвищиться,
або тому, що її понижено до фізичного мінімуму, отже, нижче нормальної вартости
робочої сили; або тому, що інші речі споживання робітників, давані мануфактурою,
стали порівняно дешевші; або тому, що робочий день став довший
або зробився інтенсивніший, і тому норма зиску в нехліборобських галузях
продукції, яка проте, реґулює хліборобський зиск, лишилась незмінна, якщо
тільки не підвищилась; абож тому, що хоч у хліборобстві й витрачається такий
самий капітал, але більш сталого і менше змінного.

Ми тут розглянули перший спосіб, у який може постати рента на землі
$А$, що до цього часу була найгірша, без того, щоб притягалось до оброблення
ще гіршу землю; а саме, коли вона постає в наслідок ріжниці індивідуальної
ціни продукції на цій землі, — ціни продукції, що до цього часу була за
реґуляційну проти тієї нової, вищої ціни продукції, по якій останній додатковий
капітал, витрачений з недостатною продуктивною силою на кращій землі,
дав потрібну додаткову кількість продукту.

Коли додаткова продукція мусила б постачатись землею $А_{-1}$, яка може дати
квартер лише за 4\pound{ ф. стерл.}, то рента з акра на $А$ підвищилася б до 1\pound{ ф. стерл}. Але в цьому випадку
земля $А_{-1}$ пересунулася б на місце $А$, як
найгірша з культивованих земель, а земля $А$ вступила б як нижчий член в
ряд родів землі, що дають ренту. Диференційна рента I змінилася б. Отже,
цей випадок лежить поза аналізою диференційної ренти II, яка виникає з різної
продуктивности послідовних витрат капіталу на тій самій дільниці землі.

Але, крім того, диференційна рента на землі $А$, може постати ще двояким
способом:

Коли за незмінної ціни, — будь-якої ціни, хоч би вона і була знижена
проти колишньої, — додаткова витрата капіталу породжує додаткову продуктивність,
що prima facie до певної межі завжди мусить статися якраз на найгіршій
землі.

\looseness=-1
Подруге, тоді, коли навпаки, продуктивність послідовних витрат капіталу
на землі $А$ понижується.

\looseness=-1
В обох випадках припускається, що стан попиту потребує збільшення
продукції.

\looseness=-1
Але, з погляду диференційної ренти, тут виступає специфічна трудність
в зв’язку з раніш викладеним законом, що за ним визначальною для всієї продукції
(або для всієї витрати капіталу) завжди є індивідуальна пересічна ціна
продукції одного квартера. Але для землі $А$, у протилежність кращим родам
землі, ціна продукції, яка обмежує для нових витрат капіталу вирівняння індивідуальної
з загальною ціною продукції, дана не поза нею. Бо індивідуальна ціна
продукції на $А$ і є та сама загальна ціна продукції, що реґулює ринкову ціну.

Припустімо:

1)~За висхідної продуктивної сили послідовних витрат
капіталу на одному акрі землі $А$, з авансованим капіталом в 5\pound{ ф. стерл.},
відповідно 6\pound{ ф. стерл.} ціни продукції, можна випродукувати замість 2 квартерів
3 квартери. Перша витрата капіталу в 2\sfrac{1}{2}\pound{ ф. стерл.} дає 1 квартер, друга — 2 квартери. В цьому
випадку 6\pound{ ф. стерл.} ціни продукції дають 3 квартери,
отже, квартер коштуватиме пересічно 2\pound{ ф. стерл.}; отже, коли 3 квартери
будуть продані по 2\pound{ ф. стерл.}, то $А$, як і давніш, не дасть ренти, але зміниться
лише основа диференційної ренти II; за реґуляційну ціну продукції стали
2\pound{ ф. стерл.} замість 3\pound{ ф. стерл.}; на найгіршій землі капітал в 2\sfrac{1}{2}\pound{ ф. стерл.}
продукує тепер пересічно 1\sfrac{1}{2}  замість 1 квартера, і це тепер офіційна родючість
\parbreak{}  %% абзац продовжується на наступній сторінці

\parcont{}  %% абзац починається на попередній сторінці
\index{i}{0186}  %% посилання на сторінку оригінального видання
законом час, ви вкладали б мені до кишені річно \num{1.000}\pound{ фунтів
стерлінґів}»\footnote{
Там же, стор. 48.
}. «Атоми часу є елементи баришу»\footnote{
«Moments are the element of profit». («Reports of the Insp. etc.
30 th April 1860», p. 56).
}.

З цього боку немає нічого характеристичнішого, як назва
«\textenglish{full times}»\footnote*{
повний час. \emph{Ред.}
} для робітників, що працюють повний час, і «half
times»\footnote*{
половина часу. \emph{Ред.}
} для дітей до тринадцятилітнього віку, яким дозволяється
працювати лише по 6 годин\footnote{
Цей вислів має офіціальне право громадянства так на фабриці,
як і у фабричних звітах.
}. Робітник є тут не що більше,
як персоніфікований робочий час. Усі індивідуальні ріжниці
сходять на ріжницю між «Vollzeitler» і «Halbzeitler»\footnote*{
робітником повночасним і робітником півчасним. \emph{Ред.}
}.

\subsection{Галузі англійської промисловости без законодавчих меж
експлуатації}

Досі ми розглядали прагнення здовжувати робочий день,
ненажерливий вовчий голод за додатковою працею, на такому
полі, де безмірні зловживання, не перевищені і навіть — як каже
один буржуазний англійський економіст, — жорстокостями еспанців
проти червоношкурих Америки\footnote{
«Ненажерливість власників фабрик призводить до того, що в погоні
за баришем вони допускаються таких жорстокостей, яких ледве чи
перевищили жорстокості еспанців підчас завойовування Америки в гонитві
за золотом» («The cupidity of mill-owners, whose cruelties in pursuit
of gain, have hardly been exceeded by those perpetrated by the Spaniards
on the conquest of America in the pursuit of gold»). (John Wade:
«History of the Middle and Working Classes», 3 id ed. London 1835, p-114).
Теоретична частина цієї книги, свого роду нарис політичної економії,
містить у собі дещо ориґінальне для свого часу, приміром, про торговельні
кризи. Щождо історичної частини, то вона є безсоромний пляґіят із Sir
М.~Eden: «History of the Poor», London 1799.
}, спричинилися, нарешті,
до того, що капітал закували у ланцюги законодавчого реґулювання.
А тепер киньмо оком на деякі галузі промисловости, де
висисання робочої сили або ще й тепер вільне від тих законодавчих
пут, або було таким ще зовсім недавно.

«Пан Бровтон, суддя графства, як голова мітингу, який
відбувся в нотінгемському міському будинку 14 січня 1860~\abbr{р.},
заявив, що серед частини міської людности, занятої виробництвом
мережива, панують такі страшні злидні й нужда, що решта цивілізованого
світу ще таких не знає\dots{} О 2, 3, 4 годині ранку 9--10-літніх
дітей виривають із їхніх брудних ліжок і примушують
тільки за мізерний харч працювати до 10, 11, 12 години вночі,
в наслідок чого нидіють їхні члени, корчиться тіло, тупіють риси
їх обличчя, і їхнє ціле людське єство дубіє в німій нерухомості, на
яку навіть глянути страшно. Це для нас не диво, що пан Малет
і інші фабриканти виступили з протестом проти всякої дискусії.
Система, як її описав панотець Монтегіо Вальпі, — це система безмежного
\index{i}{0187}  %% посилання на сторінку оригінального видання
рабства, — рабства з кожного погляду, соціяльного, фізичного,
морального й інтелектуального\dots{} Що подумати про місто,
яке скликає прилюдний мітинг на те, щоб просити про обмеження
робочого часу для чоловіків на 18 годин на добу\elli{!..} Ми деклямуємо
проти вірджінських і каролінських плянтаторів. Але хіба їхня
торговля неграми з усіма страхіттями батога й баришування людським
м’ясом огидніша, ніж це повільне душогубство, яке відбувається
для того, щоб на користь капіталістам вироблялося
серпанки й комірчини?»\footnote{
«London Daily Telegraph», з 17 січня 1860~\abbr{р.}
}

Ганчарня (Pottery) Стафордшіру була протягом останніх
22 років предметом трьох парляментських слідств. Результати
цих слідств наведено у звіті пана Скрайвена з 1841~\abbr{р.}, поданому
членам «Children’s Employment Commission», у звіті д-ра
Ґрінхов з 1860~\abbr{р.}, опублікованому за розпорядженням лікарського
урядовця Privy Council («Public Health», 3 rd Report,
I, 112--113), нарешті, y звіті пана Льон Ге з 1863~\abbr{р.}, у «First
Report of the Children’s Employment Commission» з 13 червня
1863 p. Для мого завдання досить узяти із звітів з 1860 і 1863~\abbr{рр.}
деякі свідчення дітей, що сами були об’єктом експлуатації. Із
становища дітей можна робити висновки й про становище дорослих,
а особливо дівчат і жінок, і до того ж в такій галузі промисловости,
поруч з якою бавовнопрядіння й~\abbr{т. ін.} може видаватися
дуже приємною й здоровою працею\footnote{
Порівн. Engels: «Lage der arbeitenden Klasse in England»,
Leipzig 1845, S. 249--251. (Енгельс: «Становище робітничої кляси в
Англії». Партвидав «Пролетар», 1932~\abbr{р.}, стор. 233--236).
}.

Вільгельм Вуд, дев’яти років, «почав працювати, мавши
7 років 10 місяців». Спочатку він був «van moulds» (носив до
сушні виготовлений товар у формах і приносив назад порожні
форми). Цілий тиждень день-у-день приходив о 6 годині вранці
й кінчав роботу так десь коло 9 години вечора. «Я цілий тиждень
працюю щодня до 9 години вечора. Так було, приміром, протягом
останніх 7--8 тижнів». Отже, п’ятнадцять годин праці для семилітньої
дитини! Дж.~Меррей, дванадцятилітнє хлоп’я, свідчить:
«І run moulds and turn jigger (я ношу форми та кручу колесо).
Я приходжу о 6, іноді о 4 годині вранці. Я працював цілу останню
ніч до 8 години сьогоднішнього ранку. Я не спав від минулої
ночі. Крім мене працювало ще 8 або 9 хлопчиків цілу останню
ніч без перерви. За винятком одного, всі знов прийшли сьогодні
вранці. Я дістаю 3\shil{ шилінґи} 6\pens{ пенсів} (1 таляр 5 шагів) на тиждень.
Я не дістаю більше, коли працюю цілу ніч. Останнього тижня
я працював дві ночі». Фернігав, десятилітнє хлоп’я: «Мені не
завжди лишається ціла година на обід, часто лише півгодини;
це буває щочетверга, п’ятниці й суботи»\footnote{«Children’s Employment Commission. 1 st Report etc. 1863», Appendix,
p. 16, 19, 18.
}.

Д-р Ґрінхов заявляє, що вік життя в ганчарняних округах
Stoke-upon-Trent і Wolstanton надзвичайно короткий. Хоч в
\parbreak{}  %% абзац продовжується на наступній сторінці

\input{_0188.tex}
\index{iii2}{0189}  %% посилання на сторінку оригінального видання
Одно з найкумедніших явищ є в тому, що всі противники Рікардо, які
заперечують визначення вартости виключно працею, в справі з диференційною
рентою, що випливає з ріжниць землі, надають ваги тому, що тут вартість
визначається природою, а не працею; і одночасно приписують це визначення
положенню, або, і ще більше, процентові на капітал, вкладений в землю при
обробітку. Та сама праця дає однакову вартість для продукту, створеного
протягом даного часу; але величина або кількість цього продукту, отже, і та
частина вартости, яка припадає на відповідну частину цього продукту за даної
кількости праці, залежить єдино від кількости продукту, а це знову від продуктивности
даної кількости праці, не від величини цієї кількости. Чи завдячує
ця продуктивність своїм походженням природі, чи суспільству — цілком байдуже.
Тільки в тому разі, коли вона сама коштує праці, отже, капіталу, вона
збільшує ціну продукції новою складовою частиною, чого природа сама по собі
не робить.

\section{Абсолютна земельна рента}

Аналізуючи диференційну ренту, ми виходили з припущення, що найгірша
земля не виплачує земельної ренти, або, висловлюючись загальніше, що земельну
ренту виплачує тільки така земля, для продукту якої індивідуальна ціна продукції
нижча від ціни продукції, що реґулює ринок, так що в такий спосіб
виникає надзиск, що перетворюється на ренту. Потрібно насамперед зауважити,
що закон диференційної ренти, як днференційної ренти, зовсім не залежить від
правильности чи неправильности того припущення.

Коли загальну ціну продукції, що реґулює ринок, ми назвемо Р, то Р для
продукту найгіршого роду землі А збігається з індивідуальною ціною продукції
на цій землі; тобто вона оплачує зужиткований у продукції сталий і змінний капітал
плюс пересічній зиск (= підприємницькому баришеві плюс процент).

Рента тут дорівнює нулеві. Індивідуальна ціна продукції найближчого
кращого роду землі В = Р', і Р>Р'; тобто Р оплачує більше, ніж дійсну
ціну продукції продукту на клясі землі В. Хай тепер Р — Р' = d; тому
d, надмір Р над Р', є той надзиск, що його добуває орендар з цієї кляси В.
Це d перетворюється на ренту, яку доводиться виплачувати власникові землі.
Хай для третьої кляси землі С за дійсну ціну продукції буде Р", і хай Р —
Р'' = 2d; отже, ці 2d перетворюються на ренту; так само для четвертої кляси
D індивідуальна ціна продукції хай буде Р'", а Р — Р'" = 3d, які перетворюються
на земельну ренту і т. д. Даймо тепер, що припущення, ніби для
кляси землі А рента = 0, а тому ціна її продукту = Р + 0, помилкове. Хай,
навпаки, і вона дає ренту = г. В цьому випадку маємо двоякі наслідки.

\emph{Поперше}: ціна продукту землі кляси А не реґулювалася б ціною продукції
на цій землі, а мала б деякий надмір над цією ціною, вона була б =
P — r. Бо, коли припускається, нормальний перебіг капіталістичного способу
продукції, отже, коли припускається, що надмір r, виплачуваний від орендаря
земельному власникові, не становить вирахування ані з заробітної плати, ані
з пересічного зиску на капітал, то орендар може виплачувати його лише тому,
що його продукт продається понад ціну продукції, що він, отже, дав би йому
надзиск, коли б не доводилося відступати цей надмір у формі ренти земельному
власникові. Реґуляційна ринкова ціна всього наявного на ринку продукту
всіх родів землі була б тоді не та ціна продукції, яку дає капітал взагалі
у всіх сферах продукції, тобто не ціна рівна витратам плюс пересічний
зиск, а була б ціною продукції плюс рента, Р + r, не Р. Бо ціна продукту
кляси А визначає взагалі межу реґуляційної загальної ринкової ціни, тієї ціни,
\parbreak{}  %% абзац продовжується на наступній сторінці

\parcont{}  %% абзац починається на попередній сторінці
\index{iii1}{0190}  %% посилання на сторінку оригінального видання
абож хоч і в тому самому напрямі, але не в тій самій мірі,
одним словом, якщо відбуваються двосторонні зміни, які, однак,
змінюють попереднє відношення між обома сторонами, то кінцевий
результат завжди мусить звестись до одного з двох
вищерозглянутих випадків.

Справжня трудність при загальному визначенні понять попиту
і подання полягає в тому, що визначення це, здається, зводиться
до тавтології. Розгляньмо спочатку подання, тобто продукт,
який перебуває на ринку або може бути приставлений на
ринок. Для того, щоб не вдаватись до цілком зайвих тут
деталей, візьмімо тут масу річної репродукції в кожній даній
галузі промисловості і залишмо при цьому осторонь те, що різні
товари в більшій чи меншій мірі можуть забиратися з ринку
і нагромаджуватись для споживання, скажемо, ближчого року.
Ця річна репродукція виражає насамперед певну кількість, міру
або число, залежно від того, як виміряється товарна маса, —
окремими екземплярами, чи як суцільна величина; це — не тільки
споживні вартості, що задовольняють людські потреби, але і такі
споживні вартості, що перебувають на ринку в певній даній
кількості. Подруге, ця кількість товарів має певну ринкову
вартість, яку можна виразити як кратне ринкової вартості товару
або товарної міри, що служать одиницями. Тому між
кількістю товарів, що перебувають на ринку, і їх ринковою
вартістю не існує ніякого необхідного зв’язку; тимчасом, наприклад,
як деякі товари мають специфічно високу вартість,
інші мають специфічно низьку вартість, так що дана сума вартості
може виразитись у дуже великій кількості одних і в дуже
незначній кількості інших товарів. Між кількістю товарів, що
перебувають на ринку, і ринковою вартістю цих товарів існує
тільки такий зв’язок: на даній базі продуктивності праці виготовлення
певної кількості товарів вимагає в кожній окремій
сфері виробництва певної кількості суспільного робочого часу,
хоч у різних сферах виробництва це відношення є цілком різне
і не стоїть ні в якому внутрішньому зв’язку з корисністю цих
товарів або специфічною природою їх споживних вартостей.
При всіх інших однакових умовах, якщо кількість a даного
сорту товарів коштує b робочого часу, то кількість na коштує
nb робочого часу. Далі: оскільки суспільство хоче задовольнити
потреби, хоче щоб для цієї мети був вироблений товар, воно мусить
його оплатити. Дійсно, оскільки при товарному виробництві
передбачається поділ праці, то суспільство купує ці товари,
вживаючи на їх виробництво частину робочого часу, який
є в його розпорядженні, отже, купує їх за допомогою певної
кількості робочого часу, яким воно може порядкувати. Та частина
суспільства, якій в наслідок поділу праці припадає вживати
свою працю на виробництво цих певних товарів, мусить
дістати еквівалент у суспільній праці, представленій у товарах,
які задовольняють її потреби. Але не існує ніякого необхідного,
\parbreak{}  %% абзац продовжується на наступній сторінці

\parcont{}  %% абзац починається на попередній сторінці
\index{iii2}{0191}  %% посилання на сторінку оригінального видання
земельного власника зовсім не підстава для того, щоб даром передати свою землю
до розпорядження орендареві і, виявши філантропічне ставлення до цього в
справах приятеля, запровадити crédit gratuit\footnote*{
Безплатний кредит. \emph{Прим. Ред.}
}. Таке припущення має в собі
абстрагування від земельної власности, знищення земельної власности, що її
існування саме і ставить межу для приміщення капіталу і вільного використовування
його на землі, — межу, яка зовсім не відпадає від самого міркування
орендаря, що стан збіжжевих цін дозволив би йому здобути з свого
капіталу з допомогою експлуатації землі роду $А$ звичайний зиск, коли б йому
не довелося виплачувати ренти, тобто коли б він міг на практиці ставитись до
земельної власности так, наче б її не існувало. Але монополію земельної власности,
земельну власність як межу капіталу припускається диференційною
рентою, бо без цього надзиск не перетворився б на земельну ренту, і не дістався
б земельному власникові замість орендареві. І земельна власність як межа,
продовжує існувати і там, де рента як диференційна рента відпадає, тобто на
землі $А$. Якщо ми розглянемо випадки, коли в країні капіталістичної продукції
капітал може вкладатися в землю без виплати ренти, то ми знайдемо, що всі
вони включають хоч і не юридичне, то фактичне знищення власности на землю,
знищення, яке, проте, може статися лише за цілком певних і своєю природою
випадкових обставин.

\emph{Перше}. Коли земельний власник сам є капіталіст, або капіталіст сам є
земельний власник. Коли ринкова ціна піднеслась так високо, що на тому,
що є тепер землею роду $А$, можна здобути ціну продукції, тобто покриття капіталу
плюс пересічний зиск, то він може в цьому випадку \emph{сам господарювати}
на своїй дільниці землі. Але чому? Тому, що у відношенні до нього земельна
власність не створює будь-якої межі для приміщення його капіталу.
Він може обробляти землю як простий елемент природи і тому він може керуватися
виключно міркуваннями про використання свого капіталу, капіталістичними
міркуваннями. Такі випадки трапляються на практиці, але тільки як винятки.
Так само як капіталістичне оброблення землі має за передумову роз’єднення капіталу,
що функціонує, і земельної власности, цілком так само воно виключає як
загальне правило провадження господарства самим земельним власником. Одразу
видно, що таке провадження господарства самим земельним власником є цілком
випадкове. Коли збільшений попит на збіжжя потребує оброблення більшої кількости
землі $А$, ніж її є у власників, які сами провадять господарство, коли, отже,
частина її мусить бути віддана в оренду для того, щоб вона могла взагалі оброблятися,
тоді зараз же відпадає ця гіпотетичність погляду на межу, яку земельна
власність створює для приміщення капіталу. Постає недоладне противенство,
коли виходять з відповідного капіталістичному способові продукції відокремлення
між землею і капіталом, орендарем і земельним власником, а потім,
навпаки, припускають, що господарство провадять, як загальне правило, самі
земельні власники до такого обсягу і повсюди, де капітал, коли б незалежно
від нього не існувало жодної земельної власности, не здобував бн з оброблення
землі жодної ренти. (Див. у А.~Сміта місце про ренту з копалень, цитоване
значно далі). Це знищення земельної власности є випадкове. Воно може статись
або не статись.

\emph{Друге}. В складі орендованих земель можуть бути такі окремі дільниці
землі, що за даного ріння ринкових цін не дають ренти, отже, на ділі здаються
даром, але земельний власник не вважає їх за такі, бо він бачить загальну суму
ренти з землі, віддай її в оренду, а не осібні ренти з окремих складових дільниць
його землі. В цьому випадку для орендаря, — оскільки справа йде про
нерентодайні орендовані дільниці, — земельна власність як межа приміщення
\parbreak{}  %% абзац продовжується на наступній сторінці

\parcont{}  %% абзац починається на попередній сторінці
\index{iii1}{0192}  %% посилання на сторінку оригінального видання
репродукції в наслідок нагромадження капіталу, то, при інших
незмінних умовах, потрібна додаткова кількість бавовни. Те саме
і щодо засобів існування. Робітничий клас, для того, щоб і далі
жити при звичайних пересічних умовах, мусить діставати принаймні
попередню кількість необхідних засобів існування, хоч,
може, і розподілених дещо інакше між різними сортами товарів;
якщо ж узяти до уваги щорічний ріст населення, то потрібна
ще певна додаткова кількість засобів існування; те саме
з більшими чи меншими модифікаціями можна сказати і щодо
інших класів.

Отже, здається, ніби на стороні попиту є певна, даної величини
суспільна потреба, яка для свого задоволення вимагає
певної кількості товару на ринку. Але кількісна визначеність
цієї потреби цілком еластична й хитка. Вона тільки здається
фіксованою. Якби засоби існування були дешевші або грошова
заробітна плата була вища, то робітники купували б більше
засобів існування і виявилася б більша „суспільна потреба“ на
ці сорти товарів, — при чому ми зовсім залишаємо осторонь
пауперів і т. д., „попит“ яких стоїть нижче найвужчих меж їх
фізичної потреби. Коли б, з другого боку, подешевшала, наприклад,
бавовна, то попит капіталістів на бавовну виріс би, в бавовняну
промисловість було б вкладено більше додаткового
капіталу і т. д. При цьому взагалі не слід забувати, що попит
на продуктивне споживання при нашому припущенні є попит
з боку капіталіста і що справжня мета капіталіста є виробництво
додаткової вартості, так що він тільки з цією метою
виробляє певний сорт товарів. З другого боку, це не перешкоджає
тому, що капіталіст, оскільки він виступає на ринку як
покупець, наприклад, бавовни, репрезентує потребу в бавовні,
адже і для продавця бавовни байдуже, чи перетворює покупець
цю бавовну в сорочки, в бавовняний порох, чи має намір
затикати нею вуха собі і всьому світові. Але в усякому разі це
справляє великий вплив на те, якого роду покупець він є. Його
потреба в бавовні істотно модифікується тією обставиною, що
в дійсності вона тільки приховує його потребу добувати зиск. —
Межі, в яких репрезентована на \emph{ринку} потреба в товарах —
попит — кількісно відрізняється від \emph{дійсної суспільної} потреби,
звичайно, дуже різні для різних товарів; я маю на увазі ріжницю
між кількістю товарів, на яку є попит, і тією кількістю
їх, на яку був би попит при інших грошових цінах товарів або
при інших грошових або життьових умовах покупців.

Нема нічого легшого, як зрозуміти нерівномірності попиту
й подання та відхилення, що випливають звідси, ринкових цін
від ринкових вартостей. Справжня трудність полягає у визначенні
того, що слід розуміти під висловом: попит і подання
покриваються.

Попит і подання покриваються, якщо вони стоять у такому
відношенні одне до одного, що товарна маса певної галузі виробництва
\index{iii1}{0193}  %% посилання на сторінку оригінального видання
може бути продана по її ринковій вартості, — ні вище,
ні нижче. Ось перше, що нам кажуть.

Подруге: якщо товари можуть бути продані по їх ринковій
вартості, то попит і подання покриваються.

Якщо попит і подання взаємно покриваються, то вони перестають
діяти, і саме тому товари продаються по їх ринковій вартості.
Якщо дві сили рівномірно діють у протилежних напрямах,
то вони одна одну знищують, зовсім не діють назовні, і явища,
які відбуваються при цій умові, мусять бути пояснені якось
інакше, а не діянням цих двох сил. Якщо попит і подання взаємно
знищуються, то вони перестають щонебудь пояснювати,
не діють на ринкову вартість і залишають нас у цілковитому
невіданні того, чому ринкова вартість виражається саме в цій
сумі грошей, а не в будьякій іншій. Дійсні внутрішні закони
капіталістичного виробництва, очевидно, не можуть бути пояснені
з взаємодіяння попиту й подання (цілком незалежно від
глибшого аналізу цих двох суспільних рушійних сил, який сюди
не стосується), бо ці закони тільки тоді виявляються здійсненими
в чистому вигляді, коли попит і подання перестають
діяти, тобто взаємно покриваються. В дійсності попит і подання
ніколи не покриваються або, якщо і покриваються, то тільки
випадково, — отже, з наукового погляду такі випадки слід прирівняти
до нуля і розглядати як неіснуючі. Але в політичній
економії припускається, що вони покриваються. Чому? Це робиться
для того, щоб розглядати явища в їх закономірному, відповідному
їх поняттю вигляді, тобто розглядати їх незалежно від
того, якими вони здаються в наслідок руху попиту й подання.
З другого боку, для того, щоб знайти дійсну тенденцію їх руху, так
би мовити, фіксувати її. Бо відхилення від рівності мають протилежний
характер і, через те що вони завжди йдуть одне за одним,
вони урівноважуються завдяки своїм протилежним напрямам, завдяки
своїй суперечності. Отже, якщо попит і подання не покриваються
ні в одному випадку, то їх відхилення від рівності йдуть одне
за одним таким чином, — результат відхилення в одному напрямі
є той, що воно викликає відхилення в протилежному напрямі, —
що, коли розглядати підсумок руху за більш-менш довгий період
часу, подання і попит постійно покриваються; однак, вони покриваються
тільки як пересічне минулих уже коливань, тільки як
постійний рух їх суперечності. В наслідок цього ринкові ціни,
що відхиляються від ринкових вартостей, розглядувані щодо
їх пересічної, вирівнюються в ринкові вартості, при чому відхилення
від цих останніх взаємно знищуються як плюс і мінус.
І ця пересічна має не тільки теоретичне значення, вона має
також і практичну важливість для капіталу, вкладення якого розраховане
на коливання й вирівнювання протягом більш-менш певного
періоду часу.

Тому відношення між попитом і поданням пояснює, з одного
боку, тільки відхилення ринкових цін від ринкових вартостей
\parbreak{}  %% абзац продовжується на наступній сторінці

\parcont{}  %% абзац починається на попередній сторінці
\index{iii1}{0194}  %% посилання на сторінку оригінального видання
і, з другого боку, тенденцію до знищення цих відхилень, тобто
до знищення впливу відношення між попитом і поданням. (Виняткові
товари, що мають ціни, не маючи вартості, тут не розглядаються.)
Попит і подання можуть в дуже різній формі
знищувати вплив, що його викликає їх нерівність. Наприклад,
якщо падає попит, а тому й ринкова ціна, то це може привести
до того, що капітал відтягатиметься і таким чином подання
зменшиться. Але це, може привести й до того, що сама ринкова
вартість завдяки винаходам, які скорочують необхідний робочий
час, знизиться і через це зрівняється з ринковою ціною. Навпаки:
якщо попит підвищується і, отже, ринкова ціна підвищується
понад ринкову вартість, то це може привести до того,
що до цієї галузі виробництва припливе занадто багато капіталу
і виробництво зросте настільки, що ринкова ціна впаде
навіть нижче ринкової вартості; або, з другого боку, це може
привести до такого підвищення цін, яке скоротить самий попит.
В деяких галузях виробництва це може привести також до
того, що на більш-менш довгий період часу підвищиться сама
ринкова вартість, бо протягом цього часу частина продуктів,
на які є попит, мусить вироблятися при гірших умовах.

Якщо попит і подання визначають ринкову ціну, то, з другого
боку, ринкова ціна і, при дальшому аналізі, ринкова вартість
визначає попит і подання. Щодо попиту це очевидно, бо
попит рухається в напрямі, протилежному до цін, підвищується,
коли ціни падають, і навпаки. Але те саме стосується й до
подання. Бо ціни засобів виробництва, що входять у товар, який
подається на ринок, визначають попит на ці засоби виробництва,
отже й подання тих товарів, подання яких включає в собі
попит на ці засоби виробництва. Ціни на бавовну мають визначальний
вплив на подання бавовняних матерій.

До цієї плутанини — визначення цін попитом і поданням і,
поруч з цим, визначення попиту й подання цінами — долучається
ще й те, що подання визначається попитом і, навпаки,
попит визначається поданням, ринок визначається виробництвом,
а виробництво — ринком.\footnote{
Великим тупоумством є оця „дотепність“: „Where the quantity of wages,
capital, and land, required to produce an article, have become different from what
they were, that which Adam Smith calls the natural price of it, is also different,
and that price which was previously its natural price, becomes, with reference to
this alteration, its market-price; because, though neither the supply, nor the quantity
wanted may have changed (і те і друге змінюється тут якраз тому, що
ринкова вартість або — про що йде мова в А. Сміта — ціна виробництва змінюється
в наслідок зміни вартості) that supply is not now exactly enough for
those persons who are able and willing to pay what is now the cost of production,
but is either greater or less than that; so that the proportion between the supply,
and what is, with reference to the new cost of production, the effectual demand,
is different from what is was. An alteration in the rate of supply will then take
place if there is no obstacle іn the way of it, and at last bring the commodity
to its new natural price. It may then seem good to some persons to say that, as
the commodity gets to its natural price by an alteration in its supply, the natural
price is as much owing to one proportion between the demand and the supply, as
the market-price is to another; and consequently, that the natural price, just as
much as the market-price, depends on the proportion that demand and supply
bear to each other“. („The great principle of demand and supply is called into
action to to determine wat A. Smith calls natural prices as well as market-prices". —
Malthus.) [„Якщо кількість заробітної плати, капіталу й землі, потрібна для
виготовлення якогось товару, змінюється порівняно з попередньою, то змінюється
й те, що Адам Сміт називає його природною ціною, і та ціна, яка
первісно була його природною ціною, стає у відношенні до цієї зміни його
ринковою ціною; бо, хоч ні подання, ні кількість товару, на яку є попит,
може, не змінилися“ (і те і друге змінюється тут якраз тому, що ринкова
вартість або — про що йде мова в А. Сміта — ціна виробництва змінюється
в наслідок зміни вартості), „проте, подання це тепер не цілком точно відповідає
попитові тих осіб, які спроможні і хочуть заплатити те, що становить тепер
витрати виробництва; воно або більше, або менше; так що відношення між
поданням і тим, що становить тепер при нових витратах виробництва дійсний
попит, відрізняється від попереднього. Отже, якщо не буде перешкод,
настане зміна в розмірі подання і це кінець-кінцем приведе товар до його
нової природної ціни. Дехто, може, вважатиме тоді можливим сказати, що —
через те що товар досягає своєї природної ціни в наслідок зміни розміру
його подання — природна ціна так само завдячує своє існування одному відношенню
між попитом і поданням, як ринкова ціна — другому; і що, отже,
природна ціна, цілком так само як і ринкова ціна, залежить від того відношення,
в якому стоять одне до одного попит і подання“. („Великий принцип
попиту й подання покликається до дії для того, щоб так само визначити
те, що А. Сміт називає природною ціною, як і те, що він називає ринковою
ціною". — Мальтус)]. („Observations on certain verbal disputes etc. ", Лондон 1821,
стор. 60, 61). Наш мудрець не розуміє, що в даному випадку саме зміна в cost
of production [витратах виробництва], отже і в вартості, викликала зміну в попиті,
отже й у відношенні між попитом і поданням, і що ця зміна в попиті
може викликати зміну в поданні; а це доводить якраз протилежне тому, що
хоче довести наш мислитель; а саме, це доводить, що зміна у витратах виробництва
аж ніяк не регулюється відношенням між попитом і поданням, а, навпаки,
сама регулює це відношення.}

\index{iii1}{0195}  %% посилання на сторінку оригінального видання
Навіть ординарний економіст (див. виноску) розуміє, що
й без викликаної зовнішніми обставинами зміни подання чи потреби
відношення між попитом і поданням може змінитися
в наслідок зміни у ринковій вартості товарів. Навіть він мусить
згодитись, що, яка б не була ринкова вартість, попит і подання
мусять урівноважитись, щоб вона реалізувалась. Це значить,
що не відношення між попитом і поданням пояснює ринкову
вартість, а, навпаки, ця остання пояснює коливання попиту
й подання. Автор „Observations“ після місця, цитованого у виносці,
каже далі: „This proportion (між попитом і поданням),
however if we still mean by „demand“ and „natural price“, what
we meant just now, when referring to Adam Smith, must always
be a proportion of equality; for it is only when the supply is equal
to effectual demand, that is, to that demand, which will pay neither
more nor less than the natural price, that the natural price is in
fact paid; consequently, there may be two very different natural
prices, at different times, for the same commodity, and yet the
proportion which the supply bears to the demand, be in both cases
the same, namely the proportion of equality“. [„Однак, це відношення
(між попитом і поданням), якщо ми й далі розумітимем
\parbreak{}  %% абзац продовжується на наступній сторінці

\parcont{}  %% абзац починається на попередній сторінці
\index{iii1}{0196}  %% посилання на сторінку оригінального видання
під „попитом“ і „природною ціною“ те, що ми досі розуміли під
цим, покликаючись на А. Сміта, завжди мусить бути відношенням
рівності, бо тільки тоді, коли подання дорівнює дійсному
попитові, тобто попитові, який не хоче платити ні більше,
ні менше природної ціни, — тільки тоді дійсно сплачується природна
ціна; отже, в різний час той самий товар може мати дві
дуже різні природні ціни, і все ж відношення між поданням
і попитом, в обох випадках може бути однаковим, а саме
відношенням рівності“.] Отже, тут допускається, що при двох
різних natural prices [природних цінах] одного й того самого
товару в різний час попит і подання кожного разу можуть взаємно
покриватись і мусять покриватись для того, щоб товар
в обох випадках був проданий по його natural price. Але через
те що в обох випадках немає ніякої ріжниці у відношенні між
попитом і поданням, але є ріжниця у величині самої natural
price, то ця остання, очевидно, визначається незалежно від попиту
й подання і, отже, менш за все може бути ними визначена.

Для того, щоб товар продавався по його ринковій вартості,
тобто пропорціонально до вміщеної в ньому суспільно-необхідної
праці, сукупна кількість суспільної праці, вживана для
виробництва сукупної маси цього роду товарів, мусить відповідати
величині суспільної потреби в цих товарах, тобто платоспроможної
суспільної потреби. Конкуренція, коливання ринкових
цін, які відповідають коливанням відношення між попитом
і поданням, постійно намагаються звести до цієї міри сукупну
кількість праці, вжитої на кожний рід товарів.

У відношенні між попитом і поданням товарів повторюється,
поперше, відношення між споживною вартістю і міновою вартістю,
між товаром і грішми, між покупцем і продавцем; подруге,
відношення між виробником і споживачем, хоч обидва
вони можуть бути представлені третіми особами, торговцями.
При дослідженні відношення між покупцем і продавцем досить
протиставити їх, кожного окремо, один одному, щоб розвинути
це відношення. Трьох осіб досить для повної метаморфози
товару і, отже, для процесу продажу-купівлі, взятого в цілому.
$А$ перетворює свій товар у гроші $В$, якому він продає товар,
і знову перетворює свої гроші в товар, який він купує на ці
гроші в $C$; весь процес відбувається між ними трьома. Далі:
при дослідженні грошей ми припускали, що товари продаються
по їх вартості, бо не було ніякої підстави розглядати ціни, що
відхиляються від вартості, оскільки йшлося тільки про ті зміни
форми, які пророблює товар, стаючи грішми і знову перетворюючись
з грошей у товар. Раз товар взагалі продається і на
виручені гроші купується новий товар, то ми маємо перед
собою цілу метаморфозу, і для неї як такої однаково, чи стоїть
ціна товару нижче чи вище його вартості. Вартість товару зберігає
своє значення як основа, бо тільки з цієї основи можуть бути
раціонально виведені гроші, і ціна за своїм загальним поняттям
\parbreak{}  %% абзац продовжується на наступній сторінці

\parcont{}  %% абзац починається на попередній сторінці
\index{i}{0197}  %% посилання на сторінку оригінального видання
«Наші «білі раби», — вигукнув «Morning Star», орган панів
фритредерів Кобдена й Брайта, — наші білі раби запрацьовуються
на смерть і гинуть і вмирають без найменшого шуму».\footnote{
«Morning Star» з 23 липня 1863 р. «Times» скористався цим випадком
для оборони американських рабовласників проти Брайта й т. ін.
«Дуже багато з нас, — каже «Times», — гадають, що лоти, доки ми сами
вимучуємо на смерть працею наших власних молодих жінок, погрожуючи
їм ударами голоду замість свисту батога, доти ми ледве чи маємо право
йти мечем і вогнем на ті родини, що їхні члени родилися рабовласниками
та які принаймні добре годують своїх рабів і вимагають від них лише
помірної праці» («Times», а 2 липня 1863 р.). Газета торів «Standart»
розправлялась у тому самому дусі з його преподобієм Ньюмен Холлом:
«Він відлучує від церкви рабовласників, але молиться разом із порядними
людьми, що примушують працювати за собачу плату лондонських візників
та кондукторів омнібусів і т. ін. лише по 16 годин на день». Нарешті,
пролунав голос оракула, винахідника культу генія, пана Томаса Карлейля,
про якого я вже року 1850 писав: «Геній пішов к чорту, лишився культ».
В коротенькій притчі він зводить єдину величну подію сучасної історії,
американську громадянську війну, на те, що Петро з півночі з усіх сил
намагається переломити черепа Павлові з півдня, бо Петро з півночі
наймає свого робітника «поденно», а Павло з півдня — «на ціле життя».
(«Macmillan’s Magazine». Ilias Americana in nuсе. Серпневий зошит
1863 р.). Так луснув, нарешті, шумовинний пухир торійських симпатій
до міських — але ні в якому разі не до сільських! — найманих робітників.
Основа цих симпатій — це рабство!
}

«Запрацьовуватись на смерть — це є порядок дня не лише
в майстернях кравчих, але в тисячах місць, ба на кожному місці,
де справи йдуть добре\dots{} Візьмімо як приклад коваля. Як вірити
поетам, то немає в світі людини сильнішої, веселішої за коваля.
Він устає раннім ранком і викрешує іскри перед тим, як засяє
сонце; нема такої людини, що так їла б, так пила б і спала, як
він. Якщо поглянути на долю коваля чисто з фізичного боку, то,
дійсно, за помірної праці, становище його одне з найкращих.
Але ходімо за ним до міста й погляньмо на той тягар праці, який
накладають на його дужі плечі, погляньмо на місце, яке він посідає
у таблицях смертности нашої країни? У Marylebone (один із
найбільших міських кварталів Лондону) смертність ковалів становить
31 на 1000 на рік, а це на 11 перевищує пересічну смертність
дорослих чоловіків Англії. Праця ця, майже інстинктова вмілість
людини, сама по собі бездоганна, через саму лише надмірність
стає руйнаційною для людини. Людина може зробити стільки й
стільки ударів молотом на день, стільки й стільки кроків, стільки
й стільки разів дихнути, стільки й стільки зробити якоїсь роботи
й прожити пересічно, приміром, 50 років. Її примушують робити
на стільки більше вдарів, на стільки більше кроків, стільки частіш
віддихувати, а це все разом збільшує її життєве завдання на
одну четвертину на день. Вона силкується це все робити, а результат
такий, що за обмежений період вона виконує на четвертину
більшу роботу і вмирає на 37 році замість на 50».\footnote{
\emph{Dr. Richardson}: «Work and Overwork» y «Social Science Review»,
18 липня 1863.
}

\parcont{}  %% абзац починається на попередній сторінці
\index{iii1}{0198}  %% посилання на сторінку оригінального видання
з меншими витратами виробництва) або принаймні з якомога меншими
втратами вийти з цього становища, і в цьому разі він уже не
турбуватиметься про своїх сусідів, хоч його дії зачіпають не тільки
його самого, але й усіх його товаришів\footnote{
„If each man of a class could never have more than a given share, or
aliquot part of the gains and possessions of the whole, he would readily combine
to raise the gains (він саме так і робить, якщо відношення між попитом і поданням
йому це дозволяє); this is monopoly. But where each man thinks that he
may any way increase the absolute amount of his own share, though by a process
which lessens the whole amount, he will often do it; this is competition“. [„Коли б
кожен член групи ніколи не міг одержати більше даної частки або відповідної
частини доходів і володінь всієї групи, то він охоче об’єднувався б з іншими,
щоб підвищити ці доходи“ (він саме так і робить, якщо відношення між попитом
і поданням йому це дозволяє); „це — монополія. Але якщо кожен думає,
що він якимсь способом може збільшити абсолютну суму своєї власної частки,
хоч би й шляхом зменшення цілої суми, то він часто саме так і робитиме; це —
конкуренція.] („An Inquiry into those principles respecting the nature of demand
etc.“, Лондон 1821, стор. 105).
}.

Попит і подання передбачають перетворення вартості в ринкову
вартість, і, оскільки це відбувається на капіталістичній
базі, оскільки товари є продукти капіталу, передбачають капіталістичний
процес виробництва, отже, відносини, які цілком інакше
переплітаються, ніж проста купівля і продаж товарів. Тут ідеться
не про формальне перетворення вартості товарів у ціну, тобто
не про просту переміну форми; тут ідеться про певні кількісні
відхилення ринкових цін від ринкових вартостей і, далі, від цін виробництва.
При простій купівлі і продажу досить, щоб товаровиробники
як такі протистояли один одному. Попит і подання при
дальшому аналізі передбачають існування різних класів і підрозділів
класів, які розподіляють між собою сукупний дохід суспільства
і споживають його як дохід, які, отже, пред’являють попит, визначуваний
цим доходом; тимчасом як, з другого боку, для розуміння
попиту і подання, що їх утворюють між собою виробники
як такі, необхідно зрозуміти всю систему капіталістичного
процесу виробництва в цілому.

При капіталістичному виробництві справа йде не тільки про те,
щоб замість маси вартості, кинутої в циркуляцію у товарній
формі, вилучити з неї рівну масу вартості в іншій формі, —
в формі грошей або в формі іншого товару, — справа йде про те,
щоб на капітал, авансований на виробництво, здобути таку саму
додаткову вартість, або зиск, як і на кожний інший капітал такої
самої величини, або pro rata [пропорціональну до] його величини,
незалежно від того, в якій галузі виробництва він застосовується;
отже, справа йде про те, щоб продати товари мінімум
по таких цінах, які дають пересічний зиск, тобто по цінах виробництва.
В цій формі капітал сам приходить до усвідомлення
себе як \emph{суспільної сили}, в якій кожний капіталіст має частину,
пропорціональну до його участі в сукупному суспільному капіталі.

Поперше, для капіталістичного виробництва самого по собі
не має значення певна споживна вартість, взагалі специфічність
\index{iii1}{0199}  %% посилання на сторінку оригінального видання
товару, який воно виробляє. В кожній сфері виробництва
мета полягає тільки в тому, щоб виробляти додаткову
вартість, привласнювати собі в продукті праці певну кількість
неоплаченої праці. І так само в природі підлеглої капіталові
найманої праці лежить те, що вона байдуже ставиться до
специфічного характеру своєї праці, мусить змінюватись відповідно
до потреб капіталу і переходити з однієї сфери виробництва
до іншої.

Подруге, кожна сфера виробництва дійсно є остільки ж добра
і остільки ж погана, як і будь-яка інша; кожна дає той самий
зиск і кожна була б безцільною, коли б вироблювані нею товари
не задовольняли будь-якої суспільної потреби.,

Але якщо товари продаються по їх вартостях, то, як це вже
показано, в різних сферах виробництва виникають дуже різні
норми зиску, залежно від різного органічного складу вкладених
у ці сфери мас капіталу. Але капітал вилучається з сфери виробництва
з нижчою нормою зиску і кидається в іншу, яка дає
вищий зиск. В наслідок цієї постійної еміграції та імміграції,
одним словом, в наслідок свого розподілу між різними сферами
виробництва, залежно від того, де норма зиску падає і де підвищується,
капітал здійснює таке відношення між попитом і поданням,
що в різних сферах виробництва пересічний зиск
стає однаковий, і тому вартості перетворюються в ціни виробництва.
Це вирівнення капіталові вдається здійснити тим повніше,
чим вищий капіталістичний розвиток в даному національному
суспільстві, тобто чим більше відносини даної країни пристосовані
до капіталістичного способу виробництва. З прогресом капіталістичного
виробництва розвиваються і умови його; воно підпорядковує
своєму специфічному характерові і своїм імманентним
законам усю сукупність суспільних передумов, в межах яких
відбувається процес виробництва.

\looseness=-1
Постійне вирівнювання постійних нерівностей відбувається
тим швидше, 1)~чим рухливіший капітал, тобто чим легше він
може бути перенесений з однієї сфери і з одного місця в інші;
2)~чим швидше робоча сила може бути перекинута з однієї
сфери в іншу і з одного місцевого центру виробництва до
іншого. Пункт 1-й передбачає повну свободу торгівлі всередині
суспільства і усунення всіх монополій, крім природних, особливо
тих, що виникають з самого капіталістичного способу виробництва.
Далі, передбачається розвиток кредитної системи, яка
концентрує в руках окремих капіталістів неорганізовану масу
вільного суспільного капіталу; нарешті — підпорядкування різних
сфер виробництва капіталістам. Це останнє включене вже
у припущені нами передумови, раз ми допустили, що справа
йде про перетворення вартостей у ціни виробництва в усіх капіталістично
експлуатованих сферах виробництва; однак, само
це вирівнювання наштовхується на значніші перешкоди, коли
численні і масові сфери виробництва, проваджені некапіталістично
\index{iii1}{0200}  %% посилання на сторінку оригінального видання
(наприклад, землеробство у дрібних селян), вклинюються
між капіталістичні підприємства і переплітаються з ними. Нарешті,
— велика густота населення. — Пункт 2-й передбачає скасування
всіх законів, які перешкоджають робітникам переселятись
з однієї сфери виробництва в іншу або з одного місцевого
центру виробництва до якогонебудь іншого. Байдуже ставлення
робітника до змісту його праці. Якомога більше зведення праці
в усіх сферах виробництва до простої праці. Зникнення всіх професійних
передсудів у робітників. Нарешті — і це особливо —
підпорядкування робітника капіталістичному способові виробництва.
Дальший виклад цього питання належить до спеціального
дослідження конкуренції.

Із сказаного випливає, що кожний окремий капіталіст, як
і сукупність усіх капіталістів кожної окремої сфери виробництва,
бере участь в експлуатації сукупного робітничого класу сукупним
капіталом і в ступені цієї експлуатації не тільки в силу
загальної класової симпатії, але й безпосередньо економічно,
бо — якщо припустити всі інші умови, в тому числі і вартість
сукупного авансованого сталого капіталу, даними — пересічна
норма зиску залежить від ступеня експлуатації сукупної
праці сукупним капіталом.

\looseness=-1
Пересічний зиск збігається з пересічною додатковою вартістю,
яку капітал виробляє на кожні 100 одиниць; відносно додаткової
вартості щойно сказане зрозуміле само собою. Щождо пересічного
зиску, то сюди приєднується ще як один з моментів,
які визначають норму зиску, тільки вартість авансованого капіталу.
Справді, особливий інтерес, що його має капіталіст або
капітал певної сфери виробництва в експлуатації безпосередньо
занятих ним робітників, обмежується тим, щоб за допомогою
винятково надмірної праці, або за допомогою зниження заробітної
плати нижче пересічного рівня, абож за допомогою виняткової
продуктивності вживаної праці одержати додаткову вигоду,
одержати такий зиск, що перевищує пересічний. Якщо залишити
це осторонь, то капіталіст, який зовсім не вживає у своїй сфері
виробництва змінного капіталу, отже й робітників (що в дійсності,
звичайно, неможливо), був би так само дуже заінтересований
в експлуатації робітничого класу капіталом і цілком так само
діставав би свій зиск з неоплаченої додаткової праці, як і, наприклад,
той капіталіст, який (знову таки в дійсності неможливе
припущення) вживає тільки змінний капітал, тобто витрачає весь
свій капітал на заробітну плату. Але ступінь експлуатації праці
при даному робочому дні залежить від пересічної інтенсивності
праці, а при даній інтенсивності — від довжини робочого дня.
Від ступеня експлуатації праці залежить висота норми додаткової
вартості, отже, при даній загальній масі змінного капіталу
— величина додаткової вартості, а тому й величина зиску.
Той спеціальний інтерес, що його капітал певної сфери виробництва,
в відміну від сукупного капіталу, має в експлуатації
\parbreak{}  %% абзац продовжується на наступній сторінці

\input{_0201c.tex}
\parcont{}  %% абзац починається на попередній сторінці
\index{iii1}{0202}  %% посилання на сторінку оригінального видання
умовах. Виключаючи взагалі випадки криз і перепродукції, це
стосується до всіх ринкових цін, як би дуже вони не відхилялись
від ринкових вартостей або ринкових цін виробництва.
Саме ринкова ціна передбачає, що за товари того самого роду
сплачується однакова ціна, не зважаючи на те, що ці товари
можуть бути вироблені при дуже різних індивідуальних умовах
і тому можуть мати дуже різні витрати виробництва. (Про надзиски
як наслідки монополій у звичайному розумінні слова, штучних
чи природних, ми тут не говоримо.)

Але, крім того, надзиск може виникнути ще в тому випадку,
коли певні сфери виробництва перебувають в такому стані, що
вони можуть ухилитися від перетворення їх товарних вартостей
у ціни виробництва, а тому й від зведення їх зисків до пересічного
зиску. У відділі про земельну ренту ми розглянемо
дальший розвиток цих двох форм надзиску.

\section{Впливи загальних коливань заробітної плати
на ціни виробництва}

Припустім, що пересічний склад суспільного капіталу є
$80 c \dplus{} 20 v$, а зиск — 20\%. В цьому випадку норма додаткової
вартості є 100\%. Загальне підвищення заробітної плати, якщо
припустити всі інші умови однаковими, означає зниження норми
додаткової вартості. Для пересічного капіталу зиск і додаткова
вартість збігаються. Припустім, що заробітна плата підвищується
на 25\%. Та сама маса праці, яку привести в рух коштувало 20,
коштує тепер 25. Отже, ми маємо в цьому випадку, замість
$80 c \dplus{} 20 v \dplus{} 20 p$, за один оборот вартість у $80 c \dplus{} 25 v \dplus{} 15 p$.
Праця, приведена в рух змінним капіталом, як і раніш, виробляє
суму вартості в 40. Якщо $v$ підвищується з 20 до 25, то надлишок
$m$ або $p$ є вже тільки \deq{} 15. Зиск в 15 на 105 \deq{} 14\sfrac{2}{7}\%,
і це було б новою нормою пересічного зиску. Через те що
ціна виробництва товарів, вироблюваних пересічним капіталом,
збігається з їх вартістю, ціна виробництва цих товарів не змінилася
б; тому підвищення заробітної плати привело б, правда,
до зниження зиску, але не привело б до зміни вартості й ціни
товарів.

Раніше, коли пересічний зиск був \deq{} 20\%, ціна виробництва
товарів, вироблених за один період обороту, дорівнювала їх
витратам виробництва плюс зиск в 20\% на ці витрати виробництва,
отже \deq{} $k \dplus{} kp' \deq{} k \dplus{} \frac{20k}{200}$, де $k$ є змінна величина, різна
залежно від вартості засобів виробництва, що входять у товари,
і від розміру того зношування, яке основний капітал, застосований
у виробництві цих товарів, віддає продуктові. Тепер
ціна виробництва становила б $k \dplus{} \frac{14\sfrac{2}{7} k }{100}$.


\index{iii1}{0203}  %% посилання на сторінку оригінального видання
Візьмімо тепер капітал, склад якого є нижчий, ніж первісний
склад пересічного суспільного капіталу $80 c \dplus{} 20 v$ (який
перетворився тепер в $76\sfrac{4}{21}c \dplus{} 23\sfrac{17}{21}v$), наприклад, $50 c \dplus{} 50 v$.
Тут ціна виробництва річного продукту, — якщо ми для спрощення
припустимо, що весь основний капітал увійшов як зношування
в річний продукт і що час обороту такий самий, як
і в випадку I, — становила перед підвищенням заробітної плати
$50 c \dplus{} 50 v \dplus{} 20 p \deq{} 120$. Підвищення заробітної плати на 25\%
дає для тієї самої кількості приведеної в рух праці підвищення
змінного капіталу з 50 до 62\sfrac{1}{2}. Коли б річний продукт був
проданий по попередній ціні виробництва в 120, то це дало б
$50 c \dplus{} 62\sfrac{1}{2}v \dplus{} 7\sfrac{1}{2}p$, тобто норму зиску в 6\sfrac{2}{3}\%.
Але нова пересічна норма зиску є 14\sfrac{2}{7}\%, і через те що ми всі інші умови
припускаємо незмінними, цей капітал в $50 c \dplus{} 62\sfrac{1}{2}v$ так само
мусить дати вказаний зиск. Але капітал в 112\sfrac{1}{2}, при нормі зиску
в 14\sfrac{2}{7}, дає 16\sfrac{1}{14} зиску.
% REMOVED \footnote*{
% В першому німецькому виданні тут сказано: „в круглих числах 16\sfrac{1}{12}
% зиску“; відповідно до цього Енгельс обчислює потім ціну виробництва в 128\sfrac{7}{12}
% В рукопису Маркса дано точне число в 16\sfrac{3}{42}, яке нами взяте з відповідним
% скороченням дробу і застосоване при обчисленні ціни виробництва.
% \Red{Примітка ред. нім. вид. ІМЕЛ.}
% }
Отже, ціна виробництва вироблених
ним товарів є тепер $50 c \dplus{} 62\sfrac{1}{2}v \dplus{} 16\sfrac{1}{14}p \deq{} 128\sfrac{8}{14}$. Отже, в наслідок
підвищення заробітної плати на 25\% ціна виробництва
тієї самої кількості того самого товару підвищилась тут з 120
до 128\sfrac{8}{14}, або більше ніж на 7\%.

Візьмім, навпаки, сферу виробництва вищого складу, ніж пересічний
капітал, наприклад, $92 c \dplus{} 8 v$. Отже, первісний пересічний
зиск і тут \deq{} 20, і якщо ми знову припустимо, що весь
основний капітал входить у річний продукт і що час обороту
такий самий, як і в випадках І і II, то ціна виробництва товару
й тут \deq{} 120.

В наслідок підвищення заробітної плати на 25\% змінний капітал
для тієї самої кількості праці зростає з 8 до 10, отже
витрати виробництва товарів зростають з 100 до 102; з другого
боку, пересічна норма зиску впала з 20\% до 14\sfrac{2}{7}\%. Але
$100 : 14\sfrac{2}{7} \deq{} 102 : 14\sfrac{4}{7}$.
% REMOVED \footnote*{
% В першому німецькому виданні тут стоїть: „(приблизно)“. В рукопису
% Маркса цього слова немає. В дійсності тут рівняння точне, а не тільки приблизне.
% \Red{Примітка ред. нім. вид. ІМЕЛ.}
% }
Отже, зиск, що припадає тепер на 102,
становить 14\sfrac{4}{7}. І тому весь продукт продається за
$k \dplus{} kp' \deq{} 102 \dplus{} 14\sfrac{4}{7} \deq{} 116\sfrac{4}{7}$. Отже, ціна виробництва впала
з 120 до 116\sfrac{4}{7}, або майже на 3\%.
% REMOVED \footnote*{
% В першому німецькому виданні тут сказано: „більше ніж на 3\%. В рукопису
% Маркса стоїть: „на 3\sfrac{3}{7}“, тобто дано абсолютне число. В процентах воно
% дорівнює 2\sfrac{6}{7}\%. \Red{Примітка ред. нім. вид. ІМЕЛ.}
% }

Отже, в наслідок підвищення заробітної плати на 25\%:

1)~для капіталу пересічного суспільного складу ціна виробництва
товару лишилась незмінною;

2)~для капіталу нижчого складу ціна виробництва товару
\parbreak{}  %% абзац продовжується на наступній сторінці

\input{_0204.tex}
\parcont{}  %% абзац починається на попередній сторінці
\index{iii1}{0205}  %% посилання на сторінку оригінального видання
Рікардо не досліджує), досить тільки перевернути щойно наведені
міркування.

I.~Пересічний капітал $= 80 c \dplus{} 20 v \deq{} 100$; норма додаткової
вартості \deq{} 100\%; ціна виробництва \deq{} товарній вартості $= 80 c \dplus{}
20 v \dplus{} 20 p \deq{} 120$; норма зиску \deq{} 20\%. Нехай заробітна плата
впаде на одну чверть, тоді той самий сталий капітал приводитиметься
в рух 15-ма $v$ замість $20 v$. Отже, ми маємо товарну
вартість $ \deq{} 80 c \dplus{} 15 v \dplus{} 25 p \deq{} 120$. Кількість праці, вироблена $v$,
лишається  незмінною, і тільки створена ним нова вартість інакше
розподіляється між капіталістом і робітниками. Додаткова вартість
підвищилась з 20 до 25, і норма додаткової вартості
підвищилась з \frac{20}{25} до \frac{25}{15}, отже, з 100\% до 166\sfrac{2}{3}\%.
Зиск на 95 тепер \deq{} 25, отже, норма зиску на 100 \deq{} 26\sfrac{6}{19}. Новий
процентний склад капіталу тепер є $84\sfrac{4}{19}c \dplus{} 15\sfrac{15}{19}v \deq{} 100$.

II.~Нижчий склад. Первісно $50 c \dplus{} 50 v$, як вище. В наслідок
падіння заробітної плати на \sfrac{1}{4}, $v$ зводиться до 37\sfrac{1}{2}, і тим самим
весь авансований капітал зводиться до $50 c \dplus{} 37\sfrac{1}{2}v \deq{} 87\sfrac{1}{2}$. Якщо
ми застосуємо до цього нову норму зиску в 26\sfrac{6}{19}\%, то
$100 : 26\sfrac{6}{19} \deq{} 87\sfrac{1}{2} : 23\sfrac{1}{38}$. Та сама товарна маса,
яка раніш коштувала 120, коштує
тепер $87\sfrac{1}{2} \dplus{} 23\sfrac{1}{38} \deq{} 110\sfrac{10}{19}$; падіння ціни
майже на 8\%.
% REMOVED \footnote*{
% В першому німецькому виданні тут стоїть: „майже на 10\%“. В рукопису
% Маркса в цьому місці сказано: „Знижується майже на 10“, тобто дається абсолютне
% число. Точно воно дорівнює 9\sfrac{9}{19} і становить 7\sfrac{7}{19}\%. \Red{Примітка ред. нім. вид. ІМЕЛ.}
% }

\looseness=1
III.~Вищий склад. Первісно $92 c \dplus{} 8 v \deq{} 100$. Падіння заробітної
плати на \sfrac{1}{4} знижує $8 v$ до $6 v$, весь капітал до 98. Отже,
$100 : 26\sfrac{6}{19} \deq{} 98 : 25\sfrac{15}{19}$. Ціна виробництва товару,
раніш $100 \dplus{} 20 \deq{} 120$, тепер, після падіння заробітної плати, є
$98 \dplus{} 25\sfrac{15}{19} \deq{} 123\sfrac{15}{19}$;
отже, вона підвищилась більше ніж на 3\%.
% REMOVED \footnote*{
% В першому німецькому виданні тут стоїть: „майже на 4\%“. В рукопису
% Маркса тут так само дається абсолютне число (3\sfrac{15}{19}). В процентах воно
% становить 3\sfrac{3}{19}\%. \Red{Примітка ред. нім. вид. ІМЕЛ.}
% }

Отже, ми бачимо, що досить тільки повторити попередні міркування в зворотному
напрямі і з відповідними змінами: загальне падіння заробітної плати має своїм
наслідком загальне підвищення додаткової вартості, норми додаткової вартості, а
при інших незмінних умовах і норми зиску, хоч і в
іншій пропорції; далі воно має своїм наслідком падіння цін виробництва для
товарних продуктів капіталів нижчого складу і підвищення цін виробництва для
товарних продуктів капіталів вищого складу. Результат якраз протилежний до того,
що  виявився при загальному підвищенні заробітної плати\footnote{
Надзвичайно дивно, що Рікардо (який, звичайно, застосовує іншого
методу, ніж це зроблено тут, бо не розуміє процесу вирівнення вартостей в ціни
виробництва) навіть не приходить до цієї думки, а розглядає тільки перший
випадок, підвищення заробітної плати і вплив його на ціни виробництва товарів
(„Principles etc.“, Лондон 1852, стор. 26 і далі]. A servum pecus
imitatorum [рабське стадо наслідувачів] не додумалось навіть до того, щоб
зробити це, само собою зрозуміле, по суті тавтологічне застосування.
}.
\parbreak{}  %% абзац продовжується на наступній сторінці

\input{_0206.tex}
\input{_0207.tex}

\index{iii1}{0208}  %% посилання на сторінку оригінального видання
\subsection{Ціна виробництва товарів середнього складу}

Ми бачили, яким чином відхилення цін виробництва від вартостей постає в наслідок того:

1) що до витрат виробництва товару додається не додаткова вартість, вміщена в ньому, а
пересічний зиск;

2) що ціна виробництва товару, яка таким чином відхиляється від вартості, входить як елемент
у витрати виробництва інших товарів, в наслідок чого, отже, вже у витратах виробництва товару може
міститись відхилення від вартості спожитих на нього засобів виробництва, незалежно від того
відхилення, що може постати для самого цього товару в наслідок ріжниці між пересічним зиском і
додатковою вартістю.

Таким чином, можливо, що і в товарів, вироблених капіталами середнього складу, витрати виробництва
відхилятимуться від суми вартості елементів, з яких складається ця складова частина їх ціни
виробництва. Припустім, що середній склад є $80 c + 20 v$. Можливо, що в дійсних капіталах, які мають
такий склад, $80  c$ більше або менше вартості $с$, сталого капіталу, бо це $с$ складається з товарів,
ціна виробництва яких відхиляється від їх вартості. Так само $20 v$  могли б відхилятися від своєї
вартості, якщо в споживання заробітної плати входять товари, ціна виробництва яких відрізняється від
їх вартості; отже, робітник, щоб купити ці товари (замістити їх), мусить витратити більше або менше
робочого часу, отже, мусить виконати більше або менше необхідної праці, ніж потрібно було б, коли б
ціни виробництва
необхідних засобів існування збігалися з їх вартостями.

Однак, ця можливість зовсім не міняє правильності положень, встановлених для товарів
середнього складу. Кількість зиску, що припадає на ці товари, дорівнює кількості вміщеної в них
самих додаткової вартості. Наприклад, при наведеному вище капіталі з складом у $80 с + 20 v$ для
визначення додаткової вартості важливе не те, чи ці числа є вирази дійсних вартостей, а те, як вони
відносяться одне до одного; а саме, що $v = \frac{1}{5}$, а $с = \frac{4}{5}$  всього капіталу. Якщо це так, то
додаткова вартість, вироблена $v$, дорівнює, як ми це припустили вище, пересічному зискові. З другого
боку: через те що додаткова вартість дорівнює пересічному зискові, ціна виробництва = витратам
виробництва + зиск = $k + p = k + m$, на практиці дорівнює вартості товару. Тобто підвищення або
зниження заробітної плати лишає $k + p$  в цьому випадку так само незмінним, як воно лишило б
незмінною вартість товару, і викликає тільки відповідний зворотний рух,
зниження або підвищення, на стороні норми зиску. А саме, якщо в наслідок підвищення або зниження
заробітної плати тут змінилася б ціна товарів, то норма зиску в цих сферах середнього складу стала б
вищою або нижчою порівняно з її рівнем в інших
\index{iii1}{0209}  %% посилання на сторінку оригінального видання
сферах. Лиш оскільки ціна лишається незмінною, сфера
середнього складу зберігає свій рівень зиску однаковим з іншими сферами. Отже, на практиці в цій
сфері справа відбувається цілком так само, як коли б продукти цієї сфери продавались по їх дійсній
вартості. А саме, якщо товари продаються по їх дійсних вартостях, то очевидно, що при інших
однакових умовах підвищення або зниження заробітної плати викликає відповідне зниження або підвищення зиску,
але не викликає ніякої зміни вартості товарів, і що при всіх обставинах підвищення або зниження
заробітної плати ніколи не може вплинути на вартість товарів, а завжди тільки на величину додаткової
вартості.

\subsection{Підстави капіталіста для компенсації}

Уже було сказано, що, конкуренція вирівнює норми зиску різних сфер виробництва в пересічну норму
зиску і саме тим перетворює вартості продуктів цих різних сфер виробництва в ціни виробництва. І це
стається саме в наслідок постійного перенесення капіталу з однієї сфери виробництва до іншої, де в
даний момент зиск стоїть вище пересічного рівня; при цьому, однак, слід взяти до уваги коливання
зиску, зв’язані з чергуванням худих і ситих років в даній галузі промисловості на протязі даного
періоду часу. Ця безперервна еміграція та імміграція капіталу, яка відбувається між різними сферами
виробництва, породжує висхідні і низхідні рухи норми зиску, які більше чи менше взаємно
урівноважуються і через це мають тенденцію повсюди зводити норму зиску до того самого спільного й загального рівня.

Цей рух капіталів завжди викликається в першу чергу станом ринкових цін, які в одному місці
підвищують зиск понад загальний пересічний рівень, в другому — знижують його нижче цього рівня. Ми
покищо залишаємо осторонь купецький капітал, з яким ми тут ще не маємо справи і який, як це
показують пароксизми спекуляції з певними улюбленими товарами, що раптово вибухають, може з
надзвичайною швидкістю витягати маси капіталу з одної галузі застосування і так само швидко кидати
їх до іншої. Але в кожній сфері виробництва у власному розумінні слова — в промисловості,
землеробстві, рудниках і т. д. — перенесення капіталу з однієї сфери в іншу становить значні
труднощі, особливо в наслідок наявності основного капіталу. До того ж досвід показує, що коли
яканебудь галузь промисловості,
наприклад, бавовняна промисловість, в певний час дає надзвичайно високий зиск, то вона потім, в
інший час, дає дуже незначний зиск, а то навіть і збиток, так що за певний цикл років пересічний
зиск в ній приблизно такий самий, як і в інших галузях. І капітал швидко привчається зважати на цей
досвід.

Але чого конкуренція \emph{не} показує, так це визначення вартості, яке керує рухом виробництва; так це
вартостей, які стоять за
\parbreak{}  %% абзац продовжується на наступній сторінці

\parcont{}  %% абзац починається на попередній сторінці
\index{iii1}{0210}  %% посилання на сторінку оригінального видання
цінами виробництва і в кінцевому рахунку визначають їх. Навпаки,
конкуренція показує: 1) пересічні зиски, які є незалежні від
органічного складу капіталу в різних сферах виробництва, отже,
і від маси живої праці, привласненої даним капіталом у даній сфері
експлуатації; 2) підвищення і падіння цін виробництва в наслідок
зміни висоти заробітної плати — явище, яке на перший
погляд цілком суперечить вартісному відношенню товарів;
3) коливання ринкових цін, які за даний період часу зводять
пересічну ринкову ціну товарів не до ринкової \emph{вартості}, а до
ринкової ціни виробництва, яка відхиляється від цієї ринкової
вартості, дуже відмінна від неї. Всі ці явища, як \emph{здається}, в такій
самій мірі суперечать визначенню вартості робочим часом,
як і природі додаткової вартості, яка складається з неоплаченої
додаткової праці. \emph{Отже, в конкуренції все з’являється у перекрученому
вигляді}. Економічні відносини в готовому вигляді, як
вони виявляються на поверхні, в їх реальному існуванні, отже,
і в тих уявленнях, за допомогою яких носії та агенти цих
відносин намагаються їх собі з’ясувати, дуже відрізняються
від їх внутрішньої, істотної, але скритої суті (Kerngestalt) та відповідного
цій суті поняття і в дійсності перекручені та протилежні
цій суті та відповідному їй поняттю.

Далі. Коли капіталістичне виробництво досягає певного ступеня
розвитку, вирівнення різних норм зиску окремих сфер виробництва
в одну загальну норму зиску зовсім не відбувається
тільки через гру притягання і відштовхування, за допомогою
якої ринкові ціни притягають або відштовхують капітал. Після
того, як за певний період часу встановились пересічні ціни
і відповідні їм ринкові ціни, до \emph{свідомості} окремих капіталістів
доходить, що в цьому процесі вирівнення вирівнюються \emph{певні
ріжниці}, так що вони відразу включають їх у свої взаємні розрахунки.
В уявленні капіталістів ці ріжниці живуть і включаються
ними в обрахунки як підстави для компенсації.

Основне уявлення при цьому є сам пересічний зиск, —
уявлення, що рівновеликі капітали за однакові періоди часу мусять
давати рівновеликі зиски. В основі цього уявлення знов
таки лежить уявлення, що капітал кожної сфери виробництва
повинен pro rata [пропорціонально] своїй величині брати участь
в сукупній додатковій вартості, видушеній з робітників сукупним
суспільним капіталом; або що кожний окремий капітал треба
розглядати тільки як частину сукупного капіталу, а кожного
капіталіста в дійсності — як акціонера спільного підприємства,
який бере участь в сукупному зиску pro rata величині своєї частини
капіталу.

На цьому уявленні базується потім обрахунок капіталіста.
Так, наприклад, якщо капітал обертається повільніше — або тому,
що товар довше затримується в процесі виробництва, або тому,
що він мусить бути проданий на віддалених ринках, — то зиск,
який в наслідок цього вислизає з рук капіталіста, він все ж нараховує,
\index{iii1}{0211}  %% посилання на сторінку оригінального видання
отже, відшкодовує себе тим, що робить надбавку до
ціни. Абож, коли капіталовкладення, яким загрожують дуже
великі небезпеки, як, наприклад, у мореплавстві, одержують відшкодування
шляхом надбавки до ціни. Як тільки капіталістичне
виробництво, а разом з ним і страхова справа досягають певного
ступеня розвитку, небезпека фактично стає однаковою для
всіх сфер виробництва (див. Корбет); але підприємства, яким
найбільше загрожує небезпека, платять вищу страхову премію
і відшкодовують себе за це в ціні своїх товарів. На практиці
все це зводиться до того, що кожна обставина, яка робить
певне капіталовкладення менш зисковним, а друге більш зисковним,
— а всі вони в певних межах вважаються однаково необхідними,
— включається в обрахунок як раз назавжди встановлена
підстава для компенсації, при чому вже немає потреби в новій
і новій діяльності конкуренції, щоб виправдати такий мотив або
фактор обрахунку. Капіталіст забуває тільки, — або, скоріше,
не бачить, бо конкуренція йому цього не показує, — що всі ці
підстави для компенсації, які капіталісти висувають один проти
одного у взаємному обчисленні товарних цін різних галузей
виробництва, базуються просто на тому, що всі капіталісти мають
pro rata [пропорціонально] їх капіталові однакові домагання
щодо спільної здобичі, сукупної додаткової вартості. Навпаки,
через те що одержаний ними зиск відрізняється від видушеної
ними додаткової вартості, їм \emph{здається}, що їх підстави для
компенсації не вирівнюють їх участі в сукупній додатковій
вартості, а \emph{створюють самий зиск}, бо цей останній, мовляв,
виникає просто з так чи інакше мотивованої надбавки до витрат
виробництва товарів.

У всьому іншому і для пересічного зиску має силу те, що
було сказано в розділі VII, стор. 148, про уявлення капіталіста
щодо джерела додаткової вартості. Тут справа стоїть інакше
лиш остільки, оскільки при даній ринковій ціні товарів і даній
експлуатації праці заощадження на витратах виробництва залежить
від індивідуальної вправності, уважності і~\abbr{т. д.}


  \parcont{}  %% абзац починається на попередній сторінці
\index{ii}{0212}  %% посилання на сторінку оригінального видання
Припустімо тепер, навпаки, незмінну величину періоду обороту, незмінний
маштаб продукції, але, з другого боку, зміну цін, тобто падіння або
підвищення цін на сировинні та допоміжні матеріяли й працю, або перших
двох з цих елементів. Припустімо, що ціна сировинних та допоміжних
матеріялів, так само, як і заробітна плата, зменшилась на половину.
Тоді в нашому прикладі треба було б авансованого капіталу щотижня
50\pound{ ф. стерл.} замість 100, а для дев'ятитижневого періоду обороту — 450\pound{ ф.
стерл.} замість 900\pound{ ф. стерл.} 450\pound{ ф. стерл.} авансованої капітальної
вартости виділюється насамперед як грошовий капітал, але процес продукції
триватиме й далі в тому самому маштабі, з тим самим періодом
обороту і з тим самим поділом останнього. Річна маса продукту лишається
теж та сама, але вартість її на половину зменшилась. Цю зміну, яку
супроводить і зміна в поданні та в попиті на грошовий капітал, спричиняє
не прискорення обігу й не зміна маси грошей, що циркулюють. Навпаки.
Зниження вартости, зглядно ціни елементів продуктивного капіталу
наполовину справило б насамперед той вплив, що авансувалось би капітальну
вартість, зменшену наполовину для того, шоб вести підприємство
X у попередніх розмірах, а що підприємство X авансує цю капітальну
вартість насамперед у формі грошей, тобто як грошовий капітал,
то, значить, воно мало б викидати на ринок лише половину попередньої
кількости грошей. Маса грошей, поданих в циркуляцію, зменшилась би
тому, що знизились ціни елементів продукції. Такий був би перший вплив.

Але подруге: половина первісно авансованої капітальної вартости в
900\pound{ ф. стерл.} \deq{} 450\pound{ ф. стерл.}, яка а) по черзі перебігала форму грошового
капіталу, продуктивного капіталу й товарового капіталу, б) яка одночасно
постійно перебувала почасти в формі грошового капіталу, почасти
в формі продуктивного капіталу, почасти в формі товарового капіталу,
в одній поряд однієї, — ця половина виділилась би з
кругобігу підприємства X і тому надійшла б як додатковий грошовий
капітал на грошовий ринок, впливаючи на нього як додаткова складова
частина. Ці звільнені гроші, 450\pound{ ф. стерл.}, впливають як грошовий капітал
не тому, що вони є гроші, які стали надлишкові для продовження
підприємства X, а тому, що вони є складова частина первісної капітальної
вартости, і тому повинні й далі діяти як капітал, а не витрачатись як
простий засіб циркуляції. Найближчий спосіб надати їм чинности капіталу,
це подати їх на грошовий ринок як грошовий капітал. З другого боку,
можна було б також збільшити розміри продукції, залишаючи осторонь
основний капітал, вдвоє. Авансуючи той самий капітал в 900\pound{ ф.
стерл.}, можна було б провадити процес продукції в подвоєному розмірі.

З другого боку, коли б ціни поточних елементів продуктивного капіталу
підвищились наполовину, то щотижня замість 100\pound{ ф. стерл.} треба
було б 150\pound{ ф. стерл.}, отже, замість 900\pound{ ф. стерл.} — 1350\pound{ ф. стерл}. Щоб
провадити підприємство в тому самому маштабі, треба було б 450\pound{ ф.
стерл.} додаткового капіталу, і це залежно від стану грошового ринку
справляло б на нього pro tanto більший або менший тиск. Коли б на
ввесь вільний капітал, що є та ринку, ставилося вже попит, то це при
\index{ii}{0213}  %% посилання на сторінку оригінального видання
звело б до підвищеної конкуренції за вільний капітал. Коли б деяка частина
його лежала без діла, те її pro tanto покликали б до діяльности.

Але, потретє, за даних розмірів продукції, за незмінної швидкости
обороту та незмінних цін елементів поточного продуктивного капіталу,
ціна продуктів підприємства X може знизитись або підвищитись. Коли
ціна товарів, що їх подає підприємство X, знижується, то спадає ціна
його товарового капіталу з 600\pound{ ф. стерл.}, що їх воно завжди подавало
в циркуляцію, напр., до 500\pound{ ф. стерл}. Отже, шоста частина вартости авансованого
капіталу не припливає назад з процесу циркуляції (додаткову вартість,
що є в товаровому капіталі, тут не береться на увагу); вона пропадає
марно в цьому процесі. Але що вартість, зглядно ціна елементів продукції
лишається та сама, то цих 500\pound{ ф. стерл.}, які приплили назад, вистачить
лише на те, щоб замістити \sfrac{5}{6} капіталу в 600 ф. стерл, ввесь час
занятого в процесі продукції. Отже, для того, щоб і далі провадити підприємство
в тому самому маштабі, довелось би витратити 100\pound{ ф. стерл.}
додаткового грошового капіталу.

Навпаки: коли ціна продуктів підприємства X підвищиться, то підвищиться
й ціна товарового капіталу з 600\pound{ ф. стерл.}, напр., до 700\pound{ ф.
стерл}. Сьома частина його ціни, рівна 100\pound{ ф. стерл.}, приходить не з
процесу продукції, не була авансована на нього, а припливає сюди з
процесу циркуляції. Однак, для заміщення продуктивних елементів треба
лише 600\pound{ ф. стерл.}; отже, 100\pound{ ф. стерл.} звільняються.

Дослідження причин, чому в першому випадку період обороту
скорочується або подовжується, в другому випадку ціни на сировинний
матеріял та працю, і в третьому ціни поданих продуктів підвищуються
або падають, — дослідження цих причин не входить у межі цього досліду.

Але ось що входить у межі його:

\textbf{I випадок.} \emph{Незмінний маштаб продукції, незмінні ціни елементів
продукції та продуктів, зміна в періоді циркуляції, а значить, і в періоді
обороту.}

Згідно з припущенням у нашому прикладі, в наслідок скорочення періоду
циркуляції, треба авансувати всього капіталу менше на \sfrac{1}{9}; тому
капітал цей зменшується з 900 до 800\pound{ ф. стерл.}, і виділюється грошовий
капітал в 100\pound{ ф. стерл}.

Як і раніш, підприємство X дає той самий шеститижневий продукт такої
самої вартости в 600\pound{ ф. стерл.}, а що роблять цілий рік безперервно, то
воно протягом 51-го тижня дає ту саму масу продукту, вартістю в 5100\pound{ ф.
стерл}. Отже, в масі та ціні продукту, що його подає підприємство в
циркуляцію, немає жодної зміни, немає її і в тих строках, що в них підприємство
подає продукт на ринок. Але виділилось 100\pound{ ф. стерл.}, бо
через скорочення періоду циркуляції процес насичено авансуванням капіталу
лише в 800\pound{ ф. стерл.} замість попередніх 900\pound{ ф. стерл}. Ці 100\pound{ ф.
стерл.} виділеного капіталу існують у формі грошового капіталу. Але вони
зовсім не репрезентують тієї частини авансованого капіталу, що постійно
мусить функціонувати в формі грошового капіталу. Припустімо, що з
авансованого поточного капіталу I \deq{} 600\pound{ ф. стерл.} \sfrac{4}{5} \deq{} 480\pound{ ф. стерл.}
\parbreak{}  %% абзац продовжується на наступній сторінці

\parcont{}  %% абзац починається на попередній сторінці
\index{iii1}{0214}  %% посилання на сторінку оригінального видання
показано, чому це зниження виступає не в цій абсолютній формі,
а більше в тенденції до прогресивного падіння.) Отже, прогресуюча
тенденція загальної норми зиску до зниження є тільки
\emph{властивий} \emph{капіталістичному способові виробництва вираз}
прогресуючого розвитку суспільної продуктивної сили праці.
Цим не сказано, що норма зиску не може тимчасово падати і
з інших причин, але цим доведено, як само собою зрозумілу
з суті капіталістичного способу виробництва необхідність, що
з розвитком цього способу виробництва загальна пересічна норма
додаткової вартості мусить виражатись у падаючій загальній
нормі зиску. Через те що маса вживаної живої праці постійно
зменшується порівняно з масою упредметненої праці, яку вона
приводить в рух, порівняно з масою продуктивно споживаних
засобів виробництва, то й відношення тієї частини цієї живої
праці, яка неоплачена і упредметнюється в додатковій вартості,
до розміру вартості всього вживаного капіталу мусить постійно
зменшуватись. Але це відношення маси додаткової вартості до
вартості всього вживаного капіталу становить норму зиску, яка
через це мусить постійно падати.

Хоч і яким простим здається цей закон після того, що ми досі
розвинули, проте всій дотеперішній політичній економії не вдалося
відкрити його, як ми це побачимо в одному з дальших відділів.
Вона бачила явище і мучилася в суперечливих спробах
пояснити його. Але при тій великій важливості, яку цей закон
має для капіталістичного виробництва, можна сказати, що він
становить таємницю, над розв’язанням якої б’ється вся політична
економія від часів Адама Сміта, і що ріжниця між різними школами
від часів А. Сміта полягає в різних спробах розв’язати цю
таємницю. З другого ж боку, якщо взяти до уваги, що дотеперішня
політична економія хоч напомац і підходила до розрізнення
сталого і змінного капіталу, але ніколи не спромоглась
ясно сформулювати його; що вона ніколи не представляла додаткову
вартість відокремлено від зиску, а зиск взагалі ніколи
не представляла у чистому вигляді в відміну від його різних
усамостійнених одна проти одної складових частин, — як промисловий
зиск, торговельний зиск, процент, земельна рента; що
вона ніколи грунтовно не аналізувала ріжниці в органічному
складі капіталу, а тому й утворення загальної норми зиску, —
то перестає бути загадковим те, що їй ніколи не вдавалося розв’язати
цю загадку.

Ми навмисно виклали цей закон раніше, ніж показали розпад
зиску на різні усамостійнені одна проти одної категорії. Незалежність
цього викладу від розпаду зиску на різні частини,
які припадають різним категоріям осіб, прямо доводить незалежність
закону в його всезагальності від такого розпаду і від
взаємних відношень між категоріями зиску, які виникають з цього
розпаду. Зиск, про який ми тут говоримо, є тільки інша назва
самої додаткової вартості, яка тільки представлена у відношенні
\parbreak{}  %% абзац продовжується на наступній сторінці

\parcont{}  %% абзац починається на попередній сторінці
\index{iii1}{0215}  %% посилання на сторінку оригінального видання
до всього капіталу, а не у відношенні до змінного капіталу, з
якого вона виникає. Отже, падіння норми зиску виражає спадаюче
відношення самої додаткової вартості до всього авансованого
капіталу, і тому воно незалежне від будь-якого розподілу
цієї додаткової вартості між різними категоріями.

Ми бачили, що на певному ступені капіталістичного розвитку,
коли склад капіталу $c : v \deq{} 50 : 100$, норма додаткової вартості
в 100\% виражається в нормі зиску в 66\sfrac{2}{3}\% і що на вищому
ступені розвитку, коли $c : v$ як $400 : 100$, та сама норма додаткової
вартості виражається в нормі зиску тільки в 20\%. Те, що
стосується до різних послідовних ступенів розвитку в одній
країні, стосується і до різних ступенів розвитку, які існують
одночасно один поряд одного в різних країнах. У нерозвиненій
країні, де перший склад капіталу є пересічний, загальна норма
зиску була б \deq{} 66\sfrac{2}{3}\%, тимчасом як у країні другого складу капіталу,
з значно вищим ступенем розвитку, вона була б \deq{} 20\%.

Ріжниця обох національних норм зиску могла б зникнути і
навіть стати протилежною в наслідок того, що в менш розвиненій
країні праця була б менш продуктивною, тому більша
кількість праці виражалася б у меншій кількості того самого
товару, більша мінова вартість виражалася б у меншій споживній
вартості, отже, робітник мусив би вживати більшу частину
свого часу на репродукцію своїх власних засобів існування або
їх вартості і меншу частину на створення додаткової вартості,
давав би менше додаткової праці, так що норма додаткової
вартості була б нижча. Якщо, наприклад, у менш розвиненій країні
робітник працював би \sfrac{2}{3} робочого дня на себе самого і \sfrac{1}{3} на
капіталіста, то, зберігаючи припущення вищенаведеного прикладу,
та сама робоча сила оплачувалася б у розмірі 133\sfrac{1}{3} і дала б
надлишок тільки в 66\sfrac{2}{3}. Змінному капіталові в 133\sfrac{1}{3} відповідав
би сталий капітал в 50. Отже, норма додаткової вартості становила
б тут $133\sfrac{1}{3} : 66\sfrac{2}{3}= 50\%$, а норма зиску $183\sfrac{1}{3}:
66\sfrac{2}{3}$ або приблизно 36\sfrac{1}{2}\%.

Через те що ми досі ще не дослідили різних складових частин,
на які розпадається зиск, — отже, вони для нас ще не існують,
— то ми тільки для того, щоб уникнути непорозумінь,
зауважимо наперед таке. При порівнянні країн різних ступенів
розвитку, а саме країн з розвиненим капіталістичним виробництвом
і таких, де праця ще формально не підпорядкована капіталові,
хоча в дійсності робітник експлуатується капіталістом
(наприклад, в Індії, де райот господарює як самостійний селянин,
отже, його виробництво, як таке, ще не підпорядковане капіталові,
хоч лихвар може видушити з нього в формі процента
не тільки всю його додаткову працю, але навіть — капіталістично
висловлюючись — частину його заробітної плати), було б
великою помилкою, коли б хтонебудь схотів міряти висоту національної
норми зиску висотою національного рівня процента.
В такому проценті міститься весь зиск і навіть більше ніж зиск,
\parbreak{}  %% абзац продовжується на наступній сторінці

\parcont{}  %% абзац починається на попередній сторінці
\index{iii1}{0216}  %% посилання на сторінку оригінального видання
тимчасом як у країнах розвиненого капіталістичного виробництва
процент виражає тільки відповідну частину виробленої додаткової
вартості або зиску. З другого боку, тут рівень процента переважно
визначається такими відносинами (позики лихварів знаті,
власникам земельної ренти), які не мають нічого спільного з
зиском, а, навпаки, показують тільки, в якій мірі лихвар привласнює
собі земельну ренту.

В країнах різного ступеня розвитку капіталістичного виробництва
і тому різного органічного складу капіталу норма
додаткової вартості (один з факторів, що визначають норму зиску)
може стояти вище в тій країні, де нормальний робочий день
коротший, ніж у тій країні, де він довший. \emph{Поперше}, якщо
англійський робочий день у 10 годин в наслідок своєї вищої
інтенсивності дорівнює австрійському робочому дневі в 14 годин,
то при однаковому розподілі робочого дня 5 годин додаткової
праці англійця можуть на світовому ринку представляти
вищу вартість, ніж 7 годин австрійця. А \emph{подруге}, в Англії
додаткову працю може становити більша частина робочого дня,
ніж в Австрії.

Закон спадаючої норми зиску, в якій виражається та сама або
навіть зростаюча норма додаткової вартості, означає, інакше
кажучи, таке: якщо взяти якусь певну кількість пересічного
суспільного капіталу, наприклад, капітал в 100, то частина
його, представлена в засобах праці, дедалі зростає, а частина,
представлена в живій праці, дедалі зменшується. Отже, через
те що вся маса живої праці, додаваної до засобів виробництва,
зменшується порівняно з вартістю цих засобів виробництва,
то порівняно з вартістю всього авансованого капіталу
зменшується також і неоплачена праця і та частина вартості,
в якій вона виражається. Або: з усього витраченого капіталу
все менша й менша частина перетворюється в живу працю,
і тому весь цей капітал вбирає порівняно з своєю величиною
все менше й менше додаткової праці, хоч одночасно з цим відношення
неоплаченої частини вживаної праці до її оплаченої частини
може зростати. Відносне зменшення змінного і збільшення
сталого капіталу, хоч обидві ці частини абсолютно зростають,
є, як ми вже сказали, тільки інший вираз зростаючої продуктивності
праці.

Припустім, що капітал в 100 складається з $80c \dplus{} 20v$, а ці
останні \deq{} 20 робітникам. Норма додаткової вартості нехай буде
100\%, тобто робітники працюють півдня на себе, півдня на капіталіста.
Нехай у другій, менш розвиненій країні капітал буде
$20c \dplus{} 80v$, і ці останні \deq{} 80 робітникам. Але цим робітникам потрібно
\sfrac{2}{3} робочого дня для себе й тільки \sfrac{1}{3} вони працюють на
капіталіста. При всіх інших однакових умовах, у першому випадку
робітники виробляють вартість в 40, у другому — в 120.
Перший капітал виробляє $80c \dplus{} 20v \dplus{} 20m \deq{} 120$; норма зиску \deq{}
20\%; другий капітал $20c \dplus{} 80v \dplus{} 40m \deq{} 140$; норма зиску
\index{iii1}{0217}  %% посилання на сторінку оригінального видання
\deq{} 40\%. Отже, в другому випадку вона вдвоє більша, ніж у першому,
хоч у першому випадку норма додаткової вартості, \deq{} 100\%,
вдвоє більша, ніж у другому випадку, де вона становить тільки
50\%. Але зате однакової величини капітал привласнює собі в першому
випадку додаткову працю тільки 20, а в другому 80 робітників.

Закон прогресуючого падіння норми зиску або відносного
зменшення привласнюваної додаткової праці порівняно з масою
упредметненої праці, яка приводиться в рух живою працею, аж
ніяк не виключає зростання абсолютної маси праці, яка приводиться
в рух і експлуатується суспільним капіталом, а тому й зростання
абсолютної маси привласнюваної ним додаткової праці; так само
цей закон не виключає того, що капітали, які є в розпорядженні
окремих капіталістів, командують дедалі більшою масою праці,
а тому й додаткової праці, — останнє навіть у тому випадку,
коли число робітників, якими вони командують, не зростає.

Якщо взяти робітниче населення даної чисельності, наприклад,
два мільйони, якщо взяти, далі, як дані, довжину і інтенсивність
пересічного робочого дня, а також заробітну плату,
а разом з тим і відношення між необхідною і додатковою працею,
то сукупна праця цих двох мільйонів, а також їх додаткова
праця, яка виражається в додатковій вартості, завжди виробляє
вартість однакової величини. Але з зростанням маси сталого
— основного і обігового — капіталу, який приводиться в рух
цією працею, падає відношення цієї величини вартості до вартості
цього капіталу, яка зростає разом з його масою, хоч і не в тій
самій пропорції. Це відношення, а тому й норма зиску, падає, хоч
капітал командує такою самою масою живої праці, як і раніше,
і вбирає таку саму масу додаткової праці. Відношення змінюється
не тому, що зменшується маса живої праці, а тому, що збільшується
маса упредметненої вже праці, яку вона приводить в рух.
Зменшення тут відносне, не абсолютне, і в дійсності нічим
не зв’язане з абсолютною величиною приведеної в рух праці
й додаткової праці. Падіння норми зиску виникає не з абсолютного,
а тільки з відносного зменшення змінної складової частини
всього капіталу, з її зменшення. порівняно з сталою складовою
частиною.

Те саме, що має значення для даної маси праці і маси додаткової
праці, має значення і для зростаючого числа робітників, а тому, при
даних припущеннях, і для зростаючої маси праці, яка взагалі
є в розпорядженні, і зокрема для її неоплаченої частини, для
додаткової праці. Якщо робітниче населення зростає з двох мільйонів
до трьох, якщо змінний капітал, виплачений йому в формі
заробітної плати, так само становив раніше два мільйони, а тепер
становить три мільйони, а сталий капітал, навпаки, підвищується
з 4 до 15 мільйонів, то при даних припущеннях (незмінний
робочий день і незмінна норма додаткової вартості) маса додаткової
праці, додаткової вартості зростає наполовину, на 50\%,
\parbreak{}  %% абзац продовжується на наступній сторінці

\parcont{}  %% абзац починається на попередній сторінці
\index{iii1}{0218}  %% посилання на сторінку оригінального видання
з двох мільйонів до трьох. Проте, не зважаючи на це зростання
абсолютної маси додаткової праці, а тому й додаткової
вартості на 50\%, відношення змінного капіталу до сталого
впало б з $2 : 4$ до $3 : 15$, і відношення додаткової вартості до
всього капіталу було б таке (в мільйонах):
\begin{gather*}
\text{\phantom{I}I. }\phantom{1}4c \dplus{} 2v \dplus{} 2m; K \deq{} \phantom{1}6, p' \deq{} 33\sfrac{1}{3}\%.\\
\text{II. }15c \dplus{} 3v \dplus{} 3m; K \deq{} 18, p' \deq{} 16\sfrac{1}{3}\%.
\end{gather*}
\noindent{}Тимчасом як маса додаткової вартості підвищилась наполовину,
норма зиску впала наполовину порівняно з попередньою.
Але зиск є тільки додаткова вартість, обчислена на суспільний
капітал, і тому маса зиску, його абсолютна величина, розглядувана
з точки зору всього суспільства, дорівнює абсолютній величині
додаткової вартості. Отже, абсолютна величина зиску, його
сукупна маса, зросла б на 50\%, не зважаючи на величезне зменшення
цієї маси зиску відносно авансованого сукупного капіталу
або не зважаючи на величезне зменшення загальної норми зиску.
Отже, число вживаних капіталом робітників, тобто абсолютна
маса праці, яка ним приводиться в рух, тому й абсолютна маса
вбираної ним додаткової праці, тому й маса виробленої ним додаткової
вартості, тому й абсолютна маса виробленого ним зиску
\emph{може} зростати і зростати прогресивно, не зважаючи на прогресивне
падіння норми зиску. Це не тільки \emph{може} бути. Це — залишаючи
осторонь минущі коливання — \emph{мусить} так бути на базі
капіталістичного виробництва.

Капіталістичний процес виробництва є разом з тим істотно і
процес нагромадження. Ми показали, як з розвитком капіталістичного
виробництва маса вартості, яка мусить бути просто
репродукована, збережена, збільшується і зростає разом з
підвищенням продуктивності праці, навіть якщо вживана робоча
сила лишається незмінною. Але з розвитком суспільної продуктивної
сили праці ще більше зростає маса вироблюваних споживних
вартостей, частину яких становлять засоби виробництва.
А добавна праця, через привласнення якої це додаткове багатство
може бути знову перетворене в капітал, залежить не від вартості,
а від маси цих засобів виробництва (включаючи й засоби
існування), бо в процесі праці робітник має справу не з вартістю,
а з споживною вартістю засобів виробництва. Однак, самонагромадження
і дана разом з ним концентрація капіталу є
матеріальний засіб підвищення продуктивної сили. Але це зростання
засобів виробництва передбачає зростання робітничого
населення, створення населення робітників, яке відповідає додатковому
капіталові і загалом і в цілому навіть завжди перевищує
його потреби, отже, створення перенаселення робітників.
Тимчасовий надлишок додаткового капіталу порівняно з робітничим
населенням, яке є в його розпорядженні, справляв би
двоякий вплив. З одного боку, він ступнево збільшував би робітниче
\index{iii1}{0219}  %% посилання на сторінку оригінального видання
населення шляхом підвищення заробітної плати, отже,
пом’якшенням згубних впливів, що скорочують приріст робітників,
і полегшенням шлюбів; а з другого боку, шляхом застосування
методів, які створюють відносну додаткову вартість (введення
й поліпшення машин), він ще далеко швидше створив би
штучне відносне перенаселення, яке з свого боку — бо в капіталістичному
виробництві злидні породжують населення, — знов таки є
теплицею дійсного швидкого збільшення чисельності населення.
Тому з природи капіталістичного процесу нагромадження —
який є тільки моментом капіталістичного процесу виробництва —
само собою випливає, що збільшена маса засобів виробництва,
призначених для перетворення в капітал, завжди знаходить під
рукою відповідно збільшене і навіть надлишкове робітниче населення,
яке можна експлуатувати. Отже, з розвитком процесу
виробництва і нагромадження \emph{мусить} зростати маса придатної
до привласнення і привласнюваної додаткової праці, а тому й абсолютна
маса зиску, привласнюваного суспільним капіталом. Але ті
самі закони виробництва і нагромадження разом з масою сталого
капіталу підвищують у дедалі більшій прогресії і його вартість, —
швидше, ніж вони підвищують вартість змінної частини капіталу,
обмінюваної на живу працю. Отже, одні й ті самі закони зумовлюють
для суспільного капіталу зростаючу абсолютну масу
зиску і падаючу норму зиску.

Ми тут цілком залишаємо осторонь те, що та сама величина
вартості з прогресом капіталістичного виробництва і відповідного
йому розвитку продуктивної сили суспільної праці та при помноженні
галузей виробництва, отже й продуктів, представляє прогресивно
зростаючу масу споживних вартостей і насолод.

Хід розвитку капіталістичного виробництва і нагромадження
зумовлює процеси праці в дедалі більшому масштабі, отже, в дедалі
більших розмірах, і відповідно до цього зумовлює зростаюче
авансування капіталу на кожне окреме підприємство. Тому
зростаюча концентрація капіталів (супроводжена в той самий
час, хоч і в меншій мірі, зростанням числа капіталістів) є так
само однією з матеріальних умов капіталістичного виробництва
і нагромадження, як і одним із створюваних ним самим результатів.
Рука в руку і у взаємодії із цим відбувається прогресуюча експропріація
більш чи менш безпосередніх виробників. Таким чином
для одиничних капіталістів стає зрозумілим, що вони мають
у своєму розпорядженні дедалі зростаючі робітничі армії (як би
сильно не падав їх змінний капітал порівняно з сталим), що маса
привласнюваної ними додаткової вартості, а тому й зиску, зростає
одночасно з падінням норми зиску і не зважаючи на це падіння.
Якраз ті самі причини, які концентрують маси робітничих армій
під командою окремих капіталістів, збільшують також масу застосовуваного
основного капіталу, як і сировинних та допоміжних
матеріалів,— збільшують відносно швидше, ніж масу вживаної
живої праці.

\parcont{}  %% абзац починається на попередній сторінці
\index{ii}{0220}  %% посилання на сторінку оригінального видання
\frac{\num{25.000}}{5} \deq{} 5000\pound{ ф. стерл}. Коли поділити ці 5000\pound{ ф. стерл.} на 500, то матимемо число оборотів 10,
цілком таке саме, як і для цілого капіталу в 2500\pound{ ф. стерл}.

Це пересічне обчислення, що за ним вартість річного продукту ділиться на вартість авансованого
капіталу, а не на вартість частини цього капіталу, постійно застосовуваної в одному робочому періоді
(отже, в нашому прикладі, не на 400, а на 500, не на капітал І, а на капітал І \dplus{} капітал II), — це
пересічне обчислення тут, де йдеться лише про продукцію додаткової вартости, є абсолютно точне. Далі
ми побачимо, що, з іншого погляду, воно не зовсім точне, як і взагалі це пересічне обчислення не
зовсім точне. Інакше кажучи, воно задовільне для практичних цілей капіталіста,
але воно не виражає точно й гаразд усіх реальних обставин обороту.

Досі ми одну частину вартости товарового капіталу лишали цілком осторонь, а саме вміщену в ньому
додаткову вартість, спродуковану та долучену до продукту протягом процесу продукції. На неї тепер і
треба нам звернути увагу.

Коли припустити, що витрачуваний щотижня змінний капітал в 100\pound{ ф. стерл.}, продукує додаткову
вартість в 100\% \deq{} 100\pound{ ф. стерл.}, то змінний капітал в 500\pound{ ф. стерл.},  витрачуваний протягом
п’ятитижневого періоду обороту, випродукує додаткову вартість в 500\pound{ ф. стерл.}, тобто половина
робочого дня складається з додаткової праці.

Але коли 500\pound{ ф. стерл.} змінного капіталу продукують 500\pound{ ф. стерл.} додаткової вартости, то 5000\pound{ ф.
стерл.} випродукують її 500 × 10 \deq{} 5000\pound{ ф. стерл}. Але авансований змінний капітал \deq{} 500\pound{ ф. стерл}.
Відношення всієї маси додаткової вартости, спродукованої протягом року, до суми вартости
авансованого змінного капіталу ми звемо річною нормою додаткової вартости. Отже, в даному випадку,
вона \deq{} \frac{5000}{500} \deq{} 1000\%.
Коли ближче аналізувати цю норму, то виявиться, що вона дорівнює тій нормі додаткової вартости, яку
авансований змінний капітал продукує протягом одного періоду обороту, помноженій на число оборотів
змінного капіталу (а воно збігається з числом оборотів цілого обігового капіталу).

Авансований протягом одного періоду обороту змінний капітал в даному випадку \deq{} 500\pound{ ф. стерл.};
створена ним додаткова вартість теж \deq{} 500\pound{ ф. стерл}. Тому норма додаткової вартости протягом одного
періоду обороту \deq{} \frac{500m}{500v} \deq{} 100\%. Ці 100\%, помножені на 10, на число оборотів протягом року,
дають \frac{5000m}{5000v} \deq{} 1000\%.

Це має силу щодо річної норми додаткової вартости. Щождо маси додаткової вартости, здобуваної
протягом певного періоду обороту, то ця маса дорівнює вартості авансованого протягом цього періоду
змінного капіталу — в даному випадку \deq{} 500\pound{ ф. стерл.}, помноженій на норму
\parbreak{}  %% абзац продовжується на наступній сторінці

\parcont{}  %% абзац починається на попередній сторінці
\index{ii}{0221}  %% посилання на сторінку оригінального видання
додаткової вартости, в даному випадку, отже, 500 × \frac{100}{100} \deq{} 500 × 1 \deq{} 500\pound{ ф. стерл}. Коли б
авансований капітал був \deq{} 1500\pound{ ф. стерл.} при незмінній
нормі додаткової вартости, то маса додаткової вартости була б \deq{}
1500 × \frac{100}{100} \deq{} 1500\pound{ ф. стерл}.

Змінний капітал у 500\pound{ ф. стерл.}, що обертається 10 разів на рік, і
що продукує протягом року додаткову вартість в 5000\pound{ ф. стерл.}, отже,
капітал, що для нього річна норма додаткової вартости \deq{} 1000\%, ми
будемо називати капіталом А.

Припустімо тепер, що інший змінний капітал В в 5000\pound{ ф. стерл.}
авансується на цілий рік (тобто, тут на 50 тижнів) і тому обертається
лише один раз на рік. Припустімо при цьому далі, що наприкінці року
продукт оплачується в той самий день, як його виготовлено, і, значить,
грошовий капітал, що на нього його перетворюється, повертається в той
самий день. Отже, період циркуляції тут \deq{} 0, період обороту дорівнює
робочому періодові, а саме, одному рокові. Як і в попередньому випадку,
в процесі праці щотижня перебуває змінний капітал в 100\pound{ ф. стерл.},
а тому протягом 50 тижнів — в 5000\pound{ ф. стерл}. Далі, норма додаткової
вартости хай буде та сама \deq{} 100\%, тобто за однакової довжини робочого
дня половина його складається з додаткової праці. Коли ми візьмемо
5 тижнів, то вкладений змінний капітал \deq{} 500\pound{ ф. стерл.}, норма додаткової
вартости \deq{} 100\%, отже, маса додаткової вартости, створена протягом
5 тижнів \deq{} 500\pound{ ф. стерл}. Кількість робочої сили, що її тут експлуатується,
і ступінь її експлуатації, згідно з нашим припущенням, тут
точно такі самі, як і при капіталі А.

Вкладений змінний капітал в 100\pound{ ф. стерл.} щотижня створює додаткову
вартість в 100\pound{ ф. стерл.}, тому протягом 50 тижнів вкладений капітал
в 100 × 50 \deq{} 5000\pound{ ф. стерл.} створить додаткову вартість в 5000\pound{ ф. стерл}. Маса щороку створюваної
додаткової вартости буде така сама, як і в попередньому випадку \deq{} 5000\pound{ ф. стерл.}, але річна норма
додаткової
вартости цілком інша. Вона дорівнює спродукованій протягом року
додатковій вартості, поділеній на авансований змінний капітал:
\frac{5000m}{5000v} \deq{} 100\%, тимчасом як раніш для капіталу А вона дорівнювала 1000\%.

При капіталі А, як і при капіталі В, ми витрачали щотижня 100\pound{ ф. стерл.} змінного капіталу; ступінь
зростання вартости або норма додаткової
вартости цілком та сама, вона дорівнює 100\%; величина змінного
капіталу теж та сама \deq{} 100\pound{ ф. стерл}. Експлуатується цілком таку
саму кількість робочої сили, величина й ступінь експлуатації в обох випадках
однакові, робочі дні однакові і однаково поділяються на доконечну
й додаткову працю. Сума змінного капіталу, застосованого протягом
року, однакова величиною \deq{} 5000\pound{ ф. стерл.}, вона пускає в рух таку
саму масу праці й витягує з робочої сили, пущеної в рух обома рівними
капіталами, однакову масу додаткової вартости, 5000\pound{ ф. стерл}. І,
\parbreak{}  %% абзац продовжується на наступній сторінці

\input{_0222.tex}
\parcont{}  %% абзац починається на попередній сторінці
\index{iii1}{0223}  %% посилання на сторінку оригінального видання
бо вона мусить зрости навіть для того, щоб при зміненому
складі капіталу можна було вживати ту саму масу праці при
попередніх відношеннях експлуатації.

Отже, той самий розвиток суспільної продуктивної сили
праці виражається з прогресом капіталістичного способу виробництва,
з одного боку, в тенденції до прогресуючого падіння
норми зиску, а з другого боку, в постійному зростанні абсолютної
маси привласнюваної додаткової вартості або зиску; так
що загалом відносному зменшенню змінного капіталу і зиску
відповідає абсолютне збільшення обох. Ця двобічна дія, як ми
вже показали, може виразитись тільки в зростанні всього капіталу
в швидшій прогресії, ніж та, в якій падає норма зиску. Для
того, щоб при вищому складі капіталу або при відносно сильнішому
збільшенні сталого капіталу можна було вжити абсолютно
зрослий змінний капітал, весь капітал мусить зрости не
тільки відповідно до вищого складу, але ще швидше. З цього
випливає, що чим більше розвивається капіталістичний спосіб
виробництва, тим більша й більша маса капіталу потрібна для
того, щоб уживати ту саму робочу силу, і ще більша для того,
щоб уживати вирослу робочу силу. Отже, зростаюча продуктивна
сила праці на капіталістичній базі з необхідністю створює
постійне позірне перенаселення робітників. Якщо змінний капітал
становить тільки \sfrac{1}{6} всього капіталу замість колишньої
\sfrac{1}{2}, то, щоб можна було вжити ту саму робочу силу, весь
капітал мусить потроїтись; а для того, щоб можна було вжити
подвійну робочу силу, він мусить пошестеритись.

Дотеперішня політична економія, яка не зуміла була пояснити
закон падіння норми зиску, вказувала на підвищення маси
зиску, зростання абсолютної величини зиску, чи то для окремих
капіталістів, чи для суспільного капіталу, як на свого роду
підставу для утішення, але й вона базується на самих тільки
загальних місцях і можливостях.

Те, що маса зиску визначається двома факторами, поперше,
нормою зиску і, подруге, масою капіталу, вжитого для одержання
цієї норми зиску, — це просто тавтологія. Тому та обставина,
що зростання маси зиску можливе, не зважаючи на одночасне
падіння норми зиску, є тільки вираз цієї тавтології і
не допомагає ні на крок посунутися вперед, бо цілком так само
можливе й те, що капітал зростатиме без зростання маси зиску
і що він може навіть зростати і в тому випадку, коли вона
падає. 100 при 25\% дає 25, 400 при 5\% дає тільки 20.\footnote{
„We should also expect that, however the rate of the profits of stock
might diminish in consequence of the accumulation of capital on the land and the
rise of wages, yet the aggregate amount of profits would increase. Thus, supposing
that, with repeated accumulations of \num{100000}\pound{ £}, the rate of profits should fall from
20 to 19, to 18, to 17 per cent., a constantly diminishing rate; we should expect that
the whole amount of profits received by those successive owners of capital would be
always progressive; that it would be greater when the capital was \num{200000}\pound{ £}, than
when \num{100000}\pound{ £}; still greater when \num{300000}\pound{ £}; and so on, increasing, though at a
diminishing rate, with every increase of capital. This progression, howewer, is only
true for a certain time; thus, 19 per cent, on \num{200000}\pound{ £} is more than 20 on \num{100000}\pound{ £};
again 18 per cent on \num{300000}\pound{ £} is more than 19 per cent, on \num{200000}\pound{ £}; but after
capital has accumulated to a large amount, and profits have fallen, the further
accumulation diminishes the aggregate of profits. Thus, suppose the accumulation
should be \num{1000000}\pound{ £}, and the profits 7 per cent., the whole amount of profits will be
\num{70000}\pound{ £}; now if an addition of \num{100000}\pound{ £} capital bemade to the million, and profits should
fall to 6 per cent., \num{66000}\pound{ £} or a diminution of 4000\pound{ £} will be received by the owners
of stock, although the whole amount of stock will be increased from \num{1000000}\pound{ £} to
\num{1100000}\pound{ £}.“ [„Нам слід, отже, сподіватися, що хоча норма зиску на капітал може
зменшитися в наслідок нагромадження капіталу в країні і підвищення заробітної
плати, однак загальна сума зиску збільшиться. Так, якщо ми припустимо, що при
послідовному нагромадженні \num{100000}\pound{ фунтів стерлінгів} норма зиску впаде з 20\% до
19\%, до 18\% і до 17\%, тобто постійно зменшуватиметься, то слід сподіватися,
що вся сума зиску, одержувана цими послідовними власниками капіталу, постійно
зростатиме; що вона буде більша при капіталі в \num{200000}\pound{ фунтів стерлінгів},
ніж при капіталі в \num{100000}\pound{ фунтів стерлінгів}, і ще більша при капіталі
в \num{300000}\pound{ фунтів стерлінгів} і~\abbr{т. д.}, зростаючи з кожним збільшенням капіталу,
не зважаючи на зменшення норми. Однак, таке зростання має місце тільки на
протязі певного часу; так, 19\% від \num{200000}\pound{ фунтів стерлінгів} є більше, ніж 20\%
від \num{100000}\pound{ фунтів стерлінгів}, 18\% від \num{300000}\pound{ фунтів стерлінгів} знов таки
більше, ніж 19\% від \num{200000}\pound{ фунтів стерлінгів}; але після того, як капітал уже
нагромадився до великої суми, а зиски зменшились, дальше нагромадження
зменшує загальну суму зиску. Так, якщо припустимо, що нагромадження
становить \num{1000000}\pound{ фунтів стерлінгів}, а зиск 7\%, то загальна сума зиску становитиме
\num{70000}\pound{ фунтів стерлінгів}; якщо тепер до капіталу в мільйон буде
додано \num{100000}\pound{ фунтів стерлінгів} і зиск знизиться до 6\%, то власники капіталу
одержать \num{66000}\pound{ фунтів стерлінгів}, або на 4000\pound{ фунтів стерлінгів} менше, хоч
загальна сума капіталу зросла з \num{1000000}\pound{ фунтів стерлінгів} до \num{1100000}\pound{ фунтів
стерлінгів}“. \emph{Ricardo}: „Principles of Political Economy“, розд. VII („Works“
вид. Мак-Куллоха, 1852, стор. 68 [69]). В дійсності тут припускається, що капітал
зростає з \num{1000000} до \num{1100000}, тобто на 10\%, тимчасом як норма зиску
падає з 7 до 6, тобто на 14\sfrac{2}{7}\%. Hinc illae lacrimae [звідси ці сльози].} Але
якщо ті самі причини, які викликають падіння норми зиску,
\index{iii1}{0224}  %% посилання на сторінку оригінального видання
сприяють нагромадженню, тобто утворенню додаткового капіталу,
і якщо кожен додатковий капітал приводить в рух добавну
працю і виробляє добавну додаткову вартість; якщо, з другого
боку, просте зниження норми зиску включає вже і той
факт, що сталий капітал, а тому й весь старий капітал, зріс, —
то весь цей процес перестає бути таємничим. Ми далі побачимо,
до яких умисних фальшувань в обчисленнях вдаються
для того, щоб по-шахрайському відкинути можливість збільшення
маси зиску при одночасному зменшенні норми зиску.

Ми показали, як ті самі причини, які викликають тенденцію
загальної норми зиску до падіння, зумовлюють прискорене нагромадження
капіталу, а тому й зростання абсолютної величини або
загальної маси привласнюваної ним додаткової праці (додаткової
вартості, зиску). Як усе в конкуренції, а тому й у свідомості
агентів конкуренції, так і цей закон — я маю на думці цей внутрішній
і необхідний зв’язок між двома явищами, які, як здається,
одне одному суперечать — виступає у перекрученому вигляді.
Очевидно, що в межах вищенаведених пропорцій капіталіст,
який розпоряджається великим капіталом, одержує більшу масу
\parbreak{}  %% абзац продовжується на наступній сторінці

\parcont{}  %% абзац починається на попередній сторінці
\index{iii1}{0225}  %% посилання на сторінку оригінального видання
зиску, ніж дрібний капіталіст, який, видимо, одержує високий зиск. Далі, найповерховіше
спостереження конкуренції показує, що при певних обставинах, коли більший капіталіст хоче захопити
для себе місце на ринку, витиснути дрібніших капіталістів, — як, наприклад, за часів кризи, — він
використовує це практично, тобто навмисно знижує свою норму зиску, щоб витиснути з ринку дрібніших
капіталістів. Так само й купецький капітал — про який ми пізніше скажемо докладніше — показує явища,
завдяки яким зниження зиску здається наслідком розширення підприємства, а разом з тим і капіталу.
Власне науковий вираз замість помилкового розуміння ми дамо пізніше. Подібні поверхові погляди є
результатом порівнення норм зиску, одержуваних в окремих галузях підприємств залежно від того, чи
підпорядковані вони режимові вільної конкуренції чи монополії. Цілком банальне уявлення, яке
створюється в головах агентів конкуренції, ми знаходимо в нашого Рошера, а саме, що таке зниження
норми зиску є „розумніше й гуманніше“\footnote*{
„Die Grundlagen der Nationalökonomie“. 2 Aufl. Stuttgart und Augsburg 1857,
стор. 190. \Red{Примітка ред. нім. вид. ІМЕЛ.}
}. Зменшення норми зиску представлено тут як \emph{наслідок}
збільшення капіталу і зв’язаного з цим розрахунку капіталістів, що при меншій нормі зиску маса
зиску, яку вони кладуть собі в кишеню, буде більша. Все це (за винятком того, що є в А.~Сміта, про
що пізніше) основане на цілковитому нерозумінні того, що таке взагалі є загальна норма зиску, і на
тому грубому уявленні, що ціни дійсно визначаються шляхом надбавки більш-менш довільної частки зиску
до дійсної вартості товарів. Хоч які грубі ці уявлення, все ж вони з необхідністю виникають з того
перекрученого способу й вигляду, в якому імманентні закони капіталістичного виробництва виявляються
в сфері конкуренції.

\pfbreak{}

Закон, згідно з яким падіння норми зиску, викликуване розвитком продуктивної сили, супроводиться
збільшенням маси зиску, виражається і в тому, що падіння цін товарів, вироблюваних капіталом,
супроводиться відносним збільшенням мас зиску, які містяться в них і реалізуються через їх продаж.

Через те що розвиток продуктивної сили і відповідний цьому вищий склад капіталу приводить в рух
дедалі більшу кількість засобів виробництва за допомогою дедалі меншої кількості праці, то кожна
пропорціональна частина всього продукту, кожна одиниця товару або кожна певна окрема кількість
товару, яка служить одиницею міри для сукупної маси вироблених товарів, вбирає менше живої праці і
містить у собі, крім того, менше упредметненої праці як щодо зношення застосованого основного
капіталу,
так і щодо спожитих сировинних і допоміжних матеріалів. Отже, кожна одиниця товару містить у собі
меншу суму праці як упредметненої
\index{iii1}{0226}  %% посилання на сторінку оригінального видання
в засобах виробництва, так і новододаної під час виробництва. Тому ціна одиниці товару
падає. Маса зиску, яка міститься в кожній одиниці товару, може, не зважаючи на це, збільшитись, якщо
норма абсолютної чи відносної додаткової вартості зростає. Кожний окремий товар містить у собі менше
новододаної праці, але неоплачена частина її зростає в порівнянні з оплаченою. Однак, це
відбувається тільки в певних межах. Разом з дуже значним абсолютним зменшенням новододаної до кожної
одиниці товару суми живої праці, яке відбувається в ході розвитку виробництва, зменшуватиметься
абсолютно і маса неоплаченої праці, яка міститься в ній, як би вона не зростала відносно, а саме в
порівнянні з оплаченою частиною. Маса зиску, яка припадає на кожну одиницю товару, дуже
зменшуватиметься з розвитком продуктивної сили праці, не зважаючи на зростання норми додаткової
вартості; і це зменшення цілком так само, як падіння норми зиску, тільки уповільнюється здешевленням
елементів сталого капіталу та іншими наведеними в першому відділі цієї книги обставинами, які
підвищують норму зиску при незмінній і навіть при падаючій нормі додаткової вартості.

Те, що ціна окремих товарів, з суми яких складається сукупний продукт капіталу, падає, не означає
нічого іншого, як те, що дана кількість праці реалізується в більшій масі товарів, що, отже, кожна
одиниця товару містить у собі менше праці, ніж раніше. Це відбувається навіть у тому випадку, коли
ціна якоїсь частини сталого капіталу, сировинного матеріалу та ін. зростає. За винятком окремих
випадків (наприклад, коли продуктивна сила праці рівномірно здешевлює всі елементи як сталого, так і
змінного капіталу), норма зиску знижуватиметься, не зважаючи на підвищену норму додаткової вартості,
1) тому що навіть більша неоплачена частина зменшеної загальної суми новододаної праці є менша, ніж
була менша відповідна неоплачена частина більшої загальної суми, і 2) тому що вищий склад капіталу в
окремому товарі виражається в тому, що та частина його вартості, яка взагалі представляє новододану
працю, зменшується порівняно з тією частиною вартості, яка представляє сировинний матеріал,
допоміжний матеріал і зношування основного капіталу. Ця переміна у відношенні різних складових
частин ціни окремого товару, зменшення тієї частини ціни, яка представляє новододану живу працю, і
збільшення тих частин ціни, які представляють раніше упредметнену працю, є та форма, в якій у ціні
окремого товару виражається зменшення змінного капіталу порівняно з сталим. Наскільки таке зменшення
є абсолютним для капіталу даної величини, наприклад, для 100, настільки ж воно є абсолютним для
кожного окремого товару як відповідної частини репродукованого капіталу. Однак, норма зиску, якщо
тільки обчисляти її на елементи ціни окремих товарів, виступила б іншою, ніж вона є в дійсності. І
саме з такої причини:


\index{iii1}{0227}  %% посилання на сторінку оригінального видання
[Норма зиску обчислюється на весь застосований капітал, але за певний час, фактично за один рік.
Відношення виробленої за рік і реалізованої додаткової вартості або зиску до всього капіталу,
обчислене в процентах, є норма зиску. Отже, вона не неодмінно дорівнює тій нормі зиску, при якій в
основу обчислення кладеться не рік, а період обороту капіталу, про який іде мова; тільки в тому
випадку, коли цей капітал обертається саме один раз за рік, обидві ці норми збігаються.

З другого боку, зиск, одержаний на протязі року, є тільки сума зисків на товари, вироблені і продані
на протязі того самого року. Якщо ж ми обчислюватимем зиск на витрати виробництва товарів, то
одержимо норму зиску $= \frac{p}{k}$, де $р$ становить реалізований на протязі року зиск, а $k$ — суму витрат
виробництва товарів, вироблених і проданих протягом того самого часу. Очевидно, що ця норма зиску
$\frac{p}{k}$ тільки в тому випадку може збігатися з дійсною нормою зиску $\frac{p}{K}$, — маса зиску, поділена на весь
капітал, — коли $k = К$, тобто коли капітал обертається, саме один раз за рік.

Візьмімо три різні стани якогонебудь промислового капіталу.

І. Капітал в 8000\pound{ фунтів стерлінгів} виробляє і продає щороку 5000 штук товару по 30\shil{ шилінгів} за
штуку, отже, має річний оборот в 7500\pound{ фунтів стерлінгів}. На кожну штуку товару
він дає зиск в 10\shil{ шилінгів} = 2500\pound{ фунтам стерлінгів} на рік. Отже, в кожній штуці містяться 20\shil{ шилінгів} авансованого капіталу і 10\shil{ шилінгів} зиску, отже норма зиску на кожну штуку становить
$\frac{10}{20}= 50\%$. На суму в 7500\pound{ фунтів стерлінгів}, що обернулась, припадає 5000\pound{ фунтів стерлінгів}
авансованого капіталу і 2500\pound{ фунтів стерлінгів} зиску; норма зиску на кожний оборот, $\frac{p}{k}$, так само =
50\%. Навпаки, норма зиску, обчислена на весь капітал, $\frac{p}{K} = \frac{2500}{8000} = 31\sfrac{1}{4}\%$.

II. Припустім, що капітал збільшується до \num{10000}\pound{ фунтів стерлінгів}. Припустім, що в наслідок
збільшеної продуктивної сили праці він може виробляти щороку \num{10000} штук товару при витратах
виробництва в 20\shil{ шилінгів} на штуку. Він продає їх із зиском в 4\shil{ шилінги} на штуку, отже, по 24\shil{ шилінги} за штуку. Тоді ціна річного продукту = \num{12000}\pound{ фунтам стерлінгів}, з яких \num{10000}\pound{ фунтів
стерлінгів} авансованого капіталу і 2000\pound{ фунтів стерлінгів} зиску $\frac{p}{k}$ на кожну штуку $= \frac{4}{20}$, для річного
обороту $= \frac{2000}{\num{10000}}$, отже, в обох випадках = 20\%, а через те що весь
\parbreak{}  %% абзац продовжується на наступній сторінці

\parcont{}  %% абзац починається на попередній сторінці
\index{ii}{0228}  %% посилання на сторінку оригінального видання
вартости виражає не що інше, як відношення застосованого протягом
певного часу змінного капіталу до спродукованої протягом того самого
часу додаткової вартости; або — масу тієї неоплаченої праці, що її пускає
в рух змінний капітал, застосований протягом цього часу. Вона абсолютно
не має чинення до тієї частини змінного капіталу, яку авансовано,
але протягом певного часу не застосовується, отже, так само не
має ніякою чинення вона й до відношення між частиною капіталу, авансованого
в певний протяг часу, і тією частиною його, що її застосовано
протягом цього самого часу — відношення, що для різних капіталів
під впливом періодів обороту модифікується й є різне.

З наведеного вище скорше випливає, що річна норма додаткової вартости
лише в одному єдиному випадку збігається з справжньою нормою
додаткової вартости, яка виражає ступінь експлуатації праці: а саме в
тому разі, коли авансований капітал обертається тільки один раз на рік,
коли тому авансований капітал дорівнює капіталові, що обернувся протягом
року, а відношення маси додаткової вартости, спродукованої протягом
року, до капіталу, застосованого на її продукцію протягом року, збігається
і є тотожне з відношенням маси додаткової вартости, спродукованої
протягом року, до капіталу, авансованого протягом року.

A) Річна норма додаткової вартости дорівнює:

маса додаткової вартости, спродукованої протягом року: авансований змінний капітал

Але маса додаткової вартости, спродукованої протягом року, дорівнює
справжній нормі додаткової вартости, помноженій на змінний капітал,
застосований на її продукцію. Капітал, застосований на продукцію
річної маси додаткової вартости, дорівнює авансованому капіталові, помноженому
на число оборотів його, яке ми позначатимемо n. Тому формула А) перетворюється на таку:

B) Річна норма додаткової вартости дорівнює:

справжня норма додаткової вартости X аванс. змінний капітал X n: авансований змінний капітал

Наприклад, для капіталу В = 100\%Х500Х1: 5000 або 100\%. Тільки коли
n = 1, тобто, коли авансований змінний капітал обертається тільки
один раз на рік, отже, дорівнює застосованому протягом року капіталові,
або капіталові, що обернувся протягом року, — тільки тоді річна норма додаткової
вартости, дорівнює справжній нормі додаткової вартости.

Коли ми позначимо річну норму додаткової вартости М', справжню норму
додаткової вартости m', авансований змінний капітал — v, число оборотів
— n, то М' = m'vn: $v = m$'n; отже, М' = m'n, і лише тоді = m', коли
n = 1; отже, М' = m'X$1 = m$'.
\parbreak{}  %% абзац продовжується на наступній сторінці

\parcont{}  %% абзац починається на попередній сторінці
\index{i}{0229}  %% посилання на сторінку оригінального видання
сцени протягом усього часу тривання драми, так і робітники належали
тепер до фабрики протягом 15 годин, не рахуючи часу на
дорогу до фабрики й назад. Таким чином години відпочинку перетворювалися
на години примусового безділля, що гнали молодого
робітника до шинку, а молоду робітницю в дім розпусти. За
кожної нової витівки, що її день-у-день вигадував капіталіст,
щоб тримати свої машини в русі 12 або 15 годин, не збільшуючи
робочого персоналу, робітник мусів проковтнути свою їжу то в
той, то в інший шматок часу. Під час агітації за десятигодинний
робочий день фабриканти кричали, що робітнича наволоч подає
петиції, сподіваючись дістати за десятигодинну працю дванадцятигодинну
заробітну плату. Тепер вони обернули медалю. Вони
виплачували десятигодинну заробітну плату за дванадцяти й
п’ятнадцятигодинне порядкування робочими силами!\footnote{
Див. «Reports etc. for 30 th April 1849», p. 6 і докладне пояснення
«shifting system»\footnote*{
— системи пересувань. \emph{Ред.}
}, яке фабричні інспектори Хоуелл і Савндер дають
у «Reports etc. for 31 st October 1848». Див. також петицію проти
«shift system», подану королеві духівництвом Ashton’a й околиць на весні
1849~\abbr{р.}
} Так ось
у чім була річ; це було фабрикантське видання десятигодинного
закону! Це були ті самі фритредери, сповнені благодаті й любови
до людства, що підчас аґітації проти хлібних законів цілих десять
років до останнього шага обчислювали робітникам, що за вільного
довозу хліба, при тих засобах, що їх має англійська промисловість,
цілком досить було б десяти годин праці, щоб збагатити
капіталістів\footnote{
Порівн., наприклад, «The Factory Question and the Ten Hours
Bill. By R. H. Greg. 1837».
}.

Дворічний бунт капіталу увінчався нарешті присудом однієї
з чотирьох вищих судових установ Англії, Court of Exchequer,
який в одному з випадків, що дійшов до нього, 8 лютого 1850~\abbr{р.}
вирішив, що хоч фабриканти й чинили проти змісту закону
1844~\abbr{р.}, але самий цей закон містить у собі деякі слова, що роблять
його безглуздим. «Цей вирок знищив закон про десятигодинну
працю»\footnote{
\emph{F. Engels}: «Die englische Zehnstundenbill» (у видаваній мною
«Neue Rheinische Zeitung». Politish-ökonomische Revue, Aprilheft
1850», p. 13). Той самий «високий» суд так само винайшов підчас американської
громадянської війни словесну зачіпку, яка перетворювала закон
проти озброєння піратських кораблів у його пряму протилежність.
}. Маса фабрикантів, що досі боялись застосовувати
систему змін для підлітків і робітниць, ухопилися за неї тепер
обома руками\footnote{
«Reports etc. for 30 th April 1850».
}.

Але за цією, здавалось, остаточною перемогою капіталу
настав зараз же поворот. Робітники досі ставили пасивний, хоч
і впертий і день-у-день відновлюваний опір. Тепер вони почали
голосно протестувати на загрозливих мітинґах у Ланкашірі і
Йоркшірі. Значить, так званий десятигодинний закон — це лише
ошуканство, парляментське шахрайство, а на ділі він ніколи не
існував! Фабричні інспектори пильно попереджали уряд, що
\parbreak{}  %% абзац продовжується на наступній сторінці

\input{_0230.tex}
\input{_0231.tex}
\parcont{}  %% абзац починається на попередній сторінці
\index{iii1}{0232}  %% посилання на сторінку оригінального видання
додаткової праці здовження робочого дня, — цей винахід сучасної промисловості, — не змінюючи при
цьому істотно відношення вживаної робочої сили до сталого капіталу, який вона приводить в рух, і в
дійсності скорше відносно зменшуючи сталий капітал. Взагалі ж ми вже показали, — і це становить
власне таємницю тенденції норми зиску до падіння, —-що методи виробництва відносної додаткової
вартості загалом і в цілому зводяться ось до чого: з одного боку, з даної маси праці якомога більше
перетворити в додаткову вартість, з другого боку, взагалі вживати якомога менше праці порівняно з
авансованим капіталом; так що ті самі причини, які дозволяють підвищувати ступінь експлуатації
праці, не дозволяють з тим самим сукупним капіталом експлуатувати стільки ж праці, як і раніш. Такі
є протилежні тенденції, які, викликаючи підвищення норми додаткової вартості, одночасно викликають
падіння маси додаткової вартості, вироблюваної даним капіталом, а тому й падіння норми зиску. Тут
слід також згадати про масове вживання жіночої і дитячої праці, тому що при цьому вся сім’я мусить
давати капіталові більшу масу додаткової праці, ніж раніше, навіть якщо зростає загальна сума
заробітної плати, виплачуваної цій сім’ї, — випадок, який аж ніяк не є загальним явищем. — Такий
самий вплив справляє все те, що сприяє при незмінній величині застосовуваного капіталу виробництву
відносної додаткової вартості шляхом самого тільки поліпшення методів, як у землеробстві. Хоча тут
застосовуваний сталий капітал не зростає в порівнянні з змінним, оскільки ми розглядаємо цей
останній як показник уживаної робочої сили, але маса продукту зростає порівняно з ужитою робочою
силою. Те саме має місце, коли продуктивна сила праці (однаково, чи входить її продукт у споживання
робітників, чи в елементи сталого капіталу) звільняється від перешкод, що утруднюють зносини, від
самовільних обмежень або від обмежень, які стали обтяжливими з бігом часу, взагалі від усякого роду
пут, при чому цим не зачіпається відношення змінного капіталу до сталого.

Можна було б поставити питання: чи входять у число тих причин, які гальмують падіння норми зиску,
але в кінцевому рахунку завжди прискорюють його, тимчасові, але що завжди повторюються, виявляються
то в одній, то в другій галузі виробництва підвищення додаткової вартості понад загальний рівень для
капіталіста, який використовує винаходи і~\abbr{т. д.}, поки вони ще не стали загальнопоширеними. На це
питання треба відповісти позитивно.

Маса додаткової вартості, вироблена капіталом даної величини, є добуток двох множників — норми
додаткової вартості, помноженої на число робітників, занятих при даній нормі. Отже, при даній нормі
додаткової вартості маса її залежить від числа
робітників, а при даному числі робітників — від норми додаткової вартості, тобто взагалі від
складного відношення абсолютної
\index{iii1}{0233}  %% посилання на сторінку оригінального видання
величини змінного капіталу і норми додаткової вартості. Але ми показали, що пересічно ті самі
причини, які підвищують норму відносної додаткової вартості, зменшують масу вживаної робочої сили.
Проте, ясно, що збільшення або зменшення тут відбувається залежно від певного відношення, в якому
відбувається цей протилежний рух, і що тенденція до зменшення норми зиску послаблюється зокрема в
наслідок підвищення норми абсолютної додаткової вартості, яка походить із здовження робочого дня.

При дослідженні норми зиску ми взагалі виявили, що зниженню норми, яке відбувається в наслідок
зростання маси всього застосовуваного капіталу, відповідає збільшення маси зиску. Якщо розглядати
сукупний змінний капітал суспільства, то
створена ним додаткова вартість дорівнює створеному зискові. Разом з абсолютною масою додаткової
вартості виросла і норма додаткової вартості; перша виросла тому, що збільшилась вживана
суспільством маса робочої сили, друга — тому, що підвищився ступінь експлуатації цієї праці. Але
відносно капіталу даної величини, наприклад, 100, норма додаткової вартості може зрости, тоді як
маса її пересічно падає; бо норма визначається відношенням, в якому змінна частина капіталу зростає
в своїй вартості, а маса визначається, навпаки, тією відносною частиною, яку становить змінний
капітал в усьому капіталі.

Підвищення норми додаткової вартості — через те що воно відбувається і при таких обставинах, коли,
як це показано вище, не відбувається ніякого збільшення або не відбувається пропорціонального
збільшення сталого капіталу порівняно з змінним — є один з факторів, яким визначається маса
додаткової вартості, а тому й норма зиску. Цей фактор не знищує загального закону. Але він робить
те, що цей закон діє більше як тенденція, тобто як закон, абсолютне здійснення якого затримується,
уповільнюється і ослаблюється протидіючими обставинами. Але через те що ті самі причини, які
підвищують норму додаткової вартості (навіть здовження робочого дня є результат великої
промисловості), мають тенденцію зменшувати вживану даним капіталом кількість робочої сили, то одні й
ті самі причини мають тенденцію зменшувати норму зиску і уповільнювати рух цього зменшення. Якщо
одному робітникові накидають таку працю, яку раціонально виконати можуть тільки двоє, і якщо це
відбувається при таких обставинах, коли цей робітник може заступити трьох, то один робітник даватиме
стільки додаткової праці, скільки раніш давало двоє, і остільки норма додаткової вартості
підвищиться. Але він не даватиме стільки, скільки раніш давало троє, і остільки маса додаткової
вартості знизиться. Але це зниження компенсується або обмежується підвищенням норми додаткової
вартості. Якщо все населення працюватиме при підвищеній нормі додаткової вартості, то маса
додаткової вартості збільшиться, хоч населення лишиться тим
\parbreak{}  %% абзац продовжується на наступній сторінці

\parcont{}  %% абзац починається на попередній сторінці
\index{iii2}{0234}  %% посилання на сторінку оригінального видання
певного матеріяльного продукту, пшениці. Але вона не має ніякого чинення до продукції
\emph{вартости пшениці}. Оскільки вартість втілюється у пшеницю, пшениця
розглядається лише як певна кількість зрічевленої суспільної праці, цілком
байдуже до тієї особливої речовини, в якій втілюється ця праця або до особливої
споживної вартости цієї речовини. Це не суперечить тому, що 1) в інших рівних
умовах дешевина або дорожнеча пшениці залежать від продуктивности землі. Продуктивність
хліборобської праці зв’язана з природними умовами, і залежно від продуктивности
останніх та сама кількість праці втілюється в більший або меншій кількості
продукту, споживних вартостей. Наскільки велика кількість праці, втілена в
одному шефелі, це залежить від того, яку кількість шефелів дає дана кількість праці.
Від продуктивности землі тут залежить, в якій кількості продукту втілюється вартість;
але цю вартість дано незалежно від такого поділу. Вартість втілюється
у споживній вартості; а споживна вартість є умова створення вартости; але
було б безглуздям створювати таке протиставлення, де на однім боці ставиться
споживну вартість, землю, а на другому боці — вартість, і до того ще особливу
частину вартости. 2) [Тут рукопис уривається].

\subsubsection{}

Вульґарна економія в дійсності не робить нічого іншого, як тільки подоктринерському
тлумачить, систематизує і виправдує уявлення аґентів буржуазної
продукції, захоплених відносинами цієї продукції. Отже, нас не повинно
дивувати те, що якраз у формі виявлення економічних відносин, яка відчужена
від них, і в якій вони prima facie набувають вульґарного і цілком суперечливого
характеру, — а всяка наука була б зайва, коли б форма виявлення й суть
речей безпосередньо збігалися, — що якраз тут вульґарна економія почуває себе
цілком вдома, і що ці відносини здаються їй то самоочевиднішими, що більше
захований в них внутрішній зв’язок і що звичайнішими здаються вони для
буденної уяви. Тому в неї немає ніякого передчуття того, що триєдиність,
з якої вона виходить: земля — рента, капітал — процент, праця — заробітна плата
або ціна праці, є три prima facie неможливі сполучення. Насамперед перед
нами споживна вартість, \emph{земля}, яка не має вартости, і мінова вартість, \emph{рента},
таким чином соціальне відношення, взяте як річ, є поставлене в певне
відношення до природи; отже, дві неспівмірні величини ставляться в певне
відношення одна до однієї. Потім \emph{капітал — процент}. Коли під капіталом
розуміти певну суму вартости, самостійно визначену в грошах, то prima facie
є безглуздя, ніби вартість має більшу вартість, ніж вона варта. Саме у формі:
капітал — процент відпадає всяке посередництво, і капітал зводиться до своєї
найзагальнішої, але тому й нез’ясуванної з себе самої та абсурдної формули.
Саме тому вульґарний економіст і дає перевагу формулі: капітал — процент,
з таємничою властивістю вартости бути нерівною самій собі, над формулою:
капітал — зиск, бо ця вже ближче підходить до дійсного капіталістичного відношення.
А потім, турботно відчуваючи, що 4 не є 5, і тому 100 талярів не
можуть бути 110 талярами, вульґарний економіст від капіталу як вартости
вдається до речевої субстанції капіталу, до його споживної вартости як умови
продукції для праці, до машин, сирового матеріялу тощо. Таким чином, знову
щастить, замість незрозумілого першого відношення, за яким 4 \deq{} 5, створити
цілком неспівмірне відношення між споживною вартістю, річчю, на одному боці,
і певним суспільним продукційним відношенням, додатковою вартістю на другому
боці, як у випадку з земельною власністю. Скоро вульґарний економіст
доходить і до цієї неспівмірности, йому все стане ясним, він уже не почуває
потреби міркувати далі. Бо він дійшов саме до «раціонального» для буржуазного
уявлення. Нарешті, \emph{праця — заробітна плата}, ціна праці, як показано
\parbreak{}  %% абзац продовжується на наступній сторінці

\input{_0235.tex}

Інше питання — яке в наслідок своєї специфічності лежить власне поза межами нашого дослідження —
таке: чи підвищується загальна норма зиску в наслідок вищої норми зиску, яку дає капітал, вкладений
у зовнішню, особливо в колоніальну торгівлю?

\looseness=1
\disablefootnotebreak{} 
Капітали, вкладені в зовнішню торгівлю, можуть давати вищу норму зиску тому, що тут, поперше,
відбувається конкуренція з товарами, вироблюваними іншими країнами при менш легких умовах
виробництва, так що більш розвинена країна продає свої
товари вище їх вартості, хоч і дешевше, ніж конкуруючі країни. Оскільки праця більш розвиненої
країни оцінюється тут як праця вищої питомої ваги, норма зиску підвищується в наслідок того, що
праця, не оплачувана як праця вищої якості, продається як така. Те саме може мати місце відносно
тієї країни, до якої відправляються товари і з якої одержуються товари; а саме, така країна віддає
більше упредметненої праці in natura, ніж вона одержує, і все ж при цьому одержує товари дешевше,
ніж вона сама могла б їх виробити. Цілком так само, як фабрикант,
який використовує новий винахід раніше, ніж він стає загальнопоширеним, продає дешевше своїх
конкурентів, і все ж продає свої товари вище їх індивідуальної вартості, тобто специфічно вищу
продуктивну силу вживаної ним праці використовує як додаткову працю. Він реалізує таким чином
надзиск. З другого боку, щодо капіталів, вкладених у колоніях і~\abbr{т. д.}, то вони можуть давати вищі
норми зиску тому, що там в наслідок нижчого розвитку, норма зиску взагалі стоїть вище, а при умові
вживання рабів, кулі і~\abbr{т. п.}, стоїть вище і експлуатація праці. Не можна зрозуміти, чому ці вищі
норми зиску, що їх таким чином дають і відправляють до батьківщини капітали, вкладені в певні
галузі, тут, якщо тільки цьому не перешкоджає монополія, не повинні були б увіходити в процес
вирівнення загальної норми зиску і тому pro tanto [відповідно до цього] підвищувати її\footnote{
А.~Сміт тут має рацію проти Рікардо, який каже: They contend the equality of profits will be
brought about by the general rise of profits; and I am of opinion that the profits of the favoured
trade will speedily submit to the general level“ [„Вони твердять, що рівність зисків буде здійснена
через загальне підвищення зисків; а я тієї думки, що зиски підприємства, яке є в сприятливіших
умовах, швидко знизяться до загального рівня“]. ([\emph{Ricardo}:] „Works“, видання Мак-Куллоха, стор. 73).
}. 
\enablefootnotebreak{}

Особливо
цього не можна зрозуміти, якщо зазначені
\index{iii1}{0237}  %% посилання на сторінку оригінального видання
галузі капіталовкладень підлягають законам вільної конкуренції. Навпаки, Рікардо вбачається
таке: на гроші, одержані за кордоном від продажу по вищій ціні, там купуються товари і
відправляються на заміну додому; отже, ці товари продаються
всередині країни, і тому це може становити, принаймні тимчасово, особливу, порівняно з іншими
сферами, невигоду для сфер виробництва, які є в сприятливих умовах. Ця ілюзія відпадає, як тільки ми
абстрагуємось від грошової форми. Країна,
яка перебуває в сприятливіших умовах, одержує назад більше праці в обмін за меншу кількість праці,
хоч ця ріжниця, цей надлишок, як взагалі при обміні між працею і капіталом, привласнюється певним
класом. Отже, оскільки норма зиску є вища, тому
що вона взагалі вища в колоніальній країні, це при сприятливих природних умовах цієї країни може йти
рука в руку з низькими товарними цінами. Вирівнення відбувається, але вирівнення не за старим
рівнем, як гадає Рікардо.

Але та сама зовнішня торгівля розвиває всередині країни капіталістичний спосіб виробництва і тим
самим веде до зменшення змінного капіталу порівняно з сталим; з другого боку, вона створює
перепродукцію відносно закордону і тому в дальшому перебігу знов таки справляє протилежний вплив.

І таким чином взагалі виявляється, що ті самі причини, які приводять до падіння загальної норми
зиску, викликають протилежні впливи, які гальмують, уповільнюють і почасти паралізують це падіння.
Вони не знищують закону, але ослаблюють його діяння. Без цього було б незрозумілим не падіння
загальної норми зиску, а, навпаки, відносна повільність цього падіння. Таким чином закон діє тільки
як тенденція, вплив якої виразно виступає тільки при певних обставинах і на протязі довгих періодів
часу.

Раніше, ніж піти далі, ми, щоб уникнути непорозумінь, повторимо ще два, вже не раз розвинуті
положення.

\emph{Поперше}: Той самий процес, який в ході розвитку капіталістичного способу виробництва породжує
здешевлення товарів, породжує також зміну в органічному складі суспільного капіталу, застосовуваного
для виробництва товарів, а в наслідок цього і падіння норми зиску. Отже, зменшення відносних витрат
на окремий товар, а також тієї частини цих витрат, яка містить у собі зношування машин, не слід
ототожнювати з зростанням вартості сталого капіталу порівняно з змінним, хоча, навпаки, всяке
зменшення відносних витрат на сталий капітал, при незмінному або зростаючому розмірі його речових
елементів, впливає на підвищення норми зиску, тобто на зменшення pro tanto [відповідно до цього]
вартості сталого капіталу порівняно з змінним капіталом, застосовуваним в дедалі менших пропорціях.

\emph{Подруге}: Та обставина, що в окремих товарах, сукупність яких становить продукт капіталу, відношення
додаваної живої праці, яка міститься в них, до уміщених в них матеріалів праці
\parbreak{}  %% абзац продовжується на наступній сторінці

\parcont{}  %% абзац починається на попередній сторінці
\index{ii}{0238}  %% посилання на сторінку оригінального видання
можуть вийти лише з певних галузей, як, напр., сільське господарство тощо,
де працюють виключно дужі парубки. Це діється й після того, як нові підприємства
стали вже постійною галуззю продукції і, значить, після того,
як уже утворилась потрібна для них бродяча робітнича кляса. Напр.,
коли залізниця раптом почне будуватись у ширшому від пересічного
маштабі. Тоді вбирається частину резервної армії робітників, що її тиск
тримав заробітну плату на порівняно низькому рівні. Тоді заробітна плата
скрізь підвищується, навіть у тих частинах робочого ринку, де робітники
й раніш легко знаходили собі працю. Це триває доти, доки неминучий
крах знову звільняє резервну армію робітників, і заробітну плату
знову знижується до її мінімуму й нижче\footnote{
В рукопису тут вставлено таку замітку, щоб пізніш її розвинути: „Суперечність
в капіталістичному способі продукції: робітники як покупці товару,
важать для ринку. Але як продавців свого товару — робочої сили капіталістичне
суспільство має тенденцію обмежувати їх мінімумом ціни. Дальша суперечність:
ті епохи, коли капіталістична продукція напружує всі свої сили, регулярно з’являються
як епохи перепродукції, бо продуктивні сили ніколи не можна застосувати
так, щоб у наслідок цього можна було не лише випродукувати, а й зреалізувати
більше вартости; але продаж товарів, реалізація товарового капіталу, отже,
і додаткової вартости, обмежена не просто споживними потребами суспільства
взагалі, з споживними потребами такого суспільства, що його переважна
більшість завжди бідна й мусить завжди лишатися бідною. Однак це стосується
лише до наступного відділу.“ \emph{Ф.~Е.}
}.

Оскільки більший або менший протяг періоду обороту залежить від
робочого періоду у власному значенні, тобто від періоду, потрібного на
те, щоб виготувати продукт для ринку, він ґрунтується на кожного
разу даних речових умовах продукції різних капіталовкладень, на
умовах, що в хліборобстві мають більше характер природних умов продукції,
а в мануфактурі і в більшій частині видобувної промисловости
змінюються разом із суспільним розвитком самого продукційного процесу.

Оскільки протяг робочого періоду ґрунтується на величині поставок
(на кількісному розмірі, що в ньому продукт звичайно подається на ринок
як товар), він має умовний характер. Але сама ця умовність має за
матеріяльну базу розміри продукції, а тому вона є випадкова лише остільки,
оскільки ми розглядаємо її ізольовано.

Нарешті, оскільки протяг періоду обороту залежить від протягу періоду
циркуляції, він почасти зумовлюється постійною зміною ринкових
коньюнктур, більшою або меншою легкістю продажу і неминучою, відси
посталою, потребою подавати частину продукту на ближчий або дальший
ринок. Лишаючи осторонь розмір попиту взагалі, рух цін відіграє
тут головну ролю, оскільки при зниженні цін продаж навмисно обмежується,
тимчасом як продукція розвивається далі; навпаки буває при
підвищенні цін, коли продукція та продаж не відстають одне від одного,
або коли продаж може відбуватися заздалегідь. Однак, за власне матеріяльну
базу треба вважати справжнє віддалення місця продукції від
ринку збуту.



\sectionextended{Розвиток внутрішніх суперечностей закону}{%
\subsection{Загальні зауваження}}

В першому відділі цієї книги ми бачили, що норма зиску завжди виражає норму додаткової вартості
нижчою, ніж вона є. Тепер ми побачили, що навіть зростаюча норма додаткової вартості має тенденцію
виражатись у падаючій нормі зиску. Норма
зиску дорівнювала б нормі додаткової вартості тільки в тому випадку, коли $c$ було б $= 0$, тобто коли б
увесь капітал витрачався на заробітну плату. Падаюча норма зиску тільки тоді виражає падаючу норму
додаткової вартості, коли відношення між вартістю сталого капіталу і масою робочої сили, яка
приводить його в рух, лишається незмінним, або коли ця остання збільшується у відношенні до вартості
сталого капіталу.

Рікардо, досліджуючи, як він гадав, норму зиску, в дійсності досліджував тільки норму додаткової
вартості і цю останню тільки при тому припущенні, що робочий день щодо інтенсивності й довжини є
стала величина.

Падіння норми зиску і прискорене нагромадження лиш остільки є різні вирази одного й того ж процесу,
оскільки і те і друге є виразом розвитку продуктивної сили. Нагромадження, з свого боку, прискорює
падіння норми зиску, оскільки разом з ним
дана концентрація робіт у великому масштабі, а тому й вищий склад капіталу. З другого боку, падіння
норми зиску знову таки прискорює концентрацію капіталу і його централізацію шляхом експропріації
дрібних капіталістів, шляхом експропріації
останніх решток безпосередніх виробників, у яких лишається ще щонебудь експропріювати. В наслідок
цього, з другого боку, прискорюється — щодо маси — нагромадження, хоча з падінням норми зиску падає
і норма нагромадження.

З другого боку, оскільки норма зростання вартості всього капіталу, норма зиску, є стимулом
капіталістичного виробництва (подібно до того, як збільшення вартості капіталу є його єдиною метою),
падіння цієї норми уповільнює утворення нових самостійних капіталів і виступає таким чином як
загроза для розвитку капіталістичного процесу виробництва; воно сприяє перепродукції, спекуляції,
кризам, утворенню надмірного капіталу поряд з надмірним населенням. Отже, ті економісти, які,
подібно до Рікардо, вважають капіталістичний спосіб виробництва за абсолютний, відчувають тут, що
цей спосіб виробництва сам собі створює межу, і тому приписують цю межу не виробництву, а природі (в
ученні про ренту). Але важливим в їх жаху перед падаючою нормою зиску є відчуття того, що
капіталістичний спосіб виробництва в розвитку продуктивних сил має таку межу, яка не стоїть ні в
якому зв’язку з виробництвом багатства як таким;
\index{iii1}{0240}  %% посилання на сторінку оригінального видання
і ця особлива межа свідчить про обмеженість і тільки
історичний, минущий характер капіталістичного способу виробництва;
свідчить про те, що він не є абсолютний спосіб виробництва
для виробництва багатства і що, навпаки, на певному
ступені він вступає в конфлікт із своїм дальшим розвитком.

Рікардо і його школа розглядають, в усякому разі, тільки промисловий
зиск, в якому міститься і процент. Але й норма земельної
ренти має тенденцію до падіння, хоч її абсолютна маса зростає
і хоч вона може зростати й відносно, порівняно з промисловим
зиском (див. \emph{Ед.~Уест} [„Essay on Application of Capital to Land“,
Лондон 1815], який виклав закон земельної ренти \emph{раніше} від
Рікардо). Якщо ми розглядатимем сукупний суспільний капітал
$К$ і, позначимо через $р_1$ той промисловий зиск, який залишається
після відрахування процента й земельної ренти, через $z$
процент і через $r$ земельну ренту, то
$\frac{m}{K} \deq{} \frac{p}{K} \deq{} \frac{p_1 \dplus{} z \dplus{} r}{K} \deq{} \frac{p_1}{K} \dplus{} \frac{z}{K} \dplus{} \frac{r}{K}$.
Ми бачили, що хоч у ході розвитку капіталістичного
виробництва сукупна сума додаткової вартості, $m$, постійно
зростає, проте $\frac{m}{K}$ так само постійно зменшується, бо $К$
зростає ще швидше, ніж $m$. Отже, немає ніякої суперечності
в тому, що $p_1$, $z$ і $r$, кожне само по собі, можуть постійно зростати,
тимчасом як $\frac{m}{K} \deq{} \frac{p}{K}$, а також $\frac{p_1}{K}$, $\frac{z}{K}$ і $\frac{r}{K}$, кожне само
по собі, постійно зменшуються, або що $р_1$ порівняно з $z$, або
$r$ порівняно з $р_1$, абож порівняно з $р_1$ і $z$ відносно зростає.
При зростаючій сукупній додатковій вартості або зиску $m \deq{} p$, але
при одночасно падаючій нормі зиску $\frac{m}{K} \deq{} \frac{p}{K}$ відношення величин
частин $p_1$, $z$ і $r$, на які розпадається $m \deq{} p$, може як завгодно
змінюватись у межах, даних сукупною сумою $m$, при чому
на величину $m$ або $\frac{m}{K}$ це не впливає.

Взаємна зміна $p_1$, $z$ і $r$ є тільки різний розподіл $m$ між різними
рубриками. Тому і $\frac{p_1}{K}$, $\frac{z}{K}$ або $\frac{r}{K}$, норма індивідуального
промислового зиску, норма процента і відношення ренти до
сукупного капіталу можуть підвищуватись одно порівняно з одним,
хоч $\frac{m}{K}$, загальна норма зиску, падає; умовою при цьому
лишається тільки те, щоб сума всіх трьох $= \frac{m}{K}$. Якщо норма
зиску падає з 50\% до 25\%, якщо, наприклад, склад капіталу,
при нормі додаткової вартості в 100\%, змінюється з $50 c \dplus{} 50 v$
у $75 c \dplus{} 25 v$, то в першому випадку капітал в 1000 дасть зиск
\parbreak{}  %% абзац продовжується на наступній сторінці

\parcont{}  %% абзац починається на попередній сторінці
\index{ii}{0241}  %% посилання на сторінку оригінального видання
свого функціонування само підприємство через капіталізацію певної частини
додаткової вартости. Для капіталіста \emph{В} це не можливо. Частина
капіталу, що про неї мовиться, мусить складати в нього частину первісно
авансованого капіталу. В обох випадках ця частина капіталу фігуруватиме
в книгах капіталіста як авансований капітал — і ним вона є в
дійсності — бо, згідно з нашим припущенням, вона становить частину
продуктивного капіталу, доконечного для провадження підприємства
в даному маштабі. Але величезна ріжниця в тому, з якого фонду
її авансується. У \emph{В} вона дійсно є частина первісного авансованого
капіталу або капіталу, що його треба мати в розпорядженні.
Навпаки, в \emph{А} вона є частина додаткової вартости, застосованої як
капітал. Цей останній випадок показує нам як не лише акумульований
капітал, а й частина первісно авансованого капіталу може бути просто
капіталізованою додатковою вартістю.

Скоро сюди долучається розвиток кредиту, відношення первісно авансованого
капіталу й капіталізованої додаткової вартости заплутується
ще більше. Напр., \emph{А} позичає в банкіра \emph{С} частину продуктивного капіталу,
що з ним, він починає або продовжує справу протягом року. З
самого початку він не має власного капіталу, достатнього для провадження
справи. Банкір \emph{С} позичає йому суму, що складається виключно з
додаткової вартости, покладеної до нього підприємцями \emph{D}, \emph{E}, \emph{F} і т. ін.
З погляду А тут ще не йдеться про акумульований капітал. А в дійсності
для \emph{D}, \emph{E}, \emph{F} і т. ін. \emph{А} є не що інше, як аґент, що капіталізує привласнену
ними додаткову вартість.

В книзі 1, розділ 22, ми бачили, що акумуляція, перетворення додаткової
вартости на капітал, своїм реальним змістом є процес репродукції
в поширеному маштабі, все одно, чи виявляється таке поширення
екстенсивно у вигляді долучення нових фабрик до старих, чи в інтенсивному
поширенні попереднього маштабу підприємства.

Розмір продукції може поширюватись малими дозами, оскільки
частину додаткової вартости застосовується на такі поліпшення, що або
тільки підвищують продуктивну силу вживаної праці, або разом з тим
дають змогу і визискувати її інтенсивніше. Або ж, коли робочий день не
обмежено законом, досить додаткової витрати обігового капіталу (на матеріяли
продукції та заробітну плату), щоб поширити розміри підприємства,
не збільшуючи основного капіталу, що його денний протяг вживання
таким чином лише подовжується, тимчасом як період обороту його
відповідно скорочується. Або, за сприятливих ринкових коньюнктур, капіталізована
додаткова вартість може дати змогу спекулювати на сировинному
матеріялі, отже, переводити такі операції, що для них не вистачило
б первісно авансованого капіталу, й т. ін.

А проте, очевидно, що там, де порівняно велике число періодів обороту
зумовлює частішу реалізацію додаткової вартости протягом року,
будуть наставати періоди, коли не можна буде ні подовжувати робочий
день, ні заводити частинні поліпшення; тимчасом як, з другого боку, пропорційне
поширення цілого підприємства, зумовлене почасти загальним
\parbreak{}  %% абзац продовжується на наступній сторінці

\parcont{}  %% абзац починається на попередній сторінці
\index{ii}{0242}  %% посилання на сторінку оригінального видання
характером підприємства, напр., будівель, почасти поширенням фонду
робочої сили, як у сільському господарстві, можливе лише в певних
більш-менш вузьких межах, і для цього треба додаткового капіталу
такого розміру, що його може дати лише багаторічна акумуляція додаткової
вартости.

Отже, поряд справжньої акумуляції або перетворення додаткової
вартости на продуктивний капітал (і відповідної репродукції в поширеному
розмірі) відбувається акумуляція грошей, нагромадження частини додаткової
вартости як лятентного грошового капіталу, який лише пізніше,
досягши певних розмірів, має функціонувати як додатковий активний
капітал.

Так стоїть справа з погляду поодинокого капіталіста. Однак, з розвитком
капіталістичної продукції розвивається одночасно кредитова система.
Грошовий капітал, що його капіталіст ще не може застосувати в своєму
власному підприємстві, застосовує інший і платить за це йому проценти. Він
функціонує для свого власника як грошовий капітал в особливому
значенні, як особливий ґатунок капіталу, відмінний від продуктивного
капіталу. Але він діє як капітал в руках другого. Очевидно, що при
частішій реалізації додаткової вартости і при збільшенні маштабу, що
в ньому її продукується, зростає пропорція, що в ній новий грошовий
капітал, або гроші як капітал, подається на грошовий ринок, а відси
знову вбирається — принаймні більшу частину його — для поширення
продукції.

Найпростіша форма, що в ній може виявлятися цей додатковий лятентний
грошовий капітал, є форма скарбу. Можливо, що цей скарб є
додаткове золото або срібло, одержане безпосередньо або посередньо
в обміні з країнами, що продукують благородні металі. І тільки таким
способом в країні абсолютно зростає грошовий скарб. З другого боку,
можливо — і так здебільша буває, — що цей скарб є не що інше, як
гроші, вилучені з циркуляції всередині країни, що набрали форму скарбу
в руках поодиноких капіталістів. Можливо далі, що цей лятентний грошовий
капітал складається просто з знаків вартости — кредитові гроші
ми тут ще лишаємо осторонь — або з простих, потверджених леґальними
документами вимог (юридичних титулів) капіталістів до третіх осіб. В
усіх цих випадках, хоч яка буде форма буття цього додаткового грошового
капіталу, він, оскільки він є капітал in spe\footnote*{
In spe — досл.; „в надії, в перспективі“, тобто потенціяльно. \emph{Ред.}
}, репрезентує не
що інше, як додаткові та в запасі тримані юридичні титули капіталістів на
майбутню додаткову річну продукцію суспільства.

„Таким чином, маса справді акумульованого багатства, розглядувана
з кількісного боку,\dots{} надзвичайно мала порівняно з продуктивними
силами суспільства, що йому воно належить, хоч на якому щаблі цивілізації
стояло б те суспільство; або навіть порівняно з дійсним споживанням
цього самого суспільства протягом лише небагатьох років; остільки
мала, що головну увагу законодавців та політико-економів треба було б
\parbreak{}  %% абзац продовжується на наступній сторінці

\parcont{}  %% абзац починається на попередній сторінці
\index{i}{0243}  %% посилання на сторінку оригінального видання
ступеня експлуатації робочої сили. Цей цілком очевидний другий
закон є важливий для пояснення багатьох явищ, які випливають
із тенденції капіталу, що її ми маємо розвинути пізніш,
а саме тенденції якомога більше скорочувати число робітників,
що їх він вживає, або його змінну складову частину, перетворену
на робочу силу, — всупереч іншій його тенденції, а саме продукувати
якомога більшу масу додаткової вартости. Навпаки, коли
маса вживаних робочих сил або величина змінного капіталу
зростає, алеж непропорційно до зменшення норми додаткової
вартости, то маса продукованої додаткової вартости меншає.\footnote*{
У французькому виданні цей абзац подано так: «Цей абсолютно
ясний закон є важливий для розуміння складних явиш. Ми вже знаємо,
що капітал намагається продукувати якомога більше додаткової вартости;
ми побачимо пізніш, що він разом із цим намагається скоротити до
мінімуму, порівняно з розмірами підприємства, свою змінну частину,
або кількість робітників, що їх він експлуатує. Ці тенденції стають одна
одній суперечними, скоро лише зменшення одного з факторів, що визначають
суму додаткової вартости, вже не може бути компенсоване збільшенням
другого». («Le Capital etc.», v. I, ch. XI, p. 132). Ред.
}

Третій закон випливає з визначення маси продукованої додаткової
вартости двома факторами: нормою додаткової вартости й
величиною авансованого змінного капіталу. Коли дано норму
додаткової вартости, або ступінь експлуатації робочої сили,
і вартість робочої сили, або величину доконечного робочого
часу, то само собою зрозуміло, що чим більший змінний капітал,
тим більша маса продукованої вартости й додаткової вартости.
Коли дано межі робочого дня, а також межі його доконечної
складової частини, то маса вартости й додаткової вартости,
що її продукує поодинокий капіталіст, очевидно, залежить
виключно від тієї маси праці, яку він пускає в рух. Але
маса ця, за даних припущень, залежить від маси робочої сили,
або від числа робітників, яких він експлуатує; а це число, з
свого боку, визначається величиною авансованого ним змінного
капіталу. Отже, за даної норми додаткової вартости й даної вартости
робочої сили маси продукованої додаткової вартости є
просто пропорційні до величин авансованих змінних капіталів.
Та тепер уже відомо, що капіталіст ділить свій капітал на
дві частини. Одну частину він вкладає в засоби продукції. Це —
стала частина його капіталу. Другу частину він перетворює на
живу робочу силу. Ця частина становить його змінний капітал.
На базі того самого способу продукції в різних галузях продукції
відбувається різний поділ капіталу на сталу та змінну складові
частини. В тій самій галузі продукції це відношення змінюється
разом із зміною технічної основи й суспільних комбінацій процесу
продукції. Але хоч як розпадатиметься даний капітал на сталу
й змінну складові частини, чи остання відноситиметься до першої
як 1 : 2, 1 : 10, або 1 : х, — це не порушує щойно встановленого
закону, бо, згідно з попередньою аналізою, вартість сталого капіталу
хоч і з’являється знов у вартості продукту, але не увіходить
у новоутворену вартість. Щоб уживати 1.000 прядунів, потрібно,
\parbreak{}  %% абзац продовжується на наступній сторінці

\parcont{}  %% абзац починається на попередній сторінці
\index{iii1}{0244}  %% посилання на сторінку оригінального видання
виробництва вимагають застосування масового капіталу. Це зумовлює
також централізацію капіталу, тобто поглинення дрібних
капіталістів великими і втрату першими своїх капіталів. Це знов
таки є відокремлення, хоча тільки вторинного порядку, умов праці
від виробників, до яких ще належать ці дрібні капіталісти, бо
у них власна праця грає ще певну роль; взагалі праця капіталіста
стоїть у зворотному відношенні до величини його капіталу, тобто
до тієї міри, в якій він є капіталіст. Це є те відокремлення одне
від одного умов праці з одного боку і виробників з другого,
яке становить поняття капіталу, яке починається з первісним
нагромадженням (книга І, розд. XXIV), потім виявляється як постійний
процес в нагромадженні і концентрації капіталу і, нарешті,
виражається тут як централізація вже наявних капіталів у небагатьох
руках і втрата капіталів (так змінюється тепер експропріація)
багатьма. Цей процес швидко привів би капіталістичне
виробництво до краху, коли б постійно поряд з доцентровою
силою знов і знов децентралізаційно не діяли протидіючі тенденції.

\subsection{Конфлікт між розширенням виробництва
і зростанням вартості}

Розвиток суспільної продуктивної сили праці виявляється
двояко: поперше, у величині вже вироблених продуктивних сил,
в розмірі вартості і в розмірі маси виробничих умов, при яких
відбувається нове виробництво, і в абсолютній величині нагромадженого
вже продуктивного капіталу; подруге, у відносно
незначній величині витрачуваної на заробітну плату частини
капіталу порівняно з усім капіталом, тобто у відносно незначній
кількості живої праці, потрібної для репродукції і збільшення
вартості даного капіталу, для масового виробництва. А це передбачає
разом з тим концентрацію капіталу.

Відносно вживаної робочої сили розвиток продуктивної сили
виявляється знов таки двояко: поперше, в збільшенні додаткової
праці, тобто в скороченні необхідного робочого часу, потрібного
для репродукції робочої сили. Подруге, в зменшенні кількості
робочої сили (числа робітників), яка взагалі вживається для
того, щоб привести в рух даний капітал.

Обидва ці рухи не тільки йдуть рука в руку, але взаємно
зумовлюють один одного; обидва вони є явища, в яких
виражається один і той самий закон. Проте, вони діють на
норму зиску в протилежному напрямі. Сукупна маса зиску
дорівнює сукупній масі додаткової вартості, $\text{норма зиску} =
\frac{m}{K} = \frac{\text{додаткова вартість}}{\text{сукупний авансований капітал}}$. Але додаткова вартість, як сукупна сума,
визначається, поперше, її нормою, а подруге,
масою одночасно вживаної при цій нормі праці або, що є те
саме, величиною змінного капіталу. З одного боку, підвищується
один фактор, норма додаткової вартості; з другого боку, зменшується
\index{iii1}{0245}  %% посилання на сторінку оригінального видання
(відносно або абсолютно) другий фактор, число робітників.
Оскільки розвиток продуктивних сил зменшує оплачувану
частину вживаної праці, він підвищує додаткову вартість, підвищуючи
її норму; проте, оскільки він зменшує всю масу
праці, вживаної даним капіталом, він зменшує другий фактор,
число робітників, на яке треба помножити норму додаткової
вартості, щоб одержати її масу. Двоє робітників, які працюють
по 12 годин на день, не можуть дати такої ж маси додаткової вартості,
як 24 робітники, які працюють тільки по 2 години кожний,
навіть якби вони могли живитись самим повітрям і якби їм через це
зовсім не доводилось працювати на самих себе. Отже, в цьому
відношенні компенсація зменшеного числа робітників підвищенням
ступеня експлуатації праці має певні непереступні межі;
тому вона може, звичайно, затримати падіння норми зиску, але
вона не може його усунути.

Отже, з розвитком капіталістичного способу виробництва
норма зиску падає, тимчасом як маса його із збільшенням маси
застосовуваного капіталу підвищується. При даній нормі абсолютна
маса, на яку зростає капітал, залежить від його величини
в даний момент. Але, з другого боку, якщо цю величину дано,
то відношення, в якому він зростає, норма його зростання, залежить
від норми зиску. Безпосередньо підвищення продуктивної
сили (яке, крім того, як уже згадано, завжди йде рука в руку із
знеціненням наявного капіталу) може збільшити величину вартості
капіталу тільки в тому випадку, коли воно, підвищуючи
норму зиску, збільшує ту частину вартості річного продукту,
яка зворотно перетворюється в капітал. Оскільки мова йде про
продуктивну силу праці, це може статися тільки в тому випадку
(бо ця продуктивна сила безпосередньо не має ніякого відношення
до \emph{вартості} наявного капіталу), коли в наслідок підвищення
продуктивної сили або збільшується відносна додаткова
вартість, або зменшується вартість сталого капіталу, отже, здешевлюються
товари, які входять або в репродукцію робочої сили,
або в елементи сталого капіталу. Але і те і друге означає також
знецінення наявного капіталу; і те і друге йде рука в руку із
зменшенням змінного капіталу порівняно з сталим. І те і друге
зумовлює падіння норми зиску і уповільнює це падіння. Далі,
оскільки підвищена норма зиску спричиняє підвищений попит на
працю, вона впливає на збільшення робітничого населення і разом
з тим на збільшення матеріалу, придатного для експлуатації, який
тільки й робить капітал капіталом.

Але посередньо розвиток продуктивної сили праці сприяє
збільшенню наявної капітальної вартості, збільшуючи масу й різноманітність
споживних вартостей, в яких представлена та сама
мінова вартість і які становлять матеріальний субстрат, речові елементи
капіталу, матеріальні предмети, з яких складається сталий
капітал безпосередньо і змінний, принаймні, посередньо. З тим
самим капіталом і тією самою працею створюється більше речей,
\parbreak{}  %% абзац продовжується на наступній сторінці

\parcont{}  %% абзац починається на попередній сторінці
\index{iii1}{0246}  %% посилання на сторінку оригінального видання
які можуть бути перетворені в капітал, незалежно від їх мінової
вартості; речей, які можуть служити для того, щоб вбирати добавну
працю, отже й добавну додаткову працю, і таким чином
утворити додатковий капітал. Маса праці, якою може розпоряджатись
капітал, залежить не від його вартості, а від маси
сировинних і допоміжних матеріалів, машин і елементів основного
капіталу, засобів існування, тобто всього того, з чого складається
капітал, яка б не була його вартість. В той час, як зростає
таким чином маса вживаної праці, отже й додаткової праці, зростає
також вартість репродукованого капіталу і новододана до
неї додаткова вартість.

Але ці обидва моменти, які включає в собі процес нагромадження,
не можна розглядати тільки в тому стані спокійного співіснування
їх одного поряд одного, в якому їх досліджує Рікардо;
вони містять у собі суперечність, яка виявляється в суперечних
тенденціях і явищах. Антагоністичні фактори діють одночасно
один проти одного.

Одночасно з спонуканням до дійсного збільшення робітничого
населення, що випливає із збільшення частини сукупного
суспільного продукту, яка функціонує як капітал, діють фактори,
які створюють тільки відносне перенаселення.

Одночасно з падінням норми зиску зростає маса капіталів,
і рука в руку з цим відбувається знецінення наявного капіталу,
яке затримує це падіння і спонукає до прискореного нагромадження
капітальної вартості.

Одночасно з розвитком продуктивної сили розвивається
вищий склад капіталу, відносне зменшення змінної частини порівняно
з сталою.

Ці різні впливи виявляються то більше один поряд з одним
у просторі, то більше один по одному в часі; конфлікт антагоністичних
факторів періодично розв’язується (macht sich Luft)
в кризах. Кризи завжди є тільки тимчасові насильні розв’язання
наявних суперечностей, насильні вибухи, які на мить відновлюють
порушену рівновагу.

Суперечність, в її найзагальнішому виразі, полягає в тому,
що капіталістичний спосіб виробництва має тенденцію до абсолютного
розвитку продуктивних сил, незалежно від вартості і
вміщеної в ній додаткової вартості, а також незалежно від
суспільних відносин, при яких відбувається капіталістичне виробництво;
тимчасом як, з другого боку, він має своєю метою
збереження існуючої капітальної вартості і збільшення її в якнайвищій
мірі (тобто постійно прискорюване зростання цієї вартості).
Специфічна особливість капіталістичного способу виробництва
полягає в тому, щоб наявну капітальну вартість використати,
як засіб для якомога дужчого збільшення цієї вартості. Методи,
якими він цього досягає, ведуть до зменшення норми зиску,
знецінення наявного капіталу і розвитку продуктивних сил праці
за рахунок вироблених уже продуктивних сил.

\index{iii1}{0247}  %% посилання на сторінку оригінального видання
Періодичне знецінення наявного капіталу, що є імманентним
капіталістичному способові виробництва засобом затримувати
падіння норми зиску і прискорювати нагромадження капітальної
вартості шляхом утворення нового капіталу, порушує дані відносини,
в яких відбувається процес циркуляції і репродукції
капіталу, і тому воно супроводиться раптовими зупиненнями й
кризами процесу виробництва.

Відносне зменшення змінного капіталу порівняно з сталим,
яке йде рука в руку з розвитком продуктивних сил, дає стимул
зростанню робітничого населення і в той же час постійно створює
штучне перенаселення. Нагромадження капіталу, розглядуване
щодо вартості, уповільнюється, в наслідок падіння норми зиску, і
разом з тим ще більше прискорюється нагромадження споживних
вартостей, тимчасом як це останнє знов таки приводить до прискореного
ходу нагромадження, розглядуваного щодо вартості.

Капіталістичне виробництво постійно намагається перебороти
ці імманентні йому межі, але воно переборює їх тільки такими
засобами, які ставлять перед ним ці межі знову і в ще колосальнішому
масштабі.

Справжня межа капіталістичного виробництва є сам капітал,
є те, що капітал і самозростання його вартості являє собою
вихідний і кінцевий пункт, мотив і мету виробництва; що виробництво
є тільки виробництво для капіталу, а не навпаки:
засоби виробництва не є просто засобами для все ширшого й
ширшого розвитку життьового процесу суспільства виробників.
Межі, в яких тільки й може рухатись збереження і зростання
капітальної вартості, яке грунтується на експропріації і зубоженні
широких мас виробників, — ці межі постійно приходять
через це в суперечність з методами виробництва, які капітал
мусить застосовувати для своєї мети і які спрямовані до необмеженого
збільшення виробництва, до виробництва як самоцілі, до безумовного
розвитку суспільних продуктивних сил праці. Засіб —
безумовний розвиток суспільних продуктивних сил — вступає
в постійний конфлікт з обмеженою метою — збільшенням вартості
наявного капіталу. Тому, якщо капіталістичний спосіб виробництва
є історичний засіб для розвитку матеріальної продуктивної
сили і для створення відповідного їй світового ринку, то
він разом з тим становить постійну суперечність між цим його
історичним завданням і відповідними йому суспільними відносинами
виробництва.

III. Надмір капіталу при надмірі населення

З падінням норми зиску зростає той мінімум капіталу, який
потрібен окремому капіталістові для продуктивного вживання
праці, — потрібен як для експлуатації праці взагалі, так і для
того, щоб вживаний робочий час був робочим часом, необхідним
для виробництва товарів, щоб він не перевищував
\parbreak{}  %% абзац продовжується на наступній сторінці

\parcont{}  %% абзац починається на попередній сторінці
\index{iii1}{0248}  %% посилання на сторінку оригінального видання
пересічного робочого часу, суспільно-необхідного для виробництва
товарів. І одночасно зростає концентрація, бо за певними
межами великий капітал з невеликою нормою зиску нагромаджує
швидше, ніж невеликий капітал з великою нормою зиску. Ця
зростаюча концентрація, з свого боку, досягнувши певної висоти,
знов таки приводить до нового падіння норми зиску. Маса дрібних
розпорошених капіталів у наслідок цього штовхається на шлях
авантюр: спекуляцій, шахрайських кредитних і акційних підприємств,
криз. Так звана плетора [наддостаток] капіталу завжди
стосується головним чином до плетори такого капіталу, для якого
падіння норми зиску не урівноважується масою зиску, — а такі
завжди є новоутворювані свіжі паростки капіталу, — або до плетори
таких капіталів, які, будучи самі по собі нездатними самостійно
функціонувати, передаються в формі кредиту в розпорядження
керівників великих галузей підприємств. Ця плетора капіталу виростає
з тих самих обставин, які викликають відносне перенаселення,
і тому вона є явище, яке доповнює це останнє, хоч обоє
вони перебувають на протилежних полюсах: на одному боці — незанятий
капітал, на другому боці — незаняте робітниче населення.

Перепродукція капіталу, а не окремих товарів, — хоч перепродукція
капіталу завжди включає перепродукцію товарів, —
означає через це не що інше, як перенагромадження капіталу.
Щоб зрозуміти, що таке є це перенагромадження (докладніше
дослідження його ми подаємо нижче), досить тільки припустити
його абсолютним. Коли перепродукція капіталу була б абсолютною?
І при тому перепродукція, яка поширювалася б не на ту
чи іншу або декілька значних сфер виробництва, а була б абсолютною
в самому своєму об’ємі, отже, охоплювала б усі сфери
виробництва?

Абсолютна перепродукція капіталу була б у наявності в тому
випадку, коли додатковий капітал для цілей капіталістичного
виробництва був би = 0. Але метою капіталістичного виробництва
є збільшення вартості капіталу, тобто привласнення додаткової
праці, виробництво додаткової вартості, зиску. Отже,
коли б капітал зріс порівняно з робітничим населенням настільки,
що не можна було б ні здовжити абсолютний робочий час, що
його дає це населення, ні розширити відносний додатковий робочий
час (останнє, крім того, було б нездійсниме при таких обставинах,
коли попит на працю є такий значний, отже, коли є тенденція
до підвищення заробітної плати), тобто коли б зрослий
капітал виробляв тільки таку саму або навіть меншу масу додаткової
вартості, ніж до свого зростання, то мала б місце абсолютна
перепродукція капіталу; тобто зрослий капітал К + ΔК виробляв
би не більше зиску, або навіть менше зиску, ніж капітал К
до свого збільшення на ΔК. В обох випадках мало б також
місце значне і раптове падіння загальної норми зиску, але на
цей раз в наслідок переміни у складі капіталу, викликаної не розвитком
продуктивної сили, а підвищенням грошової вартості
\parbreak{}  %% абзац продовжується на наступній сторінці

\parcont{}  %% абзац починається на попередній сторінці
\index{ii}{0249}  %% посилання на сторінку оригінального видання
їхнього існування, другу — $b$, що її вони почасти витрачають на речі
розкошів, а почасти застосовують на поширення продукції; $а$ — в такому
разі репрезентує змінний капітал, $b$ — додаткову вартість. Але такий поділ
не мав би жодного впливу на величину тієї маси грошей, яка потрібна
для циркуляції цілого їхнього продукту. За інших незмінних умов, вартість
товарової маси, що циркулює, була б та сама, а значить, і маса
потрібних для цього грошей була б та сама. Крім того, при однаковому
поділі періодів обороту продуценти мусили б мати такі самі грошові запаси,
тобто постійно мати в грошовій формі таку саму частину свого
капіталу, бо, згідно з нашим припущенням, їхня продукція, як і раніш,
була б товаровою продукцією. Отже, та обставина, що частина товарової
вартости складається з додаткової вартости, абсолютно не змінює маси
грошей доконечних для провадження підприємства.

Один з супротивників Тука, що тримається формули $Г — Т — Г$, запитує
його, як капіталістові вдається постійно вилучати з циркуляції більше
грошей, ніж він подає туди. Це цілком зрозуміло. Тут ідеться не про
утворення додаткової вартости. Останнє, являючи єдину таємницю, з
капіталістичного погляду само собою зрозуміле. Застосована бо сума вартости
не була б капіталом, коли б вона не збагачувалась додатковою
вартістю. А що згідно з припущенням вона є капітал, то додаткова вартість
сама собою зрозуміла.

Отже, питання не в тім, відки береться додаткова вартість, а в тім,
відки беруться гроші, що на них вона перетворюється.

Але для буржуазної економії існування додаткової вартости зрозуміле
само собою. Отже, її не лише припускається, але разом з нею припускається
й те, що частина товарової маси, пущеної в циркуляцію, складається
з додаткового продукту, отже, репрезентує таку вартість, що її капіталіст
не кинув у циркуляцію, кидаючи туди свій капітал; що, отже, капіталіст,
разом з своїм продуктом кидає в циркуляцію певний надлишок
порівняно з своїм капіталом, а потім знову вилучає з неї цей надлишок.

Товаровий капітал, що його капіталіст подає в циркуляцію, має більшу
вартість (звідки це постає, не пояснюється або не розуміється, але з
погляду буржуазної економії c’est un fait\footnote*{
Це — факт. \Red{Ред.}
}, ніж продуктивний капітал,
що його він вилучив з циркуляції в формі робочої сили плюс засоби
продукції. Тому при цьому припущенні ясно, чому не лише капіталіст
$А$, але й $В$, $С$, $D$ і~\abbr{т. ін.} можуть постійно вилучати з циркуляції через
обмін своїх товарів більшу вартість, ніж вартість їхнього первісно авансованого
капіталу, що його потім знову й знову авансується. $А$, $В$, $С$,
$D$ і~\abbr{т. ін.} завжди подають в циркуляцію в формі товарового капіталу, —
а ця операція так само багатобічна, як і самостійно діющі капітали, —
більшу товарову вартість, ніж та, що її вони вилучають з циркуляції в
формі продуктивного капіталу. Отже, їм постійно доводиться розподіляти
між собою суму вартости (тобто кожному доводиться вилучати для себе
з циркуляції продуктивний капітал), що дорівнює сумі вартости їхніх
\parbreak{}  %% абзац продовжується на наступній сторінці

\index{iii1}{0250}  %% посилання на сторінку оригінального видання
Частина старого капіталу мусила б лишитись без діла при
всяких обставинах, лишитись без діла щодо своєї властивості як
капіталу — функціонувати і зростати в своїй вартості. Яка саме
частина лишилася б без діла, це вирішила б конкурентна боротьба.
Поки все йде добре, конкуренція, як це виявилось при
вирівненні загальної норми зиску, діє, як практичний братерський
союз класу капіталістів, так що вони спільно ділять між собою загальну
здобич пропорціонально до величини частки, вкладеної
кожним з них. Але як тільки справа йде вже не про розподіл
зиску, а про розподіл збитку, то кожний з них намагається
якомога зменшити свою участь в ньому і перекласти його на
шию іншому. Для всього класу збиток є неминучий. Але скільки
з нього припаде на кожного окремого капіталіста, наскільки
взагалі кожний з них повинен взяти участь в ньому, це стає
тоді питанням сили й хитрості, і конкуренція перетворюється
тоді в боротьбу ворогуючих братів. Протилежність між інтересами
кожного окремого капіталіста і інтересами класу капіталістів виявляється
при цьому цілком так само, як перед тим за допомогою
конкуренції проявлялась на практиці тотожність цих інтересів.

Яким же чином міг би бути знов усунений цей конфлікт
і як могли б відновитись відносини, відповідні „здоровому“
рухові капіталістичного виробництва? Спосіб усунення міститься
вже в простому констатуванні конфлікту, про усунення якого
йде мова. Він полягає в залишенні без діла і навіть частковому
знищенні капіталу, рівного своєю вартістю всьому додатковому
капіталові $ΔК$ або принаймні частині його. Хоча — як
це вже випливає з викладу конфлікту — розподіл цього збитку
ні в якому разі не поширюється рівномірно на поодинокі окремі
капітали, а вирішується в конкурентній боротьбі, в якій збиток
розподіляється дуже нерівно і в дуже різних формах, залежно
від особливих переваг або особливих завойованих уже позицій,
так що один капітал лишається лежати без діла, другий знищується,
третій має тільки відносний збиток або зазнає тільки
тимчасового знецінення і т. д.

Але при всяких обставинах рівновага відновилась би в наслідок
бездіяльності і навіть знищення капіталу в більшому чи
меншому розмірі. Це почасти поширилося б на матеріальну
субстанцію капіталу; тобто частина засобів виробництва, основний
і обіговий капітал, не функціонувала б, не діяла б як капітал;
частина підприємств, що вже почали функціонувати, припинила
б роботу. Хоча в цьому відношенні час робить своє
і погіршує всі засоби виробництва (за винятком землі), але тут
в наслідок припинення функціонування мало б місце значно
сильніше справжнє руйнування засобів виробництва. Головний
результат у цьому відношенні був би, однак, у тому, що ці засоби
виробництва перестали б діяти як засоби виробництва, — в зруйнуванні
їх функції як засобів виробництва на коротший чи довший
час.


\index{iii1}{0251}  %% посилання на сторінку оригінального видання
Найбільш руйнівного впливу, і при тому найгострішого
характеру, зазнав би капітал, оскільки він має властивість
вартості, зазнали б капітальні \emph{вартості}. Частина капітальної
вартості, яка перебуває просто у формі посвідок на майбутню
участь в додатковій вартості, в зиску, і яка в дійсності становить
тільки боргові зобов’язання в різних формах на виробництво,
відразу знецінюється з падінням доходів, на які вона розрахована.
Частина готівки золота й срібла лежить без діла, не функціонує
як капітал. Частина товарів, що перебувають на ринку, може
здійснити свій процес циркуляції і репродукції тільки шляхом
надзвичайного зниження своїх цін, отже, шляхом знецінення
того капіталу, який вона представляє. Цілком так само більше
чи менше знецінюються елементи основного капіталу. До цього
долучається ще й те, що певні припущені відношення цін
обумовлюють процес репродукції, і тому цей останній в наслідок
загального падіння цін приходить до застою і розладу. Цей
розлад і застій паралізує функцію грошей як платіжного засобу,
яка розвивається одночасно з розвитком капіталу і грунтується
на згаданих припущених відношеннях цін; він розриває у сотнях
місць ланцюг платіжних зобов’язань на певні строки і ще більше
загострюється в наслідок зумовленого цим краху (Zusammenbrechen)
кредитної системи, що розвинулась одночасно з капіталом,
і таким чином веде до сильних і гострих криз, до раптових
насильних знецінень і дійсного застою й розладу
% REMOVED \footnote*{
% В першому німецькому виданні тут стоїть: „застою й занепаду (Sturz)“;
% виправлено на підставі рукопису Маркса. \Red{Примітка ред. нім. вид, ІМЕЛ.}
% }
процесу
репродукції, і тим самим до дійсного зменшення репродукції.

Але одночасно діяли б і інші фактори. Застій виробництва
позбавив би роботи частину робітничого класу і цим поставив би
заняту частину його в такі відносини, при яких вона мусила б
згоджуватись на зниження заробітної плати навіть нижче пересічного
рівня; обставина, яка дає для капіталу такий самий результат,
як коли б при пересічній заробітній платі була підвищена
відносна чи абсолютна додаткова вартість. Період процвітання
сприяв би шлюбам серед робітників і зменшив би смертність їх
дітей, обставини, які — хоч би й яке вони означали дійсне збільшення
населення — не означають збільшення дійсно працюючого
населення, але на відношення робітників до капіталу впливають
цілком так само, як коли б збільшилося число дійсно функціонуючих
робітників. З другого боку, падіння цін і конкурентна
боротьба спонукали б кожного капіталіста підвищувати індивідуальну
вартість свого сукупного продукту понад його загальну
вартість за допомогою застосування нових машин, нових поліпшених
методів праці, нових комбінацій, тобто підвищувати продуктивну
силу даної кількості праці, знижувати відношення
змінного капіталу до сталого, і тим самим звільняти робітників,
\parbreak{}  %% абзац продовжується на наступній сторінці

\parcont{}  %% абзац починається на попередній сторінці
\index{iii2}{0252}  %% посилання на сторінку оригінального видання
які витрачається дохід, тобто правити за засоби споживання, перебігають протягом
року різні ступені, наприклад, вовняна пряжа, сукно. На одному ступені
вони становлять частину сталого капіталу, на другому — їх особисто споживається,
отже, цілком, входять в склад доходу. Можна, отже, уявити собі разом
з А. Смітом, що сталий капітал є лише позірний елемент товарової вартости,
який в загальному зв’язку зникає. Таким самим чином відбувається далі обмін
змінного капіталу на дохід. Робітник купує на свою заробітну плату частину
товарів, що становить його дохід. Одночасно він покриває цим самим для капіталіста
грошову форму змінного капіталу. Нарешті: частина продуктів, що становлять
сталий капітал, покривається або in natura, або за посередництвом обміну
між самими продуцентами сталого капіталу; процес, до якого споживачі
не мають жодного чинення. Коли спустити це з уваги, то може постати зовнішня
видимість, що дохід споживачів покриває ввесь продукт, отже і сталу частину
вартости.

5) Крім плутанини, яку вносить перетворення вартостей на ціни продукції,
виникає ще й інша в наслідок перетворення додаткової вартости на різні окремі
форми доходу, самостійні одна проти однієї і залічені до різних елементів продукції,
на зиск і ренту. При цьому забувається, що вартості товарів є основою, і що
розпадання цієї товарової вартости на окремі складові частини, і дальший розвиток
цих складових частин вартости у форми доходу, їх перетворення на відносини
різних посідачів різних чинників продукції до цих окремих складових
частин вартости, їх розподіл між цими посідачами згідно з певними категоріями
і титулами, нічого не змінює у самому визначенні вартости й законів її.
Так само мало змінюється закон вартости тією обставиною, що вирівнювання
зиску, тобто розподіл сукупної додаткової вартости між різними капіталами, і
перешкоди, що почасти (в абсолютній ренті) ставляться землеволодінням на
шляху цього вирівнювання, призводять до відхилу регуляційних пересічних цін
товарів від їхніх індивідуальних вартостей. Це впливає знов таки тільки на
добавку додаткової вартости до цін різних товарів, але не знищує самої додаткової
вартости і сукупної вартости товарів як джерела цих різних складових
частин ціни.

Тут перед нами quid pro quo, яке ми розглядаємо в дальшім розділі, і
яке неминуче зв’язане з ілюзією, що нібито вартість виникає з її власних
складових частин. А саме: спершу різні складові частини вартости товару набувають
в доходах самостійних форм, і як такі доходи їх залічують до окремих
речових елементів продукції, як до їхніх джерел, замість залічити їх до
вартости товару як до їхнього джерела. Вони дійсно залічуються до зазначених
окремих джерел, але не як складові частини вартости, а як доходи, як складові
частини вартости, що дістаються цим певним категоріям агентів продукції: робітникові,
капіталістові, земельному власникові. А проте, можна уявити собі, що
ці складові частини вартости замість виникати від розкладу товарової вартости,
навпаки, лише створюють її своїм сполученням; тоді саме й виникає чудове
порочне коло: вартість товарів виникає з суми вартости заробітної плати, зиску,
ренти, а вартість заробітної плати, зиску, ренти в свою чергу визначається
вартістю товарів і т. ін.\footnote{
«Оборотний капітал, витрачений на матеріяли, сировий матеріял і викінчені вироби, сам
складається з товарів, що їх потрібна ціна складена з тих самих елементів; так що, розглядаючи
сукупність товарів у певній країні, було б зайво зачислювати цю частину оборотного капіталу до
елементів потрібної ціни». (Storch, Cours d’Ec. Pol., II, p. 140). Під цими елементами оборотного
капіталу Шторх розуміє (основний — це тільки змінена форма оборотного) сталу частину вартости.
«Правда, що заробітна плата робітника так само як і частина зиску підприємця, яка складається з
заробітних плат, коли розглядати їх як частину засобів існування, і собі складається з товарів, що
куплені по ринковій ціні, і теж мають в собі заробітні плати, ренти на капітали, земельні ренти і
підприємницькі зиски\dots{} спостереження це доводить лише неможливість розкласти потрібну ціну на її
простіші елементи» (ib., примітка). — У своїх Considérations sur la nature du revenu national (Paris
1824)
Шторх y своїй полеміці s Сеєм, правда, розуміє все безглуздя, що до нього призводить помилкова
аналіза товарової вартости, що розкладає її без остачі тільки на доходи, і правильно висловлюється
про все безглуздя цих висновків — з погляду не поодиноких капіталістів, а нації, — але сам він не
робить і кроку вперед в аналізі prix nécessaire (потрібної ціни), відносно якої він, замість
відсувати розв’язання
завдання до безконечности, заявляє в своєму «Cours» що її неможливо розкласти на її дійсні елементи.
«Ясно, що вартість річного продукту поділяється почасти на капітали, почасти на зиски, і що кожна з
цих частин вартости річного продукту регулярно купує продукти, потрібні нації так для збереження її
капіталу, як і для відновлення її споживного фонду (р. 134--135)\dots{} Чи зможе вона (селянська родина,
що працює самостійно) жити у своїх клунях і стайнях, живитись тільки насінням
і травою, одягатися з своєї робочої худоби, витрачати свої хліборобські зваряддя? Згідно з
твердженими п. Сея, слід було б відповісти позитивно на всі ці питання (135--136)\dots{} Коли визнати, що
дохід нації дорівнює її гуртовому продуктові, тобто, що з нього не доводиться вираховувати капітали,
то доведеться також визнати, що вона може непродуктивно витратити всю вартість свого річного
продукту,
не роблячи найменшої шкоди своєму майбутньому доходові. (147) Продукти, що складають капітал нації,
не підлягають споживанню», (р. 150)
}.

\index{iii2}{0253}  %% посилання на сторінку оригінального видання
При нормальному стані продукції тільки частина новодолученої праці
вживається на продукцію і тому на покриття сталого капіталу: саме якраз та
частина, що покриває сталий капітал, витрачений у продукції засобів споживання,
речових елементів доходу. Це вирівнюється тим, що ця стала частина,
кляси II не коштує новодолучуваної праці. Але сталий капітал, який (коли розглядати
сукупний процес репродукції, що в ньому, отже, включено і те вирівнювання
кляс І і II), не є продукт новодолученої праці, хоч цей продукт неможливо
було б випродукувати без нього, — цей сталий капітал, розглядуваний з речового
боку, підлягає підчас процесу репродукції, випадковостям і небезпекам,
які можуть його зменшити. (Але далі, коли розглядати його щодо вартости, то
він також може знецінитися в наслідок зміни у продуктивній силі праці;
проте, це стосується лише до поодиноких капіталістів). Відповідно до цього частина
зиску, отже, додаткової вартости, а тому і додаткового продукту, що в ньому
(коли розглядати його з погляду вартости) репрезентується лише новодолучена
праця, служить страховим фондом. При цьому суть справи ані трохи не змінюється
від того, чи порядкує цим страховим фондом страхове товариство як
окреме підприємство, чи ні. Це є однісінька частина доходу, що не споживається
як такий, і не служить неодмінно фондом акумуляції. Чи служить вона фактично
фондом акумуляції, чи лише покриває прогріхи репродукції, це залежить від
випадку. Це також однісінька частина додаткової вартости і додаткового продукту,
отже, додаткової праці, що крім частини, яка служить для акумуляції, отже,
для поширення процесу репродукції, мусить існувати і далі по знищенні
капіталістичного способу продукції. Звичайно, це має своєю передумовою, що
частина, регулярно споживувана безпосередніми продуцентами, не лишиться обмеженою на своєму
теперішньому мінімальному рівні. За винятком додаткової
праці на тих, хто через свій вік ще не може або вже не може брати участи у продукції,
відпаде всяка праця на утримання тих, хто не працює. Коли ми уявимо
собі суспільство при його виниканні, то побачимо, що тут немає ще випродукованих
засобів продукції, отже, немає сталого капіталу, що його вартість увіходить
у продукт, і при репродукції в тому самому маштабі мусить покриватися
in natura з продукту в розмірі, визначуваному його вартістю. Але природа
безпосередньо дає тут засоби існування, їх не доводиться продукувати. Тому
вона залишає також дикунові, що йому доводиться задовольняти лише малі потреби,
час, — не на те, щоб використати ще не сущі в наявності засоби продукції для
нової продукції, а на те, щоб, крім праці, якої коштує привласнення наявних
у природі засобів існування, витрачати працю на перетворення інших продуктів
природи на засоби продукції, лук, кам’яний ніж, човен і т. ін. Процес цей,
\parbreak{}  %% абзац продовжується на наступній сторінці

\input{_0254.tex}
\parcont{}  %% абзац починається на попередній сторінці
\index{iii1}{0255}  %% посилання на сторінку оригінального видання
в падінні норми зиску такий закон, який на певному пункті
найбільш вороже виступає проти розвитку самого цього способу
виробництва і який через це мусить раз-у-раз переборюватись
кризами.

2)~В тому, що для розширення чи скорочення виробництва
вирішальним є привласнення неоплаченої праці і відношення цієї
неоплаченої праці до упредметненої праці взагалі або, висловлюючись
капіталістично, що вирішальним для цього є зиск
і відношення цього зиску до застосовуваного капіталу, отже,
певна висота норми зиску, а не відношення виробництва до
суспільних потреб, до потреб суспільно розвинених людей. Тому
капіталістичне виробництво доходить до своєї межі вже на
такому ступені розширення виробництва, який, навпаки, при
інших передумовах був би далеко недостатнім. Воно припиняється
не тоді, коли цього вимагає задоволення потреб, а тоді,
коли цього припинення вимагає виробництво і реалізація зиску.

Якщо норма зиску знижується, то, з одного боку, сили
капіталу скеровуються на те, щоб окремий капіталіст за допомогою
кращих методів і~\abbr{т. д.} знизив індивідуальну вартість
кожної одиниці своїх товарів нижче її суспільної пересічної
вартості і таким чином одержав би при даній ринковій ціні
надзиск; з другого боку, виникають грюндерські підприємства
і загальний сприятливий ґрунт для них в завзятих спробах застосування
нових методів виробництва, в нових капіталовкладеннях,
в нових авантюрах, щоб забезпечити хоч якийнебудь
надзиск, який не залежав би від загального пересічного рівня
і перевищував би його.

Норма зиску, тобто відносний приріст капіталу, має важливе
значення передусім для всіх нових паростків капіталу, які групуються
самостійно. І коли б утворення капіталів потрапило
виключно в руки деяких небагатьох уже наявних великих
капіталів, для яких маса зиску урівноважує його норму, то
взагалі згас би вогонь, який оживляє виробництво. Виробництво
охопив би сон. Норма зиску є рушійна сила капіталістичного
виробництва; виробляється тільки те і остільки, що і оскільки
може бути вироблене з зиском. Звідси острах англійських економістів
перед зменшенням норми зиску. Те, що вже сама тільки
можливість цього непокоїть Рікардо, свідчить якраз про його
глибоке розуміння умов капіталістичного виробництва. Якраз те,
що йому закидають, а саме, що він при розгляді капіталістичного
виробництва, не турбуючись про „людей“, звертає увагу
тільки на розвиток продуктивних сил, — яких би це не коштувало
жертв людьми і капітальними \emph{вартостями} — якраз це є
в нього найвизначніше. Розвиток продуктивних сил суспільної
праці є історичне завдання і виправдання капіталу. Саме цим він
несвідомо утворює матеріальні умови вищої форми виробництва.
Рікардо непокоїть те, що нормі зиску, цьому стимулові капіталістичного
виробництва, умові й рушієві нагромадження, загрожує
\index{iii1}{0256}  %% посилання на сторінку оригінального видання
небезпека в наслідок розвитку самого виробництва. А кількісне
відношення тут — усе. В дійсності в основі цього лежить щось
глибше, про що він тільки догадується. В цьому виявляється
чисто економічним способом, тобто з буржуазної точки зору,
в межах капіталістичного розуміння, з точки зору самого капіталістичного
виробництва, обмеженість капіталістичного виробництва,
його відносність, те, що воно не є абсолютний, а тільки
історичний спосіб виробництва, відповідний певній обмеженій
епосі розвитку матеріальних умов виробництва.

\subsection{Додатки}

Через те що розвиток продуктивної сили праці відбувається
дуже нерівномірно в різних галузях промисловості, і не тільки
нерівномірно щодо ступеня, а часто і в протилежному напрямі,
то звідси випливає, що пересічна маса зиску (дорівнює додаткової
вартості) мусить стояти далеко нижче тієї висоти, якої можна
було б сподіватися відповідно до розвитку продуктивної сили
в найбільш розвинених галузях промисловості. Те, що розвиток
продуктивної сили в різних галузях промисловості відбувається
не тільки в дуже різних пропорціях, але часто і в протилежному
напрямі, виникає не тільки з анархії конкуренції і своєрідності
буржуазного способу виробництва. Продуктивність праці
зв’язана також з природними умовами, які часто стають менш
вигідними в тій самій мірі, в якій зростає продуктивність, оскільки
вона залежить від суспільних умов. Звідси протилежний рух
в цих різних сферах — прогрес в одних, регрес в інших. Досить
тільки згадати, наприклад, про вплив сезонів року, від чого
залежить кількість найбільшої частини всіх сировинних матеріалів,
про вичерпання лісів, кам’яновугільних і залізнорудних
копалень і~\abbr{т. д.}

Якщо обігова частина сталого капіталу, сировинний матеріал
і~\abbr{т. д.} постійно зростає в своїй масі в міру розвитку продуктивної
сили праці, то цього не можна сказати про основний
капітал, будівлі, машини, пристрої для освітлення, опалення і~\abbr{т. д.}
Хоч машина з зростанням її розмірів стає абсолютно дорожчою,
але відносно вона стає дешевшою. Якщо п’ятеро робітників
виробляють удесятеро більше товарів, ніж раніш, то
з цієї причини витрати на основний капітал не збільшуються
вдесятеро; хоч вартість цієї частини сталого капіталу зростає
з розвитком продуктивної сили, але вона зростає далеко не в такій
самій пропорції. Ми вже не раз відзначали ріжницю між
відношенням сталого капіталу до змінного, як воно виражається
в падінні норми зиску, і тим самим відношенням, як воно,
з розвитком продуктивності праці, виражається щодо одиничного
товару та його ціни.

[Вартість товару визначається всім робочим часом, минулим
і живим, що входить у цей товар. Підвищення продуктивності праці
\parbreak{}  %% абзац продовжується на наступній сторінці

\parcont{}  %% абзац починається на попередній сторінці
\index{iii1}{0257}  %% посилання на сторінку оригінального видання
полягає саме в тому, що частка живої праці зменшується,
а частка минулої праці збільшується, але так, що загальна сума
вміщеної в товарі праці зменшується; отже, так, що жива праця
зменшується дужче, ніж збільшується минула. Минула праця,
втілена у вартості товару — стала частина капіталу — складається
почасти із зношування основного, почасти з обігового
сталого капіталу, який цілком входить у товар, — сировинного
й допоміжного матеріалу. Та частина вартості, що походить
з сировинного й допоміжного матеріалу, мусить з розвитком
продуктивності праці зменшуватись, бо ця продуктивність відносно
зазначених матеріалів виявляється саме в тому, що їх вартість
знижується. Навпаки, найхарактернішим для зростаючої
продуктивної сили праці є саме те, що основна частина сталого
капіталу дуже значно збільшується, а разом з тим так само
збільшується і та частина його вартості, яка в наслідок зношування
переноситься на товари. Для того, щоб новий метод
виробництва виявився як справжнє підвищення продуктивності,
він мусить переносити на окремий товар меншу додаткову частину
вартості, відповідну зношуванню основного капіталу,
ніж та частина вартості, яка віднімається, заощаджується в наслідок
зменшення живої праці, — одним словом він мусить зменшувати
вартість товару. Само собою зрозуміло, що він мусить
зменшувати її навіть і тоді, коли — як це буває в окремих випадках
— в утворення вартості товару входить, крім додатково зношуваної
частини основного капіталу, додаткова частина вартості
відповідно до більшої кількості або до дорожчих сировинних і
допоміжних матеріалів. Всі надбавки до вартості мусять бути
більше ніж урівноважені зменшенням вартості, яке випливає із
зменшення живої праці.

Тому це зменшення загальної кількості праці, яка входить
у товар, здавалося б, мало бути істотною ознакою підвищеної
продуктивної сили праці, незалежно від того, при яких суспільних
умовах відбувається виробництво. В суспільстві, в якому
виробники регулюють своє виробництво за складеним заздалегідь
планом, і навіть при простому товарному виробництві продуктивність
праці безумовно вимірювалась би цим масштабом.
Але як стоїть справа при капіталістичному виробництві?

Припустім, що певна капіталістична галузь виробництва
виробляє нормальну штуку свого товару при таких умовах: зношування
основного капіталу становить на штуку \sfrac{1}{2} шилінга або
марки; сировинного й допоміжного матеріалу входить у кожну
штуку на 17\sfrac{1}{2} шилінгів; заробітної плати 2 шилінги, і при нормі
додаткової вартості в 100\% додаткова вартість становить 2 шилінги.
Вся вартість = 22 шилінгам або маркам. Для спрощення
ми припускаємо, що в цій галузі виробництва капітал має пересічний
склад суспільного капіталу, що, отже, ціна виробництва
товару збігається з його вартістю, а зиск капіталіста збігається
з виробленою додатковою вартістю. В такому разі витрати
\parbreak{}  %% абзац продовжується на наступній сторінці

\parcont{}  %% абзац починається на попередній сторінці
\index{iii1}{0258}  %% посилання на сторінку оригінального видання
виробництва товару $= \sfrac{1}{2} \dplus{} 17\sfrac{1}{2} \dplus{} 2 \deq{} 20\text{ шилінгам}$, пересічна
норма зиску $\frac{2}{20} \deq{} 10\%$, а ціна виробництва штуки товару дорівнює
його вартості \deq{} 22\shil{ шилінгам} або маркам.

Припустім, що винайдено машину, яка наполовину скорочує
потрібну на кожну штуку товару живу працю, але зате
збільшує втроє ту частину вартості, яка складається з зношування
основного капіталу. Тоді справа стоятиме так: зношування
\deq{} 1\sfrac{1}{2}\shil{ шилінгам}, сировинний та допоміжний матеріал, як
і раніше, 17\sfrac{1}{2}\shil{ шилінгів}, заробітна плата 1\shil{ шилінг}, додаткова
вартість 1\shil{ шилінг}, разом 21\shil{ шилінг} або 21 марка. Вартість товару
зменшилась тепер на 1\shil{ шилінг}; нова машина безперечно підвищила
продуктивну силу праці. Але для капіталіста справа стоятиме
так: його витрати виробництва є тепер: 1\sfrac{1}{2}\shil{ шилінги} зношування,
17\sfrac{1}{2}\shil{ шилінгів} — сировинний і допоміжний матеріал,
1\shil{ шилінг} — заробітна плата, разом 20\shil{ шилінгів}, як і раніш.
Через те що норма зиску безпосередньо в наслідок застосування
нової машини не змінюється, він мусить одержати 10\%
понад витрати виробництва, що становить 2\shil{ шилінги}; отже, ціна
виробництва лишилась незмінною \deq{} 22\shil{ шилінгам}, але вона на 1\shil{ шилінг}
вища вартості. Для суспільства, яке виробляє при капіталістичних
умовах, товар \emph{не} став дешевшим, нова машина не являє
собою \emph{ніякого} поліпшення. Отже, капіталіст не має ніякого
інтересу в тому, щоб вводити нову машину. А через те що
введенням нової машини він просто зробив би нічого невартою
свою стару, ще не зношену машину, перетворив би її просто
в старе залізо, отже, зазнав би позитивного збитку, то він дуже
стережеться такої утопічної для нього дурості.

Отже, для капіталу закон зростаючої продуктивної сили праці
має не безумовне значення. Для капіталу ця продуктивна сила
підвищується не тоді, коли взагалі заощаджується жива праця,
а тільки тоді, коли на \emph{оплачуваній} частині живої праці заощаджується
більше, ніж додається минулої праці, як це вже коротко
зазначено було в книзі І, розділ XIII, 2, стор. 411\footnote*{
Стор. 297--298 рос. вид. 1935~\abbr{р.} Ред. укр. перекладу.
}. Тут
капіталістичний спосіб виробництва впадає в нову суперечність.
Його історичне покликання — нестримний розвиток продуктивності
людської праці, підготований вперед у геометричній прогресії.
Він зраджує це покликання, оскільки він, як у даному
випадку, перешкоджає розвиткові продуктивності. Цим він тільки
знову доводить, що він хиріє від старості і все більше й більше
переживає себе.]\footnote{
Вищенаведене стоїть у дужках, тому що хоч це і є переробка з примітки
оригіналу рукопису, але у викладі деяких моментів воно виходить за
межі того матеріалу, що є в оригіналі. — \emph{Ф.~Е.}
}

\pfbreak{}

В конкуренції збільшення мінімуму капіталу, який з підвищенням
продуктивної сили стає потрібним для успішного ведення
\index{iii1}{0259}  %% посилання на сторінку оригінального видання
самостійного промислового підприємства, виявляється
так: як тільки нове дорожче промислове устаткування стає
загальнопоширеним, дрібніші капітали на майбутнє виключаються
з цього виробництва. Тільки на перших порах механічних винаходів
у різних сферах виробництва дрібніші капітали можуть
в них самостійно функціонувати. З другого боку, дуже великі
підприємства, з надзвичайно високим відношенням сталого капіталу,
як залізниці, дають не пересічну норму зиску, а тільки
частину її, процент. Інакше загальна норма зиску знизилась би
ще більше. Навпаки, і тут великі капітали, зібрані в формі акцій,
знаходять собі поле для безпосереднього застосування.

Зростання капіталу, отже, нагромадження капіталу, включає
зменшення норми зиску лиш остільки, оскільки разом з цим зростанням
настають розглянуті нами вище зміни у відношенні органічних
складових частин капіталу. Однак, не зважаючи на постійні,
повсякденні перевороти в способі виробництва, та чи інша,
більша чи менша частина всього капіталу протягом певного
часу продовжує нагромаджуватися на базі даного пересічного відношення
цих складових частин, так що з зростанням цієї частини
не сполучена ніяка органічна переміна, отже й ніякі причини
падіння норми зиску. Це постійне збільшення капіталу, а тому
й розширення виробництва на основі старих методів виробництва,
яке спокійно триває далі, тимчасом як поряд з ними
вводяться вже нові методи, знов таки є причиною того, що
норма зиску зменшується не в тій мірі, в якій зростає сукупний
капітал суспільства.

Збільшення абсолютного числа робітників, не зважаючи на
відносне зменшення змінного капіталу, витрачуваного на заробітну
плату, відбувається не в усіх галузях виробництва і не
в усіх рівномірно. В землеробстві зменшення елементу живої
праці може бути абсолютним.

Зрештою, абсолютне збільшення числа найманих робітників,
не зважаючи на його відносне зменшення, є тільки потреба капіталістичного
способу виробництва. Для нього робочі сили
стають уже зайвими, як тільки немає вже необхідності примушувати
їх працювати 12--15 годин на день. Розвиток продуктивних
сил, який зменшував би абсолютне число робітників, тобто
в дійсності давав би змогу всій нації виконувати своє сукупне
виробництво за коротший час, викликав би революцію, бо він
вивів би в тираж більшість населення. В цьому знову виявляється
специфічна межа капіталістичного виробництва, а також
те, що воно ніяк не є абсолютною формою для розвитку продуктивних
сил і створення багатства, що воно, навпаки, на певному
пункті вступає в колізію з цим розвитком. Частково така колізія
виявляється в періодичних кризах, які походять з того, що
то одна, то друга частина робітничого населення робиться зайвою
в своїй старій професії. Межа капіталістичного виробництва
— надлишковий час робітників. Абсолютний надлишковий
\parbreak{}  %% абзац продовжується на наступній сторінці

\input{_0260c.tex}
\parcont{}  %% абзац починається на попередній сторінці
\index{iii1}{0261}  %% посилання на сторінку оригінального видання
(Звичайно, здешевлення сталого капіталу в усіх цих галузях
може підвищити норму зиску при незмінній експлуатації робітника.)
Як тільки новий метод виробництва починає поширюватись,
і цим фактично дається доказ того, що ці товари можуть
вироблятись дешевше, то капіталісти, які працюють при старих
умовах виробництва, мусять продавати свій продукт нижче
його повної ціни виробництва, бо вартість цього товару впала,
робочий час, потрібний їм для виробництва цього товару, стоїть
вище суспільного. Одним словом, — і це виявляється як наслідок
конкуренції, — вони так само мусять запровадити новий метод
виробництва, при якому відношення змінного капіталу до
сталого є менше.

Всі обставини, які спричиняють те, що застосування машин
здешевлює ціну товарів, вироблюваних цими машинами,
завжди зводяться до зменшення тієї кількості праці, яку вбирає
одиниця товару; а подруге, вони зводяться до зменшення
зношуваної частини машин, вартість якої входить в одиницю
товару. Чим повільнішим є зношування машин, тим на більшу
кількість товарів воно розподіляється, тим більше живої праці
заміщають вони до строку їх репродукції. В обох випадках
збільшується кількість і вартість основного сталого капіталу
порівняно із змінним.

„All other things being equal, the power of a nation to save from
its profits varies with the rate of profits, is great when they are high,
less, when low; but as the rate of profit declines, all other things do
not remain equal\dots{} A low rate of profit is ordinarily acompanied by
a rapid rate of accumulation, relatively to the numbers of the people,
as in England\dots{} a high rate of profit by a slower rate of accumulation,
relatively to the numbers of the people“. [„При всіх інших
однакових умовах спроможність нації робити заощадження з своїх
зисків змінюється із зміною норми зиску; вона більша, коли
норма зиску — висока, менша, коли вона — низька; але якщо
норма зиску знижується, всі інші умови не лишаються однаковими\dots{}
Низька норма зиску звичайно супроводиться швидким
темпом нагромадження порівняно з чисельністю населення,
як в Англії\dots{} висока норма зиску — повільнішим темпом нагромадження
порівняно з чисельністю населення“.] Приклади: Польща,
Росія, Індія і т. д. (\emph{Richard Jones}: „An Introductory Lecture on Political
Economy“, London 1833, стор. 50 і далі). Джонс правильно відзначає,
що, не зважаючи на падаючу норму зиску, inducements
and faculties to accumulate [спонуки до нагромадження і можливості
нагромаджувати] збільшуються. Поперше, в наслідок зростаючого
відносного перенаселення. Подруге, тому, що з зростанням
продуктивності праці збільшується маса споживних вартостей,
представлених тією самою міновою вартістю, отже, збільшується
маса речових елементів капіталу. Потрете, тому що
збільшується різноманітність галузей виробництва. Почетверте,
в наслідок розвитку кредитної системи, акційних товариств і т. д.
\parbreak{}  %% абзац продовжується на наступній сторінці

\input{_0262c.tex}

  \index{i}{0249}  %% посилання на сторінку оригінального видання
Відділ четвертий

Продукція відносної додаткової вартости

Розділ десятий

Поняття відносної додаткової вартости

Ту частину робочого дня, яка продукує лише еквівалент
оплаченої капіталістом вартости робочої сили, ми досі розглядали
як величину сталу, чим вона в дійсності і є за даних умов продукції,
на даному ступені економічного розвитку суспільства. Понад
цей доконечний для нього робочий час робітник міг працювати
ще 2, 3, 4, 6 і т. д. годин. Від величини цього здовження залежали
норма додаткової вартости й величина робочого дня. Якщо
доконечний робочий час був сталою величиною, то цілий робочий
день, навпаки, був величиною змінною. Припустімо тепер такий
робочий день, що його величина й поділ на доконечну й додаткову
працю є дані. Нехай лінія ас, а—————b——с, репрезентує,
приміром, дванадцятигодинний робочий день, відтинок
аb — 10 годин доконечної праці, відтинок bс — 2 години додаткової
праці. Яким чином можна збільшити продукцію додаткової
вартости, тобто здовжити додаткову працю, без усякого дальшого
здовження ас, або незалежно від усякого дальшого здовження ас?

Не зважаючи на те, що межі робочого дня ас дано, bс, здається,
можна здовжити, якщо не через здовження його поза кінцевий
пункт с, який разом з тим є кінцевий пункт робочого дня ас, то
через переміщення його початкового пункту b у протилежному напрямі,
в бік а. Припустімо, що в лінії а————b' — b—с
відтинок b'b дорівнює половині bс, тобто дорівнює одній робочій
годині. Коли тепер за дванадцятигодинного робочого дня
ас пункт b посунуто до b', то bс поширюється до b'с, додаткова
праця зростає наполовину, з 2 до 3 годин, хоч робочий день,
як і раніш, має лише 12 годин. Але це поширення додаткової
праці bс до b'с, від 2 до 3 годин, очевидно, неможливе без одночасного
скорочення доконечної праці, аb до аb', з 10 до 9 годин.
Здовженню додаткової праці відповідало б скорочення доконечної
праці, або частина того робочого часу, що його досі
робітник фактично зуживав для себе самого, перетворюється на
робочий час для капіталіста. Що змінилося б тут, так це не довжина
робочого дня, а його поділ на доконечну працю й додаткову
працю.

\index{iii1}{0250}  %% посилання на сторінку оригінального видання
Частина старого капіталу мусила б лишитись без діла при
всяких обставинах, лишитись без діла щодо своєї властивості як
капіталу — функціонувати і зростати в своїй вартості. Яка саме
частина лишилася б без діла, це вирішила б конкурентна боротьба.
Поки все йде добре, конкуренція, як це виявилось при
вирівненні загальної норми зиску, діє, як практичний братерський
союз класу капіталістів, так що вони спільно ділять між собою загальну
здобич пропорціонально до величини частки, вкладеної
кожним з них. Але як тільки справа йде вже не про розподіл
зиску, а про розподіл збитку, то кожний з них намагається
якомога зменшити свою участь в ньому і перекласти його на
шию іншому. Для всього класу збиток є неминучий. Але скільки
з нього припаде на кожного окремого капіталіста, наскільки
взагалі кожний з них повинен взяти участь в ньому, це стає
тоді питанням сили й хитрості, і конкуренція перетворюється
тоді в боротьбу ворогуючих братів. Протилежність між інтересами
кожного окремого капіталіста і інтересами класу капіталістів виявляється
при цьому цілком так само, як перед тим за допомогою
конкуренції проявлялась на практиці тотожність цих інтересів.

Яким же чином міг би бути знов усунений цей конфлікт
і як могли б відновитись відносини, відповідні „здоровому“
рухові капіталістичного виробництва? Спосіб усунення міститься
вже в простому констатуванні конфлікту, про усунення якого
йде мова. Він полягає в залишенні без діла і навіть частковому
знищенні капіталу, рівного своєю вартістю всьому додатковому
капіталові $ΔК$ або принаймні частині його. Хоча — як
це вже випливає з викладу конфлікту — розподіл цього збитку
ні в якому разі не поширюється рівномірно на поодинокі окремі
капітали, а вирішується в конкурентній боротьбі, в якій збиток
розподіляється дуже нерівно і в дуже різних формах, залежно
від особливих переваг або особливих завойованих уже позицій,
так що один капітал лишається лежати без діла, другий знищується,
третій має тільки відносний збиток або зазнає тільки
тимчасового знецінення і т. д.

Але при всяких обставинах рівновага відновилась би в наслідок
бездіяльності і навіть знищення капіталу в більшому чи
меншому розмірі. Це почасти поширилося б на матеріальну
субстанцію капіталу; тобто частина засобів виробництва, основний
і обіговий капітал, не функціонувала б, не діяла б як капітал;
частина підприємств, що вже почали функціонувати, припинила
б роботу. Хоча в цьому відношенні час робить своє
і погіршує всі засоби виробництва (за винятком землі), але тут
в наслідок припинення функціонування мало б місце значно
сильніше справжнє руйнування засобів виробництва. Головний
результат у цьому відношенні був би, однак, у тому, що ці засоби
виробництва перестали б діяти як засоби виробництва, — в зруйнуванні
їх функції як засобів виробництва на коротший чи довший
час.


\index{iii1}{0251}  %% посилання на сторінку оригінального видання
Найбільш руйнівного впливу, і при тому найгострішого
характеру, зазнав би капітал, оскільки він має властивість
вартості, зазнали б капітальні \emph{вартості}. Частина капітальної
вартості, яка перебуває просто у формі посвідок на майбутню
участь в додатковій вартості, в зиску, і яка в дійсності становить
тільки боргові зобов’язання в різних формах на виробництво,
відразу знецінюється з падінням доходів, на які вона розрахована.
Частина готівки золота й срібла лежить без діла, не функціонує
як капітал. Частина товарів, що перебувають на ринку, може
здійснити свій процес циркуляції і репродукції тільки шляхом
надзвичайного зниження своїх цін, отже, шляхом знецінення
того капіталу, який вона представляє. Цілком так само більше
чи менше знецінюються елементи основного капіталу. До цього
долучається ще й те, що певні припущені відношення цін
обумовлюють процес репродукції, і тому цей останній в наслідок
загального падіння цін приходить до застою і розладу. Цей
розлад і застій паралізує функцію грошей як платіжного засобу,
яка розвивається одночасно з розвитком капіталу і грунтується
на згаданих припущених відношеннях цін; він розриває у сотнях
місць ланцюг платіжних зобов’язань на певні строки і ще більше
загострюється в наслідок зумовленого цим краху (Zusammenbrechen)
кредитної системи, що розвинулась одночасно з капіталом,
і таким чином веде до сильних і гострих криз, до раптових
насильних знецінень і дійсного застою й розладу
% REMOVED \footnote*{
% В першому німецькому виданні тут стоїть: „застою й занепаду (Sturz)“;
% виправлено на підставі рукопису Маркса. \Red{Примітка ред. нім. вид, ІМЕЛ.}
% }
процесу
репродукції, і тим самим до дійсного зменшення репродукції.

Але одночасно діяли б і інші фактори. Застій виробництва
позбавив би роботи частину робітничого класу і цим поставив би
заняту частину його в такі відносини, при яких вона мусила б
згоджуватись на зниження заробітної плати навіть нижче пересічного
рівня; обставина, яка дає для капіталу такий самий результат,
як коли б при пересічній заробітній платі була підвищена
відносна чи абсолютна додаткова вартість. Період процвітання
сприяв би шлюбам серед робітників і зменшив би смертність їх
дітей, обставини, які — хоч би й яке вони означали дійсне збільшення
населення — не означають збільшення дійсно працюючого
населення, але на відношення робітників до капіталу впливають
цілком так само, як коли б збільшилося число дійсно функціонуючих
робітників. З другого боку, падіння цін і конкурентна
боротьба спонукали б кожного капіталіста підвищувати індивідуальну
вартість свого сукупного продукту понад його загальну
вартість за допомогою застосування нових машин, нових поліпшених
методів праці, нових комбінацій, тобто підвищувати продуктивну
силу даної кількості праці, знижувати відношення
змінного капіталу до сталого, і тим самим звільняти робітників,
\parbreak{}  %% абзац продовжується на наступній сторінці

\parcont{}  %% абзац починається на попередній сторінці
\index{iii2}{0252}  %% посилання на сторінку оригінального видання
які витрачається дохід, тобто правити за засоби споживання, перебігають протягом
року різні ступені, наприклад, вовняна пряжа, сукно. На одному ступені
вони становлять частину сталого капіталу, на другому — їх особисто споживається,
отже, цілком, входять в склад доходу. Можна, отже, уявити собі разом
з А. Смітом, що сталий капітал є лише позірний елемент товарової вартости,
який в загальному зв’язку зникає. Таким самим чином відбувається далі обмін
змінного капіталу на дохід. Робітник купує на свою заробітну плату частину
товарів, що становить його дохід. Одночасно він покриває цим самим для капіталіста
грошову форму змінного капіталу. Нарешті: частина продуктів, що становлять
сталий капітал, покривається або in natura, або за посередництвом обміну
між самими продуцентами сталого капіталу; процес, до якого споживачі
не мають жодного чинення. Коли спустити це з уваги, то може постати зовнішня
видимість, що дохід споживачів покриває ввесь продукт, отже і сталу частину
вартости.

5) Крім плутанини, яку вносить перетворення вартостей на ціни продукції,
виникає ще й інша в наслідок перетворення додаткової вартости на різні окремі
форми доходу, самостійні одна проти однієї і залічені до різних елементів продукції,
на зиск і ренту. При цьому забувається, що вартості товарів є основою, і що
розпадання цієї товарової вартости на окремі складові частини, і дальший розвиток
цих складових частин вартости у форми доходу, їх перетворення на відносини
різних посідачів різних чинників продукції до цих окремих складових
частин вартости, їх розподіл між цими посідачами згідно з певними категоріями
і титулами, нічого не змінює у самому визначенні вартости й законів її.
Так само мало змінюється закон вартости тією обставиною, що вирівнювання
зиску, тобто розподіл сукупної додаткової вартости між різними капіталами, і
перешкоди, що почасти (в абсолютній ренті) ставляться землеволодінням на
шляху цього вирівнювання, призводять до відхилу регуляційних пересічних цін
товарів від їхніх індивідуальних вартостей. Це впливає знов таки тільки на
добавку додаткової вартости до цін різних товарів, але не знищує самої додаткової
вартости і сукупної вартости товарів як джерела цих різних складових
частин ціни.

Тут перед нами quid pro quo, яке ми розглядаємо в дальшім розділі, і
яке неминуче зв’язане з ілюзією, що нібито вартість виникає з її власних
складових частин. А саме: спершу різні складові частини вартости товару набувають
в доходах самостійних форм, і як такі доходи їх залічують до окремих
речових елементів продукції, як до їхніх джерел, замість залічити їх до
вартости товару як до їхнього джерела. Вони дійсно залічуються до зазначених
окремих джерел, але не як складові частини вартости, а як доходи, як складові
частини вартости, що дістаються цим певним категоріям агентів продукції: робітникові,
капіталістові, земельному власникові. А проте, можна уявити собі, що
ці складові частини вартости замість виникати від розкладу товарової вартости,
навпаки, лише створюють її своїм сполученням; тоді саме й виникає чудове
порочне коло: вартість товарів виникає з суми вартости заробітної плати, зиску,
ренти, а вартість заробітної плати, зиску, ренти в свою чергу визначається
вартістю товарів і т. ін.\footnote{
«Оборотний капітал, витрачений на матеріяли, сировий матеріял і викінчені вироби, сам
складається з товарів, що їх потрібна ціна складена з тих самих елементів; так що, розглядаючи
сукупність товарів у певній країні, було б зайво зачислювати цю частину оборотного капіталу до
елементів потрібної ціни». (Storch, Cours d’Ec. Pol., II, p. 140). Під цими елементами оборотного
капіталу Шторх розуміє (основний — це тільки змінена форма оборотного) сталу частину вартости.
«Правда, що заробітна плата робітника так само як і частина зиску підприємця, яка складається з
заробітних плат, коли розглядати їх як частину засобів існування, і собі складається з товарів, що
куплені по ринковій ціні, і теж мають в собі заробітні плати, ренти на капітали, земельні ренти і
підприємницькі зиски\dots{} спостереження це доводить лише неможливість розкласти потрібну ціну на її
простіші елементи» (ib., примітка). — У своїх Considérations sur la nature du revenu national (Paris
1824)
Шторх y своїй полеміці s Сеєм, правда, розуміє все безглуздя, що до нього призводить помилкова
аналіза товарової вартости, що розкладає її без остачі тільки на доходи, і правильно висловлюється
про все безглуздя цих висновків — з погляду не поодиноких капіталістів, а нації, — але сам він не
робить і кроку вперед в аналізі prix nécessaire (потрібної ціни), відносно якої він, замість
відсувати розв’язання
завдання до безконечности, заявляє в своєму «Cours» що її неможливо розкласти на її дійсні елементи.
«Ясно, що вартість річного продукту поділяється почасти на капітали, почасти на зиски, і що кожна з
цих частин вартости річного продукту регулярно купує продукти, потрібні нації так для збереження її
капіталу, як і для відновлення її споживного фонду (р. 134--135)\dots{} Чи зможе вона (селянська родина,
що працює самостійно) жити у своїх клунях і стайнях, живитись тільки насінням
і травою, одягатися з своєї робочої худоби, витрачати свої хліборобські зваряддя? Згідно з
твердженими п. Сея, слід було б відповісти позитивно на всі ці питання (135--136)\dots{} Коли визнати, що
дохід нації дорівнює її гуртовому продуктові, тобто, що з нього не доводиться вираховувати капітали,
то доведеться також визнати, що вона може непродуктивно витратити всю вартість свого річного
продукту,
не роблячи найменшої шкоди своєму майбутньому доходові. (147) Продукти, що складають капітал нації,
не підлягають споживанню», (р. 150)
}.

\index{iii2}{0253}  %% посилання на сторінку оригінального видання
При нормальному стані продукції тільки частина новодолученої праці
вживається на продукцію і тому на покриття сталого капіталу: саме якраз та
частина, що покриває сталий капітал, витрачений у продукції засобів споживання,
речових елементів доходу. Це вирівнюється тим, що ця стала частина,
кляси II не коштує новодолучуваної праці. Але сталий капітал, який (коли розглядати
сукупний процес репродукції, що в ньому, отже, включено і те вирівнювання
кляс І і II), не є продукт новодолученої праці, хоч цей продукт неможливо
було б випродукувати без нього, — цей сталий капітал, розглядуваний з речового
боку, підлягає підчас процесу репродукції, випадковостям і небезпекам,
які можуть його зменшити. (Але далі, коли розглядати його щодо вартости, то
він також може знецінитися в наслідок зміни у продуктивній силі праці;
проте, це стосується лише до поодиноких капіталістів). Відповідно до цього частина
зиску, отже, додаткової вартости, а тому і додаткового продукту, що в ньому
(коли розглядати його з погляду вартости) репрезентується лише новодолучена
праця, служить страховим фондом. При цьому суть справи ані трохи не змінюється
від того, чи порядкує цим страховим фондом страхове товариство як
окреме підприємство, чи ні. Це є однісінька частина доходу, що не споживається
як такий, і не служить неодмінно фондом акумуляції. Чи служить вона фактично
фондом акумуляції, чи лише покриває прогріхи репродукції, це залежить від
випадку. Це також однісінька частина додаткової вартости і додаткового продукту,
отже, додаткової праці, що крім частини, яка служить для акумуляції, отже,
для поширення процесу репродукції, мусить існувати і далі по знищенні
капіталістичного способу продукції. Звичайно, це має своєю передумовою, що
частина, регулярно споживувана безпосередніми продуцентами, не лишиться обмеженою на своєму
теперішньому мінімальному рівні. За винятком додаткової
праці на тих, хто через свій вік ще не може або вже не може брати участи у продукції,
відпаде всяка праця на утримання тих, хто не працює. Коли ми уявимо
собі суспільство при його виниканні, то побачимо, що тут немає ще випродукованих
засобів продукції, отже, немає сталого капіталу, що його вартість увіходить
у продукт, і при репродукції в тому самому маштабі мусить покриватися
in natura з продукту в розмірі, визначуваному його вартістю. Але природа
безпосередньо дає тут засоби існування, їх не доводиться продукувати. Тому
вона залишає також дикунові, що йому доводиться задовольняти лише малі потреби,
час, — не на те, щоб використати ще не сущі в наявності засоби продукції для
нової продукції, а на те, щоб, крім праці, якої коштує привласнення наявних
у природі засобів існування, витрачати працю на перетворення інших продуктів
природи на засоби продукції, лук, кам’яний ніж, човен і т. ін. Процес цей,
\parbreak{}  %% абзац продовжується на наступній сторінці

\parcont{}  %% абзац починається на попередній сторінці
\index{ii}{0254}  %% посилання на сторінку оригінального видання
циркуляцію, а вилучає з циркуляції 6000 ф. стерл.: 5000 ф. стерл. капіталу й
1000 ф. стерл. додаткової вартости. Ці 1000 ф. стерл. перетворились на
гроші за допомогою тих грошей, що їх вона сама пустила в циркуляцію
не як капіталіст, а як споживач, що їх вона не авансувала, а витратила.
Тепер вони повертаються до неї знову як грошова форма спродукованої
нею додаткової вартости. І з цього часу ця операція повторюється
щороку. Але, починаючи з другого року, 1000 ф. стерл., витрачувані нею,
завжди є вже перетворена форма, грошова форма спродукованої нею
додаткової вартости. Вона витрачає їх щороку і щороку ж вони повертаються
до неї назад.

Коли б капітал цього капіталіста обертався протягом року частіше,
то справа від цього ані трохи не змінилась би, але, звичайно, змінився
б протяг часу, а тому й величина тієї грошової суми, що її капіталістові
понад авансований ним грошовий капітал довелось би пускати в
циркуляцію на своє особисте споживання.

Ці гроші капіталіст пускає в циркуляцію не як капітал. Але, зрозуміло,
така вже властивість капіталіста, що доки до нього повернеться з
циркуляції додаткова вартість, він може існувати на ті засоби, які є в
його посіданні.

В цьому випадку ми припускали, що грошова сума, яку капіталіст
пускає в циркуляцію на покриття свого особистого споживання до першого
зворотного припливу його капіталу, точно дорівнює спродукованій ним
додатковій вартості, і тому має бути перетворена на гроші. Очевидно,
що таке припущення відносно поодинокого капіталіста довільне. Але
воно мусить бути правильне для цілої кляси капіталістів, якщо ми
припускаємо просту репродукцію. Воно виражає лише те, що є в цьому
припущенні, а саме, що всю додаткову вартість, але і тільки її, споживається
непродуктивно; що, отже, жодної частини первісного капіталу
не споживається непродуктивно.

Ми вище припускали, що цілої продукції благородних металів
(припустімо = 500 ф. стерл.) досить лише для того, щоб покрити зношування
грошей.

Капіталісти, що продукують золото, мають увесь свій продукт у
формі золота, так ті частини його, що покривають сталий і змінний
капітал, як і ту частину його, яка складається з додаткової вартости.
Отже, частина суспільної додаткової вартости складається з золота, а
не з такого продукту, що перетворюється на золото лише в циркуляції.
Вона з самого початку складається з золота, й пускається її в циркуляцію
для того, щоб вилучити з циркуляції продукти. Це саме тут стосується
до заробітної плати, змінного капіталу і до покриття авансованого
сталого капіталу. Отже, коли одна частина кляси капіталістів пускає в
циркуляцію товарову вартість більшу (на величину додаткової вартости),
ніж авансований ними грошовий капітал, то друга частина капіталістів
пускає в циркуляцію більшу грошову вартість (більшу на додаткову
вартість), ніж товарова вартість, що її вони постійно вилучають з циркуляції
для продукції золота. Якщо частина капіталістів постійно випомповує
\index{ii}{0255}  %% посилання на сторінку оригінального видання
з циркуляції більше грошей, ніж подає в неї, то частина капіталістів,
— та, що продукує золото — постійно напомповує в неї більше
грошей, ніж вилучає з неї в засобах продукції.

Хоч частина цього продукту, золота, в 500 ф. стерл., є додаткова
вартість продуцентів золота, однак цілу цю суму призначається лише на
заміщення грошей, потрібних для циркуляції товарів; при цьому байдуже,
скільки з цієї суми йде на перетворення на гроші додаткової вартости
товарів і скільки на перетворення на гроші інших складових частин
вартости товарів.

Якщо продукцію золота перенести з даної країни в інші країни, то
це нічого не змінює в справі. Частину суспільної робочої сили й суспільних
засобів продукції в країні \emph{А} перетворено на продукт, прим., на
полотно, вартістю в 500 ф. стерл., що його вивозиться в країну \emph{В}, щоб
там купити золото. Продуктивний капітал, застосований таким чином у
країні \emph{А}, так само не подає на ринок країни \emph{А} товарів — на відміну
від грошей — як коли б його безпосередньо застосовувалось на продукцію
золота. Цей продукт \emph{А} репрезентовано в 500 ф. золота, і він
надходить в циркуляцію країни \emph{А} лише як гроші. Частина суспільної
додаткової вартости, що є в цьому продукті, існує безпосередньо як
гроші, і для країни А ніколи не існує інакше, як у формі грошей. Хоч
для капіталістів, що продукують золото, лише частина продукту репрезентує
додаткову вартість, а друга частина — покриття капіталу, однак питання
про те, яка кількість цього золота покриває, крім обігового сталого
капіталу, змінний капітал, і яка кількість репрезентує додаткову
вартість залежить виключно від тих відповідних відношень,
що в них заробітна плата й додаткова вартість перебувають до
вартости товарів, що циркулюють. Частина, що становить додаткову
вартість, розподіляється між різними членами кляси капіталістів. Хоч
вони постійно витрачають її на особисте споживання й знов одержують
її через продаж нового продукту, — саме ця купівля й продаж взагалі
лише і зумовлює циркуляцію між ними грошей, потрібних для перетворення
на гроші додаткової вартости, — однак деяка частина суспільної
додаткової вартости, хоч і в змінних кількостях, перебуває в формі
грошей в кишені капіталістів, цілком так само, як частина заробітної
плати, принаймні протягом кількох днів тижня, затримується в формі грошей
в кишенях робітників. І ця частина не обмежена тією частиною грошового
продукту, що первісно становила додаткову вартість капіталістів, які
продукують золото; як сказано, вона обмежена тією пропорцією, що в
ній вищеназваний продукт в 500 ф. стерл. взагалі розподіляється між
капіталістами і робітниками, і що в ній запас товарів, призначених для
циркуляції, складається з додаткової вартости та з інших складових частин
вартости.

А проте, частина додаткової вартости, яка існує не в інших товарах,
а в грошах поряд цих інших товарів, лише остільки складається з частини
щорічно продукованого золота, оскільки частина річної продукції золота
йде в циркуляцію для реалізації додаткової вартости. Друга частина
\parbreak{}  %% абзац продовжується на наступній сторінці

\parcont{}  %% абзац починається на попередній сторінці
\index{i}{0256}  %% посилання на сторінку оригінального видання
припускаючи, що вартість грошей не змінюється, — продукує
завжди ту саму нову вартість у 6 шилінґів, хоч би як ця сума
вартости поділялась між еквівалентом вартости робочої сили й
додатковою вартістю. Але якщо внаслідок підвищення продуктивної сили праці вартість денних засобів
існування, а тому й
денна вартість робочої сили падає з 5 шилінґів до 3, то додаткова
вартість зростає з 1 шилінґа до 3 шилінґів. Щоб репродукувати
вартість робочої сили раніше було потрібно 10, а тепер треба
лише 6 робочих годин. Чотири робочі години стали вільні, і їх
можна прилучити до сфери додаткової праці. Звідси іманентне
прагнення й постійна тенденція капіталу підвищувати продуктивну
силу праці, щоб здешевити товари та через здешевлення товарів
здешевити самого робітника.5

Абсолютна вартість товару для капіталіста, що його продукує, сама по собі байдужа. Капіталіста
цікавить лише додаткова
вартість, що міститься в товарі й що її можна реалізувати в продажі. Реалізація додаткової вартости
включає й повернення авансованої вартости. А що відносна додаткова вартість зростає просто
пропорційно до розвитку продуктивної сили праці, тимчасом як
вартість падає зворотно пропорційно до того самого розвитку,
отже, що той самий ідентичний процес здешевлює товари та збільшує додаткову вартість, яка міститься
в них, то й розв’язується
та загадка, що капіталіст, який дбає лише про продукцію мінової
вартости, постійно намагається знизити мінову вартість товарів, —
суперечність, якою один з основників політичної економії, а
саме Кене, мучив своїх супротивників, що так і не дали йому на
неї відповіді. «Ви визнаєте, — каже Кене, — що чим більше
можна без шкоди для продукції заощадити витрат та зменшити
дорогі роботи при фабрикації промислових продуктів, тим корис-

5 «У тій самій пропорції, в якій меншають видатки робітника, буде
зменшено і його заробітну плату, якщо тільки разом з цим промисловість
звільняють від обмежень» («In whatever proportion the expenses of a
labourer are diminished, in the same proportion will his wages be diminished,
if the restraints upon industry are at the same time taken off»).
(«Considerations concerningt taking off the Bounty on Corn exported etc.»,
London 1752, p. 7). «Інтереси промисловости вимагають, щоб хліб і
взагалі всякі харчові речі були якомога дешевші: бо те, що їх робить
дорожчими, робить дорожчою і працю... по всіх країнах, де промисловість вільна від обмежень, ціна на
предмети харчування мусить впливати
на ціну праці. Цю останню завжди понижують, коли дешевшають потрібні
засоби існування». («The interest of trade requires, that corn and all provisions should be as cheap
as possible; for whatver makes them dear, must
make labour dear also... in all countries, where industry is not restrained,
the price of provisions must affect the Price of Labour. This will always
be diminished when necessaries of life grow cheaper»). (Там же, стор. 3).
«Заробітну плату понижують в тій самій пропорції, в якій зростають
продуктивні сили. Правда, машини здешевлюють засоби існування, але
вони також і робітників роблять дешевшими». («Wages are decreased
in the same proportion as the powers of production increase. Machinery,
it is true, cheapens the necessaries of life, but it also cheapens the labourer
too»). («А Prize Essay on the comparative merits of Competition and
Cooperation», London 1834, p. 27).
\index{i}{0257}  %% посилання на сторінку оригінального видання
ніше це заощадження, бо воно зменшує ціну цих виробів. А, проте,
ви думаєте, що продукція багатства, яке походить з праці промисловців, полягає у збільшенні мінової
вартости їхніх виробів».\footnote{
«Ils conviennent que plus on peut, sans préjudice, épargner de frais ou
de travaux dispendieux dans la fabrication des ouvrages des artisans, plus
cette épargne est profitable par la diminution du prix de ces ouvrages.
Cependant ils croient que la production de richesse qui résulte des travaux
des artisans consiste dans l’augmentation de la valeur vénale de leurs ouvrages»).
(Quesnay: «Dialogues sur le Commerce et sur les Travaux des
Artisans», ed. Daire, Paris 1846, p. 188, 189).
}

Отже, заощадження на праці\footnote{
«Ці спекулянти, що так багато заощаджують на праці робітників,
яку вони мусили б оплатити» («Ces spéculateurs si économes du travail
des ouvriers qu’il faudrait qu’ils payassent»). (J. N. Bidaull: «Du
Monopole qui s’établit dans les arts industrielles et le commerce», Paris
1828, p. 13). «Підприємець завжди намагатиметься заощаджувати час і
працю» («The employer will be always on the stretch to economise
time and labour»). (Dugald Stewart: Works ed. by Sir W. Hamilton.
Edinburgh 1885, vol. Ill, «Lectures on Political Economy», p. 318). «їхній (капіталістів) інтерес
вимагає того, шоб продуктивні сили зуживаних
ними робітників були якомога найбільші. Тому вони звертають свою
увагу майже виключно на збільшення цієї сили». («Their (the capitalists’)
interest is that the productive powers of the labourers they employ should
be the greatest possible. On promoting that power their attention is fixed
and almost exclusively fixed»). (R. Jones: «Textbook of Lectures on
the Political Economy of Nations», Hertford 1852, Lecture III).
} внаслідок розвитку продуктивної сили праці за капіталістичної продукції
зовсім не має на меті
скорочення робочого дня. Воно має на меті лише скорочення робочого часу, доконечного для продукції
певної кількости товарів.
Те, що робітник за підвищеної продуктивної сили його праці
продукує за одну годину, приміром, вдесятеро більше товарів,
ніж раніш, отже, на кожну штуку товару потребує вдесятеро
менше робочого часу, аж ніяк не заважає тому, що його тепер,
як і раніш, примушують працювати 12 годин та продукувати
протягом 12 годин 1.200 штук товару замість 120. Навіть більше:
його робочий день разом з тим може здовжуватися, так що він
тепер продукуватиме 1.400 штук за 14 годин, і т. ін. Тому в економістів
такої породи, як от Мак Кулох, Юр, Сеніор і tutti
quanti,\footnote*{
— всі, скільки їх є. Ред.
} ми на одній сторінці читаємо, що робітник повинен дякувати капіталові за розвиток
продуктивних сил, бо цей останній
скорочує доконечний робочий час, а на другій сторінці — що
робітник мусить виявити цю вдячність, працюючи в майбутньому
15 годин замість 10. За капіталістичної продукції розвиток продуктивної сили праці має на меті
скоротити ту частину робочого
дня, протягом якої робітник мусить працювати для себе самого,
щоб саме цим здовжити другу частину робочого дня, протягом
якої він може працювати задурно на капіталіста. У якій мірі
можна досягти цього результату і не здешевлюючи товарів, це
виявиться при розгляді особливих метод продукції відносної
додаткової вартости, до чого ми тепер і переходимо.

\index{i}{0258}  %% посилання на сторінку оригінального видання
\section{Кооперація}
Капіталістична продукція, як ми бачили, починається фактично
лише там, де той самий індивідуальний капітал одночасно експлуатує
значне число робітників, отже, лише там, де процес праці
поширює свій обсяг і постачає продукти у великому маштабі.
Праця значного числа робітників у той самий час, у тому самому
приміщенні (або, як хто хоче, на тому самому полі праці), для продукції
того самого ґатунку товарів, під командою того самого
капіталіста, становить історично й логічно вихідний пункт капіталістичної
продукції. Щодо самого способу продукції мануфактура,
приміром, на початках свого розвитку ледве чи відрізняється
від цехової ремісничої промисловости чимось іншим, як
хібащо більшим числом робітників, яких одночасно експлуатує
той самий капітал. Майстерню цехового майстра тільки розширено.

Отже, спочатку ріжниця є лише кількісна. Ми бачили, що
маса додаткової вартости, яку продукує даний капітал, дорівнює
додатковій вартості, яку дає окремий робітник, помноженій на
число одночасно експлуатованих робітників. Це число само по
собі нічого не змінює в нормі додаткової вартости або в ступені
експлуатації робочої сили; щодо продукції товарової вартости
взагалі, то для неї всякі якісні зміни робочого процесу, очевидно,
не мають значення. Це випливає з природи вартости. Якщо один
дванадцятигодинний робочий день упредметнюється в 6\shil{ шилінґах},
то \num{1.200} таких робочих днів — у 6\shil{ шилінґах} × \num{1.200}. В одному
випадку в продукті втілилось 12 × \num{1.200} робочих годин, у другому
тільки 12 робочих годин. У продукції вартости велике число
має значення завжди тільки як багато окремих одиниць. Отже, для
продукції вартости немає ніякої ріжниці, чи \num{1.200} робітників продукують
поодиноко, чи спільно під командою того самого капіталу.

А проте, в певних межах тут відбувається модифікація. Праця,
упредметнена у вартості, є праця пересічної суспільної якости,
тобто виявлення якоїсь пересічної робочої сили. Але пересічна
величина існує завжди лише як пересічна багатьох різних індивідуальних
величин того самого роду. В кожній галузі промисловости
індивідуальний робітник, Петро чи Павло, більш чи менш
відхиляється від пересічного робітника. Ці індивідуальні відхилення,
які в математиці називаються «помилками», компенсуються
і зникають, скоро тільки взяти разом велике число робітників.
Славетний софіст і сикофант Едмунд Берк навіть запевняє
на підставі свого практичного досвіду як фармер, що всяка індивідуальна
ріжниця праці зникає вже «для такої незначної групи»,
як 5 рільничих наймитів, отже, п’ять перших-ліпших дорослих
англійських рільничих наймитів протягом того самого часу виконають
разом стільки ж праці, як і яких-будь інших п’ять англійських
рільничих наймитів\footnote{
«Безперечно, щодо сили, вправности та сумлінної пильности, існує
велика ріжниця між вартістю праці однієї людини і вартістю праці іншої
людини. Але я цілком певен на підставі моїх докладних спостережень, що
п’ятеро перших-ліпших людей разом постачають кількість праці, рівну
кількості праці всяких інших п'ятьох людей зазначеного мною віку; це
значить, що з цих п’ятьох робітників один має всі властивості доброго
робітника, другий є недобрий, а троє інших будуть із цього погляду середні,
наближаючись то до першого, то до другого. Отже, вже в такій
невеликій групі, як п’ятеро людей, ви знайдете повнотою все те, що взагалі
можуть дати п'ятеро людей». («Unquestionably, there is a great deal
of difference between the value of one man’s labour and that of another,
from strength, dexterity and honest application. But I am quite sure, from
my best observation, that any given five men will, in their total, afford a
proportion of labour equal to any other five within the periods of life I have
stated; that is, that among such five men there will be one possessing all
the qualifications of a good workman, one bad, and the other three middling,
and approximating to the first and the last. So that in so small a platoon
as that of even five, you will find the full complement of all that five men
can earn»). (\emph{E.~Burke}: «Thoughts and Details on Searcity», London
1800, p. 16). Порівн., опріч того, Кетле про середнього індивіда.
}. Хоч би й що, а ясно, що сукупний
\index{i}{0259}  %% посилання на сторінку оригінального видання
робочий день значного числа одночасно експлуатованих робітників,
поділений на число робітників, сам по собі є день пересічної
суспільної праці. Припустімо, що робочий день поодинокого
робітника є, приміром, 12 годин. Тоді робочий день дванадцятьох
робітників, експлуатованих одночасно, становить сукупний
робочий день у 144 години, і хоч праця кожного поодинокого
з тих дванадцятьох робітників може більш або менш відхилятися
від пересічної суспільної праці; отже, поодинокий робітник
може потребувати на ту саму роботу трохи більш або трохи менш
часу, все ж робочий день кожного поодинокого робітника, як
одна дванадцята частина сукупного робочого дня в 144 години,
має пересічну суспільну якість. Але для капіталіста, що експлуатує
дванадцятьох робітників, робочий день існує як сукупний
робочий день дванадцятьох. Робочий день кожного поодинокого
робітника існує лише як відповідна частина сукупного робочого
дня цілком незалежно від того, чи працюють ці дванадцятеро
разом, допомагаючи один одному, чи ввесь зв’язок між їхніми
працями є лише в тому, що вони працюють для того самого капіталіста.
Навпаки, коли що два з цих 12 робітників працюватимуть
у дрібного майстра, то лише випадково кожний поодинокий майстер
випродукує, можливо, ту саму масу вартости і зреалізує,
отже, загальну норму додаткової вартости. Тут завжди бувають
індивідуальні відхилення. Коли б робітник зуживав на продукцію
якогось товару значно більше часу, ніж це суспільно-потрібно,
коли б індивідуально-доконечний для нього робочий
час значно відхилявся від суспільно-доконечного або пересічного
робочого часу, то його праця не мала б значення пересічної праці,
а його робоча сила — значення пересічної робочої сили. Цієї робочої
сили або зовсім не можна було б продати, або її можна було б
продати лише за ціну, нижчу від пересічної вартости робочої сили.
Отже, завжди припускається певний мінімум працездатности, і ми
побачимо пізніш, що капіталістична продукція находить засоби,
щоб виміряти цей мінімум. А проте цей мінімум відхиляється
від пересічної величини, хоч і доводиться сплачувати пересічну
\parbreak{}  %% абзац продовжується на наступній сторінці

\input{i/_0260.tex}
\index{i}{0261}  %% посилання на сторінку оригінального видання

Економію на засобах продукції треба взагалі розглядати з
подвійного погляду. Поперше, оскільки вона здешевлює товари
й тим знижує вартість робочої сили. Подруге, оскільки вона змінює
відношення додаткової вартості до цілого авансованого капіталу,
тобто до суми вартостей його сталих та змінних складових
частин. Цей останній пункт ми розглядаємо лише у першому
відділі третьої книги цього твору, куди задля зв’язку ми відносимо
й дещо інше, що можна б уже тут розглянути. Хід аналізу
вимагає саме так розбити тему, і це, зрештою, відповідає духові
капіталістичної продукції. Саме тому, що тут умови праці протистоять
робітникові як самостійні, то й економія на них виступає
як осібна операція, яка аніскільки не обходить робітника і
тому відокремлена від методів, що підвищують його особисту продуктивність.

Та форма праці, коли багато осіб планомірно й спільно, один
поруч одного, працюють у тому самому процесі продукції або в різних,
але зв’язаних між собою процесах продукції, називається
кооперацією\footnote{
«Concours de forces» («сполучення сил»). (\emph{Destutt de Tracy}: «Traité
de la Volonté et de ses effets», Paris 1826, p. 78).
}.

Подібно до того, як сила нападу ескадрону кавалерії або сила
опору полку піхоти посутньо відмінні від суми сил нападу й
опору кожного поодинокого кавалериста й піхотинця, так і механічна
сума сил поодиноких робітників відмінна від тієї суспільної
сили, яка розвивається, коли багато рук одночасно спільно працює
над тією самою неподільною операцією, приміром, коли треба
підняти тягар, покрутити корбою, забрати із шляху якусь перешкоду\footnote{
«Є безліч таких простих операцій, що їх не можна поділити на
частки, і все ж не можна виконати їх без кооперації багатьох рук. Так,
наприклад, підняти велику колоду на віз\dots{} коротко, всяка праця, що
її не можна виконати без співробітництва багатьох рук, які одночасно
помагають одна одній у тому самому неподільному процесі праці». («There
are numerous operations of so simple a kind as not to admit a division into
parts, which cannot be performed without the cooperation of many pairs
of hands. For instance the lifting og a large tree on a wain\dots{} every thing
in short, which cannot be done unless a great many pairs ef hands help each
other in the same undivided employment, and at the same time»). (\emph{E.~G.~Wakefield}: «А View of the Art of Colonization», London 1849, p. 168).
}.
За таких обставин цього результату комбінованої
праці або зовсім не можна було б досягти поодинокими силами,
або, якщо й можна було б, то тільки протягом довшого часу або
лише в карликовому маштабі. Тут справа не тільки в підвищенні
індивідуальної продуктивної сили через кооперацію, але й у
створенні продуктивної сили, яка сама по собі мусить бути масовою
силою\footnote*{У французькому виданні це речення подано так: «Справа не тільки
в підвищенні індивідуальних продуктивних сил, але й у створенні за допомогою
кооперації нової сили, яка функціонує лише як колективна
сила» \emph{Ред.}}\footnoteA{
«Якщо одна людина зовсім не може, а десятеро людей можуть
тільки з найбільшою напругою всіх своїх сил підняти тягар вагою в тонну,
то сто людей посягнуть цього, працюючи кожен лише одним пальцем» («As
one man cannot, and 10 men must strain, to lift a tun of weight, yet
one hundred men can do it only by the strength of a finger of each of them»).
(\emph{John Bellers}: «Proposals for raising a colledge of industry», London
1696, p. 21).
}.
\index{i}{0262}  %% посилання на сторінку оригінального видання

Крім цієї нової сили, яка постає із злиття багатьох сил в одну
колективну силу, вже самий суспільний контакт при більшості
продуктивних праць викликає змагання та своєрідне зворушення
життьового духа (animal spirits), яке збільшує індивідуальну
дієздатність поодиноких осіб, так що дванадцятеро осіб
разом протягом того самого робочого дня в 144 години дадуть
далеко більший сукупний продукт, ніж дванадцять поодиноких
робітників, що з них кожен працюватиме 12 годин, або ніж один
робітник, який працюватиме день за днем 12 днів\footnote{
«Отже, в цьому (коли один фармер уживає на 300 акрах те саме
число робітників, яке 10 дрібних фармерів уживають кожен на 30 акрах),
тобто в такій пропорції робітників, є й така користь, яку не легко зрозуміти
людям, незнайомим із справою на практиці: справді, хто буде
заперечувати, що 1 відноситься до 4, як 3 відноситься до 12; однак на
практиці це не так: підчас жнив і інших спішних робіт справа йде ліпше
й успішніше, коли сполучити значне число рук разом; так, наприклад,
2 возії, 2 навантажники, 2 подавальники, 2 загрібальники й декілька
людей на скиртах або на току зроблять удвоє більше, ніж те саме число
робочих рук, поділених на різні групи по поодиноких фармах». («There
is also an advantage in the proportion of servants, which will not easily be
understood but by practical men; for it is natural to say, as 1 is to 4, so
are $3: 12$; but this will not hold good in practice; for in harvest-time
and many other operations which require that kind of despatch, by throwing
many hands together, the work is better, and more expeditiously done:
for exemple, in harvest, 2 drivers, 2 loaders, 2 pitchers, 2 rakers, and the
rest at the rick, or in the barn, will despatch double the work, that the same
number of hands would do, if divided into different gangs, on different
farms»): («An Enquiry into the Connection between the present price of
provisions and the size of farms. By a Farmer», London 1773, p. 7, 8).
}. Це випливає
з того, що людина з природи є якщо й не політична\footnote{
Арістотелеве визначення, власне кажучи, говорить, що людина з природи є міський громадянин.
Для клясичної старовини це так само
характеристичне, як і визначення Франкліна, що людина з природи є
творець знарядь, характеристичне для доби янкі.
}, як думає
Арістотель, то в усякому разі громадська тварина.

Хоч багато осіб одночасно й спільно виконують таку саму або
однорідну працю, все ж індивідуальна праця кожної особи, як
частина колективної праці, може репрезентувати різні фази
самого процесу праці, що через них предмет праці, в наслідок
кооперації, перебігає швидше. Приміром, коли мулярі складають
ряд рук, щоб подавати цеглу від основи риштовання до його
верху, то кожен з них робить те саме, а все ж поодинокі операції
становлять безперервні частини однієї спільної операції, окремі
фази, які кожна цеглина мусить перебігти в процесі праці, і
завдяки чому 24 руки колективного робітника подають цеглу
швидше, ніж дві руки поодинокого робітника, що сходить на
риштовання та спускається з нього\footnote{
«Треба ще зауважити, що такий частинний поділ праці може бути
навіть тоді, коли робітники працюють коло тієї самої справи. Наприклад,
мулярі, які подають із рук до рук цеглу на високе риштовання, виконують
усі ту саму роботу, а все ж між ними є якийсь рід поділу праці, який полягає
в тому, що кожний з них переносить цеглу на певну віддаль, і що всі
вони приставляють її на місце призначення далеко швидше, ніж це було б,
якби кожний з них сам носив свою цеглу на те високе риштовання»
(«On doit encore remarquer que cette division partielle de travail peut se
faire quand même les ouvriers sont occupés d’une même besogne. Des maçons,
par exemple, occupés de faire passer de mains en mains des briques à un
échafaudage supérieur, font tous la même besogne, et pourtant il existe
parmi eux une espèce de division de travail, qui consiste en ce que chacun
d’eux fait passer la brique par un espace donné, et que tous ensemble la
font parvenir beaucoup plus promptement à l’endroit marqué qu’ils ne le
feraient si chacun d’eux portait sa brique séparement jusqu’à l’échafaudage
supérieur»). (\emph{F.~Skarbek}: «Théorie des richesses sociales», 2 éme éd.
Paris 1840, vol. I., p. 97, 98).
}. Предмет праці перебігає
\index{i}{0263}  %% посилання на сторінку оригінального видання
ту саму просторінь за коротший час. З другого боку, комбінацію
праці маємо й тоді, коли, приміром, будівлю розпочинають одночасно
з різних боків, хоч би кооперовані робітники робили те
саме або однорідне. Комбінований робочий день в 144 години,
який охоплює предмет праці з багатьох боків у простороні, бо
комбінований робітник або робітник колективний має очі й руки
спереду й ззаду і є до певної міри всюдисущий, — той робочий
день посуває наперед виготовлення цілого продукту швидше,
ніж 12 дванадцятигодинних робочих днів більш або менш відокремлених
робітників, які мусять братися до своєї праці однобічніше.
Просторово різні частини продукту таким чином вистигають
у той самий час.

Ми підкреслювали, що багато робітників, які один одного
доповнюють, роблять те саме або однорідне, бо ця найпростіша
форма спільної праці відіграє чималу ролю і в найрозвиненішій
формі кооперації. Якщо процес праці складний, то вже сама
маса тих, що спільно працюють, дозволяє розподіляти різні
операції поміж різних робітників, отже, і виконувати їх одночасно
та через це скорочувати робочий час, потрібний, щоб виготовити
цілий продукт\footnote{
«Коли треба виконати складну працю, різні справи треба виконувати
одночасно. Один робить одне, тимчасом як другий робить друге,
і всі разом допомагають досягти результату, якого зовсім не могла б здійснити
одна людина. Один гребе, тимчасом як другий кермує стерном, а
третій закидає невід або б'є рибу бодцем, — і влови риби дають такий
результат, що був би неможливий без такого співробітництва». («Est-il
question d’exécuter un travail compliqué, plusieurs choses doivent être
faites simultanément. L’un en fait une pendant que l’autre en fait une
autre, et tous contribuent à l’effet qu’un seul homme n'aurait pu produire.
L’un rame pendant que l’autre tient le gouvernail, et qu’un troisième jette
le filet, ou harponne le poisson, et la pêche a un succès impossible sans
ce concours»). (\emph{Destutt de Tracy}: «Traité de la Volonté et de ses effets»,
Paris 1826, p. 78).
}.

У багатьох галузях продукції бувають критичні моменти,
тобто визначувані самою природою робочого процесу періоди
часу, протягом яких мусять бути досягнені певні результати
праці. Коли, приміром, треба постригти ватагу овець або зжати
та звезти хліб із певного числа морґів, то кількість і якість продукту
залежить від того, щоб операція почалася в певний час
\parbreak{}  %% абзац продовжується на наступній сторінці

\parcont{}  %% абзац починається на попередній сторінці
\index{i}{0264}  %% посилання на сторінку оригінального видання
і скінчилася в певний час. Період, що протягом його треба виконати
процес праці, є тут так само приписаний, як от за вловів
оселедців. Поодинока людина може з одного дня викраяти лише
один робочий день, приміром, у 12 годин, але кооперація, приміром,
із 100 людей збільшує дванадцятигодинний день до робочого
дня в 1.200 годин. Короткість строку праці компенсується
величиною маси праці, що її кидається у вирішальний момент
на поле продукції. Вчасний результат залежить тут від одночасного
вживання багатьох комбінованих робочих днів, а обсяг
корисного ефекту — від числа робітників, яке однак завжди
лишається меншим від числа тих робітників, що змогли б протягом
того самого часу виконати ту саму працю, працюючи кожен
окремо.\footnote{
«Виконання її (рільничої праці) у критичний момент має величезну
вагу» («The doing of it at the critical juncture, is of so much the
greater consequence»). («An Inquiry into the Connection between the present
price etc.», p. 7). «У рільництві немає важливішого фактора, ніж час»
(Liebig: «Ueber Theorie und Praxis in der Landwirtschaft», 1856, S. 23).
} Це через брак такої кооперації на заході Сполучених
штатів гине рік-у-рік сила хліба, а в тих частинах Східньої
Індії, де англійське панування знищило давню громаду, — сила
бавовни.\footnote{
«Дальше лихо, що його ледве чи хто міг сподіватись у країні,
яка вивозить праці більше, ніж усяка інша країна, за винятком хіба
Китаю та Англії, — це неможливість знайти достатню кількість робочих
рук для збирання бавовни. В наслідок цього значна частина врожаю лишається
незібрана, а другу частину його збирають із землі після того, як
бавовна вже висипалась і через це втратила належний колір і почасти
згнила; таким чином через те, що у відповідний час бракує робочих рук,
плянтатор фактично примушений відмовитися від великої частини того
врожаю, що його з такою тривогою сподівається Англія». («The next
evil is one which one would scarcely expect to find in a country which exports
more labour than any other in the world, with the exception perhaps of
China an England — the impossibility of procuring a sufficient number of
hands to clean the cotton. The consequence of this is that large quantities of
the crop are left unpicked, while another portion is gathered from the ground,
when it has fallen, and is of course discoloured and partially rotted, so
that for want of labour at the proper season the cultivator is actually forced
to submit to the loss of a large part of that crop for which England is so
anxiously looking»). (Bengal Hurkaru: «Bi-Monthly Overland Summary
of News. 22 nd July 1861»).
}

З одного боку, кооперація дозволяє поширити просторову
сферу праці, а тому для певних процесів праці, як от, приміром,
за осушування ґрунту, будування гребель, іриґації, будування
каналів, шляхів, залізниць тощо, вона потрібна вже в наслідок
просторової зв’язаности предмету праці. З другого боку, кооперація
уможливлює просторово звужувати, порівняно з маштабом
продукції, поле продукції. Це обмеження просторової сфери праці
за одночасного поширення сфери її діяння, через що заощаджується
багато непродуктивних витрат (faux frais), постає із зосередження
робітників, зближення різних процесів праці та концентрації
засобів продукції.\footnote{
«З прогресом рільництва всю ту, а, може, і ще значнішу кількість
капіталу й праці, яку колись уживали для поверхового оброблення 500 ак-
}

\index{i}{0265}  %% посилання на сторінку оригінального видання
Проти рівновеликої суми відокремлених індивідуальних робочих
днів комбінований робочий день продукує більші маси споживних
вартостей і тому зменшує робочий час, потрібний, щоб
досягти певного корисного ефекту. Чи комбінований робочий
день у даному випадку дістає цю збільшену продуктивну силу
тому, що він підносить механічну силу праці, чи тому, що поширює
її просторову сферу діяння; чи тому, що він супроти маштабу
продукції просторово звужує продукційне поле; чи тому, що в
критичний момент він пускає в рух багато праці за короткий час;
чи тому, що заохочує поодиноких осіб до змагання та напружує
їхній життєвий дух; чи тому, що він накладає печать безперервности
та багатобічности на однорідні операції багатьох осіб; чи
тому, що виконує одночасно різні операції, чи тому, що економізує
засоби продукції через спільний ужиток їх; чи тому, що надає
індивідуальній праці характеру пересічної суспільної праці, —
за всяких обставин специфічна продуктивна сила комбінованого
робочого дня є суспільна продуктивна сила праці, або продуктивна
сила суспільної праці. Вона випливає із самої кооперації. У пляномірному
співробітництві з іншими робітник стирає свої індивідуальні
межі й розвиває свою родову спроможність.\footnote{
«Сила кожної людини мінімальна, але сполука мінімальних сил
утворює спільну силу, більшу за суму цих сил, так що сили через саме
своє об’єднання можуть зменшити час та збільшити сферу своєї акції»
(«La forza di ciascuno uomo è minima, ma la riunione delle minime forze
forma una forza totale maggiore anche della somma delle forze medesime
fino a che le forze per essere riunite possono diminuere il tempo ed accrescere
lo spazio della loro azione»). (G. R. Carli примітка до P. Verri: «Meditazioni
sulla Economia Politica». vol. XV, p. 196). [«Колективна
праця дає такі результати, яких ніколи не могла б дати індивідуальна
праця. Отже, у міру того як зростатиме кількість людности, продукти
об’єднаної промисловости значно переважатимуть суму, що її ми мали
в наслідок простого складання, обчисленого на основі цього зросту\dots{}
У сфері механічних робіт так само, як і в сфері наукових робіт, людина
може протягом одного дня фактично зробити більше, ніж ізольований
індивід протягом усього свого життя. Аксіома математиків, що ціле
дорівнює сумі частин, прикладена до нашого предмету, вже не є правильна.
Щодо праці, цієї великої основи існування людства, то можна
сказати, що продукт об’єднаних зусиль значно переважає все те, що
могли б колибудь спродукувати зусилля поодиноких і розрізнених індивідів».
— Th. Sadler: «The Law of Population», London 1850].\footnote*{
Наведене тут у прямих дужках ми беремо з французького видання.
(«Le Capital etc.», v. I, ch. XIII, p. 143). \emph{Ред.}
}
}

рів, концентрується тепер для досконалішого оброблення 100 акрів».
Хоч «проти кількости вживаного капіталу й праці просторінь і скоротилася,
проте сфера продукції поширилася супроти тієї сфери, що її
раніш мав або експлуатував поодинокий незалежний аґент продукції»).
(«In the progress of culture all, and perhaps more than all the capital an
labour which once loosely occupied 500 acres, are now concentrated for the
more complete tillage of 100. Relatively to the amount of capital and labour
employed, space is concentrated, it is an enlarged sphere of production,
as compared to the sphere of production formely occupied or worked upon
by one single, independent agent of production»). (R. Jones: «An Essay,
on the Distribution of wealth. Part I. On Rent», London 1831, p. 191,
199).
\parbreak{}  %% абзац продовжується на наступній сторінці

\parcont{}  %% абзац починається на попередній сторінці
\index{i}{0266}  %% посилання на сторінку оригінального видання
Коли робітники взагалі не можуть безпосередньо співробітничати,
не бувши згуртовані, отже, коли згуртовання їх у певному
місці є умова їхньої кооперації, то наймані робітники не можуть
кооперувати без того, щоб той самий капітал, той самий капіталіст
не вживав їх одночасно, отже, і не купував одночасно їхні робочі
сили. Тому сукупна вартість цих робочих сил, або сума заробітної
плати робітників за день, тиждень і т. д., мусить вже бути нагромаджена
в кишені капіталіста раніш, ніж сами робочі сили будуть
сполучені у продукційному процесі. Заплатити 300 робітникам
відразу навіть хоч би й лише за один день — це вимагає
більшого авансування капіталу, аніж платити кільком робітникам
тиждень за тижнем протягом цілого року. Отже, число
кооперованих робітників, або маштаб кооперації, залежить насамперед
від величини того капіталу, що його поодинокий капіталіст
може авансувати на купівлю робочої сили, тобто від того обсягу,
в якому кожен поодинокий капіталіст порядкує засобами
існування багатьох робітників.

І щодо сталого капіталу справа стоїть так само, як і щодо
змінного. Приміром, видатки на сировинний матеріял для одного
капіталіста, що вживає 300 робітників, у тридцять разів більші,
ніж для кожного з тих 30 капіталістів, що кожний з них вживає
10 робітників. Правда, розмір вартости й маса матеріялу спільно
вживаних засобів праці зростають не в такій пропорції, як число
вживаних робітників, але все ж вони зростають дуже значно.
Отже, концентрація більших мас засобів продукції в руках поодиноких
капіталістів є матеріяльна умова кооперації найманих
робітників, а розмір кооперації, або маштаб продукції залежить
від розміру цієї концентрації.

Первісно певна мінімальна величина індивідуального капіталу
виступала як доконечна для того, щоб кількости одночасно
визискуваних робітників, а тому й маси продукованої додаткової
вартости вистачило для звільнення самого визискувача від ручної
праці, для перетворення дрібного майстра на капіталіста, отже,
і для того, щоб формально створити капіталістичне відношення.
Тепер вона виступає як матеріяльна умова для перетворення
багатьох розпорошених і один від одного незалежних індивідуальних
процесів праці на один комбінований суспільний процес
праці.

Так само командування капіталу над працею первісно виступало
лише як формальний наслідок того, що робітник, замість
працювати на себе, працює на капіталіста, а тому й під доглядом
капіталіста. З розвитком кооперації багатьох найманих робітників
командування капіталу розвивається на доконечність для
виконання самого процесу праці, на дійсну умову продукції.
Наказ капіталіста на полі продукції стає тепер так само доконечний,
як наказ генерала на полі бою.

Всяка безпосередньо суспільна або спільна праця у великому
маштабі потребує в більшій або меншій мірі керування, яке упосереднює
гармонію між індивідуальними діями та виконує загальні
\index{i}{0267}  %% посилання на сторінку оригінального видання
функції, що виникають із руху цілого продуктивного тіла
відмінно від руху його самостійних органів. Окремий скрипаль
дириґує собі сам, оркестра потребує дириґента. Ця функція
керування, догляду та упосереднення стає функцією капіталу,
скоро тільки підпорядкована йому праця стає кооперативною.
Як специфічна функція капіталу функція керування набирає
специфічних характеристичних ознак.

Насамперед рушійним мотивом і визначальною метою капіталістичного
процесу продукції є якомога більше самозростання
капіталу,\footnote{
«Зиск... однісінька мета продукції» («Profits.... is the sole end
of trade»). (J. Vanderlint: «Money answers all Things», London 1734,
p. 11)
} тобто якомога більша продукція додаткової вартости,
отже, якомога більший визиск робочої сили капіталом. Із зростом
маси одночасно експлуатованих робітників зростає і їхній
опір, а тому неминуче зростає і гніт капіталу, щоб перебороти
той опір. Керування капіталіста є не тільки осібна функція, що
виникає з природи суспільного процесу праці й належить до нього,
воно є одночасно й функція визиску суспільного процесу праці
і тому зумовлюється неминучим антагонізмом між визискувачем
і сировинним матеріялом його визиску. Так само із зростом розміру
засобів продукції, що протистоять найманому робітникові
як чужа власність, зростає й доконечність контролю над доцільним
уживанням цих засобів.\footnote{
Часопис англійських філістерів «Spectator» сповіщає в числі
з 3 червня 1866 р., що після заведення чогось на зразок товариського підприємства
між капіталістом та робітником у «Wirework company of Manchester»
«першим результатом було те, що раптом зменшилося псування
матеріялу, бо робітники зрозуміли, що їм, як і всім іншим власникам,
нема нащо псувати своє власне майно, а псування знаряддя та матеріялу
є, може, найбільше, після легкодушних боргів, джерело втрат у промисловості»
(«the first result was a sudden decrease in waste, the men not seeing
why they should waste their own property any more than any other master’s,
and waste is perhaps, next to bad debts, the greatest source of manufacturing
loss»). Той самий часопис викриває ось яку основну хибу в рочдельських кооперативних
спробах: «Вони показали, що робітничі асоціяції можуть
успішно порядкувати крамницями, фабриками та майже всіма формами
промисловости, і що вони надзвичайно поліпшили становище самих робітників,
але! але в такому випадку вони зовсім не залишали якогось виразного
місця для капіталіста» («They showed that associations of workmen
could manage shops, mills, and almost all forms of industry with success,
and they immensely improved the condition of the men, but then they did
not leave a clear place for masters»). Quelle horreur!\footnote*{
Який жах! Ред.
}
} Далі, кооперація найманих робітників
є лише результат діяння капіталу, який їх одночасно вживає.
Зв’язок їхніх функцій та їхня єдність як продуктивного цілого
тіла лежать поза ними, в капіталі, що їх згуртовує та тримає
вкупі. Тим-то зв’язок їхніх праць протистоїть їм ідеально як плян,
практично — як авторитет капіталіста, як сила чужої волі, що
підпорядковує їхню діяльність своїй меті.

Тому, якщо капіталістичне керування своїм змістом є двоїсте
внаслідок двоїстости самого продукційного процесу, що ним
\parbreak{}  %% абзац продовжується на наступній сторінці

\input{i/_0268.tex}
\parcont{}  %% абзац починається на попередній сторінці
\index{i}{0269}  %% посилання на сторінку оригінального видання
Це відношення анітрохи не змінюється від того, що капіталіст
замість однієї купує 100 робочих сил або складає контракт замість
з одним із 100 незалежними один від одного робітниками. Він може
вживати 100 робітників, не кооперуючи їх. Тому капіталіст
оплачує вартість 100 самостійних робочих сил, але не оплачує
комбінованої робочої сили 100 робітників. Як незалежні особи,
робітники є окремі індивіди, що стають у відношення до того
самого капіталу, а не один до одного. Кооперація між ними починається
лише в процесі праці, алеж у процесі праці вони перестають
уже належати самим собі. Вступивши до цього процесу,
вони стають складовою частиною капіталу. Як кооперовані
особи, як члени одного активного організму, вони сами є лише
осібні способи існування капіталу. Тим-то продуктивна сила,
яку робітник розвиває як суспільний робітник, є продуктивна
сила капіталу. Суспільна продуктивна сила праці розвивається
безплатно, скоро тільки робітників поставлено в певні умови,
а капітал ставить їх у ці умови. А що суспільна продуктивна
сила праці нічого не коштує капіталові і що, з другого боку, робітник
не розвиває її раніше, ніж його праця належить капіталові,
то й видається вона продуктивною силою, яку капітал має з
природи, його іманентною продуктивною силою.

Колосальним виявляється ефект простої кооперації у велетенських
спорудах давніх азійців, єгиптян, етрусків і~\abbr{т. ін.}
«За минулих часів траплялося, що ці азійські держави після
покриття своїх цивільних та військових видатків мали ще якийсь
лишок засобів існування, що його вони могли витратити на розкішні
й корисні твори. Їхнє панування над робочими силами майже
всієї рільничої людности та виключне право монарха і духівництва
порядкувати тим лишком давали їм засоби будувати ті
могутні монументи, якими вони заповнили країну\dots{} При переміщенні
тих велетенських статуй і тих величезних мас, що їх
транспорт викликає подив, марнотратно вживали майже виключно
людської праці. Для цього досить було певного числа робітників
та концентрації їхніх зусиль. Так ми бачимо, як із глибин океану,
з коралевих скель виростають острови, творячи суходіл, дарма
що кожний індивід, що бере участь у творенні їх (depositary),
є дрібний, слабий та нікчемний. Нерільничі робітники азійської
монархії мало що повинні були додати до тієї справи, крім
своїх індивідуальних фізичних зусиль; та їхнє число — це їхня
сила, а влада керування цими масами дала початок тим велетенським
спорудам. Саме концентрація в руках однієї або небагатьох
осіб тих доходів, що з них жили робітники, робила можливими
такі підприємства»\footnote{
R.~Jones: «Textbook of Lectures etc.», Hertford 1852, p. 77, 78.
Давні асирійські, єгипетські та інші колекції в Лондоні та по інших
европейських столицях роблять нас свідками тих кооперативних процесів
праці.
}. Ця влада азійських та єгипетських царів
або етруських теократів тощо в сучасному суспільстві перейшла
до капіталіста, однаково, чи виступає він як поодинокий капіталіст,
\index{i}{0270}  %% посилання на сторінку оригінального видання
чи, як це маємо в акційних товариствах, як комбінований
капіталіст.

Кооперація в процесі праці в тій формі, в якій ми находимо
її переважно на початках людської культури, у мисливських народів\footnoteA{
Linguet у своїй «Théorie des Lois civiles», мабуть, має рацію, коли
він називає полювання першою формою кооперації, а полювання на людей
(війну) першою формою полювання.
} або хоч би в рільничих індійських громадах, базується
з одного боку, на спільному володінні умовами продукції, з другого
боку — на тому, що поодинокий індивід не відірвався ще від
пуповиння племени або громади й прив’язаний до них так само
міцно, як окрема бджола до бджоляного вулика. І те і друге відрізняє
цю кооперацію від капіталістичної кооперації. Спорадичне
вживання кооперації у великому маштабі в античному світі, в
середньовіччі та в сучасних колоніях базується на безпосередніх
відносинах панування та підлеглости, здебільша на рабстві.
Навпаки, капіталістична форма кооперації з самого початку має
за свою передумову вільного найманого робітника, що продає
свою робочу силу капіталістові. Однак історично вона розвивається
протилежно до селянського господарства та незалежного
ремества, однаково, чи це останнє має цехову форму, чи ні\footnote{
Дрібне селянське господарство і незалежне ремество, що обидва
почасти становлять базу февдального способу продукції, а почасти після
його розкладу існують поруч капіталістичної продукції, разом з тим становлять
економічну основу клясичної громади за її найліпших часів, за
тих часів, коли первісна східня громадська власність уже розпалася, а
рабство не встигло ще серйозно опанувати продукцію.
}.
Супроти них капіталістична кооперація виступає не як осібна
історична форма кооперації; навпаки, сама кооперація виступає
супроти них як певна історична форма, властива капіталістичному
процесові продукції, як специфічна форма, що відрізняє
його від інших способів продукції.

Подібно до того, як суспільна продуктивна сила праці, що
розвивається в наслідок кооперації, з’являється як продуктивна
сила капіталу, так само й сама кооперація з’являється як специфічна
форма капіталістичного процесу продукції, протилежно до
процесу продукції поодиноких незалежних робітників або й дрібних
майстрів. Це — перша зміна, якої зазнає дійсний процес
праці через свою підлеглість капіталові. Ця зміна відбувається
спонтанно. Її передумова, одночасна експлуатація значного
числа найманих робітників у тому самому процесі праці, становить
вихідний пункт капіталістичної продукції, який збігається
з існуванням самого капіталу. Тим то, якщо капіталістичний
спосіб продукції, з одного боку, являє собою історичну доконечність
для перетворення процесу праці на суспільний процес, то,
з другого боку, ця суспільна форма процесу праці являє собою
методу, що її застосовує капітал на те, щоб із більшим зиском
експлуатувати процес праці, збільшуючи його продуктивну силу.

У своїй розглянутій досі простій формі кооперація збігається
з продукцією у великому маштабі, але не становить тривалої
\parbreak{}  %% абзац продовжується на наступній сторінці

\parcont{}  %% абзац починається на попередній сторінці
\index{i}{0271}  %% посилання на сторінку оригінального видання
характеристичної форми якоїсь осібної епохи розвитку капіталістичного
способу продукції. Щонайбільше, вона є приблизно
така на початках мануфактури,\footnote{
«Хіба поєднання вправности, працьовитости та змагання багатьох,
що виконують ту саму працю, не є спосіб посувати наперед цю працю?
І хіба Англія могла б якимось іншим способом довести свою вовняну мануфактуру
до такої досконалости?» («Whether the united skill, industry
and emulation of many together on the same work be not the way to advance
it? And whether it had been otherwise possible for England, to have
carried on her Woollen Manufacture to so great perfection?»). (\emph{Berkeley}: «The
Querist», London 1750, p. 56, § 521).
} зорганізованої ще на ремісничий
штиб, та в тих великих рільничих господарствах, які відповідають
епосі мануфактури й посутньо відрізняються від селянського
господарства тільки масою одночасно вживаних робітників
та розміром сконцентрованих засобів продукції. Проста кооперація
все ще є переважна форма по таких галузях продукції, де
капітал оперує у великому маштабі, а поділ праці й машини не
відіграють значної ролі.

Кооперація лишається основною формою капіталістичного способу
продукції, хоч проста її форма сама з’являється як осібна
форма поряд інших розвиненіших її форм.

\section{Поділ праці та мануфактура}
\subsection{Двояке походження мануфактури}

Кооперація, що ґрунтується на поділі праці, утворює собі
свою клясичну форму в мануфактурі. Як характеристична форма
капіталістичного процесу продукції вона домінує протягом мануфактурного
періоду у власному значенні, який триває приблизно
від середини ХVІ віку до останньої третини XVIII.

Мануфактура виникає двояким способом.

Або робітників різнорідних самостійних реместв, що через
їхні руки мусить переходити продукт аж до останньої стадії його
виготовлення, згуртовують в одній майстерні під командою того
самого капіталіста. Приміром, карета була продуктом спільної
праці великого числа незалежних ремісників, як от стельмаха,
римаря, кравця, слюсаря, мідяра, токаря, позументаря, скляра,
маляра, лаківника, позолотника і~\abbr{т. ін.} Каретна мануфактура
сполучає всіх цих різних ремісників в одній майстерні, де вони
одночасно спільно працюють. Правда, карету не можна золотити
раніш, ніж її зроблено. Але якщо одночасно роблять багато карет,
то одну якусь частину можна завжди золотити, тимчасом як
інша частина пробігає ранішу фазу продукційного процесу. До
цього часу ми все ще стоїмо на ґрунті простої кооперації, яка
находить готовим свій людський та речовий матеріял. Алеж
дуже скоро настає ґрунтовна зміна. Кравець, слюсар, мідяр і~\abbr{т. ін.}, що працює тільки коло карет, утрачає крок за кроком
\parbreak{}  %% абзац продовжується на наступній сторінці

\input{i/_0272.tex}

\index{ii}{0273}  %% посилання на сторінку оригінального видання
На основі суспільної продукції треба визначити маштаб, що в ньому
такі операції, які на довгий час відтягують робочу силу й засоби продукції,
не даючи протягом цього часу жодного продукту як корисного
наслідку, можуть провадитись без шкоди для тих галузей продукції, які
постійно або кілька разів на рік не лише відтягують робочу силу й засоби
продукції, а й дають засоби, існування й засоби продукції. За суспільної
продукції, так само, як і за капіталістичної продукції, робітники
в галузях підприємств з короткими робочими періодами, як і раніше, лише
на короткий час відтягуватимуть продукти, не даючи натомість нового
продукту, тимчасом як галузі підприємств з довгими робочими періодами,
перше ніж вони сами почнуть давати продукти, постійно відтягують
продукти на довгий час. Отже, ця обставина випливає з речових
умов відповідного процесу праці, а не з його суспільної форми. За суспільної
продукції грошовий капітал відпадає. Суспільство розподіляє робочу
силу й засоби продукції між різними галузями праці. Продуценти
можуть, правда, одержувати паперові посвідки, що ними вони вилучають
з суспільних споживних запасів ту кількість продуктів, яка відповідає їхньому
робочому часові. Ці посвідки — зовсім не гроші. Вони не циркулюють.

Тепер ми бачимо, що, оскільки потреба в грошовому капіталі випливає
з протягу робочого періоду, її зумовлено двома обставинами: п оперше,
тією, що гроші взагалі є та форма, що в ній мусить виступити
кожен індивідуальний капітал (кредит ми лишаємо осторонь) для того,
щоб перетворитись на продуктивний капітал. Це випливає з суті капіталістичної
продукції, взагалі товарової продукції. — Подруге, величину
потрібного грошового авансування зумовлює та обставина, що протягом
порівняно довгого часу суспільству постійно відбирається робочу силу
й засоби продукції, при чому протягом цього часу йому не повертається
жодного продукту, що його можна було б перетворити на гроші.
Першої обставини, а саме того, що авансовуваний капітал треба авансувати
в грошовій формі, не знищує форма самих цих грошей, тобто те,
що вони є або металеві, або кредитові гроші, або знаки вартости й~\abbr{т. ін.} На другу обставину жодного впливу не справляє те, за допомогою
яких грошових засобів або за допомогою якої форми продукції
відтягають працю, засоби існування та засоби продукції, не подаючи
натомість у циркуляцію жодного еквіваленту.
\label{original-273}
\index{i}{0274}  %% посилання на сторінку оригінального видання
2. Частинний робітник та його знаряддя

Якщо тепер підійти ближче до деталів, то насамперед ясно,
що робітник, який цілий свій вік виконує одну й ту саму просту
операцію, перетворює ціле своє тіло на її автоматично однобічний
орган і тому витрачає на це менше часу, ніж ремісник, що
виконує навпереміну цілий ряд операцій. Але комбінований
колективний робітник, що становить живий механізм мануфактури,
складається тільки з таких однобічних частинних робітників.
Тим то тут порівняно з самостійним ремеством за коротший
час продукується більше, тобто продуктивна сила праці підвищується.\footnote{
«Що більше працю якоїсь складної мануфактурної галузі розчленовано
й поділено між різними ремісниками, то ліпше й скорше цю працю виконується,
то менше витрачається часу та праці» («The more any manufacture
of much variety shall be distributed and assigned to different atrists
the same must needs be better done and with greater expedition, with less
loss of lime and labour»), («The Advantages of the East-India Trade»,
London 1720, p. 71).
}
Крім цього, якщо частинна праця всамостійнюється
у виключну функцію однієї особи, то вдосконалюється і її метода.
Постійне повторювання тієї самої обмеженої роботи й концентрація
уваги на цій обмеженій роботі навчають через досвід досягати
наміченого корисного ефекту з якнайменшою витратою сили.
А що різні ґенерації робітників завжди одночасно живуть разом
і працюють разом у тих самих мануфактурах, то набуті таким
чином способи технічної вмілости швидко вкорінюються, нагромаджуються
та передаються від однієї ґенерації до другої.\footnote{
«Легка праця є успадкована вправність» («Easy labour is transmitted
skill»). (Th. Hodgskin: «Popular Political Economy», London 1827, p. 48).
}

Мануфактура, репродукуючи всередині майстерні й систематично
розвиваючи до крайніх меж те розмежування реместв,
яке вона знайшла у містах середньовіччя, тим самим фактично
продукує віртуозність частинних робітників. З другого боку, її
перетворення частинної праці на життєву професію людини
відповідає прагненню попередніх суспільств робити ремество спадковим,
надавати йому закам’янілої форми каст або, — якщо
певні історичні умови створювали змінливість індивідів, яка
суперечила кастовому ладу, — закостенілої форми цехів. Касти
й цехи виникають із того самого природного закону, що реґулює
поділ рослин і тварин на роди й підроди, з тією лише відміною,
що на певному ступені розвитку спадковість каст або винятковість
цехів декретується як суспільний закон.\footnote{
«Вправності також\dots{} дійшли в Єгипті належного ступеня досконалости.
Бо лише в цій одній країні ремісники ні в якому разі не сміють
встрявати до занять інших громадянських кляс, а повинні працювати
лише в тій професії, яка за законом спадково належала їхньому родові\dots{}
В інших народів ми знаходимо, що ремісники поділяють свою увагу на
надто багато об'єктів\dots{} То заходяться вони коло обробітку землі, то беруться
до торговельних справ, то займаються одночасно двома або трьома
ремествами. У вільних державах вони часто бігають на народні збори\dots{}
Навпаки, в Єгипті кожного ремісника, що встрявав до державних справ
або береться одночасно до кількох реместв, піддають тяжким карам.
} «Мусліну з Даккі
\index{i}{0275}  %% посилання на сторінку оригінального видання
щодо його тонкости, ситців і інших матерій з Короманделя щодо
пишноти та тривалости фарб ще ніколи не перевищено. А проте
їх продукують без капіталу, без машин, без поділу праці або
якогось із тих інших засобів, що дають так багато переваг європейській фабрикації. Ткач — то
ізольований індивід, що на замовлення споживача виготовлює тканину, а до того — ще й на ткацькому
варстаті якнайпростішої конструкції, варстаті, що
часом складається лише з дерев’яних грубо позбиваних брусів.
У нього немає навіть апарату для натягування основи, і тому
ткацький варстат мусить лишатися розтягнутим на цілу свою
довжину, та такий він незграбний і широкий, що не може вміститися в хаті продуцента, і тому цей
останній мусить виконувати свою працю на вільному повітрі, перериваючи її повсякчас у негоду».\footnote{
«Historical and descriptive Account of British India etc. by
Hugh Murray, James Wilson etc.», Edinburgh 1832, vol. II, p. 449, 450.
Індійський ткацький варстат дуже високий, бо основу натягується вертикально.
}
Лише ця особлива вмілість, що нагромаджувалася
від покоління до покоління та спадково переходила від батька
до сина, дає індусові, як і павукові, цю віртуозність. А проте
порівняно з більшістю мануфактурних робітників такий індійський ткач виконує дуже складну працю.

Ремісник, що виконує один по одному різні частинні процеси
в продукції якогось виробу, мусить змінювати то місце, то інструменти. Перехід від однієї операції
до іншої перериває хід його праці і становить, так би мовити, пори в його робочому дні. Ці пори
звужуються, якщо він протягом цілого дня безупинно
виконує ту саму операцію, або вони зникають у міру того, як
меншає змінливість його операцій. Збільшена продуктивність
постає тут або із збільшення витрати робочої сили протягом
даного часу, отже, із зросту інтенсивности праці, або із зменшення
непродуктивного споживання робочої сили. А саме: зайва витрата
сили, що її вимагає кожний перехід од спокою до руху, компенсується при довшому триванні осягнутої
вже нормальної швидкости
праці. З другого боку, безперервність одноманітної праці ослабляє
напруження уваги та розмах життєвого духа, який саме в зміні
діяльности знаходить свій відпочинок та принаду.

Продуктивність праці залежить не тільки від віртуозности
робітника, але й від досконалосте його знарядь. Знарядь того
самого роду, як, приміром, різальних, свердлильних, поштовхових та ударних і т. ін., вживається в
різних процесах праці, і той самий інструмент у тому самому процесі праці придається до різних
операцій. Однак, скоро тільки різні операції якогось
процесу праці відокремляться одна від одної, і кожна частинна
операція набуде в руках частинного робітника якнайвідповіднішої, а через це й виключної форми, то
постає доконечність змін

Таким чином ніщо не може їм заважати пильно працювати в своїй професії\dots{} А до того, діставши багато
правил від своїх прадідів, вони ревно
дбають про те, шоб винайти нові удосконалення». (Diodorus Siculus:
«Historische Bibliothek», Bd. I, Kap. 74, S. 117, 118).
\parbreak{}  %% абзац продовжується на наступній сторінці

\input{i/_0276.tex}
\parcont{}  %% абзац починається на попередній сторінці
\index{i}{0277}  %% посилання на сторінку оригінального видання
пружин, виготівник циферблятів, виготівник спіральних пружин,
робітник, що робить дірки для каменя та вставляє рубін, виготівник
стрілок, виготівник коробки для годинника, виготівник
шрубів, позолотник з багатьма підрозділами, як от, приміром,
колісник (виріб мідяних та сталевих коліщат знову поділено),
Triebmacher, Zeigenverkmacher, acheveur de pignon (закріплює
коліщата в належних місцях, полірує facettes і т. ін.), Zapfenmacher,
planteur de finissage (вставляє в механізм різні коліщата
та пружини), finisseur de barillet (вирізує зубці, поширює дірочки
до належного розміру та закріплює установку), Hemmungmacher
і, як підрозділ цієї галузі, виготівник циліндрів, виготівник
трибків, виготівник маятників, Raquettemacher (тобто виготівник
механізму, що реґулює годинник), planteur d’échappement
(Hemmungmacher у власному значенні); далі: repasseur de barillet
(виготовлює коробку для пружини та закріпляє її установку),
ґлянсувальник сталі, ґлянсувальник коліщат, ґлянсувальник
шруб, маляр цифр, Blattmacher (покриває мідь емалем),
fabricant de pendants (виготовлює лише кільця до годинникової
коробки), finisseur de charnière (вставляє мосяжевий штифт
всередину коробки й т. ін.), faiseur de secret (виготовлює пружину,
що відкриває кришку годинника), ґравер, ciseleur, полірувальник
годинникової коробки і т. д. і т. д., нарешті, repasseur, що
складає окремі частини годинника докупи та пускає годинника
в рух. Лише небагато частин годинника переходить через різні
руки, і всі ці membra disjecta збираються лише в руках того,
хто, кінець-кінцем, сполучає їх в один цілий механізм. Це
зовнішнє відношення готового продукту до його різнорідних
елементів тут, як і в подібних роботах, лишає комбінацію частинних
робітників у тій самій майстерні випадковою. Самі
частинні праці знов таки можуть провадитись як незалежні
одне від одного ремества, як от у кантоні Ваадт та Невшатель,
тимчасом як у Женеві, приміром, існують великі мануфактури
годинників, тобто існує безпосередня кооперація частинних робітників
під командою одного капіталу. І в останньому випадку
цифербляти, пружини й коробки рідко виготовлюють у самій
мануфактурі. Комбіноване мануфактурне виробництво зисковне
тут лише за виняткових умов, бо конкуренція поміж робітниками,
що хочуть працювати вдома, надзвичайно велика, роздрібнення
продукції на масу гетерогенних процесів мало дає змоги застосовувати
спільні засоби праці; крім того, при роздрібненій фабрикації
капіталіст заощаджує собі видатки на робітні приміщення
й т. д.\footnote{
Женева в 1854 р. випродукувала 80.000 годинників, що не складає
навіть і п’ятої частини продукції годинників кантону Невшатель.
Chaux-de-Fonds, що його можна розглядати як єдину мануфактуру годинників,
сам щорічно дає удвоє більше, ніж Женева. Від 1850 і до 1861 р.
Женева постачила 750.000 годинників. Див. «Report from Geneva on
the watch Trade» в «Reports by H. M. ’s Secretaries of Embassy and Legation
on the Manufactures, Commerce etc.». № 6 1863. Якщо відсутність
зв’язку між процесами, на які розпадається продукція складних про-
} Однак становище й цих частинних робітників, які працюють
\index{i}{0278}  %% посилання на сторінку оригінального видання
хоч і вдома, але на капіталіста (Fabrikant, établisseur),
геть цілком відмінне від становища самостійного ремісника, що
працює для своїх власних клієнтів.\footnote{
В годинникарстві, в цьому клясичному прикладі гетерогенної
мануфактури, можна дуже докладно вивчити згадані вище диференціяцію
та спеціялізацію робочих інструментів, що випливають із розчленування
ремісничої праці.
}

Другий рід мануфактури, її закінчена форма, продукує
вироби, що перебігають зв’язані між собою фази розвитку, певний
ряд послідовних процесів, як от, приміром, дріт у голчаній
мануфактурі, який проходить через руки 72, а то й 92 специфічних
частинних робітників.

Оскільки така мануфактура комбінує ремества первісно розпорошені,
вона зменшує просторове відокремлення між окремими
фазами продукції виробу. Час на перехід його з однієї
стадії до одної скорочується, і так само зменшується праця, що
упосереднює ці переходи.\footnote{
«За такого тісного співжиття людей на транспортування неодмінно
мусить витрачатись менше часу» («In so close a cohabitation
of the People, the carriage must needs be less»). («The Advantages of the
East-India Trade», London 1720, p. 106).
} Порівняно з ремеством, таким способом
досягається вищої продуктивної сили, і цей виграш виникає
саме з загального кооперативного характеру мануфактури.
З другого боку, властивий мануфактурі принцип поділу праці
зумовлює ізоляцію різних фаз продукції, що усамостійнюються
одна проти одної як відповідна кількість частинних праць ремісничого
характеру. Встановлення і зберігання зв’язку поміж
ізольованими функціями вимагає постійного транспортування
виробу з одних рук до одних та з одного процесу до одного.
З погляду великої промисловости ця обставина виступає як характеристична
та іманентна принципові мануфактури обмеженість,
що удорожчує продукцію.\footnote{
«Ізоляція різних стадій мануфактури, що постає в наслідок вживання
ручної праці, надзвичайно збільшує витрати продукції, при чому
втрата виникає, головне, з самого лише переходу від одного процесу
до одного» («The isolation of the different stages of manufacture consequent
upon the employment of the manual labour adds immensely to
the cost of production, the loss mainly arising from the mere removals from
one process to another»). («The Industry of Nations», London 1855,
Part. II, p. 200).
}

Коли подивимось на певну кількість сировинного матеріялу,
приміром, ганчірок у паперовій мануфактурі або дроту в голчаній
мануфактурі, то побачимо, що цей матеріял перебігає в руках
різних частинних робітників почерговий щодо часу ряд фаз

дуктів, сама по собі дуже утруднює перетворення таких мануфактур на
машинове виробництво великої промисловости, то при продукції годинників
сюди долучаються ще дві інші перешкоди: дрібність і тендітність
їхніх елементів та їхній люксусовий характер, отже і різноманітність їх,
наприклад, ліпші лондонські фірми протягом цілого року ледве чи виробляють
тузінь годинників, які були б подібні один до одного. Фабрика
годинників Vacheron and Constantin, що з успіхом вживає машин,
дає щонайбільше три-чотири відміни годинників, різних щодо величини
й форми.
\parbreak{}  %% абзац продовжується на наступній сторінці

\parcont{}  %% абзац починається на попередній сторінці
\index{i}{0279}  %% посилання на сторінку оригінального видання
продукції, аж поки набере своєї остаточної форми. Навпаки, коли
розглядатимемо майстерню як сукупний механізм, то виявиться,
що сировинний матеріял перебуває одночасно в усіх своїх фазах
продукції. Однією частиною своїх багатьох озброєних знаряддям
рук колективний робітник, скомбінований із частинних робітників,
тягне дріт, тоді як іншими руками та знаряддям він одночасно
вирівнює його, іншими ріже його, загострює й~\abbr{т. д.} Послідовність
різних стадій процесу в часі перетворюється на одночасність
існування цих стадій у просторі. Звідси виготовлення більшої
кількости товару протягом того самого часу\footnote{
«Він (поділ праці) зумовлює економію часу, розчленовуючи
працю на різні операції, які можна виконувати всі одночасно\dots{} В наслідок
одночасного виконування всіх тих різних процесів праці, які одна
людина мусить провадити послідовно, один по одному, утворюється,
наприклад, можливість виробити велике число цілком закінчених
шпильок протягом того самого часу, якого треба на те, щоб обрізати й
загострити одну шпильку». («It (the division of labour) produces also
an economy of time, by separating the work into its different branches, all
of which may be carried on into execution at the same moment\dots{} By carrying
on all the different processes at once, which an individual must have
executed separately, it becomes possible to produce a multitude of pins
for instance completely finished in the same time as a single pin might
have been either cut or pointed»). (\emph{Dugald Stewart}: Works, edited
by Sir W.~Hamilton, Edinburgh 1855, vol. Ill, «Lectures on Political
Economy», p. 319).
}. Ця одночасність
виникає, щоправда, з загальної кооперативної форми цілого
процесу, але мануфактура не тільки находить уже готові умови
кооперації, вона почасти лише сама створює їх, розчленовуючи
ремісничу працю. З другого боку, вона досягає цієї суспільної
організації процесу праці, лише міцно приковуючи того самого
робітника до того самого деталю.

Що частинний продукт кожного частинного робітника разом
з тим є лише осібний ступінь у розвитку того самого продукту,
то один робітник постачає іншому, або одна група робітників
іншій сировинний матеріял. Результат праці одного становить
вихідний пункт для праці іншого. Отже, один робітник тут безпосередньо
дає працю іншим. Робочий час, доконечний для
досягнення у кожному частинному процесі корисного ефекту,
що його має на меті цей процес, установлюється з досвіду, і цілий
механізм мануфактури ґрунтується на тій передумові, що протягом
даного робочого часу досягається якогось даного результату.
Лише за цієї передумови різні процеси праці, процеси, що
один одного доповнюють, можуть відбуватися безперервно,
одночасно та просторово один поруч одного. Ясно, що ця безпосередня
взаємна залежність праць, а тому й робітників примушує
кожного окремого робітника витрачати на свою функцію лише
доконечний робочий час, і таким чином досягається цілком іншої
безперервности, одноманітности, правильности, порядку\footnote{
«Що більше різноманітности серед робітників мануфактури\dots{} то
більші порядок і реґулярність у кожній роботі, то менше мусить витрачатись
на неї часу, то менше мусить витрачатись праці» («The more variety
of artists to every manufacture\dots{} the greater the order and regularity of
every work, the same must needs be done in less time, the labour must
be less»). («The Advantages of the East-India Trade», London 1720, p. 68).
}, а особливо
\index{i}{0280}  %% посилання на сторінку оригінального видання
й інтенсивности праці, аніж у незалежному реместві або
навіть у простій кооперації. Та обставина, що на продукцію
якогось товару витрачається лише суспільно-доконечний робочий
час, за товарової продукції з’являється взагалі як зовнішній
примус конкуренції, бо, висловлюючись поверхово, кожний
поодинокий продуцент мусить продавати товар за його ринковою
ціною. Навпаки, у мануфактурі виготовлення даної кількости
продукту протягом даного часу стає технічним законом самого
процесу продукції\footnote{
Проте цього результату мануфактурне підприємство в багатьох
галузях продукції доходить лише недосконало, бо мануфактура не вміє
з певністю контролювати загальні хемічні й фізичні умови процесу продукції.
}.

Однак різні операції потребують неоднакового часу й тому
дають протягом однакового часу неоднакові кількості частинних
продуктів. Отже, коли той самий робітник день-у-день повинен
виконувати завжди лише ту саму операцію, то для різних
операцій мусить бути вжите відносно різне число робітників,
приміром, у черенковій мануфактурі на одного ґлянсувальника
четверо ливарників та двоє відламувачів, бо один ливарник виливає
за годину \num{2.000} черенків, один відламувач одламує \num{4.000}, а
ґлянсувальник ґлянсує начисто \num{8.000}. Тут принцип кооперації
повертається назад до своєї найпростішої форми, до одночасного
вживання багатьох робітників, що роблять однорідну роботу,
але цей принцип стає тепер виразом певного органічного відношення.
Отже, мануфактурний поділ праці не тільки спрощує
і урізноманітнює якісно відмінні органи суспільного колективного
робітника, а й утворює тривале математичне відношення
для кількісного обсягу цих органів, тобто для відносного числа
робітників або для відносної величини робітничих груп у кожній
окремій функції. Разом з якісним розчленуванням він розвиває
й кількісну норму (Regel) та пропорційність суспільного процесу
праці.

Якщо для певного маштабу продукції на основі досвіду встановлено
якнайвідповіднішу пропорційність між різними групами
частинних робітників, то поширити цей маштаб можна лише тоді,
коли вжити кратну кількість робітників кожної з цих окремих
груп\footnote{
«Якщо досвід залежно від осібної природи продукту кожної
мануфактури показав так найвигідніший спосіб поділу фабрикації на
частинні операції, як і потрібне для цього число робітників, то всі підприємства,
що не вживатимуть точної кратної кількости цього числа
робітників, будуть продукувати з більшими витратами\dots{} Це одна з
причин колосального поширення промислових підприємств». (\emph{Ch.~Babbage}:
«On the Economy of Machinery», London 1832. ch. XXI, p. 172, 173).
}. До цього долучається ще й те, що той самий індивід може
виконувати деякі роботи однаково добре, все одно, чи провадяться
вони у великому чи в малому розмірі, приміром, роботи догляду,
транспортування частинних продуктів з однієї продукційної фази
\parbreak{}  %% абзац продовжується на наступній сторінці

\parcont{}  %% абзац починається на попередній сторінці
\index{i}{0281}  %% посилання на сторінку оригінального видання
до іншої і т. д. Отже, усамостійнення цих функцій, або доручення
цих функцій окремим робітникам, стає вигідним лише із збільшенням
числа занятих робітників, але це збільшення мусить
одразу охопити всі групи в однаковій пропорції.

Окрема група, певне число робітників, що виконують ту саму
частинну функцію, складається з однорідних елементів та становить
осібний орган цілого механізму. Однак у різних мануфактурах
сама група є розчленоване робоче тіло, тимчасом як цілий
механізм утворюється через повторювання або помножування
цих продуктивних елементарних організмів. Візьмімо, приміром,
мануфактуру пляшок. Вона розпадається на три посутньо відмінні
фази. Перша, підготовча фаза: виготовлення скляної маси,
тобто суміші піску, вапна й т. ін., та перетоплення цієї суміші
на плинну скляну масу.\footnote{
В Англії піч для перетоплювання відокремлена від печі для виробу
скляних продуктів, але в Бельгії, наприклад, та сама піч служить
для обох процесів.
} В цій першій фазі працюють різні
частинні робітники, так само як і в кінцевій фазі: вийманні
пляшок із сушарні, сортуванні та пакуванні їх і т. ін. Посередині
між цими двома фазами маємо власне виробництво пляшок, або
перероблення плинної скляної маси. Коло того самого отвору
печі для виробу скляних продуктів працює група, яка в Англії
зветься «hole» (діра) і складається з одного bottle maker’a,\footnote*{
— пляшкаря. \emph{Ред.}
}
або finisher’a,\footnote*{
— робітника, що остаточно закінчує продукт. \emph{Ред.}
} одного blower’a,*** одного gatherer’a,\footnote*{
— збирача. \emph{Ред.}
}одного
putter up\footnote*{
— укладача. \emph{Ред.}
} або whetter off\footnote*{
— шліфувальника. \emph{Ред.}
} і одного taker’а.\footnote*{
— приймача. \emph{Ред.}
}
Ці п’ятеро частинних робітників становлять стільки ж окремих
органів одним-одного робочого тіла, яке може функціонувати лише
як єдність, отже, лише через безпосередню кооперацію п’ятьох.
Якщо бракує однієї з п’ятьох складових частин цього тіла, то
воно паралізується. Але та сама піч для перетоплення скла має декілька
отворів, — в Англії, наприклад, 4--6, — і в кожному з них
міститься глиняний топильний тигель із плинним склом; коло кожного
з них працює своя власна робоча група, складена з таких самих
п’ятьох робітників. Організація кожної поодинокої групи базується
тут безпосередньо на поділі праці, тимчасом як зв’язок поміж
різними однорідними групами полягає в простій кооперації,
що дає можливість через спільне вживання ощадніше використовувати
один із засобів продукції, в даному випадку, піч для перетоплення
скла. Кожна така піч з її 4--6 групами становить гуту,
а скляна мануфактура охоплює кілька таких гут разом із приладдям
і робітниками для початкової й кінцевої фази продукції.

Нарешті, мануфактура, так само, як вона виникає почасти
з комбінації різних реместв, може розвинутися на комбінацію

* * * — надимача. \emph{Ред.}
\parbreak{}  %% абзац продовжується на наступній сторінці

\parcont{}  %% абзац починається на попередній сторінці
\index{i}{0282}  %% посилання на сторінку оригінального видання
різних мануфактур. Великі англійські гути, приміром, сами
фабрикують свої глиняні топильні тиглі, бо від їхньої якости
посутньо залежить, вдасться чи не вдасться виготовлення продукту.
Мануфактура засобу продукції сполучається тут із мануфактурою
продукту. Навпаки, мануфактура продукту може бути
сполучена з мануфактурами, що для них цей продукт служить
знову за сировинний матеріял або що з їхніми продуктами він
пізніше сполучається. Так, наприклад, мануфактура кремінного
скла комбінується із шліфуванням скла та з мосяжоливарством;
останнє служить для виготовлення металевих оправ різноманітних
скляних товарів. Тоді різні комбіновані мануфактури
становлять більш або менш просторово відокремлені відділи цілої
мануфактури, а разом з тим вони є незалежні один від одного
процеси продукції, кожний із своїм власним поділом праці. Хоч
комбінована мануфактура дає деякі вигоди, проте, доки вона
ґрунтується на своїй власній основі, вона не набуває дійсної
технічної єдности. Ця єдність постає лише тоді, коли мануфактура
перетворюється на машинове виробництво.

Мануфактурний період, що незабаром проголошує зменшення
робочого часу, доконечного для продукції товарів, за свій свідомий
принцип,\footnote{
Між іншим, це можна побачити з праць \emph{W. Petty, John Bellers,
Andrew Yarranton}: «The Advantages of the East-India Trade» та \emph{J. Vanderlint}.
} спорадично розвиває й уживання машин, особливо
для деяких простих початкових процесів, які можна провадити
лише у великому маштабі та при значній витраті сили. Приміром,
у паперовій мануфактурі здавна почали перемелювати ганчір’я
на паперових млинах, а в металюрґії розробляти руду на так
званих млинах-дробарках.\footnote{
Ще наприкінці XVI віку Франція користується ступою та решетом,
щоб роздробляти та перемивати руди.
} Найелементарнішу форму всякої
машини залишила у спадщину Римська імперія у формі водяного
млина.\footnote{
Усю історію розвитку машин можна простежити на історії млинів
на збіжжя. В Англії ще й досі фабрику називають mill (млин). У
німецьких технологічних творах перших десятиліть XIX віку ми теж
ще подибуємо вираз Mühle (млин) не тільки на означення тих машин,
що їх рухають за допомогою сил природи, але й для всяких мануфактур,
що вживають механічних апаратів.
} Ремісничий період передав у спадщину великі винаходи,
як от компас, стрільний порох, друкарство та автоматичний
годинник. Однак взагалі і в цілому машини відіграють ту другорядну
ролю, яку приписує їм Адам Сміс, поряд поділу праці.\footnote{
Як побачимо докладніше з четвертої книги цього твору, А. Сміс
не виставив жодної нової тези щодо поділу праці. Але що характеризує
його як політико-економа, який резюмує мануфактурний період, так
це той наголос, що його він робить на поділі праці. Та підрядна роля,
що її А. Сміс приписує машинам, викликала на початку великої промисловости
заперечення Лодерделя, а пізніше, за розвиненішої епохи —
заперечення Юра. А. Сміс переплутує також диференціяцію інструментів,
у якій велику ролю відігравали сами частинні робітники мануфактури,
— з винаходом машин. Але тут не мануфактурні робітники
відіграють ролю, а вчені, ремісники, а то й селяни (Brindley) і т. ін.
}
\parbreak{}  %% абзац продовжується на наступній сторінці

\parcont{}  %% абзац починається на попередній сторінці
\index{ii}{0283}  %% посилання на сторінку оригінального видання
закуп його робочої сили, та з другої частини, що протягом її він продукує
додаткову вартість (зиск, ренту й т. ін.). — Саме та щоденна праця,
яку витрачається на репродукцію засобів продукції, і вартість якої
розкладається на заробітну плату й додаткову вартість, — саме ця праця
реалізується в нових засобах продукції, які заміщують сталу частину
капіталу, витрачену на продукцію засобів споживання.

Головні труднощі, що з них більшу частину уже розв’язано в попередньому
викладі, постають тоді, коли досліджують не акумуляцію, а просту
репродукцію. Тим то А. Сміс (книга II), як і раніше Кене (Tableau
économique), виходять з простої репродукції, скоро мова йде про рух
річного продукту суспільства та його репродукцію, упосереднену циркуляцією.

\subsubsection{Розклад мінової вартостіі на $v \dplus{} m$ у Сміса}

Догму А. Сміса, ніби ціна або мінова вартість (exchangeable value)
кожного поодинокого товару, — отже, і всіх товарів, сукупність яких
становить річний продукт суспільства (він слушно припускає всюди капіталістичну
продукцію), — складається з трьох складових частин (component
parts) або розкладається на (resolves itself into): заробітну плату,
зиск і ренту, можна звести на те, що товарова вартість — $v \dplus{} m$, тобто
дорівнює вартості авансованого змінного капіталу плюс додаткова вартість.
Це зведення зиску й ренти до того загального й єдиного, що
ми звемо m, ми можемо зробити саме з виразного дозволу А. Сміса, як
це видно з наступних цитат, де ми спочатку не звертаємо уваги на всебічні
пункти, тобто на всі позірні або справжні відхили від догми, що за
нею товарова вартість складається виключно з елементів, які ми позначаємо
як $v \dplus{} m$.

В мануфактурі „вартість, що її робітники долучають до матеріялів,
розкладається на\dots{}\dots{} дві частини, що з них одна оплачує їхню заробітну
плату, а друга — зиск їхньому хазяїнові на ввесь капітал, авансований ним
на матеріял і на заробітну плату“. (Кн. І, розд. 6, стор. 41). — „Хоч
мануфактуристові“ (мануфактурному робітникові) „його заробітну плату
й авансує підприємець, але в дійсності вона нічого не коштує цьому
останньому, бо звичайно вартість цієї заробітної плати, разом з зиском,
повертається (restored) в збільшеній вартості предмету, що на нього застосовано
працю „мануфактуриста“. (Кн. II, розд. З, стор. 221). Частина капіталу
(Stock), витрачена на „утримання продуктивної праці\dots{} після того як вона
служила йому (підприємцеві) в функції капіталу\dots{} становить їх (робітників)
дохід“. (Кн. II, розд. З, стор. 223).

А. Сміс у щойно цитованому розділі виразно каже: „Ввесь річний
продукт землі та праці кожної країни\dots{} сам собою (naturally) розпадається
на дві частини. Одну з цих частин, і часто найбільшу, насамперед
призначається замістити капітал і відновити засоби існування, сировинні
матеріяли й готові продукти, взяті з капіталу. Другу частину призначається
утворити дохід, чи то для власника цього капіталу, як зиск на
його капітал, чи то дохід для когобудь іншого, як ренту з його
\parbreak{}  %% абзац продовжується на наступній сторінці

\parcont{}  %% абзац починається на попередній сторінці
\index{i}{0284}  %% посилання на сторінку оригінального видання
до якоїсь однобічної функції та зв’язується з нею на цілий його
вік, то, з другого боку, різні операції праці в такій самій мірі
пристосовується до тієї ієрархії природних і придбаних здібностей\footnote{
Доктор Юр у своїй апотеозі великої промисловости гостріше
схоплює специфічний характер мануфактури, ніж попередні економісти,
що не мали його полемічного інтересу, і навіть ніж його сучасники,
як от Беббедж, який хоч і перевищує його як математик і механік, а проте
розглядає велику промисловість, власне кажучи, лише з погляду мануфактури.
Юр зауважує: «Пристосування робітника до кожної окремої
операції становить суть поділу праці». З другого боку, цей поділ він називає
«пристосуванням праць до різних індивідуальних здібностей» і характеризує
нарешті цілу мануфактурну систему як «систему градацій
відповідно до ступеня вправности», як «поділ праці за різними ступенями
вправности» й~\abbr{т. д.} (\emph{Ure}: «Philosophy of Manufacture», p. 19 —
23 і далі).
}. Тимчасом кожний процес продукції потребує певних
простих маніпуляцій, до яких здатна кожна перша-ліпша людина.
І ці маніпуляції звільняються тепер від їхнього рухомого
зв’язку з змістовнішими моментами діяльности й кам’яніють у
виключні функції.

Тим-то мануфактура утворює в кожному реместві, яке вона
захоплює, клясу так званих ненавчених (ungeschickter) робітників,
яких ремісниче виробництво строго виключало. Розвиваючи
до віртуозности одну якусь цілком зоднобічнену спеціяльність
коштом загальної працездатности, вона вже й самий брак усякого
розвитку починає робити спеціяльністю. Поруч ієрархічних
щаблів постає простий поділ робітників на навчених та ненавчених
(geschickte und ungeschickte). Для останніх витрати
на навчання відпадають цілком, для перших вони нижчі порівняно
з ремісником, бо функції їхні простіші. В обох випадках вартість
робочої сили спадає\footnote{
«Кожний професійний робітник\dots{} дістаючи, в наслідок того,
що він постійно виконує ту саму працю, змогу вдосконалюватися, стає
дешевший» («Each handicraftsman, being\dots{} enabled to perfect to himself
by practice in one point, became\dots{} a cheaper workman»). (\emph{Ure}:
«Philosophy of Manufacture», p. 19).
}. Винятки бувають лише остільки, оскільки
розчленування робочого процесу створює нові складні функції,
яких у ремісничому виробництві або зовсім не було, або якщо вони
й були, то не в такому самому розмірі. Відносне зневартнення
робочої сили, яке виникає в наслідок зникнення або зменшення
витрат на навчання, безпосередньо спричинюється до підвищеного
зростання вартости капіталу, бо все, що скорочує час, доконечний
для репродукції робочої сили, поширює сферу додаткової
праці.

\subsection{Поділ праці всередині мануфактури та поділ праці
всередині суспільства}

Ми розглянули спочатку походження мануфактури, потім
її прості елементи — частинного робітника і його знаряддя, —
нарешті, її цілий механізм. Тепер розгляньмо коротко відношення
\parbreak{}  %% абзац продовжується на наступній сторінці

\parcont{}  %% абзац починається на попередній сторінці
\index{ii}{0285}  %% посилання на сторінку оригінального видання
дохід, кінець-кінцем, походить від одного з них“ (кн. І, розд. 6,
стор. 48), — то в цих словах зібрано до купи всілякі qui pro quo.

1) Всі члени суспільства, що безпосередньо — хоч працюючи, хоч
без праці — не функціонують у процесі репродукції, можуть одержати
свою пайку річного товарового продукту, тобто засоби свого споживання,
насамперед лише з рук тих кляс, що їм у першу чергу дістається
продукт: з рук продуктивних робітників, промислових капіталістів і
землевласників. В цьому розумінні їхні доходи матеріяльно походять із
заробітної плати (продуктивних робітників), зиску й земельної ренти, а
тому вони протистоять цим первинним доходам, як доходи похідні.
Однак, з другого боку, одержувачі цих похідних в такому розумінні доходів
здобувають їх в наслідок своєї суспільної функції як королі,
попи, професори, повії, вояки тощо; це дає їм змогу вбачати в цих
своїх функціях первинні джерела їхніх доходів.

2) І тут доходить кульмінаційного пункту чудна помилка А. Сміса:
почавши з правильного визначення складових частин вартости товару і
суми тих новоспродукованих вартостей, що втілені в цих частинах; з’ясувавши
потім, як ці складові частини утворюють стільки ж різних
джерел доходу\footnote{
Я подаю це речення буквально, як воно є в рукопису, хоч у даному зв’язку
воно ніби суперечить і попередньому й безпосередньо дальшому. Цю позірну
суперечність розв’язується далі під цифрою 4: „Капітал і дохід в А. Сміса“. — Ф. Е.
}, виснувавши таким чином доходи з вартости, він іде
потім зворотним напрямком — і це лишається в нього домінантним уявленням
— і перетворює доходи з „складових частин“ (component parts)
на „первісні джерела всякої мінової вартости“, розкриваючи цим
широко двері вульґарній економії. (Див. нашого Рошера).

3) Стала частина капіталу

Подивімось тепер, яким чаклуванням намагається А. Сміс винищити
в товаровій вартості сталу частину вартости капіталу.

„Частина ціни зерна, напр., оплачує ренту землевласника“. Походження
цієї складової частини вартости так само не має чинення до тієї
обставини, що цю частину виплачується землевласникові, і що вона для
нього становить дохід у формі ренти, як походження інших складових
частин вартости не має чинення до того, що вони як зиск і заробітна
плата становлять джерела доходу.

„Друга частина оплачує заробітну плату й утримання робітників“
(і робочої худоби! — додає він до цього), „що були зайняті в продукції
зерна, а третя частина оплачує зиск фармера. Ці три частини, як здається
(seem, в дійсності так здається), „або безпосередньо, або кінець-кінцем
становлять усю ціну зерна“\footnote{
Ми вже зовсім не кажемо про те, що Адамові особливо не пощастило з
його прикладом. Вартість зерна тільки тому розкладається на заробітну плату,
зиск і ренту, що корм, спожитий робочою худобою, подано як заробітну плату
робочої худоби, а саму робочу худобу — як найманих робітників, а тому й найманого
робітника — як робочу худобу. (Додаток з рукопису II).
}. Вся ця ціна, тобто визначення її
\parbreak{}  %% абзац продовжується на наступній сторінці


\index{iii1}{0286}  %% посилання на сторінку оригінального видання
З одного боку, такий торговельний робітник є такий самий
найманий робітник, як і всякий інший. Поперше, оскільки його
праця купується на змінний капітал купця, а не на ті гроші, що
витрачаються як дохід; отже, оскільки вона купується не для
особистих послуг, а з метою самозростання вартості авансованого
на це капіталу. Подруге, оскільки вартість його робочої
сили, а тому і його заробітна плата, визначається, як і в усіх
інших найманих робітників, витратами виробництва і репродукції
його специфічної робочої сили, а не продуктом його праці.

Але між ним і робітниками, безпосередньо вживаними промисловим
капіталом, мусить існувати така сама ріжниця, яка існує
між промисловим капіталом і торговельним капіталом, а тому й
між промисловим капіталістом і купцем. Через те що купець, як
простий агент циркуляції, не виробляє ні вартості, ні додаткової
вартості (бо та добавна вартість, яку він додає до товарів своїми
витратами, зводиться до додання вартостей, які вже раніш існували,
хоч тут нав’язується питання: яким чином він удержує, зберігає
цю вартість свого сталого капіталу?), то й торговельні робітники,
вживані ним для виконання тих самих функцій, не можуть
безпосередньо створювати для нього додаткову вартість. Тут, як
і тоді, коли справа йшла про продуктивних робітників, ми припускаємо,
що заробітна плата визначається вартістю робочої сили,
отже, купець не збагачується відрахуваннями з заробітної плати,
так що в обрахунок своїх витрат він вносить не таке авансування
на працю, яке оплачувало б її тільки почасти, — іншими словами,
він збагачується не тим, що обшахровує своїх прикажчиків і т. п.

Труднощі при вивченні питання про торговельних найманих
робітників полягають зовсім не в тому, щоб пояснити, яким чином
вони виробляють зиск безпосередньо для свого наймача, хоч
безпосередньо вони не виробляють додаткової вартості (а зиск
є тільки перетворена форма її). Це питання в дійсності розв’язане
вже загальним аналізом торговельного зиску. Подібно до
того, як промисловий капітал одержує зиск в наслідок того,
що продає вміщену в товарах і реалізовану працю, за яку він
не заплатив ніякого еквіваленту, цілком так само і торговельний
капітал одержує зиск в наслідок того, що він оплачує продуктивному
капіталові не всю неоплачену працю, яка міститься
в товарі (в товарі, оскільки капітал, витрачений на його виробництво,
функціонує як відповідна частина сукупного промислового
капіталу); навпаки, при продажу товарів він примушує
заплатити собі за цю неоплачену ним частину праці, яка ще
міститься в товарах. Відношення купецького капіталу до додаткової
вартості інше, ніж відношення промислового капіталу.
Останній виробляє додаткову вартість шляхом безпосереднього
привласнювання неоплаченої чужої праці. Перший привласнює
собі частину цієї додаткової вартості, примушуючи промисловий
капітал відступати йому цю частину.

Тільки за допомогою своєї функції реалізації вартостей торговельний
\index{iii1}{0287}  %% посилання на сторінку оригінального видання
капітал функціонує в процесі репродукції як капітал,
і тому як функціонуючий капітал одержує частину з виробленої
сукупним капіталом додаткової вартості. Для кожного окремого
купця маса його зиску залежить від маси капіталу, яку він
може вжити в цьому процесі, а він тим більше може вжити
з неї на купівлю й продаж, чим більша є неоплачена праця його
прикажчиків. Саму функцію, в силу якої його гроші є капітал,
торговельний капіталіст примушує здебільшого виконувати своїх
робітників. Неоплачена праця його прикажчиків, хоч вона й не
створює додаткової вартості, створює однак йому привласнення
додаткової вартості, що своїм результатом є для цього капіталу
цілком те саме; отже, ця неоплачена праця є для нього
джерелом зиску. Інакше торговельне підприємство ніколи не
можна було б вести у великому масштабі, ніколи не можна
було б вести по-капіталістичному.

Подібно до того, як неоплачена праця робітника безпосередньо
створює для продуктивного капіталу додаткову вартість, цілком
так само неоплачена праця торговельних найманих робітників створює
для торговельного капіталу участь в цій додатковій вартості.

Трудність полягає ось у чому: оскільки робочий час і праця
самого купця не є вартостетворча праця, хоч і створює йому
участь у виробленій вже додатковій вартості, то як стоїть справа
з тим змінним капіталом, який він витрачає на купівлю торговельної
робочої сили? Чи слід цей змінний капітал прирахувати
як витрати до авансованого купецького капіталу? Якщо ні, то це,
видимо, суперечить законові вирівнення норми зиску; який капіталіст
авансовував би 150, коли б він міг рахувати як авансований
капітал тільки 100? Якщо ж слід, то це, видимо, суперечить
сутності торговельного капіталу, бо цей сорт капіталу
функціонує як капітал не в наслідок того, що він подібно до
промислового капіталу приводить в рух чужу працю, а в наслідок
того, що він сам працює, тобто виконує функції купівлі й продажу,
і саме тільки за це і цим переносить на себе частину
додаткової вартості, створеної промисловим капіталом.

(Отже, нам треба дослідити такі пункти: змінний капітал купця;
закон необхідної праці в циркуляції; яким чином праця купця
зберігає вартість його сталого капіталу; роль купецького капіталу
в сукупному процесі репродукції; нарешті, роздвоєння на
товарний капітал і грошовий капітал — з одного боку, і на товарно-торговельний
капітал і грошево-торговельний капітал —
з другого боку.)

Якби кожний купець мав лиш стільки капіталу, скільки він
міг би обертати особисто своєю власною працею, то мало б
місце безконечне роздрібнення купецького капіталу; це роздрібнення
мусило б зростати в міру того, як продуктивний капітал,
з розвитком капіталістичного способу виробництва, виробляє
в дедалі більшому масштабі і оперує дедалі більшими масами.
Отже, ми мали б зростаючу невідповідність між тим і другим.
\parbreak{}  %% абзац продовжується на наступній сторінці

\parcont{}  %% абзац починається на попередній сторінці
\index{i}{0288}  %% посилання на сторінку оригінального видання
захоплює поряд економічної сфери усякі інші сфери суспільства
та всюди закладає основу тому розвиткові професіоналізму,
спеціялізації, тій парцеляції людини, яка примусила вже А. Ферґюсона,
вчителя Адама Сміса, вигукнути: «Ми є нація гелотів,
і немає серед нас вільних людей».\footnote{
A. Ferguson: «History of Civil Society», Edinburgh 1767, Part IV,
sect II, p. 285.
}

Однак, не зважаючи на численні аналогії та зв’язки між поділом
праці всередині суспільства й поділом праці всередині майстерні,
обидва вони не тільки щодо ступеня, але й суттю відмінні.
Безперечно, найяскравіше ця аналогія виступає там, де внутрішній
зв’язок сплітає різні галузі продукції. Скотар, приміром,
продукує шкури, гарбар вичиняє шкуру, швець із вичиненої
шкури робить чоботи. Кожен продукує тут частинний продукт,
а остання готова форма — це комбінований продукт їхніх окремих
праць. Сюди треба додати ще різноманітні галузі праці, які
постачають засоби продукції скотареві, гарбареві та шевцеві.
Можна собі уявити разом з А. Смісом, що цей суспільний поділ
праці відрізняється від мануфактурного лише суб’єктивно, а саме
лише для спостерігача, який тут, у мануфактурі, одним поглядом
просторово охоплює різноманітні частинні праці, тоді як там розкиданість
їх по великих просторах та велике число робітників,
занятих у кожній окремій галузі, затемнюють зв’язок.\footnote{
У справжніх мануфактурах, каже він, поділ праці видається
більшим, бо «робітники, що працюють у кожній з різних галузей праці,
часто можуть бути сполучені в тій самій майстерні, і таким чином всіх
їх одразу охоплює око спостерігача. Навпаки, у тих великих мануфактурах
(І), які мають своїм призначенням задовольняти широкі потреби
великої маси людности, кожна окрема галузь вживає такої великої кількости
робітників, що всіх їх неможливо сполучити в тій самій майстерні...
поділ праці тут далеко не так виразно впадає на очі» («... those employed
in every different branch of the work can often be collected into the
same workhouse, and placed at once under the view of the spectator. In
those great manufactures (I), on the contrary, which are destined to supply
the great wants of the great body of the people, every different branch of
the work employs so great a number of workmen, that it is impossible
to collect them all into the same workhouse... the divisions is not near
so obvious»). (A. Smith: «Wealth of Nations», b. I, ch. 1, p. 7, 17).
Знамените місце того самого розділу, яке починається словами:
«Погляньте на життєві умови простого ремісника або поденника в
цивілізованій країні, в країні, що процвітає, і т. д.» («Observe the accomodation
of the most common artificer or day labourer in a civilized and
thriving country etc.»), розділу, де змальовано далі, як безліч різноманітних
галузей промисловості спільними силами задовольняє потреби
простого робітника, — це місце майже слово в слово списано з приміток
Б. де Мандевіля до його «Fable of the Rees, or Private Vices, Publick Benefits».
(Перше видання без приміток 1706 р., друге з примітками 1714 р.).
} Але що
саме встановлює зв’язок між незалежними працями скотаря,
гарбаря, шевця? Те, що їхні продукти існують як товари. Навпаки,
що характеризує мануфактурний поділ праці? Те, що
частинний робітник не продукує жодного товару.\footnote{
«Тут уже немає нічого, що можна було б назвати природною винагородою
за індивідуальну працю. Кожен робітник продукує лише ча-
} Тільки спільний
\index{i}{0289}  %% посилання на сторінку оригінального видання
продукт частинних робітників перетворюється на товар.58а
Поділ праці всередині суспільства упосереднюється купівлею та
продажем продуктів різних галузей праці, зв’язок між частинними
робітниками в мануфактурі — продажем різних робочих
сил тому самому капіталістові, який вживає їх як одну комбіновану
робочу силу. Мануфактурний поділ праці має собі за передумову
концентрацію засобів продукції в руках одного капіталіста,
суспільний поділ праці — розпорошення засобів продукції
між багатьма один від одного незалежними продуцентами товарів.
Тимчасом як у мануфактурі залізний закон пропорційного числа
або пропорційности реґулює (subsumiert) розподіл певних робочих
мас між певними функціями, випадок і сваволя ведуть свою
вередливу гру в поділі товаропродуцентів і їхніх засобів продукції
поміж різними суспільними галузями праці. Правда, різні сфери
продукції постійно намагаються дійти рівноваги, бо, з одного
боку, кожний товаропродуцент мусить виробляти споживну вартість,
отже, задовольняти осібну суспільну потребу, але обсяг
цих потреб кількісно є різний і внутрішній зв’язок сполучає
різні маси потреб в одну природно вирослу систему: з другого ж
боку, закон вартости товарів визначає, скільки з усього робочого
часу, який суспільство має в своєму розпорядженні, може воно
витратити на продукцію кожного окремого роду товару. Але ця
постійна тенденція різних сфер продукції дійти рівноваги виявляється
лише як реакція проти постійного нищення (Aufhebung)
цієї рівноваги. Норма, що її а priori і пляномірно дотримуються

cтину цілого, а через те, що кожна частина сама по собі не має ніякої
вартости або корисности, то тут немає нічого такого, що робітник міг би
взяти і сказати: це мій продукт, це я залишаю собі» («There is no longer
anything which we can call the natural reward of individual labour.
Each labourer produces only some part of a whole, and each part, having no
value or utility of itself, there is nothing on which the labourer can seize,
and say: it is my product, this I will keep for myself»). («Labour defended
against the claims of Capital», London 1825, p. 25). Автором цієї
прегарної праці є цитований вище Т. Годжскін.

58a Примітка до другого видання. — Янкі практично зілюстрували
цю ріжницю між суспільним і мануфактурним поділом праці. Одним
з нових податків, вигаданих у Вашинґтоні за часів громадянської війни,
був акциз у 6\% на «всі промислові продукти». Питання: що таке промисловий
продукт? Законодавець відповідає: Кожна річ є продукт,
«якщо вона зроблена» (when it is made), а вона є зроблена, якщо готова
для продажу. Ось один із багатьох прикладів. Мануфактури Нью-Йорку
і Філадельфії за старих часів «робили» парасолі з усіма їхніми причандалами.
Але що парасоля є mixtum compositum\footnote*{
— складне сполучення. Ред.
} цілком різнорідних
складових частин, то ці останні поволі поробилися продуктами незалежних
одна від одної й розкиданих по різних місцях галузей продукції,
їхні частинні продукти входили як самостійні товари в парасольну
мануфактуру, яка лише сполучає їх в одну цілість. Янкі охристили такого
роду продукти «assembled articles» (збірними продуктами) — назва,
яку вони заслужили собі саме як «збирачі» податків. Так, парасоля «збирала»
спочатку 6\% акцизу з ціни кожного із своїх елементів, а потім
знову 6\% з ціни цілого продукту.
\parbreak{}  %% абзац продовжується на наступній сторінці

\parcont{}  %% абзац починається на попередній сторінці
\index{i}{0290}  %% посилання на сторінку оригінального видання
при поділі праці всередині майстерні, при поділі праці всередині
суспільства діє лише а posteriori як унутрішня, німа, сприймана
лише в барометричних змінах ринкових цін природна доконечність,
що переборює безладну сваволю товаропродуцентів. Мануфактурний
поділ праці припускає безумовний авторитет капіталіста
супроти людей, які є лише прості члени належного йому цілого
механізму; суспільний поділ праці протиставить одного одному
незалежних товаропродуцентів, які не визнають жодного іншого
авторитету, крім конкуренції, крім того примусу, що його справляє
на них тиск їхніх взаємних інтересів, так само як у царстві
тварин bellum omnium contra omnes\footnote*{
— боротьба всіх проти всіх. \emph{Ред.}
} більш або менш підтримує
умови існування всіх видів. Тому та сама буржуазна свідомість,
яка прославляє мануфактурний поділ праці, довічне приковання
робітника до якоїсь детальної операції та безумовну підпорядкованість
частинного робітника капіталові, як організацію
праці, що підвищує її продуктивну силу, — ця сама буржуазна
свідомість так само голосно ганьбить усякий свідомий суспільний
контроль та реґулювання суспільного процесу продукції як замах
на незаймані права власности, на волю та «геніальність» індивідуального
капіталіста, геніяльність що сама себе визначає.
Це дуже характеристично, що ентузіясти-апологети фабричної
системи не вміють нічого сильнішого сказати проти всякої загальної
організації суспільної праці, як тільки те, що така організація
перетворила б ціле суспільство на одну фабрику.

Якщо анархія суспільного поділу праці й деспотія мануфактурного
поділу праці зумовлюють одна одну в суспільстві капіталістичного
способу продукції, то, навпаки, попередні форми
суспільства, в яких відокремлення промислів розвивалося стихійно,
потім кристалізувалось і, нарешті, закріпилося законом,
подають, з одного боку, образ пляномірної та авторитарної
організації суспільної праці, тимчасом як, з другого боку, вони
цілком виключають поділ праці всередині майстерні або розвивають
його в карликовому маштабі, або лише спорадично й випадково.\footnote{
«Можна\dots{} встановити, як загальне правило, що чим менше поділ
праці всередині суспільства підлягає авторитетові, тим дужче поділ
праці розвивається всередині майстерні й тим більше він тут підлягає
авторитетові однієї особи. Таким чином щодо поділу праці авторитет
у майстерні та авторитет у суспільстві зворотно пропорційні один
одному» («On peut\dots{} établir en règlegénérale, que moins l’autorité préside
à la division du travail dans l’intérieur de la société, plus la division du
travail se développe dans l’intérieur de l’atelier, et plus elle y est soumise
à l’autorité d’un seul. Ainsi l’autorité dans l’atelier et celle dans la société
par rapport à division du travail, sont en raison inverse l’une de l’autre»),
(\emph{K. Marx}: «Misère de la Philosophie», Paris 1847. p. 130, 131 — \emph{К. Маркс}:
«Злиденність філософії», Партвидав 1932~\abbr{р.}, стор. 118--119).
}

Приміром, ті старовинні дрібні індійські громади, які почасти
ще й досі існують, ґрунтуються на спільному посіданні
землі, на безпосередній сполуці рільництва й ремества та на
\parbreak{}  %% абзац продовжується на наступній сторінці

\parcont{}  %% абзац починається на попередній сторінці
\index{iii1}{0291}  %% посилання на сторінку оригінального видання
як \emph{купецький} капітал, і з того, подруге, що він примушує заплатити
собі зиск, тому що він функціонує як \emph{капітал}, тобто
тому, що він виконує працю, яка йому, як функціонуючому капіталові,
оплачується зиском. Отже, таке є питання, що його маємо
розв’язати.

Припустім, що $В$ \deq{} 100, $b$ \deq{} 10, а норма зиску \deq{} 10\%. Припустім,
що $К$ \deq{} 0, щоб без потреби не вводити знову в обрахунок
цей елемент купівельної ціни, який сюди не стосується і з
яким ми вже покінчили. Таким чином продажна ціна була б \deq{}
$В \dplus{} р \dplus{} b \dplus{} р (= В \dplus{} Вр' \dplus{} b \dplus{} bp'$, де $р'$ є норма зиску) \deq{}
100 \dplus{} 10 \dplus{} 10 \dplus{} 1 \deq{} 121.

Але коли б $b$ не витрачалось купцем на заробітну плату, —
тому що $b$ виплачується тільки за торговельну працю, отже, за
працю, потрібну для реалізації вартості товарного капіталу, що
його промисловий капітал кидає на ринок, — то справа стояла б
так: купець віддавав би свій час на те, щоб купити або продати
на $В$ \deq{} 100, і ми припустимо, що це весь час, яким він порядкує.
Коли б торговельна праця, що її репрезентує $b$ або 10,
оплачувалась не заробітною платою, а зиском, то вона передбачала
б другий купецький капітал \deq{} 100, бо 10\% від нього \deq{}
$b$ \deq{} 10. Це друге $В$ \deq{} 100 не входило б додатково в ціну товару,
але ці 10\%, звичайно, входили б у неї. Тому було б зроблено дві
операції по 100 \deq{} 200, щоб купити товарів на 200 \dplus{} 20 \deq{} 220.

Через те що купецький капітал є абсолютно не що інше, як
усамостійнена форма частини функціонуючого в процесі циркуляції
промислового капіталу, всі питання, що стосуються до нього,
мусять бути розв’язані таким способом, що проблема ставиться
насамперед у такій формі, при якій властиві купецькому капіталові
явища виступають ще не самостійно, а в безпосередньому
зв’язку з промисловим капіталом, як його відгалуження. Як контора,
в відміну від майстерні, торговельний капітал постійно
функціонує в процесі циркуляції. Отже, саме тут, в конторі
самого промислового капіталіста, треба насамперед дослідити $b$,
що про нього зараз іде мова.

Перш за все ця контора є завжди надзвичайно мала порівняно
з промисловою майстернею. Проте, ясно, що в міру розширення
масштабу виробництва збільшуються торговельні операції,
які доводиться постійно проводити для циркуляції промислового
капіталу, — як для того, щоб продати наявний у формі
товарного капіталу продукт, так і для того, щоб знову перетворювати
в засоби виробництва виручені гроші і всьому вести рахунок.
Обчислення цін, ведення книг, ведення каси, кореспонденція,
— все це належить сюди. Чим більший масштаб виробництва,
тим більші — хоч і ніяк не в такій самій пропорції — купецькі операції
промислового капіталу, отже, тим більша є праця та інші
витрати циркуляції для реалізації вартості й додаткової вартості.
В наслідок цього стає потрібним уживання торговельних
найманих робітників, які становлять власне контору. Видатки на
\parbreak{}  %% абзац продовжується на наступній сторінці

\parcont{}  %% абзац починається на попередній сторінці
\index{i}{0292}  %% посилання на сторінку оригінального видання
знову відбудовуються на тому самому місці з тією самою назвою,\footnote{
61 «У цій простій формі\dots{} мешканці країни жили від споконвічних
часів. Межі сел змінялися рідко; і хоч сами села інколи й зазнавали
ушкоджень, а то навіть і цілковитого спустошення в наслідок воєн,
голоду та пошестей, а все ж вони й далі існували цілі віки під тією самою
назвою, в тих самих межах, з тими самими інтересами, і навіть з тими
самими родинами. Загибель або поділ держав мало турбує мешканців;
доки село лишається цілим, їм байдуже, під чиєю владою воно опинилося,
якому суверенові воно підлягає; їхнє внутрішнє економічне життя
лишається незмінне». («Under this simple form\dots{} the inhabitants of the
country have lived since time immemorial. The boundaries of the villages
have been but seldom altered; and though the villages themselves have
been sometimes injured, and even desolated by war, famine, and disease,
the same name, the same limits, the same interests, and even the same
families, have continued for ages. The inhabitants give themselves non
trouble about the breaking up and division of kingdoms; while the village
remains entire, they care not to what power it is transferred ort to what
sovereign it devolves; its internal economy remains unchanged»). (\emph{Th. Stamford
Raffles}, late Lieut. Gov. of Java: «The History of Java», London 1817.
vol. II, p. 285).
} дає нам ключ до зрозуміння таємниці незмінности азійських
суспільств, незмінности, яка так гостро контрастує з постійним
розпадом і новоутворенням азійських держав та безперервною
зміною династій. Бурі, що відбуваються в хмарній царині політики,
не зачіпають структури основних економічних елементів
суспільства.

Як це вже раніше зазначалось, цехові закони пляномірно
перешкоджали перетворенню цехового майстра на капіталіста,
точно обмежуючи число підмайстрів, яких міг тримати поодинокий
майстер. Так само майстер міг вживати підмайстрів виключно
в тому реместві, де він сам був майстром. Цех із запалом
боронився проти усяких замахів купецького капіталу, —
єдиної вільної форми капіталу, що протистояла йому. Купець
міг купити всякі товари, та тільки не працю як товар. Його терпіли
лише як скупника-розповсюдника (Verleger) продуктів
ремества. Якщо зовнішні обставини викликали поступ у розвитку
поділу праці, то наявні цехи розколювались на підроди абож
поряд старих засновувалися нові цехи, однак не об’єднуючи різних
реместв в одній майстерні. Тим то цехова організація, хоч і в
якій великій мірі зумовлювані нею відокремлення, ізоляція й
розвиток реместв належать до матеріяльних умов існування
мануфактурного періоду, все ж виключала мануфактурний поділ
праці. Взагалі і в цілому, робітник і його засоби продукції
лишались зв’язані одне з одним так само, як слимак із своєю
шкаралупою, і таким чином бракувало першої основи мануфактури
— усамостійнення засобів продукції як капіталу супроти
робітника.

Тим часом як поділ праці в цілому суспільстві, хоч упосереднюється
він товаровим обміном, хоч ні, належить до найрізнорідніших
економічних суспільних формацій, мануфактурний
поділ праці є цілком специфічний витвір капіталістичного способу
продукції.


\index{i}{0293}  %% посилання на сторінку оригінального видання
\subsection{Капіталістичний характер мануфактури}
\vspace{\medskipamount}

\looseness=1
Скупчення значного числа робітників під командою того
самого капіталу становить природний вихідний пункт так кооперації
взагалі, як і мануфактури. Навпаки, мануфактурний поділ
праці розвиває зріст числа вживаних робітників у технічну
доконечність. Тепер мінімум робітників, яких мусить уживати
поодинокий капіталіст, приписується йому наявним поділом
праці. З другого боку, користі з дальшого поділу праці зумовлені
дальшим збільшенням числа робітників, яке можна перевести
лише за кратного збільшення робітників\footnote*{
У французькому виданні це місце подано так: «Щоб мати користь
з дальшого поділу праці, треба не просто збільшити число робітників, а
збільшили їх у кратному відношенні, тобто збільшити їх воднораз,
відповідно до певних пропорцій, по всіх різних групах майстерні» («Le
Capital etc.», v. І, ch. XIV, p. 156). \emph{Ред.}
}. Але разом із
змінною складовою частиною капіталу мусить зростати й стала
його частина, поряд розміру спільних умов продукції, як от
будівлі, печі й~\abbr{т. д.}, особливо мусить зростати й кількість сировинного
матеріялу, та ще й далеко швидше, ніж число робітників.
Маса сировинного матеріялу, споживана протягом даного
часу даною кількістю праці, більшає в тій самій пропорції, в
якій більшає продуктивна сила праці в наслідок поділу праці.
Отже, зріст мінімального розміру капіталу в руках поодинокого
капіталіста або щораз більше перетворювання суспільних засобів
існування та засобів продукції на капітал — це закон, що
виникає з технічного характеру мануфактури\footnote{
«Недосить того, щоб у суспільства був капітал, потрібний для
підрозділу реместв (слід було сказати: «потрібні для цього засоби
існування та засоби продукції»); крім цього, потрібно, щоб цього капіталу
нагромадилося в руках підприємців у досить значних масах, так, щоб
вони мали змогу провадити продукцію у великому маштабі\dots{} Що більше
зростає поділ праці, то щораз значнішого капіталу в знарядді,
сировинному матеріялі й~\abbr{т. д.} потребує постійне застосування того самого числа
робітників»). (\emph{Storch}: «Cours d’Economie Politique». Паризьке видання,
т. І, стор. 250, 251). «Концентрація засобів продукції та поділ
праці так само невіддільні одне від одного, як у політичній царині
концентрація суспільної влади й поділ приватних інтересів» («La
concentration des instruments de production et la division du travail sont aussi
inséparables l’une de l’autre que le sont, dans le régime politique, la
concentration des pouvoirs publics et la division des intérêts privés»).
(\emph{K.~Marx}: «Misère de la Philosophie», Paris 1847, p. 134. — \emph{K.~Маркс}:
«Злиденність філософії», Партвидав 1932~\abbr{р.}, стор. 122).
}.

В мануфактурі, як і в простій кооперації, робоче тіло, що функціонує,
є форма існування капіталу. Суспільний продукційний
механізм, складений із багатьох індивідуальних частинних робітників,
належить капіталістові. Тому продуктивна сила, що виникає
з комбінації праць, видається продуктивною силою капіталу.
Мануфактура у власному значенні не тільки підбиває під команду
й дисципліну капіталу робітника, який раніше був самостійний,
а ще й створює, крім цього, ієрархічну ґрадацію серед самих робітників.
Тимчасом як проста кооперація лишає взагалі і в цілому
\parbreak{}  %% абзац продовжується на наступній сторінці

\parcont{}  %% абзац починається на попередній сторінці
\index{i}{0294}  %% посилання на сторінку оригінального видання
незмінним спосіб праці поодиноких осіб, мануфактура до самої
основи революціонізує його та захоплює індивідуальну робочу
силу в самому її корені. Вона калічить робітника, робить із нього
якусь потвору, штучно активізуючи в ньому розвиток лише
якоїсь однієї частинної вмілости через пригнічення цілого світу
продуктивних хистів і інстинктів, так само як, наприклад, у
штатах Ля Пляти вбивають тварину, щоб добути з неї шкуру
або лій. Не тільки окремі частинні праці розподіляються між
різних індивідів, але й сам індивід розділяється, перетворюється
на автоматичне знаряддя якоїсь частинної праці\footnote{
Дуґалд Стюарт називає мануфактурних робітників «живими
автоматами\dots{} що їх уживають для детальних операцій продукції» («living
automatons\dots{} employed in the details of the work»). (Works, ed. by Sir
W.~Hamilton, Edinburgh 1855. vol. III, «Lectures on Political Economy»,
p. 318).
}, і таким чином
здійснюється безглузда байка Мененія Аґріппи, яка змальовує
людину лише як простий фраґмент її ж власного тіла\footnote{
У коралів кожний окремий індивід дійсно становить шлунок
для цілої групи. Але він постачає їй харчі, а не відбирає їх, як це робив
римський патрицій.
}. Коли первісно робітник продає свою робочу силу капіталові, бо в нього
немає матеріяльних засобів для продукції товару, то тепер сама
його індивідуальна робоча сила відмовляється служити, якщо
її не продано капіталові. Вона функціонує лише в певному сполученні,
яке існує тільки після її продажу, в майстерні капіталіста.
Зробившися через свої природні властивості нездатним робити
щось самостійне, мануфактурний робітник розвиває продуктивну
діяльність тільки як приналежність до майстерні капіталіста\footnote{
«Робітник, що опановує своє ремество в повному обсязі, може
скрізь виконувати своє ремество й добувати засоби існування; другий
(мануфактурний робітник) є не що інше, як аксесуар, що, відокремлений
від своїх товаришів, не має ані здатности, ані незалежности,
і примушений підкорятись законові, що його визнають за доцільне
накинути йому». («L’ouvrier qui porte dans ses bras tout un métier, peut
aller partout exercer son industrie et trouver des moyens de subsister: l’autre
n’est qu’un accessoire qui, séparé de ses confrères, n'a plus ni capacité,
ni indépendance, et qui se trouve forcé d’accepter la loi qu’on juge à
propos de lui imposer»). (\emph{Storch}: «Cours d’Economie Politique». Ed.~Petersbourg 1845, vol. I, p. 204).
}. Як на чолі вибраного народу було написано, що він є власність
Єгови, так само поділ праці накладає на мануфактурного робітника
печать, що таврує його на власність капіталу.

Ті знання, розум і воля, що їх, хоч і в невеликому маштабі,
розвиває в собі самостійний селянин або ремісник — подібно до
того, як дикун усю військову вмілість практикує у формі особистих
хитрощів, — все це тепер потрібне лише для майстерні як цілости.
Духовні потенції продукції поширюють свій маштаб на одному
боці, бо на багатьох вони зникають. Те, що втрачають частинні
робітники, концентрується супроти них у капіталі\footnote{
\emph{A.~Ferguson}: «History of Civil Society», Edinburgb 1767,
part IV, sect. I, p. 281: «Один може виграти те, що другий втратив».
(«The former тау hâve gained what the other has lost»).
}. Мануфактурний
\index{i}{0295}  %% посилання на сторінку оригінального видання
поділ праці протиставить їм духовні потенції матеріяльного
продукційного процесу як чужу власність та силу, що опановує
їх. Цей процес відокремлення починається у простій кооперації,
де капіталіст репрезентує супроти поодиноких робітників єдність
і волю суспільного робочого тіла. Він розвивається в мануфактурі,
яка калічить робітника, перетворюючи його на частинного
робітника. Він завершується у великій промисловості, яка відокремлює
від праці науку як самостійний фактор продукції та
примушує її служити капіталові\footnote{
«Людина науки й продуктивний робітник дуже віддалені один від
одного, і наука замість збільшувати в руках робітника його власні продуктивні
сили для нього самого, майже скрізь поставила себе проти нього.
Знання стає знаряддям, що його можна відокремити від праці та їй протипоставити».
(\emph{W.~Thompson}: «An Inquiry into the Principles of the
Distribution of Wealth», London 1824, p. 274).
}.

В мануфактурі збагачення колективного робітника, а тому
й капіталу, на суспільну продуктивну силу зумовлено збіднінням
робітника на індивідуальні продуктивні сили. «Неуцтво є
мати промисловости, як і забобонів. Розумування й фантазія
підпадають помилкам; але звичка рухати ногою або рукою не
залежить ні від одного, ні від другого. Отже, мануфактури процвітають
найкраще там, де найбільше можна звільнитися від
інтелектуальної роботи, так що майстерня може бути розглядувана
як машина, що її частинами є люди»\footnote{
\emph{A.~Ferguson}: «History of Civil Society», Edinburgh 1767,
part IV, sect. I, p. 280.
}. Справді, в середині
XVIII віку деякі мануфактури вживали охоче напівідіотів,
щоб виконувати деякі прості операції, які, однак, становили
фабричні таємниці\footnote{
\emph{J.~D.~Tuckett}: «А History of the Past and Present State of
the Labouring Population», London 1846, vol. I, p. 148.
}.

«Інтелект великої більшости людей, — каже А.~Сміс, — неминуче
розвивається з їхніх щоденних заняттів і через ці заняття.
Людина, що витратила ціле своє життя на виконування небагатьох
простих операцій\dots{} не має нагоди вправляти свій розум\dots{} Вона
взагалі стає такою тупою і темною, якою тільки й може стати
людська істота». Змалювавши туподумство частинного робітника,
Адам Сміс каже далі: «Одноманітність його життя без усяких
змін губить, ясна річ, і жвавість його розуму\dots{} Вона руйнує
навіть енерґію його тіла та робить його нездатним уживати напружено
й довгочасно своєї сили, хіба лише в тій частинній
праці, до якої його привчили. Таким чином його вправність у його
осібному реместві є, здається, властивість, що він її набуває коштом
своїх інтелектуальних, соціяльних та військових здібностей.
Але саме такий є в кожному промисловому й цивілізованому
суспільстві той стан, що в ньому неминуче мусить опинитись
кожний працюючий бідняк (the labouring poor), тобто велика маса
народу»\footnote{
\emph{A.~Smith}: «Wealth of Nations», b. V, ch. I, art. II. p. 140, 141.
Як учень А.~Ферґюсона, який розвинув усі шкідливі наслідки поділу
праці, А.~Сміс цілком ясно розумів цей пункт. На початку свого твору,
де він ex professo прославляє поділ праці, він лише мимохідь указує на
нього як на джерело суспільних нерівностей. Лише у п’ятій книзі про
державні доходи він репродукує Ферґюсона. В «Misère de la Philosophie»
я вже з’ясував усе потрібне про історичне відношення між Фергюсоном,
А.~Смісом, Лемонтеєм та Сеєм у їхній критиці поділу праці; там
я також уперше подав мануфактурний поділ праці як специфічну форму
капіталістичного способу продукції (стор. 122 і далі) («Злиденність філософії»
, Партвидав 1932~\abbr{р.}, стор. 113 і далі).
}. Щоб перешкодити тому повному знидінню народньої
\index{i}{0296}  %% посилання на сторінку оригінального видання
маси, яке виникає з поділу праці, А.~Сміс радить державну
організацію народньої освіти, хоч і в обережно гомеопатичних
дозах. Послідовно полемізує проти цього його французький
перекладач та коментатор Ґ.~Ґарньє, що за часів першої французької
імперії, природна річ, перетворився на сенатора. На його
думку, народня освіта суперечить основним законам поділу праці;
завівши її, «ми засудили б цілу нашу суспільну систему». «Як
і всякі інші поділи праці, — каже він, — поділ між працею фізичною
та розумовою\footnote{
Вже \emph{Ферґюсон} каже («History of Civil Society», part IV, sect. I.
p. 281): «І саме думання в цьому віці поділу праці стає осібною професією»
(«and thinking itself, in this age of separations, may become a peculiar
craft»).
} стає дедалі виразнішим та рішучішим у
міру того, як багатіє суспільство (він слушно прикладає цей
вислів до капіталу, земельної власности та їхньої держави).
Подібно до всіх інших і цей поділ праці є наслідок минулого та
причина майбутнього проґресу\dots{} Невже ж уряд має право протидіяти
цьому поділові праці та спиняти його в природному його
розвитку? Невже він має право частину державних прибутків
витрачати на спробу перемішати та сплутати дві кляси праці,
що прагнуть свого розділу й відокремлення?»\footnote{
\emph{G.~Garnier}, т. V його перекладу, стор. 2--5.
}.

Деяке розумове й фізичне скалічення є невіддільне навіть від
поділу праці всередині суспільства як цілости. Але що мануфактурний
період проводить це суспільне розчленування галузей
праці значно далі і що, з другого боку, цей період з властивим
йому поділом праці вперше захоплює життя індивіда в самому
його корені, то він також уперше дає матеріял та поштовх до
промислової патології\footnote{
\emph{Рамацціні}, професор практичної медицини в Падуї, опублікував
1713~\abbr{р.} свою працю «De morbis artificum», перекладену 1781 p.
французькою мовою і знову передруковану 1841~\abbr{р.} в «Encyclopédie des
Sciences Médicales, 7 ème Discours: Auteurs Classiques». Період великої
промисловости, природно, дуже збільшив його каталог робітничих хороб.
Дивись, між іншим, «Hygiène physique et morale de l’ouvrier dans les
grandes villes en général, et dans la ville de Lyon en particulier. Par le
Dr.~A.~L.~Fonterel», Paris 1858 і «Die Krankheiten, welche verschiedenen
Ständen, Altern und Geschlechtern eigentümlich sind». 6 Bände, Ulm I860.
1854 p. Society of Arts призначило слідчу комісію щодо промислової
патології. Реєстр зібраних цією комісією документів можна найти в каталозі
«Twickenham Economic Museum». Дуже важливі офіціальні «Reports
on Public Health». Див. також \emph{Eduard Reich, M.~D.}: «Ueber
die Entartung des Menschen», Erlangen 1868.
}.

«Ділити людину — це значить карати її на смерть, якщо вона
\parbreak{}  %% абзац продовжується на наступній сторінці

\input{i/_0297.tex}
\parcont{}  %% абзац починається на попередній сторінці 
\index{i}{0298}  %% посилання на сторінку оригінального видання 
of Nations» і т. ін. Він не тільки розвиває суспільну продуктивну
силу праці для капіталіста замість для робітника, але й розвиває
її через покалічення індивідуального робітника. Він створює
нові умови панування капіталу над працею. Тим-то, якщо цей
поділ праці, з одного боку, з’являється як історичний проґрес
та доконечний момент розвитку процесу економічного формування
суспільства, то, з другого боку, він з’являється як засіб
цивілізованої та рафінованої експлуатації.

Політична економія, що стає спеціяльною наукою лише за
мануфактурного періоду, розглядає суспільний поділ праці взагалі
лише з погляду мануфактурного поділу праці,\footnote{
Давніші письменники, як от Петті, анонімний автор «Advantages
of the East-India Trade» і т. д., зрозуміли капіталістичний характер
мануфактурного поділу праці ясніше, ніж А. Сміс.
} лише як
засіб з тією самою кількістю праці продукувати більше товару,
отже, здешевити товари та прискорити акумуляцію капіталу.
В гострій протилежності до цього підкреслювання кількости та
мінової вартости письменники клясичної старовини звертають
увагу виключно на якість та на споживну вартість.\footnote{
Виняток серед сучасних письменників становлять лише деякі
автори XVІІІ віку, як от Беккарія та Джемс Гарріс, які щодо поділу
праці майже виключно наслідують давніх. Напр., Беккарія каже: «Кожен
знає з досвіду, що, прикладаючи рук і розуму завжди до однорідної праці
та до виготовлювання тих самих продуктів, можна з більшою легкістю
досягти значніших та ліпших успіхів, ніж у тому випадку, коли б кожний
ізольовано сам для себе виготовляв усі потрібні йому речі... Люди поділяються
таким чином на різні кляси й стани в інтересах спільної та індивідуальної
користи». («Ciascuno prova coll’esperienza, che applicando
la mano e l’ingegno sempre allo stesso genere di opere e di produtti, egli
più facili, più abbondanti e migliori ne traca resultati, di quello che se
ciascuno isolatamente le cose tutte a se necessarie soltanto facesse... Dividendosi
in tal maniera per la comune e privata utilita gli uomini
in varie classe e condizioni»). (Cesare Beccaria: «Elementi di Economia
Publica», ed. Custodi, Parte Moderna, vol. XI, p. 28). Джеме
Гарріс, пізніш граф Малмсберійський, відомий своїми «Diaries» про
своє перебування послом у Петербурзі, сам каже в одній примітці до
свого «Dialogue concerning Happiness», London 1741, пізніше знов передрукованого
в «Three Treatises etc.», З rd ed. London 1772: «Всі докази
природности суспільства (а саме докази, засновані на «поділі занять»)
я взяв із другої книги «Республіки» Платона. (The whole argument,
to prove society natural is taken from the second book of Plato’s
republic»).
} У наслідок
роз’єднання суспільних галузей продукції товари виробляються
ліпше, різні нахили й таланти людей вибирають собі відповідні
сфери діяльности,\footnote{
Так, в «Одісеї», XIV, 228 говориться: «Ἄλλoς γὰρ τ’ἄλλοισιν ἀνὴρ ἐπιτέρπεται ἔργοις» * a
Архілох y Секста Емпірика каже: «Ἄλλος ἄλλῳ ἐπ’ἔργῳ καρδίην ἰαίνεται * *
} а без обмеження ніде не можна зробити нічого
значного. 79 Отже продукт і продуцент удосконалюються через

78 «Поλλ’ ἠπίστατο ἔργα, κακῶς δ’ ἠπίστατο πάντα»\footnote*{
Багато знав він справ, та всі погано знав. Ред.
} — Атенець, як товаропродуцент,
почував свою перевагу над спартанцем, бо цей останній міг

* Одні люди люблять одне, інші — інше. Ред.

** Одне тішить серце одного, інше — іншого. Ред.
\index{i}{0299}  %% посилання на сторінку оригінального видання 
поділ праці. Якщо ці письменники принагідно згадують і про
зріст маси продуктів, то лише щодо більшої повноти споживних
вартостей. Про мінову вартість, про подешевшання товарів вони
зовсім не згадують. Цей погляд споживної вартости панує так
у Платона,\footnote{
Платон виводить поділ праці всередині громади з багатобічности
потреб та однобічности здібностей індивідів. Його головний погляд є в
тому, що робітник мусить пристосовуватися до справи, а не справа до
робітника, як воно неминуче буває тоді, коли він разом працює в різних
ремествах, отже, в тому чи тому реместві працює як у побічному. «Бо
справі ніколи чекати на вільний час продуцента, а треба, щоб продуцент
виконував свою справу пильно і не між іншим. — Треба. — Аджеж кожна
річ продукується легше й ліпше і в більшій кількості, коли людина робить
лише одну річ, що відповідає її здібності, та в належний час, вільний
від усякої іншої роботи». («Οὐ γὰρ ἐθέλει τὸ πραττόμενον τὴν τοῦ πράττοντος σχολὴν περιμένειν, ἀλλ’ 
ἀνάγκη τὸν πράττοντα τῷ πραττομένῷ ἐπακολουθεῖν μὴ ἐν παρέργου μέρει. — Ἀνάγκη. Ἐκ δὴ τούτων πλείω
τε ἕκαστα γίγνεται καὶ κάλλιον καὶ ῥᾷον, ὅταν εἷς ἓν κατὰ φύσιν καὶ ἐν καιρῷ, σχολὴν τῶν ἄλλων ἄγων,
πράττῃ). («Respublica», lib. II, c. 12,
ed. Baiter, Orelli etc.). Подібні думки ми маємо в Тукідіда: «Geschichte
des Peloponnesischen Krieges», книга перша, відділ 142: «Морська справа
є така ж умілість, як і будь-що інше, і не можна коло неї працювати принагідно,
як коло якоїсь побічної справи, навпаки, морська справа не
дозволяє працювати коло чогось іншого навіть як побічної справи».
Якщо справа мусить чекати на робітника, каже Платон, то часто ґавиться
критичний момент продукції і продукт псується, «ἔργου καιρὸν διόλλυται». Цю
саму платонівську ідею подибуємо знов у протесті англійських власників
білилень проти того застереження фабричного закону, яке встановлює визначену
годину на їжу для всіх робітників. Їхнє підприємство, мовляв, не
може пристосовуватися до робітників, бо «в різних операціях опалювання,
промивання, біління, качання, прасування та фарбування не можна
перервати роботу у наперед визначений момент без небезпеки заподіяти
шкоду... встановлення для всіх робітників тієї самої перерви на їжу —
це значило б у певних випадках кинути коштовні продукти на небезпеку,
що вони попсуються через незакінчені операції» («in the various operations
of singeing, washing, bleaching, mangling, calendering, and dycing,
none of them can be stopped at a given moment without risk of damage...
to enforce the same dinner hour for all the workpeople might occasionally
subject valuable goods to the risk of danger by incomplete operations»).
Le platonisme où va-t-il se nicher!.\footnote*{
Куди ще може продертись платонізм! Ред.
}
} що розглядав поділ праці як основу поділу суспільства
на стани, як і в Ксенофонта,\footnote{
Ксенофонт оповідає, що не тільки велика честь діставати страви
зі столу перського короля, але що й ці страви куди смачніші, ніж ін-
} який з характеристичним для
нього буржуазним інстинктом уже ближче підходить до поділу
праці всередині майстерні. Платонова республіка, оскільки в ній

порядкувати у війні людьми, але не грішми, — як це Тукідід вкладав
в уста Перікла у промові, в якій він підцьковує атенців до пелопонеської
війни: «Σώμασί τε ἐτοιμότεροι οἱ αὐτουργοὶ τῶν ἀνθρώπων ἤ χρήμασι πολεμεῖν».\footnote*{
Люди, що працюють для задоволення власних потреб, радше
віддадуть на війну свої тіла, ніж гроші. Ред.
} (Thucydides:
«Geschichte des Peloponnesischen Krieges», книга перша, відділ
141). А проте їхнім ідеалом, навіть у матеріяльній продукції, була
αυταρχεια,\footnote*{
— автаркія. Ред.
} що протиставляється поділові праці, бо «παρ’ ὧν γὰρ τὸ εὖ, παρὰ τούτων καὶ τὸ
αὐτάρκες».\footnote*{
«з цього постає благо, а з того і незалежність». Ред.
} Треба при цьому зважити, що за часів упадку
30 тиранів не було ще й 5.000 атенців без земельної власности.
\index{i}{0300}  %% посилання на сторінку оригінального видання 
поділ праці фігурує як конститутивний принцип держави, є
лише атенська ідеалізація єгипетського кастового ладу, так само
як Єгипет в інших його сучасників, наприклад, у Ізократа,\footnote{
«Він (Бузіріс) поділив усіх на окремі касти... наказав, щоб ті
самі люди завжди працювали коло тих самих справ, бо знав, що ті люди,
які змінюють свою працю, не знають ґрунтовно ніякої справи; а ті, які
завжди працюють коло тих самих справ, зможуть кожну з них виконати
якнайдосконаліше. І ми справді бачимо, що відносно вмілости та реместв
вони куди більше перевищили своїх суперників, ніж звичайно майстер
перевищує партача, а інституції, якими вони підтримують королівське
панування та інший державний лад, були в них такі досконалі, що найславетніші
філософи, яким доводилося писати про це, вихваляли державний
лад Єгипту більше, ніж інших країн». (Isocrates: «Busiris», с. 8).
83 Порівн. Diodorus Siculus.
}
вважається за зразкову промислову країну; це значення Єгипет
зберігає ще навіть для греків доби Римської імперії.83

Підчас власне мануфактурного періоду, тобто того періоду,
коли мануфактура була панівною формою капіталістичного способу
продукції, повне здійснення властивих їй тенденцій наражається
на багато різних перешкод. Хоч мануфактура, як ми
бачили, і утворює поряд ієрархічного розчленування робітників
простий поділ їх на навчених та ненавчених, однак число останніх
лишається через переважний вплив перших дуже обмеженим.
Хоч вона й пристосовує окремі операції до різних ступенів дозрілости,
сили й розвитку її живих робочих органів і тим штовхає
до продуктивного визиску жінок та дітей, все ж взагалі і в цілому
ця тенденція розбивається об звички та опір робітників-чоловіків.
Хоч розчленування ремісничої роботи понижує витрати на
навчання, отже, і вартість робітників, проте для тяжчої детальної
праці лишається потрібним довший час на навчання, і його
з запалом зберігають робітники навіть там, де він зайвий. Ми
находимо, приміром, в Англії laws of apprenticeship\footnote*{
— закони про учнів. Ред.
} з їхнім
семирічним часом навчання в повній силі аж до кінця мануфактурного
періоду, і лише велика промисловість знищила їх. Через
те, що реміснича вправність лишається основою мануфактури,

ші. «І тут немає нічого дивного, бо як і інші вмілості надто вдосконалені по
великих містах, так само й королівські страви готуються цілком своєрідно.
Бо по дрібних містах та сама людина виробляє ліжка, двері, плуги, столи;
крім цього, вона часто ще й будує доми і задоволена, коли сама находить
такі замовлення, яких вистачає, щоб підтримати своє існування.
Цілком неможливо, щоб людина, яка робить усяку всячину, робила
все добре. Але по великих містах, де кожний поодинокий продуцент
находить багато покупців, досить і одного ремества, щоб прохарчуватися.
Часто непотрібно навіть знати ціле ремество: один виробляє чоловічі
черевики, другий — жіночі. Часто один живе з того, що лише шиє, другий
— з того, що викроює черевики; один крає одяг, інший складає кусники
докупи. Неминучий наслідок цього той, що виконавець найпростішої
праці безумовно й накраще виконує її. Так само стоїть справа і
з куховарством». (Xenophon: «Cyropaedie», lib. VIII, с. 2). Тут вважається
виключно на те, як дійти доброї якости споживної вартости,
хоч уже й Ксенофонтові відома залежність маштабу поділу праці від
обсягу ринку.
\parbreak{}  %% абзац продовжується на наступній сторінці

\parcont{}  %% абзац починається на попередній сторінці
\index{iii1}{0301}  %% посилання на сторінку оригінального видання
визначально впливає на його відношення до сукупного капіталу,
або на відносну величину купецького капіталу, необхідного для
циркуляції, бо ясно, що абсолютна величина необхідного купецького
капіталу і швидкість його обороту стоять у зворотному
відношенні одне до одного; але його відносна величина, або та
частина, яку він становить у сукупному капіталі, визначається
його абсолютною величиною, припускаючи всі інші умови
однаковими. Якщо сукупний капітал становить \num{10000}, то, коли
купецький капітал становить \sfrac{1}{10} його, він \deq{} 1000; якщо сукупний
капітал становить 1000, то \sfrac{1}{10}  його \deq{} 100. Таким чином, хоч
його відносна величина лишається тією самою, його абсолютна
величина є різна залежно від величини сукупного капіталу. Але
тут ми беремо його відносну величину, скажімо, \sfrac{1}{10}  сукупного капіталу,
як дану. Сама ця відносна величина його, однак, знов таки
визначається оборотом. При швидкому обороті його абсолютна величина,
наприклад, \deq{} 1000\pound{ фунтам стерлінгів} у першому випадку,
\deq{} 100 у другому випадку, а тому його відносна величина \deq{} \sfrac{1}{10}
При повільнішому обороті його абсолютна величина, скажімо, \deq{}
2000 в першому випадку, \deq{} 200 в другому. Тому його відносна
величина зросла з \sfrac{1}{10}  до \sfrac{1}{5} сукупного капіталу. Обставини,
які скорочують пересічний оборот купецького капіталу, наприклад,
розвиток засобів транспорту, pro tanto [відповідно до цього]
зменшують абсолютну величину купецького капіталу, отже
підвищують загальну норму зиску. В протилежному випадку, —
навпаки. Розвинений капіталістичний спосіб виробництва, порівняно
з попереднім становищем, впливає двояко на купецький
капітал: та сама кількість товарів обертається за допомогою
меншої маси дійсно функціонуючого купецького капіталу; в наслідок
швидшого обороту купецького капіталу і більшої швидкості
процесу репродукції, на якій грунтується швидший оборот,
зменшується відношення купецького капіталу до промислового
капіталу. З другого боку: з розвитком капіталістичного способу
виробництва все виробництво стає товарним виробництвом і тому
весь продукт потрапляє в руки агентів циркуляції; при чому
сюди долучається ще й те, що за попереднього способу виробництва,
яке провадилося в незначних розмірах, якщо залишити
осторонь масу продуктів, які споживалися in natura безпосередньо
самими виробниками, і масу повинностей, які виконувались
in natura, дуже значна частина виробників продавала свої
товари безпосередньо споживачам або працювала на їх особисте
замовлення. Тому, хоч за попередніх способів виробництва торговельний
капітал є більший відносно того товарного капіталу,
що його він обертає, але він

1)~абсолютно менший, бо незрівнянно менша частина сукупного
продукту виробляється як товар і мусить надходити в циркуляцію
як товарний капітал та потрапляти в руки купців; він менший
тому, що товарний капітал є менший. Але разом з тим він відносно
більший не тільки в наслідок більшої повільності його
\parbreak{}  %% абзац продовжується на наступній сторінці

\parcont{}  %% абзац починається на попередній сторінці
\index{iii1}{0302}  %% посилання на сторінку оригінального видання
обороту і порівняно з масою товарів, які він обертає. Він більший і
тому, що ціна цієї товарної маси, а тому й купецький капітал,
який належить авансувати на неї, в наслідок меншої продуктивності
праці є більші, ніж при капіталістичному виробництві;
тому та сама вартість виражається в меншій масі товарів.

2) На основі капіталістичного способу виробництва не тільки
виробляється більша маса товарів (при чому треба взяти до
уваги зменшену вартість цієї товарної маси), але одна й та сама
маса продукту, наприклад, зерна, становить більшу масу товару,
тобто дедалі більша частина її йде в торгівлю. Зрештою, в наслідок
цього зростає не тільки маса купецького капіталу, але
взагалі весь капітал, вкладуваний у циркуляцію, наприклад, у
судноплавство, залізниці, телеграф і т. д.

3) Але — і це є точка зору, виклад якої належить до „конкуренції
капіталів“ — нефункціонуючий або напівфункціонуючий
купецький капітал зростає з прогресом капіталістичного способу
виробництва, з полегшенням просування його в дрібну торгівлю,
з розвитком спекуляції і надлишку капіталу, що звільняється.

Але, якщо припустити за дану відносну величину купецького
капіталу порівняно з сукупним капіталом, ріжниця оборотів
у різних галузях торгівлі не впливає ні на величину сукупного
зиску, що припадає купецькому капіталові, ні на загальну норму
зиску. Зиск купця визначається не масою товарного капіталу,
що його він обертає, а величиною грошового капіталу, який він
авансує для опосереднення цього обороту. Якщо загальна річна
норма зиску є 15\% і якщо купець авансує 100 фунтів стерлінгів,
то, коли його капітал обертається один раз за рік, він продасть
свій товар за 115. Якщо ж його капітал обертається 5 разів за
рік, то за рік він п’ять разів продасть за 103 товарний капітал,
купівельна ціна якого є 100, отже, за цілий рік він продасть
товарний капітал в 500 за 515. Але, як і в першому випадку, це
становить на його авансований капітал в 100 річний зиск в 15.
Коли б це було не так, то купецький капітал давав би, відповідно
до числа своїх оборотів, значно вищий зиск, ніж промисловий
капітал, що суперечить законові загальної норми зиску.

Отже, число оборотів купецького капіталу в різних галузях
торгівлі безпосередньо впливає. на торговельні ціни товарів. Висота
торговельної надбавки до ціни, величина відповідної частини
торговельного зиску даного капіталу, яка припадає на ціну
виробництва одиниці товару, стоїть у зворотному відношенні
до числа оборотів або до швидкості обороту купецького
капіталу в різних галузях торгівлі. Якщо купецький капітал
обертається п’ять разів за рік, то він додає до товарного капіталу
однакової вартості тільки \sfrac{1}{5} тієї надбавки, яку додає до товарного
капіталу однакової вартості другий купецький капітал, який може
обертатися тільки один раз за рік.

Вплив пересічного часу обороту капіталів в різних галузях
торгівлі на продажні ціни зводиться до того, що, відповідно до
\parbreak{}  %% абзац продовжується на наступній сторінці

\input{i/_0303.tex}
\parcont{}  %% абзац починається на попередній сторінці 
\index{i}{0304}  %% посилання на сторінку оригінального видання 
вона знову та знову становить вихідний пункт таких революцій
щоразу, коли ремісниче або мануфактурне виробництво переходить
у машинове.

Коли придивитися ближче до виконавчої машини, або власне
робочої машини, то взагалі і в цілому знову побачимо в ній, хоч
часто і в дуже змодифікованій формі, ті самі апарати й знаряддя,
що ними працює ремісник та мануфактурний робітник, але
вже не як знаряддя людини, а як знаряддя якогось механізму,
або як механічні знаряддя. Або ціла машина є лише більш
чи менш змінене механічне видання колишніх ремісницьких
інструментів, як от у механічному ткацькому варстаті,\footnote{
Особливо в первісній формі механізованого ткацького варстату
можна з першого ж погляду пізнати давній ткацький варстат. У своїй
сучасній формі він посутньо змінений.
} або
розміщені на кістяку робочої машини чинні органи є давні
знайомі, як от веретена у прядільній машині, дроти в машині
на плетіння панчіх, пили у тартаку, ножі в різальній машині
й т. д. Відміна цих знарядь від власне тіла робочої машини
починається вже від самих їхніх народин, а саме: ці знаряддя
здебільшого все ще продукуються ремісничим або мануфактурним
способом, і лише потім прикріплюють їх до кістяка
робочої машини, спродукованої машиновим способом.\footnote{
Тільки приблизно від 1850 р. дедалі більшу частину знарядь робочих
машин починають фабрикувати в Англії машиновим способом, хоч
фабрикують їх не ті самі фабриканти, що виробляють сами машини.
Машинами для фабрикації такого механічного знаряддя є, наприклад,
automatic bobbin-making engine, card-setting engine — машини виготовляти
ткальні самопряди, машини виготовляти веретена mule і throstle.
} Отже,
виконавча машина це — такий механізм, що, одержавши відповідний
рух, виконує своїм знаряддям ті самі операції, що їх раніш
виконував робітник подібним знаряддям. Чи рушійна сила виходить
від людини, чи навіть знову таки від якоїсь машини, це
нічого не змінює в суті справи. Після перенесення власне знаряддя
від людини до механізму машина стає на місце простого знаряддя.
Ріжниця відразу впадає на очі, навіть і тоді, коли сама людина
все ще лишається першим мотором. Число робочих інструментів,
якими людина може орудувати одночасно, обмежене кількістю
її природних знарядь продукції, кількістю органів її власного
тіла. В Німеччині пробували були примусити прядуна рухати
два прядільні колеса, тобто працювати одночасно обома руками
й обома ногами. Це була надто напружена робота. Пізніше вигадали
ножаний самопряд з двома веретенами, але такі віртуози-прядуни,
які могли одночасно прясти дві нитки, траплялися майже
так рідко, як двоголові люди. Навпаки, машина jenny вже з
самого початку свого з’явлення пряде 12—18 веретенами, машина
на плетіння панчіх плете одночасно кількома тисячами дротів і
т. д. Число знарядь, що ними та сама виконавча машина одночасно
робить, від самого початку емансиповане від тієї органічної
межі, що її не могло переступити ручне знаряддя робітника.


\index{i}{0305}  %% посилання на сторінку оригінального видання
При багатьох ручних знаряддях ріжниця поміж людиною як
простою рушійною силою та як робітником, що виконує власне
ручну роботу, має почуттьово сприйману форму. Наприклад,
при самопряді нога діє лише як рушійна сила, тимчасом як рука,
що працює коло веретена, смикає та крутить, виконує власне
операцію прядіння. Саме цю останню частину ремісничого інструменту
захоплює насамперед промислова революція, залишаючи
людині поряд нової праці — наглядати за машиною та виправляти
своїми руками її хиби — насамперед ще й суто механічну ролю
рушійної сили. Навпаки, знаряддя, на які людина з самого початку
діє лише як проста рушійна сила, — приміром, крутячи корбою
млина\footnote{
Мойсей-єгиптянин каже: «Не зав’язуй рота волові, коли він молотить».
Навпаки, християнсько-германські філантропи клали на шию
кріпакові, якого вони вживали як рушійну силу для млина, великі дерев’яні
кола, щоб він не міг рукою підносити борошна до рота.
}, помпуючи, підіймаючи та спускаючи ручку ковальського
міха, товчучи в ступі й~\abbr{т. ін.}, — передусім викликають вживання
тварин, води, вітру\footnote{
Почасти брак природних водоспадів, почасти боротьба з надміром
стоячої води примусили голляндців використовувати вітер як рушійну
силу. Самий вітряк голляндці перейняли від Німеччини, де цей винахід
викликав «ввічливу» боротьбу між шляхтою, попами та імператором за те,
кому з них трьох «належить» вітер. Повітря закріпачує, казали в Німеччині,
тимчасом як Голляндію вітер зробив вільною. Що він тут закріпачував,
так це не голляндців, а землю для голляндців. Ще 1836~\abbr{р.}
в Голляндії вживали \num{12.000} вітряків у \num{6.000} кінських сил, щоб захистити
дві третини країни, не дати їм знову перетворитися на болото.
} як рухових сил. Почасти за мануфактурного
періоду, а спорадично вже задовго перед ним ці знаряддя
розвиваються на машини, алеж вони не революціонізують способу
продукції. Що вони навіть у своїй ремісничій формі вже є
машини, це виявляється в періоді великої промисловости. Смоки,
приміром, якими голляндці викачали 1836--37~\abbr{рр.} воду з Гаарлемського
озера, були сконструйовані за принципом звичайних
смоків з тією лише відміною, що їхнім толокам надавали руху не
людські руки, а циклопічні парові машини. Звичайний та дуже
неудосконалений ковальський міх ще й тепер іноді в Англії перетворюють
на механічний повітряний смок, сполучаючи його
ручку з паровою машиною. Сама парова машина, така, як її
винайдено наприкінці XVII віку, за мануфактурного періоду,
та якою вона і далі існувала до початку 80 років XVIII віку\footnote{
Правда, її вже дуже поліпшив Ватт за допомогою його першої
так званої парової машини простого чину, але в цій формі вона лишалася
простою машиною піднімати воду та ропу.
},
не викликала жодної промислової революції. Навпаки, саме створення
виконавчих машин зробило зреволюціонізоваиу парову
машину доконечною [а тому й можливою]\footnote*{
Заведене у прямі дужки ми беремо з першого німецького видання.
\emph{Ред.}
}. Скоро тільки людина,
замість діяти на предмет праці знаряддям, діє лише як рушійна
сила на виконавчу машину, то втілення цієї рушійної сили в
людські мускули стає випадковим, і вода, вітер, пара і~\abbr{т. д.} можуть
\parbreak{}  %% абзац продовжується на наступній сторінці

\input{i/_0306.tex}
\parcont{}  %% абзац починається на попередній сторінці
\index{i}{0307}  %% посилання на сторінку оригінального видання
часто вживали коней, як про це, окрім лементу тогочасних аґрономів,
свідчить уже й вираз механічної сили в кінських силах —
вираз, що зберігся до наших часів. Вітер був надто непостійний
і не піддавався контролеві; крім того, в Англії, цій батьківщині
великої промисловости, вживали переважно водяної сили вже
протягом мануфактурного періоду. Вже в XVII столітті були
спроби одним водяним колесом пустити в рух двоє жорен, отже, і
два млинові кола. Але зріст розміру передатного механізму дійшов
тепер конфлікту з недостатньою силою води, і це є одна з обставин,
що спонукала докладніше досліджувати закони тертя. Так
само неоднорідне діяння рушійної сили у млинах, що їх пускали
в рух за допомогою посування та тягнення коромисел, привело
до теорії та вживання махового колеса\footnote{
Faulhaber 1625, De Caus 1688.
}, що пізніше відіграє
таку важливу ролю у великій індустрії. Таким способом мануфактурний
період розвивав перші наукові й технічні елементи
великої промисловости. Тростільну прядільню Аркрайта від
самого початку гнала вода. Алеж і вживання сили води як
головної рушійної сили було сполучене з труднощами. Її не можна
було підняти до бажаної височини й не можна було зарадити її
бракові, деколи вона відмовлялась служити, а передусім вона
була суто льокальної природи\footnote{
Новітній винахід турбін звільняє промислову експлуатацію
водяної сили від багатьох колишніх обмежень.
}. Лише з винаходом другої Ваттової
машини, так званої парової машини двійного чину, знайдено
перший мотор, що, споживаючи вугілля й воду, сам витворює
свою рушійну силу; мотор, що його сила стоїть цілком під контролем
людини; мотор пересувний і засіб до пересовування; мотор
міський, а не, як водяне колесо, сільський; мотор, що дозволяє
концентрацію продукції по містах, замість, як водяне колесо,
розкидувати її по селах;\footnote{
«В перші часи існування текстильних мануфактур місце заснування
мануфактури залежало від існування потока з остільки достатнім
спадом, щоб він міг обертати водяне колесо; і хоч будування водяних
фабрик було першим ударом для системи домашньої мануфактури, однак
ці фабрики, що їх з доконечности будували над потоками та часто-густо
в значній віддалі одну від однієї, становили скорше частину сільської, аніж
міської системи; лише після введення сили пари замість сили води фабрики
можна було зосереджувати по містах і по тих місцевостях, де було досить
вугілля та води, потрібних для продукції пари. Парова машина — мати
мануфактурних міст». («In the early days of textile manufactures, the locality
of the factory depended upon the existence of a stream having a sufficient
fall to turn a water wheel; and, although the establishment of the water
mills was the commencement of the breaking up of the domestic system of
manufacture, yet the mills necessarily situated upon streams, and frequently
at considerable distances the one from the other, formed part of a rural rather
than an urban system; and it was not until the introduction of the steampower
as a substitute for the stream, that factories were congregated in towns and
localities where the coal and water required for the production of steam were
found in sufficient quantities. The steam-engine is the parent of manufacturing
towns»). (\emph{A. Redgrave} у «Reports of Insp. of Fact, for 30 th
April 1866», p. 36).
} мотор універсальний у своєму технологічному
\index{i}{0308}  %% посилання на сторінку оригінального видання
застосуванні й порівняно мало залежний від локальних
умов щодо свого місця застосування. Великий геній Ватта
виявляється у специфікації патенту, який він узяв у квітні 1784~\abbr{р.},
специфікації, що в ній його парову машину описано не як якийсь
винахід для окремих завдань, але як універсальний чинник
великої промисловости. Він натякає тут на такі способи її застосовування,
що з них деякі, як от, приміром, паровий молот,
заведено в життя лише більше ніж півстоліття пізніше. Однак
він сумнівався в тому, чи можна буде застосувати парову машину
до мореплавства. Його наступники, Болтон та Ватт, виставили
1851~\abbr{р.} на лондонській промисловій виставці колосальнішу парову
машину для океанських пароплавів.

Лише після того, як знаряддя перетворились із знарядь людського
організму на знаряддя механічного апарату, виконавчої
машини, тільки тоді й рухова машина набула самостійної форми,
цілком емансипованої від меж людської сили. Разом з цим та
поодинока виконавча машина, яку ми досі розглядали, зводиться
на простий елемент машинової продукції. Тепер одна рухова
машина може одночасно рухати багато робочих машин. Зі збільшенням
числа одночасно пущених у рух робочих машин зростає
й рухова машина, а передатний механізм розвивається в широченний
і складний апарат.

Тут треба розрізняти дві форми: кооперацію багатьох однорідних
машин і систему машин.

В одному випадку цілий продукт виробляє та сама робоча
машина. Вона виконує всі ті різні операції, що їх виконував своїм
знаряддям ремісник, наприклад, ткач своїм ткацьким варстатом,
або ті, що їх послідовно виконували ремісники за допомогою
різних знарядь, однаково, чи були вони самостійні ремісники,
чи члени якоїсь мануфактури\footnote{
З погляду мануфактурного поділу праці ткацтво було зовсім не
проста, а скорше складна реміснича праця, і тому механічний ткацький
варстат є машина, що виконує дуже різноманітні операції. Взагалі неправильно
думати, ніби сучасні машини первісно опанували такі операції,
які мануфактурний поділ праці вже спростив. Прядіння й ткання за мануфактурного
періоду відокремлено одне від одного на нові роди, їхнє знаряддя
поліпшено та урізноріднено, але самого процесу праці ані скільки
не поділено, він лишався ремісничий. За вихідну точку для машини є не
праця, а засіб праці.
}. Приміром, у сучасній мануфактурі
поштових конвертів один робітник за допомогою фальцу
фальцював папір, другий накладав клей, третій одгинав кляпку,
на якій друкується девізу, четвертий витискував девізу й~\abbr{т. ін.},
і при кожній з цих частинних операцій кожний окремий конверт
мусив переходити з рук до рук. Одним-одна машина виготовляти
конверти виконує всі ці операції відразу й виготовляє \num{3.000} й
більше поштових конвертів за одну годину. Одна американська
машина виготовляти паперові мішечки, виставлена па лондонській
промисловій виставці 1862~\abbr{р.}, ріже папір, намазує клей,
фальцює й виготовляє 300 штук за хвилину. Цілий процес, що в
мануфактурі є поділений і виконується послідовно, тут виконує
\parbreak{}  %% абзац продовжується на наступній сторінці

\parcont{}  %% абзац починається на попередній сторінці
\index{i}{0309}  %% посилання на сторінку оригінального видання
одна робоча машина, яка працює за допомогою комбінації різних
знарядь. Чи така робоча машина є лише механічне відродження
складнішого ремісничого знаряддя, чи вона є комбінація різнорідних
простих інструментів, спеціялізованих мануфактурою, на
фабриці, тобто в основаній на машиновому виробництві майстерні,
кожного разу знов з’являється проста кооперація, і насамперед
(робітника ми лишаємо тут осторонь) саме як зосередження у просторі
однорідних робочих машин, що одночасно разом функціонують.
Так, ткацька фабрика утворюється через сполучення
багатьох механічних ткацьких варстатів, а швацька фабрика —
через сполучення багатьох швацьких машин у тому самому робітному
приміщенні. Але тут існує технічна єдність, тому що це
велике число однорідних робочих машин одночасно та рівномірно
дістає свій рух від руху спільного першого мотора, руху, що
переноситься на них за допомогою передатного механізму, теж
почасти спільного всім їм, бо від нього розходяться лише осібні
відгалуження для кожної окремої виконавчої машини. Цілком
так само як численні знаряддя становлять лише органи однієї
робочої машини, так само й численні робочі машини становлять
тепер лише однорідні органи того самого рухового механізму.
Але система машин у власному значенні слова заступає
поодиноку самостійну машину лише там, де предмет праці послідовно
перебігає ряд зв’язаних між собою різних частинних
процесів, що їх поступінно виконує низка різнорідних виконавчих
машин, які однак одна одну доповнюють. Тут знову з’являється
характеристична для мануфактури кооперація, що ґрунтується
на поділі праці, але тепер уже як комбінація частинних
робочих машин. Специфічні знаряддя різних частинних робітників,
приміром, у вовняній мануфактурі знаряддя шаповалів,
чухральників вовни, стригунів вовни, прядунів вовни й т. ін.,
перетворюються тепер на знаряддя специфікованих робочих машин,
що з них кожна становить осібний орган для осібної функції
в системі комбінованого робочого механізму. Сама мануфактура
дає машиновій системі по тих галузях, де її заводиться вперше,
взагалі та в цілому природну основу поділу, а тому й організацію
процесу продукції.\footnote{
Перед епохою великої промисловости вовняна мануфактура була
домінантною мануфактурою Англії. Тим то за першу половину XVIII століття
саме в ній пороблено більшість експериментів. Досліди, пороблені
на овечій вовні, стали корисними і для бавовни, що її механічне
оброблення потребує менш тяжких підготовчих праць, так само як пізніше,
навпаки, механічна вовняна промисловість розвинулась на основі
механічного прядіння й ткання бавовни. Поодинокі елементи вовняної
мануфактури, як, наприклад, чесання вовни, заведено у фабричну систему
лише в останні десятиліття. «Вживання механічної сили до чесання вовни...
Дуже поширене від часів заведення «чесальної машини», особливої машини
Лістера... мало безперечно своїм наслідком те, що дуже велике число
людей лишилося без праці. Перше вовну розчісували руками здебільшого
вдома в чесальника. Тепер її звичайно розчісують на фабриці, і ручну
} Однак відразу виступає посутня ріжниця.
\index{i}{0310}  %% посилання на сторінку оригінального видання
В мануфактурі робітники мусять окремо або групами виконувати
кожний окремий частинний процес своїм ручним знаряддям.
Якщо робітник і пристосовується тут до процесу, то й процес
теж уже перед тим пристосовано до робітника. За машинової
продукції цей суб’єктивний принцип поділу відпадає. Цілий
процес розкладається тут об’єктивно, розглядуваний сам по собі,
на свої складові фази, і проблема виконання кожного частинного
процесу та сполучення різних частинних процесів розв’язується
за допомогою технічного застосування механіки, хемії й т. д.,\footnote{
«Отже, принцип фабричної системи є в заміні... поділу або розчленування
праці між робітників розкладом процесу на його посутні
складові елементи» («The principle of the factory system, then, is to substitute...
the partition of a process into its essential constituents, lor the
division or gradation of labour among artizans»). (Ure: «Philosophy of
Manufacture», p. 20).
}
при чому, природно, теоретична концепція, як і раніш, мусить
бути вдосконалена через нагромаджений широкий практичний
досвід. Кожна частинна машина постачає для тієї що безпосередньо
йде за нею, її сировинний матеріял; а що всі вони функціонують
одночасно, то продукт так само безупинно перебуває на
різних ступенях процесу свого творення, як і в процесі переходу
з однієї фази продукції в іншу. Як у мануфактурі безпосередня
кооперація частинних робітників створює певні кількісні відношення
між окремими групами робітників, так і в розчленованій
системі машин те, що частинні машини одна одній постійно дають
роботу, створює певне відношення між їх числом, їх розміром
та їхньою швидкістю. Комбінована робоча машина, тепер розчленована
система різнорідних окремих робочих машин та груп
робочих машин, є то досконаліша, що безупинніший є цілий її
рух, тобто, що з меншими перервами переходить сировинний матеріял
від своєї першої фази до останньої фази, отже, що більше
сировинний матеріял пересувається з однієї фази продукції в
іншу фазу самим механізмом замість людської руки. Якщо в
мануфактурі ізоляція окремих процесів є принцип, даний самим
поділом праці, то в розвинутій фабриці панує, навпаки, безперервність
окремих процесів.

Система машин, хоч базується вона на простій кооперації

працю усунено, за винятком деяких осібних родів праці, в яких усе ще віддають
перевагу вовні, розчісаній руками. Багато з ручних чесальників
знайшли працю на фабриках, але продукт ручного чесальника такий незначний
супроти продукту машини, що дуже велике число чесальників
лишилося без праці». («The application of power to the process of combing
wool... extensively in operation since the introduction of the «combing
machine», especially Lister’s... undoubtedly had the effect of throwing
a very large number of men out of work. Wool was formerly combed by
hand, most frequently in the cottage of the comber. It is how very generally
combed in the factory, and hand labour is superseded, except in some particular
kinds of work, in which hand-combed wool is still preferred. Many
of the handcombers found employment in the factories, but the produce of
the handcomber bears so small a proportion to that of the machine, that
the employment of a very large number of combers has passed away»). («Reports
of Insp. of Fact, for 31st October 1856», p. 16).
\parbreak{}  %% абзац продовжується на наступній сторінці

\input{i/_0311.tex}
\parcont{}  %% абзац починається на попередній сторінці
\index{i}{0312}  %% посилання на сторінку оригінального видання
частина була з’єднана в мануфактурах, де, як уже раніш згадувано,
поділ праці панував з особливою точністю. Із збільшенням числа
винаходів і зростом попиту на нововинайдені машини щораз
більше розвивався, з одного боку, розклад фабрикації машин на
різноманітні самостійні галузі, а з другого боку — поділ праці
всередині машинобудівельних мануфактур. Отже, ми тут вбачаємо
в мануфактурі безпосередню технічну основу великої промисловости.
Мануфактура продукувала машини, за допомогою яких
велика промисловість знищила ремісниче та мануфактурне виробництво
в тих галузях продукції, які вона насамперед охопила.
Отже, машинове виробництво виросло стихійно на невідповідній
йому матеріяльній основі. На певному ступені розвитку машинове
виробництво само мусило зробити переворот у цій основі, яку воно
спочатку застало готового і потім далі виробляло в її старій формі,
та створити для себе нову базу, відповідну його власному способові
продукції. Як поодинока машина лишається карликовою, поки
її пускає в рух лише людина, як система машин не могла вільно
розвиватися, поки на місце рушійних сил, які вона застала, —
худоби, вітру, а то й води — виступила парова машина, так само
й велика промисловість була паралізована в цілому своєму розвитку
доти, доки характеристичний для неї засіб продукції,
сама машина, завдячувала своє існування особистій силі та особистій
вправності, а значить, залежала від розвитку мускулів,
гостроти зору та віртуозности рук, що з ними частинний робітник
у мануфактурі й ремісник поза нею орудували своїм карликовим
інструментом. Залишаючи осторонь подорожчання машин у
наслідок такого способу виникнення їх, — обставина, що, як свідомий
мотив, панує над капіталом, — поширення промисловости,
яка провадилася вже машиновим способом, та проходження машин
у нові галузі продукції лишалися таким чином цілком залежними
від зросту тієї категорії робітників, яка через напівмистецький
характер своєї праці могла збільшуватися тільки поступінно,
а не скоками. Але на якомусь певному ступені розвитку велика
промисловість стає і технічно в суперечність із своєю ремісничою
та мануфактурною основою. Збільшення розміру рухових
машин, передатного механізму та виконавчих машин, збільшення
складности, різноманітности і точної правильности складових
частин виконавчої машини в міру того, як вона відривається від
того ремісничого зразка, що спочатку цілком визначає її будову, і
дістає вільну форму, яку визначає тільки її механічне завдання,\footnote{
Механічний ткацький варстат у своїй першій формі складається
переважно з дерева, поліпшений, сучасний — із заліза. До якої міри стара
форма засобу продукції напочатку опановує його нову форму, показує,
між іншим, найповерховіше порівняння сучасного парового ткацького
варстату з давнім, або сучасних роздмухових пристроїв на ливарнях
з першим безпорадним механічним відродженням звичайного ковальського
міха, і, може, ще влучніше, ніж усе інше, перший льокомотив, що його
пробували збудувати ще перед винаходом теперішніх льокомотивів: у
}
\index{i}{0313}  %% посилання на сторінку оригінального видання
розвиток автоматичної системи та щораз неминучіше вживання
матеріялу, який важко переробити, наприклад, заліза замість
дерева, — розв’язання всіх цих завдань, що виникали стихійно,
натрапляло всюди на особисті перешкоди, які навіть комбінований
робітничий персонал мануфактури міг усунути лише
до певного ступеня, але не по суті. Таких машин, як от, наприклад,
сучасний друкарський прес, сучасний паровий ткацький
варстат та сучасна чухральна машина, не могла дати
мануфактура.

Переворот у способі продукції в одній сфері промисловості
зумовлює переворот у способі продукції і в інших сферах. Це
має силу насамперед для таких галузей промисловости, які,
хоч суспільним поділом праці так ізольовані, що кожна з них
продукує якийсь самостійний товар, проте переплітаються одна
з однією як фази одного якогось цілого процесу. Так, машинове
прядіння зробило доконечним машинове ткання, а одне й друге
разом — механічно-хемічну революцію в білінні, друкуванні
та фарбуванні. Таким саме чином, з другого боку, революція в
прядінні бавовни зумовила винахід gin’a, машини до відділювання
бавовняних волокон від насіння, через що лише й зробилася
можлива продукція бавовни в потрібному тепер великому маштабі.\footnote{
Cottongin,\footnote*{
Машина, що вибирає зерно з бавовни. \emph{Ред.}
} винайдений одним янкі, Елія Вайтнеєм, до найновіших
часів у головному зазнав менше змін, ніж якабудь інша машина XVIII віку.
Лише останніми десятиліттями (перед 1867 р.) другий американець, пан
Імрі з Альбані, в Нью-Йорку, за допомогою простого й доцільного
поліпшення зробив машину Вайтнея антикварною річчю.
}
А революція у способі продукції промисловости і рільництва
зробила доконечною й революцію в загальних умовах суспільного
процесу продукції, тобто в засобах комунікації і транспорту.
Як засоби комунікації й транспорту суспільства, що його
pivot,\footnote*{
Точка, що довкола неї все обертається, стрижень. \emph{Ред.}
} уживаючи вислову Фур’є, було дрібне рільництво з його
домашньою допомічною промисловістю та міське ремество, уже
ніяк не могли задовольняти потреб продукції мануфактурного
періоду з його поширеним поділом суспільної праці, з його концентрацією
засобів праці та робітників і з його колоніальними
ринками, — а тому й дійсно зазнали перевороту, — так само й
засоби транспорту й комунікації, що перейшли від мануфактурного
періоду, перетворились незабаром у нестерпні пута для
великої промисловости з її гарячковим темпом продукції, з її масовими
розмірами, з її постійним перекидуванням мас капіталу й
робітників з однієї сфери продукції до іншої та з її новостворюваними
світовими ринковими зв’язками. Тим то, не кажучи вже

нього було дійсно таки дві ноги, які він навпереміну підносив, як кінь.
Тільки з дальшим розвитком механіки та з нагромадженням практичного
досвіду форма машини починає цілком визначатися принципами механіки
і тим то цілком емансипується від давньої форми того знаряддя, яке розвивається
на машину.
\parbreak{}  %% абзац продовжується на наступній сторінці

\input{i/_0314.tex}
\parcont{}  %% абзац починається на попередній сторінці
\index{iii1}{0315}  %% посилання на сторінку оригінального видання
первісна громада), привласнювачем, отже, продавцем продукту
є рабовласник, феодал, держава, яка стягає дань. Купець купує
і продає для багатьох. В його руках концентруються купівлі
й продажі, в наслідок чого купівля й продаж перестають
бути зв’язаними з безпосередніми потребами покупця (як
купця).

Але яка б не була суспільна організація тих сфер виробництва,
обмін товарів яких опосереднює купець, його майно завжди
існує як грошове майно і його гроші завжди функціонують
як капітал. Форма цього капіталу завжди є Г — Т — Г'; гроші,
самостійна форма мінової вартості — вихідний пункт, і збільшення
мінової вартості — самостійна мета. Самий товарний обмін і операції,
що опосереднюють його, — відокремлені від виробництва
і виконувані не-виробниками, — є просто засіб збільшення, збільшення
не просто багатства, але багатства в його загальній суспільній
формі, багатства як мінової вартості. Спонукальний мотив
і визначальна мета полягає в тому, щоб перетворити Г в Г + ΔГ;
акти Г — Т і Т — Г', які опосереднюють акт Г — Г', виступають
лиш як перехідні моменти цього перетворення Г в Г + ΔГ. Це
Г — Т — Г' як характерний рух купецького капіталу відрізняє його
від Т — Г — Т, тієї товарної торгівлі між самими виробниками,
кінцевою метою якої є обмін споживних вартостей.

Тому, чим менш розвинене виробництво, тим більше грошове
майно концентрується в руках купців або тим більше воно виступає
як специфічна форма купецького майна.

За капіталістичного способу виробництва, — тобто коли капітал
опановує само виробництво і надає йому цілком зміненої
і специфічної форми, — купецький капітал виступає тільки як
капітал в особливій функції. За всіх попередніх способів виробництва,
і тим більше, чим більше виробництво є безпосередньо
виробництво засобів існування виробника, купецький капітал
виступає як функція капіталу par excellence.

Отже, не становить ні найменшої трудності зрозуміти, чому
купецький капітал як історична форма капіталу з’являється задовго
до того, як капітал підпорядкував собі само виробництво.
Його існування й розвиток до певної висоти самі є історичною
передумовою для розвитку капіталістичного способу
виробництва, 1) як попередня умова концентрації грошового
майна і 2) тому що капіталістичний спосіб виробництва передбачає
виробництво для торгівлі, збут у великому масштабі і не
окремим покупцям, отже, передбачає також купця, який купує
не для задоволення своєї особистої потреби, а в своєму акті купівлі
концентрує акти купівлі багатьох. З другого боку, весь
розвиток купецького капіталу впливає таким чином, що все
більше й більше надає виробництву характеру виробництва, що
має за мету мінову вартість, дедалі більше перетворює продукти
в товари. Однак, його розвиток, узятий сам по собі, є, як
ми це побачимо зразу далі, недостатній для того, щоб викликати
\parbreak{}  %% абзац продовжується на наступній сторінці

\parcont{}  %% абзац починається на попередній сторінці
\index{i}{0316}  %% посилання на сторінку оригінального видання
сили суспільної праці. Природні сили, як от пара, вода й~\abbr{т. д.},
що їх уживається до продуктивних процесів, теж нічого не
коштують. Але як людина потребує легенів, щоб дихати, так само
потребує вона якогось «витвору людської руки», щоб продуктивно
споживати природні сили. Водяне колесо потрібне, щоб
експлуатувати рушійну силу води, парова машина — щоб експлуатувати
пружність пари. З наукою справа така сама, як і з силами
природи. Коли вже відкрито закон про відхилення магнетової
голки у сфері дії електричного струму або про утворення магнетизму
в залізі електричним струмом, то ці закони не коштують
ані шеляга.\footnote{
Наука взагалі «нічого» не коштує капіталістові, та це ані трохи
не заважає йому експлуатувати її. Капітал присвоює собі «чужу» науку
так само, як і чужу працю. Але «капіталістичне» присвоєння й «особисте»
присвоєння, чи то науки, чи то матеріяльного багатства — це цілком різні
речі. Сам д-р Юр нарікав на грубу необізнаність з механікою його любих
фабрикантів, що експлуатують машини, а Лібіґ оповідає про таке неуцтво
в хемії англійських хемічних фабрикантів, від якого волосся стає дибом.
} Але, щоб експлуатувати ці закони для телеграфії
і~\abbr{т. д.}, треба дуже дорогого та складного апарату. Як ми бачили,
машина не витискує знаряддя. З карликового знаряддя людського
організму виростає воно розміром та кількістю на знаряддя механізму,
створеного людиною. Капітал примушує тепер робітника
працювати не ручним знаряддям, а машиною, що сама орудує
своїм знаряддям. Тим то, коли на перший же погляд ясно, що
велика промисловість, включаючи в процес продукції велетенські
природні сили та природознавство, мусить надзвичайно
підвищити продуктивність праці, то ніяк не є так само ясно,
чи не купується цю підвищену продуктивну силу збільшеною
витратою праці на другому боці.\footnote*{
У французькому виданні це місце подано так: «Тим то, коли на
перший же погляд ясно, що механічна промисловість, включаючи в свій
склад науку та велетенські природні сили, надзвичайно підвищує продуктивність
праці, то можна, однак, запитати, чи не втрачається на другому
боці те, що виграється на одному, чи економізується вживанням
машин більше праці, ніж коштує їх продукція та утримання» («Le Capital
etc.», v. I, ch. XV, p. 168). \emph{Ред.}
} Подібно до кожної іншої складової
частини сталого капіталу машина не утворює вартості,
але віддає свою власну вартість продуктові, що для його продукції
вона служить. Оскільки вона має вартість, і тому переносить
цю вартість на продукт, остільки вона й становить складову
частину вартости продукту. Замість його здешевлювати, вона
удорожчує його відповідно до своєї власної вартости. І це ясна
річ, що машина та розвинена система машин, цей характеристичний
засіб праці великої промисловости, мають куди більшу вартість,
аніж засоби праці ремісничого або мануфактурного виробництва.
Насамперед треба тут зауважити, що машина завжди цілком
увіходить у процес праці й завжди тільки частинно в процес
утворення вартости. Вона ніколи не додає більше вартости,
ніж пересічно втрачає через своє зужиткування. Отже, є велика
\parbreak{}  %% абзац продовжується на наступній сторінці

\parcont{}  %% абзац починається на попередній сторінці
\index{iii1}{0317}  %% посилання на сторінку оригінального видання
виробництва зв’язуються між собою за допомогою третього члена,
виражає двоякого роду обставини. З одного боку, воно виражає
те, що циркуляція ще не опанувала виробництва, а відноситься до
нього, як до даної передумови. З другого боку, те, що процес
виробництва ще не ввібрав у себе циркуляцію просто як свій
момент. Навпаки, в капіталістичному виробництві має місце і те
і друге. Процес виробництва цілком грунтується на циркуляції, а
циркуляція є лиш момент, перехідна фаза виробництва, лиш реалізація
продукту, виробленого як товар, і заміщення елементів
його виробництва, вироблюваних як товари. Форма капіталу, що
походить безпосередньо з циркуляції — торговельний капітал —
виступає тут лиш як одна з форм капіталу в його русі репродукції.

Той закон, згідно з яким самостійний розвиток купецького капіталу
стоїть у зворотному відношенні до ступеня розвитку капіталістичного
виробництва, з особливою ясністю виявляється в історії
посередницької торгівлі (carrying trade), наприклад, у венеціанців,
генуезців, голландців і т. д., отже, там, де головний бариш
добувається не за допомогою вивозу продуктів своєї країни,
а від посередництва при обміні продуктів громад, торговельно
і взагалі економічно ще не розвинених, та від експлуатації обох
країн виробництва.\footnote{
„Жителі торговельних міст привозили з багатших країн витончені мануфактурні
товари і дорогі предмети розкоші і таким чином давали поживу для
чванливості великих землевласників, які жадібно купували ці товари і сплачували
за них величезні маси сировинного продукту своїх земель. Таким чином торгівля
значної частини Европи в цей час полягала в обміні сировинного продукту
однієї країни на мануфактурні продукти країни, промислово більш розвиненої...
Як тільки цей смак став загальнопоширеним і викликав значний попит, купці,
щоб заощадити витрати провозу, почали засновувати подібні мануфактури
у своїй власній країні“ (\emph{A. Smith}: „Wealth of Nations“, книга III, розд. Ill [вид.
Wakefield, Лондон 1835/39, т. З, стор. 41 і далі]).
} Тут перед нами купецький капітал у чистому
вигляді, відокремлений від крайніх членів, від тих сфер
виробництва, між якими він є посередником. Таке є головне
джерело його утворення. Але ця монополія посередницької торгівлі,
а з нею й сама ця торгівля, занепадає в тій самій мірі,
в якій прогресує економічний розвиток тих народів, що їх вона
експлуатувала з двох сторін і нерозвиненість яких була базою
її існування. При посередницькій торгівлі це являє собою не
тільки занепад особливої галузі торгівлі, але й падіння переваги
чисто торговельних народів і їх торговельного багатства
взагалі, яке грунтувалось на базі цієї посередницької торгівлі.
Це — тільки особлива форма, в якій у ході розвитку капіталістичного
виробництва виражається підпорядкування торговельного
капіталу промисловому. Зрештою, яскравий приклад
того, як господарює купецький капітал там, де він прямо опановує
виробництво, становить не тільки колоніальне господарство
взагалі (так звана колоніальна система), але особливо господарство
старої голландсько-ост-індської компанії.

\parcont{}  %% абзац починається на попередній сторінці
\index{ii}{0318}  %% посилання на сторінку оригінального видання
авансує промисловому капіталістові грошовий капітал (в найточнішому
значенні цього слова, тобто капітальну вартість у грошовій формі), то
справжнім пунктом повороту цих грошей є кишеня цього грошового
капіталіста. Таким чином, хоч гроші в своїй циркуляції більш або менш
переходять через усякі руки, маса грошей, що циркулюють, належить
підрозділові грошового капіталу, організованому і сконцентрованому в
формі банків тощо; спосіб, що ним цей підрозділ авансує свій капітал,
зумовлює постійний кінцевий, зворотний приплив до нього цього капіталу
в грошовій формі, хоч це знову таки упосереднюється зворотним перетворенням
промислового капіталу на грошовий капітал.

Для товарової циркуляції завжди потрібні дві умови; товари, подавані
в циркуляцію, і гроші, подавані в циркуляцію. „Процес циркуляції... не
закінчується, як безпосередній обмін продуктами, після того як споживні
вартості перемінили місця або посідачів. Гроші не зникають тому, що
вони, кінець-кінцем, випали з ряду метаморфоз якогось товару. Вони раз-у-раз
осідають у тих пунктах циркуляції, що їх звільняють ті або інші
товари“. (Книга І, розд. III, 2 п. а).

Напр., розглядаючи циркуляцію між II с і І ($v + m$), ми припустили,
що II підрозділ авансував для цієї циркуляції 500 ф. стерл. грішми. При
безмежному числі тих процесів циркуляції, що на них сходить циркуляція
між великими суспільними групами продуцентів, продуцент то
однієї, то другої групи спершу виступає як покупець, отже, подає
гроші в циркуляцію. Цілком лишаючи осторонь індивідуальні обставини,
це зумовлено вже неоднаковістю періодів продукції, а тому й оборотів
різних товарових капіталів. Отже, II на 500 ф. стерл. купує у І засобів
продукції на таку саму суму вартости, а І купує у II засобів споживання на
500 ф. стерл.; отже, гроші припливають назад до II; останній ані трохи
не збагачується таким зворотним припливом. Спочатку він подав у циркуляцію
500 ф. грішми і вилучив звідти товарів на ту саму суму вартости,
потім він продає товарів на 500 ф. стерл. і вилучає з циркуляції
таку саму суму вартости в грошах; таким чином 500 ф. стерл. припливають
назад. В дійсності II підрозділ подав таким чином у циркуляцію
на 500 ф. стерл. грошей і на 500 ф. стерл. товарів = 1000 ф. стерл.;
він вилучає з циркуляції на 500 ф. стерл. товарів і на 500 ф. стерл.
грошей. Для обміну 500 ф. стерл. товарами (І) і 500 ф. стерл. товарами
(II) циркуляція потребує лише 500 ф. стерл. грішми; отже,
хто на закуп чужого товару авансував гроші, той одержує їх
назад, продаючи власний товар. Тому, коли б спочатку І купив у II
товару на 500 ф. стерл., а потім продав би підрозділові II товару на
500 ф. стерл., то 500 ф. стерл. повернулись би до І, а не до II.

Гроші, витрачені на заробітну плату, тобто змінний капітал, авансований
у грошовій формі, в клясі І повертаються в цій формі не безпосередньо,
а посередньо, обкружним шляхом. Навпаки, в клясі II 500 ф.
стерл. заробітної плати повертаються безпосередньо від робітників до
капіталістів, як і взагалі цей зворотний приплив завжди є безпосередній
у всіх тих випадках, коли купівля та продаж між тими самими особами
\parbreak{}  %% абзац продовжується на наступній сторінці

\parcont{}  %% абзац починається на попередній сторінці
\index{i}{0319}  %% посилання на сторінку оригінального видання
з якою обертається веретено, або від числа вдарів, що їх молот
робить за одну хвилину. Деякі з тих колосальних молотів дають
70 ударів, Ryder’ова патентована ковальська машина, що вживає
парового молота невеликих розмірів на кування веретен, дає
700 ударів на одну хвилину.

Якщо дано ту пропорцію, в якій машина переносить вартість
на продукт, то величина цієї частини вартости залежить від величини
вартости самої машини\footnote{
Читач, полонений капіталістичними уявленнями, певна річ, дивується,
що тут немає мови про «процент», що його машина, пропорційно
до своєї капітальної вартости, додає до продукту. Однак, легко зрозуміти,
що машина, — тому що вона, як і будь-яка інша складова частина
сталого капіталу, не створює нової вартости, — не може додавати такої
вартости і під назвою «процент». Далі, ясно, що тут, де йдеться про продукування
додаткової вартости, не можна жодної частини її припустити
a priori під назвою «процент». Капіталістичний спосіб обчислення,
який prima facie\footnote*{
— на перший погляд. \emph{Ред.}
} видається безглуздим та суперечним законам утворення
вартости, ми пояснимо в третій книзі цього твору.
}. Що менше праці вона сама містить
у собі, то менше вартости додає вона до продукту. Що менше
вартости віддає вона, то продуктивніша вона й то більш наближається
її служба до служби сил природи. А продукція машин за
допомогою машин зменшує їхню вартість порівняно з їх розмірами
і їхньою дією.

Порівняльна аналіза цін на товари, продуковані ремісничим
або мануфактурним способом, та цін на ті самі товари як продукти
машин, дає взагалі такий результат, що в машиновому продукті
складова частина вартости, яку до нього додає засіб праці, відносно
зростає, але абсолютно меншає. Це значить, що її абсолютна
величина меншає, але її величина супроти загальної вартости
продукту, наприклад, одного фунта пряжі, більшає\footnote{
Ця додавана машиною складова частина вартости падає абсолютно
й відносно там, де машина витискує коні, взагалі робочу худобу,
уживану виключно як рухову силу, а не як машини для обміну речовин.
До речі зауважимо, що Декарт, визначаючи тварини як прості машини,
дивиться очима мануфактурного періоду, відмінно від середньовіччя,
яке вважало тварину за помічника людини так само, як пізніше й пан
фон Галлер в його «Restauration der Staatswissenschaften». Що Декарт
так само, як і Бекон, розглядав зміну способу продукції та практичне
опанування природи людиною як результат зміненої методи думання, —
це показує його «Discours de la Méthode», де, між іншим, сказано: «Можна
(за допомогою методи, яку він увів у філософію) дійти знань, дуже корисних
у житті, і замість тієї спекулятивної філософії, якої навчають по школах,
знайти практичну філософію, за допомогою якої, знаючи силу та дію
огню, води, повітря, зірок і всіх інших навкольних тіл так само достеменно,
як ми знаємо різні ремества наших ремісників, ми могли б тим
самим способом вживати їх для всього того, на що вони придатні, і таким
чином зробитися хазяїнами й владарями природи» та тим самим «пособляти
поліпшенню людського життя» («Il est possible de parvenir à des connaissances
fort utiles à la vie, et qu'au lieu de cette philosophie spéculative
qu’on enseigne dans les écoles, on en peut trouver une pratique, par laquelle,
connaissant la force et les actions du feu, de l’eau, d’air, des astres, et de
tous les autres corps qui nous environnent, aussi distinctement que nous
connaissons les divers métiers de nos artisans, nous les pourrions employer
en même façon à tous les usages auxquels ils sont propres, et ainsi nous
rendre comme maîtres et possesseurs de la nature\dots{} contribuer au
perfectionnement de la vie humaine»). У передмові до «Discourses upon Trade»
(1691 p.) сера Дудлея Норта сказано, що метода Декарта, застосована
до політичної економії, почала визволяти її від стародавніх казок і забобонних
уявлень про гроші, торговлю й~\abbr{т. д.} Загалом, однак, англійські
економісти давніх часів приєднуються до філософії Бекона й Гоббса, тимчасом
як пізніш «філософом» χατ’ ε’ξοχη'ν\footnote*{
— переважно. \emph{Ред.}
} політичної економії для Англії,
Франції та Італії став Льокк.
}.

\index{i}{0320}  %% посилання на сторінку оригінального видання
Ясна річ, що коли продукція якоїсь машини коштує стільки ж
праці, скільки заощаджується при вживанні її, то відбувається
просте переміщення праці, тобто загальна сума праці, потрібна
на продукцію товару, не меншає, або продуктивна сила праці не
більшає. Однак ріжниця між працею, якої коштує машина, і
тією працею, яку вона заощаджує, або ступінь її продуктивности,
не залежить, очевидно, від ріжниці між її власного вартістю й
вартістю того знаряддя, яке вона заміняє. Ця ріжниця триває
так довго, поки трудові витрати на машину, а тому й та частина
вартости, яку вона додає до продукту, лишаються меншими від
тієї вартости, яку робітник із своїм знаряддям додав би до предмету
праці. Тим то продуктивність машини вимірюється тим
ступенем, у якому вона заміняє людську робочу силу. За Бейнсом,
на 450 веретен-мюлів із підготовчими машинами, що їх рухає
одна парова кінська сила, припадає 2\sfrac{1}{2} робітника\footnote{
За річним звітом торговельної палати в Ессені (жовтень 1863~\abbr{р.})
сталеливарня Круппа за допомогою 161 перетопних, гартівних та цементових
печей, 32 парових машин (1800~\abbr{р.} це було приблизно загальне число
парових машин, що вживалися в Менчестері) та 14 парових молотів, —
які разом репрезентували \num{1.236} кінських сил, — 49 ковальських горен,
203 виконавчих машин та приблизно \num{2.400} робітників — випродукувала
1862~\abbr{р.} 13 мільйонів фунтів литої сталі. Тут на одну кінську силу немає
й двох робітників.
}, і кожне
автоматичне веретено mule випрядає за десятигодинного робочого
дня 13 унцій пряжі (пересічного нумера), отже, 2\sfrac{1}{2} робітника
випрядають 365\sfrac{5}{8} фунтів пряжі на тиждень. Отже, при своєму
перетворенні на пряжу приблизно 366 фунтів бавовни (для спрощення
ми залишаємо осторонь відпадки) вбирають лише 150 робочих
годин, або 15 десятигодинних робочих днів, тимчасом як із
самопрядом, якщо ручний прядун дає за 60 годин 13 унцій пряжі,
та сама кількість бавовни забрала б \num{2.700} десятигодинних робочих
днів, або \num{27.000} робочих годин\footnote{
Беббедж обчислює, що на Яві самою лише працею прядіння до
вартости бавовни додається майже 117\%. У той самий час (1832) в Англії
загальна вартість, яку при тонкопрядінні додавали до бавовни машини
і праця, становила приблизно 33\% вартости сировинного матеріялу,
(«On the Economy of Machinery», London 1832, p. 214).
}. Там, де стару методу
blockprinting, або ручного вибивання перкалю, витиснуло машинове
вибивання, одним-одна машина за допомогою одного чоловіка
або підлітка вибиває за одну годину стільки саме чотирибарвного
перкалю, скільки раніше вибивало 200 чоловіка\footnote{
Окрім того, при машиновому вибиванні заощаджується фарбу.
}.
\parbreak{}  %% абзац продовжується на наступній сторінці

\parcont{}  %% абзац починається на попередній сторінці
\index{i}{0321}  %% посилання на сторінку оригінального видання
Доки Елія Вайтней не вигадав 1793~\abbr{р.} cottongin’y, відділення одного
фунта бавовни від насіння коштувало пересічно один робочий
день. У наслідок його винаходу одна негритянка могла добувати
на день 100 фунтів бавовни, а від того часу продуктивність cottongin’y
ще значно збільшилася. Один фунт бавовняних волокон,
продукованих раніше за 50 центів, продавався пізніше по
10 центів з більшим зиском, тобто із включенням більшої кількости
неоплаченої праці. В Індії для відділення волокон від насіння
вживають інструмента, щось наче напівмашини, що зветься
churka; ним один чоловік та одна жінка чистять 28 фунтів на день.
Інструментом churka, що його винайшов перед кількома роками д-р Форбс, один
чоловік та один підліток чистять 250 фунтів на
день; там, де волів, пари або води вживають як рушійної сили,
потрібно лише декількох підлітків та дівчаток у ролі feeders
(подавальників матеріялу до машини). Шістнадцять таких машин,
що їх женуть волами, виконують щодня ту саму працю, яку раніш
виконували пересічно 750 чоловіка\footnote{
Порівн. «Paper read by Dr. Watson, Reporter on Products to the
Government of India, before the Society of Arts», 17 April 1860.
}.

Як уже згадано, парова машина при паровому плузі виконує
протягом однієї години, за 3\pens{ пенси}, або за \sfrac{1}{4}\shil{ шилінґа}, стільки ж
праці, скільки 66 осіб по 15\shil{ шилінґів} виконують протягом однієї
години. Я повертаюсь до цього прикладу, щоб запобігти помилковому
уявленню. А саме, ці 15\shil{ шилінґів} зовсім не є вираз праці, яку
додають протягом однієї години 66 робітників. Якщо відношення
додаткової праці до доконечної праці становило 100\%, то ці
66 робітників продукували за годину вартість у 30\shil{ шилінґів},
хоч в еквіваленті, який вони сами діставали, тобто в заробітній
платі в 15\shil{ шилінґів}, втілено лише 33 години. Отже, коли ми припустимо,
що машина коштує стільки ж само, скільки й річна
плата витиснених нею 150 робітників, приміром, \num{3.000}\pound{ фунтів
стерлінґів}, то ці \num{3.000}\pound{ фунтів стерлінґів} зовсім не є грошовий
вираз праці, постаченої й доданої до предмету праці цими 150 робітниками,
а є вони грошовий вираз лише тієї частини їхньої
річної праці, яка для них самих виражається в заробітній платі.
Навпаки, грошова вартість машини в \num{3.000}\pound{ фунтів стерлінґів}
виражає всю ту працю, яку витрачено на її продукцію, хоч би
в якій пропорції ця праця становила заробітну плату для
робітника й додаткову вартість для капіталіста. Отже, якщо
машина коштує стільки ж само, скільки коштує замінювана нею
робоча сила, то упредметнена в ній самій праця завжди куди
менша, ніж жива праця, яку вона заміняє\footnote{
«Ці німі діячі (машини) завжди є продукт значно меншої праці,
ніж та, яку вони заміняють, навіть і тоді, коли їхня грошова вартість
однакова» («These mute agents are always the produce of much less labour
than that which they displace, even when they are of the same money
value»). (\emph{Ricardo}: «Principles of Political Economy», 3 rd ed. London
1821, p. 40).
}.

Якщо розглядати машини виключно як засіб удешевлення
\parbreak{}  %% абзац продовжується на наступній сторінці

\parcont{}  %% абзац починається на попередній сторінці
\index{i}{0322}  %% посилання на сторінку оригінального видання
продукту, то межу вживання їх дано тим, що їх власна продукція
коштує менше праці, ніж праця, замінювана вживанням машин.
Однак, для капіталу ця межа визначається вужче. Через те, що
він оплачує не працю, якої вжито, а вартість ужитої робочої
сили, то для нього вживання машини обмежується ріжницею
між вартістю машини й вартістю тієї робочої сили, яку машина
заміняє. А що поділ робочого дня на доконечну працю й додаткову
працю по різних країнах є різний, так само як і в тій самій
країні він різний в різні періоди або в той самий період у різних
галузях продукції; що, далі, дійсна заробітна плата робітника
то падає нижче вартости його робочої сили, то підноситься понад
неї, то ріжниця між ціною машин і ціною робочої сили, яку ці
машини мають замінити, може дуже коливатися, навіть і тоді,
коли ріжниця між кількістю праці, потрібної для продукції
машини, і загальною кількістю праці, яку вона заміняє, лишається
без зміни.\footnoteA{
Примітка до другого видання. Тим то в комуністичному суспільстві
вживання машин мало б зовсім інший обсяг, ніж у суспільстві
буржуазному.
}  Але лише перша ріжниця визначає для самого
капіталіста витрати продукції товару та впливає на нього через
примусові закони конкуренції. Тим то в Англії нині винаходять
машини, яких уживають лише в Північній Америці, як у XVI
та XVII віці Німеччина винаходила машини, що їх уживала
лише Голляндія, і як деякі французькі винаходи XVIII віку
використовувано лише в Англії. Сама машина продукує в давніше
розвинених країнах через те, що її вживають у деяких галузях
підприємства, такий надмір праці (redundancy of labour, каже
Рікардо) по інших галузях, що тут падіння заробітної плати нижче
вартости робочої сили перешкоджає вживанню машин та робить
його зайвим, а часто й неможливим з погляду капіталу, зиск
якого і без того випливає із зменшення не просто вживаної ним
праці, а лише праці, ним оплаченої. По деяких галузях англійської
вовняної мануфактури останніми роками дитяча праця дуже
зменшилася, подекуди її майже витиснено. Чому? Фабричний
закон примусив до подвійної зміни дітей, що з них одна працює
6 годин, друга 4 години, або кожна лише по 5 годин. Але батьки
не хотіли продавати half-times (робітників половинного часу)
дешевше, ніж раніш продавали full-times (робітників повного
часу). Звідси заміна half-times машинами.\footnote{
«Підприємці не стануть без доконечности тримати дві зміни дітей
молодших за тринадцять років\dots{} Справді, одна кляса фабрикантів, що
прядуть вовну, тепер рідко вживає дітей молодших за 13 років, тобто
half-times. Вони позаводили нові машини та поліпшення різного роду,
які майже усувають працю дітей (тобто дітей до 13 років). Для ілюстрації
такого зменшення числа дітей я нагадаю про один процес праці, в якому
через додаток до тодішніх машин одного апарату, так званого piecing
machine, працю шістьох або чотирьох half-times, відповідно до особливости
кожної машини, може виконати один підліток (старший за
13 років). Система половинного часу» стимулювала «винахід присукувальної
машини». («Employers of labour would not unnecessarily retain two sets
fo children under thirteen\dots{} In fact one class of manufacturers, the spinners
of woollen yarn, now rarely employ children under thirteen years ages, i. e.
half-times. They have introduced improved and new machinery of various
kinds which altogether supersedes the employment of children; f. і: I will
mention one process as an illustration of this diminution in the number
of children, wherein, by thy addition of an apparatus, called a piecingmachine,
to existing machines, the work of six or four half-times, according
to the peculiarity of each machine, can be performed by one young
person\dots{} the half-time system» стимулювала «the invention of the
piecing-machine»). (Reports of Insp of Fact, for 31 st October 1858»).
} Перед забороною
вживати жіночої та дитячої (нижче десятирічного віку) праці по
\index{i}{0323}  %% посилання на сторінку оригінального видання
копальнях капітал вважав за остільки згідне із своїм моральним
кодексом, і особливо із своєю головною книгою, примушувати
голих жінок та дівчат, часто разом із чоловіками, працювати по
вугільних та інших копальнях, що лише після тієї заборони він
удався до машин. Янкі винайшли машини розбивати камінь.
Англійці їх не вживають, бо «нещасний» (wretch — технічний
вислів в англійській політичній економії на означення рільничих
робітників), що виконує цю працю, дістає оплату такої
незначної частини своєї праці, що машини удорожчили б цю
продукцію для капіталіста.\footnote{
«Машини\dots{} часто не можуть знайти вжитку доти, доки праця (він
має на оці заробітну плату) не піднесеться» («Machinery\dots{} can frequently
not be employed until labour rises»). (\emph{Ricardo}: «Principles of
Political Economy», 3 rd. ed, London 1821, p. 479).
} В Англії замість коней іноді все
ще вживають жінок, щоб тягати барки каналами тощо,\footnote{
Див. «Report of the Social Science Congress at Edinburgh. October
1863».
} бо
праця, потрібна на продукцію коней та машин, є математично
дана кількість, тимчасом як праця, потрібна на утримання жінок
із надмірної людности, є нижча від усякого обрахунку. Тим то
ніде немає безсоромнішого марнотратства людської сили на всякі
дрібниці, як саме в Англії, в цій країні машин.

\subsection{Безпосередні діяння машинового виробництва на робітників}

За вихідний пункт великої промисловості є, як це вже
показано, революція в засобі праці, а зреволюціонізований засіб
праці набирає своєї найрозвиненішої форми в розчленованій
системі машин на фабриці. Перше ніж розглядати, як до цього
об’єктивного організму додається людський матеріял, розгляньмо
деякі загальні діяння тієї революції на самого робітника.

\subsubsection{Присвоювання капіталом додаткових робочих
сил. Жіноча та дитяча праця}

Оскільки машина робить мускульну силу зайвою, вона стає
засобом, щоб уживати робітників без мускульної сили або робітників
з недостатнім фізичним розвитком, але з більшою гнучкістю
членів. Тому жіноча й дитяча праця була першим словом капіталістичного
вживання машин! Таким чином цей могутній засіб
\parbreak{}  %% абзац продовжується на наступній сторінці

\parcont{}  %% абзац починається на попередній сторінці
\index{i}{0324}  %% посилання на сторінку оригінального видання
заміщувати працю та робітників перетворився відразу на засіб
збільшувати число найманих робітників, підбиваючи під безпосереднє
панування капіталу всіх членів родини без ріжниці статі
й віку. Примусова праця на капіталіста узурпувала не тільки час
дитячих забав, а ще й час вільної праці в колі домашніх, у прийнятих
звичаєм межах, для потреб самої родини\footnote{
Підчас бавовняної кризи, що супроводила американську громадянську
війну, англійський уряд послав д-ра Едварда Сміса до Ланкашіру,
Чешіру й~\abbr{т. д.}, щоб дати звіта про стан здоров’я бавовняних робітників.
Е. Сміс, між іншим, повідомляє: «З погляду гігієни криза, крім
того, що вона витиснула робітників із фабричної атмосфери, дала чимало
й інших користей. Жінки робітників мають тепер потрібний вільний
час, щоб нагодувати груддю своїх дітей замість отруювати їх Cordial’ем
Ґодфрея (препаратом з опію). Вони тепер мають час учитися варити страви».
На нещастя, припало це куховарство на той час, коли вони не мали
чого їсти. Але ми бачимо, як капітал для свого самозростання узурпував
працю родини, потрібну для самого споживання родини. Так само кризу
використано на те, щоб по окремих школах учити дочок робітників шити.
Отже, треба було американської революції та світової кризи, щоб дівчата-робітниці,
які прядуть для цілого світу, навчилися шити.
}.

Вартість робочої сили було визначено не тільки робочим
часом, потрібним, щоб утримати поодинокого дорослого робітника,
а ще й часом, потрібним, щоб утримати робітничу родину.
Викидаючи всіх членів робітничої родини на ринок праці, машини
розподіляють вартість робочої сили чоловіка на всю його родину.
Тому вони знижують вартість його робочої сили. Може бути купівля
родини, розпарцельованої на чотири робочі сили, коштує й
більше, ніж раніш коштувала купівля робочої сили голови родини,
але зате тепер чотири робочі дні стають на місце одного дня, і їхня
ціна падає пропорційно надлишкові додаткової праці чотирьох
над додатковою працею одного. Тепер для існування однієї родини
четверо мусять постачати капіталові не тільки працю, а ще й
додаткову працю. Таким чином машина з самого початку, разом
із збільшенням людського матеріялу експлуатації, цього справжнього
поля капіталістичного визиску\footnote{
«Збільшення числа робітників було велике в наслідок дедалі
більшої заміни праці чоловіків працею жінок, а особливо праці дорослих
працею дітей. Троє дівчаток у віці 13 років із заробітною платою від 6
до 8\shil{ шилінґів} на тиждень замінили дорослого чоловіка, що його плата
коливається між 18 і 45\shil{ шилінґами}». («The numerical increase of labourers
has been great, through the growing substitution of female for male, and
above all of childish for adult, labour. Three girls of 13, at wages from of
6 sh. to 8 sh. a week, have replaced the one man of mature age, of wages
varying from 18 sh. to 45 sh.»). (\emph{Th. de Quincey}: «The Logic of Political
Economy», London 1844, p. 147 n.). Через те, що певних функцій родини,
як от, приміром, догляд та годування груддю дітей і~\abbr{т. д.}, не можна зовсім
усунути, то матері родин, конфісковані капіталом, мусять сяк чи так наймати собі заступників. Роботи, яких потребує споживання родини,
наприклад, шиття, латання й~\abbr{т. д.}, доводиться заміняти купівлею готових
товарів. Отже, зменшенню витрати хатньої праці відповідає збільшення
грошових видатків. Тому витрати продукції робітничої родини зростають
та врівноважують збільшення доходу. До цього долучається ще й те, що
економія та доцільність у використовуванні й готуванні засобів існування
стають неможливі. Про ці факти, які офіціяльна політична економія
затаює, можна знайти багатий матеріял у «Reports» фабричних інспекторів,
у звітах «Children’s Employment Commission», а особливо в «Reports
on Public Health».
}, збільшує одночасно і ступінь експлуатації.

Машини також ґрунтовно революціонізують формальний вираз
капіталістичного відношення, контракт між робітником і капіталістом.
На основі товарового обміну першою передумовою було
те, що капіталіст і робітник протистояли один одному як вільні
\index{i}{0325}  %% посилання на сторінку оригінального видання
особи, як незалежні посідачі товарів, один — як посідач грошей
та засобів продукції, другий — як посідач робочої сили. А тепер
капітал купує неповнолітніх або напівповнолітніх. Раніш
робітник продавав свою власну робочу силу, що нею він порядкував
як формально вільна особа. Тепер він продає жінку
й дітей. Він стає работорговцем\footnote{
Протилежно до того великого факту, що обмеження праці жінок
та дітей по англійських фабриках одвоювали від капіталу дорослі робітники-чоловіки,
ми знаходимо ще в найновіших звітах «Children’s Employment
Commission» справді обурливі та цілком гідні работорговців риси
у батьків-робітників щодо баришування дітьми. Але капіталістичні
фарисеї, як це можна бачити з тих самих «Reports», ще й викривають цю,
ними самими утворену, увіковічнену та експлуатовану жорстокість, яку
вони в інших випадках називають «волею праці». «Дитячу працю покликано
на поміч\dots{} навіть для того, щоб діти заробляли собі свій щоденний
шматок хліба. Без сил, потрібних, щоб витримати таку надмірну
працю, без навчання, потрібного для спрямовання їхнього дальшого життя,
їх кинуто в таке становище, що руйнувало їх фізично й морально. Єврейський
історик зауважив з приводу зруйнування Єрусалиму Титом, що
немає нічого дивовижного в тому, що підчас руйнування Єрусалиму його
так страшенно сплюндрували, коли вже якась нелюдяна мати навіть пожертвувала
свого власного сина, щоб заспокоїти муки страшного голоду».
(«Infant labour has been called into aid\dots{} even to work for their own daily
bread. Without strength to endure such disproportionate toil, without
instruction to guide their future life, they have been thrown into a situation
physically and morally polluted. The Jewish historian has remarked upon
the ovethrow of Jerusalem by Titus, that is was no wonder it should have
been destroyed, with such a signal destruction, when an inhuman mother
sacrificed her own offspring to satisfy the cravings of absolute hunger»).
(«Public Economy Concentrated», Carlisle 1833, p. 66).
}. Попит на дитячу працю
часто і своєю формою подібний до попиту на негрів-рабів,
про який дуже часто можна було читати в об’явах американських
газет. «Мою увагу, — каже, приміром, один англійський
фабричний інспектор, — звернула на себе об’ява в місцевій
газеті одного з найзначніших мануфактурних міст моєї округи.
Ось копія цієї об’яви: Потрібно 12--20 хлопчаків, не молодших
від такого віку, щоб їх можна було вважати за 13-літніх.
Плата — 4\shil{ шилінґи} на тиждень. Спитати й~\abbr{т. д.}»\footnote{
 \emph{A. Redgrave} у «Reports of Insp. of Fact, for 31 st October
1858», p. 40, 41.
}. Речення:
«щоб їх можна було вважати за 13-літніх» пояснюється тим, що
за фабричним законом дітям, молодшим від 13 років, можна
працювати тільки 6 годин. Лікар, що має визнану урядом кваліфікацію
(certifying surgeon), мусить засвідчити вік. Отже,

\parbreak{}  %% абзац продовжується на наступній сторінці

\parcont{}  %% абзац починається на попередній сторінці
\index{i}{0326}  %% посилання на сторінку оригінального видання
фабрикант домагається таких хлопчаків, які мають вигляд, наче
їм уже минуло 13 років. Раптове — іноді — зменшення числа
дітей, молодших за 13 років, експлуатованих фабрикантами, яке
так дивує в англійській статистиці за останні 20 років, було, за
словами самих фабричних інспекторів, здебільша справою тих
certifying surgeons, що перекручували вік дітей відповідно до
експлуататорської жадоби капіталістів та баришницьких потреб
батьків. В Bethnal Green, цій ославленій окрузі Лондону, щопонеділка
й щовівтірка влаштовують одкритий базар, на якому
діти обох статей, починаючи від 9-літнього віку, сами себе наймають
на лондонські шовкові мануфактури. «Звичайні умови —
1\shil{ шилінґ} 8\pens{ пенсів} на тиждень (це належить батькам) і 2\pens{ пенси} —
для мене самого, та чай». Контракти мають силу тільки на тиждень.
Сцени й мова підчас цього базару справді обурливі\footnote{
«Children’s Employment Commission. 5 th Report», London 1866,
p. 81. n.31. [До 4 видання. Шовкову промисловість у Bethnal Green тепер
майже знищено. — \emph{Ф.~Е.}].
}. Ще й досі
трапляється в Англії, що жінки беруть «хлопчаків із робітного
дому й наймають їх якомубудь покупцеві за 2\shil{ шилінґи}
6\pens{ пенсів} на тиждень»\footnote{
«Children’s Employment Commission. 3 rd Report», London 1864,
p. 53, n. 15.
}. Всупереч законодавству ще й досі щонайменше
\num{2.000} хлопчаків продається у Великобрітанії їхніми
власними батьками як живі сажотрусні машини (дарма що існують
машини для заміни їх)\footnote{
Там же, 5 th Report, p. XXIII, n. 137.
}. Зумовлена машинами революція у
правних відносинах між покупцем та продавцем робочої сили,
позбавивши всю цю угоду навіть подоби контракту між вільними
особами, дала пізніш англійському парляментові юридичну
підставу виправдуватися за втручання держави у фабричний
режим. Щоразу, коли фабричний закон обмежує дитячу працю
в нереґляментованих досі галузях промисловости 6 годинами,
знову й знову лунає голосіння фабрикантів: частина батьків,
мовляв, забирають своїх дітей із реґляментованих тепер галузей
промисловости, щоб запродати їх на такі, де ще панує «воля
праці», тобто, де діти, молодші за 13 років, примушені працювати
як дорослі, отже, на такі галузі, де за них можна дорожче
взяти. А що капітал із своєї природи левелер, тобто вимагає,
як свого природженого права, рівности в умовах експлуатації
праці в усіх сферах продукції, то й законодавче обмеження дитячої
праці в одній галузі промисловости стає причиною обмеження
її в іншій галузі.

\looseness=-1
Ми вже раніш відзначали фізичний занепад дітей, підлітків
і жінок робітників, що їх машини кидають на експлуатацію капіталу
спочатку безпосередньо по фабриках, що виростають на
основі машин, а потім посередньо, в усіх інших галузях промисловости.
Тому ми тут спинимося лише на одному пункті, на жахливій
смертності робітничих дітей у перші роки їхнього життя.
\parbreak{}  %% абзац продовжується на наступній сторінці

\parcont{}  %% абзац починається на попередній сторінці
\index{i}{0327}  %% посилання на сторінку оригінального видання
В Англії є 16 реєстраційних округ, де на \num{100.000} дітей, молодших
за один рік, припадає пересічно лише \num{9.000} смертних випадків
на рік (в одній окрузі тільки \num{7.047}), у 24 округах — понад \num{10.000},
але менш як \num{11.000}, у 39 округах понад \num{11.000}, але менш як \num{12.000},
у 48 округах понад \num{12.000}, але менш як \num{13.000}, у 22 округах
понад \num{20.000}, у 25 округах понад \num{21.000}, у 17 — понад \num{22.000},
в 11 — понад \num{23.000}, в Ноо, Wolverhampton, Asthon-under-Lyne
і Preston — понад \num{24.000}, в Nottingham, Stockport і Bradford —
понад \num{25.000}, y Wisbeach — \num{26.000} та в Менчестері — \num{26.125}\footnote{
«Sixth Report on Public Health», London 1864, p. 34.
}.
Як показав один офіціяльний лікарський дослід 1861~\abbr{р.}, причина
таких високих норм смертности лежить, залишаючи осторонь
місцеві умови, головне у праці матерів поза власною хатою і в
тих умовах, що випливають із цього, а саме в занедбуванні та
мордуванні дітей, між іншим, у невідповідному харчуванні, в
недостачі харчу, в годуванні дітей препаратами опію і~\abbr{т. д.};
до цього долучається протиприродне відчуження матерів від
своїх дітей і, як наслідок цього, навмисне виголодовування й
отруювання\footnote{
«Він (дослід 1861~\abbr{р.})\dots{} показав, крім того, що серед зазначених
умов, з одного боку, діти вмирають через те, що матері, працюючи на фабриках,
занедбують своїх дітей та зле поводяться з ними, і що, з другого
боку, матері до такої міри втрачають природні почуття до власних
дітей, що смерть їхня не завдає їм жалю, а іноді навіть\dots{} вони просто вживають
усяких заходів, щоб заподіяти їм смерть» (Там же).
}. Навпаки, в таких рільничих округах, «де жіночої
праці знаходимо найменше, там і норма смертности найнижча»\footnote{
Там же, стор. 454.
}.
Слідча комісія з 1861~\abbr{р.} дала, однак, несподіваний результат,
а саме, що в деяких суто рільничих округах уздовж Німецького
моря норма смертности дітей, молодших за один рік, майже
досягає норми смертности найбільше вславлених із цього погляду
фабричних округ. Тим то докторові Джульєну Гентерові доручено
було розслідити це явище на місці. Його звіта долучено до
«VI Report on Public Health»\footnote{
Там же, стор. 454--463. «Reports by' Dr.~Henry Julian Hunter
on the excessive mortality of infants in some rural districts of England».
}. Досі гадали, що дітей гублять малярія
та інші недуги, властиві низьким та болотяним місцевостям.
Розслід дав якраз протилежний висновок, а саме, «що та сама
причина, яка знищила малярію, тобто перетворення на родючу
землю того ґрунту, який зимою був болотом, а влітку — злиденним
пасовиськом, — що ця сама причина зумовила надзвичайно
високу норму смертности немовлят»\footnote{
Там же, стор. 35, 455, 456.
}. Сімдесят лікарів, які
в цих округах практикували і яких прослухав д-р Гентер,
були «на диво однієї думки» щодо цього пункту. А саме, одночасно
з революцією в рільничій культурі заведено тут і промислову
систему. «Заміжніх жінок, що працюють разом із дівчатами й
хлопцями, чоловік віддає за певну суму в розпорядження орендареві,
який називається «Gangmeister» та наймає ввесь гурт.
\parbreak{}  %% абзац продовжується на наступній сторінці

\parcont{}  %% абзац починається на попередній сторінці
\index{ii}{0328}  %% посилання на сторінку оригінального видання
спродукованій протягом року в підрозділі II) плюс змінна капітальна
вартість І, репродукована протягом року, і новоспродукована додаткова
вартість І (тобто плюс вартість, спродукована протягом року в підрозділі
І).

Отже, припускаючи просту репродукцію, вся вартість спродукованих
протягом року засобів споживання дорівнює новоспродукованій річній
вартості, тобто дорівнює всій вартості, спродукованій суспільною працею
протягом року, і мусить їй дорівнювати, бо при простій репродукції
споживається всю цю вартість.

Цілий суспільний робочий день розпадається на дві частини: 1) доконечна
праця; протягом року вона утворює вартість в 1500 v; 2) додаткова
праця; вона утворює додаткову вартість в 1500 m. Сума цих
вартостей = 3000, дорівнює вартості спродукованих протягом року засобів
споживання в 3000. Отже, вся вартість спродукованих протягом
року засобів споживання дорівнює всій вартості, що її продукує цілий
суспільний робочий день протягом року, дорівнює вартості суспільного змінного
капіталу плюс суспільна додаткова вартість, дорівнює цілому річному
новому продуктові.

Але ми знаємо, що хоч ці дві величини вартости збігаються, а проте,
зовсім не всю вартість товарів II, засобів споживання, спродуковано
в цьому підрозділі суспільної продукції. Вони збігаються, бо стала
капітальна вартість, що знову з’являється в II, дорівнює вартості (змінній
капітальній вартості плюс додаткова вартість), новоспродукованій
в І; тому І (v + m) може купити в II ту частину продукту, яка для продуцентів
його (в підрозділі II) репрезентує сталу капітальну вартість. Відси
видно, чому, хоч для капіталістів II вартість їхнього продукту розпадається
на c + v + m, з погляду суспільства вартість цього продукту
можна розкласти на v + m. Але справа стоїть так лише тому, що тут
II c дорівнює І (v + m), і що ці дві складові частини суспільного продукту
через обмін навзаєм обмінюються своїми натуральними формами,
так що після такого обміну II c знову існує в формі засобів продукції,
а І (v + m), навпаки, — в формі засобів споживання.

Саме ця обставина й дала А. Смісові нагоду твердити, що вартість
річного продукту розкладається на v + m. Це правильно 1) лише для
тієї частини річного продукту, яка складається з засобів споживання, і
2) правильно не в тому розумінні, що всю цю вартість спродуковано в II і
що тому вартість продукту II дорівнює змінній капітальній вартості, авансованій
в II, плюс додаткова вартість спродукована, в II. А лише в тому
розумінні, що II (c + v + m) = II (v + m) + І (v + m), або тому, що
II c = І (v + m).

З цього далі випливає ось що:

Хоч суспільний робочий день (тобто праця, витрачена цілою робітничою
клясою протягом цілого року), так само, як і кожен індивідуальний
робочий день, розпадається лише на дві частини, а саме — на доконечну
працю плюс додаткову працю, отже, хоч і вартість, спродукована цим
робочим днем, теж розпадається лише на дві частини, а саме — на змінну
\parbreak{}  %% абзац продовжується на наступній сторінці

\input{i/_0329.tex}
\parcont{}  %% абзац починається на попередній сторінці
\index{i}{0330}  %% посилання на сторінку оригінального видання
вплив задушливого та гидкого повітря на бідні діти... Я побував
у багатьох таких школах, там я бачив цілі лави дітей, що абсолютно
нічого не робили; і ці діти мали посвідки, що ходили до
школи, а в офіціяльній статистиці вони фігурують як такі, що
дістали освіту (educated)».\footnote{
Leonhard Horner у «Reports etc. for 31 st October 1857», р. 17, 18.
} В Шотляндії фабриканти силкуються
по змозі не приймати на роботу дітей, що мусять ходити до школи.
«Цього досить, щоб показати велику неприхильність фабрикантів
до пунктів закону про виховання дітей»\footnote{
Sir J. Kincaid у «Reports etc. for 31 st October 1856», p. 66.
}. У неймовірно
жахливій формі виявляється це по перкалевибійних фабриках та
інших вибійнях, що підведені під осібний фабричний закон.
За постановами цього закону, «кожна дитина, раніш ніж її можна
прийняти до такої вибійні на роботу, мусить відвідувати школу
щонайменше 30 днів та не менш як 150 годин протягом 6 місяців
безпосередньо перед тим днем, коли вона вперше починає працювати.
Протягом того часу, коли дитина працює на цих вибійних
фабриках, вона так само мусить що шість місяців на рік ходити
до школи по 30 день, або 150 годин... Відвідувати школу треба
між 8 годиною ранку й 6 годиною по півдні. Відвідування школи,
що триває менше ніж 2 1/2  години або більше як 5 годин на день,
не може вважатися за частину тих 150 годин. За звичайних обставин
діти відвідують школу вранці й по півдні протягом 30 днів
по 5 годин на день, а після цих 30 днів, дійшовши встановленої
статутами повної суми в 150 годин, скінчивши, як вони сами
висловлюються, свою книжку, вони повертаються до вибійні
й лишаються там знову 6 місяців, доки знову прийде строк іти
до школи; тоді вони знову лишаються в школі доти, доки знову
скінчать свою книжку... Дуже багато підлітків, які відвідували
школу протягом приписаних 150 годин, вертаючись до неї після
шестимісячного перебування на фабриці, знають не більше, ніж
вони знали з самого початку... Певна річ, вони знову позабували
все, чого набралися раніш, відвідуючи школу. По інших перкалевибійних
фабриках відвідування школи геть чисто залежить від
потреб фабрики. Потрібне число годин протягом кожного півріччя
поповнюється зарахуванням 3—5-годинних відвідувань, що порозкидувані,
може, і по цілому півроці. Приміром, одного дня
школу відвідують від 8 до 11 години ранку, іншого дня — від
1 до 4 години по півдні, і після того, як дитина потім знову декілька
днів не відвідувала школу, вона раптом знову приходить
від 3 до 6 години по півдні; потім, може бути, вона ходить 3 або 4 дні
або й цілий тиждень підряд, а далі знову зникає на 3 тижні або
на цілий місяць та вертається на декілька годин у вільні дні,
коли підприємець випадково її не потребує; отак дитину, так би
мовити, кидають туди та сюди (buffeted), із школи до фабрики,
з фабрики до школи, поки нараховується сума в 150 годин».\footnote{
A. Redgrave у «Reports of Insp. of Fact. for 31 st October
1857», p. 41, 42. По тих галузях англійської промисловости, де від давні-
}
\index{i}{0331}  %% посилання на сторінку оригінального видання
Додаючи переважну кількість дітей та жінок до комбінованого
робочого персоналу, машина, кінець-кінцем, ламає опір, що
його чоловік-робітник у мануфактурі ставив ще деспотизмові
капіталу.\footnote{
«Пан E., фабрикант, повідомив мене, що коло своїх механічних
ткацьких варстатів він вживає виключно жіночої праці; він дає перевагу
заміжнім жінкам, особливо жінкам, що мають дома родину, утримання
якої залежить від них; вони куди уважніші та слухняніші, ніж незаміжні,
та мусять до крайности напружувати свої сили, щоб добувати собі доконечні
засоби існування. Таким чином чесноти, властиві жіночому характерові,
повертаються їм на шкоду, — таким чином усе моральне й ніжне
їхньої природи стає засобом їхнього поневолення та джерелом їхнього
страждання». («Ten Hours Factory Bill. The Speech of Lord Ashley, 15 th
March», London 1844, p. 20).
}

b) Здовження робочого дня

Якщо машина є якнаймогутніший засіб збільшувати продуктивну
силу праці, тобто скорочувати робочий час, потрібний для
продукції товару, то, як носій капіталу, стає вона насамперед
у безпосередньо охоплених нею галузях промисловости якнаймогутнішим
засобом здовжувати робочий день поза всяку природну
межу. Вона створює, з одного боку, нові умови, що дозволяють
капіталові давати повну волю цій своїй постійній тенденції,
з другого боку, вона створює нові мотиви до загострення його
ненажерливої жадоби чужої праці.

Насамперед рух та функціонування засобу праці в машині
усамостійнюється проти робітника. Засіб праці стає сам по собі
промисловим perpetuum mobile, яке продукувало б безнастанно,
коли б воно не натрапляло на певні природні межі у своїх помічниках
— людях: на слабощі їхнього тіла й на їхню сваволю.
Тому, як капітал, — а, як такий, автомат має в особі капіталіста
свою свідомість і волю, — засіб праці є надхнений прагненням
звести людські природні межі, що ставлять йому опір, але є елястичні,
до мінімуму опору.\footnote{
«З того часу, як повсюди заведено коштовні машини, людину
примусили працювати далеко більше, ніж їй пересічно під силу» («Sіnce
the general introduction of expensive machinery, human nature has been
forced far beyond its average strength»). (Robert Owen: «Observations
on the effects of the manufacturing system», 2 nd ed. London 1817).
} Але й без того цей опір зменшується
через позірну легкість праці коло машини та більшу покірливість
і гнучкість жіночого й дитячого елементу».\footnote{
Англійці, які охоче розглядають першу емпіричну форму виявлення
речі, як її причину, часто вважають за причину довгого робочого
}

шого часу панує фабричний закон у власному значенні (не Print Work's
Act, що його ми щойно навели в тексті), перешкоди проти пунктів про
виховання за останні роки до певної міри переборено. А в тих галузях
промисловости, які не підведені під фабричний закон, ще й досі цілком
панують погляди фабриканта скла, Дж. Ґедса, який так навчав члена слідчої
комісії Вайта: «Оскільки я розумію, більша освіта, яку дістала останніми
роками певна частина робітничої кляси, є лихо. Вона небезпечна,
бо робить робітників надто незалежними». («Children’s Employment Commission.
4 th Report», London 1865, p. 253).

\index{i}{0332}  %% посилання на сторінку оригінального видання
Продуктивність машин, як ми вже бачили, стоїть у зворотному
відношенні до величини тієї складової частини вартости, яку вона
переносить на продукт. Що довший той період, протягом якого
машина функціонує, то більша маса продуктів, на яку розділяється
додана нею вартість, і то менша та частина вартости
яку вона долучає до кожного окремого товару. Період активного
життя машини, очевидно, визначається довжиною робочого дня
або триванням денного процесу праці, помноженого на число
днів, у які цей процес повторюється.

Зужиткування машини зовсім не відповідає з математичною
точністю часові користування нею.\footnote*{
На берегах свого власного екземпляра першого видання Маркс
тут дає таку примітку: «Це має силу й щодо інших витрат, пов’язаних із
машинами. Наприклад: «Кожний фабрикант знає, що коли треба розігріти
парову машину, то здобути пару на 3 години коштує стільки ж саме,
скільки і на 4 години... Звідси постає (для залізниць) маленька економія
на паливі, коли перебіг робиться на велику віддаль». («Royal Commission
on Railways», London 1867. Evidence, p. 175). Ред.
} Та навіть, коли припустити
таку відповідність, то й тоді машина, яка служить протягом
7 1/2 років по 16 годин щоденно, охоплює такий самий великий
період продукції та додає до загального продукту не більше
вартости, ніж та сама машина, що протягом 15 років служить
лише по 8 годин на день. Але в першому випадку вартість машини
була б репродукована удвоє швидше, ніж в останньому, і капіталіст
у першому випадку за допомогою цієї машини проковтнув би
за 7 1/2  років стільки ж додаткової праці, скільки в другому випадку
за 15 років.

Матеріяльне зужиткування машини є двояке. Одно випливає
з її уживання — так само, як монети стираються від циркуляції,
друге — з невживання її, як от меч без ужитку ржавіє в піхвах,
В останньому випадку вона стає здобиччю стихій. Зужиткування
першого роду стоїть більше або менше у прямому відношенні,

часу на фабриках те велике іродське грабіжництво дітей, що його капітал
на податках фабричної системи практикував по домах для бідних та
сиріт, і за допомогою якого він здобув собі цілком безвільний людський
матеріял. Так, наприклад, Філден, сам англійський фабрикант, каже:
«Очевидно, робочий день здовжувала та обставина, що велика численність
безпритульних дітей, яких приводили з різних частин країни, унезалежнювала
підприємців від робочих рук, і вони, завівши за допомогою такого
нужденного, так добутого матеріялу, звичай довгої праці, дуже легко
могли накинути це і своїм сусідам» («It is evident that the long hours of
work were brought about by the circumstance of so great a number of destitute
children being supplied from different parts of the country, that the
masters were independent of the hands, and that, having once established
the custom by means of the miserable materials which they had procured in
this way, they could impose it on their neighbours with the greater facility»).
(J. Fielden: «The Curse of the Factory System», London 1836, p. 11).
Щодо жіночої праці фабричний інспектор Савндерс каже у фабричному
звіті за 1844 рік: «Серед робітниць є жінки, які багато тижнів один по
одному, за винятком лише небагатьох днів, працюють від 6 години ранку
до 12 години ночі, маючи менше ніж 2 годин на їжу, так що 5 днів на тиждень
у них із 24 годин лишається тільки 6 годин на те, щоб дійти додому
й назад та відпочити в ліжку».
\parbreak{}  %% абзац продовжується на наступній сторінці

\parcont{}  %% абзац починається на попередній сторінці
\index{i}{0333}  %% посилання на сторінку оригінального видання
останнє зужиткування до певної міри в зворотному відношенні
до її вжитку.\footnote{
«Учинення\dots{} шкоди делікатним рухомим частинам металевого
механізму тим, що він не працює» («Occasion\dots{} injury to the delicate
moving parts of metallic mechanism by inaction»). (Ure: «Philosophy
of Manufacture», London 1835, p. 281).
}

Але, крім матеріяльного зужиткування, машина зазнає, так би
мовити, і морального зужиткування. Вона втрачає мінову вартість
або в тій мірі, в якій машини тієї самої конструкції можуть бути
репродуковані дешевше, або в тій, в якій з нею починають конкурувати
кращі машини.\footnote{
Згаданий уже раніш «Manchester Spinner» («Times», 26. Nov. 1862),
перелічуючи витрати на машини, каже: «це (а саме відрахування для покриття
зужиткування машини) призначено на повернення втрат, які постійно
постають через заміну машин, раніше ніж їх зужитковано, на інші,
нові й ліпшої конструкції» («It (allowance or deterioration of machinery)
is also intended to cover the loss which is constantly arising from the superseding
of machines before they are worn out by others of a new and better
«construction»).
} В обох випадках, хоч і яка молода
та повна життєвої сили була б машина, її вартість визначається вже
не тим робочим часом, який фактично упредметнений у ній, а тим,
що тепер є доконечний на її репродукцію або на репродукцію
кращих машин. Тим то вона є більше або менше зневартнена.
Що коротший той період, протягом якого репродукується її цілу
вартість, то менша небезпека морального зужиткування, а що
довший робочий день, то коротший цей період. Коли машини вперше
вводяться в якусь галузь продукції, то раз-у-раз постають
нові методи дешевшої репродукції їх\footnote{
«Загалом вважають, що сконструювати одним-одну машину за
новим моделем коштує уп’ятеро дорожче, ніж реконструкція тієї самої
машини за тим самим моделем». (Babbage: «On the Economy of Machinery
and Manufactures», London 1832, p. 211).
} та поліпшення, які охоплюють
не тільки поодинокі частини або апарати, але й цілу їхню
конструкцію. Тому в перший період життя машин цей осібний
мотив до подовження робочого дня діє якнайгостріше.\footnote{«Від декількох років у фабрикації тюлю пороблено такі значні та
численні поліпшення, що машину, яка добре збереглася й коштувала
первісно 1.200\pound{ фунтів стерлінґів}, через декілька років продавали за
60\pound{ фунтів стерлінґів}\dots{} Поліпшення поставали одне по одному з такою
швидкістю, що машини лишалися в руках машинобудівельників незакінченими,
бо через щасливіші винаходи вони були вже застарілі». Тим то
в цей період «бурі й натиску» фабриканти тюлю незабаром збільшили первісний
восьмигодинний робочий день до 24 годин з подвійною зміною
робітників. (Там же, стор. 281).}

За інших незмінних обставин та за даного робочого дня експлуатація
подвійного числа робітників потребує подвоєння так
тієї частини сталого капіталу, що її витрачається на машини й
будівлі, як і тієї, що її витрачається на сировинний матеріял,
допоміжні матеріяли й т. д. Із здовженням робочого дня маштаб
продукції ширшає, тимчасом як частина капіталу, витрачена
на машини та будівлі, лишається незмінна.\footnote{
«Само собою очевидно, що з припливами та відпливами на ринку
та за навперемінного поширення та скорочення попиту постійно трапляються випадки, коли фабрикант може вжити додаткового обігового капіталу,
не вживаючи додаткового основного капіталу\dots{} якщо додаткову
кількість сировинного матеріялу можна переробити без додаткових витрат
на будівлі та машини» («It is self-evident, that, amid the ebbings and
flowings of the market, and the alternate expansions and contractions of
demand, occasions will constantly recur, in which the manufacturer may
employ additional floating capital without employing additional fixed
capital\dots{} if additional quantities of raw material can be worked up without
incurring an additional expence for buildings and machinery»).
(R. Torrens: «On Wages and Combination», London 1834, p. 64).
} Тому не тільки
\index{i}{0334}  %% посилання на сторінку оригінального видання
зростає додаткова вартість, але й меншають витрати, доконечні
для того, щоб її добути. Правда, в більшій або меншій мірі це
буває завжди за всякого здовження робочого дня, але тут це
має вирішальніше значення, бо частина капіталу, перетворена
на засоби праці, тут взагалі має більше значення.\footnote{
Згадану в тексті обставину я наводжу тільки для повноти, бо
норму зиску, тобто відношення додаткової вартости до цілого авансованого
капіталу, я розглядаю лише в третій книзі.
} Дійсно,
розвиток машинового виробництва зв’язує чимраз більшу складову
частину капіталу в такій формі, в якій вона, з одного боку, може
постійно самозростати своєю вартістю, а з другого боку, втрачає
споживну вартість і мінову вартість, скоро тільки переривається
її контакт із живою працею. «Коли, — повчав пан Ешворт,
англійський бавовняний маґнат, професора Нассау В. Сеніора, —
коли рільник кидає свій заступ, то він на цей час робить некорисним
капітал у 18 пенсів. Коли один із наших людей (тобто
з фабричних робітників) кидає працю на фабриці, то він робить
некорисним капітал, який коштував 100.000\pound{ фунтів стерлінґів}».\footnote{
«When a labourer», said Mr. Ashworth, «lays down his spade, he
renders useless, for that period, a capital worth 18 d. When one of our people
leaves the mill, he renders useless a capital that has cost 100.000 pounds».
(Senior: «Letters on the Factory Act», London 1837, p. 13, 14).
}
Подумати тільки! «Зробити некорисним», хоча б лише на хвилину,
капітал, що коштував 100.000\pound{ фунтів стерлінґів}. Це — справді
жахна річ, що один із наших людей взагалі може колибудь кинути
фабрику! Чимраз більше зростання розміру машин робить «бажаним»
— визнає повчений від Ешворта Сеніор — щораз більше
й більше здовжування робочого дня.\footnote{
«Велика пропорція основного капіталу проти обігового\dots{} робить
бажаним довгий робочий день» («The great proportion of fixed to circulating
capital\dots{} makes long hours of work desirable»). Із зростом розміру
машин і т. д. «мотиви до здовження робочого дня дедалі посилюються,
бо це єдиний засіб, щоб зробити зисковною відносно велику масу основного
капіталу» («the motives to long hours of work will become greater,
as the only means by which a large proportion of fixed capital can be
made profitable»). (Там же, стор. 11--13). «На фабриці є різні витрати,
які лишаються постійними, незалежно від того, чи робочий час на
фабриці довший, чи коротший, наприклад, орендна плата за будівлі, місцеві
та загальні податки, убезпечення від огню, заробітна плата різним
постійним робітникам, псування машин та різні інші витрати, пропорція
яких до зиску меншає в такому самому відношенні, в якому
зростає розмір продукції». («Reports of Insp. of Fact, for 31 st October
1862», p. 19).
}

\index{i}{0335}  %% посилання на сторінку оригінального видання
Машина продукує відносну додаткову вартість не тільки тим,
що вона безпосередньо зневартнює робочу силу та посередньо
здешевлює її, здешевлюючи товари, потрібні для її репродукції,
але ще й тим, що, при першому спорадичному заведенні її, вона
перетворює вживану посідачем машини працю на працю вищого
ступеня і більшої ефективности (potenzierte Arbeit), суспільну
вартість машинового продукту підвищує понад його індивідуальну
вартість, і таким чином дає змогу капіталістові денну вартість
робочої сили покривати меншою частиною вартости денного продукту.
Тому підчас цього переходового періоду, коли машинове
виробництво лишається своєрідною монополією, бариші є надзвичайно
великі, і капіталіст силкується якнайґрунтовніше визискати
цей «перший час молодого кохання» за допомогою якнайбільшого
здовження робочого дня. Великий бариш загострює ненажерливу
жадобу ще більшого баришу.

У міру того, як машина стає загальним явищем у тій самій
галузі продукції, суспільна вартість машинового продукту знижується
до його індивідуальної вартости та потверджується той
закон, що додаткова вартість постає не з тих робочих сил, що їх
капіталіст замістив машиною, а, навпаки, з тих робочих сил,
яких він коло неї вживає. Додаткова вартість походить тільки
із змінної частини капіталу, і ми вже бачили, що масу додаткової
вартости визначають два фактори — норма додаткової вартости
та число одночасно вживаних робітників. За даної довжини робочого
дня норма додаткової вартости визначається тим відношенням,
що в ньому робочий день розпадається на доконечну працю
й додаткову працю. Число одночасно вживаних робітників визначається,
з свого боку, відношенням змінної частини капіталу до
сталої. Тепер ясно, що хоч як машинове виробництв збільшувало б
через піднесення продуктивної сили праці додаткову працю коштом
доконечної праці, воно доходить цього результату лише тим,
що зменшує число робітників, вживаних якимось даним капіталом.
Воно перетворює на машини, отже, на сталий капітал, що не продукує
жодної додаткової вартости, частину капіталу, який раніш
був змінний, тобто перетворювався на живу робочу силу. З двох
робітників, приміром, неможливо витиснути стільки додаткової
вартости, скільки з 24. Якщо кожний із 24 робітників дає за 12 годин
праці лише одну годину додаткової праці, то разом вони дають
24 години податкової праці, тимчасом як уся праця двох робітників
становить лише 24 години. Отже, вживання машин з метою
продукції додаткової вартости містить у собі іманентну суперечність,
бо з двох факторів додаткової вартости, що її дає капітал
даної величини, машини збільшують один фактор, норму додаткової
вартости, лише тим, що зменшують другий фактор —
число робітників. Ця іманентна суперечність виявляється, скоро
тільки з загальним поширенням машин у якійсь галузі промисловости
вартість продукованого машиновим способом товару стає
реґулятивною суспільною вартістю всіх товарів того самого
роду, і це є та суперечність, яка, не доходячи до свідомости капіталу,\footnote{
\index{i}{0336}  %% посилання на сторінку оригінального видання
Чому ця іманентна суперечність не доходить до свідомости поодинокого
капіталіста, а тому і до свідомости політичної економії, яка поділяє
його погляд, це ми побачимо з перших відділів третьої книги.
} знову таки спонукає його до найбільш насильного здовжування
робочого дня, щоб зменшення відносного числа експлуатованих
робітників компенсувати збільшенням не тільки
відносної, але й абсолютної додаткової праці.

Отже, коли капіталістичне вживання машин, з одного боку,
утворює нові могутні мотиви до безмірного здовжування робочого
дня та робить переворот у самому способі праці й характері суспільного
робочого організму в такий спосіб, що ламає опір супроти
цієї тенденції, то, з другого боку, воно, почасти через підбивання
під владу капіталу неприступних раніш верств робітничої кляси,
почасти через звільнення витиснених машиною робітників, продукує
надмірну робітничу людність,\footnote{
Одна з великих заслуг Рікарда\footnote*{
У французькому виданні тут сказано: «Одна з заслуг Сісмонді
та Рікарда в тому, що вони зрозуміли і т. д.». \emph{Ред.}
} в тому, що він зрозумів, що
машина є засіб продукції не тільки товарів, але й «redundant
population».\footnote*{
— надмірної людности. \emph{Ред.}
}
} яка примушена коритися
законові, що його диктує їй капітал. Звідси те варте уваги явище в
історії сучасної промисловости, що машина нищить усі моральні та
природні межі робочого дня. Звідси той економічний парадокс, що
наймогутніший засіб до скорочення робочого часу перетворюється
в найпевніший засіб до того, щоб увесь час життя робітника та
його родини перетворити на робочий час, що ним порядкує капітал
для збільшення своєї вартости. «Коли б, — мріяв Арістотель,
цей найбільший мислитель старовини, — коли б кожне знаряддя
могло з наказу або передчування виконувати свою власну працю
так, як майстерні витвори Дедала рухалися сами собою, або як
триніжки Гефеста з своєї власної охоти заходжувалися коло святої
праці, коли б так сами собою ткали ткацькі човники, то не потрібні
були б ані майстрові помічники, ані панові раби».\footnote{
F. Biese: «Die Philosophie des Aristoteles», Berlin 1842, Bd. II,
S. 408.
} І Антіпарос,
грецький поет за часів Ціцерона, вітав винахід водяного
млина, щоб молоти мливо, цю елементарну форму кожної
продуктивної машини, як визвольника рабинь і відновника золотого
віку!\footnote{
Подаємо тут Штольберґів переклад цього вірша, бо вірш цей,
цілком так само, як і наведені вище цитати про поділ праці, характеризує
протилежність між античними та сучасними поглядами.

«Schonet der mahlenden Hand, о Müllerinnen, und schlafet
Sanft! Es verkünde der Hahn euch den Morgen umsonst!
Däo hat die Arbeit der Mädchen den Nymphen befohlen,
Und itzt hüpfen sie leicht über die Räder dahin,
Dass die erschütterten Achsen mit ihren Speichen sich wälzen,
Und im Kreise die Last drehen des wälzenden Steins.
}

«Поганці, ах, ті поганці!» Вони, як це виявив мудрий Бастія,
\index{i}{0337}  %% посилання на сторінку оригінального видання
а перед ним ще мудріший Мак-Куллох, нічого не розуміли з
політичної економії та християнства. Вони, між іншим, не розуміли,
що машина — найвипробуваніший засіб до здовження робочого
дня. Вони виправдували рабство однієї людини лише як
засіб до повного людського розвитку іншої. Але, щоб проповідувати
рабство мас із метою перетворити небагато грубих або
напівосвічених вискочнів у «eminent spinners», «extensive sausage
makers» та «influential shoe black dealers»,\footnote*{
— «видатних прядунів», «великих ковбасників» і «впливових
продавців вакси». \emph{Ред.}
} для цього їм
бракувало специфічно-християнського почуття.

с) Інтенсифікація праці\footnote*{
У французькому виданні Маркс додає до цього таку примітку;
«Словом інтенсифікація праці ми позначаємо методи, що роблять працю
напруженішою» («Par le mot intensification nous designons les procedés
qui rend le travail plus intense»). \emph{Ред.}
}

Безмірне здовження робочого дня, що його продукують машини
в руках капіталу, приводить пізніше, як ми вже бачили, до
реакції з боку суспільства, життю якого воно загрожувало в
самому корені, а тим самим і до законодатно обмеженого нормального
робочого дня. На основі останнього набирає вирішальної
ваги явище, з яким ми вже раніш зустрічались, а саме інтенсифікація
праці. При аналізі абсолютної додаткової вартости йшлося
насамперед про екстенсивну величину праці, а ступінь її інтенсивности
припускалось за даний. Тепер ми маємо розглянути
перетворення екстенсивної величини на інтенсивну величину,
тобто на величину, вимірювану щодо ступеня.

Само собою зрозуміло, що разом із розвитком машин та з нагромадженням
досвіду спеціяльною клясою машинових робітників
природно зростає швидкість, а тим самим і інтенсивність
праці. Так, в Англії протягом цілого півстоліття здовження
робочого дня йде поруч зростання інтенсивности фабричної праці.
Однак зрозуміло, що при такій праці, де йдеться не про минущі
пароксизми, а про реґулярну одноманітність, що з дня на день
повторюється, мусить настати пункт, коли здовження робочого

Lasst uns leben das Leben das Väter, und lasst uns der Gaben
Arbeitslos uns freun, welche die Göttin uns schenkt».

(«Gedichte aus dem Griechischen übersetzt von
Christian Graf zu Stolberg. Hamburg. 1782»).

(«Руки свої бережіть, о млинарки, і спіте спокійно, —
Півні нехай сповіщають про ранок — для вас то байдуже.
Део роботу дівочу на німф відтепер всю поклала,
Німфи віднині на колесах легко і жваво танцюють,
І обертаються осі і крутяться спині із ними,
І перемелюють зерно важкеє на кам'яних жорнах.
Отже, живімо як предки жили, без утомної праці
І заживаймо дарів, подарованих нам від богині»).
\parbreak{}  %% абзац продовжується на наступній сторінці


\index{ii}{0338}  %% посилання на сторінку оригінального видання
\subsection[Капітал і дохід: змінний капітал і заробітна плата]{Капітал і дохід: змінний капітал і заробітна плата\footnotemark{}}

\label{original-338}
\noindent{}Ціла
\footnotetext{Відси за рукописом VIII.}
річна репродукція, цілий продукт цього року є продукт корисної
праці за цей рік. Але вартість цього цілого продукту більша, ніж
та частина його вартости, що в ній втілюється річна праця, робоча
сила, витрачена протягом цього року. Новоспродукована вартість
цього року, вартість, новоутворена протягом цього року в товаровій
формі, менша, ніж вартість продукту, ніж вся вартість маси товарів,
виготовлених протягом цілого року. Різність, яка буде, коли з усієї
вартости річного продукту відлічити вартість, долучену до нього працею
поточного року, не є справді репродукована вартість, а вартість, що лише
знову з’явилася в новій формі існування: вартість, перенесена на річний
продукт вартістю, яка існувала раніше від цього продукту, яка — залежно
від тривалости складових частин сталого капіталу, що брали участь у
процесі суспільної праці цього року, — може бути раннішого або пізнішого
походження, яка можливо походить з вартости засобів продукції,
що з’явились на світ минулого року або протягом ряду попередніх років.
В усякому разі це — вартість, перенесена з торішніх засобів продукції
на продукт поточного року.

Коли ми звернемось до нашої схеми, то після обміну розглянутих досі
елементів між І і II і в межах II ми матимемо:

I) $4000 c \dplus{} 1000 v \dplus{} 1000 m$ (останні 2000 реалізуються в засобах
споживання II с) \deq{} 6000.

II) $2000 с$ [репродукуються через обмін з І ($v \dplus{} m$)] \dplus{} $500 v \dplus{} 500 m$ \deq{} 3000.

Сума вартости \deq{} 9000.

Вартість, новоспродукована протягом року, міститься тільки в $v$ і $m$.
Отже, сума новоспродукованої протягом цього року вартости дорівнює
сумі $v \dplus{} m$, \deq{} 2000 І ($v \dplus{} m$) \dplus{} 1000 II ($v \dplus{} m$) \deq{} 3000. Всі інші частини
вартости продукту цього року є лише вартість, перенесена з вартости
попередніх засобів продукції, зужиткованих на річну продукцію.
Крім вартости в 3000, праця поточного року не випродукувала жодної
іншої вартости; це — вся нова вартість, спродукована нею протягом року.

Але, як ми бачили, 2000 І ($v \dplus{} m$) заміщують для II підрозділу 2000
ІІ~$с$ в натуральній формі засобів продукції. Отже, дві третини річної
праці, витрачені в категорії І, знову випродукували сталий капітал II, —
як усю його вартість, так і його натуральну форму. Отже, з суспільного
погляду, дві третини праці, витраченої протягом року, створили нову сталу
капітальну вартість, реалізовану в натуральній формі, відповідній підрозділові
II.~Отже, більшу частину річної суспільної праці витрачено на
продукцію нового сталого капіталу (капітальної вартости, що існує в
засобах продукції) для заміщення сталої капітальної вартости, витраченої
на продукцію засобів споживання. Капіталістичне суспільство в цьому
\parbreak{}  %% абзац продовжується на наступній сторінці

\parcont{}  %% абзац починається на попередній сторінці
\index{i}{0339}  %% посилання на сторінку оригінального видання
у зворотному відношенні до часу її діяння. Тому в певних межах
виграється на інтенсивності праці те, що втрачається на часі її
тривання. Але, щоб робітник дійсно таки витрачав більше робочої
сили, про це капітал дбає за допомогою методи оплати.\footnote{
Особливо за допомогою відштучної плати, форми, що її розглядається
в шостому відділі книги.
} У мануфактурах,
наприклад, у ганчарнях, де машини не відіграють
жодної ролі або відіграють незначну ролю, запровадження фабричного
закону дало разючий доказ того, що саме лише скорочення
робочого дня навдивовижу підносить реґулярність, одноманітність,
лад, безупинність та енерґію праці.\footnote{
Див. «Reports of Insp. of Fact. for 31st October 1865».
} Однак цей результат
у власне фабриці видавався сумнівним, бо залежність робітника
від безупинного та одноманітного руху машини давно створила
вже тут якнайсуворішу дисципліну. Тому, коли 1844~\abbr{р.}
почали обговорювати питання про скорочення робочого дня
нижче від 12 годин, то фабриканти майже одноголосно заявили,
що «їхні наглядачі по різних робітних приміщеннях пильнували,
щоб руки не гаяли часу», що «ступінь недріманости та уважности
робітників ледве чи можна підвищити» («the extent of vigilance
and attention on the part of the workmen») та, припускаючи, що всі
інші обставини, як, приміром, швидкість руху машин і~\abbr{т. ін.},
не змінюються, «було б безглуздям сподіватися в добре впоряджених
фабриках якогось значного результату від збільшення
уважности робітників і~\abbr{т. ін.}»).\footnote{
«Reports of Insp. of Fact. for 1844 and the quarter ending 30th
April 1845», p. 20, 21.
} Це твердження збито експериментами.
Пан Р. Ґарднер запровадив 20 квітня 1844~\abbr{р.} на двох
своїх великих фабриках у Престоні замість 12-годинного лише
11-годинний робочий день. Приблизно через рік виявився
той результат, що «за однакових витрат одержано однакову
кількість продукту, та що всі робітники заробили за 11 годин
стільки саме заробітної плати, скільки раніш за 12 годин».\footnote{
Там же, стор. 19. Через те, що відштучна плата лишилась та
сама, висота тижневої плати залежала від кількости продукту.
}
Я тут лишаю осторонь експерименти в прядільних та чухральних
відділах, бо вони були зв’язані із збільшенням швидкости машин
(на 2\%). Навпаки, у ткальному відділі, де, крім того, ткано
дуже різні сорти легеньких фантастичних квітчастих виробів,
не сталося жодних змін в об’єктивних умовах продукції. Результат
був такий: «Від 6 січня до 20 квітня 1844~\abbr{р.} за дванадцятигодинного
робочого дня пересічна тижнева заробітна плата кожного
робітника становила 10\shil{ шилінґів} 1\sfrac{1}{2}\pens{ пенса}, від 20 квітня до
29 червня 1844~\abbr{р.} за одинадцятигодинного робочого дня пересічна
тижнева заробітна плата — 10\shil{ шилінґів} 3\sfrac{1}{2}\pens{ пенса}».\footnote{
Там же, стор. 22.
} Тут за
11 годин продукували більше, ніж раніш за 12 годин, виключно
в наслідок більшої й рівномірнішої витривалости робітників та
\parbreak{}  %% абзац продовжується на наступній сторінці


\index{ii}{0340}  %% посилання на сторінку оригінального видання
Гроші, то спочатку функціонували для капіталіста як грошова
форма змінного капіталу, тепер функціонують у руках робітника як
грошова форма його заробітної плати, що її він перетворює на засоби
існування; отже, як грошова форма доходу, одержуваного ним від
завжди повторюваного продажу своєї робочої сили.

Тут перед нами лише той простий факт, що гроші покупця, в
даному разі капіталіста, з його рук переходять до рук продавця, в
даному разі продавця робочої сили, робітника. Тут не змінний капітал
двічі функціонує — як капітал для капіталіста і як дохід для робітника, —
і ті самі гроші, що спочатку існували в руках капіталіста як грошова
форма його змінного капіталу, отже, як потенціяльний змінний капітал, і
що потім, після того, як капіталіст перетворив їх на робочу силу, служать
у руках робітника як еквівалент проданої робочої сили. А те, що ті
самі гроші в руках продавця використовується інакше, ніж у руках покупця,
є явище властиве кожній купівлі та продажеві товарів.

Апологети-економісти фалшиво освітлюють справу, і це найкраще
видно, коли ми звернемо увагу виключно, — не турбуючись покищо про
дальші наслідки, — тільки на акт циркуляції $Г — Р$ ($\deq{} Г — Т$), перетворення
грошей на робочу силу на боці капіталістичного покупця, $Р — Г$ ($\deq{} Т — Г$),
перетворення товару робочої сили на гроші на боці продавця, робітника.
Вони кажуть; ті самі гроші реалізують тут два капітали; покупець —
капіталіст — перетворює свій грошовий капітал на живу робочу силу, що
її він долучає до свого продуктивного капіталу; з другого боку, продавець
— робітник — перетворює свій товар — робочу силу — на гроші й
витрачає їх як дохід, через що саме й може він знову й знов продавати
й таким чином зберігати свою робочу силу; отже, сама його робоча
сила є його капітал у товаровій формі і є постійне джерело його доходу.
А справді робоча сила є його здібність (яка постійно відновлюється,
репродукується), а не його капітал. Вона єдиний товар, що його він
постійно може й мусить продавати для того, щоб жити, і що діє як
капітал (змінний) лише в руках покупця, капіталіста. Коли якась людина
постійно мусить знову й знов продавати третій особі свою робочу силу,
тобто самого себе, то це, згідно з згаданими економістами, доводить,
що вона — капіталіст, бо їй завжди доводиться продавати „товар“ (саму
себе). В цьому розумінні й раб, хоч його раз назавжди продає як
товар третя особа, стає капіталістом, бо природа цього товару — робітникараба
— така, що покупець не тільки примушує його кожного дня робити,
а й дає йому ті засоби існування, що завдяки їм він може знову й
знов робити. — (Порівняй про це Сісмонді та Сея в листах до Малтуса).

2) Отже, те, що в обміні 1000 І~$v \dplus{} 1000$ І~$m$ на 2000 II~$с$ є сталий
капітал для одних (2000 II~$с$), стає змінним капіталом і додатковою
вартістю, тобто взагалі доходом для інших; а те, що є змінний капітал
і додаткова вартість 2000 І ($v \dplus{} m$), тобто, взагалі, доходом для одних,
стає сталим капіталом для інших.

Розгляньмо спочатку обмін I~$v$ на II~$с$, насамперед з погляду робітника.

\input{i/_0341.tex}
\index{i}{0342}  %% посилання на сторінку оригінального видання
«Через те, що кількість продуктів реґулюється переважно
швидкістю машин, то інтерес фабриканта мусить бути в тому,
щоб гнати їх із якнайбільшою швидкістю, яку тільки можна
сполучити з такими умовами: зберігання машин, щоб вони не
надто скоро псувалися, зберігання якости фабрикованих продуктів
та здатність робітника встигати за рухом без більшого напруження,
ніж яке він може безупинно розвивати. Часто буває так,
що фабрикант, дуже поспішаючися, занадто прискорює рух.
Тоді полами та поганий виріб більш ніж урівноважують швидкість,
і він примушений зробити рух машин повільнішим. А що
активний та розсудливий фабрикант знаходить осяжний максимум,
то я дійшов висновку, що неможливо за 11 годин випродукувати
стільки, як за 12 годин. Крім того, я припускав, що робітник,
якому платять відштучно, напружує свої сили до якнайбільшої
міри, скільки він може безупинно витримувати такий
ступінь інтенсивности праці».\footnote{
«Reports of Insp. of Fact. to 30 th April 1845», p. 20.
} Тим то Горнер, усупереч експериментам
Ґарднера й ін., дійшов такого висновку, що дальше скорочення
робочого дня нижче від 12 годин мусило б зменшити
кількість продукту.\footnote{
Там же, стор. 22.
} Він сам через 10 років цитує свої сумніви
з року 1845 на доказ того, як мало він ще тоді тямив в елястичності
машин та людської робочої сили, що обидві в рівній мірі
напружуються до якнайвищого ступеня через примусове скорочення
робочого дня.

А тепер перейдімо до періоду після 1847 р., від часу запровадження
закону про десятигодинний робочий день в англійських
бавовняних, вовняних, шовкових та лляних фабриках.

«Швидкість веретен на throstles зросла на 500 обертів, на
mules на 1.000 обертів на одну хвилину, тобто швидкість веретена
throstles, яка 1839 р. мала 4.500 обертів на хвилину, становить
тепер (1862 р.) 5.000 обертів, становить тепер 6.000 обертів на хвилину; отже,
швидкість збільшилася в першому випадку на \sfrac{1}{10}, а в другому —
на \sfrac{1}{5}».\footnote{
«Reports of Insp. of Fact. for 31 st October 1862», p. 62.
} Джеме Несміс, славетний цивільний інженер із
Patricroft’a коло Менчестеру, в одному листі до Леонарда Горнера
1852 р. пояснив у подробицях про поліпшення, пороблені в паровій
машині між 1848 і 1852 рр. Зауваживши, що парова кінська
сила, яка в офіціяльній фабричній статистиці все ще визначається
за ефектом її в 1828 р.,\footnote{
Це змінилося від часу «Parliamentary Return»\footnote*{
— парляментського звіту. \emph{Ред.}
} 1862 p. Тут виступає справжня парова кінська сила сучасних парових машин
та водяних коліс на місце номінальної (див. примітку 109а, стор. 318). І веретен
на сукання (Dublierspindeln) уже не переплутують із прядільними веретенами
у власному значенні (як у «Returns» 1839, 1850 та 1856); далі
для вовняних фабрик подано число «gigs»,\footnote*{
— ворсувальних машин. \emph{Ред.}
} заведено ріжницю між джу
товими та конопляними фабриками, з одного боку, і лляними — з другого;
нарешті, вперше заведено до звіту панчішне виробництво.
} є лише номінальна й може бути
\index{i}{0343}  %% посилання на сторінку оригінального видання
показником (Index) дійсної сили, він, між іншим, каже:
«Немає ніякого сумніву, що парові машини тієї самої ваги, часто
ті самі ідентичні машини, де пороблено лише сучасні поліпшення,
виконують пересічно на 50\% більше праці, ніж раніш, та що в
багатьох випадках ті самі ідентичні парові машини, які за часів
обмеженої швидкости в 220 футів на хвилину давали 50 кінських
сил, тепер, при зменшеному споживанні вугілля, дають понад 100
кінських сил\dots{} Сучасна парова машина з тією самою кількістю
номінальних кінських сил, у наслідок поліпшення в її конструкції,
зменшення розміру та поліпшення конструкції парового
казана тощо, діє з більшою силою, ніж раніш\dots{} Тому, хоч супроти
номінальної кінської сили вживається те саме число рук, що й
раніш, однак супроти робочих машин вживається менше число
рук».\footnote{
«Reports of Insp. of Fact, for 31 st October 1856», p. 11.
} 1850 р. фабрики Об’єднаного Королівства вживали 134.217
номінальних кінських сил, щоб рухати 25.638.716 веретен та
301.495 ткацьких варстатів. 1856 р. число веретен і ткацьких
варстатів становило відповідно 33.503.580 і 369.205. Коли б
потрібна кінська сила лишилася та сама як 1850 р., то 1856 р.
потрібно було б 175.000 кінських сил. Але за офіціяльними документами
число їх становило лише 161.435, отже, понад 10.000
кінських сил менше, ніж їх потрібно було б на основі розрахунку
з 1850 р.\footnote{
Там же, стор. 14, 15.
} «Останній «Return» з 1850 р. (офіціяльна статистика)
установлює той факт, що фабрична система поширюється з чимраз
більшою швидкістю, число рук у відношенні до машин зменшилося,
парова машина в наслідок економії на силі та інших
метод рухає машини більшої ваги, і що більшої кількости продукту
досягається в наслідок поліпшення робочих машин, змінених
метод фабрикації, збільшеної швидкости машин та багатьох інших
причин».\footnote{
Там же, стор. 20.
} «Великі поліпшення, які пороблено в машинах
усякого роду, дуже підвищили їхню продуктивну силу. Немає
ніякого сумніву, що скорочення робочого дня дало\dots{} стимул
до цих поліпшень. Ці поліпшення й інтенсивніше напруження
робітника призвели до того, що протягом скороченого (на 2 години,
або на одну шосту) робочого дня продукується, щонайменше,
стільки ж продукту, як раніш протягом довшого дня».\footnote{
«Reports etc. for 31 st October 1858», p. 9, 10. Порівн. «Reports
etc. for 30 th April 1860», p. 30 і далі.
}

Як зросло збагачення фабрикантів у наслідок інтенсивнішого
визиску робочої сили, доводить уже одна та обставина, що пересічне
пропорційне зростання англійських бавовняних і т. ін.
фабрик становило 1838 – 1850 рр. 32\%, а 1850 – 1856 рр. – 86\%.

Хоч який великий був проґрес англійської промисловости
за вісім років – від 1848 до 1856 – за панування десятигодинного
\index{i}{0344}  %% посилання на сторінку оригінального видання
робочого дня, все ж його знову значно перевищив дальший
шестирічний період від 1856 до 1862 р. Наприклад, 1856 р. по
шовкових фабриках було 1.093.799 веретен, а 1862 р. — 1.388.544;
ткацьких варстатів 1856 р. було 9.260, а 1862 р. — 10.709. Навпаки,
число робітників 1856 р. становило 56.131, а 1862 р. — 52.429.
Це дає збільшення числа веретен на 26,9\% і ткацьких варстатів
на 15,6\% при одночасному зменшенні робітників на 7\%. 1850 р. на
фабриках суканої вовни було в ужитку 875.830 веретен, 1856 р. —
1.324.549 (приріст на 51,2\%), а 1862 р. — 1.289.172 (зменшення
на 2,7\%). А коли відлічити веретена на сукання (Dublierspindeln),
які фігурують у переліку за 1856 р., але не фігурують у переліку
за 1862 р., то число веретен від 1856 р. майже не змінилося. Навпаки,
швидкість веретен і ткацьких варстатів від 1850 р. в
багатьох випадках подвоїлася. Число парових ткацьких варстатів
по фабриках суканої вовни 1850 р. становило 32.617,
1856 р. — 38.956, а 1862 р. — 43.048. Коло них працювало 1850 р.
79.737 осіб, 1856 р. — 87.794, а 1862 р. — 86.063, та з них було
дітей, молодших за 14 років: 1850 р. — 9.956, 1856 р. — 11.228,
а 1862 р. — 13.178. Отже, не зважаючи на значне збільшення
числа ткацьких варстатів у 1862 р. порівняно з роком 1856, загальне
число вживаних робітників зменшилось, а число експлуатованих
дітей зросло.\footnote{
«Reports of Insp. of Fact, for 31 st October 1862», p. 100 і 130.
}

27 квітня 1863 р. член парляменту Ферранд ось що заявив у
Палаті громад: «Делеґати робітників із 16 округ Ланкашіру й
Чешіру, з доручення яких я говорю, повідомили мене, що в наслідок
поліпшення машин праця на фабриках постійно зростає.
Раніш одна особа з помічниками обслуговувала два ткацькі верстати,
тепер одна особа без помічників обслуговує три варстати,
а часто-густо й чотири варстати і т. д. Як це видно з поданих
фактів, дванадцять годин праці стиснуто тепер менше, ніж у
10 робочих годин. Тому само собою зрозуміло, до якого величезного
розміру зросла тяжкість праці фабричних робітників останніми
роками».\footnote{
За допомогою сучасного парового варстату один ткач на двох
варстатах фабрикує тепер за 60 годин на тиждень 26 сувоїв певного сорту
тканини певної довжини та ширини, а раніш на старому паровому ткацькому
варстаті він міг продукувати лише 4 такі сувої. Витрати на ткання
одного такого сувою вже на початку 1850-х років спали з 2 шилінґів
9 пенсів до 5\sfrac{1}{8} пенсів.

Додаток до другого видання: «Перед З0 роками (1841) від одного прядуна
бавовняної пряжі з трьома помічниками вимагали доглядати лише
за однією парою мюлів із 300 – 324 веретенами. Тепер (кінець 1871 р.)
він з п’ятьма помічниками має доглядати мюлів, що їхнє число веретен
становить 2.200, та продукує, щонайменше, всемеро більше пряжі, ніж
1841 р.». (\emph{Alexander Redgrave}, фабричний інспектор, у «Journal of
the Society of Arts», 5 Januar 1872).
}

Тому, хоч фабричні інспектори невтомно та з повним правом
вихваляють сприятливі результати законів 1844 та 1850 рр.,
все ж вони визнають, що скорочення робочого дня викликало вже
\parbreak{}  %% абзац продовжується на наступній сторінці

\parcont{}  %% абзац починається на попередній сторінці
\index{i}{0345}  %% посилання на сторінку оригінального видання
таку інтенсивність праці, що руйнує здоров’я робітників, отже,
руйнує саму робочу силу. «В більшості бавовняних фабрик, фабрик
суканої вовни та шовкових фабрик той запал, що виснажує
робітника і є потрібний для праці коло машин, рух яких останніми
роками так надзвичайно прискорено, є, здається, однією
з причин тієї надмірної смертности від недуг на легені, яку виявив
д-р Ґрінхов у своєму найновішому вартому уваги звіті»\footnote{
«Reports of Insp. of Fact, for 31 st October 1861», p. 25, 26.
}.
Немає найменшого сумніву, що тенденція капіталу, скоро тільки
закон раз назавжди покладе край здовжуванню робочого дня,
а саме тенденція відшкодовувати себе систематичним підвищенням
ступеня інтенсивности праці та перетворенням усякого поліпшення
машин на засіб до більшого висисання робочої сили, мусить
незабаром знову привести до того поворотного пункту, де знов
стає неминучим скорочення робочого дня\footnote{
Тепер (1867 p.) у Ланкашірі почалася серед фабричних робітників
аґітація за восьмигодинний робочий день.
}. З другого боку,
швидкий поступ англійської промисловости за час від 1848~\abbr{р.} до
наших часів, тобто за періоду десятигодинного робочого дня, ще
дужче перевищує розвиток її за час від 1838 до 1847~\abbr{р.}, тобто за
періоду дванадцятигодинного робочого дня, ніж цей останній
перевищує розвиток промисловости протягом півстоліття від часу
заведення фабричної системи, тобто за періоду необмеженого
робочого дня\footnote{
Декілька нижченаведених чисел показують поступ власне «фабрик»
в Об’єднаному Королівстві, починаючи від 1848~\abbr{р.}:

\setlength{\tabcolsep}{3pt}
\begin{footnotesize}
\noindent\begin{tabularx}{\textwidth}{@{}Xrrrr@{}}
  \addlinespace
  \toprule
  & \multicolumn{4}{c}{\emph{Розмір експорту по роках}} \\
  \cmidrule{2-5}
  & 1848 & 1851 & 1860 & 1865\\
  \midrule

  \addlinespace
  \emph{Бавовняні фабрики} \\
  Бавовняна пряжа\dotfill{} & \num{135.831.162} \samewidth{ярд.}{фун.} & \num{143.966.106} \samewidth{ярд.}{фун.} & \num{197.343.655} \samewidth{ярд.}{фун.} & \num{103.751.455} \samewidth{ярд.}{фун.} \\
  Нитки до шиття\dotfill{} & \textemdash & \num{4.392.176} \samewidth{ярд.}{фун.} & \num{6.287.554} \samewidth{ярд.}{фун.} & \num{4.648.611} \samewidth{ярд.}{фун.} \\
  Бавовняні тканини\dotfill{} & \num{1.091.373.930} ярд. & \num{1.543.161.789} ярд.
     & \num{2.776.218.427} ярд.   & \num{2.015.237.851} ярд. \\

  \addlinespace
  \makecell[l]{\emph{Льнопрядні та}\\\emph{коноплепрядні фабрики}} \\
  Пряжа\dotfill{} & \num{11.722.182} \samewidth{ярд.}{фун.} & \num{18.841.326} \samewidth{ярд.}{фун.} &  \num{31.210.612} \samewidth{ярд.}{фун.}  &  \num{36.777.334} \samewidth{ярд.}{фун.} \\
  Тканини\dotfill{} &  \num{88.901.519} ярд. &   \num{129.106.753} ярд.
    &   \num{143.996.773} ярд.  &  \num{247.012.329} ярд. \\

  \addlinespace
  \emph{Шовкові фабрики} \\
  Пряжа й нитки\dotfill{} &  \num{194.815} \samewidth{ярд.}{фун.} &   \num{462.513} \samewidth{ярд.}{фун.}  &    \num{897.402} \samewidth{ярд.}{фун.}  & \num{812.589} \samewidth{ярд.}{фун.} \\
  Тканини\dotfill{}       & \textemdash & \num{1.181.455} ярд. & \num{1.307.293} ярд.
     & \num{2.869.837} ярд. \\

  \addlinespace
  \emph{Вовняні фабрики} \\
  Пряжа\dotfill{}   & \textemdash & \num{14.670.880} \samewidth{ярд.}{фун.}  &  \num{27.533.968} \samewidth{ярд.}{фун.} &\num{31.669.267} \samewidth{ярд.}{фун.} \\
  Тканини\dotfill{} & \textemdash & \num{151.231.153} ярд.  & \num{190.371.537} ярд. &\num{278.837.418} ярд. \\
\end{tabularx}
\end{footnotesize}

\setlength{\tabcolsep}{\tabcolsepdef}
\begin{footnotesize}
\centering
\noindent\begin{tabularx}{\textwidth}{@{}Xrrrr@{}}
  \addlinespace
  \toprule
  & \multicolumn{4}{c}{\emph{Вартість експорту по роках,\pound{ (фун. стерл.)}} } \\
  \cmidrule{2-5}
  & 1848 & 1851 &
    1860 & 1865 \\
  \midrule

  \addlinespace
  \emph{Бавовняні фабрики} \\
  Бавовняна пряжа\dotfill{} & \num{5.927.831} &  \num{6.634.026} & \num{9.870.875} & \num{10.351.049} \\
  Бавовняні тканини\dotfill{} & \num{16.753.369}  & \num{23.454.810}  & \num{42.141.505} & \num{46.903.796}\\

  \addlinespace
  \emph{Льнопрядні та коноплепрядні фабрики} \\
  Пряжа\dotfill{} & \num{493.449}  & \num{951.426} & \num{1.801.272} & \num{2.505.497} \\
  Тканини\dotfill{} & \num{2.802.789}  &  \num{4.107.396} & \num{4.804.803} & \num{9.155.358} \\

  \addlinespace
  \emph{Шовкові фабрики} \\
  Пряжа й нитки\dotfill{} &  \num{77.789} &  \num{196.380} & \num{826.107} & \num{768.064} \\
  Тканини\dotfill{}       & \num{510.328} & \num{1.130.398} & \num{1.587.303} & \num{1.409.221} \\

  \addlinespace
  \emph{Вовняні фабрики} \\
  Пряжа\dotfill{}   &  \num{776.975} & \num{1.484.544} & \num{3.843.450} & \num{5.424.047} \\
  Тканини\dotfill{} & \num{5.733.829} & \num{8.377.183} & \num{12.156.998} & \num{20.102.259} \\

\end{tabularx}
\end{footnotesize}


\noindent{}(Див. Сині Книги: «Statistical Abstract for the United Kingdom»,
№ 8 і № 13, London 1861 і 1866).

\noindent{}В Ланкашірі число фабрик збільшилося між 1839 і 1850~\abbr{рр.} лише на
4\%, між 1850 і 1856 — на 19\%, між 1856 і 1862 — на 33\%, тимчасом
як число занятих осіб за обидва одинадцятилітні періоди абсолютно підвищилось,
відносно ж зменшилось. Див. «Reports of Insp. of Fact. for
31 st October 1862», p. 63. В Ланкашірі мають перевагу бавовняні фабрики.
А яке відносно велике місце мають вони взагалі у фабрикації пряжі й
тканини, це видно з того, що з загального числа всіх фабрик такого роду
в Англії, Велзі, Шотляндії та Ірляндії на них самих припадає 45,2\%, а із
загального числа веретен — 83,3\%, із загального числа парових ткацьких
варстатів — 81,4\%, із загального числа парових кінських сил, що дають
рух тим фабрикам, — 72,6\%, а із загального числа занятих осіб — 58,2\%.
(Там же, стор. 62, 63).
}.

\index{i}{0346}  %% посилання на сторінку оригінального видання
\subsection{Фабрика}

На початку цього розділу ми розглядали тіло фабрики, тобто
організовану систему машин. Далі ми бачили, як машини, присвоюючи
собі жіночу й дитячу працю, збільшують людський
матеріял, що його експлуатує капітал, як вони, безмірно здовжуючи
робочий день, забирають увесь час життя робітника та,
нарешті, як їхній поступ, що дозволяє продукувати велетенські,
чимраз більші маси продукту в щораз коротший час, служить
систематичним засобом, щоб кожного моменту пускати в рух
більше праці, тобто раз-у-раз інтенсивніше визискувати робочу
силу. Тепер ми перейдемо до фабрики як до цілости, і саме в її
найрозвиненішій формі.

Д-р Юр, Піндар автоматичної фабрики описує її, з одного
боку, як «кооперацію різних кляс робітників, дорослих і недорослих,
які з вправністю і пильністю доглядають систему продуктивних
\index{i}{0347}  %% посилання на сторінку оригінального видання
машин, що її безнастанно пускає в рух центральна
сила (перший мотор)», а з другого боку, «як велетенський автомат,
складений з безлічі механічних та самосвідомих органів,
які оперують у згоді й без перерви, щоб продукувати той самий
предмет, так що всі ці органи підпорядковані одній рушійній
силі, яка сама собою рухається». Ці обидва визначення зовсім
не є ідентичні. В першому комбінований колективний робітник
або суспільне робоче тіло з’являється як домінантний суб’єкт,
а механічний автомат — як об’єкт; у другому сам автомат є
суб’єкт, а робітники як свідомі органи лише додані до його
несвідомих органів і разом з цими останніми підпорядковані
центральній рушійній силі. Перше визначення можна прикласти
до кожного можливого вживання машин у великому маштабі,
друге характеризує капіталістичне вживання машин, а тому й
сучасну фабричну систему. Тим то Юр любить також змальовувати
центральну машину, що від неї виходить рух, не тільки як
автомата, але і як автократа. «По цих величезних майстернях
благодатна сила пари збирає довкола себе міріяди своїх підданців»\footnote{
«\emph{Ure}: «Philosophy of Manufacture», р. 18.
}.

Разом із робочим знаряддям і віртуозність керувати ним
переходить від робітника до машини. Видатність знаряддя звільняється
від особистих рамок людської робочої сили. Тим самим
усунуто (aufgehoben) ту технічну основу, на якій ґрунтується
поділ праці в мануфактурі. Тому замість тієї ієрархії спеціялізованих
робітників, що характеризує мануфактуру, на автоматичній
фабриці виступає тенденція зрівняти, або знівелювати ті
праці, які мають виконувати помічники машин\footnote{
Там же, стор. 20. Порівн. \emph{K.~Marx}: «Misère de la Philosophie»,
Paris 1847. p. 140, 141. (\emph{K.~Маркс}: «Злиденність філософії», Партвидав
«Пролетар», 1932~\abbr{р.}, стор. 126, 127).
}, замість штучно
витворених ріжниць між частинними робітниками виступають
переважно природні ріжниці віку та статі.

Оскільки поділ праці відроджується на автоматичній фабриці,
він є насамперед розподіл робітників між різними спеціялізованими
машинами та розподіл мас робітників, які однак не являють
собою організованих груп, між різними відділами фабрики, де
вони працюють коло однорідних виконавчих машин, що стоять
рядом одна побіч однієї; отжеж, серед них існує тільки проста
кооперація. Розчленовану групу мануфактури тут замінено
зв’язком головного робітника з небагатьма помічниками. Посутня
ріжниця є поміж робітниками, які справді працюють коло виконавчих
машин (сюди належать декілька робітників, що доглядають
рушійної машини та опалюють її), і простими підручними
(майже виключно діти) цих машинових робітників. До підручних
у більшій або меншій мірі залічується всіх «feeders» (які
лише подають машинам матеріял праці). Побіч цих головних кляс
виступає чисельно незначний персонал, що доглядає всіх машин
\parbreak{}  %% абзац продовжується на наступній сторінці

\parcont{}  %% абзац починається на попередній сторінці
\index{iii1}{0348}  %% посилання на сторінку оригінального видання
поверхового уявлення про внутрішній зв’язок економічних відносин,
який виявляється в конкуренції. Це є спосіб для того, щоб від
змін, які супроводять конкуренцію, прийти до границь цих змін.
Його не можна прикласти до пересічного розміру процента.
Немає абсолютно ніякої підстави, чому середні умови конкуренції,
рівновага між позикодавцями й позичальниками, повинні давати
позикодавцеві розмір процента в 3, 4, 5\% і~\abbr{т. д.} на його капітал
абож певну процентну частину — 20 чи 50\% — гуртового
зиску. В тих випадках, коли справу-тут вирішує конкуренція як
така, визначення само по собі є випадковим, чисто емпіричним,
і тільки педантство або фантазерство може хотіти зобразити
цю випадковість як щось необхідне\footnote{
Так, наприклад, \emph{J.~G.~Opdyke}: „А Treatise on Political Economy“, New
York 1851, робить надзвичайно невдалу спробу пояснити загальність розміру
процента в 5\% вічними законами. Незрівняно наївніший пан \emph{Карл Арнд} в „Die
naturgemässe Volkswirtschaft gegenüber dem Monopoliengeist und dem Kommunismus
etc.“, Hanau 1845. Тут можна прочитати таке: „В природному ході виробництва
благ існує тільки \emph{одно} явище, яке — в цілком культивованих країнах — до
певної міри ніби призначене реґулювати розмір процента; це — відношення,
в якому збільшуються маси дерев у європейських лісах в наслідок їх щорічного
приросту. Цей приріст відбувається цілком незалежно від їх мінової вартості“
[як це комічно, що дерева організують свій приріст незалежно від своєї мінової
вартості!] „у відношенні 3--4 до 100. — Отже, згідно з цим“ [тому що приріст
дерев зовсім не залежить від їх мінової вартості, хоч і як дуже їх мінова вартість
може залежати від їх приросту] „не можна було б сподіватись падіння
нижче того рівня, що його він“ [розмір процента] „має в теперішній час у
найбагатших на гроші країнах“ (стор. 124 [125]). — Це заслуговує назви „розмір
процента лісового походження“, а його винахідник у тому самому творі здобуває
ще більшу заслугу перед „нашою наукою“ як „філософ собачого податку“
[стор. 420 і далі].
}. У парламентських звітах
1857 і 1858~\abbr{рр.}, які стосуються законодавства про банки і торговельної
кризи, немає нічого кумеднішого, як базікання директорів
Англійського банку, лондонських банкірів, провінціальних
банкірів і професіональних теоретиків про „real rate produced“
[фактично утворену норму], яке не йшло далі таких загальних
місць, як, наприклад, що „ціна, яка сплачується позиченим
капіталом, може мінятись із зміною подання цього капіталу“,
що „висока норма процента і низька норма зиску не можуть
довгий час існувати одна поряд одної“ та інші подібні банальності\footnote{
Англійський банк підвищує і знижує норму свого дисконту залежно від
того, припливає чи відпливає золото, хоч, звичайно, він при цьому завжди бере
до уваги норму, яка панує на відкритому ринку. „By which gambling in discounts,
by anticipation of the alterations in the bank rate, has now become half the trade
ol the great heads of they money centre“ [„В наслідок цього спекуляція на зміні
дисконту, яка передхоплює зміни банкової норми, стала тепер наполовину
заняттям великих фірм грошового центру“], — тобто лондонського грошового
ринку („The Theory of the Exchanges etc.“, стор. 113).
}.
Звичка, узаконена традиція і~\abbr{т. д.} цілком так само, як
і сама конкуренція, впливають на визначення середнього розміру
процента, оскільки він існує не тільки як пересічне число,
але й як фактична величина. Середній розмір процента мусить
уже бути припущений як законний у багатьох судових справах,
\parbreak{}  %% абзац продовжується на наступній сторінці

\parcont{}  %% абзац починається на попередній сторінці
\index{i}{0349}  %% посилання на сторінку оригінального видання
підручних на фабриці, то її можна заміняти почасти машинами\footnote{
Приклад: різні механічні апарати, які позаводжувано від часів
закону 1844~\abbr{р.} по вовняних фабриках для заміни дитячої праці. Скоро
тільки дітям самих панів фабрикантів доведеться проходити «їхню школу»
як підручним на фабриці, то ця майже незаймана сфера механіки одразу
розвинеться до дивовижних розмірів. «Ледве чи є ще така небезпечна
машина, як от selfacting-mule. Більша частина нещасливих випадків
трапляється з малими дітьми й саме через те, що вони підлазять під мюлі
тоді, як вони в русі, щоб замести долівку. Багатьох «minders» (робітників
при мюлях) притягли (фабричні інспектори) до судової відповідальности
та засудили на грошові кари за ці провини, але без будь-якої загальної
користи. Коли б будівники машин винайшли хоч одну машину
для замітання долівки, вживання якої звільнило б цих малих дітей від
потреби лазити під машини, то це було б щасливим додатком до наших
охоронних заходів». («Reports of Insp. of Factories for 31 st October
1866», p. 63).
},
а почасти вона дозволяє — через те, що вона зовсім проста —
хутко й постійно зміняти людей, обтяжених цими муками.

Хоч машина технічно знищує стару систему поділу праці,
все ж таки остання животіє на фабриці й далі, спочатку за звичкою,
як традиція мануфактури, а потім капітал систематично
репродукує та закріпляє її в ще огидливішій формі як засіб
експлуатації робочої сили. Довічна спеціальність орудувати
частинним знаряддям стає довічною спеціяльністю служити частинній
машині. Машинами зловживають, щоб самого робітника
від дитячих років перетворювати на частину частинної машини\footnote{
Після цього можна оцінити неймовірну вигадку Прудона, який
«конструює» машини не як синтезу засобів праці, а як синтезу частинних
праць для самих робітників. [Крім того, він робить остільки ж історичне,
як і дивовижне відкриття, що «машиновий період відзначається специфічним
характером, а саме найманою працею»]\footnote*{
Подане у прямих дужках ми беремо з французького видання. \emph{Ред.}
}.
}.
Таким чином не тільки значно зменшуються витрати, потрібні
для його власної репродукції, але й одночасно завершується його
безпорадна залежність від фабрики, як цілости, отже, від капіталіста.
Тут, як і всюди, треба розрізняти збільшення продуктивности,
що його зумовлює розвиток суспільного процесу продукції,
від того збільшення продуктивности, що його зумовлює
капіталістичний визиск цього процесу.

В мануфактурі і в реместві знаряддя служить робітникові,
на фабриці робітник служить машині. Там рух знаряддя праці
виходить від нього, тут — він має йти за його рухом. У мануфактурі
робітники є члени живого механізму. На фабриці існує
мертвий механізм незалежно від них, а їх додають до нього як
живі додатки. «Сумна одноманітність безконечної муки праці, з
якою той самий механічний процес знову й знову повторюється, є
подібна до муки Сізіфової; тягар праці немов скеля знову й знову
спадає на знесилених робітників»\footnote{
«\emph{F.~Engels}: «Die Lage der arbeitenden Klasse in England», Leipzig
1845, S. 217 (\emph{Ф.~Енґельс}: «Становище робітничої кляси в Англії»,
Партвидав «Пролетар», 1932~\abbr{р.}, стор. 187). Навіть цілком ординарний,
«оптимістичний фритредер, пан Молінарі, зауважує: «Людина, доглядаючи
щодня 15 годин за одноманітним рухом машини, виснажується
швидше, ніж коли вона протягом того самого часу працює фізично. Ця праця
догляду, яка, може, могла б бути за корисну гімнастику для розуму,
коли б не тривала надто довго, своєю надмірністю руйнує кінець-кінцем
і розум і саме тіло». («Un homme s’use plus vite en surveillant quinze
heures par jour l’évolution uniforme d’un mécanisme, qu’en exerçant dans
le même espace de temps, sa force physique. Ce travail de surveillance,
qui servirait peut-être d’utile gymnastique à l’intelligence, s’il n’etait pas
trop prolongé, détruit à la longue, par son excès, et l’intelligence et le corps
même»). (\emph{G. de Molinari}: «Etudes Economiques», Paris 1846, p. 49).
}. Виснажаючи до крайности
\index{i}{0350}  %% посилання на сторінку оригінального видання
нервову систему, машинова праця пригнічує багатобічну гру
мускулів і відбирає всяку змогу вільної фізичної й інтелектуальної
діяльности\footnote{
\emph{F.~Engels}, там же, стор. 216. (Партвидав «Пролетар», 1932~\abbr{р.},
стор. 186).
}. Навіть полегшення праці робиться засобом тортур,
бо машина не визволяє робітника від праці, а відбирає його
праці зміст. Всякій капіталістичній продукції, оскільки вона є
не тільки процес праці, але й процес зростання вартости капіталу,
є спільне те, що не робітник вживає умов праці, а, навпаки,
умови праці вживають робітника, але тільки при машиновій
системі це перекручення набуває технічно-очевидної реальности.
В наслідок свого перетворення на автомат засіб праці підчас
самого процесу праці протистоїть робітникові як капітал,
як мертва праця, що опановує й висисає живу робочу силу. Відокремлення
інтелектуальних сил процесу продукції від ручної
праці та перетворення їх у владу капіталу над працею завершується,
як ми вже раніш казали, у великій промисловості, побудованій
на основі машин. Частинна вправність індивідуального,
спустошеного машинового робітника зникає як незначна другорядна
річ перед наукою, перед велетенськими силами природи
і перед суспільною масовою працею, що втілені в системі машин
та разом з нею становлять владу «хазяїна» (master). Тому цей хазяїн,
у мозку якого машини нероздільно зрослися з його монополією
на них, у випадках колізії вигукує зневажливо до «рук»:
«Хай фабричні робітники в своїх власних інтересах запам’ятають,
що їхня праця в дійсності є дуже низький сорт навченої
праці; що жодної іншої праці не можна легше вивчити та що,
зважаючи на її якість, жодної праці не оплачується ліпше; що
жодної іншої праці не можна придбати за такий короткий час та
в такому великому розмірі, сяк-так привчивши найменш досвідчених
осіб. Машини хазяїна відіграють у дійсності далеко важливішу
ролю в справі продукції, ніж праця і вправність робітника,
яких можна навчитися за шість місяців і яких може навчитися
кожен сільський наймит»\footnote{
«The factory operatives should keep in wholesome remembrance
the fact that theirs is really a low species of skilled labour; and that there
is none which is more easily acquired or of its quality more amply remunerated,
or which, by a short training of the least expert can be more quickly
as well as abundantly acquired\dots{} The master’s machinery really plays
a far more important part in the business of production than the labour
and the skill of the operative, which six months education can teach, and
a common labourer саn learn». («The Master Spinners’and Manufacturer,
Defence Fund. Report of the Committee», Manchester 1854, p. 17). Пізніш
побачимо, що цей «хазяїн» співає іншої пісеньки, коли йому загрожує
небезпека втратити свої «живі» автомати.
}.

\index{i}{0351}  %% посилання на сторінку оригінального видання
Технічне підпорядковання робітника одноманітній ході засобу
праці і своєрідний склад робочого тіла з індивідів обох статей
і якнайрізнішого віку створюють казармову дисципліну, яка
розвивається на вивершений фабричний режим та цілком розвиває
раніш уже згадану працю нагляду, отже, разом з тим і
поділ робітників на ручних робітників та наглядачів за працею,
на рядових промислових солдатів, та промислових унтер-офіцерів.
«Головні труднощі в автоматичній фабриці\dots{} були\dots{} в дисципліні,
доконечній, щоб примусити людей одмовитися від їхньої
звички до нереґулярности в роботі та пристосувати їх до незмінної
реґулярности великого автомату. Але ж винайти та з успіхом
провести в життя дисциплінарний кодекс, що відповідав би
потребам та швидкості автоматичної системи — ця робота, гідна
Геркулеса, була благородним ділом Аркрайта! Навіть за наших
днів, коли цю систему зорганізовано в цілій її повноті, майже
неможливо знайти серед робітників, що вилюдніли вже в дозрілих
людей\dots{} корисних помічників для автоматичної системи»\footnote{
\emph{Ure}: «Philosophy of Manufacture», стор. 15. Хто знає біографію
Аркрайта, тому ніколи не спаде на думку назвати цього геніяльного
голяра «благородним». Поміж усіх великих винахідників XVIII віку
він безперечно був найбільший крадій чужих винаходів та наймерзенніший
суб’єкт.
}.
Фабричний кодекс, що в ньому капітал формулює свою автократію
над своїми робітниками приватноправним шляхом та самовладно,
без поділу влади, взагалі такого любого буржуазії, і
без ще улюбленішої репрезентативної системи, — цей кодекс є
лише капіталістична карикатура того суспільного реґулювання
процесу праці, яке стає потрібне при кооперації у великому маштабі
та при вживанні спільних засобів праці, особливо машин.
Місце батога в наглядача за рабами заступає карна книга наглядача.
Всі кари природно сходять до грошової кари й відраховань
із заробітної плати, а законодавча бистродумність фабричних
Лікурґів робить порушування їхніх законів, коли можливо,
ще прибутковішим для них, ніж додержування їх\footnote{
«Рабство, в кайданах якого буржуазія тримає пролетаріят, ніде
так ясно не виявляється, як у фабричній системі. Тут настає кінець
усякій свободі, і юридично, і фактично. Вранці о пів на шосту робітник
мусить бути на фабриці; якщо він спізниться на декілька хвилин, — на
нього накладають кари; якщо він спізниться на 10 хвилин — його зовсім
не впускають до фабрики до кінця сніданку, і він втрачає плату за
чверть дня. Він мусить на команду їсти, пити та спати\dots{} Деспотичний
дзвінок підіймає його з ліжка, відриває його від сніданку та обіду. А як
воно на фабриці? Тут фабрикант — абсолютний законодавець. Він видає
фабричні правила, як йому забажається; змінює та робить додатки
до свого кодексу, які йому захочеться; і хоч які безглузді ці зміни та
додатки, суди все ж таки кажуть робітникові: через те, що ви з доброї
волі згодилися на цей контракт, то мусите тепер його виконувати\dots{} Ці
робітники засуджені від дев’ятого року життя аж до самої смерти жити
під моральною та фізичною палицею». (\emph{F.~Engels}: «Die Lage der
arbeitenden Klasse in England», Leipzig 1845. S. 217 ff. — \emph{Ф.~Енґельс}:
«Становище робітничої кляси в Англії», Партвидав, 1932~\abbr{р.}, стор. 187 і
далі). Що «кажуть суди», це я поясню на двох прикладах. Один випадок
трапився в Шеффілді наприкінці 1866~\abbr{р.} Там один робітник найнявся
на два роки на металеву фабрику. Посварившися з фабрикантом,
він залишив фабрику й заявив, що ні в якому разі не працюватиме
більше на нього. Його оскаржили за зламання контракту й засудили
на два місяці в’язниці. (Коли фабрикант ламає контракт, то його можна
оскаржити лише перед цивільним судом і ризикує він лише грошовою
карою). Після того, як він одсидів ці два місяці, той самий фабрикант
закликає його на основі старого контракту повернутися до фабрики.
Робітник відмовляється. Він, мовляв, одбув уже кару за зламання контракту.
Фабрикант позиває його знову, суд знову засуджує його, дарма
що один із суддів, містер Ші, прилюдно назвав правничою потворністю,
що людину ціле життя можна періодично знову й знову карати за ту саму
провину або злочин. А цей присуд виніс не «Great Unpaid» (сільські
мирові судді), провінціяльні Dogberries, а один із найвищих судів у
Лондоні. [До четвертого видання. Тепер це скасовано. Тепер в Англії,
за винятком деяких випадків, наприклад, на громадських газівнях, робітник
за зламання контракту відповідає нарівні з підприємцем, і його
можна оскаржити лише перед цивільним судом. — \emph{Ф.~Е.}]. — Другий
випадок був у Wiltshire наприкінці листопада 1863~\abbr{р.} Якихось З0 робітниць
при парових ткацьких варстатах, працюючи в якогось Геррупа,
фабриканта сукна в Leoner’s Mill, Westbury, Leigh, улаштували страйк,
бо цей самий Герруп мав приємну звичку стягати з заробітної плати за
спізнення вранці, а саме 6\pens{ пенсів} за 2 хвилини, 1\shil{ шилінґ} за 3 хвилини й
1\shil{ шилінґ} 6\pens{ пенсів} за 10 хвилин. При 9\shil{ шилінґах} за годину це становить
4\pound{ фунти стерлінґів} 10\shil{ шилінґів} на день, тимчасом як їхня пересічна річна
плата ніколи не перевищує 10--12\shil{ шилінґів} на тиждень. Герруп доручив
також одному підліткові повідомляти трубою про фабричні години,
а цей іноді робив це перед шостою годиною вранці; якщо ж руки не з’являлися
саме тоді, коли він кінчав, то брама замикалась, а на тих, що залишалися
за брамою, накладали грошову кару; а що на фабриці не було
годинника, то нещасні руки попадали під владу молодого вартового,
інспірованого від Геррупа. Руки, що почали «страйк», матері родин
та дівчата, заявили, що знову стануть до праці, якщо вартового замінять
годинником та заведуть раціональніший карний тариф. Герруп
оскаржив 19 жінок та дівчат перед судом за зламання контракту. Серед
голосного обурення авдиторії кожну з них засудили до 6\pens{ пенсів} грошової
кари та до 2\shil{ шилінґів} 6\pens{ пенсів} судових витрат. Народна маса провела
Геррупа з суду з шиканням. Одна з улюблених операцій фабрикантів
— це карати робітників відрахуванням із заробітної плати за
кепську якість постачуваного їм матеріялу. Ця метода викликала в 1866~\abbr{р.}
загальний страйк в англійських ганчарняних округах. Звіти «Children’s
Employment Commission» (1863--1866) наводять випадки, коли робітник,
замість одержувати плату, ставав через свою працю та за допомогою
карного регламенту ше й винуватцем своїх ясновельможних «хазяїнів».
Повчальні риси бистродумности фабричних автократів у справі відрахувань
з заробітної плати дала також найновіша бавовняна криза.
«Я сам, — каже фабричний інспектор Р.~Бекер, — мусив нещодавно розпочати
судовий процес проти одного бавовняного фабриканта, бо він у цей тяжкий
та лютий час стягав з кількох «молодих» (понад тринадцять років)
робітників, що в нього працювали, по 10\pens{ пенсів} за лікарську посвідку
про вік, яка коштує йому тільки 6\pens{ пенсів} і за яку закон дозволяє стягати
лише 3\pens{ пенси}, а звичай — нічого\dots{} Другий фабрикант, щоб досягти
тієї самої мети без конфлікту з законом, накладає на кожну бідну дитину,
що на нього працює, данину в 1\shil{ шилінґ} за вивчення вмілости та
таємниці прядіння, скоро тільки лікарська посвідка визнає її за дозрілу
виконувати цю працю. Отже, десь на споді існують течії, що їх треба знати,
щоб зрозуміти такі надзвичайні явища, як страйки за таких часів, як
наші» (мовиться про страйк механічних ткачів на фабриці в Darwen у
червні 1863~\abbr{р.}) «Reports of Insp. of Fact, for 30 th April 1863 p.», p. 50,
51. (Фабричні звіти завжди йдуть далі, ніж їхні офіціяльні дати).
}.

\index{i}{0352}  %% посилання на сторінку оригінального видання
Ми відзначаємо тут лише матеріяльні умови, за яких виконується
фабрична праця. Всі чуттьові органи однаково страждають
від штучно підвищеної температури, від повітря, заповненого
відпадками сировинного матеріялу, від оглушливого шуму
й~\abbr{т. д.}, не кажучи вже про небезпеку для життя серед густо
поставлених машин, які з реґулярністю пір року подають свої
\index{i}{0353}  %% посилання на сторінку оригінального видання
промислові бюлетені про вбитих та покалічених\footnoteA{
Закони для охорони від небезпечних машин дали добрі наслідки.
«Але\dots{} тепер існують нові джерела нещасливих випадків, які ще не існували
перед 20 роками, а саме збільшена швидкість машин. Колеса, вали,
веретена і ткацькі варстати женуть тепер із збільшеною силою, яка щораз
зростає: пучки мусять швидше та певніш хапати увірвану нитку, бо
досить вагання або необережности і будуть жертви\dots{} Багато нещасливих
випадків зумовлюються старанням робітників швидше скінчити свою
працю. Треба собі пригадати, що для фабрикантів дуже важливо тримати
свої машини в безнастанному русі, тобто безупинно продукувати пряжу й
тканини. Кожна зупинка на одну хвилину — це втрата не тільки на рушійній
силі, але й на продукції. Тому наглядачі за працею, заінтересовані
в кількості продукції, підганяють робітників тримати машини в русі,
а це не менш важливо і для робітників, яким платять від ваги або від
штуки. Тому, хоч у більшості фабрик формально й заборонено чистити
машини підчас їхнього руху, на практиці це загальне явище. Сама ця
обставина спричинила за останні шість місяців 906 нещасливих випадків\dots{}
Хоч чищення відбувається день-у-день, але все ж ґрунтовне чищення
машин призначають здебільша на суботу, і воно відбувається здебільше
підчас руху машин\dots{} За цю операцію не платять, і через те робітники
силкуються якомога швидше її скінчити. Тим то число нещасливих
випадків у п’ятницю й особливо в суботу далеко більше, ніж іншими
днями тижня. У п’ятницю число нещасливих випадків перевищує пересічне
число за перші чотири дні тижня приблизно на 12\%, у суботу ж
число нещасливих випадків перевищує пересічне число за попередні
п’ять день на 25\%; але коли взяти на увагу, що фабричний день у суботу
налічує лише 7\sfrac{1}{2} годин, а інші дні тижня 10\sfrac{1}{2}, то це
перевищення становитиме більше ніж 65\%». («Reports of Insp. of Fact, for
31st October 1866». London 1867, p. 9, 15, 16, 17).
}. Заощадження
суспільних засобів продукції, що вперше визріває у фабричній
системі, немов у теплиці, перетворюється в руках капіталу разом
з тим на систематичне грабування життєвих умов робітника
підчас його праці, на грабування простору, повітря, світла
та засобів охорони робітника від небезпечних для життя або антигігієнічних
умов продукційного процесу; про влаштування якихось
вигід для робітника нічого й казати\footnote{
У першому відділі третьої книги я розповім про похід англійських
фабрикантів, що стосується до недавнього часу, проти тих статтей фабричного
закону, які мають захищати члени «рук» від небезпечних для життя
машин. Тут досить однієї цитати з офіційного звіту фабричного інспектора
Леонарда Горнера: «Я чув, як фабриканти з непробачливою легкістю
говорили про деякі нещасливі випадки, наприклад, втрата одного пальця —
це, мовляв, дрібничка. Життя й надії робітника так дуже залежать від
його пальців, що така втрата є для нього надзвичайно серйозна подія.
Слухаючи таке безглузде базікання, я питав їх: «Припустіть, що вам потрібен
один додатковий робітник і до вас з’явилося двоє, обидва з усякого
іншого погляду однаково дужі, але один не має великого або вказівного
пальця, то котрого з них ви вибрали б?» І хвилини не вагаючись, вони
висловилися за того, що має всі пальці\dots{} Ці пани фабриканти мають
фалшиві упередження проти того, що вони називають псевдофілантропічним
законодавством». («Reports of Insp. of Fact. for 31 st October
1855»). Ці пани — меткі людці, і не дурно мріють вони про бунт рабовласників!
}. Чи не правду казав Фур’є, називавши фабрики «пом’якшеною каторгою»\footnote{
На фабриках, здавна підлеглих фабричному законові з його примусовим
обмеженням робочого часу та іншими його постановами, деякі
давніші лиха зникли. Саме поліпшення машин вимагає на якомусь певному
пункті «поліпшеної конструкції фабричних будівель», що йде на
користь робітникам. (Порівн. «Reports etc. for 31 st October 1863», p. 109).
}.

\index{i}{0354}  %% посилання на сторінку оригінального видання
\subsection{Боротьба між робітником і машиною}

Боротьба між капіталістом і найманим робітником починається
разом із виникненням самого капіталістичного відношення. Вона
лютує протягом цілого мануфактурного періоду\footnote{
Див. між іншим: \emph{John Houghton}: «Husbandry and Trade improved»,
London 1727. «The Advantages of the East India Trade», 1720.
\emph{John Belters}: «Proposals for raising a Colledge of Industry», London
1696. «Хазяїни та робітники, на жаль, перебувають у постійній війні
між собою. Незмінна мета хазяїнів — діставати працю для себе якомога
дешевше, і вони не вагаються пускатись на всякі хитрощі, щоб досягти
цієї мети, тимчасом як робітники з такою самою впертістю використовують
усяку нагоду, щоб примусити своїх хазяїнів виконати їхні підвищені
вимоги». («The masters and their workmen are unhappily in a perpetual
state of war with each other. The invariable object of the former is to
get their work done as cheap as possibly: and they do not fail to employ
every artifice to this purpose, whilst the latter are equally attentive to
every occasion of distressing their masters into a compliance with higher
demands»). «An Inquiry into the causes of the Present High Prices of
Provisions», 1767, p.61, 62. (Автор — панотець Натаніел Ферстер, що цілком
стоїть на боці робітників).
}. Але лише із заведенням машин робітник починає боротися проти самого
засобу праці, цієї матеріяльної форми існування капіталу. Він
повстає проти цієї певної форми засобу продукції як матеріяльної
основи капіталістичного способу продукції.

Мало не ціла Европа пережила у XVII віці повстання робітників
проти так званої Bandmühle (або інакше Schnurmühle
або Mühlenstuhl»)\footnote*{
стьожковий млин. \emph{Ред.}
}, проти машини для ткання стьожок та брузументу\footnote{
Bandmühle винайдено в Німеччині. Італійський абат Лянчелотті
у своїй праці, що появилась у Венеції 1636~\abbr{р.}, оповідає: «Антон Міллер
з Данціґу якихось 50 років тому (Лянчелотті писав 1579~\abbr{р.}) бачив у Данціґу
дуже мудру машину, яка виготовляла одночасно 4--6 тканин; міська
рада, турбуючися про те, що цей винахід може поробити масу робітників
старцями, затаїла цей винахід, а винахідника наказала потайки задушити
або втопити». В Ляйдені таку саму машину вперше вжито 1629~\abbr{р.} Заколоти
серед брузументників примусили маґістрат спочатку заборонити її;
генеральні штати своїми різними постановами з 1623, 1639~\abbr{рр.} і~\abbr{т. д.}
мали обмежити її вживання; нарешті, її дозволено під певними умовами
постановою 15 грудня 1661~\abbr{р.} «У цьому місті, — каже Боксхерн («Inst.
Роl. 1663») про заведення стьожкової машини в Ляйдені, — якихось 20 років
тому винайдено ткацький варстат, на якому один робітник міг легше й
більше виробляти тканин, аніж багато робітників за той самий час без
варстату. Але це спричинилося до скарг та заколотів серед ткачів, і магістрат,
нарешті, заборонив уживати машини» («In hac urbe ante hos viginti
circiter annos instrumentum quidam invenerunt textorium, quo solus quis
plus panni et facilius conficere poterat, quam plures aequali tempore. Hinc
turbae ortae et querulae textorum, tandemque usus hujus instrumenti a
magistratu prohibitus est»). (\emph{Boxhorn}: «Institutiones politicae», Leyden
1663). Ту саму машину 1676 p. заборонено в Кельні, тимчасом як введення
її в Англії в той самий час спричинилось до заколотів серед робітників.
Королівським едиктом з 19 лютого 1685~\abbr{р.} заборонено вживати її по всій
Німеччині. В Гамбурзі з наказу маґістрату її прилюдно спалили. Карл VI
відновив 9 лютого 1719~\abbr{р.} едикт з 1685~\abbr{р.}, а в саксонському курфюрстві
загальний вжиток її дозволено лише 1765~\abbr{р.} Ця машина, що наробила в
світі стільки шуму, була в дійсності попередницею прядільних і ткальних
машин, отже, і промислової революції XVIII віку. Вона робила цілком
недосвідченого у ткацтві підлітка здатним пускати в рух цілий
варстат з усіма його човниками, через саме лише совання рушієм туди
й назад; у своїй поліпшеній формі вона давала воднораз 40--50 сувоїв
тканини.
}. Наприкінці першої третини XVII віку чернь підчас
заколотів знищила вітряну лісопильню, поставлену якимось голляндцем
коло Лондону. Ще на початку XVII віку тартаки,
\index{i}{0355}  %% посилання на сторінку оригінального видання
гнані водою, лише насилу перемагали в Англії народній опір,
що його підтримував парлямент. Коли 1758~\abbr{р.} Іврі збудував
першу машину стригти овець, гнану водою, її спалили кілька сот
людей, які опинилися без праці. Проти scribbling mills\footnote*{
машин для першого, грубого чесання вовни. \emph{Ред.}
} та чухральних машин Аркрайта \num{50.000} робітників, які досі жили з
чухрання вовни, звернулися з петицією до парляменту. Масове
руйнування машин в англійських мануфактурних округах протягом
перших 15 років XIX віку, зумовлене вживанням парових
варстатів і відоме під назвою руху луддітів, дало антиякобінському
урядові Сідмавта, Castlereagh’a та інших привід до
якнайреакційніших насильницьких кроків. Треба часу й досвіду,
щоб робітник навчився відрізняти машину від капіталістичного
вживання її, а тому й переносити свої напади з самих матеріяльних
засобів продукції на суспільну форму експлуатації їх\footnote{
У старомодних мануфактурах ще й за наших часів повторюються
інколи грубі форми обурення робітників проти машин. Так, наприклад,
у виробництві напилків у Шеффілді 1865~\abbr{р.}
}.

Боротьба за заробітну плату в мануфактурі припускає наявність
мануфактури й зовсім не скерована проти її існування. Якщо
хто і боровся проти утворення мануфактур, то робили це не наймані
робітники, а цехові майстри й упривілейовані міста. Тому письменники
мануфактурного періоду розуміють поділ праці переважно
як засіб заміняти робітників у можливості, а не в дійсності витискувати
з мануфактур робітників\footnote*{
У французькому виданні це речення подано так: «Письменники
мануфактурного періоду в поділі праці вбачають можливий засіб поповнювати
недостачу в робітниках, а не витискувати з мануфактур робітників,
що вже працюють». \emph{Ред.}
}. Ця ріжниця сама собою
\parbreak{}  %% абзац продовжується на наступній сторінці

\parcont{}  %% абзац починається на попередній сторінці
\index{iii1}{0356}  %% посилання на сторінку оригінального видання
й капіталіст, який працює власним капіталом. Обидва одержували
б однаковий пересічний зиск, а капітал, чи взятий у позику,
чи власний, діє як капітал лиш остільки, оскільки він
виробляє зиск. Умова повернення капіталу нічого не змінила б
у цьому. Чим більше розмір процента наближається до нуля,
отже, наприклад, знижується до 1\%, тим більше взятий у позику
капітал стає в однакове становище з власним капіталом.
Поки грошовий капітал має існувати як грошовий капітал,
він мусить знову й знову віддаватись у позику, і при тому
за існуючий процент, скажімо, за 1\%, і завжди тому самому
класові промислових і торговельних капіталістів. Поки ці останні
функціонують як капіталісти, ріжниця між тим, хто функціонує
за допомогою взятого у позику капіталу, і тим, хто функціонує за
допомогою власного капіталу, полягає тільки в тому, що один
повинен сплачувати проценти, а другий — ні; один кладе собі в
кишеню весь зиск $р$, а другий $р — z$, зиск мінус процент; чим
більше $z$ наближається до нуля, тим більше $р — z$ наближається
до $р$, отже, тим більше обидва капітали будуть в однаковому
становищі. Один мусить сплачувати капітал назад і знову брати
його в позику; а другий, поки його капітал має функціонувати,
теж мусить знову й знову авансовувати капітал для процесу
виробництва і не може ним порядкувати незалежно від цього
процесу. Зрештою лишається ще, єдина, сама собою зрозуміла
ріжниця, яка полягає в тому, що один з них є власник свого капіталу,
а другий — ні.

Тепер напрошується таке питання. Яким чином цей чисто
кількісний поділ зиску на чистий зиск і процент обертається
у якісний? Іншими словами, яким чином капіталіст, який застосовує
тільки свій власний капітал, а не взятий у позику, теж
підводить частину свого гуртового зиску під окрему категорію
процента і окремо обчислює його як такий? І, отже, далі, яким
чином усякий капітал, чи взятий у позику, чи ні, як капітал,
що дає процент, відрізняється від самого себе як капіталу, що
дає чистий зиск?

Відомо, що не кожний випадковий кількісний поділ зиску
такого роду обертається в якісний. Наприклад, декілька промислових
капіталістів об’єднуються в асоціацію для ведення підприємства
і потім розподіляють між собою зиск відповідно до
юридично укладеного договору. Інші провадять своє підприємство
кожний сам за себе, без associé [компаньйонів]. Ці останні
обчислюють свій зиск не за двома категоріями, — одну частину
як особистий зиск, а другу як компанійський зиск для неіснуючих
спільників. Отже, тут кількісний поділ не обертається
в якісний. Поділ відбувається, коли власник випадково складається
з декількох юридичних осіб; він не відбувається, коли
цього немає.

Щоб відповісти на це питання, нам доведеться ще дещо довше
спинитися на дійсному вихідному пункті утворення процента;
\parbreak{}  %% абзац продовжується на наступній сторінці

\parcont{}  %% абзац починається на попередній сторінці
\index{i}{0357}  %% посилання на сторінку оригінального видання
праці — вівці, коні і т. д., — безпосередні акти насильства становлять
тут першу передумову промислової революції. Спочатку
проганяють робітників із землі, а потім з’являються вівці. І тільки
розкрадання землі у великому маштабі, як ось в Англії, створює
для великого рільництва поле його діяльности.\footnoteA{
[До четвертого видання. — Це стосується й до Німеччини. Там,
де в нас існує велике рільництво, отже, саме на Сході, воно стало можливим
лише через застосування системи «Bauernlegen»,\footnote*{
Так називався в Німеччині процес експропріяції земель у селян;
в Англії цей процес звався «Clearing of Estates» («очищення маєтків» —
у дійсності очищення їх від людей). Див. про це далі розділ 24, §2. Ред.
} яке почалося
в XVI віці, особливо від 1648 р. — Ф. Е.].
} Тому на своїх
початках цей переворот у рільництві на позір має скорше вигляд
політичної революції.

Засіб праці у формі машини відразу стає конкурентом самого
робітника.\footnote{
«Машини й праця перебувають у постійній конкуренції» («Machinery
and labour are in constant competition»). (Ricardo: «Principles of
Political Economy». 3 rd ed., London 1821, p. 479).
} Самозростання вартости капіталу за допомогою
машини стоїть у прямому відношенні до числа робітників, умови
існування яких вона нищить. Ціла система капіталістичної продукції
ґрунтується на тому, що робітник продає свою робочу силу
як товар. Поділ праці уоднобічнює робочу силу на цілком частинну
вмілість — керувати частинним знаряддям. Скоро тільки
керування знаряддям переходить до машини, то разом із споживною
вартістю зникає й мінова вартість робочої сили. Робітник не
находить собі покупців, як паперові гроші, що виключені з обігу.
Та частина робітничої кляси, що її машини таким способом перетворюють
у надмірну людність, тобто в таку, яка безпосередньо
вже не потрібна для самозростання капіталу, з одного боку, гине
в нерівній боротьбі старого ремісничого й мануфактурного виробництва
з машиновим виробництвом, з другого боку, переповнює
всі приступніші галузі промисловости, переповнює ринок праці,
а тому знижує ціну робочої сили нижче за її вартість. Великою
втіхою для павперизованих робітників має бути те, що їхні страждання,
мовляв, почасти лише «тимчасові» («а temporary inconvenience»),
а почасти те, що машини, мовляв, лише поступінно
опановують ціле поле продукції, а через те зменшується розмір
та інтенсивність їхнього руйнаційного діяння. Одна втіха побиває
другу. Там, де машина захоплює якесь поле продукції поступінно,
вона породжує хронічні злидні серед робітничих верств, які з
нею конкурують. Там, де перехід відбувається швидко, там її
 вплив є масовий і гострий. Немає в світовій історії жахливішого
видовища, як поступінне вимирання англійських ручних бавовняних
ткачів, що тривало цілі десятиліття і, нарешті, завершилося
1838 р. Багато з них померло з голоду, багато животіло довгий
час із своїм родинами, мавши 2 1/2 пенси на день.\footnote{
Конкуренція між ручним та машиновим тканням перед заведенням
закону з 1833 р. про бідних затягувалася в Англії тим, що заро-
} Навпаки,
гостро подіяло заведення англійських бавовняних машин у
\index{i}{0358}  %% посилання на сторінку оригінального видання
Східній Індії, генерал-губернатор якої 1834—35 рр. констатував:
«Ледве чи знайдеться аналогія до цих злиднів в історії
торговлі. Рівнини Індії біліють від кісток бавовняних ткачів».
Щоправда, оскільки ці ткачі переставились, покинули це тимчасове
шиття, остільки і машини заподіяли їм лише «тимчасових
страждань». Зрештою, це «тимчасове» діяння машин є перманентне,
бо вони постійно захоплюють нові сфери продукції. Отже,
характер усамостійнення та відчужености, що його капіталістичний
спосіб продукції надає взагалі умовам праці та продуктові
праці супроти робітника, розвивається з виникненням
машин до повного антагонізму.\footnote{
«Та сама причина, яка може збільшити дохід країни (тобто, як
тут же пояснює Рікардо, доходи лендлордів та капіталістів, багатство
(wealth) яких з економічного погляду взагалі дорівнює багатству нації
(wealth of the nation), може одночасно утворити надмір людности та
погіршити становище робітника» («The same cause which may increase
the revenue of the country may at the same time render the population
redundant and deteriorate the condition of the labourer»). (Ricardo:
«Principles of Political Economy», 3 rd ed. London 1821, p. 469). «Постійна
мета й тенденція кожного вдосконалення механізму фактично є в тому,
щоб цілком збутися праці людини або зменшити її ціну, замінюючи працю
} Саме через це з виникненням
машин уперше вибухають жорстокі повстання робітників проти
засобу праці.

бітну плату, яка зменшилась далеко нижче за мінімум, поповнювано
допомогами парафій. «1827 р. високопреподобний Тернер був парохом у
Wilmslow’i, у Чешірі, в мануфактурній окрузі. Питання комітету в справі
еміграції й відповіді пана Тернера показують, як підтримували конкуренцію
людської праці проти машин. Питання: «Чи не усунено вживанням
механічних варстатів вживання ручних варстатів?» Відповідь: «Безперечно,
воно усунуло б його ще в більшій мірі, ніж це е в дійсності, коли б
ручні ткачі не мали змоги згоджуватися на зниження заробітної плати».
Питання: «Але, згоджуючися на це, чи не наймаються вони за плату,
якої не вистачає їм для їхнього існування, та чи не сподіваються вони
допомоги з парафії, щоб покрити недостачу?» Відповідь: «Так, і конкуренцію
між ручним варстатом і механічним варстатом фактично підтримує
лише податок для бідних». Отже, ганебний павперизм або еміграція — ось
вигоди, що їх мають робітники у наслідок заведення машин. Із почесних
та до певної міри незалежних ремісників їх зводять на становище плазівної
голоти, що живе з принизливого хліба добродійности. Ось що
вони називають тимчасовими труднощами». («The Rev. Mr. Turner was
in 1827 rector of Wilmslow, in Cheshire, a manufacturing district. The questions
of the Committee on Emigration, and Mr. Turner’s answers show
how the competition of human labour is maintained against machinery.
Question: «Has not the use of the power-loom superseded the use of the
hand-loom?» Answer: «Undoubtedly; it would have superseded them much
more than it has done, if the handloom weavers were not enabled to submit
to ä reduction of wages». Question: «But in submitting he has accepted wages
which are insufficient to support him, and looks to parochial contribution
as the remainder of his support?» Answer: «yes, and in fact the competition
between the hand-loom arid the power-loom is maintained out the poorrates»
. Thus degrading pauperism or expatriation, is the benefit which the
industrious receive from the introduction of machinery, to be reduced from
the respectable and in some degree independent mechanic, to the cringing
wretch who lives on the debasing bread of charity. This they call a temporary
inconvenience»). («A Prize Essay on the comparative merits of Competition
and Cooperation», London 1834, p. 29).

\index{i}{0359}  %% посилання на сторінку оригінального видання
Засіб праці вбиває робітника. Певна річ, ця безпосередня
протилежність найнаочніше виявляється тоді, коли новозаведена
машина конкурує з традиційним ремісничим або мануфактурним
виробництвом. Але й у межах самої великої промисловости постійне
поліпшування машин і розвиток автоматичної системи діють аналогічно.
«Постійна мета поліпшення машин є в тому, щоб зменшити
ручну працю або вдосконалити ланку в продукційному
ланцюзі фабрики, замінивши люський апарат залізним».\footnote{
«Reports of Insp. of Fact, for 31 st October 1858», p. 43.
}
«Застосування сили пари й води до машин, що їх досі рухалось
рукою, трапляється щодня... Незначні поліпшення в машинах,
що мають на меті заощадити на рушійній сипі, поліпшити продукт,
збільшити продукцію протягом того самого часу, витиснути дитину,
жінку або чоловіка, — такі поліпшення робиться постійно
і, хоч на око вага цих поліпшень невелика, все ж вони дають
важливі результати».\footnote{
«Reports |of Insp. of Fact, for 31 st October 1856», p. 15.
} «Повсюди, де якась операція потребує
чималої вправности та певної руки, її якомога швидше забирають
із рук надто навченого робітника, що має часто нахил до нереґулярности
всякого роду, щоб доручити її осібному механізмові,
який так добре вреґульований, що за ним може наглядати й
мала дитина».\footnote{
Ure: «Philosophy of Manufacture», p. 19. «Велика перевага
машин, що їх уживають на цегельнях, є в тому, що вони роблять хазяїна
незалежним від навчених робітників». («Children’s Employment Commission.
5 th Report», London 1866, p. 180, n. 46).

Додаток до другого видання. Пан А. Стеррок, головний управитель
машинового відділу «Great Northern Railway», висловлюється так про
будування машин (льокомотивів і т. д.): «Дорогих (expensive) англійських
робітників із дня на день потребують щораз менше. Продукція збільшується
через уживання поліпшених інструментів, а ці інструменти із свого
боку обслуговує нижчий рід праці (a low class of labour)... Раніш усі частини
парової машини продукувала неодмінно кваліфікована праця.
Ті самі частини тепер продукує менш кваліфікована праця, але з добрими
інструментами... Під інструментами я розумію машини, що їх уживають
в машинобудуванні». («Royal Commission on Railways. Minutes of Evidence»,
n. 17 862 and 17 863. London 1867).
} «За автоматичної системи талант робітника проґресивно
витискується».\footnote{
Ure: «Philosophy of Manufacture», , p. 20.
} «Поліпшення машин не тільки вимагає
зменшити число дорослих робітників, уживаних, щоб досягти
певного результату, але воно ще й заміняє одну клясу індивідів
на другу клясу, більш навчених на менш навчених, дорослих
на дітей, чоловіків на жінок. Всі ці переміни призводять до постійних
коливань у нормі заробітної плати».\footnote{
Там же, стор. 321.
} «Машини безупинно
викидають дорослих із фабрики».206 Надзвичайну еластичність
машинової системи як наслідок нагромадженого практичного
досвіду, як наслідок наявного вже розміру механічних
засобів та постійного проґресу техніки, виявив нам бурхливий

дорослих робітників-чоловіків працею жінок та дітей або працю навчених
робітників працею чорноробів». (Ure: «Philosophy of Manufacture», p.23).

205 Там же, стор. 23.
\index{i}{0360}  %% посилання на сторінку оригінального видання
розвиток цієї системи, що відбувався під тиском скорочення
робочого дня. Але хто міг би 1860 р., року зенітного розвитку
англійської бавовняної промисловости, передбачати ті чимраз
швидші поліпшення машин і відповідне витискування ручної
праці, що їх викликали три наступні роки під тиском американської
громадянської війни? Щодо цього пункту тут досить кількох
прикладів з офіціяльних даних англійських фабричних інспекторів.
Один менчестерський фабрикант заявляє: «Замість 75 чухральних
машин ми потребуємо тепер лише 12, і вони дають нам
таку саму кількість продуктів такої самої, якщо не ліпшої,
якости... Заощадження на заробітній платі становить 10 фунтів
стерлінґів на тиждень, заощадження на відпадках бавовни — 10\%».
В одній менчестерській тонкопрядільні «через прискорення руху
й заведення різних автоматичних (self-acting) процесів усунено
в одному відділі 1/4, у другому більш ніж 1/2 робітничого персоналу,
тимчасом як чесальна машина, що заступила другу чухральну
машину, дуже зменшила число робітників, занятих раніш у чухральному
відділі». Інша прядільна фабрика оцінює свої загальні заощадження
на «руках» у 10\%. Панове Джілмер, фабриканти-прядільники
в Менчестері, заявляють: «Заощадження в нашому відділі
blowing (чищення бавовни) на руках та заробітній платі, досягнуті
в наслідок заведення нових машин, ми оцінюємо у цілу третину...
у відділах jack frame і drawing frame room витрати на руки та інші
видатки зменшились приблизно на 1/3, у прядільному відділі видатки
зменшились приблизно на 1/3. Але це не все: якщо наша пряжа
йде тепер до ткача, то в наслідок застосування нових машин її
так дуже поліпшено, що ткачі продукують більше та ліпші тканини,
аніж із колишньої машинової пряжі».\footnote{
«Reports of Insp. of Fact, for 31 st October 1863», p. 108 і далі.
} Фабричний інспектор
А. Редґрев додає до цього: «Зменшення числа робітників
при збільшенні продукції швидко проґресує; по вовняних фабриках
недавно знову почалося зменшення рук, і це зменшення триває
далі; перед кількома днями один учитель, що мешкає коло Рочделя,
сказав мені, що величезне зменшення школярок по дівочих школах
зумовлене не тільки натиском кризи, а ще й тими змінами
в машинах вовняної фабрики, наслідком яких там сталося зменшення
рук пересічно на 70 робітників половинного часу».\footnote{
Там же, стор.109. Швидке поліпшення машин підчас бавовняної
кризи дозволило англійським фабрикантам зараз же по скінченні американської
громадянської війни знову миттю переповнити світовий ринок.
Уже в останні шість місяців 1866 р. тканин майже не можна було продати.
Тоді почався вивіз товарів у Китай та Індію на комісію, що, природно,
зробило «glut»\footnote*{
— пересичення ринку. Ред.
} ще інтенсивнішим. На початку 1867 р. фабриканти вдалися
до свого звичайного зарадчого способу, до зниження заробітної
плати на 5\%. Робітники опирались та заявили, теоретично цілком правильно,
що єдине, чим тут можна зарадити, — це працювати скорочений
час, чотири дні на тиждень. Після довгих вагань капітани промисловости,
як вони сами називали себе, — змушені були згодитися на це, подекуди
із зниженням заробітної плати на 5\%, подекуди без зниження її.
}

\index{i}{0361}  %% посилання на сторінку оригінального видання
Загальний результат механічних поліпшень, заведених в англійській
бавовняній промисловості під впливом американської
громадянської війни, показує оця таблиця:

\begin{center}
  \noindent\begin{tabularx}{\textwidth}{Xrrr}
     \multicolumn{4}{c}{\textbf{Число фабрик}} \\
                         & 1858 р. & 1861 р. & 1868 р. \\
   Англія та Велз\dotfill{} & 2.046   & 2.715   & 2.405 \\
   Шотляндія\dotfill{} & 152 & 163 & 131 \\
   Ірляндія\dotfill{} & 12 & 9 & 13 \\
   \cmidrule{2-4}
   Об’єднане Королівство\dotfill{} & 2.210 & 2.887 & 2.549 \\
     \addlinespace
     \multicolumn{4}{c}{\textbf{Число парових ткацьких варстатів}} \\
   Англія та Велз\dotfill{} & 275.590 & 368.125 & 344.719 \\
   Шотляндія\dotfill{} & 21.624 & 30.110 & 31.864 \\
   Ірляндія\dotfill{} & 1.633 & 1.757 & 2.746 \\
   \cmidrule{2-4}
   Об’єднане Королівство\dotfill{} & 298.847 & 399.992 & 379.329 \\
     \addlinespace
     \multicolumn{4}{c}{\textbf{Число веретен}} \\
   Англія та Велз\dotfill{} & 25.818.576 & 28.352.152 & 30.478.228 \\
   Шотляндія\dotfill{} & 2.041.129 & 1.915.398 & 1.397.546 \\
   Ірляндія\dotfill{} & 150.512 & 119.944 & 124.240 \\
   \cmidrule{2-4}
   Об’єднане Королівство\dotfill{} & 28.010.217 & 30.387.494 & 32.000.014 \\
     \addlinespace
     \multicolumn{4}{c}{\textbf{Чиcло вживаних робітників}} \\
    Англія та Велз\dotfill{} & 341.170   & 407.598 & 357.052 \\
    Шотляндія\dotfill{} & 34.698 & 41.237 & 39.809 \\
    Ірляндія\dotfill{} &  3.345 &  2.734 & 4.203 \\
    \cmidrule{2-4}
    Об’єднане Королівство\dotfill{} & 379.213& 451.569 & 401.064 \\
  \end{tabularx}
\end{center}

Отже, від 1861 до 1868 р. зникло 338 бавовняних фабрик, тобто
продуктивніший та більший машиновий механізм сконцентрувався
в руках меншого числа капіталістів. Число парових ткацьких
варстатів зменшилося на 20.663; але продукт їхній одночасно
збільшився, так що поліпшений ткацький варстат давав
тепер більше продукту, ніж старий. Нарешті, число веретен
зросло на 1.612.541, тимчасом як число вживаних робітників
зменшилося на 50.505. Отже, ті «тимчасові» злидні, що ними бавовняна
криза душила робітників, збільшив і зміцнив хуткий
та невпинний проґрес машинової системи.

Однак машина діє не тільки як непереможний конкурент,
який завжди напоготові зробити найманого робітника «зайвим».
Капітал голосно й тенденційно проголошує її силою, ворожою
робітникові, та саме як таку вживає її. Вона стає наймогутнішим
бойовим знаряддям придушувати періодичні робітничі повстання,
страйки і т. ін. проти автократії капіталу.\footnote{
«Відносини між хазяїнами й руками по фабриках флінтґлясу та пляшкового
скла — це хронічний страйк». Звідси швидкий розвиток мануфактури
пресованого скла, де головні операції виконуються за допомогою машин.
Одна фірма в Ньюкестлі, яка раніш продукувала 350.000 фунтів дутого
кремінного скла річно, тепер замість цієї кільцости продукує 3.000.500
фунтів пресованого скла». («Children’s Employment Commission. 4 th
Report 1865», p. 262, 263).
} За Ґаскелем,
\index{i}{0362}  %% посилання на сторінку оригінального видання
парова машина з самого початку була антагоністом «людської
сили», що дав капіталістам змогу розбивати щораз більші
домагання робітників, які загрожували кризою фабричній системі
на самому початку її виникнення.\footnote{
\emph{Gaskell}: «The Manufacturing Population of England», London
1833, p. 3, 4.
} Можна було б написати
цілу історію винаходів, які, починаючи від 1830 р., покликано
до життя лише як бойове знаряддя капіталу проти повстань робітників.
Ми нагадаємо передусім selfacting mule,\footnote*{
— автоматичну прядільну машину. \emph{Ред.}
} бо нею починається
нова епоха автоматичної системи.\footnote{
Деякі дуже важливі застосування машин, щоб будувати машини,
винайшов п. Ферберн під впливом страйків на його власній фабриці.
}

У своєму свідченні перед комісією, що їй доручено було дослідити
Trades-Unions, Несміс, винахідник парового молота, повідомляє
про поліпшення в машинах, які він завів у наслідок
великого та довгого страйку машинових робітників у 1851 р.,
таке: «Характеристична риса наших сучасних механічних поліпшень
— це заведення самодіяльних виконавчих машин. Все, що
тепер має робити механічний робітник, і що може зробити всякий
підліток, — це не самому працювати, а лише наглядати за прегарною
роботою машини. Цілу клясу робітників, що залежить
виключно від своєї вмілости, тепер усунено. Раніш я на одного
механіка мав чотирьох хлопців. Завдяки цим новим механічним
комбінаціям я зменшив число дорослих чоловіків з 1.500 на 750.
Наслідком цього було значне збільшення мого зиску».

Про одну машину для друку фарбами на перкалевибійних
фабриках Юр каже: «Нарешті капіталісти почали шукати способу
визволитися з-під цієї нестерпної неволі (тобто від тяжких
для них умов контракту з робітниками), покликавши собі на допомогу
джерела науки, і незабаром їх відновили в їхніх законних
правах, правах голови над іншими частинами тіла». Про один
винахід для шліхтування основи, що його безпосередньою причиною
був страйк, він каже так: «Орда незадоволених, що, окопавшися
за старими лініями поділу праці, вважала себе за непереможну,
побачила себе таким чином оточеною з флангів, а свої
оборонні засоби знищеними сучасною механічною тактикою. Вони
мусили здатися на ласку та гнів переможців». Про винахід
selfacting mule він каже: «Вона була покликана, щоб відновити
порядок серед промислових кляс\dots{} Цей винахід потверджує
розвинуту вже нами доктрину, що капітал, примусивши науку
служити йому, завжди силує бунтівничу руку праці до покірливости».\footnote{
\emph{Ure}: «Philosophy of Manufacture», стор. 367--370.
} Хоч твір Юра з’явився 1835 р., отже, за часів порівняно
мало ще розвинутої фабричної системи, все ж він лишається клясичним
виразом духу фабрики не тільки через свій щирий цинізм,
але й через ту наївність, з якою він виказує абсурдні суперечності
капіталістичного мозку. Розвинувши, приміром, «доктрину»,
що капітал за допомогою науки, взятої ним на утримання, «завжди
\parbreak{}  %% абзац продовжується на наступній сторінці

\parcont{}  %% абзац починається на попередній сторінці
\index{i}{0363}  %% посилання на сторінку оригінального видання
силує бунтівничну руку праці до покірливости», він обурюється
з того, «що з певного боку обвинувачують механічно-фізичну
науку в тому, що вона віддалася на волю деспотизмові багатих
капіталістів та погодилася бути засобом утиску бідних кляс».
Після довгого та голосного проповідування корисности для робітників
швидкого розвитку машин, він застерігає їх, що вони своїм
опором, страйками й т. ін. прискорюють розвиток машин. «Такі
ґвалтовні заколоти, — каже він, — виявляють короткозорість людини
в її найогидливішій формі, короткозорість людини, що сама
себе робить своїм катом». Декілька сторінок перед тим читаємо
протилежне: «Без тих гострих колізій та перерв, спричинюваних
помилковими поглядами робітників, фабрична система була б
розвинулася далеко швидше та з далеко більшою корисністю
для всіх заінтересованих сторін». А далі знову вигукує: «На
щастя для людности фабричних округ Великобританії поліпшення
в механіці відбуваються лише поступінно». «Несправедливо, —
каже він, — обвинувачують машини в тому, що вони зменшують
заробітну плату дорослих, витискуючи певну частину з них, через
що число їх перевищує потребу в праці. Але ж вони збільшують
попит на працю дітей та підносять таким чином рівень заробітної
плати дорослих». З другого боку, цей самий утішник боронить
низьку заробітну плату дітей тим, що «вона спиняє батьків посилати
надто рано дітей до фабрик». Ціла його книжка — то апологія
необмеженого робочого дня, і коли законодавство забороняє
мордувати 12-літніх дітей більше, ніж 12 годин на добу, то це
нагадує його ліберальній душі найтемніші часи середньовіччя.
Це не заважає йому закликати фабричних робітників до молитви
з подякою провидінню за те, що воно за допомогою машин
«дало їм вільний час розмірковувати над своїми безсмертними
інтересами».\footnote{
\emph{Ure}: «Philosophy of Manufacture», стор. 368, 7, 370, 280, 321, 281, 475.
}

\section*{6. Теорія компенсації відносно робітників, витискуваних
машинами}

Цілий ряд буржуазних економістів, як от Джемс Мілл, Мак
Куллох, Торенс, Сеніор, Джон Стюарт Мілл і інші, твердять,
що всі машини, які витискують робітників, завжди й неминуче
звільняють у той самий час відповідний капітал, щоб дати заняття
тим самим робітникам.\footnote{
Рікардо спочатку поділяв цей погляд, але пізніше з характеристичною
для нього науковою безсторонністю та любов’ю до правди виразно
відмовився від нього. Див. \emph{David Ricardo}: «Principles of Political
Economy», розд. 31. «On Machinery».
}

Припустімо, що якийсь капіталіст уживає 100 робітників,
приміром, у шпалерній мануфактурі, при річній заробітній платі
в 30\pound{ фунтів стерлінґів} на кожного. Отже, витрачуваний ним річно
змінний капітал становить \num{3.000}\pound{ фунтів стерлінґів}. Припустімо,
\parbreak{}  %% абзац продовжується на наступній сторінці

\parcont{}  %% абзац починається на попередній сторінці
\index{i}{0364}  %% посилання на сторінку оригінального видання
що він звільняє 50 робітників, а решту — 50 робітників — уживає
коло машин, які коштують йому 1.500 фунтів стерлінґів. Щоб
справу спростити, ми залишаємо осторонь будівлі, вугілля тощо.
Припустімо, далі, що споживаний кожного року сировинний
матеріял коштує, як і раніш, 3.000 фунтів стерлінґів.\footnote{
Nota bene. Я подаю ілюстрацію цілком на манір вищеназваних
економістів.
} Чи
«звільнився» через цю метаморфозу якийсь капітал? За старого
способу продукції загальна витрачена сума становила 6.000 фунтів
стерлінґів і складалася наполовину із сталого, наполовину із
змінного капіталу. Тепер вона складається з 4.500 фунтів стерлінґів
(3.000 фунтів стерлінґів на сировинний матеріял та 1.500 фунтів
стерлінґів на машини) сталого та 1.500 фунтів стерлінґів змінного
капіталу. Замість половини, змінна або перетворена на живу робочу
силу частина капіталу становить лише\sfrac{1}{4} цілого капіталу. Замість
звільнення відбувається тут зв’язування капіталу в такій формі,
що в ній він перестає обмінюватися на робочу силу, тобто відбувається
перетворення змінного капіталу на сталий. Капітал у 6.000
фунтів стерлінґів, за інших незмінних умов, може тепер давати заняття
не більш, як 50 робітникам. З кожним поліпшенням машин він
дає заняття дедалі меншому числу робітників. Коли б новозаведені
машини коштували менше за суму, що її коштували витиснута
ними робоча сила й знаряддя праці, тобто, наприклад, замість
1.500 фунтів стерлінґів лише 1.000 фунтів стерлінґів, то змінний
капітал у 1.000 фунтів стерлінґів перетворився б на сталий капітал,
тобто був би зв’язаний, а капітал у 500 фунтів стерлінґів
звільнився б. Останній, якщо припустити ту саму річну плату,
становить фонд заняття приблизно для 16 робітників, а звільнено
їх 50 — навіть багато менше, ніж для 16 робітників, бо для того,
щоб ці 500 фунтів стерлінґів перетворити на капітал, треба частину
з них знов перетворити на сталий капітал, і отже, лише частину
з них можна перетворити на робочу силу.

Але припустімо навіть, що виготовлювання нових машин дає
заняття більшому числу механіків. Чи буде це компенсацією
для викинутих на брук шпалерників? У найліпшому випадку
виготовлювання нових машин дасть заняття меншому числу
робітників, ніж витискує вживання машин. Сума в 1.500 фунтів
стерлінґів, яка репрезентує лише заробітну плату звільнених
шпалерників, репрезентує тепер у формі машин: 1) вартість засобів
продукції, потрібних на виготовлення машин; 2) заробітну
плату механікам, що їх виготовляють; 3) додаткову вартість, що
припадає їхньому «хазяїнові». Далі: машина, бувши виготовлена,
аж до самої своєї смерти не потребує, щоб її відновлювали. Отже,
щоб постійно давати заняття додатковому числу механіків, фабриканти
шпалер один по одному мусять витискувати робітників
машинами.

В дійсності ці апологети мають на думці не цей рід звільнення
капіталу. Вони мають на думці засоби існування звільнених робітників.
\index{i}{0365}  %% посилання на сторінку оригінального видання
Не можна заперечити, що у вищенаведеному випадку,
приміром, машини не тільки звільняють 50 робітників і через це
роблять їх «вільними», але разом з тим ще знищують їхній зв’язок
із засобами існування вартістю в 1.500 фунтів стерлінґів та
«звільняють» таким чином ці засоби існування. Отже, той простий
і зовсім не новий факт, що машина звільняє робітника від
засобів існування, мовою економістів означає, що машина звільняє
засоби існування для робітника або перетворює їх на капітал,
щоб уживати робітника. Як бачимо, все залежить від того, яким
способом що висловити. Nominibus mollire licet mala.\footnote*{
Можна гарними словами підсолоджувати лихо. \emph{Ред.}
}

За цією теорією засоби існування вартістю в 1.500 фунтів
стерлінґів були капіталом, що збільшив свою вартість за допомогою
праці п’ятдесятьох звільнених шпалерників. Отже, цей
капітал втрачає своє заняття, скоро тільки ті п’ятдесят робітників
звільняються від роботи, та не має і хвилини спокою, поки
не знайде нове «вміщення», де названі п’ятдесят робітників
знову зможуть споживати його продуктивно. Отже, раніш або
пізніш, капітал і робітники знову мусять зійтися, і тоді матимемо
компенсацію. Отож, страждання робітників, витиснутих машинами,
так само минущі, як і багатства цього світу.

Засоби існування в сумі 1.500 фунтів стерлінґів ніколи не протистояли
звільненим робітникам як капітал. Як капітал протистояли
їм ті 1.500 фунтів стерлінґів, які перетворено тепер на
машини. Коли ближче придивитися, то ці 1.500 фунтів стерлінґів
репрезентують лише ту частину шпалер, щорічно продукованих
за допомогою звільнених робітників, яку вони одержували
від свого хазяїна як заробітну плату не in natura,\footnote*{
— продуктами. \emph{Ред.}
} а в грошовій
формі. За ці шпалери, перетворені на 1.500 фунтів стерлінґів,
купували вони собі засоби існування на таку саму суму. Тому ці
останні існували для них не як капітал, а як товари, і вони сами
існували для цих товарів не як наймані робітники, а як покупці.
Та обставина, що машина «звільнила» їх від купівельних засобів,
перетворює їх з покупців на непокупців. Звідси зменшений
попит на ці товари. Voilà tout.\footnote*{
Оце й усе. \emph{Ред.}
} Якщо цей зменшений попит не
компенсується збільшеним попитом з іншого боку, то ринкова
ціна цих товарів меншає. Якщо це триває довго та у великому
розмірі, то постає переміщення робітників, уживаних у продукції
цих товарів. Частину капіталу, що раніше продукувала доконечні
засоби існування, репродукується в іншій формі.\footnote*{
У французькому виданні замість останніх двох речень читаємо
таке: «Але, може, це спричиниться до того, що капітал, якого уживалося
в продукції цих засобів існування, покличе до роботи як додаткових
робітників наших звільнених шпалерників? Цілком навпаки: якщо це
зниження цін триватиме деякий час, то почнуть знижувати заробітну плату
робітників, уживаних у продукції цих засобів існування. Якщо дефіцит
у збуті доконечних засобів існування триватиме довгий час, то частина
капіталу, вживана в продукції їх, відпливе звідси й шукатиме собі іншої
сфери вміщення». («Le Capital etc.», v. I, ch. XV. p. 190). \emph{Ред.}
} Підчас спадання
\index{i}{0366}  %% посилання на сторінку оригінального видання
ринкових цін та переміщення капіталу робітники, уживані
у продукції доконечних засобів існування, також «звільняються»
від якоїсь частини їхньої заробітної плати. Отже, замість
довести, що машини, звільняючи робітників од засобів існування,
одночасно перетворюють ці останні на капітал, щоб уживати
перших, пан апологет із своїм випробуваним законом попиту й
подання доводить, навпаки, що машини не тільки в тій галузі
продукції, де їх заводять, але й у тих галузях продукції, де їх
не заведено, викидають робітників на брук.

Дійсні факти, перекручені економічним оптимізмом, такі.
Витиснутих машинами робітників викидають із майстерні на
ринок праці, і вони збільшують там число робочих сил, що ними
можна порядкувати для капіталістичної експлуатації. В сьомому
відділі ми побачимо, що цей вплив машин, який нам тут змальовано
як компенсацію для робітничої кляси, спадає, навпаки, як
найстрашніша кара на робітника. Тут зауважимо лише ось
що: робітники, викинуті з однієї галузі промисловости, можуть,
щоправда, шукати заняття в якійсь іншій. Якщо вони знаходять
собі заняття, і таким чином відновлюється зв’язок між ними й
засобами існування, які були звільнені разом з ними, то це
стається за допомогою нового додаткового капіталу, що шукає
вміщення, а зовсім не того капіталу, що вже раніш функціонував
і тепер перетворений на машини. Але навіть і в такому випадку,
— які мізерні їхні перспективи! Скалічені через поділ праці,
ці бідолахи так мало чого варті поза своєю колишньою сферою
праці, що вони знаходять собі доступ лише до небагатьох нижчих,
і тому завжди переповнених та низько оплачуваних, галузей
праці.\footnote{
Один рікардіянець зауважує з приводу цього, заперечуючи проти
нісенітниць Ж. Б. Сея: «За розвиненого поділу праці вмілість робітників
може придатись тільки в тій окремій галузі, де вони навчилися її; вони
сами є своєрідні машини. Тим то абсолютно не поможе, коли верзти, як
папуга, що речі мають тенденцію знаходити свій рівень. Нам треба лише
поглянути навколо себе, і ми побачимо, що вони довгий час не можуть
знайти свого рівня, а якщо і знайдуть його, то цей рівень нижчий, ніж
він був на початку процесу». («An Inquiry into those Principles respecting
the Nature of Demand etc.», London 1821, p. 72).
} Далі, кожна галузь промисловости притягає щороку
новий потік людей, який дає їй континґент для регулярної заміни
та зросту.\footnote*{
У французькому виданні кінець цього речення подано так: «\dots{} який
дає їй континґент для заміни спрацьованої робочої сили та для того поповнення,
що його вимагає реґулярний розвиток цієї галузі». \emph{Ред.}
} Скоро тільки машини звільняють частину робітників,
що досі працювали в певній галузі промисловости, то й новий
потік промислових рекрутів перерозподіляється, і його вбирають
інші галузі праці, тимчасом як первісні жертви за переходовий
час здебільша занепадають і гинуть.

Безперечний факт, що машини сами по собі не винні у «звільненні»
робітників од засобів існування. Вони здешевлюють і
збільшують продукт у тій галузі, яку захоплюють, та спочатку
лишають ту масу засобів існування, що її продукується по інших
\parbreak{}  %% абзац продовжується на наступній сторінці

\parcont{}  %% абзац починається на попередній сторінці
\index{i}{0367}  %% посилання на сторінку оригінального видання
галузях продукції, незмінною. Отже, після заведення машин,
як і перед тим, суспільство має таку ж саму або більшу кількість
засобів існування для звільнених робітників, зовсім не кажучи
вже про ту величезну частину річного продукту, що її марно
тратять неробітники. І в цьому вся премудрість (pointe) економічної
апологетики! Суперечностей та антагонізмів, мовляв, невіддільних
від капіталістичного вживання машин, не існує, бо
вони виростають не з самих машин, а з капіталістичного вживання
їх! Отже, що машина, розглядувана сама по собі, скорочує
робочий час, а застосована капіталістично здовжує робочий день,
що сама по собі вона полегшує працю, а застосована капіталістично
підносить її інтенсивність, що сама по собі вона є перемога
людини над силою природи, а капіталістично застосована
поневолює людину силою природи, що сама по собі збільшує багатство
продуцента, а капіталістично застосована павперизує
його й т. ін., то буржуазний політико-економ просто заявляє,
що, як доводить розгляд машини самої по собі якнайяскравіше,
всі ті очевидні суперечності є проста видимість ницої дійсности,
а сами по собі, отже, і в теорії, вони зовсім не існують. Таким чином
він звільняє себе від клопоту далі ламати собі голову та,
крім того, накидає своєму супротивникові таку дурість, наче той
бореться не з капіталістичним застосуванням машин, а з самою
машиною.

Правда, буржуазний економіст не заперечує, що при цьому
бувають і тимчасові неприємності; але не буває медалі без зворотного
боку! Для нього неможливе якесь інше, а не капіталістичне
використовування машин. Отже, експлуатація робітника за допомогою
машин для нього ідентична з експлуатацією машини робітником.
Отже, той, хто викриває, яка в дійсності справа з капіталістичним
вживанням машин, той, мовляв, взагалі не хоче вживання
їх, той ворог соціяльного проґресу!\footnote{
Одним із віртуозів у цьому чванливому кретинізмі є Мак Куллох.
«Якщо корисно, — каже він, наприклад, з афектованою наївністю
восьмилітньої дитини, — щораз більше й більше розвивати вмілість робітника,
так щоб він став здатний продукувати щораз більшу кількість товарів
із тією самою або меншою кількістю праці, то не менш корисне мусить
бути й те, шоб він користувався з таких машин, які найуспішніше допомагали
б йому досягти цього результату». (Мас Culloch: «Principles
of Political Economy», London 1830, p. 182).
} Це цілком нагадує
аргументацію славного горлоріза Біл Сайкса: «Панове присяжні,
певна річ, цим комівояжерам горло перерізано. Але цей вчинок
не моя вина, це вина ножа. Невже ж задля таких тимчасових
неприємностей нам скасувати вживання ножа? Подумайте ж!
Що сталося б із рільництвом і ремеством без ножа? Хіба ж не
дає він порятунку в хірургії, хіба можна без нього вчитися анатомії?
Та ще чи не бажаний він помічник на веселих бенкетах?
Позбавте нас ножа — і ви відкинете нас назад до часів найглибшого
варварства».\footnoteA{
«Винахідник прядільної машини зруйнував Індію, що нас,
однак, мало обходить». (A. Thiers: «De la Propriété», Paris 1848). Пан
}
\index{i}{0368}  %% посилання на сторінку оригінального видання
Хоч машина неминуче й витискує робітників по тих галузях
промисловосте, де її заводять, все ж вона може викликати збільшення
праці по інших галузях праці. Але це діяння не має нічого
спільного з так званою теорією компенсації. А що кожний машиновий
продукт, наприклад, один метр машинової тканини, дешевший,
ніж витиснутий ним однорідний ручний продукт, то звідси випливає
такий абсолютний закон: якщо загальна кількість товарів,
випродукованих машиновим способом, лишається рівна загальній
кількості заміщуваних нею товарів, випродукованих ремісничим
або мануфактурним способом, то загальна сума вжитої праці
зменшується. Те збільшення праці, яке потрібне для продукції
самих засобів праці, машин, вугілля й т. д., мусить бути менше
від праці, заощадженої через уживання машин. У противному
разі машиновий продукт був би так само дорогий або й дорожчий,
ніж ручний продукт. Алеж фактично замість лишатися однаковою
вся маса машинового продукту, що його продукує зменшене
число робітників, зростає далеко понад загальну масу витиснутого
ремісничого продукту. Припустімо, що 400.000 метрів машинової
тканини продукується меншим числом робітників, ніж 100.000
метрів ручної тканини. У збільшеному вчетверо продукті міститься
вчетверо більше сировинного матеріялу. Отже, продукцію
сировинного матеріялу треба збільшити в чотири рази. Щождо
спожитих засобів праці, як от будівлі, вугілля, машини й т. д.,
то межі, в яких може зростати додаткова праця, потрібна на їхню
продукцію, змінюються відповідно до ріжниці поміж масою машинового
продукту й масою ручного продукту, яку може виготовити
те саме число робітників.

Отже, з поширенням машинового виробництва в якійсь одній
галузі промисловости більшає насамперед продукція в тих інших
галузях, які постачають їй її засоби продукції. Якою мірою
через те зростає маса занятих робітників, це залежить, за
даної довжини робочого дня й інтенсивности праці, від складу
вжитих капіталів, тобто від відношення між їхніми сталими та
змінними складовими частинами. Це відношення, з свого боку,
дуже варіює залежно від того обсягу, в якому машини вже захопили
або захоплюють ті галузі промисловости. Число робітників,
засуджених на працю по копальнях вугілля й металю, страшенно
зросло з розвитком англійської машинової системи, хоч це зростання
останніми десятиліттями уповільнюється в наслідок заведення
нових машин у гірництві.\footnote{
За переписом 1861 р. (Vol. II. London 1863) число робітників,
що працювали по копальнях вугілля Англії та Велзу, становило 246.613,
з них 73.545 молодші й 173.067 понад 20 років віку. До першої рубрики
належать 835 від п’яти до десяти років, 30.701 від десяти до п’ятнадцяти,
42.010 від п’ятнадцяти до дев’ятнадцяти років. Число заня-
} Разом з машиною увіходить
у життя новий рід робітника — її продуцент. Ми вже знаємо, що
машинове виробництво завойовує в чимраз більшому розмірі й

Тьєр переплутав тут прядільну машину з механічним ткацьким варстатом,
та «нас це, однак, мало обходить».
\index{i}{0369}  %% посилання на сторінку оригінального видання
цю галузь продукції.\footnote{
1861 р. в Англії та Велзі працювало коло продукції машин 60.807
осіб, залічуючи сюди і фабрикантів з їхніми прикажчиками й т. д. і всіх
аґентів та купців цієї галузі, але виключаючи продуцентів невеличких
машин, як от швацьких машин і т. ін., так само й продуцентів знарядь
до робочих машин, як от веретен і т. ін. Число всіх цивільних інженерів
становило 3.329,
} Далі щодо сировинного матеріялу,\footnote{
Що залізо — найважливіший сировинний матеріял, то треба тут
зауважити, що 1861 р. в Англії та Велзі був 125.771 робітник, які
працювали по залізоливарнях, з того 123.430 чоловіків, 2.341 жінка.
З-поміж перших 30.810 молодші і 92.620 понад 20 років.
} то
не підлягає ніякому сумнівові, наприклад, що бурхливий розвиток
бавовняного прядільництва з оранжерійною швидкістю прискорив
культуру бавовнику в Сполучених штатах, а разом з нею
він не тільки прискорив африканську торговлю рабами, але
одночасно зробив вирощування негрів головним промислом так
званих прикордонних рабовласницьких штатів. Коли 1790 р.
зроблено в Сполучених штатах перший перепис рабів, число їх
становило 697.000, а 1861 року — приблизно чотири мільйони.
З другого боку, не менш певне й те, що розцвіт механічної вовняної
фабрики разом з чимраз більшим перетворенням орного поля на
пасовиська для овець викликав масове вигнання сільських робітників
та перетворення їх на «зайвих». В Ірландії ще й тепер
відбувається цей процес, який її людність, що від 1845 р. встигла
вже зменшитися майже наполовину, ще далі зменшує до тієї
міри, яка точно відповідає потребам її лендлордів та англійських
панів фабрикантів вовни.

Якщо машини захоплюють попередні або проміжні ступені,
які предмет праці має перебігти раніш, ніж він набирає своєї
остаточної форми, то разом із матеріялом праці більшає й попит
на працю в проваджуваних ремісничим або мануфактурним способом
галузях промисловости, які обробляють машиновий
фабрикат. Наприклад, машинове прядіння постачало пряжу так
дешево й так багато, що ручні ткачі спочатку могли, не збільшуючи
видатків, працювати повний час. Таким чином їхній дохід
збільшився.\footnote{
«Родина з чотирьох дорослих осіб (бавовняних ткачів) з двома
дітьми, що працювали як winders, одержувала наприкінці останнього та
на початку цього століття 4 фунти стерлінґів на тиждень за десятигодинного
робочою дня: якщо робота мала нагальний характер, вони могли
заробити й більше... Раніш вони завжди страждали від недостатнього
подання пряжі». (Gaskell: «The Manufacturing Population of England»,
London 1833, p. 25—27).
} Звідси наплив людей до пряділень бавовни,
доки, нарешті, 800.000 бавовняних ткачів, що їх в Англії покликали
були до життя машини jenny, throstle та mule, були
вбиті паровими ткацькими варстатами. Так само разом з надміром
матерії на одяг, продукованої машиновим способом, зростало
число кравців, кравчих, швачок і т. д., поки не з’явилася
швацька машина.

тих на копальнях заліза, міді, олива, цини та всіх інших металів —
319.222.

\index{i}{0370}  %% посилання на сторінку оригінального видання 
Відповідно до збільшення маси сировинного матеріялу, півфабрикатів,
робочих інструментів і т. ін., що їх дає машинове
виробництво при порівняно меншому числі робітників,
оброблення цього сировинного матеріялу та півфабрикатів розподіляється
на численні підроди, отже, зростає різноманітність
галузей суспільної продукції. Машинове виробництво розвивав
суспільний поділ праці незрівнянно більше, ніж мануфактура,
бо воно в незрівнянно вищій мірі збільшує продуктивну силу
захоплених ним галузей промисловості.

Найближчий результат машинової системи в тому, що вона
збільшує додаткову вартість, а разом з тим масу продуктів, у
якій вона втілена, отже, з тією субстанцією, з якої живиться
кляса капіталістів та її похлібці, вона збільшує самі ці суспільні
верстви. Зріст багатства цих останніх та постійне відносне меншання
числа робітників, потрібних для продукції доконечних
засобів існування, породжують разом із новими потребами в розкошах
і нові засоби задоволення їх. Більша частина суспільного
продукту перетворюється на додатковий продукт, більша частина
додаткового продукту репродукується і споживається в щораз
витонченіших та різноманітніших формах. Інакше кажучи: продукція
предметів розкошів зростає.\footnote{
Ф. Енґельс у «Lage der arbeitenden Klasse in England» («Становище
робітничої кляси в Англії») відзначає злиденний стан великої частини
саме цих продуцентів предметів розкошів. Про численні нові докази
цього див. у звітах «Children’s Employmeni Commission».
} Витонченість та різноманітність
продуктів випливає так само з нових відносин світового
ринку, створюваних великою промисловістю. Не тільки
більшу кількість закордонних засобів споживання вимінюється
на тубільний продукт, але й більша маса чужих сировинних матеріялів,
складових елементів, півфабрикатів тощо ввіходить
як засіб продукції в тубільну промисловість. З розвитком цих
відносин світового ринку зростає попит на працю в транспортовій
промисловості, і остання розпадається на численні нові
підроди.\footnote{
1861 p. в Англії та Велзі в торговельній фльоті працювало 94.665
моряків.
}

Збільшення засобів продукції й засобів існування при відносному
меншанні числа робітників спонукає поширювати працю в
таких галузях промисловости, продукти яких, як от канали,
товарові доки, тунелі, мости й т. д., дають плоди у далекій будучині.
Безпосередньо на основі машинової системи або на основі
відповідного їй загального промислового перевороту, постають
цілком нові галузі продукції, а тому й нові сфери праці. Однак
участь їхня в загальній продукції навіть у найрозвиненіших
країнах незначна. Число вживаних ними робітників більшає
прямо пропорційно до того, наскільки репродукується доконечність
щонайгрубішої ручної праці. Головними галузями промисловости
цього роду в наші часи можна вважати газові заводи,
\parbreak{}  %% абзац продовжується на наступній сторінці

\parcont{}  %% абзац починається на попередній сторінці
\index{i}{0371}  %% посилання на сторінку оригінального видання
телеграфію, фотографію, пароплавство та залізниці. Перепис
1861 р. (для Англії та Велзу) подає для газової промисловости
(газові заводи, продукція механічних апаратів, аґенти газових
компаній і т. ін.) 15.211 осіб, для телеграфії — 2.399, для фотографії
— 2.366, для служби на пароплавах — 3.570 та для залізниць —
70.599, куди входять приблизно 28.000 більш-менш постійно
занятих «ненавчених» землекопів і цілий адміністративний і комерційний
персонал. Отже, загальне число індивідів у цих п’ятьох
нових галузях промисловости — 94.145.

Нарешті, надзвичайно підвищена продуктивна сила в сферах
великої промисловости, супроводжувана, як ми це спостерігаємо,
інтенсивним та екстенсивним збільшенням визиску робочої
сили по всіх інших сферах продукції, дає змогу непродуктивно
вживати щораз більшу й більшу частину робітничої кляси й таким
чином репродукувати щораз більшими масами стародавніх домашніх
рабів під назвою «кляси слуг», як от слуг, покоївок, льокаїв
і т. ін. За переписом 1861 р. вся людність Англії й Велзу налічувала
20.066.244 особи, з того 9.776.259 чоловіків та 10.289.965 жінок.
Якщо від цього відлічити всіх тих, що застарі або замолоді
для праці, всіх «непродуктивних» жінок, підлітків і дітей,
далі «ідеологічні» професії, як от урядовців, попів, юристів,
військових тощо, потім усіх тих, що їхнє виключне заняття є
споживання чужої праці в формі земельної ренти, процентів і
т. ін., насамкінець, павперів, волоцюг, злочинців і т. ін., то залишається
приблизно 8 мільйонів осіб обох статей та найрізнішого
віку, залічуючи сюди й усіх капіталістів, що так або інакше
функціонують у продукції, торговлі, фінансах тощо. З цих 8 мільйонів
припадає на:

Рільничих робітників (залічуючи сюди пастухів та
наймитів і наймичок, що живуть у фармерів).................1.908.261 осіб
Всіх, що працюють на бавовняних, вовняних, напіввовняних,
лляних, конопляних, шовкових
і джутових фабриках, на механічних в’язальнях
панчіх та коло фабрикації мережива............................... 642.607 \footnote{
З того чоловіків, старших від 13 років, лише 177.596.
}»
Всіх, що працюють по копальнях та руднях............................565.835»
Всіх, що працюють па металюрґійних заводах
(домни, вальцювальні тощо) та металевих мануфактурах усякого
роду............................................................................ 396.998 \footnote{
З того жінок 30.501.
}»
Клясу слуг.................................................................... 1.208.648 \footnote{
З того чоловіків 137.447. З цього числа в 1.208.648 виключено ввесь
персонал, що служить не у приватних осіб.

Додаток до другого видання. Від 1861 р. до 1870 р. число слуг-чоловіків
майже подвоїлося. Воно зросло до 267.671. 1847 р. сторожів дичини
було 2.694 (в аристократичних мисливських парках), а 1869 р. — 4.291.
— Молодих дівчат, що служать у лондонських дрібних буржуа, народньою
мовою називають «little slaveys» — маленькі рабині.\footnote*{
Тут у власному Марксовому примірнику 1 німецького видання є
така цитата з «Evening Star» від 11 вересня 1868 р.: «Як виснажують над-
}
}»

\index{i}{0372}  %% посилання на сторінку оригінального видання
Якщо додамо всіх тих, що працюють по текстильних фабриках,
до персоналу копалень та рудень, то матимемо 1.208.442;
якщо ж число перших додамо до персоналу всіх металюрґійних
заводів і мануфактур, то матимемо загальне число 1.039.605 —
в обох випадках менше, ніж число сучасних хатніх рабів. От
який величний результат капіталістичної експлуатації машин!

7. Відштовхування і притягування робітників із розвитком
машинового виробництва. Кризи в бавовняній промисловості

Всі серйозні представники політичної економії визнають,
що заведення в життя машин впливає неначе чума на робітників
у тих традиційних ремествах і мануфактурах, з якими машина
насамперед починає конкурувати. Майже всі вони бідкаються
над рабством фабричного робітника. Але який той великий козир,
що ним усі вони козиряють? Це те, що машини після всіх страхіть
періоду заведення їх у життя та розвитку їх, кінець-кінцем,
не зменшують, а збільшують число рабів праці! Так, політична
економія захоплюється огидною теоремою — огидною для всякого
«філантропа», що вірує у вічну природну доконечність капіталістичного
способу продукції, — теоремою, що навіть фабрика,
яка вже основана на машиновому виробництві, після певного
періоду зросту, після коротшого або довшого «переходового
часу» починає мучити більше число робітників, ніж те, яке вона
первісно викинула на брук!\footnote{
Ґаніль, навпаки, вважає за остаточний результат машинового виробництва
абсолютне зменшення числа рабів праці, що їхнім коштом годується
потім збільшене число «gens honnêtes»,\footnote*{
— порядних людей. \emph{Ред.}
} що розвивають свою відому
«perfectibilité perfectible»,\footnote*{
— здібну вдосконалюватися здібність до вдосконалення. \emph{Ред.}
} [що її так надхненно висміяв Фур’є].\footnote*{
Заведений у прямі дужки кінець речення беремо з французького
видання. \emph{Ред.}
} Хоч
}

Правда, з деяких прикладів, як от на англійських фабриках

мірною працею молодих служниць, це — ганьба для їхніх господинь.
Випадково я знайомий з багатьма з цих «рабинь», як їх дехто називає,
і співчуваю їм від усього серця. Вони мусять рано вставати та працювати
до самісінької ночі. Вони сплять у підвальних комірках із нечистю або
по горищах із пацюками. Вони харчуються покидьками. Їх лають і шельмують,
їх переслідують брутальні хазяйські сини, їх мучать 4 або 5 дітей;
під дощ їх ганяють по пиво, інколи їх б’ють розгнівані господині. Тижнями
їм не дозволяють піти до церкви. їм платять дуже мало; якщо вони
захоріють, їх відсилають до їхніх родичів, коли в них є родичі, або ж до
шпиталю, або до притулку для бідних. Не диво, що вони мають острах
і огиду до пристойної праці і готові «піти світ за очі, хоч к чорту», і це
вони, ці бідолашні маленькі рабині, залюбки й роблять. Я бачив, як вони
плакали, оповідаючи про свої страждання, побої, голод і холод, про те,
як їх прогнали з їхнього «місця», коли вони захоріли, як жили вони
тоді з продажу свого одягу, і як, нарешті, коли все було продано, вони
утопли в мерзоті, дедалі більше занепадаючи. На жаль, лише дехто їм
співчуває». \emph{Ред.}
\index{i}{0373}  %% посилання на сторінку оригінального видання
суканої шерсти та шовкових фабриках, виявилося, що на певному
ступені розвитку надзвичайне поширення фабричних галузей
може сполучатися не тільки з відносним, але й з абсолютним
зменшенням числа вживаних робітників. 1860 р., коли з наказу
парляменту розпочато спеціяльний перепис усіх фабрик Об’єднаного
Королівства, налічувалося 652 фабрики в тій частині
фабричних округ Ланкашіру, Чешіру та Йоркшіру, яку доручено
фабричному інспекторові Р. Бекерові; з цих фабрик 570 мали:
парових ткацьких варстатів — 85.622, веретен (за винятком веретен
на сукання) — 6.819.146, кінських сил у парових машинах —
27.439, у водяних колесах — 1.390, занятих осіб — 94.119.
Навпаки, 1865 р. на цих самих фабриках було: ткацьких варстатів
— 95.163, веретен — 7.025.031, кінських сил у парових машинах
— 28.925, у водяних колесах — 1.445, занятих осіб —
88.913. Отже, від 1860 р. до 1865 р. зріст цих фабрик становив
у парових ткацьких варстатах 11\%, у веретенах — 3\%, у парових
кінських силах — 5\%, тимчасом як число занятих осіб за той
самий період зменшилося на 5,5\%.\footnote{
«Reports of Insp. of Fact. for 31 st October 1865», p. 58 і далі.
Але одночасно було дано вже й матеріяльну базу для вживання чимраз
більшого числа робітників: було засновано 110 нових фабрик з 11.625 паровими
ткацькими варстатами, 628.756 веретенами, 2.695 паровими й водяними
кінськими силами (Там же).
} Між 1852 і 1862 рр. сталося
значне збільшення англійської вовняної фабрикації, тимчасом як
число вживаних робітників лишилося майже без змін. «Це показує,
в якій великій мірі новозаведені машини витиснули працю

і як мало він розуміє рух продукції, а все ж він принаймні почуває,
що машини дуже фатальна інституція, скоро заведення їх перетворює
занятих робітників на павперів, тимчасом як розвиток їх покликає до
життя більше рабів праці, ніж вони були вбили. Кретинізм його власного
погляду можна висловити лише його власними словами: «Les classes
condamnées à produire et à consommer diminuent, et les classes qui dirigent
le travail, qui soulagent, consolent et éclairent toute la population,
se multiplient... et s’approprient tous les bienfaits qui résultent de la diminution
des frais du travail, de l’abondance des productions et du bon marché
des consommations. Dans cette direction, l’espèce humaine s’élève aux plus
hautes conceptions du génie, pénètre dans les profondeurs mystérieuses de la
religion, établit les principes salutaires de la morale (яка є в тому, щоб «s’approprier
tous les bienfaits etc.»), les lois tutélaires de la liberté (liberté pour
«les classes condamnées à produire»?) et du pouvoir, de l’obéissance et de
la justice, du devoir et de l’humanité». («Кляси, засуджені на продукцію
та споживання, зменшуються, а кляси, що керують працею, дають поміч,
утіху та освіту цілому народові, збільшуються... та присвоюють собі
всі блага, що є результат зменшення витрат праці, рясности продуктів
та подешевшання предметів споживання. В цьому напрямі людський
рід підноситься до найвищих концепцій генія, доходить таємних глибин
релігії, встановлює спасенні принципи моралі (яка є в тому, щоб «присвоювати
собі всі блага й т. ін.»), закони до охорони волі (волі для «кляс,
засуджених на продукцію»?) та влади, покірливости та справедливости,
обов’язку та гуманности»). Ці теревені маємо в «Des Systèmes d’Economie
Politique etc. Par M. Ch. Ganilh». 2 ème ed. Paris 1821, vol. I, p. 224.
Порівн. там же, стор. 212.
\parbreak{}  %% абзац продовжується на наступній сторінці

\parcont{}  %% абзац починається на попередній сторінці
\index{i}{0374}  %% посилання на сторінку оригінального видання
попередніх періодів»\footnote{
«Reports etc. for 31 st October 1862», p. 79.

Додаток до другого видання. Наприкінці грудня 1871~\abbr{р.} фабричний
інспектор А.~Редґрев в одній доповіді, яку він прочитав у Бредфорді, в
«New Mechanics’ Institution», сказав: «Що мене від деякого часу почало
вражати, так це зміна вигляду вовняних фабрик. Раніш вони були переповнені
жінками й дітьми, тепер здається, що машина виконує всю працю.
На мій запит один фабрикант дав мені таке пояснення: за старої системи в
мене працювало 63 особи; після заведення поліпшених машин я зменшив
число своїх рук до 33, а недавно, в наслідок нових великих змін, я зміг
зменшити їх з 33 до 13».
}. У деяких випадках збільшення числа
занятих робітників часто є лише позірне, тобто воно завдячує не
поширенню фабрик, що вже ґрунтуються на машиновому виробництві,
а поступінному прилучуванню до них побічних галузей.
Наприклад, «збільшення числа механічних ткацьких варстатів
і фабричних робітників, занятих коло них, від 1838 до 1858~\abbr{рр.}
в (брітанській) бавовняній фабриці завдячує просто поширенню
цієї галузі промисловости; навпаки, по інших фабриках воно
завдячує тому, що там до варстатів на ткання килимів, стрічок,
полотна тощо, які до того часу пускали у рух силою людських
мускулів, почали прикладати парову силу»\footnote{
«Reports etc. for 31 st October 1856», p. 16.
}. Отже, збільшення
числа цих фабричних робітників було просто виразом зменшення
загального числа вживаних робітників. Нарешті, ми тут цілком
залишаємо осторонь те, що скрізь, за винятком металевих фабрик,
підлітки (молодші за 18 років), жінки та діти становлять геть
переважну частину фабричного персоналу.

\looseness=-1
Однак зрозуміло, що, незважаючи на масу робітників, яких фактично
витискує та в можливості заступає машинове виробництво,
число фабричних робітників із зростанням самого машинового виробництва,
вираженим у збільшеному числі фабрик того самого
роду або в поширеному розмірі наявних фабрик, кінець-кінцем,
може бути більше за число витиснутих ними мануфактурних робітників
та ремісників. Припустімо, що тижнево вживаний капітал
у 500\pound{ фунтів стерлінґів} складається, приміром, за старого способу
продукції з \sfrac{2}{5} сталої та \sfrac{3}{5} змінної складової частини, тобто 200\pound{ фунтів стерлінґів} витрачається на засоби продукції, 300\pound{ фунтів
стерлінґів} — на робочу силу, скажімо, по 1\pound{ фунту стерлінґів}
на робітника. Зі заведенням машинового виробництва склад цілого
капіталу змінюється. Він розпадається тепер, наприклад, на \sfrac{4}{5}
сталої та \sfrac{1}{5} змінної складових частин, або на робочу силу витрачається
лише 100\pound{ фунтів стерлінґів}. Отже, дві третини вживаних
раніш робітників звільняється. Якщо це фабричне виробництво
поширюється і цілий ужитий капітал за інших незмінних умов
продукції зростає з 500 до \num{1.500}, дотепер уживатиметься 300 робітників
— стільки ж, скільки й перед промисловою революцією.
Якщо вжитий капітал зростає й далі до \num{2.000}, то вживатиметься
400 робітників, отже, на \sfrac{1}{3} більше, ніж за старого способу виробництва.
Число вжитих робітників абсолютно збільшилося на 100,
а відносно, тобто проти цілого авансованого капіталу, впало на 800,
\parbreak{}  %% абзац продовжується на наступній сторінці

\parcont{}  %% абзац починається на попередній сторінці
\index{i}{0375}  %% посилання на сторінку оригінального видання
бо капітал у \num{2.000}\pound{ фунтів стерлінґів} за старого способу виробництва
вживав би не 400, а \num{1.200} робітників. Отже, відносне зменшення
числа вживаних робітників узгоджується з його абсолютним
збільшенням. Вище ми припускали, що із зростом цілого
капіталу склад його лишається незмінний, бо й умови продукції
не змінюються. Але ми вже знаємо, що з кожним кроком розвитку
машинової системи стала частина капіталу, яка складається з
машин, сировинного матеріялу тощо, зростає, тим часом як
змінна частина капіталу, витрачена на робочу силу, падає; разом
з тим знаємо, що ні за якого іншого способу продукції не буває
такого постійного поліпшення машин, а тому й таких змін у
складі цілого капіталу. Але цю постійну зміну так само постійно
переривають періоди спокою та просте кількісне поширення на
даній технічній основі. Тому число вживаних робітників зростає.
Так, число робітників на бавовняних, вовняних, суканої вовни,
лляних та шовкових фабриках Об’єднаного Королівства 1835 р.
становило лише \num{354.684}, тимчасом як 1861 р. число самих ткачів
(обох статей та найрізнішого віку, від 8 років починаючи) при
парових варстатах становило \num{230.654}. Певна річ, цей зріст буде
менш значним, коли взяти на увагу, що ще 1838 р. брітанських
ручних ткачів бавовни разом з їхніми родинами, яким вони сами
давали заняття, налічувалося \num{800.000} чоловіка,\footnote{
«Страждання ручних ткачів (бавовни і тканини з домішкою
бавовни) були предметом дослідження королівської комісії, але хоч їхні
злидні були визнані й оплакані, все ж поліпшення (!) їхнього становища
віддали на волю випадкові та часові, і можна сподіватися, що ці страждання
тепер (через 20 років!) майже (nearly) зникли, чому, певно, допомогло
сучасне велике поширення парових ткацьких варстатів». («Reports
of lnsp. of Fact, for 31 st October 1856», p. 15).
} не кажучи вже
зовсім про тих ручних ткачів, що їх витиснуто в Азії та на континенті
Европи.

В тих небагатьох увагах, що треба ще зробити про цей пункт,
ми почасти торкаємося суто фактичних відносин, до яких самий
наш теоретичний виклад ще не довів нас.

Поки машинове виробництво поширюється в якійсь галузі промисловосте
коштом традиційного ремества або мануфактури, його
успіхи настільки ж певні, як, наприклад, успіх армії, озброєної
ґвинтівками, проти армії, озброєної луками. Цей перший період,
коли машина ще тільки завойовує собі сферу діяння, має вирішальне
значення через ті надзвичайно високі зиски, що їх вона
допомагає продукувати. Ці зиски не тільки сами по собі становлять
джерело прискореної акумуляції, але вони ще й притягають у
цю сприятливу сферу продукції велику частину суспільного додаткового
капіталу, що постійно знов утворюється й шукає нового
вміщення. Особливі користі того першого періоду бурі й натиску
постійно повторюються по тих галузях продукції, де
машини заводиться вперше. Але, скоро тільки фабрична система
досягає певної ширини існування та певного ступеня зрілости,
скоро тільки, особливо, її власна технічна основа, машини, продукуються
\index{i}{0376}  %% посилання на сторінку оригінального видання
знову ж таки за допомогою машин; скоро тільки відбувається
революція в добуванні вугілля й заліза, а також в обробленні
металів і в транспортовій справі; одне слово, скоро тільки
будуть створені загальні умови продукції, відповідні великій
індустрії, — з цього моменту цей спосіб виробництва набуває елястичности,
здатности до раптового стрибкуватого поширювання, що
не має інших меж, як тільки в сировинному матеріялі та в ринку
для збуту. Машини, з одного боку, безпосередньо сприяють збільшенню
сировинного матеріялу, як, наприклад, cotton gin збільшила
продукцію бавовни.\footnote{
Про інші методи, якими машини впливають на продукцію сировинного
матеріялу, буде згадано в третій книзі.
} З другого боку, дешевина машинового
продукту та переворот у засобах транспорту й комунікації є
знаряддя завоювати чужі ринки. Руйнуючи на цих ринках ремісничу
продукцію, машинове виробництво силоміць перетворює
їх на поля продукції свого сировинного матеріялу. Так, Східня
Індія була примушена продукувати для Великобрітанії бавовну,
вовну, коноплю, джут, індиґо тощо.\footnote{
\noindent{}Вивіз бавовни із Східньої Індії до Великобрітанії:

\noindent{}1846 р. — \num{34.540.143} фунти, 1860 р. — \num{204.141.168} фунтів, 1865 р. —
\num{445.947.600} фунтів.

\noindent{}Вивіз вовни із Східньої Індії до Великобрітанії:

\noindent{}1846 р. — \num{4.570.581} фунт, 1860 р. — \num{20.214.173} фунти, 1865 р. —
\num{20.679.111} фунтів.
} Постійне «перетворювання»
робітників у «зайвих» по країнах великої промисловости примушує
до штучної еміґрації й колонізації чужих країн, які перетворюються
на місця продукції сировинного матеріялу для метрополії,
як, наприклад, Австралія перетворилася в місце продукції
вовни.\footnote{
\noindent{}Вивіз вовни з рогу Доброї Надії до Великобрітанії:

\noindent{}1846 р. — \num{2.958.457} фунтів, 1860 р. — \num{16.574.345} фунтів, 1865 р. —
\num{29.220.623} фунти.

\noindent{}Вивіз вовни з Австралії до Великобрітанії:

\noindent{}1846 р. — \num{21.789.346} фунтів, 1860 р. — \num{59.166.616} фунтів, 1865 р. —
\num{109.734.261} фунт.
} Створюється новий, відповідний до розташування головних
центрів машинового виробництва, міжнародній поділ праці,
який перетворює одну частину земної кулі переважно на поле
рільничої продукції [для другої частини земної кулі, яка стає
переважно полем промислової продукції].\footnote*{
Заведене у прямі дужки ми беремо з другого німецького видання.
\emph{Ред.}
} Ця революція стоїть
у зв’язку з переворотами в рільництві, що їх ми тут ще не розглядаємо
докладніше.\footnote{
Самий економічний розвиток Сполучених штатів є продукт европейської,
особливо англійської, великої промисловости. Сполучені штати
в їхньому теперішньому вигляді (1866 р.) все ще треба розглядати як
колонію Европи. [До четвертого видання. Від того часу вони
розвинулись у другу промислову країну світу, не втративши при цьому цілком;
свого колоніального характеру. — \emph{Ф.~Е.}].

\begin{center}
    \captionnew{Вивіз бавовни із Сполучених штатів до Великобрітанії (в фунтах):}

    \begin{tabular}{lrlr}
    1846 р. \dotfill{} & \num{401.949.393} & 1852 р.\dotfill{} &  \num{765.630.544} \\
    1859 р. \dotfill{} & \num{961.707.264} & 1860 р.\dotfill{} & \num{1.115.890.608} \\
    \end{tabular}
\end{center}


\begin{center}
    \captionnew{Вивіз збіжжя тощо із Сполучених штатів до Великобрітанії \\ (1850--1862 рр.) (в центнерах):}

    \begin{tabular}{lrr}
     & \makecell{1850 р.} & \makecell{1862 р.} \\
     \addlinespace
     Пшениця\dotfill  & \num{16.202.312} & \num{41.033.503} \\
    Ячмінь\dotfill & \num{3.669.653} &   \num{6.624.800} \\
    Овес\dotfill & \num{3.174.801}  &  \num{4.426.994}\\
    Жито \dotfill & \num{388.749} & \num{7.108}\\
    Пшеничне борошно\dotfill & \num{3.819.440} & \num{7.207.113}\\
    Гречка\dotfill & \num{1.054} & \num{19.571}\\
    Кукурудза\dotfill & \num{5.473.161} & \num{11.694.818}\\
    Веrе або Bigg (особливий рід ячменю)\dotfill & \num{2.039} & \num{7.675}\\
    Горох\dotfill & \num{811.620} & \num{1.024.722}\\
    Квасоля\dotfill & \num{1.822.972} & \num{2.037}.37\\
    \cmidrule{2-3}
    Увесь довіз\dotfill & \num{34.365.801} & \num{74.083.351}\\
    \end{tabular}
\end{center}
}

З ініціятиви пана Ґледстона Палата громад 17 лютого 1867 р.
наказала зібрати статистичні відомості про вивіз та довіз в
Об’єднане Королівство всякого роду збіжжя й борошна за час
від 1831 до 1866 р. Нижче я подаю зведення цих статистичних
\index{i}{0377}  %% посилання на сторінку оригінального видання
відомостей. Борошно перечислено на квартери збіжжя (Див.
таблицю на стор. \pageref{original-378}).

\begin{sidewaystable}
  \index{i}{0378}  %% посилання на сторінку оригінального видання
  \label{original-378}
  \centering
  \small
  \caption*{П'ятирічні періоди й 1866 рік}
  \begin{tabularx}{\textheight}{Xrrrrrrrr}
    \toprule
    
     & \makecell{1831\textendash{}1835} & \makecell{1836\textendash{}1840} & \makecell{1841\textendash{}1845}
     & \makecell{1846\textendash{}1850} & \makecell{1851\textendash{}1855} & \makecell{1856\textendash{}1860} 
     & \makecell{1861\textendash{}1865} & \makecell{1866} \\
    
    \midrule

    \addlinespace
    \makecell{Пересічно за рік} \\
    Імпорт (квартери)\dotfill{} &  \num{1.096.373} & \num{2.389.729} & \num{2.843.865} & \num{8.776.552}  & \num{8.345.237} & \num{10.913.612} & \num{15.009.871} & \num{16.457.340} \\
    
    \addlinespace
    \makecell{Пересічно за рік} \\
    
    Експорт (квартери)\dotfill{}  &   \num{225.363}  &   \num{251.770}  &    \num{139.056} &    \num{155.461}  &    \num{307.491} &     \num{341.150}  &    \num{302.754}  &  \num{216.218} \\
    
    \makehangcell{Перевага імпорту над експортом пересічно за рік\dotfill{}}
        & \num{871.010} & \num{2.137.959} & \num{2.704.809} & \num{8.621.091}  & \num{8.037.746}  & \num{10.572.462}  & \num{14.707.117}  & \num{16.241.122} \\
    
    \addlinespace
    \makecell{Людність:} \\

    \makehangcell{Пересічне число на рік у кожному періоді\dotfill{}}
        & \num{24.621.107} & \num{25.929.507} &
        \num{27.262.569} & \num{27.797.598} & \num{27.572.923} & \num{28.391.544} & \num{29.381.460} & \num{29.935.404} \\

    \makehangcell{Пересічна кількість збіжжя тощо в квартерах, що її 
        споживає за рік один індивід, при рівному розподілі між людністю,
        із надлишку проти тубільної продукції\dotfill{}}
        & 0,036 & 0,082 & 0,099 & 0,310  & 0,291  & 0,372  & 0,543  & 0,543 \\
  \end{tabularx}
\end{sidewaystable}

Величезна, стрибкувата розширність фабричної системи та
її залежність від світового ринку неминуче породжують гарячкову
продукцію і наступне переповнення ринків, із звуженням
яких настає параліч. Життя промисловости перетворюється на
послідовний ряд періодів середнього пожвавлення, розцвіту,
перепродукції, кризи й застою. Непевність і непостійність, що
їх зазнає праця і разом з нею й доля робітника через машинове
виробництво, стають нормальними з цією зміною періодів промислового
циклу. За винятком часів розцвіту, між капіталістами лютує
якнайзавзятіша боротьба за їхнє індивідуальне місце на ринку.
Це їхнє місце на ринку стоїть у прямому відношенні до дешевини
продукту. Крім створеного цим суперництва щодо вживання
поліпшених машин, які замінюють робочу силу, та нових метод
продукції, кожного разу настає такий момент, коли капіталісти
намагаються здешевити товари, силоміць понижуючи заробітну
плату нижче вартости робочої сили.\footnote{
У відозві робітників, викинутих на брук льокавтом фабрикантів
чобіт із Лестеру, до «Trade-Societies of England», липень 1866 p., між
іншим, сказано: «Ось уже років із 20 тому, як у чоботарстві Лестеру
відбувся переворот: замість зшивати почали скріпляти гвіздками. Тоді
можна було мати добру заробітну плату. Незабаром ця нова галузь промисловости
дуже поширилась. Велика конкуренція почалася між різними
фірмами, кожна з них намагалась подати найелегантніший товар. Алеж
незабаром виникла гірша конкуренція, а саме — фірми намагались побороти
одна одну на ринку нижчою ціною (undersell). Шкідливі наслідки виявилися
незабаром у пониженні заробітної плати, і ціна на працю спадала
так дуже швидко, що багато фірм платить тепер лише половину первісної
заробітної плати. А все ж, хоч заробітна плата падає нижче й нижче,
зиски з кожною зміною тарифу праці, здається, зростають». — Навіть
несприятливі періоди промисловости фабриканти використовують на те,
щоб через надмірне пониження заробітної плати, тобто безпосередньою
крадіжкою найдоконечніших засобів існування робітника, здобувати
надзвичайні зиски. Ось приклад. Мова йде про кризу шовкоткацтва в
Coventry. «Із свідчень, які я дістав так від фабрикантів, як і від робітників,
безперечно виходить, що заробітна плата понижена в більшому розмірі,
ніж цього вимагала конкуренція чужоземних продуцентів або інші обставини.
Більшість ткачів працює за заробітну плату, знижену на 30--40\%.
Моток стьожки, за який ткач перед п’ятьма роками діставав 6 або 7 шилінґів,
дає йому тепер лише 4 шилінґи 3 пенси або 3 шилінґи 6 пенсів;
за іншу працю, за яку раніш платили 4 шилінґи або 4 шилінґи 3 пенси,
він дістає тепер тільки 2 шилінґи або 2 шилінґи 3 пенси. Заробітну плату
понижено тепер більш, ніж це потрібно було для активізації попиту.
Справді, для багатьох сортів стьожок пониження заробітної плати не
супроводилося навіть якимось пониженням ціни товару». (Звіт комісара
F. D. Longe в «Children’s Employment Commission. 5 th Report 1866»,
p. 114, n. 1).
}


\index{i}{0379}  %% посилання на сторінку оригінального видання
Отже, зростання числа фабричних робітників зумовлено пропорційно
куди швидшим зростанням цілого капіталу, вкладеного
у фабрику. Але цей процес відбувається лише в межах періодів
припливу та відпливу промислового циклу. До того його завжди
перериває технічний прогрес, який то потенціяльно заступає робітника,
то витискує його фактично. Ця якісна зміна в машиновому
виробництві постійно викидає робітників із фабрики або замикав
фабричну браму перед новим потоком рекрутів, тимчасом як
просте кількісне поширення фабрик поглинає разом із викинутими
й свіжі контингенти. Таким чином робітників постійно відштовхують
або притягають, кидають ними туди й сюди, і це
супроводиться постійними змінами щодо статі, віку та вправности
завербованих.

Доля фабричного робітника унаочнюється найкраще, коли
подати короткий огляд долі англійської бавовняної промисловости.

Від 1770 до 1815 рр. бавовняна промисловість п’ять років перегнивала
період пригнічення або застою. Протягом цього першого
45-річного періоду англійські фабриканти мали монополію на
машини та світовий ринок. 1815--1821 рр. — час пригнічення;
1822--1823 рр. — розцвіт; 1824 р. — скасування закону про коаліції,
загальне велике поширення фабрик; 1825 р. — криза;
1826 р. — великі злидні та повстання серед бавовняних робітників;
1827 р. — легке поліпшення; 1828 р. — велике зростання
числа парових ткацьких варстатів і вивозу; 1829 р. — вивіз,
особливо до Індії, перевищує всі попередні роки; 1830 р. —
переповнені ринки, великі злидні; 1831--1833 рр. — тривале
пригнічення; у східньоіндійської компанії відібрано монополію
торговлі із Східньою Азією (Індією та Китаєм); 1834 р. — великий
зріст фабрик та машин, недостача рук; новий закон про бідних
активізує еміграцію сільських робітників до фабричних округ;
очищення сільських графств от дітей, торговля білими рабами.
1835 р. — великий розцвіт, але одночасно ручні бавовняні ткачі
\parbreak{}  %% абзац продовжується на наступній сторінці

\parcont{}  %% абзац починається на попередній сторінці
\index{i}{0380}  %% посилання на сторінку оригінального видання
вимирають від голоду; 1836~\abbr{р.} — великий розцвіт; 1837--1838~\abbr{рр.} —
пригнічений стан і криза; 1839~\abbr{р.} — знову пожвавлення; 1840~\abbr{р.} —
велика депресія, повстання, втручання війська; 1841--1842~\abbr{рр.} —
страшне страждання фабричних робітників; 1842~\abbr{р.} — фабриканти
викидають робітників із фабрик, щоб вимусити скасування збіжжевих
законів. Робітники багатьма тисячами напливають до
Йоркширу, звідти військо проганяє їх назад, їхніх проводирів
віддають під суд у Ланкастері. 1843~\abbr{р.} — великі злидні: 1844~\abbr{р.} —
знову пожвавлення; 1845~\abbr{р.} — великий розцвіт. 1846~\abbr{р.} — спершу
триває піднесення, потім симптоми реакції; скасування збіжжевих
законів; 1847~\abbr{р.} — криза; загальне пониження заробітної плати
на 10 і більше процентів на славу «big loaf» (коровай величезного
розміру, що його пани фритредери обіцяли робітникам підчас
агітації проти збіжжевих законів);\footnote*{
Цього пояснення вислову «big loaf» немає в німецькому тексті.
Ми беремо його з французького видання, де його подано в дужках у самому
тексті. \emph{Ред.}
} 1848~\abbr{р.} — пригнічення триває; Менчестер під військовою охороною; 1849~\abbr{р.} —
знову пожвавлення. 1850~\abbr{р.} — розцвіт. 1851~\abbr{р.} — спад товарових цін, низька
заробітна плата, часті страйки; 1852~\abbr{р.} — починається поліпшення,
страйки тривають далі, фабриканти загрожують довезти чужоземних
робітників. 1853~\abbr{р.} — зріст вивозу; восьмимісячний страйк
і великі злидні в Престоні. 1854~\abbr{р.} — розцвіт, переповнення ринків.
1855~\abbr{р.} — із Сполучених штатів, Канади, із східньоазійських
ринків надходять звістки про банкрутства; 1856~\abbr{р.} — великий
розцвіт; 1857~\abbr{р.} — криза; 1858~\abbr{р.} — поліпшення; 1859~\abbr{р.} — великий
розцвіт, зріст числа фабрик; 1860~\abbr{р.} — зеніт англійської
бавовняної промисловости; індійські, австралійські й інші ринки
так переповнено, що ще 1863~\abbr{р.} вони ледве поглинули всю заваль;
торговельний договір з Францією; величезний зріст числа фабрик
та машин; 1861~\abbr{р.} — піднесення триває якийсь час далі, реакція,
американська громадянська війна, недостача бавовни; 1862 до
1863~\abbr{р.} — повний крах.

Історія бавовняного голоду надто характеристична, щоб не
спинитись на ній хоч на одну хвилину. З наведених даних про
становище світового ринку за 1860--1861~\abbr{рр.} ми бачимо, що
бавовняний голод прийшовся фабрикантам до речі та почасти був
для них корисний — факт, визнаний у звітах менчестерської
торговельної палати, проголошений у парламенті Палмерстоном
і Дербі, потверджений подіями.\footnote{
Порівн. «Reports of Insp. of Fact, for 31 st October 1862», p. 30.
} Певна річ, 1861~\abbr{р.}
поміж \num{2.887} бавовняними фабриками Об’єднаного Королівства
багато було дрібних фабрик. За звітами фабричного інспектора
А. Редґрева, що його округа включає з тих \num{2.887} фабрик \num{2.109} фабрик,
392 фабрики з цих останніх, або 19\%, вживали кожна
менш від 10 парових кінських сил, 345 фабрик, або 16\%, вживали
від 10 до 20 сил, а \num{1.372} фабрики — 20 і більше кінських сил.\footnote{
Там же, стор. 19.
}
\parbreak{}  %% абзац продовжується на наступній сторінці

\parcont{}  %% абзац починається на попередній сторінці
\index{i}{0381}  %% посилання на сторінку оригінального видання
Більшість дрібних фабрик були ткальні, засновані підчас розцвіту
після 1858~\abbr{р.} здебільша спекулянтами, з яких один постачав
пряжу, другий — машини, третій — будівлі; заправляли колишні
overlookers\footnote*{
фабричні наглядачі. \emph{Ред.}
} та інші немаєтні люди. Ці дрібні фабриканти здебільша
позагибали. Таку саму долю була б заготовила їм торговельна
криза, до якої не припустив голод на бавовну. Хоч вони
й становили \sfrac{1}{3} числа фабрикантів, однак їхні фабрики увібрали
незрівнянно меншу частину капіталу, вкладеного в бавовняну промисловість.
Щодо розмірів кризи, то, на основі автентичного
цінування, в жовтні 1862~\abbr{р.} стояло без ніякої роботи 60,3\%
веретен та 58\% ткацьких варстатів. Ці числа стосуються до всієї
цієї галузі промисловости і, певна річ, дуже варіюють в окремих
округах. Лише дуже небагато фабрик працювало повний час (60 годин на тиждень),
решта працювала з перервами. Навіть
для тих небагатьох робітників, які працювали повний час
та за звичну відштучну плату, тижневий заробіток неминуче зменшувався
через заміну ліпшого сорту бавовни на гірший, бавовни
Sea Island на єгипетську (в тонкопрядільнях), американської та
єгипетської — на суратську (східньоіндійську), чистої бавовни —
на суміш відпадків бавовни й сурату. Коротші волокна бавовни-сурату,
її забрудненість, більша ламкість ниток, заміна борошна
при шліхтуванні основи пряжі на всякого роду важкі інґредієнти
тощо — все це зменшувало швидкість машин або число ткацьких
варстатів, за якими міг наглядати один ткач, збільшувало працю
в наслідок хибної роботи машин та зменшувало разом із масою
продукту відштучну заробітну плату. При вживанні сурату та
при повночасній праці втрата робітника доходила до 20, 30 і
більше процентів. Але більшість фабрикантів і норму відштучної
плати знизили на 5, 7\sfrac{1}{2} і 10 процентів. Тому можна зрозуміти
становище тих, що працювали лише 3, 3\sfrac{1}{2} і 4 дні на тиждень,
або тільки по 6 годин денно. 1863~\abbr{р.}, після того, як настало вже
відносне поліпшення, тижнева заробітна плата ткачів, прядунів
тощо становила 3\shil{ шилінґи} 4\pens{ пенси}, 3\shil{ шилінґи} 10\pens{ пенсів}, 4\shil{ шилінґи}
6\pens{ пенсів}, 5\shil{ шилінґів} 1\pens{ пенс} і~\abbr{т. ін.}\footnote{
«Reports of Insp. of Fact, for 31 st October 1865», p. 41--45.
} Навіть при такому злиденному
стані винахідницький дух фабрикантів щодо відраховань
від заробітної плати не завмирав. Почасти це були кари за вади
в продукті в наслідок поганої бавовни, поганих машин і~\abbr{т. ін.}
А там, де фабрикант був власником котеджів робітників, він сам
собі платив квартирну плату, відлічуючи її від номінальної
заробітної плати. Фабричний інспектор А.~Редґрев оповідає
про selfacting minders (вони наглядають за двома автоматичними
мюлями), які «за чотирнадцятиденну повну працю заробляли
8\shil{ шилінґів} 11\pens{ пенсів}; із тієї суми в них відраховували плату за
помешкання, при чому, однак, фабрикант половину цієї суми
повертав їм як подарунок, так що minders’и приносили додому
\parbreak{}  %% абзац продовжується на наступній сторінці

\parcont{}  %% абзац починається на попередній сторінці
\index{i}{0382}  %% посилання на сторінку оригінального видання
цілих 6\shil{ шилінґів} 11\pens{ пенсів.} Тижнева плата ткачів наприкінці
1862~\abbr{р.} починалась з 2\shil{ шилінґів} 6\pens{ пенсів}».\footnote{
«Reports of Insp. of Fact, for 31 st October 1863», p. 41, 42.
} Плату за помешкання
часто відраховували від заробітної плати навіть і тоді, коли
руки працювали лише короткий час.\footnote{
Там же, стор. 51.
} Не диво, що в деяких
частинах Ланкашіру почалось щось ніби голодна чума! Але найхарактеристичніше
було те, що революціонізування процесу продукції
відбувалося коштом робітника. Це були справжні experimenta
in corpora vili,\footnote*{
— експерименти на нічого не вартих тілах. \emph{Ред.}
} як от експерименти анатомів на жабах.
«Хоч я, — каже фабричний інспектор Редґрев, — подав дійсні
доходи робітників по багатьох фабриках, але з цього не можна
зробити такого висновку, ніби вони тиждень-у-тиждень дістають
ту саму суму. Доходи робітників зазнають якнайбільших
коливань через постійне експериментування («experimentalizing»)
з боку фабрикантів\dots{} їхні заробітки зростають або падають залежно
від якости бавовняної сумішки; вони то наближаються
до їхніх попередніх заробітків, відхиляючися від них лише на
15\%, то в найближчий або другий тиждень падають на 50 —
60\%».\footnote{
Там же, стор. 50, 51.
} Ці експерименти роблено не тільки коштом засобів
існування робітників. Робітники мусили поплатитись усіма своїми
п’ятьма почуттями. «Ті робітники, що працювали коло відкривання
паків бавовни, оповідали мені, що нестерпний сморід
доводить їх до непритомности. Тим робітникам, що їх уживають
до праці по відділах мішання та чухрання бавовни, порох та бруд,
що вилітають із бавовни, подразнюють дишні шляхи, викликають
кашель та стиснене дихання\dots{} Через те, що волокна короткі,
при шліхтуванні додають до пряжі багато всякого матеріялу, а
саме всяких суроґатів замість борошна, що його вживали раніш.
Звідси млості та диспепсія в ткачів. Через порох панує бронхіт,
а так само запал горла, далі шкіряні недуги через подразнення
шкури брудом, що є в сураті». З другого боку, суроґати борошна
були для панів фабрикантів за джерело наживи, бо збільшували
вагу бавовни. Ці суроґати давали те, що «15 фунтів перепряденого
сировинного матеріялу важили 26 фунтів».\footnote{
Там же, стор. 62, 63.
} У звіті фабричних
інспекторів з 30 квітня 1864~\abbr{р.} ми читаємо: «Промисловість
користується тепер цим допоміжним джерелом у справді таки
непристойній мірі. Від дуже авторитетної особи я знаю, що восьмифунтову
тканину виготовляють із 5\sfrac{1}{4} фунтів бавовни та 2\sfrac{3}{4} фунтів
шліхти. В іншій 5\sfrac{1}{4}-фунтовій тканині було 2 фунти шліхти.
Це були звичайні шертинґи для вивозу. До інших сортів іноді
додають 50\% шліхти, так що фабриканти можуть вихвалятись
та дійсно вихваляються, що вони багатіють, «продаючи тканини
дешевше, ніж номінально коштує вміщена в них пряжа».\footnote{
«Reports etc. for 30 th April 1864», p. 27.
}
\parbreak{}  %% абзац продовжується на наступній сторінці

\input{i/_0383.tex}
\parcont{}  %% абзац починається на попередній сторінці 
\index{i}{0384}  %% посилання на сторінку оригінального видання 
8. Революціонізування мануфактури, ремества та домашньої
праці великою промисловістю

а) Знищення кооперації, основаної
на реместві й поділі праці

Ми бачили, як машини знищують кооперацію, основану на
реместві, та мануфактуру, основану на поділі ремісничої
праці. Прикладом першого роду є жатка — вона заступає кооперацію
женців. Разючим прикладом другого роду є машина до
фабрикації швацьких голок. За Адамом Смісом 10 чоловіка за
його часів у наслідок поділу праці виготовляли 48.000 голок на
день. А одним-одна машина виготовляє 145.000 голок за 11-годинний
робочий день. Одна жінка або одна дівчина наглядає
пересічно за чотирма такими машинами, і тому продукує тими
машинами до 600.000 голок денно, а за тиждень понад 3.000.000.\footnote{
«Children’s Employment Commission. 4 th Report. 1864», p. Ю8,
n. 447.
}
Оскільки на місце кооперації або мануфактури стає поодинока
робоча машина, вона сама знову може стати основою ремісничого
виробництва. Однак, це відновлення ремісничого виробництва,
яке ґрунтується на машинах, становить лише перехід до фабричного
виробництва, яке звичайно з’являється кожного разу, коли
механічна рушійна сила, пара або вода замінює людські мускули,
що рухали машину. Спорадично і лише переходово дрібне виробництво
може сполучатися з механічною рушійною силою за
допомогою наймання пари, як ми це бачимо по деяких мануфактурах
Бермінґему, або за допомогою вжитку невеличких кальорійних
машин, як от у деяких галузях ткацтва тощо.\footnote{
У Сполучених штатах таке відновлення ремества на машиновій
базі трапляється часто. Саме через це концентрація, за неминучого переходу
до фабричного виробництва, йтиме там семимилевими кроками порівняно
з Европою та навіть з Англією.
} В шовкоткацтві
в Ковентрі стихійно розвинувся експеримент з «cottage-фабриками».\footnote*{
— котеджами-фабриками. Ред.
}
В центрі побудованих у формі квадрату рядів котеджів
будується так званий engine house\footnote*{
— машиновий будинок. Ред.
} для парової машини,
яку валками сполучають з ткацькими варстатами в котеджах.
У всіх таких випадках наймалось пару, приміром, за 2 1/2 шилінґи
з ткацького варстату. Цю плату за пару треба було сплачувати
щотижня, все одно, чи працювали ткацькі варстати, чи ні. Кожний
котедж мав 2—6 ткацьких варстатів, що належали робітникам,
вони були або куплені на кредит, абож найняті. Боротьба між
котеджем-фабрикою та власне фабрикою тривала понад 12 років.
Вона закінчилася повного руїною цих 300 котеджів-фабрик.\footnote{
Порівн. «Reports of Insp. of Fact, for 31 st October 1865», p. 64.
} Там
де природа процесу від самого початку не вимагала продукції у великому
маштабі, ті галузі промисловосте, які виникли за останні
\parbreak{}  %% абзац продовжується на наступній сторінці

\parcont{}  %% абзац починається на попередній сторінці
\index{i}{0385}  %% посилання на сторінку оригінального видання
десятиліття, як от виробництво ковертів, сталевих пер тощо, звичайно
проходили спочатку ремісниче виробництво, а потім мануфактурне
виробництво як короткі переходові фази до фабричного
виробництва. Найтяжчою ця метаморфоза лишається там, де мануфактурна
продукція виробу являє собою не послідовний ряд процесів
виготовлення його, а багато відокремлених процесів. Це становило,
приміром, велику перешкоду для фабрикації сталевих пер.
Однак уже перед якимись п’ятнадцятьма роками винайдено автомат,
що воднораз виконує шість окремих процесів. 1820 р. ремество
дало перші 12 тузенів сталевих пер за 7 фунтів стерлінґів 4 шилінґи,
мануфактура дала їх 1830 р. за 8 шилінґів, а фабрика в
наші часи постачає їх гуртовим торгівцям за 2--6 пенсів.\footnote{
Пан Джілло влаштував у Бермінґемі першу мануфактуру сталевих
пер у великому маштабі. Вже 1851 р. вона дала понад 180 мільйонів пер
та споживала річно 120 тонн сталевої бляхи. Бермінґем, що монополізував
цю промисловість в Об’єднаному Королівстві, продукує тепер річно
мільярди сталевих пер. Число осіб, занятих у цьому виробництві, за
переписом 1861 р. становило 1.428; з них 1.268 робітниць, що почали
працювати з п’ятьох років.
}

\subsubsection{Зворотний вплив фабричної системи
на мануфактуру й домашню працю}

З розвитком фабричної системи та переворотом у рільництві, що
супроводить цей розвиток, не тільки поширюється маштаб продукції
по всіх інших галузях промисловости, але змінюється й характер
їхній. Принцип машинового виробництва: розкладати продукційний
процес на його складові фази та розв’язувати проблеми,
що постають таким чином, за допомогою застосування механіки,
хемії і т. ін., коротко кажучи, за допомогою природничих наук, —
цей принцип усюди набуває вирішального значення. Тому машина
протискується в мануфактури, захоплюючи то той, то інший
частинний процес. Отже, тривала кристалізація організації мануфактури,
що походить із старого поділу праці, розпадається
та зазнає безперестанних змін. Крім цього, склад колективного
робітника або комбінованого робочого персоналу зазнає ґрунтовного
перевороту. Протилежно до мануфактурного періоду
плян поділу праці ґрунтується тепер на вживанні жіночої праці,
праці дітей усякого віку, ненавчених робітників, де це ще
можливо, словом, — на вживанні «cheap labour», дешевої праці,
як її характеристично звуть англійці. Це має силу не тільки для
всякої комбінованої у великому маштабі продукції — незалежно
від того, чи вживає вона машин, чи ні, — але й для так званої
домашньої промисловости, все одно, чи працюють робітники в
своїх приватних мешканнях, чи в невеличких майстернях. Ця
так звана сучасна домашня промисловість, крім назви, не має
нічого спільного із стародавньою домашньою промисловістю,
яка має за передумову незалежне міське ремество, самостійне
селянське господарство та передусім хату для робітничої родини
\parbreak{}  %% абзац продовжується на наступній сторінці

\parcont{}  %% абзац починається на попередній сторінці
\index{i}{0386}  %% посилання на сторінку оригінального видання
Її тепер перетворено на зовнішній відділ фабрики, мануфактури
або крамниці. Окрім фабричних робітників, мануфактурних робітників
і ремісників, яких капітал просторово концентрує великими
масами та якими командує безпосередньо, він незримими
нитками пускає в рух іншу армію, армію домашніх робітників,
порозкидуваних по великих містах та по селах. Приклад: фабрика
сорочок панів Тіллей в Лондондері в Ірландії вживає \num{1.000} фабричних
робітників та \num{9.000} домашніх робітників, порозкиданих по
селах\footnote{
«Children’s Employment Commission. 2 nd Report 1864»,
p. LXVIII, n. 415.
}.

Експлуатація дешевих та незрілих робочих сил у сучасній
мануфактурі стає ще безсоромнішою, ніж на фабриці у власному
значенні, бо технічної основи, що існує на фабриці, а саме заміни
мускульної сили машинами та легкости праці, в мануфактурі
здебільшого немає; крім того, в мануфактурі жіночий або ще
незрілий організм дітей якнайбезсовісніше віддають під впливи
отруйних речовин тощо. У так званій домашній праці ця експлуатація
стає ще безсоромнішою, ніж у мануфактурі, тому що здатність
робітників до опору зменшується з розпорошенням їх;
тому що поміж власне підприємцем і робітником втискується цілий
ряд хижацьких паразитів; тому що домашня праця всюди бореться
з машиновим або, принаймні, з мануфактурним виробництвом тієї
самої галузі продукції; тому що злидні відбирають у робітника
найпотрібніші умови праці — помешкання, світло, вентиляцію
й~\abbr{т. ін.}; тому що нереґулярність праці зростає, і, нарешті, тому
що в цих останніх притулках для всіх тих, кого велика промисловість
і рільництво зробили «зайвими», конкуренція між робітниками
неминуче досягає свого максимуму. Економізування
засобів продукції, що його вперше систематично розвиває й
організує машинове виробництво, економізування, що з самого
початку є разом з тим найнещадніше марнотратство робочої
сили та грабування нормальних передумов функціонування праці,
виявляє тепер свій антагоністичний та душогубний бік то дужче,
що менше в певній галузі промисловости розвинута суспільна
продуктивна сила праці й технічна основа комбінованих процесів
праці.

\manualpagebreak{}
\subsubsection{Сучасна мануфактура}

А тепер я поясню на декількох прикладах подані вище тези.
З розділу про робочий день читач уже справді знає безліч доказів
їх. Металеві мануфактури в Бермінґемі та околицях вживають,
здебільша для дуже важкої роботи, \num{30.000} дітей та підлітків,
поруч \num{10.000} жінок. Ми знаходимо їх тут в антигігієнічному
мосяжництві, на фабриках ґудзиків, за ґлязуруванням, ґальванізуванням
і дякуванням\footnote{
І навіть у шліхтуванні терпугів у Шеффілді працюють діти!
}. Надмірна праця для дорослих та
\parbreak{}  %% абзац продовжується на наступній сторінці

\parcont{}  %% абзац починається на попередній сторінці
\index{i}{0387}  %% посилання на сторінку оригінального видання
недорослих забезпечила різним лондонським друкарням газет
та книжок славетну назву «бойні»\footnoteA{
«Children’s Employment Commission. 5 th Report 1866», p. 3,
n. 24, p. 6, n. 55, 56, p. 7, n. 59, 60.
}. Таку ж саму надмірну
працю, жертвами якої є головним чином жінки, дівчата й діти,
ми бачимо і в палітурнях. Тяжка також праця недорослих у
виробництві линв, нічна праця в солоницях, по свічкових та інших
хемічних мануфактурах; у шовкоткальнях, де не користуються
механічною силою, ми бачимо вбивчу працю дітей, уживаних на
те, щоб рухати ткацькі варстати\footnote{
Там же, стор. 114, 115, n 6, 7. Комісар правильно зауважує,
що коли машина взагалі заступає людину, так тут підліток буквально
заступає машину.
}. Одна з найогидніших, найбрудніших
та найгірш оплачуваних праць, до яких переважно
вживають молодих дівчат та жінок, — це сортування лахміття.
Відомо, що Великобританія, не згадуючи вже про величезну масу
її власного лахміття, є склад для торговлі лахміттям цілого світу.
Це лахміття привозять сюди з Японії, якнайвіддаленіших штатів
Південної Америки та з Канарських островів. Але головні джерела
цього довозу є Німеччина, Франція, Росія, Італія, Єгипет,
Туреччина, Бельгія та Голландія. Лахміття служить для
удобрення, фабрикації клоччя (на матраци до ліжок), shoddy
(штучної вовни) та як сировинний матеріял для паперу. Жінки-сортувальниці
лахміття є передатниці віспи та інших заразливих
недуг, що їх першими жертвами стають вони сами\footnote{
Див. звіт про торговлю лахміттям і численні ілюстрації в «Public
Health. 8th Report», London 1866, Appendix, p. 196-208.
}. За класичний
приклад надмірної праці, тяжкої та непідхожої праці, а
тому й сполученої з нею брутальности до робітників, що їх споживають
від наймолодшого віку, можуть бути, поряд рудень та
копалень, цегельні або майстерні для виробу черепиці, де недавно
винайдену машину застосовують в Англії ще лише спорадично
(1866~\abbr{р.}). Між травнем і вереснем праця триває там від 5 години
ранку до 8 години вечора, а там, де сушіння відбувається на
вільному повітрі, — часто від 4 години ранку до 9 години вечора.
Робочий день, що триває від 5 години ранку до 7 години вечора,
вважається за «скорочений», «помірний». Дітей обох статей
уживають, починаючи від 6, навіть від 4 років життя. Вони працюють
стільки ж годин, а часто й більше, ніж дорослі. Праця
тут тяжка, а літня спека ще збільшує виснаження. В одній цегельні
в Mosley, наприклад, одна дівчина 24 років виробляла
\num{2.000} цеглин на день, їй помагало двоє малих дівчаток, які зносили
глину та складали цеглу до купи. Ці дівчатка витягали
щодня по слизьких краях цегельної ями з глибини 30 футів 10 тонн
глини і переносили її на віддаль 210 футів. «Дитина не може
пройти чистилища цегельні, не зазнавши великої моральної
деґрадації\dots{} Безсоромна мова, яку їм доводиться чути від найніжнішого
віку, неподобні, непристойні й безсоромні звички, серед
яких вони виростають в неуцтві та здичавінні, роблять із них
\parbreak{}  %% абзац продовжується на наступній сторінці

\input{i/_0388.tex}
\parcont{}  %% абзац починається на попередній сторінці
\index{i}{0389}  %% посилання на сторінку оригінального видання
цього свого права на здоров’я, вони не можуть одержати жодної
ефективної допомоги і від спеціяльних урядовців санітарної
поліції\dots{} Життя багатьох тисяч робітників та робітниць тепер
без усякої потреби нівечиться та скорочується через безмежні
фізичні страждання, що їх породжує сама лише їхня праця»\footnote{
«Public Health. 6 th Report», London 1864, p. 31.
}.
Для ілюстрації впливу робітних приміщень на стан здоров’я
робітників д-р Сімон подає таку таблицю смертности:

\begin{table}[H]
\noindent\begin{tabularx}{\textwidth}{@{}Xr@{~}lrrr@{}l@{}}
   \toprule 
     \multirowcell{2}[2.5ex][l]{Порівняння галузей \\ промисловости щодо \\ їхнього впливу \\ на здоров'я} &
     \multicolumn{2}{l}{\multirowcell{2}[2.5ex][l]{Число осіб усякого \\ віку, занятих \\ у відповідних галузях \\ промисловости}} &
     \multicolumn{4}{r@{}}{\makecell{
         Норма смертности на \num{100.000} \\
         чоловіка у відповідних \\
         галузях промисловости \\
         за віком
     }} \\
  \cmidrule(l){4-7}
     & & & ~~~25\textendash{}35 & ~~~35\textendash{}45 & ~~~45\textendash{}55 \\

  \midrule
    Рільництво в Англії та Велзі\dotfill{} & 
    $\left.\begin{array}{r@{}}\text{\num{958.265}}\end{array}\right.$& & 743 & \phantom{1.}805 & \num{1.145} \\

    Лондонські кравці\dotfill{} &
    $\left\{
    \begin{array}{r@{}}
      \text{\num{22.301}}\\ 
      \text{\num{12.379}}\end{array}\right.$& 
    $\left.
    \begin{array}{@{}l}
      \text{чоловіків}\\ 
      \text{жінок}
    \end{array} 
    \right\}$ & 958 & \num{1.262} & \num{2.093} \\
                                              

    Лондонські друкарі\dotfill{} &
    $\left.\begin{array}{r@{}}\text{\num{13.803}}\end{array}\right.$& & 894 & \num{1.747} 
    & \num{2.367} & \hang{l}{\footnotemark{}}
\end{tabularx}
\end{table}
\footnotetext{
Там же, стор. 30. Д-р Сімон зауважує, що смертність лондонських
кравців і друкарів на 25--30 році життя в дійсності куди більша,
бо їхні лондонські підприємці одержують із села велике число молодих
людей до 30 років як «учнів» та «improvers» (які хочуть досконало вивчити
своє ремество). В перепису вони фігурують як лондонці, і через те
збільшують число людей, на яке обчислюється смертність у Лондоні,
хоч вони дають відносно менше число смертних випадків, ніж лондонці.
Велика частина з них, особливо у випадках тяжких недуг, повертається
на село. (Там же).
}
\subsubsection{Сучасна домашня праця}

Тепер я звертаюсь до так званої домашньої праці. Щоб скласти
собі уявлення про цю сферу капіталістичної експлуатації, що
являє собою задній плян великої промисловости, та про її страхіття,
треба розглянути, приміром, цілком ідилічне на позір виробництво
цвяхів, що його провадять по деяких глухих селах Англії\footnote{
Тут мова про ковані цвяхи, відмінно від різаних, що їх фабрикують
машиновим способом. Див. «Children’s Employment Commission
З rd Report», p. XI, p. XIX, n. 125--130, p. 53, n. 11, p. 114, n. 487,
p. 137, n. 674.
}.
Тут досить кількох прикладів із галузей продукції мережив та
плетіння з соломи, які зовсім ще не вживають машин або конкурують
з машиновим та мануфактурним виробництвом.

Із тих \num{150.000} осіб, що працюють у виробництві мережива в
Англії, приблизно \num{10.000} підлягають фабричному законові 1861~\abbr{р.}
Величезна більшість решти \num{140.000} осіб складається з жінок,
підлітків та дітей обох статей, хоч чоловіча стать репрезентована
дуже слабо. Стан здоров’я цього «дешевого» матеріялу для експлуатації
видно з такого зіставлення д-ра Трюймена, лікаря
\parbreak{}  %% абзац продовжується на наступній сторінці

\parcont{}  %% абзац починається на попередній сторінці
\index{i}{0390}  %% посилання на сторінку оригінального видання
при головному диспансері (General Dispensary) в Нотінґемі. На
кожні 686 пацієнток, мережівниць, здебільша від 17 до 24 року
життя, сухітниць було:

1852 р.   1    на    45    1857 р.   1   на 13
1853  »   1     »      28    1858   »  1   »    15
1854  »   1     »      17    1859   »  1   »      9
1855  »   1     »      18    1860   »  1   »      8
1856  »   1     »      15    1861   »  1   »      8 \footnote{
«Children’s Employment Commission. 2nd. Report» p. XXII, n. 166.
}

Цей проґрес у поширенні сухот мусить задовольнити найоптимістичніших
проґресистів та найбрехливіших німецьких комівояжерів
вільної торговлі.

Фабричний закон 1861 р. реґулює виробництво мережива
у власному значенні, оскільки в ньому продукують машинами,
а це в Англії є звичайне явище. Галузі, що їх ми тут розглядаємо
коротко, — і саме щодо так званих домашніх робітників, а не
тих, що сконцентровані по мануфактурах, товарових домах і
т. ін., — розпадаються на: 1) finishing (остаточне оброблення
мережива, що фабрикується машиновим способом, — категорія,
що знову таки розпадається на численні підвідділи); 2) плетіння
мережива.

Lace finishing робиться як домашню роботу або по так званих
«mistresses houses»,\footnote*{
— домах хазяйок. \emph{Ред.}
} або в приватних помешканнях жінок, що
працюють сами чи з своїми дітьми. Жінки, які тримають «mistresses
houses», сами є бідні. Майстерня становить частину їхнього
приватного помешкання. Вони дістають замовлення від фабрикантів,
власників крамниць і т. ін. та вживають до праці жінок,
дівчат і малих дітей відповідно до розміру їхньої кімнати та коливань
попиту в цій галузі промисловості. Число занятих робітниць
змінюється по деяких з тих майстерень від 20 до 40, по інших від
10 до 20. Пересічний мінімальний вік, від якого діти починають
працювати, — 6 років, але декотрі мають менше ніж 5 років. Звичайно
робочий час триває від 8 години ранку до 8 години вечора,
з перервою для їжі на 1\sfrac{1}{2} години, при чому їдять нереґулярно
і часто в тих самих вонючих норах, де працюють. Коли справи
йдуть добре, то праця часто триває від 8 години (іноді від 6 години)
ранку до 10, 11 або й 12 години ночі. В англійських казармах
приписаний для кожного солдата об’єм становить 500--600 кубічних
футів, по військових лазаретах — 1.200. А по тих норах
для праці на кожну особу припадає від 67 до 100 кубічних
футів. Крім того, газове світло знищує кисень повітря. Щоб
мереживо тримати чистим, діти мусять часто скидати черевики,
навіть зимою, хоч долівка зроблена з кам’яних плит або з цегли.
«В Нотінґемі немає нічого незвичайного в тому, що в одну невеличку
кімнату, може, не більшу за 12 футів у квадраті, напихають
\parbreak{}  %% абзац продовжується на наступній сторінці

\parcont{}  %% абзац починається на попередній сторінці
\index{i}{0391}  %% посилання на сторінку оригінального видання
14--20 дітей і примушують їх 15 годин на добу займатися такою
роботою, яка сама по собі виснажує людину своєю нудністю та
монотонністю та до того ще й виконується в умовах, що якнайдужче
руйнують здоров’я. Навіть наймолодші діти працюють з
напруженою увагою та дивовижною швидкістю, майже ніколи
не дозволяючи своїм пальцям відпочити або рухатися повільніше.
Коли до них звертаються з запитаннями, то вони не підводять очей
від роботи, боячися втратити хоча б один момент». «Довга палиця»
служить для «mistresses» за засіб спонукати дітей до праці то
більше, що більше здовжується робочий час. «Діти поступінно
втомлюються та стають неспокійні, як птиці, під кінець того довгого
часу, протягом якого їх прив’язано до роботи, монотонної, шкідливої
для очей, виснажливої через одноманітність позиції тіла.
Це справжня рабська праця» («Their work is like slavery»).\footnote{
«Children’s Employment Commission. 2 nd. Report 1864» p. XIX.
XX XXI.
}
Там, де жінки працюють разом із своїми власними дітьми вдома
в сучасному значенні цього слова, тобто в найманій кімнаті,
часто в якійсь халупці на горищі, це становище ще гірше, якщо
це тільки можливо. Таку роботу роздають на 80 миль навколо
Нотінґему. Коли дитина, що працює в крамниці, виходить із неї
о дев’ятій або десятій годині вечора, то часто їй дають на дорогу
ще цілий клунок для того, щоб вона закінчила роботу вдома.
Капіталістичний фарисей, що його репрезентує один з його наймитів,
робить це, звичайно, зворушливо приказуючи: «це для
матері», алеж сам дуже добре знає, що нещасній дитині доведеться
самій присісти та допомагати матері.\footnote{
Там же, стор. XXI, XXVI.
}

Промисловість плетіння мережива поширена головним чином
у двох рільничих округах Англії: в мереживній окрузі Honiton,
20--30 миль поздовж південного берега Девонширу, включаючи
й небагато місць Північного Девону, та в другій окрузі, що
охоплює більшу частину графств Букінґему, Бедфорду, Нортґемптону
та сусідні частини Оксфордширу та Гетінґдонширу.
Котеджі рільничих поденників звичайно являють собою й приміщення
для праці. Деякі мануфактуристи вживають понад
\num{3.000} таких домашніх робітників, головне, дітей та підлітків,
виключно жіночої статі. Тут повторюються ті умови, що ми їх
описали при розгляді lace finishing. Тільки замість «mistresses
houses» тут виступають так звані «lace schools» (школи мережива),
що їх у своїх халупках тримають бідні жінки. Починаючи від
п’ятого року, а іноді й раніше, і до 12 або 15 року, працюють
діти по цих школах; протягом першого року наймолодші працюють
від 4 до 8 години, а потім від 6 години ранку до 8 та 10 години
вечора. «Загалом кажучи, кімнати — це звичайні комірки невеличких
котеджів, камін у них забито, щоб не було протягу, мешканці
іноді й зимою огріваються лише теплом свого власного
тіла. В інших випадках ці так звані шкільні кімнати — це помешкання,
\index{i}{0392}  %% посилання на сторінку оригінального видання
подібні до маленьких комірок без опалення\dots{} Переповнення
цих закутків та зумовлена цим переповненням зіпсованість
повітря доходять часто крайнього ступеня. До цього ще
долучається шкідливий вплив стоків, клозетів, гнилі й іншого
бруду, що є звичайна річ при вході до невеликих котеджів». Щодо
помешкань, то «в одній школі мережива 18 дівчат і хазяйка,
35 кубічних футів на кожну особу; в іншій, де нестерпний сморід,
18 осіб, 24\sfrac{1}{2} кубічних фута на людину. Трапляється й таке, що
в цій промисловості вживають до праці дітей 2--2\sfrac{1}{2} років».\footnote{
Там же, стор. ХХІХ, ХХХ.
}

Там, де в сільських графствах Букінґему та Бедфорду припиняється
плетіння мережива, починається плетіння з соломи. Воно
поширене по значній частині Гертфордширу та по західніх і північних
частинах Есексу. 1861 р. коло плетіння з соломи та коло
виготовлення солом’яних брилів працювало \num{40.043} особи, з них
\num{3.815} чоловічої статі всякого віку, решта — жіночої статі, при
чому \num{14.913} молодші від 20 років, з них \num{7.000} дітей. Замість шкіл
мережива з’являються тут «straw plait schools» (школи плетіння
з соломи). Тут дітей починають учити плести з соломи звичайно
від 4 року, іноді між 3 та 4 роком життя. Виховання вони, звичайно,
не дістають ніякого. Сами діти називають початкові школи
«natural schools» (натуральними школами) відмінно від тих кровососних
установ, де їх тримають за працею просто для того, щоб
вони виготовили роботу, наказану їм від їхніх напівзголоднілих
матерів — здебільша 30 ярдів на день. Ці матері потім часто
примушують їх працювати ще вдома до 10, 11, 12 години вночі.
Солома ріже їм пучки й рот, яким вони її постійно змочують.
Згідно з загальним поглядом медичних урядовців Лондону, що
його зрезюмував д-р Беллярд, 300 кубічних футів на кожну
особу становлять мінімум об’єму для спальні або робітної
кімнати. Але в школах плетіння з соломи помешкання ще менші,
ніж у школах мережива, а саме на кожну особу в них припадає
12\sfrac{2}{3}, 17, 18\sfrac{1}{2} і менше ніж 22 кубічні фути. «Менші з цих чисел, —
каже комісар Байт, — дають менший об’єм, ніж половина того,
що його заняла б дитина, коли б її запакувати в коробку з вимірами
по 3 фути кожний». Отакі радощі життя дітей до 12 або
14 років. Бідні, занепалі батьки думають тільки про те, щоб
якомога більше видушити з дітей. Діти, вирісши, звичайно,
зовсім не дбають про своїх батьків та кидають їх. «Немає нічого
дивного в тому, що неуцтво й розпуста панують серед людности,
так вихованої\dots{} Її моральність є на щонайнижчому щаблі\dots{} Багато
жінок має нешлюбних дітей, і деякі з них у такому недозрілому
віці, що сами знавці кримінальної статистики з дива німіють
перед цим фактом».\footnote{
Там же, стор. ХL, XLI.
} І батьківщина цих зразкових родин є
Англія — зразкова християнська країна Европи, як каже граф
Монталямбер, — людина, певна річ, найкомпетентніша в справах
християнства!


\index{i}{0393}  %% посилання на сторінку оригінального видання
Заробітну плату, взагалі мізерну в розглянутих щойно галузях
промисловосте (винятково максимальна плата дітей по школах
плетіння з соломи — 3\shil{ шилінґи}), зменшується ще нижче її номінальної
величини за допомогою truck-системи\footnote*{
системи виплати заробітної плати товарами. \emph{Ред.}
}, що взагалі особливо панує в округах виробництва мережива\footnote{
«Children’s Employment Commission. 1st Report 1863», p. 185.
}.

\subsubsection{Перехід сучасної мануфактури і домашньої
праці до великої промисловости. Прискорення
цієї революції через застосування фабричних
законів до цих способів продукції}

Здешевлювання робочої сили через саме лише зловживання
жіночими та недозрілими робочими силами, через просте грабування
усіх нормальних умов праці й життя та через брутальність
надмірної і нічної праці, кінець-кінцем натрапляє на певні природні
межі, що їх уже не можна переступити, а разом з цим
на ці межі натрапляє і побудоване на цих підвалинах здешевлення
товарів і капіталістична експлуатація взагалі. Скоро тільки цього
пункту, нарешті, досягнено, — а на це треба багато часу, — тоді
б’є година для заведення машин і швидкого віднині перетворення
розпорошеної домашньої праці (або й мануфактури) на фабричну
промисловість.

Найколосальніший приклад цього руху дає виробництво «wearing
apparel» (речей, що належать до одягу). За клясифікацією
«\textenglish{Children’s Employment Commission}» ця промисловість охоплює
виробників солом’яних капелюхів, дамських капелюхів, шапкарів,
кравців, milliners і dressmakers\footnote{
Millinery стосується, власне, лише до виробництва головних уборів,
але воно охоплює і виробництво жіночих пальт і мантиль, тимчасом
як dressmakers — не те саме, що й наші модистки.
}, виробників сорочок і швачок,
корсетниць, рукавичників, шевців і разом з тим багато дрібніших
галузей, як от фабрикація галстухів, комірців і~\abbr{т. ін.} Жіночий
персонал, що працює в цих галузях промисловости в Англії і
Велзі, становив 1861~\abbr{р.} \num{586.298} осіб, з них, щонайменше, \num{115.242}
молодші за 20 років, \num{16.650} молодші за 15 років. Число цих робітниць
у цілому Об’єднаному Королівстві становило (1861~\abbr{р.})
\num{750.334}. Число робітників-чоловіків, що того самого часу працювали
в капелюшній, шевській, рукавичній та кравецькій промисловості
Англії й Велзу, становило \num{437.969}, з них \num{14.964} молодші
за 15 років, \num{89.285} осіб 15--20 років, \num{333.117} — понад 20 років.
У ці числа не входить багато, належних сюди, дрібних галузей.
Але коли взяти ці числа так, як вони є, то ми матимемо лише для
Англії та Велзу, за переписом 1861~\abbr{р.}, суму в \num{1.024.277} осіб,
отже, приблизно стільки, скільки забирає рільництво і скотарство.
Тепер ми починаємо розуміти, для чого машини допомагають
\parbreak{}  %% абзац продовжується на наступній сторінці

\parcont{}  %% абзац починається на попередній сторінці
\index{i}{0394}  %% посилання на сторінку оригінального видання
якимось чарівним способом витворювати такі величезні маси продуктів
і «звільняти» такі величезні маси робітників.

Виробленням «wearing apparel» займаються ті мануфактури,
які репродукували в себе тільки той поділ праці, що його membra
disjecta\footnote*{
поодинокі члени. \Red{Ред.}
} вони находили готовими: крім того, дрібні ремісники-майстри,
що однак працюють не на індивідуальних споживачів,
як раніш, а на мануфактури й крамниці, так що часто цілі міста
й околиці заняті в таких галузях як своїм фахом, як от шевство
тощо; нарешті, у найбільшому розмірі цим займаються так звані
домашні робітники, які являють собою зовнішні відділи мануфактур,
крамниць, а то й майстерень дрібних майстрів\footnote{
Англійські millinery і dressmaking провадиться здебільша в
будинках підприємців, почасти найманими робітницями, що мешкають
у цих будинках, почасти поденницями, що мешкають осторонь.
}. Маси матеріялу праці, сировинного матеріялу, півфабрикатів тощо
постачає велика промисловість, маса дешевого людського матеріялу
(taillable à merci et miséricorde\footnote*{
відданого на ласку та гнів. \Red{Ред.}
}) складається із «звільнених»
великою промисловістю й рільництвом. Мануфактури
цієї галузі завдячували своє походження переважно потребі капіталістів
мати під рукою готову армію, що відповідала б кожному
рухові попиту\footnote{
Комісар Вайт відвідав одну мануфактуру військового одягу,
де працювало \num{1.000}--\num{1.200} осіб, майже всі жіночої статі, одну мануфактуру
чобіт з \num{1.300} особами, де майже половина були діти й підлітки, і~\abbr{т. ін.} («Children’s Employment Commission. 2nd Report», р. XVII,
n. 319).
}. Однак ці мануфактури поряд себе дозволяли й далі
існувати розпорошеному ремісничому і домашньому виробництву
як своїй широкій основі. Значну продукцію додаткової вартости
в цих галузях праці, разом із проґресивним здешевленням їхніх
продуктів, зумовило й зумовлюється, головним чином, мінімальною
заробітною платою, ледве достатньою для мізерного животіння
та сполученою з максимально можливим для людини робочим
часом. Саме дешевина людського поту й людської крови, перетворених
на товари, постійно поширювала й день-у-день далі поширює
ринок збуту, а для Англії особливо і колоніяльний ринок, де
до того ще й переважають англійські звички й англійський смак.
Нарешті, настав поворотний пункт. Основи старої методи, простої
брутальної експлуатації робочого матеріялу, що більше або
менше супроводилася систематично розвинутим поділом праці,
було вже недосить для ринку, що зростав, та для конкуренції
між капіталістами, що зростала ще швидше. Пробив час машини.
Такою вирішально революційною машиною, що рівномірно охопила
всю безліч галузей цієї сфери продукції, як от виробництво
модного вбрання, кравецтво, шевство, швацтво, капелюшництво
й~\abbr{т. ін.}, була швацька машина.

Її безпосереднє діяння на робітників приблизно таке, як і
всякої машини, що в періоді великої промисловости завойовує
\parbreak{}  %% абзац продовжується на наступній сторінці

\parcont{}  %% абзац починається на попередній сторінці
\index{i}{0395}  %% посилання на сторінку оригінального видання
нові галузі продукції. Дітей наймолодшого віку усувається. Заробітна
плата машинових робітників зростає порівняно з платою
домашніх робітників, з-поміж яких багато належать до
«найбідніших із бідних» («the poorest of the poor»). Заробітна плата
поставлених у кращі умови ремісників, що з ними конкурує
машина, падає. Нові машинові робітники — це виключно дівчата
й молоді жінки. За допомогою механічної сили вони знищують
монополію чоловічої праці в тяжчій роботі й витискують із сфери
легшої праці маси старих жінок і дітей-недолітків. Всемогутня
конкуренція вбиває найслабших робітників ручної праці. Жахливий
зріст числа випадків голодної смерти (death from starvation)
у Лондоні за останнє десятиліття йде паралельно з поширенням
машинового шиття.\footnote{
Приклад. 26 лютого 1864 р. тижневий збіт про смертність Registrar
General реєструє п’ять випадків голодної смерти. Того самого
дня «Times» повідомляє про новий випадок голодної смерти. Шість жертов
голодної смерти за один тиждень!
} Нові робітниці коло швацької машини,
що її вони пускають у рух рукою й ногою або тільки рукою,
сидьма й навстоячки, залежно від важкости, величини й спеціяльности
машини, витрачають багато робочої сили. Їхнє заняття стає
шкідливим для здоров’я в наслідок тривалости процесу, хоч він
здебільша і коротший, аніж за старої системи. Всюди, де швацькі
машини, як ось у виробництві чобіт, корсетів, капелюхів тощо,
заводиться в майстерні, що й без того були тісні й переповнені,
вони збільшують антигігієнічні впливи. «Вражіння, — каже комісар
Лорд, — при вході до низьких робітних приміщень, де працює
разом 30--40 машинових робітників, нестерпне\dots{} Спека, почасти
від газових пічок для огрівання прасок, жахлива\dots{} Навіть тоді,
коли в таких приміщеннях переважають так звані помірні робочі
години, тобто від 8 години ранку до 6 години вечора, — то все ж
день-у-день зомліває звичайно 3 або 4 особи».\footnote{
«Children’s Employment Commission. 2nd. Report 1864», p. LXVII,
n. 406--409, p. 84, n. 124, p. LXXIII, n. 441, p. 66, n. 6, p. 84, n. 126,
p. 78, n. 85, p. 76, n. 69, p. LXXII, n. 483.
}

Переворот у суспільному способі продукції, цей неминучий
продукт перетворень у засобі продукції, відбувається в строкатому
хаосі переходових форм. Вони змінюються залежно від того,
в якому обсязі та протягом якого часу швацька машина вже захопила
ту або іншу галузь промисловости, змінюються залежно
від того стану, в якому перебували робітники до заведення
машини, — залежно від переваги мануфактурного, ремісничого або
домашнього виробництва, наймової плати за робітні приміщення\footnote{
«Орендна плата за робітні приміщення є, здається, той елемент,
що відіграє вирішальну ролю, а тому в столиці стара система роздавати
роботу дрібним підприємцям і родинам трималася найдовше; так само
до неї там і поверталися найшвидше» («The rental of premises required
for work rooms seems the element which ultimately determines the point,
and consequently it is in the metropolis, that the old system of giving work
out to small employers and families has been longest retained, and earliest
returned to»). (Там же, стор. 84, n. 123). Остання фраза стосується виключно
до шевства.
}
\parbreak{}  %% абзац продовжується на наступній сторінці

\parcont{}  %% абзац починається на попередній сторінці
\index{i}{0396}  %% посилання на сторінку оригінального видання
й т. ін. Наприклад, у виробництві модного вбрання, де праця
здебільша вже була зорганізована, переважно у формі простої
кооперації, швацька машина спочатку становить лише новий
фактор мануфактурного виробництва. В кравецтві, виробництві
сорочок, у шевстві й т. ін. перехрещуються всі форми. Тут —
власне фабричне виробництво. Там — посередники дістають сировинний
матеріял від капіталіста en chef та гуртують коло швацьких
машин по «комірках» і «горищах» 10--50, а то й більше найманих
робітників. Нарешті, як то буває при всіх машинах, які не
являють собою розчленованої системи й які можна застосовувати
в карликовому розмірі, ремісники або домашні робітники із
своєю власного родиною або за участю небагатьох чужих робітників
уживають швацьких машин, що їм самим і належать.\footnote{
Цього немає в рукавичництві й т. ін., де становище робітників
ледве можна відрізнити від становища павперів.
}
В Англії тепер фактично переважає така система, що капіталіст
концентрує значне число машин у своїх будівлях і потім розподіляє
машиновий продукт для дальшого оброблення серед армії
домашніх робітників.\footnote{
«Children’s Employment Commission. 2 nd Report 1864», p. 2,
n. 122.
} Однак строкатість переходових форм
не заховує тенденції до перетворення на фабричне виробництво
у власному значенні. Цю тенденцію живить сам характер швацької
машини, що її придатність до різноманітного вжитку штовхає
до сполучення відокремлених раніш галузей продукції в
тому самому будинку й під командою того самого капіталу; її
живить та обставина, що попереднє зшивання й деякі інші операції
найдоцільніше виконувати там, де стоїть машина; нарешті,
її живить неминуча експропріяція ремісників і домашніх робітників,
що продукують власними машинами. Ця доля спіткала
їх уже почасти тепер. Невпинний зріст маси капіталу, вкладеного
у швацькі машини,\footnote{
В Лейчестері в самому лише чобітно-шевському виробництві,
що продукувало на гуртовий продаж, уже 1864 р. вживалося 800 швацьких
машин.
} спонукує збільшувати продукцію та спричинює
на ринку застої, що дає сигнал домашнім робітникам продавати
швацькі машини. Перепродукція самих таких машин примушує
продуцентів, які потребують збуту, віддавати їх на прокат
за тижневу плату, і цим самим вона утворює вбивчу конкуренцію
для дрібних власників машин.\footnote{
«Children’s Employment Commission. 2nd Report 1864», p. 84,
n. 124.
} Постійні зміни конструкції,
що все ще тривають далі, та здешевлення машин так само постійно
знецінюють старі екземпляри машин і дають змогу вживати їх
із зиском тільки великим капіталістам, що купують їх масами
по неймовірно низьких цінах. Нарешті, тут, як і в усіх подібних
процесах перевороту, вирішальне значення має заміна людини
паровою машиною. Вживання парової сили натрапляє спочатку
на суто технічні перешкоди, як от двигіт машин, труднощі щодо
\parbreak{}  %% абзац продовжується на наступній сторінці

\parcont{}  %% абзац починається на попередній сторінці
\index{iii1}{0397}  %% посилання на сторінку оригінального видання
норми процента. — І далі така мудрість. На слушне зауваження:
„Але за гроші платяться проценти“, яке, звичайно, містить у собі
запитання: яке відношення має процент, який одержує банкір,
що зовсім не торгує товарами, до цих товарів? І хіба не одержують грошей за однакові проценти
фабриканти, які витрачають
ці гроші на зовсім різних ринках, отже, на ринках, де панує
цілком різне відношення між попитом і поданням уживаних у виробництві товарів? — на це питання наш
тріумфуючий геній зауважує: якщо фабрикант купує бавовну в кредит, „тоді ріжниця між ціною за
готівку і ціною в кредит у момент скінчення строку становить міру процента“. Навпаки. Існуюча норма
процента, регулювання якої геній Нормана повинен пояснити, є масштаб ріжниці
між ціною за готівку і ціною в кредит до скінчення строку. Спочатку треба продати бавовну по її ціні
за готівку, а ця ціна визначається ринковою ціною, яка сама регулюється станом попиту
й подання. Припустім, що ціна = 1000\pound{ фунтам стерлінгів}. На цьому
між фабрикантом і бавовняним маклером справа закінчується,
оскільки йдеться про купівлю й продаж. Але сюди долучається
друга операція. Операція між позикодавцем і позичальником. Вартість у 1000\pound{ фунтів стерлінгів} дається
фабрикантові в позику бавовною, а він повинен повернути її грішми, скажімо, через три місяці. А
проценти за 1000\pound{ фунтів стерлінгів} за три місяці, визначувані
ринковою нормою процента, становитимуть тоді надбавку до і поверх ціни готівкою. Ціна бавовни
визначається попитом і поданням.
Але ціна позиченої вартості бавовни, ціна 1000\pound{ фунтів стерлінгів} на
три місяці, визначається нормою процента. І це — а саме, що сама
бавовна перетворюється таким чином у грошовий капітал — служить для пана Нормана доказом того, що
процент існував би, навіть
якби взагалі не існувало грошей. Якби взагалі не існувало грошей,
то в усякому разі не існувало б ніякої загальної норми процента.

Насамперед, вульгарне уявлення про капітал як про „товари,
вживані у виробництві“. Оскільки ці товари фігурують як капітал, їх вартість як \emph{капіталу}, в відміну
від їх вартості як \emph{товарів}, виражається в зиску, який одержується від їх продуктивного або
торговельного застосування. І норма зиску безумовно
завжди має деяке відношення до ринкової ціни куплених товарів
і до їх попиту й подання, але визначається вона ще зовсім іншими
обставинами. І що норма процента взагалі має свою межу в нормі
зиску, — в цьому немає ніякого сумніву. Але нехай пан Норман
прямо скаже нам, як визначається ця межа. А визначається
вона попитом і поданням грошового капіталу \emph{в відміну його} від
інших форм капіталу. Але можна було б далі запитати: Як визначається попит і подання грошового
капіталу? Що існує прихований зв’язок між поданням речового капіталу і поданням грошового капіталу,
— в цьому немає ніякого сумніву, так само як
і в тому, що попит промислових капіталістів на грошовий капітал визначається обставинами дійсного
виробництва. Замість
того, щоб пояснити нам це, Норман повчає нас тієї премудрості,
\parbreak{}  %% абзац продовжується на наступній сторінці

\input{i/_0398.tex}
\parcont{}  %% абзац починається на попередній сторінці
\index{i}{0399}  %% посилання на сторінку оригінального видання
досягти до робітника\footnote{
Заведення цієї й інших машин на фабриці сірників замінило в
одному відділі  230 підлітків 32 підлітками й дівчатами від 14 до 17 років.
Цю економію на робітниках 1865~\abbr{р.} проведено ще далі за допомогою застосування
парової сили.
}. Так само ще й тепер твердять у тих
галузях мануфактури мережив, які ще не підпорядковані фабричному
законові, що, мовляв, не можна встановити реґулярного
часу на їжу через те, що різні матеріяли для мережив потребують
для сушіння неоднакового часу, що коливається від трьох хвилин
до однієї години й більше. На це комісари «Children’s Employment
Commission» відповідали: «Обставини тут такі ж самі, що
й у виробництві шпалер. Декотрі з головних фабрикантів з цієї
галузі енерґійно доводили, що природа вживаних матеріялів і
різнорідність процесів, які вони проходять, не дозволяють, без великої
втрати, раптом припиняти роботу для їжі\dots{} За пунктом 6
відділу VI Factory Act’s Extension Act\footnote*{
закону про поширення сфери чинности фабричного закону. \emph{Ред.}
} (1864 p.) їм дано від
часу видання закону вісімнадцятимісячний строк, після скінчення
якого вони мусили прийняти перерви для відпочинку,
приписані фабричним актом»\footnote{
«Children’s Employment Commission. 2nd Report 1864», p. IX,
n. 50.
}. Ледве закон дістав санкцію
парляменту, як пани фабриканти вже зробили відкриття: «Той
лихий стан, якого ми сподівалися від заведення фабричного закону,
не настав. Ми не бачимо, щоб продукція будь-як ослабла.
В дійсності ми продукуємо більше протягом того самого часу»\footnote{
«Reports of Insp. oî Fact. for 31 st October 1865», p. 22.
}.
Отже, ми бачимо, що англійський парлямент, якому, певно,
ніхто не закине геніяльности, через досвід дійшов зрозуміння
того, що примусовий закон простими приписами може усунути
всі так звані природні перешкоди продукції щодо обмеження
й реґулювання робочого дня. Тому при заведенні фабричного
закону в якійсь галузі промисловості визначається строк від 6
до 18 місяців, протягом якого справа фабрикантів є усунути
технічні перешкоди. Фраза Мірабо: «Impossible? Ne me dites
jamais ce bête de mot!»\footnote*{
Неможливо? Не говоріть мені ніколи цього дурного слова! \emph{Ред.}
} стосується особливо до сучасної технології.
Але, активізуючи таким чином, мов у теплиці, розвиток
матеріяльних елементів, доконечних для перетворення мануфактурного
виробництва на фабричне, фабричний закон разом з
цим через доконечність збільшених витрат капіталу прискорює
загибіль дрібніших майстрів та концентрацію капіталу\footnote{
«Потрібних поліпшень\dots{} не можна завести в багатьох старих
мануфактурах без таких витрат капіталу, які перевищують засоби багатьох
сучасних власників\dots{} Заведення фабричних законів неодмінно
супроводиться переходовою дезорганізацією. Розмір цієї дезорганізації
прямо пропорційний до величини того лиха, якому треба зарадити».
(Там же, стор. 96, 97).
}.

Залишаючи осторонь суто технічні й технічно усовувані
перешкоди, реґулювання робочого дня наражається на безладні
\parbreak{}  %% абзац продовжується на наступній сторінці

\input{i/_0400.tex}
\index{i}{0401}  %% посилання на сторінку оригінального видання
На фабриках і мануфактурах, не підданих ще фабричному
законові, панує якнайстрашніша надмірна праця: періодично,
підчас так званого сезону, і спорадично — в наслідок раптових замовлень. У зовнішніх відділах
фабрики, мануфактури й
крамниць, тобто у сфері домашньої праці, і без того цілком нерегулярної, цілком залежної щодо
сировинного матеріялу й замовлень
від примх капіталіста, який тут не є обмежений міркуваннями
про експлуатацію будівель, машин тощо і не ризикує нічим, хіба
тільки шкурою самих робітників, — у цій сфері таким чином систематично вирощують промислову резервну
армію, якою завжди
можна порядкувати і яку протягом однієї частини року винищують якнайжорстокішим примусом до праці, а
протягом другої
частини зводять до стану голоти через брак праці. «Підприємці, — каже «Children’s Employment
Commission», — визискують
звичну нереґулярність домашньої праці, щоб у ті часи, коли
потрібна нагальна праця, здовжувати її до 11, 12, 2 години
вночі, а в дійсності, як каже поширена там фраза, здовжувати
її до «всякої години» та ще до того в помешканнях, «де самого
лише смороду досить, щоб вас збити з ніг (the stench is enough
to knock you down). Ви дійдете, може, до дверей і відчините їх,
але вас пройме жах, і ви не підете далі».\footnote{
«Children’s Employment Commission, 4th Report», p. XXXV,
n. 235 і 237.
} «Чудаки оті наші
підприємці, — каже один із вислуханих свідків, швець, — вони
гадають, нібито дитині не шкодить, коли протягом однієї половини року її на смерть виснажують
працею, а протягом другої
половини майже примушують тинятися без роботи».\footnote{
Там же, стор. 127, n. 56.
}

Як про технічні перешкоди, так само й про ці так звані «промислові звички» («usages which have grown
with the growth of
trade»\footnote*{
— звички, що розвивалися разом з розвитком промислів. \emph{Ред.}
}) заінтересовані капіталісти твердили й твердять, ніби
вони є «природні межі» продукції — улюблений лемент бавовняних лордів тих часів, коли вперше почав
їм загрожувати фабричний закон. Хоч їхня промисловість більше ніж усяка інша
спирається на світовий ринок, а тому й на судноходство, однак
досвід спростував їхню брехню. Від того часу англійські фабричні
інспектори дивляться на кожну таку «промислову перешкоду»
як на просту викрутку.\footnote{
«Щодо втрат, що їх зазнає торговля в наслідок невчасного виконання замовлень на товари, які
доводиться перевозити морем, то я пригадую
собі, що це був улюблений арґумент панів фабрикантів у 1832 та 1833 рр.
Все, що можна тепер сказати з цього приводу, не має такої ваги, як тоді,
коли пара не скоротила ще наполовину всіх дистанцій і не утворила нових
умов для перевозу. І в той час цей арґумент не витримував критики, а
тепер він зовсім не витримує її». («With respect to the loss of trade by
the non-completion of shipping orders in time, I remember that this was
the pet argument of the factory masters in 1832 and 1833. Nothing that
can be advanced now on this subject could have the force that it had then,
before steam had halved all distances and established new regulations
for transit. It quite failed at that time of proof when put to the test, and
again it will certainly fail should it have to be tried»). («Reports of Insp.
of Fact, for 31 st October 1862», p. 54, 55).
} Ґрунтовні й сумлінні досліди «Children’s
\index{i}{0402}  %% посилання на сторінку оригінального видання
Employment Commission» справді доводять, що в деяких
галузях промисловости регулювання робочого дня тільки рівномірніше розподілило б на цілий рік ту
масу праці, якої вживається вже в цих галузях;\footnote{
«Children’s Employment Commission, 4 th Report», p. XVIII,
n. 118.
} що це реґулювання є перше раціональне
обмеження людовбивчих, нісенітних, пустотливих примх моди,\footnote{
Джон Беллерс уже 1699 р. зауважує: «Непостійність мод збільшує
число бідних. Вона спричиняє два великі лиха: 1) робітники бідують
узимку від недостачі праці, бо торговці матеріями і ткачі-хазяїни не
наважуються витрачати свої капітали, щоб дати робітникам роботу, поки
настане весна й виявиться, яка буде мода; 2) по весні бракує робітників,
і ткачі-хазяїни мусять брати багато учнів, щоб задовольнити потреби
королівства протягом кварталу або півроку; це відбирає руки від рільництва, позбавляє село робочих
сил і здебільша переповнює міста жебраками; а ті, що соромляться жебракувати, зимою помирають з
голоду».
(«The uncertainty of fashions does increase necessitous Poor. It has two
great mischiefs in it: 1st) The journeymen are miserable in winter for want
of work, the mercers and master-weavers not daring to layout their stocks
to kepp the journeymen imployed before the spring comes and they know
what the fashion will then be; 2dly) In the spring the journeymen are
not sufficient, but the master-weavers must draw in many prentices, that
they may supply the trade of the kingdom in a quarter or half a year, which
robs the plow of hands, drains the country of labourers, and in a great part
stocks the city with beggars, and starves some in winter that are ashamed
to beg»). («Essays about the Poor, Manufactures etc.», p. 9).
}
які сами по собі не відповідають системі великої промисловости;
що розвиток океанського судноходства й засобів комунікації
взагалі усунув власне технічну підставу сезонової праці;\footnote{
«Children’s Employment Commission. 5 th Report», p. 171, n. 31.
}
що всі інші обставини, яких нібито не можна контролювати,
усувається поширенням будівель, додатковими машинами, збільшенням числа одночасно занятих робітників\footnote{
Так, наприклад, у свідченнях бретфордських торговців-експортерів читаємо: «За цих обставин
ясно, що немає потреби примушувати
дітей працювати по крамницях довше, ніж від 8 години ранку до 7—7\sfrac{1}{2} годин вечора. Це — справа лише
додаткових видатків і додаткових рук.
[Дітям не треба було б працювати до пізньої ночі, коли б деякі підприємці не були такі жадні на
бариші: додаткова машина коштує лише
16 або 18 фунтів стерлінґів]\dots{} Всі труднощі випливають із недостатнього
устаткування та недостатнього помешкання». (Там же, стор. 171, n. 31,
36 і 38).
} і зворотним впливом, що його всі ці зміни справляють на систему великої торгівлі.\footnote{
Там же. Один лондонський фабрикант, який, зрештою, розглядає
примусове реґулювання робочого дня як засіб захисту робітників проти
фабрикантів і самих фабрикантів проти великої торговлі, свідчить: «Нашу
промисловість притискують торговці-експортери, які, приміром, відсилаючи товари вітрильним кораблем,
хочуть до початку певного сезону
бути вже на місці і разом з тим сховати собі до кишені ріжницю між
фрахтом вітрильного корабля й пароплава; абож із двох пароплавів вони
хочуть вибрати собі той, що відпливає раніш, щоб з’явитися на закордонному ринку попереду своїх
конкурентів».
}  Проте
капітал, як він це не раз заявляв устами своїх

\parbreak{}  %% абзац продовжується на наступній сторінці

\input{i/_0403.tex}
\input{i/_0404.tex}
\index{i}{0405}  %% посилання на сторінку оригінального видання
Хоч і які мізерні в цілому постанови фабричного закону про
виховання, все ж вони оголосили початкове навчання за обов’язкову
умову праці\footnote{
За англійським фабричним законом батьки не можуть посилати
дітей молодших від 14 років до «контрольованих» фабрик, не даючи
їм одночасно початкової освіти. Фабрикант відповідає за недодержання
закону. «Навчання при фабриках обов'язкове та є умова праці» («Factory
education is compulsory, and it is a condition of labour»). («Reports of
Insp. of Fact, for 31st. October 1863», p. 111).
}. Їхній успіх вперше довів, що можна сполучати
навчання і гімнастику\footnote{
Про найкращі наслідки сполучення гімнастики (а для хлопців
і військових вправ) з обов’язковим навчанням дітей по фабриках і школах
для бідних дивись промову Н.~В.~Сеніора на сьомому, щорічному конґресі
«National Association for the Promotion of Social Science» в «Report
of Procedings etc.», London 1863, p. 63, 64, а також звіти фабричних інспекторів
з 31 жовтня 1865~\abbr{р.}, стор. 118, 119, 120, 126 і далі.
} з ручною працею, отже, і ручну
працю з навчанням і гімнастикою. З свідчень учителів фабричні
інспектори незабаром виявили, що фабричні діти, хоч їх навчають
удвоє менше, ніж звичайних школярів, здобувають стільки ж
знання, а часто й більше, ніж ті останні. «Справа проста. Ті, що
проводять у школі тільки половину дня, завжди мають свіжу
голову й майже завжди здатні й охочі вчитися. Система поперемінного
чергування праці й навчання робить одне заняття відпочинком
від другого, отже, вона значно відповідніша для дитини,
ніж безперервність одного з цих двох заняттів. Хлопчина, що від
самого ранку сидить у школі, особливо ж у спеку, ніяк не може
змагатися з якимось іншим, що жвавий і втямливий приходить до
школи від своєї праці»\footnote{
«Reports of Insp. of Fact, for 31st October 1865», p. 118. Один
наївний фабрикант шовку заявив слідчому комісарові «Children’s Employment
Commission» ось що: «Я цілком переконаний, що справжній секрет
продукувати вправних робітників знайдено у сполученні праці з навчанням,
починаючи від дитинства. Звичайно, праця не повинна бути ні
надто напружена, ні осоружна, ані нездорова. Я бажав би, щоб мої
власні діти мали працю й забави як відпочинок від школи». («Children’s
Employment Commission. 5th Report», p. 82, n. 36).
}. Дальші докази можна знайти в промові
Сеніора на соціологічному конґресі в Едінбурзі 1863~\abbr{р.} Сеніор
зазначає тут, між іншим, і те, що однобічний непродуктивний і
довгий шкільний день дітей вищих і середніх кляс без користи
збільшує працю вчителя, «тимчасом як він не тільки даремно,
але й з абсолютною шкодою для дітей забирає їм час, виснажує
їхнє здоров’я й енергію»\footnote{
Сеніор, там само, стор. 66. Яким чином велика промисловість на
певному ступені розвитку через переворот у способі матеріяльної продукції
і в суспільних продукційних відносинах робить переворот і в головах,
показує яскраво порівняння промови Н.~В.~Сеніора з 1863~\abbr{р.} з його філіппікою
проти фабричного закону 1833~\abbr{р.}, або порівняння поглядів згаданого
конґресу з тим фактом, що в певних сільських частинах Англії бідним
батькам ще й досі заборонено під загрозою голодної смерти навчати
своїх дітей. Так, наприклад, пан Снелл повідомляє як про звичайну практику
в Сомерсетшірі, що коли бідна людина подається до парафії
по допомогу, то її примушують забрати своїх дітей із школи. Так, пан
Воллестоп, піп з Feltham’y, оповідає про випадки, коли деяким родинам
відмовлено всякої допомоги за те, що «вони посилали своїх дітей до
школи»!
}. Із фабричної системи, як можна простежити
\index{i}{0406}  %% посилання на сторінку оригінального видання
в деталях у Роберта Овена, виріс зародок виховання
майбутности, яке для всіх дітей певного віку сполучатиме продуктивну
працю з навчанням і гімнастикою, і не тільки як методу
піднесення суспільної продукції, але і як єдину методу творити
всебічно розвинутих людей.

Ми бачили, що велика промисловість технічно знищує (aufhebt)
мануфактурний поділ праці, який на все життя прив’язував
геть усю людину до однієї детальної операції, але разом з цим
капіталістична форма великої промисловости репродукує той
поділ праці в ще потворнішому вигляді: на фабриці у власному
значенні — через перетворення робітника в свідомий додаток
до частинної машини, повсюди — почасти через спорадичне вживання
машини і машинової праці\footnote{
Де ремісничі машини, що їх пускає в рух людська сила, безпосередньо
або посередньо конкурують із розвинутими машинами, отже, з
машинами, які мають своєю передумовою механічну рушійну силу, там
постає велика переміна щодо робітника, який пускає машину в рух.
Первісно парова машина заміняла цього робітника, тепер він повинен
заміняти парову машину. Тому напруження й витрата його робочої сили
досягають величезних розмірів, особливо для підлітків, що засуджені
на такі тортури! Так, комісар Лендж бачив у Ковентрі й околицях 10—15-літніх
хлопців, що їх уживають крутити стьожковий варстат, не
кажучи вже про ще молодших дітей, які мусили крутити варстати менших
розмірів. «Це надзвичайно тяжка праця. Хлопчики просто заміняють
парову силу» («The boy is a mere substitute for steam power»),
(Children’s Employment Commission. 5th Report 1866», p. 114, n. 6). Про
вбійчі наслідки «цієї системи рабства», як називає її офіціяльний звіт,
там же і далі.
}, а почасти через заведення
жіночої, дитячої і некваліфікованої праці як нової основи поділу
праці. Суперечність поміж мануфактурним поділом праці і суттю
великої промисловости виявляється ґвалтовно. Вона виявляється,
між іншим, у тому жахливому факті, що велику частину дітей,
уживаних по сучасних фабриках та мануфактурах і від наймолодшого
віку прикованих до найпростіших маніпуляцій, визискують
цілими роками, при чому вони не навчаються ніякої праці, яка
згодом зробила б їх придатними хоча б у тій самій мануфактурі
або фабриці. Наприклад, в англійських друкарнях раніше практикували
перехід учнів од легших до змістовніших робіт, перехід,
що відповідав системі давньої мануфактури й ремества. Учні
проходили науку, аж поки ставали навченими друкарями. Вміти
читати й писати — це було доконечною вимогою від усіх тих,
хто бажав стати ремісником. Але все це змінилося з того часу,
як заведено друкарську машину. Вона вживає робітників двох
категорій — дорослого робітника, доглядача машини, і друкарських
хлопчаків, здебільша від 11 до 17 років, що їхня праця
виключно в тому, щоб подавати аркуші паперу в машину абож
забирати з неї надруковані аркуші. Вони виконують, особливо
в Лондоні, цю тяжку працю в деякі дні тижня по 14, 15, 16 годин
без перерви, а часто 36 годин уряд лише з двома годинами перерви
\parbreak{}  %% абзац продовжується на наступній сторінці


\index{ii}{0407}  %% посилання на сторінку оригінального видання
\subsubsection{Заміщення II с при акумуляції}

Отже, при обміні І ($v \dplus{} m$) на II с можливі різні випадки.

При простій репродукції обидві ці величини мусять дорівнювати одна
одній й заміщувати одна одну, бо інакше, як ми бачили вище, проста
репродукція не може відбуватись без порушень.

При акумуляції треба звернути увагу насамперед на норму акумуляції.
До цього часу ми в усіх випадках припускали, що норма акумуляції в I
$\deq{} \sfrac{1}{2}m$ I,
і що вона в різні роки лишалась стала. Ми припускали
тільки зміну відношення, що в ньому цей акумульований капітал поділяється
на сталий і змінний. При цьому ми мали три випадки:

1) І ($v \dplus{} \sfrac{1}{2}m$) \deq{} II $с$, яке, отже, менше, ніж І ($v \dplus{} m$). Це завджди
мусить бути так, інакше І не акумулював би.

2) І ($v \dplus{} \sfrac{1}{2}m$) більше, ніж ІІ $с$. В цьому випадку заміщення досягається
тим, що до ІІ $с$ долучається відповідна частина з II $m$, так що
ця сума \deq{} І ($v \dplus{} \sfrac{1}{2} m$). Тут заміщення для II є не проста репродукція
його сталого капіталу, а вже акумуляція, збільшення цього сталого капіталу
на частину його додаткового продукту, що її він обмінює на засоби
продукції І; це збілішення разом з тим включає, що II, крім того, відповідно
збільшує свій змінний капітал з свого власного додаткового продукту,

3) І ($v \dplus{} \sfrac{1}{2}m$) менше, ніж ІІ $с$. В цьому випадку II за допомогою
обміну не цілком репродукує свій сталий капітал, отже, він мусить покрити
недостачу купівлею в І. Та це не потребує дальшої акумуляції
змінного капіталу II, бо такою операцією тільки цілком репродукується
за величиною його сталий капітал. З другого боку, та частина капіталістів
І, яка акумулює лише додатковий грошовий капітал, через такий
обмін почасти вже здійснила акумуляцію такого роду.

Припущення простої репродукції, а саме, що І ($v \dplus{} m$) \deq{} IIc не лише
не узгоджується з капіталістичною продукцією, — це, однак, не виключає
того, що в промисловому циклі в 10--11 років сукупна продукція одного
якогось року часто буває менша, ніж попереднього року, отже,
порівняно з попереднім роком не відбувається навіть простої репродукції,
— але, крім того, при природному річному прирості людности проста
репродукція могла б відбуватись лише остільки, оскільки відповідно
більше, непродуктивного службового люду брало б участь у споживанні
тих 1500, що репрезентують сукупну додаткову вартість. Навпаки, акумуляція
капіталу, тобто справжня капіталістична продукція при цьому
була б неможлива. Отже, факт капіталістичної акумуляції виключає можливість того, що II $с \deq{}$ І ($v \dplus{} m$).

Однак, навіть при капіталістичній акумуляції могло б статись, що в
наслідок перебігу процесу акумуляції, який відбувався протягом цілого
ряду попередніх періодів продукції, II $с$ було б не лише рівне, але й навіть
більше, ніж І ($v \dplus{} m$). Це була б перепродукція в II, і її можна було б
вирівняти лише великим крахом, що в наслідок його капітал з II перемістився
б в І. — Відношення І ($v \dplus{} m$) до II $с$ зовсім не зміниться від того,
коли частина сталого капіталу II репродукує сама себе, як, наприклад,
\parbreak{}  %% абзац продовжується на наступній сторінці

\parcont{}  %% абзац починається на попередній сторінці
\index{i}{0408}  %% посилання на сторінку оригінального видання
посвячений. Велика промисловість розірвала вуаль, який ховав
од людей їхній власний суспільний процес продукції й робив
різні стихійно повідокремлювані галузі продукції загадками
одну для однієї і навіть для посвячених у кожну з них. Принцип
цієї промисловости — розкладати всякий процес продукції, взятий
сам по собі і насамперед незалежно від виконування його
рукою людини, на його складові елементи — цей принцип створив
цілком новітню науку технології. Строкаті, на позір позбавлені
зв’язку й закостенілі форми суспільного процесу продукції,
розпалися на свідомо пляномірні й, відповідно до бажаного корисного
ефекту, систематично повідокремлювані застосування
природознавства. Технологія відкрила так само ті нечисленні
великі основні форми руху, в яких неодмінно, не вважаючи на
всю різноманітність уживаних інструментів, відбувається вся
продуктивна діяльність людського організму, цілком так само,
як у механіці найскладніший характер машин не ховає того,
що машини є постійне повторювання простих механічних знарядь.
Сучасна промисловість ніколи не розглядає і не трактує наявну
форму якогось процесу продукції як остаточну. Тому її технічна
база є революційна, тимчасом як технічна база всіх попередніх
способів продукції суттю своєю була консервативна\footnote{
«Буржуазія не може існувати, не революціонізуючи постійно
знарядь продукції, отже, і продукційних відносин, отже, і всіх суспільних
відносин. Навпаки, незмінне збереження старого способу продукції було
першою умовою існування всіх попередніх промислових кляс. Постійний
переворот у продукції, невпинне стрясіння всіх суспільних станів, вічна
непевність і рух відзначають буржуазну епоху від усіх інших. Всі тривалі,
заіржавілі відносини з їхнім кортежем традиційно-поважаних ідей і
поглядів гинуть, усі новоутворені старіють раніше, ніж вони встигають
скостеніти. Все стале й непорушне випаровує, все святе втрачає святість,
і люди, нарешті, мусять тверезими очима подивитись на своє життєве
становище, на свої взаємні відносини». (\emph{F.~Engels і К.~Marx}: «Manifest
der Kommunistischen Partei», London 1848, S. 5. — \emph{K.~Маркс
і Ф.~Енґельс}: «Маніфест Комуністичної Партії», Партвидав 1932, стор. 30).
}. Машинами, хемічними процесами й іншими методами вона постійно
робить перевороти в технічній основі продукції і разом з тим
у функціях робітників і в суспільних комбінаціях робочого процесу.
Цим вона так само постійно революціонізує поділ праці
всередині суспільства й безупинно кидає маси капіталу й маси
робітників з однієї галузі продукції до іншої. Через це природа
великої промисловости зумовлює переміну праці, рух функцій,
всебічну рухливість робітника. З другого боку, вона репродукує
у своїй капіталістичній формі старий поділ праці з його закостенілими
спеціяльностями. Ми бачили, як ця абсолютна суперечність
[між технічними потребами великої промисловости і соціяльним
характером, що його вона набирає за капіталістичного ладу]\footnote*{
Заведене у прямі дужки ми беремо з французького видання. \emph{Ред.}
}
знищує увесь спокій, сталість, певність життєвого становища
робітника, постійно загрожуючи йому вибити з його рук разом
\parbreak{}  %% абзац продовжується на наступній сторінці

\input{i/_0409.tex}
\parcont{}  %% абзац починається на попередній сторінці
\index{i}{0410}  %% посилання на сторінку оригінального видання
з такими ферментами перевороту та їхньою метою, знищенням
старого поділу праці. Однак розвиток суперечностей певної історичної
форми продукції — це єдиний історичний шлях її розкладу
й утворення нової. «Ne sutor ultra crepidam!»\footnote*{
— Шевче, тримайся колодок своїх. \emph{Ред.}
} — це nec plus ultra\footnote*{
— найвищий ступінь, апогей. \emph{Ред.}
} ремісничої премудрости, стало страшенною дурістю від
того моменту, коли годинникар Ватт вигадав парову машину,
голяр Аркрайт — прядільну машину, ювелірний робітник Фултон
— пароплав.\footnote{
Джон Беллерс, справжній феномен в історії політичної економії,
ще наприкінці XVII віку з повного ясністю розумів доконечність знищити
теперішнє виховання й поділ праці, що породжують гіпертрофію й атрофію
на обох полюсах суспільства, хоч і в протилежному напрямі. Між
іншим, він чудово каже: «Вчитись у лінощах — це лише трохи щось ліпше,
ніж учитись лінощів\dots{} Фізична праця — це первісна божа установа\dots{}
Праця так само потрібна для здоров’я тіла, як харч для його життя;
бо ті неприємності, що їх людина уникає через лінощі, впадуть на неї
через недугу\dots{} Праця додає олії до лямпи життя, думання запалює її\dots{}
Дурненька дитяча праця (пророчий закид проти Базедових і сучасних
тупих наслідувачів їх) лишає дитячий розум дурненьким». («An idle
learning being little better than the Learning of Idlenes\dots{} Bodily Labour,
it’s a primitive institution of God\dots{} Labour being as proper for the
bodies health, as eating is for its living; for what pains a man saves by
Ease, he will find in Disease\dots{} Labour adds oyl to the lamp of life when
thinking inflames it\dots{} A childish silly employ, leaves the children’s minds
silly»). («Proposals for raising a Colledge of Industry of all useful Trades
and Husbandry», London 1696, p. 12, 14, 18).
}

Оскільки фабричне законодавство реґулює працю по фабриках,
мануфактурах тощо, воно спочатку здається тільки втручанням
у експлуататорські права капіталу. Навпаки, всяке регулювання
так званої домашньої праці\footnote{
Вона, зрештою, здебільша має характер домашньої праці і в
дрібних майстернях, як ми це бачили в мануфактурі мережива і в плетінні
з соломи і як це можна було б докладніше показати особливо на металевих
мануфактурах у Шеффілді, Бермінґемі й~\abbr{т. ін.}
} виявляється відразу ж
як прямий замах на patria potestas, тобто, висловлюючись сучасною
мовою, на батьківський авторитет, крок, що перед ним делікатний
і чутливий англійський парлямент з удаваним жахом
подавався назад. Однак сила фактів примусила, нарешті, визнати,
що велика промисловість руйнує разом з економічною основою
старої родини й відповідної їй родинної праці й самі старі родинні
відносини. Неминуче треба було оголосити право дітей. «На лихо,
— читаємо в кінцевому звіті «Children’s Employment Commission»
з 1866~\abbr{р.}, — з усіх виказів свідків ясно, що ні від кого
не треба так дуже боронити дітей обох статей, як від їхніх власних
батьків». Система безмірної експлуатації дитячої праці взагалі
й домашньої праці зокрема тим «підтримується, що батьки нестримно
й безконтрольно використовують самовільну й нещадну
владу над своїми молодими й тендітними нащадками\dots{} Не можна
давати батькам абсолютної влади робити з своїх дітей просто
машини, щоб добувати з них стільки та стільки тижневого заробітку\dots{}
\index{i}{0411}  %% посилання на сторінку оригінального видання
Діти й підлітки мають право на те, щоб закон захищав їх
проти зловживання батьківської влади, яке передчасно нищить
їхню фізичну силу й понижує їхній моральний та інтелектуальний
рівень».\footnote{
«Children’s Employment Commission. 5th Report», p. XXV,
n. 162 і 2nd Report, p. XXXVIII, n. 285, 289, p. XXXV, n. 191.
} Однак не зловживання батьківською владою
створило цю безпосередню або посередню експлуатацію недозрілих
робочих сил капіталом; навпаки, капіталістичний спосіб
експлуатації, знищивши економічну основу, що відповідала батьківській
владі, викликав зловживання цією владою. Хоч і яким
страшним і огидливим з’являється розклад старої родини всередині
капіталістичної системи, а все ж велика промисловість,
призначаючи жінкам, підліткам і дітям обох статей вирішальну
ролю в суспільно-організованому процесі продукції поза сферою
хатнього господарства, створює нову економічну основу для
вищої форми родини й відносин поміж обома статями. Певна
річ, однаково абсурдно вважати за абсолютну форму родини її
християнсько-германську форму, як і староримську, або старогрецьку,
або східню, які, зрештою, становлять історичний ряд
розвитку. Так само ясно, що склад комбінованого робочого персоналу
з індивідів обох статей і найрізнішого віку, хоч він у своїй
грубій, стихійно виниклій капіталістичній формі, де робітник
існує для процесу продукції, а не процес продукції для робітника,
є отруйне джерело морального зіпсуття й рабства, — за відповідних
умов мусить, навпаки, перетворитись на джерело гуманного
розвитку.\footnote{
«Фабрична праця могла б бути так само чиста і приємна, як домашня
праця, а то, може, навіть і більше» («Factory labour may be as
pure and as excellent as domestic labour, and perhaps more so»). («Reports
of Insp. of Fact, for 31 st October 1865», p. 127).
}

Доконечність перетворити фабричний закон із виняткового
закону для пряділень і ткалень, цих перших витворів машинового
виробництва, на загальний закон усієї суспільної продукції, випливає,
як ми вже бачили, з історичного ходу розвитку великої промисловости,
на задньому пляні якої зазнають цілковитого перевороту
традиційні форми мануфактури, ремества й домашньої
праці: мануфактура постійно перетворюється на фабрику, ремество
постійно перетворюється на мануфактуру, і, нарешті,
сфери ремества й домашньої праці в дивовижно короткий час
перетворюються на злиденні трущоби, де необмежено панує найшаленіша,
потворна капіталістична експлуатація. Дві обставини
відіграють кінець-кінцем вирішальну ролю: поперше, спостереження,
яке постійно повторюється, що капітал, коли він підпадає
під державний контроль лише на поодиноких пунктах
суспільної периферії, тим безмірніше відшкодовує себе в інших
пунктах;\footnote{
Там же, стор. 27, 32.
} подруге, волання самих капіталістів про рівність
умов конкуренції, тобто про рівні межі експлуатації праці.\footnote{
Масові приклади цього в «Reports of Insp. of Fact.».
}
\parbreak{}  %% абзац продовжується на наступній сторінці

\index{iii1}{0412}  %% посилання на сторінку оригінального видання
\section{Роль кредиту в капіталістичному виробництві}

Загальні зауваження, які ми досі зробили з приводу кредиту,
були такі:

I. Необхідність утворення кредиту для опосереднення вирівнення норми зиску або руху цього
вирівнення, на чому ґрунтується все капіталістичне виробництво.

II. Скорочення витрат циркуляції.

1) Головними витратами циркуляції є самі гроші, оскільки
вони самі мають вартість. Гроші заощаджуються за допомогою
кредиту трояким способом.

A. Тим, що для значної частини операцій вони зовсім відпадають.

B. Тим, що прискорюється циркуляція засобів циркуляції\footnote{
„Пересічна циркуляція банкнот Французького банку становила в 1812 році:
106 538 000 франків; в 1818 році: 101 205 000 франків, тимчасом як грошовий обіг,
загальна сума всіх надходжень і платежів, становив в 1812 році: 2 837 712 000
франків; в 1818 році: 9 665 030 000 франків. Отже, діяльність обігу у Франції
в 1818 році відносилась до діяльності обігу в 1812 році як 3: 1. Великим регулятором швидкості
циркуляції є кредит... Цим пояснюється, чому сильне тиснення на грошовий ринок звичайно збігається з
цілком заповненою циркуляцією“ („The Currency Theory Reviewed etc.“, стор. 65). — „Між вереснем 1833
року
і вереснем 1843 року в Великобританії виникло близько 300 банків, які випускали власні банкноти;
наслідком цього було скорочення в циркуляції банкнот
на 2 \sfrac{1}{2} мільйони; наприкінці вересня 1833 року вона становила: 36 035 244 фунтів стерлінгів, а
наприкінці вересня 1843 року: 33 518 544 фунтів стерлінгів“
(там же, стор. 53). — „Дивовижна діяльність шотландської циркуляції дає їй
змогу за допомогою 100 фунтів стерлінгів виконати таку ж кількість грошових
операцій, для якої в Англії потрібно 420 фунтів стерлінгів“ (там же, стор. 55.
Це останнє стосується тільки до технічного боку операції).
}.
Почасти це збігається з тим, що доведеться сказати в пункті\footnote{„До заснування банків сума капіталу, потрібна для функціонування знаряддя циркуляції, завжди була
більша, ніж цього вимагала дійсна товарна
циркуляція“ („\emph{Economist}“, 1845, стор. 238).}
}.
З одного боку, це прискорення — технічне; тобто при незмінних
величині й кількості дійсних товарних оборотів, які опосереднюють споживання, менша кількість грошей
або грошових знаків
виконує ту саму службу. Це зв’язано з технікою банкової справи.
З другого боку, кредит прискорює швидкість метаморфози товарів і разом з тим швидкість грошової
циркуляції.

C. Заміщенням золотих грошей паперовими.

2) Прискорення, за допомогою кредиту, окремих фаз циркуляції
або метаморфози товарів, потім метаморфози капіталу, і разом
з тим прискорення процесу репродукції взагалі. (З другого боку,
кредит дозволяє на більший строк відділяти один від одного акти
купівлі й продажу і служить через це базою спекуляції.) Скорочення резервних фондів, що можна
розглядати двояко: з одного
боку, як зменшення знаряддя циркуляції, з другого боку, як скорочення тієї частини капіталу, яка
завжди повинна існувати в грошовій формі
\parbreak{}  %% абзац продовжується на наступній сторінці

\input{i/_0413.tex}
\input{i/_0414.tex}
\parcont{}  %% абзац починається на попередній сторінці
\index{i}{0415}  %% посилання на сторінку оригінального видання
інспекторам, і одним махом збільшив таким чином сферу контролю
цих інспекторів більш ніж на \num{100.000} майстерень, — на 300 самих
тільки цегелень, — персонал інспекторів з винятковою дбайливістю
збільшили всього на вісім помічників, дарма що й перед
тим він був надто малий.\footnote{
Персонал фабричної інспекції складався з 2 інспекторів, 2 помічників
і 41 субінспекторів. Нових 8 субінспекторів призначено 1871~\abbr{р.}
Загальна сума видатків на проведення фабричних законів в Англії,
Шотландії й Ірландії становила 1871--1872~\abbr{рр.} лише \num{25.347}\pound{ фунтів стерлінґів},
включаючи й судові витрати на процеси проти порушень закону.
}

Отже, в цьому англійському законодавстві 1867~\abbr{р.} вражає,
з одного боку, накинута парляментові панівних кляс доконечність
у принципі погодитись на такі надзвичайні й широкі заходи
проти ексцесів капіталістичної експлуатації, а з другого боку,
половинчастість, неохота і mala fides,\footnote*{
— нечесність. \emph{Ред.}
} з якими парлямент
потім дійсно проводив у життя ці заходи.

Слідча комісія 1862~\abbr{р.} також запропонувала нову реґляментацію
гірничої промисловости, промисловости, яка від усіх інших
відрізняється тим, що в ній інтереси землевласників і промислових
капіталістів ідуть пліч-о-пліч. Протилежність цих двох
груп інтересів сприяла фабричному законодавству; відсутности
цієї протилежности досить для того, щоб пояснити проволікання
та викрути щодо гірничого законодавства.

Слідча комісія 1840~\abbr{р.} зробила такі жахливі і обурливі викриття
і викликала такий скандал на всю Европу, що парлямент
мусив рятувати своє сумління Mining Act’oм 1842~\abbr{р.}, в якому
він обмежився забороною праці під землею для жінок і дітей,
молодших від 10 років.

Потім 1860~\abbr{р.} видано Mines’ Inspection Act,\footnote*{
— закон про інспекцію над копальнями. \emph{Ред.}
} що згідно з ним
спеціяльно призначені державні урядовці мають наглядати за гірничими
підприємствами, і дітей між 10 і 12 роками не можна в
тих підприємствах вживати до праці, якщо вони не мають шкільного
посвідчення або не відвідують школи протягом певного
числа годин. Цей закон лишився цілком мертвою буквою через
на сміх мале число призначених інспекторів, мізерність їхніх
уповноважень та інші причини, які докладніше з’ясується в
дальшому викладі.

Одна з найновіших Синіх Книг про гірничі підприємства —
це «Report from the Select Committee on Mines, together with\dots{}
Evidence, 23 July 1866». Це — праця комітету, складеного з
членів нижньої палати й уповноваженого закликати й вислухувати
свідків; товстий том in folio, де сам «Report» має всього
лише п’ять рядків такого змісту: комітет нічого не може сказати,
треба переслухати ще більше свідків!

Спосіб допиту свідків нагадує cross examinations\footnote*{
— перехресний допит. \emph{Ред.}
} по англійських
судах, де адвокат безсоромними і заплутаними перехресними
\index{i}{0416}  %% посилання на сторінку оригінального видання
запитами намагається спантеличити свідка й перекрутити
його слова. Тут у ролі адвокатів виступають сами члени парламентської
слідчої комісії, між ними власники й експлуататори копалень;
свідки тут робітники копалень, здебільша кам’яновугільних.
Ця ціла фарса надто характеристична для духу капіталу, так що
не можна не подати тут декілька витягів. Для легшого огляду
я подаю результати слідства й~\abbr{т. ін.} за рубриками. Нагадую, що
питання і обов’язкові відповіді в англійських Синіх Книгах
нумеровані, і що свідки, чиї свідчення тут цитується, є робітники
кам’яновугільних копалень.

1. Праця дітей від 10 років по копальнях. Праця разом з
неминучим ходінням від і до копалень триває звичайно 14 —
15 годин, винятково довше, від 3, 4, 5 години ранку до 4--5 години
вечора (№№ 6, 452, 83). Дорослі робітники працюють двома
змінами, або по 8 годин, але для підлітків, щоб заощадити на
видатках, такої зміни нема (№№ 80, 203, 204). Малих дітей уживають
головно щоб відчиняти й зачиняти двері в різних відділах
копальні, а старших дітей — до тяжкої роботи: перевозити вугілля
й~\abbr{т. ін.} (№№ 122, 739, 1747). Довгий робочий день під землею
триває до 18 або 22 року життя, коли відбувається перехід до
власне копальневої праці (№ 161). Дітей і підлітків тепер тяжче
мордують працею, ніж колибудь у попередні часи (№№ 1663 —
67). Копальневі робітники майже одноголосно вимагають парляментського
закону про заборону копальневої праці для дітей,
молодших за 14 років. Але ось Гессей Вівіян (сам експлуататор
копальні) питає: «Чи не залежить ця вимога від більших або
менших злиднів батьків?» — А містер Брюс: «Чи це не жорстоко,
якщо батько помер або покалічений тощо, відбирати в родини
цей ресурс? Адже ця заборона мусить мати силу, як загальне
правило. Чи хочете ви підземну працю дітей до 14 років заборонити
в усіх випадках?» Відповідь: «В усіх випадках» (№№ 107
до 110). Вівіян: «А якщо працю дітей до 14 років по копальнях
заборонять, то чи не посилатимуть батьки дітей на фабрики
тощо? — Як правило, ні» (№ 174). Робітник: «Відчиняти й зачиняти
двері, здається, легко. Але це виснажна праця. Не кажучи
вже про постійний протяг, дитина сидить там немов у в’язниці,
цілком так, наче в темній тюремній камері». Буржуа Вівіян:
«А не може дитина, вартуючи при дверях, читати, якщо вона
матиме світло? — ІІоперше, вона мусила б купити собі свічку.
Але, крім того, їй цього і не дозволили б. Її поставили, щоб пильнувала
справи, вона має виконувати певний обов’язок. Я ніколи
не бачив, щоб якабудь дитина читала в копальні» (№№ 141--160).

2. Виховання. Копальневі робітники вимагають закона про
обов’язкове навчання дітей, як на фабриках. Вони заявляють,
що той пункт закону 1860~\abbr{р.}, який вимагає шкільної посвідки для
того, щоб вживати до праці дітей 10--12 років, є чисто ілюзоричний.
«Педантична» процедура допитування капіталістичними
слідчими стає тут справді забавною. (№ 115). «Чи цей закон більше
потрібний проти підприємців, чи проти батьків? — Проти тих
\parbreak{}  %% абзац продовжується на наступній сторінці

\parcont{}  %% абзац починається на попередній сторінці
\index{iii1}{0417}  %% посилання на сторінку оригінального видання
в одних суперечність усунена негативно, а в других — позитивно.

Досі ми розглядали розвиток кредитної системи — і вміщене
в ній приховане скасування (Aufhebung) капіталістичної власності — головним чином щодо промислового
капіталу. В дальших розділах ми розглядаємо кредит щодо капіталу, що дає процент, як
такого, а також його вплив на цей останній, як і ту форму, що
її він при цьому набирає; при цьому нам взагалі доведеться ще
зробити декілька зауважень специфічно економічного характеру.

Але попереду ще таке:

Якщо кредитна система виступає як головна підойма перепродукції та надмірної спекуляції в торгівлі,
то тільки тому,
що процес репродукції, який по своїй природі є еластичний,
форсується тут до крайніх меж, і при тому форсується тому,
що значна частина суспільного капіталу застосовується його
невласниками, які через це пускаються в справи цілком інакше,
ніж власник, який, оскільки він функціонує сам, боязливо зважує
межі свого приватного капіталу. Це тільки показує, що зростання вартості капіталу, основане на
антагоністичному характері
капіталістичного виробництва, допускає дійсний, вільний розвиток тільки до певного пункту, отже, в
дійсності утворює для виробництва імманентні окови й межі, які постійно прориваються
кредитною системою.\footnote{
\emph{Th. Chalmers}. [„On Political Economy etc.“ London 1832.)
} Тому кредитна система прискорює матеріальний розвиток продуктивних сил і
утворення світового ринку,
доведення яких як матеріальних основ нової форми виробництва
до певного ступеня розвитку є історичним завданням капіталістичного способу виробництва. Одночасно
кредит прискорює насильні вибухи цієї суперечності, кризи, і тим самим посилює елементи розкладу
старого способу виробництва.

Імманентний кредитній системі двобічний характер: з одного
боку, розвивати рушійну силу капіталістичного виробництва,
збагачення експлуатацією чужої праці, до найчистішої і найколосальнішої системи гри й шахрайства і
дедалі більше обмежувати число тих небагатьох, що експлуатують суспільне багатство; а з другого
боку, становити перехідну форму до нового
способу виробництва, — ця двобічність і надає головним провісникам кредиту від Ло до Ісаака Перейри
їхнього приємного мішаного характеру шахрая і пророка.

\input{i/_0418.tex}
\parcont{}  %% абзац починається на попередній сторінці
\index{i}{0419}  %% посилання на сторінку оригінального видання
що я ніде не знайшов нічого, хоч трохи подібного до жіночої
праці в копальнях. Це — чоловіча праця і праця для дужих
чоловіків. Кращі елементи поміж копальневими робітниками,
які силкуються піднестися, стати людьми, замість знаходити підтримку
в своїх дружин, через них занепадають». Після цілого
ряду перехресних запитань цих буржуа виявляється, нарешті,
таємниця їхнього «милосердя» до вдовиць, бідних родин і т. ін.:
«Власник копальні призначає певних джентельменів для головного
догляду; останні, щоб заробити панської похвали, додержують
політики зробити все якомога найекономніше, і дівчата-робітниці
одержують від 1\shil{ шилінґа} до 1\shil{ шилінґа} 6\pens{ пенсів} денно там,
де чоловік мусив би одержувати 2\shil{ шилінґи} 6\pens{ пенсів}» (№ 1816).

4. Жюрі для огляду мерців (№ 360). «Щодо слідства coroner’ів\footnote*{
В Англії — слідчий для огляду мерців у випадках наглої смерти. \emph{Ред.}
}
у ваших округах, чи задоволені робітники судовими процесами,
коли трапляються нещасливі випадки? — Ні, незадоволені»
(№ 816). «Чому ні? — Особливо тому, що на членів жюрі призначають
людей, які абсолютно нічого не тямлять у копальнях. Робітників
ніколи не закликають, хібащо лише як свідків. Загалом
до жюрі закликають сусідніх крамарів, які є під впливом власників
копалень, їхніх покупців, і не розуміють навіть технічних
висловів свідків. Ми бажаємо, щоб копальневі робітники становили
частину членів жюрі. Звичайно присуд суперечить виказам
свідків» (№ 378). «Чи не повинні жюрі бути безсторонніми? —
Так» (№ 379). «А чи будуть робітники безсторонніми? — Я не
бачу ніяких мотивів, чому б їм не бути безсторонніми. Вони
добре розуміють справу» (№ 380). «А чи не виявлятимуть вони
нахилу до того, щоб виносити несправедливо суворі присуди в
інтересах робітників? — Ні, я не думаю».

5. Фалшива міра та вага й т. ін. Робітники вимагають тижневої
виплати замість двотижневої, міряння цебер вагою, а не
на об’єм, захисту проти вживання фалшивої ваги і т. ін.
(№ 1071). «Коли цебра по-шахрайському збільшують, то робітник
може ж, повідомивши наперед за 14 день, покинути працю
в копальні? — Алеж, прийшовши на іншу копальню, він і там
знайде те саме» (№ 1072). «Але він все ж може покинути те місце,
де йому чинять кривду? — Алеж повсюди панує те саме» (№ 1073).
«Але робітник завжди може покинути своє місце, попередивши
про це за 14 днів? — Так». Цього досить!

6. Копальнева інспекція. Робітники страждають не тільки
від нещасливих випадків через вибух газів (№ 234 і далі). «Нам
доводиться так само нарікати на погану вентиляцію в кам’яновугільних
копальнях, бо люди в них ледве можуть дихати;
через це вони стають непридатними до будь-якої праці. Так,
наприклад, саме тепер у тій частині копалень, де я працюю,
отруйне повітря загнало багатьох людей на цілі тижні до ліжка.
У головних ходах здебільша повітря є досить, але мало його
\parbreak{}  %% абзац продовжується на наступній сторінці

\parcont{}  %% абзац починається на попередній сторінці 
\index{i}{0420}  %% посилання на сторінку оригінального видання 
саме по тих місцях, де ми працюємо. Коли ж якийсь робітник
надішле скаргу з приводу вентиляції до інспектора, то його звільняють,
і він стає вже «поміченим» робітником, що й деінде не
знайде собі ніякої праці. «Mining inspecting Act» 1860 р. —
це просто клапоть паперу. Інспектор — число їх занадто вже
мале — мабуть лише один раз на сім років формально відвідує
копальню. Наш інспектор — це цілком неспроможна сімдесятлітня
людина, і він має інспектувати більш ніж 130 кам’яновугільних
копалень. Опріч більшого числа інспекторів, нам треба
субінспекторів» (№ 280). «Тоді уряд повинен тримати таку армію
інспекторів, щоб вони сами, без інформації самих робітників,
могли робити все те, чого ви вимагаєте? — Це неможливо, але
вони повинні сами приходити в копальні по інформації» (№ 285).
«Чи не гадаєте ви, що наслідок цього був би такий, що відповідальність
(!) за вентиляцію й т. ін. з власників копалень спала б
на державних урядовців? — Зовсім ні; їхній обов’язок був би
примушувати виконувати наявні вже закони» (№ 294). «Коли ви
говорите про субінспекторів, то чи не маєте ви на думці людей
з меншою платою й нижчої категорії, аніж сучасні інспектори? —
Я зовсім не бажаю нижчих людей, тоді як ви можете дати кращих»
(№ 295). «Чи хочете ви більшого числа інспекторів або
людей нижчої кляси ніж інспектори? — Нам треба людей, які
сами товклися б по копальнях, людей, що не тремтіли б за свою
шкуру» (№ 296). «Коли б ваше бажання мати інспекторів нижчого
сорту здійснилося, то чи не постане небезпека з того, що
вони недосить здібні? — Ні, це є справа уряду призначити
здібних людей». Нарешті, цей спосіб допиту стає навіть для
президента слідчої комісії надто безглуздим: «Ви хочете, — питає
він, встряваючи, — людей практичних, що сами оглядали б
копальні й повідомляли б інспектора, який потім міг би використати
свої ширші знання» (№ 531). «Чи не спричинить вентиляція
по всіх цих старих копальнях багато видатків? — Так,
затрати, може, і зростуть, але людське життя буде захищене»
(№ 581). Якийсь копальневий робітник протестує проти 17 відділу
закону 1860 р.: «Тепер, коли інспектор копалень знаходить
якусь частину копальні в непридатному для праці стані, він
мусить про це повідомити власника копальні й міністра внутрішніх
справ. Після цього власник копальні має двадцять днів на
роздум; наприкінці цих 20 днів він може відмовитися зробити
будь-які зміни. Коли він відмовляється зробити зміни, то він
мусить писати до міністра внутрішніх справ і запропонувати
йому п’ять гірничих інженерів, з-поміж яких міністер мусить
призначати третейських суддів. Ми запевняємо, що в цьому випадку
сам власник копальні має можливість призначати своїх
власних суддів» (№ 586). Буржуа-екзамінатор, сам власник
копалень: «Це — суто спекулятивне заперечення» (№ 588). «Отже,
ви дуже невисокої думки про чесність гірничих інженерів? —
Я кажу, що це велика кривда й велика несправедливість»
(№ 589). «Чи не займають гірничі інженери такого офіціяльного
\parbreak{}  %% абзац продовжується на наступній сторінці

\parcont{}  %% абзац починається на попередній сторінці 
\index{i}{0421}  %% посилання на сторінку оригінального видання 
місця, що підносило б їхні рішення понад небезсторонність, якої
ви боїтесь? — Я відмовляюсь відповідати на питання про особистий
характер цих людей. Я переконаний, що в багатьох випадках
вони поводяться дуже небезсторонньо, і що треба відібрати
в них цю владу там, де на карту ставиться людське життя».
В того самого буржуа вистачає безстидства спитати: «Чи не гадаєте
ви, що й власники копалень мають утрати в наслідок вибухів?»
— І, нарешті, ще (№ 1042): «Чи не можете ви, робітники,
сами захистити свої власні інтереси, не вдаючися по допомогу
до уряду? — Ні». — 1865 р. у Великобрітанії було 3.2І7 кам’яновугільних
копалень і — 12 інспекторів. Один власник копалень
у Йоркшірі («Times», з 26 січня 1867 р.) сам обчислює, що
інспектори, коли навіть залишити осторонь їхні суто бюрократичні
справи, які забирають у них увесь час, могли б відвідати
кожну копальню лише один раз на десять років. Не диво, що
останніми роками (особливо ж і 1866 і 1867 рр.) катастрофи проґресивно
збільшувалися числом і своїми розмірами (іноді число
жертов становило 200—300 робітників). Оце вам краса «вільної»
капіталістичної продукції!

У всякому разі закон 1872 р., хоч які в нього хиби, є перший
закон, що реґулює години праці дітей, вживаних по копальнях,
і до певної міри робить відповідальними за так звані нещасливі
випадки експлуататорів і власників копалень.

Королівська комісія 1867 р. для розсліду умов праці дітей,
підлітків та жінок у рільництві опублікувала декілька дуже
важливих звітів. Зроблено різні спроби застосувати принципи
фабричного законодавства, у змодифікованій формі, до рільництва,
але всі вони досі кінчалися повною невдачею. Але на одно
я маю тут звернути увагу, а саме на існування непереможної
тенденції до загального застосування цих принципів.

Якщо, з одного боку, загальне поширення фабричного законодавства
як засобу фізичного й духовного захисту робітничої
кляси стало неминучим, то, з другого боку, як це вже зазначено,
воно повсюдно поширює й прискорює перетворення розпорошених
процесів праці карликового маштабу на комбіновані процеси
праці великого суспільного маштабу, отже, повсюдно поширює
й прискорює концентрацію капіталу й самовладне панування
фабричного режиму. Воно руйнує всі старовинні й переходові
форми, за якими панування капіталу почасти ще ховається і
замінює їх прямим, незахованим пануванням капіталу. Цим
воно повсюдно поширює й безпосередню боротьбу проти цього
панування. Тимчасом як в індивідуальних майстернях воно примушує
до одноманітности, реґулярности, порядку й економії, —
воно, в наслідок величезного поштовху, що його дають розвиткові
техніки обмеження й реґулювання робочого дня, збільшує
анархію й катастрофи капіталістичної продукції в цілому, збільшує
інтенсивність праці й конкуренцію машин з робітником. Разом
із сферами дрібного виробництва та домашньої праці воно знищує
останні притулки «зайвих» робітників, а тим самим і на-
\parbreak{}  %% абзац продовжується на наступній сторінці

\input{i/_0422.tex}
\parcont{}  %% абзац починається на попередній сторінці
\index{i}{0423}  %% посилання на сторінку оригінального видання
машин у рільництві здебільша вільний від шкідливого фізичного
впливу, справлюваного на фабричного робітника\footnote{
Докладний опис машин, уживаних в англійському рільництві,
знаходимо в «\textgerman{Die landwirtschaftlichen Geräte und Maschinen Englands,
von Dr.~W.~Hamm}». 2 Auflage 1856. У своєму нарисі про розвиток англійського
рільництва пан Гам надто некритично йде за паном Леонс де
Лявернь. [До четвертого видання. — Розуміється, тепер цей нарис застарів.
— \emph{Ф.~Е}].
},
то в «утворенні зайвих» робітників вони тут діють ще інтенсивніше
і не маючи собі в цьому опору, як ми це пізніше
побачимо в подробицях. У графствах Кембрідж і Суффолк, приміром,
площа обробленої землі за останні двадцять років дуже
поширилася, тимчасом як сільська людність за той самий період
не тільки відносно, а й абсолютно зменшилася. У Сполучених
штатах Північної Америки рільничі машини заміняють робітників
покищо лише у можливості, тобто вони дозволяють продуцентові
обробляти більшу площу землі, але не проганяють дійсно
занятих робітників. В Англії й Велзі число осіб, що працювали
у фабрикації рільничих машин, становило 1861~\abbr{р.} \num{1.034}, тим часом
як число рільничих робітників, занятих коло парових і робочих
машин, становило лише \num{1.205}.

У сфері рільництва велика промисловість діє якнайбільш
революційно в тому розумінні, що вона нищить твердиню старого
суспільства, «селянина», і висуває на його місце найманого робітника.
Таким чином потреби соціяльного перевороту й соціяльні
противенства\footnote*{
У французькому виданні тут замість «соціяльні противенства»
cказано «клясова боротьба». \emph{Ред.}
} на селі доходять такого ж рівня, як і в місті.
Замість найрутиннішого і найнераціональнішого виробництва
постає свідоме технологічне застосування науки. Капіталістичний
спосіб продукції завершує розрив того первісного родинного
зв’язку рільництва з мануфактурою, який об’єднував дитинячі,
нерозвинуті форми одного і другої. Але разом із цим цей спосіб
продукції утворює матеріяльні передумови нової, вищої
синтези, а саме спілки рільництва і промисловости, на основі
їхніх антагоністично розвинених форм. Капіталістична продукція
в міру того, як перевага міської людности, яку вона стягає до
великих центрів, щораз більшає, — нагромаджує, з одного боку,
історичну силу руху суспільства, а з другого боку, перешкоджає
обмінові речовин між людиною й землею, тобто перешкоджає
повертанню ґрунтові тих його складових частин, які людина
зужила у формі харчових засобів і одягу, отже, вона порушує
вічну природну умову тривалої родючости ґрунту. Цим самим
вона одночасно руйнує фізичне здоров’я міських робітників і
інтелектуальне життя сільських робітників\footnote{
«Ви розділяєте народ на два ворожі табори: на необтесаних мужиків
і слабовитих карликів. Боже мій! Нація, що розділилася на рільничі
й торговельні інтереси, вважає себе за здорову й називає себе
навіть освіченою й цивілізованою не наперекір, а саме в наслідок цього
потворного й неприродного поділу». («You divide the people into two
hostile camps of clownish boors and emasculated dwarfs. Good heavens!
a nation divided into agricultural and commercial interests calling itself
sane, nay styling itself enlightened and civilized, not only in spite of,
but in consequence of this monstrous and unnatural division»). (\emph{David
Urquhart}: «Familiar Words», London 1855, p. 119). Це місце показує
одночасно силу і слабість такого роду критики, яка вміє обмірковувати
та ганьбити сучасність, але не вміє її зрозуміти.
}.  Але, руйнуючи
спонтанейно посталі умови цього обміну речовин, капіталістична
\index{i}{0424}  %% посилання на сторінку оригінального видання
продукція разом з тим примушує відновити його систематично
як закон, що реґулює суспільну продукцію, і у формі, адекватній
повному розвиткові людини. У рільництві, як і в мануфактурі,
капіталістичне перетворення продукційного процесу є
разом з тим мартиролог продуцентів, засіб праці є разом з тим
засіб поневолення, засіб експлуатації й засіб павперизації робітника,
суспільна комбінація процесу праці є разом з тим організоване
пригнічення індивідуальної життьової сили робітника,
його волі й самостійности. Розпорошеність сільських робітників
по великих просторах ламає одночасно силу їхнього
опору, тимчасом як концентрація міських робітників підносить
її. У сучасному рільництві, так само і в міській промисловості
підвищення продуктивної сили й більша ефективність праці
купується ціною нищення й виснаження самої робочої сили.
І кожний проґрес капіталістичного рільництва — це не тільки
проґрес у вмілості грабувати робітника, але разом з тим і у вмілості
грабувати ґрунт, кожний проґрес у піднесенні родючости
його на даний час — це разом з тим проґрес у руйнуванні тривалих
джерел цієї родючости. Що більш якась країна, як, приміром,
Сполучені штати Північної Америки, виходить від великої
промисловости як бази свого розвитку, то швидший цей
процес руйнування\footnote{
Порівн. \emph{Liebig}: «Die Chemie in ihrer Anwendung auf Agrikultur
und Physiologie». 7. Auflage 1862, особливо також «Einleitung
in die Naturgesetze des Feldbaues» у першому томі. Вияснення неґативного
боку сучасного рільництва з погляду природознавства — це одна
з невмирущих заслуг Лібіґа. Його історичні нариси з історії рільництва,
хоч вони й не без грубих помилок, також висвітлюють деякі питання.
Можна пожалкувати, що він навмання зважується висловлювати
ось які погляди: «Продовжуване далі роздрібнювання й частіше переорювання
підвищує обмін повітря всередині поруватих частин землі, збільшує
й поновлює поверхню цих частин землі, на яку має впливати повітря;
але легко зрозуміти, що додатковий здобуток із поля не може бути,
пропорційний до витраченої на поле праці, а зростає в куди меншій пропорції».
«Цей закон, — додає Лібіґ, — уперше висловив Дж.~Ст.~Мілл у своїм
«Principles of Political Economy», v. I, p. 17, ось так: «Те, що продукт
землі за інших рівних умов зростає в дедалі меншій пропорції порівняно
до зростання числа вживаних робітників (навіть відомий закон Рікарда
пан Мілл повторює тут у фалшивому формулюванні, бо через те, що «зменшення
числа вживаних робітників» («the decrease of the labourers employed»)
в Англії постійно відбувалося поруч із проґресом рільництва, закона,
вигаданого для Англії і в Англії, не можна було б застосувати, принаймні
в Англії), — це універсальний закон рільничої промисловости». Це, — каже
далі Лібіґ, — річ досить дивна, бо Міллові була невідома основа цього закону»
(\emph{Liebig}, там же, книга 1, стор. 143 і примітка). Не кажучи вже про
помилкове тлумачення слова «праця», під яким Лібіґ розуміє щось інше,
ніж політична економія, в усякому разі «річ досить дивна», що він із Дж.~Ст.~Мілла робить першого оповісника теорії, яку Джемс Андерсон опублікував
уперше за часів А.~Сміса й повторював її в різних творах аж до
початку XIX віку, теорії, яку 1815~\abbr{р.} присвоїв собі Малтуз, взагалі
майстер у пляґіятах (ціла його теорія залюднення є безсоромний пляґіят),
яку Вест розвинув одночасно з Андерсоном і незалежно від нього,
яку Рікардо 1817~\abbr{р.} зв’язав із загальною теорією вартости й яка від
того часу під ім’ям Рікарда обійшла ввесь світ, яку 1820~\abbr{р.} звульґаризував
Джемс Мілл (батько Дж.~Ст.~Мілла) і яку, нарешті, повторює, між
іншим, і пан Дж.~Ст.~Мілл як шкільну догму, що встигла вже зробитися
банальною фразою. Безперечно, Дж.~Ст.~Мілл завдячує свій, в усякому
разі, «дивний» авторитет майже виключно подібним qui pro quo.
}. Тому капіталістична продукція розвиває
техніку й комбінування суспільного процесу продукції, але
лише так, що вона разом з цим підриває джерела виникнення
всякого багатства: землю й робітника.

\index{i}{0425}  %% посилання на сторінку оригінального видання

\chapter{Продукція абсолютної і відносної
Додаткової вартости}

\section{Абсолютна й відносна додаткова вартість}

Спочатку ми розглядали процес праці абстрактно (див. п’ятий
розділ), незалежно від його історичних форм, як процес між
людиною і природою. Там ми казали: «Коли розглядати цілий
процес праці з погляду його результату, [продукту]\footnote*{
Заведене у прямі дужки взято з французького видання. \emph{Ред.}
}, то і засоби
праці, і предмет праці, одне й друге, з’являються як засоби
продукції, а сама праця — як продуктивна праця». У примітці
сьомій був додаток: «Цього визначення продуктивної праці,
що випливає з погляду простого процесу праці, зовсім недосить
для капіталістичного процесу продукції». Це нам треба тут
розвинути далі.

Оскільки процес праці є суто індивідуальний, той самий
робітник сполучає всі ті функції, що пізніше розділяються. В індивідуальному
присвоюванні предметів природи для своїх життєвих
цілей він контролює сам себе. Пізніше його контролюють.
Поодинока людина не може впливати на природу, не пускаючи
в рух своїх мускулів під контролем свого власного мозку. Як у
системі природи голова й руки належать одне до одного, так само
і процес праці сполучає працю голови й працю рук. Пізніше ті
праці розділяються аж до ворожої протилежности. Продукт перетворюється
\index{i}{0426}  %% посилання на сторінку оригінального видання
взагалі з безпосереднього продукту індивідуального
продуцента на суспільний, на спільний продукт колективного
робітника, тобто на продукт комбінованого робочого персоналу,
що його члени беруть ближчу або дальшу участь в обробленні
предмету праці. Тому з кооперативним характером самого процесу
праці неодмінно ширшає поняття продуктивної праці та
її носія, продуктивного робітника. Щоб працювати продуктивно,
йому тепер уже не треба самому прикладати рук, а досить бути
органом колективного робітника, виконувати одну якусь його
частинну функцію. Наведене вище первісне визначення продуктивної
праці, виведене з самої природи матеріяльної продукції,
завжди зберігає свою силу для колективного робітника, розглядуваного
як ціле. Але воно вже не має сили для кожного з його
членів, взятого окремо.

Але, з другого боку, поняття продуктивної праці вужчає.
Капіталістична продукція є не тільки продукція товару, вона
з самої суті своєї є продукція додаткової вартости. Робітник
продукує не для себе, а для капіталу. Тому вже недосить того,
що він взагалі продукує. Він мусить продукувати додаткову
вартість. Тільки той робітник продуктивний, що продукує додаткову
вартість для капіталіста, або служить для самозростання
вартости капіталу. Так, шкільний учитель, якщо можна
взяти приклад з-поза сфери матеріяльної продукції, є продуктивний
робітник тоді, коли він не тільки обробляє дитячі голови,
але й себе витрачає, щоб збагатити підприємця. Те, що останній
вклав свій капітал не у фабрику ковбас, а у фабрику навчання,
нічого не змінює в цьому відношенні. Тому поняття продуктивного
робітника ні в якому разі не містить у собі тільки відношення
між діяльністю та корисним ефектом, поміж робітником та продуктом
праці; воно містить у собі ще й специфічно-суспільне,
історично постале продукційне відношення, яке робить робітника
безпосереднім засобом зростання вартости капіталу. Тому бути
продуктивним робітником — це не щастя, а біда. У четвертій
книзі цієї праці\footnote*{
Мова йде про теорії додаткової вартости, що їх Маркс гадав видати
як четверту книгу «Капіталу». \emph{Ред.}
}, що розглядає історію теорії, ми ближче побачимо,
що клясична політична економія вже віддавна зробила
продукцію додаткової вартости характеристичною вирішальною
ознакою продуктивного робітника. Тому із зміною її розуміння
природи додаткової вартости змінюється й її визначення продуктивного
робітника. Так, фізіократи заявляють, що тільки
рільнича праця продуктивна, бо тільки вона дає додаткову вартість.
Але для фізіократів додаткова вартість існує виключно
у формі земельної ренти.

Здовження робочого дня поза той пункт, коли робітник спродукував
би лише еквівалент вартости своєї робочої сили, і присвоєння
цієї додаткової праці капіталом — оце є продукція
абсолютної додаткової вартости. Продукція абсолютної додаткової
\parbreak{}  %% абзац продовжується на наступній сторінці

\input{i/_0427.tex}
\parcont{}  %% абзац починається на попередній сторінці
\index{i}{0428}  %% посилання на сторінку оригінального видання
форми репродукується подекуди і на базі великої промисловости,
хоч і з цілком зміненою фізіономією.

Коли, з одного боку, для продукції абсолютної додаткової
вартости досить лише формальної підпорядкованости праці капіталові,
досить, наприклад, того, щоб ремісники, які раніш працювали
на самих себе абож як підмайстри цехового майстра,
стали тепер як наймані робітники під безпосередній контроль
капіталіста, то, з другого боку, виявилося, що методи продукції
відносної додаткової вартости є разом з тим методи продукції
абсолютної додаткової вартости. Аджеж безмірне здовження
робочого дня виявилось як найхарактеристичніший продукт великої
промисловости. Взагалі специфічно-капіталістичний спосіб
продукції перестає бути простим засобом продукувати відносну
додаткову вартість, скоро тільки він опановує цілу галузь
продукції, і ще більше — скоро він опановує всі вирішальні
галузі. Він стає тепер загальною, суспільно-панівною формою
подукційного процесу. Як осібна метода продукувати відносну
додаткову вартість, він діє ще лише остільки, оскільки, поперше,
захоплює галузі продукції, досі лише формально підпорядковані
капіталові, отже, поширюючи сферу свого впливу; подруге,
остільки, оскільки галузі промисловости, що підпали вже під
його руку, постійно революціонізуються через зміну метод продукції.

З певного погляду ріжниця між абсолютною і додатковою вартістю
видається взагалі ілюзорною. Відносна додаткова вартість
є абсолютна, бо вона зумовлює абсолютне здовження робочого
дня поза робочий час, доконечний для існування самого робітника.
Абсолютна додаткова вартість є відносна, бо вона зумовлює:
розвиток такої продуктивности праці, що дозволяє обмежити
доконечний робочий час певною частиною робочого дня. Але
коли звернути увагу на рух додаткової вартости, то ця позірна
тотожність зникає. Скоро тільки капіталістичний спосіб продукції
виник і став загальним способом продукції, ріжниця між абсолютною
й відносною додатковою вартістю стає відчутною, коли
йдеться про підвищення норми додаткової вартости взагалі.
Коли припустити, що робочу силу оплачується за її вартістю,
то ми опиняємось перед такою альтернативою: при даній продуктивності
праці й нормальному ступені її інтенсивности норму
додаткової вартости можна підвищити тільки через абсолютне
здовження робочого дня; з другого боку, при даних межах робочого
дня норму додаткової вартости можна підвищити лише через
зміну відносних величин його складових частин, доконечної
праці й додаткової праці, а це, з свого боку, має собі за передумову
зміну продуктивности або інтенсивности праці, якщо
заробітна плата не повинна впасти нижче вартости робочої сили.

Якщо робітник потребує всього свого часу на те, щоб продукувати
засоби існування, потрібні для утримання його самого і
його родини, то йому не лишається вже часу задурно працювати
на третіх осіб. Без певного ступеня продуктивности праці в робітника
\index{i}{0429}  %% посилання на сторінку оригінального видання
не може бути такого вільного часу, без такого надлишкового
часу не може бути додаткової праці, а тому й капіталістів,
але також не може бути й рабовласників, февдальних баронів,
одне слово, жодної кляси великих власників.\footnote{
«Саме існування капіталістичних підприємців як осібної кляси
залежить від продуктивности праці» («The very existence of the master-capitalists
as a distinct class is dependent on the productiveness
of industry») - (\emph{Ramsay}: «An Essay on the Distribution of Wealth»,
Edinburgh 1836, p. 206). «Коли б праці кожної людини вистачало лише
для продукції її власних засобів існування, то не могло б бути й власности»
(«If each man’s labour were but enough to produce his own food,
there could be no property»). (\emph{Ravenstone}: «Thoughts on the Funding
System», London 1824, p. 14, 15).
}

Таким чином можна говорити про природну базу додаткової
вартости, але лише в тому цілком загальному розумінні, що в
природі немає жодної абсолютної перешкоди, яка б не дозволяла
одній людині звалювати з себе на іншу людину працю, потрібну
для її власного існування, наприклад, так само, як у природі
не існує жодних абсолютних перешкод для того, щоб одна людина
вживала для харчування м’яса іншої.\footnoteA{
На основі нещодавно зробленого обчислення лише в досліджених
уже частинах землі живе ще щонайменше чотири мільйони канібалів.
} Ні в якому разі не
слід, як це іноді робилося, сполучати містичні уявлення з
цим стихійним розвитком продуктивности праці. Тільки тоді,
коли люди тяжкою працею вибилися з свого первісного тваринного
стану, отже, коли сама їхня праця до деякої міри вже є
усуспільнена, — лише тоді постають відносини, за яких додаткова
праця однієї людини стає умовою існування іншої. На початках
культури здобуті продуктивні сили праці незначні, але так само
незначні й потреби, що розвиваються разом з розвитком засобів
для задоволення тих потреб та залежно від цього розвитку. Далі,
на тих початках культури частина суспільства, що живе з чужої
праці, є величина зникомо мала супроти маси безпосередніх продуцентів.
З проґресом суспільної продуктивної сили праці ця
частина зростає абсолютно й відносно.\footnote{
У диких індіян Америки мало не все належить робітникові, 99\%
продуктів припадає робітникові; в Англії на робітника не припадає й
двох третин» («Among the wild Indians in America, almost every thing
is the labourer’s, 99 parts of an hundred are to be put upon the account of
Labour; In England, perhaps the labourer has not \sfrac{2}{3}»). («The Advantages
of the East-India Trade etc.», London 1720, p. 73).
} Зрештою, капіталістичне
відношення постає на економічному ґрунті, який є продукт
довгого процесу розвитку. Наявна продуктивність праці,
з якої воно виходить як з основи, не є дар природи, а дар історії,
яка охоплює тисячі століть.

Якщо абстрагуватися від більш або менш розвиненої форми
суспільної продукції, то залишається, що продуктивність праці
зв’язана з природними умовами. Всі ці умови можна звести до
природи самої людини, як от раса й т. ін., та до природи, що оточує
людину. Зовнішні природні умови розпадаються з економічного
погляду на дві великі кляси: природне багатство на засоби
\parbreak{}  %% абзац продовжується на наступній сторінці

\parcont{}  %% абзац починається на попередній сторінці 
\index{i}{0430}  %% посилання на сторінку оригінального видання 
існування, отже, родючість ґрунту, багаті рибою води й т. ін.,
і природне багатство на засоби праці, як от водоспади, судноплавні
річки, дерево, металі, вугілля й т. ін. На початках культури
має переважне значення перша форма природного багатства,
на вищих ступенях розвитку — друга форма. Порівняйте,
наприклад, Англію з Індією, або — в античному світі — Атени
й Корінт з країнами на узбережжі Чорного моря.

Що менше число природних потреб, які абсолютно треба
задовольняти, і що більша природна родючість ґрунту та сприятливість
підсоння, то менший робочий час, доконечний для утримання
й репродукції продуцента. Отже, то більший може бути й
надлишок його праці на інших супроти його праці на самого себе.
Так, уже Діодор зауважує про давніх єгиптян: «Просто неймовірно,
як мало праці й витрат коштує їм виховання їхніх дітей.
Вони варять для них першу-ліпшу просту страву; дають їм
їсти й долішню частину папіруса, яку можна присмажити на
вогні, та коріння й стебла болотяних рослин, почасти сирі, почасти
варені й печені. Діти здебільша ходять без взуття й одягу,
бо повітря там дуже м’яке. Тому дитина коштує своїм батькам,
доки виросте, в цілому не більше, як двадцять драхм. Цим головне
й можна пояснити, що в Єгипті така численна людність,
у наслідок чого й можна було збудувати там такі великі споруди».\footnote{
Diodorus Siculus: «Bibliotheca historica», lib. I, c. 80.
}
Однак великі споруди давнього Єгипту своє існування
завдячують менше чисельності його людности, ніж тій обставині,
що відносно великою частиною людности можна було порядкувати
для цієї справи. Як індивідуальний робітник може давати
тим більше додаткової праці, чим менший його доконечний робочий
час, цілком так само чим менша частина робітничої людности,
потрібна для продукції доконечних засобів існування,
тим більша та її частина, якою можна порядкувати для іншої
справи.

Скоро капіталістичну продукцію дано як передумову, то, за
інших незмінних обставин і за даної довжини робочого дня, величина
додаткової праці буде змінюватися залежно від природних
умов праці, особливо ж залежно від родючости ґрунту. Але звідси
ні в якому разі не випливає протилежне, а саме те, що найродючіший
ґрунт є найвідповідніший для зростання капіталістичного
способу продукції. Цей спосіб припускає панування людини над
природою. Занадто марнотратна природа «водить людину, як
дитину, на мотузочку». Вона не робить власний розвиток людини
природною доконечністю.\footnote{
«Перше (природні багатства), будучи найсприятливішим і найкориснішим,
робить народ безтурботним, чванливим та схильним до всяких
надмірностей, тимчасом як друге приневолює до дбайливости, науки,
мистецтва та розумної політики» («The first (natural wealth), as it
is most noble and advantageous, so doth it make the people careless, proud,
and given to all excesses; whereas the second enforceth vigilancy, literature,
arts and policy»). («England’s Treasure by Foreign Trade. Or the Balance
of our Foreign Trade is the Rule of our Treasure. Written by Thomas Mun,
} Не тропічне підсоння з його буйною
\index{i}{0431}  %% посилання на сторінку оригінального видання 
рослинністю, а помірна смуга є батьківщина капіталу. Не абсолютна
родючість ґрунту, а його диференційованість, різноманітність
його природних продуктів становить природну основу
суспільного поділу праці та через зміну природних умов, серед
яких живе людина, спонукає її до урізноманітнення її власних
потреб, здібностей, засобів праці та способів праці. Доконечність
суспільно контролювати якусь силу природи, економно користуватися
нею, присвоювати її собі або приборкувати у великому
маштабі за допомогою споруд, зроблених людською рукою, —
ось що відіграє найвирішальнішу ролю в історії промисловости.
Наприклад, уреґулювання води в Єгипті,\footnote{
Доконечність обчислювати періоди розливу Ніла створила єгипетську
астрономію, а з нею й панування касти жерців як керівників
рільництва. «Сонцестояння — це той момент року, коли починається
розлив Ніла, і єгиптяни мусили стежити за цим сонцестоянням з особливою
увагою. Для них важливо було встановити цей тропічний рік для
того, щоб реґулювати свої рільничі роботи. Тому вони мусили шукати
на небі виразного знаку його повороту» («Le solstice est le moment de
l’année où commence la crue du Nil, et celui que les Egyptiens ont dû
observer avec le plus d’attention... C’était cette année tropique qu’il leur
importait de marquer pour se diriger dans leurs opérations agricoles. Ils
durent donc chercher dans le ciel un signe apparent de son retour»). (Cuvier:
«Discours sur les révolutions du globe». Ed. Hoefer. Paris 1863, p. 141).
} Льомбардії, Голляндії
й т. ін. Або в Індії, Персії й т. ін., де зрошування штучними
каналами постачає ґрунтові не тільки конче потрібну воду, але
разом з її намулом приставляє з гір мінеральне добриво. Каналізація
— ось у чому була таємниця розцвіту промисловости Еспанії
та Сіцілії під арабським пануванням.\footnote{
Однією з матеріяльних основ державної влади над малими, незв'язаними
між собою продукційними організмами Індії, було реґулювання
водопостачання. Мохаммеданські володарі Індії розуміли це краще,
ніж їхні англійські нащадки. Ми нагадаємо лише про голод 1866 р., що
коштував життя більш ніж мільйонові індусів в окрузі Оріссі Бенґальського
президентства.
}

Сприятливі природні умови завжди дають лише можливість
додаткової праці, алеж ніколи не дійсність додаткової праці,
отже, і додаткової вартости, або додаткового продукту. Різні
природні умови праці призводять до того, що та сама кількість
праці по різних країнах задовольняє різні маси потреб,\footnote{
«Немає двох країн, що давали б однакову кількість доконечних
засобів існування в однаковій достатності та при однакових затратах
} отже,

of London, Merchant, and now published for the common good by his son
John Mun», London 1669, p. 181, 182). «Я також не можу уявити собі
більшого прокляття для народу, як бути закинутим на клапоть землі,
де сама природа рясно продукує засоби існування, а підсоння вимагає
або дозволяє лише мало турбуватися про одяг і житло... Можлива й протилежна
крайність. Ґрунт, що з нього навіть працею не можна вирвати
ніякого продукту, так само недобрий, як і той ґрунт, що рясно родить
без ніякої праці». («Nor can I conceive a greater curse upon a body of people,
than to be thrown upon a spot of land, where the productions for subsistence
and food were, in great measure, spontaneous, and the climate required
or admitted little care for raiment and covering... there may be an extreme
on the other side. A soil incapable of produce by labour is quite as bad as
a soil that produces plentifully without any labour»). («An Inquiry into
the Present High Price of Provisions», London 1767, p. 10).
\index{i}{0432}  %% посилання на сторінку оригінального видання 
до того, що, за інших аналогічних обставин, доконечний робочий
час є різний. На додаткову працю вони впливають лише як
природна межа, тобто визначають той пункт, що від нього може
початися праця на інших. Ця природна межа відсувається назад
у тій самій мірі, в який промисловість проґресує. Серед західньоевропейського
суспільства, де робітник лише додатковою працею
купує дозвіл працювати для свого власного існування, легко
постає ілюзія, що давати додатковий продукт є природжена властивість
людської праці.\footnote{
«Всяка праця мусить» (це, здається, також належить до прав і
обов’язків громадянина) «давати надлишок» («Chaque travail doit
laisser un excédant»). (Proudhon).
} Але візьмімо, наприклад, жителів
східніх островів азійського архіпелагу, де саґо дико росте в лісі.
«Коли місцеві жителі, просвердливши діру в дереві, переконуються,
що стрижень уже достиг, вони зрубують дерево, ділять
його на декілька кусків, видирають стрижень, змішують його
з водою і, відцідивши воду, дістають цілком придатне до вжитку
саґове борошно. Одно дерево дає звичайно 300 фунтів, а може
дати 500—600 фунтів. Отже, там ідуть у ліс і рубають собі хліб,
як у нас рубають дерево на паливо.\footnote{
F. Shouw: «Die Erde, die Pflanze und der Mensch». 2. Auflage. Leipzig
1854, S. 148.
} Припустімо, що такому
східньоазійському рільникові потрібно 12 робочих годин на
тиждень для задоволення всіх його потреб. Сприятлива природа
безпосередньо дає йому багато вільного часу. Для того, щоб він
цей час продуктивно зуживав на самого себе, потрібен цілий ряд
історичних умов, а для того, щоб він витрачав його як додаткову
працю на чужих осіб, потрібен зовнішній примус. Коли б там
було заведено капіталістичну продукцію, наш молодець мусив би
працювати, може, 6 днів на тиждень, щоб присвоїти собі самому
продукт одного робочого дня. Сприятливість природи не пояснює,

праці. Людські потреби зростають або зменшуються залежно від суворости
або м’якости підсоння, що в ньому люди живуть; отже, розміри, що в
них мешканці різних країн мусять продукувати, не можуть бути однакові,
і не можна визначити ступінь цієї неоднаковости інакше, як тільки у
зв’язку з ступенем теплоти або холоду; звідси можна зробити й той загальний
висновок, що кількість праці, потрібної для певного числа людности,
найбільша в холодному підсонні, найменша в теплому. Бо в першому
не тільки люди більше потребують одягу, але й земля потребує
більше праці на оброблення, аніж у другому». («There are no two countries
which furnish an equal number of the necessaries of life in equal plenty,
and with the same quantity of labour. Men’s wants increase or diminish
with the severity or teinperateness of the climate they live in; consequently
the proportion of trade which the inhabitants of different countries are
obliged to carry on through necessity, cannot be the same, nor is it practicable
to ascertain the degree of variation farther than by the Degrees of
Heat and Cold; from whence one may make this general conclusion, that
the quantity of labour required for a certain number of people is greatest
in cold climates, and least in hot ones; for in the former men not only want
more clothes, but the earth more cultivating than in the latter»). («An
Essay on the Governing Causes of the Natural Rate of Interest», London
1750, p. 60). Автор цього епохального анонімного твору J. Massey.
Юм запозичив із нього свою теорію процента.
\parbreak{}  %% абзац продовжується на наступній сторінці

\input{i/_0433.tex}
\parcont{}  %% абзац починається на попередній сторінці
\index{iii1}{0434}  %% посилання на сторінку оригінального видання
відбувається не за допомогою простої кредитної операції без
будь-якої участі грошей; що, отже, при великому попиті на грошові
позики може відбуватися величезна кількість цих операцій,
не збільшуючи при цьому циркуляції. Але самий той факт,
що циркуляція Англійського банку лишається незмінною або
навіть зменшується одночасно із значним збільшенням грошових
позик, які він видає, зовсім не доводить prima facie,
як це вважають Фуллартон, Тук та інші (в наслідок їх
помилкової думки, що грошова позика є те саме, що й одержання
capital on loan [позикового капіталу], додаткового капіталу), що
циркуляція грошей (банкнот) в їх функції як засобу платежу
не збільшується й не розширюється. Тому що циркуляція банкнот
як засобу купівлі в періоди застою у справах, коли потрібні такі
великі позики, скорочується, то їх циркуляція як засобу платежу
може збільшуватись, а загальна сума циркуляції, сума банкнот,
що функціонують як засоби купівлі й платежу, все ж може
лишатись незмінною або навіть зменшуватись. Циркуляція, як
засобу платежу, банкнот, які відразу ж припливають назад до
банку, що їх видав, в очах згаданих економістів зовсім не є
циркуляція.

Коли б циркуляція грошей як засобу платежу збільшилась
у вищій мірі, ніж вона зменшилась би як циркуляція засобів
купівлі, то вся циркуляція зросла б, хоч би маса грошей,
що функціонують як засіб купівлі, значно зменшилась. І це
дійсно настає в певні моменти кризи, а саме при цілковитому
краху кредиту, коли стає неможливим не тільки продаж товарів
і цінних паперів, але й дисконт векселів і коли вже ніщо не
дійсне, крім платежу готівкою, або, як каже купець: каса. Тому
що Фуллартон та інші не розуміють, що циркуляція банкнот як
засобу платежу є характерна для таких часів грошової скрути,
то вони розглядають це явище як випадкове. „With respect again
to those examples of eager competition for the possession of banknotes,
which characterise seasons of panic and which may sometimes,
as at the close of 1825, lead to a sudden, though only temporary,
enlargement of the issues, even while the efflux of bullion
is still going on, these, I apprehend, are not to be regarded as
among the natural or necessary concomitants of a low exchange; the
demand in such cases is not for circulation (слід було б сказати:
не на циркуляцію як на засоби купівлі) but for hoarding, a demand
on the part of alarmed bankers and capitalists which arises
generally in the last act of the crisis (отже, попит на резерв засобів
платежу) after a long continuation of the drain, and is the
precursor of its termination“ [„Щодо прикладів тієї завзятої конкуренції
за оволодіння банкнотами, яка характеризує часи
паніки і яка іноді, як от наприкінці 1825 року, може привести
до раптового, хоч би тільки тимчасового збільшення
емісії банкнот, навіть тоді, коли відплив золота все ще триває,
то, на мою думку, на них не можна дивитись як на природних
\parbreak{}  %% абзац продовжується на наступній сторінці

\parcont{}  %% абзац починається на попередній сторінці
\index{i}{0435}  %% посилання на сторінку оригінального видання
її ринкової ціни (!), він начебто авансує ріжницю своєму підприємцеві
(?) і~\abbr{т. ін.}».\footnoteA{
\emph{J. St. Mill}: «Principles of Political Economy», London 1868,
p. 252, 253 passim. (Вищенаведені місця перекладено з французького
видання «Капіталу». — \emph{Ф. Е.}).
} В дійсності робітник авансує даром
капіталістові свою працю протягом тижня і~\abbr{т. д.}, з тим, щоб
наприкінці тижня і~\abbr{т. ін.} одержати її ринкову ціну; за Міллом,
це перетворює робітника на капіталіста! На пласкій рівнині
й грудка землі видається горбом; пласкість нашої сучасної буржуазії
можна зміряти калібром її «великих мислителів».

\section{Зміна величини ціни робочої сили та додаткової
вартости}

[У третьому відділі, сьомий розділ, ми аналізували норму
додаткової вартости, але лише з погляду продукції абсолютної
додаткової вартости. У четвертому відділі ми знайшли додаткові
визначення. Тут нам треба зрезюмувати все посутнє про це].\footnote*{
Заведене у прямі дужки ми беремо з другого німецького видання.
\emph{Ред.}
}

Вартість робочої сили визначається вартістю звичних доконечних
засобів існування пересічного робітника. Маса цих засобів
існування, хоч форма їхня і може змінятись, для даної епохи
й даного суспільства є дана, а тому її треба розглядати як сталу
величину. Змінюється лише вартість цієї маси. Ще два інші
фактори, входять у визначення вартости робочої сили. З одного
боку, витрати на її розвиток, які змінюються із зміною способу
продукції, з другого боку, природні ріжниці робочої сили, тобто,
чи є вона чоловіча або жіноча, дорослих робітників або підлітків.
Споживання цих різних робочих сил, знову ж таки зумовлюване
способом продукції, створює велику ріжницю у витратах репродукції
робітничої родини та у вартості дорослого робітника-чоловіка.
Однак у дальшому досліді обидва ці фактори не береться
на увагу.\footnoteA{
Розглянутий на стор. 253--254 випадок, тут, природно, також
виключено. (Примітка до третього видання — \emph{Ф. Е.}).
}

Ми припускаємо: 1) що товари продаються за їхньою вартістю;
2) що ціна робочої сили може іноді піднестися понад свою вартість,
але ніколи не падає нижче від неї.

Зробивши такі припущення, ми виявили, що відносні величини
ціни робочої сили й додаткової вартости визначаються
трьома обставинами: 1) довжиною робочого дня, або екстенсивною
величиною праці; 2) нормальною інтенсивністю, або її інтенсивною
величиною, тобто тією обставиною, що певну кількість
праці витрачається за певний час; 3) нарешті, продуктивною
силою праці, тобто тією обставиною, що та сама кількість праці
за той самий час дає, залежно від ступеня розвитку умов продукції,
\index{i}{0436}  %% посилання на сторінку оригінального видання
більшу або меншу кількість продукту. Очевидно, можливі
дуже різні комбінації, відповідно до того, чи один із трьох факторів
сталий, а два змінюються, чи два фактори сталі, а один
змінюється, чи, нарешті, всі три змінюються одночасно. Число
цих комбінацій збільшується ще й через те, що за одночасної
зміни різних факторів величина й напрям змін можуть бути
різні. Далі ми розглядаємо лише головні комбінації.

\subsection{Величина робочого дня й інтенсивність праці сталі (дані),
продуктивна сила праці змінюється}

При цьому припущенні вартість робочої сили й додаткової
вартости визначається трьома законами:

Поперше, робочий день даної величини завжди виражається
в тій самій новоспродукованій вартості, хоч би й як змінювалася
продуктивність праці і разом з нею маса продуктів, а тому й
ціна поодинокого товару.

Новоспродукована вартість дванадцятигодинного робочого
дня, є, наприклад, 6\shil{ шилінґів}, хоч маса спродукованих споживних
вартостей змінюється з продуктивною силою праці, і вартість
6\shil{ шилінґів} розподіляється, отже, на більшу або меншу кількість
товарів.\footnote*{
У французькому виданні останні два абзаци подано так: «Поперше,
робочий день даної величини продукує завжди ту саму вартість,
хоч би й як змінювалася продуктивність праці.

Якщо одна година праці нормальної інтенсивности продукує вартість
у \sfrac{1}{2}\shil{ шилінґа}, то дванадцятигодинний робочий день може спродукувати
лише вартість у 6\shil{ шилінґів.} (Ми припускаємо завжди, що вартість грошей
лишається незмінна). Якщо продуктивність праці підвищується або зменшується,
то той самий робочий день дасть більше або менше продуктів,
і вартість у 6\shil{ шилінґів} розподілиться таким чином на більшу або меншу
кількість товарів». («Le Capital etc.», v. I, ch. XVII, p. 224). \emph{Ред.}
}

Подруге, вартість робочої сили й додаткова вартість змінюються
в протилежному напрямі. Зміна продуктивної сили праці,
її зростання або зменшення, впливає на вартість робочої сили у
зворотному напрямі, на додаткову вартість — у простому.

Новоспродукована вартість дванадцятигодинного робочого
дня є стала величина, наприклад, 6\shil{ шилінґів.} Ця стала величина
дорівнює сумі додаткової вартости плюс вартість робочої сили,
яку робітник заміщує еквівалентом. Само собою зрозуміло, що
з двох частин сталої величини жодна не може збільшитися без
того, щоб друга не зменшилася. Вартість робочої сили не може
підвищитися з 3\shil{ шилінґів} до 4 без того, щоб додаткова вартість
не знизилася з 3\shil{ шилінґів} до 2, а додаткова вартість не може
підвищитися з 3\shil{ шилінґів} до 4 без того, щоб вартість робочої
сили не знизилася з 3\shil{ шилінґів} до 2. Отже, за цих обставин неможлива
ніяка зміна абсолютної величини ані вартости робочої
сили, ані додаткової вартости без одночасної зміни їхніх відносних
або пропорціональних величин. Неможливо, щоб вони одночасно
падали або підносилися.

\index{i}{0437}  %% посилання на сторінку оригінального видання 
Далі, вартість робочої сили не може падати, отже, додаткова
вартість не може підвищуватися без того, щоб не підвищувалася
продуктивна сила праці; наприклад, у вищенаведеному випадку
вартість робочої сили не може впасти з 3 до 2 шилінґів, якщо
підвищена продуктивна сила праці не дозволяє за 4 години продукувати
ту саму масу засобів існування, яка раніш потребувала
для своєї продукції 6 годин. Навпаки, вартість робочої сили не
може підвищитися з 3 шилінґів до 4 без того, щоб продуктивна
сила праці не знизилась, отже, без того, щоб не потрібно було
8 годин на продукцію тієї самої маси засобів існування, яку раніш
продукувалося за 6 годин. Звідси випливає, що збільшення продуктивности
праці знижує вартість робочої сили і тим підвищує
додаткову вартість, і, навпаки, зменшення продуктивности праці
підвищує вартість робочої сили та знижує додаткову вартість.

Формулюючи цей закон, Рікардо не помітив одного: хоч зміна
величини додаткової вартости або додаткової праці зумовлює
зворотну зміну величини вартости робочої сили або доконечної
праці, але звідси ні в якому разі не випливає, що вони змінюються
в тій самій пропорції. Вони більшають або меншають на ту саму
величину. Але пропорція, в якій кожна частина новоспродукованої
вартости або робочого дня більшає або меншає, залежить
від первісного поділу, що був перед зміною продуктивної сили
праці. Якщо вартість робочої сили була 4 шилінґи, або доконечний
робочий час — 8 годин, додаткова вартість — 2 шилінґи,
або додаткова праця — 4 години, і якщо, в наслідок підвищення
продуктивної сили праці, вартість робочої сили падає
до 3 шилінґів, або доконечна праця падає до 6 годин, то додаткова
вартість підвищується до 3 шилінґів, або додаткова праця
до 6 годин. Ту саму величину, 2 години, або 1 шилінґ, там додано,
тут однято. Але відносна зміна величин на обох сторонах
різна. Тимчасом як вартість робочої сили падає з 4 шилінґів
до 3, отже, на 1/4, або на 25\%, додаткова вартість підвищується
з 2 шилінґів до 3, отже, на 1/2, або 50\%. Звідси випливає, що
відносне збільшення або зменшення додаткової вартости, яке
постає в наслідок даної зміни продуктивної сили праці, то більше,
що менша, і то менше, що більша була первісно частина робочого
дня, яка виражається в додатковій вартості.

По-третє, збільшення або зменшення додаткової вартости є
завжди наслідок, але ніколи не причина відповідного зменшення
або збільшення вартости робочої сили.\footnote{
До цього третього закону Мак Куллох зробив, між іншим, безглуздий
додаток, ніби додаткова вартість може підвищуватися й без зниження
вартости робочої сили, в наслідок скасування податків, що їх раніш мав
платити капіталіст. Скасування таких податків аніскільки не змінює
тієї кількости додаткової вартости, яку промисловий капіталіст безпосередньо
витискує з робітника. Воно змінює лише відношення між тією
частиною додаткової вартости, яку капіталіст ховає собі до кишені, і
тією частиною, що її він мусить віддати третім особам. Отже, воно нічого не
змінює у відношенні між вартістю робочої сили й додатковою вартістю.
Таким чином, виняток Мак Куллоха доводить тільки його нерозуміння
}

\index{i}{0438}  %% посилання на сторінку оригінального видання 
А що робочий день є стала величина й виражається в сталій
величині вартости, що кожній зміні величини додаткової вартости
відповідає зворотна зміна величини вартости робочої сили, що
вартість робочої сили може змінятися лише разом із зміною продуктивної
сили праці, то, за цих умов, цілком ясно, що кожна
зміна величини додаткової вартости постає в наслідок зворотної
зміни величини вартости робочої сили. Тому, коли ми раніш
бачили, що неможлива жодна зміна абсолютних величин вартости
робочої сили й додаткової вартости без зміни їхніх відносних
величин, то тепер бачимо, що жодна зміна їхніх відносних
величин вартости неможлива без зміни абсолютної величини вартости
робочої сили.

За третім законом зміна величини додаткової вартости має
собі за передумову зміну вартости робочої сили, спричинену
зміною продуктивної сили праці. Межу зміни величини додаткової
вартости дано новою межею вартости робочої сили. Але навіть
у тому випадку, коли обставини дозволяють цьому законові
діяти, все ж можуть відбуватися проміжні зміни. Якщо, наприклад,
у наслідок підвищеної продуктивної сили праці вартість
робочої сили падає з 4 шилінґів до 3, або доконечний робочий час
падає з 8 до 6 годин, то ціна робочої сили могла б спасти лише до
З шилінґів 8 пенсів, 3 шилінґів 6 пенсів, 3 шилінґів 2 пенсів
і т. д., а тому додаткова вартість могла б підвищитися лише до
З шилінґів 4 пенсів, 3 шилінґів 6 пенсів, 3 шилінґів 10 пенсів,
і т. д. Ступінь спаду, що його мінімальна межа є 3 шилінґи,
залежить од відносної ваги, яку кидають на терези натиск капіталу,
з одного боку, опір робітників — з другого.

Вартість робочої сили визначається вартістю певної кількости
засобів існування. Із зміною продуктивної сили праці змінюється
вартість цих засобів існування, а не їхня маса. Сама ця
маса, при зростанні продуктивної сили праці, може зростати для
робітника і для капіталіста одночасно і в однаковій пропорції
без якоїбудь зміни між величинами ціни робочої сили й додаткової
вартости. Коли первісна вартість робочої сили дорівнює 3 шилінґам,
а доконечний робочий час становить 6 годин, коли додаткова
вартість теж дорівнює 3 шилінґам, або додаткова праця
становить також 6 годин, то подвоєння продуктивної сили праці,
при незмінному поділі робочого дня, лишило б незмінними ціну
робочої сили й додаткову вартість. Тільки кожна з них виражалася
б у подвійній кількості, але відповідно до цього й здешевілих
споживних вартостей. Хоча ціна робочої сили й лишалася б
незмінна, все ж вона підвищилася б понад її вартість. Коли б
ціна робочої сили впала, але не до мінімальної межі в 1 1/2 шилінґа,
визначеної її новою вартістю, а до 2 шилінґів 10 пенсів, 2 шилінґів
6 пенсів і т. д., то й це падіння ціни все ще репрезентувало б
зростання маси засобів існування. Таким чином при зростанні

загального правила — лихо, яке з ним трапляється у вульґаризуванні
Рікарда так само часто, як і з Ж. Б. Сеєм у його вульґаризуванні А. Сміса.
\parbreak{}  %% абзац продовжується на наступній сторінці

\parcont{}  %% абзац починається на попередній сторінці
\index{i}{0439}  %% посилання на сторінку оригінального видання
продуктивної сили праці ціна робочої сили могла б постійно
спадати разом з одночасним постійним зростанням маси засобів
існування робітника. Але відносно, тобто порівняно з додатковою
вартістю, вартість робочої сили постійно спадала б, і, отже,
ширшала б прірва між життєвим становищем робітника й капіталіста\footnote{
«Коли постає зміна в продуктивності промисловости, і певна
кількість праці й капіталу продукує більше або менше продуктів, то
відносна величина заробітної плати може значно змінятися, тимчасом як
кількість продуктів, що її репрезентує ця відносна величина, лишається
та сама; абож може змінятися кількість, тимчасом як відносна
величина лишається незмінною» («When an alteration takes place in the productiveness
of industry, and that either more or less is produced by a given
quantity of labour and capital, the proportion of wages may obviously
vary, whilst the quantity, which that proportion represents, remains the
same, or the quantity may vary, whilst the proportion remains the same»).
(«Outlines of Political Economy», London 1832, p. 67).
}.

Рікардо перший точно зформулював зазначені вище три закони.
Хиби його досліду такі: 1) ті особливі умови, серед яких
мають силу ці закони, він вважає за само собою зрозумілі, загальні
й виключні умови капіталістичної продукції. Він не визнає
ніякої зміни ні в довжині робочого дня, ні в інтенсивності
праці, так що в нього продуктивна сила праці сама собою стає
єдиним змінним фактором; — 2) але, — і це куди більше вносить
помилок у його аналізу, — як і всі інші економісти, він ніколи
не досліджував додаткову вартість як таку, тобто незалежно
від її осібних форм, як от зиск, земельна рента й~\abbr{т. ін.} Тому
він закони про норму додаткової вартости безпосередньо скидає
до однієї купи із законами про норму зиску. Як уже сказано,
норма зиску є відношення додатковоі вартости до цілого
авансованого капіталу, тимчасом як норма додаткової вартости
є відношення додаткової вартости лише до змінної частини
цього капіталу. Припустімо, що капітал у 500\pound{ фунтів стерлінґів}
($С$) поділяється на сировинний матеріял засоби праці й~\abbr{т. ін.},
разом 400\pound{ фунтів стерлінґів} ($с$) і на 100\pound{ фунтів стерлінґів} заробітної
плати ($v$); далі, що додаткова вартість дорівнює 100\pound{ фунтам
стерлінґів} ($m$). Тоді норма додаткової вартости\[
   \frac{m}{v} \deq{} \frac{100\text{\pound{ фунтів стерлінґів}}}{100\text{\pound{ фунтів стерлінґів}}} \deq{} 100\%
\]

\noindent{}Норма ж зиску $\frac{m}{C} \deq{} \frac{100\text{\pound{ фунтів стерлінґів}}}{500\text{\pound{ фунтів стерлінґів}}} \deq{} 20\%$. Крім того,
ясно, що норма зиску може залежати від обставин, які ні в якому
разі не впливають на норму додаткової вартости. Пізніше, в
третій книзі цього твору, я доведу, що та сама норма додаткової
вартости може виражатися в якнайрізніших нормах зиску, і
різні норми додаткової вартости, при певних обставинах, можуть
виражатись у тій самій нормі зиску.


\index{i}{0440}  %% посилання на сторінку оригінального видання
\manualpagebreak{}
\subsection{Сталий робочий день, стала продуктивна сила праці,
інтенсивність праці змінюється}

Зростання інтенсивносте праці припускає збільшену витрату
праці протягом того самого часу. Тому інтенсивніший робочий
день утілюється в більшій кількості продуктів, аніж менш
інтенсивний з тим самим числом годин. Правда, і при підвищенні
продуктивної сили праці той самий робочий день дає
більше продуктів. Але в останньому випадку вартість одиниці
продукту спадає, бо він коштує менше праці, ніж раніш; у першому
ж випадку вона лишається незмінна, бо продукт коштує
стільки ж праці, що й раніш. Число продуктів тут зростає, але
ціна їхня не спадає. З їхнім числом зростає сума їхніх цін, тимчасом
як при підвищенні продуктивної сили праці та сама сума
вартосте лише виражається в збільшеній масі продуктів. Отже,
при однаковому числі годин інтенсивніший робочий день утілюється
в більшій новоспродукованій вартості, отже, при незмінній
вартості грошей, — у більшій кількості грошей. Новоспродукована
протягом нього вартість змінюється відповідно до того,
як відхиляється його інтенсивність від нормального суспільного
ступеня. Отже, той самий робочий день виражається не в сталій,
як раніш, а у змінній новоспродукованій вартості, наприклад,
інтенсивніший дванадцятигодинний робочий день у 7\shil{ шилінґах},
8\shil{ шилінґах} і~\abbr{т. д.} замість 6\shil{ шилінґів}, як це було за дванадцятигодинного
робочого дня звичайної інтенсивносте. Ясно, що коли
новоспродукована протягом робочого дня вартість змінюється,
скажімо, з 6\shil{ шилінґів} на 8\shil{ шилінґів}, то й обидві частини цієї
новоспродукованої вартости, ціна робочої сили й додаткова вартість,
можуть одночасно зростати, в однаковій або неоднаковій
мірі. Ціна робочої сили й додаткова вартість можуть одна й друга
зрости одночасно з 3\shil{ шилінґів} до 4, коли новоспродукована вартість
зростає з 6 до 8\shil{ шилінґів.} Тут підвищення ціни робочої
сили не включає неодмінно підвищення її ціни понад її вартість.
Навпаки, воно може супроводитися падінням її вартости. Це
завжди буває тоді коли підвищення ціни робочої сили не компенсує
її прискореного зужитковування.

Ми знаємо, що, за деякими минущими винятками, зміна
продуктивности праці тільки тоді викликає зміну величини вартости
робочої сили а тому й величини додаткової вартости, коли
продукти розглядуваної галузі промисловости стають предметами
звичайного споживання робітника. Тут ця межа відпадає.
Хоч величина праці змінюється екстенсивно або інтенсивно, цій
зміні її величини відповідає зміна величини новоспродукованої
нею вартости, незалежно від природи того предмету, в якому
ця вартість виражається.

Коли б інтенсивність праці одночасно й рівномірно підвищилася
в усіх галузях промисловости, то новий вищий ступінь інтенсивности
праці став би звичайним, нормальним суспільним ступенем,
і тому перестав би вважатись за екстенсивну величину.
\parbreak{}  %% абзац продовжується на наступній сторінці

\parcont{}  %% абзац починається на попередній сторінці
\index{i}{0441}  %% посилання на сторінку оригінального видання
Однак навіть і тоді пересічні ступені інтенсивности праці лишалися
б у різних націй різні і тому модифікували б застосування
закону вартости до робочих днів різних націй. Інтенсивніший
робочий день однієї якоїсь нації виражається в більшій сумі
грошей, ніж менш інтенсивний робочий день іншої нації.12

III. Продуктивна сила та інтенсивність праці сталі, робочий
день змінюється

Робочий день може змінятися в двох напрямах: він може
скорочуватись або здовжуватись. [При наших нових даних ми
маємо такі закони:

Робочий день втілюється прямо пропорціонально своїй довжині
в більшій або меншій вартості; остання, отже, є величина
змінна, а не стала.

Всяка зміна у відношенні величин додаткової вартости й вартости
робочої сили випливає із зміни абсолютної величини додаткової
праці, отже, і додаткової вартости.

Абсолютна вартість робочої сили може змінятися лише в
наслідок зворотної дії, що її справляє здовження додаткової
праці на ступінь зужитковування робочої сили. Отже, всяка зміна
її абсолютної вартости є наслідок, але ніколи не причина зміни
величини додаткової вартости.

У цьому розділі, як і в наступних, ми припускатимемо завжди,
що робочий день, який первісно становить 12 годин, — шість
годин доконечної праці і шість годин додаткової праці — продукує
вартість у 6 шилінґів, що з неї одна половина дістається
робітникові, а друга — капіталістові.

Розгляньмо спочатку скорочення робочого дня, приміром,
а 12 годин до 10. Тепер він дає лише вартість у 5 шилінґів. Додаткова
праця спадає з 6 годин до 4, додаткова вартість — з 3 шилінґів
до 2 шилінґів. Це зменшення її абсолютної величини
призводить до зменшення її відносної величини. Вона відносилась
до вартости робочої сили як 3: 3, а тепер — лише як 2: 3. Зате
вартість робочої сили, хоч вона й лишається та сама, зростає
в своїй відносній величині; вона відноситься тепер до додаткової
вартости як 3: 2, а не як 3: 3].\footnote*{
Заведене у прямі дужки ми беремо з французького видання («Le
Capital etc.», v. І, ch. XVII, p. 226--227). \emph{Ред.}
}

12 «За інших однакових умов англійський фабрикант може за даний
час пустити в рух значно більшу суму праці, ніж чужоземний фабрикант,
так що це урівноважує ріжницю між 60-годинним тижнем у нас
і 72-або 80-годинним по інших країнах» («All things being equal,
the English manufacturer can turn out a considerably larger amount of
work in a given time than a foreign manufacturer, so much as to counterbalance
the difference of the working days, between 60 hours a week here
and 72 or 80 elsewhere»). («Reports of Insp. of Fact, for 31 st October 1885»,
p. 65). Більше законодавче скорочення робочого дня в континентальних
фабриках було б найпевнішим засобом зменшити цю ріжницю між робочою
годиною на континенті і в Англії.
\parbreak{}  %% абзац продовжується на наступній сторінці

\parcont{}  %% абзац починається на попередній сторінці
\index{i}{0442}  %% посилання на сторінку оригінального видання
1) Скорочення робочого дня за даних умов, тобто, коли продуктивність
та інтенсивність праці не змінюються, залишає
вартість робочої сили, а тому й доконечний робочий час незмінним.
Воно зменшує додаткову працю й додаткову вартість. З абсолютною
величиною останньої падає і її відносна величина, тобто
її величина проти незмінної величини вартости робочоі сили.
Тільки через пониження її ціни нижче від її вартости може капіталіст
триматися без втрат.

В усіх звичайних запереченнях проти скорочення робочого
дня виходять з тієї гіпотези, що це явище відбувається при припущених
тут обставинах, тимчасом як у дійсності, навпаки, зміни
продуктивности й інтенсивности праці або передують скороченню
робочого дня, або безпосередньо настають після нього\footnote{
«Є обставини, що компенсують це\dots{} їх вивів на світ десятигодинний
фабричний закон» («There are compensating circumstances\dots{} which
the working of the Ten Hour’s Act has brought to light»). («Reports of
Insp. of Fact, for 1st December 1848», p. 7).
}.

2) Здовження робочого дня. Хай доконечний робочий час
буде 6 годин, або вартість робочої сили 3\shil{ шилінґи}, так само додаткова
праця — 6 годин, або додаткова вартість 3\shil{ шилінґи.}
Тоді цілий робочий день становить 12 годин і виражається у
продукті вартістю в 6\shil{ шилінґів.} Якщо робочий день здовжується
на дві години, а ціна робочої сили лишається незмінна, то з
абсолютною зростає й відносна величина додаткової вартости.
Хоч величина вартости робочої сили абсолютно лишається незмінна,
відносно вона спадає. За умов пункту 1) відносна величина
вартости робочої сили не могла змінятися без зміни її абсолютної
величини. Тут, навпаки, відносна зміна величини вартости
робочої сили є результат абсолютної зміни величини додаткової
вартости.

А що новоспродукована вартість, у якій виражається робочий
день, зростає разом з його здовженням, то ціна робочої сили
і додаткова вартість можуть зростати одночасно, чи то на однакову
чи на неоднакову величину. Отже, це одночасне зростання
можливе у двох випадках — при абсолютному здовженні робочого
дня і при ростущій інтенсивності праці без такого здовження.

Із здовженням робочого дня ціна робочої сили може впасти
нижче від її вартости, хоч би номінально ця ціна й лишилася
незмінна або навіть і зросла. Адже ж денну вартість робочої сили,
як ми собі пригадуємо, оцінюється за її нормальним пересічним
триванням або за нормальним періодом життя робітника та за
відповідним, нормальним, властивим людській натурі перетворенням
життєвої субстанції на рух\footnote{
«Кількість праці, витраченої людиною протягом 24 годин, можна
приблизно визначити, досліджуючи хемічні зміни, що відбулися в її
тілі, бо зміна форм матерії показує на попередню діяльність динамічної
сили» («The amount of labour which a man had undergone in the course
of 24 hours might approximative arrived at by an-examination of the
chemical changes which had taken place in his body, changed forms in
matter indicating the anterior exercise of dynamic force»). (Grove: «On
the Correlation of Physical Forces»).
}. До певного пункту збільшене
\index{i}{0443}  %% посилання на сторінку оригінального видання
зужитковання робочої сили, невіддільне від здовження
робочого дня, можна компенсувати збільшеним відживленням її.
Поза цим пунктом зужитковування зростає в геометричній проґресії
і одночасно руйнуються всі нормальні умови репродукції
та функціонування робочої сили. Ціна робочої сили і ступінь
її експлуатації перестають бути спільномірними величинами.

\manualpagebreak{}
\subsection{Одночасні зміни тривання праці, продуктивної сили праці
та інтенсивности праці}

Тут, очевидно, можливе велике число комбінацій. Можуть
змінятися два фактори, а один лишатися сталим, або всі три фактори
можуть одночасно змінятися. Вони можуть змінятися в
однаковій або неоднаковій мірі, в тому самому або в протилежному
напрямі, і їхні зміни можуть тому почасти або цілком
навзаєм компенсуватися. А втім, аналіза всіх можливих випадків,
після висновків, поданих у пунктах І, II та III, легка. Результат
кожної можливої комбінації можна знайти, коли розглядати
почережно кожний з факторів як змінний, а інші як сталі.
Тому ми тут коротко зазначимо лише два важливі випадки.

1) Падуща продуктивна сила праці при одночасному здовженні
робочого дня.

Коли ми тут говоримо про падущу продуктивну силу праці,
то йдеться про галузі праці, що їхні продукти визначають вартість
робочої сили, отже, наприклад, про падущу продуктивну силу
праці в наслідок чимраз більшої неродючости ґрунту та відповідного
подорожчання продуктів землі. Припустімо, що робочий
день триває 12 годин, новоспродукована протягом нього вартість
становить 6\shil{ шилінґів}, з чого половина покриває вартість робочої
сили, а друга половина становить додаткову вартість. Отже,
робочий день розпадається на 6 годин доконечної праці і 6 годин
додаткової праці. Хай у наслідок подорожчання продуктів землі
вартість робочої сили підвищується з 3\shil{ шилінґів} до 4\shil{ шилінґів},
отже, доконечний робочий час — з 6 до 8 годин. Якщо робочий
день лишається незмінний, то додаткова праця спадає з б до 4 годин,
а додаткова вартість з 3 до 2\shil{ шилінґів.} Якщо робочий день
здовжується на 2 години, тобто з 12 до 14 годин, то додаткова
праця лишається 6 годин, а додаткова вартість 3\shil{ шилінґи}, але
величина цієї останньої падає порівняно з вартістю робочої сили,
вимірюваною доконечною працею. Якщо робочий день здовжується
на 4 години — з 12 до 16 годин, то відносні величини додаткової
вартости й вартости робочої сили, додаткової праці й
доконечної праці, лишаються незмінні, але абсолютна величина
додаткової вартости зростає з 3 до 4\shil{ шилінґів}, абсолютна величина
додаткової праці — з 6 до 8 робочих годин, отже, на \sfrac{1}{3},
або 33\sfrac{1}{3}\%. Отже, при зменшенні продуктивної сили праці
і одночасному здовженні робочого дня абсолютна величина
додаткової вартости може лишатись незмінна, тимчасом як її
відносна величина падає; її відносна величина може лишатись
\parbreak{}  %% абзац продовжується на наступній сторінці

\parcont{}  %% абзац починається на попередній сторінці
\index{i}{0444}  %% посилання на сторінку оригінального видання
незмінна, тим часом як її абсолютна величина зростає, і, залежно
від ступеня здовження робочого дня, можуть зростати обидві.

В період від 1799 р. до 1815 р. зростання цін на засоби існування
в Англії призвело до номінального підвищення заробітної
плати, хоч дійсна, виражена в засобах існування плата і впала.
Звідси Вест і Рікардо зробили такий висновок, що зменшення
продуктивности рільничої праці спричинилося до зменшення
норми додаткової вартости, і це припущення факту, що існував
тільки в їхній фантазії, вони зробили вихідним пунктом важливої
аналізи кількісного співвідношення між величинами заробітної
плати, зиску та земельної ренти. Але в наслідок підвищеної
інтенсивности праці й вимушеного здовження робочого часу
додаткова вартість зросла тоді і абсолютно і відносно. Це був
той період, коли безмірне здовження робочого дня здобуло собі
прав громадянства,\footnote{
«Хліб і праця рідко коли йдуть цілком пліч-о-пліч; але існує
очевидна межа, що поза нею їх ніяк не можна розлучити. Щодо незвичайних
зусиль, роблених робітничими клясами за часів дорожнечі, зусиль,
що призвели до спаду заробітної плати, констатованого у свідченнях
(а саме перед парляментськими комітетами 1814—15 рр.), то вони
становлять більше заслугу поодиноких осіб і звичайно сприяють зростові
капіталу. Але жодна гуманна людина не забажає, щоб вони лишалися
постійними й неослаблими. Як тимчасова полегкість, вони варті
подиву, але коли б вони зробилися постійним явищем, то результат був
би такий самий, як коли б людність країни збільшилася до крайніх меж,
визначених засобами її існування». («Corn and Labour rarely march
quite abreast; but there is an obvious limit, beyond which they cannot
be separated. With regard to the unusual exertions made by the labouring
classes in periods of dearness, which produce the fall of wages noticed in
the evidence (Parliamentary Committees of Inquiry 1814—15), they are
most meritorius in the individuals, an certainly favour the growth of capital.
But no man of humanity could wish to see them constant and unremitted.
They are most admirable as a temporary relif; but if they were
constantly in action, effects of a similar kind would result from them, as
from the population of a country being pushed to the every extreme limits
of its food»). (Malthus: «Inquiry into the Nature and Progress of Rent»,
London 1815, p. 48n). Робить честь Малтузові те, що він тут робить
наголос на здовженні робочого дня, про що він безпосередньо зазначає
ще в іншому місці свого памфлету, тимчасом як Рікардо й інші, не вважаючи
на наявність кричущих фактів, кладуть в основу всіх своїх дослідів
сталу величину робочого дня. Але консервативні інтереси, що їхнім
рабом був Малтуз, заважали йому бачити, що безмірне здовження робочого
дня разом з надзвичайним розвитком машин та експлуатацією жіночої
і дитячої праці мусило зробити «зайвою» велику частину робітничої
кляси, особливо після того, як припинилися зумовлений війною попит
та англійська монополія на світовому ринку. Природно, куди зручніше
було й куди більше відповідало інтересам панівних кляс, перед якими
Малтуз вклонявся чисто по-попівському, поясняти це «перелюднення»
з вічних законів природи, ніж з лише історичних природних законів
капіталістичної продукції.
} період, що його в Англії виразно характеризує
прискорене зростання капіталу на одному боці й павперизму
на другому.\footnote{
«Головна причина зростання капіталу за часів війни випливала
з великих зусиль і, може бути, з ще більших нестатків робітничих кляс,
найчисленніших у кожному суспільстві. В наслідок нужденних обставин
більше число жінок і дітей примушені були взятися до праці; а ті, які
}
\index{i}{0445}  %% посилання на сторінку оригінального видання
2) Збільшення інтенсивности і продуктивної сили праці при
одночасному скороченні робочого дня.

Підвищення продуктивної сили праці та зростання її інтенсивносте
в одному напрямі діють однаково. Одне і друге збільшує
масу продуктів, продукованих протягом певного часу. Отже,
одне і друге скорочує ту частину робочого дня, що її робітник
потребує на продукцію своїх засобів існування — або їх еквіваленту.
Абсолютну мінімальну межу робочого дня визначає
взагалі ця його доконечна складова частина, яку однак можна
скорочувати. Коли б цілий робочий день скоротився до цієї останньої,
то зникла б додаткова праця — річ за капіталістичного
режиму неможлива. Усунення капіталістичної форми продукції
дозволяє обмежити робочий день доконечною працею. Однак, за
інших незмінних умов, остання поширила б свої рамки. З одного
боку, тому що життєві умови робітника покращали б і його життєві
потреби збільшилися б. З другого боку, довелося б до доконечної
праці залічити частину теперішньої додаткової праці,
саме працю, потрібну на те, щоб утворити суспільний резервний
фонд і фонд акумуляції.

Що більше зростає продуктивна сила праці, то більше можна
скорочувати робочий день, а що більше скорочується робочий
день, то більше може зростати інтенсивність праці. З суспільного
погляду продуктивність праці зростає також з її економією. Ця
остання включає не тільки економію на засобах продукції, але й
уникання всякої некорисної праці. Тимчасом як у кожному індивідуальному
підприємстві капіталістичний спосіб продукції примушує
до економії, його анархічна система конкуренції породжує
якнайбезмірніше марнотратство суспільних засобів продукції
та робочих сил поряд безлічі функцій, тепер неминучих, але по
суті зайвих.

За даної інтенсивносте й продуктивности праці частина суспільного
робочого дня, доконечна для матеріяльної продукції,
є то коротша, отже, частина часу, завойована для вільної, розумової
й суспільної діяльносте індивідів, є то більша, що рівномірніше
поділено працю поміж усіма дієздатними членами суспільства,
що менше одна суспільна верства може звалити природну
доконечність праці з себе на інші верстви. З цього погляду,
абсолютна межа для скорочення робочого дня є вселюдність праці.
У капіталістичному суспільстві вільний час однієї, кляси створюється
перетворенням усього життя мас на робочий час.

ще раніш стали робітниками, мусили з тієї самої причини присвятити
більшу частину свого часу збільшенню продукції». («A principal cause
of the increase of capital, during the war, proceeded from the greater exertions,
and perhaps the greater privations of the labouring classes, the most
numerous in every society. More women and children were compelled, by
necessitous circumstances, to enter upon laborious occupations; and former
workmen were, from the same cause, obliged to devote a greater portion
of their time to increase production»). («Essays on Political Economy in
which are illustrated the Principal Causes of the Present National Distress»,
London 1830, p. 248).

\index{i}{0446}  %% посилання на сторінку оригінального видання
\section{Різні формули норми додаткової вартости}

Ми бачили, що норма додаткової вартости виражається в таких
формулах:
\begin{gather*}
\text{I. }\frac{\text{додаткова вартість}}{\text{змінний капітал}} \left( \frac{m}{v}\right) =
\frac{\text{додаткова вартість}}{\text{вартість робочої сили}} = \\
= \frac{\text{додаткова праця}}{\text{доконечна праця}}
\end{gather*}

Дві перші формули виражають у формі відношення вартостей
те саме, що третя виражає у формі відношення відтинків часу,
що протягом їх ці вартості продукується. Ці формули, що одна
одну доповнюють, є строго логічні. Тим то ми находимо їх у клясичній
політичній економії, правда, щодо суті, але виробленими
несвідомо. Зате ми бачимо там такі вивідні формули:
\begin{gather*}
\text{II. }\frac{\text{додаткова праця\footnotemarkZ}}{\text{робочий день}} =
\frac{\text{додаткова вартість}}{\text{вартість продукту}} = \\
= \frac{\text{додатковий продукт}}{\text{сукупний продукт}}
\end{gather*}
\footnotetextZ{
У французькому виданні Маркс заводить цю формулу в дужки
і дає до цього таку примітку: «Ми заводимо першу формулу в дужки,
бо ясно вираженого поняття додаткової праці ми не знаходимо в буржуазній
політичній економії». \emph{Ред.}
}

Ту саму пропорцію виражено тут навпереміну то у формі
робочих часів, то у формі вартостей, що в них вони втілюються,
то у формі продуктів, що в них існують ці вартості. Звичайно,
припускається, що під вартістю продукту треба розуміти лише
вартість, новоспродуковану протягом робочого дня, а сталу частину
вартости продукту виключено.

У всіх цих формулах дійсний ступінь експлуатації праці, або
норму додаткової вартости, виражено неправильно. Хай робочий
день буде 12 годин. Якщо інші припущення нашого попереднього
прикладу лишаються незмінні, то в цьому випадку дійсний
ступінь експлуатації праці виразиться в таких пропорціях:
$\frac{\text{6 годин додаткової праці}}{\text{6 годин доконечної праці}}=
\frac{\text{додаткова вартість у 3 шилінґи}}{\text{змінний капітал у 3 шилінґи}}
=100\%$.

Навпаки, за формулою II ми маємо:
\begin{gather*}
\text{II. }\frac{\text{6 годин додаткової праці}}{\text{робочий день у 12 годин}} = \\
= \frac{\text{додаткова вартість у 3 шилінґи}}{\text{новоспродукована вартість у 6 шилінґів}} = 50\%\text{.}
\end{gather*}

Ці вивідні формули в дійсності виражають ту пропорцію,
що в ній робочий день або новоспродукована протягом нього
\parbreak{}  %% абзац продовжується на наступній сторінці

\parcont{}  %% абзац починається на попередній сторінці
\index{i}{0447}  %% посилання на сторінку оригінального видання
вартість поділяються між капіталістом і робітником. Тому, якщо
розглядати їх як безпосередні вирази ступеня самозростання
капіталу, то дійшлося б такого неправильного закону: додаткова
праця або додаткова вартість ніколи не може досягти 100\%\footnote{
Див., наприклад, «\textgerman{Dritter Brief an v. Kirchmann von Rodbertus.
Widerlegung der Ricardoschen Theorie von der Grundrente und Begründung
einer neuen Rententheorie}», Berlin 1854. Я пізніше повернуся
до цього твору, який, не вважаючи на його хибну теорію земельної ренти,
доходить суті капіталістичної продукції. — [Додаток до третього видання.
— Ми бачимо тут, як доброзичливо цінував Маркс своїх попередників,
коли находив у них якийсь справжній крок наперед, якусь вірну
нову думку. Тимчасом опубліковані листи Родбертуса до Руд. Маєра
обмежують до певної міри вищенаведене визнання. Там читаємо: «Треба
врятувати капітал не тільки від праці, але й від себе самого, а цього в
дійсності можна найкраще досягти, якщо розглядати діяльність капіталіста-підприємця як народньо-й
державногосподарську функцію,
покладену на нього капіталістичною власністю, а його дохід — як певну
форму утримання, бо ми ще не знаємо ніякої іншої соціяльної організації.
Але утримання повинні бути вреґульовані і знижені, коли вони
забагато відбирають від заробітної плати. Таким способом слід також
відбити напад Маркса на суспільство — так я назвав би його книгу\dots{}
Взагалі Марксова книга — це не так дослід про капітал, як полеміка
проти теперішньої форми капіталу, яку він сплутує з самим поняттям
капіталу, з чого саме й постають його помилки». («Briefe usw. von Dr.~Rodbertus-Jagetzow, herausgegeben von Dr.~Rud. Meyer», Berlin 1881,
Bd.~I, S. 111, 48. Brief von Rodbertus). У таких ідеологічних банальностях
зникають справді сміливі напади Родбертусових «соціяльних листів».
— \emph{Ф.~Е.}].
}. А що додаткова праця завжди може становити лише певну частину
робочого дня, або додаткова вартість — лише певну частину
новоспродукованої вартости, то додаткова праця завжди є неодмінно
менша, ніж робочий день, або додаткова вартість завжди
є менша, ніж новоспродукована вартість. Але для того, щоб
відноситися одна до однієї як $\frac{100}{100}$ вони мусили б бути між собою
рівні. Щоб додаткова праця забрала цілий робочий день (тут
мова йде про пересічний день робочого тижня, робочого року
й~\abbr{т. ін.}), доконечна праця мусила б упасти до нуля. Але коли
зникає доконечна праця, то зникає й додаткова праця, бо остання
є лише функція першої. Отже, пропорція
$\frac{\text{додаткова праця}}{\text{робочий день}} \deq{} \frac{\text{додаткова вартість}}{\text{новоспродукована вартість}}$
ніколи не може досягти межі
\frac{100}{100}, а ще менше — підвищитися до $\frac{100 \dplus{} х}{100}$. Але це цілком можлива річ
для норми додаткової вартости, або дійсного ступеня експлуатації
праці. Візьмімо, приміром, обчислення пана Л. де Ляверня,
що за ним англійський рільничий робітник дістає лише чвертину,
а капіталіст (фармер) — три чверті продукту\footnote{
Ту частину продукту, яка тільки покриває витрачений сталий
капітал, само собою зрозуміло, з цього обчислення виключено. — Пан
Л. де Лявернь, сліпий прихильник Англії, подає скорше надто низьке,
ніж високе відношення.
} або його вартости, не-
\parbreak{}  %% абзац продовжується на наступній сторінці

\parcont{}  %% абзац починається на попередній сторінці
\index{i}{0448}  %% посилання на сторінку оригінального видання
від того, як потім далі розділяється та здобич між капіталістом
і земельним власником і~\abbr{т. ін.} За цим обчисленням
додаткова праця англійського сільського робітника відноситься
до його доконечної праці як $3 : 1$, процентова норма експлуатації
дорівнює 300\%.

Шкільна метода розглядати робочий день як сталу величину
зміцнилася в наслідок застосування формул II, бо тут додаткову
працю завжди порівнюється з робочим днем даної величини.
Те саме буде, коли звертати увагу виключно на поділ новоспродукованої
вартости. Робочий день, що вже упредметнився в якійсь
спродукованій вартості, є завжди робочий день даної величини.

Вираз додаткової вартости й вартости робочої сили в частинах
новоспродукованої вартости — спосіб виразу, що, зрештою,
сам виростає з капіталістичного способу продукції і що його
значення з’ясується пізніше — приховує специфічний характер
капіталістичного відношення, а саме обмін змінного капіталу
на живу робочу силу та відповідне усунення робітника від продукту.
Замість того постає фалшива видимість відносин товариства,
де робітник і капіталіст ділять між собою продукт пропорційно
до різних факторів, що утворюють його\footnote{
А що всі розвинуті форми капіталістичного процесу продукції
є форми кооперації, то немає, природно, нічого легшого, як абстрагуватися
від їхнього специфічного антагоністичного характеру та одним
махом перетворити їх на форми вільної асоціяції, як то зробив граф
А. де Ляборд у своїй книзі «De l’Esprit de l’Association dans tous les
interêts de la Communauté», Paris 1818. Янкі Г.~Керей з таким самим
успіхом принагідно витворяє такі фіґлі навіть щодо відносин системи
рабства.
}.

А втім формули II можна завжди обернути назад у формули І.~Якщо ми, наприклад, маємо: $\frac{\text{додаткова праця в 6 годин}}{\text{робочий день у 12 годин}}$
то доконечний робочий час дорівнює робочому дневі в 12 годин мінус
додаткова праця в 6 годин.

Отже, маємо:\[
\frac{\text{додаткова праця в 6 годин}}{\text{доконечна праця в 6 годин}} \deq{} \frac{100}{100}\text{.}
\]

\noindent{}Третя формула, яку я, забігаючи наперед, принагідно вже
наводив, така:
\begin{gather*}
\text{III. }
\frac{\text{додаткова вартість}}{\text{вартість робочої сили}} \deq{}
\frac{\text{додаткова праця}}{\text{доконечна праця}} \deq{} \frac{\text{неоплачена праця}}{\text{оплачена праця}}
\text{.}
\end{gather*}
Те непорозуміння, що до нього могла призвести формула
$\frac{\text{неоплачена праця}}{\text{оплачена праця}}$, непорозуміння, наче капіталіст оплачує працю,
а не робочу силу, відпадає після поданих вище міркувань.
\parbreak{}  %% абзац продовжується на наступній сторінці

\input{i/_0449c.tex}
  \parcont{}  %% абзац починається на попередній сторінці
\index{iii1}{0103}  %% посилання на сторінку оригінального видання
й нервів і мозку. Справді, тільки шляхом найпотворнішого марнотратства індивідуального розвитку
забезпечується і здійснюється
розвиток людства взагалі в цю історичну епоху, яка безпосередньо передує свідомій перебудові
людського суспільства. Через те що вся економія, про яку тут іде мова, виникає з суспільного
характеру праці, то фактично саме цей безпосередньо
суспільний характер праці породжує це марнотратство життям і
здоров’ям робітників. Характерним у цьому відношенні є питання,
поставлене вже фабричним інспектором Б. Бекером: „Все питання потребує серйозного обміркування того,
яким способом
можна найкраще відвернути це \emph{жертвування дитячим життям,
яке спричинює праця тісно скупченими масами} (\emph{congregational
labour})“ („Rep. of Insp. of Fact., Oct. 1863“, стор. 157).

\emph{Фабрики}. Сюди належить відсутність будь-яких охоронних
заходів для безпеки, комфорту й здоров’я робітників також і на
фабриках у власному значенні слова. Більша частина бюлетенів
убою, які перелічують ранених і вбитих промислової армії (дивись щорічні фабричні звіти), виходить
звідси. Так само недостача місця, повітря і~\abbr{т. д.}

Ще в жовтні 1855 року Леонард Горнер скаржився на опір
дуже значного числа фабрикантів вимогам закону про захисні
пристрої до горизонтальних валів, не зважаючи на те, що небезпека постійно доводиться нещасними,
часто смертельними
випадками, і що такі захисні пристрої і не дорогі і ніяк не заважають виробництву („Rep. of Insp. of
Fact., Oct. 1855“, стор. 6 [7]).
В цьому опорі цим та іншим постановам закону фабрикантів
відкрито підтримували неплатні мирові судді, які, самі здебільшого фабриканти або друзі'
фабрикантів, мали розв’язувати ці судові справи. Якого роду були вироки цих панів,
видно з слів вищого судді Кемпбеля з приводу одного з таких
вироків, на який йому подано було апеляційну скаргу: „Це —
не тлумачення парламентського акта, це — просто скасування
його“ (там же, стор. 11). В тому самому звіті Горнер оповідає, що на багатьох фабриках машини
пускають у рух без
попередження про це робітників. Через те що й на спиненій
машині завжди є що робити, при чому ця робота завжди виконується руками й пальцями, нещасні випадки
виникають просто
внаслідок неподачі сигналу (там же, стор. 44). Для опору
фабричному законодавству фабриканти утворили в ті часи
у Манчестері тред-юніон, так звану „National Association for the
Amendment of the Factory Laws“ („Національна асоціація для поліпшення фабричних законів“), який у
березні 1855~\abbr{р.} внесками
по 2 шилінги від кінської сили зібрав суму понад \num{50000}\pound{ фунтів
стерлінгів}, щоб з неї оплачувати судові витрати своїх членів
по судових скаргах фабричних інспекторів і вести процеси від
імени асоціації. Завдання полягало в тому, щоб довести, що killing no murder [умертвіння не є
вбивство], коли це робиться
задля зиску. Шотландський фабричний інспектор, сер Джон
\parbreak{}  %% абзац продовжується на наступній сторінці

\index{ii}{0104}  %% посилання на сторінку оригінального видання
Відділ другий

Оборот капіталу

Розділ сьомий

Час обороту й число оборотів

Ми бачили: сукупний час циркуляції даного\footnote*{
Термін „сукупний час циркуляції“ тут Маркс вживає в тому самому розумінні,
в якому він далі в цьому ж розділі вживає термін „час обороту“, тимчасом
як взагалі він з цій книзі термін „час циркуляції“ вживає в тому самому
розумінні, що і „час обігу“, тобто в розумінні того часу, що протягом його капітал
перебуває в сфері циркуляції. (Дивись розділ V). \emph{Ред.}
} капіталу дорівнює сумі
часу його обігу та часу його продукції. Це є відтинок часу від моменту
авансування капітальної вартости в певній формі до моменту, коли капітальна
вартість, що процесує, повертається в тій самій формі.

Мета, що визначає капіталістичну продукцію, завжди є зростання
авансованої вартости, чи авансовано цю вартість в її самостійній формі,
тобто в грошовій формі, чи в формі товару, так що його форма вартости
має лише ідеальну самостійність у ціні авансованих товарів.
В обох випадках ця капітальна вартість перебігає протягом свого кругобігу
різні форми існування. Її тотожність з самою собою констатується
в книгах капіталіста або в формі рахункових грошей.

Хоч візьмемо ми форму $Г\dots{} Г'$, хоч форму $П\dots{} П$, обидві форми
значать: 1) що авансована вартість функціонувала як капітальна вартість
і зросла своєю вартістю; 2) що по закінченні процесу вона повернулась
до тієї форми, в якій почала його. Зростання авансованої вартости Г і
разом з тим поворот капіталу до цієї форми (до грошової форми) виразно
помітно в $Г\dots{} Г'$. Але те саме відбувається і в другій формі. Бо
вихідний пункт для П є наявність елементів продукції, товарів даної
вартости. Ця форма має в собі зростання цієї вартости (Т' і $Г'$) і поворот
до первісної форми, бо в другому П авансована вартість
знову має форму елементів продукції, що в ній її первісно авансовано.
Раніше ми бачили:» Якщо продукція має капіталістичну форму, то
і репродукція має ту саму форму. Як процес праці за капіталістичного
способу продукції є лише засіб для процесу зростання вартости, так
\parbreak{}  %% абзац продовжується на наступній сторінці

\parcont{}  %% абзац починається на попередній сторінці
\index{iii1}{0105}  %% посилання на сторінку оригінального видання
головним чином внаслідок запровадження нових машин, які вже
з самого початку виробляються з захисними пристроями, з якими
фабрикант мириться, бо вони не вимагають від нього додаткових
витрат. Крім того, декільком робітникам удалось добитися через
суд значного відшкодування за свої втрачені руки і відстояти ствердження цих судових вироків вищою
інстанцією („Rep. of
Insp. of Fact., 30 April 1861“, стор. 31; також April 1862, стор. 17 [18]).

Цього досить в питанні про економію на засобах забезпечення
життя і членів тіла робітників (серед яких багато дітей) від небезпек, які виникають безпосередньо з
уживання їх коло машин.

\emph{Праця в закритих приміщеннях взагалі}. — Відомо, в якій
великій мірі економія на просторі, отже й на будівлях, приводить до скупченості робітників у тісних
приміщеннях. До цього
долучається ще економія на засобах вентиляції. Разом з довшим
робочим часом обидві ці причини викликають значне збільшення
хвороб органів дихання, а тому й збільшення смертності. Нижченаведені ілюстрації взяті з звіту про
„Public Health, 6-th Rep.
1863“ [народне здоров’я]; звіт складений доктором Джоном
Сімоном, добре відомим з нашої першої книги.

Подібно до того, як комбінація робітників і кооперація їх
допускає застосовування машин у великому масштабі, концентрацію засобів виробництва і економію в
застосуванні їх, так
само ця спільна робота великих мас у закритих приміщеннях
і при таких обставинах, коли вирішальним є не здоров’я робітників, а успішніше виготовлення
продукту, — ця масова концентрація робітників у тій самій майстерні є, з одного боку,
джерелом зростаючого зиску для капіталістів, а, з другого боку,
якщо вона не компенсується скороченням робочого часу і спеціальними охоронними заходами, разом з тим
причиною марнотратства життям і здоров’ям робітників.

Доктор Сімон встановлює таке загальне правило, яке він доводить масовими статистичними даними: „В
тій самій мірі, в якій
населення певної місцевості змушене спільно працювати в закритих приміщеннях, в тій самій мірі
зростає, при інших
незмінних умовах, норма смертності цієї округи внаслідок
хвороб легенів“ (стор. 23). Причина — погана вентиляція. „Мабуть, у всій Англії немає жодного
винятку з того загального
правила, що в кожній окрузі, яка має значну, проваджену
в закритих приміщеннях промисловість, збільшена смертність
цих робітників достатня для того, щоб забарвити статистику
смертності всієї округи значним переважанням хвороб легенів“
(стор. 23).

З статистики, смертності в тих галузях промисловості, в яких
працюють у закритих приміщеннях і які в 1860 і 1861~\abbr{рр.} були
досліджені санітарним відомством, виявляється: на те саме число
чоловіків між 15 і 55 роками, на яке в англійських землеробських округах припадає 100 випадків
смерті від сухот і інших
хвороб легенів, припадає: у Ковентрі — 163 випадки смерті від
\parbreak{}  %% абзац продовжується на наступній сторінці

\input{_0106.tex}
\input{_0107.tex}
\parcont{}  %% абзац починається на попередній сторінці
\index{i}{0108}  %% посилання на сторінку оригінального видання
набирає у простій циркуляції, упосереднюють лише обмін товарів
і зникають у кінцевому результаті руху. Навпаки, в циркуляції
$Г — Т — Г$ функціонують обидва, товар і гроші, лише як різні
способи існування самої вартости: гроші як її загальна, товар
як її осібна, так би мовити, лише замаскована форма існування\footnote{
«Не речовина становить капітал, а вартість цієї речовини» («Се
n’est pas la matière, qui fait le capital, mais la valeur de cette matière»).
(\emph{J.~B.~Say}: «Traité d’Economie Politique», 3-ème éd. Paris 1817, vol. II,
p. 429).
}.
Вартість постійно переходить з однієї форми в другу, ніколи не
зникаючи в цьому русі, і таким чином перетворюється на автоматичний
суб’єкт. Коли фіксувати осібні форми виявлення, що
їх у кругобігу свого життя навпереміну набирає вартість, що
самозростає, то виходять такі визначення: капітал є гроші, капітал
є товар\footnote{
«Обігові гроші (!), що вжиті з продуктивною метою, є капітал»
(«Currency (!) employed to productive purposes is capital»). (\emph{Mac Leod}:
«The Theory and Practice of Banking», London 1855, vol. I, ch. 1, p. 55).
«Капітал — це товари» («Capital is commodities»). (James Mill: «Elements
of Political Economy», London 1821, p. 74).
}. Але в дійсності вартість стає тут суб’єктом процесу,
в якому вона, постійно змінюючи свою грошову форму на товарову
й товарову на грошову, сама змінює свою величину, відштовхує
себе як додаткову вартість від самої себе як первісної
вартости, самозростає. Бо рух, що в ньому вона прилучає до себе
додаткову вартість, є її власний рух, отже, її зростання є самозростання.
Вона набула магічної якости плодити вартість через
те, що вона є вартість. Вона виплоджує живі діти або, принаймні,
несе золоті яйця.

Вартість як активний суб’єкт такого процесу, що в ньому
вона, то набираючи, то скидаючи грошову форму й товарову
форму, в цій зміні зберігає себе й зростає, потребує насамперед
самостійної форми, за допомогою якої констатується її тотожність
із нею самою. І цю форму вона має лише в грошах. Тому гроші
становлять вихідний і кінцевий пункт кожного процесу зростання
вартости. Вона дорівнювала 100\pound{ фунтам стерлінґів}, вона тепер
є 110\pound{ фунтів стерлінґів} і~\abbr{т. д.} Але сами гроші мають тут силу
лише як одна з форм вартости, бо у неї їх дві. Не набравши товарової
форми, гроші не стають капіталом. Отже, гроші тут не виступають
проти товарів вороже, як за скарботворення. Капіталіст
знає, що всі товари, хоч якими обідраними вони виглядали б,
або хоч як погано вони пахли б, по вірі й правді є гроші, так би
мовити, євреї істинного обрізання, а до того ще й чудотворний
засіб із грошей робити більше грошей.

Коли в простій циркуляції вартість товарів супроти їхньої
споживної вартости набирає щонайбільше самостійної форми
грошей, то тут вона раптом виступає як субстанція, що процесує,
сама собою рухається, субстанція, що для неї товар і гроші обоє
є лише форми; навіть більше: замість репрезентувати відношення
товарів вона вступає тепер, так би мовити, у приватне відношення
\parbreak{}  %% абзац продовжується на наступній сторінці

\parcont{}  %% абзац починається на попередній сторінці
\index{iii2}{0109}  %% посилання на сторінку оригінального видання
в 200\pound{ ф. ст.} може розглядатися як процент на капітал в \num{4.000}\pound{ ф. ст}. Капіталізована
таким чином земельна рента і становить купувальну ціну або вартість землі,
категорія, що prima facie, так само як і ціна праці є іраціональна, бо земля не є продукт
праці, отже, не має жодної вартости. Але, з другого боку, за цією іраціональною
формою ховається дійсне продукційне відношення. Коли капіталіст купує
землю, що дає річну ренту в 200\pound{ ф. ст.} за \num{4.000}\pound{ ф. ст.}, то він одержує пересічний річний
процент в 5\% з \num{4.000}\pound{ ф. ст.} — цілком так само, якби він вклав цей капітал у
процентні папери, або безпосередньо віддав його в позику з 5\%. Це зростання вартости
капіталу в \num{4.000}\pound{ ф. ст.} на 5\%. При такому припущенні він повернув би собі
за 20 років купівельну ціну свого маєтку доходами з нього. Тому в Англії купівельна
ціна землі обчислюється за певним числом years’ purchase\footnote*{
Роки, що протягом їх оплачується купівельну ціну. Прим. Ред.
}, що є лише іншим
виразом капіталізування земельної ренти. На ділі, це купівельна ціна, — не
землі, а тієї земельної ренти, яву вона дає, — обчислена відповідно до звичайного
розміру проценту. Але ця капіталізація ренти має своєю передумовою
ренту, тимчасом як ренти не можна вивести й пояснити в зворотному порядку
з її власної капіталізації. Її існування, незалежно від продажу, є тут
передумовою, вихідним пунктом.

З цього випливає, що, припускаючи земельну ренту незмінною щодо величини,
ціна землі може підвищуватись або падати в зворотному напрямку з
підвищенням і падінням розміру проценту. Коли б звичайний розмір проценту
знизився з 5 до 4\%, то річна земельна рента в 200\pound{ ф. ст.} становила б річне
зростання вартости з капіталу вже не в \num{4.000}\pound{ ф. ст.} а в \num{5.000}\pound{ ф. ст.} і таким чином
ціна тієї самої ділянки землі підвищилась би з \num{4.000} до \num{5.000}\pound{ ф. ст.} або з 20 years’
purchase до 25. Зворотне в зворотному випадку. Це незалежний від руху самої
земельної ренти і реґульований лише розміром проценту рух земельної ціни.
А що ми бачили, що з поступом суспільного розвитку норма зиску, а тому і
розмір проценту, оскільки він регулюється нормою зиску, має тенденцію
знижуватися; що, далі, навіть лишаючи осторонь норму зиску, розмір проценту
має тенденцію знижуватися в наслідок зросту позичкового грошового капіталу,
то з цього випливає, що ціна землі має тенденцію підвищуватись і незалежно
від руху земельної ренти та ціни земельних продуктів, частину якої становить
рента.

Сплутування самої земельної ренти з тією процентною формою, яку вона
набуває для покупця землі — сплутування, що ґрунтується на цілковитій
непізнанності природи земельної ренти, — мусить привести до найдивовижніших
помилкових висновків. Але що земельну власність у всіх старих країнах
вважається за особливо почесну форму власности, а купівлю земельної власности
— за особливо певне приміщення капіталу, то розмір проценту, з якого
обчислюється купувальна ціна земельної ренти, є звичайно нижчий, ніж його розмір
при інших способах приміщення капіталу, розрахованих на порівняно довший час,
так що, наприклад, покупець землі одержує на купівельну ціну її лише 4\%, тимчасом
як він одержав би на той самий капітал при іншому способі приміщення 5\%;
або, що сходить на те саме, він платить за земельну ренту більше капіталу,
ніж довелося б йому заплатити за такий самий річний грошовий дохід в інших
сферах приміщення капіталу. Пан Тьєр у своїй взагалі цілком кепській праці про
La Propriété (відбитку його промови проти Прудона, проголошеної 1849~\abbr{р.} на французьких
Національних Зборах) робить той висновок, що земельна рента низька,
тимчасом як це тільки доводить, що купувальна ціна її висока.

Та обставина, що капіталізована земельна рента видається ціною або
вартістю землі, і що земля, таким чином, продається і купується, як усякий
інший товар, є для деяких апологетів ґрунтом для виправдання земельної
\parbreak{}  %% абзац продовжується на наступній сторінці

\input{_0110.tex}
\input{_0111.tex}
\parcont{}  %% абзац починається на попередній сторінці
\index{iii2}{0112}  %% посилання на сторінку оригінального видання
аристократією і буржуазією самим собі, — з очевидністю, поза всяким сумнівом
доводять, що високі ренти і відповідний їм зріст земельних цін підчас антиякобінської
війни завдячують почасти тільки вирахуванню з заробітної плати
і її пониженню навіть нижче від фізичного рівня; тобто завдячують виплаті
частини нормальної заробітної плати земельному власникові. Різні обставини,
між іншим, знецінення грошей, спосіб додержування законів про бідних
у хліборобських округах і~\abbr{т. ін.} уможливили цю операцію в той самий час, коли
доходи орендарів колосально зростали і землевласники казково збагачувались.
А втім, одним з головних арґументів так орендарів, як і землевласників на
користь запровадження хлібних мит був той, що фізично неможливо ще більше
знизити заробітну плату сільських поденників. Це становище по суті не змінилось,
і в Англії, як в усіх європейських країнах, частина нормальної
заробітної плати по-старому входить до складу земельної ренти. Коли
граф Shaftesbury, в той час лорд Ashley, один з філантропів-аристократів,
був так надзвичайно зворушений становищем англійських фабричних робітників
і виступив їхнім парламентським оборонцем в справі аґітації за десятигодинний
день — на помсту за це оборонці промисловців опублікували статистичні
дані про заробітну плату сільських поденників у належних йому селах (див.
книга І, розділ XXIII, 5, е: британський хліборобський пролетаріят), які
ясно показали, що частина земельної ренти цього філантропа постає просто
з грабунку, що за нього чинять його орендарі над заробітною платою
хліборобських робітників. Ця публікація ще й тим цікава, що наведені в ній
факти можуть сміливо стати поряд з усім найгіршим, що розкрили комісії
1814 і 1815~\abbr{рр.} Скоро тільки обставини примушують до тимчасового підвищення
заробітної плати хліборобських робітників, так орендарі починають кричати, що
підвищення заробітної плати до нормального рівня, якого вона досягає в інших
галузях промисловости, є річ неможлива, яка неминуче зруйнує їх, коли
одночасно не знизити земельної ренти. Отже, тут робиться визнання, що під
ім’ям земельної ренти орендарі роблять вирахування з заробітної плати і виплачують
його землевласникові. Наприклад, 1849--1859~\abbr{р.} заробітна плата хліборобських
робітників в Англії підвищилась в наслідок збігу потужних обставин
як от: еміграція з Ірландії, що припинила приплив звідти хліборобських
робітників; надзвичайне вбирання хліборобської людности фабричною промисловістю;
попит на салдатів для війни; надзвичайна еміграція в Австралію
і Сполучені Штати (в Каліфорнію) та інші причини, на яких тут не доводиться
спинятися докладніше. Одночасно за цей період, за винятком 1854--1856~\abbr{рр.}, років
з кепськими урожаями, пересічні ціни збіжжя понизились більше, ніж на 16\%.
Орендарі волали про пониження рент. В окремих випадках вони досягли цього.
Взагалі ж вони з цією вимогою зазнали поразки. Вони вдалися до пониження
витрат продукції, між іншим засобом масового вживання парових
локомобілів і нових машин, які почасти замінили й витиснули з господарства
коней, а почасти, звільняючи хліборобських робітників, зумовили
штучне перелюднення, а тому і нове зниження заробітної плати. І це відбувалось,
не зважаючи на загальне відносне зменшення хліборобської людности за це
десятиліття, порівняно з ростом усієї людности, і не зважаючи на абсолютне зменшення
хліборобської людности в деяких суто-хліборобських округах\footnote{
John C.~Morton, The Forces used in Agriculture. Доповідь у Лондонському Society of Arts 1860
року, основана на автентичних документах, зібраних безпосередньо приблизно у 100 орендарів в 12
шотляндських і 35 англійських графствах.
}. Так
само, 12 жовтня 1865 року Fowcett, тоді професор політичної економії у Кембріджі,
що вмер 1884 року генерал-почтмайстером, говорив на Social Science
Congress: хліборобські поденники починають еміґрувати, і орендарі починають
\parbreak{}  %% абзац продовжується на наступній сторінці

\parcont{}  %% абзац починається на попередній сторінці
\index{iii2}{0113}  %% посилання на сторінку оригінального видання
скаржитись, що вони не зможуть платити таких високих рент, як звичайно
платили, бо в наслідок еміґрації праця дорожчає». Отже, тут висока земельна
рента прямо ототожнюється з низькою заробітною платою. І оскільки висота
земельної ціни зумовлюється цією обставиною, яка підвищує ренту, остільки підвищення
вартости землі тотожне із знеціненням праці, високий рівень земельної
ціни — з низьким рівнем ціни праці.

Те саме і у Франції. «Орендна плата підвищується, бо на однім боці підвищується
ціна хліба, вина, м’яса, городини і овочів, а на другім боці ціна
праці лишається незмінна. Коли б старі люди порівняли рахунки їхніх батьків, —
що відсунуло б нас назад майже на 100 років, — вони побачили б, що тоді ціна
робочого дня у сільській Франції була достоту така, як і тепер. Ціна м’яса від того
часу збільшилась утроє\dots{} Хто жертва цього перевороту? Чи багатий, що є власник
здаваної в оренду землі, чи бідняк, що її обробляє?\dots{} Зріст орендних
цін є доказ суспільного лиха». (Du Mécanisme de la Société en France et en
Angleterre. Par 1. Rubichon, 2-me édit. Paris 1837, p. 101).

Приклади ренти, як наслідку вирахування, з одного боку, з пересічного
зиску, з другого — з пересічної заробітної плати:

Цитований вище Мортон, сільський аґент і сільсько-господарський інженер,
каже, що в багатьох місцевостях зроблено спостереження, що рента
за великі оренди нижча, ніж за дрібні, бо «конкуренція за останні, звичайно,
більша, ніж за перші, і тому що дрібні орендарі, які рідко мають можливість
узятись до якогось іншого діла, крім хліборобства, вимушені пекучою потребою
знайти підхоже діло, часто погоджуються платити таку ренту, про яку вони
сами знають, що вона надто висока». (John С. Morton, The Resources of Estates.
London 1885. p. 116).

Проте, на його думку, в Англії ця ріжниця поступово згладжується, чому,
як він вважає, дуже сприяє еміґрація кляси саме дрібних орендарів Той самий
Мортон наводить приклад, коли в земельну ренту безперечно входить вирахування
з заробітної плати самого орендаря, а тому ще безперечніше і з заробітної
плати робітників, які в нього працюють. А саме, коли орендні дільниці
менші, ніж 70--80 акрів (30--34 гектари), при яких неможливо держати парокінний
плуг. «Коли орендар не працює своїми власними руками так само
дбало, як будь-який робітник, він не може існувати від своєї оренди. Коли він
виконання роботи покладе на своїх людей, а сам обмежиться виключно наглядом
за ними, то він, найімовірніше, дуже скоро виявить, що не зможе виплачувати
орендної плати» (1. с., р 118). З цього Мортон висновує, що коли орендарі
в краю не дуже бідні, то розміри віддаваних на оренду дільниць не повинні
бути менші за 70 акрів, щоб орендар міг держати двох або трьох коней.

Надзвичайна мудрість пана Léonce de Lavergne, Membre de l’Institut et de
la Société Centrale d’Agriculture. У своїй Economie Rurale de l’Angletterre (цитовано
з англійського перекладу, London 1855), він робить таке порівняння
річних вигід від рогатої худоби, яку у Франції вживається для роботи, а в Англії
не вживається, бо її заміняють коні (р. 42):
\begin{center}
\begin{tabular} {l c@{ } c l c@{ } c}
  \multicolumn{3}{c}{Франція:} & \multicolumn{3}{c}{Англія:} \\

  молоко & \phantom{0}4 & міл. ф. ст. & молоко & 16 & міл. ф. ст.\\

  м'ясо & 16 & \ditto{міл. ф. ст.} & м’ясо & 20 & \ditto{міл. ф. ст.}\\

  робота & \phantom{0}8 & \ditto{міл. ф. ст.} & робота &  \textemdash & \ditto{міл. ф. ст.}\\

  \cmidrule(rl){2-3}
  \cmidrule(rl){5-6}

         & 28 & міл. ф. ст. &  & 36 & міл. ф. ст.
\end{tabular}
\end{center}

Але вищий продукт для Англії тут виходить лише тому, що згідно з його
власними даними молоко в Англії коштує удвоє дорожче, ніж у Франції, тимчасом
як для м’яса він припускає однакові ціни в обох країнах (р. 35); отже, молочний
\parbreak{}  %% абзац продовжується на наступній сторінці

\parcont{}  %% абзац починається на попередній сторінці
\index{iii1}{0114}  %% посилання на сторінку оригінального видання
обміну речовин у людини, почасти ту форму, в якій предмети
споживання лишаються після споживання їх. Отже, покидьки
виробництва в хемічній промисловості є побічні продукти, які
при незначному масштабі виробництва пропадають марно; залізні
стружки, які лишаються при фабрикації машин і знову входять
як сировинний матеріал у виробництво заліза і~\abbr{т. д.} Екскременти споживання — це виділювані людиною
природні речовини,
рештки одягу у формі ганчірок і~\abbr{т. д.} Екскременти споживання
мають найбільше значення для землеробства. Щодо застосування
їх, капіталістичне господарство відзначається колосальним марнотратством; у Лондоні, наприклад, воно
не знаходить кращого
застосування для екскрементів 4\sfrac{1}{2} мільйонів людей, як з величезними витратами заражати ними Темзу.

Подорожчання сировинних матеріалів є, звичайно, спонукою
до використовування відпадів.

Загалом умовами цього повторного використання є: масовість
цих екскрементів, яка можлива тільки при роботах у великому
масштабі; поліпшення машин, завдяки чому речовини, які раніш
у своїй даній формі були непридатні до вжитку, переходять
у таку форму, в якій вони можуть бути використані в новому
виробництві; прогрес науки, особливо хемії, яка відкриває корисні властивості таких відпадів.
Правда, і в дрібному землеробстві, де поля обробляються як сади, як от у Ломбардії, південному Китаї
та Японії, також має місце значна економія
цього роду. Але, загалом кажучи, при цій системі продуктивність землеробства купується великим
марнотратством людської робочої сили, відтягуваної від інших сфер виробництва.

Так звані відпади відіграють значну роль майже в кожній
галузі промисловості. Так, наприклад, у грудневому фабричному звіті за 1863 рік [стор. 139]
наводиться як одна з головних
причин того, чому в Англії — як і в багатьох частинах Ірландії —
орендарі тільки неохоче й рідко сіють льон, ось що: „Значна
кількість відпадів\dots{} які відходять при обробітку льону в дрібних льонотіпальних фабриках, де
рушійною силою є вода
(scutch mills)\dots{} Відпадів від бавовни порівняно небагато, а при
обробленні льону їх дуже багато. Старанна робота при мочінні і механічному тіпанні льону може значно
обмежити цю
втрату\dots{} В Ірландії льон часто тіпають надзвичайно незадовільним способом, так що 28--30\% його
пропадало марно“; усе це
могло б бути усунене при застосуванні кращих машин. Костриця при цьому відпадає в такій великій
кількості, що фабричний інспектор каже: „З деяких тіпальних фабрик в Ірландії
мене повідомили, що тіпальники часто вживають у себе дома
відпади, які утворюються на цих фабриках, як паливний матеріал для своїх печей, а це ж дуже цінний
матеріал“ („Rep. of
Insp. of Fact. Oct., 1863“, стор. 140). Про відпади бавовни мова
буде далі, там, де ми розглядаємо коливання цін на сировинний
матеріал.

\input{_0115.tex}
\parcont{}  %% абзац починається на попередній сторінці
\index{i}{0116}  %% посилання на сторінку оригінального видання
ближче до цього. Перед обміном було вина на 40\pound{ фунтів стерлінґів}
у руках $А$ і збіжжя на 50\pound{ фунтів стерлінґів} у руках $В$, разом
вартости на 90\pound{ фунтів стерлінґів}. Після обміну маємо ту саму
загальну вартість в 90\pound{ фунтів стерлінґів}. Вартість, що циркулює,
не збільшилась ні одним атомом, а тільки змінився розподіл її
між $А$ й $В$. Те, що на одному боці з’являється як додаткова вартість,
є зменшення вартости на другому боці, плюс на одному
боці є мінус на другому. Та сама зміна відбулася б, коли б $А$, не
прикриваючись формою обміну, безпосередньо украв у $В$ 10\pound{ фунтів
стерлінґів}. Очевидно, що суму вартости, яка циркулює, не
можна збільшити жодною зміною в її розподілі, так само як єврей
не збільшує маси благородного металю в країні тим, що продає
за одну ґінею один фартинґ із часів королеви Ганни. Ціла
кляса капіталістів країни не може вигравати коштом самої себе.\footnote{
Детю де Трасі, хоч він є член Інституту, — а, може, саме тому,
що він є член Інституту, — був протилежної думки. «Промислові капіталісти,
— каже він, — мають свої зиски з того, що вони всі товари продають
дорожче, ніж коштувала їх продукція. Кому ж продають вони їх?
Поперше, один одному». («Traité de la Volonté et de ses effets», Paris
1826, p. 239).
}

Хоч верть, хоч круть, а факт лишається той самий. Коли обмінюється
еквіваленти, то не постає жодної додаткової вартости,
і коли обмінюється не-еквіваленти, то теж не постає жодної додаткової
вартости.\footnote{
«Обмін двох рівних вартостей не збільшує й не зменшує маси вартостей,
що є в суспільстві. Обмін двох нерівних вартостей\dots{} також нічого
не змінює в сумі суспільних вартостей, хоч і додає до майна одного те,
що він бере від майна іншого». («L’échange qui se fait de deux valeurs
égales n’augmente ni ne diminue la masse des valeurs exisstantes dans la
société. L’échange de deux valeurs inégales\dots{} ne change rien non plus à ia
somme des valeurs sociales, bien qu’il ajoute à la fortune de l’un ce qu’il
ôte de la fortune de l’autre»). (\emph{J. B. Say}: «Traité d’Economie Politique»,
Paris 1817, vol. II, p. 443, 444). Сей, звичайно, байдужий щодо висновків
з цієї тези, запозичує її майже дослівно в фізіократів. У який спосіб використовував
він для збільшення своєї власної «вартости» забуті за його
часів твори фізіократів, показує такий приклад: «Найславнішу» тезу
Monsi ur Say’a: «On n’achète des produits qu’avec des produits» («Продукти
купується лише за продукти» — там же, т. II, стор. 441) в ориґіналі
у фізіократів подано так: «Les productions ne se paient qu’avec des
productions» («Продукти оплачується лише продуктами»). (\emph{Le Trosne}:
«De l’Intérêt Social», Physiocrates, éd. Daire, Paris. 1846, p. 899).
} Циркуляція або обмін товарів не створює
жодної вартости.\footnote{
«Обмін не додає жодної вартости до продуктів» («Exchange confers
no value at all upon products»). (\emph{F. Wayland}: «The Elements of Political
Economy», Boston 1853, p. 168).
}

Звідси зрозуміло, чому в нашій аналізі основної форми капіталу,
форми, що в ній він визначає економічну організацію сучасного
суспільства, ми покищо цілком залишаємо осторонь популярні
і, так би мовити, допотопні форми капіталу — торговельний
капітал та лихварський капітал.

У русі власне торговельного капіталу форма $Г — Т — Г'$,
купити, щоб дорожче продати, з’являється в найчистішому
\parbreak{}  %% абзац продовжується на наступній сторінці

\input{_0117.tex}
\input{_0118.tex}
\input{_0119.tex}
\index{ii}{0120}  %% посилання на сторінку оригінального видання
Здебільша це залежить від площі, яка є в розпорядженні. При деяких будівлях можна надбудовувати
горішні поверхи, при інших треба поширювати в боки, тобто треба більше землі. За капіталістичної
продукції, з одного боку, багато засобів витрачається марно, а з другого боку, при поступінному
поширенні підприємства спостерігається багато випадків такого роду недоцільного поширення будівель в
боки (почасти це шкодить робочій силі), бо нічого не робиться за суспільним пляном, а все залежить
від безлічі різних обставин, засобів і т. ін., що з ними має діло капіталіст. А з цього постає
велике марнотратство продуктивних сил.

Таке повторне вкладання грошового резервного фонду частинами (тобто частини основного капіталу,
знову перетвореної на гроші) найлегше робиться в хліборобстві. Просторове обмежене поле продукції
тут якнайбільш здібне поступінно вбирати капітал. Так само стоїть справа й там, де відбувається
природна репродукція, як, напр., у скотарстві.

Основний капітал спричиняє особливі витрати на зберігання. Почасти це зберігання здійснюється самим
процесом праці; основний капітал псується, коли він не функціонує в процесі праці (див. кн. І, розд.
VI і розд. XIII. Зношування машин, що постає від їх невживання). Тому англійський закон вважає
буквально за шкоду (waste), коли орендовані ділянки не обробляється заведеним у країні способом. (W.
A. Holdsworth, Barrister at Law, „The Law of Landlord and Tenant“, London, 1857, p. 96). Це
зберігання, що походить з ужитку в процесі праці, є безплатний природний дар живої праці. Ця
властива праці сила зберігання має двоїстий характер. З одного боку, праця зберігає вартість
матеріялів праці, переносячи їх на продукт; з другого боку, оскільки вона й не переносить на продукт
вартости засобів праці, вона все ж зберігає їхню вартість, зберігаючи їхню споживну вартість тим, що
вони функціонують у процесі продукції.

Однак, для того, щоб основний капітал зберігався в належному стані, потрібні й безпосередні витрати
праці. Машини треба час від часу чистити. Тут справа йде про новододавану працю, що без неї вони
будуть непридатні до вжитку, про безпосереднє зберігання від шкідливих стихійних впливів, завжди
сполучених з продукційним процесом, отже, про зберігання машин у стані працездатности в прямому
значенні цього слова. Само собою зрозуміло, нормальну життьову тривалість основного капіталу
обчислюється, зважаючи на те, що здійсняться умови, в яких він може нормально існувати протягом
цього часу, так само як припускається, що, коли людина живе пересічно 30 років, вона також і
миється. Отже, тут ходить не про те, щоб замінити працю, яка є в машині, а про постійну новододавану
працю, потрібну в наслідок уживання машини. Тут ідеться не про ту працю, що її виконує машина, а про
ту, що прикладається до машини, тим часом як машина є не чинник продукції, а сировинний
матеріял. Капітал, витрачений на цю працю, — хоч і не входить власне в той процес праці, що йому
продукт завдячує своїм походженням, — належить до поточного капіталу. Цю працю доводиться постійно
витрачати на продукцію, а тому й вартість цієї праці завжди мусить покриватись
\parbreak{}  %% абзац продовжується на наступній сторінці

\input{_0121.tex}
\parcont{}  %% абзац починається на попередній сторінці
\index{iii1}{0122}  %% посилання на сторінку оригінального видання
той час, коли в наслідок американської громадянської війни ціна
бавовни підвищилась до нечуваного за ціле майже століття
рівня, звіт говорив зовсім інше: „Ціна, яку тепер дають за бавовняні
відпади, і повторне використання цих відпадів на
фабриці як сировинного матеріалу компенсують до певної міри
ріжницю в утраті на відпадах між індійською і американською
бавовною. Ця ріжниця становить приблизно 12 1/2\%. Втрата при
обробленні індійської бавовни становить 25\%, так що в дійсності
бавовна коштує прядільникові на 1/4 більше, ніж він за неї
платить. Втрата на відпадах не була така важлива, коли американська
бавовна коштувала 5 або 6 пенсів за 1 фунт, бо вона
не перевищувала тоді 3/4 пенса на фунт; але вона дуже важлива
тепер, коли 1 фунт бавовни коштує 2 шилінги, і втрата на відпадах
становить, отже, 6 пенсів“\footnote{
У кінцевій фразі звіту зроблено помилку. Замість 6 пенсів мусить бути З
пенси втрати на відпадах. Ця втрата становить, правда, 25\% при обробленні
індійської бавовни, але тільки 12 1/2—15\% при обробленні американської бавовни,
а мова тут іде про цю останню, при чому раніше той самий процент при
ціні в 5—6 пенсів був правильно обчислений. А втім, і при обробленні американської
бавовни, яка довозилась до Европи протягом останніх років громадянської
війни, процент відпадів часто був значно вищий, ніж у попередні
часи. — Ф. Е.
} („Rep. of Insp. of Fact., Oct.
1863“, стор. 106).

II. Підвищення й зниження вартості капіталу, звільнення
і зв’язування капіталу

Явища, які ми досліджуємо в цьому розділі, для свого повного
розвитку передбачають наявність кредитної справи і конкуренції
на світовому ринку, який взагалі становить базу й життьову атмосферу
капіталістичного способу виробництва. Але ці конкретніші
форми капіталістичного виробництва можуть бути вичерпно розглянуті
тільки після того, як буде з’ясована загальна природа
капіталу; крім того, розгляд цих форм не входить у план нашої
праці і належить до можливого продовження її. Проте, явища,
зазначені в заголовку, можуть бути тут розглянуті в загальній
формі. Вони зв’язані, поперше, між собою, а подруге, як
з нормою, так і з масою зиску. Їх треба коротко розглянути
хоч би вже тому, що вони викликають ілюзію, ніби не тільки
норма, але й маса зиску — яка в дійсності тотожна з масою
додаткової вартості — може зменшуватись або збільшуватись
незалежно від рухів додаткової вартості, чи то її маси чи її
норми.

Чи можна розглядати звільнення і зв’язування капіталу на
одному боці, підвищення і зниження його вартості на другому,
як різні явища?

Насамперед постає питання: що розуміємо ми під звільненням
і зв’язуванням капіталу? Підвищення і зниження вартості
\parbreak{}  %% абзац продовжується на наступній сторінці

\parcont{}  %% абзац починається на попередній сторінці
\index{iii2}{0123}  %% посилання на сторінку оригінального видання
до життя цю природну умову підвищеної продуктивної сили праці, в такий
спосіб як кожен капітал може воду перетворити в пару. Ця природна умова
трапляється в природі лише місцями, і там, де її немає, її неможливо створити
певного витратою капіталу. Вона зв’язана не з продуктами, створюваними працею,
як машини, вугілля тощо, а з певними природними відносинами певної частини
землі. Та частина фабрикантів, що їм належать водоспади, усувають ту частину
фабрикантів, у яких немає водоспадів, від застосування цієї природної сили,
бо земля — і тим паче земля, що має водну силу, — обмежена. Це не виключає
того, що хоч кількість природних водоспадів у певній країні обмежена, кількість
водяної сили, яку може використовувати промисловість, може бути збільшена.
Водоспад можна штучно відвести, щоб цілком використати його рушійну силу;
коли вже є водоспад, водяне колесо можна удосконалити, щоб більше використати
силу води; там, де для подачі води звичайне колесо непридатне, можна застосувати
турбіни і~\abbr{т. ін.} Посідання цією природною силою становить монополію
в руках її посідача, таку умову високої продуктивної сили вкладеного капіталу,
яку не можна створити продукційним процесом самого капіталу\footnote{
Див. про надзиск „Inquiry“ (проти Мальтуса).
}; ця природна
сила, яка може бути так монополізована, завжди зв’язана з землею. Така природна
сила не належить ні до числа загальних умов згаданої сфери продукції, ні до
числа таких її умов, що їх можна створити як загальні умови.

Тепер, коли ми собі уявимо що водоспади разом з прилежною до них землею
перебувають в руках осіб, які вважаються власниками цих частин землі,
землевласниками, то ми побачимо, що вони не дозволяють прикладати капітал
до водоспаду, використовувати його з допомогою капіталу. Вони можуть
дозволити і не дозволити використання водоспаду. Але капітал не може створити
водоспад із себе. Тому надзиск, що постає з цього використання водоспаду, постає
не з капіталу, а з застосування капіталом цієї природної сили, яку монополізувати
можна і яка монополізована. В таких обставинах надзиск перетворюється на земельну
ренту, тобто він дістається власникові водоспаду. Коли фабрикант виплачує
йому за його водоспад 10\pound{ ф. ст.} на рік, то його зиск становить 15\pound{ ф. ст.}; 15\% на
ті 100\pound{ ф. ст.}, що їх тепер досягають його витрати продукції; і він опиняється
тепер цілком в такому самому становищі, може в кращому, ніж усі інші капіталісти
його сфери продукції, що працюють з допомогою пари. Справа ані трохи
не відмінилась би від того, коли б капіталіст сам був власником водоспаду. Він,
як і раніш, одержував би надзиск в 10\pound{ ф. ст.} не як капіталіст, а як власник водоспаду,
і саме тому, що цей надмір постає не з його капіталу, як такого, а
з порядкування такою природною силою, що її можна відділити від його капіталу,
що її можна монополізувати, та яка обмежена в своїх розмірах, — саме тому, цей
надмір переворюється на земельну ренту.

\emph{Перше}: Ясно, що ця рента завжди становить диференційну ренту, бо
вона не ввіходить визначально в загальну ціну продукції товару, а навпаки,
має її за передумову. Вона завжди виникає з ріжниці між індивідуальною
ціною продукції, для окремого капіталу, який порядкує монополізованою природною
силою, і загальною ціною продукції для капіталу, взагалі вкладеного у згадану
сферу продукції.

\emph{Друге}: Ця земельна рента постає не з абсолютного підвищення продуктивної
сили застосованого капіталу — зглядно привласненої ним праці, —
що взагалі могло б призвести лише до зменшення вартости товарів; а
з більшої відносної продуктивности певних окремих капіталів, приміщених
в певну сферу продукції, порівняно з тими приміщенями капіталу, які усунені
від цих виключних, створених природою сприятливих умов підвищення
продуктивної сили. Коли б, наприклад, не зважаючи на те, що вугілля має вартість,
\index{iii2}{0124}  %% посилання на сторінку оригінального видання
а сила води не має вартости, користання парою все ж давало б рішучі
переваги, недосяжні при використанні сили води, і коли б ці переваги більше
ніж компенсували силу води, то сила води не мала б застосування і не могла б
породити жодного надзиску, а, отже, і ренти.

\emph{Третє}: Сила природи не є джерело надзиску, а лише його природна
база, бо це є природна база виключно підвищеної продуктивної сили праці. Так
взагалі споживна вартість є носій мінової вартости, а не причина її. Коли б
ту саму споживну вартість можна було створювати без праці, вона б не мала
жодної мінової вартости, але як і давніш, мала б свою природну корисність
як споживна вартість. Але, з другого боку, без споживної вартости, отже,
без такого природного носія праці, річ не має жодної мінової вартости. Коли б
різні вартості не вирівнювались у ціни продукції і різні індивідуальні ціни
продукції не вирівнювались би в загальну ціну продукції, яка реґулює ринок,
то звичайне підвищення продуктивної сили праці в наслідок використання водоспаду,
лише знизило б ціну товарів, продукованих з допомогою водоспаду, але
не підвищило б тієї частини зиску, що міститься в цих товарах; так само, як,
з другого боку, ця підвищена продуктивна сила праці взагалі не перетворювалась
би на додаткову вартість, коли б капітал продуктивну силу вживаної ним
праці, природну і суспільну, не привлащував би як свою власну.

\emph{Четверте}: Земельна власність на водоспад сама по собі не має ніякого
чинення до створення цієї частини додаткової вартости (зиску), а тому і взагалі
ціни товару, який продукується з допомогою водоспаду. Цей надзиск існував
би і тоді коли б не існувало земельної власности, коли б, наприклад,
земля, до якої належить водоспад, використовувалась фабрикантом, як безгосподарна
земля. Отже, земельна власність не створює тієї частини вартости,
яка перетворюється в надзиск, а лише дає земельному власникові, власникові
водоспаду, можливість перекласти цей надзиск з кишені фабриканта у свою
власну. Земельна власність є причина не створення цього надзиску, а його
перетворення у форму земельної ренти, отже, привласнення цієї частини зиску,
зглядно ціни товару, власником землі або водоспаду.

\emph{П’яте}: Ясно, що ціна водоспаду, отже, ціна, яку одержав би земельний
власник, коли б він продав його третій особі, або самому фабрикантові,
спочатку не входить у ціну продукції товарів, хоч входить в індивідуальні
витрати продукції даного фабриканта; бо рента виникає тут з ціни продукції
товарів того самого роду, продукованих з допомогою парових машин,
з ціни продукції, що реґулюється незалежно від водоспаду. Але, далі, ця ціна
водоспаду взагалі є іраціональний вираз, що за ним ховається реальне економічне
відношення. Водоспад, як земля взагалі, як усі сили природи, не має
жодної вартости, бо в ньому не зрічевлено жодної праці, а тому не має він
жодної ціни, яка нормально є не що інше, як виражена в грошах вартість.
Де немає вартости, там ео ipso\footnote*{
Тим самим. \emph{Пр.~Ред.}
} нічого виражати в грошах. Ця ціна є не
що інше, як капіталізована рента. Земельна власність дає власникові можливість
захоплювати ріжницю між індивідуальним зиском і пересічним зиском,
захоплюваний в такий спосіб зиск, що відновляється щорічно, може бути капіталізований
і тоді виступає як ціна самої сили природи. Коли надзиск, що
його дає фабрикантові використання водоспаду, становить 10\pound{ ф. ст.} на рік, а
пересічний процент 5\%, то ці 10\pound{ ф. ст.} на рік становлять проценти з капіталу в
200\pound{ ф. ст.} і ця капіталізація річних 10\pound{ ф. ст.}, що водоспад дає змогу власникові
його захоплювати їх у фабриканта, виступає тоді, яв капітальна вартість самого
водоспаду. Те, що водоспад не має вартости і що ціна його є звичайний
відбиток захоплюваного надзиску, капіталістично обчисленого, це одразу
\parbreak{}  %% абзац продовжується на наступній сторінці

\input{_0125.tex}
\input{_0126c.tex}
\parcont{}  %% абзац починається на попередній сторінці
\index{iii2}{0127}  %% посилання на сторінку оригінального видання
методів (кормові трави), почасти механічними засобами, які перетворюють
підґрунтя в верхній шар ґрунту, або змішують його з ним, або обробляють підґрунтя,
не переміщуючи його на поверхню.

Всі ці впливи на диференційну родючість різних земель сходять на те, що для
економічної родючости стан продуктивної сили праці, в даному разі здібність хліборобства
одразу використовувати природну родючість ґрунту, — здібність, яка різна
на різних ступенях розвитку, — становить так само момент так званої природної
родючости ґрунту, як і його хемічний склад і інші природні властивості.

Отже, ми припускаемо певний ступінь розвитку хліборобства. Ми припускаємо
далі, що ієрархія щодо родів ґрунту відповідає цьому ступеневі розвитку,
як це звичайно завжди буває щодо одночасних приміщень капіталу на різних
землях. В такому разі диференційна рента може бути представлена у висхід ій
або низхідній послідовності, бо, хоч певна послідовність дана для всієї сукупности
дійсно оброблюваних земель, проте завжди відбувається послідовний рух,
в якому складалась ця послідовність.

Припустімо землю чотирьох родів: $А$, $В$, $C$, $D$. Припустимо далі, що ціна
квартера пшениці \deq{} 3\pound{ ф. стерл.} або 60\shil{ шил.} А що рента є просто диференційна рента,
то ця ціна в 60\shil{ шил.} за квартер з найгіршої землі дорівнює ціні продукції,
тобто дорівнює капіталові плюс пересічний зиск.

Хай $А$ буде ця найгірша земля, що на 50\shil{ шил.} витрат дає 1 квартер \deq{} 60\shil{ шил.}; отже, 10\shil{ шил.} зиску, або 20\%.

Хай $В$ при цій самій витраті дає 2 кварт. \deq{} 120\shil{ шил.} Це дало б 70\shil{ шил.}
зиску, або 60\shil{ шил.} надзиску.

Хай $C$ при такій самій витраті дає 3 кварт — 180\shil{ шил.}; загальний
зиск \deq{} 130\shil{ шил.}; надзиск \deq{} 120\shil{ шил.}

Хай $D$ дає 4 кварт. \deq{} 240\shil{ шил.} \deq{} 180\shil{ шил.} надзиску.

Ми мали б тоді тоді таку послідовність:

Відповідні ренти були б для $D \deq{} 190 - 10$ шил. або ріжниця між $D$ та
$А$; для $C \deq{} 130 - 10$ шил. або ріжниця між $C$ та $А$; для $В \deq{} 70 - 10$ шил. або ріжниця
між $В$ та $А$; а загальна рента для $В$, $C$, $D$ \deq{} 6 кв. \deq{} 360\shil{ шил.}, дорівнювала б сумі ріжниць між
$D$ і $А$, $C$ і $А$, $В$ та $А$.

\begin{table}[H]
  \centering
  \caption*{Таблиця I}

  \begin{tabular}{cccccccc}
    \toprule
      \multirowcell{2}{\makecell{Рід \\землі}} &
      \multicolumn{2}{c}{Продукт} &
      \multirowcell{2}{\makecell{Авансова-\\ний капітал}} &
      \multicolumn{2}{c}{Зиск} &
      \multicolumn{2}{c}{Рента}
      \\
    \cmidrule(rl){2-3}
    \cmidrule(l){5-6}
    \cmidrule(l){7-8}
    &
    \makecell{Квар-\\тери} &
    \makecell{Ши-\\лінги} &
    &
    \makecell{Квар-\\тери} &
    \makecell{Ши-\\лінги} &
    \makecell{Квар-\\тери} &
    \makecell{Ши-\\лінги} &
    \\
    \midrule
     A  &  1  &  \phantom{0}60 & 50 & \phantom{0}\sfrac{1}{6}   &  \phantom{0}10  &   \textemdash & \textemdash \\
     B  &  2  &  120           & 50 & 1\sfrac{1}{6}  &  \phantom{0}70  &   1           & \phantom{0}60 \\
     C  &  3  &  180           & 50 & 2\sfrac{1}{6}  &  130 &   2           & 120 \\
     D  &  4  &  240           & 50 & 3\sfrac{1}{6}  &  190 &   3           & 180 \\
     \cmidrule(rl){2-3}
     \cmidrule(l){7-8}
     Разом & 10 квар. & 600 ш. &    &       &      &   6 квар. &     360 ш. \\
  \end{tabular}
\end{table}

Ця послідовність,
що становить за даних умов даний продукт, коли справу розглядати
абстрактно (а ми вже показали ті причини, що в наслідок їх така послідовність
може бути і в дійсності), може бути і в низхідному порядку (низхідному
від $D$ до $А$, від родючої землі до менш і менш родючої (так само як і в висхідному
порядку (висхідному від $А$ до $D$, від відносно неродючої до чимраз родючішої землі)
і, нарешті, перемінно, то в низхідному, то в висхідному порядку, наприклад,
від $D$ до $C$, від $C$ до $А$, від $А$ до $В$.

Процес, що відбувався при низхідній послідовності, був такий: ціна квартера
поступово підвищується, скажемо, з 15\shil{ шил.} до 60. Скоро виявилося, що
випродукованих на $D$ 4 кв. (під ними можпа розуміти мільйони) уже не
\parbreak{}  %% абзац продовжується на наступній сторінці

\parcont{}  %% абзац починається на попередній сторінці
\index{iii2}{0128}  %% посилання на сторінку оригінального видання
досить, так ціна пшениці почала підноситись доти, поки $C$ не набуло змоги
покрити недостачу подання. Тобто ціна мусила піднестись до 20\shil{ шил.} за
квартер. Скоро тільки ціна пшениці піднеслась до 30\shil{ шил.} за квартер, ак
в число оброблюваних земель могла б увійти земля $В$. а якби вона піднеслась
до 60\shil{ шилінґів}, до числа оброблюваних земель могла б увійти й земля $А$, це
не призвело б до того, що на застосований тут капітал довелось би задовольнятися
нормою зиску, нижчою за 20\%. Таким чином, для $D$ створилась би
рента спочатку в 5\shil{ шил.} з квартера \deq{} 20\shil{ шил.} з 4 кв., що тут продукується,
а потім в 45\shil{ шил.} з квартера \deq{} 180\shil{ шил.} з 4 квартерів.

Коли норма зиску з $D$ спочатку також була \deq{} 20\%, то і загальний зиск
з 4 кв. був також лише 10\shil{ шил.}, що проте, при ціні збіжжя в 15\shil{ шил.}, становило
більшу кількість збіжжя, ніж при ціні в 60\shil{ шил.} А що збіжжя входить
у репродукцію робочої сили і частина кожного квартера мусить покривати заробітну
плату, а друга — сталий капітал, то за такого припущення додаткова
вартість була вища, а тому, за інших незміних умов, вища була і норма зиску.
(Справу про норму зиску треба ще дослідити осібно і детальніше).

Коли, навпаки, послідовність була зворотна, коли процес починався з $А$,
то, — якщо довелося б ввести в обробіток нові лани, — ціна квартера спочатку
піднеслась би вище за 60\shil{ шил.}; але тому, що потрібне подання в 2 кварт. постачало
б $В$, то ціна знову понизилась би до 60\shil{ шил.}; хоч $В$ і продукує квартер
за 30\shil{ шилінґів}, проте, продається він за 60, бо його подання вистачало б
якраз тільки для того, щоб покрити попит. Так створилася б рента спочатку в
60\shil{ шил.} для $В$, і таким самим способом для $C$ і $D$, припускаючи завжди, що
ринкова ціна залишається 60\shil{ шил.}, хоч дійсна вартість, по якій $C$ і $D$ дають
квартер пшениці, дорівнює 20 і 15\shil{ шил.}; бо як і давніш потрібно подання одного
квартера, що його постачає $А$, для задоволення загальної потреби. В цьому випадку
підвищення попиту понад ту потребу, яку спочатку задовольняло $А$, потім
$А$ і $В$, могло б привести не до послідовного обробітку $В$, $C$ і $D$, а до поширення
площі обробітку взагалі і можливо, що родючіші землі входили б в обробіток
лише пізніше.

В першому ряді із збільшенням ціни рента стала б підвищуватись, а норма
зиску зменшуватись. Це зменшення могло б цілком або почасти паралізуватися
протидіющими обставинами; на цьому пункті згодом спинимося докладніше.
Не слід забувати, що загальна норма зиску визначається не додатковою вартістю
в усіх сферах продукції рівномірно. Не хліборобський зиск визначає промисловий,
а навпаки. Але про це далі.

У другому ряді норма зиску на витрачений капітал лишилась би та
сама; маса зиску визначилась би в меншій кількості збіжжя; але відносна ціна
його проти інших товарів підвищилась би. Але збільшення зиску, там де воно
відбувається, відокремлюється в формі ренти від зиску, замість того, щоб потрапити
до кишені промислових орендарів і визначитися як зиск, що зростає.
А ціна хліба за такого припущення лишилась би незмінною.

Розвиток і зріст диференційної ренти залишаються однакові так за незмінних
цін, як і за таких, що підвищуються, і так само за безперервного поступу
від гірших земель до кращих, як і за безперервного реґресу від крайніх
до гірших земель.

До цього часу ми вважали: 1)~що ціна при одній послідовності підвищується,
при другій — лишається незмінна, і 2)~що постійно відбувається перехід
від кращих земель до гірших або навпаки — від гірших до кращих.

Але припустімо, що потреба в хлібі піднялась з первісних 10 до 17 кв.;
далі, що найгірша земля $А$ витиснута другою землею $А$, яка при ціні продукції
в 60\shil{ шил.} (50\shil{ шил.} витрат, плюс 10\shil{ шил.}, що становлять 20\% зиску) дає
1\sfrac{1}{3} кварт., так що ціна продукції одного квартера \deq{} 45\shil{ шил.}; абож припустімо,
\parbreak{}  %% абзац продовжується на наступній сторінці

\parcont{}  %% абзац починається на попередній сторінці
\index{iii2}{0129}  %% посилання на сторінку оригінального видання
що перша земля $А$ поліпшилась в наслідок постійного раціонального обробітку,
або що вона при незмінності витрат стала продуктивніше оброблятися,
наприклад, в наслідок заведення конюшини тощо, так що її продукт, за незмінної
величини авансованого капіталу, збільшився до 1\sfrac{1}{3} кварт. Припустімо
далі, що землі $В$, $C$, $D$, як і давніш, дають ту саму кількість продукту, але що
почали оброблятися нові землі $А'$ пересічної між $А$ і $В$ родючости, далі $В'$, $В''$, що
містяться своєю родючістю між $В$ і $C$; в цьому випадку постали б такі явища:

\emph{Перше}: ціна продукції квартера пшениці, або її реґуляційна ринкова
ціна, зменшилась би з 60 до 45\shil{ шил.}, або па 25\%.

\emph{Друге}: відбувся б одночасний перехід від родючішої землі до менш
родючої, і від менш родючої землі до родючішої. Земля $А'$ родючіша, ніж $А$, але
менш родюча, ніж $В$, $C$, $D$, що оброблялись до цього часу; а $В'$, $В''$ родючіші, ніж
$А$, $А'$ і $В$, але менш родючі, ніж $C$ і $D$. Отже, перехід від однієї землі до другої
відбувався б у всіх напрямках; відбувся б перехід не до абсолютно
менш родючої землі проти $А$ тощо, а до відносно менш родючої, порівняно
з землями $C$ і $D$, які до цього часу були найродючіші; з другого боку, перехід
відбувався б не до абсолютно родючішої землі, а до відносно родючішої проти
земель $А$, — або $А$ і $В$, — які до цього часу були найменш родючі.

\emph{Третє}: Рента з $В$ знизилася б; а також рента з $C$ і $D$; але загальна
сума ренти, визначена в збіжжі, піднеслась би з 6 до 7\sfrac{2}{3} кв.; маса землі, що
обробляється і дає ренту, збільшилася б, а також збільшилася б і маса продукту
з 10 до 17 квар. Зиск, хоч він і лишився без перемін для $А$, визначений у
збіжжі, підвищився б; але можливо, що навіть норма зиску підвищилася б, бо
підвищилася б відносна додаткова вартість. В цьому випадку в наслідок здешевлення
засобів існування зменшилася б заробітна плата, отже, витрата на змінний капітал,
отже, і загальні витрати. Вся сума ренти, визначена в грошах, знизилась би з 360 до 345\shil{ шил.}

Подаємо нову послідовність переходу
(див. табл. II).

\begin{table}[h]
  \begin{center}
  \footnotesize
    \emph{Таблиця II}

  \begin{tabular}{c c c c c c c с c}
    \toprule
      \multirowcell{2}{\makecell{Рід \\землі}} &
      \multicolumn{2}{c}{Продукт} &
      \multirowcell{2}{\makecell{Витрата \\капіталу}} &
      \multicolumn{2}{c}{Зиск} &
      \multicolumn{2}{c}{Рента} &
      \multirowcell{2}{\makecell{Ціна про-\\дукції \\квартера}}
      \\
    \cmidrule(rl){2-3}
    \cmidrule(l){5-6}
    \cmidrule(l){7-8}
    &
    \makecell{Квар-\\тери} &
    \makecell{Ши-\\лінґи} &
    &
    \makecell{Квар-\\тери} &
    \makecell{Ши-\\лінґи} &
    \makecell{Квар-\\тери} &
    \makecell{Ши-\\лінґи} &
    \\
    \midrule
     А\phantom{''}   &  1\sfrac{1}{3}            & \phantom{0}60 & 50  &  \phantom{0}\sfrac{2}{9} & \phantom{0}10  &  \textemdash             & \textemdash    & 45\phantom{\sfrac{1}{1}} шил. \\
     А'\phantom{'}   &  1\sfrac{2}{3}            & \phantom{0}75 & 50  &  \phantom{0}\sfrac{5}{9} & \phantom{0}25  &  \phantom{0}\sfrac{1}{3} & \phantom{0}15  & 36\phantom{\sfrac{1}{1}} \ditto{шил.} \\
     B\phantom{''}   &  2\phantom{\sfrac{1}{1}}  & \phantom{0}90 & 50  &  \phantom{0}\sfrac{8}{9} & \phantom{0}40  &  \phantom{0}\sfrac{2}{3} & \phantom{0}30  & 30\phantom{\sfrac{1}{1}} \ditto{шил.} \\
     В'\phantom{'}   &   2\sfrac{1}{2}           & 105           & 50  &  1\sfrac{2}{9}           & \phantom{0}55  &  1\phantom{\sfrac{1}{1}}                       & \phantom{0}45  & 25\sfrac{2}{7} \ditto{шил.} \\
     В''             &   2\sfrac{2}{3}           & 120           & 50  &  1\sfrac{5}{9}           & \phantom{0}70  &  1\sfrac{1}{3}           & \phantom{0}60  & 22\sfrac{1}{2} \ditto{шил.} \\
     C\phantom{''}   &  3\phantom{\sfrac{1}{1}}  & 135           & 50  &  1\sfrac{8}{9}           & \phantom{0}85  &  1\sfrac{2}{3}           & \phantom{0}75  & 20\phantom{\sfrac{1}{1}} \ditto{шил.} \\
     D\phantom{''}   &  4\phantom{\sfrac{1}{1}}  & 180           & 50  &  2\sfrac{8}{9}           & 130            &  2\sfrac{2}{3}           & 120            & 15\phantom{\sfrac{1}{1}} \ditto{шил.} \\
     \cmidrule(rl){2-2}
     \cmidrule(l){7-8}
     Разом & 17 & &    &       &      &   7\sfrac{2}{3} &     345 \\
  \end{tabular}
  \end{center}
\end{table}

Нарешті, коли б, як і давніш, оброблялись тільки землі $А$, $В$, $C$, $D$, але продуктивність їхня зросла б
остільки, що земля $А$ замість 1 квартера давала б 2, $В$ замість 2 квартерів — 4,
$C$ замість 3 квартерів — 7 і $D$ замість 4 квартерів — 10, отже, коли б ті самі
причини по-різному вплинули б на різні землі, то вся продукція підвищилася
б з 10 до 23 квартерів. Припустімо, що попит в наслідок приросту
людности і пониження ціни поглинув би ці 23 квартери, в такому разі ми
мали б такий результат (див. табл. III).

Числові відношення тут, як і в попередніх таблицях, довільні, але припущення
цілком раціональні.


\index{iii2}{0130}  %% посилання на сторінку оригінального видання
Перше і головне припущення є, що поліпшення в хліборобстві нерівномірно
впливає на землі різних родів, і тут воно більше впливає на кращі землі
$C$ і $D$, ніж на $А$ і $В$. Досвід довів, що, звичайно, справа так і стоїть, хоч може
статись і зворотне. Коли б поліпшення більше впливало на гірші землі, ніж на
кращі, то рента з останніх понизилася б замість підвищитись. — Але з абсолютним
зростом родючости всіх родів землі у таблиці одночасно припускається зріст
вищої відносної родючості кращих родів землі $C$ і $D$, а тому зріст ріжниці в продукті за однакової
величини застосованого капіталу, а тому і зріст диференційної ренти.
Друге припущення є в тому, що з зростанням всього продукту відповідно зростає і загальна потреба в
ньому. \emph{Поперше}, не слід уявляти собі це зростання раптовим; воно відбувається поступово, доти, доки
не встановиться ряд III. \emph{Подруге}, невірно, нібито споживання потрібних засобів існування не зростає разом з їхнім
здешевленням. Скасування хлібних законів в Англії (дивись Newman) довело зворотне, і протилежне
уявлення постало лише тому, що великі і раптові ріжниці в урожаях, які пояснюються тільки погодою,
спричиняють то неспіврозмірне пониження, то неспіврозмірне підвищення цін збіжжя.
Коли в цьому разі раптове і скороминуще здешевлення не встигає справити повного впливу на поширення
споживання, то зворотне явище спостерігається в тому випадку, коли здешевлення випливає із зменшення
самої регуляційної ціни продукції, отже, коли воно має тривалий характер. \emph{Потрете}, частина збіжжя
може бути спожита у вигляді горілки або пива. А зростаюуче споживання обох цих продуктів ніяк не
обмежено вузькими межами. \emph{Почетверте}, тут справа залежить почасти від приросту людности, почасти
від експорту збіжжя в тих країнах, що вивозять збіжжя — як от Англія, до і
пізніше половини XYIII століття, і де тому потребу реґулюється межами не самого
тільки національного споживання. \emph{Нарешті}, збільшення і здешевлення
продукції пшениці може мати своїм наслідком, що замість жита або вівса за
головний засіб харчування маси народу стане пшениця, так що вже в наслідок
самого цього ринок для неї зросте подібно до того, як при зменшенні кількості
продукту і збільшенні його ціни може постати зворотне явище. — При цих припущеннях, отже, і при
взятих числових відношеннях, ряд III дає той наслідок,
що ціна знижується з 60 до ЗО шил. за квартер, отже на 50\%; що продукція проти ряду І зростає з 10
до 23 квартерів, отже, на 130\%; що рента з землі $В$ лишається незмінною, рента з землі $C$
подвоюється, а з $D$ більше, ніж подвоюється, і що загальна сума ренти підвищується з 18 до 22\pound{ ф.
стерл.}, отже, на 22\sfrac{1}{9}\%.

З порівняння цих трьох таблиць (при чому ряд I треба брати подвійно:
у висхідному напрямку від $А$ до $D$ і в низхідному від $D$ до $А$), що їх можна
розглядати або як дані ступені хліборобства, за даного стану суспільства, наприклад,

\begin{table}[h]
  \begin{center}
  \footnotesize
    \emph{Таблиця III}

  \begin{tabular}{c c c c c c c c c}
    \toprule
      \multirowcell{2}{\makecell{Рід \\землі}} &
      \multicolumn{2}{c}{Продукт} &
      \multirowcell{2}{\makecell{Витрата \\капіталу}} &
      \multicolumn{2}{c}{Зиск} &
      \multicolumn{2}{c}{Рента} &
      \multirowcell{2}{\makecell{Ціна про-\\дукції \\квартера}}
      \\
    \cmidrule(rl){2-3}
    \cmidrule(l){5-6}
    \cmidrule(l){7-8}
    &
    \makecell{Квар-\\тери} &
    \makecell{Ши-\\лінґи} &
    &
    \makecell{Квар-\\тери} &
    \makecell{Ши-\\лінґи} &
    \makecell{Квар-\\тери} &
    \makecell{Ши-\\лінґи} &
    \\
    \midrule
      А  &  \phantom{0}2  &  \phantom{0}60  & 50 & \phantom{0}\sfrac{1}{3}  & \phantom{0}10  & \phantom{0}0 & \phantom{00}0  &  30\\
      B  &  \phantom{0}4  &  120            & 50 & 2\sfrac{1}{3}            & \phantom{0}70  & \phantom{0}2 & \phantom{0}60  &  15\\
      C  &  \phantom{0}7  &  210            & 50 & 5\sfrac{1}{3}            & 160            & \phantom{0}5 & 150            &  8\sfrac{4}{7} \\
      D  &  10            &  300            & 50 & 8\sfrac{1}{3}            & 250            & \phantom{0}8 & 240            &  6\phantom{0} \\
      \cmidrule(rl){2-2}
      \cmidrule(l){7-8}
      Разом & 23          &                 &    &                          &                & 15           & 450           & \\
  \end{tabular}
  \end{center}
\end{table}


1)~Що ряд у своєму закінченному вигляді, — хоч би який завжди був перебіг процесу його складання, —
завжди виступає як низхідний; бо при розгляді ренти завжди виходять спочатку від землі, яка дає
максимум ренти, і лише нарешті переходять до тієї, що не дає жодної ренти.

2)~Ціна продукції на найгіршій землі, що не дає ренти, завжди становить реґуляційну ринкову ціну,
хоч ціна ця в таблиці I, коли вона склалася у висхідному порядку, тільки через те лишається
незмінна, що всю кращу землю оброблено. В цьому випадку ціна збіжжя, випродукованого на кращій
землі, є реґуляційною остільки, оскільки від кількости продукту, випродукованого на ній, залежить, в
якій мірі земля $А$ лишається реґуляційною. Коли б на землях $В$, $C$, $D$ продукувалось понад потребу, то
земля $А$ перестала б бути реґуляційною. Це й штовхнуло Шторха на те, що він за реґуляційпі визнав
найкращі землі. В цьому розумінні англійські ціни хліба реґулюються американськими.

3)~Диференційна рента постає з ріжниці в природній родючості різного ґрунту (тут положення землі ще
не береться на увагу), — ріжниці даної для кожного даного ступеня розвитку культури, отже, з
обмежености кількости кращих земель, і з тієї обставини, що однакові капітали доводиться вкладати в
неоднакові землі, які, отже, при витраті однакового капіталу дають неоднакову кількість продукту.

4)~Диференційна рента і ґрадація диференційної ренти можуть однаково виникнути так у низхідному
порядку в наслідок переходу від кращої землі до гіршої, як і навпаки, від гіршої до кращої, або в
наслідок переходу впереміжку в усіх напрямках. (Ряд І може скластися в наслідок переходу так від І)
до $А$, як і від $А$ до $D$. Ряд II охоплює обидва види руху).

5)~Залежно від способу виникнення, диференційна рента може постати за сталої, висхідної і низхідної
ціни хліборобського продукту. За низхідної ціни загальна продукція і загальна сума ренти може
підвищитись, і земельні дільниці, що не давали до цього часу ренти, можуть почати давати ренту, не
зважаючи на те, що гірша земля $А$ витиснена кращою або сама поліпшилась, і що рента з інших кращих і
навіть найкращих земель понижується (таблиця II);
цей процес може бути також зв’язаний з пониженням загальної суми ренти (в грошах). Нарешті, за
низхідних цін, зумовлених загальним поліпшенням обробітку, в зв’язку з чим кількість і ціна продукту
з найгіршої землі понижується, — рента з частини кращих земель може лишитись незмінною або
понизитися, але рента з найліпших земель може зрости. Коли ріжниця мас продукту дана, то
диференційна рента з усякої землі проти найгіршої землі без сумніву залежить від ціни, наприклад,
квартера пшениці. Але коли ціна є дана, то диференційна рента залежить від розміру ріжниці між
масами продукту, і коли з підвищенням абсолютної родючости всіх земель родючість земель відносно
більше підвищується, ніж родючість гірших, то разом з цим зростає і величина цієї ріжниці. Так
(таблиця І) при ціні в 60\shil{ шил.} рента з $B$ визначається ріжницею в продукті проти $А$, тобто, надміром в
3 квартери; тому рента $= 3 × 60 \deq{} 180$ шил. Але в таблиці III, де ціна \deq{} 30\shil{ шил.}, вона визначається
масою надмірного продукту з землі $D$ порівняно з $А \deq{} 8$ кварт.; але $8 × 30 \deq{} 240$ шил.

Разом з тим відпадає та перша фалшива передумова диференційної ренти, — яка ще панує в Веста,
Мальтуса, Рікардо, нібито диференційна рента неодмінно має за передумову перехід до земель дедалі
гіршої якости, або постійно зменшувану родючість ґрунту. Як ми вже бачили, диференційна рента може
постати при переході до земель дедалі ліпшої якости; вона може постати, коли нижчий ступінь займе
краща земля, замість колишньої гіршої; вона може постати в зв’язку з ростучим поступом хліборобства.
Умовою її виникнення є лише неоднаковість родів землі. Оскільки береться на увагу розвиток
продуктивності,
\parbreak{}  %% абзац продовжується на наступній сторінці

\parcont{}  %% абзац починається на попередній сторінці
\index{iii2}{0132}  %% посилання на сторінку оригінального видання
умова її виникнення є в тому, що підвищення абсолютної родючості всієї земельної площі не знищує
тієї неоднаковості, але збільшує її, або залишає без зміни, абож лише зменшує.

Від початку і до половини XVIII століття панувало в Англії, не зважаючи на пониження ціни золота і
срібла, безперервне падіння цін збіжжя одночасно (коли розглядати весь період) з ростом ренти,
загальної суми ренти, розміру оброблюваної земельної площі, хліборобської продукції і людности. Це
відповідає таблиці І, комбінованій з таблицею II, у висхідному напрямку, але так, що гірша земля $А$
або поліпшується, або виключається з числа земель, оброблюваних під збіжжя; це, звичайно, не значить,
що вона не буде використана для інших сільськогосподарських або промислових цілей.

Від початку XIX століття (треба точніше подати час) до 1815 року безперервне підвищення цін збіжжя
одночасно з постійним зростом ренти, загальної суми ренти, розміру оброблюваної земельної площі,
хліборобської продукції і людності. Це відповідає таблиці І у низхідному напрямку. (Тут слід навести
цитату щодо обробітку гірших земель за того часу).

За доби Петті і Давенанта скарги сільської людности і земельних власників на поліпшення і поширення
обробітку; пониження ренти на кращих землях, підвищення загальної суми ренти в наслідок поширення
площі землі, що дає ренту.

(До цих трьох пунктів навести потім дальші цитати; також щодо ріжниці у родючості різних частин
обробленої землі в країні).

Щодо диференційної ренти слід взагалі зауважити, що ринкова вартість завжди перевищує загальну ціну
продукції даної маси продуктів. Для прикладу візьмімо таблицю І. 10 кватерів всього продукту
продаються за 600\shil{ шил.}, бо ринкова ціна визначається ціною продукції на $А$, яка становить 60\shil{ шил.} за
квартер. Але дійсна ціна продукції є:

\begin{table}[H]
  \centering
  \small
  \begin{tabular}{l r@{~}r l l}
    А & 1 кварт. \deq{} & 60\shil{шил.} & & 1 кварт. \deq{} 60\shil{шил.} \\
    В & 2 кварт. \deq{} & 60\shil{шил.} & & 1 кварт. \deq{} 30\shil{шил.} \\
    C & 3 кварт. \deq{} & 60\shil{шил.} & & 1 кварт. \deq{} 20\shil{шил.} \\
    D & 4 кварт. \deq{} & 60\shil{шил.} & & 1 кварт. \deq{} 15\shil{шил.} \\
    \midrule
      &10 кварт. \deq{} &240\shil{шил.} & ~пересічно & 1 кварт. \deq{} 24\shil{шил.} \\
  \end{tabular}
\end{table}

\noindent{}Дійсна ціна продукції 10 квартерів є 240\shil{ шил.}; вони продаються за 600, тобто на 250\% дорожче.
Дійсна пересічна ціна 1 квартера є 24\shil{ шил.}: ринкова ціна — 60\shil{ шил.}, тобто теж на 250\% дорожча.

Тут маємо визначення за посередництвом ринкової вартости в тому її вигляді, як вона на базі
капіталістичного способу продукції пробивається за посередництвом конкуренції; ця остання породжує
фалшиву соціяльну вартість. Це постає з закону ринкової вартости, якому підпорядковані продукти
хліборобства.
Визначення ринкової вартости продуктів, отже, і хліборобських продуктів, є суспільний акт, хоч і акт
суспільно несвідомий і ненавмисний, акт, що неминуче ґрунтується на міновій вартості продукту, а не
на землі і ріжницях її родючості. Коли уявити собі, що капіталістична форма суспільства знищена і
суспільство організоване як свідома і плянова асоціація, то ці 10 квартерів являтимуть собою
кількість самостійного робочого часу, рівну тому, що міститься в 240\shil{ шил.} Отже, суспільство не
купувало б цього хліборобського продукту за таку кількість робочого часу, яка в 2\sfrac{1}{2}, раза більша
за робочий час, який дійсно міститься в цьому продукті. Тим самим відпала б база класу власників
землі. Це впливало б цілком так само, як здешевлення продукту на таку суму в наслідок чужоземного
довозу. Тому, оскільки справедливо було б сказати, що — в умовах збереження сучасного способу
продукції, але припускаючи, що диференційна рента діставатиметься державі — ціни земельних
продуктів, за інших незміних умов, залишились би тими самими, так само помилково було б
\parbreak{}  %% абзац продовжується на наступній сторінці

\parcont{}  %% абзац починається на попередній сторінці
\index{iii2}{0133}  %% посилання на сторінку оригінального видання
казати, що вартість продуктів залишилась би та сама в умовах заміни капіталістичної
продукції асоціацією. Тотожність ринкової ціни однорідних товарів є
спосіб, у який, на базі капіталістичного способу продукції і взагалі продукції,
що ґрунтується на обміні товарів між поодинокими продуцентами, пробивається
суспільний характер вартости. Те, що суспільство, розглядуване як споживач,
переплачує за хліборобські продукти, і що становить мінус у реалізації його
робочого часу в хліборобській продукції, це становить тепер плюс для однієї
частини суспільства, для земельних власників.

Друга обставина, важлива для того, про що говориться в дальшім розділі
під рубрикою II, така:

Справа не тільки в ренті з акра або з гектара, взагалі не тільки в ріжниці
між ціною продукції і ринковою ціною, або між індивідуальною й загальною
ціною продукції з акра, але також і в тому, скільки акрів кожного
роду землі обробляється. Тут важлива безпосередньо лише величина загальної
суми ренти, тобто сукупної ренти з усієї оброблюваної площі; але це дає
нам одночасно можливість перейти до з’ясовування того, як підвищується норма
ренти, хоч ціни не збільшуються, і хоч за низхідних цін не збільшуються
ріжниці у відносній родючості різних родів землі. Вище ми мали:
% (див. табл. I).

\begin{table}[H]
  \small
  \centering
  \caption*{Таблиця I}

  \begin{tabular}{l c c c c c}
    \toprule
      \makecell[l]{Рід\\землі} &
      \makecell{Акри} &
      \makecell{Ціна\\продукції,\pound{}} &
      \makecell{Продукт,\\кварт.} &
      \makecell{Рента\\в збіжжі, кварт.} &
      \makecell{Грошова\\рента,\pound{}}
      \\
     \midrule
     A & 1 & 3 & 1 & 0 & 0 \\
     B & 1 & 3 & 2 & 1 & 3 \\
     C & 1 & 3 & 3 & 2 & 6 \\
     D & 1 & 3 & 4 & 3 & 9 \\
     \midrule
     Сума & 4 & \textendash{} & \hang{r}{1}0 & 6 & \hang{r}{1}8 \\
  \end{tabular}
\end{table}

\noindent{}Припустімо тепер, що число оброблюваних акрів кожного розряду подвоїлося.
В такому разі ми матимемо:
% (див. табл. Iа).

\begin{table}[H]
  \small
  \centering
  \caption*{Таблиця Iа}

  \begin{tabular}{l c c c c c}
    \toprule
      \makecell[l]{Рід\\землі} &
      \makecell{Акри} &
      \makecell{Ціна\\продукції,\pound{}} &
      \makecell{Продукт,\\кварт.} &
      \makecell{Рента\\в збіжжі, кварт.} &
      \makecell{Грошова\\рента,\pound{}}
      \\
     \midrule
     A & 2 & 6 & 2 & 0 & \phantom{0}0 \\
     B & 2 & 6 & 4 & 2 & \phantom{0}6 \\
     C & 2 & 6 & 6 & 4 & 12 \\
     D & 2 & 6 & 8 & 6 & 18 \\
     \midrule
     Сума 
       & 8 & \textendash{} & \hang{r}{2}0 & \hang{r}{1}2 & 36 \\
  \end{tabular}
\end{table}

\noindent{}Ми припустимо ще 2 випадки; перший, коли продукція розширюється
на обох гірших родах землі. Отже, тоді матимемо:
%(див. табл. Іb).

\begin{table}[H]
  \small
  \centering
  \caption*{Таблиця Ib}
  \begin{tabular}{l c c c c c c}
    \toprule
      \multirowcell{2}[0ex][l]{Рід\\землі} &
      \multirowcell{2}{Акри} &
      \multicolumn{2}{c}{Ціна продукції,\pound{ ф. ст.}} &
      \multirowcell{2}{Продукт} &
      \multirowcell{2}{Рента\\в збіжжі, кварт.} &
      \multirowcell{2}{Грошова\\рента,\pound{}} \\
      \cmidrule(rl){3-4}

      &  &  на акр. & в сумі & &                    &  \\
      \midrule
      A & 4 & 3 & 12 & 4 & 0 &  \pZ{}0 \\
      B & 4 & 3 & 12 & 8 & 4 & 12 \\
      C & 2 & 3 & \pZ{}6 & 6 & 4 & 12 \\
      D & 2 & 3 & \pZ{}6  & 8 & 6 & 18 \\
     \midrule
     Сума & \hang{r}{1}2 & \textendash{} & 36 & \hang{r}{2}6 & \hang{r}{1}4 & 42 \\
  \end{tabular}
\end{table}

\noindent{}І, нарешті, коли маємо неоднакове поширення продукції і оброблюваної площі в чотирьох розрядах:
% (див. табл. Iс).

\begin{table}[H]
  \small
  \centering
  \caption*{Таблиця Iс}
  \begin{tabular}{l c c c c c c}
    \toprule
      \multirowcell{2}[0ex][l]{Рід\\землі} &
      \multirowcell{2}{Акри} &
      \multicolumn{2}{c}{Ціна продукції,\pound{ ф. ст.}} &
      \multirowcell{2}{Продукт} &
      \multirowcell{2}{Рента\\в збіжжі, кварт.} &
      \multirowcell{2}{Грошова\\рента,\pound{}} \\
      \cmidrule(rl){3-4}

      &  &  на акр. & в сумі & &                    &  \\
      \midrule
      A & 1 & 3 & \pZ{}3 & \pZ{}1 & \pZ{}0 &  \pZ{}0 \\
      B & 2 & 3 & \pZ{}6 & \pZ{}4 & \pZ{}2 & \pZ{}6 \\
      C & 5 & 3 & 15      & 15     & 10 & 30 \\
      D & 4 & 3 & 12      & 16     & 12 & 36 \\
     \midrule
     Сума & \hang{r}{1}2 & \textendash{} & 36 & 36 & 24 & 72 \\
  \end{tabular}
\end{table}

\noindent{}Насамперед, в усіх цих випадках І, І$а$, І$b$, І$с$ рента з одного акра лишається та сама; бо в
дійсності продукт однакової маси капіталу на кожному
\parbreak{}  %% абзац продовжується на наступній сторінці

\parcont{}  %% абзац починається на попередній сторінці
\index{iii2}{0134}  %% посилання на сторінку оригінального видання
акрі землі того самого роду лишився незмінний; припущено тільки, — і це кожного
даного моменту відбувається в усякій країні, — що землі різних родів перебувають
у певному відношенні до всієї оброблюваної землі; і що відношення, — а це постійно
відбувається в двох країнах при порівнянні їх одна з однією, або в тій самій
країні за різних часів, — в якому вся оброблювана земельна площа розподіляється
між різними родами землі, змінюється.

Порівнюючи Іа з І, ми бачимо, що коли площа оброблюваних земель усіх
чотирьох розрядів зростає в однаковій пропорції, то з подвоєнням кількосте
оброблюваних акрів подвоюється вся продукція, а також рента в збіжжі і грошах.

\begin{table}[h]
  \begin{center}
    \emph{Таблиця Iс.}
    \footnotesize

  \begin{tabular}{c c c c c c c}
    \toprule
      \multirowcell{2}{\makecell{Рід \\землі}} &
      \multirowcell{2}{\makecell{Акри}} &
      \multicolumn{2}{c}{Ціна продукції} &
      \multirowcell{2}{\makecell{Продукт}} &
      \multirowcell{2}{\makecell{Рента \\ в збіжжі}} &
      \multirowcell{2}{\makecell{Грошова \\рента}} \\
      \cmidrule(rl){3-4}

      &  &  На акр. & В сумі & &                    &  \\
      \midrule

      A & 1\phantom{акр.} &  3  ф. ст.                 & 3  ф. ст.         & 1  кварт.         & 0  кварт.         & 0  ф. ст.\\
      B & 2\phantom{акр.} &  3  \ditto{ф.} \ditto{ст.} & 6  \ditto{ф.} \ditto{ст.} & 4  \ditto{кварт.} & 2  \ditto{кварт.} & 6  \ditto{ф.} \ditto{ст.}\\
      C & 5\phantom{акр.} &  3  \ditto{ф.} \ditto{ст.} & 15  \ditto{ф.} \ditto{ст.}  & 15  \ditto{кварт.} & 10  \ditto{кварт.} & 30  \ditto{ф.} \ditto{ст.}\\
      D & 4\phantom{акр.} &  3  \ditto{ф.} \ditto{ст.} & 12  \ditto{ф.} \ditto{ст.}  & 16  \ditto{кварт.} & 12  \ditto{кварт.} & 36  \ditto{ф.} \ditto{ст.}\\
     \cmidrule(rl){1-1}
     \cmidrule(rl){2-2}
     \cmidrule(rl){4-4}
     \cmidrule(rl){5-5}
     \cmidrule(rl){6-6}
     \cmidrule(rl){7-7}
     Сума & 12 акр. &                 & 36 ф. ст.  & 36 кварт.        & 24  кварт.         & 72  ф. ст. \\
  \end{tabular}
  \end{center}
\end{table}

Але порівнюючи послідовно І$b$ з І$с$, ми знайдемо, що в обох випадках площа
оброблюваної землі збільшується утроє. В обох випадках вона збільшується з
4 до 12 акрів, але найбільше збільшення в I$b$ відбувається в розрядах
$а$ і $b$, в яких $а$ не дає жодної ренти, а $b$ дає найменшу
диференційну ренту, а саме з 8 новооброблюваннх акрів на землю $а$ і $b$
припадає по 3, разом 6, тимчасом,
як на $c$ і $d$ припадає лише по 1 акрові, разом 2. Іншими словами: \sfrac{3}{4} приросту
припадають на $а$ і $b$ і лише \sfrac{1}{4} на $c$ і $d$. Коли таке припустити, то в
I$b$ порівняно з І, потроєному збільшенню площі обробленої землі не відповідає
таке саме потроєне збільшення продукту, бо кількість його збільшилась з 10
не до 30, а лише до 26. З другого боку, тому, що значна частина усього приросту
постала на землі $А$, що не дає ренти, а більша частина приросту на
кращих землях постала на розряді $В$, то рента в збіжжі збільшується лише
з 6 до 14 кварт., а грошова рента — з 18 до 42 ф. стерл.

Коли ми, навпаки, порівняємо І$с$ з І, де земля, яка не дає ренти, зовсім
не збільшується в розмірі, а земля, що дає мінімальну ренту, збільшується
лише незначно, тоді як найбільший приріст припадає на $C$ і $D$, то ми побачимо,
що при потроєному збільшенні обробленої землі продукція зросла з 10 до
36 квартерів, тобто більш, ніж у три рази; рента в збіжжі збільшилася з 6 до
24 квартерів, або в чотири рази; і так само збільшилась грошова рента
з 18 до 72 ф. стерл.

В усіх цих випадках по самій суті справи ціна хліборобського продукту
лишається незмінною; в усіх випадках загальна сума ренти зростає з поширенням
культури, оскільки воно відбувається не виключно на гіршій землі, що
не дає жодної ренти. Але зріст цей різний. В тій самій мірі, в якій поширення
відбувається на кращих землях і в якій, отже, маса продукту зростає не тільки
пропорційно поширенню земельної площи, але швидше, — зростає і рента в
збіжжі і в грошах. В тій самій мірі, в якій поширення відбувається переважно
на найгіршій землі і на близьких до неї родах землі (при чому припускається,
що розряд найгіршої землі лишається той самий), в цій самій мірі загальна
сума ренти збільшується непропорційно поширенню культивованої площі. Отже,
в двох країнах, де земля $А$, що не дає ренти, однакова якістю, сума ренти
буде перебувати у зворотному відношенні до тієї відповідної частини, яку в загальній
площі обробленої землі становлять найгірші і менш родючі землі, а тому

\parbreak{}  %% абзац продовжується на наступній сторінці

\parcont{}  %% абзац починається на попередній сторінці
\index{iii2}{0135}  %% посилання на сторінку оригінального видання
і в зворотному відношенні до маси продукту, що його дають за однакової величини
вкладеного капіталу рівновеликі загальні площі землі. Отже, відношення між
кількістю найгіршої оброблюваної землі і кількостю найліпшої, в межах усієї
земельної площі країни справляє на загальну суму ренти вплив зворотний
тому, що його справляє відношення між якістю найгіршої з оброблюваних земель
і якістю ліпшої і найліпшої на ренту з акра і, тому, за інших рівних
умов і на суму ренти. Сплутування цих двох моментів дало привід до всеможливих
безглуздих заперечень проти диференційної ренти.

Отже, загальна сума ренти зростає в наслідок простого поширення культури
і сполученого з ним збільшеного застосування капіталу і праці до землі.

Але найважливіший пункт є такий: хоч згідно з припущенням відношення
рент з різних родів землі, обчислених на акр, не змінюється, а тому
не змінюється і норма ренти щодо капіталу, вкладеного в кожен акр, проте
виявляється таке: коли ми порівняємо І$а$ з І — тим випадком, коли число оброблюваних
акрів і вкладений в них капітал збільшились пропорційно, — то ми знайдемо,
що так само як загальна продукція зросла пропорційно збільшенній площі
обробленої землі, тобто обидві подвоїлись, так само зросла і загальна сума ренти.
Вона збільшилась з 18 до 36\pound{ ф. стерл.}, цілком так само, як число акрів, що
збільшилося з 4 до 8.

Коли ми візьмемо загальну площу в 4 акри, то загальна сума ренти з них
становитиме 18\pound{ ф. стерл.}, отже, пересічна рента, враховуючи землю, яка не дає
ренти, становитиме 4\sfrac{1}{2}\pound{ ф. стерл}. Таке обчислення міг зробити б, наприклад,
якийсь земельний власник, котрому належали б усі 4 акри; і таким самим
чином обчислюється статистично пересічна рента усієї країни. Загальна сума
ренти в 18\pound{ ф. стерл.} постає при застосуванні капіталу в 10\pound{ ф. стерл}.
Відношення між обома цими числами ми називаємо нормою ренти: отже, тут
180\%

Та сама норма ренти постає з І$а$, де замість 4 акрів обробляється 8,
але де землі всіх родів збільшились в однаковому відношенні. Загальна сума
ренти в 36\pound{ ф. стерл.} дає при 8 акрах і 20\pound{ ф. стерл.} застосованого капіталу
пересічну ренту в 4\sfrac{1}{2}\pound{ ф. стерл.} з акра і норму ренти в 180\%.

Якщо ми, навпаки, розглянемо І$b$, де приріст відбувся переважно на обох
гірших родах землі, то ми матимемо ренту в 42\pound{ ф. стерл.} з 12 акрів, тобто
пересічну ренту в 3\sfrac{1}{2}\pound{ ф. стерл.} з акра. Ввесь витрачений капітал є 30\pound{ ф. стерл.},
отже, норма ренти = 140\%. Отже, пересічна рента з акра зменшилась на 1\pound{ ф.
стерл.}, а норма ренти упала з 180 до 140\%. Отже, тут поруч з зростанням
загальної суми ренти з 18\pound{ ф. стерл.} до 42\pound{ ф. стерл.} відбувається зниження
пересічної ренти, обчислюваної так на акр, як і на капітал; зниження рівнобіжне,
але не пропорційне зростанню продукції. Це відбувається, не зважаючи на те,
що рента з усіх родів землі, обчислена так на акр, як і на витрачений капітал,
лишається та сама. Це відбувається тому, що \sfrac{3}{4} приросту припадає на
землю $А$, яка не дає ренти, і на землю $В$, яка дає лише мінімальну ренту.

Коли б у випадку І$b$ все поширення сталось лише на землі $А$, то ми
дали б 9 акрів на $А$, 1 на $В$, 1 на $C$ і 1 на $D$. Загальна сума ренти, як і давніш,
була б 18\pound{ ф. стерл.}, отже, пересічна рента з акра на цих 12 акрах
була б 1\sfrac{1}{2}\pound{ ф. стерл.}; 18\pound{ ф. стерл.} ренти на 30\pound{ ф. стерл.} витраченого капіталу
становили б норму ренти в 60\%. Середня рента, обчислена так на акр,
як і на застосований капітал, дуже зменшилася б, тоді як загальна сума ренти
не зросла б.

Порівняймо, нарешті, І$с$ з І і І$b$. В порівнянні з І земельна площа збільшилась
утроє і так само збільшився витрачений капітал. Загальна сума ренти
становить 72\pound{ ф. стерл.} з 12 акрів, отже, 6\pound{ ф. стерл.} з акра, проти 4\sfrac{1}{2}\pound{ ф.
стерл.} в випадку І. Норма ренти на витрачений капітал (72\pound{ ф. стерл.}: 30\pound{ ф. стерл.})
\parbreak{}  %% абзац продовжується на наступній сторінці

\parcont{}  %% абзац починається на попередній сторінці
\index{iii2}{0136}  %% посилання на сторінку оригінального видання
становить 240\% замість 180\%. Увесь продукт збільшився з 10 до
36 квартерів.

Порівняно з І$b$, де загальне число оброблених акрів, застосований капітал
і ріжниці між обробленими родами землі лишились ті самі, але розподіл
їх інший, продукт становить 36 квартерів замість 26 кварт., пересічна рента
з акра становить 6\pound{ ф. стерл.} замість 3\sfrac{1}{2} і норма ренти щодо всього авансового
капіталу тієї самої величини — 240\% замість 140\%.

Хоч як би ми стали розглядати різні становища, подані в таблицях І$а$,
І$b$, І$с$, чи як становища, що одночасно існують одно біля одного в різних країнах,
чи як послідовні становища в тій самій країні, — однаково, виявиться таке:
за сталої ціни збіжжя, сталої тому, що продукт з найгіршої землі, яка не дає
ренти, лишається той самий; за незмінної різниці у родючости різних розрядів
оброблюваної землі; при відносно однаковій кількості продукту, а, отже,
при однаковій витраті капіталу на відповідно однакові частини (акри) земельної
площі, оброблюваної в кожному розряді; при сталому в наслідок цього
відношенні між рентами з акра кожного роду землі і при однаковій нормі ренти
на капітал, вкладений у кожну дільницю землі того самого роду — виявиться,
\emph{поперше}, що сума ренти завжди зростає разом з поширенням оброблюваної
площі, а тому із збільшенням витрати капіталу, за винятком того випадку, коли
весь приріст припав би на землю, що не дає ренти. \emph{Подруге}, так пересічна
рента на акр (загальна сума ренти, поділена на все число оброблюваних
акрів), як і пересічна норма ренти (загальна сума ренти, поділена на ввесь
витрачений капітал) можуть значно варіювати, і хоч обидві в одному напрямку,
але в різних пропорціях у відношенні одна до однієї. Якщо не брати
на увагу того випадку, коли приріст відбувається лише на землі $А$,
яка не дає ренти, то виявляється, що пересічна рента на акр і пересічна
норма ренти на капітал, вкладений у хліборобство, залежить від того, які пропорційні
частини всієї оброблюваної землі становлять землі різних розрядів;
або, що сходить на те саме, від розподілу всього застосованого капіталу між
землями різної родючости. Чи багато, чи мало землі обробляється і тому (за
винятком того випадку, коли приріст припадає лише на $А$) чи більша, чи
менша є загальна сума ренти, пересічна рента на акр або пересічна рента на
застосований капітал лишається та сама, доки відношення різних родів оброблюваної
землі до всієї її площі лишається те саме. Дарма, що з поширенням
культури і збільшенням застосованого капіталу відбувається підвищення і
навіть значне підвищення загальної суми ренти, пересічна рента на акр і
пересічна норма ренти на капітал понижується, коли поширення земельних
дільниць, що не дають ренти, або дають лише незначну диференційну ренту,
зростає швидше, ніж поширення кращих земельних дільниць, що дають більшу
ренту. Навпаки, пересічна рента на акр і пересічна норма ренти на капітал підвищується в міру того,
як кращі землі починають становити відносно більшу
частину всієї площі, і тому на них припадає відносно більше застосованого
капіталу.

Таким чином, коли розглядати пересічну ренту на акр або гектар усієї
оброблюваної землі, як це звичайно робиться в статистичних працях при порівнянні
різних країн за тієї самої доби, або різних діб у тій самій країні, та
виявляється, що пересічна висота ренти на акр, а тому і загальна сума ренти
в певних пропорціях (хоч зовсім не в тих самих, а в швидше ростучих) відповідає
не відносній, а абсолютній родючості хліборобства в країні, тобто відповідає
масі продуктів, одержуваній пересічно з однакової земельної площі. Бо що
більшу частину із загальної площі становлять кращі землі, то більша маса
продуктів, одержувана з земельної площі однакової величини за однакової величини
застосованого капіталу, і то більша пересічна рента на акр. Зворотне в зворотному
\index{iii2}{0137}  %% посилання на сторінку оригінального видання
випадку. Тому здається, що рента визначається не відношенням диференційної
родючости, а абсолютною родючістю, і що таким чином закон диференційної
ренти знищується. Тому деякі явища заперечуються, або їх намагаються
пояснити несущими різницями пересічних цін хліба і ріжницями диференційної
родючости оброблюваних дільниць землі, тимчасом, як ці явища ґрунтуються
просто на тому, що відношення загальної суми ренти так до всієї площі оброблюваної
землі, як і до всього капіталу, вкладеного в землю, за однакової родючости
землі, що не дає ренти, а тому і за однакових цін продукції і за
однакової ріжниці між землями різних родів визначаються не тільки рентою
на акр, або нормою ренти на капітал, але в такій же мірі відношенням числа
акрів кожного роду до загального числа оброблюваних акрів; або, що сходить
на те саме, розподілом усього застосованого капіталу між різними родами землі.
До цього часу на цю обставину, дивовижно, зовсім не звертали уваги. В усякому
разі виявляється, і це є важливе для нашого дальшого досліду, що відносна
висота пересічної ренти на акр і пересічна норма ренти, або відношення
загальної суми ренти до всього вкладеного в землю капіталу, може збільшуватися
або зменшуватися просто в наслідок екстенсивного поширення культури,
за незмінних цін, незмінної ріжниці в родючості оброблюваних дільниць землі
і незмінної ренти з акра, або норми ренти на капітал, вкладений в акр
кожного розряду землі, що дійсно дає ренту, або на ввесь капітал, що дійсно
дає ренту.

\pfbreak

Треба зробити ще такі доповнення щодо тієї форми диференційної ренти,
яка досліджена в нас під рубрикою І, і що почасти мають також значення і
для диференційної ренти II.

\emph{Перше:} Ми бачили, як пересічна рента з акра або пересічна норма
ренти на капітал може підвищитись з поширенням культури за сталих цін і
незмінної ріжниці в родючості оброблюваних земельних дільниць. Скоро вся
земля в будь-якій країні буде привласнена, вкладення капіталу в землю, культура
і людність досягнуть певної висоти — обставини, наявність яких доводиться
припускати, скоро капіталістичний спосіб продукції став панівним і упідлеглив
собі і хліборобство, — ціна необроблюваної землі різної якости (просто припускаючи
існування диференційної ренти) визначається ціною оброблюваних дільниць
землі однакової якости і рівноцінного положення. Ціна є така сама — за вирахуванням
витрат на обробіток, що приєднується до неї — хоч ця земля і не
дає ренти. Ціна землі, звичайно, є не що інше, як капіталізована рента. Але
і в ціні оброблених земельних дільниць оплачуються лише майбутні ренти, наприклад,
одним заходом виплачується наперед ренти за 20 років, коли міродайний
розмір проценту є 5\%. Коли продається земля, вона продається як така, що дає
ренту, і перспективний характер ренти (яку розглядається тут як витвір землі, чим
вона є тільки з видимости) призводить до того, що необроблювана земля не
відрізняється від оброблюваної. Ціна необроблюваних дільниць землі, як і рента
з них, — а ціна становить лише зосереджену формулу ренти — має суто ілюзорний
характер, поки ці дільниці не будуть дійсно використані. Але вона, таким
чином, визначається a priori і реалізується, скоро знаходяться покупці. Тому,
коли дійсна пересічна рента в певній країні визначається дійсною пересічною
річною сумою ренти і відношенням цієї останньої до всієї оброблюваної площі,
то ціна необроблюваної частини земельної площі визначається ціною оброблюваної
і є тому лише відбиток вкладення капіталу та його наслідків на оброблюваних
земельних дільницях. А що за винятком найгіршої землі, землі усіх родів
дають ренту (а ця рента, як ми побачимо в рубриці II, зростає з масою капіталу
і з відповідною до цієї маси інтенсивністю культури), то і створюється таким чином
\parbreak{}  %% абзац продовжується на наступній сторінці

\parcont{}  %% абзац починається на попередній сторінці
\index{iii1}{0138}  %% посилання на сторінку оригінального видання
для грубих нумерів пряжі і важких тканин... Є побоювання, що
збільшена кількість машин, недавно установлених у камвольній
промисловості, приведе до подібної реакції і в цій галузі промисловості.
Пан Бекер обчислює, що в самому лиш 1849 році в цій
галузі промисловості продукт ткацьких верстатів збільшився на
40\%, продукт веретен на 25—30\%, а розширення промисловості
все ще продовжується в тих самих розмірах“ („Rep. of Insp.
of Fact., April\footnote{
рік. Жовтень. „Ціна бавовни продовжує... викликати
значну пригніченість в цій галузі промисловості, особливо для
таких товарів, для яких сировинний матеріал становить значну
частину витрат виробництва. Значний ріст ціни на шовк-сирець
часто приводив до пригнічення і в цій галузі“ („Rep. of Insp. of
Fact., Oct. 1850“, стор. 14). — За цитованим тут звітом комітету
королівського товариства культури льону в Ірландії, висока
ціна льону при низьких цінах інших сільськогосподарських продуктів
забезпечила тут значне розширення виробництва льону
для наступного року (стор. [31] 33).
}“, стор. 54).

1853 рік. Квітень. Великий розквіт. „Ніколи ще за ті 17 років,
протягом яких мені офіціально доводилось знайомитися з станом
фабричної округи Ланкашіра, я не спостерігав такого загального
процвітання; діяльність по всіх галузях надзвичайна“, —
каже Л. Горнер („Rep. of Insp. of Fact., April 1853“, стор. 19).

1853 рік. Жовтень. Депресія в бавовняній промисловості.
„Перепродукція“ („Rep. of Insp. of Fact., Oct. 1853“, стор. [13] 15).

1854 рік. Квітень. „Шерстяна промисловість, хоч справи в ній
йшли не жваво, повністю завантажила всі фабрики; так само
й бавовняна промисловість. Камвольна промисловість протягом
цілого минулого півріччя всюди працювала нерегулярно... В лляній
промисловості відбувалися порушення в наслідок зменшеного
подання льону й коноплі з Росії в зв’язку з кримською війною“
(„Rep. of Insp. of Fact., [April] 1854“, стор. 37).

1859 рік. „Справи в шотландській лляній промисловості все
ще в пригніченому стані... бо сировинний матеріал рідкий і дорогий;
погана якість торішнього урожаю в прибалтійських країнах,
звідки йде до нас головний довіз, справлятиме шкідливий вплив
на промисловість цієї округи; навпаки, джут, який в багатьох
грубих товарах помалу витискує льон, не є ні надзвичайно дорогий,
ні рідкий... приблизно половина машин в Денді пряде
тепер джут“ („Rep. of Insp. of Fact., April 1859“, стор. 19). —
„В наслідок високої ціни сировинного матеріалу льонопрядіння
все ще зовсім невигідне, і в той час, як усі інші фабрики працюють
повний час, ми маємо ряд прикладів спинення машин,
що перероблюють льон... Прядіння джуту... перебуває в більш
задовільному стані, бо за останній час ціна на цей матеріал
стала помірнішою“ (Rep. of Insp. of Fact., Oct. 1859“, стор. 20).

\index{iii1}{0139}  %% посилання на сторінку оригінального видання
1861—1864 рр. Американська громадянська війна. Cotton Famine [бавовняний
голод]. Найбільший приклад перерви в процесі виробництва в наслідок
недостачі й дорожнечі сировинного матеріалу

1860 рік. Квітень. „Щодо стану справ, то я радий можливості
повідомити вас, що, не зважаючи на високу ціну сировинних
матеріалів, всі галузі текстильної промисловості, за винятком
шовку, працювали протягом останнього півроку дуже добре...
В деяких бавовняних округах робітників шукали шляхом оголошень,
і робітники йшли туди з Норфолька та інших землеробських
графств... Як видно, в усіх галузях промисловості панує велика недостача
сировинного матеріалу. Тільки... ця недостача тримає нас
у певних межах. В бавовняній промисловості число новозбудованих
фабрик, розширення наявних фабрик і попит на робітників,
мабуть, ніколи ще не досягали такого високого рівня, як
тепер. Скрізь і всюди шукають сировинного матеріалу“ („Rep.
of Insp. of Fact., April 1860“ [стор. 57]).

1860 рік. Жовтень. „Стан справ у бавовняних, шерстяних
і льонопрядільних округах був добрий; в Ірландії він, як кажуть,
вже більше року навіть „дуже добрий“, і був би ще кращий,
коли б не висока ціна на сировинний матеріал. Прядільники
льону, здається, з більшим нетерпінням, ніж будьколи,
чекають відкриття індійських джерел постачання за допомогою
залізниць і відповідного розвитку індійського землеробства, щоб,
нарешті... добитися відповідного їх потребам подання льону“
(„Rep. of Insp. of Fact., Oct. 1860“, стор. 37).

1861 рік. Квітень. „Стан справ у даний момент пригнічений...
деякі бавовняні фабрики працюють неповний час і багато шовкових
фабрик працюють тільки частково. Сировинний матеріал
дорогий. Майже в усіх галузях текстильної промисловості
ціна його вища, ніж та, при якій він міг би бути перероблений
для маси споживачів“ („Rep. of Insp. of Fact., April 1861“, стор. 33).

Тепер виявилось, що в 1860 році в бавовняній промисловості
була перепродукція; наслідки цього давалися взнаки ще протягом
ближчих років. „Потрібно було від двох до трьох років,
поки світовий ринок поглинув перепродукцію 1860 року“ („Rep.
of Insp. of Fact., October 1863“, стор. 127). „Пригнічений стан
ринків бавовняних фабрикатів у Східній Азії, на початку 1860 року,
справив відповідний зворотний вплив на стан справ у Блекберні,
де пересічно 30 000 механічних ткацьких верстатів майже виключно
заняті у виробництві тканин для цього ринку. В наслідок
цього попит на працю був уже тут обмеженим багато місяців
перед тим, як став відчутним вплив бавовняної блокади...
На щастя, це уберегло багатьох фабрикантів від краху. Запаси,
поки їх тримали на складах, підвищились у своїй вартості, і таким
чином уникнуто було того жахливого знецінення, яке
інакше при такій кризі було б неминучим“ („Rep. of Insp. of
Fact., Oct. 1862“, стор. 28, 29 [30]).

\parcont{}  %% абзац починається на попередній сторінці
\index{iii2}{0140}  %% посилання на сторінку оригінального видання
ніж та, яка оброблялася до того часу, — на різних родах землі, починаючи з А
і до D, отже, наприклад, обробіток більших площ В і С, зовсім не має за свою передумову
попереднього підвищення цін хліба, подібно до того, як щорічне поширення,
наприклад, бавовнопрядіння не потребує постійного підвищення цін пряжі. Хоч
значне підвищення або пониження ринкових цін впливає на розмір продукції,
проте, залишаючи це осторонь, і при пересічних цінах, що своїм рівнем не справляють
на продукцію ні пригнобного, ні особливо оживного впливу, у хліборобстві
(як і в усіх інших галузях продукції, проваджених капіталістично) постійно
відбувається та відносна перепродукція, яка сама по собі тотожня з
акумуляцією і яка, за інших способів продукції, безпосередньо спричинюється
зростом людности, а в колоніях — постійною іміграцією. Попит постійно зростає,
і передбачаючи це, постійно вкладають в нові землі все нові й нові капітали;
хоч, залежно від обставин, капітали вкладають на створення різних хліборобських
продуктів. До цього само по собі призводить наростання нових капіталів.
Щодо окремого капіталіста, то розміри своєї продукції він припасовує до розміру
капіталу, що він ним порядкує, оскільки сам він може ще його контролювати.
Він прагне лише того, щоб захопити якомога більше місця на ринку.
Коли настає перепродукція, то він обвинувачує в цьому не себе, а своїх конкурентів.
Окремий капіталіст може розширювати свою продукцію так привласнюючи
собі порівняно більшу відповідну частину даного ринку, як і розширюючисамий
ринок.

Розділ сороковий.

Друга форма диференційної ренти.
(диференційна рента II).

До цього часу ми розглядали диференційну ренту лише як наслідок різної
продуктивности однакових капіталовкладень на однакових площах землі з різною
родючістю, так що диференційна рента визначалась ріжницею між продуктом
капіталу, вкладеного в найгіршу землю, що не дає ренти, і продуктом капіталу,
вкладеного в кращу землю. При цьому ми мали одночасне приміщення капіталів
в різні дільниці землі, так що кожному новому приміщенню капіталу відповідало
поширення обробітку землі, збільшення оброблюваної площі. Але, кінець-кінцем,
диференційна рента по суті справи була лише наслідком різної
продуктивности рівних капіталів, вкладених в землю. Чи буде будь-яка ріжниця,
коли капітали різної продуктивности вкладаються один після одного в ту
саму дільницю землі, і коли вони вкладаються один поряд одного в різні дільниці
землі — чи буде якась ріжниця, коли тільки припустити, що наслідки ті самі?

Насамперед, не можна заперечувати, що, оскільки справа йде про створення
надзиску, цілком байдуже, чи дадуть 3 ф. ст. ціни продукції, вкладені в акр
землі А, продукт в 1 квартер, так що 3 ф. ст. будуть ціною продукції і регуляційною
ринковою ціною одного квартера, тоді як 3 ф. ст. ціни продукції на акрі землі
В дадуть 2 квартери і, таким чином, надзиск в 3 ф. ст., а 3 ф. ст. ціни продукції
на акрі землі С дадуть 3 квартери і 6 ф. ст. надзиску і, нарешті, 3 ф. ст. ціни
продукції на акрі землі D дадуть 4 квартери і 9 ф. ст. надзиску; чи такий самий
наслідок буде досягнутий тим, що ці 12 ф. ст. ціни продукції, або 10 ф. ст.,
капіталу будуть вкладені з таким самим успіхом, в такій самій послідовності,
в один і той самий акр. І в тому, і в тому випадку це капітал в 10 ф. ст.
що частини його вартости, послідовно вкладені по 2 1/2 ф. ст., — однаково, чи вкладаються
вони один поряд одного на 4 акрах землі різної родючости, чи послідовно на
тому самому акрі, — в наслідок того, що продукт їхній різний, однією своєю частиною
\parbreak{}  %% абзац продовжується на наступній сторінці

\parcont{}  %% абзац починається на попередній сторінці
\index{iii2}{0141}  %% посилання на сторінку оригінального видання
не дають надзиску, тимчасом як їхні інші частини дають надзиск, відповідний
ріжниці між їхнім продуктом і продуктом того вкладення, що не дає ренти.

Надзиски і різні норми надзиску з різних частин вартости капіталу створюються
в обох випадках рівномірно. А рента є не що інше, як форма цього
надзиску, який становить її субстанцію. Але в усякому разі другий спосіб являє
собою труднощі щодо перетворення надзиску в ренту, цієї зміни форми, яка містить
в собі перенесення надзиску від капіталістичного орендаря до земельного власника.
Звідси упертий опір офіційній хліборобській статистиці з боку англійських
орендарів. Звідси боротьба між ними й землевласниками за встановлення дійсних
наслідків приміщення їхніх капіталів (Morton). Рента встановлюється саме
при оренді земель, і після цього надзиски, що постають з послідовного приміщення
капіталу, потрапляють в кишеню орендаря весь час, поки триває орендний
договір. Звідси боротьба орендарів за тривалі орендні договори і навпаки,
в наслідок переваги сили лендлордів, збільшення числа контрактів, які можна
щороку скасовувати (tenancies at will).

Тому ясно з самого початку: хоч для закону створення надзиску нічого не
змінюється від того, чи вкладено рівні капітали з різними наслідками один поряд
одного в рівновеликі земельні дільниці, чи вкладено їх послідовно один за одним в ту
саму дільницю землі, — проте, це становить значну ріжницю для перетворення
надзиску в земельну ренту. Останній спосіб замикає це перетворення, з одного
боку, у вужчі, з другого — у мінливіші межі. Тому в країнах інтенсивної культури
(а економічно під інтенсивною культурою ми розуміємо не що інше, як
концентрацію капіталу на тій самій земельній площі, замість розподілу його
між земельними дільницями, що лежать одна біля однієї) праця таксатора, як
це зазначає Morton у своїх «Resources of States», стає дуже важливою, складною
і важкою професією. При триваліших поліпшеннях землі, коли минає термін
орендного договору, штучно підвищена диференційна родючість землі збігається
з її природною, а тому і оцінка розміру ренти збігається з оцінкою розміру
ренти від земель різної родючости взагалі. Навпаки, оскільки створення надзиску
визначається висотою капіталу, вкладеного в продукцію, висота ренти, одержуваної
при певній величині цього капіталу, приєднується до пересічної ренти
країни і тому дбають про те, щоб новий орендар порядкував капіталом, достатнім
для продовження культури з колишнім ступенем інтенсивности.

При розгляді диференційної ренти II треба відзначити ще такі пункти:

Поперше: Її база і вихідний пункт, не тільки історично, але й оскільки
справа йде про рух за всякого даною моменту, є диференційна рента 1, тобто
одночасний обробіток розміщених одна поряд однієї земельних дільниць, різних
своєю родючістю і положенням; отже одночасне вживання одної поряд однієї
різних складових частин усього хліборобського капіталу на земельних дільницях
різної якости.

Історично це само собою зрозуміло. В колоніях колоністам доводиться прикладати
лише незначний капітал; за головних агентів продукції є праця і земля.
Кожен окремий голова родини намагається добитися для себе і своїх самостійного
поля дії, поряд з товаришами-колоністами. У власне хліборобстві це
взагалі мусило так відбуватися вже за докапіталістичних способів продукції. При
вівчарстві і взагалі скотарстві як самостійних галузях продукції земля експлуатується
більш або менш спільно, і з самого початку експлуатація має екстенсивний
характер. Капіталістичний спосіб продукції походить з давніших способів
продукції, за яких засоби продукції, фактично або юридично, становлять
власність самого обробника, словом, з ремісничої продукції в хліборобств. По
суті справи з ремісничої продукції лише поступово розвивається концентрація
засобів продукції і перетворення їх на капітал, що протистоїть безпосереднім
\parbreak{}  %% абзац продовжується на наступній сторінці


\index{iii1}{0142}  %% посилання на сторінку оригінального видання
За даними того самого звіту, з бавовняних робітників Ланкашіра
й Чешіра тоді працювали повний час 40 146 робітників,
або 11,3\%, неповний робочий час — 134 767 робітників, або 38\%,
зовсім без роботи було 179 721 робітник, або 50,7\%. Коли виключити
звідси дані про Манчестер і Больтон, де випрядаються
головним чином тонкі нумери, — галузь, що порівняно мало потерпіла
від недостачі бавовни, — то справа виявиться ще несприятливішою, 'а
саме: таких, що працюють повний час — 8,5\%,
неповний час — 38\%, безробітних — 53,5\% (стор. 19, 20).

„Для робітників становить істотну ріжницю, чи переробляють
вони добру чи погану бавовну. В перші місяці року, коли фабриканти
намагались тримати свої фабрики в русі тим, що вживали
всяку бавовну, яку тільки можна було купити по помірних цінах,
багато поганої бавовни потрапило на ті фабрики, де раніше звичайно
застосовували добру; ріжниця в заробітній платі робітників
була така велика, що відбулося багато страйків, бо робітники
при старій відштучній платі тепер не могли вже добути
собі зносного щоденного заробітку\dots{} В деяких випадках ріжниця
в наслідок застосовування поганої бавовни становила навіть при
повному робочому часі половину всього заробітку“ (стор. 27).

1863 рік. Квітень. „На протязі цього року зможуть бути заняті
повний час трохи більше половини бавовняних робітників“
(„Rep. of Insp. of Fact., April 1863“, стор. 14).

„Дуже серйозна невигода при застосуванні ост-індської бавовни,
яку тепер фабрики мусять споживати, полягає в тому,
що швидкість машин при цьому мусить бути дуже уповільнена.
Протягом останніх років було вжито всіх заходів для збільшення
цієї швидкості, так щоб ті самі машини виконували більше
роботи. Але зменшена швидкість зачіпає робітника в такій самій
мірі, як фабриканта, бо більшість робітників одержують відштучну
плату — прядільники стільки то за фунт випряденої
пряжі, ткачі стільки то за витканий кусок; і навіть у інших
робітників, які одержують тижневу плату, заробітна плата повинна
знизитися в наслідок зменшення виробництва. На підставі
моїх досліджень\dots{} і переданих мені даних про заробіток бавовняних
робітників на протязі цього року\dots{} виявляється зменшення
заробітної плати пересічно на 20\%, в деяких випадках
на 50\% порівняно з висотою заробітної плати 1861 року“
(стор. 13). — „Зароблена сума залежить\dots{} від того, який матеріал
переробляється\dots{} Становище робітників, щодо суми заробленої
плати, тепер (жовтень 1863 року) багато краще, ніж минулого
року в цей час. Машини поліпшено, сировинний матеріал
знають краще, і робітники легше справляються з тими труднощами,
з якими їм доводилося боротись спочатку. Минулої весни
я був у Престоні в одній швацькій школі“ [благодійна установа
для безробітних]; „дві молоді дівчини, які за день перед тим
були послані до ткацької фабрики, де, за заявою фабриканта, вони
могли б заробити 4 шилінга на тиждень, просили, щоб їх знову
\parbreak{}  %% абзац продовжується на наступній сторінці

\input{_0143.tex}
\parcont{}  %% абзац починається на попередній сторінці
\index{i}{0144}  %% посилання на сторінку оригінального видання
засоби продукції не лише для шестигодинного, але для дванадцятигодинного
процесу праці. Коли 10 фунтів бавовни вбирали
в себе 6 робочих годин і перетворювались на 10 фунтів пряжі, то
20 фунтів бавовни вберуть у себе 12 робочих годин  перетворяться
на 20 фунтів пряжі. Розгляньмо продукт здовженого процесу
праці. В 20 фунтах пряжі упредметнено тепер 5 робочих днів:
4 — у спожитій кількості бавовни й веретен, а один день бавовна
увібрала в себе протягом процесу прядіння. Але грошовий вираз
5 робочих днів є 30\shil{ шилінґів}, або 1\pound{ фунт стерлінґів} і 10\shil{ шилінґів.}
Отже, це є ціна 20 фунтів пряжі. Фунт пряжі коштує, як і
раніш, 1\shil{ шилінґ} і 6\pens{ пенсів.} Але сума вартости товарів, кинутих у
процес, становила 27\shil{ шилінґів.} Вартість пряжі становить 30\shil{ шилінґів.}
Вартість продукту зросла на одну дев’яту понад вартість,
авансовану на його продукцію. Таким чином 27\shil{ шилінґів} перетворились
на 30\shil{ шилінґів.} Вони породили додаткову вартість
у 3\shil{ шилінґи.} Нарешті трюк удався. Гроші перетворилися
на капітал.

Всі умови проблеми розв’язано й законів товарового обміну
ані трохи не порушено. Еквівалент обмінено на еквівалент.
Капіталіст як покупець платив за кожний товар — бавовну,
веретена, робочу силу — за його вартістю. А потім він зробив
те, що робить кожний інший покупець товарів: він споживав
їхню споживну вартість. Процес споживання робочої сили, який
разом з тим є процес продукції товару, дав продукт у 20 фунтів
пряжі вартістю в 30\shil{ шилінґів.} Тепер капіталіст повертається
на ринок і продає товари, тимчасом як раніш він на ньому
купував товари. Він продає 1 фунт пряжі за 1\shil{ шилінґ} 6\pens{ пенсів} —
ні на шаг більше ні менше від його вартости. І все ж він вилучає
з циркуляції на 3\shil{ шилінґи} більше, ніж первісно подав до неї.
Цілий цей процес, перетворення його грошей на капітал, відбувається
у сфері циркуляції й відбувається не в ній. Він відбувається
за посередництвом циркуляції, бо зумовлює його купівля
робочої сили на товаровому ринку; не в циркуляції, бо остання
лише підготовляє процес зростання вартости, який відбувається
у сфері продукції. Таким чином «tout pour le mieux dans le meilleur
des mondes possibles».\footnote*{
Все є якнайкраще в цьому якнайкращому з світів. \emph{Ред.}
}

Перетворюючи гроші на товари, що служать за речові
елементи утворення нового продукту або за фактори процесу
праці, прилучаючи до їхньої мертвої предметности живу робочу
силу, капіталіст перетворює вартість — минулу, упредметнену,
мертву працю — на капітал, на вартість, що сама з себе зростає,
на одушевлену потвору, що починає «працювати» так, наче вона
захоплена любовною жагою.

Коли ми тепер порівняємо процес утворення вартости й процес
зростання вартости, то побачимо, що процес зростання вартости є
не що інше, як процес утворення вартости, продовжений за межі
певного пункту. Коли процес утворення вартости триває лише
\parbreak{}  %% абзац продовжується на наступній сторінці

\parcont{}  %% абзац починається на попередній сторінці
\index{ii}{0145}  %% посилання на сторінку оригінального видання
поточний капітал, а не основний, оскільки І) вартість його цілком входить
у продукт і 2) оскільки його in natura цілком заміщено новим екземпляром
з нового продукту.

А.~Сміс каже нам, з чого складається обіговий і основний капітал.
Він перелічує ті речі, ті речові елементи, що становлять основний капітал,
і ті, що становлять обіговий капітал, ніби таке призначення властиве
цим речам речово, з природи, а не випливає з певних функцій
цих речей в капіталістичному процесі продукції. І однак в тому самому
розділі (Book II, chap. 1) він зауважує, що, хоч певна річ, напр., житлова
будівля, призначена для безпосереднього споживання „може давати
дохід своєму власникові, а значить, служити йому, \so{функціонуючи як
капітал}, однак вона не може ні давати дохід суспільству, ані служити
йому, функціонучи як капітал, отже, вона ані трохи не збільшує доходу
всього суспільства“\footnote*{
\dots{} may yield a revenue to its proprietor, and thereby serve \so{in the function
of a capital} to him, it cannot yield any to the public, nor serve in the function
of a capital to it, and the revenue of the whole body of the people can never be
in the smallest degree increased by it“(p. 186).
}. Тут А.~Сміс цілком виразно висловлює думку, що
властивість бути капіталом речі мають не як такі і не за всяких обставин,
але що це є така функція, яку вони, залежно від обставин, іноді мають,
а іноді не мають. Але те, що має силу для капіталу взагалі, те має
силу й для його підрозділів.

Ті самі речі становлять складову частину поточного або основного
капіталу залежно від того, яку функцію вони виконують в процесі праці.
Напр., худоба, як робоча худоба (засіб праці) становить речову форму
існування основного капіталу; навпаки, як худоба, відгодовувана на
заріз (сировинний матеріял), вона становить складову частину обігового
капіталу фармера. З другого боку, та сама річ може іноді функціонувати
як складова частина продуктивного капіталу, а іноді належати до
фонду безпосереднього споживання. Напр., будинок, функціонучи як місце
праці, є основна складова частина продуктивного капіталу, а функціонуючи
як житлова будівля власника, зовсім не має форми капіталу.
Ті самі засоби праці можуть у багатьох випадках функціонувати то як
засоби продукції, то як засоби споживання.

Це була одна з помилок, що випливають із Смісового уявлення: особливості
основного та обігового капіталу розглядати як особливості,
властиві речам. Аналіза процесу праці („Капітал“, книга 1, розділ V)
вже показала, як змінюються визначення засобу праці, матеріялу праці,
продукту, залежно від різної ролі, що та сама річ відіграє в цьому
процесі. Але визначення основного і не основного капіталу ґрунтуються
й собі на тих певних ролях, що їх ці елементи відіграють у процесі праці,
а, значить, і в процесі утворення вартости.

\parcont{}  %% абзац починається на попередній сторінці
\index{iii2}{0146}  %% посилання на сторінку оригінального видання
нового надзиску, при чому воно може відбутися одночасно на землях D, C, В, А.
Коли ж, навпаки, з обробітку буде витиснута гірша земля А, то реґуляційна
ціна продукції понизиться, і від відношення між зменшеною ціною одного
квартера і збільшеним числом квартерів, що створюють надзиск, залежить, чи
підвищується чи понижується визначений в грошах надзиск, а отже, і диференційна
рента. Але в усякому разі тут виявляється та варта уваги обставина,
що за зменшуваних надзисків з послідовних капіталовкладень, ціна продукції
може зменшуватися замість підвищуватись, як це здається на перший погляд.

Ці додаткові приміщення капіталу з зменшуванними додатковими здобутками
цілком відповідають тому випадкові, коли в землі, що за своєю родючістю
містяться між А і В, В і C, C і D, було б, наприклад, вкладено чотири нові
самостійні капітали по 2\sfrac{1}{2} ф. стерл., які давали б відповідно 1\sfrac{1}{2} квартери,
2\sfrac{1}{3}, 2\sfrac{2}{3} і 3 квартери. Для всіх цих чотирьох додаткових капіталів на всіх
цих родах землі створились би надзиски, потенціяльні ренти, хоч норма надзиску
порівняно з тією, що дає таке саме капіталовкладення на щоразу кращій землі,
і зменшилася б. Та цілком байдуже було б, чи вкладено ці чотири капітали
в землю D і~\abbr{т. ін.}, чи розподілені вони між D і А.

Ми підходимо тепер до посутньої ріжниці між обома формами диференційної
ренти.

Коли справа йде про диференційну ренту І, то за незмінної ціни продукції
і незмінних ріжниць, разом з загальною сумою ренти може підвищитись
пересічна рента на акр, або пересічна норма ренти на капітал; але пересічність
є лише абстракція. Дійсний рівень ренти, обчислений на акр або на капітал,
тут лишається той самий.

Навпаки, розмір ренти обчислений на акр в тих самих обставинах, може
підвищитись, хоч норма ренти, обчислена на витрачений капітал, лишається
та сама.

Припустімо, що продукція подвоюється в наслідок того, що в землі А,
В, C, D вкладалося б по 5 ф. стерл. капіталу замість 2\sfrac{1}{2} ф. стерл., тобто
в цілому 20 ф. стерл. замість 10 ф. стерл., з незмінною відносною родючістю.
Це було б цілком те саме, як коли б замість одного акра кожного з цих
родів землі оброблялося 2 акри, а витрати лишалися б ті самі. Норма зиску
залишалася б та сама, так само як і її відношення до надзиску або ренти. Але,
коли б земля А почала давати тепер 2 квартери, В — 4, C — 6, D — 8, то ціна
продукції, як і давніш, дорівнювала б 3 ф. стерл. за квартер, бо цей приріст
завдячував би своїм походженням не подвоєній родючості за незмінного розміру
капіталу, а незмінній відносній родючості за подвоєного розміру капіталу. Ці
два квартери з А коштували б тепер 6 ф. стерл., як давніш 1 квартер коштував
3 ф. стерл. Зиск на всіх чотирьох родах землі подвоївся б, але тільки
тому, що подвоївся б витрачений капітал. Але в тому самому відношенні подвоїлася
б рента, вона дорівнювала б 2 квартерам для В замість 1 квартера, 4 квартерам
для C замість 2 і 6 квартерам для D замість 3, і відповідно до цього
грошова рента для В, C, D дорівнювала б відповідно 6 ф. стерлінґів, 12 ф.
стерл., 18 ф., стерл. Так само як продукт з акра, подвоїлася б і грошова
рента з акра, отже, і ціна землі, що в ній капіталізується ця грошова рента. За
таким розрахунком підвищується рівень збіжжевої і грошової ренти, а тому і
ціна землі, бо маштаб, що ним виміряється ця ціна, акр, є земельна площа
сталої величини. Навпаки, у пропорційній висоті ренти не сталося жодної зміни,
коли обчислювати її як норму ренти щодо витраченого капіталу. Загальна сума
ренти в 36 стосується до витраченого капіталу в 20, як загальна сума ренти
в 18 до витраченого капіталу в 10. Це саме має силу і для відношення грошової
ренти з земель кожного роду до вкладеного в них капіталу; так, наприклад,
12 ф. стерл. ренти з землі C стосуються до 5 ф. стерл. капіталу, як
\parbreak{}  %% абзац продовжується на наступній сторінці

\parcont{}  %% абзац починається на попередній сторінці
\index{iii2}{0147}  %% посилання на сторінку оригінального видання
давніш 6\pound{ ф. стерл.} ренти до 2\sfrac{1}{2}\pound{ ф. стерл.} капіталу. Тут не виникає нових
ріжниць між витраченими капіталами, але виникають нові надзиски тільки
тому, що додатковий капітал вкладається в якусь з земель, що дають ренту,
або в усі землі, даючи при цьому пропорційно своїй величині той самий
продукт. Коли б подвійна витрата капіталу була зроблена, наприклад, лише
на $C$, то диференційна рента між $C$, $В$ і $D$, обчислена на капітал, залишалася б
та сама; бо хоч її маса з $C$ і подвоїлася б, але подвоївся б і вкладений
капітал.

Звідси видно, що за незмінної ціни продукції, незмінної норми зиску і
незмінних ріжниць (а тому і за незмінної норми надзиску або ренти, обчислених
на капітал), висота ренти, визначеної в продукті і в грошах, може підвищитись
з акра, а тому може підвищитись і ціна землі.

Те саме може статися при зменшуваних нормах надзиску, отже, і ренти,
тобто при зменшуваній продуктивності додаткових вкладень капіталу, що все
ще дають ренту. Коли б другі вкладення капіталу в 2\sfrac{1}{2}\pound{ ф. стерл.} не дали
подвоєного продукту, а дали б на $В$ лише З\sfrac{1}{2} квартери, на $C$ — 5 і на $D$ —
6 квартерів, то диференційна рента на $В$ для другого вкладення капіталу в 2\sfrac{1}{2} ф.
стерл. була б лише \sfrac{1}{2} квартера замість 1, на C — 1 замість 2 і на $D$ — 2 замість
3 квартерів. Відношення між рентою і капіталом для обох послідовних
витрат було б таке:

\begin{table}[H]
  \centering
  \small
  \begin{tabular}{l l l}
   
  & Перша витрата & Друга витрата \\

$В$: & Рента 3\pound{ ф. стерл.}, капітал 2\sfrac{1}{2}\pound{ ф. стерл.} 
      & Рента 1\sfrac{1}{2}\pound{ ф. стерл.}, капітал 2\sfrac{1}{2}\pound{ ф. стерл.} \\

$C$: & \ditto{Рента} 6\ditto{\pound{ ф. стерл.}, капітал} 2\sfrac{1}{2} 
      & \ditto{Рента} З\phantom{\sfrac{1}{2}}\ditto{\pound{ ф. стерл.}, капітал} 2\sfrac{1}{2} \\

$D$: & \ditto{Рента} 9\ditto{\pound{ ф. стерл.}, капітал} 2\sfrac{1}{2}
      & \ditto{Рента} 6\phantom{\sfrac{1}{2}}\ditto{\pound{ ф. стерл.}, капітал} 2\sfrac{1}{2} \\
  \end{tabular}
\end{table}

\noindent{}Не зважаючи на таку понижену норму відносної продуктивности капіталу,
а тому і надзиску, обчисленого на капітал, збіжжева і грошова рента підвищилася
б для $В$ з 1 до 1\sfrac{1}{2} квартерів (з 3 до 4\sfrac{1}{2}\pound{ ф. стерл.}), для $C$ з 2 до 3 квартерів (з 6 до 9\pound{ ф. стерл.}) і для $D$ з 3 до 5 квартерів (з 9 до 15\pound{ ф. стерл.})
В цьому випадку ріжниці для додаткових капіталів порівняно з капіталом,
вкладеним в $А$, зменшилися б, ціна продукції лишилася б та сама, але рента
на акр, а тому і ціна землі на акр підвищилася б.

Щодо комбінацій диференційної ренти II, що має за свою передумову, як
свою базу диференційну ренту І, то вони такі.

\section{Диференційна рента II.~Перший випадок: стала ціна продукції}

Таке припущення включає й те, що ринкова ціна, як і давніше, регулюється
капіталом, вкладеним в найгіршу землю $А$.

I.~Коли додатковий капітал, вкладений в якусь із земель $В$, $C$, $D$, що
дають ренту, продукує лише стільки, скільки продукує такий самий капітал на
землі $А$, тобто коли при регуляційній ціні продукції він дає лише пересічний
зиск, не даючи, отже, жодного надзиску, то вплив справлений ним на ренту, дорівнює
нулеві. Все лишається, як було давніш. Це те саме, як коли б перше-ліпше
число акрів якости $А$, найгіршої землі, було приєднано до вже оброблюваної
площі.

II.~Додаткові капітали дають на землях усіх родів додаткові продукти в кількості,
пропорційній величині цих капіталів; тобто — величина продукції зростає,
\parbreak{}  %% абзац продовжується на наступній сторінці

\parcont{}  %% абзац починається на попередній сторінці
\index{iii2}{0148}  %% посилання на сторінку оригінального видання
залежно від специфічної родючости землі кожного типу, пропорційно величині
додаткового капіталу. В XXXIX розділі ми виходили з такої таблиці І:

\begin{table}[h]
  \begin{center}
    \emph{Таблиця І}
    \footnotesize

  \begin{tabular}{c c c c c c c c c c c}
    \toprule
      \multirowcell{2}{\makecell{Рід \\землі}} &
      \multirowcell{2}{\rotatebox[origin=c]{90}{Акри}} &
      \rotatebox[origin=c]{90}{Капітал} &
      \rotatebox[origin=c]{90}{Зиск} &
      \rotatebox[origin=c]{90}{\makecell{Ціна про- \\ дукції}} &
      \multirowcell{2}{\rotatebox[origin=c]{90}{\makecell{Продукт \\ в кварт.}}} &
      \rotatebox[origin=c]{90}{\makecell{Продажна \\ ціна}} &
      \rotatebox[origin=c]{90}{Здобуток} &
      \multicolumn{2}{c}{Рента} &
      \multirowcell{2}{\makecell{Норма \\надзиску}} \\

      \cmidrule(rl){3-3}
      \cmidrule(rl){4-4}
      \cmidrule(rl){5-5}
      \cmidrule(rl){7-7}
      \cmidrule(rl){8-8}
      \cmidrule(rl){9-10}

       &  &  ф. ст. & ф. ст. & ф. ст. & & ф. ст. & ф. ст. & Кварт. & ф. ст. &  \\
      \midrule

      A & 1 &  \phantom{0}2\sfrac{1}{2} & \sfrac{1}{2} & \phantom{0}3 & \phantom{0}1 & 3 & \phantom{0}3 & 0 & \phantom{0}0 & \phantom{00}0\\
      B & 1 &  \phantom{0}2\sfrac{1}{2} & \sfrac{1}{2} & \phantom{0}3 & \phantom{0}2 & 3 & \phantom{0}6 & 1 & \phantom{0}3 & 120\% \footnotemarkZ{}\\ % ця мітка у заголовку \\
      C & 1 &  \phantom{0}2\sfrac{1}{2} & \sfrac{1}{2} & \phantom{0}3 & \phantom{0}3 & 3 & \phantom{0}9 & 2 & \phantom{0}6 & 240\%\\
      D & 1 &  \phantom{0}2\sfrac{1}{2} & \sfrac{1}{2} & \phantom{0}3 & \phantom{0}4 & 3 & 12           & 3 & \phantom{0}9 & 360\%\\
     \cmidrule(rl){1-1}
     \cmidrule(rl){2-2}
     \cmidrule(rl){3-3}
     \cmidrule(rl){5-5}
     \cmidrule(rl){6-6}
     \cmidrule(rl){8-8}
     \cmidrule(rl){9-9}
     \cmidrule(rl){10-10}

     Разом & 4 & 10 & & 12 & 10 & & 30 & 6 & 18 &\\
  \end{tabular}

  \end{center}
\end{table}
\footnotetextZ{В німецькому тексті тут стоїть «12\%, 24\%, 36\%». Очевидна помилка. \emph{Прим. Ред.}} % текст примітки прямо під заголовком

Тепер ця таблиця перетворюється на:
\begin{table}[h]
  \begin{center}
    \emph{Таблиця ІI}
    \footnotesize

  \begin{tabular}{c c c c c c c c c c c}
    \toprule
      \multirowcell{2}{\makecell{Рід \\землі}} &
      \multirowcell{2}{\rotatebox[origin=c]{90}{Акри}} &
      Капітал &
      \rotatebox[origin=c]{90}{Зиск} &
      \rotatebox[origin=c]{90}{\makecell{Ціна про- \\ дукції}} &
      \multirowcell{2}{\rotatebox[origin=c]{90}{\makecell{Продукт \\ в кварт.}}} &
      \rotatebox[origin=c]{90}{\makecell{Продажна \\ ціна}} &
      \rotatebox[origin=c]{90}{Здобуток} &
      \multicolumn{2}{c}{Рента} &
      \multirowcell{2}{\rotatebox[origin=c]{90}{\makecell{Норма \\ надзиску}}} \\

      \cmidrule(rl){3-3}
      \cmidrule(rl){4-4}
      \cmidrule(rl){5-5}
      \cmidrule(rl){7-7}
      \cmidrule(rl){8-8}
      \cmidrule(rl){9-10}

       &  &  ф. ст. & ф. ст. & ф. ст. & & ф. ст. & ф. ст. & Кварт. & ф. ст. &  \\
      \midrule

      A & 1 & 2\sfrac{1}{2} \dplus{} 2\sfrac{1}{2} \deq{} 5 & 1 & 6 & \phantom{0}2 & 3 & \phantom{0}6 & \phantom{0}0 & \phantom{0}0 & \phantom{00}0\phantom{\%}\\
      B & 1 & 2\sfrac{1}{2} \dplus{} 2\sfrac{1}{2} \deq{} 5 & 1 & 6 & \phantom{0}4 & 3 & 12           & \phantom{0}2 & \phantom{0}6 & 120\% \\ % ця мітка у заголовку \\
      C & 1 & 2\sfrac{1}{2} \dplus{} 2\sfrac{1}{2} \deq{} 5 & 1 & 6 & \phantom{0}6 & 3 & 18           & \phantom{0}4 & 12 & 240\%\\
      D & 1 & 2\sfrac{1}{2} \dplus{} 2\sfrac{1}{2} \deq{} 5 & 1 & 6 & \phantom{0}8 & 3 & 25           & \phantom{0}6 & 18 & 360\%\\
     \cmidrule(rl){1-1}
     \cmidrule(rl){2-2}
     \cmidrule(rl){3-3}
     \cmidrule(rl){6-6}
     \cmidrule(rl){8-8}
     \cmidrule(rl){9-9}
     \cmidrule(rl){10-10}

     Разом & 4 & \phantom{2\sfrac{1}{2} \dplus{} 2\sfrac{1}{2} \deq{}}20 & & & 20 & & 60 & 12 & 36 &\\
  \end{tabular}

  \end{center}
\end{table}

Тут немає потреби в тому, щоб капітал вкладати у кожний з типів землі
в подвоєному розмірі, як це є в таблиці. Закон лишається той самий, скоро
тільки на якийсь один або декілька родів землі, що дають ренту, вжито додатковий капітал, хоч би в
якому розмірі. Треба лише, щоб продукція на землях
кожного роду збільшувалася в тому самому відношенні, в якому збільшується
капітал. Рента підвищується тут виключно в наслідок збільшення вкладеного
в землю капіталу і відповідно до цього збільшення капіталу. Це збільшення
продукту і ренти, в наслідок збільшення вкладеного капіталу і пропорційно
йому, є,  щодо кількости продукту і ренти, цілком таке саме, як у тому випадку,
коли оброблювана площа рівних за якістю дільниць землі, що дають ренту,
збільшилася б, оброблючись з такою самою витратою капіталу, з якою давніш оброблялись земельні
дільниці тієї самої якости. В випадку, поданому в таблиці II, наприклад, наслідок був би той самий,
коли б додатковий капітал,
в 2\sfrac{1}{2}\pound{ ф. стерл.} на акр було вкладено в другі акри земель $B$, $C$ і $D$.


\index{iii2}{0149}  %% посилання на сторінку оригінального видання
Цей випадок не припускає далі жодного продуктивнішого застосування
капіталу, а лише застосування більшого капіталу до тієї самої площі і з тими
самими наслідками, як і до того часу.

Усі відносні величини тут лишаються ті самі. Звичайно, коли розглядати
не відносні ріжниці, а суто аритметичні, то диференційна рента з різних земель може
змінитися. Припустімо, наприклад, що додатковий капітал вкладено лише
в $В$ і $D$. Тоді ріжниця між $D$ і $А$ \deq{} 7 квартерам, давніш вона \deq{} 3; ріжниця
між $В$ і $А$ \deq{} 3 кварт., давніш вона \deq{} 1; ріжниця між $C$ і В $\deq{} - 1$, давніш
вона $= \dplus{} 1$ і~\abbr{т. ін.} Але ця аритметична ріжниця, вирішальна щодо диференційної
ренти І, оскільки в ній виражається ріжниця в продуктивності за однакового
розміру вкладеного капіталу, тут цілком не має ваги, бо вона є лише
наслідок того, чи вкладено, чи ні різні додаткові капітали, за незмінної ріжниці
для кожної рівної частини капіталу на ріжних дільницях.

IIІ.~Додаткові капітали дають надмірний продукт і створюють тому надзиски,
але при понижуваній нормі, не пропорційно їхньому збільшенню.

\begin{table}[H]
  \centering
  \caption*{Таблиця ІІІ}
  \footnotesize

  \setlength{\tabcolsep}{4pt}
  \settowidth\rotheadsize{\theadfont Продажна}
  \begin{tabular}{l c r c c r c c c c c}
    \toprule
      \thead[tl]{Рід\\землі} &
      &
      \thead[t]{Капітал} &
      \rothead{Зиск} &
      \rothead{Ціна\\продукції} &
      \thead[t]{Продукт} &
      \rothead{Продажна\\ціна} &
      \rothead{Здобуток} &
      \multicolumn{2}{c}{Рента} &
      \rothead{Норма\\надзиску} \\

      \cmidrule(rl){2-11}

       & акри  & \poundsign{} & \poundsign{} & \poundsign{} & кв. & \poundsign{} & \poundsign{} & кв. & \poundsign{}  & \% \\
      \midrule

      A & 1 & 2\hang{l}{\tbfrac{1}{2}} & \phantom{0}\hang{l}{\tbfrac{1}{2}} & \phantom{0}3 & \phantom{2 \dplus{} 1\tbfrac{1}{2} \deq{}} 1\phantom{\tbfrac{1}{2}} & 3 & \phantom{0}3\phantom{\tbfrac{1}{2}} &\phantom{0} 0\phantom{\tbfrac{1}{2}} & \phantom{0}0\phantom{\tbfrac{1}{2}} & \pZ{}\pZ{}0 \\
      B & 1 & 2\tbfrac{1}{2} \dplus{} 2\tbfrac{1}{2} \deq{} 5 & 1 & \phantom{0}6 & 2 \dplus{} 1\tbfrac{1}{2} \deq{} 3\tbfrac{1}{2}           & 3           & 10\tbfrac{1}{2}                     & \phantom{0}1\tbfrac{1}{2}           & \phantom{0}4\tbfrac{1}{2}           & \pZ{}90 \\
      C & 1 & 2\tbfrac{1}{2} \dplus{} 2\tbfrac{1}{2} \deq{} 5 & 1 & \phantom{0}6 & 3 \dplus{} 2\phantom{\tbfrac{1}{2}} \deq{} 5\phantom{\tbfrac{1}{2}} & 3 & 15\phantom{\tbfrac{1}{2}}           & \phantom{0}3\phantom{\tbfrac{1}{2}} & \phantom{0}9\phantom{\tbfrac{1}{2}} & 180\\
      D & 1 & 2\tbfrac{1}{2} \dplus{} 2\tbfrac{1}{2} \deq{} 5 & 1 & \phantom{0}6 & 4 \dplus{} 3\tbfrac{1}{2} \deq{} 7\tbfrac{1}{2}           & 3           & 22\tbfrac{1}{2}                     & \phantom{0}5\tbfrac{1}{2}           & 16\tbfrac{1}{2}                     & 330\\
     \midrule

     Разом &  & 17\hang{l}{\tbfrac{1}{2}} & 3\hang{l}{\tbfrac{1}{2}} & 21 & 17\pF{} & & 51\phantom{\tbfrac{1}{2}}  & 10\pF{} & 30\pF{} &\\
  \end{tabular}
  \setlength{\tabcolsep}{\tabcolsepdef}
\end{table}

\looseness=-1
\noindent{}При цьому третьому припущені знов таки байдуже, чи повторні додаткові
капітали вкладаються рівномірно або нерівномірно на землі різних родів або ні:
в однакових чи неоднакових відношеннях відбувається зменшення продукції
надзиску; чи всі додаткові капітали вкладаються в той самий сорт землі, що дає
ренту, чи розподіляються вони рівномірно або нерівномірно, між землями
різної якости, що дають ренту. Всі ці обставини байдужі для закону, що його тут
розвиваємо. Єдине наше припущення є в тому, що додатковий капітал,
вкладений в будь-який сорт землі, що дає ренту, дає надзиск, але в зменшуваній
пропорції проти розміру збільшення капіталу. Межі цього зменшення
коливаються в прикладах вищенаведеної таблиці, між 4 квартерами \deq{} 12\pound{ ф. стерл.},
продуктом першого капіталовкладення на найкращій землі $В$ і 1 квартером
\deq{} 3\pound{ ф. стерл.}, продуктом такого самого вкладення капіталу на найгіршій
землі $А$. Продукт з найкращої землі при витраті капіталу і становить максимальну
межу, а продукт з найгіршої землі $А$, що не дає ні ренти, ні надзиску,
становить, за однакового вкладення капіталу, мінімальну межу продукту,
який дають послідовні вкладення капіталу на будь-якого роду землях, що дають надзиск за зменшуваної
продуктивности послідовних вкладень капіталу. Як
припущення ІІ відповідає тому, що нові однакові якістю дільниці землі кращих
родів приєднується до оброблюваної площі, так що кількість якогось роду обробленої
землі збільшується, так припущення ІІІ відповідає тому, що оброблюються
\parbreak{}  %% абзац продовжується на наступній сторінці


\index{iii1}{0150}  %% посилання на сторінку оригінального видання
Збільшення норми зиску завжди походить від того, що додаткова
вартість відносно або абсолютно збільшується порівняно
з витратами її виробництва, тобто порівняно з сукупним авансованим
капіталом, інакше кажучи, від того, що ріжниця між
нормою зиску і нормою додаткової вартості зменшується.

Коливання в нормі зиску, незалежно від зміни в органічних
складових частинах капіталу чи від абсолютної величини капіталу,
можливі через те, що вартість авансованого капіталу,
в якій би формі — основній чи обіговій — він не існував,
підвищується або падає в наслідок незалежного від уже наявного
капіталу збільшення чи зменшення робочого часу, потрібного
для його репродукції. Вартість всякого товару — отже
й тих товарів, з яких складається капітал, — визначається не
тим необхідним робочим часом, який міститься в ньому самому,
а тим \emph{суспільно}-необхідним робочим часом, який потрібен для
його репродукції. Ця репродукція може відбутися при обтяжливіших
або полегшених обставинах, відмінних від умов первісного
виробництва. Якщо при змінених обставинах потрібно,
загалом кажучи, вдвоє більше або, навпаки, вдвоє менше часу,
щоб репродукувати той самий речовий капітал, то при незміненій
вартості грошей капітал, який раніше був вартий 100 фунтів
стерлінгів, тепер буде вартий 200 фунтів стерлінгів, відповідно
50 фунтів стерлінгів. Якби це підвищення вартості або
зниження вартості зачіпало всі частини капіталу в однаковій
мірі, то й зиск відповідно до цього виразився б у подвійній
або вдвоє меншій грошовій сумі. Якщо ж воно включає і зміну
в органічному складі капіталу, якщо воно підвищує або знижує
відношення змінної частини капіталу до сталої, то, при інших
однакових умовах, норма зиску зростатиме при відносному
зростанні і падатиме при відносному зменшенні змінного капіталу.
Якщо ж підвищується або падає тільки грошова вартість
(в наслідок зміни вартості грошей) авансованого капіталу, то
в тому самому відношенні підвищується або падає грошовий вираз
додаткової вартості. Норма зиску лишається незмінною.

\parcont{}  %% абзац починається на попередній сторінці
\index{i}{0151}  %% посилання на сторінку оригінального видання
зужитковує удвоє більше матеріялу, який має удвоє більшу
вартість, і зужитковує удвоє більше машин, що мають удвоє
більшу вартість, отже, зберігає в продукті двох тижнів удвоє
більше вартости, ніж у продукті одного тижня. За даних незмінних
умов продукції робітник зберігає то більше вартости, що
більше вартости він додає; але він зберігає більше вартости не
тому, що додає більше вартости, а тому, що додає її за незмінних
і незалежних від його власної праці умов.

Звичайно, в деякому відносному розумінні можна сказати,
що робітник завжди зберігає старі вартості в тій самій пропорції, в
якій він додає нову вартість. Чи піднесеться вартість бавовни з
1\shil{ шилінґа} на 2\shil{ шилінґи}, чи спаде на 6\pens{ пенсів}, робітник завжди
зберігає в продукті однієї години лише удвоє меншу вартість
бавовни, ніж у продукті двох годин, хоч би й як змінялася ця
вартість. Далі, коли змінюється продуктивність його власної праці,
коли вона підноситься або падає, то він, приміром, за одну робочу
годину випряде більше або менше бавовни, ніж раніше, і відповідно
до цього збереже більшу або меншу вартість бавовни у
продукті однієї робочої години. Але за всім тим він за дві робочі
години збереже удвоє більше вартости, ніж за одну робочу годину.

Вартість, залишаючи осторонь її суто символічний вираз у
знаках вартости, існує лише в якійсь споживній вартості, в якійсь
речі. (Сама людина, розглядувана просто лише як буття робочої
сили, є предмет природи, річ, хоч і жива, самосвідома річ, а сама
праця є речове виявлення цієї сили). Тому, коли гине споживна
вартість, то гине й вартість. Засоби ж продукції не втрачають
своєї вартости одночасно із своєю споживною вартістю, бо в наслідок
процесу праці вони втрачають первісну форму своєї споживної
вартости в дійсності лише на те, щоб у продукті набрати форми
іншої споживної вартости. Але, хоч і як важливо для вартости
існувати в якійсь споживній вартості, для неї, як це доводить
метаморфоза товарів, байдуже, в якій споживній вартості вона
існує. Звідси випливає, що в процесі праці вартість переходить
із засобу продукції на продукт лише тією мірою, якою засіб продукції
разом із своєю самостійною споживною вартістю втрачає
й свою мінову вартість. Він віддає продуктові лише ту вартість,
яку він втрачає як засіб продукції. Але з цього погляду зрізними
речовими факторами процесу праці справа стоїть неоднаково.

Вугілля, що ним опалюють машину, зникає безслідно, так
само мастиво, що ним мастять вісь колеса, і т. ін. Фарби й інші
допоміжні матеріяли зникають, але виявляються у властивостях
продукту. Сировинний матеріял становить субстанцію продукту,
але змінює свою форму. Отже, сировинний матеріял і допоміжні
матеріяли втрачають самостійну форму, в якій вони увійшли до
процесу праці як споживні вартості. Інша справа з власне засобами
праці. Інструмент, машина, фабричний будинок, бочка й
т. ін. служать у процесі праці лише доти, доки зберігають вони
свою первісну форму, доки й завтра можуть входити в процес
праці в тій самій формі, що й учора. Як за свого життя, тобто
\parbreak{}  %% абзац продовжується на наступній сторінці

\parcont{}  %% абзац починається на попередній сторінці
\index{ii}{0152}  %% посилання на сторінку оригінального видання
можуть належати до фонду споживання, отже, взагалі не належать до
суспільного капіталу, хоч і становлять елемент суспільного багатства, що
з нього капітал є лише частина. Продуцент цих речей, кажучи словами
Сміса, одержує зиск, продаючи їх. Отже, обіговий капітал! Людина, що
користується з них, їхній остаточний покупець, може використати їх, лише
вживаючи їх в процесі продукції. Отже, основний капітал!

Титули власности, напр., на залізницю, можуть щодня переходити з
рук в руки, і власники їх, продаючи ці титули, можуть одержувати зиск
навіть за кордоном; отже, титули власности на залізницю, протилежно
самій залізниці можна вивозити. А проте, сами ці речі мусять саме в тій
країні, де вони льокалізовані, або лежати без діла, або функціонувати як
основна складова частина продуктивного капіталу. Так само фабрикант
$А$ може одержахи зиск, продавши свою фабрику фабрикантові $В$, що однак
не перешкоджає фабриці й тепер, як раніше, функціонувати як основний
капітал.

Отже, якщо фіксовані в певному місці, невідокремлювані від ґрунту
засоби праці доконечно мусять — згідно з їхнім призначенням — функціонувати
як основний капітал в самій країні, хоча б для їхнього продуцента
вони функціонузали як товаровий капітал, не являючи елементів його
основного капіталу (останній складається для нього з засобів праці, що
вони потрібні на будування будівель, залізниць тощо), то відси ні в
якому разі не випливає зворотний висновок, що основний капітал мусить
складатись з нерухомих речей. Корабель або льокомотив працюють
лише рухаючись; і все ж вони функціонують — не для їхнього продуцента,
а для їхнього споживача — як основний капітал. З другого боку,
речі, що якнайочевидніше фіксовані в продукційному процесі, у ньому
живуть та вмирають і, одного разу ввійшовши в нього, вже ніколи його
не облишають, є поточні складові частини продуктивного капіталу. Напр.,
вугілля, зуживане машиною в процесі продукції, газ, що ним освітлюється
фабричну будову тощо. Вони поточні не тому, що вони разом з
продуктом матеріяльно облишають процес продукції і циркулюють як
товар, а тому, що їхня вартість цілком ввіходить у вартість товару, що
його продукувати вони допомагають, і, значить, її цілком треба покрити
через продаж товару.

В щойно цитованому місці з А. Сміса треба зазначити ще таке речення:
„Обіговий капітал, що дає\dots{} утримання робітникам, які виробляють
їх“ (мащини і т. інше).

У фізіократів частина капіталу, авансована на заробітну плату, правильно
фігурує під назвою avances annuelles протилежно до avances primitives.
З другого боку, в них виступає як складова частина продуктивного
капіталу, вживаного фармером, не сама робоча сила, а засоби існування,
що їх видається сільсько-господарським робітникам („утримання
робітників“, як каже Сміс). Це точно відповідає їхній специфічній доктрині.
А саме — у них частину вартости, долучувану працею до продукту
(цілком так само, як і ту частину вартости, що її долучають до продукту
сировинні матеріяли, знаряддя праці та інші речові складові частини
\parbreak{}  %% абзац продовжується на наступній сторінці

\parcont{}  %% абзац починається на попередній сторінці
\index{iii2}{0153}  %% посилання на сторінку оригінального видання
капітал. Це включає, за незмінних ріжниць між родами землі, зріст надпродукту,
пропорційний зростові вкладеного капіталу. Отже, випадок цей виключає всяку
додаткову витрату на землю $А$, яка вплинула б на диференційну ренту. На цій землі
норма надзиску \deq{} 0; отже, вона лишається \deq{} 0, бо ми припустили, що продуктивна
сила додаткового капіталу, а тому і норма надзиску лишаються сталими.

Але реґуляційна ціна продукції може за цих умов лише знизитися, бо замість
ціни продукції з $А$ реґуляційною стає ціна продукції ближчої якістю землі
$В$ або взагалі з будь-якої землі, кращої, ніж $А$; отже, коли б ціна продукції
на землі $C$ зробилась регуляційною, то капітал був би вилучений з $А$,
або навіть з $А$ і $В$, і таким чином всі землі, гірші, ніж $C$, випали б з конкуренції
земель, на яких сіють пшеницю. Умова, потрібна для цього за даних
припущень, є в тому, щоб надпродукт з додаткових капіталовкладень задовольняв
потребам, і щоб тому продукція на гіршій землі $А$ тощо зробилася
зайвою для поновлення подання.

Отже, візьмімо, наприклад, таблицю II, але змінимо її так, щоб замість 20
квартерів, потребу задовольняли 18 квартерів. Земля $А$ відпала б; $D$, а
разом з нею ціна продукції в 30\shil{ шил.} за кв. стала б реґуляційною. Диференційна
рента набуває тоді такої форми:

\begin{table}[H]
  \begin{center}
    \emph{Таблиця ІV}
    \footnotesize

  \begin{tabular}{c c c c c c c c c c c}
    \toprule
      \multirowcell{2}{\makecell{Рід \\землі}} &
      \multirowcell{2}{\rotatebox[origin=c]{90}{Акри}} &
      \rotatebox[origin=c]{90}{Капітал} &
      \rotatebox[origin=c]{90}{Зиск} &
      \rotatebox[origin=c]{90}{\makecell{Ціна про- \\ дукції}} &
      \multirowcell{2}{\rotatebox[origin=c]{90}{\makecell{Продукт \\ в кварт.}}} &
      \rotatebox[origin=c]{90}{\makecell{Продажна \\ ціна}} &
      \rotatebox[origin=c]{90}{Здобуток} &
      \multicolumn{2}{c}{Рента} &
      \multirowcell{2}{\makecell{Норма \\надзиску}} \\

      \cmidrule(rl){3-3}
      \cmidrule(rl){4-4}
      \cmidrule(rl){5-5}
      \cmidrule(rl){7-7}
      \cmidrule(rl){8-8}
      \cmidrule(rl){9-10}

       &  &  ф. ст. & ф. ст. & ф. ст. & & ф. ст. & ф. ст. & Кварт. & ф. ст. &  \\
      \midrule

      B & 1 &  \phantom{0}5 & 1 & \phantom{0}6 & \phantom{0}4 & 1\sfrac{1}{2} & \phantom{0}6 & 0 & \phantom{0}0 & \phantom{00}0\% \\ % ця мітка у заголовку \\
      C & 1 &  \phantom{0}5 & 1 & \phantom{0}6 & \phantom{0}6 & 1\sfrac{1}{2} & \phantom{0}9 & 2 & \phantom{0}3 & \phantom{0}60\%\\
      D & 1 &  \phantom{0}5 & 1 & \phantom{0}6 & \phantom{0}8 & 1\sfrac{1}{2} & 12           & 4 & \phantom{0}6 & 120\%\\
     \cmidrule(rl){1-1}
     \cmidrule(rl){2-2}
     \cmidrule(rl){3-3}
     \cmidrule(rl){4-4}
     \cmidrule(rl){5-5}
     \cmidrule(rl){6-6}
     \cmidrule(rl){8-8}
     \cmidrule(rl){9-9}
     \cmidrule(rl){10-10}

     Разом & 3 & 15 & 3 & 18 & 18 & & 27 & 6 & 9 &\\
  \end{tabular}

  \end{center}
\end{table}

Отже, вся рента порівняно з таблицею II знизилась би з 36\pound{ ф. стерл.}
до 9, а в збіжжі з 12 кварт, до 6; вся продукція знизилася б лише на 2
квартери, з 20 до 18. Норма надзиску, обчислена у відношенні до капіталу,
знизилася б наполовину, з 180 до 90\%\footnote*{
До 60\%, тобто знизилася б втроє, бо в таблиці II вона \deq{} \frac{36}{20} × 100 \deq{} 180\%, а в тaблиці
IV вона $= \frac{9}{15} × 100 \deq{} 60\%$. \emph{Прим. Ред.}
}. Отже, пониженню ціни продукції
тут відповідає зменшення збіжжевої і грошової ренти.

Порівняно з таблицею І, відбувається лише зменшення грошової ренти;
збіжжева рента в обох випадках дорівнює 6 квартерам; але тільки в одному
випадку вона \deq{} 18\pound{ ф. стерл.}, а в другому \deq{} 9\pound{ ф. стерл}. Для земель $C$ і $D$
збіжжева рента проти таблиці І лишилась та сама\footnote*{
Те, що сказано тут, правильне лише для землі $C$, але неправильне для землі $D$, бо в табл. І земля
$D$ дає 3 кв. ренти, а в табл. IV земля $D$ дає 4 кв. ренти. Те, що тут сказано, було б правильне, коли
взяти загальну ренту з земель $B$, $C$ і $D$. \emph{Прим. Ред.}
}. В дійсності, в наслідок
того, що додаткова продукція, досягнена з допомогою додаткового капіталу рівної
продуктивности, витиснула з ринку продукт $А$ і разом з тим усунула землю $А$
з числа конкурентних аґентів продукції — в наслідок цього в дійсності створилася
нова диференційна рента І, в якій краща земля $В$ від грає ту саму ролю,
яку давніш відігравала гірша земля $А$. В наслідок цього, з одного боку, відпадає
рента з $В$; з другого боку, згідно з припущенням, вкладення додаткового
\parbreak{}  %% абзац продовжується на наступній сторінці

\parcont{}  %% абзац починається на попередній сторінці
\index{iii2}{0154}  %% посилання на сторінку оригінального видання
капіталу нічого не змінило в ріжницях між $В$, $C$ і $D$. Тому частина продукту,
що перетворюється на ренту, зменшується.

\looseness=1
Коли б вищенаведений наслідок — задоволення попиту при виключенні
землі $А$ — був спричинений тим, що більше, ніж подвійна кількість капіталу
вкладалася б в землю $C$ або $D$ або в обидві разом, то справа набула б іншого
вигляду. Наприклад, коли б третє вкладення капіталу було зроблено на $C$:

\begin{table}[H]
  \centering
  \caption*{Таблиця ІVa}
  \footnotesize

  \settowidth\rotheadsize{\theadfont Продажна}
  \begin{tabular}{l c r c c c c c c c c}
    \toprule
      \thead[tl]{Рід\\землі} &
      &
      \rothead{Капітал} &
      \rothead{Зиск} &
      \rothead{Ціна\\продукції} &
      \rothead{Продукт} & % \\ в кварт.}}} \\ в кварт.}}}
      \rothead{Продажна\\ціна} &
      \rothead{Здобуток} &
      \multicolumn{2}{c}{Рента} &
      \rothead{Норма\\надзиску} \\

      \cmidrule(rl){2-11}

       & акри  & \makecell{\poundsign{}} & \poundsign{} & \poundsign{} & кв. & \poundsign{} & \poundsign{} & кв. & \poundsign{}  & \% \\
      \midrule

      B & 1 &  \phantom{0}5\phantom{\tbfrac{1}{2}} & 1\phantom{\tbfrac{1}{2}} & \phantom{0}6 & \phantom{0}4 & 1\tbfrac{1}{2} & \phantom{0}6\phantom{\tbfrac{1}{2}} & 0 & \phantom{0}0\phantom{\tbfrac{1}{2}}   & \phantom{00}0 \\
      C & 1 &  \phantom{0}7\tbfrac{1}{2}           & 1\tbfrac{1}{2}           & \phantom{0}9 & \phantom{0}9 & 1\tbfrac{1}{2} & 13\tbfrac{1}{2}                     & 3 & \phantom{0}4\tbfrac{1}{2}            & \phantom{0}60\\
      D & 1 &  \phantom{0}5\phantom{\tbfrac{1}{2}} & 1\phantom{\tbfrac{1}{2}} & \phantom{0}6 & \phantom{0}8 & 1\tbfrac{1}{2} & 12\phantom{\tbfrac{1}{2}}           & 4 & \phantom{0}6\phantom{\tbfrac{1}{2}}  & 120\\
     \midrule

     Разом & 3 & 17\tbfrac{1}{2} & 3\tbfrac{1}{2} & 21 & 21 & & 30\tbfrac{1}{2} & 7 & 10\tbfrac{1}{2} &\\
  \end{tabular}
\end{table}

\noindent{}Продукт з $C$ збільшився тут проти таблиці ІV з 6 кватерів до 9, надпродукт
— з 2 квартерів до 3, грошова рента зросла з 3\pound{ ф. стерл.} до 4\sfrac{1}{2}\pound{ ф.
стерл}. Проти таблиці II, де грошова рента була 12\pound{ ф. стерл.} і таблиці І,
де вона була 6\pound{ ф. стерл.}, вона навпаки зменшилась. Загальна сума ренти визначена
в збіжжі \deq{} 7 квартерів, зменшилась проти таблиці II (12 квартерів),
збільшилась проти таблиці І (6 квартерів); визначена в грошах (10\sfrac{1}{2}\pound{ ф. стерл.})
зменшилася проти обох (18\pound{ ф. стерл.} і 36\pound{ ф. стерл.}).

Коли б у землю $В$ було вкладено третій капітал в 2\sfrac{1}{2}\pound{ ф. стерл.}, то хоч
це й змінило б масу продукції, але не зачепило б ренти, бо згідно з припущенням
послідовні вкладення капіталу не вносять жодної ріжниці в землю того
самого роду, а земля $В$ ренти не дає. Навпаки, коли ми припустимо, що третій капітал вкладається в землю $D$,
замість $C$, то ми матимемо:

\begin{table}[H]
  \centering
  \caption*{Таблиця ІVb}
  \footnotesize

  \settowidth\rotheadsize{\theadfont Продажна}
  \begin{tabular}{l c r c c c c c c c c}
    \toprule
      \thead[tl]{Рід\\землі} &
      &
      \rothead{Капітал} &
      \rothead{Зиск} &
      \rothead{Ціна\\продукції} &
      \rothead{Продукт} & % \\ в кварт.}}} \\ в кварт.}}}
      \rothead{Продажна\\ціна} &
      \rothead{Здобуток} &
      \multicolumn{2}{c}{Рента} &
      \rothead{Норма\\надзиску} \\

      \cmidrule(rl){2-11}

       & акри  & \makecell{\poundsign{}} & \poundsign{} & \poundsign{} & кв. & \poundsign{} & \poundsign{} & кв. & \poundsign{}  & \% \\
      \midrule

      B & 1 &  \phantom{0}5\phantom{\tbfrac{1}{2}} & 1\phantom{\tbfrac{1}{2}} & \phantom{0}6 & \phantom{0}4 & 1\tbfrac{1}{2}  & \phantom{0}6 & 0 & \phantom{0}0 & \phantom{00}0 \\
      C & 1 &  \phantom{0}5\phantom{\tbfrac{1}{2}} & 1\phantom{\tbfrac{1}{2}} & \phantom{0}6 & \phantom{0}6 & 1\tbfrac{1}{2}  & \phantom{0}9 & 2 & \phantom{0}3 & \phantom{0}60\\
      D & 1 &  \phantom{0}7\tbfrac{1}{2}           & 1\tbfrac{1}{2}           & \phantom{0}9 & \phantom{0}12 & 1\tbfrac{1}{2} & 18           & 6 & \phantom{0}9 & 120\\
     \midrule

     Разом & 3 & 17\tbfrac{1}{2} & 3\tbfrac{1}{2} & 21 & 22 & & 33 & 8 & 12 &\\
  \end{tabular}
\end{table}

\noindent{}Тут загальна кількість продукту \deq{} 22 кварт., більша ніж удвоє проти
загальної кількости продукту таблиці І, хоч авансований капітал є лише 17\sfrac{1}{2}\pound{ ф. стерл.} проти 10\pound{ ф. стерл.}, отже, не подвоївся. Далі, загальна кількість продукту
на 2 квартерн більша, ніж загальна кількість продукту у таблиці II, хоч
в останній авансований капітал більший, а саме 20\pound{ ф. стерл.}.


\index{iii2}{0155}  %% посилання на сторінку оригінального видання
На землі $D$ збіжжева рента проти таблиці I зросла з 3\footnote*{
В німецькому тексті стоїть: «з 2 квартерів». Явна помилка, як це можна бачити з таблиці І \emph{Прим. Ред.}
} квартерів до 6
тимчасом як грошова рента лишилася, як і давніш, 9\pound{ ф. стерл}. Проти таблиці II
збіжжева рента з $D$ лишилася колишня, 6 квартерів, але грошова рента знизилась
з 18\pound{ ф. стер.} до 9\pound{ ф. стерл}.

Коли розглядати загальні суми ренти, то збіжева рента таблиці IVb = 8
квартерам, більша, ніж рента в таблиці І, що дорівнює 6 квартерам, і більша,
ніж рента в таблиці IVа, що дорівнює 7 квартерам; і навпаки, вона менша, ніж
рента в таблиці II = 12 кварт. Грошова рента в таблиці IVb = 12\pound{ ф. стерл.},
більша, ніж грошова рента в таблиці ІVа = 10\sfrac{1}{2}\pound{ ф. стерл.}, і менша від грошової
ренти таблиці І = 18\pound{ ф. стерл.} і таблиці II = 36\pound{ ф. стерл}.

Щоб по відпаданні ренти з $В$ в умовах таблиці IVb загальна сума ренти
дорівнювала такій у таблиці I, ми мусимо одержати ще на 6\pound{ ф. стерл.}
надпродукту, тобто 4 квартери по 1\sfrac{1}{2}\pound{ ф. стерл.}, що є новою ціною продукції.
Тоді ми знову маємо загальну суму ренти в 18\pound{ ф. стерл.}, як у таблиці І.~Величина
потрібного на це додаткового капіталу буде різна залежно від того, чи
вкладемо ми його в $C$ або $D$, чи розподілимо його між обома родами землі.

На $C$ капітал в 5\pound{ ф. стерл.} дає 2 квартери надпродукту, отже, 10\pound{ ф. ст.}
додаткового капіталу дадуть 4 квартери додаткового надпродукту. На $D$ було б
досить додаткової витрати в 5\pound{ ф. стерл.}, щоб випродукувати 4 квартери додаткової
збіжжевої ренти при зробленому тут основному припущенні, що продуктивність
додаткових капіталовкладень лишається та сама. Тому здобуваємо
такі наслідки.

\begin{table}[h]
  \begin{center}
    \emph{Таблиця ІVc}
    \footnotesize

  \begin{tabular}{c c c c c c c c c c c}
    \toprule
      \multirowcell{2}{\makecell{Рід \\землі}} &
      \multirowcell{2}{\rotatebox[origin=c]{90}{Акри}} &
      \rotatebox[origin=c]{90}{Капітал} &
      \rotatebox[origin=c]{90}{Зиск} &
      \rotatebox[origin=c]{90}{\makecell{Ціна про- \\ дукції}} &
      \multirowcell{2}{\rotatebox[origin=c]{90}{\makecell{Продукт \\ в кварт.}}} &
      \rotatebox[origin=c]{90}{\makecell{Продажна \\ ціна}} &
      \rotatebox[origin=c]{90}{Здобуток} &
      \multicolumn{2}{c}{Рента} &
      \multirowcell{2}{\makecell{Норма \\надзиску}} \\

      \cmidrule(rl){3-3}
      \cmidrule(rl){4-4}
      \cmidrule(rl){5-5}
      \cmidrule(rl){7-7}
      \cmidrule(rl){8-8}
      \cmidrule(rl){9-10}

       &  &  ф. ст. & ф. ст. & ф. ст. & & ф. ст. & ф. ст. & Кварт. & ф. ст. &  \\
      \midrule

      B & 1 &  \phantom{0}5\phantom{\sfrac{1}{2}} & 1\phantom{\sfrac{1}{2}} & \phantom{0}6 & \phantom{0}4 & 1\sfrac{1}{2} & \phantom{0}6 & 0 & \phantom{0}0 & \phantom{00}0\% \\
      C & 1 & 15\phantom{\sfrac{1}{2}}            & 3\phantom{\sfrac{1}{2}} & 18           & 18           & 1\sfrac{1}{2} & 27           & 6 & \phantom{0}9 & \phantom{0}60\%\\
      D & 1 &  \phantom{0}7\sfrac{1}{2}           & 1\sfrac{1}{2}           & \phantom{0}9 & 12           & 1\sfrac{1}{2} & 18           & 6 & \phantom{0}9 & 120\%\\
     \cmidrule(rl){1-1}
     \cmidrule(rl){2-2}
     \cmidrule(rl){3-3}
     \cmidrule(rl){4-4}
     \cmidrule(rl){5-5}
     \cmidrule(rl){6-6}
     \cmidrule(rl){8-8}
     \cmidrule(rl){9-9}
     \cmidrule(rl){10-10}

     Разом & 3 & 27\sfrac{1}{2} & 5\sfrac{1}{2} & 33 & 34 & & 51 & 12 & 18 &\\
  \end{tabular}

  \end{center}
\end{table}

\begin{table}[h]
  \begin{center}
    \emph{Таблиця ІVd}
    \footnotesize

  \begin{tabular}{c c c c c c c c c c c}
    \toprule
      \multirowcell{2}{\makecell{Рід \\землі}} &
      \multirowcell{2}{\rotatebox[origin=c]{90}{Акри}} &
      \rotatebox[origin=c]{90}{Капітал} &
      \rotatebox[origin=c]{90}{Зиск} &
      \rotatebox[origin=c]{90}{\makecell{Ціна про- \\ дукції}} &
      \multirowcell{2}{\rotatebox[origin=c]{90}{\makecell{Продукт \\ в кварт.}}} &
      \rotatebox[origin=c]{90}{\makecell{Продажна \\ ціна}} &
      \rotatebox[origin=c]{90}{Здобуток} &
      \multicolumn{2}{c}{Рента} &
      \multirowcell{2}{\makecell{Норма \\надзиску}} \\

      \cmidrule(rl){3-3}
      \cmidrule(rl){4-4}
      \cmidrule(rl){5-5}
      \cmidrule(rl){7-7}
      \cmidrule(rl){8-8}
      \cmidrule(rl){9-10}

       &  &  ф. ст. & ф. ст. & ф. ст. & & ф. ст. & ф. ст. & Кварт. & ф. ст. &  \\
      \midrule

      B & 1 & \phantom{0}5\phantom{\sfrac{1}{2}} & 1\phantom{\sfrac{1}{2}} & \phantom{0}6 & \phantom{0}4 & 1\sfrac{1}{2} & \phantom{0}6 & \phantom{0}0 & \phantom{0}0 & \phantom{00}0\% \\
      C & 1 & \phantom{0}5\phantom{\sfrac{1}{2}} & 1\phantom{\sfrac{1}{2}} & \phantom{0}6 & \phantom{0}6 & 1\sfrac{1}{2} & \phantom{0}9 & \phantom{0}2 & \phantom{0}3 & \phantom{0}60\%\\
      D & 1 & 12\sfrac{1}{2}                     & 2\sfrac{1}{2}           & 15           & 20           & 1\sfrac{1}{2} & 30           & 10           & 15           & 120\%\\
     \cmidrule(rl){1-1}
     \cmidrule(rl){2-2}
     \cmidrule(rl){3-3}
     \cmidrule(rl){4-4}
     \cmidrule(rl){5-5}
     \cmidrule(rl){6-6}
     \cmidrule(rl){8-8}
     \cmidrule(rl){9-9}
     \cmidrule(rl){10-10}

     Разом & 3 & 22\sfrac{1}{2} & 4\sfrac{1}{2} & 27 & 30 & & 45 & 12 & 18 &\\
  \end{tabular}

  \end{center}
\end{table}



\index{iii2}{0156}  %% посилання на сторінку оригінального видання
Загальна сума грошової ренти становила б якраз половину того, що було
в таблиці II, де додаткові капітали були вкладені за незмінних цін продукції.

Найважливіше є порівняти вищенаведені таблиці з таблицею І.

Ми бачимо, що з пониженням ціни продукції на половину, з 60 шил. до
30 шил. за квартер, загальна сума грошової ренти залишилась та сама = 18\pound{ ф.
ст.} і відповідно до цього збіжжева рента подвоїлась, саме зросла з 6 кварт. до
12 кварт. Рента з $В$ відпала; з $C$ грошова рента в ІVс збільшилась на половину,
але на половину зменшилась в ІVс; з $D$ вона лишилась та сама = 9\pound{ ф.
стерл.} у таблиці ІVс, і піднеслась з 9\pound{ ф. стерл.} до 15\pound{ ф. стерл.} у таблиції ІVd.
Продукц я піднеслась з 10 квартерів до 34 в ІVс, і до 30 квартер в в IVd;
зиск підвищився з 2\pound{ ф. стерл.} до 5\sfrac{1}{2} в ІVс і до 4\sfrac{1}{2} в IVd. Загальна сума
вкладеного капіталу зросла в одному випадку з 10\pound{ ф. стерл.} до 27\sfrac{1}{2}\pound{ ф. стерл.},
в другому — з 10 до 22\sfrac{1}{2}\pound{ ф. стерл.}; отже, обидва рази більше, ніж удвоє. Норма
ренти, рента, обчислена у відношенні до авансованого капіталу, в усіх таблицях
від IV до IVd для кожного роду землі всюди та сама, що вже було дано тим припущенням,
що норма продуктивности обох послідовних витрат капіталу на землях
усіх родів не змінюється. Проти таблиці І вона, проте, понизилась пересічно
щодо всіх родів землі і для кожного окремого роду землі. В таблиці І вона =
180\% пересічно, в таблиці ІVс вона$ = \frac{18}{27\sfrac{1}{2}} × 100 = 65\sfrac{5}{11}\%$ і
IVd = $\frac{18}{22\sfrac{1}{2}} × 100 = 80\%$. Пересічна грошова рента з акра підвищилась. Її пересічна
величина давніш в таблиці І була 4\sfrac{1}{2}\pound{ ф. стерл.} з акра для всіх 4 акрів,
а тепер у таблицях IVс і d вона дорівнює 6\pound{ ф. стерл.} з акра для 3 акрів.
Її пересічна величина для землі, що дає ренту, була раніш 6\pound{ ф. стерл.}, а тепер
зона дорівнює 9\pound{ ф. стерл.} з акра. Отже, грошова вартість ренти з акра підвищилась
і репрезентує тепер удвоє більше продукту в збіжжі, ніж давніш, але
12 квартерів збіжжевої ренти тепер становлять менше, ніж половину всього продукту
в 34, зглядно 30\footnote*{В німецькому тексті стоїть: усього «продукту в 33, зглядно 27 квартерів» Явна помилка,
як це можна бачити з таблиць ІVс і IVd. \emph{Прим. Ред.}} квартерів, тимчасом як у таблиці І 6 квартерів становлять
\sfrac{3}{5}  усього продукту в 10 квартерів. Отже, хоч рента, коли розглядати
її як відповідну частину всього продукту, а також коли обчислити її у відношенні
до витраченого капіталу, і знизилась, одначе її грошова вартість,
обчислена на акр. збільшилась, а її вартість в продукті, збільшилась ще дужче.
Коли ми візьмемо землю $D$ в таблиці IVd, то ціна продукції тут дорівнює
15\pound{ ф. стерл.}, що з них витрачений капітал = 12\sfrac{1}{2}\pound{ ф. стерл}. Грошова рента = 15\pound{ ф. стер}. У таблиці І на тій самій землі $D$ ціна продукції була 3\pound{ ф. стерл.}, витрачений
капітал = 2\sfrac{1}{2}\pound{ ф. стерл.}, грошова рента = 9\pound{ ф. стерл.}, отже, остання
утроє більша за ціну продукції й майже у чотири рази більша за витрачений
капітал. У таблиці IVd для $D$ грошова рента в 15\pound{ ф. стерл.} якраз дорівнює ціні
продукції і лише на \sfrac{1}{5}  більша за витрачений капітал. А все ж грошова рента
з акра на \sfrac{2}{3}  більша, 15\pound{ ф. стерл.} замість 9\pound{ ф. стерл}. В таблиці І збіжжева
рента в 3 квартери = \sfrac{3}{4}  усього продукту, що становить 4 квартери, в таблиці
IVd вона = 10 квартерам, половині всього продукту (20 квартерів) з акра
землі $D$. Це показує, що грошова і збіжжева рента з акра може зрости, хоч
вона і становить відносно меншу частину всього здобутку і знизилась у відношенні
до авансованого капіталу.

Вартість всього продукту в таблиці І = 30\pound{ ф. стерл.}; рента = 18\pound{ ф.
стерл.} більше від половини цієї вартости. Вартість усього продукту в таблиці
IV = 45\pound{ ф. стерл.}, що з них 18\pound{ ф. стерл.}, менш від половини, становлять
ренту.


\index{iii2}{0157}  %% посилання на сторінку оригінального видання
Причина ж того, що не зважаючи на пониження ціни на 1\sfrac{1}{2}\pound{ ф. стерл.}
за квартер, отже на 50\%, і не зважаючи на зменшення площі конкурентної
землі з 4 до 3 акрів, загальна сума грошової ренти лишається та сама, а збіжжева
рента подвоюється, тимчасом як збіжжева й грошова рента, обчислена на акр, підвищується,
— причина цього в тому, що вироблено більше квартерів надпродукту.
Ціна збіжжя знижується на 50\%, надпродукт зростає на 100\%.
Але для досягнення такого наслідку вся продукція, згідно з нашими умовами,
мусить збільшитись утроє, а капітал, вкладений у кращу землю, мусить більше
ніж подвоїтись. В якому відношенні він мусить збільшуватись, залежить насамперед
від того, як розподіляється додаткові вкладення капіталу між кращими та
найкращими землями, припускаючи завжди, що продуктивність капіталу на
кожній категорії землі зростає пропорційно його величині.

Коли б пониження ціни продукції було менш значне, то потрібно було б
менше додаткового капіталу, щоб випродукувати ту саму грошову ренту. Коли б
подання збіжжя потрібне для того, щоб вилучити $А$ з числа оброблюваних земель,
— а це залежить не тільки від кількости продукту з акра землі $А$, але
також і від того, яку частину всієї оброблюваної земельної площі становить $А$, —
отже, коли б потрібне для цього подання було більше, отже, коли б також
потрібно було і більшої маси додаткового капіталу на кращій, ніж $А$ землі, то,
за інших незмінних відношень грошова і збіжжева ренти зросли б ще більше,
не зважаючи на те, що земля $В$ перестала б давати грошову і збіжжеву ренту.

Коли б капітал, що перестав функціонувати на землі $А$, дорівнював 5\pound{ ф.
стерл.}, то для цього випадку треба було б взяти для порівняння обидві таблиці:
II і ІV$d$. Весь продукт збільшився б з 20 до 30 квартерів. Грошова рента
зменшилася б удвоє, вона дорівнювала б 18\pound{ ф. стерл.} замість 36\pound{ ф. стерл.},
збіжжева рента залишилась би та сама \deq{} 12 квартерів.

Коли б можна було випродукувати на землі $D$ 44 квартери загального
продукту \deq{} 66\pound{ ф. стерл.}, вкладаючи капітал в 27\sfrac{1}{2}\pound{ ф. стерл.}, — що відповідало б
колишньому припущенню для $D$: 4 квартери на 2\sfrac{1}{2}\pound{ ф. стерл.} капіталу, —
то загальна сума\footnote*{
Тут очевидно справа йде про загальну грошову ренту. \Red{Прим. Ред.}
} ренти знову досягла б тієї висоти, яку вона мала в таблиці
II, і таблиця набула б такого вигляду:

\begin{table}[h]
  \begin{center}
  \begin{tabular}{c c c c c}
  \toprule
  \makecell{Рід\\землі}  & \makecell{Капітал\\ф. ст.} & \makecell{Продукт в \\ квартерах} & \makecell{Збіжжева \\ рента \\ в кварт.}& \makecell{Грошова\\рента \\ф. ст.}\\
  \midrule
  B &    \phantom{0}5\phantom{\sfrac{1}{2}} & \phantom{0}4  & \phantom{0}0  & \phantom{0}0\\
  C &    \phantom{0}5\phantom{\sfrac{1}{2}} & \phantom{0}6  & \phantom{0}2  & \phantom{0}3\\
  D &   27\sfrac{1}{2}                      & 44            & 22            & 33\\
  \cmidrule(rl){1-1}
  \cmidrule(rl){2-2}
  \cmidrule(rl){3-3}
  \cmidrule(rl){4-4}
  \cmidrule(rl){5-5}
  Разом. & 37\sfrac{1}{2} &      54  &  24  &  36\\
  \end{tabular}
  \end{center}
\end{table}

Уся продукція була б 54 квартери проти 20 квартерів у таблиці II, грошова рента була
б та сама \deq{} 36\pound{ ф. стерл}. Але весь капітал був би 37\sfrac{1}{2}\pound{ ф. стерл.}, тимчасом як у таблиці II
він був \deq{} 20\pound{ ф. стерл}. Весь авансований капітал майже подвоївся б, тимчасом як продукція майже
потроїлася б; збіжжева рента збільшилася б удвоє, грошова рента залишилася б та сама.
Отже, коли ціна, за незмінної продуктивности, знижується в наслідок приміщення
додаткового грошового капіталу у кращі землі, що дають ренту, отже
в усі землі кращі від $А$, то весь капітал має тенденцію зростати не в такій
самій пропорції, як продукція і збіжжева рента; так що зростання збіжжевої ренти
може урівноважити падіння грошової ренти, яке постає в наслідок пониження ціни.
Той самий закон виявляється і в тому, що авансований капітал мусить бути більший
відповідно до того, як його вживається більше на землі $C$, ніж на $D$, — на землі,
що дає менше ренти, ніж на тій, яка дає більше ренти. Це визначає лише ось що:
\parbreak{}  %% абзац продовжується на наступній сторінці

\parcont{}  %% абзац починається на попередній сторінці
\index{i}{0158}  %% посилання на сторінку оригінального видання
зміниться робочий час, суспільно-потрібний на продукцію товару, — а та сама кількість бавовни,
приміром, за несприятливого врожаю репрезентує більшу кількість праці, ніж за сприятливого,
— то це справляє відбитий вплив на старий товар, що його завжди рахується лише за окремий екземпляр
свого ґатунку,\footnote{«Всі продукти того самого роду становлять, власне, одну масу, ціну якої визначається загалом
незалежно від особливих обставин». («Toutes les productions d’un même gepre ne forment proprement
qu’une masse, dont le prix se détermine en général et sans égard aux circonstances particulières»),
(\emph{Le Trosne}: «De l’Intérêt Social». Physiocrates, éd. Daire,
Paris 1846, p. 893).
}
вартість якого завжди вимірюється суспільно-доконечною працею, тобто працею, завжди доконечною за
наявних суспільних умов.

Вартість засобів праці, що вже функціонують у процесі продукції,
машин і т. ін., отже, і та частина вартости, яку вони віддають продуктові, може змінюватися так
само, як вартість сировинного матеріялу. Коли, приміром, у наслідок якогось нового винаходу машина
того самого роду репродукується із зменшеною витратою праці, то старі машини більш або менш
зневартнюються
й тому переносять на продукт порівняно менше вартости. Але й тут зміна вартости постає поза тим
процесом продукції,
в якому машина функціонує як засіб продукції. В цьому процесі вона ніколи не віддає більше вартости,
ніж та, яку вона має незалежно від цього процесу.

Так само, як зміна вартости засобів продукції, хоч вона й справляє на них відбитий вплив навіть
після вступу їх до процесу продукції, не змінює їхнього характеру як сталого капіталу, так і зміна в
пропорції між сталим і змінним капіталом не порушує їхньої функціональної ріжниці. Приміром,
технічні умови процесу праці можуть змінитися так, що там, де раніш 10 робітників із 10 знаряддями
невеликої вартости обробляли порівняно малу кількість сировинного матеріялу, тепер 1 робітник
дорогою машиною обробляє в сто раз більшу кількість сировинного матеріялу. В цьому випадку сталий
капітал, тобто маса вартости застосованих
засобів продукції, дуже зросла б, а змінна частина капіталу, частина капіталу, авансована на робочу
силу, дуже спала б. Однак ця зміна переінакшує лише відношення між величиною сталого й змінного капіталу, або пропорцію,
в якій увесь капітал розпадається на сталі та змінні складові частини, але вона не
порушує самої ріжниці між сталим і змінним капіталом.

\section{Норма додаткової вартости}
\subsection{Ступінь експлуатації робочої сили}
Додаткова вартість, яку авансований капітал С створив у процесі продукції, або зростання авансованої
капітальної вартости С виступає перед нами насамперед як надлишок вартости
продукту понад суму вартости елементів його продукції.


Для зручности порівняння поновимо насамперед таку таблицю:

\begin{table}[H]
  \centering
  \caption*{Таблиця І}
  \footnotesize

  \settowidth\rotheadsize{\theadfont Продажна}
  \begin{tabular}{l c r c c c c c c}
    \toprule
      \thead[tl]{Земля} &
      &
      \rothead{Капітал} &
      \rothead{Зиск} &
      \rothead{Ціна\\продукції} &
      \rothead{Продукт} &
      \multicolumn{2}{c}{Рента} &
      \rothead{Норма\\надзиску} \\

      \cmidrule(rl){2-9}

       & акри  & \makecell{\poundsign{}} & \poundsign{} & кв. & кв. & кв. & \poundsign{}  & \% \\
      \midrule

       A & 1 & 2\tbfrac{1}{2} & \tbfrac{1}{2} & 3\phantom{\tbfrac{1}{2}} & \phantom{0}1 & 0 & \phantom{0}0 & \phantom{00}0 \\
       B & 1 & 2\tbfrac{1}{2} & \tbfrac{1}{2} & 1\tbfrac{1}{2}           & \phantom{0}2 & 1 & \phantom{0}3 & 120 \\
       C & 1 & 2\tbfrac{1}{2} & \tbfrac{1}{2} & 1\phantom{\tbfrac{1}{2}} & \phantom{0}3 & 2 & \phantom{0}6 & 240\\
       D & 1 & 2\tbfrac{1}{2} & \tbfrac{1}{2} & \phantom{0}\tbfrac{3}{4} & \phantom{0}4 & 3 & \phantom{0}9 & 360\\
    \midrule
      Разом & 4 & \hang{r}{1}0\pF & & & 10 & 6 & 18 & \makecell[t]{180 \\ пересічно}\\
  \end{tabular}

\end{table}

\noindent{}Коли ми тепер припустимо, що цифра 16 квартерів, що їх даватимуть землі
$В$, $C$, $D$ за низхідної норми продуктивности, достатня для того, щоб вилучити
$А$ з числа оброблюваних земель, то таблиця III перетворюється на таку:

\begin{table}[H]
  \centering
  \caption*{Таблиця V}
  \footnotesize

  \settowidth\rotheadsize{\theadfont Продажна}
  \begin{tabular}{l c r c r c c c c c}
  \toprule

\thead[tl]{Земля} &
&
\thead[t]{Вкладення \\ капіталу} &
\rothead{Зиск} &
\thead[t]{Продукт} &
\rothead{Продажна\\ціна} &
\rothead{Здобуток} &
\multicolumn{2}{c}{Рента} &
\rothead{Норма\\надзиску} \\

  \cmidrule(rl){2-10}
  & акри  & \poundsign{} & \poundsign{} & кв. & \poundsign{} & \poundsign{} & кв. & \poundsign{}  & \% \\
  \midrule


       B & 1 & 2\tbfrac{1}{2} \dplus{} 2\tbfrac{1}{2} & 1 & 2 \dplus{} 1\tbfrac{1}{2} \deq{} 3\tbfrac{1}{2} 
            & 1\tbfrac{5}{7} & \phantom{0}6\phantom{\tbfrac{1}{2}} & 0\phantom{\tbfrac{1}{2}} & 0\phantom{\tbfrac{1}{2}} & \phantom{00}0\\
       C & 1 & 2\tbfrac{1}{2} \dplus{} 2\tbfrac{1}{2} & 1 & 3 \dplus{} 2\phantom{\tbfrac{1}{2}} \deq{} 5\phantom{\tbfrac{1}{2}} 
            & 1\tbfrac{5}{7} & \phantom{0}8\tbfrac{4}{7}           & 1\tbfrac{1}{2}           & 2\tbfrac{4}{7}           & \phantom{0}51\hang{l}{\tbfrac{2}{5}}\\
       D & 1 & 2\tbfrac{1}{2} \dplus{} 2\tbfrac{1}{2} & 1 & 4 \dplus{} 3\tbfrac{1}{2} \deq{} 7\tbfrac{1}{2}
            & 1\tbfrac{5}{7} & 12\tbfrac{6}{7}                     & 4\phantom{\tbfrac{1}{2}} & 6\tbfrac{6}{7}           & 137\hang{l}{\tbfrac{1}{5}}\\  

      \midrule

      Разом & 3 & 15 & &  \phantom{2 \dplus{} 1\sfrac{1}{2} \deq{}}16\phantom{\sfrac{1}{2}} & & 27\sfrac{3}{7} & 5\sfrac{1}{2} & 9\sfrac{3}{7} & \makecell[t]{94\hang{l}{\sfrac{3}{10}} \\ пересічно\footnotemarkZ{}}\\
  \end{tabular}
\end{table}
\footnotetextZ{Тут пересічну норму надзиску обчислено не до всього вкладеного капіталу, а тільки до капіталу, вкладеного в рентодайні дільниці $C$ і $D$. \emph{Прим. Ред.}}

\noindent{}Тут за низхідної норми продуктивности додаткових капіталів і за різного
ступеня цього зменшення на різних землях, реґуляційна ціна продукції знизилася
з 3\pound{ ф. стерл.} до 1\sfrac{5}{7}\pound{ ф. стерл}. Вкладення капіталу збільшилося наполовину з 10\pound{ ф.
стерл.} до 15\pound{ ф. стерл}. Грошова рента зменшилася майже удвоє, з 18 до 9\sfrac{3}{7}\pound{ ф.
стерл.}, але збіжжева рента лише на \sfrac{1}{2},
% REMOVED \footnote*{
% В німецькому тексті тут стоїть «\sfrac{1}{22}». Очевидна помилка. \emph{Прим. Ред.}}
з 6 квартерів до 5\sfrac{1}{2}. Весь продукт
збільшився з 10 до 16, або на 60\%.
% REMOVED \footnote*{
% В німецькому тексті тут помилково стоїть: «160\%». \emph{Прим. Ред.}}
Збіжжева рента становить небагато більше
від третини всього продукту. Авансований капітал відноситься до грошової ренти
як $15 : 9\sfrac{3}{7}$, тимчасом як давніш це відношення було $10:18$.

\paragraph{За висхідної норми продуктивности додаткових капіталів.}

Цей випадок відрізняється від варіянту І, наведеного на початку цього
розділу, де ціна продукції за незмінної норми продуктивности знижується, тільки
тим, що коли потрібна додаткова кількість продукту для того, щоб вилучити
землю $А$, то це відбувається тут швидше.

Так за низхідної, як і за висхідної продуктивности додаткових вкладень
капіталу може це різно впливати, залежно від того, як ці вкладення розподіляються
між різними родами землі. В міру того, як цей різний вплив
\parbreak{}  %% абзац продовжується на наступній сторінці

\parcont{}  %% абзац починається на попередній сторінці
\index{iii2}{0160}  %% посилання на сторінку оригінального видання
буде вирівнювати або загострювати ріжниці, диференційна рента з кращих земель,
а разом з тим і загальна сума ренти знизиться або підвищиться, як це було
вже в випадку з диференційною рентою І. В решті, це залежить від величини земельної
площі й капіталу, вилучених разом з $А$, і від відносного розміру авансованого
капіталу, потрібного за висхідної продуктивности для того, щоб дати
додаткову кількість продукту для покриття попиту.

Єдиний пункт, на дослідженні якого тут варто спинитися, і який взагалі
вертає нас до дослідження того, як цей диференційний зиск перетворюється
на диференційну ренту, є такий:

У першому випадку, коли ціна продукції лишається та сама, додатковий
капітал, вкладений в землю $А$, не справляє впливу на диференційну ренту, як
таку, бо земля $А$, як і давніш, не дає ренти, ціна її продукту лишається та
сама, і продовжує реґулювати ринок.

У другому випадку, варіянт І, коли ціна продукції за незмінної норми продуктивности
понижується, земля $А$ неодмінно відпадає, і ще в більшій мірі це
відбувається у варіянті II (низхідна ціна продукції за низхідної норми продуктивности),
бо в противному разі додатковий капітал, вкладений у землю $А$,
мусив би підвищити ціну продукції. Але тут, у варіянті III другого випадку,
коли ціна продукції понижується, бо продуктивність додаткового капіталу підвищується,
цей додатковий капітал за певних умов може бути вкладений так
в землю $А$, як і в землі кращої якости.

Припустімо, що додатковий капітал в 2\sfrac{1}{2}\pound{ ф. стерл.}, вкладений в землю
$А$, продукує 1\sfrac{1}{5} кварт. замість 1 квартера.

\begin{table}[H]
  \centering
  \caption*{Таблиця VI}
  \footnotesize
  \setlength{\tabcolsep}{4.5pt}
  \settowidth\rotheadsize{\theadfont Продажна}
  
  \begin{tabular}{l c r c c c c c c c c}
   \toprule
      \thead[tl]{Рід\\землі} &
      &
      \thead[t]{Капітал} &
      \rothead{Зиск} &
      \rothead{Ціна\\продукції} &
      \thead[t]{Продукт} &
      \rothead{Продажна\\ціна} &
      \rothead{Здобуток} &
      \multicolumn{2}{c}{Рента} &
      \rothead{Норма\\надзиску} \\

      \cmidrule(rl){2-11}

      & акри  & \poundsign{} & \poundsign{} & кв. & \poundsign{} & \poundsign{} & кв. & \poundsign{}  & \% \\
      \midrule

       A & 1 & 2\tbfrac{1}{2} \dplus{} 2\tbfrac{1}{2} \deq{} 5 & 1 & 6 & 1 \dplus{} 1\tbfrac{1}{5} \deq{} 2\tbfrac{1}{5} & 2\tbfrac{8}{11} & \phantom{0}6 & 0\phantom{\tbfrac{1}{2}} & \phantom{0}0 & \phantom{00}0\% \\
       B & 1 & 2\tbfrac{1}{2} \dplus{} 2\tbfrac{1}{2} \deq{} 5 & 1 & 6 & 2 \dplus{} 2\tbfrac{2}{5} \deq{} 4\tbfrac{2}{5} & 2\tbfrac{8}{11} & 12           & 2\tbfrac{1}{5}           & \phantom{0}6 & 120\% \\
       C & 1 & 2\tbfrac{1}{2} \dplus{} 2\tbfrac{1}{2} \deq{} 5 & 1 & 6 & 3 \dplus{} 3\tbfrac{3}{5} \deq{} 6\tbfrac{3}{5} & 2\tbfrac{8}{11} & 18           & 4\tbfrac{2}{5}           & 12           & 240\%\\
       D & 1 & 2\tbfrac{1}{2} \dplus{} 2\tbfrac{1}{2} \deq{} 5 & 1 & 6 & 4 \dplus{} 4\tbfrac{4}{5} \deq{} 8\tbfrac{4}{5} & 2\tbfrac{8}{11} & 24           & 6\tbfrac{3}{5}           & 18           & 360\%\\

      \midrule

      Разом & 4 & \phantom{2\tbfrac{1}{2} \dplus{} 2\sfrac{1}{2} \deq{}}20 & 4 & \hang{r}{2}4 & \phantom{2 \dplus{} 1\tbfrac{1}{2} \deq{}}22\phantom{\tbfrac{1}{2}} & & 60 & 13\tbfrac{1}{5} & 36 & 240\%\hang{l}{\footnotemarkZ{}}\\
  \end{tabular}
  \setlength{\tabcolsep}{\tabcolsepdef}
\end{table}
\footnotetextZ{Тут пересічну норму надзиску обчислено не до всього вкладеного капіталу, а тільки до капіталу, вкладеного в рентодайні дільниці $В$, $C$ і $D$. \Red{Прим. Ред.}} % текст примітки прямо під заголовком

Цю таблицю слід порівняти, крім основної таблиці І, і з таблицею II, в якій
подвоєне вкладення капіталу сполучається з сталою продутивністю, пропорційною
капіталовкладенню.

Згідно з припущенням, регуляційна ціна продукції понижується. Коли б
вона залишалася сталою, 3\pound{ ф. стерл.}, то найгірша земля $А$, що давніш, при
капіталовкладенні лише в 2\sfrac{1}{2}\pound{ ф. стерл.}, не давала ренти, тепер почала б давати
ренту, хоч ніякої нової найгіршої землі не було б притягнено до оброблення;
це сталося б саме в наслідок того, що продуктивність на ній збільшилася б, але
лише для частини капіталу, а не для первісно вкладеного капіталу. Перші 3\pound{ ф.
стерл.} ціни продукції дають 1 квартер; другі — 1\sfrac{1}{5} квартера; але ввесь продукт в
2\sfrac{1}{5} квартери продається тепер по його пересічній ціні. А що норма продуктивности
зростає з додатковим капіталовкладенням, то це включає й поліпшення.

\parcont{}  %% абзац починається на попередній сторінці
\index{ii}{0161}  %% посилання на сторінку оригінального видання
капіталові як сталий капітал — це з погляду процесу зростання вартости.
Або, коли тут мова повинна бути про речову ріжницю, оскільки вона
впливає на процес циркуляції, то справа така: з природи вартости, яка є
не що інше, як зречевлена праця, і з природи діющої робочої сили, яка
є не що інше, як праця, що зречевлює себе, випливає, що робоча сила
протягом періоду її функціонування постійно утворює вартість і долярову
вартість; і що те, що на боці робочої сили виявляється як рух, як
утворення вартости, на боці її продукту виявляється у формі спокою,
як уже утворена вартість. Коли робоча сила вже діяла, то капітал не
складається вже більше з робочої сили на одному боці, із засобів продукції
на другому. Капітальна вартість, витрачена на робочу силу, є тепер
вартість, що її (+ додаткову вартість) долучено до продукту. Щоб
повторити процес, треба продати продукт і на вторговані гроші знову й
знову купувати робочу силу і вводити її в продуктивний капітал. Це
надає тоді частині капіталу, витраченій на робочу силу, так само, як і частинам
його, витраченим на матеріял праці тощо, характер обігового капіталу,
протилежно до того капіталу, що лишається закріплений у засобах праці.

Коли, навпаки, другорядне визначення обігового капіталу, спільне
йому з частиною сталого капіталу (сировинними й допоміжними матеріяламн)
— саме те визначення, що вартість, витрачену на обіговий капітал,
цілком переноситься на продукт, в продукції якого його зуживається, а
не поступінно й частинами, як в основного капіталу, що вартість ця,
отже, мусить цілком заміститися через продаж продукту, — перетворити
на посутню характеристику частини капіталу, витраченої на робочу силу,
то й частина капіталу, витрачена на заробітну плату, речово мусить
складатися не з діющої робочої сили, а з речових елементів, що їх робітник
купує на свою плату, отже, з частини суспільного товарового капіталу,
яка ввіходить у споживання робітника — з засобів існування.
Основний капітал складається при такому погляді на справу з засобів
праці, що зношуються повільніше, а тому й доводиться їх рідше відновлювати,
а капітал, витрачений на робочу силу, з засобів існування, що
їх треба заміщувати швидше.

Однак межі швидшої та повільнішої зношуваности стираються.

„Харч і одяг що їх зуживає робітник, будівлі, де він працює, знаряддя,
що допомагають йому в роботі, всі ці речі з своєї природи минущі.
Але є величезна ріжниця в часі, що протягом його зберігаються
ці різні капітали: парова машина зберігається довший час, ніж корабель,
корабель — довший час, ніж одяг робітника, одяг робітника знову таки
довший час, ніж харч, що його він споживає“\footnote{
The food and clothing consumed the labourer, the buildings in which he
works, the implements with which his labour is assisted, are all of a perishable
nature. There is, however, a vast difference in the time for which these different
capitals will endure: a steam-engine will last longer than a ship, a ship than the
clothing of the labourer, and the clothing of the labourer longer than the food which
he consumes“. (Ricardo, etc., p. 27).
}.

\input{_0162.tex}

\begin{table}[H]
  \centering
  \caption*{Таблиця VIa}
  \footnotesize

  \settowidth\rotheadsize{\theadfont Продажна}
  \begin{tabular}{l c r c r c c c c}
    \toprule
      \thead[t]{Земля} &
        &
      \thead[t]{Капітал} &
      \rothead{Зиск} &
      \thead[tc]{Продукт\\з акра} &  % в квартерах
      \rothead{Продажна\\ціна} &
      \rothead{Здобуток} &
      \multicolumn{2}{c}{Рента} \\

     \cmidrule(rl){2-9}
       & акри  & \poundsign{} & \poundsign{} & кв. & \poundsign{} & \poundsign{} & кв. & \poundsign{} \\

      \midrule
       A & 1 & 2\tbfrac{1}{2} \dplus{} 2\tbfrac{1}{2} \deq{} 5 & 1 & 1 \dplus{} \phantom{0}3\phantom{\tbfrac{1}{2}} \deq{} \phantom{0}4\phantom{\tbfrac{1}{2}}   & 1\tbfrac{1}{2} & \phantom{0}6\phantom{\tbfrac{3}{4}} & \phantom{0}0\phantom{\tbfrac{1}{2}}  & \phantom{0}0\phantom{\tbfrac{1}{2}} \\
       B & 1 & 2\tbfrac{1}{2} \dplus{} 2\tbfrac{1}{2} \deq{} 5 & 1 & 2 \dplus{} \phantom{0}2\tbfrac{1}{2} \deq{} \phantom{0}4\tbfrac{1}{2}                       & 1\tbfrac{1}{2} & \phantom{0}6\tbfrac{3}{4}           & \phantom{00}\tbfrac{1}{2}                            & \phantom{00}\tbfrac{3}{4}           \\
       C & 1 & 2\tbfrac{1}{2} \dplus{} 2\tbfrac{1}{2} \deq{} 5 & 1 & 3 \dplus{} \phantom{0}5\phantom{\tbfrac{1}{2}} \deq{} \phantom{0}8\phantom{\tbfrac{1}{2}}   & 1\tbfrac{1}{2} & 12\phantom{\tbfrac{3}{4}}           & \phantom{0}4\phantom{\tbfrac{1}{2}}                  & \phantom{0}6\phantom{\tbfrac{1}{2}} \\
       D & 1 & 2\tbfrac{1}{2} \dplus{} 2\tbfrac{1}{2} \deq{} 5 & 1 & 4 \dplus{} 12\phantom{\tbfrac{1}{2}} \deq{} 16\phantom{\tbfrac{1}{2}}                       & 1\tbfrac{1}{2} & 24\phantom{\tbfrac{3}{4}}           & 12\phantom{\tbfrac{1}{2}}                            & 18\phantom{\tbfrac{1}{2}}           \\
    
      \midrule
      Разом & 4 & \phantom{2\tbfrac{1}{2} \dplus{} 2\tbfrac{1}{2} \deq{}}20 & & \phantom{2 \dplus{} 12\tbfrac{1}{2} \deq{}}32\tbfrac{1}{2} & & & 16\tbfrac{1}{2} & 24\tbfrac{3}{4}\\
 \end{tabular}
\end{table}
% REMOVED
% \footnotemarkZ{}
% \footnotetextZ{В німецькому тексті тут очевидно помилково стоїть «6» \Red{Прим. Ред.}} % текст примітки прямо під заголовком

\noindent{}Нарешті, грошова рента підвищилася б, коли б у кращі земельні дільниці,
при тому самому відносному підвищенні родючости, вкладено було більше
додаткового капіталу, ніж у землю $А$, або коли б додаткові вкладання капіталу в кращі
земельні дільниці впливали, підвищуючи норму продуктивности. В обох випадках
ріжниці зростали б.

Грошова рента понижується, коли поліпшення, що сталося в наслідок
додаткової витрати капіталу, зменшує всі ріжниці, або частину їх, впливаючи
більше на $А$, ніж на $В$ і $C$. Вона понижується то більше, що незначніше
підвищення продуктивности кращих земельних дільниць. Від відносної неоднаковости
впливу залежить, чи підвищиться збіжжева рента, чи понизиться або
залишиться без зміни.

Грошова рента підвищується, а також і збіжжева рента, або тоді, коли за
незмінної відносної ріжниці в додатковій родючості різних земель більше вкладається
додаткового капіталу в землю, що дає ренту, ніж у землю $А$, що не дає
ренти, і більше у землю, що дає вищу, ніж у землю, що дає нижчу ренту;
абож тоді, коли родючість, при однаковому додатковому капіталі, більше зростає
на кращій і найкращій землі, ніж на землі $А$, причому грошова і збіжжева
рента підвищується саме у такому відношенні, в якому це збільшення родючости
на вищих розрядах землі вище, ніж на нижчих.

Але за всяких обставин рента відносно підвищується, коли підвищена продуктивність
є наслідок додаткової витрати капіталу, а не просто наслідок
збільшеної родючости за незмінної витрати капіталу. Це є абсолютний погляд,
який показує, що тут, як і в усіх давніших випадках, рента і збільшена рента з акра
(подібно до того, як при диференційній ренті І висота пересічної ренти на всю
оброблювану площу) є наслідок збільшеної витрати капіталу на землю, при
чому байдуже, чи функціонує ця витрата з сталою нормою продуктивности за
сталих або понижених цін, чи з низхідною нормою продуктивности за сталих або
за понижених цін, чи з висхідною нормою продуктивности за понижених цін.
Бо наше припущення: стала ціна за сталої, низхідної або висхідної норми продуктивности додаткового
капіталу, і низхідна ціна, за сталої, низхідної і висхідної
норми продуктивности, зводиться ось до чого: стала норма продуктивности додаткового капіталу при
сталій або низхідній ціні, низхідна норма продуктивности
при сталій або низхідній ціні, висхідна норма продуктивности за сталої
\parbreak{}  %% абзац продовжується на наступній сторінці

\parcont{}  %% абзац починається на попередній сторінці
\index{iii2}{0164}  %% посилання на сторінку оригінального видання
і низхідної ціни. Хоч в усіх цих випадках рента може залишитися без зміни
і може понизитись, вона понизилася б значніше, коли б додаткове вживання
капіталу, за інших незмінних обставин не зумовлювало збільшення родючости. Додаткове вкладення
капіталу тоді завжди є за причину відносної висоти ренти,
хоча б абсолютно вона й понизилась.

\section{Диференційна рента II. Третій випадок: висхідна ціна продукції}
\chaptermark{Диференційна рента II. Третій~випадок}

[Підвищення ціни продукції має за свою передумову, що продуктивність землі
найгіршої якости, що не дає ренти, зменшується. Ціна продукції, взята нами за
реґуляційну, може піднестися вище від 3\pound{ ф. ст.} за кв., лише тоді, коли 2\sfrac{1}{2}\pound{ ф. ст.},
витрачені на $А$, продукуватимуть менш за 1 квартер, або 5\pound{ ф. ст.} менш за
2 квартери, або коли б довелося обробляти землю ще гіршої якости, ніж $А$.

За незмінної або навіть висхідної продуктивности другого вкладення капіталу
це було б можливе лише тоді, коли б продуктивність першого вкладення в 2\sfrac{1}{2}\pound{ф. ст.}
зменшилась. Цей випадок трапляється досить часто. Наприклад, коли виснажений
при поверховій оранці зверхній шар ґрунту дає при старій системі обробітку
дедалі менші врожаї, то витягнений на поверхню з допомогою глибшої
оранки нижній шар за раціонального обробітку починає давати вищі
урожаї, ніж давніш. Але цей сцеціяльний випадок, точно кажучи, сюди не
стосується. Пониження продуктивности першої витрати капіталу в 2\sfrac{1}{2}\pound{ ф. ст.} зумовлює для кращих
земель, навіть коли там припустити аналогічні відношення,
пониження диференційної ренти І; проте тут ми розглядаємо лише диференційну
ренту II.~Але тому, що даний спеціяльний випадок не може статися, коли не
припускається існування диференційної ренти II і тому, що він в дійсності
становить відбитий вплив модифікації диференційної ренти І на диференційну
ренту II, то ми наведемо приклад цього випадку.

\begin{table}[H]
  \centering
  \caption*{Таблиця VII}

  \footnotesize
  \setlength{\tabcolsep}{4.5pt}
  \settowidth\rotheadsize{\theadfont Продажна}

  \begin{tabular}{l c r c c r c c c c c}
    \toprule
      \thead[tl]{Рід\\землі} &
      &
      \thead[t]{Капітал} &
      \rothead{Зиск} &
      \rothead{Ціна\\продукції} &
      \thead[t]{Продукт} & % \\ в кварт.}}}
      \rothead{Продажна\\ціна} &
      \rothead{Здобуток} &
      \multicolumn{2}{c}{Рента} &
      \rothead{Норма\\ренти} \\

    \cmidrule(rl){2-11}
      & акри  & \poundsign{} & \poundsign{} & \poundsign{} & кв. & \poundsign{} & \poundsign{} & кв. & \poundsign{} & \% \\

    \midrule
      A & 1 & 2\tbfrac{1}{2} \dplus{} 2\tbfrac{1}{2} \deq{} 5 & 1 & 6 & \phantom{0}\tbfrac{1}{2} \dplus{} 1\tbfrac{1}{4} \deq{} 1\tbfrac{3}{4}                      & 3\tbfrac{3}{7} & \phantom{0}6 & 0\phantom{\tbfrac{1}{2}} & \phantom{0}0 & \phantom{00}0\\
      B & 1 & 2\tbfrac{1}{2} \dplus{} 2\tbfrac{1}{2} \deq{} 5 & 1 & 6 & 1\phantom{\tbfrac{1}{1}} \dplus{} 2\tbfrac{1}{2} \deq{} 3\tbfrac{1}{2}                     & 3\tbfrac{3}{7} & 12           & 1\tbfrac{3}{4}           & \phantom{0}6 & 120 \\
      C & 1 & 2\tbfrac{1}{2} \dplus{} 2\tbfrac{1}{2} \deq{} 5 & 1 & 6 & 1\tbfrac{1}{2} \dplus{} 3\tbfrac{3}{4} \deq{} 5\tbfrac{1}{4}                               & 3\tbfrac{3}{7} & 18           & 3\tbfrac{1}{2}           & 12           & 240\\
      D & 1 & 2\tbfrac{1}{2} \dplus{} 2\tbfrac{1}{2} \deq{} 5 & 1 & 6 & 2\pF{} \dplus{} 5\pF{} \deq{} 7\pF{} & 3\tbfrac{3}{7} & 24           & 5\tbfrac{1}{4}           & 18           & 360\\

     \midrule

      Разом & & 20 & & & 17\tbfrac{1}{2} & & 60 & 10\tbfrac{1}{2} & 36 & 240\footnotemarkZ{}\\
 
  \end{tabular}
\end{table}
\footnotetextZ{Тут, як і далі в таблицях VIII, IX, і X пересічну норму ренти обчислено не до всього
вкладеного капіталу, а тільки до капіталу, вкладеного в рентодайні дільниці. \Red{Пр.~Ред.}} % текст примітки прямо під заголовком

\noindent{}Грошова рента, як і грошовий здобуток, лишаються ті самі, що і в таблиці II.~Підвищена реґуляційна ціна продукції точнісінько покриває те, що втрачено
на кількості продукту; а що ця ціна продукції і кількість продукту змінюються
в зворотному відношенні, то само собою зрозуміло, що здобуток їх лишається
той самий.


У вищенаведеному випадку ми припускали, що продуктивна сила другого
капіталовкладення вища, ніж первісна продуктивність першого вкладення. Справа
не зміниться, коли ми припустимо для другого капіталовкладення лише таку саму
продуктивність, що її мала первісна продуктивність першого вкладення, як от у
таблиці VIII.

\begin{table}[H]
  \begin{center}
    \emph{Таблиця VIII}
    \footnotesize

  \begin{tabular}{c@{  } c@{  } c@{  } c@{  } c@{  } c@{  } c@{  } c@{  } c@{  } c@{  } c}
    \toprule
      \multirowcell{2}{\makecell{Рід\\ землі}} &
      \multirowcell{2}{Акри} &
      Капітал &
      Зиск &
      \makecell{Ціна\\ продук.} &
      \multirowcell{2}{\makecell{Продукт в\\ квартерах}} &
      \makecell{Продажна \\ ціна} &
      \makecell{Здо-\\буток} &
      \multicolumn{2}{c}{Рента} &
      \multirowcell{2}{\makecell{Норма \\надзиску}} \\

      \cmidrule(r){3-3}
      \cmidrule(r){4-4}
      \cmidrule(r){5-5}
      \cmidrule(r){7-7}
      \cmidrule(r){8-8}
      \cmidrule(r){9-9}
      \cmidrule(r){10-10}

       &  & ф. ст. & ф. ст. & ф. ст. & & ф. ст. & ф. ст. & Кварт. & ф. ст. &   \\
      \midrule
      A & 1 & 2\sfrac{1}{2} \dplus{} 2\sfrac{1}{2} \deq{} 5 & 1 & 6 & \phantom{0}\sfrac{1}{2} \dplus{} 1 \deq{} 1\sfrac{1}{2}                                 & 4 & \phantom{0}6 & 0\phantom{\sfrac{1}{2}} & \phantom{0}0 & \phantom{00}0\% \\
      B & 1 & 2\sfrac{1}{2} \dplus{} 2\sfrac{1}{2} \deq{} 5 & 1 & 6 & 1\phantom{\sfrac{0}{0}} \dplus{} 2 \deq{} 3\phantom{\sfrac{0}{0}}                       & 4 & 12           & 1\sfrac{1}{2}           & \phantom{0}6 & 120\% \\
      C & 1 & 2\sfrac{1}{2} \dplus{} 2\sfrac{1}{2} \deq{} 5 & 1 & 6 & 1\sfrac{1}{2} \dplus{} 3 \deq{} 4\sfrac{1}{4}                                           & 4 & 18           & 3\phantom{\sfrac{1}{2}} & 12           & 240\%\\
      D & 1 & 2\sfrac{1}{2} \dplus{} 2\sfrac{1}{2} \deq{} 5 & 1 & 6 & 2\phantom{\sfrac{0}{0}} \dplus{} 4 \deq{} 6\phantom{\sfrac{0}{0}} & 4 & 24           & 4\sfrac{1}{2}           & 18           & 360\%\\

     \cmidrule(r){3-3}
     \cmidrule(l){6-6}
     \cmidrule(r){8-8}
     \cmidrule(r){9-9}
     \cmidrule(r){10-10}
     \cmidrule(r){11-11}

      Разом & & \phantom{2\sfrac{1}{2} \dplus{} 2\sfrac{1}{2} \deq{}}20 & & & \phantom{2 \dplus{} 1\sfrac{1}{2} \deq{}}15 & & 60 & 9 & 36 & 240\%\\
  \end{tabular}

  \end{center}
\end{table}

І тут ціна продукції, яка підвищується в тому самому відношенні, зумовлює
те, що зменшення продуктивности цілком урівноважується так щодо здобутку,
як і щодо грошової ренти.

У своєму чистому вигляді третій випадок виступає лише за низхідної продуктивности
другого капіталовкладення, тимчасом як продуктивність першого вкладення,
як це всюди припускалось для першого і другого випадків, лишається сталою.
Диференційна рента І тут не зачіпається, зміна відбувається лише з тією частиною,
що походить з диференційної ренти II.~Ми подаємо два приклади: в
першому продуктивність другого капіталовкладення зводиться до \sfrac{1}{2}, у другому
— до \sfrac{1}{4} продуктивности першого вкладення.

\begin{table}[H]
  \begin{center}
    \emph{Таблиця IX}
    \footnotesize

  \begin{tabular}{c@{  } c@{  } c@{  } c@{  } c@{  } c@{  } c@{  } c@{  } c@{  } c@{  } c}
    \toprule
      \multirowcell{2}{\makecell{Рід\\ землі}} &
      \multirowcell{2}{Акри} &
      Капітал &
      Зиск &
      \makecell{Ціна\\ продук.} &
      \multirowcell{2}{\makecell{Продукт в\\ квартерах}} &
      \makecell{Продажна \\ ціна} &
      \makecell{Здо-\\буток} &
      \multicolumn{2}{c}{Рента} &
      \multirowcell{2}{\makecell{Норма \\ренти}} \\

      \cmidrule(r){3-3}
      \cmidrule(r){4-4}
      \cmidrule(r){5-5}
      \cmidrule(r){7-7}
      \cmidrule(r){8-8}
      \cmidrule(r){9-9}
      \cmidrule(r){10-10}

       &  & ф. ст. & ф. ст. & ф. ст. & & ф. ст. & ф. ст. & Кварт. & ф. ст. &   \\
      \midrule
      A & 1 & 2\sfrac{1}{2} \dplus{} 2\sfrac{1}{2} \deq{} 5 & 1 & 6 & 1 \dplus{} \phantom{0}\sfrac{1}{2} \deq{} 1\sfrac{1}{2}                                 & 4 & \phantom{0}6 & 0\phantom{\sfrac{1}{2}} & \phantom{0}0 & \phantom{00}0\% \\
      B & 1 & 2\sfrac{1}{2} \dplus{} 2\sfrac{1}{2} \deq{} 5 & 1 & 6 & 2 \dplus{} 1\phantom{\sfrac{0}{0}} \deq{} 3\phantom{\sfrac{0}{0}}                       & 4 & 12           & 1\sfrac{1}{2}           & \phantom{0}6 & 120\% \\
      C & 1 & 2\sfrac{1}{2} \dplus{} 2\sfrac{1}{2} \deq{} 5 & 1 & 6 & 3 \dplus{} 1\sfrac{1}{2} \deq{} 4\sfrac{1}{2}                                           & 4 & 18           & 3\phantom{\sfrac{1}{2}} & 12           & 240\%\\
      D & 1 & 2\sfrac{1}{2} \dplus{} 2\sfrac{1}{2} \deq{} 5 & 1 & 6 & 4 \dplus{} 2\phantom{\sfrac{0}{0}} \deq{} 6\phantom{\sfrac{0}{0}} & 4 & 24           & 4\sfrac{1}{2}           & 18           & 360\%\\

     \cmidrule(r){3-3}
     \cmidrule(l){6-6}
     \cmidrule(r){8-8}
     \cmidrule(r){9-9}
     \cmidrule(r){10-10}
     \cmidrule(r){11-11}

      Разом & & \phantom{2\sfrac{1}{2} \dplus{} 2\sfrac{1}{2} \deq{}}20 & & & \phantom{2 \dplus{} 1\sfrac{1}{2} \deq{}}15 & & 60 & 9 & 36 & 240\%\\
  \end{tabular}

  \end{center}
\end{table}

Таблиця IX та сама, що й таблиця VIII, тільки в таблиці VIII зменшення
продуктивности припадає на перше, в таблиці IX — на друге капіталовкладення.


\index{i}{0166}  %% посилання на сторінку оригінального видання
Жакоб, припускаючи ціну пшениці в 80\shil{ шилінґів} за квартер і пересічний урожай в 22 бушлі з одного
акра, так що один акр приносить 11\pound{ фунтів стерлінґів}, наводить для 1815~\abbr{р.} обрахунок, який через те,
що в ньому вже переведено компенсацію різних
пунктів, дуже хибний, але все ж для нашої мети придатний.

\begin{table}[H]
\caption*{Продукція вартости на 1 акр}
\noindent\begin{tabularx}{\textwidth}{@{}X*{4}{@{~}r}@{\hspace{5em}}X*{4}{@{~}r}@{}}
Насіння (пшениця)\dotfill{} & 1 & \pound{ф. ст.} & 9 &\shil{шил.} &
Десятини, податки\dotfill{} & 1 & \pound{ф. ст.} & 1 &\shil{шил.} \\
Добриво\dotfill{} & 2 & \dittomark{} & 10 & \dittomark{} &
Рента\dotfill{} & 1 & \dittomark{} &  8 & \dittomark{} \\

Заробітна плата\dotfill{} &  3  & \dittomark{} &  10 & \dittomark{} &
Зиск фармера й проц.\dotfill{}& 1 & \dittomark{} & 2 & \dittomark{} \\

\cmidrule(r{5em}){1-5}  \cmidrule{6-10} 

Разом\dotfill{} & 7 & \pound{ф. ст.} & 9 & \shil{шил.} &
Разом\dotfill{} & З & \pound{ф. ст.} & 11 & \shil{шил.}
\end{tabularx}
\end{table}

\noindent{}Додаткова вартість, припускаючи завжди, що ціна продукту дорівнює його вартості, розподіляється тут
між різними рубриками: зиск, процент, десятина й~\abbr{т. ін.} Ці рубрики для нас не мають значення. Ми
складаємо їх і як результат маємо додаткову вартість у 3\pound{ фунти стерлінґів} 11\shil{ шилінґів.} Ті 3\pound{ фунти
стерлінґів} 19\shil{ шилінґів}, що коштують насіння і добриво, ми, як сталу частину капіталу, прирівнюємо
нулеві. Лишається авансований змінний капітал у 3\pound{ фунти стерлінґів} 10\shil{ шилінґів}, замість якого
спродуковано нову вартість у 3\pound{ фунти стерлінґів} 10\shil{ шилінґів} \dplus{} 3\pound{ фунти стерлінґів} 11\shil{ шилінґів.} Отже,
$\frac{m}{v} \deq{} \frac{3\text{\pound{ фунти ст.} }11\text{\shil{ шилінґів}}}{3\text{\pound{ фунти ст.} }10\text{\shil{ шилінґів}}}$ становить більше, ніж 100\%. Робітник більш
ніж половину свого робочого дня вживає на продукцію додаткової вартости, яку різні особи під різними
приводами  розподіляють проміж себе\footnoteA{
Наведені обчислення мають значення лише як ілюстрація. Справді, ми припускаємо, що ціни
дорівнюють вартостям. У третій книзі ми побачимо,
що це прирівняння робиться не так просто навіть для пересічних цін.
}.

\subsection{Вираз вартости продукту у відносних частинах продукту}

Вернімось тепер до того прикладу, що показав нам, як капіталіст із грошей робить капітал. Доконечна
праця його прядуна
становила 6 годин, додаткова праця — стільки ж, отже, ступінь експлуатації робочої сили — 100\%.

Продукт дванадцятигодинного робочого дня є 20 фунтів пряжі вартістю в 30\shil{ шилінґів.} Не менше як \sfrac{8}{10}
вартости цієї пряжі (24\shil{ шилінґи}) становить вартість зужиткованих засобів продукції, що лише знову
з’являється (20 фунтів бавовни на 20\shil{ шилінґів}, веретена й~\abbr{т. ін.} на 4\shil{ шилінґи}), або, інакше кажучи,
складається з сталого капіталу. Решта, \sfrac{2}{10}, є нова вартість у 6\shil{ шилінґів}, яка постала підчас
процесу прядіння, що з них половина компенсує авансовану денну вартість робочої сили, або змінний
капітал, а друга половина становить додаткову вартість у 3\shil{ шилінґи.} Отже, сукупна вартість цих 20
фунтів пряжі складаєься ось як:
вартість пряжі в $30\text{\shil{ шилінґів}}
\deq{} 24\text{\shil{ шилінґи}} \dplus{}
\oversetl{v}{v-166-node-1}{3\text{\shil{ шилінґи}}} \dplus{}
\oversetr{m}{m-166-node-1}{3\text{\shil{ шилінґи.}}}$
\begin{tikzpicture}[overlay]
    \path[-,thick,black] (v-166-node-1) edge [out=5,in=175] (m-166-node-1);
\end{tikzpicture}%


\index{iii2}{0167}  %% посилання на сторінку оригінального видання

\begin{table}[h]
  \begin{center}
    \emph{Таблиця Xa}
    \footnotesize

  \begin{tabular}{c@{  } c@{  } c@{  } c@{  } c@{  } c@{  } c@{  } c@{  } c@{  } c@{  } c}
    \toprule
      \multirowcell{2}{\makecell{Рід\\ землі}} &
      \multirowcell{2}{Акри} &
      Капітал &
      Зиск &
      \makecell{Ціна\\ продук.} &
      \multirowcell{2}{\makecell{Продукт в\\ квартерах}} &
      \makecell{Продажна \\ ціна} &
      \makecell{Здо-\\буток} &
      \multicolumn{2}{c}{Рента} &
      \multirowcell{2}{Підвищення} \\

      \cmidrule(r){3-3}
      \cmidrule(r){4-4}
      \cmidrule(r){5-5}
      \cmidrule(r){7-7}
      \cmidrule(r){8-8}
      \cmidrule(r){9-9}
      \cmidrule(r){10-10}

       &  & ф. ст. & ф. ст. & ф. ст. & & ф. ст. & ф. ст. & Кварт. & ф. ст. &   \\
      \midrule
      a & 1 & \phantom{2\sfrac{1}{2} \dplus{} }5\phantom{\sfrac{1}{2}} & 1 & 6 & \phantom{1\sfrac{1}{2} \dplus{} 3 \deq{} }1\sfrac{1}{8}           & 5\sfrac{1}{3} & \phantom{0}6\phantom{\sfrac{1}{5}} & 0\phantom{\sfrac{1}{2}}  & \phantom{0}0\phantom{\sfrac{1}{1}} & 0\phantom{\sfrac{1}{5} \dplus{} 3 × 7\sfrac{1}{5}} \\
      A & 1 & 2\sfrac{1}{2} \dplus{} 2\sfrac{1}{2}                     & 1 & 6 & 1 \dplus{} \phantom{0}\sfrac{1}{4} \deq{} 1\sfrac{1}{4}           & 5\sfrac{1}{3} & \phantom{0}6\sfrac{2}{3}           & \phantom{0}\sfrac{1}{8}  & \phantom{00}\sfrac{2}{3}           & \sfrac{2}{3}\phantom{ \dplus{} 3 × 7\sfrac{1}{5}} \\
      B & 1 & 2\sfrac{1}{2} \dplus{} 2\sfrac{1}{2}                     & 1 & 6 & 2 \dplus{} \phantom{0}\sfrac{1}{2} \deq{} 2\sfrac{1}{2}           & 5\sfrac{1}{3} & 13\sfrac{1}{3}                     & 1\sfrac{3}{8}            & \phantom{0}7\sfrac{1}{3}           & \sfrac{2}{3} \dplus{} 6\sfrac{2}{3}\phantom{ 1 ×} \\
      C & 1 & 2\sfrac{1}{2} \dplus{} 2\sfrac{1}{2}                     & 1 & 6 & 3 \dplus{} \phantom{0}\sfrac{3}{4} \deq{} 3\sfrac{3}{4}           & 5\sfrac{1}{3} & 20\phantom{\sfrac{3}{5}}           & 2\sfrac{5}{8}            & 14\phantom{\sfrac{3}{5}}           & \sfrac{2}{3} \dplus{} 2 × 6\sfrac{2}{3}\\
      D & 1 & 2\sfrac{1}{2} \dplus{} 2\sfrac{1}{2}                     & 1 & 6 & 4 \dplus{} 1\phantom{\sfrac{0}{0}} \deq{} 5\phantom{\sfrac{0}{0}} & 5\sfrac{1}{3} & 26\sfrac{2}{3}                     & 3\sfrac{7}{8}            & 20\sfrac{2}{3}                     & \sfrac{2}{3} \dplus{} 3 × 6\sfrac{2}{3}\\

     \cmidrule(r){5-5}
     \cmidrule(r){6-6}
     \cmidrule(r){8-8}
     \cmidrule(r){9-9}
     \cmidrule(r){10-10}

      Разом & & & & 30 & \phantom{2 \dplus{} 1\sfrac{1}{2} \deq{}}13\sfrac{5}{8} & & 72\sfrac{2}{3} & 8\phantom{\sfrac{1}{2}} & 42\sfrac{2}{3} & \\
  \end{tabular}

  \end{center}
\end{table}

Приєднанням землі \emph{а} породжується нову диференційну ренту І; на цій
новій основі розвивається потім диференційна рента II теж у зміненому вигляді.
Земля \emph{а} має в кожній з трьох вищенаведених таблиць ріжну родючість; ряд
відповідно висхідних ступенів родючости починається лише з $А$. Відповідно до
цього розміщується і ряд висхідних рент. Рента з найгіршої рентодайної землі,
що раніш ренти не давала, становить постійну величину, яка просто приєднується
до всіх вищих рент; лише за вирахуванням цієї сталої величини ясно виступає
при порівнянні вищих рент ряд ріжниць і його паралелізм з рядом, що
визначає родючість різних земель. У всіх таблицях різні ступені родючости, починаючи
з $А$ до $D$, стосуються один до одного, як $1: 2 : 3 : 4$, і відповідно до
цього ренти стосуються одна до однієї:

\begin{tabular}{l}
в VIIa, як 1 : 1 \dplus{} 7 : 1 \dplus{} 2 × 7 : 1 \dplus{} 3 × 7,\\
в VIIIa, як 1\sfrac{1}{5}:1\sfrac{1}{5} \dplus{} 7\sfrac{1}{5} : 1\sfrac{1}{5} \dplus{} 2 × 7\sfrac{1}{5} : 1\sfrac{1}{5} \dplus{} 3 × 7\sfrac{1}{5},\\
в Xa, як \sfrac{2}{3} : \sfrac{2}{3} \dplus{} 6\sfrac{2}{3} : \sfrac{2}{3} \dplus{} 2 × 6\sfrac{2}{3} : \sfrac{2}{3} \dplus{} 3 × 6\sfrac{2}{3}.\\
\end{tabular}

Коротко: коли рента з $А \deq{} n$, а рента з землі безпосередньо вищої родючости
$= n \dplus{} m$, то ряд буде такий: $n: n \dplus{} m: n \dplus{} 2m : n \dplus{} З m$ і~\abbr{т. д.} — Ф.~Е.]

\pfbreak

[А що вищенаведений третій випадок в рукопису не був опрацьований —
там є лише його заголовок, — то завдання редактора було по змозі доповнити
це, як зроблено вище. Але йому лишається ще зробити загальні висновки, що
випливають з усього попереднього дослідження диференційної ренти II в її трьох
головних випадках і дев’ятьох похідних випадках. Але для цієї мети наведені
в рукопису випадки придаються лише дуже мало. Поперше, в них порівнюються
дільниці землі, що з них здобутки для площ однакової величини стосуються
як $1: 2 : 3 : 4$; отже, беруться ріжниці, що вже від самого початку дуже перебільшені,
і які в дальшому розвитку зроблених на цій основі припущень і обчислень
призводять до цілком насильницьких числових відношень. Але подруге,
\parbreak{}  %% абзац продовжується на наступній сторінці

\parcont{}  %% абзац починається на попередній сторінці
\index{ii}{0168}  %% посилання на сторінку оригінального видання
з природи своєї подільні, і на ті, що потребують для виготовлення порівняно
довшого зв’язного періоду. В одному випадку по сьогоднішній
продукції певної кількости пряжі, вугілля тощо, завтра не наступає нової
продукції пряжі, вугілля та ін. Інша справа з кораблями, будівлями, залізницями
тощо. Тут не тільки робота припиняється, але припиняється
і зв’язний акт продукції. Коли будову не будується далі, то зужиті на
неї засоби продукції та працю витрачено марно. Навіть коли будову знову
розпочнуть будувати, то в проміжний період вона завжди зазнає ушкодження.

Протягом цілого робочого періоду нагромаджується, наверствовуючись,
та частина вартости, що її основний капітал щоденно віддає продуктові,
поки останній цілком достигне. І тут разом з тим виявляється практична
важливість ріжниці між основним та обіговим капіталом. Основний капітал
авансується на порівняно довший час на процес продукції, його не
доводиться поновлювати раніше, ніж мине цей, може кількарічний період.
Чи переносить парова машина свою вартість щоденно частинами на пряжу,
продукт подільного робочого процесу, чи вона протягом трьох місяців
віддає її паровозові, продуктові безперервного продукційного акту, —
ця обставина абсолютно нічого не змінює у витраті капіталу, потрібного
на закуп парової машини. В одному разі її вартість припливає назад маленькими
частками, напр., щотижня, а в другому разі — більшими масами,
напр., що три місяці. Але в обох випадках парову машину поновлюється
лише один раз, може, лише через двадцять років. Доки кожен окремий
період, що протягом його вартість парової машини через продаж продукту
припливає частинами назад, коротший, ніж період її власного існування,
та сама машина функціонує й далі в процесі продукції протягом
кількох робочих періодів.

Інакше стоїть справа з обіговими складовими частинами авансованого
капіталу. Робочу силу, куплену на цей тиждень, витрачено протягом
цього тижня, і вона зречевилась у продукті. Її треба оплатити наприкінці
цього тижня. І така витрата капіталу на робочу силу щотижня повторюється
протягом трьох місяців, так що витрата частини капіталу на
один тиждень не забезпечує капіталіста проти закупу робочої сили на
наступний тиждень. На оплату робочої сили щотижня треба витрачати
новий додатковий капітал, і, якщо залишити осторонь всі кредитові відносини,
капіталіст мусить мати спроможність витрачати заробітну плату
протягом трьох місяців, хоч він і виплачує її щотижневими порціями. Так
само стоїть справа з іншими частинами обігового капіталу, сировинними
та допоміжними матеріялами. Один по одному шари праці накладаються на
продукт. Не лише вартість витраченої робочої сили, а й додаткову
вартість безперервно переноситься на продукт протягом процесу праці,
але на продукт, що ще неготовий, що не має ще форми готового товару,
а, значить, і не є ще здатний до циркуляції. Все це має силу й для капітальної
вартости, що переноситься на продукт шарами з сировинних та
допоміжних матеріялів.

Залежно від довшого або коротшого протягу робочого періоду, що
його потребує специфічна природа продукту на виготовлення його або
\parbreak{}  %% абзац продовжується на наступній сторінці

\index{iii2}{0169}  %% посилання на сторінку оригінального видання
Варіянт III: Висхідна продуктивність другої витрати (таблиця XXI); це знов
таки зумовлює низхідну продуктивність першої витрати.

Друга видозміна: земля гіршої якости, (позначувана: літерою а)
вступає в конкуренцію; земля А дає ренту.

Варіянт 1: Незмінна продуктивність другої витрати (таблиця XXII).

Варіант 2: Низхідна продуктивність (таблиця XXIII).

Варіант 3: Висхідна продуктивність (таблиця XXIV).

Ці три варіянти відповідають загальним умовам проблеми і не дають
приводу до будь-яких зауважень.

Тепер ми наведемо таблиці:

\begin{table}[h]
  \begin{center}
    \emph{Таблиця XI}
    \footnotesize

  \begin{tabular}{c@{  } c@{  } c@{  } c@{  } c@{  } c@{  } c}
    \toprule
      \multirowcell{2}{\makecell{Рід\\ землі}} &
      Ціна продукції &
      Продукт &
      \makecell{Продажна \\ ціна} &
      \makecell{Здо-\\буток} &
      Рента &
      \multirowcell{2}{Підвищення ренти} \\

      \cmidrule(r){2-2}
      \cmidrule(r){3-3}
      \cmidrule(r){4-4}
      \cmidrule(r){5-5}
      \cmidrule(r){6-6}

       & Шил. & Бушелі & Шил. & Шил. & Шил. & &   \\
      \midrule
      A & 60 & 10 & 6 & 60  & \phantom{00}0 & \phantom{00 × 0}0 \\
      B & 60 & 12 & 6 & 72  & \phantom{0}12 & \phantom{01 × }12 \\
      C & 60 & 14 & 6 & 84  & \phantom{0}24 & \phantom{0}2 × 12           \\
      D & 60 & 16 & 6 & 96  & \phantom{0}36 & \phantom{0}3 × 12           \\
      E & 60 & 18 & 6 & 108 & \phantom{0}48 & \phantom{0}4 × 12           \\

     \cmidrule(r){6-6}
     \cmidrule(r){7-7}

      & & & & & 120 & 10 × 12 \\
  \end{tabular}

  \end{center}
\end{table}

За другої витрати капіталу на тій самій землі.

Перший випадок: за незмінної ціни продукції.

Варіянт 1: за незмінної продуктивности другої витрати капіталу.

\begin{table}[h]
  \begin{center}
    \emph{Таблиця XII}
    \footnotesize

  \begin{tabular}{c@{  } c@{  } c@{  } c@{  } c@{  } c@{  } c}
    \toprule
      \multirowcell{2}{\makecell{Рід\\ землі}} &
      Ціна продукції &
      Продукт &
      \makecell{Продажна \\ ціна} &
      \makecell{Здо-\\буток} &
      Рента &
      \multirowcell{2}{Підвищення ренти} \\

      \cmidrule(r){2-2}
      \cmidrule(r){3-3}
      \cmidrule(r){4-4}
      \cmidrule(r){5-5}
      \cmidrule(r){6-6}

       & Шил. & Бушелі & Шил. & Шил. & Шил. & &   \\
      \midrule
      A & 60 + 60 = 120 & 10 + 10 = 20 & 6 & 120  & \phantom{00}0 & \phantom{00 × 0}0 \\
      B & 60 + 60 = 120 & 12 + 12 = 24 & 6 & 144  & \phantom{0}24 & \phantom{01 × }24 \\
      C & 60 + 60 = 120 & 14 + 14 = 28 & 6 & 168  & \phantom{0}48 & \phantom{0}2 × 24 \\
      D & 60 + 60 = 120 & 16 + 16 = 32 & 6 & 192  & \phantom{0}72 & \phantom{0}3 × 24 \\
      E & 60 + 60 = 120 & 18 + 18 = 36 & 6 & 216  & \phantom{0}96 & \phantom{0}4 × 24 \\

     \cmidrule(r){6-6}
     \cmidrule(r){7-7}

      & & & & & 240 & 10 × 24 \\
  \end{tabular}

  \end{center}
\end{table}

Варіянт 2: за низхідної продуктивности другої витрати капіталу: на землі
А не зроблено другої витрати.

1) Коли земля В стає землею, що не дає ренти.


\index{iii2}{0170}  %% посилання на сторінку оригінального видання

\begin{table}[H]
  \begin{center}
    \emph{Таблиця XIII}
    \footnotesize

  \begin{tabular}{c@{  } c@{  } c@{  } c@{  } c@{  } c@{  } c}
    \toprule
      \multirowcell{2}{\makecell{Рід\\ землі}} &
      Ціна продукції &
      Продукт &
      \makecell{Продажна \\ ціна} &
      \makecell{Здо-\\буток} &
      Рента &
      \multirowcell{2}{Підвищення ренти} \\

      \cmidrule(r){2-2}
      \cmidrule(r){3-3}
      \cmidrule(r){4-4}
      \cmidrule(r){5-5}
      \cmidrule(r){6-6}

       & Шил. & Бушелі & Шил. & Шил. & Шил. & &   \\
      \midrule
      A & \phantom{60 + 60 = 0}60 & \phantom{12 + 10\sfrac{1}{3} =} 10\phantom{\sfrac{2}{3}}           & 6 & \phantom{0}60 & \phantom{00}0 & \phantom{0 × 0}0 \\
      B & 60 + 60 = 120           & 12 + \phantom{0}8\phantom{\sfrac{1}{3}} = 20\phantom{\sfrac{2}{3}} & 6 & 120           & \phantom{00}0 & \phantom{0 × 0}0 \\
      C & 60 + 60 = 120           & 14 + \phantom{0}9\sfrac{1}{3} = 23\sfrac{1}{3}                     & 6 & 140           & \phantom{0}20 & \phantom{1 × }20 \\
      D & 60 + 60 = 120           & 16 + 10\sfrac{2}{3} = 26\sfrac{2}{3}                               & 6 & 160           & \phantom{0}40 & 2 × 20 \\
      E & 60 + 60 = 120           & 18 + 12\footnotemarkZ{}\phantom{/}= 30\phantom{\sfrac{2}{3}}                  & 6 & 180           & \phantom{0}60 & 3 × 20 \\

     \cmidrule(r){6-6}
     \cmidrule(r){7-7}

      & & & & & 120 & 6 × 20 \\
  \end{tabular}

  \end{center}
\end{table}
\footnotetextZ{В німецькому тексті тут стоїть «20». Очевидна помилка. Прим. Ред.} % текст примітки прямо під заголовком

2) Кола земля $В$ не стає землею, що зовсім не дає ренти.

\begin{table}[H]
  \begin{center}
    \emph{Таблиця XIV}
    \footnotesize

  \begin{tabular}{c@{  } c@{  } c@{  } c@{  } c@{  } c@{  } c}
    \toprule
      \multirowcell{2}{\makecell{Рід\\ землі}} &
      Ціна продукції &
      Продукт &
      \makecell{Продажна \\ ціна} &
      \makecell{Здо-\\буток} &
      Рента &
      \multirowcell{2}{Підвищення ренти} \\

      \cmidrule(r){2-2}
      \cmidrule(r){3-3}
      \cmidrule(r){4-4}
      \cmidrule(r){5-5}
      \cmidrule(r){6-6}

       & Шил. & Бушелі & Шил. & Шил. & Шил. & &   \\
      \midrule
      A & \phantom{60 + 60 = 0}60 & \phantom{12 + 10\sfrac{1}{3} =} 10\phantom{\sfrac{2}{3}}           & 6 & \phantom{0}60 & \phantom{00}0 & \phantom{4 ×}0\phantom{ + 3 × 21}\\
      B & 60 + 60 = 120           & 12 + \phantom{0}9\phantom{\sfrac{1}{3}} = 21\phantom{\sfrac{2}{3}} & 6 & 126           & \phantom{00}6 & \phantom{4 ×}6\phantom{ + 3 × 21}\\
      C & 60 + 60 = 120           & 14 + 10\sfrac{1}{2} = 24\sfrac{1}{2}                               & 6 & 147           & \phantom{0}27 & \phantom{4 ×}6 + 21\phantom{1 × } \\
      D & 60 + 60 = 120           & 16 + 12\phantom{\sfrac{2}{3}} = 28\phantom{\sfrac{2}{3}}           & 6 & 168           & \phantom{0}48 & \phantom{4 ×}6 + 2 × 21 \\
      E & 60 + 60 = 120           & 18 + 13\sfrac{1}{2}= 31\sfrac{1}{2}                                & 6 & 189           & \phantom{0}69 & \phantom{4 ×}6 + 3 × 21 \\

     \cmidrule(r){6-6}
     \cmidrule(r){7-7}

      & & & & & 150 & 4 × 6 + 6 × 21 \\
  \end{tabular}

  \end{center}
\end{table}

Варіянт 3: за висхідної продуктивности другої витрати капіталу; на землі
$А$ тут теж не робиться другої витрати.

\begin{table}[H]
  \begin{center}
    \emph{Таблиця XV}
    \footnotesize

  \begin{tabular}{c@{  } c@{  } c@{  } c@{  } c@{  } c@{  } c}
    \toprule
      \multirowcell{2}{\makecell{Рід\\ землі}} &
      Ціна продукції &
      Продукт &
      \makecell{Продажна \\ ціна} &
      \makecell{Здо-\\буток} &
      Рента &
      \multirowcell{2}{Підвищення ренти} \\

      \cmidrule(r){2-2}
      \cmidrule(r){3-3}
      \cmidrule(r){4-4}
      \cmidrule(r){5-5}
      \cmidrule(r){6-6}

       & Шил. & Бушелі & Шил. & Шил. & Шил. & &   \\
      \midrule
      A & \phantom{60 + 60 = 0}60 & \phantom{12 + 10\sfrac{1}{3} =} 10\phantom{\sfrac{2}{3}}           & 6 & \phantom{0}60 & \phantom{00}0 & \phantom{4 ×0}0\phantom{ + 3 × 27}\\
      B & 60 + 60 = 120           & 12 + 15\phantom{\sfrac{1}{3}} = 27\phantom{\sfrac{2}{3}}           & 6 & 162           & \phantom{0}42 & \phantom{4 ×}42\phantom{ + 3 × 27}\\
      C & 60 + 60 = 120           & 14 + 17\sfrac{1}{2} = 31\sfrac{1}{2}                               & 6 & 189           & \phantom{0}69 & \phantom{4 ×}42 + 27\phantom{1 × } \\
      D & 60 + 60 = 120           & 16 + 20\phantom{\sfrac{2}{3}} = 36\phantom{\sfrac{2}{3}}           & 6 & 216           & \phantom{0}96 & \phantom{4 ×}42 + 2 × 27 \\
      E & 60 + 60 = 120           & 18 + 22\sfrac{1}{2}= 40\sfrac{1}{2}                                & 6 & 243           & 123           & \phantom{4 ×}42 + 3 × 27 \\

     \cmidrule(r){6-6}
     \cmidrule(r){7-7}

      & & & & & 330 & 4 × 42 + 6 × 27 \\
  \end{tabular}

  \end{center}
\end{table}


\index{iii2}{0171}  %% посилання на сторінку оригінального видання
Другий випадок за низхідної ціни продукції.

Варіянт 1: за незмінної продуктивности другої витрати капіталу: земля
$А$ випадає з конкуренції, земля $В$ стає землею, що не дає ренти.

\begin{table}[h]
  \begin{center}
    \emph{Таблиця XVI}
    \footnotesize

  \begin{tabular}{c@{  } c@{  } c@{  } c@{  } c@{  } c@{  } c}
    \toprule
      \multirowcell{2}{\makecell{Рід\\ землі}} &
      Ціна продукції &
      Продукт &
      \makecell{Продажна \\ ціна} &
      \makecell{Здо-\\буток} &
      Рента &
      \multirowcell{2}{Підвищення ренти} \\

      \cmidrule(r){2-2}
      \cmidrule(r){3-3}
      \cmidrule(r){4-4}
      \cmidrule(r){5-5}
      \cmidrule(r){6-6}

       & Шил. & Бушелі & Шил. & Шил. & Шил. &  \\
      \midrule
      B & 60 \dplus{} 60 \deq{} 120 & 12 \dplus{} 12 \deq{} 24 & 5 & 120  & \phantom{00}0 & \phantom{01 × }0 \\
      C & 60 \dplus{} 60 \deq{} 120 & 14 \dplus{} 14 \deq{} 28 & 5 & 140  & \phantom{0}20 & \phantom{1 ×} 20 \\
      D & 60 \dplus{} 60 \deq{} 120 & 16 \dplus{} 16 \deq{} 32 & 5 & 160  & \phantom{0}40 & 2 × 20 \\
      E & 60 \dplus{} 60 \deq{} 120 & 18 \dplus{} 18 \deq{} 36 & 5 & 180  & \phantom{0}60 & 3 × 20 \\

     \cmidrule(r){6-6}
     \cmidrule(r){7-7}

      & & & & & 120 & 6 × 20 \\
  \end{tabular}

  \end{center}
\end{table}

Варіант 2: за низхідної продуктивности другої витрати капіталу; земля
$А$ випадає з конкуренції, земля $В$ стає землею, що не дає ренти.

\begin{table}[h]
  \begin{center}
    \emph{Таблиця XVII}
    \footnotesize

  \begin{tabular}{c@{  } c@{  } c@{  } c@{  } c@{  } c@{  } c}
    \toprule
      \multirowcell{2}{\makecell{Рід\\ землі}} &
      Ціна продукції &
      Продукт &
      \makecell{Продажна \\ ціна} &
      \makecell{Здо-\\буток} &
      Рента &
      \multirowcell{2}{Підвищення ренти} \\

      \cmidrule(r){2-2}
      \cmidrule(r){3-3}
      \cmidrule(r){4-4}
      \cmidrule(r){5-5}
      \cmidrule(r){6-6}

       & Шил. & Бушелі & Шил. & Шил. & Шил. &  \\
      \midrule
      B & 60 \dplus{} 60 \deq{} 120 & 12 \dplus{} \phantom{0}9\phantom{\sfrac{1}{2}} \deq{} 21\phantom{\sfrac{1}{2}} & 5\sfrac{5}{7} & 120  & \phantom{00}0 & \phantom{01 × }0 \\
      C & 60 \dplus{} 60 \deq{} 120 & 14 \dplus{} 10\sfrac{1}{2} \deq{} 24\sfrac{1}{2}                               & 5\sfrac{5}{7} & 140  & \phantom{0}20 & \phantom{1 ×} 20 \\
      D & 60 \dplus{} 60 \deq{} 120 & 16 \dplus{} 12\phantom{\sfrac{1}{2}} \deq{} 28\phantom{\sfrac{1}{2}}           & 5\sfrac{5}{7} & 160  & \phantom{0}40 & 2 × 20 \\
      E & 60 \dplus{} 60 \deq{} 120 & 18 \dplus{} 13\sfrac{1}{2} \deq{} 31\sfrac{1}{2}                               & 5\sfrac{5}{7} & 180  & \phantom{0}60 & 3 × 20 \\

     \cmidrule(r){6-6}
     \cmidrule(r){7-7}

      & & & & & 120 & 6 × 20 \\
  \end{tabular}

  \end{center}
\end{table}

Варіант 3: за висхідної продуктивности другої витрати капіталу; земля
$А$ залишається конкурентною. Земля $В$ дає ренту.

\begin{table}[h]
  \begin{center}
    \emph{Таблиця XVIII}
    \footnotesize

  \begin{tabular}{c@{  } c@{  } c@{  } c@{  } c@{  } c@{  } c}
    \toprule
      \multirowcell{2}{\makecell{Рід\\ землі}} &
      Ціна продукції &
      Продукт &
      \makecell{Продажна \\ ціна} &
      \makecell{Здо-\\буток} &
      Рента &
      \multirowcell{2}{Підвищення ренти} \\

      \cmidrule(r){2-2}
      \cmidrule(r){3-3}
      \cmidrule(r){4-4}
      \cmidrule(r){5-5}
      \cmidrule(r){6-6}

       & Шил. & Бушелі & Шил. & Шил. & Шил. &   \\
      \midrule
      A & 60 \dplus{} 60 \deq{} 120 & 10 \dplus{} 15 \deq{} 25 & 4\sfrac{4}{5} & 120  & \phantom{00}0 & \phantom{00 × 0}0 \\
      B & 60 \dplus{} 60 \deq{} 120 & 12 \dplus{} 18 \deq{} 30 & 4\sfrac{4}{5} & 144  & \phantom{0}24 & \phantom{01 × }24 \\
      C & 60 \dplus{} 60 \deq{} 120 & 14 \dplus{} 21 \deq{} 35 & 4\sfrac{4}{5} & 168  & \phantom{0}48 & \phantom{0}2 × 24 \\
      D & 60 \dplus{} 60 \deq{} 120 & 16 \dplus{} 24 \deq{} 40 & 4\sfrac{4}{5} & 192  & \phantom{0}72 & \phantom{0}3 × 24 \\
      E & 60 \dplus{} 60 \deq{} 120 & 18 \dplus{} 27 \deq{} 45 & 4\sfrac{4}{5} & 216  & \phantom{0}96 & \phantom{0}4 × 24 \\

     \cmidrule(r){6-6}
     \cmidrule(r){7-7}

      & & & & & 240 & 10 × 24 \\
  \end{tabular}

  \end{center}
\end{table}


Третій випадок: За висхідної ціни продукції.

А.~Коли земля $А$ не дає ренти й продовжує реґулювати ціну.

Варіянт 1: За незмінної продуктивности другої витрати капіталу, що зумовлює
низхідну продуктивність першої витрати.

\begin{table}[H]
  \begin{center}
    \emph{Таблиця XIX\footnotemarkZ{}}
    \footnotesize

  \begin{tabular}{c@{  } c@{  } c@{  } c@{  } c@{  } c@{  } c}
    \toprule
      \multirowcell{2}{\makecell{Рід\\ землі}} &
      Ціна продукції &
      Продукт &
      \makecell{Продажна \\ ціна} &
      \makecell{Здо-\\буток} &
      Рента &
      \multirowcell{2}{Підвищення ренти} \\

      \cmidrule(r){2-2}
      \cmidrule(r){3-3}
      \cmidrule(r){4-4}
      \cmidrule(r){5-5}
      \cmidrule(r){6-6}

       & Шил. & Бушелі & Шил. & Шил. & Шил. &  \\
      \midrule
      A & 60 \dplus{} 60 \deq{} 120 & 5 \dplus{} 12\sfrac{1}{2} \deq{} 17\sfrac{1}{2}                      & 6\sfrac{6}{7} & 120  & \phantom{00}0 & \phantom{01 × }0 \\
      B & 60 \dplus{} 60 \deq{} 120 & 6 \dplus{} 15\phantom{\sfrac{1}{2}} \deq{} 21\phantom{\sfrac{1}{2}}  & 6\sfrac{6}{7} & 144  & \phantom{0}24 & \phantom{1 ×} 24 \\
      C & 60 \dplus{} 60 \deq{} 120 & 7 \dplus{} 17\sfrac{1}{2} \deq{} 24\sfrac{1}{2}                      & 6\sfrac{6}{7} & 168  & \phantom{0}48 & 2 × 24 \\
      D & 60 \dplus{} 60 \deq{} 120 & 8 \dplus{} 20\phantom{\sfrac{1}{2}} \deq{} 28\phantom{\sfrac{1}{2}}  & 6\sfrac{6}{7} & 192  & \phantom{0}72 & 3 × 24 \\
      E & 60 \dplus{} 60 \deq{} 120 & 9 \dplus{} 22\sfrac{1}{2} \deq{} 31\sfrac{1}{2}                      & 6\sfrac{6}{7} & 216  & \phantom{0}96 & 4 × 24 \\

     \cmidrule(r){6-6}
     \cmidrule(r){7-7}

      & & & & & 240 & 10 × 24 \\
  \end{tabular}

  \end{center}
\end{table}

\footnotetextZ{Це є таблиця висхідної продуктивности другої витрати капіталу. Порівн. табл. XXI. \emph{Прим. Ред}}

Варіянт 2: За низхідної продуктивности другої витрати капіталу, що не виключає
незмінюваної продуктивности першої витрати.

\begin{table}[H]
  \begin{center}
    \emph{Таблиця XX}
    \footnotesize

  \begin{tabular}{c@{  } c@{  } c@{  } c@{  } c@{  } c@{  } c}
    \toprule
      \multirowcell{2}{\makecell{Рід\\ землі}} &
      Ціна продукції &
      Продукт &
      \makecell{Продажна \\ ціна} &
      \makecell{Здо-\\буток} &
      Рента &
      \multirowcell{2}{Підвищення ренти} \\

      \cmidrule(r){2-2}
      \cmidrule(r){3-3}
      \cmidrule(r){4-4}
      \cmidrule(r){5-5}
      \cmidrule(r){6-6}

       & Шил. & Бушелі & Шил. & Шил. & Шил. &  \\
      \midrule
      A & 60 \dplus{} 60 \deq{} 120 & 10 \dplus{} 5 \deq{} 15  & 8 & 120  & \phantom{00}0 & \phantom{01 × }0 \\
      B & 60 \dplus{} 60 \deq{} 120 & 12 \dplus{} 6 \deq{} 18  & 8 & 144  & \phantom{0}24 & \phantom{1 ×} 24 \\
      C & 60 \dplus{} 60 \deq{} 120 & 14 \dplus{} 7 \deq{} 21  & 8 & 168  & \phantom{0}48 & 2 × 24 \\
      D & 60 \dplus{} 60 \deq{} 120 & 16 \dplus{} 8 \deq{} 24  & 8 & 192  & \phantom{0}72 & 3 × 24 \\
      E & 60 \dplus{} 60 \deq{} 120 & 18 \dplus{} 9 \deq{} 27  & 8 & 216  & \phantom{0}96 & 4 × 24 \\

     \cmidrule(r){6-6}
     \cmidrule(r){7-7}

      & & & & & 240 & 10 × 24 \\
  \end{tabular}

  \end{center}
\end{table}

Варіянт 3: За висхідної продуктивности другої витрати капіталу, що, за даних
припущень, обумовлює низхідну продуктивність першої витрати.

\begin{table}[H]
  \begin{center}
    \emph{Таблиця XXI}
    \footnotesize

  \begin{tabular}{c@{  } c@{  } c@{  } c@{  } c@{  } c@{  } c}
    \toprule
      \multirowcell{2}{\makecell{Рід\\ землі}} &
      Ціна продукції &
      Продукт &
      \makecell{Продажна \\ ціна} &
      \makecell{Здо-\\буток} &
      Рента &
      \multirowcell{2}{Підвищення ренти} \\

      \cmidrule(r){2-2}
      \cmidrule(r){3-3}
      \cmidrule(r){4-4}
      \cmidrule(r){5-5}
      \cmidrule(r){6-6}

       & Шил. & Бушелі & Шил. & Шил. & Шил. &  \\
      \midrule
      A & 60 \dplus{} 60 \deq{} 120 & 5 \dplus{} 12\sfrac{1}{2} \deq{} 17\sfrac{1}{2}                      & 6\sfrac{6}{7} & 120  & \phantom{00}0 & \phantom{01 × }0 \\
      B & 60 \dplus{} 60 \deq{} 120 & 6 \dplus{} 15\phantom{\sfrac{1}{2}} \deq{} 21\phantom{\sfrac{1}{2}}  & 6\sfrac{6}{7} & 144  & \phantom{0}24 & \phantom{1 ×} 24 \\
      C & 60 \dplus{} 60 \deq{} 120 & 7 \dplus{} 17\sfrac{1}{2} \deq{} 24\sfrac{1}{2}                      & 6\sfrac{6}{7} & 168  & \phantom{0}48 & 2 × 24 \\
      D & 60 \dplus{} 60 \deq{} 120 & 8 \dplus{} 20\phantom{\sfrac{1}{2}} \deq{} 28\phantom{\sfrac{1}{2}}  & 6\sfrac{6}{7} & 192  & \phantom{0}72 & 3 × 24 \\
      E & 60 \dplus{} 60 \deq{} 120 & 9 \dplus{} 22\sfrac{1}{2} \deq{} 31\sfrac{1}{2}                      & 6\sfrac{6}{7} & 216  & \phantom{0}96 & 4 × 24 \\

     \cmidrule(r){6-6}
     \cmidrule(r){7-7}

      & & & & & 240 & 10 × 24 \\
  \end{tabular}

  \end{center}
\end{table}


В. Коли гірша (позначувана літерою а) земля стає землею, яка реґулює
ціну і через те $А$ починає давати ренту. Де не виключає можливости незмінюваної
продуктивности другої витрати для всіх варіянтів.

Варіянт 1: Незмінювана продуктивність другої витрати капіталу.

\begin{table}[H]
  \begin{center}
    \emph{Таблиця XXII}
    \footnotesize

  \begin{tabular}{c@{  } c@{  } c@{  } c@{  } c@{  } c@{  } c}
    \toprule
      \multirowcell{2}{\makecell{Рід\\ землі}} &
      Ціна продукції &
      Продукт &
      \makecell{Продажна \\ ціна} &
      \makecell{Здо-\\буток} &
      Рента &
      \multirowcell{2}{Підвищення ренти} \\

      \cmidrule(r){2-2}
      \cmidrule(r){3-3}
      \cmidrule(r){4-4}
      \cmidrule(r){5-5}
      \cmidrule(r){6-6}

       & Шил. & Бушелі & Шил. & Шил. & Шил. &  \\
      \midrule
      a & \phantom{60 \dplus{} 60 \deq{} }120 & \phantom{10 \dplus{} 10 \deq{} }16 & 7\tbfrac{1}{2} & 120  & \phantom{00}0  & \phantom{01 × }0 \\
      A & 60 \dplus{} 60 \deq{} 120           & 10 \dplus{} 10 \deq{} 20            & 7\tbfrac{1}{2} & 150  & \phantom{0}30 & \phantom{1 ×} 30 \\
      B & 60 \dplus{} 60 \deq{} 120           & 12 \dplus{} 12 \deq{} 24            & 7\tbfrac{1}{2} & 180  & \phantom{0}60 & 2 × 30 \\
      C & 60 \dplus{} 60 \deq{} 120           & 14 \dplus{} 14 \deq{} 28            & 7\tbfrac{1}{2} & 210  & \phantom{0}90 & 3 × 30 \\
      D & 60 \dplus{} 60 \deq{} 120           & 16 \dplus{} 16 \deq{} 32            & 7\tbfrac{1}{2} & 240  & 120           & 4 × 30 \\
      E & 60 \dplus{} 60 \deq{} 120           & 18 \dplus{} 18 \deq{} 36            & 7\tbfrac{1}{2} & 270  & 150           & 5 × 30 \\

     \cmidrule(r){6-6}
     \cmidrule(r){7-7}

      & & & & & 450 & \hang{r}{1}5 × 30 \\
  \end{tabular}

  \end{center}
\end{table}

Варіянт 2: Низхідна продуктивність другої витрати капіталу.

\begin{table}[H]
  \begin{center}
    \emph{Таблиця XXIII}
    \footnotesize

  \begin{tabular}{c@{  } c@{  } c@{  } c@{  } c@{  } c@{  } c}
    \toprule
      \multirowcell{2}{\makecell{Рід\\ землі}} &
      Ціна продукції &
      Продукт &
      \makecell{Продажна \\ ціна} &
      \makecell{Здо-\\буток} &
      Рента &
      \multirowcell{2}{Підвищення ренти} \\

      \cmidrule(r){2-2}
      \cmidrule(r){3-3}
      \cmidrule(r){4-4}
      \cmidrule(r){5-5}
      \cmidrule(r){6-6}

       & Шил. & Бушелі & Шил. & Шил. & Шил. &  \\
      \midrule
      a & \phantom{60 \dplus{} 60 \deq{} }120 & \phantom{10 \dplus{} 10\tbfrac{1}{2} \deq{} }15\phantom{\tbfrac{1}{2}}  & 8 & 120 & \phantom{00}0 & \phantom{5 × 0}0 \phantom{+ 01 × 28} \\
      A & 60 \dplus{} 60 \deq{} 120           & 10 \dplus{} \phantom{0}7\tbfrac{1}{2} \deq{} 17\tbfrac{1}{2}                       & 8 & 140 & \phantom{0}20 & \phantom{5 × }20 \phantom{+ 01 × 28} \\
      B & 60 \dplus{} 60 \deq{} 120           & 12 \dplus{} \phantom{0}9\phantom{\tbfrac{1}{2}} \deq{} 21\phantom{\tbfrac{1}{2}}   & 8 & 168 & \phantom{0}48 & \phantom{5 × }20 \dplus{} \phantom{01 × }28\\
      C & 60 \dplus{} 60 \deq{} 120           & 14 \dplus{} 10\tbfrac{1}{2} \deq{} 24\tbfrac{1}{2}                      & 8 & 194 & \phantom{0}76 & \phantom{5 × }20 \dplus{} \phantom{0}2 × 28 \\
      D & 60 \dplus{} 60 \deq{} 120           & 16 \dplus{} 12\phantom{\tbfrac{1}{2}} \deq{} 28\phantom{\tbfrac{1}{2}}  & 8 & 224 & 104           & \phantom{5 × }20 \dplus{} \phantom{0}3 × 28 \\
      E & 60 \dplus{} 60 \deq{} 120           & 18 \dplus{} 13\tbfrac{1}{2} \deq{} 31\tbfrac{1}{2}                      & 8 & 252 & 132           & \phantom{5 × }20 \dplus{} \phantom{0}4 × 28 \\

     \cmidrule(r){6-6}
     \cmidrule(r){7-7}

      & & & & & 380 & 5 × 20 \dplus{} 10 × 28 \\
  \end{tabular}

  \end{center}
\end{table}

Варіянт 3: Висхідна продуктивність другої витрати капіталу.

\begin{table}[H]
  \begin{center}
    \emph{Таблиця XXIV}
    \footnotesize

  \begin{tabular}{c@{  } c@{  } c@{  } c@{  } c@{  } c@{  } c}
    \toprule
      \multirowcell{2}{\makecell{Рід\\ землі}} &
      Ціна продукції &
      Продукт &
      \makecell{Продажна \\ ціна} &
      \makecell{Здо-\\буток} &
      Рента &
      \multirowcell{2}{Підвищення ренти} \\

      \cmidrule(r){2-2}
      \cmidrule(r){3-3}
      \cmidrule(r){4-4}
      \cmidrule(r){5-5}
      \cmidrule(r){6-6}

       & Шил. & Бушелі & Шил. & Шил. & Шил. &  \\
      \midrule
      a & \phantom{60 \dplus{} 60 \deq{} }120 & \phantom{10 \dplus{} 10\tbfrac{1}{2} \deq{} }16\phantom{\tbfrac{1}{2}}  & 7\tbfrac{1}{2} & 120\phantom{\tbfrac{1}{2}} & \phantom{00}0\phantom{\tbfrac{1}{2}} & \phantom{5 × 15 \dplus{} 15 × }0\phantom{\tbfrac{3}{4}} \\
      A & 60 \dplus{} 60 \deq{} 120           & 10 \dplus{} 12\tbfrac{1}{2} \deq{} 22\tbfrac{1}{2}                      & 7\tbfrac{1}{2} & 168\tbfrac{3}{4}           & \phantom{0}48\tbfrac{3}{4}           & \phantom{5 × }15 \dplus{} \phantom{1 × }33\tbfrac{3}{4} \\
      B & 60 \dplus{} 60 \deq{} 120           & 12 \dplus{} 15\phantom{\tbfrac{1}{2}} \deq{} 27\phantom{\tbfrac{1}{2}}  & 7\tbfrac{1}{2} & 202\tbfrac{1}{2}           & \phantom{0}82\tbfrac{1}{2}           & \phantom{5 × }15 \dplus{} 2 × 33\tbfrac{3}{4} \\
      C & 60 \dplus{} 60 \deq{} 120           & 14 \dplus{} 17\tbfrac{1}{2} \deq{} 31\tbfrac{1}{2}                      & 7\tbfrac{1}{2} & 236\tbfrac{1}{4}           & 116\tbfrac{1}{4}                     & \phantom{5 × }15 \dplus{} 3 × 33\tbfrac{3}{4} \\
      D & 60 \dplus{} 60 \deq{} 120           & 16 \dplus{} 20\phantom{\tbfrac{1}{2}} \deq{} 36\phantom{\tbfrac{1}{2}}  & 7\tbfrac{1}{2} & 270\phantom{\tbfrac{1}{2}} & 150\phantom{\tbfrac{1}{2}}           & \phantom{5 × }15 \dplus{} 4 × 33\tbfrac{3}{4} \\
      E & 60 \dplus{} 60 \deq{} 120           & 18 \dplus{} 22\tbfrac{1}{2} \deq{} 40\tbfrac{1}{2}                      & 7\tbfrac{1}{2} & 303\tbfrac{3}{4}           & 183\tbfrac{3}{4}                     & \phantom{5 × }15 \dplus{} 5 × 33\tbfrac{3}{4} \\

     \cmidrule(r){6-6}
     \cmidrule(r){7-7}

      & & & & & 581\tbfrac{3}{4} & 5 × 15 \dplus{} 15 × 33\tbfrac{3}{4} \\
  \end{tabular}

  \end{center}
\end{table}

\parcont{}  %% абзац починається на попередній сторінці
\index{i}{0174}  %% посилання на сторінку оригінального видання
знайомства з вами в кращому світі. Addio!..\footnote{
Однак пан професор мав деяку користь із своєї подорожі до Менчестеру.
В «Letters on the Factory Act» увесь чистий прибуток, «зиск» і «процент» і
навіть «something more»,\footnote*{
— щось більше. \emph{Ред}.
} залежить від
однієї неоплаченої
робочої години робітника! Роком раніш у своїх «Outlines of Political Economy»,
складених для насолоди оксфордських студентів і освічених філістерів, Сеніор,
полемізуючи проти Рікардового визначення вартости робочим часом, «відкрив», що
зиск постає з праці капіталіста, а процент з його
аскетичности, з його «поздержливости». Сама побрехенька була стара» але слово
«поздержливість» («Abstinenz») було нове. Пан Рошер правильно переклав його
німецькою мовою словом «Enthaltung» («поздержливість»). А його компатріоти,
менше биті в латині, Вірти, Шульци
й інші Міхелі, переклали його на чорнече «самовідречення» («Entsagung»).
} Сиґнал «останньої години», що її винайшов Сеніор 1836 р., наново протрубив
був 15 квітня 1848 р. в «London Economist» Джемc Вілсон, один з головних
мандаринів економічної науки, у своїй полеміці проти
закону про десятигодинний робочий день.
\subsection{4. Додатковий продукт}

Ту частину продукту (\sfrac{1}{10} від 20 фунтів пряжі, або 2 фунти пряжі, у
прикладі §2), яка репрезентує додаткову вартість, ми
називаємо додатковим продуктом (surplus produce, produit net). Як норму
додаткової вартости визначає відношення додаткової
вартости не до цілої суми капіталу, а лише до його змінної складової частини,
так і рівень додаткового продукту визначає відношення останнього не до решти
цілого продукту, а до тієї частини його, яка репрезентує доконечну працю.
Як продукція додаткової вартости є визначальна мета
капіталістичної продукції, так і ступінь багатства вимірюється не абсолютною
величиною продукту, а відносною величиною додаткового продукту.\footnote{
«Для індивіда, що має капітал у 20.000 фунтів стерлінґів, і що його зиски
становлять 2.000 фунтів стерлінґів на рік, було б цілком байдуже, чи його
капітал вживає 100 чи 1.000 робітників, чи випродуковані товари продається
за 10.000 фунтів стерлінґів чи за 20.000 фунтів стерлінґів, аби лише
його зиски в усіх цих випадках не падали нижче як 2.000 фунтів
стерлінґів. Хіба реальний інтерес націй не такий самий? Коли припустити, що
реальний чистий прибуток нації, її ренти й зиски лишаються однакові, то не має
найменшої ваги, чи нація складається з 10 чи
12 мільйонів людности». (\emph{Ricardo}: «The Principles of Political Economy»,
3 rd. ed, London 1821, p. 416). Задовго перед Рікардом Артур Юнґ, фанатик
додаткового продукту, взагалі язикатий, неспроможний
на будь-яку критику письменник, що його слава стоїть у зворотному відношенні
до його заслуг, сказав, між іншим: «Що за користь була б для сучасного
королівства з якоїсь цілої провінції, що в ній землю обробляли б на
староримський лад дрібні незалежні селяни, про мене хоч би
й як і найкраще? Яка мета була б у цьому, крім одним-однієї мети продукувати
людей («the mere purpose of breeding men»), а це саме по собі не має
ніякої мети» (is a most useless purpose»). (\emph{Arthur Young}: «Political
Arithmetic etc.», 1774, p. 47).
Додаток до примітки 34. Дивний є «великий нахил малювати чистий прибуток
корисним для робітничої кляси\dots{} та проте ясно, що це стається не через те,
що він чистий» («the strong inclination to
represent net wealth as beneficial to the labouring class\dots{} though it
is evidently not on account
of being net»). (\emph{Th. Hopkins}: «On Rent of Land etс.», London 1823, p. 126).
}

\parcont{}  %% абзац починається на попередній сторінці
\index{iii2}{0175}  %% посилання на сторінку оригінального видання
землі А, перестала брати участь у конкуренції і земля В, і земля C зробилася б
регуляційною землею, що не дає ренти.

Отже, що більше капіталу вживається на землі, що вищого розвитку досягли
у країні хліборобство і цивілізація взагалі, то вище підносяться ренти
з акра, так само як і загальна сума рент, то колосальніший стає податок, що
його виплачує суспільство великим земельним власникам у вигляді надзисків, — доки
всі роди землі, що вже підлягли обробленню, зберігають здатність до конкуренції.

Цей закон пояснює дивовижну живучість кляси великих землевласників.
Жодна інша кляса суспільства не живе так марнотратно, як ця; жодна інша
не заявляє такої претенсії на звичну «відповідну станові» розкіш, хоч би звідки
одержувано для цього гроші; жодна інша кляса не нагромаджує з таким легким
серцем боргів за боргами. А проте, вона завжди хоч і вскочить, а вискочить —
завдяки капіталові, що його інші люди вклали в землю і що дає їй ренти позавсяким
співвідношенням з зисками, що їх одержує з нього капіталіст. Але той самий закон
пояснює також, чому ця живучість великого землевласника поволі вичерпується.

Коли 1846 року скасовано було в Англії збіжжеві мита, англійські фабриканти
думали, що цим вони перетворили землевласницьку аристократію на
павперів. Замість цього вона забагатіла більше, ніж будь-коли раніш. Яким
чином це сталося? Дуже просто. По-перше, від цього часу до орендарів почали
ставити закріплену контрактом вимогу, за якою вони зобов’язувалися витрачати
щорічно по 12 ф. ст. замість 8 ф. ст. на акр, і по-друге, землевласники, що мали і в
нижній палаті дуже численних представників, асигнували собі велику державну
допомогу для дренування та інших перманентних поліпшень своїх земель. А що
цілковитого витиснення найгіршої землі не сталося, а відбулося, щонайбільше,
застосування її для іншої мети, та й то здебільша тимчасове, то ренти підвищились
відповідно до підвищенної витрати капіталу, і земельна аристократія
виграла від цього більше, ніж будь-коли раніш.

Але все минає. Трансатлантійські пароплави, а також північно південноамериканські
та індійські залізниці дали змогу цілком особливим країнам конкурувати
на європейських збіжжевих ринках. Це були, з одного боку, північноамериканські
прерії, арґентінські пампаси, степи вже від природи придатні для
обробітку плугом, незайманий ґрунт, що багато років давав багаті врожаї навіть
за примітивної культури і без добрива. Далі це були землі російських та
індійських комуністичних громад, які мусили продавати частину свого продукту,
до того ж дедалі більшу, щоб одержати гроші для виплати податків, що їх виплачувати
примушував, досить часто з допомогою катування, нещадний деспотизм
держави. Ці продукти продавалося безвідносно до цін продукції, продавалося
за ціну, яку пропонував торговець, бо селянин на строк виплати мусив
мати гроші хоч би за яку ціну. І з цією конкуренцією, — незайманої степової
землі, а також російських та індійських селян, що знемагають під податковим пресом,
— європейський орендар і селянин не міг упоратись при старих рентах. Частина
землі в Европі остаточно стала щодо продукції збіжжя, конкурентно неспроможною,
ренти всюди занепали; другий наш випадок, варіянт II: низхідна ціна
і низхідна продуктивність додаткових витрат капіталу зробився загальним
правилом для Европи, звідси лемент аґраріїв від Шотландії до Італії, від Південної
Франції до Східньої Прусії. На щастя, ще геть не всі степові землі
оброблено; їх ще надто досить для того, щоб зруйнувати все європейське велике
землеволодіння та крім того і дрібне. — Ф. Е.].

\pfbreak

Рубрики, під якими треба дослідити ренту, такі:

А. Диференційна рента.

1) Поняття диференційної ренти. Ілюстрація силою води. Перехід до власне
хліборобської ренти.
\parbreak{}  %% абзац продовжується на наступній сторінці


2)~Диференційна рента І, що походить з ріжниці в родючості різних земельних
дільниць.

3)~Диференційна рента II, що походить з послідовної витрати капіталу
на тій самій землі. Диференційна рента II підлягає дослідженню:

a) при сталій,

b) при низхідній,

c) при висхідній ціні продукції.

І далі:

d) перетворення надзиску в ренту.

4)~Вплив цієї ренти на норму зиску.

B.~Абсолютна рента.

C.~Ціна землі.

D.~Кінцеві зауваження про земельну ренту.

\plainbreak{3}

Як загальний наслідок розгляду диференційної ренти, виходить таке:

\emph{Перше:} Створення надзиску може відбуватися різними шляхами. З одного
боку, на базі диференційної ренти І, тобто на базі витрати всього
хліборобського капіталу на земельній площі, що складається з земель різної
родючости. Далі, як диференційна рента II, на базі різної диференційної продуктивности
послідовних витрат капіталу на тій самій землі, тобто на базі
більшої продуктивности, визначеної, наприклад, у квартерах пшениці, ніж та,
що постає при тій самій витраті капіталу на найгіршій землі, що не дає ренти,
але реґулює ціну продукції. Але хоч би як виникали ці надзиски, перетворення
їх у ренту, отже, їх перехід від орендаря до землевласника, завжди припускав
як попередню умову, що різні дійсні індивідуальні ціни продукції (тобто незалежно
від загальної ціни продукції, що реґулює ринок) окремих продуктів, окремих
послідовних витрат капіталу попередньо вирівнюються в індивідуальну
пересічну ціну продукції. Надмір загальної реґуляційної ціни продукції продукту
з акра над цією його індивідуальною пересічною ціною продукції становить
і визначає величину ренти на акр. При диференційній ренті І диференційні
наслідки розпізнаються сами по собі, бо вони постають на різних дільницях
землі, що лежать одна поза однією і одна біля однієї, — за такої витрати капіталу
на акр, яку береться за нормальну, і при відповідному до цієї витрати
нормальному обробленні. При диференційній ренті II їх спершу треба зробити
розрізнюваними; справді, вони мусять бути перетворені зворотно у диференційну
ренту І, а це можна зробити лише зазначеним способом. Візьмімо, наприклад,
ренту І, а це можна зробити лише зазначеним способом.
таблицю III, ст. 149.
%% TODO: add link

Земля $В$ дає в наслідок першої витрати капіталу в 2\sfrac{1}{2}\pound{ ф. стерл.} 2 квартери
з акра, а в наслідок другої витрати, однакової розміром, — 1\sfrac{1}{2} квартери; разом
3\sfrac{1}{2} квартери з того самого акра. За цими 3\sfrac{1}{2} квартерами, що виросли
на тій самій землі, не можна побачити, яка частина з них є продуктом витрати
капіталу І і яка витрати капіталу II.~Вони в дійсності становлять продукт
усього капіталу в 5\pound{ ф. стерл.}; і дійсний факт є лише в тому, що капітал
в 2\sfrac{1}{2}\pound{ ф. стерл.} дав 2 квартери, а капітал в 5\pound{ ф. стерл.} — не 4, а 3\sfrac{1}{2} квартери.
Справа ані трохи не змінилася б, якби ці 5\pound{ ф. стерл.} дали 4 квартери, так що
продукти обох витрат капіталу були б однакові, або навіть 5 квартерів, так
що друга витрата капіталу дала б надмір в 1 квартер. Ціна продукції перших
двох квартерів дорівнює 1\sfrac{1}{2}\pound{ ф. стерл.} за квартер, ціна продукції других 1\sfrac{1}{2} квартерів є 2\pound{ ф.
стерл.} за квартер. Ці 3\sfrac{1}{2} квартери разом коштують тому 6\pound{ ф. стерл}.
Це є індивідуальна ціна продукції всього продукту, а пересічно вона становить
1\pound{ ф. стерл.} 14\sfrac{2}{7}\shil{ шил.} за квартер, округло, скажімо, 1\sfrac{3}{4}\pound{ ф. стерл}. За загальної ціни
продукції в 3\pound{ ф. стерл.}, що визначається землею $А$, це дає надзиск в 1\sfrac{1}{4}\pound{ ф. стерл.}
на квартер і, отже, для 3\sfrac{1}{2} квартерів разом — 4\sfrac{3}{8}\pound{ ф. стерл}. За пересічної ціни
\parbreak{}  %% абзац продовжується на наступній сторінці

\parcont{}  %% абзац починається на попередній сторінці
\index{ii}{0177}  %% посилання на сторінку оригінального видання
часом і часом продукції, то й час зуживання вкладеного основного капіталу раз-у-раз переривається на
більш-менш протяжні періоди, як, напр., у хліборобстві при вживанні робочої худоби, знарядь праці та
машин. Оскільки цей основний капітал складається з робочої худоби, він потребує завжди однакових або
майже однакових витрат на корм і~\abbr{т. ін.}, все одно, чи в роботі вона, чи не в роботі. Щодо мертвих
засобів праці, то коли їх не вживається, вони теж дещо зневартнюються. Тому продукт взагалі
дорожчає, бо передачу вартости на продукт обчислюється не на той час, коли основний капітал
функціонує, але на той час, коли він втрачає вартість. В цих галузях продукції бездіяльність
основного капіталу, хоч сполучена вона з поточними витратами, хоч ні, становить так само умову
нормального його вжитку, як, наприклад, втрата певної кількости бавовни в процесі прядіння; так само
в кожному процесі праці непродуктивна — але неминуча — витрата робочої сили, що відбувається в
нормальних технічних умовах, береться на увагу так само, як і продуктивна. Кожне поліпшення, що
зменшує непродуктивну витрату засобів праці, сировинного матеріялу та робочої сили, зменшує також і
вартість продукту.

В сільському господарстві поєднуються й порівняно довгий робочий період і велика ріжниця між робочим
часом і часом продукції. Годскін слушно зауважує про це: „Ріжниця в часі (хоч він тут і не відрізняє
робочого часу й часу продукції), потрібному на те, щоб виготовити продукти в сільському
господарстві, і тим часом, що потрібен в інших галузях праці, є головна причина великої залежности
сільських господарств. Вони не можуть подавати свої товари на ринок раніше, ніж через рік. Протягом
цілого цього часу вони мусять боргуватись у шевця, кравця, коваля, колісника та різних інших
продуцентів, що їхніх продуктів вони потребують, і що їхні продукти можна виготувати протягом
небагатьох днів або тижнів. В наслідок цієї природної обставини і в наслідок швидкого збільшення
багатства в інших галузях праці, землевласники, що монополізували землю цілої держави, хоч вони,
крім цього, захопили й монополію законодавства, все ж таки не можуть врятувати себе й своїх слуг
фармерів від долі найбільш залежних людей в країні“. (Thomas Hodgskin, Popular Political Economy,
London, 1827, p. 147, примітка).

Всі методи, що ними в хліборобстві почасти рівномірніше розподіляється на цілий рік витрати на
заробітну плату й засоби праці, почасти скорочується оборот у наслідок культивування різноманітних
продуктів, яке уможливлює кілька зборів урожаю на рік, — всі ці методи потребують збільшення
авансовуваного обігового капіталу, витрачуваного на заробітну плату, добриво, насіння тощо. Так
буває, коли переходять від трипільного господарства з паром до сівозмінного без пару. Так буває у
Фляндрії при cultures dérobées\footnote*{
дослівно: „потайна культура“. Так зветься культура корінняків, що їх засівають
після збору основної культури; назва походить з того, що така культура, потребуючи менше часу,
вистигає між двома основними культурами, ніби потай. \emph{Ред.}
}“. „В culture dérobée застосовують корінняки; те саме поле спочатку
дає збіжжя, льон, рапс на задоволення потреб людини,
\index{ii}{0178}  %% посилання на сторінку оригінального видання
а по жнивах його засівають корінняками на годівлю худоби. Ця система, за якої рогата худоба
може ввесь час перебувати в стійлі, дає чималі запаси угноєння і стає таким чином за основу
сівозмінного господарства. В піскуватих місцевостях більше, ніж третину оброблюваної землі
відводиться під cultures dérobées, а наслідок такий, ніби оброблюваної землі побільшало на третину“.
Поряд корінняків тут культивують також конюшину та інші кормові рослини. „Рільництво доведене таким
чином до того пункту, де воно перетворюється на городництво, потребує, звичайно, порівняно чималого
основного капіталу (Anlagekapital). В Англії основний капітал обчислюється в 250 франків на гектар.
У Фляндрії основний капітал в 500 франків на гектар наше селянство, мабуть, визнало б за дуже
низький“. (Essais sur l’Economie Rurale de la Belgique par Emile de Laveleye. Paris, 1863, p. 59,
60, 63).

Візьмімо нарешті лісівництво. — „Продукція дерева посутньо відрізняється від більшости інших
продукцій тим, що тут сила природи діє самостійно і при природному поновленні не потребує сили
людської або сили капіталу. А проте, навіть там, де ліси розводять штучно, застосування сили
людської та капіталу порівняно з дією сил природи є лише незначне. Крім того ліс може добре рости на
таких ґрунтах і місцях, де хліб не удається або продукція його не оплачується. Але для
лісорозведення при правильному господарюванні потрібна також більша площа, ніж для культури хліба,
бо на маленьких парцелях не можна розбити ліс на правильні дільниці, побічних плодів майже не можна
використати, важче зберігати дерево й~\abbr{т. д.} Однак процес продукції тут сполучено також з такими
довгими періодами, що він виходить поза пляни приватного господарства, а іноді навіть поза межі
людського життя. Капітал, витрачений на закуп землі“ (при громадській продукції
цей капітал відпадає, і справа лише в тому,
скільки землі може громада відібрати під ліс від поля та пасовиська), „дає помітні плоди лише через
довгий час і обертається тільки почасти, а цілий оборот при деяких ґатунках дерев потребує до 150
років. Крім того, для правильної продукції дерева треба, щоб був запас живого дерева в 10--40 разів
більший, ніж щорічне споживання. Тому той, хто не має інших прибутків і посідає чимало площі лісу,
не може вести правильне лісове господарство“ (Kirchhof, р. 58).

Довгий час продукції (що має в собі відносно лише незначну частку робочого часу) і сполучений з ними
довгий період обороту робить лісівництво несприятливим для приватних, а значить, і для
капіталістичних підприємств, бо останні суттю своєю є приватні підприємства, хоча б замість
поодинокого капіталіста виступав капіталіст асоційований. Розвиток культури і взагалі промисловости
остільки енергійно виявив себе щодо знищення лісів, що порівняно з цим усе, зроблене ним для
підтримання й насадження лісу, є цілком незначна величина.

Особливо треба зауважити в цитаті Кірхгофа таке місце: „Крім того, для правильної продукції дерева
треба, щоб був запас живого дерева в 10--40 разів більший, ніж щорічне споживання“. — Отже, один
оборот дорівнює 10--40 і більше рокам.

\parcont{}  %% абзац починається на попередній сторінці 
\index{i}{0179}  %% посилання на сторінку оригінального видання 
вільний чи невільний, мусить до робочого часу, доконечного для
утримання себе самого, додавати надлишок робочого часу, щоб
продукувати засоби існування для власника засобів продукції,\footnote{
«Ті, що працюють... у дійсності годують і пенсіонерів, яких називають
багатими, і самих себе» («Those who labour... in reality feed both
the pensioners called the rieh, and themselves»). (Edmund Burke: «Thoughts and
Details on Scarcity». London, 1800, p. 2)
}
хоч буде цей власник атенський χαλος χαγανος,\footnote*{
— аристократ. Ред.
} етруський теократ,
civis romanus,** норманський барон, американський рабовласник,
волоський боярин, сучасний лендлорд або капіталіст.\footnote{
У своїй «Römische Geschichte» Нібур дуже наївно зауважує: «Нічого
таїти, що такі витвори, як етруські, що будять у нас подив навіть у
своїх руїнах, у маленьких (!) державах мають собі за передумову існування
панів і рабів». Багато глибше висловився Сісмонді, що «брюссельські
мережива» мають собі за передумову існування панів наймачів і найманих
робітників.
}
Та проте ясно, то коли в якійсь суспільній економічній формації
має перевагу не мінова вартість, а споживна вартість продукту,
то додаткова праця обмежується на вужчому або ширшому колі
потреб, але з самого характеру продукції не випливає безмежна
потреба додаткової праці. Тому ми находимо жахливу надмірну
працю в старовинному світі там, де йдеться про здобуття мінової
вартости в її самостійній грошовій формі — у продукції золота
й срібла. Поневільна, що тягне за собою смерть робітника, праця
є тут офіціяльна форма надмірної праці. Досить прочитати лише
Діодора Сіцілійського.\footnote{
«Не можна без жалю до їхньої злиденної долі дивитися на цих нещасних
(що працюють на копальнях золота між Єгиптом, Етіопією й
Арабією), які навіть не мають змоги подбати про чистоту свого тіла або
покрити свою голизну. Бо тут немає поблажливости, немає жалю до
хорих, покалічених, дідусів, до жіночої слабости. Всі мусять, приневолені
ударами київ, працювати й працювати аж доки смерть покладе кінець
їхнім мукам і злидням». (Diodorus Siculus: «Historische Bibliothek», Buch З,
kар. 13).
} Однак це є винятки у старовинному
світі. Але скоро тільки народи, що в них продукція рухається
ще в низьких формах рабської праці, панщини й т. ін., втягуються
у світовий ринок, опанований капіталістичним способом продукції,
в наслідок чого переважним інтересом для них стає продаж виробів
своєї продукції за кордон, — до варварських страхіть рабства,
кріпацтва й т. ін. прищеплюється цивілізоване страхіття надмірної
праці. Тому праця негрів у південних штатах Американського
союзу мала помірно-патріархальний характер доти, доки продукцію
зверталося головним чином на задоволення власних потреб.
Але в міру того як експорт бавовни стає життєвим інтересом цих
держав, в міру цього й надмірна праця негра, а в деяких місцях
споживання його життя протягом сімох робочих років, стає складовою
частиною байдужно обрахованої системи. Тут ішлося вже
не про те, щоб видушити з нього певну масу корисних продуктів.
Тут уже йшлося про продукцію самої додаткової вартости. Те ж
саме було з панщиною, приміром, у дунайських князівствах.

* * — римський громадянин. Ред.

\input{_0180c.tex}
\parcont{}  %% абзац починається на попередній сторінці
\index{iii1}{0181}  %% посилання на сторінку оригінального видання
плату плюс додаткову вартість, додаткову працю понад їх необхідні
потреби, при чому, однак, результати її належали б їм
самим. Висловлюючись капіталістичною мовою, обидва робітники
одержують рівну заробітну плату плюс рівний зиск, але разом
з тим і вартість, виражену, наприклад, у продукті десятигодинного
робочого дня. Але, поперше, вартості їх товарів були б
різні. Нехай, наприклад, з уміщеної в товарі І вартості на спожиті
засоби виробництва припадає більша частина вартості, ніж у товарі
II, і — щоб урахувати тут усі можливі ріжниці — припустімо,
що товар І вбирає більше живої праці, отже, потребує довшого
робочого часу для свого виготовлення, ніж товар II.~Отже, вартість
цих товарів І і II дуже різна. Так само різні й суми товарних вартостей,
які є продуктом праці, виконаної за даний час робітником
І і робітником II.~Норми зиску теж дуже різні для І і II,
якщо ми назвемо тут нормою зиску відношення додаткової вартості
до всієї вартості витрачених засобів виробництва. Засоби
існування, які щодня споживаються робітниками І і II протягом
виробництва і які представляють заробітну плату, становлять
тут ту частину авансованих засобів виробництва, яку ми в інших
випадках звемо змінним капіталом. Але додаткові вартості за
однаковий робочий час були б для І і II однакові, або ще точніше:
через те що І і II одержують кожний вартість продукту
одного робочого дня, вони одержують — якщо відрахувати вартість,
авансованих „сталих“ елементів — однакові вартості, одну
частину яких можна розглядати як заміщення спожитих у виробництві
засобів споживання, а другу — як додаткову вартість,
яка лишається понад це. Якщо І зробив більше витрат, то вони
заміщаються більшою частиною вартості його товару, яка заміщає
цю „сталу“ частину, і тому він повинен також більшу
частину всієї вартості свого продукту перетворити знову в речові
елементи цієї сталої частини, тимчасом як II, якщо він
менше одержав як заміщення, повинен зате настільки ж менше
знову перетворити в елементи сталої частини. Отже, при цьому
припущенні ріжниця в нормах зиску була б байдужою обставиною,
цілком так само, як нині для найманого робітника байдуже,
в якій нормі зиску виражається видушена з нього кількість
додаткової вартості, і цілком так само, як у міжнародній торгівлі
ріжниця норм зиску у різних націй є байдужа обставина
для їх товарообміну.

Отже, для обміну товарів по їх вартостях, або приблизно
по їх вартостях, потрібен значно нижчий ступінь, ніж для обміну
по цінах виробництва, для якого потрібна певна висота капіталістичного
розвитку.

Яким би чином не встановлювались або реґулювались первісно
ціни різних товарів одного відносно одного, закон вартості
керує їх рухом. Де зменшується робочий час, потрібний для
виробництва товарів, там падають і ціни; де він збільшується,
там підвищуються, при інших незмінних умовах, і ціни.

\parcont{}  %% абзац починається на попередній сторінці
\index{iii2}{0182}  %% посилання на сторінку оригінального видання
капіталу, яка продукувала б квартер дорожче за 3\pound{ ф. стерл.}, призвела б до
скорочення його зиску. Це, за недостатньої продуктивности, перешкоджає вирівнянню індивідуальної
пересічної ціни.

Візьмімо цей випадок у колишньому прикладі, коли ціна продукції на
землі $А$ в 3\pound{ ф. стерл.} за квартер реґулює ціну для землі $В$.

\begin{table}[h]
  \begin{center}
    \footnotesize

  \begin{tabular}{c@{  } c@{  } c@{  } c@{  } c@{  } c@{  } c@{  } c@{  } c@{  } c@{  } c}
    \toprule
      Капітал &
      Зиск &
      \makecell{Ціна \\ продукції} &
      Здобуток &
      \makecell{Ціна \\ продукції \\ за кварт.} &
      \multicolumn{2}{c}{Продажна ціна} &
      Надзиск &
      Витрата \\

      \cmidrule(r){1-1}
      \cmidrule(r){2-2}
      \cmidrule(r){3-3}
      \cmidrule(r){4-4}
      \cmidrule(r){5-5}
      \cmidrule(r){6-7}
      \cmidrule(r){8-8}
      \cmidrule(r){9-9}\pound{ ф. ст.} & ф. ст. & ф. ст. & Кварт & ф. ст. & \makecell{за квартер \\ ф. ст.} & \makecell{разом \\ ф. ст.} & ф. ст. & ф. ст.   \\
      \midrule
      \phantom{0}2\sfrac{1}{2}           & \phantom{1}\sfrac{1}{2} & \phantom{0}3 & 2\phantom{\sfrac{1}{2}} & 1\sfrac{1}{2}           & 3 & \phantom{0}6\phantom{\sfrac{1}{2}} & 3\phantom{\sfrac{1}{2}} & \textemdash \\
      \phantom{0}2\sfrac{1}{2}           & \phantom{1}\sfrac{1}{2} & \phantom{0}3 & 1\sfrac{1}{2}           & 2\phantom{\sfrac{1}{2}} & 3 & \phantom{0}4\sfrac{1}{2}           & 1\sfrac{1}{2}           & \textemdash \\
      \phantom{0}5\phantom{\sfrac{1}{2}} & 1\phantom{\sfrac{1}{2}} & \phantom{0}6 & 1\sfrac{1}{2}           & 4\footnotemarkZ{}\phantom{1}  & 3 & \phantom{0}4\sfrac{1}{2}           & \textemdash             & 1\sfrac{1}{2}           \\
      \phantom{0}5\phantom{\sfrac{1}{2}} & 1\phantom{\sfrac{1}{2}} & \phantom{0}6 & 1\phantom{\sfrac{1}{2}} & 6\phantom{\sfrac{1}{2}} & 3 & \phantom{0}3\phantom{\sfrac{1}{2}} & \textemdash             & 3\phantom{\sfrac{1}{2}} \\

     \cmidrule(r){1-1}
     \cmidrule(r){2-2}
     \cmidrule(r){3-3}
     \cmidrule(r){7-7}
     \cmidrule(r){8-8}
     \cmidrule(r){9-9}

       15\phantom{\sfrac{1}{2}} & 3\phantom{\sfrac{1}{2}} & 18 & & & & 18\phantom{\sfrac{1}{1}} & 4\sfrac{1}{2} & 4\sfrac{1}{2} \\
  \end{tabular}

  \end{center}
\end{table}
\footnotetextZ{В німецькому тексті тут стоїть «3». Очевидна помилка. \Red{Пр.~Ред}}

Ціна продукції 3\sfrac{1}{2} квартерів з перших двох витрат капіталу так само
становить для орендаря 3\pound{ ф. стерл.} за квартер, бо йому доводиться виплачувати
ренту в 4\sfrac{1}{2}\pound{ ф. стерл.}, при чому ріжниця між його індивідуальною ціною продукції
і загальною ціною продукції іде, таким чином, не в його кишеню. Отже,
надмір ціни продукту перших двох витрат не може йому покрити дефіциту
в продуктах третьої і четвертої витрати капіталу.

1\sfrac{1}{2} квартера від витрати капіталу 3) коштують орендареві, включаючи
і зиск, 6\pound{ ф. стерл.}; але за реґуляційної ціни в 3\pound{ ф. стерл.} за квартер він може
продати їх лише за 4\sfrac{1}{2}\pound{ ф. стерл}. Отже, він втратив би не тільки ввесь
зиск, але й понад нього \sfrac{1}{2}\pound{ ф. стерл.} або  10\% вкладеного капіталу в 5\pound{ ф. стерл}.
Його втрата з зиску і капіталу при витраті капіталу 3) дорівнювала б  1\sfrac{1}{2}\pound{ ф. стерл.}, а при витраті капіталу 4) — 3\pound{ ф. стерл.}, разом 4\sfrac{1}{2}\pound{ ф. стерл.}, якраз
стільки, скільки становить рента від продуктивніших витрат капіталу, що їхня
індивідуальна ціна продукції саме тому не може справити вирівнювального
впливу на індивідуальну пересічну ціну продукції всього продукту землі $В$, що
надмір його доводиться виплатити як ренту третій особі.

Коли б для задоволення попиту довелося випродукувати додаткові 1\sfrac{1}{2}
квартери з допомогою третьої витрати капіталу, то регуляційна ринкова ціна
мусила б піднестись до 4\pound{ ф. стерл.} за квартер. В наслідок цього підвищення
реґуляційної ринкової ціни рента на землі $В$ для першої і другої витрати капіталу
підвищилась би, а на землі $А$ створилася б рента.

Отже, хоч диференційна рента є лише формальне перетворення надзиску
в ренту, і хоч власність на землю дає тут власникові лише можливість перемістити
надзиск з рук орендаря в свої, проте виявляється, що послідовна витрата
капіталу на ту саму земельну площу або, що сходить на те саме, збільшення
капіталу, витраченого на тій самій земельній площі, за низхідної норми
продуктивности капіталу і незмінної реґуляційної ціни геть швидше доходить
до своєї межі, отже, досягає в дійсності більш або менш штучної межі, в наслідок
просто формального перетворення надзиску в земельну ренту, що є наслідком
земельної власности. Отже підвищення загальної ціни продукції, яке стає
тут доконечним за вужчих меж, ніж в інших умовах, є тут не тільки за причину
\parbreak{}  %% абзац продовжується на наступній сторінці

\parcont{}  %% абзац починається на попередній сторінці
\index{iii2}{0183}  %% посилання на сторінку оригінального видання
підвищення диференційної ренти, але саме існування диференційної ренти як
ренти є разом з тим причина ранішого й швидшого підвищення загальної ціни
продукції, щоб таким чином забезпечити збільшене подання продукту, яке стало доконечним.

Треба зауважити далі таке:

Через додаткове капіталовкладення в землю $В$ не могла б підвищитися
регуляційна ціна до 4\pound{ ф. стерл.}, як це наведено вище, коли б земля $А$, в наслідок
другої витрати капіталу, давала додаткову продукцію дешевше від 4\pound{ ф.
стерл.}, або коли б вступила в конкуренцію нова гірша, ніж $А$, земля, що на
ній ціна продукції була б хоч і вища 3, але нижча 4\pound{ ф. стерл}. Таким чином,
ми бачимо, як диференційна рента І і диференційна рента II, тимчасом як
перша є за базу для другої, одночасно правлять одна для однієї за межу, що
спричинює то послідовні витрати капіталу на тій самій земельній дільниці, то
витрати капіталу одну біля однієї на новій додатковій землі. Так само вони
обмежують одна одну і в інших випадках, коли, наприклад, черга доходить до
кращих земель.

\section{Диференційна рента і з найгіршої з оброблюваних земель}

Припустімо, що попит на збіжжя підвищується, і що подання може бути
задоволене лише через послідовні витрати капіталу з недостатньою продуктивністю
на землях, що дають ренту, або через додаткову витрату капіталу
теж з низхідною продуктивністю на землі $А$, або через витрату капіталу на
нових землях гіршої якости, ніж $А$.

Візьмімо землю $В$ як представницю земель, що дають ренту.

Щоб уможливити додаткову продукцію 1 квартера на землі $В$ (який
тут може становити 1 мільйон квартерів, як кожен акр — мільйон акрів), додаткове
капіталовкладення вимагає підвищення ринкової ціни понад 3\pound{ ф. стерл.}
за квартер, що були до цього часу за регуляційну ціну. На землях $C$ і $D$ і~\abbr{т.
ін.} родах землі з найвищою рентою, теж може бути випродуковано додатковий
продукт, але лише з низхідною додатковою продуктивною силою; проте,
припускається, що 1 квартер землі $В$ потрібен для задоволення попиту.
Коли цей один квартер можна дешевше випродукувати з допомогою додаткового
капіталу на $В$, ніж з допомогою рівної витрати додаткового капіталу
на $А$, або спускаючись до землі $А_{-1}$, яка може випродукувати квартер, наприклад,
лише за 4\pound{ ф. стерл.}, тимчасом як додатковий капітал на $А$ міг би випродукувати
квартер уже за 3\sfrac{3}{4}\pound{ ф. стерл.}, то додатковий капітал, витрачений на
$В$, почав би регулювати ринкову ціну.

Земля $А$, як і давніш, випродукувала 1 квартер за 3\pound{ ф. стерл.} $В$ теж,
як і давніш, випродукувала в цілому 3\sfrac{1}{2} квартера, що їхня індивідуальна ціна
продукції становить разом 6\pound{ ф. стерл}. Тепер, коли б на землі $В$ потрібно було
додаткової витрати в 4\pound{ ф. стерл.} ціни продукції (включаючи і зиск), щоб випродукувати ще 1 квартер,
тимчасом як на $А$ його можна випродукувати за
3\sfrac{3}{4}\pound{ ф. стерл.}, то, зрозуміла річ, він був би випродукований на $А$, а не на $В$.
Отже, припустімо, що він може бути випродукований на $В$ з 3\sfrac{1}{2}\pound{ ф. стерл.}
додаткової ціни продукції. В цьому випадку 3\sfrac{1}{2}\pound{ ф. стерл.} були б регуляційною ціною
для всієї продукції. Тоді $В$ продав би свій теперішній продукт в 4\sfrac{1}{2}  квартери за
15\sfrac{3}{4}\pound{ ф. стерл}. З цього на ціну продукції перших 3\sfrac{1}{2}  квартерів припадає
6\pound{ ф. стерл.} і на останній квартер 3\sfrac{1}{2}\pound{ ф. стерл.}, разом 9\sfrac{1}{2}\pound{ ф. стерл}. Лишається
надзиск, для ренти \deq{} 6\sfrac{1}{4}\pound{ ф. стерл}, проти лише 4\sfrac{1}{2}\pound{ ф. стерл.} колишніх.
В цьому випадку акр землі $А$ також дав би ренту в  \sfrac{1}{2}\pound{ ф. стерл.};
\parbreak{}  %% абзац продовжується на наступній сторінці

\parcont{}  %% абзац починається на попередній сторінці
\index{iii2}{0184}  %% посилання на сторінку оригінального видання
але ціну продукції в 3 1/2 ф. стерл. реґулювала б не найгірша земля А, а краща
земля В. Звичайно, при цьому припускається, що нова земля якости А такого
ж зручного положення, як оброблювана до цього часу, є неприступна, і що довелося
би зробити другу витрату капіталу на вже оброблюваній дільниці А, але
з більшою ціною продукції, або довелось би притягнути до обробітку ще гіршу
землю А1. Коли в наслідок послідовних витрат капіталу починає діяти днференційна
рента II, то може статися, що межі підвищуваної ціни продукції реґулюватимуться
кращою землею і гірша земля, база диференційної ренти І, тоді
теж може давати ренту. Таким чином, при наявності самої лише диференційної
ренти всі оброблювані землі почали б тоді давати ренту. Ми мали б тоді такі
дві таблиці, в яких під ціною продукції розуміється суму авансованого капіталу
плюс 20\%  зиску, отже на кожні 2 1/2 ф. стерл. капіталу по 1/2 ф. стерл.
зиску, разом 3 ф. стерл. (див. табл. І).

Таке становище
речей перед новою
витратою капіталу в
3 1/2 ф стерл. на В,
що дає тільки 1 квартер.
Після цієї витрати
капіталу справа
стоїть так: (див.
табл. II).

[Це знов не зовсім
вірно обчислено.
Орендареві В продукція
цих 4 1/2 квартерпів
коштує, по-перше,
9 1/2 ф. стерл.
ціни продукції і, по-друге,
41/2 ф. стерл.
ренти, разом 14 ф
стерл.; пересічно за
квартер 31/9 ф. стерл.
Ця пересічна ціна
всієї його продукції
стає через це за реґуляційну
ринкову
ціну. Тому рента на
А становила б 1/9 ф.
стерл. замість 1/2 ф. стерл., а рента на В лишалася б, як і давніш, 4 1/2 ф.
стерл.: 4 1/2 квартерн по 3 1/2 ф. стерл. = 14 ф. стерл., звідси вирахувати
9 1/2 ф. стерл. ціни продукції, лишається надзиск в 41/2 ф. стерл. Бачимо: не
зважаючи на змінені числа, приклад показує, як з допомогою диференційноі
ренти II краща земля, що вже дає ренту, може стати за регуляційну щодо ціни,
і через це вся земля, також і та, що до того часу не давала ренти, може перетворитися
в рентодайну. — Ф. Е.].

Збіжжева рента мусить підвищитись, скоро підвищується реґуляційна
ціна продукції збіжжя, отже, скоро підвищується ціна продукції квартера збіжжя
на реґуляційній землі, або реґуляційна витрата капіталу на одному з родів
землі. Це все одно, як коли б усі роди землі стали менш плодючі і продукували
б, наприклад, на кожні 21/2 ф. стерл. нових витрат капіталу лише по 5/7
квартера замість 1 квартера. Весь надмір збіжжя, що його вони продукують

Рід землі    Акри    Ціна продукції    Продукт в кварт. Продажна  ціна    Грошовий  здобуток   
Збіжжева рента    Грошова  рента
        ф. стер. ф. стер. ф. стер. ф. стер. ф. стер
А 1 3 1 3 3 0 0
В                    1    6    3 1/2    3    101/2    11/2    41/2
С                    1    6    51/2    3    16 1/2    31/2    101/2
D                    1    6    71/2    3    221/2    51/2    161/2
Разом          4    21    171/2        52 1/2    101/2    311/2

Рід землі    Акри    Ціна продукції    Продукт в кварт. Продажна  ціна    Грошовий  здобуток   
Збіжжева рента    Грошова  рента
        ф. стер. ф. стер. ф. стер. ф. стер. ф. стер
А                    1    3             1            31/2       31/2        1/7            1/2
В                    1    9 1/2       41/2    31/2       153/4        111/14    61/4
С                    1    6             51/2    31/2        191/2        311/14    131/4
D                    1    6             71/2    31/2        261/2        511/14     201/4
Разом           4    241/2    181/2           643/4    111/2            401/4
\parbreak{}  %% абзац продовжується на наступній сторінці

\parcont{}  %% абзац починається на попередній сторінці
\index{iii2}{0185}  %% посилання на сторінку оригінального видання
з тією самою витратою капіталу, перетворюється на надпродукт, який репрезентує
надзиск, а тому й ренту. Коли припустити, що норма зиску лишається
та сама, то орендар міг би купити на свій зиск меншу кількість збіжжя.
Норма зиску може лишитись та сама, коли заробітна плата не підвищиться,
або тому, що її понижено до фізичного мінімуму, отже, нижче нормальної вартости
робочої сили; або тому, що інші речі споживання робітників, давані мануфактурою,
стали порівняно дешевші; або тому, що робочий день став довший
або зробився інтенсивніший, і тому норма зиску в нехліборобських галузях
продукції, яка проте, реґулює хліборобський зиск, лишилась незмінна, якщо
тільки не підвищилась; абож тому, що хоч у хліборобстві й витрачається такий
самий капітал, але більш сталого і менше змінного.

Ми тут розглянули перший спосіб, у який може постати рента на землі
$А$, що до цього часу була найгірша, без того, щоб притягалось до оброблення
ще гіршу землю; а саме, коли вона постає в наслідок ріжниці індивідуальної
ціни продукції на цій землі, — ціни продукції, що до цього часу була за
реґуляційну проти тієї нової, вищої ціни продукції, по якій останній додатковий
капітал, витрачений з недостатною продуктивною силою на кращій землі,
дав потрібну додаткову кількість продукту.

Коли додаткова продукція мусила б постачатись землею $А_{-1}$, яка може дати
квартер лише за 4\pound{ ф. стерл.}, то рента з акра на $А$ підвищилася б до 1\pound{ ф. стерл}. Але в цьому випадку
земля $А_{-1}$ пересунулася б на місце $А$, як
найгірша з культивованих земель, а земля $А$ вступила б як нижчий член в
ряд родів землі, що дають ренту. Диференційна рента I змінилася б. Отже,
цей випадок лежить поза аналізою диференційної ренти II, яка виникає з різної
продуктивности послідовних витрат капіталу на тій самій дільниці землі.

Але, крім того, диференційна рента на землі $А$, може постати ще двояким
способом:

Коли за незмінної ціни, — будь-якої ціни, хоч би вона і була знижена
проти колишньої, — додаткова витрата капіталу породжує додаткову продуктивність,
що prima facie до певної межі завжди мусить статися якраз на найгіршій
землі.

\looseness=-1
Подруге, тоді, коли навпаки, продуктивність послідовних витрат капіталу
на землі $А$ понижується.

\looseness=-1
В обох випадках припускається, що стан попиту потребує збільшення
продукції.

\looseness=-1
Але, з погляду диференційної ренти, тут виступає специфічна трудність
в зв’язку з раніш викладеним законом, що за ним визначальною для всієї продукції
(або для всієї витрати капіталу) завжди є індивідуальна пересічна ціна
продукції одного квартера. Але для землі $А$, у протилежність кращим родам
землі, ціна продукції, яка обмежує для нових витрат капіталу вирівняння індивідуальної
з загальною ціною продукції, дана не поза нею. Бо індивідуальна ціна
продукції на $А$ і є та сама загальна ціна продукції, що реґулює ринкову ціну.

Припустімо:

1)~За висхідної продуктивної сили послідовних витрат
капіталу на одному акрі землі $А$, з авансованим капіталом в 5\pound{ ф. стерл.},
відповідно 6\pound{ ф. стерл.} ціни продукції, можна випродукувати замість 2 квартерів
3 квартери. Перша витрата капіталу в 2\sfrac{1}{2}\pound{ ф. стерл.} дає 1 квартер, друга — 2 квартери. В цьому
випадку 6\pound{ ф. стерл.} ціни продукції дають 3 квартери,
отже, квартер коштуватиме пересічно 2\pound{ ф. стерл.}; отже, коли 3 квартери
будуть продані по 2\pound{ ф. стерл.}, то $А$, як і давніш, не дасть ренти, але зміниться
лише основа диференційної ренти II; за реґуляційну ціну продукції стали
2\pound{ ф. стерл.} замість 3\pound{ ф. стерл.}; на найгіршій землі капітал в 2\sfrac{1}{2}\pound{ ф. стерл.}
продукує тепер пересічно 1\sfrac{1}{2}  замість 1 квартера, і це тепер офіційна родючість
\parbreak{}  %% абзац продовжується на наступній сторінці

\input{_0186.tex}
\parcont{}  %% абзац починається на попередній сторінці
\index{iii2}{0187}  %% посилання на сторінку оригінального видання
ціни продукції пересічною ціною продукції з $А$; отже, це тримало б ціну продукції
на вищому рівні, ніж це потрібно, і таким чином створило б ренту.
Навіть при вільному довозі хліба з-за кордону такий результат міг би скластись
або триматись, бо орендарі вимушені були б для землі, яка при зовні
визначеній ціні продукції могла б конкурувати у продукції збіжжя, не даючи
ренти, дати інше призначення, наприклад, призначити її під пасовисько, і таким
чином лише рентодайні землі були б зайняті під збіжжя, тобто лише землі,
на яких індивідуальна пересічна ціна продуктції за квартер була б нижча від
ціни продукції, визначуваної зовні. В цілому можна визнати, що в даному випадку
ціна продукції понизиться, але не до рівня пересічної ціни, і буде стояти
вище від неї, але нижче ціни продукції на гірше оброблюваній землі $А$, так
що конкуренцію нової землі $А$ буде обмежено.

\emph{2) За низхідної продуктивної сили додаткових капіталів}

Припустімо, що земля $А_{-1}$ могла б випродукувати додатковий квартер
лише за 4 ф. стерл., а земля $А$ за 3\sfrac{3}{4}, отже дешевше ніж $А_{-1}$ але на \sfrac{3}{4}
ф. стерл. дорожче, ніж квартер, випродукований першою витратою капіталу на
$А$. В цьому випадку вся ціна двох випродукованих на $А$ квартерів була б =
6\sfrac{3}{4} ф. стерл.; отже, пересічна ціна за квартер = 3\sfrac{3}{8} ф. стерл. Ціна продукції
пидвищилася б, але лише на \sfrac{3}{8} ф. стерл., тимчасом як коли б додатковий
капітал був витрачений на новій землі, яка продукує квартер за 3\sfrac{3}{4}
ф. стерл., вона підвищилася б на дальші \sfrac{3}{8} ф. стерл. до 3\sfrac{3}{4} ф. стерл., і цим
було б спричинене відповідне підвищення усіх інших диференційних рент.

Ціна продукції в 3\sfrac{3}{8} ф, стерл. за квартер на землі $А$ таким чином
вирівнялася б за пересічною ціною продукції на тій самій землі за збільшеної
витрати капіталу і стала б реґуляційною; отже, вона не дала б ренти, бо не
було б надзиску.

Але коли б цей квартер, випродукований другою витратою капіталу, був проданий
за 3\sfrac{3}{4} ф. стерл., то земля $А$ дала б тепер ренту в \sfrac{3}{4} ф. стерл.,
дала б її також і на всі акри А, на яких не зроблено додаткової витрати і
які, отже, як і давніш, продукують квартер за 3 ф. стерл. Поки існують ще
необроблені дільниці землі $А$, ціна могла б лише тимчасове підвищитись до
3\sfrac{3}{4} ф. стерл. Конкуренція нових дільниць $А$ підтримувала б ціну продукції
на 3 ф. стерл., поки не були б вичерпані всі землі $А$, що їхнє сприятливе положення
дає їм можливість продукувати квартер дешевше, ніж за 3\sfrac{3}{4} ф. стерл.
Отже, доводиться припустити це, хоч власник землі, коли один акр землі дає ренту,
не відступить орендареві другого акра без ренти.

Чи вирівняється ціна продукції відповідно до пересічної ціни, чи зареґуляційну
зробиться індивідуальна ціна продукції другої витрати капіталу в 3\sfrac{3}{4} ф. стерл.,
це залежить знов таки від того, більшого чи меншого загального поширення набула
друга витрата капіталу на наявній землі $А$. За реґуляційну ціну стає 3\sfrac{3}{4}
ф. стерл. тільки в тому випадку, коли у землевласника є досить часу для того,
щоб фіксувати як ренту той надзиск, який одержувано б при ціні в 3\sfrac{3}{4} ф.
стерл. за квартер, поки не задовольниться попиту.

\pfbreak

Щодо низхідної продуктивности землі за послідовних витрат капіталу,
слід подивитися Лібіха. Ми бачили, що послідовне зменшення додаткової продуктивної
сили витрат капіталу постійно збільшує ренту з акра, коли ціна
продукції не змінюється, і що воно може призвести до цього навіть за низхідної
ціни продукції.

Але взагалі треба відзначити таке:

З погляду капіталістичного способу продукції відносне подорожчання
продукту відбувається завжди, коли для одержання того самого продукту
\parbreak{}  %% абзац продовжується на наступній сторінці

\input{_0188c.tex}
\index{iii2}{0189}  %% посилання на сторінку оригінального видання
Одно з найкумедніших явищ є в тому, що всі противники Рікардо, які
заперечують визначення вартости виключно працею, в справі з диференційною
рентою, що випливає з ріжниць землі, надають ваги тому, що тут вартість
визначається природою, а не працею; і одночасно приписують це визначення
положенню, або, і ще більше, процентові на капітал, вкладений в землю при
обробітку. Та сама праця дає однакову вартість для продукту, створеного
протягом даного часу; але величина або кількість цього продукту, отже, і та
частина вартости, яка припадає на відповідну частину цього продукту за даної
кількости праці, залежить єдино від кількости продукту, а це знову від продуктивности
даної кількости праці, не від величини цієї кількости. Чи завдячує
ця продуктивність своїм походженням природі, чи суспільству — цілком байдуже.
Тільки в тому разі, коли вона сама коштує праці, отже, капіталу, вона
збільшує ціну продукції новою складовою частиною, чого природа сама по собі
не робить.

\section{Абсолютна земельна рента}

Аналізуючи диференційну ренту, ми виходили з припущення, що найгірша
земля не виплачує земельної ренти, або, висловлюючись загальніше, що земельну
ренту виплачує тільки така земля, для продукту якої індивідуальна ціна продукції
нижча від ціни продукції, що реґулює ринок, так що в такий спосіб
виникає надзиск, що перетворюється на ренту. Потрібно насамперед зауважити,
що закон диференційної ренти, як днференційної ренти, зовсім не залежить від
правильности чи неправильности того припущення.

Коли загальну ціну продукції, що реґулює ринок, ми назвемо Р, то Р для
продукту найгіршого роду землі А збігається з індивідуальною ціною продукції
на цій землі; тобто вона оплачує зужиткований у продукції сталий і змінний капітал
плюс пересічній зиск (= підприємницькому баришеві плюс процент).

Рента тут дорівнює нулеві. Індивідуальна ціна продукції найближчого
кращого роду землі В = Р', і Р>Р'; тобто Р оплачує більше, ніж дійсну
ціну продукції продукту на клясі землі В. Хай тепер Р — Р' = d; тому
d, надмір Р над Р', є той надзиск, що його добуває орендар з цієї кляси В.
Це d перетворюється на ренту, яку доводиться виплачувати власникові землі.
Хай для третьої кляси землі С за дійсну ціну продукції буде Р", і хай Р —
Р'' = 2d; отже, ці 2d перетворюються на ренту; так само для четвертої кляси
D індивідуальна ціна продукції хай буде Р'", а Р — Р'" = 3d, які перетворюються
на земельну ренту і т. д. Даймо тепер, що припущення, ніби для
кляси землі А рента = 0, а тому ціна її продукту = Р + 0, помилкове. Хай,
навпаки, і вона дає ренту = г. В цьому випадку маємо двоякі наслідки.

\emph{Поперше}: ціна продукту землі кляси А не реґулювалася б ціною продукції
на цій землі, а мала б деякий надмір над цією ціною, вона була б =
P — r. Бо, коли припускається, нормальний перебіг капіталістичного способу
продукції, отже, коли припускається, що надмір r, виплачуваний від орендаря
земельному власникові, не становить вирахування ані з заробітної плати, ані
з пересічного зиску на капітал, то орендар може виплачувати його лише тому,
що його продукт продається понад ціну продукції, що він, отже, дав би йому
надзиск, коли б не доводилося відступати цей надмір у формі ренти земельному
власникові. Реґуляційна ринкова ціна всього наявного на ринку продукту
всіх родів землі була б тоді не та ціна продукції, яку дає капітал взагалі
у всіх сферах продукції, тобто не ціна рівна витратам плюс пересічний
зиск, а була б ціною продукції плюс рента, Р + r, не Р. Бо ціна продукту
кляси А визначає взагалі межу реґуляційної загальної ринкової ціни, тієї ціни,
\parbreak{}  %% абзац продовжується на наступній сторінці

\parcont{}  %% абзац починається на попередній сторінці
\index{iii1}{0190}  %% посилання на сторінку оригінального видання
абож хоч і в тому самому напрямі, але не в тій самій мірі,
одним словом, якщо відбуваються двосторонні зміни, які, однак,
змінюють попереднє відношення між обома сторонами, то кінцевий
результат завжди мусить звестись до одного з двох
вищерозглянутих випадків.

Справжня трудність при загальному визначенні понять попиту
і подання полягає в тому, що визначення це, здається, зводиться
до тавтології. Розгляньмо спочатку подання, тобто продукт,
який перебуває на ринку або може бути приставлений на
ринок. Для того, щоб не вдаватись до цілком зайвих тут
деталей, візьмімо тут масу річної репродукції в кожній даній
галузі промисловості і залишмо при цьому осторонь те, що різні
товари в більшій чи меншій мірі можуть забиратися з ринку
і нагромаджуватись для споживання, скажемо, ближчого року.
Ця річна репродукція виражає насамперед певну кількість, міру
або число, залежно від того, як виміряється товарна маса, —
окремими екземплярами, чи як суцільна величина; це — не тільки
споживні вартості, що задовольняють людські потреби, але і такі
споживні вартості, що перебувають на ринку в певній даній
кількості. Подруге, ця кількість товарів має певну ринкову
вартість, яку можна виразити як кратне ринкової вартості товару
або товарної міри, що служать одиницями. Тому між
кількістю товарів, що перебувають на ринку, і їх ринковою
вартістю не існує ніякого необхідного зв’язку; тимчасом, наприклад,
як деякі товари мають специфічно високу вартість,
інші мають специфічно низьку вартість, так що дана сума вартості
може виразитись у дуже великій кількості одних і в дуже
незначній кількості інших товарів. Між кількістю товарів, що
перебувають на ринку, і ринковою вартістю цих товарів існує
тільки такий зв’язок: на даній базі продуктивності праці виготовлення
певної кількості товарів вимагає в кожній окремій
сфері виробництва певної кількості суспільного робочого часу,
хоч у різних сферах виробництва це відношення є цілком різне
і не стоїть ні в якому внутрішньому зв’язку з корисністю цих
товарів або специфічною природою їх споживних вартостей.
При всіх інших однакових умовах, якщо кількість a даного
сорту товарів коштує b робочого часу, то кількість na коштує
nb робочого часу. Далі: оскільки суспільство хоче задовольнити
потреби, хоче щоб для цієї мети був вироблений товар, воно мусить
його оплатити. Дійсно, оскільки при товарному виробництві
передбачається поділ праці, то суспільство купує ці товари,
вживаючи на їх виробництво частину робочого часу, який
є в його розпорядженні, отже, купує їх за допомогою певної
кількості робочого часу, яким воно може порядкувати. Та частина
суспільства, якій в наслідок поділу праці припадає вживати
свою працю на виробництво цих певних товарів, мусить
дістати еквівалент у суспільній праці, представленій у товарах,
які задовольняють її потреби. Але не існує ніякого необхідного,
\parbreak{}  %% абзац продовжується на наступній сторінці

\parcont{}  %% абзац починається на попередній сторінці
\index{iii2}{0191}  %% посилання на сторінку оригінального видання
земельного власника зовсім не підстава для того, щоб даром передати свою землю
до розпорядження орендареві і, виявши філантропічне ставлення до цього в
справах приятеля, запровадити crédit gratuit\footnote*{
Безплатний кредит. \emph{Прим. Ред.}
}. Таке припущення має в собі
абстрагування від земельної власности, знищення земельної власности, що її
існування саме і ставить межу для приміщення капіталу і вільного використовування
його на землі, — межу, яка зовсім не відпадає від самого міркування
орендаря, що стан збіжжевих цін дозволив би йому здобути з свого
капіталу з допомогою експлуатації землі роду $А$ звичайний зиск, коли б йому
не довелося виплачувати ренти, тобто коли б він міг на практиці ставитись до
земельної власности так, наче б її не існувало. Але монополію земельної власности,
земельну власність як межу капіталу припускається диференційною
рентою, бо без цього надзиск не перетворився б на земельну ренту, і не дістався
б земельному власникові замість орендареві. І земельна власність як межа,
продовжує існувати і там, де рента як диференційна рента відпадає, тобто на
землі $А$. Якщо ми розглянемо випадки, коли в країні капіталістичної продукції
капітал може вкладатися в землю без виплати ренти, то ми знайдемо, що всі
вони включають хоч і не юридичне, то фактичне знищення власности на землю,
знищення, яке, проте, може статися лише за цілком певних і своєю природою
випадкових обставин.

\emph{Перше}. Коли земельний власник сам є капіталіст, або капіталіст сам є
земельний власник. Коли ринкова ціна піднеслась так високо, що на тому,
що є тепер землею роду $А$, можна здобути ціну продукції, тобто покриття капіталу
плюс пересічний зиск, то він може в цьому випадку \emph{сам господарювати}
на своїй дільниці землі. Але чому? Тому, що у відношенні до нього земельна
власність не створює будь-якої межі для приміщення його капіталу.
Він може обробляти землю як простий елемент природи і тому він може керуватися
виключно міркуваннями про використання свого капіталу, капіталістичними
міркуваннями. Такі випадки трапляються на практиці, але тільки як винятки.
Так само як капіталістичне оброблення землі має за передумову роз’єднення капіталу,
що функціонує, і земельної власности, цілком так само воно виключає як
загальне правило провадження господарства самим земельним власником. Одразу
видно, що таке провадження господарства самим земельним власником є цілком
випадкове. Коли збільшений попит на збіжжя потребує оброблення більшої кількости
землі $А$, ніж її є у власників, які сами провадять господарство, коли, отже,
частина її мусить бути віддана в оренду для того, щоб вона могла взагалі оброблятися,
тоді зараз же відпадає ця гіпотетичність погляду на межу, яку земельна
власність створює для приміщення капіталу. Постає недоладне противенство,
коли виходять з відповідного капіталістичному способові продукції відокремлення
між землею і капіталом, орендарем і земельним власником, а потім,
навпаки, припускають, що господарство провадять, як загальне правило, самі
земельні власники до такого обсягу і повсюди, де капітал, коли б незалежно
від нього не існувало жодної земельної власности, не здобував бн з оброблення
землі жодної ренти. (Див. у А.~Сміта місце про ренту з копалень, цитоване
значно далі). Це знищення земельної власности є випадкове. Воно може статись
або не статись.

\emph{Друге}. В складі орендованих земель можуть бути такі окремі дільниці
землі, що за даного ріння ринкових цін не дають ренти, отже, на ділі здаються
даром, але земельний власник не вважає їх за такі, бо він бачить загальну суму
ренти з землі, віддай її в оренду, а не осібні ренти з окремих складових дільниць
його землі. В цьому випадку для орендаря, — оскільки справа йде про
нерентодайні орендовані дільниці, — земельна власність як межа приміщення
\parbreak{}  %% абзац продовжується на наступній сторінці

\parcont{}  %% абзац починається на попередній сторінці
\index{iii2}{0192}  %% посилання на сторінку оригінального видання
капіталу відпадає, і саме через договір з самим земельним власником. Але він не платить
ренти за ці дільниці тільки тому, що він платить ренту за землю, до
якої вони належать. Тут припускається якраз така комбінація, коли доводиться
звернуть до гіршого роду землі А не як до самостійного нового поля продукції,
яке покрило б недостатнє подання, а як до такого, що становить лише
неподільну смугу в кращій землі. А випадок, який ми маємо дослідити, є якраз той,
коли доводиться самостійно провадити господарство на дільницях землі роду А,
отже, коли вони мусять за наявности загальних передумов капіталістичного способу
продукції здаватися в оренду як самостійні дільниці.

\emph{Третє}: Орендар може вкласти додатковий капітал у ту саму орендовану
дільницю, хоч за сущих ринкових цін одержувана в такий спосіб додаткова
продукція дає йому лише ціну продукції, звичайний зиск, але не дає йому
змоги платити додаткову ренту. Таким чином, на одну частину капіталу, вкладеного
в землю, він виплачує земельну ренту, на другу — ні. Як мало це припущення
розв’язує проблему, видно ось з чого: коли ринкова ціна (і разом
з цим родючість землі) дає йому можливість на додатковий капітал одержувати
додатковий здобуток, який подібно до старого капіталу дає йому, крім ціни продукції,
певний надзиск, то він бо скінчення терміну орендного договору залишає
його в себе. Але чому? Тому, що поки триває термін орендного договору, відпадає
та межа для примінення його капіталу у землю, яку створює земельна
власність. Проте, та звичайна обставина, що для забезпечення йому цього надзиску
мусить розпочатися самостійний обробіток додаткової гіршої землі і її самостійне
заорендування, незаперечно доводить, що приміщення додаткового капіталу
у стару землю не досить для створення потрібного підвищеного подання.
Одно припущення виключає друге. Правда, тепер можна було б сказані: сама
рента з найгіршого роду землі А є диференційна рента, чи то порівняно з землею,
яка обробляється самим власником (проте, це трапляється у чистому вигляді
лише як випадковий виняток), чи то порівняно з додатковим приміщенням
капіталу на тих старих заорендованих дільницях землі, що не дають ренти.
Але це була б 1) така диференційна рента, що виникала б не з ріжниці родючости
різних родів землі, а тому не мала б за свою передумову того, що земля
роду А не дає ренти, і що продукти її продається по ціні продукції; і 2) та обставина,
чи дають ренту додаткові приміщення капіталу на тій самій заорендованій
дільниці, чи ні, цілком також байдужа щодо того, чи виплачує ренту новооброблювана
земля кляси А, чи ні, так само як наприклад, для заснування нового самостійного
фабричного підприємства байдуже, чи вкладе інший фабрикант тієї
самої галузі підприємств у процентні папери частину свого капіталу, не бувши в
стані її цілком використати у своєму підприємстві, чи він зробить ряд таких окремих
розширень, що не дають йому повного зиску, а проте дають більше за процент.
Це для нього справа другорядна. Навпаки, додаткові нові підприємства мусять
давати пересічний зиск і споруджуються в надії на пересічний зиск. В усякому
разі, додаткові приміщення капіталу на старих заорендованих дільницях
землі і додаткове оброблення нової землі роду А становлять межі одне для одного.
Межа, до якої в ту саму заорендовану дільницю може вкладатись додатковий
капітал за менш сприятливих умов продукції, визначається конкурентними
новими приміщеннями у землю кляси А; з другого боку, рента, яку
може давати земля цієї кляси, обмежується конкурентними додатковими приміщеннями
капіталу на старих заорендованих землях.

Проте, всі ці фалшиві викрути не розв’язують проблеми, яка в простій
поставі така: припустімо, що ринкова ціна збіжжя (яке в цьому дослідженні
є для нас за представника всякого продукту землі) достатня для того, щоб
можна було почати оброблення частин землі кляси А, і щоб капітал, вкладений
у ці нові лани, здобув ціну продукції продукту, тобто покриття капіталу плюс
\parbreak{}  %% абзац продовжується на наступній сторінці

\parcont{}  %% абзац починається на попередній сторінці
\index{i}{0193}  %% посилання на сторінку оригінального видання
лише у власних пекарнях. Під кінець тижня\dots{} тобто в четвер,
праця починається тут о 10 годині вечора й триває з незначною
лише перервою до пізньої ночі під неділю».\footnote{
Там же, стор. LXXI.
}

Щождо «underselling masters», то й буржуазний погляд розуміє,
що «неоплачена праця підмайстрів (the unpaid labour of
the men) становить основу їхньої конкуренції».\footnote{
\emph{George Read}: «The History of Baking», London 1848, p. 16.
} «Full priced
baker» (пекар, що продає за «повну ціну») виказує перед слідчою
комісією на своїх «underselling» -конкурентів як на розкрадачів
чужої праці й фалшівників. «Вони мають успіх лише
через те, що ошукують публіку, і через те, що видушують із
своїх підмайстрів 18 годин праці, оплачуючи лише 12 годин».\footnote{
«First. Report etc.», Evidence. Свідчення «full priced baker’a» Чізмена,
p. 108.
}

Фальсифікація хліба й утворення кляси пекарів, що продають
хліб нижче від повної ціни, розвинулися в Англії на початку
XVIII сторіччя, відколи занепав цеховий характер ремества й
за спиною номінального майстра-пекаря виступив капіталіст\footnote{
\emph{George Read}: «The History of Baking», London 1848. Наприкінці
XVII й початку XVIII віків посередників (аґентів), що протиснулись
у всі можливі галузі ремісництва, офіціально ганьбили, називаючи
їх «Public Nuisances».\footnote*{— суспільним лихом. \emph{Ред.}}
Приміром, від «Grand Jury»\footnote*{Велике жюрі — суд присяжних. \emph{Ред.}}
підчас чвертьрічної
сесії мирових суддів у графстві Somerset подано до Палати громад
«presentment»,\footnote*{— внесення. \emph{Ред.}}
де, між іншим, сказано: «that these factors of Blackwell Hall
are a Public Nuisance and Prejudice to the Clothing Trade and ought to be
put down as a Nuisance». (Ці посередники Blackwell Hall є суспільне лихо
й перешкода торговлі одягом, і, як таку перешкоду, їх треба знищити»).
«The Case of our English Wool etc.», London 1685) p. 6, 7).
} в образі мірошника або торговця борошном. Цим покладено основу
для капіталістичної продукції, для безмірного зловження робочого
дня та нічної праці, хоч остання навіть у Лондоні стала на тверді
ноги лише в 1824~\abbr{р.}\footnote{«First Report etc.», p. VIII.}

Після всього попереднього зрозуміло, чому звіт комісії зачисляє
пекарських підмайстрів до тих робітників, які живуть
недовго; щасливо поминувши небезпеку стати жертвою жахливої
дитячої смертности, яка є нормальне явище для всіх категорій
робітничої кляси, вони рідко доживають до 42 року життя. А, проте,
пекарний промисел завжди переповнений кандидатами. Джерела,
що постачають для Лондону ці «робочі сили», є Шотляндія, західні
рільничі округи Англії і — Німеччина.

В 1858 – 1860~\abbr{рр.} пекарські підмайстри в Ірляндії зорганізували
власним коштом ряд великих мітинґів для аґітації проти
нічної й недільної праці. Публіка з суто ірляндським запалом
стала на їхній бік, як це було, приміром, 1860~\abbr{р.} на травневому
мітинґу в Дебліні. В наслідок цього руху було дійсно успішно
заведено виключну денну працю в Wexford’і, Kilkenny, Clonmel’i,
\parbreak{}  %% абзац продовжується на наступній сторінці

\parcont{}  %% абзац починається на попередній сторінці
\index{i}{0194}  %% посилання на сторінку оригінального видання
Waterford’i й~\abbr{т. ін.} «В Limerick’y, де, як відомо, страждання
найманих підмайстрів переходять усяку міру, цей рух розбився
об опір пекарів-хазяїнів, особливо ж пекарів-мірошників. Приклад
Limerick’a призвів до назаднього руху в Eniss’i та Tripperary.
В Cork’y, де громадське обурення виявилось у найжвавішій
формі, хазяї розбили рух, використавши свою силу викинути підмайстрів
на вулицю. В Дебліні хазяї виявили якнайрішучіший
опір і, переслідуючи тих підмайстрів, що стояли на чолі аґітації,
примусили решту поступитись і згодитись на нічну та недільну
працю»\footnote{
«Report of Committee on the Baking Trade in Ireland for 1861».
}. Комісія англійського уряду, озброєного в Ірландії
з ніг до голови, жалібно нарікає, немов голосільниця, на невблаганних
пекарів-хазяїнів Дебліну, Корку й~\abbr{т. ін.} «Комітет гадає,
що робочий час обмежено природними законами, яких не можна
порушувати безкарно. Тримаючи своїх робітників під загрозою
звільнення, хазяїни примушують їх порушувати їхні релігійні
переконання, не слухатися законів країни і зневажати громадську
думку (все це останнє стосується до недільної праці), вони сіють
ворожнечу між капіталом і працею та дають приклад, небезпечний
для релігії, моральности й суспільного ладу\dots{} Комітет гадає,
що здовження робочого дня понад 12 годин є узурпаторське втручання
у родинне й приватне життя робітника, і через встрявання
в родинний побут людини і у виконання нею своїх родинних обов’язків
як сина, брата, чоловіка й батька воно призводить до лихих
моральних результатів. Праця понад 12 годин має тенденцію підточувати
здоров’я робітника, викликає передчасну старість і
ранню смерть, а тому й нещастя робітничих родин, позбавляючи
(«are deprived») їх опіки й підпори голови родини саме в такий
час, коли це їм якнайпотрібніше»\footnote{Там же.}.

Ми тільки що побували в Ірландії. По тому боці каналу, в
Шотляндії, рільничий робітник, робітник плуга, нарікає на свою
13--14-годинну працю в найсуворішому кліматі, з чотиригодинною
додатковою працею в неділю (це в країні святкувальників
суботи!)\footnote{
Публічний мітинґ рільничих робітників у Lasswade біля Glasgow
5 січня 1866~\abbr{р.} (див. «Workman’s Advocate» з 13 січня I860~\abbr{р.}). — Утворення
наприкінці 1865~\abbr{р.} тред-юньойону рільничих робітників, передучім
у Шотландії, є історична подія. В одній з найпригнобленіших рільничих
округ Англії, в Buckingamshire, наймані робітники влаштували
в березні 1867~\abbr{р.} величезний страйк з метою вибороти підвищення тижневої
заробітної плати з 9--10\shil{ шилінґів} до 12\shil{ шилінґів.} (Із попереднього
видно, що рух англійського рільничого пролетаріяту, геть чисто зламаний
від часів придушення його енерґійних демонстрацій після року 1830
і особливо після заведення нового закону про бідних, знову починається
в шістдесятих роках, доки нарешті в році 1782 стає епохальним. У ІІ томі
я повертаюсь до цього питання, а так само й до Синіх Книг про становище
англійського рільничого робітника, що з’явилися після 1867~\abbr{р.} — Додаток
до третього видання).
}, одночасно з цим перед лондонським Grand Jury стало
троє залізничників: пасажирний кондуктор, машиніст і сиґнальник.
Велика залізнична катастрофа відправила сотні пасажирів
\parbreak{}  %% абзац продовжується на наступній сторінці

\input{_0195.tex}
\parcont{}  %% абзац починається на попередній сторінці
\index{iii1}{0196}  %% посилання на сторінку оригінального видання
під „попитом“ і „природною ціною“ те, що ми досі розуміли під
цим, покликаючись на А. Сміта, завжди мусить бути відношенням
рівності, бо тільки тоді, коли подання дорівнює дійсному
попитові, тобто попитові, який не хоче платити ні більше,
ні менше природної ціни, — тільки тоді дійсно сплачується природна
ціна; отже, в різний час той самий товар може мати дві
дуже різні природні ціни, і все ж відношення між поданням
і попитом, в обох випадках може бути однаковим, а саме
відношенням рівності“.] Отже, тут допускається, що при двох
різних natural prices [природних цінах] одного й того самого
товару в різний час попит і подання кожного разу можуть взаємно
покриватись і мусять покриватись для того, щоб товар
в обох випадках був проданий по його natural price. Але через
те що в обох випадках немає ніякої ріжниці у відношенні між
попитом і поданням, але є ріжниця у величині самої natural
price, то ця остання, очевидно, визначається незалежно від попиту
й подання і, отже, менш за все може бути ними визначена.

Для того, щоб товар продавався по його ринковій вартості,
тобто пропорціонально до вміщеної в ньому суспільно-необхідної
праці, сукупна кількість суспільної праці, вживана для
виробництва сукупної маси цього роду товарів, мусить відповідати
величині суспільної потреби в цих товарах, тобто платоспроможної
суспільної потреби. Конкуренція, коливання ринкових
цін, які відповідають коливанням відношення між попитом
і поданням, постійно намагаються звести до цієї міри сукупну
кількість праці, вжитої на кожний рід товарів.

У відношенні між попитом і поданням товарів повторюється,
поперше, відношення між споживною вартістю і міновою вартістю,
між товаром і грішми, між покупцем і продавцем; подруге,
відношення між виробником і споживачем, хоч обидва
вони можуть бути представлені третіми особами, торговцями.
При дослідженні відношення між покупцем і продавцем досить
протиставити їх, кожного окремо, один одному, щоб розвинути
це відношення. Трьох осіб досить для повної метаморфози
товару і, отже, для процесу продажу-купівлі, взятого в цілому.
$А$ перетворює свій товар у гроші $В$, якому він продає товар,
і знову перетворює свої гроші в товар, який він купує на ці
гроші в $C$; весь процес відбувається між ними трьома. Далі:
при дослідженні грошей ми припускали, що товари продаються
по їх вартості, бо не було ніякої підстави розглядати ціни, що
відхиляються від вартості, оскільки йшлося тільки про ті зміни
форми, які пророблює товар, стаючи грішми і знову перетворюючись
з грошей у товар. Раз товар взагалі продається і на
виручені гроші купується новий товар, то ми маємо перед
собою цілу метаморфозу, і для неї як такої однаково, чи стоїть
ціна товару нижче чи вище його вартості. Вартість товару зберігає
своє значення як основа, бо тільки з цієї основи можуть бути
раціонально виведені гроші, і ціна за своїм загальним поняттям
\parbreak{}  %% абзац продовжується на наступній сторінці

\parcont{}  %% абзац починається на попередній сторінці
\index{i}{0197}  %% посилання на сторінку оригінального видання
«Наші «білі раби», — вигукнув «Morning Star», орган панів
фритредерів Кобдена й Брайта, — наші білі раби запрацьовуються
на смерть і гинуть і вмирають без найменшого шуму».\footnote{
«Morning Star» з 23 липня 1863 р. «Times» скористався цим випадком
для оборони американських рабовласників проти Брайта й т. ін.
«Дуже багато з нас, — каже «Times», — гадають, що лоти, доки ми сами
вимучуємо на смерть працею наших власних молодих жінок, погрожуючи
їм ударами голоду замість свисту батога, доти ми ледве чи маємо право
йти мечем і вогнем на ті родини, що їхні члени родилися рабовласниками
та які принаймні добре годують своїх рабів і вимагають від них лише
помірної праці» («Times», а 2 липня 1863 р.). Газета торів «Standart»
розправлялась у тому самому дусі з його преподобієм Ньюмен Холлом:
«Він відлучує від церкви рабовласників, але молиться разом із порядними
людьми, що примушують працювати за собачу плату лондонських візників
та кондукторів омнібусів і т. ін. лише по 16 годин на день». Нарешті,
пролунав голос оракула, винахідника культу генія, пана Томаса Карлейля,
про якого я вже року 1850 писав: «Геній пішов к чорту, лишився культ».
В коротенькій притчі він зводить єдину величну подію сучасної історії,
американську громадянську війну, на те, що Петро з півночі з усіх сил
намагається переломити черепа Павлові з півдня, бо Петро з півночі
наймає свого робітника «поденно», а Павло з півдня — «на ціле життя».
(«Macmillan’s Magazine». Ilias Americana in nuсе. Серпневий зошит
1863 р.). Так луснув, нарешті, шумовинний пухир торійських симпатій
до міських — але ні в якому разі не до сільських! — найманих робітників.
Основа цих симпатій — це рабство!
}

«Запрацьовуватись на смерть — це є порядок дня не лише
в майстернях кравчих, але в тисячах місць, ба на кожному місці,
де справи йдуть добре\dots{} Візьмімо як приклад коваля. Як вірити
поетам, то немає в світі людини сильнішої, веселішої за коваля.
Він устає раннім ранком і викрешує іскри перед тим, як засяє
сонце; нема такої людини, що так їла б, так пила б і спала, як
він. Якщо поглянути на долю коваля чисто з фізичного боку, то,
дійсно, за помірної праці, становище його одне з найкращих.
Але ходімо за ним до міста й погляньмо на той тягар праці, який
накладають на його дужі плечі, погляньмо на місце, яке він посідає
у таблицях смертности нашої країни? У Marylebone (один із
найбільших міських кварталів Лондону) смертність ковалів становить
31 на 1000 на рік, а це на 11 перевищує пересічну смертність
дорослих чоловіків Англії. Праця ця, майже інстинктова вмілість
людини, сама по собі бездоганна, через саму лише надмірність
стає руйнаційною для людини. Людина може зробити стільки й
стільки ударів молотом на день, стільки й стільки кроків, стільки
й стільки разів дихнути, стільки й стільки зробити якоїсь роботи
й прожити пересічно, приміром, 50 років. Її примушують робити
на стільки більше вдарів, на стільки більше кроків, стільки частіш
віддихувати, а це все разом збільшує її життєве завдання на
одну четвертину на день. Вона силкується це все робити, а результат
такий, що за обмежений період вона виконує на четвертину
більшу роботу і вмирає на 37 році замість на 50».\footnote{
\emph{Dr. Richardson}: «Work and Overwork» y «Social Science Review»,
18 липня 1863.
}

\index{i}{0198}  %% посилання на сторінку оригінального видання

\subsection{Денна й нічна праця. Система змін}

З погляду процесу зростання вартости сталий капітал, засоби
продукції існують лише на те, щоб вбирати в себе працю і з кожною
краплею праці — відповідну кількість додаткової праці.
Якщо вони цього не роблять, то вже саме їхнє існування становить
для капіталіста неґативну втрату, бо ж протягом того часу,
коли вони лежать без діла, вони репрезентують марно авансований
капітал: утрата ця стає позитивною, скоро тільки на поновлення
перерваної продукції треба додаткових витрат. Здовження
робочого дня поза межі природного дня геть аж у ніч діє
лише як паліятив, лише приблизно заспокоює вампірову спрагу
живої крови праці. Тому присвоєння праці протягом усіх 24 годин
доби є іманентне прагнення капіталістичної продукції. А що
фізично неможливо день і ніч безупинно висисати ті самі робочі
сили, то, щоб перемогти фізичні перешкоди, потрібно зміни робочих
сил, споживаних вдень і вночі, зміни, яка допускає різні
методи, наприклад, її можна зорганізувати так, що частина робочого
персоналу один тиждень працює вдень, а другий тиждень
вночі й~\abbr{т. ін.} Як відомо, така система змін, таке перемінне господарство
панувало за часів юнацького розцвіту англійської бавовняної
промисловости і~\abbr{т. ін.}, і процвітає в наші часи, між
іншим, на бавовняних прядільнях Московської губерні. Як система
цей 24-годинний процес продукції існує ще й нині в багатьох
досі «вільних» галузях промисловости Великобрітанії, між іншим
у домнах, кузнях, вальцювальних та інших металевих мануфактурах
Англії, Велзу й Шотляндії. Робочий процес охоплює тут,
окрім 24 годин шістьох робочих днів, здебільшого ще й 24 години
неділі. Робітники складаються з чоловіків та жінок, дорослих
і дітей обох статей. У віці дітей і молоді є всі переходові ступені
від 8 (в деяких випадках від 6) до 18 років\footnote{
«Children’s Employment Commission». Third Report. London
1864, p. IV, V, VI.
}. У деяких галузях
дівчата й жінки працюють уночі разом із чоловічим персоналом\footnote{
«Так у Стафордшірі, як і в південному Велзі, молоді дівчата й жінки
працюють у кам’яновугляних копальнях та коксувальнях не лише вдень,
а й уночі. У звітах, подаваних до парляменту, не раз зазначувано це
явище як причину великого й загальновідомого лиха. Ці жінки, що працюють
разом із чоловіками й ледве відрізняються від них своїм одягом,
покриті брудом і сажею, наражаються на небезпеку згубити свій моральний
характер через утрату самоповаги, а це є неминучий наслідок їхньої
нежіночої праці». («Both in Staffordshire and in South Wales young girl
and women are employed on the pit banks and on the coke heaps, not only
by day, but also night. This practice has been often noticed in Reports presented
to Parliament, as being attended with great and notorious evils.
These females, employed with the men, hardly distinguished from them
in their dress, and begrimed with dirt and smoke, are exposed to the deterioration
of character arising from the loss of self-respect which can hardly
fail to follow from their unfeminine occupation»). (Там же, 194, p. XXVI.~Порівн. Fourth Report (1865), 61, p. XIII). Те саме й на гутах.
}.

\parcont{}  %% абзац починається на попередній сторінці
\index{iii2}{0199}  %% посилання на сторінку оригінального видання
продукції капітал $= k$, то ріжниця їхня є в другій, змінній частині, в додатковій
вартості, що в ціні продукції $= р$, зискові, тобто дорівнює всій додатковій
вартості, обчисленій на суспільний капітал і на кожен окремий капітал, як на
пропорційну частину суспільного капіталу; але у вартості товару вона дорівнює
дійсній додатковій вартості, породженій цим окремим капіталом, та становить
інтеґральну частину породжених ним товарових вартостей. Коли вартість товару
вища за його ціну продукції, то ціна продукції $= k \dplus{} р$, вартість $= k \dplus{} p \dplus{} d$,
так що $р \dplus{} d \deq{}$ додатковій вартості, що міститься в ньому. Отже, ріжниця між
вартістю і ціною продукції $= d$, надмірові додаткової вартости, продукованої цим
капіталом понад ту, яка припадає йому відповідно до загальної норми зиску.
З цього випливає, що ціна хліборобських продуктів може бути вища від їхньої
ціни продукції, хоч вона не досягатиме їхньої вартости. З цього випливає
далі, що до певного пункту може відбуватися тривале підвищення ціни
хліборобських продуктів, перше ніж їхня ціна досягне їхньої вартости. З цього
випливає також, що тільки в наслідок монополії земельної власности надмір
вартости хліборобських продуктів над їхньої ціною продукції може стати моментом,
що визначає їхню загальну ринкову ціну. З цього випливає, нарешті,
що в цьому випадку не подорожчання продукту є причина ренти, а рента
є причиною подорожчання продукту. Коли ціна продукту з одиниці площі найгіршої
землі $= Р \dplus{} r$, то всі диференційні ренти збільшуються відповідними кратними
$r$, бо, згідно з припущенням, за реґуляційну ринкову ціну стає $Р \dplus{} r$.

Коли б пересічний склад нехліборобського суспільного капіталу
$= 85c \dplus{} 15v$ і норма додаткової вартости \deq{} 100\%, то ціна продукції дорівнювала
б 115. Коли б склад хліборобського капіталу $= 75c \dplus{} 25v$, то вартість
продукту, при тій самій нормі додаткової вартости, і реґуляційна ринкова
вартість дорівнювала б 125. Коли б хліборобський продукт вирівнявся з нехліборобським
до пересічної ціни (для короткости ми припускаємо, що в обох галузях
продукції загальна кількість капіталу однакова), то вся додаткова вартість
дорівнювала б 40, тобто 20\% на капітал в 200. Продукт так одного, як і другого
продавалось би за 120. Отже, при вирівнянні за цінами продукції пересічні ринкові
ціни нехліборобського продукту стояли б вище, а хліборобського продукту
нижче від їхньої вартости. Коли б хліборобські продукти продавалось по їхній
повній вартості, то вони були б на 5 вище, а промислові продукти
на 5 нижче, ніж по вирівнянні. Коли ринкові відносини не дозволяють продавати
хліборобські продукти по їхній повній вартості, виторговувати весь надмір
над ціною продукції, то це призводить до середнього між обома крайніми
пунктами стану; промислові продукти будуть продаватися трохи вище від їхньої
вартости, а хліборобські продукти трохи вище від їхньої ціни продукції.

\looseness=1
Хоч земельна власність може нагнати ціну хліборобських продуктів вище
від їхньої ціни продукції, проте не від земельної властности, а від загального стану
ринку залежить, в якій мірі ринкова ціна, піднявшись над ціною продукції, наближається
до вартости, і, отже, в якій мірі додаткова вартість, створена в хліборобстві
понад даний пересічний зиск, перетворюється на ренту, абож бере участь у загальному
вирівнянні додаткової вартости в пересічний зиск. В усякому випадку
ця абсолютна рента, що виникає з надміру вартости над ціною продукції, становить
просто частину хліборобської додаткової вартости, перетворення цієї додаткової
вартости на ренту, захоплювання цієї додаткової вартости земельним власником;
цілком так само, як диференційна рента виникає з перетворення надзиску в ренту, захопленпя його
земельною власністю, при загальній реґуляційній ціні продукції. Ці обидві форми ренти є єдино
нормальні. Рента, крім цих форм, може ґрунтуватися лише на власне монопольній ціні, яку не
визначається ані ціною продукції, ані вартістю товарів, а потребою і виплатоспроможністю покупців, і
розгляд якої стосуються до вчення про конкуренцію, де досліджується справжній рух ринкових цін.


\index{iii2}{0200}  %% посилання на сторінку оригінального видання
Коли б уся земля певної країни, придатна для хліборобства, була вже
здана в оренду, — при чому припускається, як загальне явище, капіталістичний
спосіб продукції і нормальні відносини, — то не було б такої землі, яка не
давала б ренти, але могли б існувати такі приміщення капіталу, окремі частини
капіталу вкладеного в землю, які не давали б ренти; бо, скоро земля здана в
оренду, земельна власність перестає діяти як абсолютна межа потрібного вкладення
капіталу. Як відносна межа вона продовжує ще діяти і після цього
в такій мірі, в якій перехід до земельного власника долученного до землі капіталу
ставить тут перед орендарем дуже визначені межі. Тільки в цьому випадку
вся рента перетворилася б на диференційну ренту, яка
визначається не ріжницями в якості землі, а ріжницями між надзисками, що
постають від останніх приміщень капіталу на певній землі, і рентою, яка виплачувалася
б за оренду землі найгіршої кляси. Як межа земельна власність
діє абсолютно лише остільки, оскільки допущення до землі взагалі, як до сфери
приміщення капіталу, зумовлює данину земельному власникові. Коли це допущення
сталося, останній уже не може протиставити ніяких абсолютних меж
кількісному розмірові приміщення капіталу на даній дільниці землі. Будуванню
будинків взагалі кладеться межу земельною власністю третьої особи на ту дільницю
землі, на якій мається збудувати будинок. Але скоро лише ця земля
орендована під будівлю будинків, то вже від орендаря залежить, чи бажає він
збудувати на ній високий чи низький будинок.

Коли б пересічний склад хліборобського капіталу був такий самий або
вищий, ніж пересічний склад суспільного капіталу, то абсолютна рента, знов
таки в щойно дослідженому розумінні, відпала б; тобто відпала б рента, яка
відрізняється так від диференційної ренти, як і від ренти, що ґрунтується на
власне монопольній ціні. Тоді вартість хліборобського продукту не була б
вища від його ціни продукції, і хліборобський капітал пускав би в рух не
більшу кількість праці, отже, реалізував би також не більшу кількість додаткової
праці, ніж нехліборобський капітал. Те саме сталося б тоді, коли б з
проґресом культури склад хліборобського капіталу зрівнявся із пересічним
складом суспільного капіталу.

На перший погляд здається за суперечність припускати, що, з одного
боку, склад хліборобського капіталу підвищується, отже, зростає його стала частина
проти змінної, а з другого боку, що ціна хліборобського продукту має
піднестися остільки високо, щоб нова і гірша, ніж колишня, земля могла виплачувати
ренту, яка в цьому випадку могла б виникнути лише з надміру ринкової
ціни над вартістю і ціною продукції, коротко, лише з монопольної ціни
продукту.

Тут треба відрізняти таке.

Розглядаючи створення норм зиску, ми, насамперед, бачили, що капітали,
які, технологічно розглядувані, складені однаково, тобто порівняно з кількістю
машин і сирового матеріялу пускають в рух однакову кількість праці,
можуть, проте, бути різного складу в наслідок того, що сталі частини цих капіталів
мають різну вартість. Сировий матеріял або машини можуть бути в
одному випадку дорожчі, ніж у другому. Щоб пустити в рух таку саму масу
праці (а це, згідно з припущенням, було б потрібне для перероблення такої ж
самої маси сирового матеріялу), в одному випадку довелося б авансувати більший
капітал, ніж у другому, тому що, наприклад, з капіталом 100 я не можу пустити
в рух однакової кількости праці, коли сировий матеріял, який теж доводиться
оплачувати з цих 100, в одному випадку коштує 40, в другому 20.
Але те, що технологічно ці капітали все ж складені однаково, негайно виявилося
б, скоро ціна дорожчого сирового матеріялу знизилася б до рівня дешевшого.
Відношення вартости змінного і сталого капіталу тоді стали б однакові,
\parbreak{}  %% абзац продовжується на наступній сторінці

\parcont{}  %% абзац починається на попередній сторінці
\index{iii2}{0201}  %% посилання на сторінку оригінального видання
хоч у технічному відношенні між ужитою живою працею та масою і природою
застосованих умов праці не сталося жодної зміни. З другого боку, коли розглядати
справу виключно з погляду складу за вартістю, капітал порівняно низького
органічного складу в наслідок простого підвищення вартостей його сталих частин
міг би справити таке вражіння, ніби він піднісся на один ступінь з капіталом
вищого органічного складу. Хай дано капітал $= 60c + 40v$, тому що він
застосовує багато машин і сирового матеріялу, порівняно з живою працею, і
другий капітал $= 40c + 60v$, тому що він вживав багато живої праці (60\%),
мало машин (скажімо, 10\%) і відносно до робочої сили мало, до того ж
ще і дешевого, сирового матеріялу (скажімо, 30\%); таким чином в наслідок простого
підвищення вартости сирових і допоміжних матеріялів з 30 до 80, склад
міг би зрівнятися так, що в другому капіталі на 10 в машинах припадало б 80
в сировому матеріялі і 60 робочої сили, тобто $90c + 60v$, що, визначене в процентах,
теж дорівнювало б $60c + 40v$, при чому не сталося б жодної зміни в технічному
складі. Отже, капітали однакового органічного складу можуть мати
різний вартісний склад, і капітали однакового процентного вартісного складу
можуть стояти на різних ступенях органічного складу отже, виражати різні ступені
розвитку суспільної продуктивної сили праці. Отже, сама лише обставина,
що за вартісним складом хліборобський капітал стояв би на загальному рівні, ще
не доводила б того, що суспільна продуктивна сила праці досягла у нього
такого самого рівня. Вона могла б лише показувати, що власний продукт цього
капіталу, який знову таки становить частину умов його продукції, є дорожчий,
або що допоміжні матеріяли, от як добриво, котрі давніш були просто під руками,
тепер доводиться довозити здалека тощо.

Але, залишаючи це осторонь, треба взяти на увагу своєрідний характер
хліборобства.

Припустімо, що вживання в хліборобстві машин, які зберігають працю,
хемічних допоміжних засобів тощо, набирають тут ширшого розміру, отже, що сталий
капітал технічно зростає не тільки щодо вартости, але й щодо маси, порівняно
я масою ужитої робочої сили; в такому разі в хліборобстві (як і в гірничій
промисловості) справа йде не тільки про суспільну, але і про природну продуктивність
праці, яка залежить від природних умов праці. Можливо, що збільшення
суспільної продуктивної сили в хліборобстві лише компенсує, або навіть
не зовсім компенсує зменшення природної сили — ця компенсація завжди може
впливати лише протягом деякого часу, — так що, не зважаючи на технічний
розвиток, продукт не здешевлюється, а лише гальмується його ще більше подорожчання.
Можливо також, що при висхідній ціні збіжжя абсолютна маса
продукту зменшується, тимчасом як відносний надпродукт зростає; це можливо
саме при відносному збільшенні сталого капіталу, що складається переважно
я машин або худоби, при чому доводиться покривати тільки його зношування,
і при відповідному зменшенні змінної частини капіталу, яка витрачається на
заробітну плату, що її постійно доводиться покривати з продукту цілком.

Але можливо також, що з поступом хліборобства потрібно буде лише помірне
підвищення ринкової ціни над пересічною для того, щоб могла оброблятись
і одночасно давати ренту така земля гіршої якости, яка при нижчому рівні
технічних допоміжних засобів потребувала б вищого підвищення ринкової ціни.

Може здатися, що та обставина, що, наприклад, у скотарстві, коли воно провадиться
в великих розмірах, маса вжитої робочої сили дуже мала проти сталого
капіталу, який є в самій худобі, може здатися, що ця обставина цілком суперечить
тому, що хліборобський капітал, обчислений у процентах, пускає в рух робочої
сили більше, ніж нехліборобський пересічний суспільний капітал. Але тут слід
відзначити, що при розгляді ренти ми виходимо, як з вирішної, з тієї частини
хліборобського капіталу, яка продукує основний рослинний засіб харчування,
\parbreak{}  %% абзац продовжується на наступній сторінці

\parcont{}  %% абзац починається на попередній сторінці
\index{iii1}{0202}  %% посилання на сторінку оригінального видання
умовах. Виключаючи взагалі випадки криз і перепродукції, це
стосується до всіх ринкових цін, як би дуже вони не відхилялись
від ринкових вартостей або ринкових цін виробництва.
Саме ринкова ціна передбачає, що за товари того самого роду
сплачується однакова ціна, не зважаючи на те, що ці товари
можуть бути вироблені при дуже різних індивідуальних умовах
і тому можуть мати дуже різні витрати виробництва. (Про надзиски
як наслідки монополій у звичайному розумінні слова, штучних
чи природних, ми тут не говоримо.)

Але, крім того, надзиск може виникнути ще в тому випадку,
коли певні сфери виробництва перебувають в такому стані, що
вони можуть ухилитися від перетворення їх товарних вартостей
у ціни виробництва, а тому й від зведення їх зисків до пересічного
зиску. У відділі про земельну ренту ми розглянемо
дальший розвиток цих двох форм надзиску.

\section{Впливи загальних коливань заробітної плати
на ціни виробництва}

Припустім, що пересічний склад суспільного капіталу є
$80 c \dplus{} 20 v$, а зиск — 20\%. В цьому випадку норма додаткової
вартості є 100\%. Загальне підвищення заробітної плати, якщо
припустити всі інші умови однаковими, означає зниження норми
додаткової вартості. Для пересічного капіталу зиск і додаткова
вартість збігаються. Припустім, що заробітна плата підвищується
на 25\%. Та сама маса праці, яку привести в рух коштувало 20,
коштує тепер 25. Отже, ми маємо в цьому випадку, замість
$80 c \dplus{} 20 v \dplus{} 20 p$, за один оборот вартість у $80 c \dplus{} 25 v \dplus{} 15 p$.
Праця, приведена в рух змінним капіталом, як і раніш, виробляє
суму вартості в 40. Якщо $v$ підвищується з 20 до 25, то надлишок
$m$ або $p$ є вже тільки \deq{} 15. Зиск в 15 на 105 \deq{} 14\sfrac{2}{7}\%,
і це було б новою нормою пересічного зиску. Через те що
ціна виробництва товарів, вироблюваних пересічним капіталом,
збігається з їх вартістю, ціна виробництва цих товарів не змінилася
б; тому підвищення заробітної плати привело б, правда,
до зниження зиску, але не привело б до зміни вартості й ціни
товарів.

Раніше, коли пересічний зиск був \deq{} 20\%, ціна виробництва
товарів, вироблених за один період обороту, дорівнювала їх
витратам виробництва плюс зиск в 20\% на ці витрати виробництва,
отже \deq{} $k \dplus{} kp' \deq{} k \dplus{} \frac{20k}{200}$, де $k$ є змінна величина, різна
залежно від вартості засобів виробництва, що входять у товари,
і від розміру того зношування, яке основний капітал, застосований
у виробництві цих товарів, віддає продуктові. Тепер
ціна виробництва становила б $k \dplus{} \frac{14\sfrac{2}{7} k }{100}$.


\index{iii1}{0203}  %% посилання на сторінку оригінального видання
Візьмімо тепер капітал, склад якого є нижчий, ніж первісний
склад пересічного суспільного капіталу $80 c \dplus{} 20 v$ (який
перетворився тепер в $76\sfrac{4}{21}c \dplus{} 23\sfrac{17}{21}v$), наприклад, $50 c \dplus{} 50 v$.
Тут ціна виробництва річного продукту, — якщо ми для спрощення
припустимо, що весь основний капітал увійшов як зношування
в річний продукт і що час обороту такий самий, як
і в випадку I, — становила перед підвищенням заробітної плати
$50 c \dplus{} 50 v \dplus{} 20 p \deq{} 120$. Підвищення заробітної плати на 25\%
дає для тієї самої кількості приведеної в рух праці підвищення
змінного капіталу з 50 до 62\sfrac{1}{2}. Коли б річний продукт був
проданий по попередній ціні виробництва в 120, то це дало б
$50 c \dplus{} 62\sfrac{1}{2}v \dplus{} 7\sfrac{1}{2}p$, тобто норму зиску в 6\sfrac{2}{3}\%.
Але нова пересічна норма зиску є 14\sfrac{2}{7}\%, і через те що ми всі інші умови
припускаємо незмінними, цей капітал в $50 c \dplus{} 62\sfrac{1}{2}v$ так само
мусить дати вказаний зиск. Але капітал в 112\sfrac{1}{2}, при нормі зиску
в 14\sfrac{2}{7}, дає 16\sfrac{1}{14} зиску.
% REMOVED \footnote*{
% В першому німецькому виданні тут сказано: „в круглих числах 16\sfrac{1}{12}
% зиску“; відповідно до цього Енгельс обчислює потім ціну виробництва в 128\sfrac{7}{12}
% В рукопису Маркса дано точне число в 16\sfrac{3}{42}, яке нами взяте з відповідним
% скороченням дробу і застосоване при обчисленні ціни виробництва.
% \Red{Примітка ред. нім. вид. ІМЕЛ.}
% }
Отже, ціна виробництва вироблених
ним товарів є тепер $50 c \dplus{} 62\sfrac{1}{2}v \dplus{} 16\sfrac{1}{14}p \deq{} 128\sfrac{8}{14}$. Отже, в наслідок
підвищення заробітної плати на 25\% ціна виробництва
тієї самої кількості того самого товару підвищилась тут з 120
до 128\sfrac{8}{14}, або більше ніж на 7\%.

Візьмім, навпаки, сферу виробництва вищого складу, ніж пересічний
капітал, наприклад, $92 c \dplus{} 8 v$. Отже, первісний пересічний
зиск і тут \deq{} 20, і якщо ми знову припустимо, що весь
основний капітал входить у річний продукт і що час обороту
такий самий, як і в випадках І і II, то ціна виробництва товару
й тут \deq{} 120.

В наслідок підвищення заробітної плати на 25\% змінний капітал
для тієї самої кількості праці зростає з 8 до 10, отже
витрати виробництва товарів зростають з 100 до 102; з другого
боку, пересічна норма зиску впала з 20\% до 14\sfrac{2}{7}\%. Але
$100 : 14\sfrac{2}{7} \deq{} 102 : 14\sfrac{4}{7}$.
% REMOVED \footnote*{
% В першому німецькому виданні тут стоїть: „(приблизно)“. В рукопису
% Маркса цього слова немає. В дійсності тут рівняння точне, а не тільки приблизне.
% \Red{Примітка ред. нім. вид. ІМЕЛ.}
% }
Отже, зиск, що припадає тепер на 102,
становить 14\sfrac{4}{7}. І тому весь продукт продається за
$k \dplus{} kp' \deq{} 102 \dplus{} 14\sfrac{4}{7} \deq{} 116\sfrac{4}{7}$. Отже, ціна виробництва впала
з 120 до 116\sfrac{4}{7}, або майже на 3\%.
% REMOVED \footnote*{
% В першому німецькому виданні тут сказано: „більше ніж на 3\%. В рукопису
% Маркса стоїть: „на 3\sfrac{3}{7}“, тобто дано абсолютне число. В процентах воно
% дорівнює 2\sfrac{6}{7}\%. \Red{Примітка ред. нім. вид. ІМЕЛ.}
% }

Отже, в наслідок підвищення заробітної плати на 25\%:

1)~для капіталу пересічного суспільного складу ціна виробництва
товару лишилась незмінною;

2)~для капіталу нижчого складу ціна виробництва товару
\parbreak{}  %% абзац продовжується на наступній сторінці

\input{_0204.tex}
\parcont{}  %% абзац починається на попередній сторінці
\index{iii1}{0205}  %% посилання на сторінку оригінального видання
Рікардо не досліджує), досить тільки перевернути щойно наведені
міркування.

I.~Пересічний капітал $= 80 c \dplus{} 20 v \deq{} 100$; норма додаткової
вартості \deq{} 100\%; ціна виробництва \deq{} товарній вартості $= 80 c \dplus{}
20 v \dplus{} 20 p \deq{} 120$; норма зиску \deq{} 20\%. Нехай заробітна плата
впаде на одну чверть, тоді той самий сталий капітал приводитиметься
в рух 15-ма $v$ замість $20 v$. Отже, ми маємо товарну
вартість $ \deq{} 80 c \dplus{} 15 v \dplus{} 25 p \deq{} 120$. Кількість праці, вироблена $v$,
лишається  незмінною, і тільки створена ним нова вартість інакше
розподіляється між капіталістом і робітниками. Додаткова вартість
підвищилась з 20 до 25, і норма додаткової вартості
підвищилась з \frac{20}{25} до \frac{25}{15}, отже, з 100\% до 166\sfrac{2}{3}\%.
Зиск на 95 тепер \deq{} 25, отже, норма зиску на 100 \deq{} 26\sfrac{6}{19}. Новий
процентний склад капіталу тепер є $84\sfrac{4}{19}c \dplus{} 15\sfrac{15}{19}v \deq{} 100$.

II.~Нижчий склад. Первісно $50 c \dplus{} 50 v$, як вище. В наслідок
падіння заробітної плати на \sfrac{1}{4}, $v$ зводиться до 37\sfrac{1}{2}, і тим самим
весь авансований капітал зводиться до $50 c \dplus{} 37\sfrac{1}{2}v \deq{} 87\sfrac{1}{2}$. Якщо
ми застосуємо до цього нову норму зиску в 26\sfrac{6}{19}\%, то
$100 : 26\sfrac{6}{19} \deq{} 87\sfrac{1}{2} : 23\sfrac{1}{38}$. Та сама товарна маса,
яка раніш коштувала 120, коштує
тепер $87\sfrac{1}{2} \dplus{} 23\sfrac{1}{38} \deq{} 110\sfrac{10}{19}$; падіння ціни
майже на 8\%.
% REMOVED \footnote*{
% В першому німецькому виданні тут стоїть: „майже на 10\%“. В рукопису
% Маркса в цьому місці сказано: „Знижується майже на 10“, тобто дається абсолютне
% число. Точно воно дорівнює 9\sfrac{9}{19} і становить 7\sfrac{7}{19}\%. \Red{Примітка ред. нім. вид. ІМЕЛ.}
% }

\looseness=1
III.~Вищий склад. Первісно $92 c \dplus{} 8 v \deq{} 100$. Падіння заробітної
плати на \sfrac{1}{4} знижує $8 v$ до $6 v$, весь капітал до 98. Отже,
$100 : 26\sfrac{6}{19} \deq{} 98 : 25\sfrac{15}{19}$. Ціна виробництва товару,
раніш $100 \dplus{} 20 \deq{} 120$, тепер, після падіння заробітної плати, є
$98 \dplus{} 25\sfrac{15}{19} \deq{} 123\sfrac{15}{19}$;
отже, вона підвищилась більше ніж на 3\%.
% REMOVED \footnote*{
% В першому німецькому виданні тут стоїть: „майже на 4\%“. В рукопису
% Маркса тут так само дається абсолютне число (3\sfrac{15}{19}). В процентах воно
% становить 3\sfrac{3}{19}\%. \Red{Примітка ред. нім. вид. ІМЕЛ.}
% }

Отже, ми бачимо, що досить тільки повторити попередні міркування в зворотному
напрямі і з відповідними змінами: загальне падіння заробітної плати має своїм
наслідком загальне підвищення додаткової вартості, норми додаткової вартості, а
при інших незмінних умовах і норми зиску, хоч і в
іншій пропорції; далі воно має своїм наслідком падіння цін виробництва для
товарних продуктів капіталів нижчого складу і підвищення цін виробництва для
товарних продуктів капіталів вищого складу. Результат якраз протилежний до того,
що  виявився при загальному підвищенні заробітної плати\footnote{
Надзвичайно дивно, що Рікардо (який, звичайно, застосовує іншого
методу, ніж це зроблено тут, бо не розуміє процесу вирівнення вартостей в ціни
виробництва) навіть не приходить до цієї думки, а розглядає тільки перший
випадок, підвищення заробітної плати і вплив його на ціни виробництва товарів
(„Principles etc.“, Лондон 1852, стор. 26 і далі]. A servum pecus
imitatorum [рабське стадо наслідувачів] не додумалось навіть до того, щоб
зробити це, само собою зрозуміле, по суті тавтологічне застосування.
}.
\parbreak{}  %% абзац продовжується на наступній сторінці

\input{_0206.tex}
\input{_0207c.tex}
\parcont{}  %% абзац починається на попередній сторінці
\index{ii}{0208}  %% посилання на сторінку оригінального видання
завжди звільнятися 300\pound{ ф. стерл}. Навпаки, коли щотижня витрачається
300\pound{ ф. стерл.}, то для робочого періоду ми маємо 1800\pound{ ф. стерл.}, для періоду
циркуляції 900\pound{ ф. стерл.}; отже, періодично звільнятиметься вже 900\pound{ ф. стерл.}
замість 300\pound{ ф. стерл}.

D.~Ввесь капітал, напр., в 900\pound{ ф. стерл.}, треба поділити на дві частини,
які раніш: 600\pound{ ф. стерл.} для робочого періоду і 300\pound{ ф. стерл.} для періоду
циркуляції. Частина, дійсно витрачувана на процес праці, зменшиться в
наслідок цього на одну третину, з 900 до 600\pound{ ф. стерл.}, і тому розмір продукції
зменшиться на одну третину. З другого боку, 300\pound{ ф. стерл.}
функціонують лише для того, щоб зробити робочий період безперервним,
так, щоб на процес праці щотижня протягом року можна було витрачати
по 100\pound{ ф. стерл}.

Беручи абстрактно, цілком байдуже, чи роблять 600\pound{ ф. стерл.} протягом
6 × 8 \deq{} 48 тижнів (продукт \deq{} 4800\pound{ ф. стерл.}), чи весь капітал в 900\pound{ ф. стерл.}
витрачається на процес праці протягом 6 тижнів, а потім протягом
3 тижнів періоду циркуляції він лежить без діла; в останньому випадку
він працював би на протязі 48 тижнів $6 × 5\sfrac{1}{3} \deq{} 32$ тижні (продукт \deq{}
900 × 5\sfrac{1}{3} \deq{} 4800\pound{ ф. стерл.}) і 16 тижнів лежав би без діла. Але, не кажучи
вже про більше псування основного капіталу протягом 16 тижнів, коли
він лишається бездіяльний, та подорожчання праці, що її доведеться оплатити
за ввесь рік, хоч вона діє лише протягом частини його, така реґулярна
перерва продукційного процесу взагалі несполучна з продукцією
сучасної великої промисловости. Сама ця безперервність є продуктивна
сила праці.

Коли ми тепер ближче придивимось до звільненого капіталу, в дійсності
до капіталу, що його дію припинено, то виявиться, що чимала
частина його завжди мусить мати форму грошового капіталу. Зупинімось
на прикладі: робочий період 6 тижнів, період циркуляції 3 тижні, щотижнева
витрата 100\pound{ ф. стерл}. Посередині другого робочого періоду,
наприкінці 9-го тижня, припливають назад 600\pound{ ф. стерл.}, що з них протягом
решти робочого періоду треба витратити лише 300\pound{ ф. стерл}.
Отже, наприкінці другого робочого періоду з цієї суми звільняться
300\pound{ ф. стерл}. В якому стані перебувають ці 300\pound{ ф. стерл.}? Припустімо,
що \sfrac{1}{3} треба витратити на заробітну плату, \sfrac{2}{3} на сировинні та допоміжні
матеріяли. Отже, з 600\pound{ ф. стерл.}, що приплили назад, 200\pound{ ф. стерл.}, призначені
на заробітну плату, перебувають у грошовій формі, а 400\pound{ ф.
стерл.} — у формі продуктивного запасу, у формі елементів поточної
частини сталого продуктивного капіталу. А що для другої половини
робочого періоду II, треба лише половини цього продуктивного запасу,
то друга половина його протяюм 3 тижнів перебуває в формі надлишкового
продуктивного запасу, тобто запасу, що перевищує потреби одного
робочого періоду. Але капіталіст знає, що з цієї частини (\deq{} 400\pound{ ф. стерл.})
приплилого капіталу для поточного робочого періоду потрібна тільки
половина (\deq{} 200\pound{ ф. стерл.}). Отже, від ринкових умов залежатиме, чи
перетворить він знову ці 200\pound{ ф. стерл.} одразу цілком або тільки почасти на
надлишковий продуктивний запас, чи, вичікуючи сприятливих ринкових умов
\parbreak{}  %% абзац продовжується на наступній сторінці

\parcont{}  %% абзац починається на попередній сторінці
\index{iii2}{0209}  %% посилання на сторінку оригінального видання
виявиться, що в наслідок збільшення маси ренти підвищився і її рівень. Той
самий акр, що давав 2\pound{ ф. стерл.} ренти, дає тепер 4\pound{ ф. стерл.}\footnote{
Одна з заслуг Родбертуса, що до його важливої праці про ренту ми вернемося в книзі IV,
є в тому, що він розвинув цей пункт. Але, поперше, він помиляється, припускаючи, ніби для капіталу
зріст зиску завжди виявляється як і зріст капіталу, так що при збільшенні маси зиску відношення
лишається
те саме. А проте, це невірно, бо коли склад капіталу змінюється, норма зиску, не зважаючи на
незмінну експлуатацію праці, може підвищитись саме тому, що відносна вартість сталої частини
капіталу
проти змінної знизилася. — Подруге, він помиляється, трактуючи це відношення грошової ренти до
кількісно певної дільниці землі, наприклад, до одного акра, як щось таке, що взагалі припускає
тисячна
економія в її дослідженнях про підвищення або пониження ренти. Це знов невірно. Вона постійно
розглядає норму ренти у відношенні до продукту, — оскільки вона розглядає ренту в її натуральній
формі, — і у відношенні до авансованого капіталу, — оскільки вона розглядає ренту як грошову ренту,
—
бо це в дійсності є раціональні вирази.
}.

Відношення певної частини додаткової вартости, грошової ренти, — бо гроші
є самостійний вираз вартости, — до землі само по собі є безглузде й іраціональне;
бо це не співмірні величини, що тут виміряються одна одною, певна
споживна вартість, дільниця землі на стільки-от квадратових футів з одного
боку, і вартість, точніше, додаткова вартість — з другого. В дійсності це не виражає
нічого іншого, а тільки те, що в даних відносинах власність на стільки-от квадратових
футів землі дає земельному власникові можливість уловлювати певну кількість
неоплаченої праці, реалізованої капіталом, який риється на цих квадратових футах,
як свиня у картоплі (в рукопису тут стоїть в дужках, але закреслене: Лібіх). Але
prima facie цей вираз є те саме, як коли б ми здумали говорити про відношення
п’ятифунтової банкноти до діяметра землі. Однак, до посередництва тих іраціональних
форм, в яких виступають і на практиці резюмуються певні економічні
відносини, практичним носіям цих відносин у їхньому житті-бутті немає
жодного діла; а що вони привикли рухатися в цих посередницьких відносинах,
то їхній розум ані трохи не спотикається на них. Цілковита суперечність для них
не має рішуче нічого таємничого. У формах проявлення, відчужених від внутрішнього
зв’язку і безглуздих, коли їх узяти самих по собі, вони почувають себе
так само вдома, як риба у воді. Тут справедливе те, що Геґель сказав про
відомі математичні формули: те, що звичайний людський розум вважає за
і раціональне, є раціональне, а раціональне для нього є сама і раціональність.

Отже, коли розглядати справу у відношенні до самої площі землі, то підвищення
маси ренти виражається цілком так само, як підвищення норми ренти;
а звідси труднощі, що постають, коли умови, які пояснювали б один випадок,
відсутні в іншому випадку.

Але ціна землі може підвищитись навіть тоді, коли ціна продукту землі
зменшується.

В цьому випадку в наслідок дальшого диференціювання може збільшитися
диференційна рента, а тому й ціна кращих земель. Або ж коли цього немає, то
при збільшеній продуктивній силі праці ціна хліборобського продукту може понизитись,
але так, що це буде більш, ніж урівноважено збільшенням продукції. Припустімо,
що квартер коштував 60\shil{ шил.} Коли на тім самім акрі при тому самому
капіталі будуть випродуковані 2 квартери замість одного, і квартер понизиться
до 40 шил, то 2 квартери дадуть 80 шил, так що вартість продукту того самого
капіталу на тому самому акрі підвищиться на одну третину, хоч ціна акра
понизилась на одну третину. Як це можливо без того, щоб продукт продавався
вище його ціни продукції або вартости, було показано при дослідженні диференційної
ренти. В дійсності це можливо тільки в два способи. Або гіршу
землю вилучається з конкуренції, але ціна кращої землі зростає, коли
диференційна рента зростає, отже, коли загальне поліпшення діє нерівномірно
на різні роди землі. Або ж на найгіршій землі та сама ціна продукції
(і та сама вартість, коли виплачується абсолютну ренту) в наслідок підвищення
\index{iii2}{0210}  %% посилання на сторінку оригінального видання
продуктивности праці виражається у збільшеній масі продукту. Продукт
становить тепер ту саму вартість, що й давніш, але ціна його складових
частин понизилася, тимчасом, як число цих частин збільшилося. Коли вживається
той самий капітал, це неможливе, бо в цьому випадку та сама вартість
виражається в якій завгодно масі продукту. Але це можливе, коли витрачено
додатковий капітал на гіпс, гуано тощо, коротко кажучи, на такі поліпшення,
що вплив їхній триває багато років. Умова цього є в тому, щоб ціна одного квартера
хоч і знизилась, але не в такому самому відношенні, як зростає число квартерів.

III.~Ці різні умови підвищення ренти, а тому і ціни землі взагалі або окремих
родів землі, можуть почасти конкурувати між собою, почасти вони виключають
одна одну і можуть діяти лише навперемінки. Але з вище розвинутого, випливає,
що з підвищення ціни землі не можна без дальших околичностей робити
висновку, що рента підвищилась, і з підвищення ренти, яке завжди спричинює
підвищення ціни землі, не можна без дальших околичностей робити висновку,
що продукт землі збільшився\footnote{
Про падіння земельних цін при підвищенні ренти як про факт дивись Passy.
}.

\pfbreak

Замість звернутися до дійсних природних причин виснаження ґрунту, які,
проте, в наслідок стану хліборобської хемії в той час були невідомі усім економістам,
що писали про диферецційну ренту, — по допомогу звернулися до того
поверхового погляду, що в просторово обмежений лан не можна вкласти необмежену
масу капіталу; паприклад, Westminster Rewiew заперечує Річардові
Джонсові, що не можливо було б прогодувати цілу Англію обробітком Soho Square.
Хоч це вважається за особливу невигоду хліборобства, але справедливе як раз
зворотне. У хліборобстві можна продуктивно провадити послідовні приміщення
капіталу тому, що сама земля діє як знаряддя продукції, тимчасом як цього зовсім
немає, або є лише в дуже вузьких межах у випадку з фабрикою, де земля
функціонує лише як фундамент, як місце, як просторова операційна база. Правда,
можна — так і робить велика промисловість — саме на відносно невеликім, проти
парцельованого ремесла, просторі концентрувати велику продукційну споруду.
Але за даного ступеня розвитку продуктивної сили завжди потрібен певний простір,
і будування в висоту теж має свої певні практичні межі. Поширення продукції
за ці межі потребує і поширення простору землі. Основний капітал, вкладений
у машини тощо, не поліпшується споживанням, а навпаки, зношується. Внаслідок
нових винаходів і тут можуть статися окремі поліпшення, але, припускаючи
даний ступінь розвитку продуктивної сили, машина при споживанні може
лише погіршуватись. При швидкому розвитку продуктивної сили всю сукупність
старих машин доводиться заміняти вигіднішими, отже, вони гинуть. Навпаки,
земля, коли вона правильно обробляється, дедалі поліпшується. Та перевага
землі, що послідовні приміщення капіталу можуть дати вигоду без втрати колишніх,
одночасно має в собі можливість різної продуктивности цих послідовних
приміщень капіталу.

\section{Генеза капіталістичної земельної ренти}

\subsection{Вступ.}

Треба з’ясувати собі, в чому власне є труднощі трактування земельної
ренти з погляду сучасної економії, як теоретичного виразу капіталістичного
способу продукції. Цього ще не розуміє навіть величезне число новітніх письменників,
про що свідчить всяка нова спроба з’ясувати земельну ренту «по
новому». Новіша тут майже завжди є в повороті до давно вже побореного погляду.
\index{iii2}{0211}  %% посилання на сторінку оригінального видання
Трудність не в тому, щоб взагалі з’ясувати створений хліборобським
капіталом додатковий продукт і відповідну до нього додаткову вартість. Це
питання, радше, є вже розв’язане аналізою додаткової вартости, створюваної
всяким продуктивним капіталом, хоч би в яку сферу він був вкладений. Трудність
є в тому, що треба показати, звідки після того, як додаткова вартість
вирівнялась між різними капіталами на пересічний зиск, на відповідну до
їхніх відносних величин пропорційну частину всієї додаткової вартости, створеної
всім суспільним капіталом у всіх сферах продукції, — звідки після цього вирівняння,
після того як розподіл усієї додаткової вартости, яка взагалі може
бути розподілена, вже очевидно стався — звідки ж тут після цього береться
ще й та надмірна частина цієї додаткової вартости, яку капітал, вкладений
в землю, виплачує в формі земельної ренти земельному власникові.
Цілком лишаючи осторонь практичні мотиви, які спонукали сучасних економістів
як оборонців промислового капіталу проти земельної власности
досліджувати це питання, — мотиви, які ми накреслимо ближче в розділі
про історію земельної ренти, — це питання становило для них, як для теоретиків,
переважний інтерес. Визнати, що появлення ренти на капітал, вкладений
в хліборобство, завдячує особливій дії самої сфери приміщення, властивостям,
належним земній корі, як такій, це значило б відмовитись від самого
поняття вартости, отже, відмовитися від усякої можливости наукового
пізнання в цій галузі. Саме звичайне спостереження, що ренту виплачується
з ціни продукту землі, а це так і є навіть в тому випадку, коли її виплачується
в натуральній формі, скоро тільки орендар здобуває свою ціну продукції,
— показує, оскільки безглуздо надмір цієї ціни над звичайною ціною
продукції, отже, відносну дорожнечу хліборобського продукту, пояснювати надміром
природної продуктивности хліборобської промисловости над продуктивністю
інших галузей промисловости; бо, навпаки, що продуктивніша праця, то
дешевша кожна складова частина її продукту, тому що тим більша маса споживних
вартостей, в якій репрезентована та сама кількість праці, отже, та сама вартість.

Отже, при аналізі ренти вся трудність була в тому, що треба було
пояснити надмір хліборобського зиску над пересічним зиском, з’ясувати не додаткову
вартість, а властиву цій сфері продукції надмірну додаткову вартість,
отже, знов таки не «чистий продукт», а надмір цього чистого продукту над
чистим продуктом інших галузей промисловости. Сам пересічний зиск є продукт,
витвір процесу соціяльного життя, що відбувається в цілком певних історичних
продукційних відносинах, продукт, що має своєю передумовою, як ми
бачили, дуже широкосяжні посередницькі ланки. Для того, щоб взагалі можна було
говорити про надмір над пересічним зиском, сам цей пересічний зиск мусить
взагалі скластися як маштаб і — як це відбувається за капіталістичного способу
продукції, — як регулятор продукції. Отже, в таких суспільних формах, де ще
немає капіталу, який виконує ту функцію, що вимушує всю додаткову працю
і привласнює в першу чергу собі всю додаткову вартість, отже, де капітал ще
не упідлеглив собі суспільної праці, або упідлеглив її лише місцями, — взагалі
не може бути мови про ренту в сучасному значенні, про ренту як надмір
над пересічним зиском, тобто над пропорційною частиною всякого індивідуального
капіталу в додатковій вартості, спродукованій усім суспільним капіталом. Те
що, наприклад, пан Passy (дивись далі) говорить вже про ренту в первісному стані
як про надмір над зиском, як про надмір над історично-певного суспільною
формою додаткової вартости, так що за п. Passy ця форма могла б, мабуть,
існувати і без суспільства, — свідчить лише про його наївність.

Для колишніх економістів, які взагалі лише починали аналізу капіталістичного
способу продукції, ще нерозвиненого за їхнього часу, аналіза ренти
або взагалі не становила жодних труднощів, або становила лише труднощі цілком
\index{iii2}{0212}  %% посилання на сторінку оригінального видання
іншого характеру. Петті, Кантільйон, взагалі письменники, що ближче стоять
до доби февдалізму, беруть земельну ренту як нормальну форму додаткової
вартости взагалі, тимчасом як зиск для них ще не визначився і поєднується
з заробітною платою, або, щонайбільше, виступає як та частина цієї
додаткової вартости, що її капіталіст витискує з земельного власника. Отже, вони
виходять з такого стану, коли, поперше. хліборобська людність становить ще рішучу
переважну частину нації, і коли, подруге, земельний власник ще є тією
особою, яка, користуючись монополією земельної власности, у першу чергу привласнює
надмірну працю беспосередніх продуцентів, коли, отже, земельна власність
все ще є головна умова продукції. Для них ще не могло існувати такої
постави питання, що, навпаки, з погляду капіталістичного способу продукції,
намагається дослідити, яким чином земельна власність досягає того, що
віднімає від капіталу частину спродукованої ним (тобто вичавленої з безпосередніх
продуцентів) і в першу чергу привласненої вже ним додаткової вартости.

\emph{У фізіократів} труднощі вже іншого характеру. Як дійсно перші систематичні
тлумачі капіталу, вони намагалися аналізувати природу додаткової вартости
взагалі. Для них ця аналіза збігається з аналізою ренти, однісінької
форми, в якій для них існує додаткова вартість. Капітал, що дає ренту, або
хліборобський капітал, є для них однісінький капітал, що продукує додаткову
вартість, і пущена ним в рух хліборобська праця є однісінька, що створює додаткову
вартість, отже, з капіталістичного погляду цілком послідовно однісінька
продуктивна праця. Продукцію додаткової вартости вони цілком слушно вважають
за визначальний момент. Їм, залишаючи осторонь інші заслуги, про які мова
буде в книзі ІV, належить насамперед та велика заслуга, що від торговельного
капіталу, який функціонує тільки в сфері циркуляції, вони звернулись до продуктивного
капіталу, протилежно до меркантильної системи, яка за своїм грубим
реалізмом була справжньою вульґарною економією тієї доби, що її практичними
інтересами було відсунуто цілком на задній плян початки наукової аналізи
у Петті та його послідовників. Між іншим, тут, при критиці меркантильної системи,
мова йде лише про її погляди на капітал та додаткову вартість. Вже
давніш ми відзначали, що продукцію на світовий ринок і перетворення продукту
на товар, а тому і на гроші, монетарна система справедливо проголосила за передумову
і умову капіталістичної продукції. В її продовженні, в меркантильній системі,
переважну ролю відіграє вже не перетворення товарової вартости на гроші, а створення
додаткової вартости, але розглядається воно з іраціонального погляду сфери
циркуляції, до того ж так, що ця додаткова вартість виступає в формі додаткових
грошей, в надмірі торговельного балансу. Разом з тим справді характеристичне
для заінтересованих купців і фабрикантів того часу і адекватне тому періодові
капіталістичного розвитку, який вони репрезентують, є те, що при перетворенні
хліборобських февдальних громад на промислові, і при відповідній промисловій
боротьбі націй на світовому ринку, справа залежить від прискореного розвитку
капіталу, що досягається не так званим природним шляхом, а примусовими заходами.
Величезна ріжниця є в тому, чи перетворюється національний капітал на промисловий
поступово і повільно, чи це перетворення прискорюється в часі, в наслідок податків,
що ними вони в формі охоронних мит оподатковували переважно земельних
власників, середніх і дрібних селян і ремесло, в наслідок прискореної експропріяції
самостійних безпосередній продуцентів, в наслідок насильницької прискореної
акумуляції і концентрації капіталів, коротко, в наслідок прискореного
створення умов капіталістичного способу продукції. Разом з тим це становить
величезну ріжницю в капіталістичній і промисловій експлуатації природної національної
продуктивної сили. Тому національний характер меркантильної системи
в устах її оборонців є не просто фраза. З тієї притоки, що їх ніби цікавить тільки
багатство нації та допоміжні ресурси держави, вони в дійсності проголошують
\parbreak{}  %% абзац продовжується на наступній сторінці

\parcont{}  %% абзац починається на попередній сторінці
\index{iii2}{0213}  %% посилання на сторінку оригінального видання
інтереси кляси капіталістів і збагачення взагалі за конечну мету держави і прокламують
буржуазне суспільство протилежно старій надземній державі. Але разом
з цим виявляється свідомість того, що розвиток інтересів капіталу і кляси капіталістів,
капіталістичної продукції, зробився за базу національної сили і національної
переваги в сучасному суспільстві.

Далі, у фізіократів справедливе те, що на ділі вся продукція додаткової вартости,
отже, і ввесь розвиток капіталу, розглядуваний з боку природної бази,
ґрунтується на продуктивності хліборобської праці. Коли б люди взагалі не могли
продукувати протягом одного робочого дня більше засобів існування, отже, в вузькому
розумінні, більше хліборобських продуктів, ніж потрібно кожному робітникові
для його власної репродукції, — коли б денної витрати всієї його робочої сили
було досить лише для того, щоб випродукувати засоби існування, потрібні для
його особистого споживання, то взагалі не могло б бути мови ані про додатковий
продукт, ані про додаткову вартість. Продуктивність хліборобської праці, що
перебільшує індивідуальну потребу робітника, є база всякого суспільства, і насамперед
база капіталістичної продукції, яка дедалі більшу частину суспільства
відриває від продукції безпосередніх засобів існування і перетворює її, за висловом
Стюарта, в free heads\footnote*{
Free heads — дослівно вільні голови, тобто вільні робочі руки. Прим. Ред.
}, дає можливість користатися нею в інших сферах.

Але, що сказати про тих новіших письменників-економістів, котрі як
Daire, Passy та інші, на схилі життя всієї клясичної економії, навіть на її смертельній
постелі, повторюють найпервісніші уявлення про природні умови додаткової
праці, і, отже, додаткової вартости взагалі, і гадають, ніби вони цим дають
щось нове і переконливе про земельну ренту після того як цю земельну ренту
вже давно описано як осібну форму і специфічну частину додаткової вартости?
Саме для вульгарної економії характеристичне таке: те, що на певнім пережитім ступені
розвитку було нове, оригінальне, глибоке і слушне, вона повторює в
такий час, коли воно є тривіяльне, відстале і фалшиве. Вона визнає таким
чином, що в неї не має навіть передчуття про проблеми, які цікавили клясичну
економію. Вона сплутує їх з питаннями, що могли ставитись лише на нижчому
ступені розвитку буржуазного суспільства. Так само стоїть справа з її безнастанним
та самозадоволеним пережовуванням фізіократичних засад про вільну
торгівлю. Ці засади давно втратили всякий теоретичний інтерес, хоч би як
практично вони цікавили ту або іншу державу.

У власне натуральному господарстві, де хліборобський продукт зовсім не
вступає в процес циркуляці, або вступає в нього лише дуже незначна частина
цього продукту, і навіть лише порівняно незначна частка тієї частини продукту,
яка становить дохід земельного власника, — як наприклад, в багатьох
староримських лятифундіях, в віллах Карла Великого, а також (дивись Vincard,
Histoire du travail) в більшій чи меншій мірі протягом усього середньовіччя, —
продукт і додатковий продукт великих маєтків зовсім не складається тільки
з продуктів хліборобської праці. Він охоплює також і продукти промислової
праці. Домашня реміснича і мануфактурна праця, як допоміжна продукція
при хліборобстві, що становить базу, є умова того способу продукції, на якому
ґрунтується це натуральне господарство так у давній і середньовічній Европі, якще
до нашого часу — і в індійській громаді, де її традиційна організація ще не
зруйнована. Капіталістичний спосіб продукції цілком знищує це сполучення:
процес, який у великому маштабі можна вивчити особливо на прикладі Англії
за останню третину XVIII століття. Голови, що виросли у більш чи менш напівфевдальних
суспільствах, Гереншванд, наприклад, ще в кінці XVIII століття
вбачають у цьому відокремленні хліборобства від мануфактури одчайдушно сміливий
суспільний експеримент, незрозуміло ризикований спосіб існування. І навіть
\parbreak{}  %% абзац продовжується на наступній сторінці

\parcont{}  %% абзац починається на попередній сторінці
\index{iii1}{0214}  %% посилання на сторінку оригінального видання
показано, чому це зниження виступає не в цій абсолютній формі,
а більше в тенденції до прогресивного падіння.) Отже, прогресуюча
тенденція загальної норми зиску до зниження є тільки
\emph{властивий} \emph{капіталістичному способові виробництва вираз}
прогресуючого розвитку суспільної продуктивної сили праці.
Цим не сказано, що норма зиску не може тимчасово падати і
з інших причин, але цим доведено, як само собою зрозумілу
з суті капіталістичного способу виробництва необхідність, що
з розвитком цього способу виробництва загальна пересічна норма
додаткової вартості мусить виражатись у падаючій загальній
нормі зиску. Через те що маса вживаної живої праці постійно
зменшується порівняно з масою упредметненої праці, яку вона
приводить в рух, порівняно з масою продуктивно споживаних
засобів виробництва, то й відношення тієї частини цієї живої
праці, яка неоплачена і упредметнюється в додатковій вартості,
до розміру вартості всього вживаного капіталу мусить постійно
зменшуватись. Але це відношення маси додаткової вартості до
вартості всього вживаного капіталу становить норму зиску, яка
через це мусить постійно падати.

Хоч і яким простим здається цей закон після того, що ми досі
розвинули, проте всій дотеперішній політичній економії не вдалося
відкрити його, як ми це побачимо в одному з дальших відділів.
Вона бачила явище і мучилася в суперечливих спробах
пояснити його. Але при тій великій важливості, яку цей закон
має для капіталістичного виробництва, можна сказати, що він
становить таємницю, над розв’язанням якої б’ється вся політична
економія від часів Адама Сміта, і що ріжниця між різними школами
від часів А. Сміта полягає в різних спробах розв’язати цю
таємницю. З другого ж боку, якщо взяти до уваги, що дотеперішня
політична економія хоч напомац і підходила до розрізнення
сталого і змінного капіталу, але ніколи не спромоглась
ясно сформулювати його; що вона ніколи не представляла додаткову
вартість відокремлено від зиску, а зиск взагалі ніколи
не представляла у чистому вигляді в відміну від його різних
усамостійнених одна проти одної складових частин, — як промисловий
зиск, торговельний зиск, процент, земельна рента; що
вона ніколи грунтовно не аналізувала ріжниці в органічному
складі капіталу, а тому й утворення загальної норми зиску, —
то перестає бути загадковим те, що їй ніколи не вдавалося розв’язати
цю загадку.

Ми навмисно виклали цей закон раніше, ніж показали розпад
зиску на різні усамостійнені одна проти одної категорії. Незалежність
цього викладу від розпаду зиску на різні частини,
які припадають різним категоріям осіб, прямо доводить незалежність
закону в його всезагальності від такого розпаду і від
взаємних відношень між категоріями зиску, які виникають з цього
розпаду. Зиск, про який ми тут говоримо, є тільки інша назва
самої додаткової вартості, яка тільки представлена у відношенні
\parbreak{}  %% абзац продовжується на наступній сторінці

\parcont{}  %% абзац починається на попередній сторінці
\index{iii1}{0215}  %% посилання на сторінку оригінального видання
до всього капіталу, а не у відношенні до змінного капіталу, з
якого вона виникає. Отже, падіння норми зиску виражає спадаюче
відношення самої додаткової вартості до всього авансованого
капіталу, і тому воно незалежне від будь-якого розподілу
цієї додаткової вартості між різними категоріями.

Ми бачили, що на певному ступені капіталістичного розвитку,
коли склад капіталу $c : v \deq{} 50 : 100$, норма додаткової вартості
в 100\% виражається в нормі зиску в 66\sfrac{2}{3}\% і що на вищому
ступені розвитку, коли $c : v$ як $400 : 100$, та сама норма додаткової
вартості виражається в нормі зиску тільки в 20\%. Те, що
стосується до різних послідовних ступенів розвитку в одній
країні, стосується і до різних ступенів розвитку, які існують
одночасно один поряд одного в різних країнах. У нерозвиненій
країні, де перший склад капіталу є пересічний, загальна норма
зиску була б \deq{} 66\sfrac{2}{3}\%, тимчасом як у країні другого складу капіталу,
з значно вищим ступенем розвитку, вона була б \deq{} 20\%.

Ріжниця обох національних норм зиску могла б зникнути і
навіть стати протилежною в наслідок того, що в менш розвиненій
країні праця була б менш продуктивною, тому більша
кількість праці виражалася б у меншій кількості того самого
товару, більша мінова вартість виражалася б у меншій споживній
вартості, отже, робітник мусив би вживати більшу частину
свого часу на репродукцію своїх власних засобів існування або
їх вартості і меншу частину на створення додаткової вартості,
давав би менше додаткової праці, так що норма додаткової
вартості була б нижча. Якщо, наприклад, у менш розвиненій країні
робітник працював би \sfrac{2}{3} робочого дня на себе самого і \sfrac{1}{3} на
капіталіста, то, зберігаючи припущення вищенаведеного прикладу,
та сама робоча сила оплачувалася б у розмірі 133\sfrac{1}{3} і дала б
надлишок тільки в 66\sfrac{2}{3}. Змінному капіталові в 133\sfrac{1}{3} відповідав
би сталий капітал в 50. Отже, норма додаткової вартості становила
б тут $133\sfrac{1}{3} : 66\sfrac{2}{3}= 50\%$, а норма зиску $183\sfrac{1}{3}:
66\sfrac{2}{3}$ або приблизно 36\sfrac{1}{2}\%.

Через те що ми досі ще не дослідили різних складових частин,
на які розпадається зиск, — отже, вони для нас ще не існують,
— то ми тільки для того, щоб уникнути непорозумінь,
зауважимо наперед таке. При порівнянні країн різних ступенів
розвитку, а саме країн з розвиненим капіталістичним виробництвом
і таких, де праця ще формально не підпорядкована капіталові,
хоча в дійсності робітник експлуатується капіталістом
(наприклад, в Індії, де райот господарює як самостійний селянин,
отже, його виробництво, як таке, ще не підпорядковане капіталові,
хоч лихвар може видушити з нього в формі процента
не тільки всю його додаткову працю, але навіть — капіталістично
висловлюючись — частину його заробітної плати), було б
великою помилкою, коли б хтонебудь схотів міряти висоту національної
норми зиску висотою національного рівня процента.
В такому проценті міститься весь зиск і навіть більше ніж зиск,
\parbreak{}  %% абзац продовжується на наступній сторінці

\input{_0216.tex}
\input{_0217.tex}
\parcont{}  %% абзац починається на попередній сторінці
\index{iii2}{0218}  %% посилання на сторінку оригінального видання
панщиною або крепаків. Тимчасом ясно, що при тому примітивному і нерозвиненому
стані, на якому ґрунтується це суспільне продукційне відношення і
відповідний йому спосіб продукції, традиція мусить відігравати переважну ролю.
Далі ясно, що тут, як і всюди, переважна частина суспільства заінтересована
в тому, щоб усвячувати суще, як закон, і ті його межі, які дано звичаєм і
традицією, фіксувати як законні. Проте, лишаючи все інше осторонь, це стається
само собою, скоро постійна репродукція бази сущого стану, відношення,
що лежить в його основі, набуває з перебігом часу уреґульованої і упорядкованої
форми; і ця уреґульованість і цей порядок сами є доконечний момент всякого
способу продукції, коли він має набути суспільної сталости і незалежности від
звичайного випадку або сваволі. Уреґульованість і порядок є саме форма суспільного
зміцнення даного способу продукції, і тому його відносної емансипації
від просто сваволі і звичайного випадку. Він досягає цієї форми при застійному
стані так процесу продукції, як і відповідних до нього суспільних відносин
через просту повторну репродукцію їх самих. Коли ця форма проіснувала протягом
певного часу, вона зміцнюється, як звичай і традиція, і нарешті усвячується
як виразний закон. А що форма цієї додаткової праці, панщинна праця,
ґрунтується на нерозвиненості всіх суспільних продуктивних сил праці, на
примітивності самого способу праці, то і мусить вона природно віднімати у
безпосереднього продуцента незрівняно меншу відповідну частину всієї праці,
ніж за розвинених способів продукції й особливо за капіталістичної продукції.
Припустімо, наприклад, що панщинна праця на земельного власника первісно становила
два дні на тиждень. Ці два дні панщинної праці на тиждень таким чином
усталились, вони є стала величина, законно уреґульована звичаєвим або писаним
правом. Але продуктивність решти днів тижня, що ними може порядкувати
сам безпосередній продуцент, є величина змінна, яка мусить розвиватися в
процесі його досвіду, — цілком так само, як нові потреби, з якими він знайомиться,
цілком так само як поширення ринку для його продукту, ростуча забезпеченість
порядкування для самого себе цією частиною своєї робочої сили,
підганятиме його до підвищеного напруження робочої сили, при чому не слід
забувати, що вживання цієї робочої сили зовсім не обмежується хліборобством,
але охоплює й сільську домашню промисловість. Тут дана можливість певного
економічного розвитку, зрозуміла річ, залежно від більш або менш сприятливих
обставин, від природженого расового характеру тощо.

\subsubsection{Рента продуктами}

Перетворення відробітної ренти на ренту продуктами, економічною мовою
висловлюючись, нічого не змінює в суті земельної ренти. Суть земельної ренти при
таких умовах, які ми розглядаємо тут, в тому, що земельна рента є однісінька
панівна і нормальна форма додаткової вартости, або додаткової праці; а це в
свою чергу виражається в тому, що вона становить однісіньку додаткову працю
або однісінький додатковий продукт, що його безпосередній продуцент, який
\emph{посідає} умови праці, що потрібні для його власної репродукції, повинен
дати \emph{власникові} такої умови праці, яка в цьому стані охоплює все, тобто
власникові землі; і що з другого боку тільки земля і протистоїть йому, як
умова праці, що перебуває в чужій власності, відокремлена проти нього і
персоніфікована у земельному власникові. Коли рента продуктами становить
панівну і найрозвиненішу форму земельної ренти, вона все ж постійно в більшій
або меншій мірі супроводиться рештками попередньої форми, тобто ренти, що
її виплачується безпосередньо працею, отже, панщинною працею, і це однаково,
чи є земельним власником приватна особа чи держава. Рента продуктами має
своєю передумовою вищий культурний рівень безпосереднього продуцента, отже
\parbreak{}  %% абзац продовжується на наступній сторінці

\input{_0219.tex}
\parcont{}  %% абзац починається на попередній сторінці
\index{ii}{0220}  %% посилання на сторінку оригінального видання
\frac{\num{25.000}}{5} \deq{} 5000\pound{ ф. стерл}. Коли поділити ці 5000\pound{ ф. стерл.} на 500, то матимемо число оборотів 10,
цілком таке саме, як і для цілого капіталу в 2500\pound{ ф. стерл}.

Це пересічне обчислення, що за ним вартість річного продукту ділиться на вартість авансованого
капіталу, а не на вартість частини цього капіталу, постійно застосовуваної в одному робочому періоді
(отже, в нашому прикладі, не на 400, а на 500, не на капітал І, а на капітал І \dplus{} капітал II), — це
пересічне обчислення тут, де йдеться лише про продукцію додаткової вартости, є абсолютно точне. Далі
ми побачимо, що, з іншого погляду, воно не зовсім точне, як і взагалі це пересічне обчислення не
зовсім точне. Інакше кажучи, воно задовільне для практичних цілей капіталіста,
але воно не виражає точно й гаразд усіх реальних обставин обороту.

Досі ми одну частину вартости товарового капіталу лишали цілком осторонь, а саме вміщену в ньому
додаткову вартість, спродуковану та долучену до продукту протягом процесу продукції. На неї тепер і
треба нам звернути увагу.

Коли припустити, що витрачуваний щотижня змінний капітал в 100\pound{ ф. стерл.}, продукує додаткову
вартість в 100\% \deq{} 100\pound{ ф. стерл.}, то змінний капітал в 500\pound{ ф. стерл.},  витрачуваний протягом
п’ятитижневого періоду обороту, випродукує додаткову вартість в 500\pound{ ф. стерл.}, тобто половина
робочого дня складається з додаткової праці.

Але коли 500\pound{ ф. стерл.} змінного капіталу продукують 500\pound{ ф. стерл.} додаткової вартости, то 5000\pound{ ф.
стерл.} випродукують її 500 × 10 \deq{} 5000\pound{ ф. стерл}. Але авансований змінний капітал \deq{} 500\pound{ ф. стерл}.
Відношення всієї маси додаткової вартости, спродукованої протягом року, до суми вартости
авансованого змінного капіталу ми звемо річною нормою додаткової вартости. Отже, в даному випадку,
вона \deq{} \frac{5000}{500} \deq{} 1000\%.
Коли ближче аналізувати цю норму, то виявиться, що вона дорівнює тій нормі додаткової вартости, яку
авансований змінний капітал продукує протягом одного періоду обороту, помноженій на число оборотів
змінного капіталу (а воно збігається з числом оборотів цілого обігового капіталу).

Авансований протягом одного періоду обороту змінний капітал в даному випадку \deq{} 500\pound{ ф. стерл.};
створена ним додаткова вартість теж \deq{} 500\pound{ ф. стерл}. Тому норма додаткової вартости протягом одного
періоду обороту \deq{} \frac{500m}{500v} \deq{} 100\%. Ці 100\%, помножені на 10, на число оборотів протягом року,
дають \frac{5000m}{5000v} \deq{} 1000\%.

Це має силу щодо річної норми додаткової вартости. Щождо маси додаткової вартости, здобуваної
протягом певного періоду обороту, то ця маса дорівнює вартості авансованого протягом цього періоду
змінного капіталу — в даному випадку \deq{} 500\pound{ ф. стерл.}, помноженій на норму
\parbreak{}  %% абзац продовжується на наступній сторінці

\parcont{}  %% абзац починається на попередній сторінці
\index{ii}{0221}  %% посилання на сторінку оригінального видання
додаткової вартости, в даному випадку, отже, 500 × \frac{100}{100} \deq{} 500 × 1 \deq{} 500\pound{ ф. стерл}. Коли б
авансований капітал був \deq{} 1500\pound{ ф. стерл.} при незмінній
нормі додаткової вартости, то маса додаткової вартости була б \deq{}
1500 × \frac{100}{100} \deq{} 1500\pound{ ф. стерл}.

Змінний капітал у 500\pound{ ф. стерл.}, що обертається 10 разів на рік, і
що продукує протягом року додаткову вартість в 5000\pound{ ф. стерл.}, отже,
капітал, що для нього річна норма додаткової вартости \deq{} 1000\%, ми
будемо називати капіталом А.

Припустімо тепер, що інший змінний капітал В в 5000\pound{ ф. стерл.}
авансується на цілий рік (тобто, тут на 50 тижнів) і тому обертається
лише один раз на рік. Припустімо при цьому далі, що наприкінці року
продукт оплачується в той самий день, як його виготовлено, і, значить,
грошовий капітал, що на нього його перетворюється, повертається в той
самий день. Отже, період циркуляції тут \deq{} 0, період обороту дорівнює
робочому періодові, а саме, одному рокові. Як і в попередньому випадку,
в процесі праці щотижня перебуває змінний капітал в 100\pound{ ф. стерл.},
а тому протягом 50 тижнів — в 5000\pound{ ф. стерл}. Далі, норма додаткової
вартости хай буде та сама \deq{} 100\%, тобто за однакової довжини робочого
дня половина його складається з додаткової праці. Коли ми візьмемо
5 тижнів, то вкладений змінний капітал \deq{} 500\pound{ ф. стерл.}, норма додаткової
вартости \deq{} 100\%, отже, маса додаткової вартости, створена протягом
5 тижнів \deq{} 500\pound{ ф. стерл}. Кількість робочої сили, що її тут експлуатується,
і ступінь її експлуатації, згідно з нашим припущенням, тут
точно такі самі, як і при капіталі А.

Вкладений змінний капітал в 100\pound{ ф. стерл.} щотижня створює додаткову
вартість в 100\pound{ ф. стерл.}, тому протягом 50 тижнів вкладений капітал
в 100 × 50 \deq{} 5000\pound{ ф. стерл.} створить додаткову вартість в 5000\pound{ ф. стерл}. Маса щороку створюваної
додаткової вартости буде така сама, як і в попередньому випадку \deq{} 5000\pound{ ф. стерл.}, але річна норма
додаткової
вартости цілком інша. Вона дорівнює спродукованій протягом року
додатковій вартості, поділеній на авансований змінний капітал:
\frac{5000m}{5000v} \deq{} 100\%, тимчасом як раніш для капіталу А вона дорівнювала 1000\%.

При капіталі А, як і при капіталі В, ми витрачали щотижня 100\pound{ ф. стерл.} змінного капіталу; ступінь
зростання вартости або норма додаткової
вартости цілком та сама, вона дорівнює 100\%; величина змінного
капіталу теж та сама \deq{} 100\pound{ ф. стерл}. Експлуатується цілком таку
саму кількість робочої сили, величина й ступінь експлуатації в обох випадках
однакові, робочі дні однакові і однаково поділяються на доконечну
й додаткову працю. Сума змінного капіталу, застосованого протягом
року, однакова величиною \deq{} 5000\pound{ ф. стерл.}, вона пускає в рух таку
саму масу праці й витягує з робочої сили, пущеної в рух обома рівними
капіталами, однакову масу додаткової вартости, 5000\pound{ ф. стерл}. І,
\parbreak{}  %% абзац продовжується на наступній сторінці

\input{_0222.tex}
\input{_0223.tex}
\parcont{}  %% абзац починається на попередній сторінці
\index{ii}{0224}  %% посилання на сторінку оригінального видання
вартости, хоч яке різне буде відношення цього змінного капіталу, застосованого
протягом певного часу, до змінного капіталу, авансованого на
той самий час, і значить, хоч яке різне буде також і відношення утворених
мас додаткової вартости, не до застосованого, а до взагалі авансованого
змінного капіталу. Неоднаковість цього відношення, замість суперечити
розвиненим законам продукції додаткової вартости, навпаки,
потверджує їх і є неминучий наслідок їх.

Розгляньмо перший п’ятитижневий період продукції капіталу \emph{В}. Наприкінці
5-го тижня 500 ф. стерл. застосовано й зужито. Новостворена
вартість = 1000 ф. стерл., отже, $\frac{500m}{500v}=100\%$. Цілком так, як при капіталі \emph{А}.

Та обставина, що при капіталі \emph{А} додаткова вартість реалізується разом з
авансованим капіталом, а при \emph{В} — ні, нас тут покищо не обходить, бо
тут покищо йдеться лише про продукцію додаткової вартости і про
відношення її до змінного капіталу, авансованого під час її продукції.
Навпаки, коли ми обчислимо відношення додаткової вартости \emph{В} не
до тієї частини авансованого капіталу в 5000 ф. стерл., що її застосовано й
тому зужито протягом продукції цієї додаткової вартости, а до самого
цього цілого авансованого капіталу, то матимемо $\frac{500m}{5000v} = \sfrac{1}{10} = 10\%$.

Отже, для капіталу \emph{В} 10\%, а для капіталу \emph{А} 100\%, тобто вдесятеро
більше. Коли б тут сказали: така ріжниця в нормі додаткової вартости
для однакових величиною капіталів, що пускають у рух однакову кількість
праці, та ще праці, яка однаковою мірою поділяється на оплачену й
неоплачену, суперечить законам продукції додаткової вартости, — то
відповідь була б проста й випливала б з першого погляду на фактичні
відношення: для \emph{А} виражається справжня норма додаткової вартости,
тобто відношення додаткової вартости, спродукованої протягом 5 тижнів
змінним капіталом в 500 ф., до цього змінного капіталу в 500 ф. стерл.
Для \emph{В}, навпаки, обчислення робиться таким способом, що не має жодного
чинення ні до продукції додаткової вартости, ні до відповідного їй
визначення норми додаткової вартости. 500 ф. стерл. додаткової вартости,
спродуковані змінним капіталом у 500 ф. стерл., обчислюється
власне не в їхньому відношенні до 500 ф. стерл. змінного капіталу, авансованого
протягом продукції цієї додаткової вартости, а в їхньому відношенні
до капіталу в 5000 ф. стерл., що \sfrac{9}{10} його, 4500 ф. стерл., не мають
жодного чинення до продукції цієї додаткової вартости в 500 ф. стерл., а
скорше мають лише поступінно функціонувати протягом наступних 45 тижнів;
отже, вони зовсім не існують для продукції протягом перших 5 тижнів,
що про них тільки й мовиться тут. Отже, в цьому випадку ріжниця
в нормі додаткової вартости капіталів \emph{А} і \emph{В} не становить жодної
проблеми.

Порівняймо тепер річні норми додаткової вартости для капіталів \emph{В} і \emph{А}.
Для капіталу \emph{В} ми маємо $\frac{5000m}{5000v} = 100\%$;
для капіталу \emph{А} $\frac{5000m}{500v} = 1000\%$.
\parbreak{}  %% абзац продовжується на наступній сторінці

\input{_0225.tex}
\input{_0226.tex}
\index{i}{0227}  %% посилання на сторінку оригінального видання

Звичайно, всі ці викрути нічого не помогли. Фабричні інспектори
вдалися до суду. Але незабаром на міністра внутрішніх
справ сера Джорджа Грея спала така хмара петицій від фабрикантів,
що в обіжнику з 5 серпня 1848~\abbr{р.} він наказав інспекторам
«не позивати взагалі за порушення букви закону, поки не буде
доведене зловживання Relaissystem’ою з метою примусити підлітків
і жінок працювати понад десять годин». Після цього фабричний
інспектор Ф.~Стюарт дозволив так звану систему змін протягом
п’ятнадцятигодинного періоду фабричного дня для цілої Шотляндії,
де вона незабаром знов розцвіла, як колись. Навпаки,
англійські фабричні інспектори заявили, що міністер не має
жодної диктаторської влади припинити чинність закону, і далі
провадили судові переслідування проти Proslavery rebels.

Алеж нащо було притягати до суду, коли суди, county magistrates\footnote{
Ці «county magistrates», «great unpaid»\footnote*{величні неоплачувані.  \emph Ред. },
як їх називає В.~Кобe — це щось наче безплатні мирові судді, що їх обирають із почесних
осіб графства. В дійсності вони являють собою патримоніяльні суди панівних
кляс.
},
виправдували притягуваних до права? По цих судах
засідали пани фабриканти, щоб самих себе судити. Ось приклад.
Якийсь Іскрідж із бавовнопрядної фірми Кершоу, Лізе і К°
подав був фабричному інспекторові своєї округи схему Relaissystem,
призначену для його фабрики. Одержавши відмову,
він спочатку тримався пасивно. Декілька місяців пізніш якийсь
індивід, на ім’я Робінзон, теж бавовняник, і коли не П’ятниця,
то в усякому разі родич Іскріджа, став перед Borough Justices\footnote*{мировими суддями. \emph {Ред.}}
у Стокпорті, обвинувачуваний у тому, що завів у себе таку
систему змін, яку вигадав Іскрідж. Засідало четверо суддів,
серед них три бавовняні фабриканти, з тим самим неминучим
Іскріджем на чолі. Іскрідж виправдав Робінзона й заявив: що є
законоправне для Робінзона, те справедливе й для Іскріджа.
Покликаючись на свій власний судовий присуд, що набрав правної
сили, він зараз же завів цю систему й на своїй власній фабриці\footnote{
«Reports etc. for 30 th April 1849», p. 21, 22. Порівн. подібні
приклади там же, crop. 4, 5.
}.
Певна річ, уже самий склад таких суддів був явним порушенням
закону\footnote{
Законом 1 і 2 Вільяма IV, с. 24, s. 10, відомим під назвою фабричного
закону сера Джона Гобговза, забороняється кожному посідачеві
бавовнопрядної або ткацької фабрики, а також і батькові, синові або
братові такого посідача виконувати обов'язки мирового судді в питаннях,
які стосуються до фабричного закону.
}. «Такі судові фарси, — каже інспектор Хоуелл, —
аж волають по ліки\dots{} або пристосуйте закон до таких присудів,
або віддайте вирішення справ не такому вже порочному трибуналові,
який свої присуди пристосував би до закону\dots{} в усіх таких
випадках. Дуже бажано, щоб посада судді була платна!»\footnote{
«Reports etc. for 30 th April 1849».
}

Коронні юристи проголосили фабрикантську інтерпретацію
закону 1848~\abbr{р.} за недоладну, але рятівники суспільства не дали
\parbreak{}  %% абзац продовжується на наступній сторінці

\parcont{}  %% абзац починається на попередній сторінці
\index{i}{0228}  %% посилання на сторінку оригінального видання
себе збити з пантелику. «Після того, — оповідає Леонард Горнер, —
як я спробував примусити виконувати закон, розпочавши 10 процесів
у 7 різних судових округах, і лише в одному випадку найшов
підтримку в суддів\dots{} я вважаю за некорисні дальші переслідування
за оминання закону. Та частина закону, що її укладено
з метою створити одностайність у робочих годинах\dots{} вже не існує
більше в Ланкашірі. Так само я абсолютно не маю, як і мої помічники,
ніяких засобів, щоб упевнитися, що по тих фабриках, де
панує так звана Relaissystem, підлітків і жінок не примушують
працювати більш як 10 годин. Наприкінці квітня 1849~\abbr{р.} вже
114 фабрик у моїй окрузі працювали за цією методою, і число
їх останніми часами швидко зростає. Загалом же вони працюють
тепер 13\sfrac{1}{2} годин, від шостої години ранку до пів на восьму вечора;
в деяких випадках вони працюють 15 годин, від пів на шосту
ранку до пів на дев’яту вечора»\footnote{
«Reports etc. for 30 th April 1849», p. 5.
}. Вже у грудні 1848~\abbr{р.} Леонард
Горнер мав список 65 фабрикантів і 29 фабричних доглядачів,
які одноголосно заявляли, що жодна система контролю не може
за такої системи змін перешкодити поширенню якнайінтенсивнішої
надмірної праці\footnote{
«Reports etc. for 31 st October 1849», p. 6.
}. То тих самих дітей і підлітків переводять
із прядільні до ткальні й~\abbr{т. д.}, то протягом 15 годин їх
кидають (shifted) з однієї фабрики до однієї\footnote{
«Reports etc. for 30 th April 1849», p. 21.
}. Як можна контролювати
таку систему змін, «яка зловживає словом зміна, щоб із
безмежною різноманітністю перемішувати робочі руки, як карти,
і день-у-день так пересовувати години праці й відпочинку
різних осіб, що один і той самий повний асортимент рук ніколи
не працює разом на тому самому місці в той самий час»!\footnote{
«Reports etc. for 1 st December 1848», p. 95.
}

Але й залишаючи цілком осторонь дійсну надмірну працю,
ця так звана система змін була таким витвором фантазії капіталу,
що його ніколи не перевищив Фур’є у своїх гумористичних
нарисах «courtes séances»\footnote*{
коротких сеансів. \Red{Ред.}
}, з тією лише ріжницею, що притягання
праці тут перетворилося на притягання капіталу. Подивімось
на ці схеми, утворені фабрикантами і прославлені добрячою
пресою як зразок того, «що можна зробити з розумною мірою
дбайливости й методичности» («what a reasonable degree of care
and method can accomplish»). Робочий персонал розділювано
іноді на 12--15 категорій, що знову раз-у-раз зміняли свої
складові частини. Протягом п’ятнадцятигодинного періоду фабричного
дня капітал притягав робітника то на 30 хвилин, то на
годину, потім знову відштовхував його, щоб знову притягти
його на фабрику й знов одштовхнути, ганяючи його то туди, то сюди
розрізненими шматками часу, але постійно не випускаючи його
із своїх рук, доки десятигодинну працю не буде цілком закінчено.
Як на театральній сцені, мали виступати ті самі особи навпереміну
в різних явах різних дій. Але, як актор належить до
\parbreak{}  %% абзац продовжується на наступній сторінці

\parcont{}  %% абзац починається на попередній сторінці
\index{i}{0229}  %% посилання на сторінку оригінального видання
сцени протягом усього часу тривання драми, так і робітники належали
тепер до фабрики протягом 15 годин, не рахуючи часу на
дорогу до фабрики й назад. Таким чином години відпочинку перетворювалися
на години примусового безділля, що гнали молодого
робітника до шинку, а молоду робітницю в дім розпусти. За
кожної нової витівки, що її день-у-день вигадував капіталіст,
щоб тримати свої машини в русі 12 або 15 годин, не збільшуючи
робочого персоналу, робітник мусів проковтнути свою їжу то в
той, то в інший шматок часу. Під час агітації за десятигодинний
робочий день фабриканти кричали, що робітнича наволоч подає
петиції, сподіваючись дістати за десятигодинну працю дванадцятигодинну
заробітну плату. Тепер вони обернули медалю. Вони
виплачували десятигодинну заробітну плату за дванадцяти й
п’ятнадцятигодинне порядкування робочими силами!\footnote{
Див. «Reports etc. for 30 th April 1849», p. 6 і докладне пояснення
«shifting system»\footnote*{
— системи пересувань. \emph{Ред.}
}, яке фабричні інспектори Хоуелл і Савндер дають
у «Reports etc. for 31 st October 1848». Див. також петицію проти
«shift system», подану королеві духівництвом Ashton’a й околиць на весні
1849~\abbr{р.}
} Так ось
у чім була річ; це було фабрикантське видання десятигодинного
закону! Це були ті самі фритредери, сповнені благодаті й любови
до людства, що підчас аґітації проти хлібних законів цілих десять
років до останнього шага обчислювали робітникам, що за вільного
довозу хліба, при тих засобах, що їх має англійська промисловість,
цілком досить було б десяти годин праці, щоб збагатити
капіталістів\footnote{
Порівн., наприклад, «The Factory Question and the Ten Hours
Bill. By R. H. Greg. 1837».
}.

Дворічний бунт капіталу увінчався нарешті присудом однієї
з чотирьох вищих судових установ Англії, Court of Exchequer,
який в одному з випадків, що дійшов до нього, 8 лютого 1850~\abbr{р.}
вирішив, що хоч фабриканти й чинили проти змісту закону
1844~\abbr{р.}, але самий цей закон містить у собі деякі слова, що роблять
його безглуздим. «Цей вирок знищив закон про десятигодинну
працю»\footnote{
\emph{F. Engels}: «Die englische Zehnstundenbill» (у видаваній мною
«Neue Rheinische Zeitung». Politish-ökonomische Revue, Aprilheft
1850», p. 13). Той самий «високий» суд так само винайшов підчас американської
громадянської війни словесну зачіпку, яка перетворювала закон
проти озброєння піратських кораблів у його пряму протилежність.
}. Маса фабрикантів, що досі боялись застосовувати
систему змін для підлітків і робітниць, ухопилися за неї тепер
обома руками\footnote{
«Reports etc. for 30 th April 1850».
}.

Але за цією, здавалось, остаточною перемогою капіталу
настав зараз же поворот. Робітники досі ставили пасивний, хоч
і впертий і день-у-день відновлюваний опір. Тепер вони почали
голосно протестувати на загрозливих мітинґах у Ланкашірі і
Йоркшірі. Значить, так званий десятигодинний закон — це лише
ошуканство, парляментське шахрайство, а на ділі він ніколи не
існував! Фабричні інспектори пильно попереджали уряд, що
\parbreak{}  %% абзац продовжується на наступній сторінці

\input{_0230.tex}
\input{_0231.tex}

  
\index{i}{0478}  %% посилання на сторінку оригінального видання
\chapter{Процес акумуляції капіталу}

\noindent{}Перетворення певної грошової суми на засоби продукції та
робочу силу є перший рух, що його пророблює певна кількість
вартости, яка повинна функціонувати як капітал. Цей рух відбувається
на ринку, у сфері циркуляції. Друга фаза руху, процес
продукції, закінчується, скоро тільки засоби продукції
перетворено на товари, вартість яких перевищує вартість їхніх
складових частин, отже, містить у собі первісно авансований
капітал плюс додаткова вартість. Ці товари треба потім знову
кинути у сферу циркуляції. Треба їх продати, зреалізувати їхню
вартість у грошах, ці гроші знову перетворити на капітал і
так знову й знову повторювати цей процес. Цей кругобіг, що
постійно пророблює ті самі послідовні фази, становить циркуляцію
капіталу.

\disablefootnotebreak{}
Перша умова акумуляції капіталу та, щоб капіталістові вдалося
продати свої товари та знову перетворити на капітал найбільшу
частину одержаних таким чином грошей. У дальшому
викладі ми припускаємо, що капітал перебігає свій процес нормальним
способом. Ближча аналіза цього процесу належить до
другої книги\footnote*{
У другому німецькому виданні цей початок розділу викладена
так: «Ми бачили, як капітал продукує додаткову вартість у формі товару.
Лише через продаж товару реалізується вміщена в ньому додаткова
вартість разом з капітальною вартістю, авансованою на його продукцію.
Тому процес акумуляції капіталу має за передумову процес його циркуляції.
Розгляд цього останнього ми відкладаємо до наступної книги.
Реальні умови репродукції, тобто безперервної продукції з являються
почасти лише в сфері циркуляції, а почасти вони можуть бути досліджені;
лише після аналізи процесу циркуляції. Однак це ще не все». \emph{Ред.}
}.
\enablefootnotebreak{}

Капіталіст, що продукує додаткову вартість, тобто висисає
неоплачену працю безпосередньо з робітників та фіксує її в товарах,
є, правда, перший присвоювач, алеж він зовсім не є
останній власник цієї додаткової вартости. Він мусить потім
поділитися нею з тими капіталістами, що виконують інші функції
в цілій суспільній продукції, з земельним власником і~\abbr{т. ін.}
Тому додаткова вартість розпадається на різні частини. Її частини
припадають різним категоріям осіб та набирають одна проти
одної різних самостійних форм, як от зиск, процент, торговельний
зиск, земельна рента і~\abbr{т. ін.} Ці перетворені форми додаткової
вартости можна буде розглянути лише в третій книзі.

\input{i/_0479.tex}
\input{i/_0480.tex}
\parcont{}  %% абзац починається на попередній сторінці
\index{i}{0481}  %% посилання на сторінку оригінального видання
продукті\footnote{
«Заробітну плату, як і зиск, треба розглядати як частину готового
продукту» («Wages as well as profits are to be considered each oj
them as really a portion of the finished product»!. (\emph{G.~Ramsay}: «An Essay
on the Distribution of Wealth», Edinburgh 1836 p. 142). «Частина продукту,
що припадає робітникові у формі заробітної плати». (\emph{J.~Mill}:
«Elements of Political Economy». Переклад Parissot’a, Paris 1823 p. 34).
}. Це — частина продукту, постійно репродукованого
самим робітником, яка постійно припливає до нього назад у
формі заробітної плати. Правда, капіталіст виплачує йому цю
товарову вартість грішми. Але ці гроші є лише перетворена форма
продукту праці [або, точніше, певної частини продукту праці]\footnote*{
Заведене у прямі дужки беремо з другого німецького видання. \emph{Ред.}
}.
Тимчасом як робітник перетворює частину засобів продукції
на продукт, частина його попереднього продукту знов перетворюється
на гроші. Його праця минулого тижня або останнього
півріччя є те, чим оплачують його сьогоднішню працю або працю
найближчого півріччя. Ілюзія, яку утворює грошова форма,
вмить зникає, скоро тільки ми замість поодинокого капіталіста
й поодинокого робітника розглядатимемо клясу капіталістів і
клясу робітників. Кляса капіталістів постійно дає клясі робітників
у формі грошей чеки на частину продукту, випродукованого
клясою робітників і присвоєного клясою капіталістів. Ці
чеки робітник так само постійно повертає клясі капіталістів і
таким чином відбирає від неї ту частину свого власного продукту,
що припадає йому самому. Товарова форма продукту і грошова
форма товару замасковують цей процес.

Отже, змінний капітал\footnote*{
У французькому виданні Маркс тут робить таку примітку: «Змінний
капітал тут розглядається виключно як фонд для оплати найманих
робітників. Відомо, що в дійсності він стає змінним лише з того моменту,
коли куплена ним робоча сила функціонує вже в процесі продукції».
\emph{Ред.}
} є лише осібна історична форма виявлення
фонду засобів існування або робочого фонду, що його
робітник потребує для свого утримання й своєї репродукції і
що його він за всяких систем суспільної продукції завжди мусить
сам продукувати й репродукувати. Робочий фонд постійно
припливає до нього у формі засобів платежу за його працю лише
тому, що його власний продукт постійно віддалюється від нього
у формі капіталу. Але ця форма виявлення робочого фонду нічого
не змінює в тому, що капіталіст авансує робітникові його
власну упредметнену працю\footnote{
«Коли капітал вживають на авансування робітникам заробітної
плати, то він нічого не додає до фонду, призначеного для підтримання
праці» («When capital is employed in advancing to the workmen his wages
it adds nothing to the funds for the maintenance of labour»). (\emph{Cazenove}
у примітці до його видання праці Малтуза «Definitions in Political Economy»,
London 1853, p. 22).
}. Візьмімо селянина-кріпака. Він
працює своїми власними засобами продукції на своєму власному
полі, наприклад, три дні на тиждень. Три інші дні на тиждень
він одробляє панщину в панському маєтку. Він постійно репродукує
свій власний робочий фонд, і цей фонд ніколи не набирає
\index{i}{0482}  %% посилання на сторінку оригінального видання
супроти нього форми засобів платежу, авансованих йому
від третьої особи за його працю. Зате і його неоплачена примусова
праця ніколи не набирає форми добровільної та оплаченої
праці. Коли завтра поміщик присвоїть собі поле, робочу худобу,
насіння, коротко — засоби продукції селянина-кріпака,
то цей останній відтепер муситиме продавати свою робочу силу
сеньйорові. За інших незмінних обставин він, як і раніш, працюватиме
6 днів на тиждень: 3 дні на себе, 3 дні на колишнього
сеньйора, що тепер перетворився на пана-наймача робітників.
Як і раніш, він споживатиме засоби продукції як засоби продукції
і переноситиме їхню вартість на продукт. Як і раніш, певна
частина продукту ввіходитиме в репродукцію. Але так само як
панщанна праця набирає форми найманої праці, так само й робочий
фонд, що його, як і раніш, продукує й репродукує селянин-кріпак,
набирає форми капіталу, авансованого селянинові від
колишнього сеньйора. Буржуазний економіст, що його обмежений
мозок не може відокремити форму виявлення від того, що
в ній виявляється, заплющує очі перед тим фактом, що навіть
ще й тепер на земній кулі робочий фонд лише винятково виступає
у формі капіталу\footnote{
«Засоби існування робітників авансуються капіталістами робітникам
навіть менше, ніж на одній четвертині земної кулі». (\emph{Richard
Jones}: «Textbook of Lectures on the Political Economy of Nations», Hertford
1852, p. 16).
}.

Правда, змінний капітал втрачає характер вартости, авансованої
із власного фонду капіталіста\footnoteA{
«Хоч мануфактурний робітник дістає свою заробітну плату як
аванс від хазяїна, проте фактично це не коштує хазяїнові ніяких витрат,
бо сума цієї заробітної плати звичайно повертається назад разом із зиском
у підвищеній вартості речі, на яку вжито цю працю» («Though the
manufacturer has his wages advanced to him by his master, he in reality
costs him no expense, the value of these wages being generally restored,
together with a profit, in the improved value of the subject ліроп
which his labour is bestowed»). (\emph{A.~Smith}: «Wealth of Nations», b. II,
ch. 3, p. 355).
}, лише тоді, коли ми розглядатимемо
процес капіталістичної продукції в безперервній течії
його відновлення. Однак цей процес мусить десь і колись початися.
Тому, з того погляду, що його ми досі трималися, річ
імовірна, що капіталіст якогось часу за допомогою якоїсь первісної
акумуляції, незалежної від чужої неоплаченоі праці,
став посідачем грошей і тому міг виступити на ринку як
покупець робочої сили\footnote*{
У французькому виданні замість останніх двох речень читаємо
таке: «Однак, раніше ніж відновитися, цей процес мусив був початися
й тривати певний відтинок часу, протягом якого робітник не міг ще бути
оплачений з його власного продукту, ані жити з повітря. Отже, чи не слід
було б нам припустити, що капіталістична кляса, з’явившись уперше
на ринку праці, вже нагромадила своєю власною працею та власними
заощадженнями скарби, що дали їй змогу авансувати робітникам засоби
існування у формі грошей. Погодьмось поки що на таке рішення цієї
проблеми, яку ми докладніше розглянемо в розділі про так звану первісну
акумуляцію». («Le Capital etc.», v. I, ch. XXIII, p. 248--249). \emph{Ред.}
}. А втім проста безперервність капіталістичного
\index{i}{0483}  %% посилання на сторінку оригінального видання
процесу продукції, або проста репродукція, зумовлює
ще інші своєрідні зміни, що стосуються не тільки до змінної
частини капіталу, а й до всього капіталу в цілому.

Якщо додаткова вартість, створювана періодично, наприклад,
щороку, капіталом у \num{1.000}\pound{ фунтів стерлінґів}, становить 200\pound{ фунтів
стерлінґів}, і якщо цю додаткову вартість щороку споживається,
то ясно, що після п’ятирічного повторювання того самого
процесу сума спожитої додаткової вартости дорівнює 200 × 5, або
дорівнює первісно авансованій капітальній вартості в \num{1.000}\pound{ фунтів
стерлінґів}. Коли б річну додаткову вартість споживано лише
частинно, наприклад, лише наполовину, то той самий результат
ми мали б після десятирічного повторювання продукційного
процесу, бо 100 × 10 = \num{1.000}. Взагалі кажучи, авансована
капітальна вартість, поділена на щорічно споживану додаткову
вартість, дає число років або число періодів репродукції, після
скінчення яких первісно авансований капітал споживається
капіталістом і тому зникає. Уявлення капіталіста, що він споживає
продукт чужої неоплаченої праці, додаткову вартість,
та зберігає первісну капітальну вартість, абсолютно нічого не
може змінити в цьому факті. Коли мине якесь певне число років,
присвоєна ним капітальна вартість дорівнює сумі додаткової
вартости, присвоєної ним без еквіваленту протягом того самого
числа років, а спожита ним сума вартости дорівнює первісній
капітальній вартості. Правда, він зберігає у своїх руках капітал,
що його величина не змінилася — капітал, що з нього частина,
будівлі, машини й~\abbr{т. ін.}, існувала вже тоді, коли він пустив у рух
своє підприємство. Але тут ідеться про вартість капіталу, а не
про його матеріяльні складові частини. Коли хтось споживе
все своє майно, поробивши таку кількість боргів, що дорівнюють
вартості цього майна, то якраз усе це майно й репрезентує лише
загальну суму його боргів. І так само, коли капіталіст спожив
еквівалент свого авансованого капіталу, то вартість цього капіталу
репрезентує лише загальну суму присвоєної ним задурно
додаткової вартости. Жодного атома вартости його старого капіталу
вже далі не існує.

Отже, цілком незалежно від усякої акумуляції, проста безперервність
процесу продукції, або проста репродукція, після
коротшого або довшого періоду неминуче перетворює кожний
капітал у нагромаджений капітал, або в капіталізовану додаткову
вартість. Навіть якщо при своєму вступі в продукційний процес
капітал був власністю підприємця, особисто ним заробленою,
все одно, раніш, або пізніш, він стає присвоєною без еквіваленту
вартістю, або матеріялізацією, в грошовій чи іншій формі, неоплаченої
чужої праці.

Як ми бачили в четвертому розділі, для того, щоб перетворити
гроші на капітал, недосить наявности продукції вартости й товарової
циркуляції. Для цього мусили насамперед протистояти
один одному як покупець і продавець на одному боці посідач
вартости або грошей, на другому — посідач вартостетворчої
\parbreak{}  %% абзац продовжується на наступній сторінці

\input{i/_0484.tex}
\parcont{}  %% абзац починається на попередній сторінці
\index{i}{0485}  %% посилання на сторінку оригінального видання
на продукти вищої вартости, ніш вартість авансованого капіталу.
Це його продуктивне споживання. Одночасно воно є й
споживання його робочої сили капіталістом, що купив її. З другого
боку, робітник витрачає гроші, заплачені за його робочу
силу, на засоби існування: це його особисте споживання. Отже,
продуктивне й особисте споживання робітника цілком різні.
В першому він діє як рушійна сила капіталу і належить капіталістові;
у другому він належить собі самому й виконує життєві
функції поза процесом продукції. Результат першого є життя
капіталіста, результат другого — життя самого робітника.

Розглядаючи «робочий день» тощо, ми, між іншим, виявили,
що робітника часто примушують робити з свого особистого споживання
простий епізод процесу продукції. В цьому випадку
він додає до себе засоби існування, щоб підтримувати в русі
свою робочу силу, так само як до парової машини додають вугілля
й воду, до колеса — мастива й~\abbr{т. ін.} В цьому випадку його засоби
споживання є лише засоби споживання одного із засобів продукції,
його особисте споживання — безпосередньо продуктивне
споживання. Однак це з’являється як зловживання, що неістотне
для капіталістичного процесу продукції\footnote{
Россі не декламував би з приводу цього пункту з такою бундючністю.
коли б він справді збагнув таємницю «продуктивного споживання».
}.

Інакше справа виглядає, коли ми розглядаємо не поодинокого
капіталіста й не поодинокого робітника, а клясу капіталістів
і клясу робітників, не ізольований процес продукції товару, а
капіталістичний процес продукції в його течії і в його суспільному
обсягу. — Коли капіталіст перетворює частину свого капіталу
на робочу силу, то тим самим він збільшує вартість цілого
свого капіталу. Він одним махом забиває двох зайців. Він має
зиск не тільки з того, що одержує від робітника, а ще й з того,
що він робітникові дає. Капітал, відчужений в обмін на робочу
силу, перетворюється на засоби існування, споживання яких
служить для того, щоб репродукувати мускули, нерви, кістки,
мозок наявних робітників і продукувати нових робітників. Тому
особисте споживання робітничої кляси в межах абсолютної
доконечности є зворотне перетворення засобів існування, відчужених
капіталом за робочу силу, на робочу силу, яку капітал
знову може експлуатувати. Воно є продукція й репродукція
найдоконечнішого для капіталіста засобу продукції — самого
робітника. Отже, особисте споживання робітника лишається
моментом продукції і репродукції капіталу, однаково, чи відбувається
воно всередині майстерні, фабрики й~\abbr{т. ін.}, чи поза ними,
в межах процесу праці чи поза ним, цілком так само, як чищення
машин незалежне від того, чи відбувається воно підчас процесу
праці, чи підчас певних перерв цього процесу. Справа ані
трохи не змінюється від того, що робітник здійснює своє особисте
споживання задля себе самого, а не задля капіталіста. Адже і
споживання робочої худоби не перестає бути доконечним момен-
\parbreak{}  %% абзац продовжується на наступній сторінці

\parcont{}  %% абзац починається на попередній сторінці
\index{i}{0486}  %% посилання на сторінку оригінального видання
продукційного процесу від того, що худоба сама споживає
те, що їсть. Постійне зберігання і репродукція робітничої кляси
лишається постійною умовою репродукції капіталу. Виконання
цієї умови капіталіст може спокійно полишити інстинктові
робітників до самозбереження й розмножування. Капіталіст
дбає лише про те, щоб якомога обмежити їхнє особисте споживання
на найдоконечнішому, і, як небо від землі, він далекий від
тієї південно-американської грубости, з якою робітників примушують
їсти поживніший харч замість менш поживного\footnote{
«Робітники в копальнях Південної Америки, що їхня щоденна
праця (найтяжча, мабуть, у світі) є в тому, щоб витягати на своїх плечах
на поверхню землі вантаж руди в 180--200 фунтів з глибини 450 футів,
харчуються лише хлібом та бобами; вони воліли б харчуватися
самим хлібом, але їхні пани, відкривши, що на самому хлібі вони не
можуть працювати так дуже, поводяться з ними, як із кіньми, і примушують
їх їсти боби; а боби далеко багатші на кісткову золу, ніж хліб».
(\emph{Liebig}: «Die Chemie in ihrer Anwendung auf Agrikultur und Physiologie»,
7 Auflage, 1862, част. 1, стор. 194, примітка).
}.

Тому капіталіст і його ідеолог, політико-економ, розглядають
як продуктивне споживання лише ту частину особистого споживання
робітника, що потрібна для увіковічнення робітничої
кляси, отже, що дійсно мусить бути спожита для того, щоб капітал
міг споживати робочу силу; а те, що робітник споживає поверх
того для своєї насолоди, є непродуктивне споживання\footnote{
\emph{James Mill}: «Eléments d’Economie Politique», Paris 1823, стор. 238
і далі.
}.
Коли б акумуляція капіталу спричинила підвищення заробітної
плати, а тому й збільшення засобів споживання робітника без
збільшеного споживання робочої сили капіталом, то додатковий
капітал був би спожитий непродуктивно\footnote{
«Коли б ціна на працю піднеслася так високо, що, не зважаючи
на приріст капіталу, не можна було б уживати більше праці, то я сказав
би, що такий приріст капіталу споживається непродуктивно» (\emph{Ricardo}:
«Principles of Political Economy», 3rd ed., London 1821, p. 163).
}. Справді, особисте
споживання робітника є для нього самого непродуктивне, бо воно
репродукує лише індивіда, що має потреби; воно є продуктивне
для капіталіста і для держави, бо воно є продукування сили,
що продукує чуже багатство\footnote{
«Єдине продуктивне споживання у власному значенні слова є
споживання або руйнування багатства (він має на думці споживання
засобів продукції) капіталістом з метою репродукції\dots{} Робітник\dots{} є
продуктивний споживач для особи, що вживає його, і для держави, але,
точно кажучи, не для себе самого». (\emph{Malthus}: «Definitions in Political
Economy», London 1853, p. ЗО).
}.

Отже, з суспільного погляду робітнича кляса, навіть поза
безпосереднім процесом праці, є так само приналежність капіталу,
як і мертве знаряддя праці. Навіть її особисте споживання
є в певних межах лише момент процесу репродукції капіталу.
Але цей процес, постійно віддаляючи продукт праці робітничої
кляси від її полюса до протилежного полюса капіталу, дбає
про те, щоб ці самосвідомі знаряддя продукції не втекли. Особисте
споживання робітників дбає, з одного боку, про їхнє власне збереження
\index{i}{0487}  %% посилання на сторінку оригінального видання
й репродукцію, а з другого боку, знищуючи засоби
існування, воно дбає про те, щоб вони постійно знову й знов
з’являлися на ринку праці. Римський раб був прикований
кайданами, а найманий робітник прив’язаний незримими нитками
до свого власника. Видимість його незалежности підтримує
постійна зміна індивідуальних панів-наймачів і юридична фікція
контракту.

Колись капітал, де це йому здавалося потрібним, здійснював
своє право власности на вільного робітника за допомогою примусового
закону. Так, наприклад, до 1815~\abbr{р.} еміґрацію машинобудівельних
робітників в Англії було заборонено під загрозою
тяжкої кари.

Репродукція робітничої кляси включає також передачу і
нагромаджування вправности від одного покоління до другого\footnote{
«Єдина річ, про яку можна сказати, що її нагромаджують і заздалегідь
підготовляють, — це вправність робітника\dots{} Акумуляція і нагромадження
вправної праці, ця найважливіша операція, провадиться щодо
великої маси робітників без жодного капіталу». (\emph{Hodgskin}: «Labour
Defended etc.» p. 13).
}.
До якої міри капіталіст вважає існування такої вправної
робітничої кляси за одну з належних йому умов продукції, розглядає
її в дійсності як реальне існування свого змінного капіталу,
виявляється тоді, коли криза загрожує йому її втратою.
Як відомо, в наслідок американської громадянської війни й
бавовняного голоду, що її супроводив, було викинуто на брук
більшість бавовняних робітників у Ланкашірі й~\abbr{т. ін.} З надр
самої робітничої кляси, як і з інших верств суспільства, залунав
заклик до державної допомоги та добровільних національних
пожертов, щоб уможливити еміґрацію «зайвих» робітників до
англійських колоній або до Сполучених штатів. Тоді «Times»
(24 березня 1863~\abbr{р.}) опублікував листа Едмунда Потера, колишнього
президента менчестерської торговельної палати. В Палаті
громад його лист цілком справедливо названо «маніфестом
фабрикантів»\footnote{
«Цей лист можна розглядати як маніфест фабрикантів» («Thal
letter might be looked upon as the manifesto of the manufacturers»).
(\emph{Ferrand}: «Подання з приводу бавовняного голоду, засідання Палати
громад з 27 квітня 1863~\abbr{р.}»).
}. Ми подаємо тут із нього деякі характеристичні
місця, де без прикрас говориться про право власности капіталу
на робочу силу.

«Бавовняним робітникам можуть сказати, що їх забагато на
ринку праці\dots{} що їх, може, треба б зменшити на одну третину,
і тоді настане нормальний попит на останні дві третини\dots{} Громадська
думка наполягає на еміґрації. Хазяїн (тобто бавовняний
фабрикант) не може добровільно згодитися на зменшення подання
праці; він вважає, що це було б так само несправедливо, як
і неправильно\dots{} Якщо еміґрацію підтримують із громадських
фондів, то він має право вимагати, щоб його вислухали, а може
і протестувати». Цей Потер пояснює далі, яка корисна бавовняна
промисловість, як «вона, безперечно, відтягла людність з Ірляндії
\index{i}{0488}  %% посилання на сторінку оригінального видання
та з англійських рільничих округ», яка вона величезна розміром,
як вона 1860~\abbr{р.} дала \sfrac{5}{13} усієї англійської експортної
торговлі, як вона через декілька років знову зросте через поширення
ринку, особливо індійського, і через примусовий достатній
«довіз бавовни по 6\pens{ пенсів} за фунт». Він каже далі: «Час — один,
два, може, три роки — випродукує потрібну кількість\dots{} І я хотів би
тоді поставити питання, чи варта ця промисловість того, щоб її
підтримувати, чи варто тримати машини (а саме живі робочі
машини) в порядку і чи не найбільша дурість думати про те, щоб
відмовитися від них? Я думаю, що так. Я визнаю, що робітники
не є власність («І allow that the workers are not a property»),
не власність Ланкашіру й хазяїнів; але вони — сила обох; вони
є духовна й навчена сила, що її не можна замістити протягом
життя однієї ґенерації; навпаки, інші машини, коло яких вони
працюють («the mere machinery which they work»), можна здебільшого
з користю замінити й поліпшити за дванадцять місяців\footnote{
Пригадаймо собі, що той самий капітал співає іншої пісеньки за
звичайних обставин, коли йдеться про зниження заробітної плати. Тоді
«хазяїни», як один, заявляють (див. четвертий відділ, примітку 188):
«Хай фабричні робітники в своїх власних інтересах запам’ятають, що їхня
праця в дійсності є дуже низький сорт навченої праці; що жодної іншої
праці не можна легше вивчити та що, зважаючи на її якість, жодної
праці не оплачується ліпше; що жодної іншої праці не можна придбати
за такий короткий час та в такому великому розмірі, сяк-так привчивши
найменш досвідчених осіб. Машини хазяїна (що їх, як ми тепер
чуємо, можна з користю замінити й поліпшити за дванадцять місяців)
відіграють у дійсності далеко важливішу ролю в справі продукції, ніж
праця і вправність робітника (яких тепер не можна замінити за тридцять
років), яких можна навчитися за шість місяців і яких може навчитися
кожен сільський наймит».
}.
Заохочуйте до еміґрації робочої сили, або дозволяйте (!)
її, але ж, що тоді станеться з капіталістом? («Encourage
or allow the working power to emigrate, and what of the capitalist?»
Цей крик серця нагадує придворного маршала Кальба)\dots{}
Зберіть вершки робітників — і основний капітал буде в значній
мірі зневартнений, а оборотний капітал не зможе боротися при
недостатньому поданні праці нижчого сорту\dots{} Нам кажуть,
що робітники сами бажають еміґрації. Це дуже природна річ,
що вони це роблять\dots{} Зменшуйте, придушуйте бавовняне виробництво,
відбираючи в нього робочі сили (by taking away its working
power), зменшуючи, приміром, на третину або на 5 мільйонів
видатки в заробітній платі, але що тоді станеться з найближчою
над робітниками вищою клясою, — з дрібними крамарями?..
Що станеться з земельними рентами, з платою за наймання котеджів?..
З дрібним фармером, кращим домовласником і землевласником?
А тепер скажіть, чи може бути якийсь плян самозгубніший
для всіх кляс країни, ніж оцей плян ослабити націю
експортом її найкращих фабричних робітників і зневартненням
частини її найпродуктивнішого капіталу й багатства?» «Я раджу
позику в 5--6 мільйонів, розподілену на 2 або 3 роки, що нею
порядкуватимуть приставлені до адміністрації опіки над бідними
\parbreak{}  %% абзац продовжується на наступній сторінці

\parcont{}  %% абзац починається на попередній сторінці
\index{i}{0489}  %% посилання на сторінку оригінального видання
в бавовняних округах спеціяльні комісари, керуючись спеціяльними
законодавчими нормами і застосовуючи примусової праці,
щоб підтримати моральну цінність одержувачів милостині\dots{}
Чи може бути щось гірше для земельних власників або для хазяїнів
(«сап anything be worse for landowners or masters»), ніж
позбутися своїх найкращих робітників і здеморалізувати та
збентежити решту через широку спустошливу еміґрацію і спустошення
цілої провінції щодо вартости й капіталу?»

Потер, цей вибраний адвокат бавовняних фабрикантів, розрізняє
дві групи «машин», при чому і ті і другі належать капіталістові,
тільки одні стоять у його фабриці, а другі вночі й
неділями перебувають поза фабрикою в котеджах. Одні мертві,
другі живі. Мертві машини не тільки щодня погіршуються та
зневартнюються, але через невпинний технічний проґрес значна
частина з наявної маси їх постійно так дуже старіється, що їх
з вигодою і протягом небагатьох місяців можна замінити на нові.
Живі машини, навпаки, поліпшуються, що довше вони функціонують,
що більше вони від покоління до покоління нагромаджують
вправности. «Times», між іншим, так відповів цьому
фабричному маґнатові:

«Пан Е. Потер так пройнявся почуттям надзвичайної й абсолютної
ваги бавовняних хазяїнів, що для збереження цієї кляси
й увіковічнення її промислу хотів би замкнути півмільйона робітничої
кляси проти її волі у великий моральний робітний дім.
Чи варта ця промисловість того, щоб її підтримувати? — питає
Потер. Певна річ, всіма чесними засобами, — відповідаємо ми.
Чи варто тримати машини в порядку? — знову питає Потер.
Тут ми збентежені. Під машинами Потер розуміє людські машини,
бо він запевняє, що не має на думці розглядати їх як абсолютну
власність. Ми мусимо признатися, що ми не вважаємо «за варте»,
а то навіть і за можливе тримати ці людські машини в порядку,
тобто замикати їх і мастити, поки їх не потребуватимуть. Людська
машина має властивість іржавіти від бездіяльности, хоч
як дуже ви її маститимете та чиститимете. А до того людська
машина, як ми це бачимо, може сама з себе видавати пару та вибухати
або шаліти по наших великих містах у танку св. Вітта.
Можливо, як це каже Потер, і потрібен довший час на репродукцію
робітників, але, маючи машиністів і гроші, ми завжди
знайдемо заповзятих, загартованих промислових людей, щоб
зробити з них більше фабричних хазяїнів, аніж ми зможемо їх
використати\dots{} Пан Потер базікає про нове пожвавлення промисловости
через 1, 2, 3 роки й вимагає від нас, щоб ми не заохочували
або не дозволяли еміґрації робочої сили! На його думку,
це природна річ, що робітники хочуть еміґрувати, але він вважає,
що нація мусить цих півмільйона робітників разом із \num{700.000} тих,
що з ними зв’язані, замкнути проти їхнього бажання в бавовняних
округах і — неминучий наслідок — придушити силою їхнє
незадоволення та підтримувати їх самих милостинею, і все це,
сподіваючись, що якоїсь днини, може, їх знову треба буде бавовняним
\index{i}{0490}  %% посилання на сторінку оригінального видання
хазяїнам\dots{} Настав час, коли велика громадська думка
цих островів мусить щось зробити, щоб урятувати «цю робочу
силу» від тих, що хочуть поводитися з нею так, як вони поводяться
з вугіллям, залізом і бавовною» («to save this, «working
power» from those who would deal with it as they deal with iron
coal and cotton»).\footnote{
«Times» з 24 березня 1863~\abbr{р.}
}

Стаття «Times’a» була тільки jeu d’esprit.\footnote*{
— гра словами. \emph{Ред.}
} «Велика громадська
думка» була в дійсності така, як думка Потера, — що фабричні
робітники є рухома приналежність фабрик. Їхній еміґрації
стали на перешкоді.\footnote{
Парламент не вотував жодного фартинга на еміґрацію, а ухвалив
тільки закони, що давали муніципалітетам можливість тримати робітників
між життям і смертю або експлуатувати їх, не платячи їм нормальної
заробітної плати. Навпаки, коли три роки пізніше спалахнула пошесть
на худобу, парлямент грубо знехтував навіть парламентською етикетою й
негайно вотував мільйони на відшкодування мільйонерам з лендлордів,
фермери яких і без того не мали ніякої шкоди завдяки піднесенню ціни
на м’ясо. Звіряче виття землевласників на відкритті парламенту 1866~\abbr{р.}
показало, що не треба бути індусом, щоб падати навколішки перед коровою
Сабала, ані Юпітером, щоб перетворитися на бика.
} Їх замкнули в «моральний робітний
дім» бавовняних округ, і вони, як і раніш, становлять «силу
(the strength) бавовняних хазяїнів Ланкашіру».

Отже, капіталістичний процес продукції самим своїм перебігом
репродукує відокремлення робочої сили від умов праці.
Тим самим він репродукує й увіковічнює умови експлуатації
робітника. Він постійно примушує робітника продавати свою
робочу силу, щоб жити, і постійно дає капіталістові змогу купувати
її, щоб багатіти.\footnote{
«Робітник вимагав засобів існування, щоб жити, підприємець
вимагав праці, щоб мати бариш» («L’ouvrier demandait de la subsistance
pour vivre, le chef demandait du travail pour gagner»). («\emph{Sismondi}: «Nouveaux
Principes d’Economie Politique», vol. I, p. 91).
} Тепер уже не випадок протиставить на
товаровому ринку капіталіста й робітника як покупця і продавця.
Механізм самого процесу постійно відкидає одного назад на
товаровий ринок як продавця його робочої сили і постійно перетворює
його власний продукт на купівельний засіб другого.
Фактично робітник належить капіталові раніш, ніж він продав
себе капіталістові. Його економічну підлеглість\footnote{
Грубо сільська форма цієї нідлеглости існує в графстві Дергем.
Це одно з тих небагатьох графств, де обставини не забезпечують фармерові
безперечного права власности на рільничих поденників. Гірнича
індустрія дозволяє їм вибирати. Тому тут, усупереч загальному правилу,
фармер бере в оренду тільки ті землі, на яких є котеджі для робітників.
Плата за наймання котеджів становить частину заробітної плати. Ці котеджі
звуться «hind’s houses\footnote*{
— доми слуг. \emph{Ред.}
}». Їх винаймають робітникам з певними февдальними
зобов’язаннями, з умовою, що зветься «bondage» (кріпацька
залежність) і, наприклад, зобов’язує робітника на той час, коли він працює
деінде, посилати на працю свою дочку й~\abbr{т. ін.} Сам робітник називається
bondsman, кріпак. Ці відносини показують нам з цілком нового
боку і особисте споживання робітника як споживання для капіталу, або
продуктивне споживання: «Цікаво спостерігати, що навіть екскременти
цього bondsman’a його всевладний пан, лічачи все, зараховує до своїх побічних
доходів\dots{} Фармер не дозволяє будувати в навкольності ніяких кльозетів,
крім його власних, і не терпить щодо цього ніякого порушення своїх сюзеренних
прав». («Public Health, VII th Report 1864», p.188).
} упосереднює
\index{i}{0491}  %% посилання на сторінку оригінального видання
і одночасно замасковує періодичне поновлення його самопродажу,
переміна його індивідуальних хазяїнів-наймачів і коливання
ринкових цін праці.\footnote{
Пригадаймо собі, що при праці дітей і~\abbr{т. ін.} зникає навіть ця формальність
самопродажу.
}

Отже, капіталістичний процес продукції, розглядуваний в
його загальному зв’язку, або як процес репродукції, продукує
не тільки товар, не тільки додаткову вартість, — він продукує
й репродукує саме капіталістичне відношення, капіталіста на
одному боці, найманого робітника — на другому.\footnote{
«Капітал має за передумову найману працю, наймана праця має за передумову
капітал. Вони взаємно зумовлюють одне одного: вони взаємно породжують одне
одного. Хіба робітник на бавовняній фабриці
продукує лише бавовняні тканини? Ні, він продукує капітал. Він продукує
вартості, які знову служать для того, щоб командувати над його працею, щоб за
допомогою її створювати нові вартості». (\emph{К. Marx}:
«Lohnarbeit und Kapital» у «Neue Rheinische Zeitung», № 266, 7 April 1849).
Статті, опубліковані під цим заголовком в «Neue Rheinische Zeitung», є уривки
лекцій, що їх я на цю тему читав у німецькому
робітничому товаристві у Брюсселі; друкування їх перервала Лютнева
революція.\footnote*{
Статті ці з’явилися потім окремою брошурою і під тією ж назвою.
Є українське видання: Партвидав «Пролетар» 1932~\abbr{р.} \emph{Ред.}
}
}

\section{Перетворення додаткової вартости на капітал}

\subsection{Капіталістичний процес продукції в поширеному маштабі.
Перетворення законів власности товарової продукції на закони
капіталістичного присвоєння}

Раніше нам треба було дослідити, як додаткова вартість виникає
з капіталу, тепер треба дослідити, як із додаткової вартости
виникає капітал. Вживання додаткової вартости як капіталу
або зворотне перетворення додаткової вартости на капітал, називається
акумуляцією капіталу.\footnote{
«Акумуляція капіталу: вживання частини доходу як капіталу» («Accumulation of
Capital: the employment of a portion or revenue as capital»). (\emph{Malthus}:
«Definitions etc.», ed. Cazenove, p. 11).
«Перетворення доходу на капітал» («Conversion of revenue into capital»).
(\emph{Malthus}: «Principles of Political Economy», 2nd. ed. London 1836, p. 320).
}

Розгляньмо цей процес насамперед з погляду поодинокого капіталіста. Припустімо,
наприклад, що прядільний фабрикант авансував капітал у \num{10.000}\pound{ фунтів стерлінґів},
з них чотири п’ятих на бавовну, машини й~\abbr{т. д.}, і одну п’ятину на заробітну
плату. Нехай він щороку продукує \num{240.000} фунтів пряжі
вартістю в \num{12.000}\pound{ фунтів стерлінґів}. При нормі додаткової вартости в 100\%
додаткова вартість міститься в додатковому продукті
\parbreak{}  %% абзац продовжується на наступній сторінці

\parcont{}  %% абзац починається на попередній сторінці
\index{i}{0492}  %% посилання на сторінку оригінального видання
або в чистому продукті в \num{40.000} фунтів пряжі, тобто в шостині
гуртового продукту вартістю в \num{2.000}\pound{ фунтів стерлінґів}, яка
реалізується в його продажу. [Якщо ці \num{2.000}\pound{ фунтів стерлінґів}
знову авансуються як капітал, то первісний капітал зростає
з \num{10.000}\pound{ фунтів стерлінґів} до \num{12.000}\pound{ фунтів стерлінґів}, тобто
відбулася акумуляція. Насамперед не має значення, чи цей додатковий
капітал додається до старого, чи він самостійно зростає
своєю вартістю]\footnote*{
Заведене у прямі дужки ми беремо з другого німецького видання.
\emph{Ред.}
}. Сума вартости в \num{2.000}\pound{ фунтів стерлінґів} є сума
вартости в \num{2.000}\pound{ фунтів стерлінґів}. По цих грошах не чуєш і не
бачиш, що вони є додаткова вартість. Характер вартости як додаткової
вартости показує нам, яким чином вона дісталася до
рук свого власника, але нічого не змінює в природі вартости
або грошей.

Отже, щоб перетворити новоприбулу до нього суму в \num{2.000}\pound{ фунтів
стерлінґів} на капітал, прядільний фабрикант, за інших незмінних
обставин, авансує з цієї суми чотири п’ятини на купівлю
бавовни й~\abbr{т. д.} і одну п’ятину на купівлю нових робітників-прядунів,
які знайдуть на ринку ті засоби існування, що їхню вартість
він їм авансував. Тоді цей новий капітал \num{2.000}\pound{ фунтів
стерлінґів}, функціонуватиме у прядільництві та з свого боку
приноситиме додаткову вартість у 400\pound{ фунтів стерлінґів}.

Капітальна вартість була первісно авансована в грошовій
формі; навпаки, додаткова вартість з самого початку існує як
вартість якоїсь певної частини гуртового продукту. Якщо цей
продукт продається, перетворюється на гроші, то капітальна
вартість знову набирає своєї первісної форми, а додаткова вартість
змінює свій первісний спосіб буття. Однак від цього моменту
капітальна вартість і додаткова вартість, одна і друга, є грошові
суми, і їхнє зворотне перетворення на капітал відбувається цілком
тим самим способом. І одну і другу капіталіст витрачає на
купівлю товарів, що дають йому змогу знову розпочати виготовлювати
свій продукт, і до того ж цього разу в поширеному
маштабі. Але, щоб купити ці товари, він мусить знайти їх на ринку.

Його власна пряжа циркулює лише тому, що він виносить
свій річний продукт на ринок, як це роблять із своїми товарами
також і всі інші капіталісти. Але перш, ніж ці товари з’явилися
на ринку, вони вже містилися в фонді річної продукції, тобто в
загальній масі предметів усякого роду, на які перетворюється
протягом року ціла сума поодиноких капіталів або цілий суспільний
капітал, що з нього кожний поодинокий капіталіст має
у своїх руках лише певну частину. Процеси, що відбуваються на
ринку, лише переміщують поодинокі складові частини річної
продукції, пересилають їх із рук у руки, але не можуть ні збільшити
цілу суму річної продукції, ані змінити природу випродукованих
предметів. Отже, той ужиток, що його можна зробити
з цілого річного продукту, залежить од власного складу цього
продукту, алеж зовсім не від циркуляції.

\index{i}{0493}  %% посилання на сторінку оригінального видання
Насамперед річна продукція мусить постачити всі ті предмети
(споживні вартості), що з них можна замістити спожиті
протягом року речові складові частини капіталу. Коли відняти
ці предмети, то лишається чистий, або додатковий продукт, у
якому міститься додаткова вартість. А з чого складається цей
додатковий продукт? Може, з речей, призначених на задоволення
потреб і примх кляси капіталістів, отже, з речей, що входять
у їхній споживний фонд? Коли б це було так, то додаткову
вартість прогулялося б усю без остачі, і відбувалася б лише
проста репродукція.

Щоб акумулювати, треба частину додаткової вартости перетворювати
на капітал. Але, не творячи чуда, можна перетворювати
на капітал лише такі речі, які можна вживати в процесі
праці, тобто засоби продукції, і далі такі речі, з яких робітник
може себе утримувати, тобто засоби існування. Отже, частина
річної додаткової праці мусить бути вжита на виготовлення додаткових
засобів продукції і засобів існування понад ту їхню
кількість, яка була потрібна, щоб замістити авансований капітал.
Одно слово, додаткову вартість лише тому можна перетворювати
на капітал, що додатковий продукт, вартістю якого вона
є, вже містить у собі речові складові частини нового капіталу.\footnoteA{
Ми абстрагуємось тут від міжнародньої торговлі, за допомогою
якої нація може перетворювати предмети розкошів на засоби продукції
або на засоби існування, і навпаки. Щоб розглянути предмет досліду в
його чистоті, вільним від перешкодних побічних обставин, ми мусимо тут
увесь торговельний світ розглядати як одну націю і припустити, що капіталістична
продукція всюди вкоренилася й опанувала всі галузі промисловости.
}

Щоб заставити ці складові частини фактично функціонувати
як капітал, кляса капіталістів потребує якогось додатку праці.
Якщо експлуатація занятих уже робітників не зростає ті екстенсивно,
ані інтенсивно, то треба вжити додаткових робочих сил.
Про це також подбав уже механізм капіталістичної продукції,
репродукуючи робітничу клясу як клясу, залежну від заробітної
плати, клясу, що її звичайної заробітної плати вистачає на
те, щоб не тільки забезпечити своє утримання, але й розмноження.
Капіталові треба ці додаткові робочі сили різного віку, постачувані
йому щороку робітничою клясою, долучити тільки до
додаткових засобів продукції, що вже містяться в річній продукції,
— і перетворення додаткової вартости на капітал є готове.
Розглядувана конкретно акумуляція сходить на репродукцію
капіталу в чимраз ширшому маштабі. Кругобіг простої репродукції
змінюється й перетворюється, за висловом Сісмонді, на
спіралю.\footnoteA{
Сісмондівська аналіза акумуляції має ту велику хибу, що Сісмонді
надто вже задовольняється фразою: «Перетворення доходу на капітал»,
не досліджуючи матеріяльних умов цієї операції.
}

Вернімось тепер назад до нашого прикладу. Це стара історія:
Аврам породив Ісаака, Ісаак породив Якова й~\abbr{т. д.} Первісний
капітал у \num{10.000}\pound{ фунтів стерлінґів} дає додаткову вартість
\parbreak{}  %% абзац продовжується на наступній сторінці

\parcont{}  %% абзац починається на попередній сторінці 
\index{i}{0494}  %% посилання на сторінку оригінального видання 
у 2.000 фунтів стерлінґів, яка капіталізується. Новий капітал у
2.000 фунтів стерлінґів дає нову додаткову вартість у 400 фунтів
стерлінґів; ця остання знову капіталізується, отже, перетворюється
на другий додатковий капітал, що дає нову додаткову
вартість у 80 фунтів стерлінґів, і т. д.

Ми залишаємо тут осторонь ту частину додаткової вартости,
що її споживає капіталіст. Так само мало цікавить нас у даний
момент те, чи додаткові капітали додаються до первісного капіталу,
чи відокремлюються від нього, щоб самостійно зростати
своєю вартістю; чи використовує їх той самий капіталіст, що
їх нагромадив, чи він передає їх до інших рук. Ми мусимо лише
не забувати, що поруч новоутворених капіталів первісний капітал
і далі репродукує себе й продукує додаткову вартість, і що
те саме має силу для кожного акумульованого капіталу у відношенні
до створеного ним додаткового капіталу.

Первісний капітал утворився в наслідок авансування 10.000
фунтів стерлінґів. Звідки має їх їхній власник? Він їх добув
своєю власною працею і працею своїх предків! — відповідають
нам в один голос представники політичної економії,21с і це їхнє
припущення дійсно здається одним-єдиним, що узгоджується
з законами товарової продукції.

Цілком інакше стоїть справа з додатковим капіталом у 2.000
фунтів стерлінґів. Процес його постання ми знаємо цілком докладно.
Він є капіталізована додаткова вартість. Від самого початку
він не містить у собі жодного атома вартости, що не походив
би з неоплаченої чужої праці. Засоби продукції, до яких долучається
додаткова робоча сила, і так само засоби існування,
з яких вона себе утримує, є не що інше, як інтегральні складові
частини додаткового продукту, отієї данини, яку кляса капіталістів
щорічно вириває в робітничої кляси. Коли кляса капіталістів
за якусь частину цієї данини купує в робітничої кляси
додаткову робочу силу навіть за повну ціну, так що еквівалент
обмінюється на еквівалент, то все таки це давно відома операція
завойовника, що купує у переможених товари за їхні власні,
пограбовані в них гроші.

Якщо додатковий капітал уживає до праці свого власного
продуцента, то цей останній мусить, поперше, і далі збільшувати
вартість первісного капіталу і, крім того, відкуповувати продукт
своєї попередньої праці за більшу працю, ніж той продукт коштував.
Коли розглядати це як оборудку між клясою капіталістів
і робітничою клясою, то справа ані трохи не зміниться,
коли за допомогою неоплаченої праці занятих досі робітників
уживатимуть до праці додаткових робітників. Капіталіст, може,
перетворює додатковий капітал на машину, яка викидає продуцента
цього додаткового капіталу на брук, заміняючи його
кількома дітьми. У всіх випадках робітнича кляса своєю додатне
«Первісна праця, якій його капітал завдячує своє походження»
(«Le travail primitif auquel son capital a dû sa naissance»). (Sismondiz
«Nouveaux Principes d’Economie Politique», éd. Paris, vol. I, p. 109).
\index{i}{0495}  %% посилання на сторінку оригінального видання 
новою працею протягом даного року створила капітал, що наступного
року вживатиме до праці додаткових робітників. Оце
й є те, що зветься: «утворювати капітал капіталом».

За передумову акумуляції першого додаткового капіталу в
2.000 фунтів стерлінґів була сума вартости в 10.000 фунтів стерлінґів,
авансована капіталістом і належна йому силою його
«первісної праці». Навпаки, передумова другого додаткового
капіталу в 400 фунтів стерлійґів є не що інше, як попередня акумуляція
першого, акумуляція тих 2.000 фунтів стерлінґів, що
їхньою капіталізованою додатковою вартістю є цей додатковий
капітал у 400 фунтів стерлінґів. Власність на минулу неоплачену
працю з’являється тепер одним-однією умовою теперішнього
присвоєння живої неоплаченої праці в щораз більшому й
більшому розмірі. Що більше капіталіст акумулював, то більше
може він акумулювати.

Оскільки додаткова вартість, що з неї складається додатковий
капітал № І, була результатом купівлі робочої сили за частину
первісного капіталу, купівлі, що відповідала законам товарового
обміну і з юридичного погляду припускає лише вільне порядкування
на боці робітника його власними здібностями, а на боці
власника грошей або товарів — належними йому вартостями;
оскільки додатковий капітал № II і т. ін. є лише результат додаткового
капіталу № І, отже, наслідок цього першого відношення;
оскільки кожна поодинока оборудка завжди відповідає
законові товарового обміну, отже, капіталіст завжди купує
робочу силу, а робітник завжди продає її, припустімо, навіть
за її дійсною вартістю, остільки ясно, що закон присвоєння, або
закон приватної власности, що ґрунтується на товаровій продукції
й товаровій циркуляції, перетворюється через свою власну,
внутрішню, неминучу діялектику на свою пряму протилежність.
Обмін еквівалентів, що виступав як первісна операція, зазнав
таких змін, що тепер лише на позір відбувається обмін, бо, по-перше,
частина капіталу, обміняна на робочу силу, сама є лише
частина продукту чужої праці, присвоєного без еквіваленту, і,
по-друге, її продуцент, робітник, мусить не тільки замістити її,
а замістити її ще й з новим додатком. Відношення обміну між
капіталістом і робітником стає таким чином тільки позірністю,
властивою процесові циркуляції, лише формою, що є чужа самому
змістові й тільки містифікує його. Постійна купівля й продаж
робочої сили — це форма. Змісте той, що капіталіст частину
упредметненої вже чужої праці, яку він безупинно присвоює
собі, не даючи за неї жодного еквіваленту, постійно знову обмінює
на більшу кількість живої чужої праці. Спочатку право
власности здавалося нам основаним на власній праці. Ми мусили,
принаймні, визнати це припущення, бо лише рівноправні власники
товарів протистоять один одному, а засіб до присвоєння чужого

22 «Праця створює капітал, раніш ніж капітал починає вживати
праці» («Labour creates capital, before capital employs labour»). (E. G.
Wakefield: «England and America», London 1833, vol. II, p. 110).
\parbreak{}  %% абзац продовжується на наступній сторінці

\input{i/_0496.tex}
\input{i/_0497.tex}
\parcont{}  %% абзац починається на попередній сторінці 
\index{i}{0498}  %% посилання на сторінку оригінального видання 
за свій товар справедливу ціну, всю до останнього шага, і, по-друге,
ця справа взагалі ані трохи не обходить робітника В. Чого вимагає
В, і що має він право вимагати, так це те, щоб капіталіст
платив йому вартість його робочої сили. «І один і другий ще
виграли: робітник — тому, що йому авансовано продукти його
праці [слід би сказати: неоплаченої праці інших робітників]
раніше, ніж ту працю було виконано [слід би сказати: раніше,
ніж його праця створила продукт]; хазяїн — тому, що праця
цього робітника була варта більше, ніж його заробітна плата
[слід би сказати: випродукувала більше вартости, ніж вартість
його заробітної плати]».\footnote*{
«Tous deux gagnaient encore: l'ouvrier parce qu’on lui avançait les
fruits de son travail [слід би сказати: du travail gratuit d’autres ouvriers]
avant qu’il fût fait [слід би сказати: avant que le sien ait porté de fruit],
le maître, parce que le travail de cet ouvrier valait plus que son salaire
[слід би сказати: produisait plus de valeur que celle de son salaire]».
(Sismondi, 1. c., p. 135).
}

Правда, справа виглядає цілком інакше, коли ми розглядаємо
капіталістичну продукцію в безперервній течії її поновлення
й замість поодинокого капіталіста й поодинокого робітника
беремо на увагу сукупність, клясу капіталістів і проти
неї клясу робітників. Але тим самим ми приклали б маштаб,
цілком невластивий товаровій продукції.

У товаровій продукції протистоять один одному лише незалежні
один від одного продавець і покупець. їхні взаємовідносини
кінчаються тоді, коли минає строк складеної між ними угоди.
Якщо оборудка відновлюється, то вже в наслідок нової угоди,
що не має нічого спільного з попередньою й лише випадково
зводить знову того самого покупця з тим самим продавцем.

Отже, якщо судити про товарову продукцію або якийсь до
неї належний процес за її власними економічними законами,
то треба кожен акт обміну розглядати сам по собі, поза всяким
зв’язком його з попереднім і наступним актами обміну. А що акти
купівлі й продажу відбуваються лише поміж поодинокими індивідами,
то неприпустимо шукати в них відносин між цілими
суспільними клясами.

Хоч який довгий є ряд послідовних періодичних репродукцій
і попередніх акумуляцій, які проробив той капітал, що сьогодні
функціонує, він завжди зберігає свою первісну незайманість.
Поки в кожному акті обміну, — беручи кожний акт відокремлено, —
зберігаються закони обміну, спосіб присвоювання може зазнати
цілковитого перевороту, не порушуючи ніяк права власносте,
відповідного товаровій продукції. Те ж саме право власносте
має силу так напочатку, коли продукт належить продуцентові
і коли останній, обмінюючи еквівалент на еквівалент, може багатіти
лише через свою власну працю, як і в капіталістичний період,
коли суспільне багатство в чимраз більшій і більшій мірі
стає власністю тих, що мають змогу постійно знову й знову присвоювати
собі неоплачену працю інших.

\index{i}{0499}  %% посилання на сторінку оригінального видання
Цей результат стає неминучим, скоро тільки робітник сам
вільно продає свою робочу силу як товар. Але тільки відтепер товарова
продукція стає загальною й робиться типовою формою продукції;
тільки відтепер кожний продукт продукується з самого
початку на продаж і всяке продуковане багатство проходить через
циркуляцію. Тільки тоді, коли наймана праця стає базою товарової
продукції, ця остання накидає себе цілому суспільству;
але тільки тоді розгортає вона також всі заховані в ній потенції.
Сказати, що втручання найманої праці перекручує товарову
продукцію, значить сказати, що товарова продукція, щоб лишитися
неперекрученою, мусить не розвиватися. В тій самій мірі,
в якій товарова продукція розвивається за своїми власними
іманентними законами на капіталістичну продукцію, в тій самій
мірі закони власности товарової продукції перетворюються на
закони капіталістичного присвоювання\footnote{
Тим то можна дивуватися з хитромудрости Прудона, що хоче знищити
капіталістичну власність, протиставлячи їй вічні закони власности
товарової продукціїї!
}.

Ми бачили, що навіть при простій репродукції кожен авансований
капітал, хоч яким чином його первісно придбано, перетворюється
на акумульований капітал, або на капіталізовану
додаткову вартість. Але в потоці продукції ввесь первісно авансований
капітал взагалі стає безконечно малою величиною (magnitudo
evanescens у математичному розумінні) порівняно з безпосередньо
акумульованим капіталом, тобто із зворотно перетвореною
на капітал додатковою вартістю або додатковим продуктом,
однаково, чи функціонує він у руках того, хто його нагромадив,
чи в чужих руках. Тому політична економія визначає капітал
взагалі як «акумульоване багатство» (перетворену додаткову
вартість або дохід), «що його знову вживають до продукції додаткової
вартости»\footnote{
«Капітал, тобто акумульоване багатство, що його вживають, щоб
одержати зиск» («Capital, viz: accumulated wealth employed with a
view to profit»). (\emph{Malthus}: «Principles of Political Economy», p. 262).
«Капітал\dots{} складається з багатства, що заощаджується з доходу, і вживається,
щоб одержати зиск» («Capital\dots{} consists of wealth saved from
revenue, and used with a view to profit»). (\emph{R.~Jones}: «An Introductory
Lecture on Political Economy», London 1833, p. 262).
}, а капіталіста як «власника додаткового
продукту»\footnote{
«Посідачі додаткового продукту або капіталу» («The possessors
of surplusproduce or capital»). («The Source and Remedy of the National
Difficulties. A Letter to Lord John Russel», London 1821).
}. Той самий погляд, хоч і в іншій формі, маємо і
в тому вислові, що ввесь наявний капітал є акумульований або
капіталізований процент, бо процент є лише частина додаткової
вартости\footnote{
«Капітал із складними процентами на кожну частину заощадженого
капіталу до такої міри виріс, що все багатство світу, яке дає дохід,
віддавна є вже процент від капіталу» («Capital, with compound interest
on every portion of capital saved, is so all engrossing, that all the wealth
in the world from which income is derived, has long ago become the
interest on capital»). («London Economist», 19 July 1859).
}

\index{i}{0500}  %% посилання на сторінку оригінального видання
\subsection{Хибне розуміння в політичній економії репродукції
в поширеному маштабі
}

Перше ніж перейти до деяких докладніших визначень акумуляції
або зворотпого перетворення додаткової вартости на капітал,
треба усунути двозначність, вигадану клясичною економією.

Так само як товари, що їх капіталіст купує для свого власного
споживання за якусь частину додаткової вартости, не служать
йому за засоби продукції і зростання вартости, так само і праця,
яку він купує, щоб задовольняти свої природні й соціяльні
потреби, не є продуктивна праця. Замість, купуючи ті товари
і працю, перетворювати додаткову вартість на капітал, капіталіст,
навпаки, споживає або витрачає її як дохід. Всупереч
старошляхетському принципові, що, як слушно каже Геґель,
«полягає у споживанні того, що є в наявності» і особливо яскраво
виявляється в розкошах особистих послуг, для буржуазної економії
мало вирішальну вагу оповістити акумуляцію капіталу за
перший обов’язок громадянина і невтомно проповідувати таке:
не можна акумулювати, якщо проїдати цілий свій дохід замість
чималу частину його витрачати на вербування додаткових продуктивних
робітників, що дають більше, ніж коштують. З другого
боку, політичній економії треба було боротися з народнім
забобоном, що сплутує капіталістичну продукцію з скарботворенням\footnote{
«Жоден політико-економ нашого часу не може під заощадженням
розуміти лише скарботворення; але, поза цією обмеженою й недостатньою
операцією, не можна собі уявити іншого значення даного
вислову щодо народнього багатства, як лише того, що випливає з різного
вжитку заощадженого і є основане на реальній ріжниці між різними
родами праці, утримуваними коштом заощаджень» («No political
economist of the present day can by saving mean mere hoarding: and
beyond this contracted and inefficient proceeding, no use of the term in reference
to the national wealth can well be imagined, but that which must
arise from a different application of what is saved, founded upon a real
distinction between the different kinds of labour maintained by it»).
(Malthus: «Principles of Political Economy», p. 38, 39).
} і тому гадає, ніби нагромаджене багатство є багатство, захищене
від руїни в його наявній натуральній формі, отже, вилучене
із споживання або врятоване і від циркуляції. Замикати гроші
та не пускати їх у циркуляцію — це було б методою, якраз протилежною
перетворенню їх на капітал, а нагромаджувати товари
в розумінні скарботворення — було б чистим безглуздям\footnoteA{
Так, у Бальзака, що так ґрунтовно вивчив усі відтінки скупости,
старий лихвар Ґобзек змальовується уже здитинілим, коли він починав
збирати собі скарб із нагромаджених товарів.
}.
Акумуляція товарів у великих масах є результат застою циркуляції
або перепродукції\footnote{
«Акумуляція капіталу\dots{} припинення обміну\dots{} перепродукція».
(«Accumulation of stocks\dots{} non-exchange\dots{} overprodustion»).
(\emph{Th. Corbet}: «An Inquiry into the Causes and Modes of the Wealth of
Individuals» London 1841, p. 14).
}. Проте в народній уяві постає, з одного
боку, картина дібр, нагромаджених у споживному фонді багатіїв
та повільно споживаних, а з другого боку, творення запасів —
\parbreak{}  %% абзац продовжується на наступній сторінці

\parcont{}  %% абзац починається на попередній сторінці
\index{i}{0501}  %% посилання на сторінку оригінального видання
явище, властиве всім способам продукції й що на ньому ми на
хвилину спішимося в аналізі процесу циркуляції\footnote*{
У французькому виданні це речення подано так: «Звичайний спосіб
вислову сплутує також капіталістичну акумуляцію, що є процес
продукції, з двома іншими економічними явищами, а саме: з нагромадженням
у споживному фонді багатіїв дібр, які споживаються лише
повільно, та з творенням запасів споживання — явищем, властивим усім
способам продукції». («Le Capital etc.», v. I, ch. XXIV, p. 257). \emph{Ред.}
}.

Отже, в цьому розумінні клясична політична економія має
рацію, коли вона підкреслює як характеристичний момент процесу
акумуляції те, що додатковий продукт мусить споживатись
продуктивними робітниками, а не непродуктивними. Але тут починається
й її помилка. А.~Сміт завів моду малювати акумуляцію як
просте споживання додаткового продукту продуктивними робітниками,
або малювати капіталізацію додаткової вартости як просте
перетворення її на робочу силу. Послухаймо, наприклад, Рікарда:
«Треба зрозуміти, що всі продукти країни споживаються; але величезна
ріжниця, яку тільки можна собі уявити, є в тому, чи споживаються
вони тими, що репродукують якусь іншу вартість, чи тими,
що її не репродукують. Коли ми кажемо, що дохід заощаджується
й додається до капіталу, то ми розуміємо під цим, що ту частину
доходу, про яку кажуть, що її додається до капіталу, споживається
продуктивними, а не непродуктивними робітниками. Немає
більшої помилки, як припускати, що капітал збільшується через
неспоживання»\footnote{
\emph{Ricardo}: «Principles of Political Economy», 3rd, ed. London
1821, p. 163, примітка.
}. Немає більшої помилки, як та, що її за
А.~Смітом проказують Рікардо і всі пізніші економісти, а саме,
що «ту частину доходу, про яку кажуть, що її додається до капіталу,
споживається продуктивними робітниками». За цим уяввленням
вся додаткова вартість, що перетворюється на капітал,
ставала б змінним капіталом. Навпаки, вона, як і первісно авансована
вартість, поділяється на сталий капітал і змінний капітал,
на засоби продукції й робочу силу. Робоча сила є та форма, що
в ній змінний капітал існує в процесі продукції. В цьому процесі
її саму споживає капіталіст. Вона ж своєю функцією, працею,
споживає засоби продукції. Одночасно гроші, заплачені при
купівлі робочої сили, перетворюються на засоби існування, що
їх споживає не «продуктивна праця», а «продуктивні робітники».
За допомогою аналізи, цілком хибної в своїй основі, А.~Сміт
доходить такого недоладного результату, що хоч кожний індивідуальний
капітал і поділяється на сталу і змінну складову
частину, все ж суспільний капітал сходить лише на змінний
капітал, або його витрачають лише на виплату заробітної плати.
Нехай, наприклад, фабрикант сукна перетворює \num{2.000}\pound{ фунтів стерлінґів}
на капітал. Одну частину цих грошей він витрачає на купівлю
ткачів; другу — на купівлю вовняної пряжі, машин і~\abbr{т. д.}
Але люди, що в них він купує пряжу й машини, знову оплачують
частиною з тих грошей працю і~\abbr{т. д.}, поки всі \num{2.000}\pound{ фунтів стерлінґів}
\index{i}{0502}  %% посилання на сторінку оригінального видання
будуть витрачені на заробітну плату, або поки
цілий продукт, що його репрезентують \num{2.000}\pound{ фунтів стерлінґів},
буде спожитий продуктивними робітниками. Ми бачимо: вся
сила цього арґументу лежить у словах «і~\abbr{т. д.}», що посилають
нас від Понтія до Пілата. Дійсно, А.~Сміт уриває свій дослід
саме там, де починаються його труднощі\footnote{
Не вважаючи на свою «Логіку», Дж.~Ст.~Мілл ніде навіть і не
помічає цієї хибної аналізи своїх попередників, яка навіть у межах буржуазного
горизонту, просто з погляду фахівця, потребує поправок. Він
скрізь реєструє з догматизмом школяра плутанину думок своїх учителів.
Так само й тут: «Сам капітал згодом цілком сходить на заробітну плату,
і навіть коли він через продаж продукту відновлюється, він потім знову
перетворюється на заробітну плату» («The capital itself in the long run
becomes entirely wages, and when replaced by the sale of produce becomes
wages again»).
}.

Поки ми беремо на увагу лише фонд річної продукції в цілому,
річний процес репродукції легко зрозуміти. Але всі складові
частини річної продукції треба винести на товаровий ринок, і
тут саме починаються труднощі. Рухи поодиноких капіталів і
особистих доходів перехрещуються між собою, переплутуються,
губляться в загальній зміні місць — у циркуляції суспільного
багатства — у тій зміні місць, що спантеличує спостерігача та
ставить дослідові дуже заплутані завдання. У третьому відділі
другої книги я подам аналізу дійсного зв’язку всіх тих явищ.
[Там виявиться, що догма А.~Сміта, успадкована всіма його послідовникам,
перешкоджала політичній економії зрозуміти навіть
елементарний механізм суспільного процесу репродукції]\footnote*{
Заведене у прямі дужки ми беремо з другого німецького видання.
\emph{Ред.}
}. Велика
заслуга фізіократів у тому, що вони в своєму «tableau économique»
вперше зробили спробу дати картину річної продукції
в тому вигляді, в якому вона виходить із циркуляції\footnote{
А.~Сміт у своєму викладі процесу репродукції, отже, і процесу
акумуляції, в деякому відношенні не тільки не зробив жодного поступу,
але зробив рішучий крок назад порівняно з своїми попередниками, особливо
порівняно з фізіократами. З його ілюзією, згаданою в тексті, пов’язана
справді казкова догма, також перейнята від нього політичною економією,
що ціна товарів складається із заробітної плати, зиску (процента)
і земельної ренти, отже, лише із заробітної плати й додаткової вартости.
Виходячи з цієї бази, Шторх принаймні наївно признається: «Неможливо
розкласти доконечну ціну на її найпростіші елементи» («Il est
impossible de résoudre le prix nécessaire dans ses éléments les plus simples»).
(\emph{Storch}: «Cours d’Economie Politique», ed. Petersbourg 1815,
vol. II, p. 140, примітка). Гарна економічна наука, що проголошує за
неможливе розкласти ціну товарів на її найпростіші елементні Докладніше
про це питання ми скажемо в третьому відділі другої і в сьомому
відділі третьої книги.
}.

А втім, само собою зрозуміло, що політична економія не проминула
використати в інтересах кляси капіталістів тезу А.~Сміта,
ніби всю перетворену на капітал частину чистого продукту споживає
робітнича кляса.

\input{i/_0503.tex}
\parcont{}  %% абзац починається на попередній сторінці
\index{i}{0504}  %% посилання на сторінку оригінального видання
(«не має жодної дати»). Лише остільки його власна минуща
доконечність криється в минущій доконечності капіталістичного
способу продукції. Але остільки ж рушійним мотивом його
діяльности є не споживна вартість і не споживання, а мінова
вартість та її збільшення. Як фанатик зростання вартости, він
нещадно примушує людство до продукції задля продукції, отже,
до розвитку суспільних продуктивних сил і до створення тих
матеріяльних умов продукції, які тільки й можуть становити
реальну базу вищої суспільної форми, що її основний принцип
є повний і вільний розвиток кожного індивіда. Лише як персоніфікація
капіталу капіталіст є респектабельний. У цій ролі
він так само, як і збирач скарбів, пройнятий жагою абсолютного
збагачування. Але те, що в збирача скарбів становить індивідуальну
манію, у капіталіста є діяння суспільного механізму,
в якому він є лише одне колесо. Крім того, розвиток капіталістичної
продукції робить доконечним невпинне збільшення капіталу,
вкладеного в промислове підприємство, а конкуренція накидає
кожному індивідуальному капіталістові іманентні закони
капіталістичного способу продукції як зовнішні примусові закони.
Конкуренція примушує його невпинно збільшувати свій
капітал, щоб зберегти його, а збільшувати його він може лише
за допомогою проґресивної акумуляції.

Тим то, оскільки вся діяльність капіталіста є лише функція
капіталу, обдарованого в його особі волею і свідомістю, його
власне приватне споживання в його очах має значення грабежу
в акумуляції його капіталу подібно до того, як в італійській
бухгальтерії приватні видатки фігурують на сторінці дебету
капіталіста проти його капіталу. Акумуляція — це завойовання
світу суспільного багатства. Разом з масою експлуатованого людського
матеріялу вона поширює безпосереднє й посереднє панування
капіталіста.\footnote{
На прикладі старомодної, хоч і постійно відновлюваної форми
капіталіста, — на прикладі лихваря, Лютер дуже добре унаочнює властолюбство
як елемент жадоби до збагачення. «Поганці могли збагнути
своїм розумом, що лихвар тричі злодій і душогуб. Ми ж, християни, так
шануємо їх, що мало не молимося на них задля їхніх грошей\dots{} Той, хто
висисає в другого його харч, хто грабує і краде, так само є душогубець
(оскільки це від нього залежить), як і той, що голодом мордує когось
та заганяє на той світ. Але лихвар робить усе це, і все ж він спокійно
сидить у своєму кріслі, хоч і мав би висіти на шибениці, де б його шматувало
стільки ворон, скільки він накрав золотих, якби тільки на ньому
було стільки м’яса, шоб усі ті ворони могли пошматувати те м’ясо та
поділити між собою. Малих злодіїв вішають на шибениці\dots{} Малих злодіїв
тримають по в’язницях, а великі ходять собі, пишаючися, в золоті
та шовках\dots{} Отже, немає й більшого ворога людини на землі (крім чорта),
як скнара та лихвар, бо він хоче бути богом над усіма людьми. Турки,
вояки, тирани теж лихі люди, однак вони мусять давати людям жити й
визнають, що вони лихі люди й вороги; вони можуть, і-навіть мусять
іноді змилуватися над деким. Але лихвар і скнара хотів би, шоб увесь
світ пропадав з голоду, спраги, суму й нужди; він хотів би все, що навколо
нього є, мати лише собі, щоб усяк діставав усе від нього, наче від бога,
і був навіки його кріпаком. Він носить пишні мантії, золоті ланцюжки
}
\index{i}{0505}  %% посилання на сторінку оригінального видання
Але первородний гріх діє повсюди. З розвитком капіталістичного
способу продукції, акумуляції й багатства капіталіст
перестає бути простим утіленням капіталу. Він починає відчувати
«людське почуття» до свого власного Адама; до того ж він
стає настільки освіченим, що починає глузувати з фанатичного
аскетизму, як із забобону старомодного збирача скарбів. Тимчасом
як клясичний капіталіст плямує індивідуальне споживання
як прогріх проти своєї функції й як «поздержливість» від
акумуляції, модернізований капіталіст у силі зрозуміти акумуляцію
як «відречення» від особистої насолоди. «Ах, дві душі
живуть у його грудях, одна хоче розлучитися з другою!» («Zwei
Seelen wohnen, ach! in seiner Brust, die eine will sich von der
andern trennen!»).

На історичних початках капіталістичного способу продукції —
а кожний капіталістичний вискочень індивідуально пророблює
цю історичну стадію — жага збагачення й скупість панують як
абсолютні пристрасті. Але проґрес капіталістичної продукції
створює не тільки світ насолод. Разом із спекуляцією і кредитовою
справою він відкриває тисячі джерел раптового збагачення. На
певному щаблі розвитку деякий умовний ступінь марнотратства,
що є разом з тим виставою на показ багатства, а тому й кредитоспроможності!,
стає навіть діловою доконечністю для «нещасного»
капіталіста. Розкоші входять у видатки капіталу на представництво.
До того ж капіталіст багатіє не пропорційно до своєї особистої
праці і свого особистого не-споживання, як, приміром,
збирач скарбів: він багатіє в міру того, як висисає чужу робочу
силу і примушує робітника зрікатися всіх життєвих насолод.
Тому, хоч марнотратство капіталіста ніколи не має щирого
характеру марнотратства февдального пана-гуляки, навпаки,
в основі його завжди криється якнайогидливіше скнарство й
найдріб’язковіша ощадність, проте його марнотратство зростає
із зростом його акумуляції, при чому одне одному не перешкоджає.
Разом із тим у благородних грудях капіталіста розвивається
фавстівський конфлікт між жагою акумуляції і жагою насолод.

й персні, пестить свою пику, видає себе за добру побожну людину
й пишається цим\dots{} Лихвар же — страшелезна потвора, як той вовкулак,
що все плюндрує, гірший, ніж Какус, Геріон або Антус. Але він
прибирається й удає із себе побожного, щоб ніхто не бачив, де подіваються
ті воли, що їх він утягує задом у свій барліг. Але Геркулес повинен
чути, як ревуть воли й кричать полонені, повинен шукати Какуса навіть
у скелях і ярах, повинен визволити волів від лиходія. Бо Какус є лиходій,
і той лиходій — побожний лихвар, що все краде, грабує та пожирає.
І однак удає, ніби він нічого лихого не заподіяв, і ніхто не може викрити
його лиходійства, бо волів оін утягнув у свій барліг задом, і вони лишають
такі сліди, ніби їх випустили з барлогу. Так і лихвар хоче обдурити
світ, наче він дає користь і дає світові волів, тимчасом як він
захоплює їх собі й пожирає\dots{} І коли грабіжників, розбійників і напасників
колесують і стинав ть їм голови, то в скільки разів більше слід
би колесувати всіх лихварів, вимотувати з них жили\dots{} проганяти їх,
проклинати їх та стинати їм голови». (Martin Luther: «An die Pfarrherrn,
wider den Wucher zu predigen», Wittenberg 1540).

\index{i}{0506}  %% посилання на сторінку оригінального видання
«Промисловість Менчестеру, — читаємо в одному творі, опублікованому
доктором Ейкіном 1795~\abbr{р.}, — можна поділити на
чотири періоди. В першому періоді фабриканти були змушені
вперто працювати, щоб підтримати своє життя». Особливо збагачувались
вони, обкрадаючи батьків, що віддавали їм своїх дітей
як учнів і мусили дорого платити за це, тимчасом як учнів цих
капіталісти мучили голодом. З другого боку, пересічні зиски
були низькі, і акумуляція потребувала великої ощадности.
Вони жили як збирачі скарбів, і далеко не споживали навіть
процентів од свого капіталу. «У другому періоді вони почали
набувати дрібні маєтки, але працювали так само вперто, як і
раніш», бо безпосередня експлуатація праці коштує праці, як
це знає всякий погонич рабів, «і жили в тому самому скромному
стилі, як і раніш\dots{} У третьому періоді почалися розкоші, підприємства
поширювалися через посилання в кожне торговельне
місто королівства верхівців (кінних комівояжерів) по замовлення.
Мабуть, що перед 1690~\abbr{р.} було небагато, а то й зовсім не було,
капіталів від \num{3.000} до \num{4.000}\pound{ фунтів стерлінґів}, набутих у промисловості.
Однак близько того часу або трохи пізніше промисловці
вже нагромадили грошей і почали будувати собі кам’яні
будинки замість будинків із дерева й глини\dots{} Ще в перші десятиліття
XVIII віку один менчестерський фабрикант, що почастував
своїх гостей пінтою чужоземного вина, викликав пересуди
своїх сусід, що докірливо кивали головою». Перед появою машин
фабриканти, сходячись увечорі в шинках, ніколи не споживали
більш, як склянку пуншу за 6\pens{ пенсів} і жмут тютюну за 1\pens{ пенс.}
Лише 1758~\abbr{р.} — і це становить епоху — побачили «особу, справді
заняту в промисловості, у своєму власному екіпажі»! Четвертий
період», остання третина XVIII віку, «відзначається великими
розкошами і марнотратством, що спиралися на поширення
справ».\footnote{
\emph{Dr. Aikin}: «Description of the Country from 30 to 40 miles round
Manchester», London 1795, p. 182 і далі.
} Що сказав би сердечний доктор Ейкін, коли б він
воскрес і поглянув би тепер на Менчестер!

Акумулюйте, акумулюйте! В цьому Мойсей і пророки! «Промисловість
постачає матеріял, що його акумулює ощадність».\footnote{
\emph{A. Smith}: «Wealth of Nations», b. II. ch. ІІІ, p. 367.
}
Отже, заощаджуйте, заощаджуйте, тобто перетворюйте якомога
більшу частину додаткової вартости або додаткового продукту
знову на капітал! Акумуляція задля акумуляції, продукція
задля продукції — в цій формулі клясична політична економія
висловила історичну місію буржуазного періоду. Вона навіть ні
на одну хвилину не чинила собі ілюзії щодо тих мук, у яких родиться
багатство,\footnote{
Навіть Ж. Б. Сей каже: «Заощадження багатих постають коштом
бідних» («Les épargnes des riches se font aux dépens des pauvres»). «Римський
пролетар жив майже цілком коштом суспільства\dots{} Можна б майже
сказати, що сучасне суспільство живе коштом пролетарів, коштом тієї
частини, яку воно відбирає в них із винагороди за працю». (\emph{Sismondi}:
«Etudes etc.», vol. I, p. 24).
} але яка користь із нарікання на історичну
\parbreak{}  %% абзац продовжується на наступній сторінці

\parcont{}  %% абзац починається на попередній сторінці
\index{i}{0507}  %% посилання на сторінку оригінального видання
доконечність? Коли для клясичної політичної економії пролетар
є лише машина продукувати додаткову вартість, то
й капіталіста вона розглядала лише як машину перетворювати
цю додаткову вартість на додатковий капітал. Вона трактує
його історичну функцію з надзвичайною серйозністю. Щоб ґарантувати
серце капіталіста від лихого конфлікту між жагою
насолод і жадобою до збагачення, Малтуз на початку двадцятих
років цього століття обстоював такий поділ праці, що призначав
справу акумуляції капіталістові, дійсно занятому в продукції,
а справу марнотратства — іншим учасникам додаткової вартости,
земельній аристократії, людям, що дістають утримання від держави,
церкви й~\abbr{т. ін.} Надзвичайно важливо, каже він, «відокремити
пристрасть до видатків від пристрасти до акумуляці і» («the
passion for expenditure and the passion for accumulation»)\footnote{
\emph{Malthus}: «Principles of Political Economy», p. 319, 320.
}.
Пани капіталісти, що давно вже поперетворювалися на розкішників
і світських людей, зчинили галас. Як, — вигукнув один
їхній проводир, рікардіянець, — пан Малтуз проповідує високі
земельні ренти, високі податки й~\abbr{т. ін.} для того, щоб через непродуктивних
споживачів постійно підганяти промисловців! Щоправда,
продукція, продукція в щораз ширшому маштабі, таке
наше гасло. Але «через такий процес продукція куди більше
гальмується, ніж розвивається. Крім того, не зовсім справедливо
(nor is it quite fair) підтримувати таким чином у ледарстві певне
число осіб для того лише, щоб підганяти інших, при чому з характеру
цих осіб можна бачити («who are likely, from their characters»),
що вони успішно функціонуватимуть, коли їх примусити
функціонувати»\footnote{
«An Inquiry into those principles respecting the Nature of Demand
etc., p. 67.
}. Але, якщо цей рікардіянець вважає за несправедливе
підганяти промислового капіталіста до акумуляції,
збираючи жир з його юшки, то він, навпаки, вважає за доконечне
звести заробітну плату робітника по змозі на мінімум,
«щоб підтримати його працьовитість». Він ні на хвилину не
затаює й того, що таємниця добування додаткової вартости
(Plusmacherei) — це присвоювання неоплаченої праці. «Збільшений
попит на роботу з боку робітників — це значить не що
інше, як їхній нахил брати менше з свого власного продукту для
самих себе, а більшу частину з нього лишати для своїх підприємців;
і коли кажуть, що це в наслідок зменшення споживання
(з боку робітників) викликає glut (переповнення ринку, перепродукцію),
то я на це можу відповісти лише, що glut — це синонім
високого зиску»\footnote{
Там же, стор. 50.
}.

Вчені суперечки про те, як найкорисніше для акумуляції
поділити витягнену з робітників здобич між промисловим капіталістом
і неробою\dash{}землевласником тощо, припинилися перед
липневою революцією. Незабаром після того міський пролетаріят
ударив на сполох у Ліоні, а сільський пролетаріят в Англії
\parbreak{}  %% абзац продовжується на наступній сторінці

\parcont{}  %% абзац починається на попередній сторінці
\index{i}{0508}  %% посилання на сторінку оригінального видання
пустив червоного півня. По цей бік каналу швидко зростав оуенізм,
по той бік — сен-сімонізм і фур'єризм. Тоді настав час
для вульґарної політичної економії. Саме за рік перед тим, як
Нассав В.~Сеніор із Менчестеру відкрив, що зиск (включаючи
і процент) із капіталу є продукт неоплаченої «останньої дванадцятої
години праці», він сповістив світові про своє друге відкриття.
«Я, — врочисто сказав він тоді, — заміняю слово капітал,
розглядуваний як знаряддя продукції, на слово поздержливість
(Abstinenz)»\footnote{
\emph{Senior}: «Principes fondamentaux de l’Economie Politique». Trad.
Arrivabene, Paris 1836, p. 308. Але для прихильників старої клясичної
школи це було вже трохи занадто безглуздо. «Пан Сеніор замінює вислів
«праця й капітал» на вислів «праця й поздержливість»\dots{} Поздержливість
— це просте заперечення. Не поздержливість, а споживання продуктивно
вживаного капіталу становить джерело зиску». (John Cazenove
у примітці до його видання праці Малтуза «Definitions in Political
Economy», London 1853, стор. 130, примітка). Навпаки, Джон. Ст.~Мілл на одній сторінці списує Рікардову теорію зиску, а на другій приймає
Сеніорову теорію «нагороди за поздержливість» («remuneration
of abstinence»). Банальні суперечності так само рідні для нього, як чужа
для нього геґелівська «суперечність», це джерело всякої діялектики.

Додаток до другого видання. Вульґарному економістові ніколи не
впадала в голову та проста думка, що всяку людську дію можна розглядати
як «поздержливість» від протилежної дії. Їсти — значить поздержуватися
від посту, ходити — поздержуватися від стоянки, працювати —
поздержуватися від ледарства, ледарювати — поздержуватися від праці
й~\abbr{т. д.} Ці пани добре зробили б, коли б подумали над словами Спінози:
«Determinatio est negatio» («Визначення — це заперечення»).
}. Це — незрівнянний зразок «відкрить» вульґарної
економії! Економічну категорію вона заміняє на сикофантську
фразу. Voila tout\footnote*{
Ось і все. \emph{Ред.}
}. «Коли дикун, — навчає Сеніор, — робить
лук, то він займається промисловістю, але не практикує
поздержливости». Це пояснює нам, як і чому за попередніх суспільних
становищ засоби праці фабрикувалося «без поздержливости»
капіталіста. «Що більше суспільство проґресує, то більше
вимагає воно поздержливости»\footnote{
Senior, там же, стор. 342.
}, саме від тих, хто займається
працею присвоювання собі чужої праці та її продукту. Всі умови
процесу праці перетворюються відтепер на відповідну кількість
актів поздержливости капіталіста. Що збіжжя не тільки їдять,
а й сіють, то це — через поздержливість капіталіста! Що вино
витримують певний час, то це теж через поздержливість капіталіста!\footnote{
«Ніхто\dots{} не сіятиме, наприклад, своєї пшениці, лишаючи її
12 місяців у землі, або не триматиме свого вина цілі роки в льоху замість
Одразу спожити ці речі або їхній еквівалент, коли він не сподіватиметься
одержати таким способом збільшену вартість і~\abbr{т. ін.}» (No one\dots{} will sow
his wheat, f. i., and allow it to remain a twelve-month in the ground,
or leave his wine in a cellar for years, instead of consuming these things
or their equivalent at once — unless he expects to acquire additional value
etc.». (\emph{Scrope}: «Political Economy». Ed.~A.~Potter, New Уогк 1841,
p. 133, 134).
} Капіталіст грабує свою власну плоть, коли «позичає(!)
робітникові знаряддя продукції», іншими словами, коли,
сполучивши їх з робочою силою, він вживає їх як капітал, замість
\index{i}{0509}  %% посилання на сторінку оригінального видання
з’їдати парові машини, бавовну, залізниці, добриво, робочих
коней тощо або, як це собі по-дитячому уявляє вульґарний
економіст, протринькати «їхню вартість» на розкоші
й інші засоби споживання\footnote{
«Нестатки, що їх бере на себе капіталіст, позичаючи (цього
евфемізму вжито на те, щоб за випробованою манерою вульґарних економістів
ідентифікувати найманого робітника, визискуваного промисловим
капіталістом, із самим промисловим капіталістом, що позичає гроші
в капіталістів-кредиторів!) свої знаряддя продукції робітникові, замість
присвятити їхню вартість своєму власному споживанню, перетворивши
їх на предмети споживання або втіх» («La privation que s’impose le
capitaliste, en prêtant ses instruments de production au travailleur au lieu
d’en consacrer la valeur à son propre usage, en la transformant en objets
d'utilité ou d’agrément»). (\emph{G.~de~Molinari}: «Etudes Economiques», Paris
1846, p. 36).
}. Як саме кляса капіталістів має
це зробити, — це таємниця, що її досі вперто зберігає вульґарна
економія. Досить. Світ живе лише з самокатування капіталіста,
цього сучасного покаянного поклонника Вішну. Не тільки
акумуляція, а й просте «збереження капіталу потребує постійного
напруження, щоб устояти проти спокуси з’їсти його»\footnote{
«La conservation d’un capital exige\dots{} un effort\dots{} constant pour
résister à la tentation de le consommer». (\emph{Courcelle Seneuil}: «Traité
théorique et pratique des entreprises industrielles», 2 ème éd. Paris 1857, p. 20).
}.
Отже, проста гуманність, очевидно, вимагає визволити капіталіста
від цього мучеництва і спокуси, визволити його таким самим
способом, яким недавно через скасування рабства визволено
ґеорґійського рабовласника від тяжкої дилеми — чи прогуляти
геть чисто на шампанське додатковий продукт, видушений із
негрів-рабів, чи знову перетворити його частково на додаткову
кількість негрів і землі.

\enablefootnotebreak{}
В найрізніших суспільно-економічних формаціях відбувається
не тільки проста репродукція, але ще й — правда, в різних
розмірах — репродукція в поширеному маштабі. Щораз більше
продукують і більше споживають, отже, і більше продукту перетворюють
на засоби продукції. Але цей процес не є акумуляція
капіталу, а тому й не є він функція капіталіста, доки засоби продукції
робітника, а тому і його продукт і його засоби існування
не протистоять ще йому в формі
капіталу\footnote{\label{footnote-47}«Осібні кляси доходу, що найбільше сприяють проґресові національного
капіталу, змінюються на різних стадіях розвитку, а тому вони
цілком різні в націй, що стоять на різних ступенях розвитку\dots{} Зиск\dots{}
на давніших стадіях суспільного розвитку\dots{} є незначне джерело акумуляції
порівняно із заробітною платою й рентою\dots{} Коли сили національної
праці до певної міри зростають, то відносне значення зиску як
джерела акумуляції зростає». («The particular classes of income which
yield the most abundantly to the progress of national capital, change at
different stages of their progress, and are therefore entirely different in
nations occupying different positions in that progress\dots{} Profits\dots{} unimportant
source of accumulation, compared with wages and rents, in the
earlier stages of society\dots{} When a considerable advance in the powers of
national industry has actually taken place, profits rise into comparative
importance as a source of accumulation»). (\emph{Richard Jones}: «Textbook
etc.», p. 16, 21).
}. Померлий перед
кількома роками Річард Джонс, наступник Малтуза на катедрі
\parbreak{}  %% абзац продовжується на наступній сторінці

\parcont{}  %% абзац починається на попередній сторінці
\index{i}{0510}  %% посилання на сторінку оригінального видання
політичної економії у східньоіндійському коледжі в Haileybury,
влучно пояснює це на двох великих фактах. Через те, іцо
найбільша частина індійського народу самостійні господарі-селяни,
то їхній продукт, їхні засоби праці й засоби існування
ніколи не існують «у формі («in the shape») фонду, який заощаджується
з чужого доходу («saved from Revenue»), а тому
й не перебігають вони попереднього процесу акумуляції» («а
previous process of accumulation».\footnote{
Там же, стор. 36. [До 4 вид — Це, певно, недогляд, цього місця
не можна було знайти. — Ф. Е.].
} З другого боку, в тих провінціях,
де англійське панування найменше зруйнувало стару
систему, нерільничих робітників вживають до праці безпосередньо
вельможі, до яких припливає частина сільського додаткового продукту
в формі данини або земельної ренти. Частину цього продукту
вельможі споживають у натуральній формі, другу частину
їхні робітники перетворюють для них на предмети розкошів
і всякі інші засоби споживання, а решта становить заробітну
плату робітників, які є власники своїх знарядь праці. Продукція
й репродукція в поширеному маштабі відбуваються тут без
жодного втручання того дивовижного святого, того лицаря сумної
постаті, — «нездержливого» капіталіста.

4. Обставини, що визначають розмір акумуляції незалежно від
тієї пропорції, в якій додаткова вартість поділяється на капітал
і дохід: Ступінь експлуатації робочої сили. — Продуктивна
еила праці. — Зростання ріжниці між застосовуваним і споживаним
капіталом. — Величина авансованого капіталу

Якщо припустити те відношення, що в ньому додаткова вартість
розпадається на капітал і дохід, за дане, то величина акумульованого
капіталу залежить, очевидно, від абсолютної величини
додаткової вартости. Коли припустити, що 80\% капіталізується,
а 20\% з’їдається, то акумульований капітал становитиме
2.400 фунтів стерлінґів або 1.200 фунтів стерлінґів залежно
від того, чи становить ціла сума додаткової вартости 3.000 фунтів
стерлінґів, чи 1.500 фунтів стерлінґів. Тому при визначенні
величини акумуляції діють всі ті обставини, що визначають масу
додаткової вартости. [Обставини, що визначають величину додаткової
вартости, ми докладно розвинули в розділах про продукцію
додаткової вартости].\footnote*{
Заведене у прямі дужки ми беремо з другого німецького видання.
\emph{Ред.}
} Ми ще раз розглядаємо їх тут
разом, але лише остільки, оскільки вони щодо акумуляції дають
нам нові погляди.

Як ми собі пригадуємо, норма додаткової вартости залежить
насамперед від ступеня експлуатації робочої сили. Політична
економія так високо цінує цю ролю, що вона принагідно ідентифікує
прискорення акумуляції через збільшення продуктивної
сили праці з прискоренням її через збільшення експлуатації
\parbreak{}  %% абзац продовжується на наступній сторінці

\parcont{}  %% абзац починається на попередній сторінці
\index{i}{0511}  %% посилання на сторінку оригінального видання
робітника.\footnote{
«Рікардо каже: «На різних стадіях розвитку суспільства акумуляція
капіталу або засобів уживати працю (тобто експлуатувати її)
є більш або менш швидка і в усіх випадках мусить залежати від продуктивних
сил праці. Продуктивні сили праці взагалі найбільші там, де
існує надмір родючої землі». Якщо в цьому реченні продуктивні сили
праці означають мізерність тієї частини кожного продукту, яка припадає
тим, що продукують його своєю ручною працею, то ця теза є тавтологія,
бо частина, яка залишилась, є той фонд, що з нього, коли цього захочеться
власникові його («if the owner pleases»), можна акумулювати капітал.
Але здебільшого цього не буває там, де країна є найродючіша».
(«Observations on certain verbal disputes etc.», p. 74, 75).
} У відділах про продукцію додаткової вартости ми
постійно припускали, що заробітна плата принаймні дорівнює
вартості робочої сили. Однак на практиці примусове зниження
заробітної плати нижче від цієї вартости відіграє надто важливу
ролю, і тому ми мусимо хоч на хвилину спинитися на ньому
В певних межах воно фактично перетворює доконечний фонд
споживання робітника на акумуляційний фонд капіталу.

«Заробітні плати, — каже Дж. Мілл, — не мають продуктивної
сили; вони — ціна продуктивної сили; заробітні плати не
беруть участи, поряд самої праці, в продукції товарів, так само
як і ціна самих машин. Коли б працю можна було мати, не купуючи
її, заробітні плати були б зайві».\footnote{
«J. St. Mill: Essays on some unsetled Questions of Political Economy»,
London 1844, p. 90.
} Але коли б робітники
могли жити з повітря, то їх і не можна було б купити ні за яку ціну.
Отже, зниження заробітної плати до нуля є межа в математичному
розумінні, що її ніколи не можна досягти, хоч до неї можна
завжди наближатися. Постійна тенденція капіталу — це знизити
заробітну плату до цього нігілістичного рівня. Часто цитований
мною письменник XVIII століття, автор «Essay он Trade and
Commerce», виказує лише якнайінтимнішу таємницю душі англійського
капіталу, заявляючи, що історичне життєве завдання
Англії — це знизити англійську заробітну плату до французького
й голляндського рівня.\footnote{
«An Essay on Trade and Commerce», London 1770, p. 44. У грудні
I860 p. і в січні 1867 р. «Times» умістив подібні серцевиливи англійських
посідачів копалень, де описувалось щасливий стан бельгійських
копальневих робітників, які не вимагали і не діставали нічого більше,
а тільки те, що було доконечно потрібне, щоб жити для своїх «хазяїнів».
Бельгійські робітники терплять багато, і все це для того, щоб фігурувати
в «Times» як зразкові робітники! Страйк бельгійських копальневих
робітників (коло Marchiennej на початку лютого 1867 р., придушений
порохом і оливом, був відповіддю на цю атестацію.
} Він, між іншим, наївно каже: «Коли
наша біднота (символічна назва робітників) хоче жити в розкошах...
то, певна річ, її праця мусить бути дорога... Погляньте
тільки на силу-силенну зайвих речей («heap of superfluities), —
аж волосся стає диба, — що їх споживають наші мануфактурні
робітники, як ось горілка, джин, чай, цукор, чужоземні
овочі, міцне пиво, ситець, табака, тютюн і т. ін.».\footnote{
Там же, стор. 44, 46.
} Він цитує
твір одного фабриканта з Нортгемптоншіру, що, звівши очі до
неба, лементує: «Праця на цілу третину дешевша у Франції,
\parbreak{}  %% абзац продовжується на наступній сторінці

\parcont{}  %% абзац починається на попередній сторінці
\index{i}{0512}  %% посилання на сторінку оригінального видання
ніш в Англії, бо французька біднота тяжко працює та дуже
ощадна щодо харчу й одягу; вони споживають головно хліб,
овочі, городину, корінці та сушену рибу; вони дуже рідко їдять
м’ясо, і коли пшениця дорога, то й дуже мало їдять хліба».\footnote{
Фабрикант із Нортгемптоншіру чинить тут ріа fraus,\footnote*{
— благочестивий обман. \emph{Ред.}
} який йому
можна вибачити, бо це є порив серця. Він порівнює нібито життя англійських
і французьких мануфактурних робітників, алеж у вищецитованих
словах, як він і сам пізніше признається в замішанні, змальовує він
життя французьких рільничих робітників!
}
«До того ще, — каже далі наш есеїст, — треба додати, що вони
п’ють лише воду або подібні неміцні напитки, так що вони дійсно
надзвичайно мало витрачають грошей\dots{} Подібного стану речей,
безперечно, дуже тяжко добитися, але його можна досягти, як
це виразно доводить наявність його так у Франції, як і в Голландії».\footnote{
Там же, стор. 70, 71. Примітка до третього видання. Нині, завдяки
конкуренції на світовому ринку, що склалася з того часу, ми значно
посунулися наперед. «Якщо Китай, — заявляє своїм виборцям член парляменту
Степлтон, — стане великою промисловою країною, то я не бачу,
як робітнича людність Европи могла б витримати боротьбу, не знижуючись
до рівня своїх конкурентів» («Times», 3 вересня 1873 р.). Отже,
не континентальні, а китайські заробітні плати є вже тепер бажана
мета англійського капіталу.
} Два десятиріччя пізніш один американський шахрай,
янкі, що дістав титул барона, Бенжамен Томсон (інакше граф
Румфорд), з великим успіхом розвивав перед богом і людьми ті
самі філантропічні ідеї. Його «Essays» — це куховарська книга
з рецептами всякого роду, як дорогі нормальні страви робітників
заміняти на суроґати. Ось особливо вдатний рецепт цього
дивовижного «філософа»: «П’ять фунтів ячменю, п’ять фунтів
кукурудзи, на 3 пенси оселедців, на 1 пенс соли, на 1 пенс оцту,
на 2 пенси перцю й городини — разом на суму в 20\sfrac{3}{4} пенсів
маємо юшку для 64 осіб; за пересічних цін на хліб ці витрати
можна навіть знизити до \sfrac{1}{4} пенса на людину (менше ніж 3 пфеніґи)».\footnote{
Benjamin Thomson: «Essays, political, economical and philosopical
etc.», 3 volumes, London 1796--1802, vol. I, p. 288. У своєму «The
State of the Poor, or an History of the Labouring Classes in England etc.»
cep Ф. M. Еден дуже рекомендує злиденну румфордову юшку начальникам
робітних домів і докірливо нагадує англійським робітникам,
що, мовляв, «у Шотляндії є багато родин, які замість пшениці, жита й
м’яса цілі місяці живуть вівсяними крупами та ячним борошном, перемішаним
лише з водою й сіллю, і все таки живуть дуже комфортабельно»
(«and that very comfortably too»). (Там же, книга II, розд. 2, стор. 503).
Подібні «вказівки» ми мали і в XIX віці. «Англійські рільничі робітники,
— читаємо, наприклад, — не хочуть їсти мішанини з гірших сортів
жита. В Шотляндії, де виховання краще, цей забобон, мабуть, невідомий».
(\emph{Charles Н. Parry, M. D.}: «The Question of the Necessity of the
existing Cornlaws considered», London 1816, p. 69). Однак той самий
Пері нарікає, що англійський робітник тепер (1815 р.) дуже підупав
порівняно з часами Едена (1797 р.).
}
З проґресом капіталістичної продукції фальсифікація
товарів зробила зайвими ідеали Томсона.\footnote{
Зі звітів останньої парляментської слідчої комісії у справі фальсифікації
засобів існування бачимо, що навіть фальсифікація ліків в
Англії є не виняток, а правило. Наприклад, аналіза 34 проб опію, купленого
в 34 різних лондонських аптеках, виявила, що 31 проба були фальсифіковані
домішкою макових головок, пшеничного борошна, ґуми,
глини, піску й т. ін. Багато з них не мали й атома морфіну.
}

\index{i}{0513}  %% посилання на сторінку оригінального видання
Наприкінці XVIII століття і протягом перших десятиліть
XIX століття англійські фармери й лендлорди примусом добилися
абсолютно мінімальної заробітної плати, виплачуючи рільничим
поденникам у формі заробітної плати менше ніш мінімум,
і додаючи їм решту у формі допомоги від парафій. Ось приклад
тих фарсів, що їх витворяли англійські dogberries при «легальному»
встановленні тарифу заробітної плати: «Коли сквайри
1795 р. встановлювали заробітну плату для Speenhamland’y,
вони саме тоді обідали, але, очевидно, гадали, що робітники
чогось такого не потребують\dots{} Вони вирішили, що тижнева
плата має бути 3 шилінґи на людину поки буханець хліба
у 8 фунтів 11 унцій коштує 1 шилінґ і має рівномірно зростати
доти, доки буханець коштуватиме 1 шилінґ 5 пенсів.
Коли ціна хліба піднесеться ще вище, то заробітна плата
має пропорційно меншати, поки ціна буханця дійде 2 шилінґів,
і тоді харчі робітника будуть на \sfrac{1}{5} менші, ніш раніш».\footnote{
\emph{G. В. Newnham} (barrister at law): «А Review of the Evidence
before the Committees of the two Houses of Parliament on the Cornlaws»,
London 1815, p. 28 n.
}
1814 року в слідчому комітеті палати лордів запитали якогось
А. Беннета, великого фармера, суддю, адміністратора дому для
бідних і таксатора заробітної плати: «Чи додержується якоїсь
пропорції між вартістю денної праці й допомогою робітникам
від парафій?» Відповідь: «Так. Тижневий дохід кожної родини
доповнюється поверх її номінального заробітку настільки, щоб
можна було купити буханець вагою в 1 ґальон (8 фунтів 11 унцій)
і мати ще 3 пенси на людину\dots{} Ми припускаємо, що буханця
вагою в 1 ґальон досить на утримання кожної особи в родині
протягом тижня; а 3 пенси — то на одяг; коли парафія захоче
сама постачати одяг, то ці три пенси вона відраховує. Ця практика
панує не тільки всюди на захід від Вілтшіру, але, на мою
думку, і в цілій країні».\footnote{
Там же, стор. 19, 20.
} «Таким чином, — вигукує один буржуазний
письменник того часу, — фармери протягом багатьох років
спричинювали деґрадацію поважної кляси своїх земляків, примушуючи
їх шукати собі притулку в робітних домах\dots{} Фармер
збільшив свій власний дохід тим, що перешкоджав акумуляції
фонду найпотрібнішого споживання на боці робітників».\footnote{
\emph{Ch. H. Parry}: «The Question of the Necessity of the existing Cornlaws
considered», London 1816, p. 77, 69. Панове лендлорди з свого боку
не тільки «винагородили» себе за антиякобінську війну, яку вони вели
від імени Англії, а ще й надзвичайно збагатіли. «За вісімнадцять років
їхні ренти збільшились удвоє, утроє, вчетверо, а у виняткових випадках
навіть ушестеро». (Там же, стор. 100, 101).
} Яку
ролю в утворенні додаткової вартости, а тому і в утворенні
фонду акумуляції капіталу відіграє за наших днів безпосереднє
грабування із доконечного фонду споживання робітника, показала
\parbreak{}  %% абзац продовжується на наступній сторінці

\parcont{}  %% абзац починається на попередній сторінці
\index{i}{0514}  %% посилання на сторінку оригінального видання
вже, наприклад, так звана домашня праця (див. розділ XIII,
8, с.). Дальші факти про це ми подаємо далі в цьому відділі.

\enlargethispage{\baselineskip}
\looseness=-1
Хоч у всіх галузях промисловости тієї частини сталого капіталу,
яка складається із засобів праці, мусить вистачати для
певного числа робітників, визначуваного величиною вкладеного
капіталу, проте ця частина зовсім не мусить завжди зростати в
тій самій пропорції, в якій зростає число занятих робітників.
Припустімо, що на якійсь фабриці 100 робітників за восьмигодинної
праці дають 800 робочих годин. Коли капіталіст схоче
збільшити цю суму наполовину, то він може найняти 50 нових
робітників; але тоді він мусить авансувати й новий капітал,
не тільки для заробітної плати, а й на засоби праці. Однак, він
також може примусити попередніх 100 робітників працювати
12 годин замість 8, і тоді вистачить наявних уже засобів праці,
які тоді лише швидше зужитковуватимуться. Таким чином новододавана
праця, створена через більше напруження робочої сили,
може збільшити додатковий продукт і додаткову вартість, субстанцію
акумуляції, без відповідного збільшення сталої частини
капіталу.

\looseness=-1
У добувальній промисловості, наприклад, у копальнях, сировинні
матеріяли не становлять складової частини авансованого
капіталу. Тут предмет праці не є продукт попередньої праці, а
є gratis\footnote*{
безплатно. \emph{Ред.}
} дарований природою. Наприклад, металева руда, мінерали,
кам’яне вугілля, каміння й~\abbr{т. ін.} Тут сталий капітал складається
майже виключно із засобів праці, що дуже добре можуть
служити і при збільшеній кількості праці (наприклад, при денних
і нічних змінах робітників). Однак, припускаючи всі інші
обставини за незмінні, маса й вартість продукту зростатимуть
прямо пропорційно до вжитої праці. Як і першої днини продукції,
тут ідуть пліч-о-пліч первісні творці продукту, а тому й творці
речових елементів капіталу: людина й природа. Завдяки елястичності
робочої сили сфера акумуляції поширюється без попереднього
збільшення сталого капіталу.

У рільництві не можна поширити площу оброблюваної землі,
не авансуючи додаткового насіння і добрива. Але скоро це вже
авансовано, то самий лише механічний обробіток поля дивовижно
впливає на збільшення маси продукту. Більша кількість
праці, давана попередньою кількістю робітників, збільшує таким
чином родючість, не потребуючи нових авансувань на засоби
праці. Це знову таки безпосереднє діяння людини на природу,
яке стає безпосереднім джерелом збільшеної акумуляції без
участи якогось нового капіталу.

Насамкінець, у промисловості у власному значенні слова
кожна додаткова витрата на працю має своєю передумовою відповідну
додаткову витрату на сировинні матеріяли, але не неодмінно
і на засоби праці. А що добувальна промисловість і рільництво
постачають фабричній промисловості її сировинні матеріяли
\parbreak{}  %% абзац продовжується на наступній сторінці

\input{i/_0515.tex}
\parcont{}  %% абзац починається на попередній сторінці
\index{i}{0516}  %% посилання на сторінку оригінального видання
техніки, — то продуктивніші і, розглядувані щодо розміру
їхньої дієздатности, дешевші машини, знаряддя, апарати і~\abbr{т. ін.}
стають на місце старих. Старий капітал репродукується в продуктивнішій
формі, не кажучи вже про невпинні зміни деталів
у наявних засобах праці. Друга частина сталого капіталу, сировинний
матеріял і допоміжний матеріял, репродукується невпинно
протягом року, а якщо він походить із рільництва, то здебільшого
щорічно. Отже, кожне заведення ліпшої методи й~\abbr{т. ін.}
діє тут майже одночасно і на додатковий капітал і на той капітал,
що вже функціонує. Кожний проґрес на полі хемії не тільки урізноманітнює
число корисних речовин і застосування вже відомих,
поширюючи тим самим разом із зростанням капіталу сферу його
прикладання. Одночасно він навчає повертати екскременти процесу
продукції і споживання назад у кругобіг процесу репродукції,
отже, він створює нові матеріяли для капіталу без попередньої
витрати капіталу. Подібно до того, як через просте підвищення
напруження робочої сили збільшується експлуатація
природних багатств, так само наука й техніка створює для капіталу,
що функціонує, незалежну від даної його величини силу
поширюватись. Вони одночасно впливають і на ту частину первісного
капіталу, що увійшла в стадію свого відновлення. У своїй
новій формі капітал захоплює для себе задурно той суспільний
проґрес, що відбувся за спиною його старої форми. Правда, цей
розвиток продуктивної сили супроводиться частинним зневартненням
капіталів, що функціонують. Оскільки це зневартнення
дає себе гостро відчувати через конкуренцію, головний тягар
його спадає на робітника: капіталіст намагається поповнити свої
втрати через підвищену експлуатацію робітника.

Праця переносить на продукт вартість спожитих нею засобів
продукції. З другого боку, вартість і маса засобів продукції,
що їх пускає в рух дана кількість праці, зростає пропорційно
до того, як праця стає продуктивнішою. Отже, хоч та сама кількість
праці й додає до своїх продуктів завжди лише ту саму суму
нової вартости, а все ж із зростом продуктивности праці зростає
та стара капітальна вартість, яку вона одночасно переносить
на продукти.

Наприклад, коли англійський і китайський прядун працюватимуть
однакове число годин і з однаковою інтенсивністю, то за
тиждень вони обидва вироблять рівні вартості. Не зважаючи
на цю рівність, існує величезна ріжниця між вартістю тижневого
продукту англійця, що працює за допомогою потужного
автомата, і китайця, що має лише самопряд. За той самий час,
за який китаєць випрядає один фунт бавовни, англієць випрядає
декілька сот фунтів. У кілька сот разів більша сума старих вартостей
збільшує вартість продукту англійця, в якому ті старі
вартості зберігаються в новій корисній формі, і таким чином
можуть знову функціонувати як капітал. «1782~\abbr{р.}, — повідомляє
Ф.~Енґельс, — увесь збір вовни за три попередні роки (в Англії)
лежав ще необроблений через брак робітників і мусив би ще лежати,
\index{i}{0517}  %% посилання на сторінку оригінального видання
коли б не прийшли на допомогу нововинайдені машини, що
й перепряли вовну»\footnote{
\emph{F.~Engels}: «Lage der arbeitenden Klasse in England», Leipzig
1845, S. 20. (\emph{Ф.~Енґельс}: «Становище робітничої кляси в Англії», Партвидав
«Пролетар», 1932~\abbr{р.} стор. 62, 63.).
}. Упредметнена у формі машин праця,
певна річ, не створила безпосередньо жодного робітника, але
вона дала змогу невеликому числу робітників через додаток
порівняно невеликої кількости живої праці не тільки продуктивно
спожити вовну й додати до неї нову вартість, а й зберегти її стару
вартість у формі пряжі й~\abbr{т. ін.} Тим самим вона разом з тим дала
засіб і імпульс до поширеної репродукції вовни. Це є природна
властивість живої праці — зберігати стару вартість, створюючи
нову вартість. Тому із зростом дієздатности, розміру й вартости
засобів продукції, отже, з акумуляцією, яка супроводить розвиток
продуктивної сили праці, праця зберігає й увіковічнює в
вавжди нових формах чимраз більшу й більшу капітальну вартість\footnote{
Клясична політична економія через недостатню аналізу процесу
праці й процесу зростання вартости ніколи не розуміла гаразд цього
важливого моменту, репродукції, як це можна, приміром, бачити у Рікарда.
Він каже, наприклад: хоч яка буде зміна продуктивної сили,
«мільйон людей продукує на фабриках завжди ту саму вартість». Це
правда, коли дано екстенсивність і ступінь інтенсивности їхньої праці.
Але це не перешкоджає — і Рікардо недобачає цього в деяких своїх висновках
— тому, що за різної продуктивної сили своєї праці мільйон людей
перетворює на продукт дуже різні маси засобів продукції, а тому й зберігає
у своєму продукті дуже різні маси вартости, отже, вартості виготовлених
ними продуктів є дуже різні. Між іншим, треба сказати, що на цьому
прикладі Рікардо даремно силкувався пояснити Ж.~Б.~Сеєві ріжницю
між споживною вартістю (яку він називає тут wealth, речовим багатством)
і міновою вартістю. Сей відповідає: «Щодо тих труднощів, які
зазначає Рікардо, кажучи, що мільйон людей, вживаючи вдосконаленіших
способів продукції, може спродукувати вдвоє, утроє більше багатств,
не продукуючи більше вартости, то ці труднощі зникнуть, коли ми, як
це й слід, розглядатимемо продукцію як обмін, в якому віддають продуктивні
послуги своєї праці, своєї землі і своїх капіталів, щоб одержати
за це продукти. За допомогою цих продуктивних послуг ми дістаємо всі
продукти, що є на світі\dots{} Отож\dots{} ми є тим багатші, наші продуктивні
послуги мають тим більшу вартість, чим більшу кількість корисних
речей ми дістаємо за ці послуги в обміні, називаному продукцією».
«Quant à la difficulté qu’élève Mr.~Ricardo en disant que, par des procédés
mieux entendus, un million de personnes peuvent produire deux fois,
trois fois autant de richesses, sans produire plus de valeurs, cette diffuculté
n’en est pas une lorsque l’on considère, ainsi qu’on le doit, la production
comme un échange dans lequel on donne les services productifs de son travail,
de sa terre, et de ses capitaux, pour obtenir des produits. C’est par
le moyen de ces sevrices productifs que nous acquérons tous les produits
qui sont au monde\dots{} Or\dots{} nous sommes d’autant plus riches, nos services
productifs ont d’autant plus de valeur, q’uils obtiennent dans l’échange
appelé production, une plus grande quantité de choses utiles»). (\emph{J.~B.~Say}:
«Lettres à M.~Maithus» Paris 1820, p. 168, 169). «Труднощі», які має
вияснити Сей, — вони існують для нього, а не для Рікарда, — є ось у
чому: чому не збільшується вартість споживних вартостей, коли зростає
їхня кількість у наслідок підвищення продуктивної сили праці? Відповідь:
труднощі розв’язується тим, що споживну вартість будемо ласкаві
називати міновою вартістю. Мінова вартість є річ, що так або інакше
(one way or another) зв’язана з обміном. Отже, назвімо продукцію «обміном»
праці й засобів продукції на продукт — і стане ясно як день, що
ми дістанемо тим більше мінової вартости, чим більше споживних вартостей
дає нам продукція. Іншими словами, що більше споживних вартостей,
наприклад, панчіх, дає один робочий день фабрикантові панчіх,
то багатший він на панчохи. Однак раптом Сеєві спадає на думку, що «зі
збільшенням кількости» панчіх їхня «ціна» (яка, природно, не має нічого
спільного з міновою вартістю) падає, «бо конкуренція примушує їх (продуцентів)
віддавати продукти за стільки, скільки вони їм коштують»
(«parce que la concurrence les (les producteurs) oblige à donner les produits
pour ce qu’ils leur coûtent»). Звідки ж береться зиск, коли капіталіст
продає товари за цінами, що їх вони йому коштують? Але облишмо
це. Сей заявляє, що в наслідок підвищеної продуктивности кожен дістав
тепер в обмін за той самий еквівалент дві пари панчіх замість однієї, як
це було раніш, і~\abbr{т. д.} Результат, до якого він доходить, є саме та теза
Рікарда, яку він хотів був збити. Після такої величезної напруги думки
він, тріюмфуючи, звертається до Малтуза з такими словами: «Така, мій
пане, є добре пов’язана доктрина, що без неї, я це заявляю, неможливо
пояснити якнайбільші труднощі в політичній економії й особливо питання,
яким чином можливо, щоб нація стала багатшою тоді, коли вартість її
продуктів меншає хоч багатство і складається з вартостей» («Telle est,
monsieur, la doctrine bien liée sans laquelle il est impossible, le je déclare,
d’expliquer les plus grandes difficultés de l’économie politique et notamment,
comment il se peut qu’une nation soit plus riche lorsque ses produits
diminuent de valeur, quoique la richesse soit de la valeur»). (Там же, стор.
170). Один англійський економіст зауважує з приводу подібних фокусів
у «Листах» Сея: «Ці афектовані манери патякати («those affected ways
of talking») становлять y цілому те, що пан Сей залюбки називає своєю
доктриною і що він радить Малтузові викладати в Гертфорді, як це вже
робиться «в багатьох місцях Европи». Він каже: «Коли в усіх цих тезах
найдете дещо парадоксальним, то погляньте на ті речі, що їх ці тези виражають,
і я смію сподіватися, що вони здаватимуться вам дуже простими
й дуже розумними» («Si vous trouvez une physionomie de paradoxe â
toutes ces propositions, voyez les choses qu’elles expriment, et j’ose croire
qu’elles vous paraîtront fort simples et fort raisonnables»). Безсумнівно,
але в результаті того самого процесу вони видаватимуться всім, чим хочете,
та тільки не ориґінальним або важливим». («An Inquiry mto those
Principles respecting the Nature of Demand etc.», p. 116. 110).
}. Ця природна сила праці здається силою самозбереження
\index{i}{0518}  %% посилання на сторінку оригінального видання
того капіталу, до якого долучено працю, — цілком так само,
як суспільні продуктивні сили праці здаються його властивостями,
і так само, як постійне присвоювання додаткової праці
капіталістом здається постійним самозростанням вартости капіталу.
Всі сили праці здаються силами капіталу, як усі форми
вартости товару — формами грошей.

Із зростанням капіталу зростає ріжниця між застосованим
і спожитим капіталом. Інакше кажучи, зростає вартість і речова
маса засобів праці, як от будівлі, машини, дренажні труби, робоча
худоба, апарати всякого роду, — засобів праці, що протягом
довшого або коротшого періоду функціонують в постійно повторюваних
процесах продукції, або служать, щоб досягти певних
корисних ефектів, у повному своєму обсягу, тимчасом як зужитковуються
вони лише поступінно і тому втрачають свою вартість
лише частинами, отже, і лише частинами переносять її на продукт.
У тій самій мірі, в якій ці засоби праці служать як продуктотворці,
не додаючи до продукту вартости, отже, застосовуються
цілком, а споживаються лише частинно, в цій самій мірі,
\parbreak{}  %% абзац продовжується на наступній сторінці

\parcont{}  %% абзац починається на попередній сторінці 
\index{i}{0519}  %% посилання на сторінку оригінального видання 
вони, як ми вже згадували раніш, роблять ту саму дарову службу,
що й сили природи, — вода, пара, повітря, електрика й т. ін.
Ця дарова служба минулої праці, охоплена й одушевлена живою
працею, акумулюється із зростом маштабу акумуляції.

А що минула праця завжди фігурує в одягу капіталу, тобто
пасив праці робітників А, В, С і т. ін. фігурує як актив неробітника
X, то буржуа й політико-економи не знаходять слів
для вихвалення заслуг минулої праці, яка, на думку шотляндського
генія Мак Куллоха, мусить навіть діставати якусь окрему
винагороду (процент, зиск і т. ін.).\footnote{
Мак Куллох вибрав патент на «wages of past laboure»\footnote*{
— винагороду за минулу працю. Ред.
} * далеко
раніш, ніж Сеніор вибрав патент на «wages of abstinence».\footnote*{
— винагороду за поздержливість. Ред.
}
} Отже, чимраз більше й
більше значення минулої праці, яка у формі засобів продукції
бере участь у живому процесі праці, приписують її відчуженій
від самого робітника формі, а саме формі праці, що являє собою
його минулу й неоплачену працю, — тобто приписують її
формі капіталу. Практичні аґенти капіталістичної продукції
та її ідеологічні базікала так само нездібні уявити собі засіб
продукції окремо від антагоністичної суспільної характеристичної
маски, властивої йому за наших часів, як рабовласник
нездібний уявити собі самого робітника окремо від його характеристичних
рис як раба.

[Нарешті, за інших незмінних обставин, величина випродукованої
додаткової вартости, а тому й акумуляція, визначається
величиною авансованого капіталу].***

За даного ступеня експлуатації робочої сили маса додаткової
вартости визначається числом одночасно визискуваних робітників,
а це останнє відповідає, хоч і в змінній пропорції, величині
капіталу. [Із загальною величиною капіталу зростає і його
змінна складова частина, хоч і не в такій самій пропорції. Що
більший той маштаб, у якому продукує індивідуальний капіталіст,
то більше число робітників, що їх він одночасно експлуатує,
або маса неоплаченої праці, яку він собі присвоює\footnoteA{
[У третій книзі ми побачимо, що на пересічну норму зиску різних
сфер продукції не впливає властивий кожній із них поділ капіталу на
сталий і змінний елемент, і так само побачимо, що це явище лишена
иозір суперечить викладеним нами законам про природу й продукцію
додаткової вартости].***
}].*** Отже,
що більше зростає капітал через послідовні акумуляції, то більше
зростає й сума вартости, яка розпадається на фонд споживання
та фонд акумуляції. Тим-то капіталіст може жити розкішніш
і разом з тим більше «поздержуватися». І кінець-кінцем усі
пружини продукції діють то енерґійніше, що більше розширюється
разом із масою авансованого капіталу маштаб продукції.

*** — Заведене у прямі дужки ми беремо з другого німецького видання.
Ред.


\index{i}{0520}  %% посилання на сторінку оригінального видання

\subsection{Так званий робочий фонд}

У перебігу нашого досліду виявилося, що капітал є не стала
величина, а елястична частина суспільного багатства, частина,
що постійно змінюється залежно від поділу додаткової вартости
на дохід і додатковий капітал. Ми бачили, далі, що навіть за
даної величини капіталу, який функціонує, захоплені капіталом
робоча сила, наука й земля (під нею економічно треба розуміти
всі предмети праці, які природа дає без допомоги людини) становлять
його елястичні потенції, які в певних межах дають йому
простір, незалежний від його власної величини. При цьому ми
залишали осторонь усі відносини процесу циркуляції, що спричинюють
дуже різні ступені діяльности тих самих мас капіталу.
Через те, що ми припускали як передумову межі капіталістичної
продукції, отже, суто стихійний лад суспільного процесу продукції,
ми залишали осторонь також всяку, безпосередньо й пляномірно
здійснювану, раціональнішу комбінацію наявних засобів
продукції й робочих сил. Клясична політична економія віддавна
любила розглядати суспільний капітал як сталу величину
з сталим ступенем діяльности. Але цей забобон на рівень догми
підніс й закріпив лише прафілістер Єремія Бентам, цей тверезо-педантичний
шорстко-язикатий оракул плазовитого буржуазного
розуму XIX віку\footnote{
Порівн. між іншим: \emph{J.~Bentham}: «Theorie des Peines, et des Récompenses»,
trad. Et.~Dumont. 3 éme éd. Paris 1826, vol. II, liv. 4, ch. 2.
}. Бентам серед філософів є те
саме, що Мартін Туппер серед поетів. Обох їх можна було
зфабрикувати лише в Англії\footnote{
Єремія Бентам — суто англійське явище. Ніколи й ні в одній
країні ніхто ще, — не виключаючи навіть нашого філософа Хрістіяна
Вольфа, — з таким самозадоволенням не проповідував таких доморослих
банальностей. Принцип корисности — це зовсім не винахід Бентама.
Він лише бездарно повторяв те, що талановито виклали Гельвецій та
інші французи XVIII віку. Коли ми, наприклад, хочемо знати, що корисно
для собаки, то ми мусимо пізнати природу собаки. Самої цієї природи
не можна конструювати з «принципу корисности». Якщо прикласти
цей принцип до людини, якщо ми хочемо на основі принципу корисности
розглядати всякий людський вчинок, рух, відносини тощо,
то спочатку треба вивчити людську природу взагалі, а потім людську
природу, як вона змодифікована в кожній історичній епосі. Бентам цими
дрібничками не турбується. З найнаївнішою сухістю він вважає сучасного
дрібного буржуа (Spiessbürger), спеціяльно англійського, за типічну
людину. Що корисне для цієї ориґінальної типічної людини й світу її,
те корисне само по собі. Цим маштабом міряє він минуле, сучасне й майбутнє.
Наприклад, християнська релігія «корисна», бо вона з релігійного
погляду забороняє ті самі злочини, що їх карний кодекс осуджує з
юридичного. Мистецька критика «шкідлива», бо вона заважає шановним
людям зазнавати насолойи з творів Мартіна Туппера й~\abbr{т. д.} Отаким мотлохом
наповнив гори книжок цей бравий чоловічок, що його девізою було
«nulla dies sine lіnеа»\footnote*{
жодного дня без рядка. \emph{Ред.}
}. Коли б у мене була відвага мого приятеля Г.~Гайне,
я назвав би пана Єремію генієм буржуазної дурости.
}. Коли прийняти його догму, то
стають цілком незрозумілими найзвичайнісінькі явища процесу
продукції, як от, наприклад, раптові його поширення і звуження,
\parbreak{}  %% абзац продовжується на наступній сторінці

\parcont{}  %% абзац починається на попередній сторінці
\index{i}{0521}  %% посилання на сторінку оригінального видання
і навіть акумуляція.\footnote{
«Політико-економи надто схильні розглядати\dots{} певну кількість
капіталу й певне число робітників як знаряддя продукції однорідної
сили й певної однорідної інтенсивности функціонування\dots{} Ті\dots{}
хто твердять\dots{} що товари е єдині фактори продукції\dots{} доводять, що
продукцію взагалі не можна поширити, бо для такого поширення треба б спочатку
збільшити засоби існування, сировинні матеріяли і знаряддя, а це фактично
сходить на те, що ніякого зросту продукції не може бути без попереднього
її зросту, або, інакше кажучи, що ніякий зріст продукції неможливий».
(\emph{S. Bailey}: «Money and its Vicissitudes», p. 58 і 70). Бейлі
критикує цю догму головним чином з погляду циркуляції.
} Догму цю використовували і сам Бентам,
і Малтуз, і Джеме Мілл, Мак Куллох і т. д. з апологетичною
метою, а саме з метою представити одну частину капіталу, змінний
капітал, або капітал, що обмінюється на робочу силу, як
сталу величину. Створено байку, що речове існування змінного
капіталу, тобто та маса засобів існування, яку він репрезентує
для робітників, або так званий робочий фонд, є нібито осібна
частина суспільного багатства, визначена природними межами,
що їх ніяк не можна переступити. Щоб пустити в рух ту частину
суспільного багатства, що має фунціонувати як сталий капітал,
або, висловлюючи це речево, як засіб продукції, потрібна
певна маса живої праці. Вона є дана технологічними умовами. Але
не дані ані число робітників, потрібне, щоб пустити в рух цю
масу праці, бо воно змінюється разом із зміною ступеня експлуатації
індивідуальної робочої сили, ані ціна цієї робочої сили;
дана лише мінімальна межа тієї ціни, до того ж дуже еластична.
Факти, що лежать в основі цієї догми, такі. З одного боку, робітник
не має голосу при поділі суспільного багатства на засоби
споживання неробітників і на засоби продукції. З другого
боку, робітник лише у сприятливих виняткових випадках може
поширити так званий «робочий фонд» коштом «доходів» багатих.\footnote{
Дж. Ст. Мілл каже у своєму творі «Principles of Polit. Economy»,
т. II, стор. 259, 260: «Продукт праці за наших часів поділяється у зворотній
пропорції до праці: найбільша частина припадає тим, що ніколи
не працюють, дальша щодо величини частина — тим, чия праця майже
лише номінальна, і таким чином, за низхідною скалею, нагорода щораз
більше спадає в міру того, як праця стає тяжча й неприємніша, так що
за найвтомнішої і найвиснажливішої фізичної праці людина не може з
певністю розраховувати навіть на задоволення життєвих потреб». Щоб
уникнути непорозуміння, зауважу, що коли таких людей, як Дж. Ст.
Мілл і т. д., і треба ганьбити за суперечності між їхніми старими економічними
догмами і їхніми сучасними тенденціями, то все ж було б цілком
несправедливо змішувати їх до купи із зграєю вульгарно-економічних
апологетів.
}
До якої недоладної тавтології приводить спроба перебрехати
капіталістичні межі робочого фонду на його суспільні
природні межі, показує, між іншим, професор Фавсет: «Обіговий
капітал\footnote{
\emph{Н. Fawcett}. Prof, of Political Economy at Cambridge: «The Economic
Position of the British Labourer», London 1865, p. 120.
} якоїсь країни, — каже він, — це є робочий фонд
тієї країни. Тим то, щоб обрахувати пересічну грошову плату,
яку дістає кожний робітник, нам треба просто лише поділити
\parbreak{}  %% абзац продовжується на наступній сторінці

\parcont{}  %% абзац починається на попередній сторінці
\index{i}{0522}  %% посилання на сторінку оригінального видання
цей капітал на число робітничої людности».\footnote{
Я нагадую тут читачеві, що категорії змінний капітал і сталий
капітал вперше почав уживати я. Політична економія, починаючи від
А. Сміса, сплутує визначення, що містяться в них, з тими ріжницями
форм основного і обігового капіталу, що постають із процесу циркуляції.
Докладніше про це в другій книзі, другий відділ.
} Отже, це значить,
що спочатку ми підсумовуємо дійсно заплачені індивідуальні
заробітні плати, а потім уже твердимо, що результат цього додавання
становить суму вартости «робочого фонду», дарованого
богом і природою. Нарешті, визначену таким способом суму
ділимо на число робітників, щоб знову найти, скільки пересічно
може припасти кожному робітникові індивідуально. Це надзвичайно
хитромудра процедура. Вона не заважає панові Фавсетові
за одним духом сказати: «Ціле багатство, що його щорічно
акумулюється в Англії, поділяється на дві частини. Одну
частину застосовують в Англії для підтримання нашої власної
промисловости. Другу його частину експортують до інших
країн\dots{} Та частина, що її застосовують у нашій промисловості,
становить лише незначну частину багатства, що його щорічно
акумулюється в цій країні».\footnote{
\emph{Н. Fawsett}, там же, стор. 123, 122.
} Отже, більшу частину додаткового
продукту, який щорічно наростає і який відбирають в англійського
робітника без еквіваленту, капіталізується не в Англії,
а в чужих краях. Алеж разом з експортованим таким чином
додатковим капіталом експортують і частину «робочого фонду»,
вигаданого богом і Бентамом».\footnote{
Можна було б сказати, що не тільки капітал, а й робітників щороку
вивозять з Англії у формі еміґрації. Однак, у тексті немає навіть і згадки
про майно переселенців, які здебільшого неробітники. Велика частина
їх сини фармерів. Англійський додатковий капітал, що його вивозять
щороку за кордон для того, щоб мати з нього проценти, становить далеко
більшу величину порівняно з щорічною акумуляцією, аніж щорічна
еміґрація порівняно з щорічним приростом людности.
}\footnote*{
Що у французькому виданні теорію «робочого фонду» і критику
її подано повніше, наводимо тут відповідні уривки з цього видання:

«Капіталісти, їхні співвласники, їхні слуги та їхній уряд щороку
прогулюють чималу частину додаткового продукту. Крім того, вони
затримують у своїх споживних запасах багато повільно споживаних
предметів, придатних для репродукції, і перетворюють на непродуктивні
багато робочих сил, уживаючи їх для своїх особистих послуг.
Отже, частина багатства, що капіталізується, ніколи не досягає тієї
величини, якої вона могла б досягти. Її відношення до сукупного суспільного
багатства змінюється при всякій зміні в поділі додаткової вартости
на особистий дохід і додатковий капітал, а пропорція, в якій відбувається
цей поділ, постійно варіює під впливом коньюнктур, що на них ми
тут не зупиняємось. Для нас досить сконстатувати, що капітал зовсім
не є наперед визначена і фіксована частина суспільного багатства, а,
навпаки, змінна і хитка його частина\dots{}

Догма, ніби суспільний капітал завжди є певна фіксована величина,
не лише суперечить найзвичайнішим явищам продукції, як от рухам її
поширення і звуження, але вона робить незрозумілою і саму акумуляцію\dots{}
Цю догму Бентам і його прихильники — Мак-Куллох, Мілл та інші —
застосовували переважно до тієї частини капіталу, що обмінюється на
робочу силу і що її вони називають «фондом заробітної плати», або «робочим
фондом». На їх погляд це є осібна частина суспільного багатства
вартість певної кількости засобів існування, що розмірам їх сама природа
завжди ставить фатальні межі, які робітнича кляса марно намагається
переступити. Отже, сума, належна до розподілу між найманими робітниками,
є наперед визначена, а звідси випливає, що коли частина, яка
дістається кожному робітникові, є занадто мала, то це тому, що число
робітників занадто велике, і що бідність робітників, кінець-кінцем,
є наслідок не суспільного ладу, а природних умов.

Насамперед, межі, які капіталістична система ставить споживанню
продуцента, є «природні» лише в умовах цієї системи, цілком так
само, як батіг функціонує як «природна» спонука лише в умовах рабства.
В дійсності природі капіталістичної продукції властиве обмеження
частини продуцента тим, що є доконечне для підтримання його робочої
сили, так само, як ій властиве і захоплення додаткового продукту
капіталістом. Природі цієї системи продукції властиве також і те, що
додатковий продукт, який дістається капіталістові, поділяється ним
самим на дохід і додатковий капітал, тимчасом як робітник може лише
у виняткових випадках збільшити свій фонд споживання коштом фонду
споживання неробітників. «Багатий, — каже Сісмонді, — диктує закони
бідним\dots{} бо він сам переводить поділ річної продукції і залишає все
те, що він зве доходом, для самого себе, а все те, що він зве капіталом,
він відступає бідним, щоб вони з цього зробили для нього дохід» (Читай:
щоб вони з цього зробили для нього додатковий дохід). (\emph{Sismondi}: «Nouveaux
Principes d’Economie Politique», v. I, p. 107--108)\dots{}

Отже, насамперед, економісти мусили б довести, що капіталістичний
спосіб суспільної продукції, не зважаючи на те, що він зовсім
недавно виник, все ж є незмінний і «природний» спосіб продукції. Але,
навіть припускаючи розміри капіталістичної системи за дані, неправда,
«що фонд заробітної плати» є наперед визначений величиною суспільного
багатства, або величиною суспільного капіталу.

Через те, що суспільний капітал є лише мінлива й хитка частина
суспільного багатства, фонд заробітної плати, являючи собою лише
певну частину цього капіталу, не може бути фіксованою і наперед визначеною
частиною суспільного багатства; з другого боку, відносна величина
фонду заробітної плати залежить від тієї пропорції, що в ній суспільний
капітал поділяється на капітал сталий і капітал змінний, а ця
пропорція, як ми вже бачили і як ми це докладно покажемо в дальших
розділах, не залишається незмінною протягом процесу акумуляції».
(«Le Capital etc.», ch. XXIV, § 5, p. 267--268). \emph{Ред.}
}

\index{i}{0523}  %% посилання на сторінку оригінального видання

\section{Загальний закон капіталістичної акумуляції}

\subsection{Зростання попиту на робочу силу разом з акумуляцією
при незмінному складі капіталу}

У цьому розділі ми розглядаємо той вплив, що його справляє
зростання капіталу на долю робітничої кляси. Найважливіший
фактор при цьому дослідженні — це склад капіталу й зміни,
що їх він зазнає в перебігу процесу акумуляції.

Склад капіталу треба розуміти в двоякому значенні. З боку
вартости він визначається тим відношенням, що в ньому капітал
поділяється на сталий капітал, або вартість засобів продукції,
і на змінний капітал, або вартість робочої сили, загальну суму
заробітних плат. З боку речовини, що функціонує в процесі
продукції, кожний капітал поділяється на засоби продукції й
живу робочу силу; з цього боку склад капіталу визначається відношенням
\index{i}{0524}  %% посилання на сторінку оригінального видання
між масою застосованих засобів продукції, з одного
боку, і масою праці, потрібного для застосування тих засобів —
з другого. Перший я називаю вартостевим складом капіталу, а
другий — технічним складом капіталу. Між тим і другим існує
щільне взаємовідношення. Щоб висловити це взаємовідношення,
я називаю органічним складом капіталу його вартостевий склад,
оскільки останній визначається його технічним складом і відбиває
зміни технічного складу. Де говориться просто про склад капіталу,
там треба завжди розуміти його органічний склад.

Численні поодинокі капітали, вкладені в певну галузь продукції,
більш або менш відрізняються між собою щодо складу.
Пересіччя їхніх поодиноких складів дає нам склад цілого капіталу
цієї галузі продукції. Нарешті, загальне пересіччя цих пересічних
складів усіх галузей продукції дає нам склад суспільного
капіталу якоїсь країни, і тільки про нього в останній інстанції
й буде далі мова.

Зростання капіталу включає і зростання його змінної, або
перетвореної на робочу силу складової частини. Частина додаткової
вартости, перетвореної на додатковий капітал, мусить завжди
знову перетворюватися на змінний капітал, або на додатковий
робочий фонд. Коли ми припустимо, що разом з іншими незмінними
обставинами незмінним лишається і склад капіталу, тобто,
що завжди потрібно тієї самої маси робочої сили для того, щоб
пустити в рух певну масу засобів продукції, або сталого капіталу,
то в такому випадку попит на працю й фонд засобів існування
робітників, очевидно, зростатиме пропорційно до зросту капіталу,
і то швидше, що швидше зростатиме капітал. Через те, що
капітал щорічно продукує додаткову вартість, частину якої
щороку додається до первісного капіталу, через те, що сам цей
приріст щороку зростає із збільшенням розміру капіталу, який
уже функціонує, і через те, насамкінець, що при особливому збудженні
жадоби до збагачення, як от, наприклад, при відкритті
нових ринків, нових сфер вкладення капіталу в наслідок розвитку
нових суспільних потреб і т. ін., маштаб акумуляції можна раптом
поширити самою лише зміною поділу додаткової вартости
або додаткового продукту на капітал і дохід, — через це потреби
акумуляції капіталу можуть випередити зріст робочої сили або
число робітників, попит на робітників — випередити подання
їх, і тому заробітні плати можуть підвищитися. Це, кінець-кінцем,
навіть мусить статися, якщо вищенаведені передумови й
далі існують без зміни. Через те, що кожного року вживається
більше робітників, ніж у попередньому році, то раніш або пізніш
мусить настати момент, коли потреби акумуляції починають
переростати звичайне подання праці, коли, отже, настає підвищення
заробітної плати. Нарікання на це лунають в Англії
протягом цілого XV й першої половини XVIII століття. Однак
більш або менш сприятливі обставини, при яких наймані робітники
зберігаються й розмножуються, не змінюють нічого в основному
характері капіталістичної продукції. Як проста репродукція
\index{i}{0525}  %% посилання на сторінку оригінального видання
безупинно репродукує саме капіталістичне відношення,
капіталістів на одному боці, робітників на другому, так і репродукція
в поширеному маштабі, або акумуляція, репродукує
капіталістичне відношення в поширеному маштабі — більше капіталістів
або більших капіталістів на цьому полюсі, більше
найманих робітників на тому. Репродукція робочої сили, що невпинно
мусить входити до складу капіталу як засіб збільшення
його вартости і не може відокремитись від нього, — робочої сили,
що її підлеглість капіталові лише маскується зміною індивідуальних
капіталістів, яким вона продає себе, становить, в дійсності,
момент репродукції самого капіталу. Отже, акумуляція
капіталу є збільшення пролетаріяту.\footnote{
\emph{К. Marx}: «Lohnarbeit und Kapital». (\emph{K. Маркс}: «Наймана праця
і капітал», Партвидав «Пролетар» 1932). — «За однакового пригнічення
мас що більше в країні пролетарів, то вона багатша» («A égalité
d’oppression des masses, plus un pays a de prolétaires et plus il est riche»).
(\emph{Colins}: «L’Economie Politique, Source des Révolutions et des Utopies
prétendues Socialistes», Paris 1857, vol. III, p. 331). Під «пролетарем»
y політичній економії треба розуміти не що інше, як найманого робітника,
який продукує «капітал» і збільшує його вартість і якого викидають
на брук, скоро тільки він стає зайвим для потреб самозростання
«пана капіталу», як називає цю персону Пекер. «Хоробливий пролетар
пралісу» — це лише чемна фантазія Рошера. Пралісовик є власник того
пралісу й поводиться з пралісом як із своєю власністю, так само безцеремонно
як оранґутанґ. Отже, він не є пролетар. Він був би ним тільки
тоді, коли б праліс експлуатував його, а не він — цей праліс. Щождо
стану його здоров'я, то він витримає порівняння не тільки з сучасним
пролетарем, а й з сифілітичними й золотушними «порядними особами».
А втім, під пралісом пан Вільгельм Рошер розуміє, певно, свою рідну
Lüneburger Heide.
}

Клясична політична економія так добре розуміла цю тезу,
що Адам Сміс, Рікардо й інші, як уже згадано раніш, навіть помилково
ототожнюють акумуляцію із споживанням всієї капіталізованої
частини додаткового продукту продуктивними робітниками,
або з перетворенням її на додаткових найманих робітників.
Уже 1696 р. Джон Беллерс каже: «Коли б якась людина
мала \num{100.000} акрів і стільки ж фунтів стерлінґів грошей і стільки
ж худоби, то чим була б ця багата людина без робітників, як не
робітником? А через те, що робітники роблять людей багатими,
то що більше робітників, то більше багатих\dots{} Праця бідних — то
копальні багатих».\footnote{
«As the Labourers make men rich, so the more Labourers, there
will be the more rich men\dots{} the Labour of the Poor being the Mines of the
Rich». (\emph{John Bellers}: «Proposals for raising a Colledge of Industry»,
London 1696, p. 2).
} Те саме каже й Бернар де Мандевіль
на початку XVIII віку: «Де власність має достатній захист, там
легше було б жити без грошей, ніж без бідних, бо хто ж тоді
працював би?.. Так само, як робітників треба захищати від голодної
смерти, так само не повинні вони одержувати нічого
такого, що варто заощаджувати. Якщо інколи хтось із найнижчої
кляси через незвичайну працьовитість та недоїдання підноситься
понад той стан, у якому він виріс, то ніхто не сміє перешкоджати
йому в цьому: аджеж безперечно, що наймудріша річ
\parbreak{}  %% абзац продовжується на наступній сторінці

\input{i/_0526.tex}
\parcont{}  %% абзац починається на попередній сторінці
\index{i}{0527}  %% посилання на сторінку оригінального видання
а не своїм власним здібностям, які ані трохи не кращі, ніж в
інших; не володіння землею і грішми, а панування над працею
(«the command of labour») відрізняє багатих від бідних\dots{} Бідним
відповідає не стан занедбаности або рабства, а догідний і
ліберальний стан залежности («а state of easy and liberal dependence»),
а людям, що мають власність, відповідає достатній вплив
і авторитет над тими, що на них працюють. Такий стан залежности,
як це знає кожен знавець людської натури, доконечний
для вигоди самих робітників»\footnote{
\emph{Eden}: «The State of the Poor, or an History of the Labouring Glasses
in England», vol. I, b. 1, ch. 1, p. 1, 2 і передмова, p. XX.
}. Сер Ф.~М.~Еден, до речі зауважити,
є єдиний учень Адама Сміса, що протягом XVIIІ століття
зробив дещо важливе\footnote{
Якщо читач нагадає нам Малтуза, що його «Essay on Population»
появився 1798~\abbr{р.}, то я нагадаю, що ця праця у своїй першій формі є не
що інше, як по-школярському поверховий і по-попівському пишномовний
пляґіят з де Фо, Сера Джемса Стюарта, Тавнсенда, Франкліна, Уоллеса
та інших і не має в собі ані однісінької самостійно продуманої тези. Велика
сенсація, яку викликав цей памфлет, пояснюється виключно партійними
інтересами. Французька революція знайшла в Брітанському королівстві
палких оборонців; «принцип залюднення», що повільно вироблявся у
XVIII віці та що його потім підчас великої соціяльної кризи під звуки
сурм і барабанний бій проголосили як непомильну протиотруту супроти
теорій Кондорсе й інших, англійська олігархія вітала з великою радістю,
вбачавши в ньому великого гасителя всіх прагнень до дальшого розвитку
людства. Малтуз, надзвичайно здивований своїм успіхом, заходився тоді
коло того, щоб стару схему заповнити поверхово скомпільованим матеріялом
і додати до нього новий, не Малтузом відкритий, а ним лише
присвоєний. — До речі зауважимо тут, що хоч Малтуз був попом англіканської
церкви, а все ж він дав чернечу обітницю на безженство.
Саме це — одна з умов членства (fellowship) в протестантському кембріджському
університеті. «Ми не дозволяємо, щоб члени колегій були
жонаті. Хто ожениться, той повинен зараз вийти з членів колегії» («Socios
collegiorum maritos esse non permittimus, sed statim postquam quis
uxorem duxerit, socius collegii desinat esse»). («Reports of Cambridge University
Commission», p. 172). Ця обставина вигідно відрізняє Малтуза
від інших протестантських попів, що відкинули католицьку заповідь
попівського безженства і в такій мірі засвоїли заповідь «плодіться й
розмножуйтеся» як свою специфічну біблійну місію, що вони повсюди
в справді непристойних розмірах допомагають збільшувати людність,
тимчасом як робітникам вони проповідують «принцип залюднення».
Характеристично, що економічна пародія гріхопадіння, адамове яблуко,
«непереможне бажання», «перепони, що силкуються притупити стріли
Купідона» («urgent appetite», «the checks which tend to blunt the shafts
of Cupid»), як весело каже піп Тавнсенд, що цей дражливий пункт монополізували
й тепер монополізують пани представники протестантської
теології або, точніше, церкви. За винятком венеціянського ченця Ортеса,
оригінального й талановитого письменника, більшість проповідників
принципу залюднення — це протестантські попи. Такий, наприклад,
Брукер, що в йоги творі «Théorie du Système animal», Ley de 1767 вичерпано
всю сучасну теорію залюднення, до якої подала ідеї короткочасна
суперечка між Кене і його учнем Мірабо-батьком на цю саму тему,
потім ідуть піп Уоллес, піп Тавнсенд, піп Малтуз і його учень, архіпіп
Т.~Чалмерс, не кажучи вже про дрібніших попів-писак того самого напряму.
Первісно над політичною економією працювали філософи, такі,
як Гоббс, Лок, Юм, комерційні й державні люди, як от Томас Мор, Тімпл,
Сюллі, деВітт, Норт, Ло „Вандерлінт, Кантільйон, Франклін, а особливо
над теорією її з великим успіхом працювали медики, як от Петті, Барбон,
Мандевіль, Кене. Ще в середині XVIII віку піп Текер, видатний економіст
свого часу, прохає вибачення за те, що він займався мамоною. Пізніше,
а саме одночасно з «принципом залюднення», настав час протестантських
попів. Неначе передчуваючи це партацтво, Петті, що вважає людність
за базу багатства і в, так само як і Адам Сміс, непримиренний
ворог попів, каже: «Релігія найкраще процвітає там, де священники найбільше
підпадають усмирению плоті, так само як право найкраще процвітає
там, де адвокати вмирають з голоду». Тому він радить протестантським
попам, якщо вони не хочуть іти за прикладом апостола Павла
і «умерщвляти свою плоть» безженством, «принаймні не плодити більше
попів («not to breed more Churchmen»), аніж їх могли б поглинути наявні
парафії (benefices); тобто коли в Англії й Велзі існує лише \num{12.000} парафій,
то нерозумно наплоджувати \num{24.000} попів («it will not be safe to breed
\num{24.000} ministers»), бо \num{12.000} незабезпечених завжди намагатимуться
здобути собі засоби існування, а як можуть вони найлегше досягти цього,
як не ходячи серед народу та переконуючи його в тому, що ті \num{12.000} попів,
що мають парафії, отруюють душі, заморюють їх голодом та вказують
їм неправдивий шлях до неба?» (\emph{Petty}: «A Treatise on Taxes and
Contributions», London 1667, p. 57). Ставлення Адама Сміса до протестантського
попівства його часів характеризується ось чим. В «А Letter to A.~Smith,
L.~L.~D.~On the Life, Death and Philosophy of his Friend David Hume.
By One of the People called Christians», 4th ed. Oxford 1784 англіканський
єпископ д-р Херн із Норвіча докоряє А.~Смісові за те, що він в одному
відкритому листі до пана Стрехена «бальзамує свого приятеля Давіда»
(тобто Юма), що він оповідає публіці, як «Юм на своєму смертному ліжку
розважував себе Лукіяном і Вайстом», і що він навіть мав нахабство написати:
«Я завжди вважав Юма так за його життя, як і після його смерти
таким близьким до ідеалу цілком мудрої й доброчесної людини, як це тільки
дозволяють слабощі людської натури». Єпископ з обуренням вигукує:
«Чи то воно гаразд з вашого боку, мій пане, змальовувати нам як цілком
мудрий і доброчесний характер і побут людини, що була пройнята невигойною
антипатією до всього того, що зветься релігією, і напружувала
кожний свій нерв, щоб, оскільки це від неї залежало, стерти з людської
пам’яті навіть назву релігія?» (Там же, стор. 8). «Але не журіться ви,
приятелі правди, атеїзмові недовго жити» (стор. 17). Адам Сміс — «є
гидкий нечестивець («the atrocious wickedness»), він пропагує у країні
атеїзм (саме своєю «Theory of moral sentiments»)\dots{} Ми знаємо ваші хитрощі,
пане докторе! Ви добрий задум мали, але цим разом ви рахували без господаря.
На прикладі високошановного Давіда Юма ви хочете напоумити
нас, що атеїзм — це єдиний живлющий лік («cordial») для занепалого
духу і єдина протиотрута супроти страху перед смертю\dots{} Глузуйте ж
собі з руїн Вавилону та вітайте озвірілого лиходія Фараона»! (Там же,
стор. 21, 22). Один з ортодоксальних слухачів колегії, де навчав А.~Сміс,
пише після його смерти: «Приятелювання Сміса з Юмом\dots{} перешкоджало
йому бути християнином\dots{} Він вірив Юмові в усьому на слово.
Коли б Юм сказав йому, що місяць — зелений сир, він був би йому повірив.
Тим то він вірив йому, що немає бога й чудес\dots{} Своїми політичними
принципами він наближався до республіканізму». («The Bee». By James
Anderson. 18 volumes. Edinburgh 1791--93, vol. ІІІ, p. 166, 165). Піп
T.~Чалмерс запідозрює А.~Сміса в-тому, що він вигадав категорію «непродуктивних
робітників» просто із злости спеціально для протестантських
попів, не зважаючи на їхню благословенну працю в божому вертограді.
}.

\index{i}{0528}  %% посилання на сторінку оригінального видання
За тих найсприятливіших для робітників умов акумуляції,
які ми досі припускали, відношення залежности робітників од
капіталу прибирається у зносні, або, як каже Еден, у «приємні
й ліберальні» форми. Замість ставати із зростом капіталу інтенсивнішим,
воно стає тільки екстенсивнішим, тобто сфера експлуатації
й, панування капіталу лише поширюється разом із збільшенням
\index{i}{0529}  %% посилання на сторінку оригінального видання
його власного розміру й числа його підданців. Більша
частина їхнього власного додаткового продукту, який чимраз
більше зростає і перетворюється в дедалі більших розмірах на
додатковий капітал, припливає до них назад у формі засобів
платежу, так що вони можуть поширювати межі свого споживання,
краще влаштовувати свій споживний фонд одягу, меблів
і~\abbr{т. д.} і скласти невеликий грошовий резервний фонд. Але як
кращий одяг, ліпший харч, ліпше поводження і більший пекуліюм
не нищать відношення залежности й експлуатації раба,
так само це не нищить відношення залежности й експлуатації
найманого робітника. Підвищення ціни праці в наслідок акумуляції
капіталу свідчить справді лише про те, що розміри й вага
золотого ланцюга, що його сам найманий робітник уже викував
для себе, дозволяють ослабити напругу цього ланцюга. В суперечках
навколо цього предмету здебільшого не добачали головного,
а саме differentia specifica\footnote*{
відмінної ознаки. \emph{Ред.}
} капіталістичної продукції.
Робочу силу тут купують не для того, щоб через її послуги або
її продукт задовольняти особисті потреби її покупця. Мета покупця
— збільшити вартість свого капіталу, продукувати товари,
що містять у собі більше праці, аніж він оплачує, отже,
що містять у собі таку частину вартости, яка нічого не коштує
йому і яку він проте реалізує через продаж товарів. Продукція
додаткової вартости, або нажива — це абсолютний закон капіталістичного
способу продукції. Робоча сила може знаходити
собі покупців лише остільки, оскільки вона зберігає засоби продукції
як капітал, репродукує свою власну вартість як капітал
і в неоплаченій праці дає джерело додаткового капіталу\footnote{
Примітка до 2 видання. «Однак межа зайняття робітників так
промислових, як і сільських однакова: а саме можливість для підприємця
добувати зиск із продукту їхньої праці\dots{} Якщо норма заробітної
плати зростає так високо, що зиск хазяїна падає нижче пересічного
зиску, то хазяїн перестає вживати робітників або вживає їх лише за
тієї умови, щоб вони згодились на зниження заробітної плати». (\emph{John
Wade}: «History of the Middle and Working Classes», 3rd. ed., London
1835, p. 241).
}. Отже,
умови продажу робочої сили, незалежно від того, чи вони більш
чи менш сприятливі для робітників, містять у собі доконечність
постійного повторювання її продажу і репродукцію багатства
як капіталу в щораз ширшому розмірі. Заробітна плата, як ми
бачили, з самої природи своєї постійно зумовлює постачання
робітником певної кількости неоплаченої праці. Залишаючи
цілком осторонь випадки зростання заробітної плати при зниженні
ціни праці тощо, збільшення її означає в найкращому
випадку лише кількісне зменшення неоплаченої праці, що її
мусить давати робітник. Це зменшення ніколи не може дійти до
такого пункту, де воно загрожувало б самій капіталістичній
системі. Залишаючи осторонь насильні конфлікти щодо рівня
заробітної плати, — а вже Адам Сміс показав, що взагалі і в цілому
\index{i}{0530}  %% посилання на сторінку оригінального видання
в таких конфліктах хазяїн завжди лишається хазяїном, —
підвищення ціни праці, що випливає з акумуляції капіталу,
припускає таку альтернативу:

Або ціна праці й далі зростає, бо зріст її не заважає проґресові
акумуляції; в цьому немає нічого дивного, бо каже А.~Сміс,
«навіть за зниженого зиску капітали все ж зростають; вони навіть
зростають швидше, ніж раніш\dots{} Великий капітал навіть
за меншого зиску взагалі зростає швидше, ніж малий капітал за
великого зиску» («Wealth of Nation», р. 189). У цьому випадку
очевидно, що зменшення неоплаченої праці аж ніяк не заважає
капіталові поширювати своє панування. — Або, — і це є другий
бік альтернативи, — акумуляція в наслідок підвищення ціни
праці слабшає, бо притупляється спонукливий стимул баришу.
Акумуляція меншає. Але з її зменшенням зникає причина її
зменшення, а саме зникає диспропорція між капіталом і робочою
силою, приступною для експлуатації. Отже, механізм капіталістичного
процесу продукції сам усуває ті тимчасові перешкоди,
які він утворює. Ціна праці знову спадає до рівня, що
відповідає потребам зростання капіталу, все одно, чи цей рівень
нижчий, вищий або рівний тому, що його вважалося за нормальний
перед початком зростання заробітної плати. Ми бачимо:
в першому випадку не зменшення абсолютного або відносного
зростання робочої сили або робітничої людности робить капітал
надмірним, а, навпаки, збільшення капіталу робить недостатньою
приступну для експлуатації робочу силу. У другому випадку
не збільшення абсолютного або відносного зростання робочої
сили або робітничої людности робить капітал недостатнім, а,
навпаки, зменшення капіталу робить надмірною приступну для
експлуатації робочу силу, або, точніше, її ціну. Оці абсолютні
рухи акумуляції капіталу відбиваються як відносні рухи в масі
приступної для експлуатації робочої сили, і тому здається, нібито
їх спричиняє власний рух останньої. Вживаючи математичного
вислову: величина акумуляції є незалежна змінна, величина
заробітної плати — залежна, а не навпаки. Так, у промисловому
циклі підчас фази кризи загальний спад товарових цін виражається
як підвищення відносної вартости грошей, а підчас фази
розцвіту — загальне підвищення товарових цін виражається як
спад відносної вартости грошей. Так звана Currency-школа
робить із цього висновок, що за високих цін циркулює замало
грошей, а за низьких — забагато. Її неуцтво й повне нерозуміння
фактів\footnote{
Порівн. \emph{К.~Marx}: «Zur Kritik der Politischen Oekonomie»,
S. 166 і далі. (\emph{К.~Маркс}: «До критики політичної економії», ДВУ 1926,
стор. 171 і далі).
} находять собі гідну паралелю в тих економістів,
які ті явища акумуляції пояснюють тим, що в одному випадку
існує замало, а в другому забагато найманих робітників.

Закон капіталістичної продукції, що лежить в основі нібито
«природного закону залюднення», сходить просто ось на що:
відношення між капіталом, акумуляцією й нормою заробітної
\parbreak{}  %% абзац продовжується на наступній сторінці

\parcont{}  %% абзац починається на попередній сторінці
\index{i}{0531}  %% посилання на сторінку оригінального видання
плати є не що інше, як відношення між неоплаченою працею,
перетвореною на капітал, і новододаваною працею, потрібного, щоб
пустити в рух додатковий капітал. Отже, це зовсім не відношення
між двома незалежними одна від одної величинами, між величиною
капіталу, з одного боку, і кількістю робітничої людности —
з другого; навпаки, це в останній інстанції тільки відношення
між неоплаченою та оплаченою працею тієї самої робітничої людности.
Коли кількість неоплаченої праці, постачуваної робітничою
клясою й нагромаджуваної клясою капіталістів, зростає
досить швидко, так що вона може перетворюватися на капітал
лише за допомогою надзвичайного додатку оплаченої праці, то
заробітна плата підвищується і, за всіх інших незмінних умов,
неоплачена праця відносно меншає. Але скоро тільки це зменшення
доходить до того пункту, коли додаткову працю, з якої
живиться капітал, не постачається вже в нормальній кількості,
то постає реакція: капіталізується меншу частину доходу, акумуляція
слабшає, а висхідний рух заробітної плати змінюється
на протилежний. Отже, підвищення ціни праці ніколи не може
вийти за ті межі, які не тільки лишають недоторканими основи
капіталістичної системи, а й забезпечують її репродукцію в
щораз більшому маштабі. Отже, закон капіталістичної акумуляції,
змістифікований на закон природи, в дійсності виражає
лише те, що природа акумуляції виключає всяке таке зменшення
ступеня експлуатації праці або всяке таке підвищення ціни
праці, яке серйозно могло б загрожувати постійній репродукції
капіталістичного відношення й репродукції його в щораз ширшому
маштабі. Інакше й не може бути за такого способу продукції,
коли робітник існує для потреб збільшення наявних вартостей
замість, навпаки, речовому багатству існувати для потреб
розвитку робітника. Як у релігії людину опановує витвір її власної
голови, так за капіталістичної продукції її опановує витвір
її власної руки.\footnoteA{
«Але коли ми повернемося тепер до нашого першого досліду,
де показано\dots{} що сам капітал е лише продукт людської праці\dots{} то здається
цілком незрозумілим, яким чином людина могла опинитися під
пануванням свого власного продукту, капіталу, і підкоритися йому;
а що це в дійсності в безперечний факт, то мимоволі постає питання, яким
чином робітник із пана над капіталом, як творець його, міг зробитися
рабом капіталу?» (\emph{Von Thünen}: «Der isolierte Staat», Zweiter Teil, Zweite
Abteilung, Rostock 1863, S. 5, 6). Заслуга Тінена в тому, що він поставив
це питання. Але відповідь його просто дитяча.
}

\subsection{Відносне зменшення змінної частини капіталу з проґресом
акумуляції і концентрації, що її супроводить}

На думку самих економістів, до підвищення заробітної плати
призводить не наявний розмір суспільного багатства й не величина
вже надбаного капіталу, а виключно лише безупинний
зріст акумуляції капіталу та ступінь швидкости її зросту
\parbreak{}  %% абзац продовжується на наступній сторінці

\parcont{}  %% абзац починається на попередній сторінці
\index{i}{0532}  %% посилання на сторінку оригінального видання
(\emph{A. Smith}: «Wealth of Nation», кн. І, розд. 8). Досі ми розглядали
лише одну особливу фазу цього процесу, ту, коли приріст капіталу
відбувається за незмінного технічного складу капіталу.
Але цей процес іде поза межі цієї фази.

Скоро загальні основи капіталістичної системи вже дані, то
в перебігу акумуляції завжди постає такий момент, коли розвиток
продуктивности суспільної праці стає наймогутнішою
підоймою акумуляції. «Та сама причина, — каже А. Сміс, —
яка підвищує заробітну плату, а саме збільшення капіталу,
спонукує до піднесення продуктивних здібностей праці й дає
змогу меншій кількості праці продукувати більшу кількість
продуктів»\footnote*{
Замість останніх двох абзаців у першому німецькому виданні
тут читаємо таке:

«Розвинене в попередньому підрозділі має силу лише за тієї передумови,
що в перебігу акумуляції відношення між масою засобів продукції і
масою робочої сили, що пускає їх у рух, лишається незмінним, отже,
за передумови, що попит на працю зростає відповідно до зростання капіталу.
Це припущення фігурує в Адама Сміса в його аналізі акумуляції
як сама собою зрозуміла аксіома. [До певної міри воно завжди лишається
правильним, бо хоч і як можуть революціонізуватись технологічні умови
процесу продукції, всеж протягом коротшого або довшого періоду то в одній,
то в другій сфері продукції акумуляція капіталу, або поширення маштабу
продукції відбувається на вже даній технологічній базі. Отже, у цих межах
попит на працю зростає разом з акумуляцією. Але сама наявна база
безупинно революціонізується]\footnotemarkZ{}.
В перебігу акумуляції відбувається
велика революція у відношенні між масою засобів продукції і масою робочої
сили, що пускає їх у рух. Ця революція відбивається на мінливому
складі капітальної вартости, на її поділі на сталу і змінну складові частини,
тобто на мінливому відношенні між її частинами, що перетворюються
на засоби продукції і на робочу силу». \emph{Ред.}
}\footnotetextZ{
Заведене у прямі дужки Маркс з другого видання вилучив. Дальшу
фразу він тут починає так: «Але Адам Сміс недобачає, що в перебігу
акумуляції і т. д\dots{}». \emph{Ред.}
}.

Залишаючи осторонь природні умови, як родючість ґрунту
тощо, і вправність незалежних продуцентів, що працюють ізольовано,
вправність, яка, однак, виявляється більше якісно, в якості
продукту, аніж кількісно, в масі продукту, ступінь суспільної
продуктивности праці виражається у відносній величині розміру
засобів продукції, що їх якийсь робітник протягом даного часу
перетворює на продукт з тим самим напруженням робочої сили.
Маса засобів продукції, що за їхньою допомогою він функціонує,
зростає разом з продуктивністю його праці. При цьому ці засоби
продукції відіграють двояку ролю. Зростання одних є наслідок,
зростання інших — умова зростання продуктивности праці.
Наприклад, при мануфактурному поділі праці і при застосуванні
машин за той самий час перероблюють більше сировинного матеріялу,
отже, більша маса сировинного матеріялу й допоміжних
матеріялів увіходить у процес праці. Це — наслідок зростання
продуктивности праці. З другого боку, маса застосованих машин,
робочої худоби, мінерального добрива, дренажних труб і~\abbr{т. д.}
є умова зростання продуктивности праці. Те ж саме стосується
\parbreak{}  %% абзац продовжується на наступній сторінці

\parcont{}  %% абзац починається на попередній сторінці
\index{i}{0533}  %% посилання на сторінку оригінального видання
і до маси засобів продукції, сконцентрованих у будівлях, велетенських
домнах, засобах транспорту й~\abbr{т. ін.} Але чи є зростання
величини розміру засобів продукції порівняно з долученою до
них робочою силою умова чи наслідок — усе одно воно виражає
зростання продуктивности праці. Отже, збільшення продуктивности
праці виявляється у зменшенні маси праці порівняно до
тієї маси засобів продукції, що її пускає в рух ця праця, або у
зменшенні величини суб’єктивного фактора процесу праці порівняно
з його об’єктивними факторами.

[З постанням великої промисловости в Англії винайдено
спосіб перетворювати чавун з коксом на ковке залізо. Цей
спосіб, який звуть пудлінґуванням, і який полягає в тому, що
в печах особливої конструкції чавун очищають од вуглецю,
спричинив величезне поширення домен, вживання повітронагрівних
апаратів і~\abbr{т. ін.}, коротко кажучи, таке збільшення знарядь
праці і матеріялів праці, урухомлюваних тією самою кількістю
праці, що незабаром можна було постачати залізо у досить великій
кількості і досить дешево для того, щоб у багатьох випадках
витиснити з ужитку камінь і дерево. А що залізо й вугілля є
великі підойми сучасної промисловости, то значення цього новозаведеного
способу ніяк не можна перебільшити.

Однак пудлінґувач, робітник, занятий очищенням чавуну,
виконує ручну операцію; отож величина печей, що їх він може
обслуговувати, обмежена його особистими здібностями, і саме
ця межа затримує тепер те дивовижне піднесення металюрґійної
промисловости, що почалося 1780 року, року винаходу пудлінґування.

«Факт той, — вигукує «The Engineering», один з органів
англійських інженерів, — що застарілий спосіб ручного пудлінґування
є не що інше, як рештки від варварства (the fact is that the old
process of hand-puddling is little better than a barbarism)\dots{}
Сучасна тенденція нашої промисловости в тому, щоб на різних
ступенях фабрикації обробляти чимраз більші маси матеріялу.
Тим то ми бачимо, що майже щороку постають чимраз
більші домни, чимраз важкіші парові молоти, чимраз могутніші
вальцівні варстати і велетенські знаряддя, що їх застосовують
у багатьох галузях металюрґії. Серед цього загального зросту —
зросту засобів продукції проти уживаної праці — спосіб пудлінґування
залишився майже незмінним і ставить тепер нестерпні
межі промисловому розвиткові\dots{} Тим то по всіх великих заводах
починають замінювати ручний спосіб пудлінґування на
печі з автоматичним перемішуванням, що дає можливість колосально
навантажувати печі цілком незалежно від меж ручної
праці». («The Engineering», 13 June, 1874).

Отже, пудлінґування, після того, як воно революціонізувало
металюрґійну промисловість і викликало величезне збільшення
засобів праці і матеріялів праці, оброблюваних певною кількістю
праці, стало в перебігу акумуляції економічним гальмом. Від
цього гальма промисловість тепер намагається визволитися новими
\parbreak{}  %% абзац продовжується на наступній сторінці

\input{i/_0534.tex}
\input{i/_0535.tex}
\parcont{}  %% абзац починається на попередній сторінці
\index{i}{0536}  %% посилання на сторінку оригінального видання
додаткового продукту, який, з свого боку, є творчий елемент
акумуляції. Отже, вони є разом з тим методи продукції капіталу
капіталом, або методи прискореної акумуляції. Безперервне
перетворювання додаткової вартости знову на капітал виражається
в зростанні величини капіталу, що входить у процес продукції.
З свого боку це стає основою поширеного маштабу продукції,
основою тих метод підвищення продуктивної сили праці,
які супроводять це поширення, і основою прискореної продукції
додаткової вартости. Отже, якщо певний ступінь акумуляції
капіталу являє собою умову специфічно капіталістичного способу
продукції, то цей останній, з свого боку, спричинює прискорену
акумуляцію капіталу. Тому з акумуляцією капіталу розвивається
специфічно капіталістичний спосіб продукції, а із специфічно
капіталістичним способом продукції — акумуляція капіталу.
Ці обидва економічні фактори силою того складного взаємовідношення,
через яке вони один одному дають поштовх,
зумовлюють ту зміну в технічному складі капіталу, наслідком
якої змінна складова частина стає щораз меншою й меншою порівняно
із сталою складовою частиною.

Кожний індивідуальний капітал є більша або менша концентрація
засобів продукції з відповідним пануванням над більшою
або меншою армією робітників. Кожна акумуляція стає засобом
нової акумуляції. Вона поширює із збільшенням маси багатства,
що функціонує як капітал, його концентрацію в руках індивідуальних
капіталістів, а тому поширює й основу продукції
у великому маштабі і основу специфічно капіталістичних метод
продукції. Зростання суспільного капіталу відбувається через
зростання багатьох індивідуальних капіталів. Якщо припустити
всі інші умови за незмінні, то індивідуальні капітали, а разом
з ними й концентрація засобів продукції зростають у тій пропорції,
в якій вони становлять певні частини цілого суспільного
капіталу. Разом з тим від первісного капіталу відриваються
паростки й функціонують як нові самостійні капітали. При цьому
велику ролю відіграє, між іншим, поділ майна в родинах капіталістів.
Тому з акумуляцією капіталу більш або менше зростає
й число капіталістів. Дві обставини характеризують цей рід
концентрації, що безпосередньо спирається на акумуляцію,
або, краще сказати, є з нею ідентична. Поперше: зростання концентрації
суспільних засобів продукції в руках індивідуальних
капіталістів, за інших незмінних умов, є обмежене ступенем
зростання суспільного багатства. Подруге: кожна частина суспільного
капіталу, вкладена в кожну осібну сферу продукції,
є поділена між багатьма капіталістами, які протистоять один
одному як незалежні товаропродуценти, що один з одним конкурують.
Отже, акумуляція й концентрація, що її супроводить,
не тільки роздрібнюються по багатьох пунктах, але й зростання
капіталів, що функціонують, перехрещується з утворенням нових
капіталів і роздрібненням старих. Тому, якщо акумуляція виявляється,
з одного боку, як щораз більша концентрація засобів
\parbreak{}  %% абзац продовжується на наступній сторінці

\parcont{}  %% абзац починається на попередній сторінці 
\index{i}{0537}  %% посилання на сторінку оригінального видання 
продукції й панування над працею, то, з другого боку, вона
виявляється як взаємне відштовхування багатьох індивідуальних
капіталів.

Цьому роздрібненню цілого суспільного капіталу на багато
індивідуальних капіталів або взаємному відштовхуванню його
частин протидіє їхнє притягання. Це вже не проста, ідентична
з акумуляцією концентрація засобів продукції й панування над
працею. Це концентрація утворених уже капіталів, знищення
їхньої індивідуальної самостійности, експропріяція капіталіста
капіталістом, перетворення багатьох дрібних капіталів на незначне
число великих капіталів. Цей процес відрізняється від
першого тим, що він має за свою передумову лише зміну в розподілі
тих капіталів, які вже існують і функціонують, отже, його
поле діяльности не обмежене абсолютним зростанням суспільного
багатства або абсолютними межами акумуляції. Капітал зростає
великими масами тут, в одних руках, бо він зникає там, з багатьох
рук. Це — централізація у власному значенні слова, відмінно
від акумуляції й концентрації.

Законів цієї централізації капіталів або притягання капіталу
капіталом ми не можемо тут досліджувати. Досить буде коротких
фактичних вказівок. Конкуренційна боротьба провадиться через
здешевлення товарів. Дешевина товарів залежить, за інших
незмінних обставин, від продуктивности праці, а ця остання
залежить від маштабу продукції. Тим то більші капітали побивають
дрібніші. Пригадаймо собі, далі, що з розвитком капіталістичного
способу продукції зростає мінімальний розмір індивідуального
капіталу, потрібного на те, щоб провадити підприємство
в нормальних умовах. Тому дрібніші капітали ринуть
у такі сфери продукції, що їх велика промисловість опановує лише
спорадично або не цілком. Конкуренція лютує тут просто пропорційно
до числа й зворотно пропорційно до величини капіталів,
що борються між собою. Вона завжди кінчаєтеся загином багатьох
дрібних капіталістів, що їхні капітали почасти переходять до
рук переможців, а почасти гинуть. Крім цього, разом з капіталістичною
продукцією постає цілком нова сила, кредит, що спочатку
потайки прокрадається як скромний помагач акумуляції,
незримими нитками стягує в руки індивідуальних або асоційованих
капіталістів грошові засоби, розпорошені більшими або
меншими масами по поверхні суспільства; але незабаром він стає
новою і страшною зброєю в конкуренційній боротьбі і, кінець-кінцем,
перетворюється на велетенський соціяльний механізм
для централізації капіталів.

Тією самою мірою, як розвивається капіталістична продукція
й акумуляція, розвиваються також конкуренція і кредит,
ці обидві наймогутніші підойми централізації. Поруч цього
проґрес акумуляції збільшує матеріял, що його можна централізувати,
тобто збільшує поодинокі капітали, тимчасом як поширення
капіталістичної продукції утворює, з одного боку,
суспільну потребу, а з другого — технічні засоби для тих потужтшх
\index{i}{0538}  %% посилання на сторінку оригінального видання 
промислових підприємств, що здійснення їх зв’язане з попередньою
централізацією капіталу. Отже, за наших часів сила
взаємного притягання поодиноких капіталів і тенденція до централізації
дужча, ніж коли-будь раніш. Але, хоч відносне поширення
й енерґія руху в напрямі централізації визначається до
деякої міри досягнутою вже величиною капіталістичного багатства
й вищістю економічного механізму, проте проґрес централізації
зовсім не залежить від позитивного зростання величини
суспільного капіталу. І саме це й відрізняє централізацію від
концентрації, яка є лише інший вираз репродукції в поширеному
маштабі. Централізація може відбуватися через просту зміну
в розподілі капіталів, що вже існують, через просту зміну кількісного
угруповання складових частин суспільного капіталу.
Капітал тут може в одних руках зрости до велетенських розмірів,
бо там його витягнуто з багатьох поодиноких рук. У кожній
даній галузі підприємства централізація досягла б своєї крайньої
межі, коли б усі вкладені в неї капітали злилися в одинодним
капітал У кожному даному суспільстві цієї межі було б
досягнуто лише в той момент, коли цілий суспільний капітал
було б сполучено або в руках одного окремого капіталіста, або
в руках одним-одного товариства капіталістів.

Централізація доповнює справу акумуляції, даючи промисловим
капіталістам змогу поширювати маштаб своїх операцій.
Чи буде цей останній результат наслідком акумуляції або централізації;
чи відбувається централізація насильницьким шляхом
анексії, — коли деякі капітали стають такими потужними
центрами притягання для інших, що вони руйнують їхню індивідуальну
з’єднаність і потім притягують до себе ці роз’єднані
частини; чи злиття маси капіталів уже утворених або таких,
що перебувають у процесі творення, відбувається лагіднішим
способом, через утворення акційних товариств, — економічний
ефект в усіх цих випадках лишається той самий. Зростання розмірів
промислових підприємств повсюди становить вихідний
пункт для ширшої організації спільної праці багатьох, для
ширшого розвитку її матеріяльних рушійних сил, тобто для
проґресивного перетворення розрізнених і рутинних процесів
продукції на суспільно комбіновані й науково організовані
процеси продукції.

Але ясно, що акумуляція, поступінне збільшення капіталу
за допомогою репродукції, яка з колової форми переходить у
спіралю, є надто повільний процес порівняно з централізацією,
що потребує лише зміни кількісного угруповання інтеґральних
частин суспільного капіталу. Світ і досі був би ще без залізниць,
коли б йому довелось чекати, доки акумуляція доведе поодинокі
капітали до такого розміру, що зробив би їх здатними до буду77b
[До четвертого видання. — Найновіші англійські й американські
«трести» прагнуть уже цієї мети, силкуючися з’єднати принаймні всі
великі підприємства певної галузі промисловосте в одно велике акційне
товариство з практичною монополією. — Ф. Е.].
\index{i}{0539}  %% посилання на сторінку оригінального видання 
вання залізниць. Навпаки, централізація за допомогою акційних
товариств досягла цього наче одним махом руки. Збільшуючи
і прискорюючи таким чином діяння акумуляції, централізація
одночасно поширює і прискорює ті перевороти в технічному
складі капіталу, що збільшують його сталу частину коштом його
змінної частини й тим зменшують відносний попит на працю.

Маси капіталу, що їх миттю збиває до купи централізація,
репродукуються та збільшуються так само, як і інші капітали,
тільки швидше, і таким чином вони стають новими могутніми
підоймами суспільної акумуляції. Отже, коли говорять про прогрес
суспільної акумуляції, то під нею за наших часів мовчки
розуміють і діяння централізацій.

Додаткові капітали, утворені в перебігу нормальної акумуляції
(див. розділ XXII, І), служать переважно як засоби
експлуатації нових винаходів, відкрить тощо, одним словом,
промислових удосконалень. Але з часом і для старого капіталу
приходить момент відновлення його голови й членів, момент,
коли він змінює свою шкуру й теж відроджується в такій удосконаленій
технічній формі, коли досить меншої маси праці,
щоб пускати в рух більшу масу машин і сировинних матеріялів.
Абсолютне зменшення попиту на працю, що звідси неминуче
випливає, буде, ясна річ, то більше, що більше в наслідок руху
централізації є вже нагромаджені великими масами капітали,
які пророблюють цей процес відновлення.\footnote*{
Наводимо тут переклад цього абзацу за другим німецьким виданням,
де його подано повніше: «Зростання розміру індивідуальних мас
капіталу стає матеріяльною базою постійного перевороту в самому способі
продукції. Капіталістичний спосіб продукції безупинно завойовує
такі галузі праці, що зовсім ще не підпорядковані йому або підпорядковані
лише спорадично або лише формально. Поруч цього на ґрунті
цього ж способу продукції виростають нові галузі праці, що а самого початку
належать до нього. Нарешті, в галузях праці, проваджуваних
уже капіталістично, продуктивна сила праці виростає, наче в теплиці.
В усіх цих випадках число робітників знижується порівняно до маси
оброблюваних ними засобів продукції. Чимраз більша частина капіталу
перетворюється на засоби продукції, чимраз менша — на робочу силу.
Разом із збільшенням розмірів, концентрації і технічного діяння
засобів продукції, проґресивно зменшується їхня роля як засобів, що
дають заняття робітникам. Паровий плуг куди ефективніший засіб продукції,
ніж звичайний плуг, але вмішена в ньому капітальна вартість
куди меншою мірою є засіб, що дає заняття робітникам, ніж коли б її
було зреалізовано в звичайному плузі. Спочатку якраз долучення нового
капіталу до старого дозволяє поширити і технічно зреволюціонізувати
речові умови процесу продукції. Але незабаром зміна складу і технічна
перебудова більшою або меншою мірою охоплює ввесь старий капітал,
для якого настав строк репродукції і який тому наново репродукується.
Ця метаморфоза старого капіталу до певної міри так само не залежить від
абсолютного зростання суспільного капіталу, як і централізація. Але ця
остання, що лише інакше розподіляє наявний суспільний капітал і з’єднує
багато старих капіталів в один капітал, і собі діє як потужний чинник
у цій метаморфозі старого капіталу». Ред.
}

Отже, з одного боку, додатковий капітал, що утворився в
розвитку акумуляції, притягує порівняно з своєю величиною
\parbreak{}  %% абзац продовжується на наступній сторінці

\parcont{}  %% абзац починається на попередній сторінці
\index{i}{0540}  %% посилання на сторінку оригінального видання
дедалі менше й менше робітників. З другого боку, старий капітал,
що періодично репродукується в новому складі, відштовхує
дедалі більше й більше робітників, що їх він раніше вживав.

\subsection[Прогресивне утворення відносного перелюднення,
або промислової резервної армії]{Прогресивне утворення відносного перелюднення,
або промислової резервної армії\footnotemarkZ{}}
\footnotetextZ{У додатках подаємо уривок з цього параграфа за французьким
виданням. Див. стор. 736. \emph{Ред.}}

Акумуляція капіталу, що первісно виступала лише як кількісне
поширення, відбувається, як ми бачили, при безперервній
якісній зміні його складу, при постійному збільшенні його сталої
складової частини коштом змінної\footnoteA{
Примітка до третього видання. — У власному примірнику Маркса
тут на берегах книги є така увага: «Тут для пізнішого треба зауважити:
коли поширення є лише кількісне, то за більшого або меншого капіталу
в тій самій галузі промисловости зиски відносяться один до одного як
величини авансованих капіталів. Якщо кількісне поширення діє і якісно,
то разом з тим підноситься норма зиску для більшого капіталу». [\emph{Ф. Е.}].
}.

Специфічно капіталістичний спосіб продукції, відповідний
до нього розвиток продуктивної сили праці, спричинена ним
зміна в органічному складі капіталу не тільки йдуть пліч-о-пліч
з розвитком акумуляції або з зростанням суспільного багатства.
Вони йдуть уперед куди швидше, бо проста акумуляція,
або абсолютне поширення цілого капіталу супроводиться централізацією
його індивідуальних елементів, а технічний переворот
у додатковому капіталі супроводиться технічним переворотом
у первісному капіталі. Отже, з проґресом акумуляції
відношення сталої частини капіталу до змінної, коли воно було
первісно 1: 1, змінюється в 2: 1, 3: 1, 4: 1, 5: 1, 7: 1 і~\abbr{т. д.},
так що із зростанням капіталу на робочу силу замість \sfrac{1}{2} його
загальної вартости перетворюється проґресивно лише \sfrac{1}{3}, \sfrac{1}{4},
\sfrac{1}{5} \sfrac{1}{6} \sfrac{1}{8} і~\abbr{т. д.}, а на засоби продукції, навпаки, — \sfrac{2}{3}, \sfrac{3}{4}, \sfrac{4}{5},
\sfrac{5}{6}, \sfrac{7}{8} і~\abbr{т. д.} Отже, через те, що попит на працю визначається не
розміром цілого капіталу, а розміром його змінної складової
частини, то із зростанням цілого капіталу попит на працю проґресивно
падає, замість, як це раніше припускалося, більшати
пропорційно до цього зростання. Він падає відносно проти величини
цілого капіталу і в щораз швидшій проґресії із зростанням
цієї величини. Щоправда, із зростанням цілого капіталу зростає
і його змінна складова частина, або додавана до нього робоча
сила, але зростає вона в щораз меншій пропорції. Павзи, що протягом
їх акумуляція діє як просте поширення продукції на
даній технічній основі, скорочуються. Але мало того, що прискорена
в чимраз більшій проґресії акумуляція цілого капіталу
потрібна, щоб поглинути певне додаткове число робітників або
навіть щоб дати заняття тим робітникам, які вже функціонують
та в наслідок постійної метаморфози старого капіталу втрачають
роботу. Це щораз більше зростання акумуляції й централізації
ще й собі перетворюється на джерело нових змін у складі капіталу,
\index{i}{0541}  %% посилання на сторінку оригінального видання
або нового прискореного зменшення його змінної складової
частини порівняно із сталою. Це відносне зменшення змінної
складової частини капіталу, що прискорюється із зростанням
цілого капіталу і до того ж куди швидше, ніж його власне зростання,
видається на другому боці, навпаки, щораз швидшим
абсолютним зростанням робітничої людности, щораз швидшим,
ніж зростання змінного капіталу або засобів для праці цієї людности.
У дійсності ж капіталістична акумуляція постійно продукує,
і саме пропорційно до своєї енерґії і свого розміру, відносну,
тобто для середніх потреб самозростання капіталу надмірну,
а тому й зайву, або додаткову робітничу людність.

Розглядаючи цілий суспільний капітал, ми бачимо, що рух
його акумуляції то викликає періодичні зміни, то моменти цього
руху одночасно розподіляються між різними сферами продукції.
У деяких сферах відбувається зміна в складі капіталу без зростання
його абсолютної величини, в наслідок простої концентрації;
у деяких сферах абсолютне зростання капіталу зв’язане з абсолютним
зменшенням його змінної складової частини, або вбируваної
ним робочої сили; у деяких сферах то капітал і далі зростає
на своїй даній технічній основі і пропорційно до свого зростання
притягує додаткову робочу силу, то постає органічна зміна капіталу
і зменшується його змінна складова частина; в усіх сферах
зростання змінної частини капіталу, а тому й числа занятих
робітників, завжди зв’язане з великими коливаннями й тимчасовим
утворенням перелюднення, однаково, чи набирає воно
помітнішу форму відштовхування занятих уже робітників, чи
менш помітну, але не менш дійову форму утрудненого поглинення
додаткової робітничої людности її звичайними відвідними каналами\footnote{
Перепис в Англії та Велзі показує, між іншим:

Всіх осіб, що працювали в рільництві (залічуючи сюди власників,
фармерів, садівників, пастухів і~\abbr{т. д.}), було 1851~\abbr{р.} \num{2.011.447}, 1861~\abbr{р.} —
\num{1.924.110}, зменшення — \num{87.337}. Вовняна мануфактура: 1851~\abbr{р.} — \num{102.714}
осіб, 1861~\abbr{р.} — \num{79.242}; шовкові фабрики: 1851~\abbr{р.} — \num{111.940}; 1861~\abbr{р.} —
\num{101.678}; ситцевибійники: 1851~\abbr{р.} — \num{12.098}, 1861~\abbr{р.} — \num{12.556}; це незначне
збільшення, поруч величезного поширення підприємства, означає
велике відносне зменшення числа занятих робітників. Капелюшники:
1851~\abbr{р.} — \num{15.957}, 1861~\abbr{р.} — \num{18.814}: виробники солом’яних і жіночих капелюхів:
1851~\abbr{р.} — \num{20.393}, 1861~\abbr{р.} — \num{18.176}; солодівники: 1851~\abbr{р.} —
\num{10.566}, 1861~\abbr{р.} — \num{10.677}; виробники свічок: 1851~\abbr{р.} — \num{4.949} осіб, 1861~\abbr{р.} —
\num{4.686}. Це зменшення є, між іншим, наслідок поширення газового освітлення.
Гребінники: 1851~\abbr{р.} — \num{2.038}, 1861~\abbr{р.} — \num{1.478}; пильщики 1851~\abbr{р.} —
\num{30.552}, 1861~\abbr{р.} — \num{31.647}, невеличке збільшення в наслідок розвитку
машин до пиляння; голкарі: 1851~\abbr{р.} — \num{26.940}, 1861~\abbr{р.} — \num{26.130}, зменшення
в наслідок машинової конкуренції; робітники по копальнях цинку й
міді: 1851~\abbr{р.} — \num{31.360}, 1861~\abbr{р.} — \num{32.041}. Навпаки, бавовнопрядні та бавовноткальні:
1851~\abbr{р.} — \num{371.777}, 1861~\abbr{р.} — \num{456.646}; кам’яновугляні копальні:
1851~\abbr{р.} — \num{183.389}, 1861~\abbr{р.} — \num{246.613}. «Взагалі збільшення числа робітників
після 1851~\abbr{р.} найбільше в таких галузях, де досі ще не застосовувано
машин з успіхом». («Census of England and Wales for 1861», vol. III.
London 1863, p. 35--39).
}.
Разом з величиною суспільного капіталу, що вже функціонує,
і з ступенем його зростання, з поширенням маштабу
продукції й маси робітників, пущених у рух, з розвитком про-
\parbreak{}  %% абзац продовжується на наступній сторінці

\parcont{}  %% абзац починається на попередній сторінці
\index{i}{0542}  %% посилання на сторінку оригінального видання
сили їхньої праці, з поширенням і збільшенням усіх
джерел багатства поширюється й маштаб, в якому більше притягування
робітників капіталом зв’язане з більшим їх відштовхуванням,
зростає швидкість зміни органічного складу капіталу
та його технічної форми й ширшає коло тих сфер продукції, що
їх то одночасно, то навпереміну охоплює ця зміна. Отже, робітнича
людність, разом з продукованою нею самою акумуляцією
капіталу, продукує в щораз більшому розмірі засоби, які роблять
саму її відносно
зайвою\footnote{\label{footnote-79}Деякі видатні економісти
клясичної школи більше передчували,
ніж розуміли закон проґресивного зменшення відносної величини змінного
капіталу і його вплив на становище кляси найманих робітників.
Найбільша заслуга в цій справі належить Джонові Бартону, хоч і він, як
усі інші, сплутує сталий капітал з основним, а змінний з обіговим. Він
каже: «Попит на працю залежить від зростання обігового капіталу, а
не основного. Коли б це була правда, що відношення поміж цими двома
відмінами капіталу за всяких часів і серед усяких обставин однакове,
то з цього випливало б, що число занятих робітників є пропорційне до
багатства держави. Але таке припущення не має й тіні ймовірности. Що
більше розвиваються промисли й поширюється цивілізація, то більше
й більше основний капітал переважає обіговий. Сума основного капіталу,
вживаного для продукції однієї штуки англійського мусліну, щонайменше
всотеро, а може і в тисячу разів більша, ніж основний капітал, що його
вживають на продукцію такої самої штуки індійського мусліну. А обіговий
капітал відносно в сто або тисячу разів менший\dots{} Коли б усю суму річних
заощаджень додавано до основного капіталу, то це все ж не спричинило
б жодного впливу на зростання попиту на працю». («The demand for
labout depends on the increase of circulating and not of fixed capital. Were
it true that the proportion between these two sorts of capital is the same at
all times, and in all circumstances, then, indeed, it follows that the number
of labourers employed is in proportion to the wealth of the state.
But such a proposition has not the semblance of probability. As arts are
cultivated, and civilization is extended, fixed capital bears a larger and
larger proportion to circulating capital. The amount of fixed capital employed
in the production of a piece of British muslin is at least a hundred, probably
a thousand times greater than that employed in a similar piece of
Indian muslin. And the proportion of circulating capital is a hundred or thousand times less\dots{} the whole of the annual savings, added to the fixed
capital, would have no effect in increasing the demand for labour»). (\emph{John Barton}: «Observations of the circumstances which influence the Condition
of the Labouring Classes of Society», London 1817, p. 16, 17). «Та
сама причина, в наслідок якої зростає чистий дохід країни, може одночасно
на другому боці зробити людність надмірною і погіршити становище
робітників» («The same cause which may increase the net revenue of the country
may at the same time render the population redundant, and deteriorate
the condition of the labourer»). (\emph{Ricardo}: «Principles of Political Economy»,
3 rd cd. London 1821, p. 469). Із збільшенням капіталу «попит (на
працю) відносно дедалі зменшується» («the demand (for labour) will be ina diminishing
ratio»). (Там же, стор. 480, примітка). «Сума капіталу, призначена
на утримання праці, може варіювати незалежно від якихбудь змін
у загальній сумі капіталу\dots{} Великі коливання в наявній кількості роботи
й великі страждання можуть ставати частішими в міру того, як сам капітал
зростає». (The amount of capital devoted to the maintenance of labour
may vary, independently of any changes in the whole amount of capital\dots{}
Great fluctuations in the amount of employment, and great suffering may
become more frequent as capital itself becomes more plentiful»). (\emph{Richard Jones}: «An Introductory Lecture on Political Economy», London 1833, p. 13).
«Попит (на працю) зростатиме\dots{} не пропорційно до акумуляції загального
капіталу\dots{} Тому всяке збільшення національного капіталу, призначеного
на репродукцію, з проґресом суспільства справлятиме щораз
менший вплив на становище робітника» («Demand (for labour) will rise\dots{}
not in proportion to the accumulation of the general capital\dots{} Every
augmentation, therefore to the national stock destined for reproduction,
comes, in the progress of society, to have a less and less influence upon the
condition of the labourer»). (\emph{G.~Ramsay}: «An Essay on the Distribution,
of Wealth», Edinburgh 1836, p. 90, 91).

}. Це є властивий капіталістичному
способові продукції закон населення, як і кожному осібному
історичному способові продукції в дійсності властиві свої осібні
закони населення, що мають історичне значення. Абстрактний
закон населення існує тільки для рослин і тварин, і то лише
остільки, оскільки вони не зазнають історичного впливу людини.

Але якщо надмірна робітнича людність є доконечний продукт
акумуляції, або розвитку багатства на капіталістичній основі,
то це перелюднення, з свого боку, стає підоймою капіталістичної
акумуляції і навіть умовою існування капіталістичного способу
продукції. Воно утворює резервну промислову армію, якою капітал
може порядкувати і яка абсолютно належить капіталові,
так, наче б він виростив її своїм власним коштом. Воно створює
для змінних потреб самозростання капіталу завжди готовий,
приступний для експлуатації людський матеріял, незалежно
від меж дійсного приросту людности. З акумуляцією й розвитком
продуктивної сили праці, що супроводить акумуляцію,
зростає сила раптового поширення капіталу не тільки через те,
\index{i}{0543}  %% посилання на сторінку оригінального видання
що зростають еластичність капіталу, який функціонує, і те абсолютне
багатство, що з нього капітал становить лише деяку
еластичну частину, не тільки через те, що кредит при кожній
особливій принаді, одразу ж віддає надзвичайну частину цього
багатства як додатковий капітал до розпорядження продукції,
— технічні умови самого процесу продукції, машини, засоби
транспорту й~\abbr{т. ін.}, в якнайбільшому маштабі уможливлюють
якнайшвидше перетворення додаткового продукту на
додаткові засоби продукції. Маса суспільного багатства, що зростає
з проґресом акумуляції, й що її можна перетворити на додатковий
капітал, несамовито рине в старі галузі продукції, що
їхній ринок раптом поширюється, або в нововідкривані галузі»
як ось залізниці й~\abbr{т. ін.}, що потреба на них випливає з розвитку
старих галузей продукції. В усіх таких випадках треба, щоб була
можливість раптом і без скорочення маштабу продукції в інших
сферах кидати великі маси людей на вирішальні пункти. Ці
маси постачає перелюднення. Характеристичний життьовий шлях
сучасної промисловости, ця форма перериваного невеликими
коливаннями десятирічного циклу періодів середнього оживлення,
продукції під високим тисненням, кризи й застою, ґрунтується
на постійному творенні, більшому або меншому поглиненні й
новоутворенні промислової резервної армії, або перелюднення.
З свого боку, мінливість фаз промислового циклу збільшує
людський матеріял для перелюднення і стає одним з найенерґійніших
факторів репродукції перелюднення.


\index{i}{0544}  %% посилання на сторінку оригінального видання
Цей своєрідний життьовий шлях сучасної індустрії, якого ми не
знаходимо ні за однієї з попередніх епох людства, був також
неможливий і в період дитинства капіталістичної продукції. Склад
капіталу тоді змінювався лише дуже повільно. Отже, його акумуляції
відповідало в цілому пропорційне зростання попиту на
працю. Хоч і який повільний був проґрес акумуляції капіталу
порівняно з сучасною епохою, але й він наражався на природні
межі приступної для експлуатації робітничої людности, межі, що
їх можна було усунути лише насильницькими засобами, про які
згадаємо згодом. Раптове й стрибкувате поширення маштабу
продукції є передумова його раптового скорочення; останнє
знову викликає перше, але перше неможливе без людського
матеріялу, що ним можна порядкувати, воно неможливе без
збільшення числа робітників, незалежного від абсолютного зростання
людности. Це збільшення створюється тим простим процесом,
що постійно «звільняє» частину робітників, за допомогою
метод, які зменшують число занятих робітників порівняно з
вирослою продукцією. Отже, ціла форма руху сучасної промисловости
виростає з постійного перетворювання певної частини
робітничої людности на незаняті або напівзаняті руки. Поверховість
політичної економії виявляється, між іншим, у тому, що
поширення і скорочення кредиту, простий симптом періодичних
змін у промисловому циклі, вона вважає за причини цих періодичних
змін. Як небесні тіла, скоро їх кинуто в певний рух,
знову й знову повторюють його, цілком так само й суспільна
продукція, скоро її кинуто в той рух навперемінного поширення
і скорочення, знову й знову повторює цей рух. Наслідки стають
із свого боку причинами, а навперемінні фази цілого процесу,
що постійно репродукує свої власні умови, набирають форми періодичности.
[Лише з того часу, як машинова продукція глибоко
вкоренилася і почала справляти переважний вплив на всю національну
промисловість; коли завдяки їй зовнішня торговля
почала переважати внутрішню, коли світовий ринок поступінно
захопив собі широкі простори в Новому Світі, Азії і Австралії;
коли, нарешті, промислові нації, що вступили між собою в конкуренцію,
стали досить численними, — лише з цього часу починаються
періодичні цикли, що їхні послідовні фази охоплюють
ряд років і що завжди ведуть до загальної кризи, якою закінчується
один цикл і починається другий. До цього часу період
тривання цих циклів становив 10—11 років, але немає ніяких
підстав вважати це число за стале. Навпаки, з розвинутих нами
законів капіталістичної продукції треба зробити той висновок,
що це число є змінне і що період тривання циклів буде поступінно
зменшуватися»].\footnote*{
Заведене у прямі дужки ми беремо з французького видання. Ред.
}

Скоро тільки періодичність промислових фаз укорінюється,
то навіть політична економія починає розуміти, що продукція
відносної надмірної людности, тобто людности, надмірної проти
\parbreak{}  %% абзац продовжується на наступній сторінці

\parcont{}  %% абзац починається на попередній сторінці
\index{i}{0545}  %% посилання на сторінку оригінального видання
середніх потреб самозростання капіталу, є умова існування сучасної
промисловости.

«Припустімо, — каже Г. Мерівел, раніш професор політичної
економії в Оксфорді, а потім урядовець англійського міністерства
колоній, — припустімо, що нація з нагоди якоїсь кризи напружить
свої сили, щоб за допомогою еміґрації позбутися кількох сот
тисяч зайвих бідних. Який був би з цього наслідок? Такий, що
при першому ж відновленні попиту на працю була б недостача
робітників. Хоч і як швидко відбуватиметься репродукція людей,
вона в усякому разі потребує для заміни дорослих робітників
переміжну часу однієї ґенерації. Але зиски наших фабрикантів
залежать переважно від спроможности використовувати сприятливий
момент жвавого попиту й таким чином відшкодовувати
себе за часи застою. Цю спроможність фабрикантам забезпечує
тільки панування над машинами й ручною працею. Для них
повинні знайтись вільні руки; вони повинні бути здібні в разі
потреби дужче напружувати або зменшувати активність своїх
операцій відповідно до стану ринку, бо інакше вони ніяк не зможуть
серед шаленої конкуренції втримати ту перевагу, на якій
основано багатство цієї країни».\footnote{
\emph{Н. Merivale}: «Lectures on Colonization and Colonies», London 1841
and 1842, vol. I, p. 146.
} Навіть Малтуз у перелюдненні,
яке він з свого обмеженого погляду пояснює абсолютним
надмірним приростом робітничої людности, а не тим, що вона
стає відносно надмірною, визнає доконечність для сучасної промисловости.
Він каже: «Мудрі звички щодо шлюбу, доведені
до певної височини серед робітничої кляси якоїсь країни, яка
залежить головним чином від мануфактури й торговлі, були б
для цієї країни шкідливі\dots{} Відповідно до самої природи людности,
приріст робітників не може бути поданий на ринок у наслідок
особливого попиту раніше, ніж мине 16 або 18 років, а перетворення
доходу на капітал через заощадження може відбуватися
куди швидше; країні завжди загрожує, що її робочий фонд зростатиме
швидше, ніж людність».\footnote{
«Prudential habits with regard to marriage carried to a considerable
extent among the labouring class of a country mainly depending upon manufactures
and commerce might injure it\dots{} From the natute of a population,
an increase of labourers cannot be brought into market, in consequence
of a particular demand, till after the lapse of 16 or-18 years, and the conversion
of revenue into capital, by saving, may take place much more rapidly;
a country is always liable to an increase in the quantity of the funds
for the maintenance of labour faster than the increase of population»).
(\emph{Malthus}: «Principles of Political Economy», p. 254, 319, 320). У цій праці
Малтуз відкриває, нарешті, за допомогою Сісмонді, прегарну трійцю капіталістичної
продукції: перепродукцію — перелюднення — переспоживання,
три справді любісінькі почвари (three very delicate monsters, indeed)!
Порівн. \emph{F. Engels}: «Umrisse zu einer Kritik der Nationalökonomie» in
Deutsch-Französische Jahrbücher, herausgegeben von Arnold Rüge
und Karl Marx, Paris 1844, S. 107 ff.
} Оголосивши таким чином
постійну продукцію відносного перелюднення робітників доконечністю
капіталістичної акумуляції, політична економія цілком
\index{i}{0546}  %% посилання на сторінку оригінального видання
на кшталт старої діви, вкладає в уста свого «прегарного
ідеалу», капіталіста, такі слова, звернені до «зайвих» робітників,
викинутих на брук додатковим капіталом, що вони сами його
створили: «Ми, фабриканти, збільшуючи капітал, з якого ви
мусите жити, робимо для вас усе, що можемо; а ви мусите зробити
решту, пристосовуючи свою чисельність до засобів існування».\footnote{
\emph{Harriet Martineau}: «The Manchester Strike», 1842, p. 101.
}

Для капіталістичної продукції ні в якому разі недосить тієї
кількости вільної робочої сили, що її дає природний приріст
людности. Для свого вільного розвитку вона потребує промислової
резервної армії, незалежної від цієї природної межі.

Досі ми припускали, що збільшення або зменшення змінного
капіталу точно відповідає збільшенню або зменшенню числа
занятих робітників.

Однак і за незмінного або навіть зменшеного числа робітників
змінний капітал, що панує над ними, зростає, якщо індивідуальний
робітник дає більше праці і в наслідок цього зростає
його заробітна плата, хоч ціна праці лишається незмінна, а то
навіть падає, тільки повільніше, ніж зростає маса праці. Тоді
приріст змінного капіталу стає показником більшої кількости
праці, але не більшої кількости занятих робітників. Кожний
капіталіст має абсолютний інтерес у тому, щоб видушити певну
кількість праці з меншого, а не з більшого числа робітників,
хоча б останнє коштувало так само дешево, а то й дешевше.
В останньому випадку видатки на сталий капітал зростають
пропорційно до маси праці, пущеної в рух, у першому випадку
вони зростають далеко повільніше. Що більший маштаб продукції,
то вирішальніший є цей мотив. Його вага зростає з акумуляцією
капіталу.

Ми бачили, що розвиток капіталістичного способу продукції
і продуктивної сили праці — одночасно причина й наслідок
акумуляції — дають капіталістові спроможність, за однакової витрати
змінного капіталу, через екстенсивнішу або інтенсивнішу
експлуатацію індивідуальних робочих сил пускати в рух більше
праці. Далі ми бачили, що він за ту саму капітальну вартість
купує більше робочої сили, щораз більше витискуючи навчених
робітників менш навченими, дозрілих робітників — недозрілими,
чоловіків — жінками, дорослих — підлітками й дітьми.

Отже, з проґресом акумуляції більший змінний капітал, з
одного боку, пускає в рух більше праці, не наймаючи більшого
числа робітників, з другого боку, змінний капітал тієї самої
величини пускає в рух більше праці за тієї самої маси робочої
сили і, нарешті, витискуючи робочі сили вищої якости, пускає
в рух більше робочих сил нижчої якости.

Тому продукція відносного перелюднення або звільнення
робітників іде ще швидше, ніж технічний переворот процесу
продукції, і без того прискорюваний проґресом акумуляції, і
\parbreak{}  %% абзац продовжується на наступній сторінці

\input{i/_0547.tex}
\parcont{}  %% абзац починається на попередній сторінці 
\index{i}{0548}  %% посилання на сторінку оригінального видання 
не вистачило б для провадження національної продукції в її
теперішньому маштабі. Велика більшість «непродуктивних»
тепер робітників мусила б перетворитись на «продуктивних».

Взагалі і в цілому загальні коливання заробітної плати реґулюються
виключно поширенням і зменшенням промислової резервної
армії, що відповідають зміні періодів промислового циклу.
Отже, вони визначаються не рухом абсолютного числа робітничої
людности, а тим змінним відношенням, що в ньому робітнича кляса
розпадається на активну й резервну армію, тобто збільшенням
та зменшенням відносних розмірів перелюднення, ступенем, у
якому перелюднення то поглинається, то знову звільняється.
Для сучасної промисловости з її десятилітнім циклом і його періодичними
фазами, які, крім того, з проґресом акумуляції перериваються
неправильними коливаннями, що чимраз швидше йдуть
одне по одному, — це справді був би прегарний закон, що робив
би рух капіталу залежним від абсолютного руху маси людности
замість, навпаки, реґулювати попит і подання праці поширенням
і скороченням капіталу, тобто відповідно до його кожноразових
потреб самозростання, отже, реґулювати таким чином,
що ринок праці видається то відносно неповним, у наслідок
поширення капіталу, то знову переповненим, у наслідок скорочення
капіталу. Однак така є догма політичної економії. За цією
догмою в наслідок акумуляції капіталу зростає заробітна плата.
Підвищена заробітна плата стимулює швидше розмноження робітничої
людности, і це розмноження триває доти, доки ринок
праці переповнюється, отже, триває доти, доки капітал стане
відносно недостатнім проти подання праці. Заробітна плата падає,
і тепер ми маємо зворотний бік медалі. В наслідок падання
заробітної плати робітнича людність поволі рідшає, так що проти
неї капітал знову стає надмірний, абож, як це пояснюють інші,
падання заробітної плати й відповідне підвищення експлуатації
робітника знову прискорює акумуляцію, тимчасом як нижча
заробітна плата одночасно затримує зростання робітничої кляси.
Таким чином знову постає таке відношення, коли подання праці
нижче від попиту на працю, заробітна плата зростає й т. д.
Яка прегарна метода руху для розвинутої капіталістичної продукції!
Поки в наслідок підвищення заробітної плати міг би
настати будь-який позитивний зріст дійсно працездатної людности,
декілька разів минув би той час, що протягом його треба
провести промислову кампанію та вирішити справу в бою.

Між 1849 і 1859 рр., одночасно з падінням цін на збіжжя,
практично сталось лише номінальне підвищення заробітної
плати в англійських рільничих округах: наприклад, у Wiltshire
тижнева плата зросла з 7 до 8 шилінґів, у Dorsetshire — з 7 або
8 шилінґів до 9 шилінґів і т. ін. Це був наслідок надзвичайного
відпливу надмірної рільничої людности, спричиненого потребами
війни, масовим поширенням будування залізниць, фабрик, гірничих
підприємств і т. ін. Що нижча заробітна плата, то вищі процентові
числа, що в них виражається всяке, хоч би й яке незначне
\parbreak{}  %% абзац продовжується на наступній сторінці

\input{i/_0549.tex}
\input{i/_0550.tex}
\parcont{}  %% абзац починається на попередній сторінці
\index{i}{0551}  %% посилання на сторінку оригінального видання
залежить від тиску відносного перелюднення; отже, скоро тільки вони намагаються за допомогою
тред-юньйонів і~\abbr{т. ін.} організувати пляномірну взаємодію занятих і незанятих робітників, щоб знищити
або послабити руйнаційні для їхньої кляси наслідки того природного закону капіталістичної
продукції, — так капітал і його сикофант, політико-економ, зчиняють галас про порушення «вічного» і, так би
мовити, «святого» закону попиту й подання. Аджеж усякий зв’язок між занятими й незанятими
порушує, мовляв, «чисте» діяння того закону. А, з другого боку, скоро тільки, наприклад, у колоніях,
несприятливі обставини перешкоджають створенню промислової резервної армії, а разом з нею і
абсолютної залежности робітничої кляси від кляси капіталістів, то капітал і його банальний
Санчо-Панчо\footnote*{
дієва особа з роману Сервантеса «Дон-Кіхот». \emph{Ред.}
} підіймають
бунт проти «святого» закону попиту й подання і намагаються приборкати його примусовими засобами.

\subsection{Різні форми існування відносного перелюднення. Загальний закон капіталістичної акумуляції}

Відносне перелюднення існує в усяких можливих відтінках. Кожний робітник належить до нього протягом
того часу, коли він напівзанятий або зовсім незанятий. Якщо залишити осторонь ті великі форми
перелюднення, що періодично повторюються,
форми, що їх надає перелюдненню зміна фаз промислового циклу, так що воно буває то гостре, підчас
криз, то хронічне, підчас застою, — то воно постійно має три форми: текучу, лятентну і застійну.

В центрах сучасної промисловости — фабриках, мануфактурах, металюрґійних заводах, копальнях і~\abbr{т. д.}
— робітників то відштовхують, то знову в більшому розмірі притягують, так що взагалі і в цілому
число занятих більшає, хоч і в щораз
меншій пропорції порівняно з маштабом продукції. Перелюднення існує тут у текучій формі.

Так на фабриках у власному значенні, як і по всіх великих майстернях, де машини відіграють певну
ролю або, принаймні, заведено сучасний поділ праці, потрібна маса робітників чоловічої статі, що не
дійшли ще юнацького віку. Коли ці робітники доходять цього віку, то тільки дуже мале число з них
лишається в тих самих галузях промисловости, а більшість із них реґулярно звільняють. Вони
становлять той елемент текучого перелюднення, що зростає разом із зростанням розмірів промисловости.
Частина з них еміґрує й фактично лише мандрує слідком за тим капіталом, що еміґрує. Один із
наслідків цього є те, що жіноча людність зростає швидше, ніж чоловіча, як про це свідчить Англія. Та
обставина, що природний приріст робітничої маси не насичує
потреб акумуляції капіталу і проте разом з тим їх перевищує, —
\parbreak{}  %% абзац продовжується на наступній сторінці

\parcont{}  %% абзац починається на попередній сторінці
\index{i}{0552}  %% посилання на сторінку оригінального видання
є суперечність самого руху капіталу. Він потребує більших мас робітників молодшого віку, менших мас
— дорослого. Ця суперечність не більш кричуща за ту другу, що нарікають на брак робочих рук у той
самий час, коли багато тисяч викинуто на
брук через те, що поділ праці приковує їх до якоїсь певної галузі продукції\footnote{
Тимчасом як протягом останнього півріччя 1866~\abbr{р.} в Лондоні лишилося без роботи \num{80.000}--\num{90.000}
робітників, у фабричному звіті про це саме півріччя читаємо: «Здається, не зовсім правда, що попит
створює подання саме в ту хвилину, коли це потрібно. Щодо праці справа стояла інакше, бо протягом
останнього року багато машин не працювало через брак рук». («It does not appear absolutely true to
say that demand
will always produce supply just at the moment when it is needed. It has not done so with labour, for
much machinery had been idle last year for want of hands»). («Report of Insp. of Fact. for 31st
October 1866», p. 81).
}. До того ж, капітал
споживає робочу силу так швидко, що здебільшого робітник середнього віку є вже більше або менше
виснажений. Він попадає в ряди зайвих або його витискують
із вищого щабля на нижчий щабель. Саме в робітників великої промисловости ми натрапляємо на
найкоротший протяг життя. «Д-р. Лі, санітарний урядовець Менчестеру, сконстатував, що в тому місті
пересічний протяг життя заможної кляси
38 років, а робітничої кляси — лише 17 років. У Ліверпулі він становить для першої кляси 35 років,
для другої — 15. Отже, з цього випливає, що упривілейована кляса має асиґнату на життя (have a lease
of life) понад удвоє більшу, ніж її менш щасливі співгромадяни»\footnoteA{
Промова, що її виголосив на відкритті санітарної конференції в Бермінґемі 15 січня 1875~\abbr{р.} Дж.~Чемберлен, тодішній мер міста, теперішній (1883) міністер торговлі.
}. За цих обставин для абсолютного
зростання цієї частини пролетаріяту потрібна така форма, при якій чисельність її зростала б, не
зважаючи на швидке виснажування її елементів. Отже, потрібна швидка зміна поколінь робітників.
(Цей закон не має сили для решти кляс людности).
Цю суспільну потребу задовольняється ранніми шлюбами, — неминучий наслідок відносин, серед яких
живуть робітники великої промисловости, — і тією премією, яку дає експлуатація дітей робітників за
продукцію їх.

Скоро тільки капіталістична продукція опановує рільництво, або в міру того, як вона опановує
рільництво, попит на сільську робітничу людність абсолютно меншає з акумуляцією капіталу, що тут
функціонує, так що відштовхування робітничої людности
тут не доповнюється, як у нерільничій промисловості, більшим притяганням. Тому частина сільської
людности завжди готова перейти в ряди міського або мануфактурного пролетаріяту і лише вичікує
сприятливих умов для цього перетворення. (Слова мануфактура тут уживається в розумінні всякої
нерільничої
промисловости)\footnote{
За переписом 1861~\abbr{р.} в Англії і Велзі налічувалося «781 місто з \num{10.960.998} жителями, тимчасом як
по селах і сільських парафіях налічувалося лише \num{9.105.226} жителів\dots{} У перепису 1851~\abbr{р.} фігурувало
580 міст, що їх людність приблизно дорівнювала людності прилеглих до них
сільських округах. Але тимчасом як у сільських округах людність протягом наступних десятьох років
зросла лише на півмільйона, в 580 містах вона зросла на \num{1.554.067}. Приріст людности по селах
становить 6,5\%, по містах — 17,3\%. Ріжниця в нормі приросту є наслідок переселення з сел до міст.
Три чверті загального приросту людности припадає на міста». («Census etc.», vol. III, p. 11, 12).
}. Отже, це джерело відносного перелюднення
\index{i}{0553}  %% посилання на сторінку оригінального видання
б’є безперервно. Але постійний приплив до міст має своєю передумовою постійне лятентне перелюднення
в самих селах, що його розміри стають помітні лише тоді, коли вивідні канали відкриваються винятково
широко. Тим то плату сільського робітника
знижують до мінімуму і він завжди стоїть однією ногою в болоті павперизму.

Третя категорія відносного перелюднення, застійна, становить частину активної робітничої армії, але
разом із тим надзвичайна нереґулярність її занять дає капіталові невичерпний резервуар вільної
робочої сили. Її життєве становище падає нижче за
пересічний нормальний рівень робітничої кляси, і саме це робить її широкою основою осібних галузей
капіталістичної експлуатації. Її характеризують максимум робочого часу й мінімум заробітної плати.
Ми вже вивчили під рубрикою домашньої праці її головну форму. Вона рекрутується постійно із зайвих
робітників великої промисловости й рільництва, і особливо також з робітників галузей промисловости,
що гинуть, тих галузей, де ремісничу продукцію перемагає мануфактурна, мануфактурну — машинова. Її
розміри більшають у міру того, як з розмірами та енерґією акумуляції проґресує «творення зайвих»
робітників. Але разом з тим вона становить той елемент робітничої кляси, який сам себе репродукує й
увіковічнює і який бере порівняно більшу участь у загальному зростанні робітничої кляси, ніж решта
її елементів. Справді, не тільки число народжень і випадків смерти, але й абсолютна величина родин є
зворотно пропорційна до височини заробітної плати, отже, і до маси засобів існування, що ними
порядкують різні категорії робітників. Цей закон капіталістичного суспільства звучав би якимсь
безглуздям серед дикунів, а то й навіть серед цивілізованих колоністів. Він нагадує нам про масову
репродукцію індивідуально малосильних і жорстоко переслідуваних видів тварин\footnote{
«Злидні\dots{} здається\dots{} сприяють розмножуванню» («Poverty\dots{} seems\dots{} favourable to generation»).
(\emph{A.~Smith}: «Wealth of Nations», b. I, ch. 8, p. 195). На думку ґалянтного й дотепного абата Ґаліяні
це навіть надзвичайно мудра установа божа: «Господь зробив так, що людей, які виконують дуже корисну
роботу, родиться найбільше» («Iddio fa che
gli uomini che esercitano mestieri di prima utilità nasconoabbondantemente»). (\emph{Galiani}: «Della
Moneta», vol. III збірки Custodi «Scrittori Clas sici Italiani di Economia Politica». Parte Moderna.
Milano 1801, p. 78). «Злидні аж до крайніх меж голоду й епідемій не гальмують зросту людности, а
мають тенденцію абільшувати її» («Misery, up to the extreme point of famine and pestilence, instead
of checking, tends to increase population»).
(\emph{S.~Laing}: «National Distress», 1844, p. 69). Зілюструвавши це статистичними даними, Лен каже далі:
«Коли б усі жили в достатках, то земля швидко лишилася б без людей» («If the people were all in easy
circumstances, the world would soon be depopulated»).
}.


\index{i}{0554}  %% посилання на сторінку оригінального видання 
Насамкінець, найнижча верства відносного перелюднення
перебуває у сфері павперизму. Залишаючи осторонь волоцюг,
злочинців, проституток, коротко — власне люмпенпролетаріят,
ця верства суспільства складається з трьох категорій. По-перше,
працездатні. Досить лише поверхово переглянути статистику
англійського павперизму, і ми побачимо, що маса його більшає
з кожною кризою й меншає з кожним оживленням справ. По-друге,
сироти й діти павперів. Це кандидати промислової резервної
армії; в періоди великого розцвіту, як, наприклад, 1860 р., вони
швидко й масами заповнюють ряди активної робітничої армії.
Потрете, занепалі, збіднілі, непрацездатні. Це саме ті індивіди,
що гинуть од своєї нерухливости, спричиненої поділом праці,
ті, що переживають нормальний вік робітника, нарешті, жертви
промисловости, що їхнє число зростає з поширенням небезпечних
машин, копалень, хемічних фабрик і т. д. — каліки, хорі,
вдови тощо. Павперизм є дім інвалідів активної робітничої армії
і баляст промислової резервної армії. Утворення відносного
перелюднення включає й утворення павперизму, доконечність
першого включає й доконечність другого, разом із відносним
перелюдненням павперизм становить умову існування капіталістичної
продукції й розвитку багатства. Він належить до faux
frais\footnote*{
— непродуктивних витрат. Ред.
} капіталістичної продукції, що їх капітал уміє однак здебільша
звалити з себе самого на плечі робітничої кляси і дрібної
середньої кляси.

Що більше суспільне багатство, капітал, який функціонує,
розміри й енерґія його зростання, отже і абсолютна величина пролетаріату
і продуктивна сила його праці, то більша резервна промислова
армія. Робоча сила, що нею можна порядкувати, розвивається
через ті самі причини, що й експансивна сила капіталу.
Отже, відносна величина резервної промислової армії зростає
разом з потенціями багатства. Але що більша ця резервна армія
проти активної робітничої армії, то масовіше є стале перелюднення,
що його злидні стоять у зворотній пропорції до мук його
праці. Нарешті, що більша жебрацька верства робітничої кляси
й промислова резервна армія, то більший офіціяльний павперизм.
Це є абсолютний, загальний закон капіталістичної акумуляції.
У своєму здійсненні він, як і всі інші закони, модифікується
різноманітними обставинами, що аналіза їх сюди не
належить.

Можна зрозуміти безглуздість тієї економічної премудрости,
яка проповідує робітникам пристосовувати свою чисельність до
потреб самозростання капіталу. Механізм капіталістичної продукції
й акумуляції постійно пристосовує цю чисельність до цих
потреб самозростання. Перше слово цього пристосування є утворення
відносного перелюднення або промислової резервної армії,
а останнє слово — це злидні щораз більших верств активної
робітничої армії й баляст павперизму.

\index{i}{0555}  %% посилання на сторінку оригінального видання
Закон, що за ним щораз більшу масу засобів продукції, у
наслідок проґресу продуктивности суспільної праці, можна пускати
в рух із щораз меншою витратою людської сили, — цей
закон на капіталістичній основі, де не робітник уживає засобів
праці, а засоби праці вживають робітника, виражається в тому,
що, чим вища продуктивна сила праці, тим більший тиск робітників
на засоби їхньої роботи, отже, тим непевніша умова їхнього
існування: продаж власної сили для збільшування чужого багатства
або для самозростання капіталу. Отже, швидше зростання
засобів продукції і продуктивности праці, швидше, ніж
зростання продуктивної людности, виражається за капіталізму,
навпаки, в тому, що робітнича людність завжди зростає швидше,
ніж потреби самозростання капіталу.

У четвертому відділі, аналізуючи продукцію відносної додаткової
вартости, ми бачили, що за капіталістичної системи всі
методи підвищення суспільної продуктивної сили праці відбуваються
коштом індивідуального робітника; всі засоби для розвитку
продукції перетворюються на засоби поневолення й експлуатації
продуцента, калічать робітника, роблячи з нього несповналюдину,
принижують його до стану додатку до машини, з муками
його праці знищують і її зміст, відчужують від робітника духовні
сили процесу праці в тій самій мірі, в якій наука сполучається
з цим останнім як самостійна сила; вони спотворюють умови, серед
яких працює робітник, підбивають його підчас процесу праці
під якнайдріб’язковішу, ненависну деспотію, ціле його життя
перетворюють на робочий час, його жінку й дітей кидають під
джеґґернавтові колеса капіталу. Але всі методи продукції додаткової
вартости є разом з тим методи акумуляції, і всяке поширення
акумуляції стає, навпаки, засобом розвитку цих метод.
Звідси випливає, що в міру того, як акумулюється капітал,
становище робітника мусить гіршати, хоч яка б була його плата —
висока чи низька. Нарешті, той закон, що завжди тримає відносне
перелюднення, або промислову резервну армію, в рівновазі
з розмірами й енерґією акумуляції, приковує робітника до капіталу
міцніше, аніж молот Ґефеста прикував Прометея до скелі.
Цей закон зумовлює акумуляцію злиднів, що відповідає акумуляції
капіталу. Отже, акумуляція багатства на одному полюсі
є разом з тим акумуляція злиднів, мук праці, рабства, неуцтва,
здичавіння й моральної деґрадації на протилежному полюсі,
тобто на боці тієї кляси, що продукує свій власний продукт як
капітал.

Цей антагоністичний характер капіталістичної акумуляції\footnote{
«З дня на день стає ясніше, що відносини продукції, в яких
рухається буржуазія, мають не однорідний, простий характер, а двоїстий
характер; що в тій самій пропорції, в якій продукується багатство,
продукуються і злидні; що в тій самій пропорції в якій відбувається розвиток
продуктивних сил, розвивається й сила поневолення; що ці відносини
продукують буржуазне багатство, тобто багатство буржузної кляси, лише
постійно знищуючи багатство поодиноких членів цієї кляси і створюючи
пролетаріят, що дедалі більше зростає» («De jour en jour il devient donc
plus clair que les rappotrs de production dans lesquels se meut la bourgeoisie
n’ont pas un caractère un, un caractère simple, mais un caractère de duplicité;
que dans les mêmes rapports dans lesquels se produit la richesse, la
misère se produit aussi: que dans les mêmes rapports dans lesquels il y a
développement des forces productives, il y a une force productive de répression;
que ces rapports ne produisent la richesse bourgeoise, c’est à dire
la richesse de la classe bourgeoise, qu’en anéantissant continuellement la
richesse des membres intégrants de cette classe et en produisant un prolétariat
toujours croissant». (\emph{K.~Marx}: «Misère de la Philosophie», p. 116.
— \emph{K.~Маркс}: «Злиденність філософії», Партвидав 1932, стор. 110).
}
зазначали в різних формах політико-економи, хоч вони почасти
\index{i}{0556}  %% посилання на сторінку оригінального видання
сплутують з ним аналогічні, але посутньо відмінні явища передкапіталістичних
способів продукції.

Венеціанський чернець Ортес, один із найбільших письменників\dash{}економістів
XVIII віку, розглядає антагонізм капіталістичної
продукції як загальний природний закон суспільного
багатства. «Економічне добро й економічне зло в якійсь нації
завжди зрівноважуються (il bene ed il male economico in una
nazione sempre all’istessa misura), повнява дібр в одних є завжди
недостача дібр в інших (la copia dei beni in alcuni sempre eguale
alla mancanza di esse is altri). Велике багатство небагатьох завжди
супроводиться абсолютним грабуванням доконечного в далеко
більшого числа інших. Багатство якоїсь нації відповідає її людності,
а злидні її відповідають її багатству. Працьовитість одних
вимушує ледарство інших. Бідні й нероби є неминучий продукт
багатих і працьовитих» і~\abbr{т. д.}\footnote{
\emph{G.~Ortes}: «Delia Economia Nazionale libri sei», 1777, y Custodi.
Parte Moderna, vol. XXI, p. 6, 9, 22, 25 etc. Ортес каже (там же, стор. 32):
«Замість вигадувати нікчемні системи, як зробити народи щасливими,
я хочу обмежитися на розсліді причин їхнього нещастя» («In luoco di
progettar sistemi inutili per la felicità de popoli, mi limiterô a investigare
la ragione delia loro infelicità»).
} Через якихось десять років
після Ортеса англікансько-протестантський піп Тавнсенд цілком
грубим способом розхвалював злидні як доконечну умову
багатства. «Законодатний примус до праці є пов’язаний із чималими
труднощами, насильством і шумом, тимчасом як голод не
тільки є мирний, мовчазний, безупинний натиск, але, являючи
собою якнайприроднішу спонуку до промисловости і праці, викликає
якнайдужче напруження». Отже, все сходить на те, щоб
для робітничої кляси зробити голод перманентним, і про це, за
Тавнсендом, дбає принцип залюднення, який є особливо активний
серед бідних. «Це є, здається, природний закон, що бідні
до певної міри легкодумні (improvident) (а саме так легкодумні,
що приходять на світ без золотої ложки в роті), так що завжди
знаходяться люди (that there always may be some) для виконання
найнижчих, найбрудніших і найпаскудніших функцій у суспільстві.
Запас людського щастя (the fund of human happiness) через
те дуже збільшується, делікатніші люди (the more delicate) увільнені
від мук праці й можуть без перешкод іти за своїм вищим
покликанням і т. д\dots{} Закон про бідних має тенденцію зруйнувати
гармонію і красу, симетрію й порядок цієї системи, що її
\parbreak{}  %% абзац продовжується на наступній сторінці

\input{i/_0557.tex}
\parcont{}  %% абзац починається на попередній сторінці
\index{i}{0558}  %% посилання на сторінку оригінального видання
плати»]\footnote*{
\emph{Cherbuliez}: «Riche ou Pauvre», p. 146. — Заведене у прямі дужки
ми беремо з французького видання. \emph{Ред.}
}. — Нарешті, Детю де Трасі, буржуазний доктринер з
холодною, як у риби, кров’ю, брутально заявляє: «Бідні нації —
це ті, де народові добре, а багаті нації — де народ звичайно бідний»\footnote{
\emph{Destutt de Trasy}: «Traité de la Volonté et de ses effets», Paris
1826, p. 231: «Les nations pauvres, c’est là où le peuple est à son aise: et
les nations riches, c’est là où il est ordinairement pauvre».
}.

\subsection{Ілюстрація загального закону капіталістичної акумуляції}

\subsubsection{Англія 1846--1866~\abbr{рр.}}

Жоден із періодів сучасного суспільства не є такий сприятливий
для вивчення капіталістичної акумуляції, як період останніх
двадцяти років. Здається, наче він найшов торбу фортуни.
Але з-поміж усіх країн клясичний приклад дає знову таки
Англія, бо вона посідає перше місце на світовому ринку,
тільки в ній цілковито розвинувся капіталістичний спосіб продукції,
і, нарешті, заведення тисячолітнього царства вільної
торговлі від 1846~\abbr{р.}, відібрало від вульґарної політичноїї економії
її останній притулок. Про титанічний проґрес продукції, що зумовив
знову таки значну перевагу останньої половини двадцятирічного
періоду над першою, вже досить зазначено в четвертому
відділі.

Хоч абсолютне зростання англійської людности за останнє півстоліття
було дуже велике, однак відносне зростання, або норма
приросту, безперервно падало, як показує ця таблиця, запозичена
з офіціяльного перепису.

Щорічний процентовий приріст людности Англії й Велзу
за десятиріччями:

\begin{table}[H]
\centering
\noindent\begin{tabular}{lr}
   1811\textendash{}1821 & 1,533\% \\
   1821\textendash{}1831 & 1,446\% \\
   1831\textendash{}1841 & 1,326\% \\
   1841\textendash{}1851 & 1,216\% \\
   1851\textendash{}1861 & 1,141\% \\
\end{tabular}
\end{table}
\noindent{}Розгляньмо тепер, з другого боку, зростання багатства. Найпевнішу
точку опори дає тут рух зисків, земельних рент і~\abbr{т. ін.},
що підлягають прибутковому оподаткуванню. Приріст зисків, що
підпадають оподаткуванню (фармерів і деяких інших рубрик сюди
не включено), становив для Великобританії від 1853 до 1864~\abbr{р.}
50,47\% (або 4,58\% пересічно за рік)\footnote{
«Tenth Report of the Commissioners of H. M’s. Inland Revenue».
London 1866, p. 38.
}, приріст людности протягом
того самого періоду — приблизно 12\%. Збільшення земельних
рент, що підпадають оподаткуванню (сюди належать
будинки, залізниці, копальні, рибальство й~\abbr{т. ін.}), становило
від 1853 до 1864~\abbr{р.} 38\%, або 3\sfrac{5}{12}\% річно, при чому найдужче
збільшення припадало на такі рубрики:

\index{i}{0559}  %% посилання на сторінку оригінального видання
\begin{table}[H]
\centering

\noindent\begin{tabular}{lrr}
& \makecell[r]{Приріст річного\\ доходу 1864~\abbr{р.}\\ проти 1853~\abbr{р.}} &
 \makecell[r]{Збільшення \\ за рік} \\

Від будинків\dotfill{} & 38,60\% & 3,50\% \\

\ditto{Від} каменярень\dotfill{} & 84,76\% & 7,70\% \\
\ditto{Від} копалень\dotfill{} & 68,85\% & 6,26\% \\
\ditto{Від} чавуноливарень\dotfill{} & 39,92\% & 3,63\% \\
\ditto{Від} рибальства\dotfill{} & 57,37\% & 5,21\% \\
\ditto{Від} газівень\dotfill{} & \makebox[0pt][r]{1}26,02\% & \hang{r}{1}1,45\% \\
\ditto{Від} залізниць\dotfill{} & 83,29\% & 7,57\%\hang{l}{\footnote{Там же.}} \\
\end{tabular}
\end{table}
 
\noindent{}Якщо порівняти між собою щочотирирічки періоду 1853--1864~\abbr{рр.},
то ступінь збільшення доходів невпинно зростає. Приміром,
для доходів, що походять із зиску, він 1853--1857~\abbr{рр.}
становить 1,73\% на рік, 1857--1861~\abbr{рр.} — 2,74\% на рік і 1861--1864~\abbr{рр.}
9,30\% на рік. Загальна сума доходів, що підпадають прибутковому
оподаткуванню, становила в Об’єднаному королівстві
1856~\abbr{р.} \num{307.068.898}\pound{ фунтів стерлінґів}, 1859~\abbr{р.} — \num{328.127.416}\pound{ фунтів
стерлінґів}, 1862~\abbr{р.} — \num{351.745.241}\pound{ фунт стерлінґів}, 1863~\abbr{р.} —
\num{359.142.897}\pound{ фунтів стерлінґів}, 1864~\abbr{р.} — \num{362.462.279}\pound{ фунтів стерлінґів},
1865~\abbr{р.} — \num{385.530.020}\pound{ фунтів стерлінґів}\footnote{
Цих чисел для порівняння досить, але коли розглядати їх абсолютно,
то вони фалшиві, бо щорічно «затаюється» може більше ніж
100 мільйонів фунтів стерлінґів доходів. Нарікання Commissioners of Irland
Revenue на систематичне шахрайство, особливо з боку купців і промисловців,
повторюються в кожному їхньому звіті. Приміром, читаємо:
«Одно акційне товариство показало свій належний до оподаткування зиск
у \num{6.000}\pound{ фунтів стерлінґів}, таксатор визначив його у \num{88.000}\pound{ фунтів стерлінґів},
і, кінець-кінцем, податок виплачено з цієї суми. Друга компанія
показала зиск у \num{190.000}\pound{ фунтів стерлінґів} і мусила признатися, що
дійсна сума є \num{250.000}\pound{ фунтів стерлінґів}». (Там же, стор. 42).
}.

Акумуляцію капіталу одночасно супроводили його концентрація
й централізація. Хоч для Англії не існувало офіціяльної
рільничої статистики (в Ірляндії вона існує), проте 10 графств подали
її з власної волі. Тут виявився з неї такий результат, що від
1851 до 1861~\abbr{р.} число фарм, нижчих за 100 акрів, зменшилося з
\num{13.583} до \num{26.567}, отже \num{5.016} фарм сполучилося з більшими фармами\footnote{
«Census etc.», vol. III, p. 29. Твердження Джона Брайта, що 150
землевласникам належить половина англійської землі, а 12 — землевласникам
половина шотляндської, не збито.
}.  Від 1815 до 1825~\abbr{р.} з рухомого майна, що підпадало спадщинному
податкові, не було жодного понад 1 мільйон фунтів стерлінґів;
навпаки, від 1825 до 1835~\abbr{р.} їх було 8, від 1856 до червня
1859~\abbr{р.}, тобто за 4\sfrac{1}{2} роки, — 4\footnote{
«Fourth Report etc. of Inland Revenue», London 1860, p. 17.
}. Однак централізацію найкраще
можна побачити з короткої аналізи прибуткового податку
в рубриці D (зиски, за винятком фармерських і~\abbr{т. ін.}) за роки
1864 і 1865. Спочатку зауважу, що доходи з цього джерела,
які розміром не нижчі за 60\pound{ фунтів стерлінґів}, підпадають
income tax\footnote*{
— прибутковому податкові. \emph{Ред.}
}. Ці доходи, що підпадають оподаткуванню, становили
в Англії, Велзі й Шотляндії 1864~\abbr{р.} \num{95.844.222}\pound{ фунтів стерлінґів}
\index{i}{0560}  %% посилання на сторінку оригінального видання
і 1865~\abbr{р.} — \num{105.435.579}\pound{ фунтів стерлінґів}\footnote{
Це — чисті доходи, отже, після того, коли вже відлічено з них
певні суми, визначені законом.
}, число оподаткованих
осіб 1864~\abbr{р.} — \num{308.416} на загальну кількість людности в
\num{23.891.009}, 1865~\abbr{р.} — \num{332.431} особа на загальну кількість людности
в \num{24.127.003}. Про розподіл цих доходів за обидва роки
довідуємося з цієї таблиці:
\begin{table}[H]
\centering
\noindent\begin{tabular}{lrrrr}
\toprule
& \multicolumn{2}{c}{\makecell{Рік, що кінчається \\ 5 квітня 1864~\abbr{р.}}} & 
\multicolumn{2}{c}{\makecell{Рік, що кінчається \\ 5 квітня 1865~\abbr{р.}}} \\
\cmidrule(rl){2-3}
\cmidrule(rl){4-5}
& \makecell[r]{Доходи від \\ зиску,\pound{(фунтів стерлінґів)}}&
\makecell[r]{Число \\ осіб} &
\makecell[r]{Доходи від \\  зиску ,\pound{(фунтів стерлінґів)}} &
\makecell[r]{Число \\ осіб} \\
\midrule
Загальний дохід\dotfill{} & \num{95.844.222} & \num{308.416} & \hang{r}{1}\num{05.435.738} & \num{332.431} \\
З того\dotfill{} & \num{57.028.289}  & \num{23.334}  & \num{64.554.297}  & \num{24.265} \\
\ditto{З} \ditto{того}\dotfill{}& \num{36.415.225} & \num{3.619}  &  \num{42.535.576} &  \num{4.021} \\
\ditto{З} \ditto{того}\dotfill{} & \num{22.809.781} &  832  &  \num{27.555.313}  &     973 \\
\ditto{З} \ditto{того}\dotfill{} & \phantom{0}\num{8.744.762}  &  91   &   \num{11.077.238}  &     107 
\end{tabular}
\end{table}
\noindent{}1855~\abbr{р.} в Об’єднаному королівстві випродуковано \num{61.453.079}
тонн вугілля вартістю в \num{16.113.167}\pound{ фунтів стерлінґів}, 1864~\abbr{р.} —
\num{92.787.873} тонни вартістю в \num{23.197.968}\pound{ фунтів стерлінґів}; 1855~\abbr{р.} —
\num{3.218.154} тонни чавуну вартістю в \num{8.045.385}\pound{ фунтів стерлінґів},
1864~\abbr{р.} — \num{4.767.951} тонну вартістю в \num{11.919.877}\pound{ фунтів стерлінґів}.
1854~\abbr{р.} довжина залізниць, експлуатованих в Об’єднаному
королівстві, становила \num{8.054} милі з капіталовкладенням
у \num{286.068.794}\pound{ фунти стерлінґів}, 1864~\abbr{р.} довжина у милях становила
\num{12.789}, а вкладений капітал — \num{425.719.613}\pound{ фунтів стерлінґів}.
Загальний експорт і імпорт Об’єднаного королівства
становив 1854~\abbr{р.} \num{268.210.145}\pound{ фунтів стерлінґів}, 1865~\abbr{р.} —
\num{489.923.285}. Нижченаведена таблиця показує рух експорту:

\begin{table}[H]
\centering
\noindent\begin{tabular}{lr}

\emph{Роки} & \emph{\pound{Фунтів стерлінґів}} \\

1846\dotfill{} & \num{58.842.377} \\

1849\dotfill{} & \num{63.596.052} \\

1856\dotfill{} & \num{115.826.948} \\

1860\dotfill{} & \num{135.842.817} \\

1865\dotfill{} & \num{165.862.402} \\

1866\dotfill{} & \num{188.917.563}\hang{l}{\footnote{
В цей момент, березень 1867~\abbr{р.}, індійсько-китайський ринок
уже знову переповнений комісійними товарами брітанських бавовняних
фабрикантів. 1866~\abbr{р.} почалося зниження заробітної плати бавовняних
робітників на 5\%, в 1867~\abbr{р.} в наслідок подібних операцій стався страйк
\num{20.000} робітників у Престоні. [Це був пролог кризи, що вибухла одразу
після того. — \emph{Ф. Е.}].
}} \\
\end{tabular}
\end{table}

\noindent{}Після цих небагатьох даних стає зрозумілим тріюмфальний
крик генерального реєстратора брітанського народу: «Хоч і
як швидко зростала людність, вона не встигала за проґресом промисловости
й багатства»\footnote{
«Census etc.», там же, стор. 11.
}. А тепер вдаймося до безпосередніх
аґентів цієї промисловости або до продуцентів цього багатства,
до робітничої кляси. «Це одна з найсумніших характеристичних
\parbreak{}  %% абзац продовжується на наступній сторінці

\parcont{}  %% абзац починається на попередній сторінці
\index{i}{0561}  %% посилання на сторінку оригінального видання
рис соціяльного становища країни, — каже Ґледстон, — що одночасно
зі зменшенням споживної сили народу та збільшенням
статків і злиднів робітничої кляси відбувається постійна акумуляція
багатства у вищих клясах і постійний приріст капіталу».\footnote{
Ґледстон у Палаті громад 13 лютого 1843 р.: «It is one of the
most melancholy features in the social state of this country that we see,
beyond the possibility of denial, that while there is at this moment, a decrease
in the consuming powers of the people, an increase of the pressure
of privations and distress; there is at the same time a constant accumulation
of wealth in the upper classes, an increase in the luxuriousness of their
habits, and of their means of enjoyment». («Times», 14 Februar 1843. —
«Hansard», 13 Februar).
} Так говорив цей єлейний міністер у Палаті громад
13 лютого 1843 р. Двадцять років пізніше, 16 квітня 1863 р.,
подаючи на розгляд бюджет, він каже: «Від 1842 р. до 1852 р.
доходи цієї країни, що підпадають оподаткуванню, зросли на
6\%\dots{} За вісім років, від 1853 до 1861 р., вони збільшились, якщо
взяти за основу доходи 1853 р., на 20\%. Це такий дивовижний
факт, що він майже неймовірний\dots{} Це приголомшливе збільшення
багатства й сили\dots{} обмежується геть чисто на заможних
клясах, але\dots{} але воно мусить давати посередню користь і робітничій
людності, бо воно здешевлює предмети загального споживання,
— в той час, як багаті стали багатшими, бідні в усякому
разі стали менш бідними. Я не наважуся сказати, що крайності
бідности змінилися».\footnote{
«From 1842 to 1852 the taxable income of the country increased by
6 per cent\dots{} In the 8 years from 1853 to 1861, it had increased from the
basis taken in 1853, 20 per cent! The fact is so astonishing as to be almost
incredible\dots{} this intoxicating augmentation of wealth and power\dots{} entirely
confined to classes of property\dots{} must be of indirect benefit to the
labouring population, because it cheapens the commodities of general consumption
— while the rich have been growing richer the poor have been
growing less poor! at any rate, whether the extremes of poverty are less,
I do not presume to say». Ґледстон у Палаті громад 16 квітня 1863 р.
«Morning Star», 17 квітня.
} Яка слабенька прикінцева частина цього
періоду!

Якщо робітнича кляса лишилася «бідною», тільки «менш
бідною» порівняно з тим «приголомшливим збільшенням багатства
й сили», яке вона спродукувала для кляси власників, то
відносно вона лишилася так само бідною, як і раніш. Якщо
крайності бідности не зменшилися, то вони збільшилися, бо
збільшилися крайності багатства. Щождо подешевшання засобів
існування, то офіціяльна статистика, наприклад, дані лондонського
сирітського дому (Orphan Asylum), показує подорожчання
на 20\% пересічно за три роки 1860--1862 порівняно з роками
1851--1853. В наступні три роки, 1863--1865, проґресивне
подорожчання м’яса, масла, молока, цукру, соли, вугілля й
сили інших доконечних засобів існування.\footnote{
Див. офіціяльні дані в Синій Книзі: «Miscellaneous Statistics
of the United Kingdom». Part VI, London 1866, crop. 260--273 і далі.
Замість статистики сирітських домів за докази могли б служити й деклямації
міністерських журналів, що обстоюють придане для дітей королівського
дому. Там ніколи не забувають про дорожнечу засобів існування.
} Наступна бюджетова
\index{i}{0562}  %% посилання на сторінку оригінального видання
промова Ґледстона, з 7 квітня 1864 р., — це піндарівський
дитирамб на проґрес у збагаченні й на стримуване «злиднями»
щастя народу. Він каже про маси, що стоять «на краю
павперизму», про галузі продукції, «де заробітна плата не
зросла», і наприкінці резюмує щастя робітничої кляси в таких
словах: «Людське життя в дев’ятьох із десятьох випадків це
просто боротьба за існування».\footnote{
«Think of those who are on the border of that region (pauperism)»,
«wages\dots{} in others not increased\dots{} human life is but, in nine cases out of ten,
a struggle for existence» (Ґледстон у Палаті громад 7 квітня 1864 р.).
«Hansard» дає таку версію цього резюме: «Висловлюючи це в загальнішій
формі: «Що таке людське життя в більшості випадків, як не боротьба
за існування» («Again, and yet more at large, what is human life but,
in the majority of cases, a struggle for existence»). — Постійні кричущі суперечності
в бюджетових промовах Ґледстона з 1863 і 1864 рр. один англійський
письменник характеризує такою цитатою з Мольєра:

«Voilà l’homme en effet. Il va du blanc au noir.
Il condamne au matin ses sentiments du soir.
Importun à tout autre, à soi même incommode,
Il change à tous moments d’esprit comme de mode».

(«Така людина: зараз біле, далі чорне.
Що ввечорі хвалила, засуджує уранці.
Усім набридла і самій собі як тягар,
І щохвилини настрій змінює як моду»).

(«Th. Theory of Exchanges etc.»,
London 1864, p. 135).
} Професор Фавсет, не зв’язаний,
як Ґледстон, офіціяльними міркуваннями, прямо заявляє:
«Я, звичайно, не заперечую, що грошова плата підвищилася із
цим збільшенням капіталу [останніми десятиріччями], але ця
позірна користь у значних розмірах знову пропадає через те,
що багато потрібних засобів існування щораз дорожчає на його
думку, через падіння вартости благородних металів]\dots{} Багаті
швидко стають ще багатшими (the rich grow rapidly richer),
тимчасом як у побуті робітничих кляс не помітно ніякого поліпшення\dots{}
Робітники стають майже рабами крамарів, що в них
вони позаборговувались».\footnote{
H. Fawcett: «The Economie Position of the British Labourer»,
London 1865. p. 67, 82. Щождо дедалі більшої залежності! робітників од
крамарів, то це є наслідок щораз частіших коливань і перерв у їхній
занятості.
}

У розділах про робочий день і машини ми розкрили ті обставини,
в яких брітанська робітнича кляса створила «приголомшливе
збільшення багатства й сили» для маєтних кляс. Однак
нас тоді цікавив переважно робітник підчас його суспільної
функції. Щоб цілком висвітлити закони капіталістичної акумуляції,
треба також на хвилину зупинитись на становищі робітника
поза майстернею, на тому, яке його харчове й житлове становище.
Рамки книги цієї примушують нас насамперед взяти тут на увагу
найгірш оплачувану частину промислового пролетаріату і рільничих
робітників, тобто більшість робітничої кляси.

\index{i}{0563}  %% посилання на сторінку оригінального видання
Перед цим іще одно слово про офіціальний павперизм, тобто
про ту частину робітничої кляси, яка втратила умову свого існування,
продаж робочої сили, і животіє з громадської милостині.
(Офіціяльна статистика налічувала в Англії\footnote{
До Англії завжди залічують Велз, до Великобрітанії — Англію,
Велз і Шотляндію, до Об’єднаного королівства — ці три країни й Ірляндію.
}. 1855~\abbr{р.} \num{851.369} павперів,
1856~\abbr{р.} — \num{877.767}, 1865~\abbr{р.} — \num{971.433}. У наслідок бавовняного
голоду число їх зросло 1863 і 1864~\abbr{рр.} до \num{1.079.382}
й \num{1.014.978}. Криза 1866~\abbr{р.}, що найтяжче вразила Лондон, створила
в цьому центрі світового ринку, більшому числом жителів
за королівство Шотляндію, приріст павперів на 19,5\% для
1866~\abbr{р.} порівняно з 1865~\abbr{р.} і на 24,4\% порівняно з 1864~\abbr{р.} і ще
більший приріст для перших місяців 1867~\abbr{р.} порівняно з
1866~\abbr{р.} Аналізуючи статистику павперизму, треба звернути увагу
на два пункти. З одного боку, рух зменшення і збільшення маси
павперів відбиває в собі періодичні зміни промислового циклу.
З другого боку, офіціяльна статистика стає чимраз більш фальшивим
показником дійсних розмірів павперизму в міру того,
як з акумуляцією капіталу розвивається клясова боротьба, а
тому й почуття самоповаги в робітників. Наприклад, варварство
в поводженні з павперами, про що останніми двома роками так
голосно кричала англійська преса («Times», «Pall Mall Gazette»
і~\abbr{т. д.}) — явище старе. Енґельс констатує в 1844~\abbr{р.} цілком
ті самі страхіття й цілком те саме минуще лицемірне обурення
«сенсаційної літератури». Але страшне збільшення числа випадків
голодної смерти («death by starvation») в Лондоні протягом
останніх десятьох років безумовно свідчить про щораз більше зростання
огиди робітників до рабства робітних домів\footnote{
Своєрідне освітлення проґресові, який стався від часів А. Сміса,
дає та обставина, що для нього слово workhouse, робітний дім, деколи ще
рівнозначне слову manufactory, мануфактура. Наприклад, у нього на
початку розділу про поділ праці: «Тих робітників, що працюють у різних
галузях праці, часто можна скупчувати в тому самому робітному домі»
(«Those employed in every different branch of the work can often be collected
into the same workhouse»);
}, цих карних
закладів для бідности.

\subsubsection{Погано оплачувані верстви брітанської промислової робітничої
кляси}

Звернімося тепер до погано оплачуваних верств промислової
робітничої кляси. Підчас бавовняного голоду 1862~\abbr{р.} Privy
Council\footnote*{
— Таємна державна рада. \emph{Ред.}
} доручив докторові Смісові розслідити стан харчування
збіднілих бавовняних робітників Ланкашіру й Чешіру. Довголітнє
попереднє спостереження привело його до результату, що
«для того, щоб запобігти недугам від голоду» (starvation diseases),
щоденні харчі жінки мусять мати в собі пересічно щонайменш
\num{3.900} ґранів вуглецю і 180 ґранів азоту, щоденні харчі
чоловіка — пересічно щонайменше \num{4.300} ґранів вуглецю і 200 ґра-
\parbreak{}  %% абзац продовжується на наступній сторінці

\parcont{}  %% абзац починається на попередній сторінці
\index{i}{0564}  %% посилання на сторінку оригінального видання
азоту, — для жінок приблизно стільки поживних речовин,
скільки їх є у двох фунтах доброго пшеничного хліба, для чоловіка — на
\sfrac{1}{9} більше; для дорослих жінок і чоловіків — щонайменше
\num{28.600} ґранів вуглецю і \num{1.330} ґранів азоту пересічно на тиждень.
Практика на диво ствердила його обчислення: воно цілком погоджувалося
з тією мізерною кількістю харчів, на яку нужденний
стан звів споживання бавовняних робітників. Вони одержували
в грудні 1862~\abbr{р.} \num{29.211} ґранів вуглецю і \num{1.295} ґранів азоту
на тиждень.

1863~\abbr{р.} Privy Council наказав розслідувати нужденний стан
тієї частини англійської робітничої кляси, що харчувалася найгірше.
Д-р Сімон, медичний урядовець Privy Council, вибрав
для цієї праці вищезгаданого доктора Сміса. Розслід його охоплює,
з одного боку, рільничих робітників, а з другого — шовкоткачів,
швачок, виробників шкуряних рукавичок, панчішників,
рукавичників і шевців. Останні категорії, за винятком панчішників,
є виключно міські робітники. При дослідженні дотримували
правила — вибирати в кожній категорії якнайздоровіші
й відносно забезпеченіші родини.

Загальний результат дослідження був той, що «тільки в
одній із досліджених кляс міських робітників кількість споживаного
азоту трохи перевищувала ту абсолютну мінімальну
міру, нижче якої постають недуги від голоду; у двох клясах
була недостача, а в одній з них навіть дуже велика недостача
у споживанні так азотових, як і вуглецевих харчів; з досліджених
рільничих родин більш як п’ятина одержувала вуглецевих
харчів менше, ніж доконечно, більш як \sfrac{1}{3} одержувала
азотових харчів менш, ніж доконечно, а в трьох графствах
(Berkshire, Oxfordshire і Somersetshire) пересічно панувала недостача
мінімальної кількости азотових харчів»\footnote{
«Public Health. 6th Report etc. for 1863», London 1864, p. 13.
}. Серед рільничих
робітників найгірші харчі діставали робітники Англії —
найбагатшої частини Об’єднаного королівства\footnote{
Там же. стор, 17.
}. Серед сільських
робітників недостатнє харчування припадало переважно
на жінок і дітей, бо «чоловік мусить їсти, щоб виконувати свою
роботу». Ще більша нужда лютувала серед досліджених міських
категорій робітників. «Вони харчуються так погано, що випадки
жорстокої нужди, яка руйнує здоров’я, мусять траплятися дуже
часто [все це є «поздержливість» капіталіста! а саме, він здержується
від того, щоб платити за засоби існування, доконечні
просто для животіння його «рук»!]\footnote{
Так же, стор. 13.
}.

Нижченаведена таблиця дає можливість порівняти стан харчування
згаданих вище суто міських категорій робітників з
тим харчовим мінімумом, що його визначив д-р Сміс, і з
кількістю харчів бавовняних робітників за часів їхньої найбільшої
нужди:
\parbreak{}  %% абзац продовжується на наступній сторінці

\parcont{}  %% абзац починається на попередній сторінці
\index{i}{0565}  %% посилання на сторінку оригінального видання
Обидві статі                                                        Пересічна
       Пересічна
                                                                                тижнева
                    тижнева
                                                                                кількість вуглецю
            кількість азоту
                                                                                (ґранів)
                      (ґранів)

П’ять міських галузей промисловости..     28.876                                      1.192
Безробітні фабричні робітники
Ланкаширу\dotfill     28.211
1.295
Мінімальна кількість, запропонована
для ланкашірських робітників при
рівному числі чоловіків і жінок\dotfill      28.600                                      1.330
\footnote{
Там же, додаток, стор. 232.
}

Половина, \sfrac{60}{125}, із досліджених категорій промислових робітників
зовсім не споживала пива, 28\% — молока. Пересічна
тижнева кількість рідких поживних речовин коливалася від
7 унцій на родину в швачок до 24 унцій у панчішників. Більшість
тих, що не споживали молока, складалася з лондонських
швачок. Кількість споживаного на тиждень хліба коливалася
від 7\sfrac{3}{4} фунтів у швачок до 11\sfrac{1}{4} фунтів у шевців і становила пересічно
9,9 фунта на тиждень на дорослого. Кількість цукру (сиропу
й т. ін.), коливалася від 4 унцій на тиждень у виробників
шкуряних рукавичок до 11 унцій у панчішників; ціла пересічна
кількість на тиждень для всіх категорій — 8 унцій на одного
дорослого. Загальна пересічна кількість масла (жиру й т. ін.)
на тиждень — 5 унцій на дорослого. Пересічна тижнева кількість
м’яса (сала й т. ін.) на дорослого коливалася від 7\sfrac{1}{4} унцій у шовкоткачів
до 18\sfrac{1}{4} унцій у виробників шкуряних рукавичок;
загальна пересічна кількість для різних категорій — 13,6 унцій.
Щотижнева витрата на харчі для дорослих становила такі загальні
пересічні числа: шовкоткачі — 2 шилінґи 2\sfrac{1}{2} пенса,
швачки — 2 шилінґи 7 пенсів, виробники шкуряних рукавичок
— 2 шилінґи 9\sfrac{1}{2} пенсів, шевці — 2 шилінґи 7\sfrac{3}{4} пенсів,
панчішники — 2 шилінґи 6\sfrac{1}{4} пенсів. Для шовкоткачів з Macclesfield’у
тижнева пересічна кількість становила лише 1 шилінґ
8\sfrac{1}{2} пенсів. Найгірше харчувалися швачки, шовкоткачі й виробники
шкуряних рукавичок.\footnote{
Там же, стор. 232. 233.
}

У своєму загальному санітарному звіті д-р Сімон каже про
цей стан харчування таке: «Кожний, хто обізнаний з медичною
практикою серед бідних або з пацієнтами шпиталів, однаково,
чи живуть вони по шпиталях, чи поза ними, потвердить, що випадки,
коли недостача харчів породжує або загострює недуги,
дуже численні\dots{} Однак із санітарного погляду сюди долучається
ще інша, дуже важлива обставина. Треба пригадати собі, що
позбавлення харчових засобів терпиться лише з великим опором
і що, звичайно, дуже недостатнє харчування є лише наслідок
\parbreak{}  %% абзац продовжується на наступній сторінці

\parcont{}  %% абзац починається на попередній сторінці
\index{i}{0566}  %% посилання на сторінку оригінального видання
інших попередніх нестатків. Задовго перед тим, як недостатнє
харчування почне впливати на здоров’я, задовго перед тим, як
фізіолог почне лічити ґрани азоту й вуглецю, між якими коливається
життя й голодна смерть, домашнє господарство вже геть
чисто позбувається всякого матеріяльного комфорту. Одяг і
опалення стають ще злиденніші, ніж харч. Немає достатнього
захисту від суворої негоди; щодо житла, то обмеження доходить
такої міри, що воно породжує або загострює недуги; ледве помітні
сліди домового начиння або меблів, навіть чистота стає
дуже дорогою; підтримувати її важко. А проте, коли з самоповаги
ще й робляться спроби підтримувати її, то кожна така спроба
сполучена з додатковими муками голоду. Оселяються там, де
найдешевше можна найняти притулок, у кварталах, де санітарна
поліція має якнайменші успіхи, де каналізація найгірша, де
найнезначніша комунікація, найбільше бруду, де наймізерніше
й найгірше водопостачання, а по містах — де найбільша недостача
світла й повітря. Це — ті небезпеки для здоров’я, яких
неминуче зазнає бідність, коли вона сполучена з недостачею харчів.
Коли сума всіх цих злигоднів має страшенний вплив на
життя, то вже проста недостача харчів сама по собі жахлива\dots{}
Це — дуже болючі думки, особливо коли пригадати собі, що
бідність, про яку тут мовиться, це не є бідність з своєї вини,
викликана ледарством. Це — бідність робітників. І навіть щодо
міських робітників, ту працю, за яку вони купують собі цю мізерну
дещицю харчу, здебільша здовжують понад усяку міру.
А проте лише в дуже умовному розумінні можна сказати, що
з цієї праці можна себе самого утримати\dots{} Це номінальне утримання
самого себе у дуже великій мірі є не що інше, як коротший
або довший обхідний шлях до павперизму»\footnote{
Там же, стор. 14, 15.
}.

\looseness=1
Внутрішній зв’язок між муками голоду якнайпрацьовитіших
верств робітників і брутальним або рафінованим марнотратством
багатих, основаним на капіталістичній акумуляції, можна
розкрити лише через пізнання економічних законів. Інша справа
з житловими умовами. Кожний безсторонній спостерігач бачить,
що чим масовіший характер має централізація засобів продукції,
тим більше відповідне скупчення робітників на тому самому
просторі, отже, чим швидша капіталістична акумуляція, тим
злиденніші житлові умови робітників. «Поліпшення» (improvements)
міст через зламання погано побудованих кварталів,
будування палаців для банків, універсальних крамниць і~\abbr{т. д.},
будування вулиць для комерційних зносин і розкішних екіпажів,
заведення кінних залізниць і~\abbr{т. д.}, — ці «поліпшення»,
що супроводять проґрес багатства, очевидно, заганяють бідних
у щораз гірші й щораз густіш заповнювані закутки. З другого
боку, кожний знає, що дорожнеча помешкань є зворотно пропорційна
до їхньої якости, і що будівельні спекулянти з більшим
зиском і з меншими витратами експлуатують копальні бідности,
\parbreak{}  %% абзац продовжується на наступній сторінці

\parcont{}  %% абзац починається на попередній сторінці
\index{i}{0567}  %% посилання на сторінку оригінального видання
ніж колись експлуатовано копальні Потозі. Антагоністичний
характер капіталістичної акумуляції, а тому й капіталістичних
відносин власности взагалі\footnote{
«Ніде права особи не жертвовано так одверто й безсоромно на
користь праву власности, як у житлових умовах робітничої кляси. Кожне
велике місто являє собою місце людських жертов, вівтар, на якому рік-у-рік
убивають тисячі людей для Молоха ненажерливости». (\emph{S. Laing}:
«National Distress», 1844, р. 150).
} стає тут такий наочний, що навіть
офіціальні англійські звіти про цей предмет переповнені єретичними
нападами на «власність і її права». Це лихо так поширилося
з розвитком промисловости, акумуляцією капіталу, зростом
та «прикрашуванням» міст, що лише острах перед пошесними
недугами, які не щадять навіть «шановних», викликав від 1847
до 1864~\abbr{р.} не менше як десять санітарно-поліційних парляментських
актів, а перелякана буржуазія деяких міст, як от Ліверпуль,
Ґлезґо і~\abbr{т. д.}, почала втручатися в цю справу через свої
муніципалітети. Проте, д-р Сімон у своєму звіті з 1865~\abbr{р.} вигукує:
«Взагалі кажучи, цей лихий стан в Англії лишається без
контролю». З наказу Privy Council в 1864~\abbr{р.} досліджено житлові
умови сільських робітників, в 1865~\abbr{р.} — бідніших кляс по містах.
Майстерні праці д-ра Джульяна Гентера надруковано в
сьомому й восьмому звітах «Public Health». Про сільських робітників
я говоритиму пізніше. Щождо міських житлових умов,
то я насамперед наведу загальну увагу д-ра Сімона: «Хоч
мій офіціяльний погляд, — каже він, — виключно медичний, проте
звичайна гуманність не дозволяє мені не звертати уваги на другий
бік цього лиха. Дійшовши вищого ступеня, це лихо майже
неминуче зумовлює таке заперечення всякої звичайности, таке
брудне змішування тіл і фізичних функцій, таку одверту наготу
статей, що все це має звірячий, а не людський характер. Зазнавати
таких впливів — це зневага, яка стає то глибша, що довше
вона триває. Для дітей, що народилися під цим прокляттям,
воно є хрещення на ганьбу (baptism into infamy). І безнадійним
понад усяку міру є бажання, щоб люди, поставлені в такі умови,
в інших відношеннях прагнули тієї атмосфери цивілізації, що її
суть у фізичній і моральній чистоті».\footnote{
«Public Health, Eighth Report», London 1866, p. 14, примітка.
}

Перше місце щодо переповнення помешкань або й щодо абсолютної
непридатности їх для людського житла посідає Лондон.
«Дві обставини, — каже д-р Гентер, — певні: по-перше, в Лондоні
є щось з 20 великих колоній, кожна з яких має приблизно
\num{10.000}  осіб, що їхнє злиденне становище переважає все, що будь-коли
бачили деінде в Англії, і це становище є майже цілком результат
поганого стану їхнього житла; по-друге, переповненість і зруйнованість
домів по цих колоніях тепер значно гірші, ніж двадцять
років тому».\footnote{
Там же, стор. 89. Щодо дітей із цих колоній д-р Гентер каже:
«Ми не знаємо, як виховували дітей перед цією епохою тісного скупчення
бідних, і сміливим пророком був би той, хто хотів би наперед сказати,
чого можна сподіватися від дітей, які серед умов, у цій країні безприкладних,
виховуються тепер на небезпечні у майбутній своїй практиці кляси,
проводячи час до півночі з особами всякого віку, п'яними, розпусними та
лайливими». (Там же, стор. 56).
} «Не буде перебільшенням сказати, що
\index{i}{0568}  %% посилання на сторінку оригінального видання
життя в багатьох частинах Лондону й Ньюкестлю в пекло».\footnote{
Там же, стор. 62.
}

Але й тій частині робітничої кляси, що живе в кращих умовах,
а також дрібним крамарям та іншим елементам дрібної
середньої кляси, в Лондоні чимраз більше дається взнаки прокляття
цих мізерних житлових умов у міру того, як проґресують
«поліпшення», а з ними й зламання старих будинків і кварталів,
у міру того, як зростає число фабрик і наплив людей до головного
міста, нарешті, в міру того, як разом із міською земельною
рентою зростає плата за квартиру. «Плата за квартиру стала
така непомірна, що небагато робітників може оплатити більш
ніж одну кімнату».\footnote{
«Report of the Officer of Health of St. Martin’s in the Fields», 1865.
} У Лондоні немає майже жодної домовласности,
що не була б обтяжена безліччю «middlemen’ів».\footnote*{
— посередників. \emph{Ред.}
} Ціна
землі в Лондоні завжди дуже висока порівняно з річними доходами
з неї, бо кожний покупець спекулює на те, щоб раніш або
пізніше знову продати її за Jury Price (за ціну, визначену присяжними
при експропріяціях), або щоб вишахрувати надзвичайне
підвищення її вартости через близькість її до якогось великого
підприємства. Наслідок цього є реґулярна торговля контрактами
найму, що їх скуповують, коли їхній реченець наближається
до кінця. «Від джентльменів у цій справі можна сподіватися, що
вони робитимуть так, як роблять, а саме видушуватимуть з
квартирантів якомога більше, а самий дім передаватимуть своїм
наступникам у якомога злиденнішому стані».\footnote{
«Public Health. Eighth Report», London 1866. p. 91.
} Плата за квартиру
— тижнева, і ці панки нічим не ризикують. У наслідок
того, що залізниці будується в межах міста, «недавно одного
суботнього вечора у східній частині Лондону можна було бачити,
як сила родин, вигнаних із своїх старих помешкань, тинялися
з своїми злиденними пожитками на плечах, ніде не находячи
собі притулку, крім лише в робітному домі.\footnote{
Там же, стор. 88.
} Робітні
доми вже переповнені, а ухвалені парляментом «поліпшенням
почали ще тільки проводити в життя. Коли робітників проганяють
в наслідок руйнування їхніх старих домів, то вони не покидають
своєї парафії або принаймні оселяються на її межі, в найближчій
парафії. «Ясна річ, вони силкуються оселитися якомога
ближче до місця своєї праці. Наслідок той, що замість двох кімнат
родина мусить оселитися в одній кімнаті. Навіть при підвищеній
платі нове помешкання є гірше, ніж те погане, з якого їх вигнано.
Вже половині робітників на Strand’і доводиться ходити дві милі
до місця праці». Цей Strand, що його головна вулиця справляє
на чужинця імпозантне вражіння багатством Лондону, може
бути за приклад того, як напаковують людей у Лондоні. В одній
парафії цього Strand’у санітарний урядовець налічив 581 особу
\parbreak{}  %% абзац продовжується на наступній сторінці

\parcont{}  %% абзац починається на попередній сторінці
\index{i}{0569}  %% посилання на сторінку оригінального видання
акр, не зважаючи на те, що він залічив у цю територію і полону
ширини Темзи. Само собою зрозуміло, що всякі санітарно-поліційні
заходи, які, як це досі робилося в Лондоні, через
зламання негодящих домів виганяють робітників з одного кварталу,
служать лише для того, щоб їх щільніше напаковувати
л іншому. «Або, — каже д-р Гентер, — цілу цю процедуру,
як безглузду, треба цілком припинити, або мусить пробудитися
громадське співчуття (!) до того, що тепер, не перебільшуючи,
можна назвати національним обов’язком, а саме до того, щоб
дати притулок людям, які через брак капіталу не можуть його
сами собі придбати, але можуть дати відшкодування власникам
квартир періодичними виплатами»\footnote{
Там же, стор. 89.
}. Дивовижна річ ця капіталістична
справедливість! Землевласник, домовласник, комерсант,
коли в нього що експропріюють задля «поліпшень», як от
будова залізниць, будова нових вулиць і~\abbr{т. д.}, не тільки дістає
повне відшкодування. За своє вимушене «зречення» він, крім
того, мусить за божими й людськими законами мати як нагороду
ще й не абиякий зиск. А робітників з дружинами, дітьми й пожитками
викидають на брук, і якщо вони занадто великими масами
ринуть до міських кварталів, в яких муніципалітет особливо
стежить за добропристойністю, то їх переслідує санітарна поліція!

На початку XIX віку в Англії, окрім Лондону, не було жодного
міста, що налічувало б \num{100.000} мешканців. Тільки п’ять
мало понад \num{50.000}. Тепер існує 28 міст, що мають понад \num{50.000} мешканців.
«Результатом цієї зміни був не тільки величезний приріст
міської людности, але й те, що старі битком набиті дрібні
міста тепер стали центрами, з усіх боків позабудованими, без
якогобудь вільного допливу свіжого повітря. Через те, що ці
міста стали вже неприємними для багатих, вони їх залишають,
оселюючись на веселіших передмістях. Наступники цих багатіїв
оселюються у великих домах, при чому кожна родина, часто ще
з квартирантами, дістає по одній кімнаті. Таким чином людність
втискується в доми, не для неї призначені, до її потреб зовсім
непристосовані, в оточенні, що справді понижує дорослих і руйнує
дітей»\footnote{
Там же, стор. 55, 56.
}. Що швидше в якомусь промисловому або торговельному
місті акумулюється капітал, то швидше припливає
приступний для експлуатації людський матеріял, то злиденніші
імпровізовані житла робітників. Тим то Ньюкестл над Тайном,
як центр кам’яновугільної й гірничої округи, що чимраз дужче
розвивається, посідає після Лондону друге місце в житловому
пеклі. Не менше, як \num{34.000} осіб живе там по окремих комірках.
У наслідок абсолютної небезпеки для громадського здоров’я
з наказу поліції в Ньюкестлі і Ґетшеді порозвалювано недавно
чимало домів. Будування нових домів посувається дуже повільно,
а промисловість розвивається дуже швидко. Тому 1865~\abbr{р.}
місто було переповнене більше ніж будь-коли раніш. Ледве
чи можна було найняти хоч одну комірку. Д-р Імблтон із шпиталю
\index{i}{0570}  %% посилання на сторінку оригінального видання
для хорих на тиф у Ньюкестлі каже: «Безперечно, причина
тривання й поширення тифу є переповнення помешкань людьми
та нечистота цих помешкань. Доми, де звичайно живуть робітники,
стоять у глухих заулках і подвір’ях. Щодо світла, повітря,
простору й чистоти це є справді зразки недостатности й негігієнічности,
ганьба для кожної цивілізованої країни. Там чоловіки
жінки й діти лежать ночами вкупі, поперемішувані як попало.
Щодо чоловіків, то нічна зміна невпинною течією йде по
денній, так що ліжка ледве встигають прохолонути. Доми погано
забезпечено водою і ще гірше кльозетами, вони страшенно нечисті,
не вентилюються, поширюють заразу»\footnote{
Там же, стор. 149.
}. Тижнева плата
за такі діри становить від 8\pens{ пенсів} до 3\shil{ шилінґів.} «Ньюкестл
над Тайном, каже д-р Гентер, являє собою приклад того, як
одне з найкращих племен поміж нашими земляками через зовнішні
умови, а саме через стан помешкань і вулиць, занепадає
часто майже до стану дикого виродження»\footnote{
Там же, стор. 50.
}.

У наслідок постійного припливу й відпливу капіталу й праці
житловий стан якогось промислового міста може бути сьогодні
стерпний, а на завтра стає огидний. Або ж міська влада, нарешті,
отямлюється і починає усувати щонайгірші непорядки. Але на
завтра ж хмарою сарани насувають обідрані ірляндці або занепаді
англійські рільничі робітники. Їх запихають у льохи й
комори, або порядний колись дім для робітників перетворюють
на помешкання, де персонал змінюється так швидко, як солдатські
постої підчас тридцятирічної війни. Приклад: Bradford.
Саме там муніципальні філістери заходилися коло міської реформи.
Крім того, там 1861~\abbr{р.} було ще \num{1.751} незаселений дім.
Аж ось справи пішли добре, як про це нещодавно так любо
розводився солодкувато-ліберальний професор Форстер, приятель
негрів. Певна річ, з поліпшенням справ постає повідь
від хвиль «резервної армії», або «відносного перелюднення»,
що постійно хвилюється. Найгидкіші мешкання по льохах
та коморах, зареєстрованих у списку\footnote{
Список, що його склав аґент одного товариства для забезпечення
робітників у Bradford’i:

\begin{center}
\scriptsize
% See longtable manual
\setcounter{LTchunksize}{2}
\begin{longtable}{lc@{ }c@{ }r@{ }l}
Vulcanstreet. Nr. 122\dotfill{} & 1 & кімната & 16 & осіб \\
Lumleystreet. Nr. 13\dotfill{} & 1 & \ditto{кімната} & 11 & осіб \\
Bowerstreet. Nr. 41\dotfill{} & 1 & \ditto{кімната} & 11 & осіб \\
Portlandstreet. Nr. 112\dotfill{} & 1 & \ditto{кімната} & 10 & осіб \\
Hardystreet. Nr. 17\dotfill{} & 1 & \ditto{кімната} & 10 & осіб \\
Northstreet. Nr. 18\dotfill{} & 1 & \ditto{кімната} & 16 & осіб \\
\ditto{Northstreet.} Nr. 17\dotfill{} & 1 & \ditto{кімната} & 13 & осіб \\
Wymerstreet. Nr. 19\dotfill{} & 1 & \ditto{кімната} & 8 & дорослих \\
Jowettstreet. Nr. 56\dotfill{} & 1 & \ditto{кімната} & 12 & осіб \\
Georgestreet. Nr. 150\dotfill{} & 1 & \ditto{кімната} & 3 & родини \\
Rifle-Court, Marygate. Nr. 11\dotfill{} & 1 & \ditto{кімната} & 11 & осіб \\
Marshallstreet. Nr. 28\dotfill{} & 1 & \ditto{кімната} & 10 & осіб \\
\ditto{Marshallstreet.} Nr. 49\dotfill{} & 1 & \ditto{кімната} & 3 & родини \\
Georgestreet. Nr. 128\dotfill{} & 1 & \ditto{кімната} & 18 & осіб \\
\ditto{Georgestreet.} Nr. 130\dotfill{} & 1 & \ditto{кімната} & 16 & осіб \\
Edwardstreet. Nr. 4\dotfill{} & 1 & \ditto{кімната} & 17 & осіб \\
Jorkstreet. Nr. 34\dotfill{} & 1 & \ditto{кімната} & 2 & родини \\
Salt Piestreet\dotfill{} & 1 & \ditto{кімната} & 26 & осіб \\
\multicolumn{5}{c}{\emph{Льохи}} \\
Regent Square\dotfill{} & 1 & льох & 8 & осіб \\
Acrestreet\dotfill{} & 1 & \ditto{льох} & 7 & осіб \\
Robert’s Court. Nr. 33\dotfill{} & 1 & \ditto{льох} & 7 & осіб \\
Back Prattstreet, помешкання \\
\indentdef{}використовується як мідярня\dotfill{} & 1 & \ditto{льох} & 7 & осіб \\
Ebenezerstreet. Nr. 27 \dotfill{} & 1 & \ditto{льох} & 6 & осіб \\
\multicolumn{5}{l}{(«Public Health. Eighth Report», стор. 111).}
\end{longtable}\end{center}}, що його одержав
д-р Гентер від аґента одного страхового товариства, займали
здебільшого добре оплачувані робітники. Вони заявляли, що
\index{i}{0571}  %% посилання на сторінку оригінального видання
охоче платили б за кращі помешкання, коли б їх можна було
дістати. Тимчасом вони із своїми родинами занепадають і хоріють,
а солодкувато-ліберальний Форстер, член парляменту, проливає
сльози захоплення з приводу благодаті вільної торговлі й зисків
видатних голів Bradford’у від вовняних підприємств. У звіті
з 5 вересня 1865~\abbr{р.} д-р Белл, один із бредфордських лікарів для
бідних, пояснює жахливу смертність хорих на тиф у його окрузі
їхніми житловими умовами: «В одному льоху на \num{1.500} кубічних
футів живе десять осіб\dots{} На вулицях. Vincentstrasse, Green
Air Place і the Leys є 223 доми з \num{1.450} мешканцями, 435 ліжками
і 36 кльозетами\dots{} На кожне ліжко — а під ліжком я розумію
всякий жмут брудного ганчір’я або купу стружок — припадає
пересічно 3,3 особи, на декотрі 4 й 6 осіб. Багато спить без ліжка
просто на підлозі, не роздягаючись, молоді чоловіки й жінки,
жонаті й нежонаті — все це як попало, одне побіч одного. Чи
треба ще додати, що ці житла здебільша темні, вогкі, брудні,
смердючі нори, цілком непридатні для людського мешкання?
Це — центри, звідки поширюються недуги й смерть, що виривають
свої жертви навіть з-серед заможних (of good circumstances),
які допустили до того, щоб ця моровиця гноїлася в нашому
середовищі»\footnote{
Там же, стор. 114.
}.

Третє після Лондону місце щодо житлових злиднів посідає
Брістол. «Тут, в одному з найбагатших міст Европи, якнайбільший
надмір глибокої бідности («blank poverty») і житлових
злиднів»\footnote{
Там же, стор. 50.
}.

\subsubsection{Бродяча людність}

А тепер звернімося до верстви людности, сільської своїм походженням,
але здебільша занятої в промисловості. Вона становить
легку піхоту капіталу, яку відповідно до своїх потреб
він кидає то в один пункт, то в інший. Коли вона не в поході,
то «стоїть табором». Працю бродячих робітників використовують
на різні будівельні й дренажні операції, вироблення цегли,
випалювання вапна, будову залізниць тощо. Вони є рухлива
колона, що переносить у ті місцевості, навколо яких вони отаборюються,
заразливі недуги: віспу, тиф, холеру, скарлятину
й~\abbr{т. ін.}\footnote{
«Public Health. Seventh Report», London 1865, p. 18.
} У підприємствах із значною витратою капіталу, як
\parbreak{}  %% абзац продовжується на наступній сторінці

\parcont{}  %% абзац починається на попередній сторінці
\index{i}{0572}  %% посилання на сторінку оригінального видання
от будування залізниць тощо, підприємець сам здебільшого постачає
своїй армії дерев’яні курені й т. ін., імпровізовані селища
що не мають ніяких гігієнічних засобів, не підлягають ніякому
контролеві місцевої влади, але дуже вигідні для пана підприємця,
що подвійно визискує робітників — як промислових солдатів
і як квартирантів. Залежно від того, скільки дір має курінь —
одну, дві чи три, мешканцеві, тобто копальникові й т. ін., доводиться
платити на тиждень 1, 3, 4, шилінґи.\footnote{
Там же, стор. 165.
} Досить буде одного
прикладу. У вересні 1864 р., — повідомляє д-р Сімон, — міністер
внутрішніх справ сер Джордж Ґрей одержав такого звіта
від голови Nuisance Removal Committee\footnote*{
Комітет у справі боротьби в антисанітарними умовами. \emph{Ред.}
} в парафії Sevenoaks:
«Іще 12 місяців тому віспа в цій парафії була цілком невідома.
Незадовго перед цим почалися роботи коло будови залізниці
від Lewisham до Tunbridge. Опріч того, що головні роботи провадились
у безпосередньому сусідстві з цим містом, тут ще й
улаштовано головне депо цілого підприємства. Тому тут працювало
багато робітників. Через те, що неможливо було помістити
їх усіх у котеджах, то підприємець, пан Джей, побудував уздовж
залізничної колії на різних пунктах курені для житла робітників.
Ці курені не мали жодної вентиляції, ані зливів на нечисть;
крім того, вони з доконечности були надмірно переповнені,
бо кожний наймач мусив приймати інших мешканців, хоч би
яка численна була його власна родина, і хоч у кожному курені
було лише дві кімнати. За лікарським звітом, що ми його одержали,
наслідок був той, що ці бідолахи мусили ночами зносити
всі муки задухи, щоб захистити себе від заразних випарів з брудних
калюж і кльозетів, що були зараз же під вікнами. Нарешті,
один лікар, що мав нагоду відвідати ці курені, передав нашому
комітетові скаргу. В якнайгіркіших висловах говорив він про
стан цих так званих помешкань і побоювався дуже серйозних
наслідків, якщо не вживеться деяких санітарних заходів. Приблизно
рік перед тим згаданий Джей зобов’язався збудувати дім,
куди негайно мали ізолювати занятих у нього робітників, скоро
вони захоріють на заразливі недуги. Наприкінці останнього
липня він повторив цю обіцянку, але не зробив найменшого
кроку, щоб виконати її, дарма що від того часу трапилось кілька
випадків віспи і двоє чоловіка від неї померло. 9 вересня лікар
Келсон повідомив мене про нові випадки віспи в цих куренях,
змальовуючи їхній стан як жахний. Для вашої (міністра) інформації
мушу я додати, що в нашій парафії є ізольований дім, так
званий пошесний дім, де ходять за парафіянами, недужими на
заразливі хороби. Цей дім ось уже кілька місяців постійно переповнений
пацієнтами. В одній родині померло п’ятеро дітей від
віспи і пропасниці. Від 1 квітня до 1 вересня цього року трапилось
не менш як 10 смертних випадків од віспи, 4 з них у згаданих
куренях, у цих джерелах пошестей. Подати число занедужань
\index{i}{0573}  %% посилання на сторінку оригінального видання
неможливо, бо родини, де вони трапляються, ховають їх
у якнайбільшій тайні».\footnote{
Там же, стор. 18, примітка. Опікун бідних у Chapei-en-le-Frith-Union
повідомляє генерального реєстратора: «В Doveholes у великому
горбі вапняного попелу пороблено багато печер. Ці печери служать
за житла для копальників та інших робітників, занятих коло будування
залізниць. Печери тісні, вогкі, без зливів на нечисть і без кльозетів.
У них немає жодного вентиляційного приладу, за винятком відтулини у
склепінні, і ця відтулина одночасно служить і за димар. Віспа лютує і
вже спричинила декілька смертних випадків (серед троглодитів). (Там же,
примітка 2).
}

Робітники на вугільних і інших шахтах належать до найкраще
оплачуваних категорій британського пролетаріату. Якою ціною
вони купують свою заробітну плату, це показано вже раніш.\footnote{
Подробиці, наведені на стор. 415 і дальших, стосуються саме до
робітників у кам'яновугільних копальнях. Про ще гірший стан у руднях
порівн. сумлінний звіт Royal Commission з року 1864.
}
Я кину тут оком на їхні житлові умови. Експлуататор шахт,
хоч буде це власник їх, хоч наймач їх, звичайно будує певне
число котеджів для своїх рук. Вони дістають котеджі й вугілля
на опал «даремно», тобто це становить частину заробітної плати,
що її видається in natura. Хто не дістає такого помешкання,
одержує замість цього 4\pound{ фунти стерлінґів} річно. Гірничі округи
швидко приманюють до себе дуже численну людність, що складається
з самих гірників, а також ремісників, крамарів тощо,
які групуються навколо гірників. Як і скрізь, де густа залюдненість,
земельна рента й тут висока. Тому гірнопромисловець
намагається на якнайтіснішому будівельному терені при вході
в шахту побудувати стільки котеджів, скільки треба, щоб понапихати
туди всі свої робочі руки разом з їхніми родинами.
Коли недалеко відкривають нові шахти або знов починають
працювати в старих, тіснота збільшується. При будуванні котеджів
переважає лише один погляд — «поздержливість» капіталіста
від усяких не абсолютно неминучих витрат готівкою.
«Помешкання шахтарів та інших робітників, що зв’язані з копальнями
Northumberland і Durham, — каже д-р Джуліян Гентер,
— у пересічному є, мабуть, чи не найгірше й найдорожче
з усього того, що в цьому напрямі дає у великому маштабі
Англія, за винятком хіба подібних округ у Monmouthshire. Найбільше
лихо у тому, що кожна кімната надто переповнена, площа
густо забудована великим числом будинків, бракує води й
немає кльозетів, часто вживають методи ставити будинки один
над одним або розділяти їх in flats (так що різні котеджі становлять
поверхи, які вертикально лежать один над одним)\dots{} Підприємець
поводиться з усією колонією так, наче вона лише стоїть
табором, а не живе постійно».\footnote{
Там же, стоp. 180, 182.
} «Виконуючи дані мені інструкції,
— каже д-р Стивенс, — я відвідав більшу частину великих
гірничих селищ Durham Union’у\dots{} За дуже небагатьма винятками,
треба сказати, що ніде не звертають найменшої уваги на
заходи охорони здоров’я мешканців\dots{} Всі гірники прикріплені
\parbreak{}  %% абзац продовжується на наступній сторінці

\parcont{}  %% абзац починається на попередній сторінці
\index{i}{0574}  %% посилання на сторінку оригінального видання
(«bound» — вислів, що, як і bondage, походить із часів епохи
кріпацтва), — на дванадцять місяців до орендаря («lessee») або
власника копалень. Коли ж вони виявляють своє незадоволення
або якимсь іншим способом дошкуляють доглядачеві («viewer»),
то він робить у своїй книжечці значок або помітку біля їхнього
ймення і звільняє їх при відновленні річного контракту\dots{} Мені
здається, що жодна з форм trucksystem’n не може бути гіршою,
ніж та, що панує в цих густо позаселюваних округах. Робітник
примушений одержувати як частину своєї заробітної плати дім
у заразливому оточенні. Він не може сам собі допомогти. Він
усіма сторонами — кріпак (he is to all intents and purposes a
serf). Ще питання, щоб хто міг допомогти йому, окрім його власника,
але цей власник радиться насамперед із своїм балянсом,
і результат цього майже безпомилковий. Власник постачає робітникові
також і воду. Хоч добра вона, хоч погана, хоч дістає
її робітник, хоч ні, він мусить за неї платити або, точніше, дозволити
вивертати за неї з заробітної плати».\footnote{
Там же, стор. 515, 517.
}

У конфлікті з «громадською думкою» або навіть із санітарною
поліцією капітал зовсім не соромиться «виправдувати»
ті почасти небезпечні, почасти ганебні умови, в які він заганяє
працю й домашнє життя робітника, тим, що це потрібно для того,
щоб вигідніше визискувати його. Так стоїть справа, коли він
«поздержується» від пристроїв для охорони від небезпечних машин
на фабриці, від вентиляційних і убезпечливих засобів на шахтах
і т. ін. Так само тут стоїть справа і з житлами для гірників. «Як
виправдання ганебних житлових умов, — каже в своєму офіціальному
звіті д-р Сімон, лікарський урядовець Privy Council, —
наводять те, що копальні звичайно експлуатують, беручи їх в
оренду, що строк орендного контракту (в копальнях здебільша
21 рік) занадто короткий, щоб орендарям варто було влаштовувати
як слід помешкання для робітників, ремісників і т. ін.,
яких притягає до себе підприємство; коли б у нього навіть і був
намір бути щедрішим з цього боку, то в цьому йому став би на
перешкоді власник. А саме цей останній одразу почав би вимагати
надзвичайно високої додаткової ренти за привілей будувати
на його ґрунті порядне й пристойне селище для робітників,
які добувають його підземну власність. Ця заборонна ціна, якщо
не пряма заборона, залякує і тих, що за інших умов хотіли б
будувати селища\dots{} Я не хочу докладніше досліджувати гідність
цього виправдання, і так само не хочу досліджувати, на кого
остаточно спали б додаткові видатки за будову порядних помешкань
— на землевласника, на орендаря копалень, на робітника,
чи на суспільство\dots{} Але, зважаючи на такі ганебні факти, як
ось ті, що їх викривають долучені звіти [д-ра Гентера, Стівенса
й т. ін.], треба вжити заходів, щоб усунути їх. Права
на земельну власність використовують так, що учиняють велику
суспільну несправедливість. Як власник копальні землевласник
\parbreak{}  %% абзац продовжується на наступній сторінці

\parcont{}  %% абзац починається на попередній сторінці
\index{i}{0575}  %% посилання на сторінку оригінального видання
запрошує промислову колонію працювати у його маєтках, а
потім він як власник поверхні землі не дає змоги зібраним ним
робітникам найти відповідне помешкання, доконечне для їхнього
влиття. Орендар копалень [капіталістичний експлуататор] не
має жодного грошового інтересу опиратися цьому двозначному
торгові, бо він добре знає, що коли вимоги землевласника непомірні,
то наслідки цього спадуть не на нього, що робітники, на
яйих вони спадуть, надто несвідомі, щоб знати свої права на здоров’я,
і що ані якнайнепристойніші житла, ані якнайгниліша
вода ніколи не будуть приводом до страйку».135

d) Вплив криз на найкраще оплачувану
частину робітничої кляси

Раніш ніж перейти до власне рільничих робітників, я хочу
ще показати на одному прикладі, як впливають кризи навіть на
найкраще оплачувану частину робітничої кляси, на її аристократію.
Пригадаймо собі ось що: 1857 рік приніс одну з тих
великих криз, що ними кожного разу завершується промисловий
цикл. Найближча криза припала на 1866 р. Ця криза, що у власне
фабричних округах була антиципована в наслідок бавовняного
голоду, який загнав чимало капіталу із звичайної сфери його
вміщення у великі центри грошового ринку, цим разом набрала
переважно фінансового характеру. Сиґналом її вибуху в травні
1866 р. було банкрутство одного з велетенських лондонських
банків, слідом за яким наступив крах безлічі спекуляційних
фінансових товариств. Однією з великих лондонських галузей
промисловости, що їх вразила катастрофа, була будова кораблів
із заліза. Маґнати цієї галузі промисловости за часів спекуляції
не лише перепродукували понад усяку міру, але, крім
того, ще й перейняли на себе контракти на величезні замовлення,
сподіваючись, що джерело кредиту й далі залишатиметься невичерпним.
Тепер же постала страшенна реакція, яка і в інших
галузях лондонської промисловости\footnote{
Там же, стор. 16.
} триває до цього часу,
кінець березня 1867 р. Для характеристики становища робітників
наведімо таке місце з докладного звіту одного кореспон-

137 «Масове голодування лондонських бідняків!(«Wholesale starvation
of the London Poor!»)\dots{} Останніми днями мури лондонських будинків
позаклеювано величезними плякатами з такою дивовижною об’явою:
«Ситі бики, зголоднілі люди! Ситі бики покинули свої кришталеві
палаци, щоб відгодувати багатіїв у їхніх розкішних світлицях, тимчасом
як зголоднілі люди гинуть і вмирають у своїх злиденних норах». Плякати
з цим зловішим написом постійно поновлюються. Ледве позривають і
позаліплюють одну партію, як одразу на тому самому або іншому прилюдному
місці знов появляється нова\dots{} Це нагадує ті передвіщання, що підготовляли
французький народ до подій 1789 р. Нині, коли англійські
робітники з своїми жінками й дітьми вмирають від холоду і з голоду,
мільйони англійських грошей, продукт англійської праці, вкладають
у російські, еспанські. італійські й інші закордонні позики». («Reynolds'
Newspaper, 20 Januar 1867»).
\index{i}{0576}  %% посилання на сторінку оригінального видання
цента «Morning Star», що з початком 1867 р. відвідав головні
центри злиднів. «У східній частині Лондону, в округах Poplar
Millwall, Greenwich, Deptford, Limehouse і Canning Town щонайменше
15.000 робітників разом із своїми родинами живуть у
якнайтяжчих злиднях, поміж ними понад 3.000 навчених механіків.
їхні резервні фонди вичерпано в наслідок шести-восьмимісячного
безробіття\dots{} Багато зусиль коштувало мені протиснутись
до воріт робітного дому (в Роріаг’і), бо його облягла зголодніла
юрба. Вона чекала на хлібні картки, але час роздавання
їх ще не настав. Подвір’я являє собою великий квардрат із
піддашшям навколо мурів. Кучугури снігу густо вкривали
кам’яний брук на середині подвір’я. Тут деякі невеличкі площі
були загороджені івовим тином, наче кошари для овець, де гарної
години працюють чоловіки. В день моїх відвідин кошари так
були позасипувані снігом, що ніхто не міг у них сидіти. Однак
під захистом підашшя чоловіки розбивали брукняк. Кожний з
них сидів на великому бруковому камені і тяжким молотом бив
по обмерзлому ґраніту. доки набивав з нього 5 бушлів. Тоді його
денна робота кінчалась, і він діставав 3 пенси і хлібну картку.
У другій частині подвір’я стояла злиденна дерев’яна хатина.
Відчинивши двері, ми побачили, що вона була повна чоловіків,
які тулилися один до одного, щоб зігрітись. Вони дерли клоччя
з корабельної линви і сперечалися між собою, хто з них при мінімумі
харчів може найдовше працювати, бо витривалість була
тут point d’honneur.* В цьому одному лише робітному домі діставало
допомогу 7.000 осіб, серед них сотні таких, що 6 або 8 місяців
тому заробляли вправною працею найвищу в цій країні
заробітну плату. Число їх було б удвоє більше, коли б не те,
що багато з них, навіть вичерпавши всі свої грошові заощадження,
все-таки не наважуються вдаватися по допомогу до парафії,
доки в них іще лишається що-будь заставляти\dots{} Покинувши робітний
дім, я пішов вулицями здебільша з одноповерховими будинками,
що їх так багато в Роріаг’і. Моїм поводирем був член
комітету безробітних. Перший дім, куди ми зайшли, був дім
залізничника, що 27 тижнів був уже без роботи. Я найшов його
з цілою його родиною в задній кімнатці. В кімнатці були ще
деякі меблі, її ще опалювали. Це конче треба було, щоб захистити
голі ноги маленьких дітей від холоду, бо день був страшенно
зимний. На тарілці проти вогню лежало клоччя, і його жінка
й діти дерли у відплату за хліб з робітного дому. Чоловік працював
в одному з вищеописаних подвір’їв за хлібну картку й
З пенси на день. Тепер він прийшов додому обідати, дуже зголоднілий,
як сказав він нам з гіркою посмішкою, а його обід
складався з кількох шматків хліба з смальцем і склянки чаю
без молока\dots{} Дальші двері, куди ми постукали, відчинила жінка
середнього віку, яка, не сказавши й слова, провела нас у малюсіньку
задню кімнатку, де мовчки сиділа ціла її родина, втупивши

— справою чести. \emph{Ред.}
\parbreak{}  %% абзац продовжується на наступній сторінці

\input{i/_0577.tex}
\index{i}{0578}  %% посилання на сторінку оригінального видання
Через те, що серед англійських капіталістів повелася мода
змальовувати Бельґію як рай для робітників, бо «воля праці»
або, що те саме, «воля капіталу» там не обмежена ні деспотизмом
тред-юньйонів, ані фабричними законами, скажімо тут кілька
слів про «щастя» бельґійського робітника [що його пригноблюють
лише духовенство, земельна аристократія, ліберальна буржуазія
і бюрократія, але аж ніяк не тред-юньйони і не фабричні закони].\footnote*{
Заведене у прямі дужки беремо з другого німецького видання. \emph{Ред.}
}
Напевно, ніхто не був глибше посвячений у всі таємниці
цього щастя, ніж небіжчик пан Дюкпетіо, головний інспектор
бельґійських в’язниць та добродійних установ і член бельгійської
центральної статистичної комісії. Загляньмо до його твору
«Budgets économiques des classes ouvrières en Belgique», Bruxelles
1855. Тут ми знаходимо, між іншим, бельгійську середню
робітничу родину, що її щорічні видатки й доходи обчислено
на основі дуже докладних даних і що її умови харчування потім
порівнюється з умовами харчування солдата, флотського матроса
й арештанта. Родина «складається з батька, матері й чотирьох
дітей». Із цих шістьох осіб «четверо можуть цілий рік займатися
корисною працею»; припускається, «що між ними немає ані
хорих, ані нездатних до праці», що вони не роблять ані «видатків
на релігійні, моральні й інтелектуальні потреби, за винятком
невеличкої оплати за місце в церкві», ані «вкладок до ощадних
кас і до кас забезпечення на старість», ані «видатків на предмети
розкошів або інших зайвих видатків». Однак батько і старший
син палять і заходять неділями до шинку, а на це їм треба цілих
86 сантимів на тиждень. «Із загального зіставлення заробітної
плати, що її дістають робітники різних галузей промисловости,
випливає... що найвища пересічна денна заробітна плата становить
1 франк 56 сантимів для чоловіків, 89 сантимів для жінок,
56 сантимів для хлопців і 55 сантимів для дівчат. За таким обчисленням
доходи родини становили б щонайбільше 1.068 франків
на рік... У родині, що її ми визнали за типову, ми підрахували
загальну суму всяких можливих доходів. Але, якщо припустимо,
що й мати одержує заробітну плату, то ми тим самим
лишаємо хатнє господарство без керівництва; хто тоді піклуватиметься
про домівку, про маленькі діти? Хто тоді варитиме, пратиме,
лататиме? Ця дилема щодня стає перед очима робітників».

Отже, бюджет родини такий:

Батько...........    300 робочих днів по 1,56 франка = 468 франків
Мати.............    300         »         »      »    0,89       »       = 267       »
Син...............    300         »         »      »    0,56       »       = 168       »
Дочка...........     300         »         »      »    0,55       »       = 165       »

                                             Разом.......... 1.068 франків

Річні видатки родини і її дефіцит становили б, коли б робітник
діставав харчі:
Флотського матроса...   1.828 франків — дефіцит 760 франків
Солдата......................   1.473       »                   »         405       »
Арештанта.................   1.112       »                   »          44         »

\index{i}{0579}  %% посилання на сторінку оригінального видання
«Ми бачимо, що мало робітничих родин може харчуватись
принаймні так, як арештанти, не кажучи вже про те, щоб харчуватись
так, як матроси або солдати. Кожний бельгійський арештант
коштував у 1847--1849~\abbr{рр.} пересічно 63 сантими на день,
що проти щоденних витрат на утримання робітника дає ріжницю
в 13 сантимів. Витрати на адміністрацію й догляд вирівнюються
тим, що арештант не платить за квартиру\dots{} Але як пояснити, що
велике число, ми могли б сказати величезна більшість, робітників
живе у ще скупіших умовах? Лише тим, що вони вживають
таких заходів, що їхня таємниця відома тільки їм самим; вони
зменшують свою щоденну порцію, їдять житній хліб замість пшеничного;
їдять менше м’яса або й зовсім його не їдять; те саме
з маслом і приправами; ціла родина тулиться в одній або двох
комірках, де дівчата й хлопці сплять разом, часто на тому самому
солом’янику; вони заощаджують на одягу, білизні, засобах підтримувати
чистоту, відмовляють собі приємностей неділями,
коротко — вони засуджують себе на якнайприкріші нестатки.
Скоро робітник дійде до цієї останньої межі, то незначне підвищення
ціни засобів існування, якась затримка в роботі, якась
недуга збільшують його злидні й цілковито руйнують його. Борги
наростають, кредит вичерпується, одяг і найдоконечніші меблі
мандрують у льомбард, і, нарешті, родина просить вписати її
у реєстр бідних».\footnote{
\emph{Ducpétiaux}: «Budgets économiques des classes ouvrières en Belgique»,
Bruxelles 1855, p. 151, 154, 155.
} Дійсно, в цьому «раю капіталістів» щонайменша
зміна ціни найдоконечніших засобів існування тягне
за собою зміну числа смертних випадків і злочинів! (Див. маніфест
Maatschappij: «De Vlamingen VooruitI» Brussel 1860,
p. 15, 16). Ціла Бельґія налічує \num{930.000} родин, з них, за офіціальною
статистикою, \num{90.000} багатих (виборців) = \num{450.000} осіб; \num{190.000}
родин дрібної середньої кляси, міської і сільської, що значна
частина її завжди попадає до лав пролетаріяту = \num{1.950.000} осіб.
Нарешті, \num{450.000} робітничих родин = \num{2.250.000} осіб, що з них
зразкові родини зазнають щастя, яке змалював Ducpétiaux. Із \num{450.000} робітничих
родин понад \num{200.000} фігурують у списку бідних!

\subsubsection{Брітанський рільничий пролетаріят}

Антагоністичний характер капіталістичної продукції й акумуляції
ніде не виявляться брутальніш, ніж у проґресі англійського
сільського господарства (включаючи і скотарство) і в
реґресі англійського сільського робітника. Раніш ніж перейдемо
до його сучасного становища, киньмо оком назад. Сучасне
рільництво в Англії веде свій початок від середини XVIII століття,
хоча переворот у відносинах земельної власности, що з
нього як бази походить змінений спосіб продукції, датується куди
ранішим часом.

\input{i/_0580.tex}
\index{i}{0581}  %% посилання на сторінку оригінального видання

Ми вже раніш відзначали становище сільських робітників
наприкінці антиякобінської війни, протягом якої так надзвичайно
позбагачувалися земельні аристократи, фармери, фабриканти,
купці, банкіри, біржові лицарі, постачальники до армії
й т.ін. Номінальна заробітна плата підвищилась почасти в наслідок
знецінення банкнот, почасти в наслідок, незалежного від
цього, зросту цін на найдоконечніші засоби існування. Але дійсний
рух заробітної плати можна сконстатувати дуже простим
способом, не вдаючись у непотрібні тут подробиці. Закон про
бідних і відповідна адміністрація були в 1814~\abbr{р.} ті самі, що і в
1795~\abbr{р.} Пригадаймо собі, як цей закон застосовувано на селі:
у формі милостині парафія доповняла номінальну заробітну
плату до номінальної суми, потрібної для простого животіння
робітника. Відношення між заробітною платою, що її платить
фармер, і тим дефіцитом її, що його поповнює парафія, показує
нам таке: поперше, наскільки заробітна плата впала нижче її
мінімуму, подруге, міру, в якій сільський робітник складався
з найманого робітника й павпера, або міру, в якій його перетворювано
на кріпака його парафії. Ми виберемо графство, яке
репрезентує пересічні умови всіх інших графств. 1795~\abbr{р.} пересічна
тижнева заробітна плата в Northamptonshire становила
7\shil{ шилінґів} 6\pens{ пенсів}, загальна сума річних видатків родини з
6 осіб — 36\pound{ фунтів стерлінґів} 12\shil{ шилінґів} 5\pens{ пенсів}, загальна сума
її доходів — 29\pound{ фунтів стерлінґів} 18\shil{ шилінґів}, дефіцит, поповнюваний
парафією, — 6\pound{ фунтів стерлінґів} 14\shil{ шилінґів} 5\pens{ пенсів.}
У тому самому графстві тижнева заробітна плата становила 1814~\abbr{р.}
12 шил. 2\pens{ пенси}, загальна сума річних видатків родини з 5 осіб —
54\pound{ фунти стерлінґів} 18\shil{ шилінґів} 4\pens{ пенси}, загальна сума її доходів
— 36\pound{ фунтів стерлінґів} 2\shil{ шилінґи}, дефіцит, поповнюваний
парафією, — 18\pound{ фунтів стерлінґів} 6\shil{ шилінґів} 4\pens{ пенси}\footnote{
\emph{Parry}: «The Question of the Necessity of the existing Cornlaws
considered», London 1816, p. 86.
}, в 1795~\abbr{р.}
дефіцит становив менш ніж четвертину заробітної плати, в
1814~\abbr{р.} — більше, ніж половину. Само собою зрозуміло, що за
таких обставин зник 1814~\abbr{р.} і той невеличкий комфорт, що його
бачив Ідн у котеджі сільського робітника\footnote{
Там же, стор. 213.
}. З усіх тварин,
що їх тримає фармер, робітник, цей instrumentum vocale\footnote*{
— говорюще знаряддя. \emph{Ред.}
}, лишився
відтепер тією, яку мучать якнайбільше, годують якнайгірше і
з  якою поводяться щонайбрутальніше.

Такий стан речей спокійно тривав далі, доки «бурхливі повстання
1830~\abbr{р.} виявили нам (тобто панівним клясам) при світлі
підпалених скирт хліба, що під поверхнею рільничої Англії
злидні й глухе бунтівниче незадоволення палають так само буйно,
як і під поверхнею промислової Англії»\footnote{
\emph{S.~Laing}: «National Distress», 1844, p. 62.
}. Седлер охристив тоді
в палаті громад сільських робітників «білими рабами» («white
slaves»); з уст якогось єпископа пролунав цей самий епітет у палаті
\index{i}{0582}  %% посилання на сторінку оригінального видання
лордів. Е.~Дж.~Векфілд, найвидатніший економіст того періоду,
каже: «Сільський робітник південної Англії і не раб
і не вільна людина, — він павпер»\footnote{
«England and America», London 1833, vol. I, p. 47.
}.

Час, що безпосередньо передував скасуванню хлібних законів,
по-новому освітлив становище сільських робітників. З одного
боку, в інтересі буржуазних агітаторів було показати, як
мало ті охоронні закони захищають дійсних продуцентів хліба.
З другого боку, промислова буржуазія кипіла гнівом з приводу
того, що земельні аристократи викривали умови фабричної роботи,
з приводу того, що ці наскрізь зіпсовані, безсердечні і
знатні нероби виявляли вдаване співчуття до страждань фабричного
робітника та «дипломатичний запал» до фабричного законодавства.
Є давня англійська приказка, що коли два злодії
чублять один одного, то з цього завжди буде якась користь.
І дійсно, галаслива, пристрасна суперечка поміж двома фракціями
панівної кляси про те, яка з них якнайбезсоромніше експлуатує
робітника, допомогла і справа і зліва вияснити правду. Граф
Шефтсбері, інакше лорд Ешлі, стояв на чолі аристократичного
філантропічного походу проти фабрик. Тим то в 1844 й
1845~\abbr{рр.} він був улюбленою темою для «Morning Chronicle», що
викривав становище рільничих робітників. Ця газета, найзначніший
тодішній ліберальний орган, надіслала до селянських округ
власних комісарів, які зовсім не задовольнилися загальним
описом і статистикою, а опублікували імена так тих робітничих
родин, що їх становище вони дослідили, як і їхніх панів-землевласників.
Нижченаведена таблиця подає заробітну плату,
яку платять у трьох селах, у сусідстві Blanford’a, Wimbourne
і Poole. Села ці — власність містера Дж.~Бенкса і графа Шефтсбері.
Треба зауважити, що цей папа «low church»\footnote*{
— низької церкви. \emph{Ред.}
}, цей голова англійських
пієтистів, так само як і згаданий Бенкс, із злиденної
заробітної плати робітників відбирав ще в них значну частину
під приводом плати за квартиру.

\begin{small}
  \settowidth\rotheadsize{Тижневий дохід}
\begin{longtable}{ccc@{~}cc*{4}{c@{~}c}}
  
  \toprule

  \rotcell{а) Дітей} &
    \rotcell{b) Членів родин} &
    \multicolumn{2}{l}{
      \rotatebox[origin=c]{90}{\parbox[l]{\rotheadsize}{\raggedright с) Тижнева зароб. плата чоловіків}}
    } &
    \rotcell{
       d) Тижнева \\ заробітна \\ плата дітей
    } & 
    \multicolumn{2}{l}{
      \rotatebox[origin=c]{90}{\parbox[l]{\rotheadsize}{е) Тижневий дохід цілої родини }}
    } &
    \multicolumn{2}{l}{
      \rotatebox[origin=c]{90}{\parbox[l]{\rotheadsize}{f) Тижнева квартирна плата }}
    } &
    \multicolumn{2}{l}{
      \rotatebox[origin=c]{90}{\parbox[l]{\rotheadsize}{g) Загальний тижневий заробіток з відрахуванням квартирної плати }}
    } &
    \multicolumn{2}{l}{
      \rotatebox[origin=c]{90}{\parbox[l]{\rotheadsize}{h) Тижневий заробіток на людину }}
    } 
    \\  
  %   \makevertcell{ \\
%           
%             


  \addlinespace
    \multicolumn{13}{c}{Перше село} \\

  & & 
    ш. & п. &
    &
    ш. & п. &
    ш. & п. &
    ш. & п. &
    ш. & п. \\

  2 & 4 &
    8 & 0 & \emptycell{} &
    8 & 0 &
    2 & 0 &
    6 & 0 &
    1 & 6\phantom{\sfrac{1}{3}} \\

  3 & 5 &
    8 & 0 & \emptycell{} &
    8 & 0 &
    1 & 6 &
    6 & 6 &
    1 & 3\sfrac{1}{3} \\

  2 & 4 &
    8 & 0 & \emptycell{} &
    8 & 0 &
    1 & 0 &
    7 & 0 &
    1 & 9\phantom{\sfrac{1}{3}} \\

  2 & 4 &
    8 & 0 & \emptycell{} &
    8 & 0 &
    1 & 0 &
    7 & 0 &
    1 & 9\phantom{\sfrac{1}{3}} \\
  
  6 & 8 &
    7 & 0 & 1\textendash{}1 ш. 6 п. &
    10 & 6 &
    2 & 0 &
    8 & 6 &
    1 & 0\sfrac{1}{4} \\

  3 & 5 &
    7 & 0 & 1\textendash{}2 ш. 0 п. &
    7 & 6 &
    1 & 4 &
    5 & 8 &
    1 & 1\sfrac{1}{2} \\

  \addlinespace
    \multicolumn{13}{c}{Друге село} \\

  6 & 8 &
    7 & 0 & 1\textendash{}1 ш. 6 п. &
    10 & 0 &
    1 & 6\phantom{\sfrac{1}{2}} &
    8 & 6\phantom{\sfrac{1}{2}} &
    1 & \phantom{0}0\sfrac{3}{4} \\

  6 & 9 &
    7 & 0 & 1\textendash{}1 ш. 6 п. &
    7 & 0 &
    1 & 3\sfrac{1}{2} &
    5 & 8\sfrac{1}{2} &
    0 & \phantom{0}8\sfrac{1}{2} \\

  8 & \hang{r}{1}0 &
    7 & 0 & \emptycell{} &
    7 & 0 &
    1 & 3\sfrac{1}{2} &
    5 & 8\sfrac{1}{2} &
    0 & \phantom{0}7\phantom{\sfrac{1}{2}} \\

  4 & 6 &
    7 & 0 & \emptycell{} &
    7 & 0 &
    1 & 6\sfrac{1}{2} &
    5 & 5\sfrac{1}{2} &
    0 & 11\phantom{\sfrac{1}{2}} \\

  3 & 5 &
    7 & 0 & \emptycell{} &
    7 & 0 &
    1 & 6\sfrac{1}{2} &
    5 & 5\sfrac{1}{2} &
    1 & \phantom{0}1\phantom{\sfrac{1}{2}} \\

  \addlinespace
    \multicolumn{13}{c}{Третє село} \\

  4 & 6 &
    7 & 0 & \emptycell{} &
    7 & 0 &
    1 & \phantom{0}0 &
    6 & 0 &
    1 & 0\phantom{\sfrac{1}{2}} \\

  3 & 5 &
    7 & 0 & 1\textendash{}2 ш. \phantom{0}0 п. &
    \hang{r}{1}1 & 6 &
    0 & 10 &
    \hang{r}{1}0 & 8 &
    2 & 1\sfrac{1}{2} \\

  0 & 2 &
    5 & 0 & 1\textendash{}2 ш. 16 п. &
    5 & 0 &
    1 & \phantom{0}0 &
    4 & 0 &
    2 & 0\hang{l}{\footnotemark{}}\phantom{\sfrac{1}{2}} \\
\end{longtable}
\end{small}
\footnotetext{«London Economist», 29 березня 1845~\abbr{р.}, р. 290.}
\index{i}{0583}  %% посилання на сторінку оригінального видання

\noindent{}Скасування збіжжевих законів дало надзвичайний поштовх
англійському рільництву. Дренажні роботи якнайбільшого маштабу\footnote{
Земельна аристократія сама авансувала собі для цієї мети фонди
з державної скарбниці, звичайно через парлямент, за дуже низький
процент, що його фармери мали виплачувати їй удвоє.
},
нова система годівлі худоби в стайнях і засіву штучних
кормових трав, заведення механічних апаратів до угноювання,
нові способи обробляти глинкуватий ґрунт, збільшене вживання
мінерального добрива, застосування парової машини й
усякого роду нових робочих машин і~\abbr{т. д.}, взагалі інтенсивніша
культура — ось що характеризує цю епоху. Пан Пезі, президент
королівського рільничого товариства, твердить, що (відносні)
господарські витрати через заведення нових машин зменшились
майже удвоє. З другого боку, швидко збільшився позитивний
дохід від землі. Основною умовою нових метод були збільшені
капіталовкладення на акр, отже, і прискорена концентрація
фарм\footnote{
Зменшення числа середніх фармерів ясно видно з рубрик перепису:
«Син фармера, онук, брат, племінник, дочка, онучка, сестра, племінниця»,
одно слово, члени родини самого фармера, що працюють у
нього. Ці рубрики налічували 1851~\abbr{р.} \num{216.851} особу, 1861~\abbr{р.} лише \num{176.151}
особу. Від 1851 до 1871~\abbr{р.} число фарм, менших від 20 акрів, зменшилось
більш, ніж на 900, число фарм від 50 до 77 акрів спало з \num{8.253} до \num{6.370};
те саме й з усіма іншими фермами, меншими від 100 акрів. Навпаки,
протягом тих самих двадцятьох років число великих фарм збільшилось;
число фарм від 300 до 500 акрів збільшилось з \num{7.771} до \num{8.410}, число
фарм, більших за 500 акрів, — з \num{2.755} до \num{3.914}, число фарм, більших
за 1000 акрів, — з 492 до 582.
}. Одночасно й засівна площа збільшилась від 1846~\abbr{р.}
до 1856~\abbr{р.} на \num{464.119} акрів, не кажучи вже про величезні площі
у східніх графствах, які з кролячих загородок і злиденних пасовиськ
немов чарами перетворено в буйні збіжжеві лани. Ми вже
знаємо, що одночасно з цим зменшилось загальне число осіб,
занятих у рільництві. Щождо власне рільників обох статей і
\parbreak{}  %% абзац продовжується на наступній сторінці

\parcont{}  %% абзац починається на попередній сторінці
\index{i}{0584}  %% посилання на сторінку оригінального видання
різного віку, то число їх зменшилося з \num{1.241.269} в 1851~\abbr{р.} до
\num{1.163.217} в 1861~\abbr{р.}\footnote{
Число пастухів овець збільшилося з \num{12.517} до \num{25.559}.
} Тим-то, коли генеральний англійський
реєстратор справедливо зауважує: «Приріст числа фармерів і
сільських робітників од 1801~\abbr{р.} не стоїть ні в якій пропорції
до приросту рільничого продукту»,\footnote{
«Census etc.», vol. II, р. 36.
} то ця диспропорція ще
куди більше стосується до останнього періоду, коли позитивне
зменшення сільської робітничої людности відбувалось рівнобіжно
із збільшенням оброблюваної площі, інтенсифікацією культури,
нечуваною акумуляцією капіталу, вкладеного в землю і призначеного
для її оброблювання, нечуваним в історії англійської
аґрономії зростанням кількости рільничого продукту, буйним
підвищенням ренти землевласників і надзвичайним зростанням
багатства капіталістичних фармерів. Коли до цього всього додати
ще безперервне швидке поширення міських ринків збуту й
панування вільної торговлі, то сільського робітника post tot
discrimina rerum,\footnote*{
— після стількох різних пригод. \emph{Ред.}
} нарешті, поставлено в такі умови, які secundum
artem\footnote*{
— згідно з теорією. \emph{Ред.}
} мусіли були зробити його безмежно щасливим.

Навпаки, професор Роджерс доходить такого висновку, що
становище сільського робітника за наших часів надзвичайно
погіршало не лише порівняно з становищем його попередника
в останній половині XIV століття і в XV столітті — про це й
казати годі — але й порівняно з становищем його попередника
періоду від 1770~\abbr{р.} до 1780~\abbr{р.}, що «він знов став кріпаком», і
до того ж кріпаком погано годованим і погано забезпеченим щодо
житла.\footnote{
\emph{Rogers}: «A History of Agriculture and Prices in England», Oxford
1866, vol. I, p. 693, «The peasant has again become a serf». Там же, стор. 10.
П. Роджерс належить до ліберальної школи, він особистий приятель
Кобдена і Брайта, отже, зовсім не laudator temporis acti.\footnote*{
— хвалій минулих часів. \emph{Ред.}
}
} Д-р Джуліян Гентер у своєму епохальному звіті про
житлові умови сільських робітників каже: «Кошти існування
hind’a (назва сільського робітника за часів кріпацтва) фіксовано
в розмірі якнайнижчої суми, що на неї він може жити\dots{} його
заробітна плата й притулок не стоять ні в якій пропорції до зиску,
що мають видушити з нього. Він є нуль у рахунках фармера.\footnote{
«Public Health. Seventh Report», London 1865, p. 242. «The cost of
the hind is fiked at the lowest possible amount on which he can live\dots{} the
supplies of wages or shelter are not calculated on the profit to be derived
from him. He is a zero in farming calculations». Тим-то немає нічого незвичайного
в тім, що або домовласник збільшує квартирну плату для робітника,
коли він почує, що робітник заробляє дещо більше, або фармер
зменшує робітникові заробітну плату, «бо його жінка найшла собі
роботу». (Там же).
}
Засоби його існування завжди розглядають як сталу величину»,\footnote{
Там же, стор. 135.
}
«Щождо дальшого зменшування його доходу, то він може сказати:
nihil habeo, nihil curo.\footnote*{
Нічого не маю, ні про що не дбаю. \emph{Ред.}
} Йому будучина не страшна, бо
\parbreak{}  %% абзац продовжується на наступній сторінці

\parcont{}  %% абзац починається на попередній сторінці
\index{i}{0585}  %% посилання на сторінку оригінального видання
він нічого не має, окрім того, що абсолютно доконечне для його
існування. Він дійшов точки замерзання, що з неї, як із бази,
виходять усі обрахунки фармера. Хай буде що буде, хоч щастя,
хоч нещастя, його це не торкається»\footnote{
Там же, стор. 134.
}.

В 1863~\abbr{р.} проведено офіціяльне дослідження умов харчування
і праці злочинців, засуджених на заслання й на громадські примусові
роботи. Результати його подано у двох товстих синіх
книгах. «Старанне порівняння, — читаємо там, між іншим, —
харчу злочинців по в’язницях Англії із харчами павперів по
робітних домах і вільних сільських робітників цієї ж країни
доводить безперечно, що перші дістають далеко кращі харчі,
ніж будь-яка з тих двох інших кляс»\footnote{
«Report of the Commissioners\dots{} relating to Transpotration and
Penal Servitude», London 1863, p. 42, 50.
}, тимчасом як «маса
праці, якої вимагають від засуджених на громадські примусові
роботи, становить приблизно половину того, що виконує звичайний
сільський робітник»\footnote{
Там же, стор. 77. «Memorandum by the Lord Chief Justice».
}. Ось деякі характеристичні свідчення
свідків. Джон Сміс, директор едінбурзької в’язниці, свідчить.
№ 5056: «Харчі по англійських в’язницях далеко кращі, ніж
харчі звичайних сільських робітників». № 5075: «Це факт, що
звичайні рільничі робітники Шотляндії дуже рідко дістають якебудь
м’ясо». № 3047: «Чи не знаєте ви якихбудь підстав для
того, що злочинців треба годувати далеко краще (much better),
ніж звичайних сільських робітників? — Звичайно, не знаю».
№ 3048: «Чи не вважаєте ви за доцільне робити дальші експерименти,
щоб харчі арештантів, засуджених на громадські примусові
роботи, наблизити до харчів вільних сільських робітників»?\footnote{
Там же, том II, «Evidence».
} «Сільський робітник, — читаємо там, — міг би сказати:
«Я тяжко працюю й не дістаю досить їсти. Коли я був у
в’язниці, я працював не так тяжко, а харчів мав удостачу, і тому
мені краще бути у в’язниці, ніж на волі»\footnote{
Там же, том I, Appendix, стор. 280.
}. Із таблиць, доданих
до першого тому звіту, складено таке порівняльне зведення.

\begin{center}
\captionnew{Тижнева кількість харчів{\mdseries\footnoteA{Там же, стор. 274, 275.}}}
\bigskip
\noindent\begin{tabularx}{\textwidth}{@{}Xrrrr@{}}
  \toprule
  &
  \makecell[r]{Азотові \\  складові \\ частини \\ (унцій)} &
  \makecell[r]{Безазотові \\ складові \\ частини \\ (унцій)} &
  \makecell[r]{Мінеральні \\ складові \\ частини \\ (унцій)} &
  \makecell[r]{~ \\ Загальна \\ сума \\ (унцій)}\\
  \cmidrule{2-5}

  \makehangcell{Злочинець у портлендській в'язниці\dotfill{}} 
    & 28,95 & 150,06 & 4,68 & 183,69 \\
  Матрос королівської фльоти& 29,63 & 152,91 & 4,52 & 187,06  \\
  Солдат\dotfill{}          & 25,55 & 114,49 & 3,94 & 143,98\\
  Каретник (робітник)\dotfill{}  & 24,53 & 162,06 & 4,23 & 190,82\\
  Складач\dotfill{}          & 21,24 & 100,83 & 3,12 & 125,19\\
  Сільський робітник\dotfill{}  & 17,73 & 118,06 & 3,29 & 139,08 \\

\end{tabularx}
\end{center}


\index{i}{0586}  %% посилання на сторінку оригінального видання
\noindent{}Загальний результат досліджень лікарської комісії 1863~\abbr{р.}
про стан харчування гірше харчованих кляс народу вже відомий
читачеві. Читач пригадує собі, що харчі більшої частини родин
сільських робітників стоять нижче мінімальної міри, потрібної,
«щоб забезпечити себе від хороб у наслідок голоду». Так стоїть
справа особливо по всіх суто рільничих округах — Cornwall, Devon
Somerset, Witts, Stafford, Oxford, Berks і Herts. «Кількість
харчів, що їх дістає сільський робітник, — каже д-р Сімон, —
більша, ніж показує пересічна кількість, бо сам він дістає куди
більшу частину засобів існування, доконечну для його праці,
аніж інші члени його родини; в найбідніших округах він дістає
майже все м’ясо або сало. Та кількість харчів, що припадає на
жінку, а так само й на дітей у період їхнього швидкого зросту,
в багатьох випадках і майже по всіх графствах недостатня, особливо
щодо азоту»\footnote{
«Public Health. Sixth Report 1863», p. 238, 249, 261, 262.
}. Наймити й наймички, що живуть у самих
фармерів, харчуються добре. Число їх із \num{288.277} в 1851~\abbr{р.} спало
до \num{204.962} в 1861~\abbr{р.} «Праця жінок на полі, — каже д-р Сміс, —
хоч би й якими взагалі шкідливими наслідками вона супроводилася,
за сучасних обставин є дуже корисна для родини, бо дає
їй засоби на взуття, одяг, квартирну плату, і таким чином змогу
краще харчуватись»\footnote{
Там же, стор. 262.
}. Одним із найвизначніших результатів
цього дослідження було виявлення того факту, що сільський
робітник в Англії харчується куди гірше, ніж в інших частинах
Сполученого королівства («is considerably the worst fed»), як
це видно з нижченаведеної таблиці.


\begin{table}[H]
\centering
\caption*{Тижневе споживання вуглецю й азоту пересічно \\ на одного сільського робітника}
\noindent\begin{tabular}{lrr}
\toprule
& \makecell[r]{Вуглецю\\(грани)} & \makecell[r]{Азоту\\(грани)} \\
\cmidrule{2-3}
Англія\dotfill{} & 40,673 & 1,594 \\
Велз\dotfill{} & 48,354 & 2,031 \\
Шотляндія\dotfill{} & 48,980 & 2,348 \\
Ірляндія\dotfill{} & 43,366 & 2,439\hang{l}{\footnotemark{}}\\
\end{tabular}
\end{table}

\footnotetext{
Там же, стор. 17. Англійський сільський робітник дістає лише
четвертину тієї кількости молока й лише половину тієї кількости хліба,
яку дістає ірляндський сільський робітник. Кращі умови харчування
ірландського сільського робітника відзначив уже А.~Юнґ на початку
цього століття в своїй «Tour through Ireland». Причина цього та, що бідний
ірляндський фармер куди гуманніший, ніж багатий англійський.
А щодо Велзу, то наведені в тексті дані не стосуються до його південнозахідньої
частини. «Всі тамошні лікарі згоджуються на тому, що збільшення
проценту смертности від туберкульози, золотухи й~\abbr{т. ін.} інтенсивно
вростає з погіршанням фізичного стану людности, і це погіршання всі
приписують злидням. Денне утримання сільського робітника обчислюють
там у 5\pens{ пенсів}, у багатьох округах фармер (що й сам бідує) платить іще
менше. Шматок засоленого м’яса, сухий, як тверде червоне дерево, і ледве
чи вартий важкого процесу травлення, або шматок сала є за приправу до
великої кількости юшки з борошна й цибулі або до вівсянки, і це день-у-день
становить обід сільського робітника\dots{} Проґрес промисловости
для нього мав такі наслідки, що дебеле домотканне сукно витиснено в
цьому суворому й вогкому підсонні дешевими бавовняними тканинами,
а міцніші напої — «номінальним» чаєм\dots{} Після багатьох годин перебування
на вітрі й дощі рільник вертається до свого котеджу, щоб присісти
біля печі, де горить торф або кавалки, збиті з глини й покидьків кам’яного
вугілля, що, згораючи, виділюють цілі хмари вуглекислоти й сульфатної
кислоти. Стіни хатини пороблено з глини й каменю, долівка —
гола земля, що була тут і перед будуванням хатини, дах — маса понакидуваної,
непошитої соломи. Кожну щілину заткнуто, щоб не виходило
тепло, і в цій атмосфері диявольського смороду, на брудній землі, часто
висушуючи на своєму тілі свою однісіньку одіж, він сідає вечеряти а
дружиною й дітьми. Акушери, примушені проводити частину ночі в цих
хатах, описували, як їхні ноги грузли у брудній земляній долівці і як
їм доводилося — легенька собі справа! — продовбувати дірку в стіні, щоб
здобути собі хоч трохи свіжого повітря. Численні свідки різного ранґу
свідчать, що недосить харчований (underfed) селянин кожної ночі зазнає
цих і інших шкідливих для його здоров’я впливів; результат цього —
квола й золотушна людність, про це справді маємо більш ніж досить
доказів\dots{} Повідомлення парафіяльних урядовців у Caermarthenshire і Cardiganshire
виразно потверджують такий самий стан речей. Сюди треба
додати ще більше лихо — поширення ідіотизму. А тепер ще декілька слів
про кліматичні умови. Буйні південно-західні вітри пронизують усю
країну вісім-дев’ять місяців на рік, вони навівають страшні зливи,
що спадають переважно на західніх схилах гір. Дерева трапляються
рідко, хіба лише по затишних місцях; там, де вони не захищені, їх нищить
вітер. Хатини туляться під гірськими терасами, часто по ярах або
каменярнях; лише найдрібніша порода овець і місцева рогата худоба
можуть жити на таких пасовиськах\dots{} Молодь еміґрує до східніх гірничих
округ Glamorgan і Monmouth\dots{} Caermarthenshire — це розсадник
шахтарів і їхній інвалідний дім\dots{} Людність ледве-ледве підтримує
свою чисельність на тому самому рівні. Так, у Cardiganshire було:

\begin{center}
\noindent\begin{tabular}{lrr}
& \emph{1851~\abbr{р.}} & \emph{1861~\abbr{р.}} \\
Чоловічої статі\dotfill{}& \num{45.155} & \num{44.446} \\
Жіночої статі & \num{22.459} & \num{52.955} \\
\cmidrule(r){2-3}
& \num{97.614} & \num{97.401}
\end{tabular}
\end{center}
(Звіт д-ра Гентера в «Public Health. Seventh Report 1864», London
1865, p. 498--502 passim.).
}

\index{i}{0587}  %% посилання на сторінку оригінального видання
\noindent{}«Кожна сторінка звіту д-ра Гентера, — каже д-р Сімон
у своєму офіціяльному санітарному звіті, — свідчить про недостатню
кількість і злиденну якість помешкань нашого сільського
робітника. І ось уже багато років, як стан його з цього боку проґресивно
гіршає. Тепер сільському робітникові куди важче підшукати
помешкання, а коли й підшукає, то воно куди менше
відповідає його потребам, аніж це було, може, декілька століть
тому\dots{} Особливо швидко зростає це лихо протягом останніх 30
або 20 років, і житлові умови селянина тепер надзвичайно сумні.
Він тут цілком безпорадний, хібащо ті, кого він збагачує своєю
працею, захочуть завдати собі клопоту поводитися з ним із певного
роду жалісливістю та ласкавістю. Чи найде він житло на
тій землі, яку обробляє, чи буде те житло придатне для людей
чи лише для свиней, чи буде при ньому невеличкий садок, що так
полегшує гніт злиднів, — усе це залежить не від його готовости
\parbreak{}  %% абзац продовжується на наступній сторінці

\parcont{}  %% абзац починається на попередній сторінці 
\index{i}{0588}  %% посилання на сторінку оригінального видання 
або спромоги платити відповідну квартирну плату, а від того
вжитку, який будуть ласкаві зробити інші «із свого права порядкувати
своєю власністю так, як їм забажається». Хоч би яка
велика була фарма, немає такого закону, що вимагав би збудувати
на ній певну кількість помешкань для робітників, не кажучи
вже зовсім про пристойність цих помешкань; так само
закон не дає робітникові найменшого права на ту землю, для якої
його праця так само доконечна, як дощ і сонячне світло... Ще
одна загальновідома обставина кидає на терези тяжку вагу проти
нього... Це — вплив закону про бідних з його постановами про
оселення й податок на користь бідним.\footnote{
1865 р. цей закон дещо поліпшено. Досвід незабаром нам покаже,
що така латанина ані трішечки не допомагає.
} Під його впливом кожна
парафія має грошовий інтерес у тому, щоб обмежити на мінімумі
число сільських робітників, які живуть у ній, бо, на нещастя,
рільнича праця, замість ґарантувати сільському робітникові,
що тяжко працює, та його родині певну й постійну незалежність,
веде його здебільша, довшим або коротшим обхідним
шляхом, до павперизму, — павперизму, до якого протягом цілого
того шляху робітник так стоїть близько, що всяка хороба
або якийсь тимчасовий брак праці примушують його одразу звертатися
по допомогу до парафії; і тим то кожне оселення рільничої
людности в якійсь парафії, очевидно, є для неї збільшення
податку на користь бідним. Великим землевласникам 168 досить
лише вирішити, що в їхніх маєтках не повинно бути жител для
робітників, — і вони одразу звільняються від половини своєї
відповідальности за бідних... У якій мірі англійська конституція
й закони мали за мету встановити такого роду безумовну
земельну власність, що дає лендлордові силу «робити з своєю
власністю що йому забажається», поводитися з рільниками
як із чужоземцями і проганяти їх із своєї території, — це питання,
що обговорення його не входить у рамки моїх завдань...
Це право виганяти — не просто теорія. Воно реалізується
на практиці в якнайбільшому маштабі. Це одна з обставин, що
мають вирішальний вплив на житлові умови сільського робітника...
Про розміри лиха можна судити на основі останнього
перепису, який показав, що протягом останніх 10 років руйнування
домів, не зважаючи на дедалі більший місцевий попит на
них, проґресувалй у 821 різних округах Англії; таким чином в
1861 р. людність, яка проти 1851 р. зросла на 5 1/3\%, при чому
ми зовсім не беремо на увагу осіб, що примушені жити не в тих
парафіях, де вони працюють, — позбивано в помешкання, площа
яких зменшилась на 4 1/2\%... Скоро тільки процес вилюднення
завершується, — каже д-р Гентер, — у результаті його постає
показне село (showvillage), де число котеджів зведено до незнач-

163 Щоб зрозуміти дальшу цитату, зазначимо, що close villages (закритими
селами) називають такі села, що їхні власники один
або два великі лендлорди, a open villages (відкритими селами) — такі,
що їхні землі належать багатьом дрібним власникам. Саме в цих останніх
будівельні спекулянти і можуть будувати котеджі й хати.
\index{i}{0589}  %% посилання на сторінку оригінального видання 
ної кількости й де ніхто не сміє жити, крім чабанів, садівників
і сторожів дичини, — цих постійних слуг, з якими шановне панство
поводиться з ласкою, звичайною для цієї кляси слуг.\footnote{
Таке показне село має дуже принадний вигляд, але воно таке
саме нереальне, як і села, що їх бачила Катерина II підчас своєї подорожі
до Криму. Останніми часами навіть і чабана часто виганяють з цих showvillages.
Напр., біля Market Harborough є пасовисько для овець, що
займає майже 500 акрів, де потрібна праця лише однієї людини. Для скорочення
далеких переходів цими просторими рівнинами, гарними пасовиськами
Leicester’a й Northampton’a, чабанові звичайно давали котедж
на фармі. Тепер йому дають тринадцятий шилінґ на помешкання, що його
він мусить собі шукати далеко у відкритому селі.
}
дле земля потребує оброблення, і ми бачимо, що робітники,
заняті на ній, живуть не в земельного власника, а приходять з
відкритого села, що лежить, може, на віддалі трьох миль, де
їх пустили до себе численні дрібні домовласники після того, як
по закритих селах зруйнували котеджі робітників. Там, де справи
наближаються до цього результату, котеджі своїм злиденним
виглядом звичайно свідчать про долю, що на неї вони засуджені.
Вони перебувають на різних щаблях природної руїни. Поки
дах тримається, робітникові дозволяють платити за помешкання
ренту, і він часто дуже радий, що йому це дозволяють, хоча б
йому доводилося платити таку ціну, як і за добре помешкання.
Але жодного ремонту, жодних полагоджень, крім таких, що їх
зможе зробити сам пребідний мешканець. Коли ж котедж стає,
нарешті, цілком непридатний для житла, то це означає лише,
що число зруйнованих котеджів збільшилось на один і остільки
менше доведеться надалі платити податку на бідних. Тимчасом
як великі землевласники таким способом звалюють із себе податок
на користь бідним через вилюднення належної їм землі, найближче
містечко або відкрите село приймає до себе викинутих
робітників; я кажу найближче, але оце «найближче» може лежати
три-чотири милі від фарми, де робітник має день-удень
тяжко працювати. Таким чином, до його денної праці, так, наче
це була б дурничка, долучається потреба день-у-день проходити
7—8 миль, щоб заробити собі на щоденний хліб. Всі сільські
роботи, що їх виконують його дружина й діти, відбуваються
тепер серед таких самих важких обставин. Але й це ще не все
лихо, що його спричиняє йому віддаленість від місця роботи.
В одкритому селі будівельні спекулянти скуповують шматки землі
і якомога густіше забудовують їх найдешевшими халупами
всякого роду. І в таких злиденних житлах, які, навіть тоді,
коли вони виходять на чисте поле, мають найжахливіші характеристичні
риси найпоганіших міських жител, туляться рільничі
робітники Англії...\footnote{
«Доми робітників (у відкритих селах, які, звісно, завжди переповнені)
звичайно збудовані рядами, задніми стінами до крайньої лінії
того шматка землі, що його будівельний спекулянт називає своїм. Через
те світло й повітря може проходити до них лише з фасаду». (Звіт д-ра
Гентера в «Public Health. Seventh Report 1864», London 1865, p. 135).
Дуже часто власник пивниці або сільський крамар здає також у найми
} З другого боку, не треба собі уявляти,
\index{i}{0590}  %% посилання на сторінку оригінального видання 
що робітник, який живе навіть на оброблюваній ним землі, знаходить
собі таке помешкання, що на нього заслуговує його продуктивне
життя. Навіть у князівських маєтках робітничі котеджі
часто мають якнайзлиденніший характер, Є лендлорди, що вважають
і стайню за досить добре житло для своїх робітників
і їхніх родин, а проте не соромляться видушувати якнайбільше
грошей з винаймання таких помешкань.\footnote{
Винаймач помешкання (фармер або лендлорд) безпосередньо або
посередньо збагачується з праці людини, що їй він платить 10 шилінґів
на тиждень, а потім знову відбирає в цього бідолахи 4 або 5 фунтів
стерлінґів річної квартирної плати за хати, що на вільному ринку не
варті й 20 фунтів стерлінґів, але зберігають свою штучну ціну, бо власник
має силу сказати: «Бери мою хату, або йди геть звідси і шукай собі,
не мавши від мене атестації, якесь інше пристановище»... Коли людина
хоче поліпшити своє становище й піде на залізницю укладати шини або
на каменярню, то та сама сила знову каже йому: «Працюй в мене за що
низьку плату, або йди геть за тиждень після попередження; забирай свою
свиню, коли вона в тебе є, і поміркуй про те, що ти дістанеш за картоплю,
яка росте на твоєму городі». Коли ж проганяти робітника не в інтересах
власника (або фармера), то він у таких випадках іноді вважає за краще
підвищити квартирну плату, щоб покарати робітника, за те, що він
покинув у нього служити» (Д-р Гентер, там же, стор. 132).
} Хай це буде лише
напівзавалена халупа з однією кімнатою для спання, без печі,
без кльозета, з вікнами, що не відчиняються, без водопостачання,
крім якогось рівчака, без садка, — робітник безпорадний проти
такої несправедливости. А наші санітарно-поліційні закони
(The Nuisances Removal Acts) — це мертва буква. Бо ж проводити
їх доручено тим саме власникам, що здають у найми
такі діри... Виняткові веселіші картини не повинні засліплювати
нас та закривати перед нами величезну силу фактів, що є ганьба
англійської цивілізації. Дійсно, жахний мусить бути стан речей,
коли, не зважаючи на очевидну потворність теперішніх
помешкань, компетентні спостерігачі одноголосно доходять такого
висновку, що навіть ці повсюдно нікчемні помешкання є ще
безмірно менше лихо, ніж просто кількісний брак помешкань.
Вже віддавна переповнення помешкань сільських робітників

і помешкання. Тоді поряд із фармером він є другий пан для сільського
робітника. Останній мусить бути одночасно і його покупцем. «З 10 шилінґами
на тиждень мінус 4 фунти квартирної плати на рік він зобов’язаний
купувати чай, цукор, борошно, мило, свічки й пиво по цінах,
які сподобається визначити крамареві» (Там же, стор. 134). Ці відкриті
села — це дійсно «карні колонії» англійського рільничого пролетаріату.
Багато з цих котеджів — це чисті постоялі двори, через які проходить
уся бродяча наволоч з околиці. Селянин і його родина, що серед,
найбрудніших обставин часто справді навдивовижу зберегли путящість
і чистоту характеру, тут цілком гинуть. Серед знатних Шейлоків це,
звичайно, мода по-фарисейському знизувати плечима на адресу будівельних
спекулянтів, дрібних власників і відкритих сел. Вони дуже
добре знають, що їхні «закриті села й показні села» є місце народження
«відкритих сел» і не могли б існувати без цих останніх. «Без дрібних
власників одкритих сел найбільша частина сільських робітників мусіла б
спати під деревами тих маєтків, де вони працюють» (Там же, стор. 135).
Система «відкритих» і «закритих» сел панує по всій середній і всій
східній Англії.
\parbreak{}  %% абзац продовжується на наступній сторінці

\parcont{}  %% абзац починається на попередній сторінці
\index{i}{0591}  %% посилання на сторінку оригінального видання
становило предмет глибоких турбот не лише для осіб, що піклуються
про здоров’я, але й для всіх взагалі, що цінять пристойне
й моральне життя. Бо автори звітів про поширення пошесних
недуг у сільських округах кажуть знову й знову, в одноманітних
висловах, — що, здається, поробилися вже стереотипними, — про
переповнення помешкань, як про причину, що цілком зводить
нанівець усі спроби спинити розвиток пошести, скоро вона вже
з’явилась. Так само знову й знову доводили вони, що, хоч сільське
життя і добре впливає на здоров’я, скупченість людности,
що так дуже прискорює поширення пошесних недуг, сприяє
також і поширенню незаразливих недуг. І особи, які зазначали
такий стан речей, не замовчували й дальшого лиха. Навіть у тих
випадках, коли вони у своїй первісній темі торкались тільки
гігієни, вони були майже примушені звернути увагу й на інший
бік справи. Їхні звіти, що зазначають те, як часто трапляється, що
дорослі люди обох статей, жонаті й нежонаті, позбивані до купи
(huddled) у тісних спальнях, мусили викликати переконання,
що серед описаних обставин якнайгрубішим способом порушується
почуття сорому та пристойности і майже неминуче руйнується
всяку моральність\dots{}\footnote{
«Молоде подружжя не являє собою повчального прикладу для
дорослих братів і сестер, що сплять у тій самій кімнат: і хоч не можна
зареєструвати таких прикладів, однак є досить даних, які доводять правдивість
того твердження, що великих страждань, а часто й смерти зазнають
жінки, які допустились кровозмішення» (Д-р Гентер, там же, стор. 137).
Один сільський поліцайський урядовець, що протягом багатьох років
був розшукачем у найгірших кварталах Лондону, каже про дівчат свого
села: «Такої грубої аморальности з молодого віку, такого нахабства
та безсоромности я ніколи не бачив протягом усього того часу, що я був
поліцаєм у найгірших частинах Лондону\dots{} Вони живуть як свині, дорослі
парубки й дівчата, матері й батьки, усі сплять укупі в одній кімнаті»
(«Children’s Employment Commission. 6th Report», London 1867, Appendix,
p. 77, n. 155
} Напр., у додатку до мого останнього звіту
д-р Орд у своєму звіті про вибух пропасниці у Wing’y і в Виcidnghamshir’i
згадує про те, як туди прийшов із Wingrave один
парубок, недужий на пропасницю. Перші дні своєї хороби він
спав в одній кімнаті з дев’ятьма іншими особами. За два тижні
деякі з цих осіб теж захоріли, за кілька тижнів на пропасницю
занедужало 5 із цих 9 осіб, а одна померла! Одночасно
д-р Гарвей, лікар шпиталю St.~Georges, що за часів епідемії відвідував
Wing з приводу приватної практики, звітує в тому самому
дусі: «Одна молодиця, хора на пропасницю, спала вночі
в тій самій кімнаті з батьком, матір’ю, своєю нешлюбною дитиною,
двома парубками, її братами, і двома сестрами, що з них
кожна також мала нешлюбну дитину, разом 10 осіб. Декілька
тижнів раніш у тій самій кімнаті спало 13 дітей»\footnote{
«Pubic Health. Seventh Report: 1867», p. 9--14 passim.
}.

Д-р Гентер обслідував \num{5.375} котеджів сільських робітників,
не лише в суто рільничих округах, алей по всіх графствах
Англії. З цих \num{5.375} котеджів \num{2.195} мали лише по одній спальні
(часто це разом з тим і мешкальна кімната), \num{2.930} — лише по дві
\parbreak{}  %% абзац продовжується на наступній сторінці

\parcont{}  %% абзац починається на попередній сторінці
\index{i}{0592}  %% посилання на сторінку оригінального видання
і 250 — більше ніж дві. Я подаю тут декілька прикладів становища
в тузені цих графств.

\paragraph{Bedfordshire}

Wrestlingworth: Спальні біля 12 футів завдовжки й 10 футів
завширшки, хоч є багато й менших. Маленька одноповерхова
хатина часто буває розділена перегородкою з дощок на дві
спальні, одне ліжко часто стоїть у кухні в 5 футів 6 цалів заввишки.
Квартирна плата — 3\pound{ фунти стерлінґів}. Мешканці мають
сами будувати для себе кльозети, власник будинку дає лише
яму. Скоро лише хтонебудь збудує кльозет, то з нього користуються
всі сусіди. Дім на ймення Річардсон неосяжної краси.
Його вапняні стіни випинають, як жіноча сукня підчас реверансу.
Один кінець причілка опуклий, другий увігнутий, і на
останньому, на нещастя, стоїть димар, — крива труба з глини
й дерева, що нагадує хобот слона. Довга жердина служить за
підпору, щоб не впав димар; двері й вікна ромбічної форми.
З 17 оглянутих домів лише в чотирьох більше ніж по одній
спальні, і ці чотири були переповнені. В хатах з однією спальнею
містилося 3 дорослих і 3 дітей, одне подружжя з шістьма
дітьми й~\abbr{т. ін.}

Dunton: Висока квартирна плата, від 4 до 5\pound{ фунтів стерлінґів},
тижнева заробітна плата чоловіків 10\shil{ шилінґів.} Вони сподіваються
заробити на квартирну плату плетінням із соломи, що ним займається
родина. Що вища квартирна плата, то більше число осіб
мусить тиснутися в однім помешканні, щоб його оплатити. Шестеро
дорослих, що живуть з 4 дітьми в одній спальні, платять за неї
З фунти 10\shil{ шилінґів.} Найдешевшу хатину в Dunton’i, 15 футів
завдовжки, 10 футів завширшки, винаймають за 3\pound{ фунти стерлінґів}.
Лише одна з 14 обслідуваних хат мала дві спальні. Недалечко
від села стоїть хата, знадвору загиджена її мешканцями, долішня
частина дверей на 5 цалів зникла, бо цілком зогнила, надвечір
діру хитро закривають зсередини декількома цеглинами й завішують
рогожкою. Половина вікна разом із шибкою й рямою
зникла без сліду. Тут у голій хаті без меблів тулилося 3 дорослих
і 5 дітей. Dunton не гірший, аніж решта сел Biggleswade Union’у.

\paragraph{Berkshire}

Beenham: У червні 1864~\abbr{р.} в одному cot (одноповерховім
котеджі) жили чоловік, жінка, 4 дітей. Одна дочка прийшла із
служіння додому хора на скарлятину. Вона вмерла. Одна дитина
занедужала і вмерла. Мати й одна дитина були хорі на тиф, коли
до них запросили д-ра Гентера. Батько й одна дитина спали
надворі, але як важко було тут забезпечити ізоляцію, видно з
того, що на битком набитому базарному майдані бідолашного
села лежала білизна з дому зарази, чекаючи, щоб її хтось поправ.
— Квартирна плата за дім Н. 1\shil{ шилінґ} на тиждень; однісінька
спальня для подружжя й 6 дітей. Одну хату винаймають
за 8\pens{ пенсів} (на тиждень), 14 футів 6 цалів завдовжки, 7 футів
\parbreak{}  %% абзац продовжується на наступній сторінці

\input{i/_0593.tex}
\parcont{}  %% абзац починається на попередній сторінці
\index{i}{0594}  %% посилання на сторінку оригінального видання
5. Essex.

По багатьох парафіях цього графства людність меншає рівнобіжно
із зменшенням числа котеджів. Однак не менш ніж у
22 парафіях зруйнування будинків не спинило зростання людности
або не призвело до її виселення, яке відбувається всюди під
назвою «переселення до міст». У Fingringhoe, парафії, що займає
3.443 акри, в 1851 р. було 145 хат, у 1861 р. — вже лише 110, але
людність не хотіла йти геть і умудрилась зростати навіть за
таких умов. У Ramsden Crags 1851 р. 252 особи жили в 61 хаті
а в 1861 р. 262 особи тулилися вже в 49 хатах. У Basilden в 1851 р
на 1.827 акрах жило 157 осіб у 35 хатах, наприкінці цього десятиліття
— 180 осіб у 27 хатах. У парафіях Fingringhoe, South
Farnbridge, Widford, Basilden і Ramsden Crags у 1851 p. на
8.449 акрах жило 1.392 особи в 316 хатах, в 1861 р. на тій самій
площі — 1.473 особи у 249 хатах.

6. Herefordshire.

Це маленьке графство більше, ніж яке інше в Англії, потерпіло
від «духу виселення». В Nadby переповнені котеджі,
звичайно з двома спальнями, належать здебільша фармерам.
Вони легко здають їх у найми за 3 або 4\pound{ фунти стерлінґів} на рік
і платять заробітну плату в 9 шилінґів на тиждень!

7. Huntingdonshire.

У Hartford’i в 1851 р. було 37 хат, незабаром після цього
в цій маленькій парафії на 1.720 акрів зруйновано 19 котеджів;
число мешканців становило в 1831 р. 452 особи, в 1852 р. — 832,
а в 1861р. — 341. Досліджено 14 cots з однією спальнею в кожному.
В одному з них живе одне подружжя, 3 дорослі сини, одна
доросла дівчина, 4 дітей, разом 10; в іншому — 3 дорослих,
6 дітей. Одна з цих кімнат, де спало 8 осіб, мала 12 футів ІОцалів
завдовжки, 12 футів 2 цалі завширшки, 6 футів 9 цалів заввишки;
пересічно, не відраховуючи площі виступів, на людину припадало
130 кубічних футів. У 14 спальнях — 34 дорослих і 33 дітей. Ці
котеджі рідко мають садочки, але багато мешканців могло орендувати
маленькі шматки землі по 10 або 12 шилінґів за rood (\sfrac{1}{4} акра).
Ці allotments\footnote*{
— парцелі. \emph{Ред.}
} лежать далеко від хат, сами хати не мають кльозетів.
Члени родини мусять ходити або на свої парцелі, щоб там
полишати свої екскременти, або, як це, вибачайте за слово, робиться
тут, наповняти ними шухляду шафи. Як вона заповниться,
її висувають і випорожнюють там, де її вміст потрібний. В Японії
кругобіг умов життя відбувається охайніше.

8. Lincolnshire.

Langtoft: Один чоловік живе тут у хаті Wright’a з своєю дружиною,
її матір’ю й 5 дітьми; в хаті є кухня, комірка-полоскальний
над кухнею — спальня; кухня й спальня мають 12 футів

\parbreak{}  %% абзац продовжується на наступній сторінці

\parcont{}  %% абзац починається на попередній сторінці
\index{i}{0595}  %% посилання на сторінку оригінального видання
2 цалі завдовжки, 9 футів 5 цалів завширшки, вся площа має
21 фут 2 цалі довжини, 9 футів 5 цалів ширини. Спальня — це
кімната на горищі, стіни звужуються до стелі, як голова цукру,
з фасаду відкривається дахове віконце. Чому він живе тут?
Садок? Надзвичайно маленький. Квартирна плата? Висока —
1\shil{ шилінґ} 3\pens{ пенси} на тиждень. Близько до місця його праці?
Ні, хата стоїть на віддалі 6 миль від місця його праці, так що
йому доводиться день-у-день маршувати по 12 миль. Він живе
тут тому, що тут здавали в найми cot, а він хотів мати cot для
себе самого, хоч і де б то було, хоч і по якій би ціні, хоч і в якому
стані. Нижченаведена таблиця подає нам статистичні відомості
про 12 хат у Langtoft’i з 12 спальнями, що в них живуть 38 дорослих
і 36 дітей:

\begin{center}
\begin{small}
\captionnew{12 хат у Langloft’i}

\settowidth\rotheadsize{Число дітей}

\noindent\begin{tabular}{*{5}{c}@{\hspace{1em}}|@{\hspace{1em}}*{5}{c}}
  \toprule
  \rotcell{Хати} &
  \rotcell{Спальні} &
  \rotcell{Число дорослих} &
  \rotcell{Число дітей} &
  \makecell{Загальне \\ число \\ мешканців} &
  \rotcell{Хати} &
  \rotcell{Спальні} &
  \rotcell{Число дорослих} &
  \rotcell{Число дітей} &
  \makecell{Загальне \\ число \\ мешканців} \\
  \midrule
1  &  1  &  3  &  5  &  8  &  1  &  1  &  3  &  3  &  6\\
1  &  1  &  4  &  3 & 7  &  1  &  1  &  3 & 2  &  5\\
1  &  1  &  4  &  4  &  8  &  1  &  1  &  2  &  \textemdash{} & 2\\
1  &  1  &  5  &  4 & 9  &  1  &  1  &  2 & 3  &  5\\
1  &  1  &  2  &  2  &  4  &  1  &  1  &  3  &  3  &  6\\
1  &  1  &  5  &  3  &  8  &  1  &  1  &  2  &  4  &  6\\
\end{tabular}
\end{small}
\end{center}

\paragraph{Kent.}

Kennington був надзвичайно переповнений у 1859~\abbr{р.}, коли
з’явилася дифтерія і парафіяльний лікар організував офіціальний
дослід становища найбіднішої кляси людности. Він виявив, що
в цій місцевості, де потребують багато праці, багато cots зруйновано,
а нових не збудовано. В одній окрузі стояли 4 будинки,
так звані birdcages (пташині клітки), в кожному з них було по
4 кімнати таких розмірів у футах і цалях:

\begin{table}[h]
  \centering
  \begin{tabular}{lr}
    Кухня\dotfill{} & 9,5 × 8,11 × 6,6 \\
    Комірка-полоскальня & 8,6 × 4,6\phantom{0} × 6,6 \\
    Спальня\dotfill{}&8,5 × 5,10 × 6,3 \\
    Спальня\dotfill{}&8,3 × 8,4\phantom{0} × 6,3 \\
  \end{tabular}
\end{table}

\paragraph{Northamptonshire.}
Brinworth, Pickford i Floore: В цих селах зимою тиняються
по вулицях 20 — 30 робітників, не находячи праці. Фармери не
завжди як слід обробляли землю під збіжжя й корінняки, і лендлорд
збагнув, що йому корисніше сполучити всі свої оренди
в дві або три. Звідси недостача роботи. Тимчасом як по одному
боці рову поле потребує обробітку, по другому боці ошукані
робітники кидають на нього пожадливі погляди. Воно й не диво,
що, виснажені гарячковою надмірною працею влітку й напівголодні
\index{i}{0596}  %% посилання на сторінку оригінального видання
зимою, робітники кажуть на своєму власному діалекті,
що «the parson and gentlefolks seem frit to death at them»\footnoteA{
«Піп і шляхтич, здається, заприсяглися замордувати їх на
смерть».
}.

У Floore є приклади, що в спальні найменшого розміру живе
подружжя з 4, 5, 6 дітьми, або 3 дорослих з 5 дітьми, або подружжя
з дідом і 6 дітьми, хорими на скарлятину, і~\abbr{т. ін.}; у двох
хатах з двома спальнями — 2 родини, кожна складається з 8
і 9 дорослих.

\paragraph{Wiltshire.}

Stratton: Досліджено 31 хату, 8 з них мають лише одну
спальню. Pentill у тій самій парафії. Один cot винаймають за
1\shil{ шилінґ} 3\pens{ пенси} на тиждень; в ньому живе 4 дорослих і 4 дітей;
крім добрих стін, у ньому немає нічого доброго, починаючи від
долівки з погано обтесаного каменю й кінчаючи зігнилою солом’яною
стріхою.

\paragraph{Worcestershire.}

Тут хати не так жорстоко поруйновано; однак від 1851~\abbr{р.} до
1861~\abbr{р.} число мешканців на хату збільшилося з 4,2 до 4,6.

Badsey: Тут багато котеджів і садочків. Декотрі фармери
заявляють, що cots є «а great nuisance here, because they bring
the poor» («cots — велике лихо, бо принаджують бідноту»).
Один джентлмен каже: «Бідним від цього зовсім не краще; коли
збудувати 500 cots, їх розхоплять, наче булочки; справді, що
більше їх будують, то більше їх потрібно», отже, на його погляд,
хати покликають до життя мешканців, а мешканці, ясна річ,
натискують на «засоби мешкання». — З приводу цього вислову
д-р Гентер зауважує: «Але ж ці бідняки мусять звідкілясь
приходити, а що в Badsey немає особливої приваби, як от милостиня,
то, певно, мусить існувати відштовхування їх від якогось
ще невигіднішого місця, що й жене їх сюди. Коли б кожний міг
знайти недалечко від місця своєї праці cot і клаптик землі, то
напевне віддав би йому перевагу над Badsey, де йому за свій
клаптик землі доводиться платити удвоє дорожче, ніж фармерові
за свій».

Постійна еміграція до міст, постійне «створення перелюднення»
на селі через концентрацію фарм, перетворення нив на пасовиська,
застосування машин і~\abbr{т. ін.}, ідуть пліч-о-пліч з постійним
виганянням сільської людности через руйнування котеджів.
Що рідше заселена округа, то більше її «відносне перелюднення»,
то більший тиск останнього на засоби заробітку, то більший
абсолютний надмір сільської людности проти її житлових
засобів, отже, то більші по селах місцеве перелюднення й скупченість
людей з її наслідками — пошесними недугами. Скупченість
мас людей у порозкиданих дрібних селах і торгових містечках
відповідає ґвалтовному спустошенню людности на поверхні
\parbreak{}  %% абзац продовжується на наступній сторінці

\parcont{}  %% абзац починається на попередній сторінці
\index{i}{0597}  %% посилання на сторінку оригінального видання
землі. Безперервне «створення надміру» сільських робітників,
не зважаючи на зменшення числа їх і одночасне зростання маси
їхнього продукту, є джерело їхнього павперизму. Їхній евентуальний
павперизм є мотив виселення їх і головне джерело їхніх житлових
злиднів, які остаточно ламають силу їхнього опору й роблять
їх справжніми рабами землевласників\footnote{
«Благородне заняття сільського поденника надає гідности навіть
його становищу. Він не раб, а солдат миру, і заслуговує на те, щоб лендлорд,
який захопив собі право примушувати його до такої самої праці,
якої країна вимагає від солдата, дав йому помешкання, відповідне для
жонатої людини. За свою працю він так само, як і солдат, не дістає ринкової
ціни. Як і солдата, його забирають молодим, несвідомим, обізнаним
лише з своєю професією і своєю місцевістю. Ранній шлюб і різні закони
про осілість впливають на нього так само, як вербування рекрутів і військовий
карний кодекс на солдата». («The heavenbom employment of
the hind gives dignity even to his position. Не is not a slave, but a soldier
of peace, and deserves his place in married man’s quarters, to be provided
by the landlord, who has claimed a power of enforced labour similar to that
the country demande of a military soldier. Не no more receives marketprice
for his work than does a soldier. Like the soldier he is caughtyoung
ignorant, knowing only his own trade and his own locality. Early marriage and
the operation of the various laws of settlement affect the one as enlistment
and the Mutiny Act affect the other»). (Dr. \emph{Hunter} y «Public Health. Seventh
Report 1864», London 1865, p. 132). Іноді якогось винятково м’якосердого
лендлорда охоплює сум перед пустелею, що її він сам створив. «Дуже
сумно бути одному в своїх маєтках», казав граф Лейчестерський, коли
його поздоровили з закінченням будови його замку Holkham. «Я оглядаюсь
навколо й не бачу жодного будинку, крім свого власного. Я — велетень
башти велетнів і пожер усіх своїх сусідів».
} і фармерів, так що мінімум
заробітної плати стає для них природним законом. З другого
боку, село, не зважаючи на своє постійне відносне перелюднення,
є разом з тим недосить залюднене. Це виявляється не лише як
місцеве явище в таких пунктах, звідки людність занадто швидко
відпливає до міст, копалень, на будову залізниць тощо; це виявляється
повсюди так підчас жнив, як і на весні й улітку підчас
тих численних моментів, коли дуже старанне й інтенсивне
англійське рільництво потребує додаткових рук. Сільських
робітників завжди занадто багато для середніх і занадто мало для
виняткових або тимчасових потреб рільництва\footnote{
Подібний рух спостерігався останніми десятиліттями у Франції,
в міру того як капіталістична продукція там опановувала рільництво й
гнала «надмір» сільської людности до міст. Так само тут спостерігається
й погіршення житлових та інших умов коло джерела «надмірних рук».
Про своєрідний «prolétariat foncier»\footnote*{
— сільський пролетаріят. \emph{Ред.}
}, витворений системою парцель, див.,
між іншим, раніш цитовану працю \emph{Colins’a}: «L’Economie Politique», і
\emph{Karl Marx}: «Der Achtzehnte Brumaire des Louis Bonaparte», 2 Aufl.
Hamburg 1869 p., стор. 91 і далі. (\emph{К.~Маркс}: «Вісімнадцяте Брюмера Люї
Бонапарта», Партвидав «Пролетар» 1932, стор. 100 і далі). В 1846~\abbr{р.}
міська людність Франції становила 24,42\%, сільська — 75,58\%, в 1861~\abbr{р.}
міська — 28,86\% сільська — 71,14\%. Протягом останніх п’яти років
зменшення проценту сільської людности ще більше. Уже в 1846~\abbr{р.} П’ер
Дюпон співає в своєму «Ouvriers»:

\settowidth{\versewidth}{Sous les combles, dans les décombres,}
\begin{verse}[\versewidth]
«Mal vêtus, logés dans des trous,\\
Sous les combles, dans les décombres,\\
Nous vivons avec les hiboux\\
Et les larrons, amis des ombres».
\end{verse}

\settowidth{\versewidth}{Де сови лиш та злодії нічні}
\begin{verse}[\versewidth]
(«Усі в дранті, тут на горищах,\\
Серед руїн, в льохах живемо ми,\\
Де сови лиш та злодії нічні\\
Ховаються, охочі до пітьми»).\\
\end{verse}
}. Тому в офіціяльних
документах маємо зареєстровані суперечні скарги з
тих самих місцевостей на одночасну недостачу і надмір робочих
рук. Тимчасова або місцева недостача робочих рук не спричиняє
підвищення заробітної плати, а спричиняє приневолення жінок
і дітей до польових робіт і вживання робітників, щораз молодших
віком. Скоро тільки ця експлуатація жінок і дітей набирав
більших розмірів, вона й собі стає новим способом перетворювати
сільських робітників-чоловіків у зайвих і тримати їхню
заробітну плату на найнижчому рівні. На сході Англії процвітає
\index{i}{0598}  %% посилання на сторінку оригінального видання
чудовий плід цього cercle vicieux\footnote*{
— зачарованого кола. \emph{Ред.}
} — так звана Gangsystem
система артілей або ватаг (Gang-oder Bandensystem), про яку
я скажу тут декілька слів\footnote{
Шостий і останній «Report of Children’s Employment Commission»,
опублікований наприкінці березня 1867~\abbr{р.}, каже лише про систему
рільничих артілей.
}.

Система артілей процвітає майже виключно в Lincolnshire
Huntingdonshire, Cambridgeshire, Norfolk, Suffolk і Nott nghamshire,
спорадично — в сусідніх графствах Nothampton, Bedford
і Rutland. Як приклад візьмімо тут Lincolnshire. Значна
частина цього графства — це нова земля, колишнє болото, абож
земля, як і в інших названих східніх графствах, відвойована від
моря. Парова машина наробила чудес при осушуванні. Колишня
драговина й пісковий ґрунт красіють тепер у наряді буйного
збіжжя і дають якнайвищу ренту. Те саме стосується й до штучно
здобутого наносного ґрунту, як от на острові Axholme та інших
парафіях на побережжі Trent’y. В міру того як виникали нові
фарми, не лише не будували нових котеджів, але руйнували і
старі, а робітників постачали з відкритих сел, віддалених за
кілька миль, порозкидуваних здовж сільських доріг, що в’ються
схилами горбів. Лише там людність раніш знаходила собі захист
від тривалих зимових поводей. Робітники, що живуть на фармах
розміром від 400 до \num{1.000} акрів (їх тут звуть «confined labourers»\footnote*{
— «прикріплені робітники». \emph{Ред.}
}),
служать виключно для постійних важких польових робіт, виконуваних
кіньми. На кожні 100 акрів (1 акр — 40,49 ара, або
\num{1.584} пруських морґів) пересічно припадає ледве один котедж.
Один фармер, що орендував колишню драговину, свідчить перед
слідчою комісією: «Моя фарма має більш ніж 320 акрів, все це
саме орне поле. Котеджів на ній немає. Тепер у мене живе один
робітник. Чотири робітники, що доглядають моїх коней, живуть
\parbreak{}  %% абзац продовжується на наступній сторінці

\parcont{}  %% абзац починається на попередній сторінці 
\index{i}{0599}  %% посилання на сторінку оригінального видання 
в околиці. Легку працю, яка потребує багато робочих рук, виконують
артілі».\footnote{
«Children's Employment Commission. Sixth Report». Evidence,
p. 37, n. 173.
} Земля потребує багато легкої польової роботи:
виполювати бур’ян, обкопувати, робити деякі операції з
гноєм, вибирати каміння й т. ін. Все це роблять артілі або організовані
ватаги, що живуть у відкритих селах.

Артіль складається з 10—40 або 50 осіб, а саме, з жінок,
підлітків обох статей (з 13 до 18 років), хоч хлопці, дійшовши
13 років, здебільша покидають ватаги, і, нарешті, з дітей обох
статей (з 6 до 13 років). На чолі стоїть Gangmaster (староста
артілі); це завжди звичайний сільський робітник, здебільша
нікчемна людина, розпусник, волоцюга і п’яниця, але з певним
духом підприємливости і savoir faire.\footnote*{
— спритности. Ред.
} Він навербовує артіль,
яка працює під його проводом, а не під проводом фармера. З останнім
він договорюється здебільша відштучно, і його дохід, що
пересічно не дуже перевищує заробіток звичайного сільського
робітника,178 майже цілком залежить від умілости, з якою він
за найкоротший час зможе добути від своєї ватаги якнайбільше
праці. Фармери відкрили, що жінки працюють як слід лише під
диктатурою чоловіків, але що, з другого боку, жінки й діти,
скоро вони вже почали працювати, витрачають свої життєві
сили з справжньою загарливістю, — це знав уже Фур’є, — тимчасом
як дорослий робітник-чоловік хитрує, щоб якомога заощадити
свої сили. Староста ватаги переходить від одного маєтку
до іншого, і таким чином його ватага працює 6—8 місяців на
рік. Тим то він для робітничої родини дохідніший і певніший
наймач, аніж окремий фармер, що тільки принагідно вживає
дітей до праці. Ця обставина так зміцнює його вплив по відкритих
селах, що здебільша дітей можна найняти лише за його посередництвом.
Наймання фармерам дітей поодинці, поза артіллю, —
це його побічне заняття.

«Темний бік» цієї системи — це надмірна праця дітей і підлітків,
величезні переходи, які їм щодня доводиться робити туди
й назад до маєтків, віддалених на 5,6 а іноді й 7 миль, і, нарешті,
деморалізація «артілі». Хоч староста артілі, що його в деяких
місцевостях звуть «the driver» (підганяч), і озброєний довгою
палицею, проте він дуже рідко вживає її, і скарги на брутальне
поводження є виняток. Він — демократичний імператор або щось
наче щуролов із Гамельну.\footnote*{
Німецька легенда про щуролова з Гамельну (старовинне місто
над Везером у провінції (Ганновер), що своєю грою на сопілці заманив
усіх пацюків із міста, а потім і дітей, яких він завів у підземелля. Ред.
} Отже, він потребує популярности
серед своїх підданих і прив’язує їх до себе циганськими звичаями,
що процвітають під його опікою. Груба невгамовність, весела

173 Однак деякі старости артілі доробляються до того, що стають
фармерами, які мають 500 акрів землі, або власниками цілого ряду будинків.
\parbreak{}  %% абзац продовжується на наступній сторінці

\parcont{}  %% абзац починається на попередній сторінці
\index{i}{0600}  %% посилання на сторінку оригінального видання
розпуста і найбезсоромніше нахабство панують у ватазі. Здебільше
в шинку розплачується староста ватаги; потім він, похитуючись,
вертається додому на чолі ватаги, піддержуваний справа
й зліва кремезними бабами; позад нього скачуть діти й підлітки
співаючи глумливих і непристойних пісень. По дорозі додому звичайним
явищем є те, що Фур’є називає «фанерогамія».\footnote*{
— прилюдне злягання. \emph{Ред.}
} Часто тринадцятилітні
й чотирнадцятилітні дівчатка вагітніють від своїх
однолітків-хлопців. Відкриті села, що постачають континґент
для ватаг, стають Содомом і Гоморрою\footnote{
«Половину дівчат у Ludford’i зіпсували ватаги» (там же, Appendix,
стор. 6, п. 32).
} і дають удвоє більше
нешлюбних народжень, аніж решта королівства. Ми вже раніше
зазначали, що з морального погляду можуть дати виховані в
такій школі дівчата, ставши молодицями. їхні діти, якщо опій
не заподіє їм смерти, є природжені рекрути ватаги.

Ватага у своїй щойно описаній клясичній формі називається
публічною, громадською, або бродячою ватагою (public, common
or tramping gang). Бо бувають ще й приватні ватаги (private
gangs). Склад їх такий самий, що й громадських, тільки в них
менше людей, і працюють вони під проводом не старости ватаги,
а якогось старого сільського наймита, що його фармер не може
застосувати якось краще. Циганський гумор тут зникає, але, як
кажуть усі свідки, плата і поводження з дітьми тут гірші.

Система ватаг, що останніми роками щораз більше поширюється,
існує очевидно не заради старости ватаги.\footnote{
«Ця система дуже поширилась останніми роками. В деяких місцевостях
її заведено лише недавно, в інших, де вона існує давніше, до ватаг
вербують щораз більше і щораз молодших дітей» (там же, стор. 79,
п. 174).
} Вона існує
для збагачення великих фармерів\footnote{
«Дрібніші фармери не вживають праці ватаг». «Її не вживають
на поганій землі, а вживають на такій, що дає ренту від 2\pound{ фунтів стерлінґів}
до 2\pound{ фунтів стерлінґів} 10\shil{ шилінґів} з акра» (там же, стор. 17 і 14).
} або лендлордів.\footnote{
Одному з цих панів його ренти так припадають до смаку, що він
обурено заявив слідчій комісії, ніби ввесь галас зчинився лише через
назву системи. Коли б замість «ватаги» охристити її «юнацьким промислово-рільничим
кооперативним товариством для самостійного заробітку».
то все було б all right.\footnote*{
— гаразд. \emph{Ред.}
}
} Для фармера
немає дотепнішої методи підтримувати свій робочий персонал
нижче нормального рівня і все ж завжди мати напоготові
додаткові руки для всякої додаткової праці, за якомога менші
гроші видушувати якомога більше праці\footnote{
«Праця ватаг дешевша від усякої іншої праці; ось причина, чому
її уживають», каже один колишній староста ватаги (там же, стор. 17
і 14). «Система ватаг безперечно найдешевша для фермера й так само
безперечно найзгубніша для дітей», каже один фармер (там же, стор. 1о,
п. 3).
} і робити дорослих
робітників-чоловіків «зайвими». Після попередніх пояснень
можна зрозуміти, чому, з одного боку, визнають більше або менше
\parbreak{}  %% абзац продовжується на наступній сторінці

\parcont{}  %% абзац починається на попередній сторінці
\index{i}{0601}  %% посилання на сторінку оригінального видання
безробіття сільських робітників, а з другого — заявляють, що
система ватаг «доконечна» в наслідок браку робітників-чоловіків
та еміґрації їх до міст.\footnote{
«Без сумніву, багато робіт, що їх виконують тепер у ватагах
діти, раніш виконували чоловіки й жінки. Там, де до праці вживають
жінок і дітей, тепер безробітних чоловіків більш, ніж було раніш» (mor
men are out of work) (там же, стор. 43, п. 202). Але, з другого боку, між
іншим, читаємо: «Робітниче питання (labour question), у багатьох рільничих
округах, особливо тих, що продукують збіжжя, набуває такого
серйозного характеру в наслідок еміґрації і тієї легкости переселятись
до великих міст, яку дають залізниці, що я [«я» — це сільський аґент
одного великого лендлорда] вважаю дитячу працю за абсолютно доконечну»
(там же, стор. 80, n. 180). The labour question (робітниче питання)
в англійських рільничих округах, на відміну від решти цивілізованого
світу, означає власне the landlords’and farmers’question (лендлордське
й фармерське питання): яким чином, не зважаючи на щораз більший
відплив сільської людности, увічнити на селі достатнє «відносне перелюднення»,
а цим самим і «мінімум заробітної плати» для сільського
робітника?
} Поле, очищене від бур’яну, і людський
бур’ян Лінколншіру й т. ін. — це протилежні полюси
капіталістичної продукції.\footnote{
Вище цитований мною «Public Health Report», де з приводу
смертности дітей сказано мимохідь і про систему ватаг, лишився невідомий
пресі, а тим то й англійській публіці. Навпаки, останній звіт «Children’s
Employment Commission» дав пресі бажану «сенсаційну» поживу.
Тимчасом як ліберальна преса запитувала, яким чином шляхетні джентлмени
й леді та священики державної церкви, що ними кишіє Лінколншір,
яким чином ці персонажі, що посилали до антиподів свої спеціяльні
«місії для поліпшення звичаїв у дикунів Південного океану», могли допустити,
щоб навіч перед ними зросла така система в їхніх маєтках, —
шляхетна преса обмежувалась міркуваннями про грубу зіпсованість
селян, здатних продавати своїх дітей у таке рабство! Однак, за тих проклятих
обставин, на які «шляхетні» засудили селянина, було б зрозуміло,
коли б він навіть з’їдав своїх власних дітей. З чого дійсно можна
дивуватись, так це з пристойности, яку він ще здебільша зберіг. Автори
офіціяльних звітів доводять, що батьки навіть в округах із системою ватаг
ставляться до цієї системи з огидою. «У зібраних нами свідченнях можна
найти багато доказів того, що батьки в багатьох випадках були б вдячні
за такий примусовий закон, що дав би їм змогу опиратися спокусам і
натискові, яким вони часто підпадають. То парафіяльний урядовець,
то підприємець, загрожуючи їм звільненням, примушує їх посилати дітей
на заробітки замість до школи\dots{} Кожне марне витрачання часу й сил,
усі страждання, що їх спричиняє селянинові і його родині надзвичайна
й некорисна втома, кожний випадок, коли батьки можуть приписати
моральний занепад своєї дитини переповненню котеджів або розкладницькому
впливові системи ватаг, — все це пробуджує у грудях цих трудящих
бідолах почуття, які, певно, легко зрозуміти і які нема потреби описувати
докладніш. Вони свідомі, що їм завдають багато фізичних і моральних
мук обставини, за які вони зовсім не відповідальні, і на які вони,
коли б на те була їхня сила, ніколи не дали б своєї згоди, і проти яких
вони неспроможні боротися» (там же, стор. XX, п. 82 і XXIII, п. 96).
}

f. Ірляндія

Закінчуючи цей відділ, ми мусимо ще на хвилинку спинитися
на Ірландії. Насамперед подаємо факти, що служать нам
за вихідний пункт.

\index{i}{0602}  %% посилання на сторінку оригінального видання
Людність Ірляндії зросла в 1841 р. до 8.222.664 осіб, в 1851 р.
вона зменшилась до 6.623.985, в 1861 р. до 5.850.309, в 1866 р. —
до 5\sfrac{1}{2} мільйонів, тобто приблизно до свого рівня 1801 р. Зменшення
почалося з 1846 голодного року, так що менше ніж за
20 років Ірляндія втратила більше як \sfrac{5}{16} своєї людности.\footnote{
Людність Ірляндії: в 1801 р. — 5.319.867 осіб, в 1811 р. — 6.084.996,
в 1821 р. — 6.869.544, в 1831 р. — 7.828.347, в 1841 р. — 8.222.664 особи.
}
Загальне число еміґрантів від травня 1851 р. до липня 1865 р.
становило 1.591.487 осіб, число еміґрантів за останні п’ять років,
від 1861 до 1865 р., становило більш ніж \sfrac{1}{2} мільйона осіб. Число
заселених будинків зменшилось від 1851 р. до 1861 р. на 52.990.
Від 1851 р. до 1861 р. число фарм розміром від 15 до 30 акрів зросло
на 61.000, число фарм понад 30 акрів — на 109.000, тимчасом
як загальне число всіх фарм зменшилось на 120.000, отже зменшення,
спричинене виключно знищенням фарм нижче 15 акрів,
тобто централізацією їх.

Зменшення людности, певна річ взагалі і в цілому супроводилось
зменшенням маси продуктів. Для нашої мети досить
розглянути п’ятиріччя 1861--1865, протягом якого еміґрувало
понад \sfrac{1}{2} мільйона і абсолютна кількість людности спала
більш ніж на \sfrac{1}{3} мільйона (див. таблицю А).

\begin{table}[ht]
  \begin{flushright}
    \emph{Таблиця А}
  \end{flushright}
  \caption*{Худоба}
  \noindent\begin{tabularx}{\textwidth}{X@{}ccccc}
    \toprule
      \multirowcell{2}{\makecell{Роки}} &
      \multicolumn{2}{c}{Коні} &
      \multicolumn{3}{c}{Рогата худоба} \\
    \cmidrule(rl){2-3}
    \cmidrule(l){4-6}
    &
    \makecell{Загальна \\ кількість} &
      Зменшення &
    \makecell{Загальна \\ кількість} &
    Зменшення &
    Збільшення
    \\
    \midrule
      1860\dotfill{}& 619.811 & \textemdash{} & 3.606.374 & \textemdash{} & \textemdash{} \\
      1861\dotfill{}& 614.232 & \phantom{0}5.993 & 3.471.688 & 138.316 & \textemdash{} \\
      1862\dotfill{}& 602.894 & 11.338 & 3.254.890 & 216.798 & \textemdash{} \\
      1863\dotfill{}& 579.978 & 22.916 & 3.144.231 & 110.695 & \textemdash{} \\
      1864\dotfill{}& 562.158 & 17.820 & 3.262.294 & \textemdash{} & 118.063 \\
      1865\dotfill{}& 547.867 & 14.291 & 4.493.414 & \textemdash{} & 231.120 \\
  \end{tabularx}
\end{table}

\begin{table}[ht]
  \noindent\begin{tabularx}{\textwidth}{X@{}cccccc}
  \toprule
    \multirowcell{2}{\makecell{Роки}} &
    \multicolumn{3}{c}{Вівці} &
    \multicolumn{3}{c}{Свині}\\
  \cmidrule(rl){2-4}
  \cmidrule(l){5-7}
  &
  \makecell{Загальна \\ кількість} &
  \makecell{Змен-\\шення} &
  \makecell{Збіль-\\шення} &
  \makecell{Загальна \\ кількість} &
  \makecell{Змен-\\шення} &
  \makecell{Збіль-\\шення}
  \\
  \midrule
    1860\dotfill{}& 3.542.080 & \textemdash{} & \textemdash{} & 1.271.072 & \textemdash{} & \textemdash{} \\
    1861\dotfill{}& 3.556.050 & \textemdash{} & \phantom{0}13.970 & 1.102.042 & 169.030 & \textemdash{} \\
    1862\dotfill{}& 3.456.132 & \phantom{0}99.918 & \textemdash{} & 1.154.324 & \textemdash{} & \phantom{0}52.282 \\
    1863\dotfill{}& 3.308.204 & 147.982 & \textemdash{} & 1.067.458 & \phantom{0}86.866 & \textemdash{} \\
    1864\dotfill{}& 3.366.941 & \textemdash{} & \phantom{0}58.737 & 1.058.480 & \phantom{00}8.978 & \textemdash{} \\
    1865\dotfill{}& 3.688.742 & \textemdash{} & 321.801 & 1.299.893 & \textemdash{} & 241.413 \\
  \end{tabularx}
\end {table}

\index{i}{0603}  %% посилання на сторінку оригінального видання
З попередньої таблиці маємо такий результат:
\begin{center}
  \newcolumntype{Y}{>{\centering\arraybackslash}X}
  \noindent\begin{tabularx}{\textwidth}{Y Y Y Y}
      Коні & Рогата худоба & Вівці & Свині \\
      Абсолютне зменшення & Абсолютне зменшення & Абсолютне збільшення & Абсолютне збільшення \\
      \num{72.358} & \num{116.626} & \num{146.608} & \num{28.819}\hang{l}{\footnote{Результат був би ще несприятливіший, коли б ми пішли ще далі
      назад. Так, овець 1865 р. було \num{3.688.742}, а року 1856 — \num{3.694.294}; свиней
      року 1865 було \num{1.299.893}, а року 1858 — \num{1.409.833}.}}\\

  \end{tabularx}
\end {center}

Звернімось тепер до рільництва, що постачає засоби існування
для худоби й людей. У дальшій таблиці обчислено збільшення
або зменшення для кожного окремого року порівняно з безпосередньо
попереднім роком. Збіжжя обіймає пшеницю, овес,
ячмінь, жито, квасолю й горох, зеленина — картоплю, турнепс,
білі й червоні буряки, капусту, моркву, пастернак, вику й т. ін.

\setlength{\tabcolsep}{2pt}

\begin{table}[h]\small
  \settowidth\rotheadsize{шення}

  \begin{flushright}
    \emph{Таблиця В}
  \end{flushright}
  \caption*{Збільшення або зменшення засівної площі й лук (зглядно толок) в акрах}
  \noindent\begin{tabular}{cccccccccc}
  \toprule
    \multirowcell{2}{\makecell{Роки}} &
    Збіжжя &
    \multicolumn{2}{c}{Зеленина} &
    \multicolumn{2}{c}{\makecell{Луки й ко-\\нюшина}} &
    \multicolumn{2}{c}{Льон} &
    \multicolumn{2}{c}{\makecell{Загальна кількість \\ землі для рільни-\\цтва і скотарства}} \\

    \cmidrule(l){2-2}
    \cmidrule(l){3-4}
    \cmidrule(l){5-6}
    \cmidrule(l){7-8}
    \cmidrule(l){9-10}
   &
  \makecell{Змен-\\шення} &
  \makecell{Змен-\\шення} &
  \makecell{Збіль-\\шення} &
  \makecell{Змен-\\шення} &
  \makecell{Збіль-\\шення} &
  \makecell{Змен-\\шення} &
  \makecell{Збіль-\\шення} &
  \makecell{Змен-\\шення} &
  \makecell{Збіль-\\шення} \\
  \midrule
    1861 & \phantom{0}\num{15.701} & \phantom{0}\num{36.974} & \textemdash{} & \num{47.969} & \textemdash{} & \textemdash{} &  \phantom{0}\num{19.271} & \phantom{0}\num{81.873} & \textemdash{} \\
    
    1862 & \phantom{0}\num{72.734} & \phantom{0}\num{74.785} & \textemdash{} & \textemdash{} &  \phantom{0}\num{6.623} & \textemdash{} & \phantom{00}\num{2.055} & \num{138.841} & \textemdash{} \\
    
    1863 & \num{144.719} & \phantom{0}\num{19.358} & \textemdash{} & \textemdash{} &  \phantom{0}\num{7.724} & \textemdash{} & \phantom{0}\num{63.922} & \phantom{0}\num{92.431} & \textemdash{} \\
    
    1864 & \num{122.437} & \phantom{00}\num{2.317} & \textemdash{} & \textemdash{} & \num{47.486} & \textemdash{} & \phantom{0}\num{87.761} & \textemdash{} & \num{10.493} \\
    
    1865 & \phantom{0}\num{72.450} & \textemdash{} & \num{25.241} & \textemdash{} & \num{68.970} & \num{50.159} & \textemdash{} & \phantom{0}\num{28.218} & \textemdash{} \\
    
    1861\textemdash{}1865 & \num{428.041} & \num{107.984} & \textemdash{} & \textemdash{} & \num{82.834} & \textemdash{} & \num{122.850} & \num{330.860} & \textemdash{} \\
  \end{tabular}
\end{table}

\setlength{\tabcolsep}{\tabcolsepdef}

В 1865 році в рубриці «луки» сталося збільшення на \num{127.470} акрів,
головно через те, що площа в рубриці «необроблена пуста
земля й торфовища» зменшилась на \num{101.543} акри. Коли порівняти
1865 рік з 1864, то зменшення збіжжя становитиме \num{246.667}
квартерів, із них пшениці — \num{48.999} квартерів, вівса — \num{166.605} квартерів,
ячменю — \num{29.982} квартери й т. ін.; зменшення кількости
картоплі, хоч оброблювана під нею площа в році 1865 і збільшилась,
становило \num{446.398} тонн і т. ін. (див. таблицю \emph{С}).

Від руху людности й рільничої продукції Ірляндії перейдімо
до руху в гаманці її лендлордів, великих фармерів і промислових
капіталістів. Він відбивається у зменшенні і збільшенні прибуткового
податку. Щоб зрозуміти дальшу таблицю \emph{D}, треба зауважити,
що рубрика D (зиски, за винятком зисків фармерів) обіймає
і так звані «професійні» зиски, тобто доходи адвокатів,
лікарів і т. ін., а рубрики С й Е, які тут не перелічені окремо,
обіймають і доходи урядовців, офіцерів, державних синекуристів,
держців державних цінних паперів і т. д.

\setlength{\tabcolsep}{3pt}
\begin{sidewaystable}
  \index{i}{0604}  %% посилання на сторінку оригінального видання
  \footnotesize
  \begin{center}
    \captionnew{\emph{Таблиця С}. Збільшення або зменшення площі обробленої землі, кількости продукту на акр і \\ загальної кількости продукту в році 1865 порівняно з роком 1864\footnotemark{}}
  \end{center}

  \begin{tabularx}{\textheight}{XrrrrXrrccr@{~}lrrr}

    \toprule
    \makecell{\multirow{3}{*}{Продукти}} & 
    \multicolumn{2}{c}{\makecell{Кількість акрів\\обробленої землі}} & 
    \multicolumn{2}{c}{\makecell{Збільшення \\ або зменшення\\в 1865 р.}} & 
    \multicolumn{3}{c}{\makecell{Кількість \\ продуктів на акр }} & 
    \multicolumn{2}{c}{\makecell{Збільшення \\ або зменшення\\ в 1865 р.}} & 
    \multicolumn{5}{c}{\makecell{Загальна кількість продукту}}
    \\

    \cmidrule(rl){2-3}
    \cmidrule(rl){4-5}
    \cmidrule(rl){6-8}
    \cmidrule(rl){9-10}
    \cmidrule(rl){11-15}
    & 
    \makecell{\multirow{2}{*}{1864}} & 
    \makecell{\multirow{2}{*}{1865}} & 
    \makecell{\multirow{2}{*}{$+$}} & 
    \makecell{\multirow{2}{*}{$-$}} & 
    
    \multicolumn{2}{c}{{\multirow{2}{*}{1864}}} & 
    \makecell{\multirow{2}{*}{1865}} & 
    \makecell{\multirow{2}{*}{$+$}} & 
    \makecell{\multirow{2}{*}{$-$}} & 
    
    \multicolumn{2}{c}{{\multirow{2}{*}{1864}}} & 
    \makecell{\multirow{2}{*}{1865}} & 
    \makecell{Збіль-\\шення} &
    \makecell{Збіль-\\шення} \\

    \cmidrule(rl){14-15}
    & & & & & & & & & & & & & \multicolumn{2}{c}{1865} \\
    \midrule

    Пшениця & 276.483 & 266.989 & \emptycell{} & 9.494 &
      Пшениця цент. & 13,3 & 13,0 & \emptycell{} & 0,3 &
      875.782 & кв. & 826.783 & \emptycell{} & 48.999 кв. \\

    Овес & 1.814.886 & 1.745.228 & \emptycell{} & 69.658 &
      Овес & 12,1 & 12,3 & 0,2 & \emptycell{} &
      7.826.332 & & 7.659.727 & \emptycell{} & 166.605 \ditto{кв.}\\

    Ячмінь & 172.700 & 177.102 & 4.402 & \emptycell{} &
      Ячмінь & 15,9 & 14,9 & \emptycell{} & 1,0 &
      761.909 & & 732.017 & \emptycell{} & 29.892 \ditto{кв.}\\

    \makehangcell{Шот\-лян\-д\-сь\-кий яч\-мінь (Be\-re)} & 8.894 & 10.091 & 1.197 & \emptycell{} &
      Шотл. ячм. & 16,4 & 14,8 & \emptycell{} & 1,6 &
      15.160 & & 13.989 & \emptycell{} & 1.171 \ditto{кв.}\\

    Жито & \emptycell{} & \emptycell{} & \emptycell{} & \emptycell{} & 
      Жито & 8,5 & 10,4 & 1,9 & \emptycell{} & 
      12.680 & & 18.364 & 5.684 кв. & \emptycell{} \\

    Картопля & 1.039.724 & 1.066.260 & 26.536 & \emptycell{} &
      Картопля тонн & 4,1 & 3,6 & \emptycell{} & 0,5 &
      4.312.388 & тонн & 3.865.990 & \emptycell{} & 446.398 \ditto{кв.}\\

    Турнепс & 337.355 & 334.212 & \emptycell{} & 3.143 &
      Турнепс & 10,3 & 9,9 & \emptycell{} & 0,4 &
      3.467.659 & & 3.301.683 & \emptycell{} & 165.976 \ditto{кв.}\\
    
    Білі буряки & 14.073 & 14.839 & 316 & \emptycell{} &
      Білі бур. & 10,5 & 13,3 & 2,8 & \emptycell{} &
      147.284 & & 191.937 & 44.653 \samewidth{кв.}{т.} & \emptycell{} \\
    
    Капуста & 31.831 & 33.622 & 1.801 & \emptycell{} &
      Капуста & 9,3 & 10,4 & 1,1 & \emptycell{} & 
      297.375 & & 350.252 & 52.87 \phantom{кв.} & \emptycell{} \\
    
    Льон & 301.693 & 251.433 & \emptycell{} & 50.260 &
      \makehangcell{Льон  (Sto\-nes \\ в~14~ф.)} & 34,2 & 25,2 & \emptycell{} & 9,0 &
      64.506 & & 39.561 & \emptycell{} & 24.945 \ditto{кв.}\\

    Сіно & 1.609.569 & 1.678.493 & 68.924 & \emptycell{} &
      Сіно тонн & 1,6 & 1,8 & 0,2 & \emptycell{} &
      2.607.153 & & 3.068.707 & 461.554 \phantom{кв.} & \emptycell{}
  \end{tabularx}

\footnotetext{Дані тексту складено з матеріялів: «Agricultural Statistics, Ireland. General Abstracts,
Dublin», за роки 1860 і дальші і «Agricultural Statistics. Ireland. Tables showing the Estimated
Average Produce etc. Dublin, 1866».
Відомо, що це є офіціяльна статистика, яку щорічно подають парляментові.

Додаток до другого видання. Офіціяльна статистика показує для 1872 р. зменшення площі обробленої
землі на 134.915 акрів порівняно з роком 1871. Сталося «збільшення» зеленини — турнепсу, білих
буряків і т. ін.; «зменшення» площі обробленої землі на 16.000 акрів пшениці, на 14.000 акрів вівса,
на 4.000 акрів ячменю й жита, на 66.632 акри картоплі, на 34.667 акрів льону й на 30.000 акрів менше
під луками, конюшиною, викою, свиріпою. Площа землі, що була під культурою пшениці, протягом
останніх 5 років зменшилася в такій послідовності: 1868 р. — 285.000 акрів, 1869 р. — 280.000 акрів,
1870 р. — 259.000 акрів, 1871 р. — 244.000 акрів, 1872 р. — 228.000 акрів. Для року 1872 маємо
збільшення заокруглено на 2.600 коней, на 80.000 штук рогатої худоби, на 68.609 овець, і зменшення
числа свиней на 236.000.
}
\end{sidewaystable}
\setlength{\tabcolsep}{\tabcolsepdef}
\index{i}{0605}  %% посилання на сторінку оригінального видання
\begin{table}
\begin{flushright}
  \emph{Таблиця D}
\end{flushright}

\caption*{Доходи, що підлягають оподаткуванню, у фунтах стерлінґів}
\footnotesize
  \noindent\begin{tabularx}{\textwidth}{p{2cm} X X X X X X}

  \toprule
& 1860 & 1861 & 1862 & 1863 & 1864 & 1865 \\
\cmidrule{2-7}
\mbox{Рубрика А}
\mbox{Земельна рента} & 13.893.829 & 13.003.554 & 13.398.938 & 13.494.091 & 13.470.700
& 13.801.616 \\
Рубрика В
\mbox{Зиски фармерів} & 2.765.387 & 2.773.644 & 2.937.899 & 2.938.823 & 2.930.874 & 2.946.072 \\
Рубрика D
\mbox{Промисловий} і інший зиск & 4.891.652 & 5.836.203 & 4.858.800 & 4.846.497 & 4.546.147 & 4.850.199 \\
\mbox{Сума всіх рубрик}
від А до Е & 22.962.885 & 22.998.394 & 23.597.574 & 23.658.631 & 23.236.298 &    23.930.340\footnotemark{} % ця мітка у заголовку
\\
 % текст примітки прямо під заголовком

  \end{tabularx}

\end{table}
\footnotetext{«Tenth Report of the Commissioners of Inland Revenue», London
1866.}

Під рубрикою D збільшення прибутку від 1853 до 1864 р.
становило пересічно лише 0,93\%, тимчасом як у Великобрітанії
воно за той самий період становило 4,58\%. Дальша таблиця показує
розподіл зисків (за винятком зисків фармерів) у 1864 і
1865 рр.

\begin{table}
  \begin{flushright}
    \emph{Таблиця Е}
  \end{flushright}

\caption*{Рубрика D. Доходи, що складаються з різних зисків (понад 60 фунтів
  стерлінґів) в Ірландії}
\small

  \noindent\begin{tabularx}{\textwidth}{X X X}
\toprule
    & 1864 р. & 1865 р. \\
\cmidrule{2-3}
    & Фунтів стерлінґів &  Фунтів стерлінґів \\
  Загальний річний дохід &
  4.368.610, розподілений між особами 17.467 &
  4.669.979, розподілений між 18.081 особою \\

  Річний дохід понад 60 ф. ст. і нижче за 100 ф. ст &
  238.626, розподілений між 5.015 особами &
  222.575, розподілений між 4.703 особами \\

  Із загального річного доходу &
  1.979.066, розподілений між 11.321 особами &
  2.028.471, розподілений між 12.184 особами \\

  Решта загального річного доходу &
  2.150.818, розподілений між 1.131 особою &
  2.418.933, розподілений між 1.194 особами \\

  &
  1.083.906, розподілений між 910 особами &
  1.097.937, розподілений між 1.044 особами \\

  &
  1.066.912, розподілений між 121 особою &
  1.320.996, розподілений між 186 особами \\
  % TODO: потрібно застосувати пакет multirow, і збільшити фігурну дужку
  З нього: \Bigg\{ &
  430.535, розподілений між 105 особами &
  584.458, розподілений між 122 особами \\

  &
  646.377, розподілений між 26 особами &
  736.448, розподілений між 28 особами \\

  &
  262.610, розподілений між 3 особами &
  274.528, розподілений між 3 особами\footnotemark{}
  \\

  \end{tabularx}
\end{table}
\footnotetext{Загальний річний дохід під рубрикою D відхиляється тут від
  чисел попередньої таблиці, бо закон допускає деякі відрахування.}

\index{i}{0606}  %% посилання на сторінку оригінального видання
Англія країна розвиненої капіталістичної продукції й переважно
промислова країна, стекла б кров’ю від такого кровопуску,
якого зазнала Ірляндія. Але тепер Ірляндія є лише рільнича
округа Англії, відділена від неї широким каналом, округа, що
постачає їй збіжжя, вовну, худобу, промислових і військових
рекрутів.

Збезлюднення призвело до того, що багато землі лишилось
необробленою, кількість рільничого продукту дуже зменшилась;\footnote{
Якщо продукт зменшується відносно також і на акр, то не треба
забувати, що Англія протягом півтора віків посередньо експортувала,
ірляндський ґрунт, не залишаючи рільникам навіть засобів відновлювати
складові частини ґрунту.
}
не зважаючи на поширення площі, призначеної для
скотарства, в деяких його галузях сталося абсолютне зменшення,
в інших — ледве помітний проґрес, що раз-у-раз переривався
зворотним рухом. А все ж разом із зменшенням людности
безупинно зростали земельні ренти й фармерські зиски,
хоч останні не так стало, як перші. Причину цього зрозуміти
легко. З одного боку, разом з централізацією фарм і перетворенням
орної землі на пасовиська щораз більша частина сукупного
продукту перетворювалась на додатковий продукт. Додатковий
продукт зростав, хоч сукупний продукт, якого він становить
частину, меншав. З другого боку, грошова вартість цього
додаткового продукту зростала ще швидше, ніж його маса, бо
англійські ринкові ціни на м’ясо, вовну й т. ін. протягом останніх
двадцятьох років і особливо протягом останніх десятьох років
раз-у-раз зростали.

Роздрібнені засоби продукції, що для самого продуцента служать
за засоби праці й існування і що не зростають своєю вартістю
за допомогою прилучення до себе чужої праці, так само
не є капітал, як не є товар продукт, споживаний його власним
продуцентом. Хоч маса засобів продукції, застосованих у рільництві,
і зменшилась разом із зменшенням людности, проте
маса капіталу, застосованого в рільництві, збільшилась, бо
частина раніш роздрібнених засобів продукції перетворилась на
капітал.

Цілий капітал Ірляндії, вкладений поза рільництвом — у
промисловість і торговлю, нагромаджувався протягом останніх
двох десятиліть поволі і з постійними великими коливаннями.
Зате тим швидше розвивалась концентрація його індивідуальних
складових частин. Нарешті, хоч і яке невеличке було його абсолютне
зростання, але відносно, порівняно із зменшенням кількости
людности, він значно збільшився.

Отже, перед нашими очима тут у великому маштабі розгортається
процес, що кращого за нього ортодоксальна економія й
бажати не може для ствердження своєї догми, що злидні виникають
з абсолютного перелюднення і що рівновага відновлюється
через зменшення людности. Це експеримент куди важливіший,
\parbreak{}  %% абзац продовжується на наступній сторінці

\input{i/_0607.tex}
\parcont{}  %% абзац починається на попередній сторінці
\index{i}{0608}  %% посилання на сторінку оригінального видання
частина пустирів і торфовищ, що їх раніш не використовували,
служить тепер для поширення скотарства. Дрібні й середні
фармери — я залічую сюди всіх тих, що обробляють не більше
як 100 акрів землі — все ще становлять приблизно \sfrac{8}{10} із загального
числа\footnoteA{Примітка до другого видання. Згідно з однією таблицею в \emph{Murphy}:
«Ireland, Industrial, Political and Social», 1870, 94,6\% усіх земель
є фарми, менші від 100 акрів кожна і 5,4\% — фарми понад 100 акрів.}.  Конкуренція капіталістичної рільничої продукції
щораз більше й більше душить їх, і тим то вони постійно постачають
клясі найманих робітників нових рекрутів. Однісінька
велика промисловість Ірляндії, фабрикація полотна, потребує
порівняно мало дорослих робітників-чоловіків і, не зважаючи
на її поширення після подорожчання бавовни в 1861--1866~\abbr{рр.},
вона взагалі вживає лише порівняно незначну частину людности.
Як усяка інша велика промисловість, вона постійними коливаннями
у своїй власній сфері постійно продукує відносне перелюднення,
навіть і за абсолютного зростання маси робітників, яку
вона поглинає. Злидні сільської людности є за п’єдестал для велетенських
фабрик сорочок тощо, робітнича армія яких розпорошена
здебільшого по селах. Тут ми знову таки бачимо змальовану
раніш систему домашньої праці — систему, де недостатня
плата за роботу і надмірна праця служать за методичні засоби
продукувати «зайвих» робітників. Нарешті, хоч зменшення людности
не має тут таких руйнаційних наслідків, як у країні з розвиненою
капіталістичною продукцією, проте й тут воно відбувається
не без постійного зворотного впливу на внутрішній ринок.
[Еміґрація лишає по собі не лише порожні будинки, але й зруйнованих
квартироздавців]\footnote*{Заведене у прямі дужки ми беремо з другого німецького видання. \emph{Ред.}}. Та прогалина, що її створює тут
еміґрація, не лише зменшує місцевий попит на працю, але також
і доходи дрібних крамарів, ремісників, взагалі дрібних промисловців.
[Кожне нове виселення перетворює частину дрібної середньої
кляси на пролетарів]\footnotemarkZ[\value{footnoteZ}]. Звідси зменшення доходів між
60 і 10\pound{ фунтами стерлінґів} у таблиці Е.

Ясну картину становища сільських поденників в Ірляндії
ми маємо у звітах інспекторів ірляндської адміністрації в справах
про бідних (1870)\footnoteA{
«Reports from the Poor Law Inspectors on the wages of Agricultural
Labourers in Dublin, 1870». Порівн. також «Agricultural Labourers
(Ireland) Return etc. dated 8th March 1861», London, 1862.
}. Урядовці такого уряду, що тримається
лише за допомогою баґнетів і стану облоги, то явного, то прихованого,
мусять бути обережними у висловах, чим їхні колеґи
в Англії нехтують; а проте вони не дозволяють своєму урядові
уколисувати себе ілюзіями. За їхніми відомостями, рівень
заробітної плати на селі, і досі все ще дуже низький, все ж за
останні двадцять років підвищився на 50--60\% і становить
тепер пересічно 6--9\shil{ шилінґів} на тиждень. Але за цим позірним
\index{i}{0609}  %% посилання на сторінку оригінального видання
підвищенням криється реальне зниження заробітної плати,
бо воно навіть не урівноважує того підвищення цін на доконечні
засоби існування, що сталося за той час. Доказ — нижченаведений
витяг з офіціяльних звітів одного ірляндського робітного
дому.

\vspace{-2\medskipamount}
\begin{table}[H]
  \centering
  \caption*{Пересічні тижневі витрати на утриманця однієї людини}
  \noindent\begin{tabular}{@{}llll@{}}
    \toprule
    Роки & Харчі & Одяг & Разом \\
    \midrule
    Від 29 вересня 1848~\abbr{р.} 
    до 29 вересня 1849~\abbr{р.} & 1\shil{ шилінґ} 3\tbfrac{1}{4}\pens{ пенса} &  3\pens{ пенси}  &  1\shil{шил.} 6\tbfrac{1}{4}\pens{ пенса} \\

    Від 29 вересня 1868~\abbr{р.} 
    до 29 вересня 1869~\abbr{р.} & 2\shil{ шилінґи} 7\tbfrac{1}{4}\pens{ пенса} & 6\pens{ пенсів} & 3\shil{шил.} 1\tbfrac{1}{4}\pens{ пенса} \\
  \end{tabular}
\end{table}
\vspace{-2\medskipamount}

\noindent{}Отже, ціна доконечних засобів існування підскочила майже
вдвоє, а ціна одягу рівно вдвоє, аніж перед двадцятьма роками.

\looseness=-1
Навіть коли залишити осторонь цю диспропорцію, то саме
порівняння заробітних плат, визначених у грошах, далеко ще
не дає правдивого висновку. Перед голодом велику частину заробітної
плати на селі видавали in natura, грішми виплачували
лише дуже невеличку частину; нині грошова виплата стала загальним
правилом. Вже з цього випливає, що, хоч який буде
рух реальної заробітної плати, її грошовий рівень мусив підвищитися.
«Перед голодом сільський поденник мав шматок
землі, де він культивував картоплю і відгодовував свиней та
дробину. Нині він мусить не тільки купувати собі всі засоби
існування, але він втрачає й ті доходи, що він мав із продажу
свиней, дробини і яєць»\footnote{
Там же, стор. 291.
}. Справді, раніше сільські робітники
зливалися з дрібними фармерами і здебільша становили
лише ар’єрґард середніх і великих фарм, де вони находили для
себе заняття. Лише від часу катастрофи 1846~\abbr{р.} вони почали становити
частину кляси власне найманих робітників, окрему верству,
зв’язану із своїми панами-наймачами лише грошовими відносинами.
Ми вже знаємо, який був їхній житловий стан перед 1846~\abbr{р.}
Від того часу він ще більше погіршав. Деяка частина сільських
поденників, що, зрештою, з дня на день меншає, живе ще на
землях фармерів у переповнених хатинах, що їхній огидний стан
далеко перевищує все те найгірше, що виявили нам у цьому відношенні
англійські рільничі округи. І такий є стан речей повсюди,
за винятком деяких округ в Ulster’i, на півдні у графствах
Cork, Limerick, Kilkenny та інших; на сході у Wicklow’i
Wexford’i і~\abbr{т. д.}; у центрі в King’s і Queen’s County, Dublin’i
і~\abbr{т. д.}; на півночі в Down’i, Antrim’y, Tyrone і~\abbr{т. д.}; на заході
в Sligo, Roscommon’i, Mayo, Galway і~\abbr{т. д.} «Це, — вигукує один
\parbreak{}  %% абзац продовжується на наступній сторінці

\parcont{}  %% абзац починається на попередній сторінці 
\index{i}{0610}  %% посилання на сторінку оригінального видання 
в інспекторів, — ганьба для релігії й цивілізації цієї країни».\footnoteA{
Там же, стор. 12.

і87b Там же, стор. 12.
}
Щоб зробити стерпнішим життя сільських наймитів в їхніх льохах,
у них систематично відбирають клаптики землі, що від непам’ятних
часів належали до тих мешкань. «Усвідомлення цього
роду немилости, що її вони зазнають від лендлордів та їхніх управителів,
викликало в сільських поденників відповідне почуття
антагонізму й ненависти до тих, що поводяться з ними як з безправною
расою».187b

Першим актом революції в рільництві було те, що в якнайбільшому
маштабі й немов би на даний згори сиґнал поруйновано
хатини, розташовані на місцях роботи. Таким чином багато робітників
мусіло шукати притулку по селах і містах. Там їх, як
мотлох який, поскидано на горища, у вертепи, в льохи й закамарки
найгірших кварталів. Так тисячі ірляндських родин, що,
навіть за свідченнями англійців, пройнятих національними забобонами,
відзначалися незвичайною прихильністю до родинного
вогнища, безтурботною веселістю й чистотою родинних звичаїв,
опинилися раптом у розсадниках пороку. Чоловіки мусять тепер
шукати роботи в сусідніх фармерів, які їх наймають лише поденно,
отже, на умовах найнепевнішої форми заробітної плати;
при цьому «їм тепер доводиться далеко ходити до фарми й назад,
часто мокнути до рубчика й зазнавати інших негод, що часто
ведуть до занепаду сил, хороб і до злиднів».\footnoteA{
Там же, стор. 25.
}

«Міста рік-у-рік мусіли приймати всіх тих робітників, що
їх вважалося за надмір у сільських округах»,\footnoteA{
Там же, стор. 27.
} і після цього
ще дивуються, «що по містах та містечках є надмір робітників,
а по селах їх не вистачає!»\footnoteA{
Там же, стор. 26.
} В дійсності, цю недостачу відчувається
лише «за часів нагальних рільничих робіт, на весні
і восени, тимчасом як в інші пори року багато робітників лишаються
без роботи»;\footnoteA{
Там же, стор. 1.
} «після жнив, від жовтня до весни, ледве
чи є для них якась робота»,\footnoteA{
Там же, стор. 25.

і87h Там же, стор. 25.
} і навіть тоді, коли в них є робота,
«вони часто втрачають цілі дні й мусять зносити всілякі
перерви в роботі».1871h

Ці наслідки революції в рільництві, тобто наслідки перетворення
орної землі на пасовиська, застосування машин, якнайсуворішого
заощадження праці й т. ін., ще більше загострюються
тими зразковими лендлордами, які, замість споживати свої ренти
за кордоном, ласкаві жити в Ірляндії на своїх маєтках. Для того,
щоб закон попиту й подання лишався цілком непорушним, ці
пани витягають «тепер майже всю потрібну для них працю із
своїх дрібних фармерів, які таким чином примушені працювати
\parbreak{}  %% абзац продовжується на наступній сторінці

\parcont{}  %% абзац починається на попередній сторінці
\index{i}{0611}  %% посилання на сторінку оригінального видання
на своїх сеньйорів за заробітну плату взагалі нижчу від заробітної
плати звичайних поденників, не кажучи вже про ті невигоди
і втрати, які постають для них у наслідок того, що в критичну
пору сівби або жнив вони мусять занедбувати свої власні поля».\footnoteA{
Там же, стор. 30.
}

Отже, незабезпеченість та іреґулярність заняття, часті й довготривалі
перерви у роботі — всі ці симптоми відносного перелюднення
фігурують у звітах інспекторів адміністрації в справах
про бідних, як так само численні тяготи ірляндського рільничого
пролетаріяту. Ми пригадуємо собі, що подібні явища ми бачили
й серед англійського сільського пролетаріяту. Але ріжниця
в тому, що в Англії, промисловій країні, промислову резервну
армію рекрутують на селі, тимчасом як в Ірляндії, у рільничій
країні, рільничу резервну армію рекрутують у містах, притулках
вигнаних сільських робітників. В Англії зайві сільські робітники
перетворюються на фабричних, в Ірляндії ж, загнані в міста,
вони, справляючи тиск на заробітну плату по містах, все ж залишаються
сільськими робітниками і примушені завжди вертатися
назад у села, щоб знайти собі роботу.

Автори офіціяльних звітів резюмують свої висновки про
матеріяльний стан рільничих поденників так: «Хоч вони живуть
надзвичайно ощадно, проте їхньої заробітної плати ледве вистачає
на те, щоб здобути харчі для себе й своєї родини і заплатити
за своє житло; на одяг вони потребують додаткових доходів\dots{}
Атмосфера їхніх мешкань разом з іншими злиднями робить цю
клясу часто здобиччю тифу і сухот».\footnoteA{
Там же, стор. 21, 13.
} Після цього немає чого
дивуватися, що, за одноголосним свідченням авторів звітів,
хмура незадоволеність охоплює ряди цієї кляси, що вона сумує
за минулим, ненавидить теперішнє, зневірюється в будучині,
«піддається згубним впливам демагогів» і охоплена лише однією
idée fixe — еміґрувати до Америки. Така та блаженна країна,
на яку збезлюднення, велика малтузіянська панацея, перетворило
зелений Ерін!

Щоб побачити, як благоденствують ірляндські мануфактурні
робітники, досить одного прикладу:

«Підчас моєї останньої інспекторської подорожі на півночі
Ірляндії, — каже англійський фабричний інспектор Роберт Бекер,
— мене вразило, як один вправний ірляндський робітник
силкувався із своїх злиденних коштів дати освіту своїм дітям.
Я точно передаю його оповідання, як я чув його з власних його
уст. Що він справді вправний фабричний робітник, видно з того,
що його вживали до виробу товарів на менчестерський ринок.
Джонсон: З професії я beetler, і працюю від 6 години ранку до
11 години вночі, від понеділка до п’ятниці; суботами ми кінчаємо
о 6 годині вечора і маємо 3 години на обід і відпочинок. В мене
п’ятеро дітей. За цю працю я дістаю 10 шилинґів 6\pens{ пенсів} на тиждень;
моя дружина теж працює і заробляє на тиждень 5\shil{ шилінґів.}
\index{i}{0612}  %% посилання на сторінку оригінального видання
Старша дочка, дванадцяти років, доглядає хати. Вона наша
куховарка й однісінька помічниця. Вона підготовляє менших
до школи. Моя дружина встає разом зі мною і йде теж зі мною.
Одна дівчина, що проходить повз нашої хати, будить мене о 5\sfrac{1}{2}
годині ранку. Перед тим, як іти на роботу, ми нічого не їмо.
Вдень дванадцятилітня дочка доглядає менших дітей. Снідаємо
о 8 годині і для цього приходимо додому. Чай п’ємо раз на тиждень;
звичайно ми їмо юшку (stirabout), іноді з вівсяного борошна,
іноді з кукурудзяного, залежно від того, що можемо дістати.
Взимку до кукурудзяного борошна додаємо трохи цукру й води.
Влітку копаємо потроху картоплю, що сами садимо на клаптику
землі, а коли картопля кінчається, знову вертаємося до юшки.
Так іде з дня на день, у неділю і будні, цілий рік. Скінчивши
роботу, я ввечорі завжди почуваю надзвичайну втому. Трошки
м’яса нам винятково доводиться бачити, але дуже рідко. Троє
з наших дітей ходять до школи, і за те ми платимо за кожне по
1\pens{ пенсу} на тиждень. Наша квартирна плата становить на тиждень
9\pens{ пенсів}, торф і опалення коштують щонайменше 1\shil{ шилінґ} 6\pens{ пенсів}
на два тижні».\footnote{
«Reports of Insp. of Fact, for 31st October 1866», p. 96.
} Така ірляндська заробітна плата, таке
ірляндське життя!

Справді, злидні Ірляндії знову стали в Англії темою дня.
Наприкінці 1866 й на початку 1867~\abbr{р.} один з ірляндських земельних
маґнатів, лорд Дюфрен, взявся на сторінках «Times’a» за
розв’язання цього питання. «Яка гуманність з боку такого великого
пана!»

Із таблиці \emph{Е} видно, що в 1864~\abbr{р.} із загального зиску в \num{4.368.610}\pound{ фунтів стерлінґів} троє тільки капіталістів (Plusmacher) поклали до
своєї кишені \num{262.610}\pound{ фунтів стерлінґів}, а в 1865~\abbr{р.} тих самих троє
віртуозів «поздержливости» з \num{4.669.979}\pound{ фунтів стерлінґів} загального
зиску дістали \num{274.448}\pound{ фунтів стерлінґів}; в 1864~\abbr{р.} 26 капіталістів
дістали \num{646.377}\pound{ фунтів стерлінґів}; в 1865~\abbr{р.} 28 капіталістів —
\num{736.448}\pound{ фунтів стерлінґів}; в 1864~\abbr{р.} 121 капіталіст — \num{1.066.912}\pound{ фунтів стерлінґів} в 1865~\abbr{р.} 186 капіталістів — \num{1.320.996}\pound{ фунтів
стерлінґів}; в 1864~\abbr{р.} \num{1.131} капіталіст — \num{2.150.818}\pound{ фунтів стерлінґів},
майже половину загального річного зиску; в 1865~\abbr{р.}
\num{1.194} капіталісти дістали \num{2.418.933}\pound{ фунти стерлінґів} — більше, ніж
половину загального річного зиску. Але та левина частина,
яку проковтує з усієї річної суми земельних рент зовсім мале
число земельних маґнатів Англії, Шотляндії та Ірляндії, така
потворно велика, що англійська державна мудрість вважає за
доцільне не давати про розподіл земельної ренти такого самого
статистичного матеріялу, як про розподіл зиску. Лорд Дюфрен
— один із цих земельних маґнатів. Гадати, що ренти й зиски
колибудь можуть бути «надмірні», або що плетора (plethora)\footnote*{
— повнява. \emph{Ред.}
}
ренти і зисків є в якомусь зв’язку з плеторою народніх злиднів,
це, звичайно, є уявлення так само «непочтиве», як і «нездорове»
\parbreak{}  %% абзац продовжується на наступній сторінці

\parcont{}  %% абзац починається на попередній сторінці
\index{i}{0613}  %% посилання на сторінку оригінального видання
(unsound). Лорд тримається фактів. А факт є той, що в міру того,
як меншає кількість ірляндської людности, ірляндські ренти
зростають, що збезлюднений «добродійне» для земельного власника,
отже, і для землі, отже, і для народу, що є лише приналежність
землі. Отож він заявляє, що Ірляндія все ще перелюднена,
і що потік еміґрації пливе все ще занадто поволі. Щоб бути цілком
щасливою, Ірляндія мусить позбутися принаймні ще \sfrac{1}{3} мільйона
робітників. Не думайте собі, що цей, до того всього ще й
поетичний, лорд є лікар із школи Sangrado, який завжди, коли
він не помічав у свого недужого поліпшення, приписував йому
кровоспуск, потім знову кровоспуск, поки, нарешті, в недужого
разом з його кров’ю пропадала і його хороба. Лорд Дюфрен
вимагає нового кровоспуску лише в \sfrac{1}{3} мільйона людей
замість майже 2 мільйонів, кровоспуску, без якого дійсно ніяк
неможливо завести тисячолітнього блаженного царства на Еріні.
Докази подати не важко.
\setlength{\tabcolsep}{5pt}
\begin{table}[H]
  \caption*{Число і розмір фарм в Ірляндії 1864~\abbr{р.}}
  \small
  \centering
  \noindent\begin{tabular}{*{8}{c}}
  \toprule
  \multicolumn{2}{c}{1} &
  \multicolumn{2}{c}{2} &
  \multicolumn{2}{c}{3} &
  \multicolumn{2}{c}{4} \\

  \multicolumn{2}{c}{Фарми не більш} &
  \multicolumn{2}{c}{Фарми від 2} &
  \multicolumn{2}{c}{Фарми від 6} &
  \multicolumn{2}{c}{Фарми від 16} \\

  \multicolumn{2}{c}{від 1 акра} &
  \multicolumn{2}{c}{до 5 акрів} &
  \multicolumn{2}{c}{до 15 акрів} &
  \multicolumn{2}{c}{до 30 акрів} \\

  \cmidrule(l){1-2}
  \cmidrule(l){3-4}
  \cmidrule(l){5-6}
  \cmidrule(l){7-8}
  
  Число & Акри & Число & Акри & Число & Акри & Число & Акри \\
  \num{48.653} & \num{25.394} & \num{82.037} & \num{288.916} & \num{176.368} & \num{1.836.310} & \num{136.578} & \num{3.051.343}\\
  \\
  \toprule
  \multicolumn{2}{c}{5} &
  \multicolumn{2}{c}{6} &
  \multicolumn{2}{c}{7} &
  \multicolumn{2}{c}{8} \\

  \multicolumn{2}{c}{Фарми від 31} &
  \multicolumn{2}{c}{Фарми від 51} &
  \multicolumn{2}{c}{Фарми понад } &
  \multicolumn{2}{c}{\multirow{2}{*}{Загальна площа}} \\

  \multicolumn{2}{c}{до 50 акрів} &
  \multicolumn{2}{c}{до 100 акрів} &
  \multicolumn{2}{c}{100 акрів} &
  \\

  \cmidrule(l){1-2}
  \cmidrule(l){3-4}
  \cmidrule(l){5-6}
  \cmidrule(l){7-8}
  Число & Акри & Число & Акри & Число & Акри & \multicolumn{2}{c}{Акри} \\
  \num{71.961} & \num{2.906.274} & \num{54.247} & \num{3.983.880} & \num{31.927} & \num{8.227.807} &
  \multicolumn{2}{c}{\num{29.319.924}\hang{l}{\footnotemarkA{}}}
  \end{tabular}
\end{table}
\setlength{\tabcolsep}{\tabcolsepdef}
\footnotetextA{Загальна площа включає також торфовища й пустирі.}

Централізація знищила між 1851 і 1861~\abbr{рр.} переважно фарми
перших трьох категорій — нижче 1 і не вище 15 акрів. Вони
мусять зникнути передусім. Це дає \num{307.058} «зайвих» фармерів,
або \num{1.228.232} особи, коли при низькому пересічному обрахунку
покласти 4 особи на родину. При неймовірному припущенні, що
по закінченні революції в рільництві \sfrac{1}{4} з них знову знайде собі
роботу, все ж лишається \num{921.174} особи, що мусять еміґрувати. Категорії
4, 5 і 6, більші за 15 і не більші за 100 акрів, як це давно
відомо в Англії, занадто дрібні для капіталістичного рільництва,
а для вівчарства це зовсім незначні величини. Отже, при
тому самому припущенні, що й раніш, мусять еміґрувати ще
\num{788.761} особа, разом \num{1.709.532}. А що l’appétit vient en
\index{i}{0614}  %% посилання на сторінку оригінального видання
mangeant\footnote*{
— апетит приходить під час їди. \emph{Ред.}
}, то великі землевласники незабаром відкриють, що Ір-
ляндія із 3\sfrac{1}{2} мільйонами людности все ще бідна країна, а бідна,
тому що перелюднена, отже, збезлюднення її мусить піти ще
значно далі, щоб вона могла виконати своє справжнє призначення
бути за пасовисько для овець і рогатої худоби Англії\footnoteA{
Як окремі земельні власники й англійське законодавство пляномірно
використовували голод і викликані ним обставини, щоб силоміць
провести революцію в рільництві і звести людність Ірляндії до кількости,
вигідної для лендлордів, це я покажу докладніше у третій книзі
цього твору, у відділі про земельну власність. Там я повернуся й до становища
дрібних фермерів і сільських робітників. Тут я подам лише одну
цитату. Нассау В.~Сеніор у своєму посмертному творі «Journals, Conversations
and Essays relating to Ireland». 2 volumes. London 1868,
vol. II, p. 282 каже, між іншим, ось що: «Влучно зауважив д-р Ґ., що в
нас є закон про бідних, і що він є могутнє знаряддя, щоб забезпечити
перемогу лендлордам; друге знаряддя — еміґрація. Жоден друг Ірляндії
не побажає, щоб війна (між лендлордами й дрібними кельтськими
фармерами) тривала далі, — ще менш, щоб вона скінчилась перемогою
фармерів\dots{} Що швидше вона (ця війна) скінчиться, що швидше Ірляндія
перетвориться на пасовиська (grazing country) з порівняно нечисленною
людністю, якої треба для пасовиськ, то краще для всіх кляс». — Англійські
хлібні закони 1815~\abbr{р.} забезпечували Ірляндії монополію вільно довозити
хліб у Великобрітанію. Таким чином вони штучно сприяли рільництву.
У 1846~\abbr{р.} разом із скасуванням хлібних законів одразу знищено
і цю монополію. Не кажучи вже про всі інші обставини, лише цієї
події було досить, щоб надати потужного поштовху перетворенню ірляндської
орної землі на пасовиська, концентрації фарм і вигнанню дрібних
селян. Після того, як протягом 1815--1846~\abbr{рр.} уславляли родючість
ірляндського ґрунту і вселюдно оголосили, що з самої природи
його призначено виключно на культивування пшениці, тепер англійські
аґрономи, економісти, політики раптом зробили відкриття, що він придатний
лише для культивування кормових трав! Пан Леонс де Лявернь
поспішив повторити це по той бік каналу. Треба бути такою «серйозною»
людиною, як пан Лявернь, щоб йняти віри таким наївним теревеням.
}.

Ця корисна метода, які і все гарне на цьому світі, має свій
поганий бік. Рівнобіжно з акумуляцією земельної ренти в Ірляндії
ірляндці акумулюються в Америці. Ірляндець, що його виганяють
вівці та бики, з’являється по той бік океану, як феній.
І проти старої владарки морів повстає чимраз грізніш велетенська
молода республіка.

\settowidth{\versewidth}{Scelusque fraternae necis.} 
\begin{verse}[\versewidth]
Acerba fata Romanos agunt \\
Scelusque fraternae necis\footnote*{
Жене римлян сувора доля і злочин братовбивства. \emph{Ред.}
}.
\end{verse}

\section{Так звана первісна акумуляція}

\subsection{Таємниця первісної акумуляції}

Ми бачили, як гроші перетворюються на капітал, як за допомогою
капіталу утворюється додаткова вартість, а з додаткової
вартости — додатковий капітал. Але акумуляція капіталу має за
передумову додаткову вартість, додаткова вартість — капіталістичну
\index{i}{0615}
стичну продукцію, а ця остання — наявність великих мас капіталу
й робочої сили в руках товаропродуцентів. Таким чином,
увесь цей рух, здається, обертається у зачарованому колі, з
якого ми не можемо вийти інакше, як припустивши, що капіталістичній
акумуляції передувала «первісна» акумуляція («previous
accumulation» у Адама Сміса), акумуляція, що є не результатом
капіталістичного способу продукції, а його вихідним пунктом.

Ця первісна акумуляція відіграє в політичній економії приблизно
ту саму роль, що й первісний гріх у теології. Адам покуштував
яблука, і таким чином гріх увійшов у рід людський.
Походження цієї акумуляції пояснюють, оповідаючи про нього
анекдот давноминулих часів. За дуже давніх часів були, з
одного боку, працьовиті, розумні й насамперед ощадливі обранці,
а з другого боку — ледарі, голодранці, які прогулювали все,
що мали, і навіть ще більше. Правда, теологічна леґенда про
гріхопадіння розповідає нам, як засуждено людину їсти хліб
у поті чола свого; навпаки, історія економічного гріхопадіння
викриває, як це сталося, що є люди, які цього зовсім не потребують.
Так сталось, що перші нагромадили багатство, а в
других кінець-кінцем нічого було продавати, крім власної
шкури. І з цього гріхопадіння починаються злидні великої маси
людей, що їм все ще, не зважаючи на всю їхню працю, нічого
продати, крім себе самих, і багатство небагатьох, яке невпинно
зростає, хоч вони дуже давно перестали працювати. Такі безглузді
дитячі байки пережовує, наприклад, пан Тьєр, який,
щоб захистити propriété\footnote*{
— власність. \emph{Ред.}
}, з державно-урочистою серйозністю
підносить ці байки французам, що колись були такі дотепні.
Скоро тільки справа торкається питання про власність, то стає
священним обов’язком кожного дотримуватись поглядів дитячого
букваря як єдино правильних для всякого віку й усіх ступенів
розвитку\footnote*{
У французькому виданні Маркс додає до цього таку примітку:
«Ґете, роздратований цими дурницями, висміює їх у такому діалозі:

Вчитель: Скажи мені, звідки взялося багатство твого батька?

Дитина: Від діда.

Вчитель: А звідки воно взялося в діда?

Дитина: Від прадіда.

Вчитель: А в прадіда?

Дитина: Він загарбав його». \emph{Ред.}
}. Як відомо, в дійсній історії велику ролю відіграють
завоювання, поневолення, розбій, коротко кажучи, насильство.
Але в лагідній політичній економії з давніх часів панувала
ідилія. Право й «праця» з давніх часів були єдиним засобом
збагачення, звичайно, щоразу за винятком «поточного року».
А в дійсності методи первісної акумуляції — це все, що хочете,
тільки не ідилія.

Гроші й товари, так само, як засоби продукції й засоби існування,
самі собою не є капіталом. Їх треба перетворити на капітал.
Але саме це перетворення може відбуватися лише за певних
обставин, які сходять ось на що: два дуже різні сорти
\parbreak{}

\input{i/_0616.tex}
\parcont{}  %% абзац починається на попередній сторінці
\index{i}{0617}  %% посилання на сторінку оригінального видання
людини людиною. Однак лицарі промисловости витиснули лицарів
меча лише тим, що вони скористалися з подій, які сталися
без жодної їхньої участи. Вони піднеслися за допомогою таких
самих ницих засобів, якими колись користувалися римські пущені
з рабства, щоб стати владарями своїх патронів.

Вихідним пунктом розвитку, що утворив так найманого робітника,
як і капіталіста, було рабство робітника. Проґрес полягав
у зміні форми цього рабства, у перетворенні февдальної експлуатації
на капіталістичну експлуатацію. Щоб зрозуміти перебіг
цього процесу, нам зовсім не треба заглиблюватись у дуже
далеке минуле. Хоч перші початки капіталістичної продукції
спорадично можна спостерігати ще в XIV і XV століттях по
деяких містах на побережжі Середземного моря, однак капіталістична
ера починається лише з XVI століття. Там, де вона настає,
вже давно знищено кріпацтво і з давнього часу почався
занепад краси й гордощів середньовіччя — вільних міст.

В історії первісної акумуляції епохальне значення мають всі
ті перевороти, які служили за підойму піднесення кляси капіталістів,
що формувалася; але надто важливе значення мають
ті моменти, коли величезні маси людей раптом і силоміць відривано
від засобів їхнього існування й викидувано на ринок праці
як вільних, мов птахи, пролетарів. Експропріяція землі в сільського
продуцента, селянина, становить основу цілого процесу.
[Тому ми повинні розглянути її насамперед].\footnote*{
Заведене у прямі дужки беремо з другого німецького видання. \emph{Ред.}
} Історія її в
різних країнах набирає різного забарвлення, перебігає різні
фази в різній послідовності і в різні історичні епохи. Клясичну
форму вона має лише в Англії, яку ми тому й беремо як приклад.\footnote{
В Італії, де капіталістична продукція розвинулась найраніше,
найраніше відбувся і розклад кріпацьких відносин. Кріпака тут визволено
раніш, ніж він устиг забезпечити собі яке-небудь право давности на
землю. Тому визволення відразу перетворило його на вільного, мов
птах, пролетаря, який до того ж находить нових панів у тих містах, що
здебільша збереглися ще від римської епохи. Коли революція на світовому
ринку, що почалася з кінця XV століття, знищила торговельну
перевагу північної Італії, то постав рух зворотного напряму. Робітників
масами витискували з міст на село, де відтоді дрібна культура, організована
за типом садівництва, набула нечуваного розквіту.
}

\subsection{Експропріяція землі в сільської людности}

В Англії кріпацтво зникло фактично наприкінці XIV століття.
Величезна більшість людности\footnote{
«Дрібні землевласники, які обробляли власні поля власного працею
і тішилися скромними достатками\dots{} становили тоді куди більшу
частину нації, ніж тепер\dots{} Не менш, як 160.000 землевласників, що разом
із своїми родинами становили більше однієї сьомої частини всієї людности,
жили з того, що обробляли свої дрібні Freehold-дільниці [Freehold —
цілковита власність на землю]. Пересічний дохід цих дрібних землевласників\dots{}
оцінюється в 60--70\pound{ фунтів стерлінґів}. Обчислено, що число
тих, хто обробляв власну землю, було більше, ніж число орендарів чужої
землі». (\emph{Macaulay}: «History of England», 10th ed. London 1854,
vol. I, p. 333--334). — «Ще в останній третині XVII століття \sfrac{4}{5}
англійської людности були рільники» (там же, стор. 413). — Я цитую Маколея,
бо він, як систематичний фальсифікатор історії, по змозі применшує
подібні факти.
} складалась тоді — і ще
більше в XV столітті — з вільних селян, які господарювали
самостійно, хоч за якими февдальними вивісками ховалася їхня
власність. У великих панських маєтках вільний фармер витиснув
\index{i}{0618}  %% посилання на сторінку оригінального видання
управителя (bailiff, Vogt, який раніше сам був кріпаком. Рільничі
наймані робітники складалися почасти з селян, що використовували
свій вільний час, працюючи у великих землевласників,
почасти із самостійної, відносно й абсолютно малочисельної
кляси найманих робітників у власному значенні слова. Останні
фактично разом з тим теж були селянами, що самостійно господарювали,
бо, крім своєї заробітної плати, вони діставали котедж
і 4 або й більш акрів поля. Окрім того, вони спільно з справжніми
селянами користалися з громадської землі, на якій вони пасли
свою худобу й яка разом з тим давала їм паливо — дрова, торф
і т. ін.\footnote{
Ніколи не слід забувати, що навіть кріпак був не тільки власником
— правда, власником, що мусив платити данину — земельних
парцель, що належали до його дому, але й співвласником громадської
землі. «Селянин є тут (у Шльонську) кріпак» («Le paysan у (en Silésie)
est serf»). Проте ці кріпаки (serfs) є посідачі громадської землі. «Досі
ще не вдалося схилити жителів Шльонську до поділу громадських
земель, тимчасом як у Новій Марці немає вже жодного села, де б цього
поділу не проведено з якнайбільшим успіхом» («On n’a pas pu encore
engager les Silésiens au partage des communes, tandis que dans la nouvells
Marche, il n’y a guère de village où ce partage ne soit exécuté avec le plue
grand succès»). (\emph{Mirabeau}: «De la Monarchie Prussienne», London 1788,
vol. II, p. 125, 126).
} По всіх країнах Европи февдальна продукція характеризується
поділом землі поміж якомога більшим числом васалів.
Сила февдального пана, як і всякого суверена, спиралась не на
розміри його ренти, а на число його підданців, а останнє залежало
від числа селян, що господарювали самостійно.\footnote{
Японія з її суто февдальною організацією земельної власности
та з її розвиненим дрібноселянським господарством дає куди вірніший
образ європейського середньовіччя, ніж усі наші книги з історії, здебільшого
подиктовані буржуазними забобонами. Занадто воно вже вигідно
бути «ліберальним» коштом середньовіччя.
} Тим-то,
хоч англійська земля після норманського завоювання була
поділена на величезні баронства, що з них окремі часто охоплювали
900 давніших англосаксонських лордств, проте вона була
вкрита дрібними селянськими господарствами, що лише де-не-де
перемежалися великими панськими маєтками. Та і відносини,
за одночасного розквіту міст, яким відзначається XV століття,
уможливили те народнє багатство, яке так красномовно
змальовує канцлер Фортескю у своїх «Laudibus Legum Angliae»,
але вони виключали капіталістичне багатство.

Пролог до перевороту, що створив основу капіталістичного
способу продукції, відбувся в останній третині XV і в перші
десятиліття XVI століть. Масу вільних, як птиці, пролетарів
було викинуто на робітничий ринок у наслідок розпуску февдальних
\index{i}{0619}  %% посилання на сторінку оригінального видання
дружин, які, як слушно зауважує Джемс Стюарт, «всюди
марно наповняли будинки й двори». Хоч королівська влада,
сама продукт буржуазного розвитку, у своєму прагненні абсолютного
суверенітету силоміць прискорювала розпуск тих дружин,
проте вона зовсім не була його єдиною причиною. Сами
великі февдали, що стояли в якнайгострішій опозиції до королівської
влади й парляменту, створили куди численніший пролетаріят,
силоміць зганяючи селян із тієї землі, на яку селяни
мали таке саме февдальне право, як і сами великі февдали, і
узурпуючи їхні громадські землі. Безпосередній поштовх до цього
в Англії дав головним чином розквіт фляндрійської вовняної
мануфактури й відповідний зріст цін на вовну. Стару февдальну
шляхту поглинули февдальиі війни, нова ж була дитиною свого
часу, для якого гроші були силою над силами. Тому гаслом її
стало перетворення орного поля на пасовиська для овець. Геррісон
у своїх «Description of England. Prefixed to Holinshed’s
Chronicles» описує, як експропріація дрібних селян руйнує
країну. «What care our great incroachers!» (Що нашим великим
узурпаторам до того!). Селянські житла й робітничі котеджі
силоміць поруйновано або засуджено на руїну. «Коли порівняти, —
каже Геррісон, — давніші інвентарі кожного лицарського маєтку,
то виявиться, що безліч хат і дрібних селянських господарств
зникло, що земля годує тепер далеко менше людей, що багато
міст занепало, хоч деякі нові міста проквітають\dots{} Я міг би багато
чого розповісти про міста й села, які поруйновано задля
овечих пасовиськ і в яких позалишалися самі тільки панські
замки». Нарікання таких старих хронік завжди трохи прибільшені,
— проте вони точно змальовують вражіння, яке революція
в продукційних відносинах справила на самих сучасників.
Порівнюючи твори канцлера Фортескю й Томаса Мора, ми виразно
бачимо ту безодню, що відділяє XV і XVI століття. Із свого
золотого віку англійська робітнича кляса, як слушно каже Торнтон,
без ніяких переходових ступенів попала в залізний.

Законодавство злякалося цього перевороту. Воно ще не
стояло на тій височині цивілізації, де «Wealth of the Nation»,\footnote*{
— національне багатство. \emph{Ред.}
}
тобто утворення капіталу й нещадна експлуатація та павперизація
народньої маси, вважається за ultima Thule\footnote*{
— вершину. \emph{Ред.}
} всякої державної
мудрости. У своїй історії Генріха VII Бекон каже: «Того
часу (1489) збільшилися нарікання на перетворення орного поля
в пасовиська (для овець тощо), за якими легко може доглядати
кілька чабанів; а фарми, що здавалися в доживотну оренду, на декілька
років або на рік (з чого жила велика частина yeomen’ів\footnote*{
— вільних рільників. \emph{Ред.}
})
перетворено на панські маєтки. Це привело до занепаду народу,
а через це й до занепаду міст, церков, десятин\dots{} У лікуванні
цього лиха король і парлямент виявили в той час мудрість, гідну
подиву\dots{} Вони вжили заходів проти цієї узурпації громадських
\parbreak{}  %% абзац продовжується на наступній сторінці

\input{i/_0620.tex}
\input{i/_0621.tex}
\input{i/_0622.tex}
\parcont{}  %% абзац починається на попередній сторінці
\index{i}{0623}  %% посилання на сторінку оригінального видання
панським «полюбовницям». Навіть наймані сільські робітники
були ще тоді співпосідачами громадських земель. Десь близько
1750 р. yeomanry зникли,\footnote{
«А Letter to Sir Т. С. Bunbury, Brt.: On the High Price of Provisions.
By a Suffolk Gentleman», Ipswich 1795 p. 4. Навіть фанатичний
оборонець великого фармерства, автор «Inquiry into the Connection
betveen the present Price of provisions and the size of Farms», London
1773, p. 139, каже: «Я дуже жалкую за тим, що позвикали наші yeomanry,
ця кляса людей, що дійсно підтримувала незалежність нашої нації;
з сумом бачу, що їхні землі опинились тепер у руках монополістів-лордів
і позаорендовані дрібними фармерами на умовах, мало чим кращих від
тих умов, у яких були васалі, завжди готові відгукнутися на заклик при
всякій капосній справі» («І most truly lament the loss of our yeomanry, that
set of men who really kept up the independence of this nation; and sorry I am
to see their land now in the hands of monopolizing lords, tenanted out to
small farmers, who hold their leases on such conditions as to be little better
than vassals ready to attend a summons on every mischievous occasion»).
} а останніми десятиліттями XVIII
століття зник і останній слід громадської власности рільників.
Ми залишаємо тут осторонь суто економічні пружини революції
у рільництві. Нас цікавлять її насильні підойми.\footnote*{
У французькому виданні останні два речення подано так:
«Залишаючи осторонь суто економічні впливи, що підготували експропріацію
рільників, ми переходимо тут до підойм, вжитих, щоб насильно
прискорити цю експропріяцію». («Le Capital etc.», ch. XXVII,
p. 319). Ред.
}

За реставрації Стюартів землевласники законодатним шляхом
добилися тієї узурпації, яка повсюди на континенті відбулася й
без законодатних церемоній. Вони знищили февдальний земельний
лад, тобто скинули з себе всякі повинності щодо держави,
«відшкодували» державу податками, накинутими на селянство
й решту народньої маси, присвоїли собі сучасне право приватної
власности на маєтки, на які вони мали лише февдальні права,
і, нарешті, октроювали ті закони про оселення (law of settlement),
які, mutatis mutandis, мали такий самий вплив на англійських
селян рільників, як едикт татарина Бориса Годунова на російське
селянство.

«Glorious Revolution» (славетна революція) привела до влади
разом з Вільгельмом III Оранським\footnote{
Про особисту мораль цього буржуазного героя можна судити,
між іншим, ось із чого: «Обдарування леді Оркней великими маєтками
в Ірляндії у 1695 р. — це явний доказ королівської прихильности й впливу
леді... Милі послуги леді Окрней були, кажуть, нечисті послуги устами».
(«The large grant of lands in Ireland to Lady Orkney, in 1695, is a public
instance of the king’s affection, and the lady’s influence... Lady Orkney’s
endearing offices are supposed to have been-foeda labiorum ministeria»).
(У Sloane Manuscript Collection, у Брітанському музеї, № 4224.
Манускрипт називається: «The character and behaviour of King William,
Sunderland etc. as represented in Original Letters to the Duke of
Shrewsbury from Somers, Halifax, Oxford, Secretary Vernon etc.». У ньому
повно курйозів).
} земельних і капіталістичних
присвоювачів додаткової вартости (Plusmacher). Вони освятили
нову еру, практикуючи в колосальних розмірах крадіж
державних маєтків, що перед тим мав лише помірні розміри.
Державні землі роздаровувано, продавано за безцінь або приєднувано
\index{i}{0624}  %% посилання на сторінку оригінального видання
до приватних маєтків простою узурпацією.\footnote{
«Незаконне відчуження коронних земель, почасти через продаж,
почасти через дарування, становить скандальний розділ в англійській
історії... величезне ошукання нації (gigantic fraud on the nation)».
(F. W. Newman: «Lectures on Political Economy», London 1851, p. 129,
130). — [Подробиці про те, як сучасні англійські великі землевласники
придбали свої маєтки, див. в «Our old Nobility. By Noblesse Oblige», London
1879. — Ф. E.].
} Все це
робилося без найменшого додержання етикету законности. Присвоєне
таким шахрайським способом державне майно разом з
понаграбовуваним церковним майном, оскільки останнє не втрачено
підчас республіканської революції, становить основу сучасних
княжих доменів англійської олігархії.\footnote{
Див., наприклад, памфлет E. Burke про герцоґську родину
Бедфордів, нащадком якої є Джон Рессел, «the tomtit of liberalism».
} Буржуазні
капіталісти сприяли цій операції, між іншим, для того, щоб
перетворити землю на предмет вільної торговлі, поширити сферу
великої рільничої продукції, збільшити приплив із села вільних,
як птиці, пролетарів і т. д. До того ж нова земельна аристократія
була природною союзницею нової банкократії, цієї фінансової
шляхти, що тільки но вилупилася з яйця, і великих мануфактуристів,
що тоді спирались на охоронні мита. Англійська буржуазія
так само правильно чинила в своїх інтересах, як шведські
міщани, що, навпаки, спільно з своєю економічною твердинею
— селянством, підтримували королів, що силоміць відбирали
від олігархії коронні землі (починаючи від 1604 р. й
пізніше, за Карла X й Карла XI).

Громадська власність — цілком відмінна від щойно розглянутої
державної власности — була старогерманською інституцією,
що й далі існувала під покровом февдалізму. Ми бачили
вже, як насильна узурпація цієї громадської власности, що її
здебільшого супроводило перетворення орної землі на пасовиська,
почалася наприкінці XV століття й тривала далі в XVI столітті.
Але тоді цей процес відбувався лише як індивідуальний
насильний акт, проти якого законодавство даремно боролося протягом
150 років. Проґрес XVIII століття виявляється в тому,
що тепер сам закон стає знаряддям грабування народньої землі,
хоч поруч із цим великі фармери вшивають і своїх дрібних незалежних
приватних метод.\footnote{
«Фармери забороняють cottager’aм (бурлакам) тримати якубудь
живу істоту, крім самих себе, під тією причіпкою, що коли вони триматимуть
худобу або дробину, то крастимуть корм з клунь. Вони кажуть
також: тримайте cottager’iв у біді, якщо ви хочете, щоб вони були працьовитими.
Але справжній факт той, що фармери таким способом узурпують
усі права на громадські землі». («A Political Enquiry into the Consequences
of enclosing Waste Lands», London 1785, p. 75).
} Парляментська форма цього грабування
є «Bills for Inclosures of Commons» (закони про обгороджування
громадських земель), інакше кажучи, декрети, за допомогою
яких землевласники сами собі дарують народню землю
у приватну власність, — декрети експропріяції народу. Сер Ф. М.
Ідн у своїй хитромудрій оборонній адвокатській промові
\parbreak{}  %% абзац продовжується на наступній сторінці

\input{i/_0625.tex}
\parcont{}  %% абзац починається на попередній сторінці
\index{i}{0626}  %% посилання на сторінку оригінального видання
коло них заняття й утримання»\footnote{
Reverend Addington: «An Inquiry into the Reasons for and against
enclosing open-fields», London 1772, p. 37 — 43 і далі.
}. Під приводом обгородження
сусідні лендлорди присвоювали собі не тільки землі, що лежали
перелогом, але часто-густо ще й землі, оброблювані або самою
громадою, або орендарями, які наймали їх у громади за певну
плату. «Я кажу тут про обгороджування відкритих полів і земель,
уже оброблених. Навіть ті письменники, що боронять
inclosures, визнають, що воно збільшує монополію великих
фарм, підвищує ціни засобів існування і продукує збезлюднення\dots{}
Навіть обгороджування гулящих земель, як це тепер практикують,
відбирає у бідняків частину їхніх засобів існування і надзвичайно
збільшує фарми, що і без того вже занадто великі»\footnote{
Dr. R. Price: «Observations on Reversionary Payments 6 th ed.
By W. Morgan», London 1805, vol. II, p. 155. Прочитайте Форстера,
Едінґтона, Кента, Прайса й Джемса Андерсона й порівняйте з ними жалюгідне
сикофантське базікання Мак Куллоха у його каталогу «The
Literature of Political Economy», London 1845.
}.
«Коли земля, — каже д-р Прайс, — попадає в руки небагатьох
великих фармерів, то дрібні фармери [що про них він раніш говорив
як про «масу дрібних власників і фармерів, що утримують
себе та свої родини продуктами оброблюваної ними землі, вівцями,
птицею, свиньми і~\abbr{т. ін.}, яких вони пасуть на громадській
землі, так що їм мало доводиться купувати засобів існування»]
перетворюються на людей, що примушені заробляти на життя
працею на інших і купувати на ринку все їм потрібне\dots{} Може,
тепер більше працюють, бо більше силують до праці\dots{} Міста й
мануфактури зростатимуть, бо до них зганяють більше людей,
які шукають роботи. Такий є неминучий вплив концентрації
фарм, і так вона фактично впливала в цьому королівстві протягом
багатьох років»\footnote{
Там же, стор. 147.
}. Загальний вплив inclosures він резюмує
так: «У цілому становище нижчих народніх кляс майже з
кожного боку погіршало, дрібні землевласники й фармери позведені
до рівня поденників і наймитів, і в той самий час куди тяжче
стало за таких умов заробляти на життя»\footnote{
Там же, стор. 159. Пригадаймо собі стародавній Рим. «Багаті
захопили у свої руки більшу частину неподілених земель. Тогочасні
обставини викликали в них упевненість, що земель у них уже не відберуть,
і тому вони поскуповували сумежні дільниці бідняків, почасти за
згодою останніх, почасти відбираючи їх силою, так що замість поодиноких
нив вони почали обробляти великі маєтки. При цьому вони для
рільництва й скотарства вживали рабів, бо вільних людей у них забрали б
від праці до військової служби. Володіння рабами давало їм ше йту
велику користь, що раби, звільнені від військової служби, могли спокійно
розмножуватись і мали багато дітей. Таким чином вельможні постягали
до своїх рук усі багатства, і ціла країна аж кишіла рабами. Навпаки,
італійців ставало раз-у-раз менше, їх нищили злидні, податки й військова
служба. Коли ж наставали мирні часи, то вони були засуджені
на повне безділля, бо землею володіли багачі, які на її оброблення вживали
рабів замість вільних людей». (Арріаn: «Römische Bürgerkriege»,
1, 7). Це місце стосується до часу перед законом Ліцінія. Військова служба,
що так дуже прискорила руйнування римських плебеїв, була також
головним засобом, що ним Карл Великий дуже прискорив перетворення
вільних селян на феодально залежних і кріпаків.
}. Справді, узурпація
громадської землі й революція в рільництві, що супроводила
цю узурпацію, мали такий гострий вплив на рільничих робітників,
що, як каже сам Ідн, між 1765 і 1780~\abbr{рр.} їхня заробітна
\parcont{}
\index{i}{0627}  %% посилання на сторінку оригінального видання
плата почала падати нижче мінімуму, і її доводилося доповнювати
офіціальною допомогою для бідних. Їхньої заробітної плати,
каже він, «ледве вистачало, щоб задовольнити доконечні життєві
потреби».

Послухаймо на хвилину ще одного оборонця inclosures і противника
д-ра Прайса. «Хибний висновок, ніби відбувається
збезлюднення, якщо не видно більше людей, що марнують свою
працю у відкритому полі\dots{} Коли після перетворення дрібних
селян на людей, що мусять працювати на інших, пускається у
рух більше праці, то це ж користь, якої нація [до неї «перетворені»,
ясна річ, не належать] мусить собі бажати\dots{} Продукту
буде більше, коли їхню комбіновану працю вживатимуть на
одній фармі; таким чином утворюється додатковий продукт для
мануфактур, і через це число мануфактур, цих розсипищ золота
нашої нації, зростає пропорційно до кількости продукованого
збіжжя»\footnote{
«An Inquiry into the Connection between the present Price of
Provisons etc.», p. 124, 129. Подібне, але з протилежною тенденцією, ми
читаємо в іншого автора: «Робітників проганяють з їхніх котеджів і
примушують іти до міст шукати там роботи — але це дає більше додаткового
продукту, і таким чином капітал зростає» («Working men are driven
from their cottages, and forced into the towns to seek for employment; —
but then a larger surplus is obtained, and thus Capital is augmented»).
(«The Perils of the Nation», 2 nd ed. London 1843, p. XIV).
}.

Зразок стоїчного душевного спокою, з яким політико-економ
розглядає якнайнахабніші порушення «святого права власносте»
й найгрубіші насильства над особою, скоро тільки вони
потрібні для створення основи капіталістичного способу продукції,
показує нам, між іншим, сер Ф. М. Ідн, отой до того
ще на торійський штаб забарвлений «філантроп». Ціла низка грабунків,
жорстокостей і народніх злигоднів, що супроводили насильну
експропріяцію народу, починаючи від останньої третини
XV аж до кінця XVIII століття, приводить його тільки до
такого «утішного» кінцевого висновку: «Треба було встановити
належну (due) пропорцію між орною землею та пасовиськами.
Ще протягом цілого XIV й більшої частини XV століття
один акр пасовиська припадав на 2, 3 й навіть 4 акри орного
поля. В середині XVI століття ця пропорція змінилася так, що
2 акри пасовиська припадили на 2 акри орного поля, а пізніше
2 акри пасовиська на 1 акр орного поля, аж поки, нарешті, встановилась
пропорція — 3 акри пасовиська на 1 акр орного поля».

У XIX столітті зникла, звичайно, навіть і згадка про зв’язок
поміж рільником і громадською власністю. Не кажучи вже
про пізніші часи, чи одержало колибудь селянство хоч шеляг
відшкодування за ті \num{3.511.770} акрів громадської землі, які
\parbreak{}  %% абзац продовжується на наступній сторінці

\parcont{}  %% абзац починається на попередній сторінці
\index{i}{0628}  %% посилання на сторінку оригінального видання
між 1801 і 1831~\abbr{рр.} загарбовано в нього і парляментськими актами подаровано
лендлордами лендлордам?

Нарешті, останнім великим процесом експропріяції земель у
рільників був так званий Clearing of Estates («очищення маєтків»,
у дійсності очищення їх від людей). «Очищення» — це кульмінаційний
пункт усіх розглянутих досі англійських метод експропріяції.
Як ми бачили в попередньому відділі, при розгляді сучасних відносин,
тепер, коли вже немає незалежних селян, яких треба було б проганяти,
справа доходить до «очищення» землі від котеджів, так що рільничі
робітники на оброблюваній ними землі навіть не находять уже місця,
потрібного для їхніх жител. А що означає «Clearing of Estates» у
власному значенні слова, це ми можемо пізнати лише в горішній Шотляндії,
цій обітованій країні сучасної літератури романів. Там цей процес
визначається своїм систематичним характером, тим широким маштабом,
що в ньому він воднораз відбувається (в Ірляндії землевласники за
одним разом очищають землю від декількох сел; у горішній
Шотляндії йдеться про очищення земельних просторів
завбільшки як німецькі герцоґства) — і, нарешті, особливою формою
загарбовуваної земельної власности.

Кельти горішньої Шотляндії складалися з кланів, з яких
кожний був власником заселеної ним землі. Представник клану,
його голова, або «великий чоловік», був тільки номінальним власником цієї землі,
цілком так само, як англійська королева є номінальна власниця всієї
національної території. Коли англійському урядові пощастило придушити
внутрішні війни поміж цими «великими чоловіками» і їхні
постійні напади на рівнини долішньої Шотляндії, то голови кланів
зовсім не покинули свого старого розбійницького ремества;
вони тільки змінили його форму. Своєю власною владою вони перетворили своє
номінальне право власности на приватне право власности,
а що вони при цьому наражались на опір з боку членів клану, то вони
вирішили просто прогнати їх силою. «Якийбудь англійський король міг би з
таким самим правом претендувати на те, щоб позаганяти своїх підданих у море»
— каже професор Ньюмен\footnote{
«А king of England migth as well claim to drive all his subjects into the sea».
(\emph{F.~W.~Newman}: «Lectures on Political Economy», London 1851, p. 132).
}. Цю революцію що почалась у Шотляндії після останнього повстання претендента,
можна простежити в її перших фазах у творах сера Джемса Стюарта\footnote{
Стюарт каже: «Ренти з цих земель (він помилково переносить цю економічну
категорію на данину, що її taskmen\footnote*{
васаль. \emph{Ред.}
} платить голові клану) є цілком незначні порівняно з розмірами останніх;
щождо числа осіб, які живуть з оренди, то мабуть виявиться, що шматок
землі в гірських місцевостях Шотляндії прогодовує вдесятеро більше людей,
аніж земля такої самої вартости в найбагатших провінціях».
(«Works etc., ed. by General Sir James Steuart, his son», London 1801, vol. I,
ch. 16, p. 104).
} й Джемса Андерсона\footnote{
\emph{James Anderson}: «Observation on the means of exciting a spirit of
National Indusrty etc.», Edinburgh 1777.
}. У XVIII столітті
\index{i}{0629}  %% посилання на сторінку оригінального видання
прогнаним із сел ґаелам одночасно заборонено еміґрувати,
щоб силоміць позаганяти їх у Ґлезґо та в інші фабричні міста\footnote{
В 1860~\abbr{р.} насильно експропрійованих вивезено до Канади, при
чому їм понадавано багато брехливих обіцянок. Декотрі з них повтікали
в гори й на сусідні острови. За ними погнались поліціянти, втікачі вступили
з ними в бійку і таки повтікали.
}.
Як приклад тієї методи, що панувала в XIX столітті\footnote{
«У гірських місцевостях, — писав у 1814~\abbr{р.} Б’юкенен, коментатор
А.~Сміта, — старі відносини власности день-у-день зазнають насильного
перевороту\dots{} Лендлорд, не звертаючи уваги на спадкових орендарів
(цю категорію тут теж ужито помилково), віддає землю тому, хто дає найбільшу
ціну, і коли цей останній є меліоратор (improver), то він одразу заводить
нову систему культури. Земля, раніше рясно вкрита дрібними рільниками,
була заселена відповідно до того, скільки вона давала продукту;
за нової системи поліпшеної культури й збільшених рент треба продукувати
якнайбільше продукту з якомога меншими витратами, і задля цієї
мети усувають усі ті руки, що поробилися некорисними\dots{} Викинуті з
рідних сел шукають собі засобів існування по фабричних містах тощо».
(\emph{David Buchanan}: «Observations on etc. A.~Smith’s Wealth of Nations»,
Edinburgh 1814, vol. IV, p. 144). «Шотляндські вельможі експропріювали
цілі родини так, наче виполювали бур’ян; вони поводилися з селами
та їхньою людністю так, як індійці у своїй помсті з лігвами хижих звірів.
Людину продають за смушок, за баранячу ногу, навіть за щось
менше\dots{} Підчас нападу на північні провінції Китаю на нараді монголів
було запропоновано винищити людність і її землю перетворити на пасовиська.
Цю пропозицію виконало багато лендлордів горішньої Щотляндії
у своїй власній країні проти своїх власних земляків». (\emph{George
Ensor}: «An Inquiry concerning the Population of Nations», London 1818,
p. 215, 216).
}, тут досить
навести «очищення», пороблені герцоґинею Сотерлендською.
Ця економічно освічена особа, скоро взяла в свої руки управління,
зараз же вирішила розпочати радикальне економічне лікування
краю й перетворити на пасовиська для овець ціле графство,
що його людність попередніми подібними процесами була
вже зменшена до \num{15.000} душ. Від 1814 до 1820~\abbr{р.} цих \num{15.000} жителів,
приблизно \num{3.000} родин, систематично проганяли й винищували.
Всі їхні села позруйновано й попалено, всі їхні поля
поперетворювано на пасовиська. Між брітанськими солдатами,
присланими для екзекуції, та місцевою людністю доходило до
боїв. Одна стара жінка згоріла в полум’ї своєї хати, не схотівши
залишити її. Таким чином ця мадам присвоїла собі \num{794.000} акрів
землі, що від незапам’ятних часів належала кланові. Для прогнаних
тубільців вона відвела на узмор’ї приблизно \num{6.000} акрів
землі, по 2 акри на родину. Ці \num{6.000} акрів до того часу лежали
пустирем, не приносячи їхній власниці ніякого доходу. У своїх
благородних почуттях герцоґиня пішла так далеко, що здала
цю землю в оренду пересічно по 2\shil{ шилінґи} 6\pens{ пенсів} ренти за акр
членам клану, які протягом століть проливали за її рід свою кров.
Всю землю, загарбану в клану, вона поділила на 29 великих
овечих фарм, посадивши на кожній одним-одну родину, здебільшого
англійських фармерських наймитів. У 1825~\abbr{р.} замість
\num{15.000} ґаелів там було вже \num{131.000} овець. Викинута на узмор’я
частина тубільців намагалася прожити з рибальства. Вони поробилися
\index{i}{0630}  %% посилання на сторінку оригінального видання
амфібіями й жили, як каже один англійський письменник,
наполовину на землі й наполовину на воді, але і з того
і з другого вони жили тільки наполовину\footnote{
Коли нинішня герцоґиня Сотерлендська з великою пишністю
приймала в Лондоні пані Бічер-Стов, авторку «Хижини дядька Тома»,
щоб показати свою симпатію до рабів-негрів американської республіки, —
підчас громадянської війни, коли кожне «благородне» англійське серце
співчувало рабовласникам, вона разом з іншими аристократками благорозумно
забула про цю симпатію, — я змалював у «New-York Tribune»
становище сотерлендських рабів. (Частину моєї статті Кері подав у витягах
у «The Slave Trade», London 1853, p. 202, 203). Мою статтю передруковано
в одній шотляндській газеті, і вона викликала «ввічливу» полеміку
поміж цією газетою і сикофантами Сотерлендів.
}.

Але бравим ґаелам довелося ще тяжче спокутувати своє гірсько\dash{}романтичне
ідолопоклонство перед «великими людьми»
клану. Запах риби лоскотав великим людям у носі. Вони занюхали
тут щось зисковне і заорендували узмор’я великим риботорговцям
з Лондону. Ґаелів прогнано вдруге\footnote{
Цікаві подробиці про цю торговлю рибою подає пан \emph{Давид Уркварт}
«Portfolio, New Series». — \emph{Н.~В.~Сеніор} y своєму вищецитованому посмертному
творі «Journals, Conversations and Essays relating to Ireland»,
London 1868, кваліфікує «процедуру в Sutherlandshire, як одне з найдобродійніших
очищень (clearings), що їх люди пам’ятають».
}.

Але кінець-кінцем частину овечих пасовиськ перетворено
на мисливські парки. Як відомо, в Англії нема справжніх лісів.
Дичина по парках вельмож — це конституційна домашня худоба,
гладка, як лондонський aldermen\footnote*{
член міської ради. \emph{Ред.}
}. Тим то Шотляндія є останнє
пристановище цієї «благородної пристрасти». «У гірських місцевостях,
— каже Сомерс в 1848~\abbr{р.}, — лісова площа значно
поширилась. Тут, по цей бік Gaick’a, ви бачите новий ліс Glenfeshie,
а там, по другий бік, новий ліс Ardverikie. Там же ви маєте
й Blak-Mount, величезну пущу, нещодавно тільки заведену.
Із сходу на захід, від околиць Aberdeen’a й до скель Oban’а,
тягнеться тепер безперервна смуга лісів, тимчасом як по інших
частинах гірського краю стоять нові ліси Loch Archaig, Glengarry,
Glenmoriston і ін. Перетворення земель ґаелів на пасовиська\dots{}
загнало їх на неродючі землі. Тепер олені й сарни (Rotwild)
починають витискувати овець, кидаючи цим ґаелів у ще
гірші злидні\dots{} Мисливські парки\footnoteA{
У шотляндських «deer forests» (мисливських парках) немає жодного
дерева. Овець виганяють геть, на їхнє місце в голі гори приганяють
оленів, і це називають «deer forest». Отже, тут немає навіть лісової культури.
} й народ не можуть існувати
одне побіч одного. В усякому разі хтобудь із них мусить очистити
місце. Якщо місця для полювання протягом найближчої чверти
віку зростатимуть щодо кількости й простору, як і минулої
чверти, то жодного ґаела не залишиться на його рідній землі.
Цей рух серед землевласників гірських місцевостей спричинено
почасти модою, аристократичними примхами, мисливським запалом
тощо, а почасти землевласники торгують дичиною, маючи
на меті виключно зиск. Бо це факт, що шматок гірської землі,
\parbreak{}  %% абзац продовжується на наступній сторінці

\parcont{}  %% абзац починається на попередній сторінці
\index{i}{0631}  %% посилання на сторінку оригінального видання
призначений для полювання, у багатьох випадках дає куди
більш зиску, ніж пасовисько для овець\dots{} Аматор, що шукає
місця для полювання, дає за нього таку ціну, яку тільки дозволяє
йому глибина його кишені. На гірські місцевості звалилося горе
не менш жорстоке від того, що його зазнала Англія через політику
норманських королів. Для оленів і сарн приділяють дедалі
більше простору, тимчасом як людей заганяють у дедалі тісніше
коло\dots{} У народу відбирають одну вільність по одній\dots{} І поневолення
день-у-день зростає. «Очищування» й проганяння
народу власники практикують як твердий принцип, як сільськогосподарську
доконечність, цілком так само, як по диких місцевостях
Америки й Австралії викорінюють дерева та кущі, і ця
операція проходить спокійно, діловито».\footnote{
\emph{Robert Sommers}: «Letters from the Highlands; or the Famine
of 1847», London 1848, p. 12--28 і далі. Ці листи появилися спочатку
в «Times’i». Англійські економісти, звичайно, пояснювали голод серед
ґаелів у 1847 році — перелюдненням. У всякому разі, ґаели, мовляв,
«натискували» на свої засоби існування. — «Clearing of Estates», або,
як воно називалося у Німеччині, Bauernlegen, розвинулось у Німеччині
з особливою силою після тридцятилітньої війни і ще в 1790~\abbr{р.} викликало
селянські повстання в саксонському курфюрстві. Воно панувало особливо
у східній Німеччині. В більшості пруських провінцій тільки Фрідріх II
забезпечив селянам право власности. Здобувши Шльонськ, він примусив
землевласників відбудувати хати, клуні тощо й забезпечити селянські
господарства худобою та знаряддям. Йому потрібні були солдати для
його армії і платники податків для його державної скарбниці. А в тім
як приємно жилося селянам за Фрідріха II з його фінансовою політикою
та системою урядування, цією мішаниною деспотизму, бюрократизму
й февдалізму, можна побачити з ось яких слів його великого прихильника
Мірабо: «Одним із головних багатств рільника північної Німеччини
є льон. Та, на нещастя для роду людського, це тільки знаряддя проти
злиднів, а не шлях до добробуту. Безпосередні податки, панщина й інші
февдальні повинності всякого роду руйнують німецького селянина, який
ще до того платить посередні податки на все, що купує\dots{} у довершення
його руйнації він не сміє продавати своїх продуктів, де і як захоче; він
не сміє купувати потрібні йому продукти в тих купців, що продавали б
їх йому за найдешевшу ціну. Всі ці причини непомітно руйнують його,
і він не був би спроможний платити в строк безпосередніх податків,
коли б не займався прядінням; останнє являє собою для нього підмогу,
бо дає корисне заняття його дружині, дітям, слугам, наймитам і йому
самому. Але яке ж це злиденне життя, навіть з отією підмогою! Влітку
підчас оранки та жнив він працює, як каторжник; він лягає о дев’ятій
годині і встає о другій, щоб тільки упоратися з своєю працею; взимку
він мусів би відживлятися, маючи більше спочинку; але йому не вистачить
збіжжя на хліб і на насіння, коли він продасть харчові продукти,
щоб сплатити податки. Отже, він мусить прясти, щоб заткати цю діру\dots{}
мусить прясти з найбільшою пильністю та запопадливістю. Таким чином
селянин узимку лягає опівночі або о першій годині і встає о п’ятій-шостій
удосвіта; або він лягає о дев’ятій і встає о другій — і так день-у-день
ціле своє життя, за винятком неділь\dots{} Це надмірно довге неспання й ця
надмірна праця виснажують організм людини; ось чому на селі чоловіки
й жінки старіються далеко швидше, ніж по містах». («Le lin fait donc
une des grandes richesses du cultivateur dans le Nord d’Allemagne. Malheureusement
pour l’espèce humaine, ce n’est qu’une ressource contre la
misère, et non un moyen de bien-être. Les impôts directs, les corvées, les
servitudes de tout genre, écrasent le cultivateur allemand, qui paie encore
les impôts indirects dans tout ce qu’il achète\dots{} et pour comble de ruine, il
n’ose pas vendre ses productions où et comme il le veut; il n’ose pas acheter
ce dont il a besoin aux marchands qui pourraient le lui livrer au meilleur
prix. Toutes ces causes le ruinent insensiblement, et il se trouverait hors
d'état de payer les impôts directs à l’échéance sans la filerie: elle lui offre
une ressource, en occupant utilement sa femme, ses enfants, ses servants, ses valets,
et lui même: mais quelle pénible vie, même aidée de secours! En été, il
travaille comme un forçat au labourage et à la récorlte; il se couche’ à
9 heures et se lève à deux, pour suffire aux travaux; en hiver il devrait réparer
ses forces par un plus grand repos; mais il manquera de grains pour le
pain et les semailles, s’il se défait des denrées qu’il faudrait vendre pour payer
les impôts. Il faut done filer pour suppléer à ce vide\dots{} il faut y apporter
la plus grande assiduité. Aussi le paysan se couche-t-il en hiver à
minuit, une heure, et se lève à cinq ou six; ou bien il se couche à neuf, et
se lève à deux, et cela tous les jours de sa vie si ce n’est le dimanche. Cet
excès de veille et de travail usent la nature humaine, et de là vient qu’hommes
et femmes vieillissent beaucoup plutôt dans les campagnes que dans
les villes»). (\emph{Mirabeau}: «De la Monarchie Prussienne», Londres 1788,
vol. III, p. 212, 222).

Додаток до другого видання. У квітні 1866 року, 18 років після
опублікування вищецитованої праці Роберта Сомерса, професор Леон
Леві прочитав у Society of Arts доповідь про перетворення овечих пасовиськ
на мисливські парки, що в ній він змалював проґрес у спустошенні
в шотляндських гірських місцевостях. Він каже, між іншим, таке: «Проганяння
людности й перетворювання орного поля на пасовиська для
овець становили найвигідніший засіб діставати доходи без витрат\dots{} Заміна
овечих пасовиськ мисливськими парками, стала звичайним явищем
по гірських місцевостях. Овець виганяють дикі звірі, як раніше виганяли
людей, щоб очистити місце для овець\dots{} У Форфаршірі можна пройти
від маєтків графа фон Делгуза до маєтків Джон Ґротс, не виходячи
зовсім з мисливських лісів. У багатьох (із цих лісів) уже давно живуть
лисиці, дикі коти, куниці, тхорі, ласки й альпійські зайці, недавно
найшли собі туди шлях кролі, білиці й пацюки. Величезні земельні простори,
які в шотляндській статистиці фігурували як винятково родючі
та обширні луки, нині стоять поза всякою культурою і поліпшеннями і
призначені виключно на мисливську забаву небагатьох осіб, тай то ця
забава триває лише недовгий період року».

Лондонський «Economist» від 2 червня 1866~\abbr{р.} пише «Одна шотляндська
газета між іншими новинами з останнього тижня подає й таку:
«Одну з найкращих овечих фарм у Sutherlandshire, за яку недавно, як
вийшов строк оренди, давали \num{1.200}\pound{ фунтів стерлінґів} річної ренти, перетворено
на deer forest!» Февдальні інстинкти виявляються так само\dots{}
як за тих часів, коли нормандський завойовник\dots{} зруйнував 36 сел, щоб
створити new forest\dots{} Два мільйони акрів, що охоплюють деякі з найродючіших
земель Шотляндії, перетворено на цілковиту пустелю. Природні
трави з Glen Tilt’a вважалось за найпоживнішу пашу у графстві Perth;
мисливський ліс у Ben Aulder давав найкращу траву великій окрузі
Badenoch; частина лісу Blak-Mount була найкращим у Шотляндії пасовиськом
для чорних овець. Про розміри земельної площі, спустошеної
для аматорів полювання, можна собі скласти уявлення з того факту, що
ця площа простором далеко більша, аніж ціле графство Perth. Як багато
джерел продукції втрачає країна в наслідок цього насильного спустошення,
видно з того, що лісова площа Ben Aulder могла б прогодувати
\num{15.000} овець, і що площа цього лісу становить лише \sfrac{1}{30} частину всієї
мисливської площі Шотляндії\dots{} Вся ця мислівська земля є цілком непродуктивна\dots{}
з неї така сама користь, як коли б її затопити у хвилях
Північного моря. Міцна рука законодавства мусіла б покласти край цим
імпровізованим пустелям».
}

\index{i}{0632}  %% посилання на сторінку оригінального видання
Грабування церковних маєтків, шахрайське відчуження державних
земель, крадіж громадських земель, узурпаторське і з
нещадним тероризмом проведене перетворення февдальної й
кланової власности на сучасну приватну власність, — такі є
ідилічні методи первісної акумуляції. Вони завоювали поле
для капіталістичного рільництва, прилучили землю до капіталу
й утворили для міської промисловости потрібний приплив вільних
як птиці, пролетарів.

\index{i}{0633}  %% посилання на сторінку оригінального видання
\subsection{Криваве законодавство проти експропрійованих, починаючи
з кінця XV століття. Закони для зниження заробітної плати}

Вигнаних у наслідок розпуску февдальних дружин і ґвалтовної,
переводжуваної поштовхами експропріяції земель, цих
вільних, як птиці, пролетарів, мануфактура, що тоді поставала,
ніяк не могла поглинути так само швидко, як вони з’являлися
на світ. З другого боку, ці люди, раптово вибиті з їхньої звичайної
життєвої колії, не могли так само раптом призвичаїтись до
дисципліни нових обставин. Вони масами перетворювалися на
жебраків, розбійників, волоцюг, почасти з власного нахилу,
але здебільшого під примусом обставин. Звідси криваве законодавство
проти волоцюзтва по всіх країнах Західньої Европи
наприкінці XV і протягом усього XVI століття. Батьків теперішньої
робітничої кляси покарано насамперед за те, що їх перетворено
на волоцюг і павперів. Законодавство розглядало
їх як «добровільних» злочинців і виходило з того припущення,
що від їхньої доброї волі залежить і далі працювати серед старих
обставин, які вже не існували.

В Англії це законодавство почалося за Генріха VII.

Генріх VIII, 1530: старі й нездатні до праці жебраки дістають
дозвіл жебракувати. Навпаки, працездатних волоцюг слід карати
батогами й замикати до в’язниць. Їх слід прив’язувати ззаду до
тачки й катувати, доки потече з їхнього тіла кров, а потім узяти
від них присягу, що вони повернуться туди, де народились,
або туди, де перебували останні три роки, і «візьмуться до роботи»
(to put himself to labour). Яка жорстока іронія! Акт 27
Генріха VIII повторює цей закон та загострює його новими додатками:
якщо когобудь удруге зловлять на волоцюзтві, то його
треба знову покарати батогами та відрізати йому піввуха; а
коли кого утретє зловлять на волоцюзтві, то його, як тяжкого
злочинця й ворога громадянства, треба покарати на смерть.

Едвард VI: статут першого року його королювання, 1547,
приписує віддавати кожного, хто ухиляється від праці, у рабство
тій особі, що донесе на нього як на нероба. Хазяїн повинен годувати
свого раба хлібом і водою, давати йому легкі напої й такі
м’ясні покидьки, які він вважатиме за відповідні. Він має право
батогами й кайданами силувати його до всякої, навіть найогиднішої
праці. Коли раб відлучиться на два тижні, то його слід
засудити на довічне рабство й наложити на його лоб або щоку
тавро «S»; коли ж він утече втретє, то його слід покарати на
смерть як зрадника держави. Хазяїн може його продати, відписати
у спадщину, віддати в найми, як раба, цілком так само,
\parbreak{}  %% абзац продовжується на наступній сторінці

\parcont{}  %% абзац починається на попередній сторінці
\index{i}{0634}  %% посилання на сторінку оригінального видання
як усяке інше рухоме майно або худобу. Коли раби задумають
що проти панства, то їх так само слід покарати на смерть. Мирові
судді повинні на заяву панів розшукувати рабів-утікачів.
Коли виявиться, що волоцюга три дні тинявся без праці, то його
слід відіслати до місця його народження, випекти на його грудях
розпеченим залізом тавро «V» і, закувавши в кайдани, вживати
його там на дорожні та всякі інші подібні роботи. Коли волоцюга
неправильно показує своє місце народження, то він на кару
за це мусить стати довічним рабом цього місця, його мешканців
або корпорації, і на нього слід накласти тавро «S». Кожний
має право відібрати у волоцюг їхніх дітей і тримати їх при собі
як учнів, хлопців до 24 років, дівчат до 20 років. Коли вони втечуть,
то до вищезазначеного віку вони мусять бути рабами їхніх
хазяїнів, які мають право на своє бажання заковувати їх у
кайдани, бити батогами й~\abbr{т. ін.} Кожний хазяїн може накидати
залізний ланцюжок на шию, ноги або руки свого раба, щоб краще
його пізнавати й бути певнішим, що він не втече\footnote{
Автор «Essay on Trade etc.», 1770, зауважує: «За королювання
Едварда VI англійці, здається, цілком серйозно почали підохочувати
мануфактури й давати бідним заняття. Це видно з одного вартого уваги
статуту, де сказано, що на всіх волоцюг треба накладати тавра, і~\abbr{т. ін.}
(Там же, стор. 8).
}. Остання
частина цього статуту передбачає, що деякі бідні повинні працювати
на ту округу або тих осіб, що дають їм їсти й пити та знаходять
для них працю. Такий рід парафіяльних рабів зберігся
в Англії аж до XIX віку під назвою roundsmen (Umgeher).

Єлисавета, 1572: жебраків понад 14 років, що не мають дозволу
жебракувати, слід люто бити батогами та випікати їм тавра
на лівому вусі, коли ніхто не згоджується взяти їх на службу
на два роки; коли це повториться, то жебраків понад 18 років
слід покарати на смерть, якщо ніхто не згоджується взяти їх
на службу на два роки; коли їх спіймають на цьому втретє, то
їх слід нещадно покарати на смерть як державних зрадників.
Аналогічні статути: 18 Єлисавети с. 13 і 1597\footnoteA{
Томас Мор каже у своїй «Утопії»: «Так то й трапляється, що
жаденний і ненаситний обжера, оця справжня чума своєї батьківщини,
захоплює тисячі акрів землі, обгороджує їх тином або живоплотом, або
силою і кривдами може так зацькувати їхніх власників, що вони примушені
продати все своє майно. Тим або іншим способом, не києм, то палицею,
їх примушують виселятися — цих бідних, простих, бідолашних людей!
Мужчини, женщини, чоловіки, жінки, сироти, вдовиці, охоплені
горем матері з немовлятками, всі члени родини, бідні на засоби існування,
але багаті числом, бо рільництво потребувало багато робочих рук.
Вони бредуть геть, кажу я, з своїх рідних місць, до яких вони звикли,
і ніде не находять собі місця відпочинку; продаж усього їхнього домашнього
скарбу, хоч і невеликої вартости, міг би за інших обставин принести
деяку виручку, але, раптом опинившися на вулиці, вони мусять продати
його за безцінь. І коли вони проблукають таким чином аж поки проїдять
останню шажину, то що ж їм лишається, як не красти? Але тоді
їх вішають, додержуючи всіх форм закону. Або жебракувати? Але тоді
їх кидають у в’язницю, як волоцюг, за те, що вони тиняються й не працюють,
хоч їм ніхто не хоче дати роботу, як би жагуче вони її добивалися».
З-поміж цих бідних утікачів, що їх, як каже Томас Мор, просто
таки примушують красти, «за королювання Генріха VIII покарано на
смерть \num{72.000} великих і малих злодіїв». (Hollinshed: «Description of
England», vol. I, p. 186). За часів Єлисавети волоцюг вішали цілими лавами;
звичайно не проходило року, щоб там або деінде не повісили 300
або 400 осіб». (Strype: «Annals of the Reformation and Establishment
of Religion, and other Various Occurrences in the Church of England
during Queen Elisabeth’s Happy Reign», 2 nd ed. 1725, vol. II). За тим самим
Стріпом в Сомерсетшірі за один лише рік покарано на смерть 40 осіб,
потавровано 35, покарано батогами 37, а 183 «очайдушних злочинців»
випущено на волю. Однак, каже, він, «у це велике число обвинувачених,
у наслідок недбальства мирових суддів і безглуздого співчуття
з боку народу, не входить і п’ятина гідних кари злочинців». Він додає:
«Інші графства Англії були не у кращому становищі, ніж Сомерсетшір,
а багато навіть у гіршому».
}.

\index{i}{0635}  %% посилання на сторінку оригінального видання
Яків І: особу, що тиняється й жебрачить вважається за волоцюгу.
Мирові судді в Petty Sessions\footnote*{
малих сесіях. \emph{Ред.}
} уповноважені віддавати
таких осіб на прилюдну кару батогами й замикати їх у в’язниці
на шість місяців, спіймавши їх перший раз, і на два роки, спіймавши
їх удруге. Підчас ув’язнення їх слід карати батогами так часто
й так багато, як це вважають за відповідне мирові судді\dots{} Непоправних
і небезпечних волоцюг слід таврувати, випікаючи
їм на лівому плечі літеру «R», і вживати до примусових праць,
а коли їх іще раз спіймають на жебрацтві — нещадно карати
на смерть. Ці постанови мали силу аж до початку XVIII віку,
скасовано їх тільки актом 12 Анни с. 23.

Подібні закони були й у Франції, де в середині XVII століття
утворилось у Парижі так зване «королівство волоцюг»
(royaume des truands). Ще на початку королювання ЛюдовікаХІV
(ордонанс від 13 липня 1777~\abbr{р.}) кожну здорову людину між 16 і
60 роками засилали на ґалери, коли вона не мала засобів існування
й певної професії. Подібні постанови є в статуті Карла V
для Нідерляндів від жовтня 1537~\abbr{р.}, перший едикт штатів і міст
Голляндії від 19 березня 1614~\abbr{р.}, плякат Сполучених Провінцій
від 25 червня 1649~\abbr{р.} і~\abbr{т. д.}

Таким чином, сільську людність, силоміць позбавлену землі,
вигнану й перетворену на волоцюг, за допомогою жахливо терористичних
законів, батогами, тавруванням і катуванням привчили
до дисципліни, доконечної для системи найманої праці.

Мало того, що умови праці виступають на одному полюсі як
капітал, а на другому полюсі як люди, що не мають на продаж
нічого, крім своєї власної робочої сили. Недосить також і того,
що їх примушують добровільно продавати себе. З розвитком капіталістичної
продукції розвивається робітнича кляса, що в наслідок
свого виховання, традиції, звичок визнає вимоги цього способу
продукції за само собою зрозумілі закони природи. Організація
розвиненого капіталістичного процесу продукції ламає
всякий опір; постійне створювання відносного перелюднення
тримає закон попиту й подання, а тим то й заробітну плату
в межах, відповідних до потреби самозростання капіталу; німий
гніт економічних відносин закріпляє панування капіталіста над
\parbreak{}  %% абзац продовжується на наступній сторінці

\input{i/_0636.tex}
\parcont{}  %% абзац починається на попередній сторінці 
\index{i}{0637}  %% посилання на сторінку оригінального видання 
аніж тоді».\footnote{
Sophisms of Free Trade. By a Barrister», London 1850 p., p. 206).
Він лукаво додає: «Ми завжди готові були виступити в обороні підприємців.
Невже ж таки нічого не можна зробити для робітників?»
} Закон установив тариф заробітної плати для міста
й села, для відштучної й поденної праці. Сільські робітники
повинні найматися на рік, міські ж «на вільному ринку». Під
загрозою ув’язнення забороняється платити заробітну плату
вишу, ніж її приписує статут, але тих, що беруть таку вищу
плату, карають важче, ніж тих, що платять її. Так, ще статут
Єлисавети про учнів, в розділах 18 і 19, визначає десятиденне
ув’язнення для того, хто заплатить вищу заробітну плату, і
тритижневе для того, хто її бере. Статут із р. 1360 посилив ці
кари й навіть уповноважував хазяїна шляхом фізичного примусу
приневолювати робітників працювати на основі законного тарифу.
Всі спілки, угоди, присяги й т. ін., що ними зобов’язалися поміж
собою мулярі й теслярі, проголошено за недійсні. Об’єднання
робітників розглядається як важкий злочин, починаючи від
XIV віку аж до 1825 р., коли скасовано закони проти об’єднань.
Дух робітничого статуту з р. 1349 і всіх пізніших яскраво виявляється
в тому, що держава, щоправда, встановлювала максимум
заробітної плати, але аж ніяк не мінімум її.

В XVI столітті становище робітників, як це відомо, дуже
погіршало. Грошова плата підвищилася, але зовсім не пропорційно
до зневартнення грошей і відповідного зросту товарових цін.
Отже, на ділі заробітна плата спала. Проте закони, що мали на
меті зменшення заробітної плати, все ще мали силу; все ще відрізували
вуха й таврували тих, «кого ніхто не хотів брати на
службу». Статут про учнів 5, Єлисавети с. З, уповноважує мирових
суддів встановлювати певну заробітну плату і змінювати
її відповідно до пори року й зміни товарових цін. Яків І поширив
це реґулювання праці також і на ткачів, прядунів та на всякі
інші категорії робітників,\footnote{
З одного параграфу статуту 2, Якова І, с. 6, видно, що деякі
фабриканти сукна, які були разом з тим і мировими суддями, дозволяли
собі офіціяльно визначати тариф заробітної плати у своїх власних майстернях.
У Німеччині дуже часто видавали статути для пониження
заробітної плати, особливо після тридцятирічної війни. «Поміщикам
дуже дошкуляв брак челяді й робітників у збезлюднених місцевостях.
Всім мешканцям села заборонено було винаймати кімнати неодруженим
чоловікам і жінкам; про всіх таких осіб треба було доносити урядові
й замикати їх до в’язниці, якщо вони не хотіли бути слугами, навіть
і тоді, коли вони утримували себе з якоїсь іншої роботи, працюючи на
полі у селян за поденну плату або навіть торгуючи грішми і збіжжям.
} а Ґеорґ II поширив закони проти
робітничих об’єднань на всі мануфактури.

За власне мануфактурного періоду капіталістичний спосіб
продукції настільки зміцнився, що міг зробити законодатне
реґулювання заробітної плати так само можливим, як і зайвим,
а все ж, про всякий випадок, хотілося мати напоготові зброю
з старого арсеналу. Ще акт 8 Ґеорґа II забороняв давати кравцям
підмайстрам Лондону й околиць більш ніж 2 шилінґи 7 1/2 пенса
поденної плати, за винятком випадків загального трауру; ще
\index{i}{0638}  %% посилання на сторінку оригінального видання 
акт 13 Ґеорґа III с. 68 уповноважував мирових суддів реґулювати
заробітну плату шовкоткачів; ще 1796 р. треба було двох
присудів вищих судових інстанцій, щоб вирішити, чи судові
накази мирових суддів про заробітну плату мають силу і для
нерільничих робітників; ще 1799 р. один парляментський акт
потвердив, що плату копальневих робітників Шотляндії має
реґулювати статут Єлисавети і два шотляндські акти 1661 і 1671 рр
Але якого радикального перевороту зазнали за той час економічні
обставини, показав один нечуваний в англійській палаті
громад випадок. Тут, де більш ніж протягом 400 років фабриковано
закони про той максимум, що його ні в якому разі не
могла перевищувати заробітна плата, Вайтбред запропонував
у 1796 р. встановити законодатно мінімум заробітної плати для
рільничих робітників. Піт спротивився цьому, згоджуючись
однак, що «становище бідних жахливе» (cruel). Нарешті, в
1813 році закони про реґулюваннн заробітної плати скасовано.
Вони стали смішною аномалією від того часу, коли капіталіст
почав реґулювати працю на фабриці своїм приватним законодавством,
а плату сільського робітника почали доповнювати до доконечного
мінімуму податком на користь бідних. Але постанови
робочих статутів щодо контрактів між хазяїнами й робітниками,
щодо строків звільнення й т. ін., постанови, за якими хазяїна,
що зламав контракт, можна позивати лише до цивільного суду,
а робітника, що зламав контракт, до карного суду, — мають ще
й тепер повну силу.

Жорстокі закони проти об’єднань впали в 1825 р. в наслідок
грізної позиції, що її заняв пролетаріят. А проте вони впали
лише частково. Деякі прекрасні рештки старих статутів зникли
лише в 1859 році. Нарешті, парляментський акт з 29 червня
1871 р. через законодатне визнання тред-юньойонів претендував
усунути останні сліди цього клясового законодавства. Але
інший парляментський акт, виданий того самого дня (An act
to amend the criminal law relating to violence, threats and molestation),
фактично відновлював попередній стан у новій формі.
Цими парляментськими викрутасами всі засоби, що ними могли
користатися робітники підчас страйку або льокавту (страйк
об’єднаних фабрикантів, що одночасно закривають фабрики),
виключено з загального права і підведено під виняткові карні
закони, що їхня інтерпретація належала самим фабрикантам
у їхній ролі мирових суддів. Два роки перед тим та сама палата

(«Kaiserliche Privilegien und Sanctiones für Schlesien», I, 125). Цілих
сто років у наказах князів не вгавають гіркі нарікання поміщиків на
злісну й непокірливу голоту, що не хоче коритися суворому режимові,
не хоче задовольнятися приписаною законом платою; поодиноким поміщикам
заборонено було давати вищу плату, ніж установлює такса, вироблена
властями округи. А проте умови служби були після війни
часом кращі, ніж сто років пізніше; у Шльонську челядь ще в 1652 р.
діставала м’ясо двічі на тиждень, а в нашому столітті там по деяких
округах челядь дістає м’ясо лише тричі на рік. І заробітна плата була
після війни вища, ніж у наступних століттях». (G. Freitag).
\parbreak{}  %% абзац продовжується на наступній сторінці

\parcont{}  %% абзац починається на попередній сторінці
\index{i}{0639}  %% посилання на сторінку оригінального видання
громад і той самий пан Ґледстон з усім відомою добропорядністю
подали законопроєкта про скасування всіх виняткових карних
законів проти робітничої кляси. Але далі, ніж до другого читання,
не дійшло, і таким чином справу затягувано так довго, доки
нарешті «велика ліберальна партія», об’єднавшися з торі, набралася
сміливости рішуче виступити проти того самого пролетаріяту,
що поставив її до влади. Не задовольнившись цією зрадою
«велика ліберальна партія» дозволила англійським суддям,
що завжди виляли хвостом, вислуговуючись перед панівними
клясами, відкопати знову старі закони проти «конспірації»
і застосовувати їх проти робітничих об’єднань. Як бачимо,
лише проти волі й під натиском мас англійський парлямент відмовився
від законів проти страйків і тред-юньойнів, після того,
як сам він з безсоромним егоїзмом цілих п’ятсот років був за
перманентного тред-юньйона капіталістів.

На самому початку революційної бурі французька буржуазія
зважилася відібрати в робітництва щойно завойоване право
асоціяцій. Декретом з 14 червня 1791 р. вона проголосила всі
робітничі об’єднання за «замах на волю й деклярацію прав людини»,
караний штрафом у 500 фунтів і позбавленням права
активного громадянства на один рік.\footnote{
Перша стаття цього закону каже: «Через те, що скасування
всякого роду корпорацій громадян того самого стану й тієї самої професії
є одна з корінних основ французької конституції, забороняється відбудовувати
подібні корпорації хоч під яким приводом і в хоч якій формі»
(«L’anéantissement de toutes espèces de corporations des citoyens du même
état et profession étant l’une des bases fondamentales de la constitution
française, il est déféndu de les rétablir de fait sous quelque prétexte et
sous quelque forme que ce soit»). Стаття четверта каже, що коли «громадяни,
які належать до тієї самої професії, ремества або фаху, змовляться
між собою, укладуть угоду з тією метою, щоб спільно відмовитися
або лише за певну ціну згоджуватися допомагати своїм ремеством і праце,
— всі оті змови й угоди\dots{} будуть оголошені за протиконституційні
й за замах на волю й деклярацію прав людини й т. ін.» (\dots{} des citoyen
attachés aux mêmes professions, arts et métiers prenaient des délibérations,
faisaient entre eux des conventions tendantes à refuser de concert ou
à n’accorder qu’a un prix déterminé le secours de leur industrie ou de
leurs travaux, les dites délibérations et conventions\dots{} seront déclarées
inconstitutionelles et attentatoires à la liberté et à la déclaration des
droits de l’homme etc.»), отже, за державний злочин, цілком так само,
як у старих робітничих статутах». («Révolutions de Paris». Paris 1791,
vol. VIII, p. 523).
} Цей закон, що державнополіційними
заходами втиснув конкуренцію між капіталом і
працею у рамки, вигідні для капіталу, пережив революції і зміни
династій. Навіть терористичний уряд і той лишив його непорушним.
І лише нещодавно його викреслено з Code Pénal. Немає
нічого характеристичнішого за привід, що ним мотивують
цей буржуазний державний переворот. «Хоч і бажана річ, —
каже Шапельє, доповідач цього закону, — щоб заробітна плата підвищилась
понад теперішній її рівень, для того, щоб той, хто її
дістає, звільнився від абсолютної, майже рабської залежности,
зумовлюваної недостачею доконечних засобів існування», однак
\parbreak{}  %% абзац продовжується на наступній сторінці

\parcont{}  %% абзац починається на попередній сторінці
\index{i}{0640}  %% посилання на сторінку оригінального видання
робітники не сміють порозуміватись між собою щодо своїх інтересів,
не сміють спільно діяти, щоб послабити «свою абсолютну,
майже рабську залежність», бо саме цим «вони порушують свободу
своїх сі-devant maîtres\footnote*{
— колишніх цехових хазяїнів. \emph{Ред.}
}, теперішніх підприємців (свободу
тримати робітників у рабстві!), бо об’єднання проти деспотії
колишніх цехових хазяїнів є — відгадайте! — є відновлення цехів,
скасованих французькою конституцією\footnote{
\emph{Bûchez et Roux: } «Histoire Parlementaire», t. X, p. 193—195.
}.
\subsection{Генеза капіталістичних фермерів}
Розглянувши процес ґвалтовного створення вільних, як птиці,
пролетарів, криваву дисциплину, що перетворила їх на найманих
робітників, брудні державні заходи, що, разом із збільшенням
ступеня експлуатації праці, поліційними засобами збільшували
акумуляцію капіталу, ми опинилися перед питанням: звідки взялися
первісно капіталісти? Бо експропріяція сільської людности
створює безпосередньо лише великих землевласників. Що ж до
генези фармерів, то ми можемо її, так би мовити, намацати рукою,
бо це повільний процес, що триває цілі століття. Сами кріпаки,
що поряд з ними існували і вільні дрібні землевласники, перебували
в душе різних майнових відносинах, а тому й емансипація
їх відбулася серед дуже різних економічних умов.

В Англії перша форма фармера є bailiff\footnote*{– управитель маєтка. \emph{Ред.}
}, що сам є кріпак.
Його становище подібне до становища староримського villicus’a,
але з вужчою сферою діяльности. У другій половині XIV століття
bailiff’а замінює фармер, якому лендлорд постачає насіння,
худобу й рільниче знаряддя. Становище його не дуже відрізняється
від становища селянина. Він лише більше експлуатує
найманої праці. Швидко він стає metayer, фармером, що замість
грошової ренти платить землевласникові частиною продукту.
Він постачає одну частину потрібного для рільництва капіталу,
лендлорд — другу. Цілий продукт обидва ділять між собою у
пропорції, визначеній контрактом. В Англії ця форма швидко
зникає, щоб віддати місце фармерові у власному значенні, який
збільшує вартість свого власного капіталу, вживаючи найманих
робітників, і частину додаткового продукту віддає лендлордові
грішми або in natura як земельну ренту.

Поки, протягом XV віку незалежний селянин і рільничий
наймит, що поруч служби з найму ще й самостійно господарював,
збагачуються своєю працею, становище фармера і розміри його
продукції лишаються однаково помірні. Рільнича революція
останньої третини XV віку, що потім тривала майже цілий XVI вік
(за винятком, однак, останніх десятиліть), збагачує фармера з
такою швидкістю, з якою вона збіднює сільську людність\footnote{
«Фармери, — каже Гаррісон y своєму «Description of England», —
що їм раніше важко було платити 4\pound{ фунти стерлінґів} ренти, платять тепер 40, 50, 100\pound{ фунтів стерлінґів} і вважають, що вони зробили невигідну операцію,
коли по закінченні орендного контракту вони не зможуть відкласти
для себе суми, що дорівнювала б ренті за шість-сім років».
}.
\index{i}{0641}  %% посилання на сторінку оригінального видання
Узурпація громадських випасів і~\abbr{т. ін.} дозволяє фармерові значно
збільшити кількість своєї худоби майже без витрат, а худоба дає
йому багато угноєння для землі.

В ХVІ столітті сюди долучається ще один вирішально-важливий
момент. Того часу орендні контракти були довготермінові,
часто на 99 років. Безперервне падіння вартости благородних
металів, а тому й грошей дало фармерам золоті плоди. Передусім
воно, не кажучи вже про всі інші вищерозглянуті обставини,
понизило заробітну плату. Частину цієї заробітної плати долучувано
до фармерського зиску. Невпинний зріст цін на збіжжя,
вовну, м’ясо, одне слово, на всі рільничі продукти, збільшував
грошовий капітал фермера без жодної його участи, тимчасом
як земельна рента, яку він мусив платити, зменшувалась у наслідок
зневартнення грошей підчас тривання контракту\footnote{
Про вплив зневартнення грошей у XVI столітті на різні кляси
суспільства див. «А Compendious or Briefe Examination of Certayne Ordinary
Complaints of Diverse of our Countrymen in these our Days. By
W.~S., Gentleman», London 1581. Діялогічпа форма цього твору сприяла
тому, що його довго приписували Шекспірові, і ше року 1751 цей твір
знову видано під його йменням. Автор цього твору є Вільям Стафорд.
В одному місці лицар (knight) резонує так;

Лицар: «Ви, мій сусіде, рільнику, ви, пане крамарю, і ви, добродій
котлярю, так само і всі інші ремісники, ви можете гаразд зарадити
собі. Бо наскільки всі речі стали тепер дорожчі ніж були раніш, настільки
ви підвищуєте ціни своїх товарів і продукти своєї праці, які продаєте.
А ми не маємо нічого, що могли б продавати, ми не маємо нічого, на що
могли б підвищити ціну, щоб відшкодувати себе за ті речі, які ми мусимо
купувати». В іншому місці лицар так запитує доктора: «Скажіть мені,
будь ласка, яких саме людей ви маєте на думці? І насамперед хто, на
ваш погляд, не терпить при цьому жодних втрат?» Доктор: «Я маю на
увазі всіх тих, що живуть із купівлі й продажу; бо коли вони дорого
купують, то потім так само дорого і продають». Лицар: «А хто ті, що,
як ви кажете, виграють на цьому?» Доктор: «Звичайно, всі ті, що господарюють
на маєтках або фармах, платячи стару ренту; бо, платячи за старою
нормою, вони продають за новою, тобто вони платять за свою землю
дуже дешево, а всі продукти, що виростають на ній, продають дорого\dots{}»
Лицар: «Ну, а що ж то за люди ті, що, як ви кажете, втрачають на цьому
більше, ніж ті виграли». Доктор: «Вся шляхта, джентлмени та всі інші,
що живуть з фіксованої ренти або з фіксованого утримання, або сами
не господарюють на своїй землі, або не займаються купівлею і продажем».
(Knight: «You, my neighbour, the husbandman, you Maister Mercer,
and you Goodman Copper, with other artificers, may save yourselves
metely well. For as much as all things are deerer than they were, so much
do you arise in the pryce of your wares and occupations that yee sell agayne.
But we have nothing to sell where by we might advance ye pryce there of,
to countervaile those things that we must buy agayne». — «I pray you, what
be those sorts that ye meane. And, first, of those that yee thinke should
have no losse hereby?» — Doktor: «I meane all these that live by buying
and selling, for, as they buy deare, they sell thereafter». — Knight: «What
is the next sorte that yee say would win by it?» — Doktor: «Marry, all
such as have takings or fearmes in their owne manurance (тобто cultivation)
at the old rent, for where they pay after the olde rate, they sell after the newe — that is, they pays for their lande good cheape, and sell all things
growing thereof deare\dots{}» Knight: «What sorte is that which, ye sayde
should have greater losse hereby, than these men had profit? — Doktor:
«It is all noblemen, gentlemen, and all other that live either by a stinted
rent or stypend, or do not manure (cultivate) the ground, or doe occupy no
buying and selling»).
}. Таким
чином фармер багатів одночасно коштом своїх найманих
\index{i}{0642}  %% посилання на сторінку оригінального видання
робітників і коштом свого лендлорда. Отже, немає нічого дивного
в тому, що наприкінці XVI віку Англія мала клясу багатих
для тодішніх часів «капіталістичних фармерів»\footnote{
У Франції régisseur, на початку середніх віків управитель і збирач
повинностей на користь февдальним панам, швидко стає homme
d’affaires\footnote*{
— ділком. \emph{Ред.}
}, що за допомогою вимагання, шахрайства й~\abbr{т. ін.}, виростає
на капіталіста. Ці régisseurs сами були іноді вельможними панами. Наприклад:
«Оцей рахунок пан Жак де Торес, лицар-кастелян у
Безансоні, подає своєму панові, що веде рахунки в Діжоні для пана
герцога і графа Бургундського, про ренти, які належались від зазначеного
кастелянства від 25 грудня 1359~\abbr{р.} до 28 грудня 1360 р». («C’est
li compte que messire Jacques de Thoraine, chevalier chastelain sor Besançon
rent es seigneur tenant les comptes à Dijon pour monseigneur le
duc et comte de Bourgoigne, des rentes appartenant à la dite chastellenie,
depuis XXVe jour de décembre MCCCLIX jusq’au XXVIIIe jour de décembre
MCCCLX»). (\emph{Alexis Monteil} : «Traité des Matériaux, manuscrits etc.» v. I,
p. 234 і далі). Вже тут видно, як у всіх сферах суспільного життя левина
пайка попадає до рук посередників. Наприклад, в економічній царині фінансисти,
биржовики, купці, дрібні крамарі збирають вершки з усіх справ;
у царині громадського права адвокат скубе супротивників; у політиці
депутат має більше значення, ніж виборець, міністер — більше ніж суверен
; у релігії бога відсувають на задній плян святі «посередники», а цих
останніх знову таки витискують попи, які й собі є неминучі посередники
між добрим пастирем і його вівцями. У Франції, як і в Англії, великі
февдальні території були поділені на безліч дрібних господарств, але
на умовах куди менш сприятливих для сільської людности. В XIV столітті
виникли оренди, фарми або terriers. Число їх невпинно зростало і
значно перевищило \num{100.000}. Вони платили грішми або in natura земельну
ренту, що її розмір коливався від \sfrac{1}{12} до \sfrac{1}{5} продукту. Terriers називалися
денами, підленами і~\abbr{т. ін.} (fiefs, arrière-fiefs), залежно від вартости й
розміру доменів, що з них деякі мали лише декілька арпенів. Всі ці
terriers мали в тій або іншій мірі судову владу над людністю; такої влади
було чотири ступені. Можна зрозуміти, який гніт відчурала сільська
людність під владою усіх цих дрібних тиранів. Монтейль каже, що у Франції
за тих часів було \num{160.000} судів там, де тепер досить \num{4.000} судових установ
(залічуючи сюди й мирових суддів).
}.

\subsection {Зворотний вплив рільничої революції на промисловість.
Утворення внутрішнього ринку для промислового капіталу}

Експропріяція і зганяння сільської людности, що відбувалися
поштовхами й постійно відновлювалися, постачали, як ми бачили,
для міської промисловости щораз нові маси пролетарів, які
стояли цілком поза цеховими відносинами, — мудра обставина,
яка примушує старого А.~Андерсона (не треба сплутувати його
з Джемсом Андерсоном) в його історії торговлі увірувати в безпосереднє
втручання провидіння. Ми мусимо ще хвилину спинитися
на цьому елементі первісної акумуляції. Розрідженню незалежної,
самостійно господарюючої сільської людности відпові-
\parbreak{}  %% абзац продовжується на наступній сторінці

\parcont{}  %% абзац починається на попередній сторінці
\index{i}{0643}  %% посилання на сторінку оригінального видання
не лише згущення промислового пролетаріату, як ось Жофруа
Сент-Ілер пояснює згущення світової матерії в одному
місці розрідженням її в іншому\footnote{
У своїх «Notions de Philosophie Naturelle», Paris 1838.
}. Не зважаючи на зменшення
числа обробників землі, вона давала стільки ж продуктів, скільки
й раніш, а то й більше, бо революція у відносинах земельної
власности відбувалася в супроводі поліпшених метод обробітку
землі, поширеної кооперації, концентрації засобів продукції й~\abbr{т. ін.}, бо сільських найманих робітників не тільки примушували
інтенсивніше працювати\footnote{
Цей пукт підкреслює сер Джемс Стюарт.
}, але й поле продукції, на якому
вони працювали сами на себе, дедалі більше скорочувалось.
Отже, разом із звільненням частини сільської людности звільняються
також і її колишні засоби існування. Вони перетворюються
тепер на речовий елемент змінного капіталу. Селянин,
позбавлений власности, мусить купувати вартість цих засобів
існування у свого нового пана, у промислового капіталіста, в
формі заробітної плати. З тубільним сировинним матеріялом,
постачуваним для промисловости рільництвом, сталося те саме,
що і з засобами існування. Він перетворився на елемент сталого
капіталу.

Припустімо, наприклад, що частину вестфальських селян,
які за часів Фрідріха II всі займалися прядінням, хоч і не шовку,
то льону, силоміць експропрійовано й зігнано з землі, а другу
частину їх, ту, що залишилась, перетворено на наймитів великих
фармерів. Одночасно виростають великі льонопрядні й ткацькі
підприємства, де ці «звільнені» працюють як наймані робітники.
Льон має цілком такий самий вигляд, як і раніш. Жодне волоконце
в ньому не змінилось, але нова соціяльна душа вселилась
у його тіло. Він становить тепер частину сталого капіталу власника
мануфактури. Поділений раніше поміж безлічі дрібних
продуцентів, що сами обробляли його й разом із своїми родинами
випрядали маленькими порціями, він тепер зосереджений у руках
одного капіталіста, що примушує інших прясти і ткати на
нього. Додаткова праця (Extraarbeit), витрачена в прядінні
льону, реалізувалася раніш у додатковому доході (Extraeinkommen)
безлічі селянських родин або також — за часів Фрідріха
II — у податках pour le roi de Prusse\footnote*{
для пруського короля. \emph{Ред.}
}. Тепер вона реалізується
в зиску небагатьох капіталістів. Веретена і ткацькі варстати,
порозділені раніш по селах, зосереджені тепер по небагатьох
великих робітних казармах, так само як робітники і сировинний
матеріял. І веретена, і ткацькі варстати, і сировинний
матеріял перетворюються відтепер із засобів незалежного існування
прядунів і ткачів на засоби панування\footnote{
«Я дозволю вам, — каже капіталіст, — мати честь служити мені
з умовою, що ви віддасте мені те небагато, що у вас залишилося, за ту
тяжку працю, яку я виконую, командуючи вами» (Je permettrai que
vous ayez l’honneur de me servir, à condition que vousme donnez le peu
qui vous reste pour la peine que je prends de vous commander»). (\emph{J.~J.~Rousseau}:
«Discours sur l’Economie Politique». Geneve 1760, p. 70).
} над ними й висисання
з них неоплаченої праці. По великих мануфактурах і по
\index{i}{0644}  %% посилання на сторінку оригінального видання
великих фармах зовсім не пізнати, що вони постали із об’єднання
багатьох дрібних майстерень і через експропріацію багатьох
дрібних незалежних продуцентів. Однак безстороннього глядача
це не введе в обман. За часів Мірабо, цього лева революції, великі
мануфактури ще називалися manufactures réunies, тобто об’єднаними
майстернями, як ми говоримо про об’єднані поля. «Звертають
увагу, — каже Мірабо, — лише на великі мануфактури,
що в них сотні людей працюють під керівництвом одного директора
і що їх звичайно звуть об’єднаними мануфактурами (manufactures
réunies). Навпаки, на ті майстерні, де працює дуже багато
робітників порізно і кожний на власну руку, навряд чи
хто й оком скине. їх зовсім відсувають на задній плян. Цедуже
велика помилка, бо лише вони становлять дійсно важливу складову
частину народнього багатства\dots{} Об’єднана фабрика (fabrique
réunie) на диво збагачує одного або двох підприємців, але
робітники — це лише краще або гірше оплачувані поденники,
що не беруть ніякої участи в добробуті підприємця. Навпаки,
роз’єднана фабрика (fabrique séparée) нікого не збагачує, але
зате маса робітників живе в добробуті\dots{} Число працьовитих і
хазяйновитих робітників зростатиме, бо в розумному способі
життя, в працьовитості вони бачать засіб, щоб посутньо поліпшити
своє становище, замість здобувати собі невеличке підвищення
заробітної плати, що ніколи не може мати важливого
значення для будучини і в найкращому разі тільки дозволить
робітникам трохи краще жити з дня на день. Лише роз’єднані
індивідуальні мануфактури, сполучені здебільша з дрібним сільським
господарством, є вільні»\footnote{
\emph{Mirabeau}: «De la Monarchie Prussienne», Londres, 1788, vol. III,
p. 20--109 і далі. Коли Мірабо гадає, що роз’єднані майстерні є
також економічніші й продуктивніші, ніж майстерні «об’єднані», і ці
останні вважає лише за штучні, тепличні рослини, що виросли під охороною
уряду, то це пояснюється тодішнім станом більшої частини континентальних
мануфактур.
}. Експропріяція і зганяння
частини сільської людности не тільки звільняють разом із самими
робітниками їхні засоби існування і їхній матеріял праці для
промислового капіталу, а й створюють внутрішній ринок.

Справді, ті події, що перетворюють дрібних селян на найманих
робітників, а засоби їхнього існування і праці — на речові
елементи капіталу, створюють разом із цим для цього останнього
внутрішній ринок. Раніш селянська родина продукувала
і обробляла засоби існування й сировинні матеріяли, що їх вона
потім здебільшого сама ж і споживала. Ці сировинні матеріяли
й засоби існування тепер стали товарами; великий фармер продає
їх; мануфактури є його ринок. Пряжа, полотно, грубі вовняні
вироби — речі, що для них сировинний матеріял був у руках
кожної селянської родини, при чому кожна родина пряла і ткала
їх для власного споживання — перетворюються тепер на мануфактурні
\index{i}{0645}  %% посилання на сторінку оригінального видання
вироби, для яких саме рільничі округи становлять
ринок збуту. Численні тут і там порозкидані споживачі, досі
обслуговувані багатьма дрібними продуцентами, що працювали
на власну руку, сконцентровуються тепер в один великий ринок,
обслуговуваний промисловим капіталом\footnote{
«Коли двадцять фунтів вовни непомітно перетворюються на потрібний
протягом року для родини робітника одяг через власну працю
цієї родини, в перервах поміж іншими її роботами, то це не впадає в очі,
але винесіть цю вовну на ринок, відішліть її на фабрику, звідти до маклера,
потім до торгівця — і ви матимете великі комерційні операції і
номінальний капітал у двадцять разів більший за вартість продукту\dots{}
Таким чином робітничу клясу визискується на те, щоб підтримувати
нужденну фабричну людність, паразитарну клясу крамарів і фіктивну
комерційну, грошову й фінансову систему» («Twenty pounds of wool
converted unobtrusively into the yearly clothing of a labourer’s family
by its own industry in the intervals of other work — this makes no show;
but bring it to market, send it to the factory, thence to the broker, thence
to the dealer, and you will have great commercial operations, and nominal
capital engaged to the amount of twenty times its value\dots{} The working
class is thus emerced to support a wretched factory population, a parasitical
shopkeeping class, and a fictitious commercial, monetary and financial
system»). (\emph{David Urquhart}: «Familiar Words», London 1855, p. 120).
}. Так пліч-о-пліч
з експропріяцією селян, що раніш господарювали самостійно, і
з відокремленням їх від засобів продукції відбувається нищення
сільської підсобної промисловости, процес відокремлення мануфактури
від рільництва. І лише знищення сільської домашньої
продукції може надати внутрішньому ринкові країни тих розмірів
і тієї сталости, що їх потребує капіталістичний спосіб продукції.

Однак період мануфактури у власному значенні слова ще не
приводить до радикального перевороту. Пригадаймо, що мануфактура
опановує національну продукцію лише частково, спорадично,
і завжди спирається на міське ремество й домашню
сільську підсобну промисловість як на свою широку базу.
Якщо мануфактура знищує домашню сільську підсобну промисловість
в одній формі, в осібних галузях продукції, у певних
пунктах, то вона створює їх знову в інших пунктах, бо вона до
певної міри потребує її для оброблювання сировинного матеріялу.
Тим то вона створює нову клясу дрібних рільників, що для них
оброблювання землі є лише підсобна галузь, а головне їхнє заняття
є промислова праця, що її продукт вони — безпосередньо
або посередньо через купця — продають мануфактурі. Це —
причина, хоч і не головна, того явища, яке насамперед спантеличує
дослідника англійської історії. Починаючи від останньої
третини XV століття, дослідник натрапляє там на постійні,
лише іноді на короткий час притихлі, нарікання на зріст капіталістичного
господарства на селі й на проґресивне нищення
селянства. З другого боку, він завжди знову знаходить там це
селянство, хоч і в зменшеній кількості і в дедалі гірших умовах\footnote{
Виняток являють собою часи Кромвела. Поки існувала республіка,
всі верстви англійської народньої маси піднеслися з того занепаду,
в якому вони були за Тюдорів.
}.
Головна причина цього ось у чому: в Англії навпереміну переважає
\index{i}{0646}  %% посилання на сторінку оригінального видання
то рільництво, то скотарство, і залежно від цих періодів
змін коливаються й розміри селянської продукції. Лише велика
промисловість з її машинами дає сталу базу для капіталістичного
рільництва, радикально експропріює величезну більшість сільської
людности й вивершує відокремлення рільництва від домашньої
сільської промисловости, вириваючи її коріння: прядіння
і ткацтво\footnote{
Текет знає, що з мануфактур у власному значенні слова та із
зруйнування сільських або домашніх мануфактур виникає із заведенням
машин велика вовняна промисловість (\emph{Tuckett}: «A History of the Past
and Present State of the Labouring Population», vol. I, p. 139--143).
«Плуг, ярмо були винаходом богів і заняттям героїв; хіба ж ткацький
варстат, веретено й прядка менш благородні походженням? Ви відокремлюєте
прядку від плуга, веретено від ярма, і маєте фабрики й робітні
доми, кредит і паніку, дві ворожі нації, рільничу й комерційну».
(\emph{David Urquhart}: «Familiar Words», London 1855, p. 122). Але ось з’являється
Kepi і обвинувачує Англію, звичайно, не без підстав, у тому,
що вона намагається перетворити всі інші країни у виключно рільничі,
щоб стати для них фабрикантом. Він запевняє, що таким чином зруйновано
Туреччину, бо там «ніколи не дозволялось (Англією) землевласникам
і рільникам зміцнити своє становище через природну спілку плуга
з ткацьким варстатом, борони з молотком». («The Slave Trade», p. 125).
На його погляд, сам Уркварт є один із головних винуватців зруйнування
Туреччини, де він в інтересах Англії пропагував вільну торговлю.
Але найкраще те, що Кері, до речі великий холоп Росії, хоче за допомогою
протекційної системи спинити той процес відокремлення, що його вона
в дійсності прискорює.
}. Тим то лише вона завойовує для промислового
капіталу ввесь внутрішній ринок\footnote{
Філантропічні англійські економісти, як от Мілл, Роджерс, Ґолдвін,
Сміс, Фавсет і~\abbr{т. ін.}, та ліберальні фабриканти, як от Джон Брайт
і компанія, запитують англійських земельних аристократів, як бог
запитував Каїна про брата його Авеля: де поділись тисячі наших freeholder’iв?\footnote*{
самостійних селян. \emph{Ред.}
}.
Та ви то сами звідки взялися? Із знищення цих freeholder’iв.
Чому ви не питаєте далі: де поділись наші самостійні ткачі, прядуни,
ремісники?
}.

\subsection{Генеза промислового капіталіста}

Генеза промислового\footnote{
«Промисловий» уживається тут у протилежність до «рільничого».
В розумінні «категорії» фармер є такий самий промисловий капіталіст,
як і фабрикант.
} капіталіста відбувалася не з такою
поступінністю, як генеза фармера. Без сумніву, деякі дрібні
цехові майстрі, і ще більше самостійні дрібні ремісники і навіть
наймані робітники перетворювались на дрібних капіталістів,
а потім, поволі, за допомогою більше поширюваної експлуатації
найманої праці й відповідної акумуляції, — на капіталістів sans
phrase\footnote*{
попросту. \emph{Ред.}
}. У дитячому періоді капіталістичної продукції справа
здебільша стояла так, як у дитячому періоді середньовічного
міського ладу, де питання про те, хто з кріпаків-утікачів повинен
бути майстром, а хто слугою, вирішувано здебільша залежно
від того, хто з них раніш утік. Однак черепаша хода цієї методи
зовсім не відповідала торговельним потребам нового світового
\parbreak{}  %% абзац продовжується на наступній сторінці

\parcont{}  %% абзац починається на попередній сторінці
\index{i}{0647}  %% посилання на сторінку оригінального видання
ринку, створеного великими відкриттями кінця XV століття.
Але середньовіччя залишило по собі дві різні форми капіталу,
що визрівають серед якнайрізніших суспільно-економічних
формацій і що їх перед епохою капіталістичної продукції вважається
за капітал quand même\footnote*{
— не зважаючи ні на що. \emph{Ред.}
} — лихварський капітал і купецький
капітал. «За теперішніх часів усе суспільне багатство
попадає спочатку до рук капіталіста\dots{} він платить ренту земельному
власникові, заробітну плату робітникові, податки й десятину
збирачеві й затримує для себе самого велику, в дійсності
найбільшу частину річного продукту праці, частину, що день-у-день
зростає. Капіталіста можна тепер розглядати як первісного
власника цілого суспільного багатства, хоч жодний закон
не передав йому права на цю власність\dots{} Цю зміну щодо власности
зумовлено тим, що брали проценти на капітал\dots{} і не менш
дивно, що законодавці цілої Европи хотіли перешкодити цьому
законами проти лихварства\dots{} влада капіталіста над цілим багатством
країни — це цілковита революція в праві власности;
алеж яким законом, або якою низкою законів зумовлено цю
революцію?».\footnote{
«The Natural and Artificial Rights of Property Contrasted», London
1832, p. 98, 99. Автор цієї анонімної праці — Т. Годжскін.
} Авторові слід було б пригадати, що революцій
не роблять за допомогою законів.

Перетворенню грошового капіталу, утвореного лихварством
і торговлею, на промисловий капітал заважав февдальний режим
на селі, цеховий режим у місті.\footnote{
Ще 1794 р. дрібні виробники сукна Лідсу послали до парляменту
депутацію з проханням видати закона, що заборонив би купцям ставати
фабрикантами. (Dr. Aikn: «Description of the Country from thirty to forty
miles round Manchester», London 1795).
} Ці обмеження впали з розпуском
февдальних дружин, з експропріяцією й частинним зігнанням
сільської людности. Нова мануфактура постала в морських
експортових гаванях або в таких пунктах усередині
країни, які були поза контролем старих міст і їхнього цехового
ладу. Звідси, в Англії, люта боротьба corporate towns\footnote*{
— міст з цеховим корпоративним ладом. \emph{Ред.}
} проти
цих нових розсадників промисловости.

Відкриття копалень золота й срібла в Америці, винищення,
поневолення й поховання живцем тубільної людности в копальнях,
розпочате завоювання й розграбовування Східньої Індії,
перетворення Африки на загороду для комерційного полювання
на чорношкірих — така була світова зоря капіталістичної ери
продукції. Оці ідилічні процеси є головні моменти первісної
акумуляції капіталу. За ними слідом іде торговельна війна европейських
націй, арена її — ціла земна куля. Ця війна починається
відпадом Нідерляндії від Еспанії, набирає велетенських розмірів
в англійській антиякобінській війні й ще тепер триває у війнах
проти Китаю за опій і т. ін.

\input{i/_0648.tex}
\parcont{}  %% абзац починається на попередній сторінці
\index{i}{0649}  %% посилання на сторінку оригінального видання
провінція Яви, що 1750 р. налічувала понад 80.000 мешканців,
1811 р. мала їх лише 8.000. Оце вам doux commerce!\footnote*{
— лагідна торговля. \emph{Ред.}
}

Англійська східньоіндійська компанія, як відомо, крім політичної
влади у Східній Індії, здобула собі виключну монополію
на торговлю чаєм, як і взагалі на торговлю з Китаєм і на транспорт
товарів з Европи та до Европи. Але мореплавство по узбережжі
Індії й між островами, як і торговля всередині Індії,
зробилися монополією вищих службовців цієї компанії. Монополії
на сіль, опій, бетель\footnote*{
Бетель — рослина, що належить до перцевих; впливає збудно на
нервову систему. В тропічній Азії дуже поширена звичка жувати цю рослину.
\emph{Ред.}
} та інші товари були невичерпними
джерелами багатства. Службовці компанії сами визначали ціни
на товари й обдирали нещасних індусів, як сами хотіли. Генерал-губернатор
брав участь у цій приватній торговлі. Його фаворити
діставали контракти на умовах, що дозволяли їм краще за альхеміків
виробляти золото з нічого. Великі багатства виростали
як гриби після дощу, первісна акумуляція відбувалась повним
ходом без авансування жодного шилінґа. Судовий процес Воррен
Гастінґза аж кишить такими прикладами. Подаємо один із
них. Один контракт на постачання опію передано якомусь Сюлейвенові
в момент його від’їзду — з офіціяльного доручення —
до частини Індії, дуже віддаленої від районів продукції опію.
Сюлейвен продає свій контракт за 40.000 фунтів стерлінґів якомусь
Біннові; того самого дня Бінн перепродує його за 60.000 фунтів
стерлінґів, а останній покупець і виконавець контракту заявляє,
що й він після всього цього здобув величезний бариш.
За одним документом, внесеним до парляменту, компанія та її
службовці примусили індусів за час від 1757 р. до 1766 р. подарувати
їм 6 мільйонів фунтів стерлінґів! В 1769—1770 рр.
англійці створили голод, закупивши ввесь риж і відмовившись
перепродувати його інакше, як за казкові ціни.\footnote{
1866 р. в самій тільки провінції Оріса померло з голоду понад
мільйон індусів. Проте намагалися збагатити індійську державну касу
за допомогою цін, за якими голодним продавали засоби існування.
}

Поводження з тубільцями було, звичайно, найжорстокіше
на плянтаціях, призначених виключно для експортної торговлі,
як от у Західній Індії, а також у багатих і густо залюднених
країнах, що стали жертвою грабіжництва й убивства, як от Мехіко
та Східня Індія. Однак і в колоніях у власному значенні
слова виявився християнський характер первісної акумуляції
капіталу. Пуритани Нової Англії, ці тверезі віртуози протестантизму,
ухвалили в 1703 р. на своїй Assembly\footnote*{
— законодавчі збори. \emph{Ред.}
} видавати премію
в 40 фунтів стерлінґів за кожний скальп індійця і за кожного
спійманого червоношкурого, в 1720 р. — премію в 100 фунтів
стерлінґів за кожний скальп, а в 1744 р., після того, як Massachusetts-Bay
проголосив одно плем’я бунтарським, ухвалено платити
\index{i}{0650}  %% посилання на сторінку оригінального видання
такі ціни: за чоловічий скальп 12 років і старше — 100 фунтів стерлінґів у новій валюті, за
спійманого чоловіка — 105 фунтів стерлінґів, за спійману жінку або дитину — 55 фунтів стерлінґів, за
жіночий або дитячий скальп — 50 фунтів стерлінґів!
Кілька десятиліть пізніш колоніяльна система помстилась на нащадках цих побожних отців-піліґримів,
що й собі стали бунтарями. З намови й за гроші англійців їх усіх tomahawked.\footnote*{
— повбивано томагавками. \emph{Ред.}
} Брітанський парлямент
проголосив кровожерство і скальпування
за «засоби, дані йому богом і природою».

Колоніяльна система надзвичайно прискорила розвиток торговлі й мореплавства. «Монопольні товариства»
(Лютер) були могутніми підоймами концентрації капіталу. Колонії забезпечували ринок збуту для
мануфактур, що народжувались, а монополія на ринку забезпечувала їм збільшену акумуляцію капіталу.
Скарби, здобуті поза Европою безпосереднім плюндруванням, поневолюванням, грабіжництвом і
вбивствами, припливали в метрополію і тут перетворювались на капітал. Голляндія, яка перша цілком
розвинула колоніяльну систему, вже 1648 р. дійшла вершини своєї торговельної могутности. В її «майже
виключному посіданні була східньоіндійська торговля й засоби
комунікації поміж европейським південним заходом і північним сходом. Її рибальство, мореплавство й
мануфактури були розвиненіші, ніж у всіх інших країнах. Капітали цієї республіки були, мабуть,
значніші, ніж капітали всіх інших країн Европи разом» (Gülich: «Geschichtliche Darstellung des
Handels etc.» Iena 1830, В. I, p. 371). Ґюліх забуває додати, що народні маси Голляндії вже в 1648
р. більше терпіли від надмірної праці, були більш збіднілі й пригнічені, ніж народні маси всіх інших
країн Европи разом.

За наших часів промислова перевага веде за собою торговельну перевагу. Навпаки, за власне
мануфактурного періоду торговельна перевага забезпечує перевагу промисловости. Відси та вирішальна
роля, яку в ті часи відігравала колоніяльна система. Це був той «чужий бог», що засів на вівтарі
поруч із старими божками Европи й одного чудового дня одним ударом усіх їх поскидав. Колоніяльна
система оголосила здобування зиску за останню й єдину мету людства.

Система публічного кредиту, тобто державних боргів, що її початки ми знаходимо в Ґенуї й Венеції вже
за середньовіччя, захопила цілу Европу підчас мануфактурного періоду. Колоніяльна система з її
морською торговлею і торговельними війнами була за теплицю, що прискорювала її розвиток. Так вона
вкоренилася насамперед у Голляндії. Державний борг, тобто відчуження держави — однаково, чи
деспотичної, чи конституційної, чи республіканської — накладає свою печать на капіталістичну еру.
Однісінька частина так званого національного багатства, що дійсно входить у спільне володіння
сучасних народів,
\index{i}{0651}  %% посилання на сторінку оригінального видання
— це їхні державні борги.\footnoteA{
Вільям Коббет зауважує, що в Англії всі громадські установи
називаються «королівськими», але борг зате там є «національний» (national debt).
} Тому цілком послідовна є та сучасна доктрина, що народ стає тим
багатший, чим більше він заборговується. Державний кредит стає символом віри капіталу. І з
виникненням державної заборгованости місце гріха проти
святого духа, що за нього немає прощення, заступає зламання довіри до державного боргу.

Державний борг стає за одну з найсильніших підойм первісної акумуляції. Немов доторкаючись чарівною
паличкою, він наділяє непродуктивні гроші продуктивною силою й перетворює їх таким чином на капітал,
не потребуючи при тому виставляти
їх на небезпеку та самому зазнавати турбот, нерозривно зв’язаних з вкладанням грошей у промислові
підприємства й навіть у лихварські операції. Державні кредитори в дійсності не дають нічого, бо
позичені суми перетворюються на легко переказувані
боргові посвідки, які функціонують у їхніх руках цілком так само, як коли б це була така сама сума
готівки. Але державний борг не тільки створив таким чином клясу нероб-рантьє і імпровізоване
багатство тих фінансистів, що відіграють ролю посередників
поміж урядом і нацією — а також багатство відкупників податків, купців і приватних фабрикантів, що
їм значна частина кожної державної позики робить послугу як капітал, наче з неба спалий. Державний
борг, крім того, викликав акційні товариства,
торговлю всякими цінними паперами, ажіотаж, одно слово, біржову гру й сучасну банкократію.

З самого зародження свого великі банки, прикрашені національними титулами, були лише товариствами
приватних спекулянтів, що ставали на бік урядів і, завдяки одержаним привілеям, були спроможні
позичати їм гроші. Тому для акумуляції
державного боргу немає вірнішого мірила, ніж послідовне підвищення акцій цих банків, повний розквіт
яких починається з моменту заснування Англійського банку (1694 р.). Англійський банк почав з того,
що позичав урядові свої гроші з 8\%; одночасно
парлямент уповноважив його карбувати гроші з того самого капіталу, ще раз позичаючи його публіці у
формі банкнот. Цими банкнотами він міг дисконтувати векселі, давати позики під товари й закуповувати
благородні металі. Минуло небагато
часу, і ці кредитові гроші, зфабриковані самим банком, стали готівкою, що нею Англійський банк
видавав позики державі і сплачував коштом держави проценти від державних позик. Мало того, що банк
давав однією рукою, щоб одержати більше другою; навіть і тоді, коли він одержував, він лишався
вічним кредитором нації на всю віддану суму до останнього шага. Помалу він став також і доконечним
сховищем металевих скарбів країни й центром тяжіння для всього торговельного кредиту.
В той самий час, коли в Англії перестали палити відьом, там почали вішати фалшівників банкнот. Яке
вражіння справила на
\parbreak{}  %% абзац продовжується на наступній сторінці

\parcont{}  %% абзац починається на попередній сторінці
\index{i}{0652}  %% посилання на сторінку оригінального видання
сучасників раптова поява цієї зграї банкократів, фінансистів, рантьє, маклерів,
stockjobber’iв\footnote*{
— біржовиків, що спекулюють на цінних державних паперах, акціях тощо. \emph{Ред.}
} і
біржових вовків, це можна бачити з тогочасних творів, наприклад, творів Белінґброка.\footnoteA{
«Коли б татари захопили тепер Европу, то дуже важко було б з’ясувати їм, хто такий є в нас
фінансист» («Si les Tartares inondaient aujourd’hui l’Europe, il faudrait bien des affaires pour
leur faire entendre ce que c’est qu’un financier parmi nous»). (Montesquieu: «Esprit des lois», éd.
Londres 1769, vol. IV, p. 33).
}

Разом з державними боргами виникла система міжнароднього кредиту, за якою часто криється одне з
джерел первісної акумуляції капіталу в того або іншого народу. Так підлоти венецької хижацької
системи становлять таку скриту основу капіталістичного
багатства Голляндії, якій занепадуща Венеція позичала великі грошові суми. Такі самі відносини були
між Голляндією та Англією. Вже на початку ХVІІІ століття голляндські мануфактури були значно
перевищені, і Голляндія перестала бути панівною торговельною і промисловою нацією. Тому 1700 — 1776~\abbr{рр.} за одне з головних занять Голляндії стає позичання величезних капіталів, особливо своєму
могутньому конкурентові — Англії. Подібні відносини маємо нині між Англією та Сполученими
штатами. Багато капіталів, що сьогодні з’явилися без метричного свідоцтва в Сполучених штатах, є
лише капіталізована вчора в Англії кров дітей.

А що державні борги спираються на державні доходи, які мусять покривати річні проценти й інші
подібні платежі, то сучасна податкова система стала доконечним доповненням системи національних
позик. Позики дають урядові змогу покривати
надзвичайні видатки таким чином, що платник податків не відчуває цього відразу, але згодом ці позики
все ж вимагають підвищення податків. З другого боку, підвищення податків, спричинене нагромадженням
один по одному роблених боргів, примушує
уряд при нових надзвичайних видатках щоразу брати нові позики. Тому сучасна фіскальна система, що її
вісь становлять податки на найдоконечніші засоби існування (отже, їх подорожчання), має в собі самій
зародок автоматичного підвищування податків. Надмірне оподаткування — це не якийсь винятковий
випадок: навпаки, це — принцип. Тому в Голляндії, де вперше заведено цю систему, великий патріот Де
Вітт вихвалював її у своїх «Maximes» як найкращу систему зробити найманого робітника покірливим,
скромним, працьовитим та\dots{} переобтяженим надмірною працею. Однак той руйнаційний вплив, що його
сучасна фіскальна система справляє на становище найманих робітників, нас тут цікавить менш, ніж
зумовлена нею насильна експропріяція селянина, ремісника, коротко — всіх складових частин нижчої
верстви середньої кляси. Про це немає двох думок навіть серед буржуазних економістів.
Експропріяторське діяння
\parbreak{}  %% абзац продовжується на наступній сторінці

\parcont{}  %% абзац починається на попередній сторінці
\index{i}{0653}  %% посилання на сторінку оригінального видання
фіскальної системи ще більше посилюється через протекційну систему, що є одна з складових частин
фіскальної системи.

Велика роля, яку відіграють державні борги й відповідна їм фіскальна система в капіталізації
багатств та експропріяції мас, призвела цілий ряд письменників, як от Коббета, Дублдея і інших, до
того, що вони помилково шукали в цьому головну
причину злиднів сучасних народів.

Система протекціонізму була штучним засобом фабрикувати фабрикантів, експропріювати незалежних
робітників, капіталізувати національні засоби продукції та існування, насильно скорочувати перехід
від стародавнього способу продукції до сучасного. Европейські держави билися за патент на цей
винахід і, ставши раз на службу капіталістам (Plusmacher’aм), вони для цієї мети не тільки грабували
начисто свій власний народ: посередньо — через охоронні мита, безпосередньо — через експортові
премії і~\abbr{т. ін.} В залежних від них сусідніх країнах вони силоміць
винищували всю промисловість, як наприклад, ірляндську вовняну мануфактуру, що її знищила Англія. На
европейському континенті за прикладом Кольбера цей процес зроблено ще куди простішим. Первісний
капітал промисловців припливав тут до
них почасти безпосередньо з державної скарбниці. «Нащо, — каже Мірабо, — так далеко шукати причин
розцвіту мануфактури в Саксонії перед семилітньою війною? 180 мільйонів державних боргів!»\footnote{
«Pourquoi aller chercher si loin la cause de l’éclat manufacturier de la Saxe avant la guerre?
Cent quatre-vingt millions de dettes faites par les souverains!». (Mirabeau: «De la Monarchie
Prussienne», Londres 1788, vol. VI, p. 101).
}.

Колоніяльна система, державні борги, податковий тягар, протекціонізм, торговельні війни й~\abbr{т. ін.} —
всі ці паростки власне мануфактурного періоду колосально розростаються за дитячого періоду великої
промисловости. Народження цієї останньої
відсвятковано величезним іродовим побиттям дітей. Фабрики набирали робітників, як королівська фльота
матросів, насильною рекрутацією. Хоч який байдужий сер Ф.М.Ідн щодо страхіть експропріяції землі в
сільської людности, що тривала від останньої
третини XV століття аж до його часів, до кінця XVIII століття, хоч як самозадоволено він вітає цей
процес, «доконечний», щоб «утворити» капіталістичне рільництво та «правильне відношення між орною
землею й пасовиськом», а все ж і він не виявляє такого самого економічного розуміння щодо
доконечности крадіння дітей і рабства їх для перетворення мануфактурного виробництва на фабричне та
встановлення правильного відношення між капіталом і робочою силою. Він каже: «Може, варто було б
замислитися публіці над тим, чи може будь-яка мануфактура, що для успішного свого функціонування
мусить красти з котеджів і робітних домів дітей і примушувати їх позмінними групами тяжко працювати
більшу частину ночі, позбавляючи їх відпочинку; мануфактура, яка до того ж так збиває
\parbreak{}  %% абзац продовжується на наступній сторінці

\parcont{}  %% абзац починається на попередній сторінці
\index{i}{0654}  %% посилання на сторінку оригінального видання
докупи осіб обох статей різного віку й різних нахилів, що заразливість прикладу
мусить привести до зіпсованости й розпусти, — чи може така мануфактура збільшити
суму національного й індивідуального щастя?».\footnote{
\emph{Eden}: «The State of the Poor», v. II, ch. I, p. 421.
} «В Дербішірі, Нотінґемшірі й особливо
Ланкашірі, — каже Фелден, — ужито недавно винайдені машини на великих фабриках,
побудованих над річками, що могли пускати в рух водяне колесо. Одразу постала
потреба в тисячах робочих рук у цих місцевостях, віддалених від міст; і особливо
Ланкашір, до того часу порівняно мало залюднений і неродючий, потребував тепер
перш за все людности. Потрібні були насамперед маленькі й меткі дитячі руки.
Одразу ж повівся звичай набирати учнів (!) із лондонських, бірмінґемських та
інших парафіяльних робітних домів. Таким чином багато-багато тисяч
цих маленьких безпорадних істот від 7 до 13 або 14 років життя вивезено на
північ. У хазяїна (тобто крадія дітей) повівся звичай одягати, годувати й
приміщувати своїх учнів у «будинку для учнів» близько фабрик. Він наймав
доглядачів, що мали стежити за їхньою працею. В інтересах цих доглядачів
за рабами було примушувати дітей працювати понад усяку міру, бо їхня плата
залежала від кількости продукту, що її можна було видушити
з дітей. Природний наслідок цього була жорстокість\dots{} По багатьох фабричних
округах, особливо ж у Ланкашірі, цих невинних і беззахисних істот, відданих на
волю фабрикантів, катували з надзвичайною жорстокістю. Їх замордовували до
смерти надмірною працею\dots{} їх били батогами, заковували в кайдани й
катували з найвишуканішою витонченістю й жорстокістю; у багатьох випадках їх
виснажували голодом до шкури-кости, і все ж батогом примушували до праці\dots{}
В деяких випадках їх доводили навіть до самогубства!.. Чудові й романтичні
долини Дербішіру, Нотінґемшіру та Ланкашіру, заховані від
громадського ока, стали жахливим місцем катувань і — часто вбивства!\dots{}
Зиски фабрикантів були величезні. Це лише розпалювало їхню вовчу ненажерливість.
Вони почали заводити нічну працю, тобто тримали напоготові для нічної праці
групу робітників, що заступала другу групу робітників, знесилених денною працею;
денна йшла до ліжок, які тільки но покинула нічна група, і навпаки. У
Ланкашірі є народній переказ, що ці ліжка ніколи не простигали».\footnote{
\emph{John Fielden}: «The Curse of the Factory System», London 1836, p. 5, 6.
Про всі ті гидоти, які творилися від початку фабричної системи порівн.
Dr. \emph{Aikin}: «Description of the Country from thirty to forty miles round
Manchester», London 1795, p. 129 та \emph{Gisborne}: «Enquiry into the duties of
men», 1795, vol II. — Через те, що парова машина перенесла
фабрики від сільських водоспадів до центру міст, то «прихильний до поздержливости»
капіталіст (Plusmacher) находив дитячий матеріял під рукою, так що не треба було
насильно транспортувати рабів із робітних домів. — Коли сер Роберт Піл
(батько «міністра уважливости») запропонував у 1815 р. біл
для охорони дітей, Ф. Горнер (світило «Bullion Komitees\footnote*{
— комітет у справах зливків. \emph{Ред.}
} та інтимний приятель Рікарда) заявив у палаті громад:
«Загальновідомий факт, що разом з цінними речами одного банкрута
призначено на продаж і продано з авкціона, як частину його майна, банду
фабричних дітей, коли можна вжити такого слова. Два роки тому (року
1813) перед King’s Bench\footnote*{
— найвищим судом. \emph{Ред.}
} розглядувано огидний випадок. Ішлося про
групу хлопчиків. Одна лондонська парафія віддала їх якомусь фабрикантові,
що від себе знову передав їх якомусь іншому. Нарешті, декілька філантропів
знайшло їх у стані повного виснаження від голоду («absolute
famine»). З другим випадком, ще огиднішим, його познайомили, як члена
парламентської слідчої комісії. Декілька років тому одна лондонська
парафія склала контракт з одним ланкашірським фабрикантом, за яким
він зобов’язувався на двадцять здорових купованих ним дітей приймати
одного ідіота».
}

\index{i}{0655}  %% посилання на сторінку оригінального видання
З розвитком капіталістичної продукції протягом мануфактурного
періоду громадська думка Європи позбулася останніх решток
сорому й сумління. Нації цинічно пишались кожною підлотою,
що була засобом для акумуляції капіталу. Прочитайте, приміром,
наївні торговельні анали, складені щирим А. Андерсоном.

Тут, як тріумф англійської державної мудрости, розхвалюється
той факт, що Англія за Утрехтським миром на основі
угоди асієнто\footnote*{
— угода щодо торговлі рабами. \emph{Ред.}
} вимусила від Еспанії привілей, що давав їй
право на торговлю неграми, яку вона досі вела лише між Африкою
й англійською Західньою Індією, вести й між Африкою та
еспанською Америкою. Англія здобула право аж до 1743 р. постачати
еспанській Америці щорічно 4.800 негрів. Це забезпечувало
їй разом з тим офіціяльне прикриття для контрабанди.
Ліверпул виріс як велике місто на ґрунті торговлі рабами.
Вона становить його методу первісної акумуляції. І ще й по сей
день «поважні» ліверпулські громадяни лишилися Піндарами
работорговлі, яка — порівн. вже цитований вище твір д-ра Ейкіна
з 1795 р. — «підносить дух комерційної підприємливости аж до
пристрасти, створює славних моряків і приносить колосальні
гроші». В 1730 р. в Ліверпулі коло торговлі рабами працювало
15 кораблів, в 1751р. — 53 кораблі, в 1760 р. — 74, в 1770 р. —
96 і в 1792 р. — 132 кораблі.

Бавовняна промисловість, завівши в Англії рабство дітей,
дала разом з тим поштовх до перетворення рабовласницького
господарства Сполучених штатів, доти більш або менш патріярхального,
на комерційну систему експлуатації. Взагалі приховане
рабство найманих робітників в Европі потребувало, як основи,
рабства sans phrase\footnote*{
— попросту. \emph{Ред.}
} у Новому Світі.\footnote{
«В 1790 р. в англійській Західній Індії 10 рабів припадало на
одну вільну людину, у французькій — 14 на одну, в голляндській — 23
на одну». (\emph{Henry Barougham}: «An Inquiry into the Colonial Policy of
the European Powers», Edinburgh 1803, vol. II, p. 74).
}

Tantae molis erat\footnote*{
Стільки праці коштувало. \emph{Ред.}
} розв’язати «вічні природні закони»
капіталістичного способу продукції, вивершити процес відокремлення
\index{i}{0656}  %% посилання на сторінку оригінального видання
робітників від умов праці, перетворити на одному
полюсі суспільні засоби продукції та засоби існування на капітал,
а на протилежному — народню масу на найманих робітників,
на вільних «працюючих бідняків», — цей витвір мистецтва
сучасної історії.\footnote{
Вислів «labouring poor»\footnote*{
— працюючі бідняки. \emph{Ред.}
} подибуємо в англійських законах від
часу, коли кляса найманих робітників стає помітна. Labouring poor протистоять,
з одного боку, «idle poor»,\footnote*{
— біднякам-неробам. \emph{Ред.}
} жебракам і т. ін., з другого боку,
тим робітникам, що не є ще цілком обскубані кури, а є ще власники
своїх засобів праці. Із законів вислів «labouring poor» перейшов
до політичної економії, починаючи від Колпепера, Дж. Чайлда й ін.,
аж до А. Сміса й Ідна. По цьому можна судити, яка bonne foi\footnote*{
— сумлінність. \emph{Ред.}
} в Едмунда
Берка, цього «execrable political cantononger»\footnote*{
— огидливого політичного крамаря. \emph{Ред.}
}, коли він вислів
«labouring poor» зве «execrable political cant».\footnote*{
— огидливим політичним перекрученням. \emph{Ред.}
} Цей сикофант, що,
бувши на утриманні англійської олігархії, відігравав ролю романтика проти
французької революції, так само, як на початку заворушень в Америці
він, бувши на утриманні північно-американських колоній, відігравав
ролю ліберала проти англійської олігархії, в дійсності був наскрізь ординарним
буржуа: «Закони торговлі є закони природи, отже, і закони самого
бога» (\emph{Е. Burke}: «Thoughts and Details on Scarcity», ed. London
1800, p. 31, 32). He диво, що він, вірний законам бога й природи, завжди
продавав себе самого на найкращому ринку! У творах панотця Тукера —
Тукер був піп і торі, але зрештою цілком пристойна людина й путящий
політико-економ — можна знайти дуже гарну характеристику цього
Едмунда Берка за його ліберальних часів. При тій огидливій безхарактерності,
яка панує тепер і побожно вірить у «закони торговлі», треба
знову й знов таврувати таких Берків, що від своїх наступників відрізняються
лише одним — талантом!
} Коли гроші, як каже Ож’є, «родяться на
світ із природними кривавими плямами лише на одній щоці»,\footnote{
\emph{Marie Augier}: «Du Crédit Public», Paris 1842, p. 265.
}
то капітал, що родиться на світ, прискає кров’ю й брудом від
голови до ніг із усіх своїх пор.\footnote{
«Капітал, — каже «Quarterly Reviewer», — уникає заколотів
і сварок і з природи своєї боязкий. Це цілковита правда, алеж не вся
правда. Капітал боїться відсутности зиску або дуже малого зиску, як
природа боїться порожнечі. При відповідному зиску капітал стає відважним.
При певних 10 процентах його можна вживати повсюди; при 20 процентах
він стає жвавим; при 50 процентах він абсолютно готовий ризикувати;
за 100 процентів він топче ногами всі людські закони; 300 процентів
— і немає такого злочину, що на нього він не ризикнув би, навіть
під загрозою шибениці. Коли заколоти й сварки дають зиск, він заохочує
і до заколотів і до сварок. Докази — контрабанда і торговля рабами».
(\emph{Т. J. Dunning}: «Trades-Unions and Strikes», London 1860, p. 36).
}

\subsection{Історична тенденція капіталістичної акумуляції}

На що ж сходить первісна акумуляція капіталу, тобто його
історична генеза? Оскільки вона не є безпосереднє перетворення
рабів і кріпаків на найманих робітників, отже, не є проста
зміна форми, вона означає лише експропріяцію безпосередніх
продуцентів, тобто розклад приватної власности, основаної на
власній праці.

\index{i}{0657}  %% посилання на сторінку оригінального видання
Приватна власність, як протилежність до суспільної, колективної
власности, існує лише там, де засоби праці й зовнішні
умови праці належать приватним особам. Але залежно від того,
чи є ці приватні особи робітники або неробітники, змінюється
й характер самої приватної власности. Безмежна різноманітність
відтінків, які вона являє на перший погляд, відбивають лише
проміжні стани, що лежать між обома цими крайностями.

Приватна власність робітника на його засоби продукції є
основа дрібного виробництва, а дрібне виробництво є доконечна
умова розвитку суспільної продукції й вільної індивідуальности
самого робітника. Правда, цей спосіб продукції існує також у
рамках рабства, кріпацтва й інших відносин залежности. Але
він процвітає, виявляє всю свою енергію, здобуває клясичну
адекватну форму тільки там, де робітник є вільний приватний
власник своїх, ним самим уживаних умов праці, селянин — ріллі,
яку він обробляє, ремісник — інструменту, що на ньому він грає,
як віртоуз.

Цей спосіб продукції має за передумову роздрібнення землі
й усіх інших засобів продукції. Він виключає так концентрацію
засобів продукції, як і кооперацію, поділ праці всередині того
самого продукційного процесу, суспільне панування над природою
й реґулювання її, а також вільний розвиток суспільних продуктивних
сил. Він можливий лише за вузьких примітивних
меж продукції й суспільства. Захотіти його увіковічнити, це
значило б — як справедливо каже Пекер — «декретувати
загальну помірність». Але на якомусь певному ступені розвитку
він сам породжує матеріяльні засоби свого власного знищення.
З цієї хвилини в надрах суспільства починають ворушитися
сили та пристрасті, що почувають себе скутими цим способом
продукції. Він мусить бути знищений, і його знищується. Знищення
його, перетворення індивідуальних і роздрібнених засобів
продукції на суспільно-сконцентровані, отже, перетворення
карликової власности багатьох на колосальну власність
небагатьох, отже, експропріяція в широких народніх мас землі,
засобів існування і знарядь праці — оця жахлива й тяжка
експропріяція народньої маси становить передісторію капіталу.
Вона охоплює цілу низку насильних метод, що з них ми коротко
розглянули лише ті, які становили епоху як методи первісної
акумуляції капіталу. Експропріяцію безпосередніх продуцентів
проводиться з найнещаднішим вандалізмом і під тиском
якнайпідліших, якнайбрудніших, найдріб’язковіших і найшаленіших
пристрастей. Приватну власність, здобуту працею власника,
основану, так би мовити, на зрощенні поодинокого незалежного
робітника з його умовами праці, витісняє капіталістична
приватна власність, основана на експлуатації чужої, але формально
вільної праці\footnote{
«Ми перебуваємо в цілком нових суспільних умовах\dots{} ми намагаємось
відокремити кожний рід власности від кожного роду праці»
(«Nous sommes dans une condition tout-à-fait nouvelle de la société\dots{}
nous tendons à séparer toute espèce de propriété d’avec toute espèce de
travail»). («\emph{Sismondi}: «Nouveaux Principes de l’Economie Politique»,
vol. II, p. 434).
}.

\index{i}{0658}  %% посилання на сторінку оригінального видання
Скоро тільки цей процес перетворення в достатній мірі розклав
старе суспільство углиб і вшир, скоро тільки робітників
перетворено на пролетарів, а їхні умови праці на капітал, скоро
тільки капіталістичний спосіб продукції став на власні ноги,
дальше усуспільнення праці і дальше перетворення землі та
інших засобів продукції на суспільно-експлуатовані, отже, на
спільні засоби продукції, і тим то й дальша експропріяція приватних
власників набуває нової форми. Тепер експропріяції
підлягає вже не робітник, що сам веде самостійне господарство,
а капіталіст, що експлуатує багатьох робітників.

Ця експропріяція здійснюється наслідком гри іманентних законів
самої капіталістичної продукції, через централізацію капіталів.
Один капіталіст побиває багатьох. Пліч-о-пліч із цією
централізацією або експропріяцією багатьох капіталістів небагатьма
розвивається кооперативна форма процесу праці в
щораз ширших, більших розмірах, розвивається свідоме технічне
застосування науки, пляномірна експлуатація землі, перетворення
засобів праці на такі засоби праці, що їх можна
вживати тільки колективно, економізування всіх засобів продукції
через вживання їх як засобів продукції комбінованої,
суспільної праці, вплетіння всіх народів у сіть світового ринку,
а разом з тим інтернаціональний характер капіталістичного режиму.
Разом з постійним меншанням числа маґнатів капіталу,
що узурпують і монополізують усі вигоди цього процесу перетворення,
зростає маса злиднів, пригноблення, рабства, виродження,
експлуатації, але разом з тим і обурення робітничої
кляси, щораз більшої й більшої числом, що її навчає, об’єднує
й організує механізм самого процесу капіталістичної продукції.
Монополія капіталу стає путами того способу продукції, що
зріс за неї і під нею. Централізація засобів продукції та усуспільнення
праці досягають такого пункту, коли вони стають
несполучні з їхньою капіталістичною оболонкою. Її розривається.
Б’є остання година капіталістичної приватної власности.
Експропріяторів експропріюють.

Капіталістичний спосіб присвоєння, що випливає з капіталістичного
способу продукції, а тому й капіталістична власність
є перше заперечення індивідуальної приватної власности, основаної
на власній праці. Але капіталістична продукція з доконечністю
природного процесу породжує заперечення самої себе.
Це є заперечення заперечення. Воно відбудовує не приватну
власність, а індивідуальну власність на основі завоювань капіталістичної
ери, на основі кооперації і спільного володіння землею
й засобами продукції, спродукованих самою ж працею.

Перетворення роздрібненої приватної власности, основаної
на власній праці індивідуумів, на капіталістичну власність, є,
\parbreak{}  %% абзац продовжується на наступній сторінці

\parcont{}  %% абзац починається на попередній сторінці
\index{i}{0659}  %% посилання на сторінку оригінального видання
звичайно, процес куди довший, тяжчий і важчий, аніж перетворення
капіталістичної власности, фактично вже основаної на суспільній
продукції, на суспільну власність. Там ішлося про експропріяцію
народньої маси небагатьма узурпаторами, тут ідеться
про експропріацію небагатьох узурпаторів народньою масою\footnote{
«Проґрес промисловости, що його безвольний і нездатний до
опору носій є буржуазія, ставить на місце ізолювання робітників через
конкуренцію їхнє революційне об’єднання через асоціяцію. Отже, з розвитком
великої промисловости з-під ніг буржуазії вибивається саму основу,
на якій вона продукує й присвоює собі продукти. Вона продукує насамперед
своїх власних могильників. Її загибіль і перемога пролетаріату
однаково неминучі\dots{} З усіх кляс, що протистоять нині буржуазії, тільки
пролетаріят є справді революційна кляса. Всі інші кляси занепадають і
гинуть з розвитком великої промисловости; пролетаріят є її найпитоміший
продукт. Середні стани, дрібний промисловець, дрібний купець, ремісник,
селянин, — всі вони борються проти буржуазії, щоб забезпечити від
занепаду своє існування як середніх станів\dots{} вони реакційні, бо вони
намагаються повернути назад колесо історії». (К.~Marx und F.~Engels:
«Manifest der kommunistischen Partei», London 1847, S. 9, 11. — K.~Маркс
і Ф.~Енґельс: «Маніфест комуністичної партії», Партвидав «Пролетар»
1932~\abbr{р.}, стор. 39, 37).
}.

\section[Сучасна теорія колонізації]{Сучасна теорія колонізації\footnotemark{}}

Політична економія принципово сплутує два дуже різні
роди приватної власности, що з них один оснований на власній
праці продуцента, другий — на експлуатації чужої праці.
Вона забуває, що цей другий не лише становить пряму протилежність
першого, але й виростає тільки на його могилі.
\footnotetext{
Тут ідеться про дійсні колонії, про незайману землю, що її колонізують
вільні іміґранти. Сполучені штати, з економічного погляду, все
ще є колонія Европи. Зрештою сюди належать і такі старовинні плянтації,
де знищення рабства зробило цілковитий переворот у відносинах.
}

На заході Европи, батьківщині політичної економії, процес
первісної акумуляції капіталу більш або менш завершений.
Капіталістичний режим тут або просто підбив собі всю національну
продукцію, або там, де відносини менш розвинені, він, принаймні,
посередньо контролює належні до застарілого способу
продукції суспільні верстви, що й далі існують поряд нього й
поступінно занепадають. До цього готового світу капіталу політико-економ
з тим більшою запопадністю й тим більшим
зворушенням прикладає уявлення про право і власність, належні
до передкапіталістичного світу, чим голосніше кричать
факти проти його ідеології.

Інша справа в колоніях. Капіталістичний режим там повсюди
наражається на перешкоди з боку продуцента, що як посідач
своїх власних умов праці збагачує своєю працею самого
себе, а не капіталіста. Суперечність цих двох діяметрально протилежних
економічних систем виявляється тут на практиці
в їхній боротьбі. Там, де капіталіст має за своєю спиною силу
\parbreak{}  %% абзац продовжується на наступній сторінці

\parcont{}  %% абзац починається на попередній сторінці
\index{i}{0660}  %% посилання на сторінку оригінального видання
своєї метрополії, він намагається силоміць усунути з свого шляху спосіб продукції й присвоєння,
оснований на власній праці. Той самий інтерес, що в метрополії штовхає сикофанта капіталу,
політико-економа, проголошувати теоретичну тотожність капіталістичного способу продукції з його
власною протилежністю, — той самий інтерес спонукає його тут «to make a clean breast of it»\footnote*{
очистити своє сумління. \emph{Ред.}
} і гучно
проголосити протилежність цих двох способів продукції. З цією метою він доводить, що розвиток
суспільної продуктивної сили праці, кооперація, поділ праці, вживання машин у великому маштабі й~\abbr{т.
ін.} неможливі без експропріяції робітників та відповідного перетворення засобів продукції на
капітал. В інтересах так званого національного багатства він шукає штучних засобів створення
народньої бідности. Його апологетичний панцер кришиться тут на шматочки, як трухлява губка.

Велика заслуга Е.~Ґ.~Векфілда не в тому, що він сказав щось нове про колонії\footnote{
Небагато променів світла, що їх Векфілд кинув на суть самих колоній, цілковито передбачили
Мірабо-батько, фізіократи й ще багато раніш англійські економісти.
}, а в тому, що в
колоніях він розкрив правду про капіталістичні відносини в метрополії. Як протекційна система на
своїх початках\footnote{
Пізніше вона стає тимчасовою доконечністю в міжнародній конкуренційній боротьбі. Але хоч і які
були б її мотиви, наслідки її лишаються ті самі.
} намагалася фабрикувати капіталістів
у метрополії, так теорія колонізації Векфілда, що її Англія довгий час силкувалася здійснити
законодавчим способом, намагається фабрикувати найманих робітників у колоніях. Це він називає
«systematic colonization», систематичною колонізацією.

\looseness=1
Насамперед Векфілд відкрив у колоніях, що володіння грішми, засобами існування, машинами й іншими
засобами продукції ще не робить із людини капіталіста, коли бракує такого додатку, як найманий
робітник, другої людини, що примушена добровільно сама себе продавати. Він відкрив, що капітал не є
річ, а суспільне відношення між людьми, упосереднене речами\footnote{
«Негр є негр\dots{} Лише за певних відносин він стає рабом. Бавовнопрядна машина є машина для
прядіння бавовни. Лише за певних відносин вона стає капіталом. Вирвана з цих відносин, вона так само
не є капітал, як золото само по собі не є гроші або цукор — ціна цукру\dots{} Капітал є суспільне
продукційне відношення. Він — історичне продукційне відношення». (Karl Marx: «Lohnarbeit und
Kapital», «Neue Rheinische Zeitung», № 226 з 7 квітня 1849~\abbr{р.} — Карл Маркс: «Наймана праця і
капітал». Партвидав «Пролетар» 1932~\abbr{р.}, стор. 22, 23).
}. Пан Піл, — скаржиться він нам, —
взяв із собою з Англії на Лебединий берег у Новій Голляндії засобів існування та засобів продукції
на суму \num{50.000}\pound{ фунтів стерлінґів}. Пан Піл був такий передбачливий, що, крім того, взяв із собою
\num{3.000} осіб із робітничої кляси — чоловіків, жінок і дітей. Але, прибувши на місце призначення, Піл
залишився без жодного
\parbreak{}  %% абзац продовжується на наступній сторінці

\parcont{}  %% абзац починається на попередній сторінці
\index{i}{0661}  %% посилання на сторінку оригінального видання
слуги, який міг би постелити йому постіль або набрати води з річки\footnote{
Е.~G.~Wakefield: «England and America». London 1833, vol. II, p. 33.
}. Безталанний пан Піл! Він усе
передбачив, та забув лише експортувати англійські продукційні відносини на Лебединий берег.

Щоб зрозуміти дальші відкриття Векфілда, потрібні два попередні зауваження. Ми знаємо, що коли
засоби продукції та засоби існування є власність безпосереднього продуцента, то вони не є капітал.
Вони стають капіталом лише за таких умов, коли вони разом з тим служать за засоби експлуатації та за
засоби упідлеглення робітника. Але ця їхня капіталістична душа в голові політико-економа з’єднана
таким тісним подружнім зв’язком із їхньою речовою субстанцією, що він за всяких обставин називає їх
капіталом, навіть і тоді, коли вони є сáме протилежність капіталу. Так стоїть справа й у Векфілда.
Далі: роздрібнення засобів продукції, як індивідуальної власности багатьох незалежних один від
одного, самостійно господарюючих робітників, він називає рівним поділом капіталу. З
політико-економом трапляється те саме, що і з февдальним юристом. Цей останній і на суто грошові
відносини наклеює свої февдальні правні етикетки.

«Коли б, — каже Векфілд, — капітал був поділений поміж усіма членами суспільства рівними пайками, то
жодна людина не була б заінтересована в тому, щоб акумулювати капіталу більш, ніж вона може
застосувати своїми власними руками. Так до певної міри стоїть справа в нових американських колоніях,
де жадоба до земельної власности перешкоджає існуванню кляси найманих робітників»\footnote{Там же, т. І, стор. 17, 18.
}. Отже, поки
робітник має змогу акумулювати для себе самого, — а це він може робити, поки він лишається власником
своїх засобів продукції, — доти капіталістична акумуляція й капіталістичний спосіб продукції
неможливі. Бракує доконечної для цього кляси найманих робітників. Але як же тоді в старій Европі
здійснено експропріяцію в робітника його умов праці, яким чином, отже, створено там капітал і
найману працю? За допомогою contrat social\footnote*{ суспільного договору. \emph{Ред.}} дуже ориґінального характеру. «Людство\dots{} засвоїло собі
просту методу активізувати акумуляцію капіталу», яка, звичайно, від часів Адама здавалась йому
останньою й єдиною метою його буття: «воно поділилось на власників капіталу і власників праці\dots{} цей
поділ був результатом добровільного порозуміння та погодження» (Kombination)\footnote{Там же, стор. 18.}.Одне слово, маса
людства сама себе експропріювала на славу «акумуляції капіталу». А тепер треба б думати, що інстинкт
цього самовідданого фанатизму мусив би вільно виявитися саме в колоніях, де тільки й існують люди й
умови, які могли б перенести contrat
\index{i}{0662}  %% посилання на сторінку оригінального видання
social із царства мрій у царство дійсности. Але навіщо тоді взагалі «систематична колонізація»
протилежно до природної колонізації? Але, але: «сумнівно, чи в північних штатах американського союзу
хоч десята частина людности належить до категорії найманих робітників\dots{} В Англії\dots{} велика маса
народу складається з найманих робітників»\footnote{
Там же, стор. 42, 43, 44.
}. В дійсності нахилу до самоекспропріяції на славу
капіталові в трудящого людства так небагато, що рабство, навіть за Векфілдом, є єдина природна
основа колоніяльного багатства. Його систематична колонізація є просто pis aller\footnote*{
Pis aller — французький вираз: щось, до чого вдаються, коли немає нічого кращого. \emph{Ред.}
}, бо ж йому
доводиться мати справу з вільними людьми, а не з рабами. «Перші еспанські поселенці на Сан-Домінґо
не діставали робітників із Еспанії. Але без робітників [тобто без рабства] капітал був би загинув
або принаймні скоротився б до таких дрібних розмірів, що всякий індивід міг би застосувати його
своїми власними руками. Так воно в дійсності й сталося в останній заснованій англійцями колонії, де
великий капітал у насінні, худобі й знарядді загинув через недостачу найманих робітників, і де жоден
поселенець не має капіталу більше, ніж він може застосувати своїми власними руками»\footnote{
Там же, т. II, стор. 5.
}.

Ми бачили: експропріяція землі в народніх мас становить основу капіталістичного способу продукції.
Навпаки, суть вільних колоній у тому, що маса землі є ще народня власність, і тому кожний поселенець
може частину її перетворити на свою приватну власність і на свій індивідуальний засіб продукції, не
перешкоджаючи цим пізнішому поселенцеві зробити те саме\footnote{
«Щоб стати елементом колонізації, земля не лише повинна бути необробленою, але й бути
громадською власністю, яку можна перетворити на приватну власність». (Там же, т. II, стор. 125).
}. В цьому таємниця так процвітання колоній
як і їхніх болячок — їхнього опору проти вселення капіталу. «Де земля дуже дешева й усі люди вільні,
де кожний може з свого бажання дістати шматок землі для самого себе, там праця не лише дуже дорога,
беручи до уваги ту пайку, що припадає робітникові з його продукту, але й взагалі важко хоч за
якубудь ціну дістати комбіновану працю»\footnote{
Там же, т. І, стор. 247.
}. А що в колоніях немає ще відокремлення робітника від
умов праці й їхньої основи, від землі, або відокремлення таке існує лише спорадично або на занадто
обмеженому просторі, то там ще немає й відокремлення рільництва від промисловости і не знищена ще
сільська домашня промисловість. Але звідки ж тоді там узятися внутрішньому ринкові для капіталу? «За
винятком рабів та їхніх хазяїнів, що скомбіновують капітал і працю для великих підприємств, жодна
частина людности Америки не працює виключно коло рільництва.
\index{i}{0663}  %% посилання на сторінку оригінального видання
Вільні американці, що сами обробляють землю, мають одночасно ще багато інших занять. Частину
вживаних ними меблів і знарядь вони звичайно виготовлюють сами. Вони часто будують свої власні
будинки й постачають продукти своєї власної
промисловости на якнайдальші ринки. Вони одночасно прядуни й ткачі, вони фабрикують мило й свічки,
взуття і одяг для свого власного вжитку. В Америці рільництво часто є побічне заняття коваля,
мірошника або крамаря»\footnote{
Там же, стор. 21, 22.
}. Де ж тут лишається серед таких чудаків поле для «поздержливости»
капіталіста?

Велика принадність капіталістичної продукції в тому, що вона не лише постійно репродукує найманого
робітника як найманого робітника, але й пропорційно до акумуляції капіталу завжди продукує відносне
перелюднення найманих робітників.
Таким чином закон попиту й подання праці утримується в належній колії, коливання заробітної плати
вганяється у межі, вигідні для капіталістичної експлуатації, і, нарешті, ґарантується стільки
доконечну соціяльну залежність робітника від
капіталіста, те відношення абсолютної залежности, що його політико-економ може у себе дома, в
метрополії, пишномовно перебріхувати на вільне договірне відношення між покупцем і продавцем, між
двома однаково незалежними посідачами товарів,
посідачем товару капітал і посідачем товару праця. Але в колоніях ця чудова ілюзія зникає. Абсолютна
кількість людности тут зростає куди швидше, ніж у метрополії, бо багато робітників приходить тут на
світ уже дорослими, і все ж ринок праці тут завжди неповний. Закон попиту й подання праці тут цілком
крахує. З одного боку, старий світ постійно вкидає туди капітал, жаждущий експлуатації, охоплений
потребою в
поздержливості; з другого боку, реґулярна репродукція найманих робітників як найманих робітників
наражається на якнайнеприємніші й почасти непереможні перешкоди. Де ж тут думати про продукцію
зайвих найманих робітників пропорційно
до акумуляції капіталу! Сьогоднішній найманий робітник на завтра стає незалежним, самостійно
господарюючим селянином або ремісником. Він зникає з ринку праці, та тільки не в робітний дім. Це
постійне перетворювання найманих робітників на незалежних продуцентів, що працюють не на капітал, а
на самих себе, і збагачують не пана капіталіста, а самих себе, із свого боку надзвичайно шкідливо
впливає на стан
ринку праці. Не тільки ступінь експлуатації найманого робітника лишається до непристойности низький.
Найманий робітник, крім цього, втрачає разом із своєю залежністю й почуття залежности від
поздержливого капіталіста. Відси всі ті прикрості,
що їх так відважно, так пишномовно й так зворушливо змальовує нам Е.~Ґ Векфілд.

Подання найманої праці, — скаржиться він, — і непостійне, і нерівномірне, і недостатнє. Воно «не
лише завжди занадто
\parbreak{}  %% абзац продовжується на наступній сторінці

\parcont{}  %% абзац починається на попередній сторінці
\index{i}{0664}  %% посилання на сторінку оригінального видання
мале, але й неґарантоване».\footnote{
Там же, т. II, стор. 116.
} «Хоч продукт, призначений до розподілу між робітником і капіталістом,
і великий, але робітник бере собі таку велику частину, що він швидко стає капіталістом\dots{} Навпаки,
небагато людей, навіть коли вони живуть надзвичайно довго, можуть нагромадити великі маси
багатства».\footnote{
Там же, т. І, стор. 131.
} Робітники ні в якому разі не дозволяють капіталістові здержуватися від оплати їм за
найбільшу частину їхньої праці. Капіталістові ані крихти не допоможе, якщо він навіть
остільки хитрий, що разом із своїм власним капіталом імпортує із Европи й своїх власних найманих
робітників. «Вони незабаром перестають бути найманими робітниками, вони незабаром перетворюються на
незалежних селян, а то навіть і на конкурентів своїх колишніх хазяїнів на самому ринку найманої
праці».\footnote{
Там же, т. II, стор. 5.
} Уявіть собі, який жах! Чесний капіталіст за свої власні гроші сам імпортував з Европи
своїх власних живих конкурентів! Та це ж світу кінець! Не диво, що Векфілд скаржиться на недостатню
залежність і недостатнє почуття залежности в найманих робітників по колоніях. «У наслідок високої
заробітної плати, — каже його учень Мірвел, — в колоніях є палке жадання дешевої і покірнішої праці,
жадання такої кляси, що їй капіталіст
міг би диктувати свої умови, а не щоб вона йому диктувала їх\dots{} У країнах із старою цивілізацією
робітник, хоч і є вільний, але за силою природного закону залежить від капіталіста; в колоніях ця
залежність мусить бути створена штучними засобами».\footnote{
Merivale: «Lectures on Colonization and Colonies», London 1841 and 1842, vol. II, p. 235— 314 і
далі. Навіть лагідненький вульґарний економіст-фрітредер Молінарі каже: «В колоніях, де рабство
скасовано без заміни примусової праці на відповідну кількість вільної праці, ми бачили щось
протилежне тому, що бачимо щодня на власні очі. Ми
бачили, як прості робітники із свого боку експлуатують промислових підприємців, вимагають від них
заробітної плати, значно вищої від тієї законної частки їхнього продукту, що припадає їм.
Плянтатори, не маючи змоги дістати за свій цукор ціну достатню, щоб покрити підвищення заробітної
плати, мусіли покривати це збільшення спочатку із своїх зисків, а пізніш навіть із своїх капіталів.
Багато плянтаторів таким
чином зруйновано, іншим довелося закрити свої підприємства, щоб уникнути
неминучої руїни. Без сумніву, краще якщо загинуть нагромаджені капітали, ніж як загинуть цілі
покоління людей (яка великодушність з боку пана Молінарі!); але чи не краще було б, коли б не
загинули ні ті, ні ці?» («Dans les colonies où l’esclavage a été aboli sans que le travail forcé se
trouvât remplacé par une quantité équivalente de travail libre,
on a vu s’opérer la contre-partie du fait qui se réalise tous les jours sous nos yeux. On a vu les
simples travailleurs exploiter à leur tour les entrepreneurs d’industrie, exiger d’eux des salaires
hors de toute proportion avec la part légitime qui leur revenait dans le produit. Les planteurs, ne
pouvant obtenir de leurs sucres un prix suffisant pour couvrir la hausse du salaire, ont été obligés
de fournir l’excédant, d’abord sur leurs profits, ensuite sur leurs
capitaux mêmes. Une foule de planteurs ont été ruinés de la sorte, d’autres ont fermé leurs ateliers
pour échapper à une ruine imminente\dots{} Sans doute, il vaut mieux voir périr des accumulations de
capitaux, que des générations d'hommes: mais ne vaudrat-il pas mieux que ni les unes ni les autres périssent?» (Molinari: «Etudes Economiques», Paris 1846, p. 51, 52). Пане Молінарі,
пане Молінарі! Що це буде з десятьма заповідями, з Мойсеєм та пророками, із законом попиту й
подання, коли в Европі «entrepreneur»\footnote*{
— підприємець. \emph{Ред.}
} може скорочувати part légitime\footnote*{
— законну пайку. \emph{Ред.}
} робітника, а в Західній
Індії робітник part légitime підприємця. І скажіть, будь ласка, що це таке, ота «part légitime», що
її, як ви сами призналися, капіталіст
в Европі щодня не доплачує? Молінарі страшенно хочеться там, у колоніях, де робітники такі «прості»,
що «експлуатують» капіталістів, поліційними заходами надати належної чинности законові попиту й
подання, що в інших випадках діє автоматично.
}

\index{i}{0665}  %% посилання на сторінку оригінального видання
Які ж то, на думку Векфілда наслідки цього сумного стану в колоніях?\footnote*{
У другому німецькому виданні це речення зформульовано так: «Який же то результат панівної в
колоніях системи приватної власности, основаної на власній праці, а не на експлуатації чужої
праці?». \emph{Ред.}
} «Варварська система
розпорошености» продуцентів і національного майна.\footnote{
Wakefield: «England and America», London 1833, vol. II, p. 52.
} Роздрібнення засобів
продукції поміж численних самостійно господарюючих власників нищить з централізацією капіталу всі
основи комбінованої праці. Кожне розраховане на довгий час підприємство, що поширюється на багато
років і потребує витрати основного капіталу, наражається, переводячи свої справи, на перешкоди. В
Европі капітал не гає ані хвилинки, бо робітнича кляса становить там його живу приналежність, її там
з лишком, і він завжди може нею порядкувати. Але в колоніяльних країнах! Векфілд
оповідає надзвичайно сумну анекдоту. Він мав розмову з кількома капіталістами з Канади й штату
Нью-Йорк, де хвилі еміґрації часто спиняються, лишаючи по собі осад «зайвих» робітників. «Наш
капітал, — зідхає один з персонажів мелодрами, — наш капітал був напоготові для багатьох операцій,
що для свого виконання потребують чималого часу; але хіба ми могли починати такі операції з
робітниками, які — ми знали це — незабаром повернули б нам спину? Коли б ми були певні, що зможемо
вдержати в себе працю цих еміґрантів, ми охоче були б їх негайно найняли, та ще й за високу ціну. Ще
більше: навіть упевнені, що втратимо їх, ми все ж були б їх найняли, коли б були певні, що матимемо
нове подання праці, відповідно до наших потреб».\footnote{Там же, стор. 191, 192.}

Після того, як Векфілд так пишно змалював контраст між англійським капіталістичним рільництвом з
його «комбінованою» працею і розпорошеним американським селянським господарством, він мимохіть
пробовкнувся й про зворотний бік
медалі. Він змальовує американську народню масу як заможну, незалежну, підприємливу й порівняно
освічену, тимчасом як «англійський рільничий робітник є жалюгідний голодранець (a miserable wretch),
павпер\dots{} У якій іншій країні, крім Північної Америки й деяких нових колоній, заробітна плата за
вільну працю, вживану в рільництві, хоч у якійбудь вартій згадки
\parbreak{}  %% абзац продовжується на наступній сторінці

\parcont{}  %% абзац починається на попередній сторінці
\index{i}{0666}  %% посилання на сторінку оригінального видання
мірі перевищує найдоконечніші засоби існування робітника?.. Без сумніву, робочі коні в сільському
господарстві мають в Англії далеко кращий корм, аніж англійські рільничі робітники, бож коні є цінне
майно»\footnote{
Там же, т. І, стор. 47, 246.
}. Але never mind\footnote*{— що з того. \emph{Ред.}}, адже національне багатство з природи тотожнє з народніми злиднями.

Але як же вилікувати колонії від цієї антикапіталістичної болячки? Коли б хто хотів за одним махом
перетворити всю землю з народньої власности на приватну власність, то цим би він, правда, знищив
корінь зла, але разом з тим — і колонії. Майстерність у тім, щоб одним пострілом убити двох зайців.
Треба, щоб уряд надав незайманій землі штучну ціну, незалежну від закону попиту й подання, ціну, що
примусить еміґранта
працювати довший час найманим робітником, доки він заробить досить грошей, щоб купити собі землю\footnote{
«Ви кажете, що завдяки присвоєнню землі й капіталів, людина, яка не має нічого, крім своїх рук,
находить собі роботу та створює собі дохід\dots{} навпаки, лише завдяки індивідуальному присвоєнню
землі, стається те, що є люди, які не мають нічого, крім своїх рук. Ставлячи людину в безповітряний
простір, ви захоплюєте собі атмосферу. Те саме ви робите, захоплюючи собі землю\dots{} Це все одно, що
кинути людину в простір, де немає багатств, щоб зробити її життя залежним від вашої волі». («C’est,
ajoutez-vous, grâce à l’appropriation du sol et des capitaux que l’homme, qui n’a que ses bras,
trouve de l’occupation, et se fait un revenu\dots{} c’est au contraire, grâce à l’appropriation
individuelle du
sol qu’il se trouve des hommes n’ayant que leurs bras\dots{} Quand vous mettez un homme dans le vide,
vous vous emparez de l’atmosphère. Ainsi faites-vous, quand vous vous emparez du sol. C’est le
mettre dans le vide de richesse, pour ne le laisser vivre qu’à votre volonté»). (\emph{Colins}: «L’Economie
Politique, Source des Révolutions et des Utopies prétendues Sосіаlistes», Paris 1857, vol. III, p.
267--271 passim.).
} й перетворитись на незалежного селянина. З другого боку, фонду, створеного через продаж земель по
ціні, порівняно неприступній для найманого робітника, отже, цього грошового фонду, видушеного із
заробітної плати через порушення святого закону попиту й подання, уряд повинен уживати в міру його
зростання на те, щоб імпортувати з Европи в колонії голоту і підтримувати таким
чином для пана капіталіста ринок найманої праці повним. За таких обставин tout sera pour le mieux
dans le meilleur des mondes possibles\footnote*{
Все буде якнайкраще в цьому найкращому із світів. \emph{Ред.}}. Оце — велика таємниця «систематичної колонізації». «За цим
пляном, — вигукує тріюмфуючи Векфілд, —
подання праці мусить бути стале й реґулярне; бо, поперше, через те, що жоден робітник не має змоги
купити собі землі доти, доки він не попрацює певний час за гроші, всі еміґранти-робітники, працюючи
комбінованими групами як наймані робітники, продукували б своєму підприємцеві капітал для вживання
ще більшої кількости праці; подруге, кожний, що кинув би найману працю і став би земельним
власником, саме через купівлю землі забезпечував би певний фонд, щоб приставляти
\index{i}{0667}  %% посилання на сторінку оригінального видання
нових робітників у колонії»\footnote{
\emph{Wakefield}. Там же, т. II, стор. 192.
}. Октройована державою ціна землі мусить, звичайно, бути
«достатня» (sufficient price), тобто така висока, «щоб перешкоджати робітикам ставати незалежними
селянами доти, доки не з’являться інші, щоб заступити їхнє місце на ринку найманої праці»\footnote{
Там же, стор. 45.
}. Ця
«достатня ціна землі» є не що інше, як пом’якшене означення викупних грошей, які робітник платить
капіталістові за дозвіл покинути ринок найманої праці й заходитися коло обробітку землі. Спочатку
робітник мусить створити панові капіталістові «капітал», щоб він міг експлуатувати більше число
робітників, а потім він мусить приставити на ринок праці «заступника», якого його коштом уряд
транспортує із-за моря для його колишнього пана капіталіста.

Надзвичайно характеристично, що англійський уряд протягом багатьох років запроваджував цю методу
«первісної акумуляції капіталу», рекомендовану паном Векфілдом для вжитку спеціяльно по колоніяльних
країнах. Фіяско було, звичайно, таке саме ганебне, як фіяско з банковим актом Піла. Потік еміґрації
лише повернувся від англійських колоній до Сполучених штатів. Тимчасом проґрес капіталістичної
продукції в Европі, супроводжуваний дедалі більшим утиском з боку уряду, зробив рецепт Векфілда
зайвим. З одного боку, величезний і невпинний потік людей, що рік-у-рік тече до Америки, залишає на
сході Сполучених штатів застійні осади, бо хвиля еміґрації з Европи кидає туди людей на робітничий
ринок швидше, ніж друга хвиля еміґрації встигає занести їх на захід. З другого боку, американська
громадянська війна потягла за собою колосальний національний борг, а разом з ним податковий тиск,
народження найпідлішої фінансової аристократії, роздаровування величезної частини громадських земель
товариствам спекулянтів для експлуатації залізниць, копалень і~\abbr{т. ін.}, — одно слово, вона потягла за
собою якнайшвидшу централізацію капіталу. Отже, велика республіка перестала бути обітованою землею
для робітників-еміґрантів. Капіталістична продукція йде там велетенськими кроками вперед, хоч спад
заробітної плати й залежність найманого робітника далеко ще не зведені до европейського нормального
рівня. Безсоромне марнотратне роздаровування англійським урядом необроблених колоніяльних земель
аристократам і капіталістам, яке сам Векфілд голосно засуджує, разом із потоком людей, що їх
притягають копальні золота, і з конкуренцією, яку імпорт англійських товарів створює навіть
найдрібнішому ремісникові, породили, особливо в Австралії\footnote{
Скоро Австралія стала своєю власною законодавицею, вона звичайно, видала закони, сприятливі для
переселенців, але марнотратство земель, що його перевели вже англійці, стоїть на перешкоді. «Перша й
найважливіша мета, яку ставить новий земельний закон з року
1862, є в тому, щоб полегшити народові змогу розселюватися» («The first and main object at which the
new Land Act of 1862 aims, is to give increased facilities for the settlement of the people»). («The
Land Law of Victoria by the Hon. G.~Duffy, Minister of Public Lands», London 1862, p. 3).
}, достатнє «відносне перелюднення
\index{i}{0668}  %% посилання на сторінку оригінального видання
робітників», так що майже кожний поштовий корабель приносить із собою лихі звістки про
переповнення австралійського ринку праці — «glut of the Australian labour-market», — а проституція
процвітає там подекуди так само пишно, як і на Haymarket у Лондоні.

Однак нас цікавить тут не стан колоній. Нас цікавить лише таємниця, відкрита в Новому Світі
політичною економією Старого Світу і гучно проголошена нею: капіталістичний спосіб продукції й
акумуляції, отже, і капіталістична приватна власність зумовлюють знищення приватної власности,
основаної на власній праці, тобто зумовлюють експропріяцію робітника.

  \addtocontents{toc}{\protect\newpage}
  \setcounter{footnote}{0}% Reset footnote counter
\bookpages{Додаток}{Фрагмент «Капіталу» у~перекладі Івана~Франка}
 
\cftaddnumtitleline{toc}{book}{Додаток}{Фрагмент «Капіталу»}{}
\addtocontents{toc}{\protect\vspace{-2.4em}}
\addcontentsline{toc}{book}%
  {\protect\numberline{}{у~перекладі Івана~Франка}}%


% ЧОГО vtu ХОЧЕМО? 
% Вперше без підпису надруковано польськаю мовою в газеті «Ргаса», 1879, 18 серпня, під назвою «Czego my chcemy?». В перекладі українською мовою вперше надруковано у вид.: Франко І. Твор и. В 20-ти т., т, 19, с. 215-217. ПодаЕться за першодруком. 
% ВЛАСНІСТЬ ГРУНТОВА І Уі ІСТОРІЯ 
% Вперше надруковано в кн.: Лавле Е. де. Власність грунтова і її історія. Переклав Іван Франко. Львів, 1879 («Дрібна бібліотека», VI), с. 34. (Передмова до книги). ПодаЕться за першодруком. С. 28. лавеле  Еміль де (1822-1892) — бельгійський бур-жуаsний iсторик i •    економlст. Б ю к е р Карл (1847-1930) — німецький буржуазний еконо• міст, історик народного господарства і статистик. 
% ДОПОВНЕННЯ ДО «ОСНОВ СУСПІЛЬНОУ 
% ЕкономІт" 
% Вперше надруковано в журн. «Культура», 1926, Ns 4-9, с. 56-57, та в кн.: Іван Франко, К., 1926, с. 164-166. Це герша передмова до перекладу XXIV розділу «Капіталу» К. Маркса, доданого І. Франком до написаного ним підручника «Ос• нови суспільної економії», який мав вийти у світ в кінці 1879 — на початку 1880 р., але не був надрукований; рукопис його загуб- лено. ПодаЕться за автографом, який зберігся в архіві І. Франка ((р. 3, Nё 448). 
% ГДРУГА ЛЕРЕДМОВА ДО ПЕРЕКЛАДУ 
% 24-го РОЗДІЛУ ПРАцІ К. МАРКСА «КАП[ТАЛв. т. 1] 
% Вперше надруковано в ж,урн. «Культура», 1926, Ns 4-9, с. 57-58, та в кн.: Іван Франко. К., 1926, с. 167---168. Передмову до українського перекладу 24-го розділу «Капі- талу» К. Маркса, який І. Франко мав намір видати окремим ви- пуском «Дрібної бібліотеки» (див. коментар до перекладу в цьому томі)подаеться написано, ймовірно, на початку 1880 року.  за автографом, який э6ерігаЕться в архіві І.Франка, ф. 3, Ns 448. С. 32. ...щоби сама гграця стала товаром...- Тут франківський виклад змісту першого тому «Капіталу» К. Марк- са неточний. К. Маркс мав на уваэі не працю, а робоцу силу. С. 33. Текст перекладу подаЕться у розділі «3 наукових пере- кладів» (с. 581--609). 
% 616 



\section*{Доповненя до „Основ суспільної економії“\protect\footnotemarkZ{}}
\nonumsectioncft{Доповненя до „Основ суспільної економії“}{.~}{Іван Франко}

\footnotetextZ{Вперше надруковано в журн. «Культура», 1926, № 4--9, с. 56--57, та в кн.: Іван Франко, К., 1926, с.~164--166.

Подається за автографом: відділ рукописних фондів і текстології Інституту літератури ім. Т.~Г.~Шевченка НАН України. — Ф. 3. — Од. зб. 448. — 14 арк. 
}

\noindent{}В самім початку „Основ суспільної економії“ сказано було, що економія, се наука абстрактна, т. є. що ціль єї не є виключно — розслідити закони економічні \emph{теперішної} суспільности, але \emph{загальні} закони праці людської. А позаяк с переміною суспільного ладу в протягу віків і закони ті проявляются щораз то в інших формах, випливаючих конечно з даного ладу, то наука економічна не може ніякої с тих форм вважати сталою і незмінною. Не може, значит, і нинішних форм уважати сталими, а мусит шукати таких форм, котрі \emph{після нашого теперішного знаня} булиб відповіднійші для суспільної праці і суспільного добробутку, ніж нинішні форми.

С тої то причини в сістематичнім викладі основ сусп. економії ми не могли давати надто широкого місця вислідам про \emph{нинішний} лад, а ограничились тілько головним єго нарисом. При викладі абстрактної теорії праці се була конечна річ, — але прецінь ніхто не заперечит, що на практиці для кождого дуже важне — знати передовсім докладно теперішний лад, єго почин і розвиток. Таке знанє вже тим корисне, що замісць теоретичних засад подає масу фактів, котрі самі прут розум до таких а таких виводів, між тим коли ті самі виводи, подані без підставних фактів, усякому можут видатися хиткими та схопленими з воздуха мріями. Для того то думаєм ми, що поповнимо подекуди конечний недостаток теоретичного викладу, подаючи в „Доповнених“ обширнійший огляд деяких питань, не порушених або з боку ткнених в самім викладі.

Одна з найважнійших недостач усякого чисто теоретичного викладу та, що приходится виключати з него всякі ширші \emph{історичні} перегляди. Правда, се не є недостача конечна, бо остаточно мож би бути вірним теорії, подаючи перегляд розвитку та впадку всіх економічних порядків від почину цівілізації аж до тепер. Але не кажучи вже о тім, що для такої загальної історії економічного розвитку призбирано доси дуже ще мало матеріялу, — в нашім підручнику такий виклад був би неможливий вже й за недостачею місця. А говорити обширно про розвиток одного — ниніншого — ладу, не казавши нічо про розвиток їнчих, се значилоб вважати сей лад чимось важнійшим від прочих, між тим коли в історії, як і в зрості кождого орґанізму, кожда фаза розвитку для вислідника рівноважна.

Але вважаючи потрібним познайомити наших читателів з історичним розвитком сучасного, капіталістичного ладу, ми робимо се в „Доповненях“. А для своєї ціли ми не можем найти кращого провідника над Карля Маркса, котрий в однім розділі своєї книжки „Das Kapital“ списав короткий, хоть яркий перегляд того, як розвивалася капіталістична продукція. С тим розділом ми й хочемо познакомити наших читателів.


\section*{[Друга передмова до перекладу 24-го розділу праці К.~Маркса «Капітал», т. І]\protect\footnotemarkZ{}}
\nonumsectioncft{[Друга передмова до перекладу 24-го розділу праці К.~Маркса «Капітал», т. І]}{.~}{Іван Франко}

\footnotetextZ{Вперше надруковано в журн. «Культура», 1926, № 4--9, с. 57--58, та в кн.: Іван Франко, К., 1926, с.~167--168.

Подається за автографом: відділ рукописних фондів і текстології Інституту літератури ім. Т.~Г.~Шевченка НАН України. — Ф. 3. — Од. зб. 448. — 14 арк. 
}

\noindent{}В першій части своєї великої економічної праці про „Капітал“ стараєсь Карль Маркс вияснити передовсім, \emph{як повстає капітал}? В тій ціли виказує він поперед усего, що єдиним жерелом усякої вартости є праця людська, котра з матеріалів сирих, даних природою, і при помочи сил природи витворює предмети вжиточні для чоловіка. Коли предмети такі витворюются не для власного вжитку самого витвірця, а для заміни за їнші, тоді вони звутся товарами. Капіталістична продукція полягає на витворюваню товарів, але не всяка продукція, де витворюются товарі, є вже капіталістична. До того потрібно ще одної дуже важної вимінки: \emph{щоби сама праця стала товаром}, т. є. щоб на торзі за певний товар (гроші) мож було заміняти (купити) працю людську.

Звичайно під назвою капіталу у нас розуміются беззглядно гроші. Се по части хибно. Гроші, як бачимо, тоді тілько стают капіталом, коли за них купуєся на торзі робуча сила.

Але праця людська, се не є звичайний товар. Се товар живий, котрий має тоту властивість, що \emph{надає вартість} другим предметам, і надає єї більше, ніж кілько сам коштує. Торгова ціна праці, так як і ціна кождого товару, означена звичайними економічними правилами, с котрих найважнійше — кошт витвореня товару, т. є. в тім разі — кошт удержаня робітника і єго робучої сили. Таку ціну платит капіталіст робітникови за єго працю. Між тим робітник в тім часі, на котрий нанявся, витворює далеко більше, ніж кілько виносит єго плата. Він витворив \emph{надзвишку вартости} понад вартість своєї плати, — тота надзвишка, се зиск капіталіста, — вона побільшує єго капітал. Значит, уся капіталістична продукція полягає на твореню надзвишки, котра задармо дістаєсь капіталістови. Цілий розвиток економічний капіталістичної продукції полягає на тім, що капіталісти всіми силами старалися до крайної можности вбільшити тоту надвишку. Вбільшити єї мож було двома способами: або продовжуючи день робучий (надвишка абсолютна), або приневолюючи робітників в коротшім часі працювати з більшою натугою (релятівна надвишка). Оба ті способи витрібували капіталісти, і то перший з них (продовженє робучого дня) до такої крайности, що аж уряд, затрівожений робітницькими розрухами, мусів вдатися в те діло і ограничити стало довготу робучого дня. Від тоді капіталістична продукція і доси пре в другий бік, — стараєсь той означений правно день робучий як найдоскональше використати, раз-ураз заводячи нові машини, котрі до крайности упрощуют і прискорюют продукцію, а до обслуги вимагают як найменшого числа рук.

Се головні думки, виведені Марксом з безмірної маси фактів, нагромаджених в єго книжці. При кінци книжки розбирає він ще одно важне питане: Яким способом почалася тота капіталістична продукція? Як і на якім ґрунті та при якій управі виріс той дивний порядок, оснований на щоденнім хитрім визиськуваню, на крайній бідносте незлічимих мас народа, а крайнім богацтві немногих щасливців? Сесь важний розділ Марксової книжки — прекрасний культурно-історичний очерк — зрозумілий буде і окремо від цілої книжки і ми хочемо познакомити з ним нашу громаду, як для самої єго великої стійности наукової, так і для того, щоб заохотити всіх, хто тілько владає німецькою мовою, до читаня цілої Марксової книжки. Звичайно говорится про дуже трудний і незрозумілий спосіб писаня у Маркса. Се мож би сказати хіба про перший розділ єго книжки, — а о?кілько такий суд справедливий що до прочих розділів, най посвідчит тота часть, котра отсе переведена.

\setcounter{chapter}{23}
\sectionextended[%
Початок і історичний розвиток капіталістичної продукції в Англії
]{%
Початок і історичний розвиток капіталістичної продукції в Англії\footnotemarkZ{}}{%
\subsection{Первісне нагромадженє капіталу}}
\markboth{%
Початок і історичний розвиток капіталістичної продукції в Англії}{%
Фрагмент «Капіталу» у~перекладі Івана~Франка}

Ми бачили,
\footnotetextZ{Вперше надруковано в журн. «Культура», 1926, № 4--9, с. 61--87.
Подається за автографом перекладача: відділ рукописних фондів і текстології Інституту літератури ім. Т.~Г.~Шевченка НАН України. — Ф. 3. — Од. зб. 448. — 14 арк. Кінець автографа не зберігся. 

Переклад зроблено з другого німецького видання: \textgerman{Das Kapital. Kritik der politischen Oekonomie. Von Karl Marx. Erster Band. Zweite verbesserte Aufgabe. Hamburg. Verlag von Otto Meissner, 1872.} Про це є згадка І. Франка на початку тексту перекладу «Гл[яди] К. Marx. Das Kapital, 2 вид. з р. 1872, стор. 742--794».}
що гроші стают капіталом тоді, коли служат
до купованя робучої сили. Ми бачили, що капітал
раз~у~раз намагає — творити надзвишку вартости, а надзвишка
вбільшує капітал. Між тим щоб капітал міг нагромаджуватись,
мусит уже вперед витворюватись надзвишка;
щоб могла витворюватись надзвишка, мусит істнувати капіталістична
продукція, а щоб тота істнувала, мусит уже
вперед більша маса капіталу бути нагромаджена в руках
поєдинчих богатирів. Здаєсь затим, що весь той процес
полягає на якімось „первіснім“ нагромадженю, котре мало
місце перед капіталістичною продукцією, котре, значит, не
було випливом капіталістичної продукції, а єї жерелом.

\index{franko}{0062}
Тото первісне нагромадженє капіталу („previous accumulation“,
як каже А.~Сміт) грає в суспільній економії
майже таку саму ролю, як „гріхопаденіє“ в теольоґії. Адам
зїв яблоко і через те стягнув гріх на рід людський. Початок
гріха обяснений казкою про давнину. Колись-колись
в давнину були з одного боку пильні вибранці, а з другого —
ліниві нероби. Через те сталося, що перші нагромадили
богацтво, а другі зійшли на таке, що остаточно не мали
вже що продавати крім себе самих. І від того гріхопаденія
почалася бідність великої маси, котра ще й доси, хоть і як
тяжко працює, не має що продавати крім себе самих, —
і богацтво деяких, що й доси змагаєся, хоть самі вони
давно перестали працювати\footnote{
Такі безглузді дитиньства плете ще д. Тйер (звісний французький
муж стану) дотепним колись французам с повагою великого мудрця —
для оборони святої власности. Ну і справді, — скоро діло йде о власність,
то святий обовязок кождого — міцно стояти на становищи букваря,
ще й других переконувати, що те становище для всякого „віка
і возраста“ єдино відповідне і належне.
}. В правдивій історії грали, як
звісно, завойованя, гнет, рабунки, вбійства, — одним словом,
усілякі насиля велику ролю. Але в сумирній політичній
економії з давен-давна — все іділлія. Право і „праця“, се
здавна були єдині способи до збогаченя, тілько, розумієся,
завсігди с тим застереженєм, що аж „сего року воно щось
не так“. Але на ділі способи первісного нагромадженя капіталу
були всякі, які хочете, — тілько не іділлічні.

Гроші і товар не є зразу капіталом, таксамо, як не
є ним зразу средства продукційні і знадоби до житя. Вони
мусят бути перемінені в капітал. Але та переміна може настати
тілько серед певних обставин, котрі зводятся ось на
що: двоякі дуже відмінні посідачі товарів мусят стати супротів
себе і зіткнутися с собою, — з одного боку властивці
грошей, средств продукційних і знадіб до житя, котрим
о то йде, щоб свою суму вартостей побільшити купівлею
чужої робучої сили; а з другого боку вільні робітники, продавці
власної робучої сили і, значит, продавці \so{праці}.
Вільні вони мусят бути в двоякім значіню, т. є. щоб ані
самі вони беспосередно не були средствами продукційними,
як невольники, кріпаки і т. д., ані шоб вони самі не посідали
средств продукційних, як ґазди-селяне, дрібні властивці
ґрунтові і т. д. Такий розділ товарив між дві крайности
— се основні вимінки для капіталістичної продукції.
Без відділеня робітників від власности не може настати
капіталістична продукція. Але скоро вона раз настала, то
не тілько підтримує те відділенє, але й сама доводит до
него раз~у~раз на~ново і раз~у~раз на більший розмір. Коли
затим спитаємо: де є жерело капіталістичного ладу? то
\parbreak{}

\parcont{}  %% абзац починається на попередній сторінці
\index{iii1}{0063}  %% посилання на сторінку оригінального видання
Мальтус, Сеніор, Торренс і т. д., ці явища наводяться безпосередньо
як докази того, ніби капітал просто в своєму речовому
існуванні, незалежно від того суспільного відношення до
праці, в якому він саме й стає капіталом, є, поряд з працею
і незалежно від праці, самостійним джерелом додаткової вартості.
— 2) Під рубрикою витрат, куди належить заробітна плата
цілком так само, як і ціна сировинного матеріалу, зношування
машин і т. д., видушування неоплаченої праці здається тільки
заощадженням на оплаті одного з тих предметів, які входять
у витрати, тільки меншою платою за певну кількість праці;
цілком так само, як відбувається заощадження, коли дешевше
купують сировинний матеріал або зменшують зношування машин.
Таким чином видушування додаткової праці втрачає свій
специфічний характер; його специфічне відношення до додаткової
вартості затемнюється; і цьому затемнінню дуже допомагає
і дуже його полегшує, як показано в книзі І, відділ VI,
представлення вартості робочої сили в формі заробітної плати.

Через те що всі частини капіталу однаково здаються джерелами
надлишкової вартості (зиску), то капіталістичне відношення
містифікується.

Той спосіб, яким додаткова вартість за допомогою переходу
через норму зиску перетворюється в форму зиску, є, однак,
тільки дальший розвиток того переплутання суб’єкта і об’єкта,
яке відбувається уже в процесі виробництва. Вже там ми бачили,
як усі суб’єктивні продуктивні сили праці здаються продуктивними
силами капіталу. З одного боку, вартість, минула праця,
яка панує над живою працею, персоніфікується в капіталісті;
з другого боку, навпаки, робітник виступає просто як предметна
робоча сила, як товар. З цього перекрученого відношення неминуче
виникає вже в самому простому відношенні виробництва
відповідне перекручене уявлення, перенесена з цього відношення
свідомість, яка розвивається далі в наслідок перетворень і модифікацій
власне процесу циркуляції.

Спроба представити закони норми зиску безпосередньо як закони
норми додаткової вартості, або навпаки, є цілком хибна, як
у цьому можна пересвідчитися на прикладі школи Рікардо. В голові
капіталіста, звичайно, ці закони не розрізняються. У виразі m: K
додаткова вартість вимірюється вартістю всього капіталу, авансованого
на її виробництво і почасти в цьому виробництві цілком спожитого,
а почасти тільки застосованого в ньому. Відношення m: K в
дійсності виражає ступінь зростання вартості всього авансованого
капіталу, тобто, взяте відповідно до його раціонального, внутрішнього
зв’язку і природи додаткової вартості, воно показує,
яке є відношення величини, на яку змінився змінний капітал, до
величини всього авансованого капіталу.

\index{franko}{0064}

\subsection{Вивласненє хліборобів}

В Англії щезло кріпацтво дійсно в послідній части \RNum{14} віку. Огромна більшість людности тоді, а ще
більше в \RNum{15} віці, се були свобідні хлібороби, дрібні посідачі ґрунтів, ґазди, — хоть власність їх і
була прикрита різними феодальними прикривками\footnote{
Ще при кінци \RNum{17} віку звиж \sfrac{4}{5} усеї англійської людности були самостійні ґазди-хлібороби, як се
стверджує Маколєй. (Macaulay: „The History of England“, Lond. 1854, v. I, p. 413). Я покликуюсь на
Маколєя тим радше, що він сістематично фальшує історію і подібні факти стараєсь о кілько мож
„обкроювати“.
}. В більших панських добрах замісць давнійших
кріпаків-совтисів (bailiff) настали тепер свобідні арендаторі. Наємні робітники до хліборобства, се
були по части самостійні ґазди-хлібороби, котрі при вільнім часі йшли до пана на заробок, а по части
була се відрубна, стосунково і абсолютно мала верства властивих наймитів. І ті послідні на ділі були
також самостійними ґаздами, бо крім платні одержували від пана також поле коло 4 екрів завбільшки і
коттедж (хату). Притім порівно с прочими ґаздами вони допущені були до вживаня громадського ґрунту,
т. є. толоки, де паслась їх худоба, і ліса, відки вони брали топливо, дерево, торф і пр.\footnote{
Не тре забувати, що навіть і кріпак був не тілько властивцем — хоть за оплатою — тих часток
ґрунту, котрі належали до єго дому, але був також співвластивцем громадських ґрунтів. (Порівн., що
каже Мірабó про шльонських хліборобів в книжці: „De la Monarchie Prussiennе“, Londres 1788).
} У всіх краях Европи ціхує феодальну продукцію поділ ґрунту поміж як мож найбільше підданих. Сила
феодального пана, як і сила кождого короля, полягала не в великости єго доходів, а в многоті єго
підданих, а многота сеся залежала від многоти самостійних ґаздів, осілих на єго добрах\footnote{
Японія зі своїм чисто феодальним упорядкованєм ґрунтової власности і розвитим дрібним
ґосподарством хліборобським вказує далеко вірнійший образ середновікової Европи, ніж усі накупі наші
історії, звичайно закаламучені буржоазними пересудами. Се, бач, дуже вигідна річ —
„ліберальствувати“ на кош(т) середних віків!}. Хоть
затим англійський край по норманськім завойованю поділено на величезні баронства, с котрих одно
нераз містило в собі 900 анґльосаских льордств, то прецінь край той був покритий дрібними
хліборобськими ґаздівствами, серед котрих тілько декуди розлягалися великі панські добра. Такі
стосунки при рівночаснім росцвіті міст, котрий наступив в \RNum{15} віці, сприяли заможности люду, яку
описує канцлєр державний Фортеске в своїх „Laudes Legum Angliae“, але при них не можливе було
капіталістичне богацтво.

\index{franko}{0065}
Перший крок перевороту, що поклав основу капталістичній продукції, припадає в послідній третині 15 і
в першій чверти 16 віку. Тоді скасовано феодальне дворацтво, котре, як справедливо замічає Джемс
Стюерт, „залякало  всі хати і двори безхосенно“. Через те викинено масу голих пролєтаріїв на
робучий торг. Хоть королівська власть, що й сама виросла з буржуазного розвитку, намагаючи до
неограниченого панованя, силою скасувала те великопанське дворацтво, то прецінь вона не була єдиною
причиною нового перевороту. Ні, в упертім опорі протів королівства та
парляменту витворили великі пани-феодали далеко більшу масу пролєтаріяту, прогонюючи силою
хліборобів з ґрунту і посідлости, хоть хлібороби мали до тих ґрунтів більше право, ніж вони, і
забираючи для себе громадські ґрунти. Беспосередний товчок до того в Англії дав іменно росцвіт
фляндрійської вовняної мануфактури і звязане з ним підскоченє цін вовни. Стара феодальна шляхта
вигибла в великих феодальних війнах, а нова шляхта — се були діти свого часу, для котрих гроші були
силою понад всі сили. З вірного поля пасовиська для овець! — се став тепер їх загальний оклик.
Гаррізен в своїй „Description of England. Prefixed to Holinshed’s Chronicles“ описує, як
вивласнюванє дрібних ґаздів руйнує край. „Але що нашим великим самозванцям до того?“ Мешканя ґаздів
та коттеджі робітників валят вони силою або прогнавши людей лишают пустками. „Коли перездримо
давнійші інвентарі кождої домінії, то побачимо, що незлічимі хати та дрібні ґаздівства пощезали, що
ґрунт годує далеко меньше люда, що богато міст підупало, хоть деякі нові підносятся\dots{} Мож би
чимало наросповідатися про місточка та села, зруйновані для того, щоб було місце на толоки для
овець; тілько самотні панські двори стоят серед тих толок“. Правда, наріканя тих старих літописів
усе пересаджені, але вони досадно малюют те вражінє, яке на самих сучасників робив переворот
обставин продукційних. Порівнанє між письмами канцлєрів Фортеске і Томаса Моруса вказує наглядно
пропасть між 15. а 16. віком. „Із золотого віку — каже справедливо Зорнтон — попали англійські
робітники без ніяких перехідних ступнів прямо в зелізну“.

Праводавство злякалось сего перевороту. Воно не стояло ще на такім високім ступни цівілізації, де
„богацтво народне“, т. є. богацтво капіталістів і безграничне висисанє та зубожінє маси люду
становит верх премудрости
політичної. В своїй історії Генріха VII. каже Бекон: „В тім часі (1489) посипалися скарги на то, що
вірне поле перемінюєсь в пасовиська, котрих лехко може дозирати кілька пастухів. Ґрунти, що вперед
виарендовувались на кілька літ, на доживотну або щорічну умову, тепер зіллято разом
\index{franko}{0066}
с панськими. Се підкопало добробуток люду, а через те й міста, церкви, десятини\dots{} Щоб зарадити
тому лиху, проявили король і парлямент дивну на ті часи мудрість\dots{} Вони видали право протів того
обезлюднюючого край загарбуваня громадських ґрунтів (depopulating inclosures) і невідлучної
від него обезлюднюючої ґосподарки толочної (depopulating pasture[s])“. Оден акт Генріха VII. з р.
1489 заказує руйнувати хліборобські хати, до котрих належит що найменьше 20 екрів ґрунту. Генріх
VIII відновив той самий указ. Говорится там між їншим, що „многі аренди і огромні отари, особливо
овець, нагромаджуются в немногих руках, через що дохід
з ґрунту дуже вбільшився, а рільництво дуже підупало, церкви і хати повалено, дивовижні маси народа
стали неспосібні вдержувати себе і свої родини“. Указ наказує затим відбудовувати повалені хутори,
означує, кілько має бути вірного поля в стосунку до овечих толок і т. д. Їнший акт з р. 1533
жалуєсь, що деякі властивці мают по 24000 овець, і ограничує їх число на 2000\footnote{
В своїй „Утопії“ говорит Томас Морус про дивовижний край, де
„вівці їдят людей“.
}. Наріканя народа і
праводавство протів вивласнюваня дрібних арендаторів та хліборобів, що почалось від Генріха VII і
трівало зо 150 літ
— не помогли нічо. Чому не помогли, пояснює нам Бекон, сам того не знаючи. „Акт Генріха VII, — каже
він в своїх „Essays, civil and moral“, Sect. 20, — був глибоко і дивно обдуманий. Він утворив
сільскі ґаздівства і хліборобські доми певного нормального розміру, т. є. вдержав для них таку
пропорцію ґрунту, котра давала їм змогу плодити на світ підданих доста заможних і не придавлених
нуждою, так що плуг був в руках властивців, а не наємників\footnote{
Бекон пояснює далі звязок між свобідним, заможним селянством
а доброю інфантерією. „Се була дивно важна річ для сили і мужности
королівства — мати аренди достаточного розміру, щоб дільних мужів
забеспечити від нужди і велику часть ґрунту краєвого запевнити в посіданє джоменам, т. є. людім
середної заможности між шляхтою а халупниками (cottagers) та наймитами. Бо се загальна думка
найліпших знавців воєнного діла\dots{} що головна сила армії, се інфантерія або піхота. Але щоб
витворити добру інфантерію, тре людей вихованих не в притиску ані в нужді, але свобідно і в певній
заможности. Коли затим яка держава вросте переважно в шляхту та делікатне панство, а хлібороби та
ратаї зійдут на простих зарібників та наймитів або халупників, т. є. жебраків з власною хатою, то
така держава може мати добру кінницю, але доброї піхоти не буде мати. Се видно в Італії і Франції і
деяких других заграничних краях, де справді все або шляхта або нужденні зарібники\dots{} Дійшло там до
того, що ті краї мусят уживати наємного зброду Швейцарів та др. для своєї піхоти: відти то й пішло,
що ті держави мают богато людий, а мало вояків“. („The Reign of Henry VII.“ і т. д.).
}. А між
\parbreak{}

\input{franko/_0067.tex}
\index{franko}{0068}
Але сесі беспосередні наслідки реформації не були
найтривкійші. Церковна власність, се була реліґійна підпора
старосвіцьких порядків ґрунтових. Впала вона, то й їм не
довго було вже встоятись.

Ще в послідних десятилітях 17. віку було джоменів
(самостійних ґаздів хліборобів) більше ніж арендаторів.
Вони творили головну силу Кромвеля і — як свідчит сам
Маколєй — визначувались дуже корисно супротів роспитих
паничів та їх прислужників — сільских попів. Ще навіть
сільскі наємники були співвластивцями громадського ґрунту.
Аж около 1750. щезли джомени зовсім, а в послідних десятиліттях
18. віку щезли послідні сліди громадських ґрунтів
хліборобських. Ми ту не берем на ввагу чисто економічних
двигачів рільничого перевороту, але глядимо тілько на пoсторонні,
насильні товчки.

За реставрації Стюартів перевели великі властивці
ґрунтів правним способом такий самий рабунок, який в прочій
Европі робився і без правних оборотів. Вони знесли
феодальні ґрунтові порядки, т. є. скасували всі ті повинности,
які припадали державі з ґрунтів, „відшкодували“ державу
тим, що наложили податки на хліборобів та прочу
масу народа, а самі забрали в тісну приватну власність усі
добра, над котрими вперед мали лиш феодальну зверхність,
і накинули вкінци народови такі права осідленя (laws of
settlement), котрі, mutatis mutandis, так само повліяли на
англійських хліборобів, як указ татарина Бориса Ґодунова
на россійських хліборобів.

„Преславна революція“ (glorious Revolution) з Вільгельмом
III Оранським дала панованє в руки ґрунтових та капіталістичних
богатирів. Вони почали нову еру тим, що до
роскраданя державних ґрунтів, котре доси велося скромно
і тайком, взялися тепер на кольосальний розмір. Ті ґрунти
роздаровувано, продавано за песі гроші або й прямо без
даня рації прилучувано до приватних дібр\footnote{
„Безправна рострата коронних дібр чи то через продаж, чи через
роздарованє, становит огидну картку англійської історії\dots{} Се величезне
окраденє народа\dots{}“ (F.~W.~Newmann: „Lectures on Political Economy.
London, 1851“. стор. 129, 130).
}. Все то робилося
без найменьшої вваги на правні формальности. Ті закрадені
добра державні ураз із церковним фурфантєм, яке
\parbreak{}

\parcont{}
\index{franko}{0069}
ще не було розгарбане за революції, се основа нинішних
князівських посідлостей англійської оліґархії\footnote{
Прошу прочитати н. пр. Е.~Борка памфлєт про родину герцоґів
Бедфорд, котрої потомок, льорд Джон Россель — один з головних стовпів
теперішного лібералізму.
}. Капіталісти
з міщан радо дивилися на ті операції, між їншим і для
того, бо ґрунти через те робилися чистим товаром, а сільскі
пролєтарії, обідрані до крихти, чим раз більше тислися
до міст за роботою. Вони поступали зовсім відповідно для
власної користи, так само, як шведські міщане, котрих економічною
опорою було селянство і котрі затим дружно с селянами
помогали королям (від р. 1604, пізнійше під Карлом
X і Карлом XI) силою видирати коронні добра з рук
маґнатів.

Власність громадська, се була староґерманська встанова,
котра животіла під покривкою феодальства. Ми бачили,
як тоті громадські ґрунти силою загарбувано, при
чім по більшій части рілю перемінювано в толоки. Се почалося
с кінцем \RNum{15} віку і трівало далі в \RNum{16} Але тоді було
се все такі особистим насилєм, супротів котрого праводавство
дармо боролося цілих 150 літ. Поступ \RNum{18} віку проявляєся
тим, що само право від тепер починає підпирати рабунок
громадських ґрунтів, хоть великі арендаторі побіч
того не закидают і своїх дрібних незалежних способиків на
власну руку\footnote{
„Арендаторі заказуют коттеджерам (халупникам) держати будь
яку будь живу тварь крім себе самих, а то тому, бо як будут держати
худобу або дріб, то будут мусіли з їх стоділ красти пашу. У них є приповідка:
„держи халупника в бідности, то вдержиш го в пильности“.
А властиво все діло ту в тім, що арендаторі таким способом привласнили
собі виключне право на громадські ґрунти“, („А Political Enquiry into
the Consequences of enclosing Waste Lands. Lond. 1785“, стор. 75).
}.  Парляментарною формою, в якій відбувалися
ті рабунки, були „Bills for Inclosures of Commons“ (Закони
про прилученє громадських ґрунтів). Се були декрети, котрими
сільскі льорди роздаровували власність народну самі
собі на власність приватну, — правдиві декрети обдираня
народа. Сер Ф.~М.~Еден, котрий хитро, як правдивий адвокат,
доказує, що ґрунти громадські, се властиво приватна
власність сільских льордів, що настали намісць феодалів,
— сам же зараз збиває всі свої докази, коли домагався
„загальної постанови парляменту для прилученя громадських
ґрунтів (до дібр приватних)“, — значит, признає, що
для їх переміни в приватну власність конечно треба парляментарного
замаху, — а з другого боку сам домагався
від праводавства „відшкодованя“ для вивласнених бідаків.

Між тим коли замісць незалежних їоменів (ґаздів) настали
„tenants-at-will“, т. є. дрібні арендаторі на оден рік,
\parbreak{}

\parcont{}
\index{franko}{0070}
льокайський і від самоволі лєндльордів залежний збрід, розросталися тимчасом з рабунку державних
дібр, а ще більше з сістематичного загарбуваня громадських ґрунтів ті великі аренди, котрі в 18. в.
звано арендами капіталовими або купецькими. Чим більше вони розросталися, тим більше селян витискано
з їх давних домівок, тим більше пролєтарів перлося до міст, до промислу.

Але \RNum{18} вік не понимав ще так досконало, як \RNum{19}, що „богацтво національне“, а вбожество народне —
одно й то само. Про те горячі спори в тогочасній економічній літературі зза „прилучуваня громадських
ґрунтів“. З великої
маси матеріялу, який маю під руками, подаю отсе кілька виривків, бо в них живо малюєся тодішне
положінє.

„В многих округах в Гертфордшайрі“, пише з обуренєм Томас Урайт, „зіллято 24 аренди, кожда пересічно
в 50--150 екрів, усего в 3 аренди“. „В Нортгемтоншайрі і Лінкольншайрі загалом поприлучувано
громадські ґрунти до приватних дібр, а повсталі відси нові льордства поперевертано в толоки. Через
те в многих льордствах не ореся тепер і 50 екрів, де вперед орано 1500\dots{} Звалища колишних хат,
стоділ, стаєнь і т. д., се єдині сліди по давнійших мешканцях. З соток домів і родин де в яких селах
полишалося по 8--10. В найбільшій части округів, де прилучуванє почалося ледво від 15--20 літ назад,
уже властивців ґрунтових дуже мало супротів того, що було вперед. Се ще звичайна річ, коли 4 або 5
богатих годівників худоби посідают недавно позлучувані льордства, на котрих уперед жило 20--30
арендаторів і богато
% REMOVED \footnote*{В рукописі: богати.}
дрібних властивців та комірників. Всіх їх з родинами й цілим спрятком
повикидано гет, а з ними й богато таких родин, котрі у них зарабляли собі прожиток“. (Се пише ч.
Аддіґтон). І прилучували сусідні лєндльорди на підставі Bills for enclosures не тілько перелоги, але
часто й управні ґрунти, котрі громада або винаймала поєдинчим ґаздам за певною оплатою, або
оброблювала спільно. Говорю ту про прилучуванє царини
і загалом управних ґрунтів. Навіть писателі, котрі боронят „прилучуваня“, признают, що воно в тім
разі вменьшило управу піль, підняло в гору ціни за живність і причинилося до обезлюдненя сіл\dots{} А
навіть прилучуванє пустих ґрунтів, яке тепер відбуваєся, відбирає бідному часть утриманя
і вбільшує аренди, котрі й так уже за великі“\footnote{
Др.~Річард Прайс в своїй кнпжці: „Observations on Reversionary
Payments“, т. II, стор. 155. Прошу читати Форстера, Аддінґтона, Кента,
Прайса і Джемса Андерсона, а порівнати се з нужденною балаканкою
та вонючими похвалами Мак-Кельльока в єго списі: „The Literature of
Political Economy. Lond. 1845“.
}. „Коли“,
\parbreak{}

\parcont{}
\index{franko}{0071}
каже Річард Прайс, „всі ґрунти будут в руках кількох великих арендаторів, то з дрібних арендаторів (про них Прайс
казав уперед ось що: „множество дрібних властивців і арендаторів, що вдержуют самі себе й свої родини добутками з ґрунту, котрий оброблюют, доходами з овець, дробу, свиней і т. д.,
котрі випасают на громадських толоках, так, що для вдержаня їм мало що приходится докуповувати“)
пороблятся люде, котрі будут мусіли працею заробляти на прожиток собі і другим, і все, чого їм
треба, будут мусіли купувати на торзі\dots{} Бути може, що праці тоді буде більше, бо більше буде примусу\dots{} Міста й
мануфактури будут змагатися, бо до них напхаєся більше людей шукаючих заняття. Се тота дорога, по
котрій зовсім природно пре концентрація аренд і по котрій вона дійсно довгі вже літа чим раз далі
посуває Англію“. Загальне вліянє „прилучень“ ось як описує Прайс: „Взагалі положінє нижчих верстов
народа майже в кождім згляді погіршилося. Дрібні властивці та арендаторі зруйновані та зведені до
стану наємннків та комірників; а рівночасно й о прожиток в тім стані стало далеко тяжше“\footnote{
В наведеній книжці Р.~Прайса, стор. 147, 159. Се нагадує стародавний Рим, котрого порядки ось як
описує Аппіан в „Історії римських війн домашних“, кн. І, 7: „Богачі забрали в свої руки няйбільшу
часть неподілених ґрунтів. Вони задуфали на обставини часу, що їм тих ґрунтів ніхто вже не відбере, і
скуповували проте сусідні частки бідних, по части за їх згодою, а по части відбирали їм силою, — так, що замісць
поєдинчих  піль богачі оброблювали переважно обширні лани. Притім
вони уживали невольників до управи поля і годівлі худоби, бо свобідних
людей позабирано їм від праці до війська. Посіданє невольників приносило їм ще й тоту велику
користь, що невольники — вільні від військової служби — могли без перепони множитися і плодили
богато дітей. Таким способом постягали маґнати всі богацтва до себе, і цілі околиці вкриті були
невольниками. А правдивих Італьців ставало між тим усе меньше, — їх руйнували: бідність, податки та
військова служба. А хоть часом і настав супокій, то вони зовсім не могли підпомочися, бо весь ґрунт
був у богацьких руках, котрі замісць свобідних людей воліли мати до праці невольників“. Сесь уступ
описує часи перед правом Ліцінія. Військова служба, котра так прудко прискорила руїну римських
плєбеїв, була також головним средством, при помочи котрого Карло Великий перемінив вольних німецьких
селян в кріпаків так швидко, мов петрушку в розсаднику зростив.
}. І справді, наслідки забору громадських ґрунтів і доконаного тим забором перевороту в рільництві далися
так прудко і прикро почути сільским робітникам, що, як сам Еден признає, між 1765 а 1780 плата їх
почала знижуватися до крайної границі і уряд мусів поповнювати єї датками запомоговими. „Плата їх“,
каже Еден, „не вистатчала вже зовсім для потреб житя“.
\par{}

\parcont{}  %% абзац починається на попередній сторінці
\index{ii}{0072}  %% посилання на сторінку оригінального видання
між авансованою заробітною платою і купівельною ціною, що її платить
останній споживач, повинна являти зиск з капіталу. Він розподіляється
між фабрикантом, гуртовим купцем і роздрібним торговцем, з того часу,
як вони розподілили між собою свої функції, а виконана робота лишилась
та сама, хоч здійснили її три особи й три гатунки капіталу
замість одного“.
% Примітку видалено (вона не взазить)
% \footnote*{
% Le commerce emploie un capital considérable qui paraît, au premier coup
% d’oeil, ne point faire partie de celui dont nous avons détaillé la marche. La valeur
% des draps accumulés dans les magasins du marchand-drapier semble d'abord tout-à-fait étrangère à
% cette partie de la production annuelle que le riche donne au
% pauvre comme salaire pour le faire travailler. Ce capital n’a fait cependant que
% remplacer celui dont nous avons parlé. Pour saisir avec clarté le progrès de la
% richesse, nous l’avons prise à sa création, et nous l’avons suivie jusqu’à sa consommation.
% Alors le capital employé dans la manufacture de draps, par exemple, nous a paru
% toujours le même; échangé contre le revenu du consommateur, il ne s’est partagé
% qu’en deux parties: l’une a servi de revenu au fabricant comme profit, l’autre a servi de revenu aux
% ouvriers comme salaire, tandis qu’ils fabriquaient de nouveau drap. —

% „Mais on trouva bientôt que, pour l’avantage de tous, il valait mieux que les
% diverses parties de ce capital se remplaçassent l’une l’autre, et que, si cent mille
% écus suffisaient à faire toute la circulation entre le fabricant et le consommateur,
% ces cent mille écus se partageassent également entre le fabricant, le marchand en
% gros et le marchand en détail. Le premier, avec le tiers seulement, fit le même
% ouvrage qu’il aurait fait avec la totalité, parcequ’au moment où sa fabrication était
% terminée, il trouvait le marchand acheteur beaucoup plus tôt qu’il n’aurait trouvé le consommateur.
% Le capital du marchand en gros se trouvait de son côté beaucoup
% plus tôt remplacé par celui du marchand en détail\dots{} La différence entre les sommes, des salaires
% avancés et le prix d’achat du dernier consommateur devait faire le profit des capitaux. Elle se
% répartit entre le fabricant, le marchand et le détaillant depuis qu’ils eurent divisé entre eux
% leurs fonctions, et l’ouvrage accompli fut le même quoiqu il eût employé trois personnes et trois
% fractions de capitaux, au lieu d’une.
% („Nouveaux Principes d’Economie Politique, Livre II, ch. VIII, éd. 1827. p. 138--140).
% }.

„Всі (торговці) посередньо сприяли продукції, бо вона має на меті
споживання, і тому її можна вважати за вивершену лише тоді, коли
вона подала спродуковану річ до розпорядження споживачеві“.
% Разом з попередньою
% \footnote*{
% „Tous concouraient indirectement à la production; car celle-ci, avant pour
% objet la consommation, ne peut être considérée comme accomplie que quand elle a
% mis la chose produite à la portée du consommateur“, (lb., p. 137).
% }.

Розглядаючи загальні форми кругобігу, і взагалі в усій цій другій
книзі, ми беремо гроші як металеві гроші й лишаємо осторонь
символічні гроші — звичайні знаки вартости, що є лише виключно приналежність
деяких держав, а також кредитові гроші, які ще не розвинулись.
Це, поперше, відповідає історичному розвиткові; кредитові гроші
не відіграють жодної ролі, або лише незначну ролю, в першу добу капіталістичної
продукції. Подруге, доконечність такого порядку дослідження
обґрунтовується також теоретично тим, що всі критичні досліди над
циркуляцією кредитових грошей, що їх маємо з боку Тука й інших,
примушували їх завжди повертатись до розгляду того, як стояли б справи
на основі чистої металевої циркуляції. Але не треба забувати, що металеві
гроші можуть так само правити за купівельний засіб, як і за виплатний
засіб. Дбаючи про спрощення, ми взагалі в цій II книзі беремо їх лише
в першій функціональній формі.

\parcont{}
\index{franko}{0073}
цілі німецькі князівства), а також окремою формою ґрунтової власности, котру так насильно перемінюют
в приватну власність. Ті ґрунти, то була власність повіту (clan), — начальник або „великий чоловік“
був тілько титулярним властивцем, як представник повіту, так само, як королева англійська є
титулярною властителькою всего ґрунту Англії. Тот переворот, котрий в Шотляндії почався по посліднім
повстаню претендента, мож слідити в перших єго початках у письмах Джемса Стеєрта і Джемса Андерсона\footnote{
Стеєрт каже: „Рента в тих околицях (він хибно називає рентою тоту оплату, яку обивателі повіту
(taskmen) складали начальникови повіту) зовсім незначна в стосунку до обширности піль, але що до
числа осіб, котрих удержує одна аренда, мож сміло твердити, що оден кусник ґрунту в шотлянських
горах виживлює десять раз більше людей, ніж так само заобширний ґрунт в найбогатших рівнинах“.
}. В 18. віці заборонено притім Ґелям, прогнаним з ґрунтів, виселюватись в чужі краї, щоб їх таким
способом силою попхнути до Ґлязґова і других фабричних міст\footnote{
1860 вивожено тих насильно вивласнених хліборобів до Канади, отуманивши їх фальшивими
обіцянками. Деякі повтікали в гори і на сусідні пусті острови. Поліція пустилася за ними в погоню,
прийшло до бійки і втікачі здужали вирватися та порозбігатись.
}. За примір
методи пануючої в девятнайцятім віці\footnote{
„В шотляндських горах“, каже Бюкенен, коментатор А. Сміта, 1814, „день в день насильно затираєся
давний власностевий порядок\dots Сільский льорд, без огляду на дідичних арендаторів (знов хибно
названі тексмени) винаймає ґрунт тому, хто найбільше платит, а коли той належит до меліораторів
(imprower), то зараз заводит новий спосіб управи поля. Ґрунт, давнійше покритий дрібними
властивцями, був в стосунку до своєї плодовитости досить заселений; при новім сістемі поліпшеної
управи і побільшеної ренти одержуєсь як мож найбільше плодів як мож найменьшим коштом, і для таго
віддалюются робітники, котрі стали тепер непотрібними. Ті вигнанці з рідних хат шукают відтак
утриманя в фабричних містах і т. д. (David Buchanan: „Observations on A. Smith’s Wealth of Nations.
Edinb. 1814“.) „Шотляндські маґнати вивласнили цілі родини, немов хопту випололи: вони так обійшлися
з селами й людністю, як Інди розїдлі пімстою з дикими звірями по норах\dots Чоловіка продают
за овече руно, за волове стегно, ба ні, ще за меньшу дрібницю\dots Підчас нападу на північні
провінції Хіни була на раді Монголів така думка, щоб усіх мешканців витратити, а їх край перемінити
в степ. Тоту раду богато північно-шотляндських маґнатів дословно виповнили в своїм власнім краю і на
своїх власних земляках“. (Джордж Ензер: „An Inquiry concerning the Population of Nations. Lond.
1818“. Стор. 215, 216.
} досить буде ту навести „обчищуваня“ герцоґині Созерлєнд.
Тота в економії вишколена особа постановила зараз в початку свого панованя взятися до радікального
ліку економічного, і ціле ґрафство, в котрім задля давнійших подібних процесів осталось
\index{franko}{0074}
було всего лиш 15000 люда, перемінити в толоку для овець. Від 1814 до 1820 сістематично
прогонювано та нищено тих 15000 мешканців, т. є. майже 3000 родин. Всі їх села поруйновано і
попалено, всі їх поля пороблено толоками. Англійських жовнірів викомендерувано там для еґзекуції, і
між ними а мешканцями прийшло до бійки. Одна
стара баба згоріла враз іс хатою, с котрої не хтіла вступитися. І таким способом присвоїла собі
вельможна герцоґиня 794000 екрів ґрунту, котрі споконвіку належали до
повіту. Вигнаним мешканцям визначила вона на морськім узберіжю около 6000 екрів, по 2 екри на
родину. Тих 6000 екрів лежали доси пусто і не давали властительці ніякого
доходу. Герцоґиня так далеко зайшла в своїй щедрости, що винаймила екр пересічно по 2 шілінґи 6
пенсів для тих самих селян, котрі много сот літ проливали кров свою за
вельможну герцоґську родину. Увесь зрабований ґрунт повіту поділила герцоґиня на 29 великих аренд
для випасаня овець; в кождій аренді осіла тілько одна родина, переважно англійські наємні
арендаторі. 1825 р. замісць 15000 Ґелів на їх ґрунтах жило вже 181000 овець. А родини, вивержені на
морський беріг, старалися жити риболовством. З них поробилися земноводяні, і вони жили, як каже
писатель, на половину в воді, а на половину на березі, тілько що ні ту ні там не могли найти
достаточного прожитку\footnote{
Коли теперішна герцоґиня Созерлєнд витала в Льондоні з великою парадою міссіс Бічер Стоу,
авторку „Хати дядька Томи“, щоб виставити на показ свою прихильність для муринів-невольників в
американській републіці — чого вона і єї співарістократки певно не булиб зробили підчас домашної
війни американської, бо тоді кожде „шляхотне“ англійське серце було прихильне плянтаторам — в той
сам час описав
я в газеті „New-York-Tribune“ побут невольників созерлєндських. (Деякі місця тої статі навів Керей в
своїй „The Slave Trade. London 1853“.). Мою статю перепечатала одна шотляндська ґазета і викликала
дуже чемну перепалку між тою ґазетою а підхлібниками та похвальками герцоґів Созерлєндів.
}.

Але небораки Ґелі мусіли ще раз відпокутувати свою романтичну наклінність для „великих мужів“, т. є.
для начальників повітових (Сlanchef). Запах риб, котрими прокормлювались земноводяні Ґелі, ударив
великим мужам в ніс. Вони завітрили тут щось зисковного і заарендували морське узберіжє великим
льондонським гендлярам риб. Ґелів другий раз вигнано на штири вітри\footnote{
Цікаву історію того рибного торгу найде читатель у д. Девіда Оркуарта в єго книжці: „Portofolio.
New Series“. Сеніор водній іс своїх посмертних статей називає „процедуру в Созерлєндшайрі“ одним з
найблагодатнійших очищень від віків.
}.

Аж вкінци одну часть пасовиськ назад перемінено
\parbreak{}

\parcont{}  %% абзац починається на попередній сторінці
\index{i}{0075}  %% посилання на сторінку оригінального видання
(зглядно купівель) або частинних метаморфоз, що в них ті самі
монети лише один раз змінюють місце, або пророблюють лише
один обіг, а з другого боку — багато почасти паралельних,
почасти посплітуваних між собою більш-менш багаторозгалужених
рядів метаморфоз, що в них ті самі монети пророблюють
більш або менш значну кількість обігів. Однак загальне число
обігів усіх однойменних монет, що перебувають у циркуляції,
дає пересічне число обігів окремих монет, або пересічну швидкість
грошового обігу. Маса грошей, що їх на початку, приміром,
денного процесу циркуляції кидають у нього, визначається,
певна річ, сумою цін товарів, що циркулюють одночасно й просторово
один побіч одного. Але в межах процесу одна монета
стає, так би мовити, відповідальною за інші. Коли одна прискорює
швидкість свого обігу, то цим затримується швидкість обігу
іншої або остання й зовсім вилітає із сфери циркуляції, бо ця
сфера може поглинути лише таку масу золота, яка, помножена
на пересічне число обігів поодиноких її елементів, дорівнює сумі
цін, що мають бути зреалізовані. Тому, коли зростає число обігів
монет, то маса їх, що перебуває в циркуляції, меншає. Коли
число обігів монет меншає, то маса їх зростає. Через те, що за
даної пересічної швидкости обігу маса грошей, яка може функціонувати
як засіб циркуляції, є дана, то досить лише кинути
в циркуляцію, приміром, певну кількість однофунтових банкнот,
щоб витягти з неї рівно стільки саме золотих соверенів, — трюк
добре відомий усім банкам.

Як в обігу грошей взагалі виявляється лише процес циркуляції
товарів, тобто їхній кругобіг через протилежні метаморфози,
так у швидкості грошового обігу виявляється швидкість зміни
товарових форм, безупинне встрявання одного ряду метаморфоз
в інший, сквапність обміну речовин, швидке зникання товарів
зі сфери циркуляції й так само швидка заміна їх новими товарами.
Отже, у швидкості обігу грошей виявляється поточна єдність
протилежних фаз, що одна одну доповнюють, перетворення
споживної форми на форму вартости і зворотне перетворення
форми вартости на споживну форму, або єдність обох процесів,
продажу й купівлі. Навпаки, в загаянні грошового обігу виявляється
відокремлення й усамостійнення цих процесів як протилежностей,
застій переміни форм, а тому і обміну речовин. Звідки
постає цей застій, цього, певна річ, з самої циркуляції пізнати
не можна. Вона показує лише саме явище. Вульґарний погляд,
помічаючи, що з загаянням грошового обігу гроші не так часто
з’являються і зникають на всіх пунктах периферії циркуляції,
шукає пояснення цього явища в недостатній кількості засобів
циркуляції.\footnote{
«Через те, що гроші становлять... загальну міру купівель і продажів,
кожний, хто має щось на продаж, але не находить покупця, схиляється
до думки, що брак грошей у королівстві або країні є причина,
через яку він не може збути свої товари, і таким чином усі скаржаться на
«брак грошей»; але це велика помилка... Чого хочуть ті, які кричать,
}

\index{i}{0076}  %% посилання на сторінку оригінального видання
Отже, загальна кількість грошей, що функціонують протягом
даного періоду часу як засоби циркуляції, визначається, з одного
боку, сумою цін усіх товарів, що циркулюють, а з другого боку —
повільнішим або швидшим потоком їхніх протилежних процесів
циркуляції, від якого залежить, яку частину з тієї суми цін можна
зреалізувати за допомогою тих самих монет. Але сума цін товарів
залежить так від маси, як і від ціни кожного роду товару. Та ці
три фактори: рух цін, маса товарів, що циркулюють, і, нарешті,
швидкість обігу грошей можуть змінятися в різних напрямах
і в різних пропорціях; отже, сума цін, що має бути зреалізована,
а тому й зумовлювана нею маса засобів циркуляції, може таким
чином пророблювати численні комбінації. Ми зазначимо тут лише
ті, що найважливіші в історії товарових цін.

що немає грошей?... Фармер скаржиться... він думає, що коли б у країні
було більше грошей, він дістав би добру ціну за свої товари... Отже, він,
здається, потребує не грошей, а доброї ціни за своє збіжжя й за свою худобу,
що їх він хоче продати, але не може... Чому він не може одержати
доброї ціни?.. 1) Або тому, що в країні є забагато збіжжя або худоби,
так що більшість людей, що приходять на ринок, мають потребу продавати,
так само як він, і лише меншість має потребу купувати, 2) або
тому, що зменшився звичайний вивіз за кордон... 3) або тому, що падає
споживання, коли люди, приміром, через зубожіння, не можуть витрачати
на предмети споживання стільки, скільки витрачали раніш. Отже,
не збільшення кількости грошей допоможе фармерові продати свої продукти,
а усунення однієї з цих трьох причин, які дійсно натискають на
ринок... Так само потребують грошей купець і крамар, тобто вони не можуть
збути своїх товарів через застій на ринку... нація досягає найбільшого
розвитку тоді, коли багатства швидко переходять із рук до рук».
(«Money being... the common measure of buying and selling, every body
who has anything to sell, and cannot procure chapmen for it, is presently
apt to think, that want of money in the kingdom, or country, is the cause
why his goods do not do off; and so, want of money is the common cry;
which is a great mistake... What do these people want, who cry out for
money?.. The Farmer complains... he thinks that were more money in the
country, he should have a price for his goods... Then it seems money is not
his want, but a Price for his corn and cattle, which he would sell, but cannot...
why cannot he get a price?.. 1) Either there is too much corn and cattle
in the country, so that most who come to market have need of selling,
as he has, and few of buying: or, 2) There wants the usual vent abroad by
Transportation... Or, 3) The consumption fails, as when men, by reason of
poverty, do not spend so much in their houses as formerly they did, wherefore
it is not the increase of specifick money, which would at all advance
the farmer’s goods, but the removal of any of these three causes, which
do truly keep, down the market... The merchant and shopkeeper want money
in the same manner, that is, they want a vent for the goods they deal in,
by reason that the markets fail... a nation never thrives better, than when
riches are tost from hand to hand»). (Sir Dudley North: «Discourses upon
Trade», London 1691, p. 11—15 passim). Всі шахрайства Гереншванда
сходять на те, що суперечності, які виникають із природи товару й тому
виявляються в циркуляції товарів, можна усунути через збільшення
засобів циркуляції. З популярної ілюзії, яка застої в процесі продукції
і процесі циркуляції приписує бракові засобів циркуляції, зрештою,
ніяк не випливає зворотне, а саме, що дійсний брак засобів циркуляції
в наслідок, приміром, офіціяльних махінацій з «regulation of
currency»\footnote*{
— реґулювання засобів обігу. \emph{Ред.}
} не може із свого боку викликати застоїв.

\parcont{}
\index{franko}{0077}
а з беззглядною жорстокістю переведена переміна феодальної та окружної (Clan-)
власности в новійшу приватну власність, — ось які іділлічні були способи
первісного нагромадженя капіталу. Вони здобули ґрунт для капіталістичного
рільництва, втягли землю в обсяг капіталу, а міському промислови
достатчили потрібних „рук“, т. є. вольного і голого пролєтаріяту.

\subsection{Кроваві устави протів пролєтаріїв при кінци XV. віку.}

Вольний і голий пролєтаріят, вигнаний с хат і ґрунтів
через скасованє феодальних дворів і через насильне раз-заразом
вивласнюванє, не міг відразу перелятися весь до
новоповстаючих мануфактур так швидко, як швидко сам
повстав. А при тімже се були люде, викинені раптово с привичного
способу житя, — а такі люде не швидко можут
застосоватися до яких небудь нових, непривичних порядків.
На першій порі з них поробилися маси жебраків, розбійників,
волоцюг, — деякі з наклінности, а найбільша часть під гнетом обставин. С
кінцем XV. і підчас цілого XVI. віку бачимо проте в цілій Західній Европі
кроваві устави протів волоцюгів. Батьки нинішної робітницької верстви мусіли
на самім вступі відбути страшну кару, — за що? За то, що їх перемінено в волоцюг
та голоту. Праводавці вважали їх „добровільними переступцями“ і думали, що
тілько від їх доброї волі залежит — працювати далі серед давних обставин, котрі
між тим зо світа щезли.

В Англії почалось те праводавство під Генріхом VII.

Генріх VIII., 1530: Старі і неспосібні до праці жебраки одержуют дозвіл на
жебрацтво. За то здорові й міцні волоцюги карані будут батогами й арештом. Вони
мают бути привязані ззаду до тачок і бичовані доти, доки не поплине кров з їх
тіла, — відтак мусят зложити присягу, вернути на місце уродженя або там, де
пробули послідні 3 роки, і „засісти до праці“ (to put himself to labour). Що за
безсердечна насмішка! В 27 уст. Генріха VIII повторена попередна устава, але
заострена новими додатками. Як кого другий раз зловят на волоцюгованю, то такого
бичувати ще раз і відтяти му пів вуха. За третим разом непоправного волоцюгу,
як тяжкого злочинця і ворога суспільности — вкарати смертю.

Едуард VI.: Устава с першого року єго панованя 1547, наказує, що скоро хто
отягаєся від праці, той має бути присуджений на невольника тій особі, котра
донесла урядови о єго неробстві. Пан має годувати невольника хлібом і водою,
слабими напитками і такими обрізками мяса, які му видадутся відповідними. Він
має право всилувати го батогами \index{franko}{0078}
та зелізними ланцами до всякої, хотьби й як гидкої роботи. Коли невольник на 14
день віддалится, то зістає засуджений на віковічну неволю і має бути на чолі
або на лици напятнований буквою S, а коли до трох раз утече, то має бути
вкараний смертю, як зрадник держави. Пан може го продати, передати в наслідство,
визичити другому в неволю, зовсім так, як усяке друге рухоме добро, як худобу.
Коли невольники в чім небудь станут супротів панів, то мают також бути покарані
смертю. Мирові судьї повинні за отриманим остереженєм слідити за волоцюгами.
Коли покажеся, що такий волоцюга три дни волочився без діла, то такого
відставити на місце, де родився, роспеченим зелізом напятнувати на груди буквою
V і тамій в зелізних ланцюхах уживати до замітаня вулиці або до якої небудь
їншої служби. Коли волоцюга подасть фальшиво місце вродженя, то за кару має
бути віковічним невольником тої громади, тих мешканців або того товариства і
напятнований буквою S. Кождий має право відобрати у волоцюги єго діти і яко
помічників та термінаторів держати хлопців до 24, дівчат до 20 літ. Коли вони
втечут, то мают аж до тих літ бути невольниками майстра, а тому вільно їх
заковувати в ланци, бити і пр., як му сподобаєсь. Кождий пан може заложити
зелізну обручку на шию, руку або ногу свого невольника, щоби міг го ліпше
пізнати і бути певним, що му не втече\footnote{
Автор книжки „Essay on Trade and Commerce“ 1770, каже: „Під панованєм Едварда
VI. взялись були Англічане зовсім, здаєсь, серйозно до піддвигненя мануфактур і
затрудненя бідних. Се бачимо з одної дивовижної устави, в котрій приписуєсь, що
всі волоцюги мают бути пятновані, і т. д. (Essay on Trade and Commerce, стор.
8).
}. Послідна часть тої устави наказує, щоб
деяких бідних брали на себе громади або поєдинчі люде; ті мают їм давати їсти
й пити і старатись для них о роботу. Тот рід громадських невольників удержувався
в Англії гет ще в 19. віці під назвою roundsmen (люде, що ходят від хати до
хати).

Єлисавета, 1572: жебраки без дозволу і віком понад 14 літ мают бути без
милосердя бичовані і напятновані на лівім вусі, хіба що їх хто схоче взяти на
два роки на службу; в разі повтореня, коли мают над 18 літ, мают бути — смертю
карані, скоро їх ніхто не схоче взяти на два роки на службу; за третим разом
мают без милосердя як зрадники державні бути покарані смертю. Подібна також 18.
устава Єлисавети, розділ 13, і устава з р. 1597 \footnote{
Томас Морус каже в своїй „Утопії“: „Так то дієся, що оден захланний і неситий
ненаїсник, правдива чума нашої вітчини, може тисячі екрів ґрунту збити до купи
і обпалькувати, обгородити одним плотом, або силою та кривдою до того довести
єго властивців, що вони будут мусіли все спродувати. Сяким чи таким способом,
чи там гнись чи ломайся, він присилує їх забиратися, — бідні, прості, нещасливі
душі! Мужчини й женщини, чоловіки й жінки, сироти без батьків, удови, плачучі
матері с пеленковими дітьми, і вся челядь, убога добром, а богата
роботами, бо рільництво вимагає богато рук. І волочутся вони, кажу вам,
з знакомих, рідних місць, не находячи пристанівку. Якби при й не таких
обставинах, то моглиб бодай що то вторгувати за свій, хоть і не дуже
цінний, домашний спряток; але раптово повикидувані, мусят усе продавати
за песій гріш. А коли перебурлачат послідний свій гріш, то щож
тоді мают робити, як не красти, а відтак, боже добрий, по всій формі та
правді згинути на шибеници або пуститися на жебри. А й тоді ще їх
попрут до вязниць як волоцюгів, що-ді плентаются, а нічо не робят.
А що там судови до того, що їх ніхто не хоче взяти на роботу, хоть би
й як радо самі на ню напрошувались!“ І таких бідних утікачів, котрих
но словам Томаса Моруса присилувано до крадіжи, „за панованя Генріха
VIII., повішено 72000 великих та дрібних злодіїв“. (Ноllingshed, Dеscription
of England, т.~І, стор. 186). За часів Єлисавети „вішано волоцюгів
цілими рядами; а прецінь не було такого року, в котрім би на
однім або другім пляцу не повішено їх 300--400“ (Strype`s Annals, т. II),
Той сам Страйп свідчит, що в Соммерcетшайрі за оден рік повішено 40
люда, напятновано 35, бито батогами 37, а випущено 183 „непоправних
злочинців“. А такій, каже він, „те велике число оскаржених не становит
ще й пятої части всіх злочинців, дякувати недбальству мирових судів
і глупому милосердю народа“. Він додає: „Прочі англійські ґрафства
зовсім не стояли ліпше від Соммерсетшайра, а богато стояло в тім згляді
ще далеко гірше“.
}.

\index{franko}{0079}
Яков І: Кождий, хто ходит від села до села і жебрає,
узнаєсь волоцюгою. Мирові суді мают право засудити го на
прилюдне бичованє і за першим разом на 6 місяців, за
другим на 2 роки тюрми. Підчас сидженя в тюрмі мают
бути так часто і так богато бичовані, як се мировий судя
узнасть за добре\dots{} Непоправні і небеспечні волоцюги мают
бути на лівім плечи напятновані буквою R і заставлені до
робіт примусових, а як їх ще коли придиблют на жебранині,
то мают бути без милосердя і без сповіди повішені. Ті устави
(в рукоп. „уставі“), правосильні аж до перших літ 18. віку,
знесені зістали доперва 12. уст. Анни, розд. 23.

Подібні устави бачимо і в Франції, де в половині 17.
віку завязалось було ціле царство волоцюгів (truands) в Парижи.
Ще в початку панованя Людовіка XVI. (Указ з дня
13. липня 1777) кождий здорово збудований чоловік від 16
до 60 літ віку, скоро був без удержаня і не мав означеного
занятя, мав бути висланий на ґалєри. Подібні також: устава
Карля V. для Нідерляндів з 6. жовтня 1537, перший едікт
держав і міст голяндських з 19. марта 1614., оповіщенє Сполучених
провінцій з д. 25. червня 1649 і богато других.
Ось яким способом, — батогами, пятнованєм та тортурами
на підставі нелюдських, кровавих устав увігнано мужиків, \parbreak{}

\parcont{}
\index{franko}{0080}
насилу обрабованих з ґрунту, хат і майна, насилу пороблених
злодіями та волоцюгами, в ті тверді рами карности,
конечної при сістемі наємної праці.

IV.    Устави для знищеня робучої плати.

Не досить того, що знадоби продукції розділюются:
на однім боці сам капітал (в руках властивців богатирів),
а на другім боці сама праця, т. є. люде, котрі нічо не мают
на продаж крім своєї праці. Не досить ще присилувати
тих людей до того, щоб добровільно себе самих запродували.
В дальшім ході капіталістичної продукції виростає
вже верства робітників, котра з вихованя, традиції, привички
признає вимоги того способу продукована природними законами,
чимось таким, що й бути інакше не може. Впорядкованє
видосконаленого капіталістичного процесу продукційного
перемагає всі запори; ненастанне повставанє релятівного
перелюдненя\footnote*{
Звісно, що перелюдненєм звеся то, коли де-небудь є забагато
людей, т. є. властиво більш людей, ніж може вижити. А релятівне перелюдненє
значит, що тілько в певнім місци і серед певних обставин є для
певного діла забогато людей, так, що всі вони не можут приміститися,
і одна часть з них дармує. Кождий пійме, що вже сама проява такого
релятівного перелюдненя є знаком нездорових економічних обставин.
Між тим, як побачимо далі, ціла капіталістична продукція нерозлучно
звязана с релятівним перелюдненєм, котре змоглося в краях промислових
особливо від заведеня парових машин, через що мілійони рук робітницьких
стратили роботу (Прим. перев.).
} вдержує довіз робучих рук і попит
за працею, значит, і робучу плату на такій висоті, яка кориснійша
для підростаючого капіталу; німий примус економічних
обставин довершує панованя капіталіста над робітником.
Позаекономічна, беспосередна сила входит все ще
в уживанє, але вже лиш виїмково. При звичайнім ході діла
досить є — лишити робітника під властю „природних законів
продукції“, т. є. лишити го в залежности від капіталу,
витвореній і навіки забеспеченій самими вимінками
продукційними. Але сего не мож зробити в тій історичній
хвили, коли капітал і етична продукція інощо зароджуєсь.
Підростаюча буржоазія потребує і уживає власти державної,
щоб „реґулювати“ робучу плату, т. є. втискати єї в такі
границі, які найкориснійші для баришництва, продовжувати
день робучий і вдержувати самого робітника в „належитій“
степени залежности. Се також дуже важний причинок до
т. зв. первісного нагромадженя капіталу.

Верства наємних робітників, що повстала в послідній
половині 14. віку, становила тоді і в слідуючих столітях
тілько дуже незначну часть людности, котрої становище
\parbreak{}

\parcont{}
\index{franko}{0081}
притім міцно обезпечували самостійні ґаздівства по селах,
а цехові звязки по містах. По селах і містах не було великої
суспільної ріжниці між майстрами а робітниками.
Підчиненє праці під капітал було тілько формальне, т. є.
продукція сама не мала ще на собі окремої капіталістичної
ціхи. Попит за наємною працею змагався прото дуже швидко
за кождим нагромадженєм капіталу, — між тим рук готових
найматися до праці прибувало дуже поволи. Велика
часть витворів суспільних, що пізнійше стала фондом вбільшуючим
капітал, тоді переходила ще в руки робітника для
єго власного зужитку.

Праводавство про наємну працю, згори вже вицілене
на визискуванє робітника і в своїм розвитку йому завсігди
однаково неприхильне, почалося в Англії від виданя „Устави
робітницької“ (Statute of Labourers) Едвардом III., 1349.
Рівночасно видано в Франції Указ 1350 р. в імени короля
Жана. Англійські і французькі устави виходят рівнобіжно
і зовсім однакі що до змісту.

Устава робітницька зістала видана за про голосні наріканя
послів. „Давнійше“, каже наївно оден Торі, „жадали
бідні такої великої плати за роботу, що промисл і богацтво
були загрожені. Тепер плата така низька, що знов грозит
промисловії й богацтву і то може ще небеспечнійше ніж
тоді“. Установлено правну тарифу платну для міст і сіл,
за роботу (в рукоп. „робуту“) на дни й від штуки. Сільскі робітники
повинні винайматися на рік, міські „с прилюдного
торгу“. Під карою тюрми заборонено платити висшу плату
від означеної в уставі; а хто бере більшу плату, того кара
виносит більше, ніж сама плата. Так само ще в розд. 18
і 19. устави о учениках ремісницьких, виданої за Єлисавети,
грозится карою 10 день тюрми тому, хто платит більше,
а 21 день тюрми тому, хто бере більшу плату від правом
приписаної. Устава з р. 1360. заострила кари і навіть дала
майстрам право силувати робітників мусом до праці за таку
плату, яка означена в тарифі. Всякі звязки, угоди, присяги
і т. д., котрими взаїмно сполучилися теслі з мулярами,
узнані неважними. Стоваришеня робітницькі караются як
тяжка провина від 14. віку до 1825, в котрім скасовано
устави протів стоваришень. Дух „Робітницької устави“ з р.
1349 і її потомків просвічує ясно й с тих устав протів стоваришень.
Се тота сама засада: держава приписує, кілько
мож найбільше платити робітникови, але хрань боже, щоб
хоть натякнула на те, кілько мож йому найменьше платити!

В 16. віці, як звісно, положінє робітників дуже погіршилося.
Правда, грішми плачено більше, тількож що ціна
прошей стала меньша, а ціна товарів без міри більша. На
ділі затим і плана вменышилася. А прецінь устави для єї
зпиженя трівают далі порівно з обрізуванєм вух та пятнованєм
\index{franko}{0082}
тих, „котрих піхто не хоче взяти на службу. Єлисаветина
5 устава про учеників ремісницьких, уст. 3. надає
мировим судям власть становити де в яких реміслах плату
і змінювати ї відповідно до пори року і ціни товарів. Яков
I ростягнув ту саму реґуляцію робітницької плати на ткачів,
прядільників і на всі можливі розряди робітників\footnote{
З одної примітки до устави 2. за Якова І, розд. 6. видно, що
деякі суконники позваляли собі самі яко мирові суды урядово діктувати
платну тарифу в своїх варстатах. — В Німеччині, а іменно по 30-літній
війні, виходнт богато устав для знижуваня робучої плати. „Помічникам
на безлюдних ґрунтах дуже прикро давалась чути недостача слуг і робітників.
Всім мужикам-ґаздам заказано приймати в комірне мужчин та
женщин вільного стану; про всіх таких комірників повинно доноситися
урядови, а той запирає їх в тюрму, скоро не хотят стати слугами, хоть би
й без того мали яке їнше вдержанє, хоть би працювали у  мужиків за поденщину
або навіть торгували грішми та збіжєм. (Цісарські прівілєї та
ухвали для Шльонська, І, стор. 125). Через цілих сто літ роздаются в приписах
князів та поміщиків раз-відразу гіркі наріканя на злосливих
і здуфалих слуг, що не хотят піддатися важким условинам, не хотят вдоволюватися
платою правом приписаною. Виходят накази, щоб поєдинчпй
поміщик не смів своїм слугам платити більше, ніж кілько весь краєвий
збір покладе в таксу. А прецінь условини служби по війні нераз ще
бувают ліпші, ніж були 100 літ опісля. В р. 1052 діставали ще слуги на
Шльонську по два рази до тижня мясо; а ще в нашім столітю іменно
там були такі округи, де слуги діставали мясо хіба три рази до року.
І поденщина (плата за день роботи) по 30-літній війні була більша ніж
в слідуючих столітях“ (Ґустав Фрейтаг).
}, Джордж
II ростягнув устави протів робітницьких товариств на всі
мануфактури. В властивій порі мануфактуровій капіталістична
продукція була вже досить сильною, щоб правну
реґуляцію робучої плати зробити непотрібною, а то й неможливою,
але все такі ще на всякий злучай не закидувано
того перестарілого оружя. Ще 8. устава Джорджа II заказує
давати кравецьким челядникам в Льондоні і околици більше
понад 2 шіллінґи і півосьма пенса денної плати, окрім хіба
в разах загальної жалоби. ІЦе 13 уст. Джорджа III, розд.
68. повіряє мировим судям реґульованє робучої плати у виробників
шовку. Ще 1796 тре було двох декретів висших
судів для рішеня, чи накази мирових судьїв що до робучої
плати мают вагу і для нерільничих робітників. Ще 1799.
потвердила ухвала парляменту, що плата копальників шотландських
уреґульована уставою Єлисавети і двома шотляндськими
актами з р. 1661 і 1671. А який між тим переворот
доконався у всіх обставинах, доказала подія нечувана
в англійській палаті панів. Ту, де від звиш 400 літ
фабриковано устави виключно о тім, понад яку міру не
може ніяк переступити робуча плата, — ту поставив 1799
\parbreak{}

\parcont{}
\index{franko}{0083}
Уайтбрід внесок устави, яка може бути найменьша плата
для робітників рільничих\dots Хоть Пітт супротивлявся тому
внескови, то прецінь і сам признав, що „положінє вбогих
страшенне (cruel)“. Вкінци 1813 скасовано устави про реґуляцію
плати. Вони стались смішним недоріцтвом, відколи
капіталіст порядив у своїй фабриці після власних приватних
прав, а плата рільничого робітника давно впала понизше
мінімум конечного до прожитку, і мусіла до висоти
того мінімум доповнюватися с „податку на бідних“. Постанови
„Устави робітницької“ що до згоди між майстром
а наємним робітником, що до вимовленя терміну і т. д.,
постанови дозволяючі тілько цівільну скаргу па недодержуючого
умови майстра, а крімінальну скаргу на недодержуючого
умови робітника, — ті постанови стоят ще й доси
в повній силі. Нелюдські ухвали супротів стоваришень
скасовано 1825 з ляку перед грізною поставою пролєтаріяту.
Парлямепт зніс їх дуже нерадо\footnote{
Деякі останки устави протів стоваришень знесено аж 1859 р.
(Додаток до 2. вид.) Устава з 29. жовтня 1871. зносит всі устави
протів стоваришень і урядово признає „Робучі Звязки“ (Trades Unions).
Але в однім додатковім акті с того самого дня, п. н. „An Act to amend
the Criminal Law relating to violence, threats and molestation” — устави
протів стоваришень щасливо воскресли в новій формі. Сесь акт піддає
іменно робітників за вживанє деяких средств воєнних протів майстрів
під окремі устави крімінальні, а судят робітників на підставі тих устав
самі ж майстри, яко мирові судьї. Два роки перед тим та сама палата
послів і тот сам Ґлядстон. що 1871 винайшли нові проступкп на робітників,
вихвалювали при другім єго читанню один внесок до устави, в котрім
чесним способом роблено конець всяким окремим праводавствам
протів робітників. Вихвалювали, вихвалювали, тай хитро-мудро стали на
другім читанню. (Звісно, що в Англійськім парляменті кождий внесок,
заким одержит силу права, мусит бути три рази читаний і більшістю голосів
принятий. Прим, перев.) Цілі два роки відволікано сю справу, аж
поки „велике ліберальне сторонництво“ не звязалось зі своїми противниками
і не почулося задосить сильним, щоб разом стати — протів спільного
ворога — робітників.
},  той сам парлямент, що
сам довгі столітя с цинічним безвстидством виступав як
неустаюче стоваришенє капіталістів супроті робітників.

Сейчас в початках революційної бурі поквапилась французька
буржоазія інощо здобуте право стоваришень знов
видерти робітникам. В декреті с 14. червня 1791 оголосила
вона, що всі робітницькі стоваришеня, се „замах на свободу
і признані права чоловіка“, за котрий накладаєсь кара
500 ліврів і позбавлене па рік актівних прав горожанських.
Се право, котре конкуренційну боротьбу між капіталом
а працею силою поліційно-державною втискає в такі границі,
які вигідні для капіталу, перетрівало революції та зміни
\parbreak{}

\parcont{}  %% абзац починається на попередній сторінці
\index{iii2}{0084}  %% посилання на сторінку оригінального видання
«і тому цією операцією ви мусити порушити вексельний курс, бо закордонний
борг не оплачено в наслідок того, що ваш експорт не має відповідного імпорту.
— Це правило для всіх країн взагалі».

Лекція Вілсона сходить на те, що всякий експорт без відповідного імпорту
становить одночасно імпорт без відповідного експорту; бо в продукцію товарів,
що їх експортують, ввіходять чужоземні, отже, імпортовані товари. Перед
нами припущення, що всякий такий експорт ґрунтується на неоплаченому
імпорті або породжує його, — отже, породжує борг закордонові, або ґрунтується
на ньому. Це — помилкова річ, навіть, коли не вважати на ті дві обставини,
що 1)~Англія має даремний імпорт, не платячи за нього жодного еквівалента;
напр., частину свого індійського імпорту. Індійський імпорт вона може обмінювати
на американський імпорт, експортуючи останній без еквівалентного імпорту;
щож до вартости, то в усякім разі Англія експортувала тільки те, що їй нічого
не коштувало; 2)~Англія може й оплатила імпорт, напр., американський, що
утворює додатковий капітал; коли вона той імпорт споживає непродуктивно,
напр., на військові припаси, то це не утворює боргу проти Америки та не
впливає на вексельний курс з Америкою. Newmarch суперечить сам собі в
посвідченнях 1934 та 1935, й Wood звертає його увагу на це в 1938: «Коли
жодна частина товарів, ужитих на виготовлення речей, що їх ми вивозимо без
зворотного припливу» [військові видатки] «не походить з тієї країни, куди ці
речі експортуються, то яким способом це впливатиме на вексельний курс з цією
країною? Нехай торговля з Турцією перебуває у звичайному стані рівноваги;
яким способом вивіз військових припасів до Криму вплине на вексельний курс
між Англією та Турцією?» — Тут Newmarch втрачає свою рівновагу, забуваючи,
що саме на це просте питання він дав уже слушну відповідь під № 1934, він
каже: «Ми вже, мені здасться, вичерпали практичне питання, а тепер увіходимо
в дуже високу ділянку метафізичної дискусії».

[Вілсон має ще й інше формулювання того свого твердження, що на вексельний
курс впливає всяке перенесення капіталу з однієї країни до іншої, однаково,
чи відбувається воно у формі благородного металу, чи у формі товарів.
Вілсон, природно, знає, що на вексельний курс впливає рівень проценту, а
саме, відношення чинних норм проценту в тих двох країнах, що їхній взаємний
вексельний курс розглядається. Отже, коли він буде в стані довести, що надмір
капіталу взагалі, отже, передусім надмір товарів всякого роду, в тім і благородного
металу, має разом з іншими обставинами вплив на рівень проценту, визначаючи
його, то він буде уже на крок ближче до своєї мети; перенесення
значної частини цього капіталу з однієї країни до іншої мусить змінити рівень
проценту в обох країнах, і то саме в протилежному напрямку а тому другою
чергою мусить воно змінити й вексельний курс між обома країнами. — \emph{Ф.~Е.}].

В Economist’і, що його він тоді редаґував, за рік 1847, на стор. 475,
він пише:

1)~«Очевидно, що такий надмір капіталу, який виявляється у великих запасах
всякого роду, в тім і благородного металу, неминуче мусить привести не тільки
до низьких цін на товари взагалі, але й до нижчого рівня проценту за ужиток
капіталу.

2)~Коли ми маємо запас товарів, достатній для того, щоб обслужити
потреби країни протягом двох наступних років, то порядкування цими
товарами протягом даного періоду можна здобути за далеко нижчу норму, ніж
тоді, коли того запасу вистачить ледви чи на два місяці.

3)~Всякі позики грошей,
хоч і в якій формі їх робитиметься, являють лише передачу порядкування над
товарами від однієї особи до іншої. Тому, коли товарів є понад міру, грошовий
процент мусить бути низький, а коли товарів обмаль, він мусить бути високий.

\parcont{}
\index{franko}{0085}
капітал через ужитє наємних робітників і одну часть надвишки витворів, грішми чи натурою, платят
дідичови яко ренту ґрунтову. Доки в 15. віці незалежний мужик, а також сільский наймит, що попри
наймитство й сам про себе веде ґосподарство, збогачуются самі власною працею, доти й обставини тай
обсяг продукційний арендатора остаются дуже скромні. Переворот в рільництві, що почався в послідній
третині 15. віку і трівав через цілий 16 вік крім єго послідних десятиліть, збогатив го майже так
само прудко, як прудко зубожив мужиків\footnote{
„Арендаторі“, каже Гаррізен в своїй „Description of England“, „котрим давнійше годі було
заплатити 4 ф. шт. ренти, платят тепер по 40, 50 та 100 ф. шт. і ще кажут, що їм зле повелося, коли
по упливі арендового контракту не зложили бодай тілько готівки, кілько виносит 6--7-милітна рента“.
}. Загарбанє громадських пасовиск і т. д. дозволяє му
богато побільшувати число худоби майже без ніяких видатків, а між тим худоба достатчувала му далеко
більше обірнику для поправи ґрунту. В 16. віці причинюєсь ще одна рішучо важна обставина. Тоді
арендові контракти були довгі, нераз де з на 99 літ. А ту в 16. віці вартість золота та срібла, а
разом з ним і вартість грошей раз-ураз вменьшуєсь, і арендаторам се принесло золоті плоди. Не
зважаючи на прочі, вперед згадані обставини, арендаторі першим ділом вменьшили робочу плату. Те, що
урвано робітникам на платі, побільшувало
арендовий зиск. А з другого боку ціна збіжя, вовни, мяса, — одим словом, всіх плодів рільничих,
раз-ураз вбільшуєсь, через що змагаєся грошевий капітал арендатора
без єго причинку, — а притім ще ренту ґрунтову дідичови платит він давними, стратившими на вартости,
грішми. Таким способом він збогачуєсь рівночасно на кошт своїх наймитів і свого дідича. Не диво (в
рукоп. „даво“) затим, що вже с кінцем 16. віку витворилась в Англії окрема верства як на тодішні
обставини богатих „капіталістичних“ арендаторів\footnote{
В Франції з „Regisseur-ів“, т. є. панських окономів та тивунів середновікових поробилися швидко
т. зв. hommes d'affaires, т. є. люде, що туманництвом та шахрайством подороблялися капіталів. Такі
окономи, то були нераз великі пани. Як в Англії, так і в Франції великі феодалні добра поділені були
на богато дрібних ґосподарств, але з условинами далеко гіршими для мужиків. В~14. віці повстают і ту
аренди, звані ту „fermes“ або „terriers“. Число їх раз-ураз змагалося і дійшло гет понад
100000. Вони платили чи то грішми чи натурою ренту ґрунтову, котра виносила від 12-тої до 5-тої
части річного здобутку. Ті terriers були цілими або частковими леннами як до вартости і обєму
ґрунтів, котрі нераз виносили заледво кілька прутів. Всі арендаторі мали до певної степені (степенів
було штири) власть судову над мужиками, жиючими
на їх ґрунтах. Лехко поняти, якого притиску мусів дізнавати люд від
усіх тих дрібних тиранів. Монтейль каже, що тоді було в Франції 160000
судів, де тепер вистарчає (враз із мировими судами) 4000 трибуналів.
}.
\index{franko}{0086}
\subsection{Вліянє рільничого перевороту па промисл.
Промисловий капітал здобуває собі в краю ринок
відбутовий}

Раптове і частими нападами повторюване вивласнюванє
та прогонюванє мужиків достатчило, як ми бачили,
міському промислови раз-заразом маси пролєтаріїв, не належачих
зовсім до ніяких цехових звязків, — дуже мудра
подія, про котру старший Андерзен (не треба го мішати
з Джемсом Андерзеном) в своїй історії торговлі каже, що
се прямо боже провидініє так зробило. Ще хвилю мусимо
задержатися над тим складником первісного нагромадженя
капіталів. Не тілько що по селах убуло незалежного, самоґосподаруючого
мужицтва, а по містах прибуло промислового
пролєтаріяту, так, як після Жаффроа Сент-Улєра світової
матерії в одних місцях убуває, між тим коли в других
місцях вона згущаєсь. Помимо меньшого числа оброблюючих
рук ґрунт видавав прото однако або й ще більше
плодів, бо разом с переворотом в ґрунтових відносинах
власностевих настали також ліпші способи управи, більша
кооперація, зосередженє средств продукційних і т. д., а з другого
сільські наємники не тілько силувані були до тяжшої
праці — на се головно напирає сер Джемс Стеарт, —
а й обсяг їх домашної продукції, де вони працювали самі
на себе, чим раз більше вменьшувався. З освободженєм
одної части мужицтва освободжені зістали також єго давні
средства прожитку. Вони стают тепер матеріяльним складником
\so{змінного} капіталу\footnote*{
Звісно, що Маркс ділит капітал на постійний (constant) і змінний
(variabel), а то після того, чи в довшім протягу продукції вартість
єго зміняєся, чи ні. І так машини, сирий матеріял, будинки фабричні
і т. д., се капітал постійний, бо продукція не змінює в загальній сумі
єго вартости, а то, що убуде вартости на машинах і приладах і пр.,
котрі зуживаются при роботі, прибуває самим витворам, котрі через переробку
зискуют на вартости. Між тим друга часть капіталу, а іменно
тота, котра йде на наймленє і удержанє робітника і містится в понятю
робучої плати, се капітал змінний, бо по кождім процесі продукційнім
капіталіст добуває з него більше, ніж видав. Робітник витворює вартість
більшу, ніж тота, яку одержав в формі робучої плати. (Прим. перев.)
}. Бездомний та немаючий мужик
мусит окупувати собі ті средства прожитку від свого
нового пана, промислового капіталіста, в формі робучої
плати. Як зі средствами прожитку, так само сталося й з домашним
рільничим сирим матеріялом, котрого переробкою
займався промисл. Той сирий матеріял став частиною \so{постійного}
\index{franko}{0087}
капіталу. Се бачимо не тілько в Англії. За часів
Фрідріха II. бачимо н. пр., що часть вестфальських мужиків,
котрі всі прядут лен, — хоть ще не шовк, — насилу
вивласнено і прогнано з хат і ґрунтів, а прочу часть перемінено
в наймитів великих арендаторів. Рівночасно повстают
великі прядильні і ткальні льну, де „освободжені“ наймаются
на роботу. Лен виглядає так само, як виглядав уперед.
Ані одно волоконце в нім не змінилося, але нова соціяльна
душа вступила в єго тіло. Тепер він становит часть постійного
капіталу панів мануфактуристів. Давнійше розділений
між множество дрібних витвірців, котрі го самі управляли
і пряли, він тепер згромадився в руках одного капіталіста,
котрий других заставляв для себе прясти і ткати. Виложена
в прядильни надвишка праці становила давнійше надвишку
доходу незлічених родин мужицьких, або також — за часів
Фрідріха II, йшла на extra-податки pour le roi de Prusse.
Тепер вона становит зиск немногих капіталістів. Веретена
і ткацькі станки, давнійше розсіяні широко по краю, тепер
стовпилися в кількох великих касарнях робучих, так само
й робітники, так само й сирий матеріял. І веретена і ткацькі
станки і сирі матеріяли зі средств незалежного прожитку
для прядильників і ткачів від тепер переміеюются в средства
командованя над ними і висисаня з них бесплатної
праці. По великих мануфактурах не видно того так, як по
\linebreak[4]
\makebox[\linewidth]{\dotfill}

\begin{center}
\emph{[На цьому уривається збережений рукопис Франка]}
\end{center}

  \cleardoublepage
  \defaultfontfeatures{ 
  Path = fonts/ ,
  Scale=MatchLowercase,
}

% 
% Text fonts
%

\setmainfont{Alegreya}[
  Extension=.otf,
  ItalicFont=*-Italic,
  BoldFont=*-Bold,
  BoldItalicFont=*-BoldItalic,
  BoldFeatures={SmallCapsFont=*SC-Bold},
  SmallCapsFont=*SC-Regular,
  SmallCapsFeatures={%
    LetterSpace=10,
    WordSpace=2.5
  },
  SlantedFont = Alegreya,
  SlantedFeatures={FakeSlant=0.13},
  Scale = 0.93925
]
\setsansfont{AlegreyaSans}[
  Extension=.otf,
  UprightFont=*-Regular,
  ItalicFont=*-Italic,
  BoldFont=*-Bold,
  BoldFeatures={SmallCapsFont=*SC-Bold},
  BoldItalicFont=*-BoldItalic,
  SmallCapsFont=*SC-Regular,
  SmallCapsFeatures={%
    LetterSpace=10,
    WordSpace=3
  },
]
\newfontfamily{\letterspacefont}{AlegreyaSans}[
  Extension=.otf,
  UprightFont=*-ExtraBold,
  ItalicFont=*-Italic,
  BoldFont=*-Black,
  BoldItalicFont=*-BoldItalic,
  LetterSpace=15,
  WordSpace=4
]
\newfontfamily{\tablefont}{AlegreyaSans}[
  Extension=.otf,
  UprightFont=*-Regular,
  ItalicFont=*-Italic,
  BoldFont=*-Bold,
  BoldItalicFont=*-BoldItalic,
  Numbers={Monospaced,Lining}
]
\newfontfamily{\greekfont}{Alegreya}[
  Script=Latin,
  Extension=.otf,
]

% 
% Math fonts
%

\usepackage{unicode-math}

\setmathfont{STIX2Math}[% operators
  Extension = .otf ,
  StylisticSet = 01 ,
  Scale=MatchLowercase,
]

\setmathfont{Alegreya.otf}[% numbers
  range = {up},
  Script=Latin,
  script-features={},
  sscript-features={}
]

\setmathfont{Alegreya-Italic.otf}[% italic letters
  range = {it},
  Script=Latin,
  script-features={},
  sscript-features={}
]

%% Alllow cyrilic letters in math
\DeclareSymbolFont{cyrletters}{\encodingdefault}{\familydefault}{m}{it}
%% All letters
\newcommand{\makecyrmathletter}[1]{%
  \begingroup\lccode`a=#1\lowercase{\endgroup
  \Umathcode`a}="0 \csname symcyrletters\endcsname\space #1
}
\count255="409
\loop\ifnum\count255<"44F
  \advance\count255 by 1
  \makecyrmathletter{\count255}
\repeat

%% Fake slant fot г, д, п, т

\DeclareSymbolFont{cyrletterssl}{\encodingdefault}{\familydefault}{m}{sl}
\newcommand{\makecyrmathlettersl}[1]{%
  \begingroup\lccode`a=#1\lowercase{\endgroup
  \Umathcode`a}="0 \csname symcyrletterssl\endcsname\space #1
}
\makecyrmathlettersl{"433} % г
\makecyrmathlettersl{"434} % д
\makecyrmathlettersl{"43F} % п
\makecyrmathlettersl{"442} % т

\end{document}
