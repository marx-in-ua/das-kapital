\documentclass{kapital}
%% should be a class option
\renewcommand{\parbreak}{\unskip\ignorespaces}
\renewcommand{\parcont}{\unskip\ignorespaces}

%% proper quote marks 
\newunicodechar{„}{«}
\newunicodechar{“}{»}

%% ditto mark
\renewcommand{\dittomark}{~}

%% overfull boxes
\vfuzz=12pt
\hfuzz=1pt

\newcommand{\Year}{2023}
% \newcommand{\Year}{2024}
% \newcommand{\Year}{2025}

\newcommand{\City}{Київ}


\newcommand{\VolumeNumber}{Том І}
\newcommand{\BookNumber}{Книга перша}
\newcommand{\BookTitle}{Процес продукції капіталу}

\newcommand{\BookSource}{Переклад з четвертого німецького видання}
\newcommand{\BookAuthors}{за редакцією Д.~Рабіновича і~С.~Трикоза}
\newcommand{\SourceYear}{Харків, 1933}
\newcommand{\BookAddFirst}{}
\newcommand{\BookAddSecond}{}

\newcommand{\Biblio}{Т. І. — Кн. І: Процес продукції капіталу /
Пер. із 4 нім. вид.; За ред. Д. Рабіновича і~С.~Трикоза.}

\newcommand{\ISBN}{ISBN 000-000-0000-00-0 (том І)}

\newcommand{\FullSource}{\VolumeNumber. \BookNumber: \BookTitle. \BookSource{} \BookAuthors{} — \SourceYear{}}

\newcommand{\VolumeNumberDe}{Erster Band}
\newcommand{\BookNumberDe}{Buch I}
\newcommand{\BookTitleDe}{Der Produktionsprocess des Kapitals}
\newcommand{\AuflageDe}{Vierte, durchgesehene Auflage}
\newcommand{\AuflageDeSecond}{Herausgegeben von Friedrich Engels}
\newcommand{\HerausgegebenDe}{}
\newcommand{\PublisherDe}{Verlag von Otto Meissner}
\newcommand{\CityDe}{Hamburg}
\newcommand{\YearDe}{1890}

\begin{document}
  \input{common/front.tex}
  \nonumsection{Андрію Річицькому}{}{}

Присвячується Андрію Річицькому (справжнє ім'я — Пісоцький Анатолій Андрійович), видатному українському вченому, перекладачеві та політичному діячу. В 1920-х роках він почав працювати над першим і єдиним українським перекладом «Капіталу» Маркса з німецької, проте встиг завершити лише перший том. Репресований у 1934 році, після чого його ім’я не згадувалося у наступних виданнях «Капіталу» зовсім, попри те, що вони базувалися на його роботі. Був реабілітований у 1990. 

  \nonumsection{Чи застарів «застарілий» Маркс?}{~}{Іван Дзюба}

Почати з того, що Маркс застарівав уже не один раз. Спершу — ще після 
революцій 1848 року, які розвивалися не за логікою «Комуністичного 
маніфесту». Потім — після невдачі Паризької Комуни. Далі — коли його 
відмодельовували в протилежні боки «ревізіоністи» (від Бернштейна до 
Каутського) і Ленін, більшовики, Сталін\ldots{} А скільки в нього 
застарілих окремих формул і тез! Наприклад, колись знамените про 
«ідіотизм сільського життя». Це ж як воно звучить тепер, коли світ 
потерпає від набагато глибшого і страшнішого ідіотизму мегаполісів?!


Про те, що Карл Маркс застарів, знає у нас навіть той, хто взагалі 
нічого не знає (особливо він). Але, хоч як дивно, всупереч нашому знанню 
і незнанню, у західних університетах його праці поважно вивчають 
(звісно, з певної критичної позиції), про нього пишуть видатні 
соціологи й філософи як про одного з великих мислителів людства, його 
перевидають і шукають у нього стежок до пояснення економічних криз 
сучасного світу тощо, — а 5-го травня цього року широко відзначалося 
його 200-річчя. Але все це — «за бугром». У нас же будь-який 
малоосвічений публіцист може при нагоді поглузувати з «двох 
німецьких гномів» (це довелося зустріти недавно в інтернетному 
дописі). Таке от бачення (власне, небачення) історії, такий рівень 
культури мислення, таке розуміння динаміки інтелектуального розвитку 
людства, за якої насправді нові осягнення виростають із 
«застарілого», а заперечуване відходить, тільки стимулювавши саме 
заперечення, або й повертається невпізнане.


Ще одна біда — коли говорити про «масову людину» — брак історичного 
підходу в поцінуванні культурних явищ та феноменів думки, понятійних 
категорій. Багато хто в простоті душевній гадає, що це Маркс придумав 
класи, класову боротьбу, пролетаріат, революції та інший клопіт, отож 
усі біди від нього, Маркса. Таким чином на нього ніби падає 
відповідальність за століття (або й тисячоліття) соціальних 
конфліктів і майнових битв на нашій планеті. Хай так. Але на 
виправдання Маркса можна сказати, що в нього було багато попередників. 
Не будемо зазирати в біблійні, або античні, чи й середньовічні часи, а 
звернімося до ХIХ ст., в якому й визрівало те, що згодом дістало назву 
марксизму.

\looseness=1
Отож: у межах європейських феодальних монархій народжується і набирає 
сил буржуазія, що здобуває економічні й політичні позиції, 
використовуючи суперечності між монархом і його васалами та свої 
фінансово-майнові важелі; відбувається нагромадження капіталу, 
розвивається фабрична промисловість, змінюється характер виробничих 
і суспільних відносин та способи експлуатації робітника, колишні 
дрібні власники й обезземелені селяни стають знекоріненою і 
безправною «робочою силою», зростає безробіття. Пролетаризація 
охоплює цілі суспільні верстви, збільшуючи зубожіння люду й набираючи 
катастрофічного характеру. Як відповідь на біди, що їх приніс масі 
населення, насамперед трудовому людові, бурхливий розвиток 
жорстокого й хижацького капіталізму; як відповідь на загострення 
соціальних антагонізмів та масових злиднів — виникають, з одного 
боку, стихійні бунти, наприклад, луддитів та інших руйначів машин, 
потім і бунти та масові революційні рухи, а з другого — народжуються 
соціальні міфи й утопії та спроби окремих гуманістично мислячих, з 
чутливою соціальною совістю особистостей, як правило з освічених, 
упривілейованих станів, запропонувати моделі подолання кричущих 
суспільних дисгармоній і шляхи влаштування справедливих відносин, 
бодай у окремих локальних осередках, якщо не взагалі у світі. Так 
народжується утопічний соціалізм, яскрава плеяда теоретиків якого 
(К.~А.~Сен-Симон, Ш.~Фур'є, Р.~Оуен, а також Т.~Мюнцер, Т.~Кампанелла, Мореллі, 
Ж.~Мельє, Дж.~Уїнстлі, Г.~Б.~Маблі, Г.~Бабьйоф, Т.~Дезамі) за всієї відмінності 
поміж собою в конкретних позиціях і національному представництві 
були суголосними в критиці реального стану супільств як 
невідповідного поняттям про гідне життя, справедливість, моральність 
і доцільність, — а тому й неприйнятного для людського розуму й 
совісті. Їхні проекти ідеального суспільства базувалися на 
ідеалістичних уявленнях про рівність, свободу і братерство, про 
нібито добрі від природи моральні засади людини. На зміну приватній 
власності мало прийти велике колективне виробництво із справедливим 
розподілом і забезпеченням потреб кожного; в такому суспільстві буде 
подолано різницю між розумовою і фізичною працею, суперечність між 
містом і селом. (Мрія ця супроводжувала і ще супроводжуватиме чи не всю 
історію світу, вона позачасова!) Щоб прийти до такого суспільства 
справедливості, до здійснення цієї споконвічної мрії людства, треба 
було всього лиш переконати людність у його перевагах. Одначе ця проста 
і зрозуміла справа чомусь не вдавалася, як не вдавалися і спроби 
жертовних мрійників подати власний приклад організацією 
соціалістичних комун або фаланстерів. Побудувати комунізм чи 
соціалізм в окремо взятій громаді виявилося неможливим.


За цих умов і постає необхідність в іншому, неутопічному, 
реалістичному підході, обгрунтованому не моральною риторикою, а 
економічно, ідеологічно, політично, з орієнтацією на докорінну, 
найпевніше силову, перебудову всього суспільства. А на яку соціальну 
силу можна покладатися?


1842 року з'являється праця німецького вченого-юриста Лоренца фон 
Штайна (1815--1890) «Соціалізм і комунізм у сьогоднішній Франції». Це було 
за шість років до європейських революцій 1848-го і до появи 
«Комуністичного маніфесту» Маркса й Енгельса. Лоренц фон Штайн був 
одним із перших, хто розробляв теорію пролетаріату (невдовзі, 1845-го, 
з'являється праця Маркса і Енгельса «Свята родина», присвячена цій 
темі, але вже із розробленням стратегії дій революційного 
пролетаріату). Штайн показав, що клас пролетарів неминуче з'являється 
внаслідок появи і діяльності класу капіталістів. За вільноринкової 
економіки свободу і права мають власники, а не робітники. Учений-юрист 
ліберальних переконань, з лівих молодогегельянців, він гадав, що певні 
правові норми могли б допомогти пролетарям урівнятися з 
капіталістами і, таким чином, соціальної справедливості можна було б 
досягти без революції, шляхом реформ.

\subsection*{Молодий Маркс}

По-іншому розумів справу Карл Маркс. Він також вийшов із 
молодогегельянства, але швидко переріс його (праця Маркса і Енгельса 
«Німецька ідеологія», 1845--1846, містила розгорнуту критику ідеалізму 
Гегеля й непослідовності матеріалізму Феєрбаха). В його особі 
визначилися і рідкісно поєдналися філософ, ідеолог, соціолог, 
політичний діяч, журналіст-пропагандист, а згодом і економіст. Він уже 
був відомий як автор численних журналістських публікацій та наукових 
праць, присвячених обговоренню політичних і філософських проблем, 
полеміці з іншими теоретиками й ідеологами, аналізові тогочасного 
буржуазного суспільства. Досвід практичної роботи, широкий світогляд, 
філософська системність критичного мислення і потужний інтелект дали 
йому можливість узагальнити й переосмислити здобутки німецької 
філософії, французьких і англійських соціалістичних та комуністичних 
теорій, англійської політекономії (як відомо, Енгельс називав три 
джерела марксизму: німецька філософія, англійська політекономія, 
праці французьких істориків) — і прийти до принципово нових 
висновків. Вони чітко викладені в «Комуністичному маніфесті» 
авторства Маркса й Енгельса, по суті полемічному щодо утопічних або 
ретроградних ідей попередників. Не реформи, не регулювання ринку, не 
обмеження приватної власності на засоби виробництва, а повна їх 
націоналізація й одержавлення способів розподілу, що — уявлялося — 
зробить неможливою експлуатацію людини людиною, приведе до 
ліквідації класів та створення безкласового суспільства, в якому 
вільний розвиток кожного буде умовою вільного розвитку всіх. Для 
цього пролетаріат має взяти владу в свої руки революційним шляхом. Хоч 
є у Маркса й неоднозначні думки на цю тему. Революція відбудеться тоді, 
коли пролетаріат стане більшістю в суспільстві, але тоді він може 
прийти до влади й мирним шляхом. Зокрема, припускалося, що в країнах, де 
вже утвердився парламентський лад (Англія, США), пролетаріат може 
прийти до влади, перемігши на виборах. Саме на це згодом орієнтувалися 
соціал-демократичні партії II Інтернаціоналу, але які цього так і не 
дочекалися.


Картина майбутнього соціалістичного суспільства та шляхи його 
творення в «Комуністичному маніфесті» не обговорені скільки-небудь 
конкретно. Це була не так наукова праця, як 
політично-пропагандистський документ узагальнювального характеру. 
До речі, не слід забувати, що «Комуністичний маніфест» Маркс і Енгельс 
написали не з власного задуму, а на прохання міжнародної робітничої 
організації «Союз Комуністів». І точна його назва —  «Маніфест 
Комуністичної Партії». Тобто: вже існував досить організований 
робітничий рух, який потребував ідеологічного осмислення, і цю 
потребу мали задовольнити Маркс і Енгельс, вибір на яких упав, 
звичайно ж, не випадково. Але факт, що від самого початку не вони 
інспірували організований робітничий рух (у чому їх подеколи 
«звинувачували»), а робітничий рух їх «інспірував». Інша річ, що вони 
своїми ідеями надали нової якості й потужної енергії цьому рухові. 


«Маніфест Комуністичної Партії» завершував першу фазу діяльності 
«молодого» Маркса, в якій означилися основні його ідеї, що дістануть 
дальший розвиток, але вже й тоді своєю сукупністю були новим (хоч і не 
беззаперечним, і не беззаперечно новим у всьому) словом у науці й стали 
відомі під назвою історичний матеріалізм. Це, зокрема, погляд на 
історію людства крізь призму класової боротьби, соціальних 
антагонізмів, які і є рушієм розвитку (тут Маркс поглибив поняття 
класової боротьби, введене в обіг французькими істориками). Це 
твердження про неминучість революційних змін у суспільствах 
унаслідок суперечності між зростанням засобів виробництва й 
інерційністю суспільних відносин, боротьби між класом експлуататорів 
і класом експлуатованих. Це погляд на суму економічних відносин у 
суспільстві як на той базис, на якому виростає складна світоглядна, 
юридична, політична, ідеологічна, художня та ін. надбудова, що 
змінюється із зміною базису (теза, яка зазнавала і зазнає спростувань, 
почасти і через її профанацію вульгаризаторами марксизму: сам Маркс 
мав на увазі не пряму підпорядкованість надбудви базисові, а складну й 
багатоетапну опосередкованість зв'язку між базисом і надбудовою, хоча 
точних меж між одним і другим він не визначив, як і не наголосив 
зворотного впливу надбудови на базис). Далі, це важлива думка про те, що 
старий лад не відходить доти, доки не вичерпає своїх можливостей, а 
новий не приходить йому на зміну, доки не визріли передумови для нього. 
Навколо цих та інших Марксових ідей десятиліттями точилися суперечки 
між марксистами й антимарксистами, між ортодоксами й ревізіоністами, 
догматиками й реформаторами тощо.


Критики Маркса здебільше не охоплювали сукупності його поглядів та 
їхньої діалектики, їхньої часом вільної гри в концерті Марксових ідей. 
Так, один із непримиренних його негаторів, видатний мислитель ХХ ст. 
Арнольд Дж.~Тойнбі у «Дослідженні історії» писав: «Німецький єврей 
Карл Маркс намалював у барвах, які запозичив з 
апокаліптичних видінь відкинутої ним традиційної релігії, 
страхітливу картину відокремлення пролетаріату й класової війни, яку 
він розв'яже. Величезне враження, яке справив цей марксистський 
матеріалістичний апокаліпсис на стільки мільйонів умів, почасти 
пояснюється політичною войовничістю Марксової схеми, бо хоч вона й 
становить ядро загальної філософії історії, вона також являє собою 
революційний заклик до збройної боротьби» (Арнольд Дж.~Тойнбі. 
Дослідження історії. Т.~1. -- К., 1995. -- С.~362). 


Мусимо визнати, що ущиплива іронія Тойнбі стосується Марксової 
риторики чи метафорики, але не зачіпає суті, змісту його послання. Так 
само небагато дає і ревний пошук юдаїстських коренів у марксизмі. 
«Маркс поставив богиню «Історична Необхідність» на місце Єгови, а на 
місце євреїв, богообраного народу, — внутрішній пролетаріат 
західного світу. Його Мессіанське Царство — це диктатура 
Пролетаріату, але грандіозна будівля Єврейського Апокаліпсису легко 
вгадується під цим благеньким укриттям» (там само, с.~391). Безперечно! 
Але розпізнавання цієї метафорики не є спростуванням марксизму, бо ця 
метафорика давно вже стала складником європейського мислення, — хіба 
що за всіма ідеалами комунізму доведеться бачити проповіді Христа і 
зводити справу до цього. Не випадково ж існує християнський соціалізм, 
був християнський комунізм, який заперечувано ще в «Комуністичному 
маніфесті». 


Власне, Тойнбі іронізує фактично з «молодого» Маркса, часів до 
написання «Капіталу», і, як історик, бере до уваги насамперед його 
узагальнені історіософські моделі, що не вкладалися в циклопічну 
будову тойнбівского «Дослідження історії», яке охоплювало не одне 
тисячоліття і в масштабі якого марксизм міг здаватися епізодом.


\subsection*{Марксів «Капітал»}


\ldots{}«Молодий» Маркс був філософом, ідеологом, політичним публіцистом, 
але ще не економістом. «Зрілий» Маркс, критично опанувавши досягнення 
сучасної йому економічної науки, насамперед англійської, розпочинає 
фундаментальне дослідження капіталізму як формації, 
капіталістичного способу виробництва, — типологічно, за Марксовим 
визначенням, відмінного від азійського, античного й феодального 
розвитком продуктивних сил та способом експлуатації людини людиною 
(це дуже важлива частина Марксового вчення), — його очевидних та 
прихованих механізмів, його «таємниць» і перспектив та меж. Так 
з'являється перший том його «Капіталу» — праці, що справила 
величезний вплив на розвиток людської думки і на політичну історію 
людства. (Свій задум Маркс не встиг довести до кінця, і другий та третій 
томи «Капіталу» готував Енгельс з Марксових чернеток.)  


Маркс показав, що капіталізм — принципово новий історичний і 
економічний феномен: у тому сенсі, що для нього характерний не обмін 
товарів за допомогою грошей, як це було на докапіталістичних етапах 
історії людства, а обмін грошей за допомогою товарів. Через це метою 
капіталіста є грошовий прибуток, заради якого він готовий на все. А що 
є джерелом прибутку? Як створюється додаткова вартість? Це, сказати б, 
головна «таємниця» капіталізму, без розкриття якої не можна мати 
адекватного бачення його і не можна опрозорити його міфологію. Маркс 
зосереджується на цій «таємниці» і створює теорію вартості, теорію 
заробітної платні і теорію додаткової вартості — найбільшої 
«таємниці» капіталізму, що відтак перестає бути таємницею. 
Скрупульозний Марксів аналіз показує, що робочий день трудівника 
складається з двох частин — праці, яка повернеться в його зарплатню, і 
додану працю. Тобто, додаткова вартість — це неоплачена частина праці 
робітника. Праця робітника — товар, але дивовижний товар, єдиний 
товар, який виробляє вартість, вищу за власну вартість! Звідси — 
прибутки капіталіста, які тим більші, чим вище співвідношення між 
доданою вартістю і заробітною платнею. Це співвідношення є 
\emph{нормою}\emph{ }\emph{експлуатації}.


Можна, мабуть. сказати, що до Маркса категорія \emph{праці} виступала у 
суспільній свідомості (принаймні у вульгарно-матеріалістичному 
мисленні або в моралістичному) узагальнено, нерозчленованою: як 
джерело усякого багатства. На таке уявлення впливала не в останню 
чергу й протестантська трудова мораль. Певні ілюзії існували і в 
німецькому робітничому русі. Так, філософ-робітник Іосиф Дицген 
вважав, що праця — це Рятівник, удосконалення праці зробить те, чого не 
зміг досягти жоден Визволитель. Натомість Маркс не тільки показав, 
кому реально дістаються плоди праці, а й проаналізував економічні 
«складові» праці, її місце в процесі експлуатації робітника. 


Марксова демістифікація капіталізму, розкриття його механізму 
експлуатації мали не тільки наукове й політичне значення, але не в 
останню чергу й етичне, гуманістичне. Вони дали потужний поштовх 
робітничому революційному рухові спочатку в Європі, а потім і в усьому 
світі. Вони змінили світ. Зрештою змінили і самий капіталізм. І коли 
кажуть, що капіталізм давно вже не той, про який писав Маркс, то треба 
додати, шо став він «не тим» (хоч і не зовсім «не тим») завдяки зокрема й 
Марксу: капіталізмові нічого не залишалося, як змінитися під потужним 
тиском робітничого революційного руху, профспілкового руху, впливу на 
суспільства комуністичних і соціал-демократичних партій, — зрештою, 
і, мабуть, не в останню чергу, внаслідок власних внутрішніх 
суперечностей як джерела руху і завдяки невикористаним резервам, про 
можливість яких говорив Маркс (пригадаймо його тезу про те, що старий 
лад ніколи не сходить зі сцени, поки не вичерпає всіх своїх 
можливостей, певна річ, і здатності до змін).


Тут не буду говорити про те, як інтерпретували Маркса його 
послідовники (сам він якось саркастично сказав, що не хотів би бути 
марксистом), як розвивав марксизм В.~І.~Ленін і як на місці марксизму 
утворилося нове вчення — \emph{марксизм-ленінізм}. Це окрема велика тема. 
Але нагадаю про те, що в перше десятиліття радянської влади над 
вивченням Маркса й Енгельса трудилися спеціально створені солідні 
наукові інституції, які публікували свої праці, відбувалися дискусії 
тощо. В московському Інституті Маркса-Енгельса під керівництвом 
філософів-марксистів Д.~Рязанова та І.~Рубіна досліджували 
першоджерела, публікували невідомі твори. Тобто, в автентичному 
марксизмі бачили джерело ідей, що могли допомогти зрозуміти реальні 
суспільно-політичні процеси, орієнтуватися в будівництві нового 
суспільства. Ще жили такі ілюзії. Історична школа М.~Покровського з 
марксистських позицій гостро викривала російський імперіалізм. В 30-і 
роки, коли Сталін утвердив свій спрощений (ще набагато спрощеніший, 
ніж ленінський) варіант марксизму, всі ці структури ліквідовано, 
провідні вчені, дослідники й популяризатори Маркса були репресовані 
то як меншовики, то як троцькісти, а єдиним законним речником 
марксизму зробився сам Сталін.


Не менш цікаве й те, що коїлося з Марксом-Енгельсом і з марксизмом 
після розвалу СРСР у нашій самостійній Україні. Їхні твори опинилися у 
спецфондах. Посилатися на них — моветон, ознака «совковості», 
відсталості мислення й антипатріотизму. Та про це далі. А спочатку про 
те, яке місце посідав Маркс у політичній свідомості видатних 
українців минулого, чи мав він якусь «причетність» до визвольної 
боротьби українців?


\subsection*{Від Франка до української діаспори}

Дивно було б припускати, що Маркс, який став «душею» всіх 
комуністичних і соціалістичних рухів, залишиться «чужим» для України, 
яка шукала вирішення своїх національних проблем, що були водночас і 
соціальними. Марксом поважно цікавилися М.~Драгоманов, М.~Павлик, І.~Франко,
Леся Українка. Іван Франко 1879 року зробив перший український 
переклад частини «Капіталу» (фрагмент друкується у цьому виданні). 
Професор Київського університету Микола Зібер, видатний економіст і 
соціолог, перший в Україні й Росії популяризував ідеї Маркса. Він 
зустрічався з Марксом і Енгельсом у Лондоні. 1885 року опублікував працю 
«Д.~Рикардо и К.~Маркс в их общественно-экономических исследованиях», 
яку Маркс читав і прихильно цитував. Учень М.~Зібера Сергій 
Подолинський також зустрічався з Марксом і Енгельсом та листувався з 
ними; він був автором перших марксистських праць — популярних брошур 
— українською мовою, в яких застосовував Марксові ідеї до аналізу 
проблем українського селянства: «Про хліборобство» (1874), «Парова 
машина» (1875), «Про багатство та бідність» (1876), «Життя і здоров'я людей 
на Україні» (1879), «Ремесла і фабрики на Україні» (1880) та ін. Вони 
друкувалися, зрозуміло, в Галичині, але розповсюджувалися по всій 
Україні зусиллями Драгоманова, Павлика і київської «Громади». У 
Львові й Чернівцях 1892 року друкуються брошурами українські переклади 
з Маркса й Енгельса, а перший український переклад «Комуністичного 
маніфесту» виходить 1902 року у Львові. Цікавий етап у розповсюдженні 
марксистських ідей у Російській імперії — це розквіт т.~зв. 
«легального марксизму», найяскравіше представленого у Києві: В.~Кістяковський,
 С.~улгаков, М.~Ратнер, М.~Туган-Барановський (пізніше 
виступав з критикою Маркса). З «легальним марксизмом» уперто боровся 
Ленін, який бачив у ньому джерело ревізіонізму.


Якщо на перших порах популяризацією марксизму захоплювалися ліберали 
й народники, то з розвитком в Україні соціал-демократичного руху він 
стає елементом партійних програм. До марксизму апелювала створена 1905 
року на основі РУПу (Революційної Української Партії) — УСДРП 
(Українська Соціал-Демократична Робітнича Партія), визначними діячами 
якої були В.~Винниченко, С.~Петлюра, Д.~Антонович, Л.~Юркевич,
М.~Ковальський, М.~Тимченко та~ін. При цьому україноцентричні 
соціал-демократи звертаються до марксизму для висвітлення 
колоніального становища України та обстоювання ідеї національного й 
соціального визволення України. Видатним науковцем і політичним 
діячем цього гатунку був Микола Порш, один із чільних діячів РУПу та 
УСДРП, міністр в урядах УНР, соціолог і статистик, автор праць «Із 
статистики України» (1907), «Пролетаріат на Україні» (1907), «Про автономію 
України» (1907), «Автономія України і соціал-демократія» (1917), «Україна і 
Росія на робітничому ринку» (1918), «Україна в державному бюджеті Росії» 
(1918) та ін. Він же переклав українською мовою перший том «Капіталу» 
Маркса (не був виданий). Напередодні першої світової війни в Україні 
зростає мережа соціал-демократичної преси: «Дзвін» у Києві, «Воля», 
«Вперед», «Робітник», «Наш голос» — у Львові. Одним із організаторів і 
активних публіцистів у них був Володимир Левицький, автор книжок 
«Нарис розвитку українського робітничого руху в Галичині» (1914), 
«Царская Россия и украинский вопрос» (1919), «Соціалістичний 
інтернаціонал і поневолені народи» та ін. Українські марксисти 
зберігали європейське обличчя марксизму і відмежовувалися від 
марксизму ленінського. Про такий «лібералізований» марксизм можна 
говорити і стосовно Володимира Винниченка та інших лідерів УСДРП. 


Під час Світової війни українська соціал-демократична преса Галичини 
(в підросійській Україні вона була заборонена) не просто осуджувала 
варварське кровопролиття, а викривала загарбницький, 
імперіалістичний характер війни. Глибокий аналіз її причин з 
марксистських позицій дали В.~Левицький та М.~Залізняк. 


Тут слід нагадати, що в цей самий час значна частина європейських 
соціал-демократів, як відомо, розбіглася по «національних квартирах» 
і так чи інакше ставала по боці «своїх» урядів. 


Антивоєнна й правдиво інтернаціоналістична позиція українських 
соціал-демократів парадоксальним чином обернулася проти них у 
визвольну добу 1918--1919 рр., коли вони відігравали провідну роль у 
Центральній Раді та в урядах УНР.~Як соціалісти і марксистські 
налаштовані політики, вони сподівалися на розуміння і мирну 
домовленість із соціалістами й марксистами «великого братнього» 
народу. Але виявилося, що то зовсім інакший «соціалізм» і зовсім 
інакший «марксизм». Ставлення до загрози російсько-більшовицької 
агресії в різних колах УСДРП було різне, і це спричинило внутрішню 
боротьбу й розколи, що також додалося до причин поразки. 


Ще у складнішому становищі опинилися українські соціал-демократи 
після приходу більшовиків в Україну. Вони не могли ігнорувати того 
факту, що більшовицькі гасла неабияк впливали на українські маси, а 
Радянська Росія немовби очолила світовий революційний і 
соціалістичний рух, за яким бачилося майбутнє. Невблаганний 
історичний процес диктував необхідність стратегії і тактики, 
відповідної до нових і непередбачуваних умов, здатної забезпечити 
можливість впливати на події і не бути відкинутими на задвірки 
історії. Тут неминучими стали нові незгоди й розколи. Частина 
вчорашніх соціал-революціонерів та соціал-демократів обирає 
співпрацю, на певних умовах, з більшовиками, сподіваючись таким чином 
впливати на характер перетворень і обстоювати українські національні 
інтереси, як вони їх уявляли. На перших порах ці сподівання почасти 
справджувалися, оскільки більшовики потребували підтримки досить 
сильної партії боротьбистів і йшли на деякі поступки. Але в міру 
зміцнення своєї влади вони дедалі більше утискували своїх 
ситуативних союзників. Тим часом і серед більшовиків, у тодішній 
КП(б)У, були, хоч і не переважальні, «націонал-ухильницькі» (на 
офіційному парткерівному жаргоні) сили, пов'язані зі своїм народом і 
відповідальні за його долю принаймні в тому розумінні, що хотіли 
бачити його рівноправним з іншими в уявлюваному комуністичному 
суспільстві, яке мало привести до вільного розвитку всіх народів. 
Зрештою, сума вагомих чинників — спротив українського села, опозиція 
національної інтелігенції, наявність різних ідеологічних елементів 
та різного бачення історичної перспективи у самій партії, слабкість 
її позицій в українському та інших національних суспільствах, — а не в 
останню чергу й претензія на роль маяка антиколоніальних революцій на 
Сході, для яких комуністична Україна мала стати переконливим і 
звабливим прикладом, — змусила більшовицьку партію, на виконання 
нового курсу Леніна, вдатися до політики «українізації», ширше —
«коренізації», оскільки йшлося й про інші колонізовані народи. Тобто, 
це був пошук надійного опертя в неросійських народах. В Україні ця 
політика пов'язувалася з лідерами націонал-комунізму, давніми 
партійними діячами Олександром Шумським та Миколою Скрипником, які 
прагнули дати їй марксистське обгрунтування. Відповідні дослідження 
проваджувано в Українському Інституті Марксизму та Ленінізму (УІМЛ, 
1922--1931). Професійна марксистська методологія з різною мірою успіху 
впроваджувалася в різних галузях суспільних наук. Серед яскравих 
представників цього штибу мислення можна назвати А.~Річицького, 
одного із засновників УКП (Української Комуністичної Партії), 
сподвижника М.~Скрипника, наукового працівника УІМЛ, автора праць на 
літературні й Марксівські теми, редактора першого видання «Капітала» 
Маркса українською мовою (1927--1929); філософа-марксиста, поета, 
публіциста і літературознавця В.~Юринця; історика М.~Яворського, школа 
якого працювала до погрому 30-х років. 


Доля УІМЛ була такою ж, як і московського Інституту Маркса-Енгельса, 
хіба що набагато трагічнішою, бо стала частиною тотальних репресій 
проти українських наукових і культурних установ та їхніх діячів — під 
моторошний акомпанемент голодомору. 


Зрозуміло, що після цього будь-які серйозні роботи в галузі марксизму 
стали неможливими і втратили сенс, черга реорганізацій закінчилася 
створенням Інституту історії партії при ЦК КПУ — як філіалу Інституту 
марксизму-ленінізму при ЦК КПРС. 


Марксистська фразеологія стала способом придушення самостійного 
мислення, і не дивно, що на час розвалу СРСР марксизм був у суспільстві 
остаточно скомпрометований, хоча долинали ще якісь відгомони 
європейського неомарксизму й були спроби створювати нелегальні 
робітничі гуртки з орієнтацією на «справжній марксизм» (про це ми 
могли довідатися з великим запізненням, у 90-і роки, — після того, як СБУ 
опублікувала секретні матеріали провокаційної кагебістської «Справи 
,,Блок``», по якій «проходили» не тільки «українські буржуазні 
націоналісти», а й широкий спектр інших «підривних елементів»). 


Прикметно, що з настанням горбачовської «гласності» та після здобуття 
Україною незалежності Маркса у нас зовсім не стало. Його уникали, мов 
якогось «совєтського» маркера, і офіціоз, і рухівська опозиція. Щодо 
офіціозу зрозуміло: йому з Марксом не було про що говорити, та й 
некомфортно. А \mbox{РУХ}ом він залишився непрочитаний. На мій погляд, велика 
помилка \mbox{РУХ}у й одна з причин його досить швидкого занепаду — 
абсолютизація національного питання й невміння розкрити всю 
конкретність його пов'язаності з соціальним. Це саме те, чого можна 
було повчитися у Маркса. Але Маркс вважався завербованим у офіційну 
совєтчину (пригадую: коли я в своєму самвидавському опусі 
«Інтернаціоналізм чи русифікація» рясно посилався на погляди Маркса 
й Енгельса, як і Леніна, з національного питання, на його листування, в 
якому фактично заперечується його власна, з «Комуністичного 
маніфесту», теза про те, що пролетаріат не має вітчизни, і говориться 
протилежне: щоб успішно вести свою боротьбу, пролетаріат повинен 
насамперед визволити чи об'єднати свою вітчизну, — багато хто навіть 
із прихильних до мене були подивовані, часом і неприємно, або 
сприймали це як курйоз чи риторичний прийом). Можна зрозуміти: Маркс 
такий далекий, а українське національне питання таке пекуче, що 
багатьом воно здавалося самодостатнім. Серед рухівців панувало 
стихійне переконання: національне — головне, соціальне — 
підпорядковане. Вся історія людства, м'яко кажучи, не підтверджувала 
цього, але гіркі розчарування варті того, щоб їх пережити самому. 
Вкотре наочно виявилося, що в дилемі національне-соціальне (до якої, 
власне, і не повинен допускати розумний політик!) національне стає 
пріоритетом для героїчної меншості, а соціальне — для решти. Героїчна 
меншість може творити революції, але парламенти обирає негеролїчна 
більшість. Тож український виборець у масі своїй голосував не за 
безкорисливих патріотів української мови (або й історії), а за хижих 
демагогів, які обіцяли швидке і фантастичне полагодження житейських 
проблем. І в захисники трудящих на політичній арені перевдягалися 
їхні найжорстокіші експлуататори, захребетники — як оті донецькі 
вугільно-металургійні барони, які невеличку частину прибутків, 
здертих з каторжної праці робітників, витрачали на фінансування 
організованих ними експедицій цих робітників під Верховну Раду чи 
Кабмін, для стукання шахтарськими касками, — так, ніби це українські 
урядовці, а не вони, донецькі барони, винні в несплаті заробітку і в 
жахливому занедбанні техніки безпеки та постійних катастрофах. 


Ні, не став РУХ реальним захисником трудового люду. Як не стали ним і 
профспілки, спосіб організації яких, структура і зміст роботи, права і 
можливості залишаються далекими від прозорості. Може, я помиляюся, але 
мені здається, що ні в кого немає ніякої концепції — ані марксистської, 
ані неомарксистської, ані просто немарксистської, ані навіть 
антимарксистської — захисту трудящих за умов нашого дикого 
капіталізму. Ані концепції, ані продуманих ідей, ані якоїсь — 
політичної чи моральної — гуманістичної настанови, Про це останнє 
кажу тому, що автентичний марксизм — насамперед гуманістичне вчення! 
Воно має глибоке коріння в історії людських борінь за справедливість, 
виразно перегукується з етикою шукань істини, пропонує свого роду 
соціологію пізнання. Як філософ і письменник (у широкому значенні 
слова), Маркс не чужий феноменології, в нього знаходять елементи 
екзистенціалізму. Може, найважливіше чи, принаймні, найцікавіше для 
гуманітаристики в «Капіталі» —це дослідження товарного фетишизму й 
відчуження праці, які фундаментальним чином діють у напрямку 
збіднення світу людини, її знелюднення. Це чинники універсальні, яким 
людство ще не знайшло противаги і не знати, чи колись знайде. Тут, може, 
найважливіші з Марсових відкриттів, і вони варті не меншої уваги, ніж 
його суто економічні осягнення. 


Ще\ldots{} Про Маркса часто говорили й писали, говорять і пишуть, що він 
нібито зневажав духовну творчість або ставився до неї догматично. Як 
на мене, це прикре непорозуміння. Маркс добре знав історію культури, 
його твори не бідні на апеляцію до її фактів та на глибокі думки про 
літературу, великих письменників минулого і сучасників. А прочитайте 
його листування, прилучіться до обсягу його естетичних переживань і 
читацьких реакцій! Отут іще один незнаний нам Маркс!


\ldots{}На закінчення варто додати, що в той час, як в Україні «набридлий» 
за радянські часи Маркс для науковців перестав бути актуальним (хоча б 
для цитування, про вивчення й мови немає), а політологам і публіцистам 
було не до Маркса (та й попиту ніякого), — «націоналісти», а власне 
інтелектуали в українській діаспорі про нього не забували. Крім тих, 
хто поважно студіював Маркса (Р.~Роздольський, відомий як один із 
кращих знавців економічних і національних поглядів Маркса, в 
молодості один із організаторів КПЗУ, якийсь час співробітник, під 
керівництвом Д.~Рязанова й І.~Рубіна, московського Інституту 
Маркса-Енгельса, після його ліквідації працював у архівах Відня, 
Львова, Кракова, у 1942--1945 — в'язень німецьких концтаборів, від 1945 — у 
Детройті, автор багатьох досліджень про Маркса, зокрема й 
неомарксистського тлумачення «Капіталу» — от така дивовижна людина, 
варта біографічного роману або кіно- чи телесеріалу; Панас Феденко, 
один із організаторів УСДРП та лідер її в еміграції; історик і 
публіцист В.~Голубничий; були ті, хто розумів його роль у розвитку 
суспільних наук та в політичній історії людства і знаходив йому 
належне місце в системі своїх ідеологічних оцінок постатей і 
феноменів сучасності, як-от Іван Лисяк-Рудницький. Можна говорити про 
певну пов'язаність з марксизмом Івана Багряного та інших ідеологів 
УРДП — Української Революційно-Демократичної Партії, яка за складних 
умов політичного розбрату в еміграції мужньо обстоювала ідею єдності 
українців на основі не «філологічного паріотизму», а спільності 
корінних інтересів соціальної справедливості й прагнення до свободи. 
Ідеї Івана Багряного могли б уберегти український політикум, 
насамперед рухівців та пізніших «правих», від прикрих помилок та 
неуваги до соціальної сторони української проблематики, — але, на 
жаль, вони не знайшли належного відгуку та й просто місця у вузькому 
кругозорі наших «патріотів» (не кажучи вже про байдужих до України). 


Окремо слід сказати про солідну працю видатного українського 
історика в США (власне, й американського історика) Романа Шпорлюка 
«Націоналізм і комунізм» (Оксфорд, 1988; український переклад Георгія~Касьянова — К., «Основи», 1998). Праця має підзаголовок: «Карл Маркс проти 
Фрідріха Ліста», але фактично Марксова полеміка з німецьким 
економістом і теоретиком націоналізму Лістом — це лише сюжетний 
стрижень праці, який обростає великим фактичним матеріалом та 
інтелектуальними розважаннями автора й лектурою на тему взаємодії 
марксизму й націоналізму як двох великих проектів модернізації 
суспільств — проектів принципово суперечних один одному, але й 
суголосних багато в чому та навіть «повчальних» один до одного — аж 
такою мірою, що в процесі суспільного розвитку ХIХ--ХХ ст. марксизм 
помітно «націоналізувався», а націоналізм — почасти «омарксизмився».


Праця Романа Шпорлюка, на мій погляд, особливо важлива для українців 
тим, що виводить уявлення про націоналізм з провінційних вимірів у 
глобальні, знайомить нашого читача з інтерпретацією націоналізму 
широким колом сучасних європейських істориків, соціологів, філософів; 
те ж саме стосується і марксизму, якого українське суспільство — 
виглядає — так і не освоїло, хоч у ньому є ще немало інтелектуальних 
резервів для нас. Вони ждуть свого «будителя». І, може, якимось 
імпульсом стане сподівана публікація українського перекладу 
«Капіталу». 

\begin{flushright}
  \emph{Іван Дзюба}
\end{flushright}

{\small 22 липня 2018 р.}

  \pagenumbering{arabic}
  \mainmatter
  \thispagestyle{empty}
\null\vspace{6cm}

\noindent\hspace{1.5cm}
\begin{minipage}{\linewidth-1.5cm}
\scmain 
\raggedright
Присвячується моєму незабутньому другові\newline
Сміливому, вірному, благородному,\newline
Передовому борцеві пролетаріяту

\smallskip
{\Large\bfseries\scsans Вільгельмові Вольфу}

\smallskip
Народився в Тарнау 21 червня 1809 року\newline
Помер на вигнанні в Менчестері 9 травня 1864 року
\end{minipage}

\cleardoublepage
  \disablefootnotebreak{}
\input{ii/_!0005.tex}
\parcont{}  %% абзац починається на попередній сторінці
\index{ii}{0006}  %% посилання на сторінку оригінального видання
Берліні 1859~\abbr{р.} під тією самою назвою. В ньому викладено на сторінках
1--220 (зшитки I--V), а потім знову на сторінках від 1159 до 1472
(зшитки XIX--XXIII) теми, досліджені в І книзі „Капіталу“, починаючи з
перетворення грошей на капітал і до кінця; це є перша наявна редакція
цих тем. На сторінках від 973 до 1158 (зшитки XVI--XVIII) мовиться
про капітал і зиск, про норму зиску, купецький капітал і грошовий капітал,
— отже, про теми, пізніше розвинуті в рукопису до III книги. Навпаки,
теми, викладені в II книзі, а також дуже багато тем, розглянутих
пізніше в III книзі, тут окремо ще не розроблені. Їх трактується мимохідь,
саме в відділі, що становить головну частину рукопису, сторінки
220--972 (зшитки VI--XV): \emph{Теорії додаткової вартости}. В цьому відділі
подано докладну критичну історію центрального пункту політичної
економії, а саме теорії додаткової вартости, і разом з тим, розвинуто в
формі полеміки з попередниками більшість пунктів, досліджених пізніше
окремо та в логічному зв’язку в рукопису, що стосується до II та III
книг. Я маю на думці опублікувати критичну частину цього рукопису,
викинувши з нього багато місць, докладно розглянутих у книгах II і
III, — як IV книгу „Капіталу“\footnote*{
Цей рукопис після смерти \emph{Енгельса} виготовив до друку й видав під назвою
„Теорії додаткової вартости“ \emph{К.~Кавтський}. \Red{Ред.}
}. Хоч який цінний цей рукопис, однак, у
ньому мало з чого можна було скористатися для цього видання II книги.

Дальший датою рукопис є рукопис III книги. Його написано, принаймні
більшу частину, в 1854 і 1865 році. Лише після того, як він
був готовий в основному, Маркс почав обробляти першу книгу надрукованого
1867 року першого тому. Цей рукопис III книги я обробляю
тепер до друку.

З найближчого періоду — по виданні книги І — маємо для II книги зібрання
чотирьох рукописів in folio, що їх перенумерував сам Маркс з
І до IV. З них рукопис І (150 сторінок), написаний, мабуть, 1865 або
1867~\abbr{р.}, є перше самостійне, але більш-менш уривчасте оброблення
книги II в її теперішній побудові. І в цьому рукопису також не можна
було нічого використати. Рукопис III складається почасти з зібрання
цитат і посилань на Марксові зшитки з виписами — все це стосується переважно
до першого відділу II книги, — а почасти він є оброблення поодиноких
пунктів, а саме критики Смітових засад про основний та обіговий
капітал та про джерело зиску; далі висвітлено відношення норми
додаткової вартости до норми зиску, що стосується до III книги. Посилання
дали мало нового, бо в наслідок пізніших редакцій з них годі
було користатись для II і III книг; отже, їх теж здебільша довелось відкласти.
— Рукопис IV є готове до друку оброблення першого відділу та
першого розділу другого відділу книги II, і його тут у відповідних місцях
використано. Хоч виявилось, що цей рукопис написано раніше, ніж
рукопис II, однак, з нього можна було добре скористатись для відповідної
частини книги, бо він більш закінчений формою; досить було
зробити деякі додатки з рукопису II. — Цей останній рукопис є одним-одне
\index{ii}{0007}  %% посилання на сторінку оригінального видання
більш-менш викінчене оброблення книги II, і датовано його
1870 роком. У примітках для остаточної редакції, що про них мова
буде зараз, зазначено виразно: „В основу треба покласти друге оброблення“.

Після 1870 року знову постала перерва, зумовлена, головним чином,
хоробливим станом Маркса. Своїм звичаєм Маркс присвятив цей час
студіям. Агрономія, американські та особливо російські земельні відносини,
грошовий ринок і банківська справа, нарешті, природничі науки:
геологія і фізіологія, а особливо самостійні математичні роботи становлять
зміст численних Марксових записних зшитків цього часу. На початку
1877 року він почував себе так добре, що знову міг узятись
до своєї справжньої роботи. Кінцем березня 1877~\abbr{р.} датовано посилання
й замітки з чотирьох вищезгаданих рукописів, які являли основу
того нового перероблення II книги, що початок його є в рукопису V
(56 сторінок in folio). Він охоплює перші чотири розділи й ще мало
оброблений; посутні пункти розроблено в примітках під текстом; матеріял
скорше зібрано, ніж просіяно, але це останній викінчений виклад цієї найважливішої
частини першого відділу. — Першу спробу зробити з цього
рукопис, готовий до друку, являє рукопис VI (написаний \emph{після} жовтня
1877 року й до липня 1878 року); в ньому лише 17 сторінок чвертьаркушевих,
які охоплюють більшу частину першого розділу; а другу — і
останню — спробу являє рукопис VII, датований „2 липня 1878~\abbr{р.}“, що
має лише 7 сторінок in folio.

Того часу Маркс, здається, зрозумів, що, як не буде ґрунтовного перевороту
в стані його здоров’я, він ніколи не матиме змоги закінчити оброблення
другої й третьої книги так, щоб це задовольняло його самого. І справді
на рукописах V--VIII відбилась у багатьох місцях надмірно напружена
боротьба з пригнітною недугою. Найскладнішу частину першого відділу
було перероблено наново в рукопису V, решта першого відділу і цілий
другий відділ (за винятком розділу сімнадцятого) не являли жодних
значних теоретичних труднощів; навпаки, третій відділ, — репродукція
та циркуляція суспільного капіталу, — на його думку, конче треба
було переробити. А саме, в рукопису II розглядалось репродукцію
спочатку без зв’язку з грошовою циркуляцією, що її упосереднює, а потім
ще раз у зв’язку з нею. Це треба було усунути і взагалі так переробити
цілий відділ, щоб він відповідав поширеному кругозорові автора.
Так постав рукопис VIII, зшиток, що має лише 70 сторінок чвертьаркушевих;
але скільки Маркс зумів утиснути в ці сторінки, це доводить
порівняння відділу III у друкованому вигляді, по вилученні з нього
місць, узятих з рукопису II.

І цей рукопис подає лише попереднє трактування предмету, при чому
насамперед малось на увазі визначити й розвинути новоздобуті, порівняно
з манускриптом II, погляди, залишаючи осторонь пункти, що про них
не можна було сказати нічого нового. Чималу частину розділу XVII
другого відділу, якій, окрім того, деяким чином стосується третього відділу,
внову перероблено й поширено. Логічна послідовність часто уривається,
\index{ii}{0008}  %% посилання на сторінку оригінального видання
виклад подекуди має прогалини, і особливо наприкінці він цілком
уривчастий. Але те, що Маркс хотів сказати, так або інакше тут сказано.

Такий матеріял для II книги, що з нього я, як сказав Маркс не задовгий
час до своєї смерти своїй дочці Елеонорі, повинен був „дещо зробити“.
Це доручення я зрозумів у найвужчому його значенні; де лише
можна було, я обмежив мою роботу простим вибором між різними редакціями,
і саме так, що в основу завжди покладав останню з даних редакцій,
порівнявши її з попередніми. Справжні, тобто не лише технічні
труднощі являв при цьому тільки перший і третій відділи, але ці труднощі
були не абиякі. Я дбав про те, щоб розв’язати їх виключно в
авторовому дусі.

Цитати в тексті я здебільша перекладав там, де їх наведено на потвердження
фактів, або там, де ориґінал є до послуг кожного, хто хоче
докладно обізнатися з питанням, прим., у цитатах з А.~Сміта. Тільки в
розділі X не можна було зробити цього, бо тут безпосередньо критикується
англійський текст. У цитатах з І тому посилання зроблено на
сторінки другого видання його, останнього, яке вийшло за життя
Маркса.

Для III книги, крім першої обробки в рукопису „Zur Kritik“, згаданих
частин рукопису III і деяких коротеньких приміток, зроблених подекуди
в записних зшитках, маємо лише ось що: зазначений вище рукопис
in folio від 1864--1865~\abbr{р.}, розроблений майже так само повно, як
і рукопис II книги II, і, нарешті, зшиток з 1875~\abbr{р.}: відношення норми
додаткової вартости до норми зиску, викладене математично (в рівнаннях).
Підготовка цієї книги до друку йде швидким темпом. За думкою,
що в мене склалась до цього часу, вона являтиме, головним чином, технічні
труднощі, за винятком, звичайно, деяких дуже важливих відділів.

\pfbreak{}

\vspace*{\fill}
Тут буде до речі розбити те обвинувачення проти Маркса, що його
поширювали спочатку потихеньку й поодинці, а тепер, після смерти його
проголосили за безперечний факт німецькі катедерсоціялісти й державні
соціялісти та їхні прихильники, — обвинувачення, ніби Маркс учинив
пляґіят у Родбертуса. В іншому місці\footnote{
У передмові до „Das Elend der Philosophie. Antwort auf Proudhons Philosophie
des Elends von Karl Marx. Deutsch von E.~Bernstein und K.~Kautsky.
Stuttgart 1885“ (\emph{К.~Маркс}. „Злидні філософії“. Відповідь на „Філософію злиднів“
\emph{Прудона}).
} я вже сказав усе найпосутніше
з цього приводу, але лише тут можу навести рішучі докази.

Обвинувачення це, оскільки я знаю, вперше трапляється в „\textgerman{Emanzipationskampf
des vierten Standes}“ P.~Maєpa, стор. 43: „З цих оголошених
друком праць“ (праць Родбертуса, датованих до останньої половини
тридцятих років) „Маркс, \emph{як це можна довести}, почерпнув більшу частину
своєї критики“. Поки не було дальших доказів, я, звичайно, міг
припускати, що вся „довідність“ цього твердження сходить на те, що
Родбертус упевнив п. Маєра в цьому. — 1879~року виступає на кін сам Родбертус
\index{ii}{0009}  %% посилання на сторінку оригінального видання
і в зв’язку з своєю працею „\textgerman{Zur Erkenntniss unserer staatswirtschaftlichen
Zustände}“ (1842) пише Й.~Целлерові („\textgerman{Tübinger Zeitschrift für
die Gesamte Staatswissenschaft}“, 1879, S. 219) ось що: „Ви побачите, що
в цього“ (з розвинутих тут думок) „непогано скористався\dots{} Маркс, звичайно,
не посилаючись на мене“. Це повторює, не довго думаючи, за ним
і його посмертний видавець, Т.~Козак („Das Kapital von Rodbertus“, Berlin
1884, Einleitung, S.~XV). — Нарешті, у виданих 1881 року Р.~Маєром
„Briefe und sozialpolitischen Aufsätze von Dr.~Rodbertus-Jagetzow“ Родбертус
прямо каже: „Тепер я бачу, як мене \emph{обікрали} Шефле й Маркс, не
посилаючись на мене“ (Brief № 60, S. 134). А в другому місці претенсія
Родбертуса набирає виразнішої форми: „Відки \emph{виникає додаткова
вартість} капіталіста, це я показав у моєму третьому „Соціяльному листі“
\emph{посутньо так само}, як Маркс, тільки коротше та виразніше“ (Brief
№ 48, S. 111).

Про всі ці обвинувачення в плягіяті Маркс ніколи нічого не знав.
В його примірнику „Emanzipationskampf“ — розрізано тільки частину,
що стосується до Інтернаціоналу, а решту книги розрізав уже я після
його смерти. Тюбінґенської „Zeitschrift“ він ніколи не бачив. „Briefe“ etc.
до Р.~Маєра також були йому невідомі, і мою увагу на місце, що стосується
„обкрадання“, лише 1887 року, ласкаво звернув сам п. д-р Маєр.
Навпаки, лист №48 був Марксові відомий; п. Маєр сам з своєї ласки
подарував ориґінал молодшій дочці Марксовій. Маркс, що до нього звичайно
дійшло потайне шушукання про таємні джерела його критики, які
треба шукати у Родбертуса, показав мені цього листа й додав, що тут
він має, нарешті, автентичне свідчення про те, на що, власне, претендує
сам Родбертус; коли він не каже нічого більш, то для нього, тобто для
Маркса, справа йде на добре; а що Родбертус уважає свій виклад за
коротший та виразніший, то він може дати йому й це задоволення.
Маркс справді гадав, що цим листом Родбертуса вичерпано всю справу.

І так можна було гадати то більше, що, як я добре знаю, вся літературна
діяльність Родбертуса лишалась невідома Марксові до 1859 року,
коли його власна критика політичної економії не лише в основному, але
й у найважливіших подробицях була готова. Свої економічні студії він
почав 1843 року в Парижі, вивчаючи великих англійців і французів; з
німців він знав лише Рав і Ліста, і цього йому було досить. Ні Маркс,
ні я не знали нічогісінько про існування Родбертуса, поки 1848 року
не довелось нам критикувати в „Neue Rheinische Zeitung“ його промову,
як берлінського депутата, і його вчинки, як міністра. Ми були так необізнані,
що запитували райнських депутатів, хто це такий Родбертус,
що так швидко зробився міністром. Але й вони не могли нам нічого
сказати про економічні праці Родбертуса. Навпаки, що Маркс і без допомоги
Родбертуса вже тоді дуже добре знав, не лише звідки, але також
і \emph{як} „виникає додаткова вартість капіталіста“, — це доводять його
„Misère de la Philosophie“ 1847 року і лекції про найману працю та капітал,
прочитані 1847~\abbr{р.} в Брюсселі й опубліковані 1849~\abbr{р.} в „Neue
Rheinische Zeitung“, під № 264--69. Тільки щось 1859~\abbr{р.} Маркс довідався
\index{ii}{*0010}  %% посилання на сторінку оригінального видання
від Ляссаля, що є також економіст Родбертус, і потім знайшов
його „Третій соціяльний лист“ у Брітанському музеї.

Такі фактичні обставини. А як стоїть справа з тим змістом, що його
ніби „украв“ Маркс у Родбертуса? „Відки виникає додаткова вартість
капіталіста, — каже Родбертус, — це я показав у моєму третьому соціяльному
листі так само, як і Маркс, тільки коротше та виразніше“. Отже,
ось де центральний пункт: теорія додаткової вартости; і справді, не
можна сказати, на що інше міг би Родбертус претендувати з Маркса,
як на свою власність. Таким чином, Родбертус виголошує тут себе за
справжнього автора теорії додаткової вартости, що її викрав у нього Маркс.

Що ж каже нам третій соціяльний лист про постання додаткової
вартости? Він каже просто, що „рента“, — а Родбертус має на увазі тут
разом і земельну ренту і зиск, — постає не з „додатку вартости“ до
вартости товарів, але „в наслідок віднімання вартости, що його зазнає заробітна
плата, або, інакше кажучи, в наслідок того, що заробітна плата
являє лише частину вартости продукту“, а за достатньої продуктивности
праці „немає потреби, щоб вона дорівнювала природній міновій вартості
її продукту для того, щоб від неї лишалася ще частина на покриття
капіталу (!) й на ренту“\footnote*{
Цитовані місця є в Родбертуса, а саме в „Soziale Briefe an von Kirchmann,
Dritter Brief“, Berlin, 1851, S. 87. (Примітка Кавтського до Volksausgabe. 1926~\abbr{р.}).
\Red{Ред.}
}. При цьому нам не кажуть, що являє собою
ця „природна мінова вартість“ продукту, що при ній нічого не залишається,
на „покриття капіталу“, а, значить, і на покриття сировинного
матеріялу та зношування знарядь праці.

На щастя, нам припадає констатувати, яке враження справило це епохальне
відкриття Родбертусове на Маркса. В рукопису „Zur Kritik etc.“, в зшитку X,
на стор. 445 і далі, читаємо ми: „Відхилення. Пан Родбертус. Нова теорія
земельної ренти“. Тільки з цього погляду розглядається тут третій соціяльний
лист. З Родбертусовою теорією додаткової вартости справу взагалі
закінчено таким іронічним зауваженням: „Пан Родбертус спочатку досліджує
стан речей у країні, де не відокремлено посідання землею від посідання
капіталом, і доходить потім \emph{важливого} висновку, що рента (а її
він розуміє, як цілу додаткову вартість) просто дорівнює неоплаченій
праці або кількості продуктів, що в ній цю працю втілено.“

Капіталістичне людство вже протягом багатьох століть продукувало
додаткову вартість і помалу дійшло того, що почало замислюватись над
її постанням. Перший погляд виник із безпосередньої купецької
практики: додаткова вартість постає з додатку до вартости продукту.
Цей погляд панував серед меркантилістів, але вже Джемс Стюарт побачив,
що при цьому те, що один виграє, другий неминуче втрачає. Не зважаючи
на це, цей погляд тримався ще й далі довгий час, особливо серед
соціялістів; але з клясичної науки його витиснув А.~Сміт.

У „Багатстві народів“, кн. І, розділ VI, він каже: „Скоро капітал
(stock) нагромадився в руках поодиноких осіб, деякі з них звичайно застосують
\index{ii}{*0011}  %% посилання на сторінку оригінального видання
його так, що посадять за роботу старанних людей, дадуть їм
сировинний матеріял і засоби існування для того, щоб здобути \emph{зиск},
продаючи продукти їхньої праці, або того, що \emph{їхня праця додала до
вартости того сировинного матеріялу\dots{} Вартість}, що її робітники
\emph{додають до сировинного матеріялу}, поділяється тут на \emph{дві частини} —
з них однією оплачується \emph{їхню заробітну плату}, а друга становить
\emph{зиск підприємця} на всю суму, що її він авансував на сировинний
матеріял та заробітну плату.“ І трохи далі: „Скоро вся земля в якій-будь
країні стане приватною власністю, землепосідачі, як і інші люди,
воліють за краще жати там, де вони не сіяли, і вимагають земельної
ренти навіть за природні продукти землі\dots{} Робітник\dots{} мусить \emph{відступити}
землепосідачеві \emph{деяку частину} з того, що він зібрав або виробив
своєю \emph{працею}. Ця частина, або — що те саме — ціна цієї частини становить
\emph{земельну ренту}“.

З приводу цього місця Маркс зауважує в зазначеному рукопису
„Zur Kritik etc.“, на стор. 253: „Отже, А.~Сміт розуміє додаткову вартість,
— а саме додаткову працю, надлишок виконаної та зречевленої в
товарі праці \emph{проти} оплаченої праці, тобто проти праці, що одержала
свій еквівалент у заробітній платні, — як \emph{загальну категорію}, при чому
власно зиск і земельна рента являють лише її відгалуження“.

Далі А.~Сміт каже в книзі І, розд. VIII: „Скоро земля стала приватною
власністю, землепосідач вимагає частину майже всіх продуктів, що
їх робітник може виробити або зібрати на ній. Його земельна рента
становить \emph{перше відрахування з продукту праці, прикладеної до землі}.
Але хлібороб рідко коли має засоби утримувати себе до збору врожаю.
Його утримання звичайно авансується йому з капіталу (stock) підприємця,
орендаря, що не мав би жодного інтересу давати йому роботу, коли б
хлібороб \emph{не ділився з ним продуктом своєї праці} або не повертав йому
капіталу разом з зиском. Цей зиск є \emph{друге відрахування} з праці, прикладеної
до землі. Продукт майже всякої праці зазнає такого самого
відрахування для зиску. В усіх галузях промисловости для більшости робітників
потрібен підприємець, який авансував би їм до закінчення праці
сировинний матеріял і заробітну плату та утримання. Цей підприємець
\emph{поділяє} з ними \emph{продукт їхньої праці}, або вартість, що її вони додають
до перероблюваного сировинного матеріялу, і ця частка становить його
зиск“.

Маркс додає до цього (рукопис, ст. 256): „Отже, А.~Сміт по-простому
визначає тут земельну ренту і зиск на капітал як прості відрахування з
продукту робітника або з вартости його продукту, що дорівнює праці,
долученій ним до сировинного матеріялу. Але це відрахування, як довів
раніше сам А.~Сміт, може складатися лише з частини праці, яку робітник
додав до матеріялу понад ту кількість праці, яка оплачує тільки
його заробітну плату або дає еквівалент його заробітної плати, отже, з
додаткової праці, з неоплаченої частини його праці“.

„Відки виникає додаткова вартість капіталіста“ і, крім того, землевласника,
це знав, як бачимо, ще А.~Сміт; Маркс відверто визнає це ще
\parbreak{}  %% абзац продовжується на наступній сторінці

\enablefootnotebreak{}
\parcont{}  %% абзац починається на попередній сторінці
\index{ii}{*0012}  %% посилання на сторінку оригінального видання
1861~\abbr{р.}, тимчасом як Родбертус і юрба його прихильників, що ростуть,
як гриби під теплим літнім дощем державного соціялізму, здається, зовсім
забули про це.

„Однак, — каже далі Маркс, — додаткову вартість, як таку, Сміт не
відрізняв, як осібну категорію від особливих форм, що їх вона набирає
в зиску та земельній ренті. Відси в нього, а ще більше в Рікардо, багато
помилок і хиб у досліді“. — Це речення цілком стосується до Родбертуса.
Його „рента“ є просто сума земельної ренти й зиску; про земельну
ренту він склав собі цілком хибну теорію, а зиск він приймає без якогобудь
перегляду таким, як знайшов його у своїх попередників. — Навпаки,
додаткова вартість Марксова є \emph{загальна форма тієї} суми вартости, що
її привлащують без якогобудь еквіваленту власники засобів продукції,
і яка за цілком своєрідними законами, що їх уперше відкрив Маркс,
розкладається на особливі, \emph{перетворені} форми зиску та земельної ренти.
Ці закони викладається в III книзі, де вперше виявиться, як багато треба
проміжних ланок для того, щоб від загального розуміння додаткової
вартости дійти до розуміння того, як вона перетворюється на зиск і
земельну ренту, отже, до розуміння законів розподілу додаткової вартости
серед кляси капіталістів.

Рікардо йде вже значно далі порівняно з А.~Смітом. Він обґрунтовує
своє розуміння додаткової вартости на тій новій теорії вартости, що в
зародковій формі хоч і є вже в А.~Сміта, але ним майже завжди забувається
в його дослідженнях, — теорії, що стала за відпровідний пункт
усієї дальшої економічної науки. З того, що вартість товару визначається
кількістю праці, зреалізованої в товарах, він висновує розподіл між
робітниками й капіталістами тієї кількости вартости, яку долучено до
сировинного матеріялу працею, її розподіл на заробітну плату й зиск
(тобто в даному разі додаткову вартість). Він доводить, що вартість
товарів лишається та сама, хоч як змінюється відношення між цими двома
частинами, — закон, що для нього він припускав лише поодинокі винятки.
Він навіть висновує деякі головні закони щодо взаємного відношення між
заробітною платою і додатковою вартістю (взятою в формі зиску), хоч і в
дуже загальному розумінні (Маркс. Капітал, І, розділ XV, А), і доводить, що
земельна рента є надлишок над зиском, надлишок, — який відпадає в певних
умовах. — У жодному з цих пунктів Родбертус не пішов далі, ніж Рікардо.
Внутрішні суперечності теорії Рікардо, що на них загинула його школа, лишились
або зовсім невідомі Родбертусові, або довели його лише до утопічних
вимог („Zur Erkenntniss“ etc., S. 130) замість економічних розв’язань.

Однак ученню Рікардо про вартість і додаткову вартість не довелось
чекати „Zur Erkenntniss“ etc. Родбертуса, щоб здобути соціялістичного
використання. На стор. 609 першого тому „Капіталу“ (2 нім. видання)
наведено цитату: „The possessors of surplus produce or capital („Посідачі
додаткового продукту або капіталу“) з праці „The Source and Remedy of
the National Difficulties. A Letter to Lord John Russell“. London 1821\footnote*{
Див. українське видання, розд. XXII.~І. \emph{Ред.}
}.
\parbreak{}  %% абзац продовжується на наступній сторінці

\parcont{}  %% абзац починається на попередній сторінці
\index{ii}{*0013}  %% посилання на сторінку оригінального видання
Ця праця, що на її значення повинен був би звернути увагу вже один
вислів: Surplus produce or capital, є памфлет на 40 сторінок, що його
Маркс видобув з непам’яті; там сказано:

„Хоч як це випадало б капіталістові (з погляду капіталіста), він
завжди може привласнювати лише додаткову працю (surplus labour) робітника,
бо робітник повинен жити“ (ст. 23). Але \emph{як} робітник живе,
і тому оскільки велика може бути додаткова праця, що її привлащує
капіталіст, — це дуже відносно. „Коли вартість капіталу зменшується
не в такому відношенні, як збільшується його маса, то капіталіст буде,
витискувати з робітника продукт кожної робочої години понад той мінімум,
що з нього може існувати робітник\dots{} капіталіст може, кінець-кінцем
сказати робітникові: не треба тобі їсти хліб, бо можна прожити й на
буряках та картоплі; і ми вже дійшли цього“ (ст. 24). „Коли робітника
можна довести до такого стану, що він харчуватиметься картоплею замість
хліба, то, безперечно, правильно, що при цьому можна більше здерти з
його праці; тобто, коли, харчуючись хлібом, він повинен був на утримання
себе та своєї сім’ї \emph{залишати для себе працю понеділка й вівторка},
то, годуючись картоплею, він матиме для себе тільки \emph{половину понеділка};
а друга половина понеділка і ввесь вівторок  \emph{звільняться} або на користь
державі або  \emph{для капіталістів}“ (ст. 26). „Безперечно (it is admitted),
що сплачувані капіталістам інтереси, чи в формі ренти, проценту або
підприємецького зиску, сплачуються з праці інших“ (ст. 23). Отже, тут
ми маємо цілком Родбертусову „ренту“, тільки замість „ренти“ сказано
„інтереси“.

Маркс робить таке до цього зауваження (рукопис „Zur Kritik“,
ст. 852): „Цей мало відомий памфлет, — а видано його тоді, коли почав
звертати на себе увагу „неймовірний латальник“ Мак-Куллох, — являє
великий крок наперед порівняно з Рікардо. Додаткову вартість або
„зиск“, як зве її Рікардо (часто також додатковий продукт, surplus
produce), або interest, як називає її автор памфлету, останній визначає як
surplus labour, як додаткову працю, — працю, що її робітник виконує безплатно,
виконує понад ту кількість праці, що нею покривається вартість
його робочої сили, тобто що нею продукується еквівалент його заробітної
плати. Так само, як важливо було звести  \emph{вартість до праці}, так
само важливо було додаткову вартість (surplus value), \emph{виражену в додатковому
продукті} (surplus produce), звести до \emph{додаткової праці}
(surplus labour). \emph{Це власне сказав уже А.~Сміт, і це становить головний
момент у тому, що розвинув Рікардо}. Але ніде в них це не
висловлено в абсолютній формі й не установлено точно. Потім далі, на
стор. 859 рукопису, сказано: „А, проте, автора полонили ті економічні
категорії, що були до нього. Так само, як у Рікардо сплутування додаткової
вартости й зиску призводить до неприємних суперечностей, так
само сталось і з ним тому, що він назвав додаткову вартість
інтересом капіталу. А, проте, він стоїть вище від Рікардо тією стороною, що
він перший зводить усяку додаткову вартість до додаткової праці і, хоч зве
додаткову вартість інтересом капіталу, однак, разом з тим підкреслює,
\index{ii}{*0014}  %% посилання на сторінку оригінального видання
що interest of capital він розуміє, як загальну форму
додаткової праці на відміну від її особливих форм, ренти, проценту і
підприємецького зиску\dots{} Але назву однієї з цих особливих форм, interest,
він усе ж бере як назву загальної форми. І цього досить, щоб він знову
заплутався в економічній тарабарщині“ (в рукопису стоїть „slang“).

Цей останній пункт якнайточніше стосується й до Родбертуса. І він
також у полоні економічних категорій, що були до нього. І він зве
додаткову вартість ім’ям однієї з її перетворених підпорядкованих їй форм,
ім’ям ренти, — ренти, що її він до того ж зробив зовсім невизначеною.
Наслідок цих двох помилок був той, що він знову вдається в економічну
тарабарщину, не прокладає критично шляхів далі за Рікардо і замість
того піддається спокусі зробити з своєї недоробленої теорії, що не
вилупилася ще з шкаралупи, основу утопії, з якою він, як і завжди,
прийшов дуже пізно. Памфлет, виданий 1821 року, цілком упередив
„ренту“ \emph{Родбертуса} від 1842 року.

\vtyagnut
Наш памфлет є лише крайній аванпост тієї багатої літератури, що
двадцятими роками обернула теорію вартости й додаткової вартости
Рікардо в інтересах пролетаріяту проти капіталістичної продукції, била
буржуазію її власною зброєю. Весь Оуенівський комунізм, оскільки він
бере участь в економічній полеміці, спирається на Рікардо. Але поряд
нього був ще цілий ряд письменників, що з них Маркс уже 1847~\abbr{р.} в
полеміці проти Прудона („Misère de la philosophie“, p. 49) згадує лише деяких:
Едмонда, Томпсона, Годскіна і~\abbr{т. ін.} і~\abbr{т. ін.}, „і ще чотири сторінки et
cetera“. З цих численних праць я беру одну, першу-ліпшу: „Ап Inquiry
into the Principles of the Distribution of Wealth, most conducive to Human
Happiness, by William Thompson; a new edition, London 1850“. Цей твір,
написаний 1822~\abbr{р.}, вперше видано 1827~\abbr{р.} Багатство, що його привлащують
непродуктивні кляси, тут теж усюди визначається, як відрахування з
продукту робітника, і це в досить енергійних висловах. „Повсякчасне
намагання того, що ми звемо суспільством, було в тому, щоб обманою
або умовлянням, залякуванням або примусом спонукати продуктивного
робітника виконувати працю за якомога меншу частину продукту його
власної праці“ (стор.~28). „Чому ж робітник не може одержувати абсолютно
ввесь продукт своєї роботи?“ (ст.~32). „Цю компенсацію, що її
капіталісти вимушують від продуктивних робітників під назвою земельної
ренти, або зиску, вимагають за користування землею або іншими речами\dots{}
Що всі фізичні матеріяли, на яких або за допомогою яких позбавлений
власности продуктивний робітник, що нічого не має, крім своєї здібности
продукувати, тільки й може виявити цю свою продуктивну здібність, —
що всі ці матеріяли є в посіданні інших осіб, котрих інтереси протилежні
інтересам робітника, а згода їх є передумова його діяльности, — то чи не
залежить і чи не повинно залежати від ласки цих капіталістів те, яку
\emph{частину витворів його власної праці} вони побажають дати йому, в
нагороду за цю працю? (ст.~125)\dots{} порівняно з величиною \emph{утриманого
продукту}, все одно, чи зветься він податком, зиском або крадіжкою\dots{}
ці відрахування“ (ст.~126) і~\abbr{т. ін.}


\index{ii}{*0015}  %% посилання на сторінку оригінального видання
Треба визнати, трохи соромно мені писати ці рядки. Я ще можу
припустити, що антикапіталістична англійська література двадцятих і
тридцятих років цілком невідома в Німеччині, хоч Маркс іще в „Misère
de la philosophie“ безпосередньо звертав на неї увагу і дещо з неї —
памфлет 1821~\abbr{р.}, Равенстона, Годскіна та ін. часто цитував у першому
томі „Капіталу“. Але що не лише Literatus vulgaris\footnote*{
Вульґарний літератор. \emph{Ред.}
}, „який справді
нічого не навчився“ і з розпачу хапається за поли Родбертусового сурдута,
а також і професор з чином і гідністю, який „бундючиться
своєю вченістю“, до того забув свою клясичну політичну економію, що
серйозно закидає Марксові, ніби той у Родбертуса украв такі речі, що
їх можна знайти вже у А.~Сміта й Рікардо, — це доводить, як низько
занепала тепер офіційна політична економія.

Але тоді що ж нового сказав Маркс про додаткову вартість? Як
сталося, що Марксова теорія додаткової вартости справила таке вражіння,
як блискавка з чистого неба, і до того ж у всіх цивілізованих країнах,
тимчасом як теорії всіх його соціялістичних попередників, а між ними й
Родбертуса, не справили жодного враження?

Історія хемії може нам пояснити це на прикладі.

Як відомо, ще наприкінці XVIII століття панувала флогістична теорія,
згідно з якою суть кожного процесу горіння в тому, що від горящого
тіла відокремлюється інше, гіпотетичне тіло, абсолютна горюча речовина,
яку позначали назвою флогістон. Ця теорія була достатня, щоб пояснювати
більшість відомих тоді хемічних явищ, хоч інколи й не без деякого
натягання. Але 1774 року Прістлей відкрив відміну газу, „що був такий
чистий або такий вільний від флогістону, що порівняно з ним звичайне
повітря було вже чимось попсованим“. Він дав цьому газові назву дефлогістоване
повітря. Скоро після нього такий самий газ відкрив у Швеції
Шеле й довів, що він є в атмосфері. Він виявив також, що цей газ
зникає, коли в ньому або в звичайному повітрі спалювати якесь тіло, а
тому назвав його вогнеповітрям. „Отже, з цих даних він зробив той
висновок, що сполука, яка постає в наслідок з’єднання флогістону з
однією із складових частин повітря (тобто підчас горіння) „є не що
інше, як огонь або тепло, яке й зникає, проходячи крізь скло“\footnote{
Roscoe-Schorlemmer. Ausführliches Lehrbuch der Chemie. Braunschwelg, 1877,
I. ст. 13, 18.
}.

Прістлей, як і Шеле, описали кисень, але вони не знали, що саме
було в їхніх руках. Вони „були в полоні“ флогістичних „категорій, що
їх знайшли вони в попередників“. Елемент, що йому судилось перевернути
всі флогістичні погляди й революціонізувати хемію, марно пропадав
в їхніх руках. Але Прістлей негайно повідомив про своє відкриття
Лявуазьє в Парижі, а Лявуазьє, керуючись цим новим фактом, переглянув
усю флогістичну хемію і перший відкрив, що нова відміна повітря
є новий хемічний елемент, що підчас горіння не \emph{вилучається} з горящого
тіла отой таємничий флогістон, а що цей новий елемент \emph{сполучається} з
\parbreak{}  %% абзац продовжується на наступній сторінці

\input{ii/_!0016.tex}
\parcont{}  %% абзац починається на попередній сторінці
\index{ii}{*0017}  %% посилання на сторінку оригінального видання
процес утворення додаткової вартости в його справжньому перебігу, — а
цього не зробив ніхто з його попередників; отже, він констатував у
самому капіталі відмінності, що з ними дати собі раду зовсім не могли
ні Родбертус, ні буржуазні економісти, — відмінності, які дають ключ до
розв’язання найзаплутаніших економічних проблем, — і найкращий доказ
цього знову є книга II, а ще більше книга IIІ, як це виявиться далі.
Далі він досліджував і саму додаткову вартість і виявив обидві її форми:
абсолютну й відносну додаткову вартість, і показав ту різну, але і в
тому й у другому разі вирішувальну ролю, яку відігравала вона в історичному
розвитку капіталістичної продукції. На основі додаткової вартости
він розвинув першу раціональну теорію заробітної плати, яку ми
маємо, і вперве висвітлив основні риси історії капіталістичної акумуляції
та виклав її історичні тенденції.

\vtyagnut{}
А що ж Родбертус? Прочитавши все це, він — як і всякий тенденційний
економіст! — вважає це за „напад на суспільство“, вважає, що він уже сам
сказав куди коротше та виразніше, відки постає додаткова вартість, і
вважає, нарешті, що хоч усе це слушно для „сучасної форми капіталу“,
тобто для капіталу, як він існує історично, але не слушно для „поняття
про капітал“, тобто для утопічного уявлення пана Родбертуса про капітал.
Цілком те саме, що з стариком Прістлеєм, який до кінця свого
життя вірив у флогістон і не хотів визнавати кисню. Тільки Прістлей
справді перший описав кисень, тимчасом як Родбертус у своїй додатковій
вартості або радше в своїй „ренті“ тільки знову відкрив загальник,
а Маркс поводився протилежно до Лявуазьє і був вищий від того, щоб
твердити, що він перший відкрив самий \emph{факт} існування додаткової
вартости.

Все інше, що зробив Родбертус у політичній економії, стоїть на
такому самому рівні. Його перероблення додаткової вартости на утопію
Маркс критикував ненароком уже в „Misère de la Philosophie“;
все, що можна було ще сказати про це, я сказав у передмові до німецького
перекладу цієї праці. Його пояснення торговельних криз недостатнім
споживанням робітничої кляси є вже в Сісмонді в „Nouveaux Principes
de l’Economie Politique“, книга IV, розділ IV\footnote{
„Ainsi donc, par la concentration des fortunes entre un petit nombre des propriétaires,
le marché intérieur se reserre toujours plus, et l'industrie est toujours plus
réduite à chercher ses débouchés dans les marchés étrangers, où de plus grandes révolutions
les menacent“. Nouv. Princ., éd. 1819, 1, p. 336. („Отже, в наслідок концентрації
багатств у руках небагатьох власників унутрішній ринок дедалі більше
скорочується, і промисловості доводиться дедалі більше шукати собі місця збуту
на закордонних ринках, де їй загрожують превеликі перевороти“ (а саме криза
1817~р., що її далі й описується).
}.Тільки Сісмонді
при цьому завжди мав на увазі світовий ринок, тимчасом як горизонт
Родбертуса не поширюється поза пруський кордон. Його міркування про
те, чи з капіталу, чи з доходу походить заробітна плата, належать до
схоластики і остаточно їх усувається третім відділом цієї другої книги
„Капіталу“. Його теорія ренти лишилась виключно його здобутком і
\parbreak{}  %% абзац продовжується на наступній сторінці

\parcont{}  %% абзац починається на попередній сторінці
\index{ii}{*0018}  %% посилання на сторінку оригінального видання
може спокійно лежати, поки вийде друком рукопис Марксів, що її критикує\footnote*{
Цю книгу видано за редакцією К.~Кавтського під назвою: „Theorien über den Mehrwert“. Zweiter
Band. Erster Teil. („Теорії додаткової вартости“. Том II, част. І). \Red{Ред.}
}\dots{} Нарешті, його проєкти
щодо визволення старопруського землеволодіння від гніту капіталу знову цілком утопічні; а саме вони
обминають єдине практичне питання, про яке тут ідеться, — питання про те, як старопруський юнкер,
одержуючи, приміром, \num{20.000} марок щороку й витрачаючи, приміром, \num{30.000} марок, може все таки не
робити боргів?

Школа Рікардо біля 1830 року скрахувала на додатковій вартості. Те, чого вона не могла розв’язати,
лишилось то більше нерозв’язним для її наступниці, вульґарної економії. Два пункти, що на них вона
зазнала загибелі, такі.

Поперше. Праця є міра вартости. Але жива праця в обміні на капітал має меншу вартість, ніж
зречевлена праця, що на неї її обмінюється. Заробітна плата, вартість певної кількости живої праці,
завжди менша, ніж вартість продукту, утворюваного цією самою кількістю живої праці, або того
продукту, що в ньому ця праця втілюється. Коли
так уявляти це питання, то його в дійсності не можна розв’язати. Маркс поставив його правильно й
тому дав на нього відповідь. Не праця має вартість. Як діяльність, що утворює вартість, вона так
само не може мати особливої вартости, як важкість не може мати особливої ваги, тепло — особливої
температури, електрика особливої сили струму. Купується й продається, як товар, не праця, а робоча
\emph{сила}. Скоро вона стає товаром, її вартість вимірюється працею, втіленою в ній, як у суспільному
продукті; ця вартість дорівнює праці, суспільно доконечній для її продукції та репродукції. Отже,
купівля і продаж робочої сили на основі такої її вартости зовсім не суперечать економічному законові
вартости.

Подруге. Згідно з законом вартости Рікардо, два капітали, що вживають однакової кількости однаково
оплачуваної живої праці, за всіх інших однакових умов, продукують протягом однакового часу продукти
однакової вартости, а також додаткову вартість або зиск однакового розміру. А коли вони вживають
неоднакової кількости живої праці, то не можуть продукувати додаткову вартість, або, як кажуть
рікардіянці, зиск однакового розміру. А в дійсності маємо протилежне. У дійсності однакові
капітали протягом однакового часу продукують пересічно однаковий зиск незалежно від того, чи багато,
чи мало вживають вони живої праці. Отже, тут маємо суперечність законові вартости, що її помітив ще
Рікардо, і яку його школа теж не могла розв’язати. Родбертус також не міг не помітити цієї
суперечности, але замість розв’язати її, він зробив з неї один з вихідних пунктів своєї утопії („Zur
Erkenntniss“, S. 131). Цю суперечність розв’язав Маркс уже в рукопису „Zur Kritik“; за пляном
„Капіталу“ це розв’язання подається в книзі III.~До її опублікування пройдуть іще місяці. Отже,
економісти, які хочуть відкрити в особі Родбертуса таємне джерело й незрівняного попередника Маркса,
\parbreak{}  %% абзац продовжується на наступній сторінці

\parcont{}  %% абзац починається на попередній сторінці
\index{ii}{*0019}  %% посилання на сторінку оригінального видання
мають тут нагоду довести, що може дати Родбертусова економія. Коли
вони доведуть, як може та мусить утворюватись однакова пересічна
норма зиску не лише без порушення закону вартости, але, навпаки, на
його основі, тоді ми будемо розмовляти з ними далі. А тимчасом чи не
будуть вони ласкаві поспішити? Блискучі досліди цієї II книги й її цілком
нові результати в досі майже недосліджених галузях являють лише
попередні тези до змісту III книги, яка розвиває остаточні висновки Марксового
викладу суспільного процесу репродукції на капіталістичній
основі. Коли вийде ця III книга, менше буде розмов про якогось економіста
Родбертуса.

\manualpagebreak{}

Друга й третя книги „Капіталу“, як часто казав мені Маркс, повинні
бути присвячені його дружині.

\begin{flushright}
\emph{Фрідріх Енґельс}
\end{flushright}

{\small Лондон, день Марксового народження, 5 травня 1885~\abbr{р.}}

\manualpagebreak{}

\noindent{}Друге видання, що його тут подано, є в головному дослівний передрук
першого. Виправлено друкарські помилки, усунуто деякі стилістичні
хиби, викинуто кілька коротких абзаців, де були тільки повторення.

Третя книга, що являла цілком несподівані труднощі, тепер майже
готова в рукопису. В разі, що буду здоровий, її можна буде почати
друкувати ще цієї осени.

\begin{flushright}
\emph{Фрідріх Енґельс}
\end{flushright}

{\small Лондон, 15 липня 1893~\abbr{р.}}

\pfbreak{}

Щоб полегшити орієнтування, подаємо коротеньке зіставлення місць,
узятих з окремих рукописів II--VIII.

\subsubsection*{Відділ перший}

Стор.~\pageref{original-3} з рукопису~II. — 
Стор.~\pageref{original-4} з рукопису~VII. — 
Стор.~\pageref{original-13}--\pageref{original-16} з рукопису~VI. —
Стор.~\pageref{original-16}--\pageref{original-76} з рукопису~V. — 
Стор.~\pageref{original-76}--\pageref{original-79-1} замітка,
знайдена між витягами з книжок. —
Стор.~\pageref{original-79-2} до кінця — рукопис IV; однак
вставлено: стор.~\pageref{original-85}--\pageref{original-86} місце з рукопису~VIII; стор.~\pageref{original-89} і \pageref{original-94} замітки з рукопису~II.

\subsubsection*{Відділ другий}

Початок, ст.~\pageref{original-104}--\pageref{original-112} є кінець рукопису~IV.
Відси до кінця відділу, ст.~\pageref{original-266}, все з рукопису II.

\subsubsection*{Відділ третій}

Розділ \RNum{18} (стор.~\pageref{original-267}--\pageref{original-273}) з рукопису~II.

\noindent{}Розділ \RNum{19}: \rmntonum{I} і \rmntonum{II} (стор.~\pageref{original-274}--\pageref{original-298}) з рукопису~VIII.
— \rmntonum{III} (стор.~\pageref{original-298}--\pageref{original-300-1}) з рукопису~II.
\index{ii}{*0020}  %% посилання на сторінку оригінального видання

\noindent{}Розділ \RNum{20}:

\rmntonum{I} (стор.~\pageref{original-300-2}--\pageref{original-302-1}) з рукопису II,
лише прикінцевий абзац з рукопису~VIII.

\rmntonum{II} (стор.~\pageref{original-303}--\pageref{original-305-1}) в головному з рукопису II.

\rmntonum{III}, \rmntonum{IV}, \rmntonum{V} (стор.~\pageref{original-305-2}--\pageref{original-325-1}) з рукопису VIII.

\rmntonum{VI}, \rmntonum{VII}, \rmntonum{VIII}, \rmntonum{IX}
(стор.~\pageref{original-325-2}--\pageref{original-337}) з рукопису II.

\rmntonum{X}, \rmntonum{XI}, \rmntonum{XII}
(стор.~\pageref{original-338}--\pageref{original-373-1}) з рукопису VIII.

\rmntonum{XIII} (стор.~\pageref{original-373-2}--\pageref{original-380-1}) з рукопису II.

\noindent{}Розділ \RNum{21}: (стор.~\pageref{original-380-2}--\pageref{original-409}) увесь з рукопису VIII.



% Тут залишив ориґінальний текст, 
% бо замінив римські на арабські
% 
% Розділ 18 (стор. 267--274) з рукопису II.
% Розділ 19: I і II (стор. 274--299) з рукопису VIII. — III (стор. 299--300)
% з рукопису II.

% Розділ 20: І (стор. 300--302) з рукопису II, лише прикінцевий
% абзгц з рукопису VIII.

% II (стор. 300--305) в головному з рукопису II.

% Ill, IV, V (стор. 305--325) з рукопису VIII.

% VI, VII, VIII, IX (стор. 325--337) з рукопису II.

% X, XI, XII (стор. 338--373) з рукопису VIII.

% XIII (стор. 373--380) з рукопису II.

% Розділ 21: (стор. 380--409) увесь з рукопису VIII.

  \bookpaget{\BookNumber{}}{\BookTitle{}}
  \setcounter{footnote}{0}% Reset footnote counter

\bookpages{Додаток}{Фрагмент «Капіталу» у~перекладі Івана~Франка}{}
  \addtocontents{toc}{\protect\booktocentry{Додаток}{Фрагмент «Капіталу»\protect\par у~перекладі Івана~Франка}}

\nonumsection{Чи застарів «застарілий» Маркс?}{~}{Іван Дзюба}

Почати з того, що Маркс застарівав уже не один раз. Спершу — ще після 
революцій 1848 року, які розвивалися не за логікою «Комуністичного 
маніфесту». Потім — після невдачі Паризької Комуни. Далі — коли його 
відмодельовували в протилежні боки «ревізіоністи» (від Бернштейна до 
Каутського) і Ленін, більшовики, Сталін\ldots{} А скільки в нього 
застарілих окремих формул і тез! Наприклад, колись знамените про 
«ідіотизм сільського життя». Це ж як воно звучить тепер, коли світ 
потерпає від набагато глибшого і страшнішого ідіотизму мегаполісів?!


Про те, що Карл Маркс застарів, знає у нас навіть той, хто взагалі 
нічого не знає (особливо він). Але, хоч як дивно, всупереч нашому знанню 
і незнанню, у західних університетах його праці поважно вивчають 
(звісно, з певної критичної позиції), про нього пишуть видатні 
соціологи й філософи як про одного з великих мислителів людства, його 
перевидають і шукають у нього стежок до пояснення економічних криз 
сучасного світу тощо, — а 5-го травня цього року широко відзначалося 
його 200-річчя. Але все це — «за бугром». У нас же будь-який 
малоосвічений публіцист може при нагоді поглузувати з «двох 
німецьких гномів» (це довелося зустріти недавно в інтернетному 
дописі). Таке от бачення (власне, небачення) історії, такий рівень 
культури мислення, таке розуміння динаміки інтелектуального розвитку 
людства, за якої насправді нові осягнення виростають із 
«застарілого», а заперечуване відходить, тільки стимулювавши саме 
заперечення, або й повертається невпізнане.


Ще одна біда — коли говорити про «масову людину» — брак історичного 
підходу в поцінуванні культурних явищ та феноменів думки, понятійних 
категорій. Багато хто в простоті душевній гадає, що це Маркс придумав 
класи, класову боротьбу, пролетаріат, революції та інший клопіт, отож 
усі біди від нього, Маркса. Таким чином на нього ніби падає 
відповідальність за століття (або й тисячоліття) соціальних 
конфліктів і майнових битв на нашій планеті. Хай так. Але на 
виправдання Маркса можна сказати, що в нього було багато попередників. 
Не будемо зазирати в біблійні, або античні, чи й середньовічні часи, а 
звернімося до ХIХ ст., в якому й визрівало те, що згодом дістало назву 
марксизму.

\looseness=1
Отож: у межах європейських феодальних монархій народжується і набирає 
сил буржуазія, що здобуває економічні й політичні позиції, 
використовуючи суперечності між монархом і його васалами та свої 
фінансово-майнові важелі; відбувається нагромадження капіталу, 
розвивається фабрична промисловість, змінюється характер виробничих 
і суспільних відносин та способи експлуатації робітника, колишні 
дрібні власники й обезземелені селяни стають знекоріненою і 
безправною «робочою силою», зростає безробіття. Пролетаризація 
охоплює цілі суспільні верстви, збільшуючи зубожіння люду й набираючи 
катастрофічного характеру. Як відповідь на біди, що їх приніс масі 
населення, насамперед трудовому людові, бурхливий розвиток 
жорстокого й хижацького капіталізму; як відповідь на загострення 
соціальних антагонізмів та масових злиднів — виникають, з одного 
боку, стихійні бунти, наприклад, луддитів та інших руйначів машин, 
потім і бунти та масові революційні рухи, а з другого — народжуються 
соціальні міфи й утопії та спроби окремих гуманістично мислячих, з 
чутливою соціальною совістю особистостей, як правило з освічених, 
упривілейованих станів, запропонувати моделі подолання кричущих 
суспільних дисгармоній і шляхи влаштування справедливих відносин, 
бодай у окремих локальних осередках, якщо не взагалі у світі. Так 
народжується утопічний соціалізм, яскрава плеяда теоретиків якого 
(К.~А.~Сен-Симон, Ш.~Фур'є, Р.~Оуен, а також Т.~Мюнцер, Т.~Кампанелла, Мореллі, 
Ж.~Мельє, Дж.~Уїнстлі, Г.~Б.~Маблі, Г.~Бабьйоф, Т.~Дезамі) за всієї відмінності 
поміж собою в конкретних позиціях і національному представництві 
були суголосними в критиці реального стану супільств як 
невідповідного поняттям про гідне життя, справедливість, моральність 
і доцільність, — а тому й неприйнятного для людського розуму й 
совісті. Їхні проекти ідеального суспільства базувалися на 
ідеалістичних уявленнях про рівність, свободу і братерство, про 
нібито добрі від природи моральні засади людини. На зміну приватній 
власності мало прийти велике колективне виробництво із справедливим 
розподілом і забезпеченням потреб кожного; в такому суспільстві буде 
подолано різницю між розумовою і фізичною працею, суперечність між 
містом і селом. (Мрія ця супроводжувала і ще супроводжуватиме чи не всю 
історію світу, вона позачасова!) Щоб прийти до такого суспільства 
справедливості, до здійснення цієї споконвічної мрії людства, треба 
було всього лиш переконати людність у його перевагах. Одначе ця проста 
і зрозуміла справа чомусь не вдавалася, як не вдавалися і спроби 
жертовних мрійників подати власний приклад організацією 
соціалістичних комун або фаланстерів. Побудувати комунізм чи 
соціалізм в окремо взятій громаді виявилося неможливим.


За цих умов і постає необхідність в іншому, неутопічному, 
реалістичному підході, обгрунтованому не моральною риторикою, а 
економічно, ідеологічно, політично, з орієнтацією на докорінну, 
найпевніше силову, перебудову всього суспільства. А на яку соціальну 
силу можна покладатися?


1842 року з'являється праця німецького вченого-юриста Лоренца фон 
Штайна (1815--1890) «Соціалізм і комунізм у сьогоднішній Франції». Це було 
за шість років до європейських революцій 1848-го і до появи 
«Комуністичного маніфесту» Маркса й Енгельса. Лоренц фон Штайн був 
одним із перших, хто розробляв теорію пролетаріату (невдовзі, 1845-го, 
з'являється праця Маркса і Енгельса «Свята родина», присвячена цій 
темі, але вже із розробленням стратегії дій революційного 
пролетаріату). Штайн показав, що клас пролетарів неминуче з'являється 
внаслідок появи і діяльності класу капіталістів. За вільноринкової 
економіки свободу і права мають власники, а не робітники. Учений-юрист 
ліберальних переконань, з лівих молодогегельянців, він гадав, що певні 
правові норми могли б допомогти пролетарям урівнятися з 
капіталістами і, таким чином, соціальної справедливості можна було б 
досягти без революції, шляхом реформ.

\subsection*{Молодий Маркс}

По-іншому розумів справу Карл Маркс. Він також вийшов із 
молодогегельянства, але швидко переріс його (праця Маркса і Енгельса 
«Німецька ідеологія», 1845--1846, містила розгорнуту критику ідеалізму 
Гегеля й непослідовності матеріалізму Феєрбаха). В його особі 
визначилися і рідкісно поєдналися філософ, ідеолог, соціолог, 
політичний діяч, журналіст-пропагандист, а згодом і економіст. Він уже 
був відомий як автор численних журналістських публікацій та наукових 
праць, присвячених обговоренню політичних і філософських проблем, 
полеміці з іншими теоретиками й ідеологами, аналізові тогочасного 
буржуазного суспільства. Досвід практичної роботи, широкий світогляд, 
філософська системність критичного мислення і потужний інтелект дали 
йому можливість узагальнити й переосмислити здобутки німецької 
філософії, французьких і англійських соціалістичних та комуністичних 
теорій, англійської політекономії (як відомо, Енгельс називав три 
джерела марксизму: німецька філософія, англійська політекономія, 
праці французьких істориків) — і прийти до принципово нових 
висновків. Вони чітко викладені в «Комуністичному маніфесті» 
авторства Маркса й Енгельса, по суті полемічному щодо утопічних або 
ретроградних ідей попередників. Не реформи, не регулювання ринку, не 
обмеження приватної власності на засоби виробництва, а повна їх 
націоналізація й одержавлення способів розподілу, що — уявлялося — 
зробить неможливою експлуатацію людини людиною, приведе до 
ліквідації класів та створення безкласового суспільства, в якому 
вільний розвиток кожного буде умовою вільного розвитку всіх. Для 
цього пролетаріат має взяти владу в свої руки революційним шляхом. Хоч 
є у Маркса й неоднозначні думки на цю тему. Революція відбудеться тоді, 
коли пролетаріат стане більшістю в суспільстві, але тоді він може 
прийти до влади й мирним шляхом. Зокрема, припускалося, що в країнах, де 
вже утвердився парламентський лад (Англія, США), пролетаріат може 
прийти до влади, перемігши на виборах. Саме на це згодом орієнтувалися 
соціал-демократичні партії II Інтернаціоналу, але які цього так і не 
дочекалися.


Картина майбутнього соціалістичного суспільства та шляхи його 
творення в «Комуністичному маніфесті» не обговорені скільки-небудь 
конкретно. Це була не так наукова праця, як 
політично-пропагандистський документ узагальнювального характеру. 
До речі, не слід забувати, що «Комуністичний маніфест» Маркс і Енгельс 
написали не з власного задуму, а на прохання міжнародної робітничої 
організації «Союз Комуністів». І точна його назва —  «Маніфест 
Комуністичної Партії». Тобто: вже існував досить організований 
робітничий рух, який потребував ідеологічного осмислення, і цю 
потребу мали задовольнити Маркс і Енгельс, вибір на яких упав, 
звичайно ж, не випадково. Але факт, що від самого початку не вони 
інспірували організований робітничий рух (у чому їх подеколи 
«звинувачували»), а робітничий рух їх «інспірував». Інша річ, що вони 
своїми ідеями надали нової якості й потужної енергії цьому рухові. 


«Маніфест Комуністичної Партії» завершував першу фазу діяльності 
«молодого» Маркса, в якій означилися основні його ідеї, що дістануть 
дальший розвиток, але вже й тоді своєю сукупністю були новим (хоч і не 
беззаперечним, і не беззаперечно новим у всьому) словом у науці й стали 
відомі під назвою історичний матеріалізм. Це, зокрема, погляд на 
історію людства крізь призму класової боротьби, соціальних 
антагонізмів, які і є рушієм розвитку (тут Маркс поглибив поняття 
класової боротьби, введене в обіг французькими істориками). Це 
твердження про неминучість революційних змін у суспільствах 
унаслідок суперечності між зростанням засобів виробництва й 
інерційністю суспільних відносин, боротьби між класом експлуататорів 
і класом експлуатованих. Це погляд на суму економічних відносин у 
суспільстві як на той базис, на якому виростає складна світоглядна, 
юридична, політична, ідеологічна, художня та ін. надбудова, що 
змінюється із зміною базису (теза, яка зазнавала і зазнає спростувань, 
почасти і через її профанацію вульгаризаторами марксизму: сам Маркс 
мав на увазі не пряму підпорядкованість надбудви базисові, а складну й 
багатоетапну опосередкованість зв'язку між базисом і надбудовою, хоча 
точних меж між одним і другим він не визначив, як і не наголосив 
зворотного впливу надбудови на базис). Далі, це важлива думка про те, що 
старий лад не відходить доти, доки не вичерпає своїх можливостей, а 
новий не приходить йому на зміну, доки не визріли передумови для нього. 
Навколо цих та інших Марксових ідей десятиліттями точилися суперечки 
між марксистами й антимарксистами, між ортодоксами й ревізіоністами, 
догматиками й реформаторами тощо.


Критики Маркса здебільше не охоплювали сукупності його поглядів та 
їхньої діалектики, їхньої часом вільної гри в концерті Марксових ідей. 
Так, один із непримиренних його негаторів, видатний мислитель ХХ ст. 
Арнольд Дж.~Тойнбі у «Дослідженні історії» писав: «Німецький єврей 
Карл Маркс намалював у барвах, які запозичив з 
апокаліптичних видінь відкинутої ним традиційної релігії, 
страхітливу картину відокремлення пролетаріату й класової війни, яку 
він розв'яже. Величезне враження, яке справив цей марксистський 
матеріалістичний апокаліпсис на стільки мільйонів умів, почасти 
пояснюється політичною войовничістю Марксової схеми, бо хоч вона й 
становить ядро загальної філософії історії, вона також являє собою 
революційний заклик до збройної боротьби» (Арнольд Дж.~Тойнбі. 
Дослідження історії. Т.~1. -- К., 1995. -- С.~362). 


Мусимо визнати, що ущиплива іронія Тойнбі стосується Марксової 
риторики чи метафорики, але не зачіпає суті, змісту його послання. Так 
само небагато дає і ревний пошук юдаїстських коренів у марксизмі. 
«Маркс поставив богиню «Історична Необхідність» на місце Єгови, а на 
місце євреїв, богообраного народу, — внутрішній пролетаріат 
західного світу. Його Мессіанське Царство — це диктатура 
Пролетаріату, але грандіозна будівля Єврейського Апокаліпсису легко 
вгадується під цим благеньким укриттям» (там само, с.~391). Безперечно! 
Але розпізнавання цієї метафорики не є спростуванням марксизму, бо ця 
метафорика давно вже стала складником європейського мислення, — хіба 
що за всіма ідеалами комунізму доведеться бачити проповіді Христа і 
зводити справу до цього. Не випадково ж існує християнський соціалізм, 
був християнський комунізм, який заперечувано ще в «Комуністичному 
маніфесті». 


Власне, Тойнбі іронізує фактично з «молодого» Маркса, часів до 
написання «Капіталу», і, як історик, бере до уваги насамперед його 
узагальнені історіософські моделі, що не вкладалися в циклопічну 
будову тойнбівского «Дослідження історії», яке охоплювало не одне 
тисячоліття і в масштабі якого марксизм міг здаватися епізодом.


\subsection*{Марксів «Капітал»}


\ldots{}«Молодий» Маркс був філософом, ідеологом, політичним публіцистом, 
але ще не економістом. «Зрілий» Маркс, критично опанувавши досягнення 
сучасної йому економічної науки, насамперед англійської, розпочинає 
фундаментальне дослідження капіталізму як формації, 
капіталістичного способу виробництва, — типологічно, за Марксовим 
визначенням, відмінного від азійського, античного й феодального 
розвитком продуктивних сил та способом експлуатації людини людиною 
(це дуже важлива частина Марксового вчення), — його очевидних та 
прихованих механізмів, його «таємниць» і перспектив та меж. Так 
з'являється перший том його «Капіталу» — праці, що справила 
величезний вплив на розвиток людської думки і на політичну історію 
людства. (Свій задум Маркс не встиг довести до кінця, і другий та третій 
томи «Капіталу» готував Енгельс з Марксових чернеток.)  


Маркс показав, що капіталізм — принципово новий історичний і 
економічний феномен: у тому сенсі, що для нього характерний не обмін 
товарів за допомогою грошей, як це було на докапіталістичних етапах 
історії людства, а обмін грошей за допомогою товарів. Через це метою 
капіталіста є грошовий прибуток, заради якого він готовий на все. А що 
є джерелом прибутку? Як створюється додаткова вартість? Це, сказати б, 
головна «таємниця» капіталізму, без розкриття якої не можна мати 
адекватного бачення його і не можна опрозорити його міфологію. Маркс 
зосереджується на цій «таємниці» і створює теорію вартості, теорію 
заробітної платні і теорію додаткової вартості — найбільшої 
«таємниці» капіталізму, що відтак перестає бути таємницею. 
Скрупульозний Марксів аналіз показує, що робочий день трудівника 
складається з двох частин — праці, яка повернеться в його зарплатню, і 
додану працю. Тобто, додаткова вартість — це неоплачена частина праці 
робітника. Праця робітника — товар, але дивовижний товар, єдиний 
товар, який виробляє вартість, вищу за власну вартість! Звідси — 
прибутки капіталіста, які тим більші, чим вище співвідношення між 
доданою вартістю і заробітною платнею. Це співвідношення є 
\emph{нормою}\emph{ }\emph{експлуатації}.


Можна, мабуть. сказати, що до Маркса категорія \emph{праці} виступала у 
суспільній свідомості (принаймні у вульгарно-матеріалістичному 
мисленні або в моралістичному) узагальнено, нерозчленованою: як 
джерело усякого багатства. На таке уявлення впливала не в останню 
чергу й протестантська трудова мораль. Певні ілюзії існували і в 
німецькому робітничому русі. Так, філософ-робітник Іосиф Дицген 
вважав, що праця — це Рятівник, удосконалення праці зробить те, чого не 
зміг досягти жоден Визволитель. Натомість Маркс не тільки показав, 
кому реально дістаються плоди праці, а й проаналізував економічні 
«складові» праці, її місце в процесі експлуатації робітника. 


Марксова демістифікація капіталізму, розкриття його механізму 
експлуатації мали не тільки наукове й політичне значення, але не в 
останню чергу й етичне, гуманістичне. Вони дали потужний поштовх 
робітничому революційному рухові спочатку в Європі, а потім і в усьому 
світі. Вони змінили світ. Зрештою змінили і самий капіталізм. І коли 
кажуть, що капіталізм давно вже не той, про який писав Маркс, то треба 
додати, шо став він «не тим» (хоч і не зовсім «не тим») завдяки зокрема й 
Марксу: капіталізмові нічого не залишалося, як змінитися під потужним 
тиском робітничого революційного руху, профспілкового руху, впливу на 
суспільства комуністичних і соціал-демократичних партій, — зрештою, 
і, мабуть, не в останню чергу, внаслідок власних внутрішніх 
суперечностей як джерела руху і завдяки невикористаним резервам, про 
можливість яких говорив Маркс (пригадаймо його тезу про те, що старий 
лад ніколи не сходить зі сцени, поки не вичерпає всіх своїх 
можливостей, певна річ, і здатності до змін).


Тут не буду говорити про те, як інтерпретували Маркса його 
послідовники (сам він якось саркастично сказав, що не хотів би бути 
марксистом), як розвивав марксизм В.~І.~Ленін і як на місці марксизму 
утворилося нове вчення — \emph{марксизм-ленінізм}. Це окрема велика тема. 
Але нагадаю про те, що в перше десятиліття радянської влади над 
вивченням Маркса й Енгельса трудилися спеціально створені солідні 
наукові інституції, які публікували свої праці, відбувалися дискусії 
тощо. В московському Інституті Маркса-Енгельса під керівництвом 
філософів-марксистів Д.~Рязанова та І.~Рубіна досліджували 
першоджерела, публікували невідомі твори. Тобто, в автентичному 
марксизмі бачили джерело ідей, що могли допомогти зрозуміти реальні 
суспільно-політичні процеси, орієнтуватися в будівництві нового 
суспільства. Ще жили такі ілюзії. Історична школа М.~Покровського з 
марксистських позицій гостро викривала російський імперіалізм. В 30-і 
роки, коли Сталін утвердив свій спрощений (ще набагато спрощеніший, 
ніж ленінський) варіант марксизму, всі ці структури ліквідовано, 
провідні вчені, дослідники й популяризатори Маркса були репресовані 
то як меншовики, то як троцькісти, а єдиним законним речником 
марксизму зробився сам Сталін.


Не менш цікаве й те, що коїлося з Марксом-Енгельсом і з марксизмом 
після розвалу СРСР у нашій самостійній Україні. Їхні твори опинилися у 
спецфондах. Посилатися на них — моветон, ознака «совковості», 
відсталості мислення й антипатріотизму. Та про це далі. А спочатку про 
те, яке місце посідав Маркс у політичній свідомості видатних 
українців минулого, чи мав він якусь «причетність» до визвольної 
боротьби українців?


\subsection*{Від Франка до української діаспори}

Дивно було б припускати, що Маркс, який став «душею» всіх 
комуністичних і соціалістичних рухів, залишиться «чужим» для України, 
яка шукала вирішення своїх національних проблем, що були водночас і 
соціальними. Марксом поважно цікавилися М.~Драгоманов, М.~Павлик, І.~Франко,
Леся Українка. Іван Франко 1879 року зробив перший український 
переклад частини «Капіталу» (фрагмент друкується у цьому виданні). 
Професор Київського університету Микола Зібер, видатний економіст і 
соціолог, перший в Україні й Росії популяризував ідеї Маркса. Він 
зустрічався з Марксом і Енгельсом у Лондоні. 1885 року опублікував працю 
«Д.~Рикардо и К.~Маркс в их общественно-экономических исследованиях», 
яку Маркс читав і прихильно цитував. Учень М.~Зібера Сергій 
Подолинський також зустрічався з Марксом і Енгельсом та листувався з 
ними; він був автором перших марксистських праць — популярних брошур 
— українською мовою, в яких застосовував Марксові ідеї до аналізу 
проблем українського селянства: «Про хліборобство» (1874), «Парова 
машина» (1875), «Про багатство та бідність» (1876), «Життя і здоров'я людей 
на Україні» (1879), «Ремесла і фабрики на Україні» (1880) та ін. Вони 
друкувалися, зрозуміло, в Галичині, але розповсюджувалися по всій 
Україні зусиллями Драгоманова, Павлика і київської «Громади». У 
Львові й Чернівцях 1892 року друкуються брошурами українські переклади 
з Маркса й Енгельса, а перший український переклад «Комуністичного 
маніфесту» виходить 1902 року у Львові. Цікавий етап у розповсюдженні 
марксистських ідей у Російській імперії — це розквіт т.~зв. 
«легального марксизму», найяскравіше представленого у Києві: В.~Кістяковський,
 С.~улгаков, М.~Ратнер, М.~Туган-Барановський (пізніше 
виступав з критикою Маркса). З «легальним марксизмом» уперто боровся 
Ленін, який бачив у ньому джерело ревізіонізму.


Якщо на перших порах популяризацією марксизму захоплювалися ліберали 
й народники, то з розвитком в Україні соціал-демократичного руху він 
стає елементом партійних програм. До марксизму апелювала створена 1905 
року на основі РУПу (Революційної Української Партії) — УСДРП 
(Українська Соціал-Демократична Робітнича Партія), визначними діячами 
якої були В.~Винниченко, С.~Петлюра, Д.~Антонович, Л.~Юркевич,
М.~Ковальський, М.~Тимченко та~ін. При цьому україноцентричні 
соціал-демократи звертаються до марксизму для висвітлення 
колоніального становища України та обстоювання ідеї національного й 
соціального визволення України. Видатним науковцем і політичним 
діячем цього гатунку був Микола Порш, один із чільних діячів РУПу та 
УСДРП, міністр в урядах УНР, соціолог і статистик, автор праць «Із 
статистики України» (1907), «Пролетаріат на Україні» (1907), «Про автономію 
України» (1907), «Автономія України і соціал-демократія» (1917), «Україна і 
Росія на робітничому ринку» (1918), «Україна в державному бюджеті Росії» 
(1918) та ін. Він же переклав українською мовою перший том «Капіталу» 
Маркса (не був виданий). Напередодні першої світової війни в Україні 
зростає мережа соціал-демократичної преси: «Дзвін» у Києві, «Воля», 
«Вперед», «Робітник», «Наш голос» — у Львові. Одним із організаторів і 
активних публіцистів у них був Володимир Левицький, автор книжок 
«Нарис розвитку українського робітничого руху в Галичині» (1914), 
«Царская Россия и украинский вопрос» (1919), «Соціалістичний 
інтернаціонал і поневолені народи» та ін. Українські марксисти 
зберігали європейське обличчя марксизму і відмежовувалися від 
марксизму ленінського. Про такий «лібералізований» марксизм можна 
говорити і стосовно Володимира Винниченка та інших лідерів УСДРП. 


Під час Світової війни українська соціал-демократична преса Галичини 
(в підросійській Україні вона була заборонена) не просто осуджувала 
варварське кровопролиття, а викривала загарбницький, 
імперіалістичний характер війни. Глибокий аналіз її причин з 
марксистських позицій дали В.~Левицький та М.~Залізняк. 


Тут слід нагадати, що в цей самий час значна частина європейських 
соціал-демократів, як відомо, розбіглася по «національних квартирах» 
і так чи інакше ставала по боці «своїх» урядів. 


Антивоєнна й правдиво інтернаціоналістична позиція українських 
соціал-демократів парадоксальним чином обернулася проти них у 
визвольну добу 1918--1919 рр., коли вони відігравали провідну роль у 
Центральній Раді та в урядах УНР.~Як соціалісти і марксистські 
налаштовані політики, вони сподівалися на розуміння і мирну 
домовленість із соціалістами й марксистами «великого братнього» 
народу. Але виявилося, що то зовсім інакший «соціалізм» і зовсім 
інакший «марксизм». Ставлення до загрози російсько-більшовицької 
агресії в різних колах УСДРП було різне, і це спричинило внутрішню 
боротьбу й розколи, що також додалося до причин поразки. 


Ще у складнішому становищі опинилися українські соціал-демократи 
після приходу більшовиків в Україну. Вони не могли ігнорувати того 
факту, що більшовицькі гасла неабияк впливали на українські маси, а 
Радянська Росія немовби очолила світовий революційний і 
соціалістичний рух, за яким бачилося майбутнє. Невблаганний 
історичний процес диктував необхідність стратегії і тактики, 
відповідної до нових і непередбачуваних умов, здатної забезпечити 
можливість впливати на події і не бути відкинутими на задвірки 
історії. Тут неминучими стали нові незгоди й розколи. Частина 
вчорашніх соціал-революціонерів та соціал-демократів обирає 
співпрацю, на певних умовах, з більшовиками, сподіваючись таким чином 
впливати на характер перетворень і обстоювати українські національні 
інтереси, як вони їх уявляли. На перших порах ці сподівання почасти 
справджувалися, оскільки більшовики потребували підтримки досить 
сильної партії боротьбистів і йшли на деякі поступки. Але в міру 
зміцнення своєї влади вони дедалі більше утискували своїх 
ситуативних союзників. Тим часом і серед більшовиків, у тодішній 
КП(б)У, були, хоч і не переважальні, «націонал-ухильницькі» (на 
офіційному парткерівному жаргоні) сили, пов'язані зі своїм народом і 
відповідальні за його долю принаймні в тому розумінні, що хотіли 
бачити його рівноправним з іншими в уявлюваному комуністичному 
суспільстві, яке мало привести до вільного розвитку всіх народів. 
Зрештою, сума вагомих чинників — спротив українського села, опозиція 
національної інтелігенції, наявність різних ідеологічних елементів 
та різного бачення історичної перспективи у самій партії, слабкість 
її позицій в українському та інших національних суспільствах, — а не в 
останню чергу й претензія на роль маяка антиколоніальних революцій на 
Сході, для яких комуністична Україна мала стати переконливим і 
звабливим прикладом, — змусила більшовицьку партію, на виконання 
нового курсу Леніна, вдатися до політики «українізації», ширше —
«коренізації», оскільки йшлося й про інші колонізовані народи. Тобто, 
це був пошук надійного опертя в неросійських народах. В Україні ця 
політика пов'язувалася з лідерами націонал-комунізму, давніми 
партійними діячами Олександром Шумським та Миколою Скрипником, які 
прагнули дати їй марксистське обгрунтування. Відповідні дослідження 
проваджувано в Українському Інституті Марксизму та Ленінізму (УІМЛ, 
1922--1931). Професійна марксистська методологія з різною мірою успіху 
впроваджувалася в різних галузях суспільних наук. Серед яскравих 
представників цього штибу мислення можна назвати А.~Річицького, 
одного із засновників УКП (Української Комуністичної Партії), 
сподвижника М.~Скрипника, наукового працівника УІМЛ, автора праць на 
літературні й Марксівські теми, редактора першого видання «Капітала» 
Маркса українською мовою (1927--1929); філософа-марксиста, поета, 
публіциста і літературознавця В.~Юринця; історика М.~Яворського, школа 
якого працювала до погрому 30-х років. 


Доля УІМЛ була такою ж, як і московського Інституту Маркса-Енгельса, 
хіба що набагато трагічнішою, бо стала частиною тотальних репресій 
проти українських наукових і культурних установ та їхніх діячів — під 
моторошний акомпанемент голодомору. 


Зрозуміло, що після цього будь-які серйозні роботи в галузі марксизму 
стали неможливими і втратили сенс, черга реорганізацій закінчилася 
створенням Інституту історії партії при ЦК КПУ — як філіалу Інституту 
марксизму-ленінізму при ЦК КПРС. 


Марксистська фразеологія стала способом придушення самостійного 
мислення, і не дивно, що на час розвалу СРСР марксизм був у суспільстві 
остаточно скомпрометований, хоча долинали ще якісь відгомони 
європейського неомарксизму й були спроби створювати нелегальні 
робітничі гуртки з орієнтацією на «справжній марксизм» (про це ми 
могли довідатися з великим запізненням, у 90-і роки, — після того, як СБУ 
опублікувала секретні матеріали провокаційної кагебістської «Справи 
,,Блок``», по якій «проходили» не тільки «українські буржуазні 
націоналісти», а й широкий спектр інших «підривних елементів»). 


Прикметно, що з настанням горбачовської «гласності» та після здобуття 
Україною незалежності Маркса у нас зовсім не стало. Його уникали, мов 
якогось «совєтського» маркера, і офіціоз, і рухівська опозиція. Щодо 
офіціозу зрозуміло: йому з Марксом не було про що говорити, та й 
некомфортно. А \mbox{РУХ}ом він залишився непрочитаний. На мій погляд, велика 
помилка \mbox{РУХ}у й одна з причин його досить швидкого занепаду — 
абсолютизація національного питання й невміння розкрити всю 
конкретність його пов'язаності з соціальним. Це саме те, чого можна 
було повчитися у Маркса. Але Маркс вважався завербованим у офіційну 
совєтчину (пригадую: коли я в своєму самвидавському опусі 
«Інтернаціоналізм чи русифікація» рясно посилався на погляди Маркса 
й Енгельса, як і Леніна, з національного питання, на його листування, в 
якому фактично заперечується його власна, з «Комуністичного 
маніфесту», теза про те, що пролетаріат не має вітчизни, і говориться 
протилежне: щоб успішно вести свою боротьбу, пролетаріат повинен 
насамперед визволити чи об'єднати свою вітчизну, — багато хто навіть 
із прихильних до мене були подивовані, часом і неприємно, або 
сприймали це як курйоз чи риторичний прийом). Можна зрозуміти: Маркс 
такий далекий, а українське національне питання таке пекуче, що 
багатьом воно здавалося самодостатнім. Серед рухівців панувало 
стихійне переконання: національне — головне, соціальне — 
підпорядковане. Вся історія людства, м'яко кажучи, не підтверджувала 
цього, але гіркі розчарування варті того, щоб їх пережити самому. 
Вкотре наочно виявилося, що в дилемі національне-соціальне (до якої, 
власне, і не повинен допускати розумний політик!) національне стає 
пріоритетом для героїчної меншості, а соціальне — для решти. Героїчна 
меншість може творити революції, але парламенти обирає негеролїчна 
більшість. Тож український виборець у масі своїй голосував не за 
безкорисливих патріотів української мови (або й історії), а за хижих 
демагогів, які обіцяли швидке і фантастичне полагодження житейських 
проблем. І в захисники трудящих на політичній арені перевдягалися 
їхні найжорстокіші експлуататори, захребетники — як оті донецькі 
вугільно-металургійні барони, які невеличку частину прибутків, 
здертих з каторжної праці робітників, витрачали на фінансування 
організованих ними експедицій цих робітників під Верховну Раду чи 
Кабмін, для стукання шахтарськими касками, — так, ніби це українські 
урядовці, а не вони, донецькі барони, винні в несплаті заробітку і в 
жахливому занедбанні техніки безпеки та постійних катастрофах. 


Ні, не став РУХ реальним захисником трудового люду. Як не стали ним і 
профспілки, спосіб організації яких, структура і зміст роботи, права і 
можливості залишаються далекими від прозорості. Може, я помиляюся, але 
мені здається, що ні в кого немає ніякої концепції — ані марксистської, 
ані неомарксистської, ані просто немарксистської, ані навіть 
антимарксистської — захисту трудящих за умов нашого дикого 
капіталізму. Ані концепції, ані продуманих ідей, ані якоїсь — 
політичної чи моральної — гуманістичної настанови, Про це останнє 
кажу тому, що автентичний марксизм — насамперед гуманістичне вчення! 
Воно має глибоке коріння в історії людських борінь за справедливість, 
виразно перегукується з етикою шукань істини, пропонує свого роду 
соціологію пізнання. Як філософ і письменник (у широкому значенні 
слова), Маркс не чужий феноменології, в нього знаходять елементи 
екзистенціалізму. Може, найважливіше чи, принаймні, найцікавіше для 
гуманітаристики в «Капіталі» —це дослідження товарного фетишизму й 
відчуження праці, які фундаментальним чином діють у напрямку 
збіднення світу людини, її знелюднення. Це чинники універсальні, яким 
людство ще не знайшло противаги і не знати, чи колись знайде. Тут, може, 
найважливіші з Марсових відкриттів, і вони варті не меншої уваги, ніж 
його суто економічні осягнення. 


Ще\ldots{} Про Маркса часто говорили й писали, говорять і пишуть, що він 
нібито зневажав духовну творчість або ставився до неї догматично. Як 
на мене, це прикре непорозуміння. Маркс добре знав історію культури, 
його твори не бідні на апеляцію до її фактів та на глибокі думки про 
літературу, великих письменників минулого і сучасників. А прочитайте 
його листування, прилучіться до обсягу його естетичних переживань і 
читацьких реакцій! Отут іще один незнаний нам Маркс!


\ldots{}На закінчення варто додати, що в той час, як в Україні «набридлий» 
за радянські часи Маркс для науковців перестав бути актуальним (хоча б 
для цитування, про вивчення й мови немає), а політологам і публіцистам 
було не до Маркса (та й попиту ніякого), — «націоналісти», а власне 
інтелектуали в українській діаспорі про нього не забували. Крім тих, 
хто поважно студіював Маркса (Р.~Роздольський, відомий як один із 
кращих знавців економічних і національних поглядів Маркса, в 
молодості один із організаторів КПЗУ, якийсь час співробітник, під 
керівництвом Д.~Рязанова й І.~Рубіна, московського Інституту 
Маркса-Енгельса, після його ліквідації працював у архівах Відня, 
Львова, Кракова, у 1942--1945 — в'язень німецьких концтаборів, від 1945 — у 
Детройті, автор багатьох досліджень про Маркса, зокрема й 
неомарксистського тлумачення «Капіталу» — от така дивовижна людина, 
варта біографічного роману або кіно- чи телесеріалу; Панас Феденко, 
один із організаторів УСДРП та лідер її в еміграції; історик і 
публіцист В.~Голубничий; були ті, хто розумів його роль у розвитку 
суспільних наук та в політичній історії людства і знаходив йому 
належне місце в системі своїх ідеологічних оцінок постатей і 
феноменів сучасності, як-от Іван Лисяк-Рудницький. Можна говорити про 
певну пов'язаність з марксизмом Івана Багряного та інших ідеологів 
УРДП — Української Революційно-Демократичної Партії, яка за складних 
умов політичного розбрату в еміграції мужньо обстоювала ідею єдності 
українців на основі не «філологічного паріотизму», а спільності 
корінних інтересів соціальної справедливості й прагнення до свободи. 
Ідеї Івана Багряного могли б уберегти український політикум, 
насамперед рухівців та пізніших «правих», від прикрих помилок та 
неуваги до соціальної сторони української проблематики, — але, на 
жаль, вони не знайшли належного відгуку та й просто місця у вузькому 
кругозорі наших «патріотів» (не кажучи вже про байдужих до України). 


Окремо слід сказати про солідну працю видатного українського 
історика в США (власне, й американського історика) Романа Шпорлюка 
«Націоналізм і комунізм» (Оксфорд, 1988; український переклад Георгія~Касьянова — К., «Основи», 1998). Праця має підзаголовок: «Карл Маркс проти 
Фрідріха Ліста», але фактично Марксова полеміка з німецьким 
економістом і теоретиком націоналізму Лістом — це лише сюжетний 
стрижень праці, який обростає великим фактичним матеріалом та 
інтелектуальними розважаннями автора й лектурою на тему взаємодії 
марксизму й націоналізму як двох великих проектів модернізації 
суспільств — проектів принципово суперечних один одному, але й 
суголосних багато в чому та навіть «повчальних» один до одного — аж 
такою мірою, що в процесі суспільного розвитку ХIХ--ХХ ст. марксизм 
помітно «націоналізувався», а націоналізм — почасти «омарксизмився».


Праця Романа Шпорлюка, на мій погляд, особливо важлива для українців 
тим, що виводить уявлення про націоналізм з провінційних вимірів у 
глобальні, знайомить нашого читача з інтерпретацією націоналізму 
широким колом сучасних європейських істориків, соціологів, філософів; 
те ж саме стосується і марксизму, якого українське суспільство — 
виглядає — так і не освоїло, хоч у ньому є ще немало інтелектуальних 
резервів для нас. Вони ждуть свого «будителя». І, може, якимось 
імпульсом стане сподівана публікація українського перекладу 
«Капіталу». 

\begin{flushright}
  \emph{Іван Дзюба}
\end{flushright}

{\small 22 липня 2018 р.}

\parcont{}  %% абзац починається на попередній сторінці
\index{iii1}{0061}  %% посилання на сторінку оригінального видання
частини упредметненої в ньому праці, яку він оплатив. Додаткова
праця, що міститься в товарі, нічого не коштує капіталістові,
хоч робітникові вона цілком так само коштує праці, як
і оплачена, і хоч вона цілком так само, як і оплачена, створює
вартість і входить у товар як вартостетворчий елемент. Зиск
капіталіста постає з того, що він має для продажу щось, чого
він не оплатив. Додаткова вартість, відповідно зиск, складається
саме з надлишку товарної вартості понад витрати її виробництва,
тобто з надлишку всієї суми праці, вміщеної в товарі, понад
вміщену в ньому оплачену суму праці. Таким чином додаткова
вартість, звідки б вона не виникала, є надлишок понад увесь
авансований капітал. Отже, цей надлишок стоїть у такому відношенні
до всього капіталу, яке виражається дробом \frac{m}{K}, де
$К$ означає весь капітал. Таким чином одержуємо \emph{норму зиску}
\frac{m}{K} \deq{} \frac{m}{c+v}, у відміну від норми додаткової вартості \frac{m}{v}.

Величина додаткової вартості у її відношенні до змінного
капіталу зветься нормою додаткової вартості; величина додаткової
вартості у її відношенні до всього капіталу зветься нормою зиску.
Це два різні виміри тієї самої величини, які в наслідок ріжниці в
масштабах виражають одночасно різні пропорції або відношення
однієї і тої самої величини.

З перетворення норми додаткової вартості в норму зиску
слід виводити перетворення додаткової вартості в зиск, а не
навпаки. І справді, вихідним пунктом історично була норма зиску.
Додаткова вартість і норма додаткової вартості є, відносно, те
невидиме і суттєве, що треба розкрити, тимчасом як норма
зиску, а тому й така форма додаткової вартості як зиск виявляються
на поверхні явищ.

Щодо окремого капіталіста, то ясно, що єдине, що його
інтересує, це відношення додаткової вартості або надлишку вартості,
ради якого він продає свої товари, до всього капіталу,
авансованого на виробництво товару; тимчасом як певне відношення
цього надлишку до окремих складових частин капіталу
і його внутрішній зв’язок з цими складовими частинами не тільки
не інтересує його, але він ще й заінтересований в тому, щоб
оповити туманом це певне відношення і цей внутрішній зв’язок.

Хоча надлишок вартості товару понад витрати його виробництва
виникає в безпосередньому процесі виробництва, але реалізується
він тільки в процесі циркуляції, — і він тим легше набуває видимості
виникнення з процесу циркуляції, що в дійсності, серед
конкуренції, на дійсному ринку, від ринкових відносин залежить,
чи реалізується цей надлишок, чи ні, і в якому розмірі. Тут
немає потреби пояснювати, що коли товар продається вище
або нижче його вартості, то має місце тільки інший розподіл
додаткової вартості, і що цей інший розподіл, змінене
відношення, в якому різні особи ділять між собою додаткову вартість,
\index{iii1}{0062}  %% посилання на сторінку оригінального видання
нічого не змінює ні в величині, ні в природі додаткової
вартості. В дійсному процесі циркуляції не тільки відбуваються
перетворення, які ми розглянули в книзі II, але вони збігаються
з дійсною конкуренцією, з купівлею і продажем товарів вище
або нижче їх вартості, так що для окремого капіталіста реалізована
ним самим додаткова вартість залежить так само від
взаємного ошукування, як і від безпосередньої експлуатації
праці.

В процесі циркуляції поряд робочого часу починає діяти час
циркуляції, який цим самим обмежує масу додаткової вартості,
яку можна реалізувати за певний період. На безпосередній
процес виробництва впливають визначально ще й інші моменти,
які виникають з циркуляції. І те і друге, безпосередній
процес виробництва і процес циркуляції, постійно переходять
один в один, пронизують один одного, і тим самим постійно
перекручують свої характерні відмінні ознаки. Виробництво додаткової
вартості, як і вартості взагалі, набуває в процесі циркуляції,
як показано вище, нових визначень; капітал перебігає
круг своїх перетворень; нарешті, він вступає з свого, так би
мовити, внутрішнього органічного життя в зовнішні життьові
відносини, у відносини, де один одному протистоять не капітал
і праця, а з одного боку капітал і капітал, з другого боку
індивіди знов таки просто як покупці і продавці; час циркуляції
і робочий час перехрещуються на своєму шляху, і таким
чином здається, ніби вони обидва в однаковій мірі визначають
додаткову вартість; та первісна форма, в якій протистоять один
одному капітал і наймана праця, замасковується в наслідок втручання
відносин, які, як здається, незалежні від неї; сама додаткова
вартість здається не продуктом привласнення робочого часу,
а надлишком продажної ціни товарів понад витрати їх виробництва,
в наслідок чого витрати виробництва легко можуть здаватися
дійсною вартістю (valeur intrinsèque) товарів, так що зиск
здається надлишком продажної ціни товарів понад їх імманентну
вартість.

Правда, під час безпосереднього процесу виробництва природа
додаткової вартості постійно доходить до свідомості капіталіста,
як це вже при розгляді додаткової вартості показала
нам його жадоба до чужого робочого часу і~\abbr{т. д.} Але: 1)~сам
безпосередній процес виробництва є тільки минущий момент,
який постійно переходить у процес циркуляції, як і цей останній
переходить у нього, так що ясніше чи туманніше проблискуюча
в процесі виробництва догадка про джерело здобутого у ньому
баришу, тобто про природу додаткової вартості, щонайбільше
виступає як момент рівноправний з тим уявленням, ніби реалізований
надлишок походить від руху, який не залежить від
процесу виробництва і виникає з самої циркуляції, отже, руху,
який належить капіталові незалежно від його відношення до
праці. Адже навіть сучасними економістами, як от Рамсей,
\parbreak{}  %% абзац продовжується на наступній сторінці

\parcont{}  %% абзац починається на попередній сторінці
\index{iii1}{0063}  %% посилання на сторінку оригінального видання
Мальтус, Сеніор, Торренс і~\abbr{т. д.}, ці явища наводяться безпосередньо
як докази того, ніби капітал просто в своєму речовому
існуванні, незалежно від того суспільного відношення до
праці, в якому він саме й стає капіталом, є, поряд з працею
і незалежно від праці, самостійним джерелом додаткової вартості.
2)~Під рубрикою витрат, куди належить заробітна плата
цілком так само, як і ціна сировинного матеріалу, зношування
машин і~\abbr{т. д.}, видушування неоплаченої праці здається тільки
заощадженням на оплаті одного з тих предметів, які входять
у витрати, тільки меншою платою за певну кількість праці;
цілком так само, як відбувається заощадження, коли дешевше
купують сировинний матеріал або зменшують зношування машин.
Таким чином видушування додаткової праці втрачає свій
специфічний характер; його специфічне відношення до додаткової
вартості затемнюється; і цьому затемнінню дуже допомагає
і дуже його полегшує, як показано в книзі І, відділ VI,
представлення вартості робочої сили в формі заробітної плати.

Через те що всі частини капіталу однаково здаються джерелами
надлишкової вартості (зиску), то капіталістичне відношення
містифікується.

Той спосіб, яким додаткова вартість за допомогою переходу
через норму зиску перетворюється в форму зиску, є, однак,
тільки дальший розвиток того переплутання суб’єкта і об’єкта,
яке відбувається уже в процесі виробництва. Вже там ми бачили,
як усі суб’єктивні продуктивні сили праці здаються продуктивними
силами капіталу. З одного боку, вартість, минула праця,
яка панує над живою працею, персоніфікується в капіталісті;
з другого боку, навпаки, робітник виступає просто як предметна
робоча сила, як товар. З цього перекрученого відношення неминуче
виникає вже в самому простому відношенні виробництва
відповідне перекручене уявлення, перенесена з цього відношення
свідомість, яка розвивається далі в наслідок перетворень і модифікацій
власне процесу циркуляції.

Спроба представити закони норми зиску безпосередньо як закони
норми додаткової вартості, або навпаки, є цілком хибна, як
у цьому можна пересвідчитися на прикладі школи Рікардо. В голові
капіталіста, звичайно, ці закони не розрізняються. У виразі \frac{m}{K}
додаткова вартість вимірюється вартістю всього капіталу, авансованого
на її виробництво і почасти в цьому виробництві цілком спожитого,
а почасти тільки застосованого в ньому. Відношення \frac{m}{K} в
дійсності виражає ступінь зростання вартості всього авансованого
капіталу, тобто, взяте відповідно до його раціонального, внутрішнього
зв’язку і природи додаткової вартості, воно показує,
яке є відношення величини, на яку змінився змінний капітал, до
величини всього авансованого капіталу.


\index{iii1}{0064}  %% посилання на сторінку оригінального видання
Величина вартості всього капіталу сама по собі не стоїть
у будь-якому внутрішньому відношенні до величини додаткової
вартості, принаймні не стоїть безпосередньо. Щодо своїх речових
елементів весь капітал мінус змінний капітал, отже, сталий капітал,
складається з речових умов здійснення праці, з засобів праці
і матеріалу праці. Для того, щоб певна кількість праці реалізувалась
у товарах і, отже, утворила вартість, потрібна певна
кількість матеріалу праці і засобів праці. Залежно від особливого
характеру додаваної праці існує певне технічне відношення
між масою праці і масою засобів виробництва, до яких повинна
бути додана ця жива праця. Отже, остільки існує також певне
відношення між масою додаткової вартості або додаткової праці
і масою засобів виробництва. Якщо, наприклад, час, необхідний
для виробництва заробітної плати, становить 6 годин на день,
то робітник мусить працювати 12 годин, щоб дати 6 годин додаткової
праці, щоб створити додаткову вартість у 100\%. Він
споживає за ці 12 годин удвоє більше засобів виробництва, ніж
за ці 6 годин. Але від цього додаткова вартість, яку він додає
за 6 годин, зовсім не стає в будь-яке безпосереднє відношення
до вартості засобів виробництва, спожитих за ці 6 чи навіть
за ці 12 годин. Ця вартість тут не має ніякого значення; ідеться
тільки про технічно необхідну масу. Чи сировинний матеріал або
засоби праці дешеві, чи дорогі, це не має ніякого значення;
аби тільки вони мали потрібну споживну вартість і були наявні
в технічно встановленій пропорції до тієї живої праці, яку треба
поглинути. Однак, якщо я знаю, що за одну годину перепрядається
$х$ фунтів бавовни, які коштують $а$ шилінгів, то я, звичайно,
знаю і те, що за 12 годин перепрядається 12 $х$ фунтів
бавовни \deq{} 12 $а$ шилінгам, і тоді я можу обчислити відношення
додаткової вартості до вартості цих 12, так само як і до вартості
цих 6. Але відношення живої праці до \emph{вартості} засобів
виробництва тут привходить лиш остільки, оскільки $а$ шилінгів
служать назвою для $х$ фунтів бавовни; бо певна кількість бавовни
має певну ціну, а тому й навпаки, певна ціна може служити
показником певної кількості бавовни, поки ціна бавовни
не зміниться. Якщо я знаю, що для того, щоб привласнити 6 годин
додаткової праці, я повинен примушувати працювати 12 годин,
отже, мушу мати напоготові бавовни на 12 годин, і якщо я знаю
ціну цієї потрібної для 12 годин кількості бавовни, то посередньо
існує відношення між ціною бавовни (як показником необхідної
кількості) і додатковою вартістю. Навпаки, з ціни сировинного
матеріалу я ніколи не можу зробити висновок про масу сировинного
матеріалу, яка може бути перепрядена, наприклад, за
одну годину, а не за 6. Отже, немає ніякого внутрішнього, необхідного
відношення між вартістю сталого капіталу, — а тому
і між вартістю всього капіталу ($= c \dplus{} v$) і додатковою вартістю.

Якщо норма додаткової вартості відома і величина додаткової
вартості дана, то норма зиску виражає не що інше, як
\parbreak{}  %% абзац продовжується на наступній сторінці

\index{franko}{0065}

\looseness=1
Перший крок перевороту, що поклав основу капталістичній продукції, припадає в послідній третині \RNum{15} і
в першій чверти \RNum{16} віку. Тоді скасовано феодальне дворацтво, котре, як справедливо замічає Джемс
Стюерт, „злягло  всі хати і двори безхосенно“. Через те викинено масу голих пролєтаріїв на
робучий торг. Хоть королівська власть, що й сама виросла з буржуазного розвитку, намагаючи до
неограниченого панованя, силою скасувала те великопанське дворацтво, то прецінь вона не була єдиною
причиною нового перевороту. Ні, в упертім опорі протів королівства та
парляменту витворили великі пани-феодали далеко більшу масу пролєтаріяту, прогонюючи силою
хліборобів з ґрунту і посідлости, хоть хлібороби мали до тих ґрунтів більше право, ніж вони, і
забираючи для себе громадські ґрунти. Беспосередний товчок до того в Англії дав іменно росцвіт
фляндрійської вовняної мануфактури і звязане з ним підскоченє цін вовни. Стара феодальна шляхта
вигибла в великих феодальних війнах, а нова шляхта — се були діти свого часу, для котрих гроші були
силою понад всі сили. З вірного поля пасовиська для овець! — се став тепер їх загальний оклик.
Гаррізен в своїй „Description of England. Prefixed to Holinshed’s Chronicles“ описує, як
вивласнюванє дрібних ґаздів руйнує край. „Але що нашим великим самозванцям до того?“ Мешканя ґаздів
та коттеджі робітників валят вони силою або прогнавши людей лишают пустками. „Коли перездримо
давнійші інвентарі кождої домінії, то побачимо, що незлічимі хати та дрібні ґаздівства пощезали, що
ґрунт годує далеко меньше люда, що богато міст підупало, хоть деякі нові підносятся\dots{} Мож би
чимало наросповідатися про місточка та села, зруйновані для того, щоб було місце на толоки для
овець; тілько самотні панські двори стоят серед тих толок“. Правда, наріканя тих старих літописів
усе пересаджені, але вони досадно малюют те вражінє, яке на самих сучасників робив переворот
обставин продукційних. Порівнанє між письмами канцлєрів Фортеске і Томаса Моруса вказує наглядно
пропасть між \RNum{15} а \RNum{16} віком. „Із золотого віку — каже справедливо Зорнтон — попали англійські
робітники без ніяких перехідних ступнів прямо в зелізну“.

\looseness=1
Праводавство злякалось сего перевороту. Воно не стояло ще на такім високім ступни цівілізації, де
„богацтво народне“, т. є. богацтво капіталістів і безграничне висисанє та зубожінє маси люду
становит верх премудрости
політичної. В своїй історії Генріха VII каже Бекон: „В тім часі (1489) посипалися скарги на то, що
вірне поле перемінюєсь в пасовиська, котрих лехко може дозирати кілька пастухів. Ґрунти, що вперед
виарендовувались на кілька літ, на доживотну або щорічну умову, тепер зіллято разом
\index{franko}{0066}
с панськими. Се підкопало добробуток люду, а через те й міста, церкви, десятини\dots{} Щоб зарадити
тому лиху, проявили король і парлямент дивну на ті часи мудрість\dots{} Вони видали право протів того
обезлюднюючого край загарбуваня громадських ґрунтів (depopulating inclosures) і невідлучної
від него обезлюднюючої ґосподарки толочної (depopulating pastures)“. Оден акт Генріха VII з р.~
1489 заказує руйнувати хліборобські хати, до котрих належит що найменьше 20 екрів ґрунту. Генріх
VIII відновив той самий указ. Говорится там між їншим, що „многі аренди і огромні отари, особливо
овець, нагромаджуются в немногих руках, через що дохід
з ґрунту дуже вбільшився, а рільництво дуже підупало, церкви і хати повалено, дивовижні маси народа
стали неспосібні вдержувати себе і свої родини“. Указ наказує затим відбудовувати повалені хутори,
означує, кілько має бути вірного поля в стосунку до овечих толок і т. д. Їнший акт з р.~1533
жалуєсь, що деякі властивці мают по \num{24000} овець, і ограничує їх число на 2000\footnote{
В своїй „Утопії“ говорит Томас Морус про дивовижний край, де
„вівці їдят людей“.
}. Наріканя народа і
праводавство протів вивласнюваня дрібних арендаторів та хліборобів, що почалось від Генріха VII і
трівало зо 150 літ
— не помогли нічо. Чому не помогли, пояснює нам Бекон, сам того не знаючи. „Акт Генріха~VII, — каже
він в своїх „Essays, civil and moral“, Sect. 20, — був глибоко і дивно обдуманий. Він утворив
сільскі ґаздівства і хліборобські доми певного нормального розміру, т. є. вдержав для них таку
пропорцію ґрунту, котра давала їм змогу плодити на світ підданих доста заможних і не придавлених
нуждою, так що плуг був в руках властивців, а не наємників\footnote{
Бекон пояснює далі звязок між свобідним, заможним селянством
а доброю інфантерією. „Се була дивно важна річ для сили і мужности
королівства — мати аренди достаточного розміру, щоб дільних мужів
забеспечити від нужди і велику часть ґрунту краєвого запевнити в посіданє джоменам, т. є. людім
середної заможности між шляхтою а халупниками (cottagers) та наймитами. Бо се загальна думка
найліпших знавців воєнного діла\dots{} що головна сила армії, се інфантерія або піхота. Але щоб
витворити добру інфантерію, тре людей вихованих не в притиску ані в нужді, але свобідно і в певній
заможности. Коли затим яка держава вросте переважно в шляхту та делікатне панство, а хлібороби та
ратаї зійдут на простих зарібників та наймитів або халупників, т. є. жебраків з власною хатою, то
така держава може мати добру кінницю, але доброї піхоти не буде мати. Се видно в Італії і Франції і
деяких других заграничних краях, де справді все або шляхта або нужденні зарібники\dots{} Дійшло там до
того, що ті краї мусят уживати наємного зброду Швейцарів та др. для своєї піхоти: відти то й пішло,
що ті держави мают богато людий, а мало вояків“. („The Reign of Henry VII“ і т.~д.).
}. А між
\parbreak{}


\index{iii2}{0067}  %% посилання на сторінку оригінального видання
Про розділ банку на два відділи та про надмірне піклування в справі забезпечення
розміну банкнот Тук висловлюється так перед C.~D 1848/57:

Більші коливання рівня проценту в 1847 році проти років 1837 та 1839
завдячували лише розділові банку на два відділи. (3010). — Забезпечености банкнот
не було порушено ані в 1825, ані в 1837 та 1839 роках. (3015). — Попит
на золото в 1825 році мав на меті тільки заповнити порожняву, що утворилася
в наслідок цілковитого дискредитування однофунтівок-банкнот провінціяльних
банків; цю порожняву можна було заповнити тільки золотом, поки Англійський
банк не почав теж видавати однофунтівки-банкноти. — (3022). В листопаді та
грудні 1825 року не було ані найменшого попиту на золото для вивозу. (3023).

«Щодо дискредитування банку всередині країни та закордоном, то припинення
виплати дивідендів та вкладів мало б куди тяжчі наслідки, ніж припинення
оплати банкнот (3028)».

«3035. Чи не сказали б ви, що кожна обставина, яка, кінець-кінцем,
загрожує небезпекою розмінові банкнот, могла б в момент комерційного пригнічення
породити нові та серйозні труднощі? — Аж ніяк».

Протягом 1847 року «збільшене видання банкнот, може бути, допомогло б
знову поповнити золотий скарб банку, як це сталося в 1825 році». (3058).

Newmarch свідчить перед В А. 1857: «1357. Перший лихий вплив\dots{}
цього відокремлення обох відділів [банку] та розділу золотого запасу на дві частини,
розділу, що неминуче випливав з такого відокремлення, був той, що банкові
операції Англійського банку, отже, цілу ту ділянку його операцій, що ставить
його в безпосередній зв’язок з торговлею країни, провадилось далі лише за допомогою
половини суми попереднього запасу. В наслідок цього розділу запасу
дійшло до того, що банк мусив підвищувати норму свого дисконту, скоро запас
банкового відділу зменшувався хоч трохи. Тому цей зменшений запас зумовлював
ряд раптових змін у нормі дисконту. — 1358. Таких змін, починаючи від
1844 року [до червня 1857 року], було, може, з 60, тимчасом коли протягом
такого самого часу перед 1844 роком вони ледве чи становили дюжину». Особливий
інтерес має теж свідчення Palmer’a, що від 1811 року був директором, а
деякий час управителем англійського банку, перед C.~D. комісією лордів (1848--57).

«828. В грудні 1825 року банк ще зберіг приблизно \num{1.100.000}\pound{ ф. ст.}
золота. Він мусив би тоді, безперечно, цілком збанкрутувати, коли б тоді був
цей акт (1844 року). В грудні він видав, на мою думку, 5 або 6 мільйонів
банкнот протягом одного тижня, й це значно полегшило тодішню паніку.

«825. Перший період [від 1 липня 1825 року], коли сучасне банкове
законодавство збанкрутувало б, якщо банк спробував би довести до кінця вже
розпочаті операції, був 28 лютого 1837 року; в ті часи в розпорядженні банку
було \num{3.900.000} до 4 мільйонів ф. ст., і він зберіг би тоді лише \num{650.000}\pound{ ф. ст.}
в запасі. Другий такий період був у 1839 році й тривав від 9 липня до 5 грудня.
— 826. Яка була сума запасу в цьому випадку? — 5 вересня запас
складався з дефіциту в цілому на суму \num{200.000}\pound{ ф. ст.} (the reserve was minus
altogether \num{200.000} ф. ст). На 5 листопада запас зріс приблизно до 1--1\sfrac{1}{2}
мільйонів. — 830. Акт 1844 року заважав би банкові підтримувати торговлю
з Америкою. — 831. Три головні американські фірми збанкрутували\dots{} Майже
кожну фірму, що провадила американські операції, позбавлено кредиту, і
коли б у ті часи банк не прийшов на поміч, то я не думаю, щоб більше, як
1 або 2 фірми, могли витримати, — 836. Скруту 1837 року не можна рівняти
з скрутою 1847 року. В 1837 році вона обмежилася головне на американських
операціях». — 838. (На початку червня 1837 року дирекція банку дискутувала
питання, як зарадити тій скруті). «В цій справі декотрі з панів боронили думку\dots{}
що найправильнішим принципом було б підвищити рівень проценту, через що
товарові ціни впали б; коротко, зробити гроші дорожчими, а товари дешевшими,
\parbreak{}  %% абзац продовжується на наступній сторінці


Але сесі беспосередні наслідки реформації не були
найтривкійші. Церковна власність, се була реліґійна підпора
старосвіцьких порядків ґрунтових. Впала вона, то й їм не
довго було вже встоятись.

Ще в послідних десятилітях \RNum{17} віку було джоменів
(самостійних ґаздів хліборобів) більше ніж арендаторів.
Вони творили головну силу Кромвеля і — як свідчит сам
Маколєй — визначувались дуже корисно супротів роспитих
паничів та їх прислужників — сільских попів. Ще навіть
сільскі наємники були співвластивцями громадського ґрунту.
Аж около 1750 щезли джомени зовсім, а в послідних десятиліттях
\RNum{18} віку щезли послідні сліди громадських ґрунтів
хліборобських. Ми ту не берем на ввагу чисто економічних
двигачів рільничого перевороту, але глядимо тілько на пoсторонні,
насильні товчки.

За реставрації Стюартів перевели великі властивці
ґрунтів правним способом такий самий рабунок, який в прочій
Европі робився і без правних оборотів. Вони знесли
феодальні ґрунтові порядки, т.~є. скасували всі ті повинности,
які припадали державі з ґрунтів, „відшкодували“ державу
тим, що наложили податки на хліборобів та прочу
масу народа, а самі забрали в тісну приватну власність усі
добра, над котрими вперед мали лиш феодальну зверхність,
і накинули вкінци народови такі права осідленя (laws of
settlement), котрі, mutatis mutandis, так само повліяли на
англійських хліборобів, як указ татарина Бориса Ґодунова
на россійських хліборобів.

„Преславна революція“ (glorious Revolution) з Вільгельмом
III Оранським дала панованє в руки ґрунтових та капіталістичних
богатирів. Вони почали нову еру тим, що до
роскраданя державних ґрунтів, котре доси велося скромно
і тайком, взялися тепер на кольосальний розмір. Ті ґрунти
роздаровувано, продавано за песі гроші або й прямо без
даня рації прилучувано до приватних дібр\footnote{
„Безправна рострата коронних дібр чи то через продаж, чи через
роздарованє, становит огидну картку англійської історії\dots{} Се величезне
окраденє народа\dots{}“ (F.~W.~Newmann: „Lectures on Political Economy.
London, 1851“. стор. 129, 130).
}. Все то робилося
без найменьшої вваги на правні формальности. Ті закрадені
добра державні ураз із церковним фурфантєм, яке
\parbreak{}

\parcont{}  %% абзац починається на попередній сторінці
\index{i}{0069}  %% посилання на сторінку оригінального видання
товару іншим разом з тим призводить до того, що в руках третьої
особи лишається товар-гроші\footnote{
Примітка до другого видання. Хоч і як впадає на очі це явище,
однак політико-економи здебільша не помічають його, особливо ж вульґарні
прихильники вільної торговлі.
}. Циркуляція завжди стікає грошовим
потом.

\disablefootnotebreak{}
Нема нічого безглуздішого, як та догма, ніби циркуляція
товарів доконечно зумовлює рівновагу продажів і купівель, з
тієї причини, що кожний продаж є купівля й vice versa\footnote*{
навпаки. \emph{Ред.}
}. Коли
цим хочуть сказати, що число дійсно проведених продажів дорівнює
такому самому числу купівель, то це пласка тавтологія.
Але цією догмою хочуть довести, що продавець веде за собою
на ринок свого покупця. Продаж і купівля є тотожний акт як
взаємне відношення між двома полярно протилежними особами:
посідачем товарів і посідачем грошей. Вони становлять два полярно
протилежні акти як учинки тієї самої особи. Тим то тотожність
продажу й купівлі містить у собі й те, що товар стає некорисним,
коли він, кинутий в альхемічну реторту циркуляції, не
виходить із неї у формі грошей, коли посідач товарів його не продасть,
отже, коли посідач грошей його не купить. Ця тотожність
містить у собі далі те, що цей процес, якщо він удасться, становить
певну павзу, певний період в житті товару, який може тривати
довше або коротше. Через те, що перша метаморфоза товару є
одночасно продаж і купівля, цей частинний процес є разом з
тим самостійний процес. Покупець має товар, продавець має
гроші, тобто товар, який зберігає форму, що робить його здатним
до циркуляції, незалежно від того, чи він раніше або пізніше
знову з’явиться на ринку. Ніхто не може продати без того, щоб
хтось інший не купив. Але ніхто не потребує негайно купувати
через те тільки, що він сам щось продав. Циркуляція товарів
ламає часові, місцеві й індивідуальні межі обміну продуктів
саме тим, що наявну тут безпосередню тотожність між відчуженням
через обмін власного продукту праці й привласненням
чужого вона розколює на два протилежні акти — продаж і купівлю.
А що ці самостійні один проти одного процеси становлять
унутрішню єдність, — то це так само говорить і про те, що їхня
внутрішня єдність рухається в зовнішніх протилежностях\footnote*{
У французькому виданні це речення подано так: «Правда, що купівля
є доконечне доповнення продажу, але не менш справедливо, що
їхня єдність є єдність протилежностей». («Le Capital etc.», v. I, ch. III, p. 47).
\emph{Ред.}
}.
Коли зовнішнє усамостійнення внутрішньо несамостійних актів, —
бож вони доповнюють один одного, — доходить до якогось певного
пункту, то єдність їхня проявляється ґвалтовно — через
кризу. Іманентна товарові протилежність споживної вартости й
вартости, приватної праці, що разом з тим мусить з’являтися
як безпосередньо суспільна праця, осібної конкретної праці,
що разом з тим має значення тільки абстрактної загальної праці,
між персоніфікацією речей і зречевленням осіб, — ця іманентна
\parbreak{}  %% абзац продовжується на наступній сторінці

\parcont{}  %% абзац починається на попередній сторінці
\index{iii2}{0070}  %% посилання на сторінку оригінального видання
за нею в наші часи провадиться значну частину операцій. Такі люди охоче
гублять 20, 30 та 40\% на одній відправі товару кораблем; ближча операція
може вернути їм ті втрати. Коли їм раз-по-раз не щастить, тоді вони гинуть;
і саме такі випадки ми часто бачили останніми часами; торговельні фірми
збанкрутували, не залишивши жодного шилінґа в активі.

«4791. Нижчий рівень проценту [протягом останніх 10 років], звичайно,
має для банкірів несприятливий вплив, але, не подаючи вам для огляду торговельних
книг, мені було б дуже тяжко пояснити вам, оскільки теперішній
зиск [його власний] вищий від попереднього. Коли рівень проценту низький у
наслідок надмірного видання банкнот, то в нас є багато вкладів; коли рівень
проценту високий, то це дає нам безпосередній бариш. — 4794. Коли гроші можна
мати за помірний процент, то ми маємо більший попит на них; ми більше
визичаємо; такий вплив має це [для нас, банкірів] у цьому випадку. Якщо він
підноситься, то ми одержуємо за ті позики більше, ніж то годиться; ми одержуємо
більше, ніж повинні б мати».

Ми бачили, що всі експерти вважають кредит банкнот Англійського банку
за непохитний. А проте, банковий акт покладає цілком точно суму 9--10
мільйонів золотом для забезпечення розміну тих банкнот. Святість та непорушність
цього скарбу здійснюється, отож, цілком інакше, ніж у давніх збирачів
скарбів. W. Brown (Liverpool) свідчить перед C. D. 1847/58, 2311 так: «Щодо
тієї користи, яку ці гроші [металевий скарб в емісійному відділі] давали в ті
часи, так їх так само добре можна було б кинути в море; аджеж не можна було
навіть найменшої частини їх ужити, не ламаючи того парламентського акту».

Підприємець — будівничий Е. Capps, що його ми вже раніше згадували,
той самий, що з його свідчень узято характеристику сучасної лондонської системи
будівництва (Книга II, розд. XII), так резюмує свій погляд на банковий
акт 1844 року (В. А. 1857): «5508. Отже, ви взагалі тієї думки, що сучасна
система [банкового законодавства] дуже зручна установа для того, щоб періодично
кидати зиски промисловости до грошової торби лихваря? — Я такої думки.
Я знаю, що в будівельній справі ця система мала такий вилив».

Як згадано, шотландські банки примушено банковим актом 1845 року до
такої системи, що наближалась до англійської. На них поклали обов’язок тримати
золото в запасі на покриття банкнот, що вони видаватимуть понад суму,
усталену для кожного банку. Який вилив це мало, про це подаємо тут кілька
свідчень перед В. C. 1857.

Kennedy, управитель одного шотландського банку: «3375. Чи перед заведенням
банкового акту 1845 було в Шотландії дещо таке, що можна було б
назвати золотою циркуляцією? — Нічого подібного. — 3376. Чи від того часу
збільшилась кількість золота в циркуляції? — Ані найменше; люди не хочуть
мати золота (the people dislike gold)» — 3450. Ті приблизно 900.000 ф. ст. золота,
що їх шотландські банки мусять тримати, починаючи від 1845 року, на його думку,
тільки шкодять та «непожиточно поглинають рівну собі частину капіталу Шотландії».

Далі, Anderson, управитель Union Bank of Scotland: «3558. Єдиний значний
попит на золото, що його Англійський банк мав з боку шотландських
банків, був з нагоди закордонних вексельних курсів? — Це так; і цей попит
не зменшився від того, що ми тримаємо золото в Едінбурзі. — 3590. Поки ми
тримаємо ту саму суму цінних паперів в Англійському банкові» [або в приватних
банках Англії], «ми маємо ту саму силу, що й раніш, до того, щоб
викликати відплив золота з Англійського банку».

Насамкінець, ще одна стаття з Economist’a (Wilson): «Шотландські банки
тримають у своїх лондонських аґентів вільні суми готівкою; останні тримають
ті суми в англійському банкові. Це дає шотландським банкам змогу порядкувати
металевим скарбом банку в межах цих сум, а той скарб є завжди тут, на тому
\parbreak{}  %% абзац продовжується на наступній сторінці

\parcont{}  %% абзац починається на попередній сторінці
\index{iii2}{0071}  %% посилання на сторінку оригінального видання
місці, де його уживають, коли доводиться робити закордонні платежі». Цю систему
порушив акт 1845 року: «В наслідок акту, виданого для Шотландії
року 1845, останнім часом відбувся значний відплив золотих монет з Англійського
банку, щоб попередити той лише можливий попит на них в Шотландії,
що, може бути, й ніколи не постав би\dots{} Від цього часу одну значну суму
реґулярно закріпляють в Шотландії, а друга теж значна сума раз-у-раз мандрує
туди та сюди між Лондоном та Шотландією. Якщо настає час, коли шотландський
банкір чекає збільшеного попиту на свої банкноти, то відправляється
туди скриньку з золотом з Лондону; а коли цей час минув, то та сама скринька,
здебільша навіть не бувши одкрита, вертається назад до Лондону». (Economist
23 жовтня 1847~\abbr{р.}).

[А що каже з приводу всього цього батько банкового акту, банкір Samuel
Jones Loyd, alias лорд Оверстон?

Вже 1848 року він повторив перед C.~D. комісією лордів, що «грошову
скруту та високий рівень проценту, викликані недостатньою кількістю капіталу,
не можна полегшити збільшеним виданням банкнот» (1514), дарма що
самого лише \emph{дозволу} урядового листа з 25 жовтня 1847 року на збільшене
видання банкнот було досить, щоб збити тій кризі вістря.

Він лишається при тій думці, що «висока норма рівня проценту та пригнічений
стан фабричної промисловости були неминучим наслідком зменшення
\emph{матеріяльного} капіталу, що його можна було ужити для промислових та комерційних
цілей». (1604). А проте, пригнічений стан фабричної промисловости
протягом місяців був у тому, що матеріяльний товаровий капітал понад міру
наповнював комори, але його просто не сила було продати, і саме тому матеріяльний
продуктивний капітал цілком або напів лежав без діла, щоб не продукувати
ще більше того товарового капіталу, що його не сила було продати.

І перед банковою комісією 1857 року він каже: «Через гостре та ретельне
додержування засад акту 1844 року все відбувалося реґулярно та легко, грошова
система певна та непохитна, розцвіт країни безперечний, громадське довір’я
до акту 1844 року щодня зростає. Коли комісія бажає ще дальших
практичних доказів тому, що ті принципи, на які спирається цей акт, здорові,
та доказів тих благодійних наслідків, що їх він забезпечує, то на це ось правдива
та достатня відповідь: подивіться навколо себе; погляньте на сучасний
стан справ в нашій країні, погляньте на задоволення народу; погляньте на
багатства та розцвіт всіх кляс суспільства, і тоді, побачивши все це, комісія
буде в стані вирішити, чи схоче вона повстати проти дальшого існування акту,
що ним досягнуто таких наслідків». (В.~C. 1857, № 4189).

На цей дитирамб, що його Оверстон заспівав перед Комісією 14 липня,
відповідь дано було 12 листопада того самого року, тим листом до дирекції
банку, що ним уряд припиняв чинність чудотворного закону 1844 року, щоб
врятувати бодай те, що можна ще було врятувати. — Ф.~Е.]

\sectionextended{Благородний метал та вексельний курс}{%
\subsection{Рух золотого скарбу}}

Щодо нагромадження банкнот підчас скрути треба зауважити, що тут повторюється
те саме явище збирання скарбів у благородному металі, що в початковому
періоді суспільного розвитку завжди бувало за неспокійних часів. Акт
1844 року в своїй чинності являє інтерес тому, що він воліє перетворити ввесь
наявний у країні благородний метал на засоби циркуляції; він намагається вирівняти
відплив золота скороченням, а прилив золота збільшенням засобів циркуляції. Але
\parbreak{}  %% абзац продовжується на наступній сторінці

\index{franko}{0072}

Послухаймо ще хвильку, що говорит оден защитник „прилучуваня“ а противник Р.~Прайса: „Зовсім
фальшива тота гадка, що край обезлюднів, бо не видно-ді людей працюючих в чистім поли. Коли їх
тепер убуло по селах, то за то прибуло їх по містах\dots{} Коли дрібні ґазди-рільники перемінились в
наємних робітників, то через те сама кількість добутої праці стає більша, а се прецінь користь
пожадана для суспільности (тілько що, розумієся, самі „перемінені“ не належат до тої
суспільности!)\dots{} Добутку буде більше, коли скомбінована праця тих наємників буде ужита в \so{одній}
аренді; таким способом повстане надвишка витворів, котра піде до мануфактур, а через те й
мануфактур, тих жерел нашого богацтва, стане більше в стосунку до витвореної многоти збіжя“.

Незамутимий супокій, з яким суспільний економіст глядит на найзухвальше топтанє „святого права
власности“, на найгидше знущанє над людьми, коли йно все то робится для того, щоб покласти підвалину
капіталістичній продукції, проявляє між їншими торій і „філянтроп“ сер Ф.~М.~Еден. Цілий ряд
рабунків, головництв і притисків народних, серед яких відбувалося вивласнюванє люду від послідної
третини \RNum{15} до кінця \RNum{18} віку, викликає у него тілько сей супокійно-радісний вивід: „Належита
пропорція між вірними полями а толоками мусіла бути встановлена. Ще в цілім \RNum{14} і найбільшій части
\RNum{15} віку на оден екр толоки приходилося 2, 3, а навіть 4 екри вірного поля. В половині \RNum{16} віку
перемінилася пропорція: 3 екри толоки приходили на 2 екри рілі, а пізнійше 2 екри толоки на 1 екр
рілі, аж поки вкінци не вийшла належита пропорція: 3 екри толоки на 1 екр рілі“.

В \RNum{19} віці защезла, розумієся, й память про звязок між хліборобами а власністю громадською. Не
згадую вже зовсім о найпослідних часах, — але чи одержали селяне хоть оден шелюг відплати за тих
\num{3511770} екрів громадського ґрунту, котрі їм зрабовано між роками 1801 а 1831 і котрі с
парляментарними формальностями сільскі льорди подарували собі самим?

Послідний великий процес вивласнюваня хліборобів, се вкінци т. зв. „\textenglish{Clearing of Estates}“ (обчищуванє
дібр, або радше вимітанє з них людей). Тото „обчищуванє“, се вершок усіх англійських способів, які
ми доси бачили. Там, де вже не осталося незалежних ґаздів-хліборобів, доходит до вимітаня коттеджів,
так, що хліборобські робітники не можут уже найти й кусничка місця для замешканя на тім ґрунті,
котрий оброблюют. Властиве „обчищуванє дібр“ відзначуєся нечуваною сістематичністю і огромним
розміром, в якім тота операція нараз виконуєсь (в Шотляндії н.~пр.
вона відбувалася нараз на просторах таких завбільшки, як
\parbreak{}

\parcont{}
\index{franko}{0073}
цілі німецькі князівства), а також окремою формою ґрунтової власности, котру так насильно перемінюют
в приватну власність. Ті ґрунти, то була власність повіту (clan), — начальник або „великий чоловік“
був тілько титулярним властивцем, як представник повіту, так само, як королева англійська є
титулярною властителькою всего ґрунту Англії. Тот переворот, котрий в Шотляндії почався по посліднім
повстаню претендента, мож слідити в перших єго початках у письмах Джемса Стеєрта і Джемса Андерсона\footnote{
Стеєрт каже: „Рента в тих околицях (він хибно називає рентою тоту оплату, яку обивателі повіту
(taksmen) складали начальникови повіту) зовсім незначна в стосунку до обширности піль, але що до
числа осіб, котрих удержує одна аренда, мож сміло твердити, що оден кусник ґрунту в шотлянських
горах виживлює десять раз більше людей, ніж так само заобширний ґрунт в найбогатших рівнинах“.
}. В \RNum{18} віці заборонено притім Ґелям, прогнаним з ґрунтів, виселюватись в чужі краї, щоб їх таким
способом силою попхнути до Ґлязґова і других фабричних міст\footnote{
1860 виводжено тих насильно вивласнених хліборобів до Канади, отуманивши їх фальшивими
обіцянками. Деякі повтікали в гори і на сусідні пусті острови. Поліція пустилася за ними в погоню,
прийшло до бійки і втікачі здужали вирватися та порозбігатись.
}. За примір
методи пануючої в девятнайцятім віці\footnote{
„В шотляндських горах“, каже Бюкенен, коментатор А.~Сміта, 1814, „день в день насильно затираєся
давний власностевий порядок\dots{} Сільский льорд, без згляду на дідичних арендаторів (знов хибно
названі тексмени), винаймає ґрунт тому, хто найбільше платит, а коли той належит до меліораторів
(imprower), то зараз заводит новий спосіб управи поля. Ґрунт, давнійше покритий дрібними
властивцями, був в стосунку до своєї плодовитости досить заселений; при новім сістемі поліпшеної
управи і побільшеної ренти одержуєсь як мож найбільше плодів як мож найменьшим коштом, і для того
віддалюются робітники, котрі стали тепер непотрібними. Ті вигнанці з рідних хат шукают відтак
утриманя в фабричних містах і т. д. (David Buchanan: „Observations on A.~Smith’s Wealth of Nations.
Edinb. 1814“.) „Шотляндські маґнати вивласнили цілі родини, немов хопту випололи: вони так обійшлися
з селами й людністю, як Інди розїдлі пімстою з дикими звірями по норах\dots{} Чоловіка продают
за овече руно, за волове стегно, ба ні, ще за меньшу дрібницю\dots{} Підчас нападу на північні
провінції Хіни була на раді Монголів така думка, щоб усіх мешканців витратити а їх край перемінити
в степ. Тоту раду богато північно-шотляндських маґнатів дословно виповнили в своїм власнім краю і на
своїх власних земляках“. (Джордж Ензер: „An Inquiry concerning the Population of Nations. Lond.
1818“. Стор. 215, 216.)
} досить буде ту навести „обчищуваня“ герцоґині Созерлєнд.
Тота в економії вишколена особа постановила зараз в початку свого панованя взятися до радікального
ліку економічного, і ціле ґрафство, в котрім задля давнійших подібних процесів осталось
\index{franko}{0074}
було всего лиш \num{15000} люда, перемінити в толоку для овець. Від 1814 до 1820 сістематично
прогонювано та нищено тих \num{15000} мешканців, т. є. майже 3000 родин. Всі їх села поруйновано і
попалено, всі їх поля пороблено толоками. Англійських жовнірів викомендерувано там для еґзекуції, і
між ними а мешканцями прийшло до бійки. Одна
стара баба згоріла враз іс хатою, с котрої не хтіла вступитися. І таким способом присвоїла собі
вельможна герцоґиня \num{794000} екрів ґрунту, котрі споконвіку належали до
повіту. Вигнаним мешканцям визначила вона на морськім узберіжю около 6000 екрів, по 2 екри на
родину. Тих 6000 екрів лежали доси пусто і не давали властительці ніякого
доходу. Герцоґиня так далеко зайшла в своїй щедрости, що винаймила екр пересічно по 2\shil{ шілінґи}
6\pens{ пенсів} для тих самих селян, котрі много сот літ проливали кров свою за
вельможну герцоґську родину. Увесь зрабований ґрунт повіту поділила герцоґиня на 29 великих аренд
для випасаня овець; в кождій аренді осіла тілько одна родина, переважно англійські наємні
арендаторі. 1825 р. замісць \num{15000} Ґелів на їх ґрунтах жило вже \num{181000} овець. А родини, вивержені на
морський беріг, старалися жити риболовством. З них поробилися земноводяні, і вони жили, як каже
писатель, на половину в воді, а на половину на березі, тілько що ні ту ні там не могли найти
достаточного прожитку\footnote{
Коли теперішна герцоґиня Созерлєнд витала в Льондоні з великою парадою міссіс Бічер Стоу,
авторку „Хати дядька Томи“, щоб виставити на показ свою прихильність для муринів-невольників в
американській републіці — чого вона і єї співарістократки певно не булиб зробили підчас домашної
війни американської, бо тоді кожде „шляхотне“ англійське серце було прихильне плянтаторам — в той
сам час описав
я в газеті „New-York-Tribune“ побут невольників созерлєндських. (Деякі місця тої статі навів Керей в
своїй „The Slave Trade. London 1853“.). Мою статю перепечатала одна шотляндська ґазета і викликала
дуже чемну перепалку між тою ґазетою а підхлібниками та похвальками герцоґів Созерлєндів.
}.

Але небораки Ґелі мусіли щераз відпокутувати свою романтичну наклінність для „великих мужів“, т. є.
для начальників повітових (Сlanchef). Запах риб, котрими прокормлювались земноводяні Ґелі, ударив
великим мужам в ніс. Вони завітрили тут щось зисковного і заарендували морське узберіжє великим
льондонським гендлярам риб. Ґелів другий раз вигнано на штири вітри\footnote{
Цікаву історію того рибного торгу найде читатель у д. Девіда Оркуарта в єго книжці: „Portofolio.
New Series“. Сеніор в одній іс своїх посмертних статей називає „процедуру в Созерлєндшайрі“ одним з
найблагодатнійших очищень від віків.
}.

Аж вкінци одну часть пасовиськ назад перемінено
\parbreak{}

\parcont{}
\index{franko}{0075}
в місця для польованя. Звісна річ, що в Англії нема правдивих гір. Дичина в маґнацьких парках, се
констітуційна домашна худоба, товста, як льондонські ольдермени. Шотляндія, се затим послідне місце
для „шляхотного занятя“ — стрілецтва. „В шотляндських горах“, каже Сомерс 1848, „ліси дуже стали
обширні. Ось з одного боку від Ґейка маєте новий ліс Ґлєнфшай, а з другого боку також новий ліс
Ордверікай. Поряд з ними бачите Блік-Моунт, огромну пустиню, свіжо заложено. Від сходу до заходу,
від околиць Обердіна аж до урвищ Обена тягнутся тепер без перерви ліси. А й по других частях
Шотляндії находятся нові ліси, як ось в Льоч Орчейґ, Ґлєнджеррі, Ґлєнморістен і др\dots{} Переміна
ґрунтів в овечі толоки прогнала Шотляндців на неврожайні пустарі. Тепер серни та лиси починают
витискати овець, а Шотляндців кидати ще в страшнійшу нужду\dots{} „Ліси для дичини“\footnote{
Шотляндські „ліси для дичини“ (deer forests) не мают і одного деревця. Се звичайні голі толоки,
с котрих вигнано вівці, а на їх місце нагнано оленів, — тай ось і преславетний „ліс для дичини“. Про
засів та плеканє лісів і казки нема!
}, а народ(и) не
можут істнувати побіч себе. Або одні, або другі мусят уступити місця. Нехай тілько число і розмір
польовань в слідуючих 25 роках зможеся так, як в минувших, то певно не здиблете й одного Шотляндця в
єго ріднім краю. А се змаганє між шотляндськими маґнатами походит по части з моди, панської бути,
забагів на польованє і т. д., а по части вони займаются дичиною для зиску. Бо кусень гористого
простору, затичений для польованя, нераз далеко більше дає зиску, ніж колиб був толокою. Той, кому
забаглося польованя і шукає такого обшару, платит за него тілько, накілько статчит єго кішеня,
— (а се вже певно, що бідному чоловікови не до польованя!)\dots{} Відти то сплило на шотляндські гори
тілько недолі, кілько єї сплило на Англію ізза політики норманських
королів. Оленям віддано огромні простори до волі, а людей зігнано в тісні і чим раз тіснійші
закамарки\dots{} Одну вільність за другою видирано народови\dots{} І притиск той ще
день-денно змагаєся. Властивці з засади і загалом вимітают і виганяют народ, мов сіно косят, — мов
ті австральські та американські осадники відвічні ліси витинают, і тота операція поступає чим раз
далі, спокійно, с холодною розвагою та обрахунком\footnote{
Robert Somers: „Letters from the Highlands: or, the Famine of 1847. Lond. 1848“, стор. 12--28. Ті
листи надрукувала зразу ґазета Таймc. Англійські економісти, розумієся, зараз ростолкували, що
Шотляндці бідуют і мрут з голоду задля — перелюдненя. Сяк чи так, а їсти не було що. І в Німеччині
не чужа тота операція „Clearing of Estates“, — єї ту прізвано „Bauernlegen“ (обалюванє мужиків), і
вона особливо далася чути по 30-тилітній війні. Ще 1790 в одній части Саксонії вона викликала хлопські бунти. А найбільше
ширилося „обалюванє мужиків“ в східній Німеччині. В найбільшій части німецьких провінцій аж Фрідріх
II запевнив селянам право власности. По завойованю Шльонська він присилував дідичів до відбудованя
хат, стоділ і т. д., а також до заосмотреня мужицьких осад худобою та ґосподарськими знадобами.
Фрідріх II потребував жовнірів для війська і оподаткованих людей для побільшеня державного
скарбу. Впрочім селянам і під Фрідріхом II жилось далеко не гарно, як се мож побачити с письм єго
головного похвальця, Мірабо.

В цвітню 1872, 18 літ по виданю згаданої ту книжки Сомерса читав проф. Лєоні Лєві в „Society of
Arts“ відчит про переміну овечих пасовиск в ліси для дичини, де вказує дальший розвиток спустошеня в
шотлянських височинах. Між їншим каже він: „Обезлюдненє і переміна ґрунтів в голі толоки, се був
найвигіднійший для панів спосіб — получити доходи без видатків\dots{} Тепер же в височинах звичайно
пороблено с толок „deer forest-и“. Дичина прогнала овець так, як недавно вівці прогнали були людей. Мож вандрувати від
дібр ґрафа Дельгаузі в Форфершайрі аж до Джона o’Ґротса, не виходячи зовсім з лісів. В многих іс
тих лісів замешкуют лиси, дикі коти, куни, тхорі, ласиці та альпейські заяці; від недавна
росплодилися там також крілики, вивірки та щурі. Огромні простори, котрі в шотлянській статистиці
значились „надзвичайно врожайні і розляглі пастівники“, позбавлені тепер всякої управи і поправи і служат виключно для мисливської
забави кількох осіб, тай то лиш короткий час в році!“

Льондонський „Economist“ з 2 червця 1866 каже: „Одна шотлянська ґазета доносит послідного тижня між
їншими новинами ось що: Одна з найкращих овечих аренд в Созерлєндшайрі, за котру недавно, за упливом
біжучого арендового контракту, давано річної ренти 1200\pound{ фунтів штерлінґів}, тепер зістає перемінена в
„deer forest!“. Феодальні інстінкти проявляются й тепер так само, як тоді, коли норманський
завойовник зруйнував 36 сіл, щоб закласти Ню-форс (Новий ліс)\dots{} Два мілійони екрів, самих
найурожайнійших в Шотляндії, опустошуются тепер до крихітки. Природна трава Ґлєн-Тільту належала до
найпоживнійших в ґрафстві Перз; теперішний дір-форст Бен-Альдер був найкращим пасовиском в цілім
лісистім Бедноч; одна часть теперішного Блік-Моунт-форста була найліпшим на всю Шотляндію пасовиском
для чорномордих овець. Про обсяг ґрунтів опустошеннх для стрілецької примхи мож виробити собі
яке-таке понятє, зваживши, що вони обіймают далеко більше простору, ніж ціле ґрафство Перз. Що через
те насильне опустошенє стратив край на жерелах продукції, мож оцінити с того, що форс Бен-Альдер міг
% REMOVED \footnote*{В рукописі: між.}
прокормити \num{15000} овець і що він становит лиш \sfrac{1}{30} всіх „диких лісів“ шотлянських.
Весь той „дикий“ ґрунт зовсім не продуктівний\dots{} На одно б вийшло, як би був запався в фалі
Північного моря. Сильна рука праводавства повинна би прецінь зупинити розріст і творенє таких
самовільних пустинь“.
}.

\index{franko}{0076}
Рабунок дібр церковних, злодійське загарбуванє державних маєтків, крадіж громадських ґрунтів,
безправна \parbreak{}

\parcont{}
\index{franko}{0077}
а з беззглядною жорстокістю переведена переміна феодальної та окружної (Clan-)
власности в новійшу приватну власність, — ось які іділлічні були способи
первісного нагромадженя капіталу. Вони здобули ґрунт для капіталістичного
рільництва, втягли землю в обсяг капіталу, а міському промислови
достатчили потрібних „рук“, т. є. вольного і голого пролєтаріяту.

\subsection{Кроваві устави протів пролєтаріїв при кінци XV віку}

Вольний і голий пролєтаріят, вигнаний с хат і ґрунтів
через скасованє феодальних дворів і через насильне раз-заразом
вивласнюванє, не міг відразу перелятися весь до
новоповстаючих мануфактур так швидко, як швидко сам
повстав. А при тімже се були люде, викинені раптово с привичного
способу житя, — а такі люде не швидко можут
застосоватися до яких небудь нових, непривичних порядків.
На першій порі з них поробилися маси жебраків, розбійників,
волоцюг, — деякі з наклінности, а найбільша часть під гнетом обставин. С
кінцем XV і підчас цілого XVI віку бачимо проте в цілій Західній Европі
кроваві устави протів волоцюгів. Батьки нинішної робітницької верстви мусіли
на самім вступі відбути страшну кару, — за що? За то, що їх перемінено в волоцюг
та голоту. Праводавці вважали їх „добровільними переступцями“ і думали, що
тілько від їх доброї волі залежит — працювати далі серед давних обставин, котрі
між тим зо світа щезли.

В Англії почалось те праводавство під Генріхом VII.

Генріх VIII, 1530: Старі і неспосібні до праці жебраки одержуют дозвіл на
жебрацтво. За то здорові й міцні волоцюги карані будут батогами й арештом. Вони
мают бути привязані ззаду до тачок і бичовані доти, доки не поплине кров з їх
тіла, — відтак мусят зложити присягу, вернути на місце уродженя або там, де
пробули послідні 3 роки  і „засісти до праці“ (to put himself to labour). Що за
безсердечна насмішка! В 27 уст. Генріха VIII повторена попередна устава, але
заострена новими додатками. Як кого другий раз зловят на волоцюгованю, то такого
бичувати ще раз і відтяти му пів вуха. За третим разом непоправного волоцюгу,
як тяжкого злочинця і ворога суспільности — вкарати смертю.

\looseness=-1
Едуард VI: Устава с першого року єго панованя 1547, наказує, що скоро хто
отягаєся від праці, той має бути присуджений на невольника тій особі, котра
донесла урядови о єго неробстві. Пан має годувати невольника хлібом і водою,
слабими напитками і такими обрізками мяса, які му видадутся відповідними. Він
має право всилувати го батогами \index{franko}{0078}
та зелізними ланцами до всякої, хотьби й як гидкої роботи. Коли невольник на 14
день віддалится, то зістає засуджений на віковічну неволю і має бути на чолі
або на лици напятнований буквою S, а коли до трох раз утече, то має бути
вкараний смертю, як зрадник держави. Пан може го продати, передати в наслідство,
визичити другому в неволю, зовсім так, як усяке друге рухоме добро, як худобу.
Коли невольники в чім небудь станут супротів панів, то мают також бути покарані
смертю. Мирові судьї повинні за отриманим остереженєм слідити за волоцюгами.
Коли покажеся, що такий волоцюга три дни волочився без діла, то такого
відставити на місце, де родився, роспеченим зелізом напятнувати на груди буквою
V і тамій в зелізних ланцюхах уживати до замітаня вулиці або до якої небудь
їншої служби. Коли волоцюга подасть фальшиво місце вродженя, то за кару має
бути віковічним невольником тої громади, тих мешканців або того товариства і
напятнований буквою S. Кождий має право відобрати у волоцюги єго діти і яко
помічників та термінаторів держати хлопців до 24, дівчат до 20 літ. Коли вони
втечут, то мают аж до тих літ бути невольниками майстра, а тому вільно їх
заковувати в ланци, бити і пр., як му сподобаєсь. Кождий пан може заложити
зелізну обручку на шию, руку або ногу свого невольника, щоби міг го ліпше
пізнати і бути певним, що му не втече\footnote{
Автор книжки „Essay on Trade and Commerce“ 1770, каже: „Під панованєм Едварда
VI взялись були Англічане зовсім, здаєсь, серйозно до піддвигненя мануфактур і
затрудненя бідних. Се бачимо з одної дивовижної устави, в котрій приписуєсь, що
всі волоцюги мают бути пятновані, і т. д. (Essay on Trade and Commerce, стор.
8).
}. Послідна часть тої устави наказує, щоб
деяких бідних брали на себе громади або поєдинчі люде; ті мают їм давати їсти
й пити і старатись для них о роботу. Тот рід громадських невольників удержувався
в Англії гет ще в \RNum{19} віці під назвою roundsmen (люде, що ходят від хати до
хати).

Єлисавета, 1572: жебраки без дозволу і віком понад 14 літ мают бути без
милосердя бичовані і напятновані на лівім вусі, хіба що їх хто схоче взяти на
два роки на службу; в разі повтореня, коли мают над 18 літ, мают бути — смертю
карані, скоро їх ніхто не схоче взяти на два роки на службу; за третим разом
мают без милосердя як зрадники державні бути покарані смертю. Подібна також \RNum{18}
устава Єлисавети, розділ 13, і устава з р. 1597 \footnote{
Томас Морус каже в своїй „Утопії“: „Так то дієся, що оден захланний і неситий
ненаїсник, правдива чума нашої вітчини, може тисячі екрів ґрунту збити до купи
і обпалькувати, обгородити одним плотом, або силою та кривдою до того довести
єго властивців, що вони будут мусіли все спродувати. Сяким чи таким способом,
чи там гнись чи ломайся, він присилує їх забиратися, — бідні, прості, нещасливі
душі! Мужчини й женщини, чоловіки й жінки, сироти без батьків, удови, плачучі
матері с пеленковими дітьми, і вся челядь, убога добром, а богата
ротами, бо рільництво вимагає богато рук. І волочутся вони, кажу вам,
з знакомих, рідних місць, не находячи пристанівку. Якби при й нетаких
обставинах, то моглиб бодай що то вторгувати за свій, хоть і не дуже
цінний, домашний спряток; але раптово повикидувані, мусят усе продавати
за песій гріш. А коли перебурлачат послідний свій гріш, то щож
тоді мают робити, як не красти, а відтак, боже добрий, по всій формі та
правді згинути на шибеници або пуститися на жебри. А й тоді ще їх
попрут до вязниць як волоцюгів, що-ді плентаются а нічо не робят.
А що там судови до того, що їх ніхто не хоче взяти на роботу, хоть би
й як радо самі на ню напрошувались!“ І таких бідних утікачів, котрих
по словам Томаса Моруса присилувано до крадіжи, „за панованя Генріха
VIII, повішено \num{72000} великих та дрібних злодіїв“. (Ноllingshed, Dеscription
of England, т.~І, стор. 186). За часів Єлисавети „вішано волоцюгів
цілими рядами; а прецінь не було такого року, в котрім би на
однім або другім пляцу не повішено їх 300--400“ (Strype`s Annals, т.~II).
Той сам Страйп свідчит, що в Соммерcетшайрі за оден рік повішено 40
люда, напятновано 35, бито батогами 37, а випущено 183 „непоправних
злочинців“. А такій, каже він, „те велике число оскаржених не становит
ще й пятої части всіх злочинців, дякувати недбальству мирових судів
і глупому милосердю народа“. Він додає: „Прочі англійські ґрафства
зовсім не стояли ліпше від Соммерсетшайра, а богато стояло в тім згляді
ще далеко гірше“.
}.

\index{franko}{0079}
Яков І: Кождий, хто ходит від села до села і жебрає,
узнаєсь волоцюгою. Мирові суді мают право засудити го на
прилюдне бичованє і за першим разом на 6 місяців, за
другим на 2 роки тюрми. Підчас сидженя в тюрмі мают
бути так часто і так богато бичовані, як се мировий судя
узнасть за добре\dots{} Непоправні і небеспечні волоцюги мают
бути на лівім плечи напятновані буквою R і заставлені до
робіт примусових, а як їх ще коли придиблют на жебранині,
то мают бути без милосердя і без сповіди повішені. Ті устави,
% REMOVED \footnote*{В рукописі: уставі.}
правосильні аж до перших літ \RNum{18} віку,
знесені зістали доперва \RNum{12} уст. Анни, розд. 23.

Подібні устави бачимо і в Франції, де в половині \RNum{17}
віку завязалось було ціле царство волоцюгів (truands) в Парижи.
Ще в початку панованя Людовіка XVI (Указ з дня
13 липня 1777) кождий здорово збудований чоловік від 16
до 60 літ віку, скоро був без удержаня і не мав означеного
занятя, мав бути висланий на ґалєри. Подібні також: устава
Карля V для Нідерляндів з 6 жовтня 1537, перший едікт
держав і міст голяндських з 19 марта 1614., оповіщенє Сполучених
провінцій з д. 25 червня 1649 і богато других.
Ось яким способом, — батогами, пятнованєм та тортурами
на підставі нелюдських, кровавих устав увігнано мужиків,
\parbreak{}

\parcont{}
\index{franko}{0080}
насилу обрабованих з ґрунту, хат і майна, насилу пороблених
злодіями та волоцюгами, в ті тверді рами карности,
конечної при сістемі наємної праці.

\vspace{-\medskipamount}
\subsection{Устави для знищеня робучої плати}
\vspace{-\bigskipamount}

\disablefootnotebreak{}
Не досить того, що знадоби продукції розділюются:
на однім боці сам капітал (в руках властивців богатирів),
а на другім боці сама праця, т. є. люде, котрі нічо не мают
на продаж крім своєї праці. Не досить ще присилувати
тих людей до того, щоб добровільно себе самих запродували.
В дальшім ході капіталістичної продукції виростає
вже верства робітників, котра з вихованя, традиції, привички
признає вимоги того способу продукованя природними законами,
чимось таким, що й бути інакше не може. Впорядкованє
видосконаленого капіталістичного процесу продукційного
перемагає всі запори; ненастанне повставанє релятівного
перелюдненя\footnote*{
Звісно, що перелюдненєм звеся то, коли де небудь є забагато
людей, т. є. властиво більш людей, ніж може вижити. А релятівне перелюдненє
значит, що тілько в певнім місци і серед певних обставин є для
певного діла забогато людей, так що всі вони не можут приміститися,
і одна часть з них дармує. Кождий пійме, що вже сама проява такого
релятівного перелюдненя є знаком нездорових економічних обставин.
Між тим, як побачимо далі, ціла капіталістична продукція нерозлучно
звязана с релятівним перелюдненєм, котре змоглося в краях промислових
особливо від заведеня парових машин, через що мілійони рук робітницьких
стратили роботу (\emph{Прим.~перев.}).
} вдержує довіз робучих рук і попит
за працею, значит, і робучу плату на такій висоті, яка кориснійша
для підростаючого капіталу; німий примус економічних
обставин довершує панованя капіталіста над робітником.
Позаекономічна, беспосередна сила входит все ще
в уживанє, але вже лиш виїмково. При звичайнім ході діла
досить є — лишити робітника під властю „природних законів
продукції“, т. є. лишити го в залежности від капіталу,
витвореній і навіки забеспеченій самими вимінками
продукційними. Але сего не мож зробити в тій історичній
хвили, коли капітал і етична продукція інощо зароджуєсь.
Підростаюча буржоазія потребує і уживає власти державної,
щоб „реґулювати“ робучу плату, т. є. втискати єї в такі
границі, які найкориснійші для баришництва, продовжувати
день робучий і вдержувати самого робітника в „належитій“
степени залежности. Се також дуже важний причинок до
т. зв. первісного нагромадженя капіталу.
\enablefootnotebreak{}

\looseness=-1
Верства наємних робітників, що повстала в послідній
половині \RNum{14} віку, становила тоді і в слідуючих столітях
тілько дуже незначну часть людности, котрої становище
\parbreak{}

\parcont{}
\index{franko}{0081}
притім міцно обезпечували самостійні ґаздівства по селах
а цехові звязки по містах. По селах і містах не було великої
суспільної ріжниці між майстрами а робітниками.
Підчиненє праці під капітал було тілько формальне, т. є.
продукція сама не мала ще на собі окремої капіталістичної
ціхи. Попит за наємною працею змагався прото дуже швидко
за кождим нагромадженєм капіталу, — між тим рук готових
найматися до праці прибувало дуже поволи. Велика
часть витворів суспільних, що пізнійше стала фондом вбільшуючим
капітал, тоді переходила ще в руки робітника для
єго власного зужитку.

Праводавство про наємну працю, згори вже вицілене
на визискуванє робітника і в своїм розвитку йому завсігди
однаково неприхильне, почалося в Англії від виданя „Устави
робітницької“ (Statute of Labourers) Едвардом III, 1349.
Рівночасно видано в Франції Указ 1350 р. в імени короля
Жана. Англійські і французькі устави виходят рівнобіжно
і зовсім однакі що до змісту.

Устава робітницька зістала видана за про голосні наріканя
послів. „Давнійше“, каже наівно оден Торі, „жадали
бідні такої великої плати за роботу, що промисл і богацтво
були загрожені. Тепер плата така низька, що знов грозит
промисловії й богацтву і то може ще небеспечнійше ніж
тоді“. Установлено правну тарифу платну для міст і сіл,
за роботу
%REMOVED (в рукоп. „робуту“)
на дни й від штуки. Сільскі робітники
повинні винайматися на рік, міські „с прилюдного
торгу“. Під карою тюрми заборонено платити висшу плату
від означеної в уставі; а хто бере більшу плату, того кара
виносит більше, ніж сама плата. Так само ще в розд. 18
і 19 устави о учениках ремісницьких, виданої за Єлисавети,
грозится карою 10 день тюрми тому, хто платит більше,
а 21 день тюрми тому, хто бере більшу плату від правом
приписаної. Устава з р. 1360 заострила кари і навіть дала
майстрам право силувати робітників мусом до праці за таку
плату, яка означена в тарифі. Всякі звязки, угоди, присяги
і т. д., котрими взаїмно сполучилися теслі з мулярами,
узнані неважними. Стоваришеня робітницькі караются як
тяжка провина від \RNum{14} віку до 1825, в котрім скасовано
устави протів стоваришень. Дух „Робітницької устави“ з р.
1349 і єї потомків просвічує ясно й с тих устав протів стоваришень.
Се тота сама засада: держава приписує, кілько
мож найбільше платити робітникови, але хрань боже, щоб
хоть натякнула на те, кілько мож йому найменьше платити!

В \RNum{16} віці, як звісно, положінє робітників дуже погіршилося.
Правда, грішми плачено більше, тількож що ціна
грошей стала меньша а ціна товарів без міри більша. Наділі
затим і плата вменьшилася. А прецінь устави для єї
зниженя трівают далі порівно з обрізуванєм вух та пятнованєм
\index{franko}{0082}
тих, „котрих ніхто не хоче взяти на службу. Єлисаветина
5 устава про учеників ремісницьких, уст. 3 надає
мировим судям власть становити де в яких реміслах плату
і змінювати її відповідно до пори року і ціни товарів. Яков
I ростягнув ту саму реґуляцію робітницької плати на ткачів,
прядільників і на всі можливі розряди робітників\footnote{
З одної примітки до устави 2 за Якова І, розд. 6 видно, що
деякі суконники позваляли собі самі яко мирові судьї урядово діктувати
платну тарифу в своїх варстатах. — В Німеччині, а іменно по 30-літній
війні, виходит богато устав для знижуваня робучої плати. „Поміщикам
на безлюдних ґрунтах дуже прикро давалась чути недостача слуг і робітників.
Всім мужикам-ґаздам заказано приймати в комірне мужчин та
женщин вільного стану; про всіх таких комірників повинно доноситися
урядови, а той запирає їх в тюрму, скоро не хотят стати слугами, хоть би
й без того мали яке їнше вдержанє, хоть би працювали у  мужиків за поденщину
або навіть торгували грішми та збіжєм. (Цісарські прівілєї та
ухвали для Шльонська, І, стор. 125). Через цілих сто літ роздаются в приписах
князів та поміщиків раз~відразу гіркі наріканя на злосливих
і здуфалих слуг, що не хотят піддатися важким условинам, не хотят вдоволюватися
платою правом приписаною. Виходят накази, щоб поєдинчий
поміщик не смів своїм слугам платити більше, ніж кілько весь краєвий
збір покладе в таксу. А прецінь условини служби по війні нераз ще
бувают ліпші, ніж були 100 літ опісля. В р. 1652 діставали ще слуги на
Шльонську по два рази до тижня мясо; а ще в нашім столітю іменно
там були такі округи, де слуги діставали мясо хіба три рази до року.
І поденщина (плата за день роботи) по 30-літній війні була більша, ніж
в слідуючих столітях“ (Ґустав Фрейтаґ).
}, Джордж
II ростягнув устави протів робітницьких товариств на всі
мануфактури. В властивій порі мануфактуровій капіталістична
продукція була вже досить сильною, щоб правну
реґуляцію робучої плати зробити непотрібною, а то й неможливою,
але все такі ще на всякий злучай не закидувано
того перестарілого оружя. Ще 8 устава Джорджа II заказує
давати кравецьким челядникам в Льондоні і околици більше
понад 2\shil{ шіллінґи} і півосьма пенса денної плати, окрім хіба
в разах загальної жалоби. Ще 13 уст. Джорджа III, розд.
68 повіряє мировим судям реґульованє робучої плати у виробників
шовку. Ще 1796 тре було двох декретів висших
судів для рішеня, чи накази мирових судьїв що до робучої
плати мают вагу і для не-рільничих робітників. Ще 1799
потвердила ухвала парляменту, що плата копальників шотляндських
уреґульована уставою Єлисавети і двома шотляндськими
актами з р. 1661 і 1671. А який між тим переворот
доконався у всіх обставинах, доказала подія нечувана
в англійській палаті панів. Ту, де від звиш 400 літ
фабриковано устави виключно о тім, понад яку міру не
може ніяк переступити робуча плата, — ту поставив 1799
\parbreak{}

\parcont{}
\index{franko}{0083}
Уайтбрід внесок устави, яка може бути найменьша плата
для робітників рільничих\dots{} Хоть Пітт супротивлявся тому
внескови, то прецінь і сам признав, що „положінє вбогих
страшенне (cruel)“. Вкінци 1813 скасовано устави про реґуляцію
плати. Вони стались смішним недоріцтвом, відколи
капіталіст порядив у своїй фабриці після власних приватних
прав, а плата рільничого робітника давно впала понизше
мінімум конечного до прожитку, і мусіла до висоти
того мінімум доповнюватися с „податку на бідних“. Постанови
„Устави робітницької“ що до згоди між майстром
а наємним робітником, що до вимовленя терміну і т. д.,
постанови дозволяючі тілько цівільну скаргу на недодержуючого
умови майстра, а крімінальну скаргу на недодержуючого
умови робітника, — ті постанови стоят ще й доси
в повній силі. Нелюдські ухвали супротів стоваришень
скасовано 1825 з ляку перед грізною поставою пролєтаріяту.
Парлямент зніс їх дуже нерадо\footnote{
Деякі останки устави протів стоваришень знесено аж 1859 р.

(Додаток до 2 вид.) Устава з 29 червня 1871 зносит всі устави
протів стоваришень і урядово признає „Робучі Звязки“ (Trades Unions).
Але в однім додатковім акті с того самого дня, п. н. „An Act to amend
the Criminal Law relating to violence, threats and molestation” — устави
протів стоваришень щасливо воскресли в новій формі. Сесь акт піддає
іменно робітників за вживанє деяких средств воєнних протів майстрів
під окремі устави крімінальні, а судят робітників на підставі тих устав
самі ж майстри, яко мирові судьї. Два роки передтим та сама палата
послів і тот сам Ґлядстон, що 1871 винайшли нові проступки на робітників,
вихвалювали при другім єго читанню один внесок до устави, в котрім
чесним способом роблено конець всяким окремим праводавствам
протів робітників. Вихвалювали, вихвалювали, тай хитро-мудро стали на
другім читанню\footnote*{Звісно, що в Англійськім парляменті кождий внесок,
заким одержит силу права, мусит бути три рази читаний і більшістю голосів
принятий. (\emph{Прим. перев.})}. Цілі два роки відволікано сю справу, аж
поки „велике ліберальне сторонництво“ не звязалось зі своїми противниками
і не почулося задосить сильним, щоб разом стати — протів спільного
ворога — робітників.
},  той сам парлямент, що
сам довгі столітя с цинічним безвстидством виступав як
неустаюче стоваришенє капіталістів супроті робітників.

Сейчас в початках революційної бурі поквапилась французька
буржоазія інощо здобуте право стоваришень знов
видерти робітникам. В декреті с 14 червня 1791 оголосила
вона, що всі робітницькі стоваришеня, се „замах на свободу
і признані права чоловіка“, за котрий накладаєсь кара
500 ліврів і позбавленє на рік актівних прав горожанських.
Се право, котре конкуренційну боротьбу між капіталом
а працею силою поліційно-державною втискає в такі границі,
які вигідні для капіталу, перетрівало революції та зміни
\parbreak{}

\parcont{}  %% абзац починається на попередній сторінці
\index{iii2}{0084}  %% посилання на сторінку оригінального видання
«і тому цією операцією ви мусити порушити вексельний курс, бо закордонний
борг не оплачено в наслідок того, що ваш експорт не має відповідного імпорту.
— Це правило для всіх країн взагалі».

Лекція Вілсона сходить на те, що всякий експорт без відповідного імпорту
становить одночасно імпорт без відповідного експорту; бо в продукцію товарів,
що їх експортують, ввіходять чужоземні, отже, імпортовані товари. Перед
нами припущення, що всякий такий експорт ґрунтується на неоплаченому
імпорті або породжує його, — отже, породжує борг закордонові, або ґрунтується
на ньому. Це — помилкова річ, навіть, коли не вважати на ті дві обставини,
що 1)~Англія має даремний імпорт, не платячи за нього жодного еквівалента;
напр., частину свого індійського імпорту. Індійський імпорт вона може обмінювати
на американський імпорт, експортуючи останній без еквівалентного імпорту;
щож до вартости, то в усякім разі Англія експортувала тільки те, що їй нічого
не коштувало; 2)~Англія може й оплатила імпорт, напр., американський, що
утворює додатковий капітал; коли вона той імпорт споживає непродуктивно,
напр., на військові припаси, то це не утворює боргу проти Америки та не
впливає на вексельний курс з Америкою. Newmarch суперечить сам собі в
посвідченнях 1934 та 1935, й Wood звертає його увагу на це в 1938: «Коли
жодна частина товарів, ужитих на виготовлення речей, що їх ми вивозимо без
зворотного припливу» [військові видатки] «не походить з тієї країни, куди ці
речі експортуються, то яким способом це впливатиме на вексельний курс з цією
країною? Нехай торговля з Турцією перебуває у звичайному стані рівноваги;
яким способом вивіз військових припасів до Криму вплине на вексельний курс
між Англією та Турцією?» — Тут Newmarch втрачає свою рівновагу, забуваючи,
що саме на це просте питання він дав уже слушну відповідь під № 1934, він
каже: «Ми вже, мені здасться, вичерпали практичне питання, а тепер увіходимо
в дуже високу ділянку метафізичної дискусії».

[Вілсон має ще й інше формулювання того свого твердження, що на вексельний
курс впливає всяке перенесення капіталу з однієї країни до іншої, однаково,
чи відбувається воно у формі благородного металу, чи у формі товарів.
Вілсон, природно, знає, що на вексельний курс впливає рівень проценту, а
саме, відношення чинних норм проценту в тих двох країнах, що їхній взаємний
вексельний курс розглядається. Отже, коли він буде в стані довести, що надмір
капіталу взагалі, отже, передусім надмір товарів всякого роду, в тім і благородного
металу, має разом з іншими обставинами вплив на рівень проценту, визначаючи
його, то він буде уже на крок ближче до своєї мети; перенесення
значної частини цього капіталу з однієї країни до іншої мусить змінити рівень
проценту в обох країнах, і то саме в протилежному напрямку а тому другою
чергою мусить воно змінити й вексельний курс між обома країнами. — \emph{Ф.~Е.}].

В Economist’і, що його він тоді редаґував, за рік 1847, на стор. 475,
він пише:

«Очевидно, що такий надмір капіталу, який виявляється у великих запасах
всякого роду, в тім і благородного металу, неминуче мусить привести не тільки
до низьких цін на товари взагалі, але й до нижчого рівня проценту за ужиток
капіталу1). Коли ми маємо запас товарів, достатній для того, щоб обслужити
потреби країни протягом двох наступних років, то порядкування цими
товарами протягом даного періоду можна здобути за далеко нижчу норму, ніж
тоді, коли того запасу вистачить ледви чи на два місяці2). Всякі позики грошей,
хоч і в якій формі їх робитиметься, являють лише передачу порядкування над
товарами від однієї особи до іншої. Тому, коли товарів є понад міру, грошовий
процент мусить бути низький, а коли товарів обмаль, він мусить бути високий3). Коли
\parbreak{}  %% абзац продовжується на наступній сторінці

\parcont{}
\index{franko}{0085}
капітал через ужитє наємних робітників і одну часть надвишки витворів, грішми чи натурою, платят
дідичови яко ренту ґрунтову. Доки в \RNum{15} віці незалежний мужик, а також сільский наймит, що попри
наймитство й сам про себе веде ґосподарство, збогачуются самі власною працею, доти й обставини тай
обсяг продукційний арендатора остаются дуже скромні. Переворот в рільництві, що почався в послідній
третині \RNum{15} віку і трівав через цілий \RNum{16} вік крім єго послідних десятиліть, збогатив го майже так
само прудко, як прудко зубожив мужиків\footnote{
„Арендаторі“, каже Гаррізен в своїй „Description of England“, „котрим давнійше годі було
заплатити 4\pound{ ф. шт.} ренти, платят тепер по 40, 50 та 100\pound{ ф. шт.} і ще кажут, що їм зле повелося, коли
по упливі арендового контракту не зложили бодай тілько готівки, кілько виносит 6--7-милітна рента“.
}. Загарбанє громадських пасовиск і т. д. дозволяє му
богато побільшувати число худоби майже без ніяких видатків, а між тим худоба достатчувала му далеко
більше обірнику для поправи ґрунту. В \RNum{16} віці причинюєсь ще одна рішучо важна обставина. Тоді
арендові контракти були довгі, нераз де з на 99 літ. А ту в \RNum{16} віці вартість золота та срібла, а
разом з ним і вартість грошей раз~у~раз вменьшуєсь, і арендаторам се принесло золоті плоди. Не
зважаючи на прочі, вперед згадані обставини, арендаторі першим ділом вменьшили робучу плату. Те, що
урвано робітникам на платі, побільшувало
арендовий зиск. А з другого боку ціна збіжя, вовни, мяса, — одим словом, всіх плодів рільничих,
раз~у~раз вбільшуєсь, через що змагаєся грошевий капітал арендатора
без єго причинку, — а притім ще ренту ґрунтову дідичови платит він давними, стратившими на вартости,
грішми. Таким способом він збогачуєсь рівночасно на кошт своїх наймитів і свого дідича. Не диво
% REMOVED (в рукоп. „даво“)
затим, що вже с кінцем \RNum{16} віку витворилась в Англії окрема верства як на тодішні
обставини богатих „капіталістичних“ арендаторів\footnote{
В Франції з „Regisseur-ів“, т. є. панських окономів та тивунів середновікових поробилися швидко
т. зв. hommes d'affaires, т. є. люде, що туманництвом та шахрайством подороблялися капіталів. Такі
окономи, то були нераз великі пани. Як в Англії, так і в Франції великі феодалні добра поділені були
на богато дрібних ґосподарств, але з условинами далеко гіршими для мужиків. В~\RNum{14} віці повстают і ту
аренди, звані ту „fermes“ або „terriers“. Число їх раз~у~раз змагалося і дійшло гет понад
\num{100000}. Вони платили чи то грішми чи натурою ренту ґрунтову, котра виносила від 12-тої до 5-тої
части річного здобутку. Ті terriers були цілими або частковими леннами як до вартости і обєму
ґрунтів, котрі нераз виносили заледво кілька прутів. Всі арендаторі мали до певної степені (степенів
було штири) власть судову над мужиками, жиючими
на їх ґрунтах. Лехко поняти, якого притиску мусів дізнавати люд від
усіх тих дрібних тиранів. Монтейль каже, що тоді було в Франції \num{160000}
судів, де тепер вистарчає (враз із мировими судами) 4000 трибуналів.
}.
\index{franko}{0086}

\subsection{Вліянє рільничого перевороту на промисл.
Промисловий капітал~здобуває собі в краю ринок
відбутовий}

Раптове і частими нападами повторюване вивласнюванє
та прогонюванє мужиків достатчило, як ми бачили,
міському промислови раз~за~разом маси пролєтаріїв, не належачих
зовсім до ніяких цехових звязків, — дуже мудра
подія, про котру старший Андерзен (не треба го мішати
з Джемсом Андерзеном) в своїй історії торговлі каже, що
се прямо боже провидініє так зробило. Ще хвилю мусимо
задержатися над тим складником первісного нагромадженя
капіталів. Не тілько що по селах убуло незалежного, самоґосподаруючого
мужицтва, а по містах прибуло промислового
пролєтаріяту, так, як після Жаффроа Сент-Улєра світової
матерії в одних місцях убуває, між тим коли в других
місцях вона згущаєсь. Помимо меньшого числа оброблюючих
рук ґрунт видавав прото однако або й ще більше
плодів, бо разом с переворотом в ґрунтових відносинах
власностевих настали також ліпші способи управи, більша
кооперація, зосередженє средств продукційних і т. д., а з другого боку
сільські наємники не тілько силувані були до тяжшої
праці — на се головно напирає сер Джемс Стеарт, —
а й обсяг їх домашної продукції, де вони працювали самі
на себе, чим раз більше вменьшувався. З освободженєм
одної части мужицтва освободжені зістали також єго давні
средства прожитку. Вони стают тепер матеріяльним складником
\so{змінного} капіталу\footnote*{
Звісно, що Маркс ділит капітал на постійний (constant) і змінний
(variabel), а то після того, чи в довшім протягу продукції вартість
єго зміняєся, чи ні. І так машини, сирий матеріял, будинки фабричні
і т. д., се капітал постійний, бо продукція не змінює в загальній сумі
єго вартости, а то, що убуде вартости на машинах і приладах і пр.,
котрі зуживаются при роботі, прибуває самим витворам, котрі через переробку
зискуют на вартости. Між тим друга часть капіталу, а іменно
тота, котра йде на наймленє і удержанє робітника і містится в понятю
робучої плати, се капітал змінний, бо по кождім процесі продукційнім
капіталіст добуває з него більше, ніж видав. Робітник витворює вартість
більшу, ніж тота, яку одержав в формі робучої плати. (\emph{Прим. перев.})
}. Бездомний та немаючий мужик
мусит окупувати собі ті средства прожитку від свого
нового пана, промислового капіталіста, в формі робучої
плати. Як зі средствами прожитку, так само сталося й з домашним
рільничим сирим матеріялом, котрого переробкою
займався промисл. Той сирий матеріял став частиною \so{постійного}
\index{franko}{0087}
капіталу. Се бачимо не тілько в Англії. За часів
Фрідріха II бачимо н. пр., що часть вестфальських мужиків,
котрі всі прядут лен, — хоть ще не шовк, — насилу
вивласнено і прогнано з хат і ґрунтів, а прочу часть перемінено
в наймитів великих арендаторів. Рівночасно повстают
великі прядильні і ткальні льну, де „освободжені“ наймаются
на роботу. Лен виглядає так само, як виглядав уперед.
Ані одно волоконце в нім не змінилося, але нова соціяльна
душа вступила в єго тіло. Тепер він становит часть постійного
капіталу панів мануфактуристів. Давнійше розділений
між множество дрібних витвірців, котрі го самі управляли
і пряли, він тепер згромадився в руках одного капіталіста,
котрий других заставляє для себе прясти і ткати. Виложена
в прядильни надвишка праці становила давнійше надвишку
доходу незлічених родин мужицьких, або також, за часів
Фрідріха II, йшла на extra-податки pour le roi de Prusse.
Тепер вона становит зиск немногих капіталістів. Веретена
і ткацькі станки, давнійше розсіяні широко по краю, тепер
стовпилися в кількох великих касарнях робучих, так само
й робітники, так само й сирий матеріял. І веретена і ткацькі
станки і сирі матеріяли зі средств незалежного прожитку
для прядильників і ткачів від тепер перемінюются в средства
командованя над ними і висисаня з них бесплатної
праці. По великих мануфактурах не видно того так, як по
\linebreak[4]
\makebox[\linewidth]{\dotfill}
\centerline{\emph{[На цьому уривається збережений рукопис Франка]}}

  \disablefootnotebreak{}

\index{ii}{0104}  %% посилання на сторінку оригінального видання
\chapter{Оборот капіталу}

\section{Час обороту й число оборотів}

\label{original-104}
Ми бачили: сукупний час циркуляції даного\footnote*{
Термін „сукупний час циркуляції“ тут Маркс вживає в тому самому розумінні,
в якому він далі в цьому ж розділі вживає термін „час обороту“, тимчасом
як взагалі він в цій книзі термін „час циркуляції“ вживає в тому самому
розумінні, що і „час обігу“, тобто в розумінні того часу, що протягом його капітал
перебуває в сфері циркуляції. (Дивись розділ V). \Red{Ред.}
} капіталу дорівнює сумі
часу його обігу та часу його продукції. Це є відтинок часу від моменту
авансування капітальної вартости в певній формі до моменту, коли капітальна
вартість, що процесує, повертається в тій самій формі.

Мета, що визначає капіталістичну продукцію, завжди є зростання
авансованої вартости, чи авансовано цю вартість в її самостійній формі,
тобто в грошовій формі, чи в формі товару, так що його форма вартости
має лише ідеальну самостійність у ціні авансованих товарів.
В обох випадках ця капітальна вартість перебігає протягом свого кругобігу
різні форми існування. Її тотожність з самою собою констатується
в книгах капіталіста або в формі рахункових грошей.

Хоч візьмемо ми форму $Г\dots{} Г'$, хоч форму $П\dots{} П$, обидві форми
значать: 1) що авансована вартість функціонувала як капітальна вартість
і зросла своєю вартістю; 2) що по закінченні процесу вона повернулась
до тієї форми, в якій почала його. Зростання авансованої вартости $Г$ і
разом з тим поворот капіталу до цієї форми (до грошової форми) виразно
помітно в $Г\dots{} Г'$. Але те саме відбувається і в другій формі. Бо
вихідний пункт для $П$ є наявність елементів продукції, товарів даної
вартости. Ця форма має в собі зростання цієї вартости ($Т'$ і $Г'$) і поворот
до первісної форми, бо в другому $П$ авансована вартість
знову має форму елементів продукції, що в ній її первісно авансовано.

Раніше ми бачили: „Якщо продукція має капіталістичну форму, то
і репродукція має ту саму форму. Як процес праці за капіталістичного
способу продукції є лише засіб для процесу зростання вартости, так
\parbreak{}  %% абзац продовжується на наступній сторінці

\parcont{}  %% абзац починається на попередній сторінці
\index{iii2}{0105}  %% посилання на сторінку оригінального видання
собі за передмову, з одного боку, визволення безпосередного продуцента з стану
простої приналежности до землі (в формі підвладного, кріпака, невільника і~\abbr{т. ін.})
а, з другого боку, експропріяцію землі в маси народу.

В цьому розумінні монополія на земельну власність є історична передумова й
лишається постійною основою капіталістичного способу продукції, як і всіх попередніх
способів продукції, що спираються на визиск мас в тій або іншій
формі. Але та форма земельної власности, що її знаходить капіталістичний
спосіб продукції на початку свого розвитку, не відповідає йому. Форму, що йому
відповідає, утворює лише він сам, підпорядковуючи хліборобство капіталові:
а тому й февдальна земельна власність, власність клану, або дрібна селянська
власність з громадою марки, хоч і які різні їхні юридичні форми,
перетворюються на економічну форму, що відповідає цьому способові продукції.
Одним з великих результатів капіталістичного способу продукції є те, що, з одного
боку, він перетворює хліборобство з простої емпіричної та механічної
традиційної методи найнерозвинутішої частини суспільства на свідомий науковий
ужиток аґрономії, оскільки це взагалі можливо серед умов, даних приватною
власністю\footnote{
Цілком консервативні аґрикультурні хеміки, як от, напр., В.~Johnston, визнають, що дійсно
раціональне хліборобство скрізь надибує непереможні межі в приватній власності. Те саме визнають
письменники, оборонці ex professe монополії приватної власности на землю, як от напр., пан Charles
Comte у двотомній праці, що має собі за спеціальну мету боронити приватну власність. «Народ» — каже
він — «не може досягнути того ступеня добробуту та сили, що визначається його природою, якщо
кожна частина тієї землі, що його годує, не одержить призначення, найбільш згідного з загальним
інтересом. Щоб значно розвинути свої багатства, мусила б по змозі єдина та передусім освічена воля
взяти до своїх рук розпорядок над кожним окремим кавалком своєї території та кожний кавалок зуживати
так, щоб тим допомагати поспіхові всіх інших. Але існування такої волі\dots{} не сила було б погодити
з поділом землі на приватні земельні ділянки\dots{} та з даною кожному власникові змогою майже абсолютно
порядкувати своїм майном». — Johnston, Comte і~\abbr{т. д.}, розглядаючи суперечність між власністю та
раціональною аґрономією, мають на оці тільки потребу обробляти землю певної країни як одну цілість.
Але залежність культури окремих продуктів землі від коливань ринкових цін, та невпинна зміна цієї
культури разом з тими коливаннями цін, увесь дух капіталістичної продукції, що простує до
безпосереднього
найближчого грошового зиску, — це все стоїть у суперечності до хліборобства, що йому
доводиться господарювати серед сукупних постійних життьових умов послідовних різних людських
ґенерацій.
Яскравий приклад цього є ліси, що ними іноді господарюють до певної міри в дусі громадських
інтересів
тільки там, де ті ліси не становлять приватної власности, а підлягають державному управлінню.
}; що, з одного боку, він цілком звільняє земельну власність від
відносин панування та підлеглости, а з другого боку, цілком відлучає землю
як умову праці від земельної власности та земельного власника, для якого та
земля не становить нічого більше, крім певного грошового податку, що його
він бере від промислового капіталіста фармера за посередництвом своєї монополії:
остільки рве цей зв’язок земельного власника з землею, що земельний власник
цілий свій вік може прожити в Костянтинополі, дарма що його земельна власність
буде в Шотландії. Оттак земельна власність одержує свою суто-економічну
форму, скидаючи з себе всі свої попередні політичні й соціяльні лямівки та
зв’язки, коротко — всі ті традиційні додатки, що їх, як ми пізніше побачимо, сами
капіталісти й їхні теоретичні проводирі в запалі своєї боротьби з земельною
власністю проголосили некорисною та безглуздою надмірністю. З одного боку,
раціоналізація хліборобства, що вперше дає змогу провадити його на суспільних
основах, з другого боку, доведення земельної власности до абсурду, — це великі
заслуги капіталістичного способу продукції. Як і всі свої інші історичні кроки
поступу, так само й ці, купив він передусім ціною повного зубожіння безпосередніх
продуцентів.

Раніше, ніж перейти до самої теми, треба зробити ще кілька попередніх
уваг, щоб уникнути непорозумінь.

Отже, передумова капіталістичного способу продукції така: дійсні хлібороби
— то наймані робітники, що мають працю від капіталіста, фармера, який
\parbreak{}  %% абзац продовжується на наступній сторінці

\parcont{}  %% абзац починається на попередній сторінці
\index{iii2}{0106}  %% посилання на сторінку оригінального видання
провадить сільське господарство, тільки як осібне поле експлуатації капіталу, як
приміщення свого капіталу в осібній сфері продукції. Цей фармер-капіталіст у
певні реченці, напр., щороку, платить земельному власникові, власникові визискуваної
ним землі, певну, контрактом усталену грошову суму (цілком так, як позикоємець
грошового капіталу платить певний процент) за дозвіл уживати свій капітал
на цьому осібному полі продукції. Ця грошова сума зветься земельною рентою,
все одно, чи платиться її від орної землі, будівельної ділянки, копалень, рибальства,
лісів і~\abbr{т. ін.} Її платиться протягом всього часу, що на нього земельний
власник за контрактом визичив, винайняв землю орендареві. Отже, земельна
рента становить тут ту економічну форму, що в ній земельна власність економічно
реалізується, даючи вартість. Далі, ми маємо тут усі три кляси — найманого
робітника, промислового капіталіста, земельного власника, — що всі разом
та одна проти однієї являють кістяк новітнього суспільства.

Капітал може бути зафіксований в землі, долучений до неї, почасти
більше тимчасово, як от при поліпшеннях хемічної натури, удобреннях і~\abbr{т. ін.},
почасти більше постійно, як от при дренажі, зрошувальних спорудах, нівелюваннях,
господарчих будівлях і~\abbr{т. ін.} В іншому місці я назвав капітал, що
отак долучається до землі, la terre-capital\footnote{
Misère de la Philosophie р. 165. Там я розрізняю terre-matière і terre-capital. «Досить лише
примістити до ділянок землі, вже перетворених на засоби продукції, нові суми капіталу, — і ми
збільшуємо la terre-capital, ані трохи не збільшуючи la terre-matière, тобто простору землі\dots{} La
terre-capital так само не є вічний, як і всякий інший капітал. t. La terre-capital e основний
капітал, але основний капітал зужитковується так само, яві оборотні капітали».
}. Він належить до категорії основного
капіталу. Процент за долучений до землі капітал та за поліпшення, що їх вона
тим способом одержує, як знаряддя продукції, може становити частину тієї
ренти, що її платить фармер земельному власникові\footnote{
Я кажу «може», бо в певних обставинах цей процент регулюється законом земельної ренти, а тому як
от при конкуренції нових земель, що мають велику природну родючість, може зникнути.
}, однак ця частина не
являє собою власне земельної ренти, що її платиться за користування землею
як такою, однаково, чи перебуває та земля в природному стані, — чи її культивується.
Коли б ми систематично — що до нашого плану не належить — розглядали
земельну власність, то треба було б докладно з’ясувати цю частину
доходу земельного власника. Тут досить буде сказати кілька слів про це.
Капіталовкладання більш тимчасового характеру, що їх викликають звичайні
процеси продукції в хліборобстві, всі без винятку переводить фармер. Ці вкладання,
як і простий обробіток землі взагалі, коли його проводять до певної
міри раціонально, отже коли він не сходить до брутального виснажування
ґрунту, як от, напр., в колишніх американських рабовласників, — проти чого
однак панове земельні власники забезпечують себе в контракті, — ці вкладання
поліпшують ґрунт\footnote{
Дивись James Anderson і Сагеу.
}, збільшують кількість продуктів землі та перетворюють
землю з простої матерії на землю — капітал. Оброблене поле більш варте, ніж
необроблене тієї самої природної якости. І основні капітали, що долучені до
землі на довший час, зужитковуються протягом довгого часу, витрачає, здебільшого,
фармер, а в деяких сферах іноді тільки сам фармер. Коли ж усталений в
договорі час оренди мине — і це одна з причин, чому з розвитком капіталістичного
способу продукції земельний власник силкується по змозі дужче скоротити
час оренди, — то пороблені в землі поліпшення як приналежність невідійманна
від субстанції, від землі, припадають як власність власникові тієї землі.
Роблячи новий орендний контракт, земельний власник додає до власне земельної
ренти процент на капітал, долучений до землі; однаково, чи винаймає він землю
тому фармерові, що ті поліпшення поробив, — чи якомусь іншому фармерові.
Таким чином його рента бубнявіє; або, коли він хоче продати землю — ми далі
\parbreak{}  %% абзац продовжується на наступній сторінці

\parcont{}  %% абзац починається на попередній сторінці
\index{ii}{0107}  %% посилання на сторінку оригінального видання
наприклад, коли час обороту $о$ становить три місяці, то $n \deq{} \sfrac{12}{3} \deq{} 4$; капітал
робить чотири обороти на рік, або обертається чотири рази.
Коли $о \deq{} 18$ місяцям, то $n \deq{} \sfrac{12}{18} \deq{} \sfrac{2}{3}$ або капітал протягом року проходить
лише \sfrac{2}{3} часу свого обороту. Коли час обороту його дорівнює кільком
рокам, то він, отже, обчислюється одним роком, повтореним кілька
разів.

Для капіталіста час обороту його капіталу є час, що протягом його він
мусить авансувати свій капітал для того, щоб він збільшився вартістю
й повернувся в своїй первісній формі.

Перш ніж перейти до ближчого розгляду того впливу, що його
оборот справляє на процес продукції та процес зростання вартости, треба
розглянути дві нові форми, що підступають до капіталу із процесу циркуляції
та впливають на форму його обороту.

\sectionextended{Основний капітал і обіговий капітал}{\subsection{Відмінності форми}}

В книзі І, розділ VI, ми бачили, що частина сталого капіталу зберігає
ту певну споживну форму, що в ній вона увіходить у процес
продукції, проти тих продуктів, що їх утворенню вона сприяє. Отже,
протягом більш або менш довгого періоду, в постійно повторюваних
процесах праці, вона завжди виконує ті самі функції. Такі, напр., майстерні,
машини і~\abbr{т. ін.}, коротко кажучи — все те, що ми об’єднуємо
під назвою \emph{засоби праці}. Ця частина сталого капіталу віддає свою
вартість продуктові, в міру того, як вона разом з своєю споживною
вартістю втрачає свою мінову вартість. Цю передачу вартости або перехід
вартости таких засобів продукції на продукт, що в утворенні його
вони беруть участь, визначається за пересічним обчисленням; її вимірюється
пересічним протягом функціонування засобів продукції, від
того моменту, коли вони ввіходять в процес продукції, і до моменту
коли вони цілком зносяться, знищаться, коли їх треба буде замінити на
нові екземпляри такого ж роду, або репродукувати.

Отже, своєрідність цієї частини сталого капіталу — власне засобів
праці — ось у чому:

Частину капіталу авансується в формі сталого капіталу, тобто в формі
засобів продукції, що функціонують як чинники процесу праці, поки
зберігають ту самостійну споживну форму, що в ній вони ввіходять
у процес праці. Готовий продукт, а значить і продуктотворчі елементи,
оскільки їх перетворено на продукт, виштовхується з продукційного
процесу, щоб перейшли вони як товар з сфери продукції до сфери циркуляції.
\index{ii}{0108}  %% посилання на сторінку оригінального видання
Навпаки, засоби праці, ввійшовши в сферу продукції, вже ніколи
не облишають її. Їх міцно прив’язує до неї їхня функція. Частину
авансованої капітальної вартости \emph{фіксується} в цій формі, визначуваній
функцією засобів праці в продукційному процесі. В міру функціонування,
а тому і в міру зношування засобів праці частина їхньої вартости
переходить на продукт, а друга лишається зафіксована в засобах
праці, а значить, і в продукційному процесі. Фіксована таким чином
вартість завжди меншає, поки засоби праці не відслужили свого часу;
тому вартість їхня протягом більш або менш довгого періоду розподіляється
на масу продуктів, що виходять з ряду постійно повторюваних
процесів праці. Але поки засоби праці все ще діють як засоби праці,
тобто, поки їх не треба заміняти на нові екземпляри такого самого роду,
вартість сталого капіталу ввесь час лишається фіксована в них, тимчасом
як друга частина первісно фіксованої в них вартости переходить на
продукт, а тому циркулює, як складова частина товарового запасу. Що
триваліші засоби праці, що повільніше вони зношуються, то довший час
вартість сталого капіталу лишається фіксована в цій споживній формі.
Але хоч яка буде тривалість засобів праці, пропорція, що в ній вони
передають свою вартість, завжди стоїть у зворотному відношенні до
загального часу функціонування їх. Коли з двох машин однакової
вартости одна зношується протягом п’ятьох років, а друга протягом десятьох,
то за однаковий час перша віддає вдвоє більше вартости, ніж
друга.

Ця частина капітальної вартости, фіксована в засобах праці, циркулює
так само, як і всяка інша частина. Ми взагалі бачили, що вся капітальна
вартість перебуває в постійній циркуляції, і тому в цьому розумінні
ввесь капітал є обіговий капітал. Але циркуляція розглядуваної
тут частини капіталу своєрідна. Поперше, вона циркулює не в своїй
споживній формі, але циркулює лише її вартість, і до того лише поступінно,
частинами, в міру того, як вона переходить на продукт, що циркулює
як товар. Протягом усього часу, коли функціонує ця частина капіталу,
деяка частка її вартости лишається фіксована в ній як самостійна
проти товарів, що продукуванню їх вона допомагає. В наслідок цієї
особливости ця частина сталого капіталу набирає форми \emph{основного
капіталу}. Протилежно до нього, всі інші речові складові частини капіталу,
авансованого на продукційний процес, становлять, навпаки, \emph{обіговий}
або \emph{поточний} капітал.

Частина засобів продукції, — саме такі допоміжні матеріяли, що їх
споживають самі засоби праці підчас свого функціонування, як от вугілля
в паровій машині, або такі, що лише допомагають процесові,
напр., світильний газ тощо, — речово не ввіходить у продукт. Тільки її
вартість становить частину вартости продукту. В своїй власній циркуляції
продукт несе в циркуляцію і вартість таких засобів продукції. Це в
них спільне з основним капіталом. Але в кожному процесі праці, куди
вони ввіходять, їх зужитковується цілком, і тому треба для кожного нового
процесу праці замінити їх цілком на нові екземпляри того самого
\parbreak{}  %% абзац продовжується на наступній сторінці

\parcont{}  %% абзац починається на попередній сторінці
\index{ii}{0109}  %% посилання на сторінку оригінального видання
роду. Підчас свого функціонування вони не зберігають своєї самостійної
споживної форми. Отже, підчас їхнього функціонування жодна частина
капітальної вартости не лишається фіксована в своєму першому споживному
вигляді, в своїй натуральній формі. Та обставина, що ця частина
допоміжних матеріялів речово не входить в продукт, але ввіходить у
вартість продукту лише своєю вартістю, як частина його вартости, і що
в зв’язку з цим функціонування таких матеріялів міцно прикріплено до
сфери продукції, — призвело деяких економістів, напр., Рамсая, до того,
що вони (одночасно сплутуючи основний і сталий капітал) зачислили їх
до категорії основного капіталу.

Частина засобів продукції, що речово ввіходить у продукт, отже,
сировинний матеріял і~\abbr{т. ін.}, набуває в наслідок цього почасти таких
форм, що в них вона може пізніше ввійти в особисте споживання як
засоби споживання. Власне засоби праці, речові носії основного капіталу,
споживається лише продуктивно, і не можуть вони ввійти в особисте
споживання, бо вони не ввіходять у продукт або в ту споживну вартість,
що її вони допомагають утворити, а, навпаки, зберігають проти
неї свою самостійну форму ввесь час, поки вони цілком зносяться. Виняток
становлять тільки засоби транспорту. Корисний ефект, що його
вони дають підчас свого продуктивного функціонування, тобто підчас
перебування в сфері продукції, — зміна місця, ввіходить одночасно в особисте
споживання, напр., пасажира. Він також оплачує тут споживання,
як оплачує користування з інших засобів споживання. Ми бачили, що
сировинний матеріял і допоміжні матеріяли іноді зливаються один з
одним, напр., у хемічній фабрикації. Те саме буває й з засобами праці
та допоміжними матеріялами й сировинним матеріялом. Напр., в хліборобстві
речовини, вкладені для поліпшення ґрунту, почасти ввіходять як продуктотворчі
елементи в рослинний продукт. З другого боку, їхня дія
розподіляється на відносно довгий період, напр., 4--5 років. Тому частина
їх речово ввіходить у продукт і разом з тим переносить свою
вартість на продукт, тимчасом як друга частина зберігає свою стару
споживну форму і фіксує в ній свою вартість. Вона і далі існує як засіб
продукції і тому набирає форми основного капіталу. Як робоча худоба
віл є основний капітал. А якщо його з’їдають, він функціонує вже не як
засіб праці, отже, не як основний капітал.

Причина (Bestimmung), що надає частині капітальної вартости, витраченій
на засоби продукції, характеру основного капіталу, є виключно в
своєрідному способові циркуляції цієї вартости. Цей особливий спосіб
циркуляції випливає з того особливого способу, що ним засоби праці
віддають свою вартість продуктові, або з того способу, в який вони виступають
як вартіснотворчі чинники підчас продукційного процесу. А цей останній
й собі випливає з особливого способу функціонування різних засобів
праці в процесі праці.

Відомо, що та сама споживна вартість, що виходить як продукт з
одного процесу праці, ввіходить у другий як засіб продукції. Тільки
функціонування продукту як засобу праці в продукційному процесі робить
\index{ii}{0110}  %% посилання на сторінку оригінального видання
його основним капіталом. Навпаки, коли він ще тільки сам виходить
з процесу, він зовсім не є основний капітал. Напр., машина, як продукт,
зглядно товар фабриканта-машинобудівника належить до його
товарового капіталу. Основним капіталом вона стає лише в руках покупця,
капіталіста, що продуктивно її вживає.

Припускаючи всі інші умови за однакові, ступінь зв’язаности основного
капіталу зростає разом із тривалістю засобів праці. Саме від цієї
тривалости залежить величина ріжниці між капітальною вартістю, фіксованою
в засобах праці, і тією частиною капітальної вартости, яку вона
в повторюваних процесах праці віддає продуктові. Що повільніше відбувається
ця передача вартости, — а вартість передається з засобів праці
при всякому повторенні того самого процесу праці, — то більший фіксований
капітал, то більша ріжниця між капіталом, застосованим у продукційному
процесі, і капіталом, що в ньому зужитковується. Скоро ця
ріжниця зникає, це значить, що засіб праці віджив свій час і разом із
своєю споживною вартістю втратив свою вартість. Він перестав бути
носієм вартости. Через те, що засіб праці, як і кожний інший речовий
носій сталого капіталу, віддає свою вартість продуктові лише в тих розмірах,
в яких разом із споживною вартістю він втрачає і вартість, то
очевидно, що як повільніше втрачає він свою споживну вартість, що
довше він перебуває в продукційному процесі, то й довший буде період,
протягом якого вартість сталого капіталу лишається в ньому фіксована.

Коли засіб продукції, що не є засіб праці у власному розумінні,
напр., допоміжний матеріял, сировинний матеріял, напівфабрикат тощо,
перенесенням своєї вартости, а тому й способом циркуляції своєї вартости
відіграє таку саму ролю як засоби праці, то він так само є речовий
носій, форма існування основного капіталу. Так буває при вищезгаданих
земельних меліораціях, коли в ґрунт додається хемічні складові
частини, що їхнє діяння поширюється на багато продукційних періодів
або років. Тут частина вартости і далі існує поряд продукту в
своїй самостійній формі або в формі основного капіталу, тимчасом як друга
частина вартости передається на продукт, а тому разом з ним циркулює.
В цьому разі в продукт входить не лише частина вартости основного капіталу,
а й споживна вартість, та субстанція, що в ній існує ця частина вартости.

Лишаючи осторонь основну помилку — сплутування категорій: основний
і обіговий капітал з категоріями: сталий і змінний капітал — плутанина
в дотеперішньому визначенні понять в економістів ґрунтується насамперед
на таких пунктах.

Певні властивості, речово належні засобам праці, вони перетворюють
на безпосередні властивості основного капіталу, напр., таку, як фізична
нерухомість хоча б будинку. Але завжди легко довести, що інші засоби
праці, що, як такі, теж є основний капітал, мають протилежні властивості,
напр., фізична рухомість хоча б корабля.

Але економічну визначеність форми, що походить з циркуляції вартости,
вони сплутують з речовою властивістю; ніби речі, які самі собою
взагалі не є капітал, а робляться ним лише в певних суспільних відносинах,
\index{ii}{0111}  %% посилання на сторінку оригінального видання
могли б \emph{самі собою} та із своєї природи бути капіталом в
тій або іншій певній формі, основним або обіговим. Ми бачили в книзі
І, розділ V, що засоби продукції в кожному процесі, хоч при яких суспільних
умовах він відбувається, завжди поділяються на засоби праці й
предмет праці. Але тільки за капіталістичного способу продукції обидва
вони робляться капіталом, саме „продуктивним капіталом“, як це
визначено в попередньому розділі. Разом з тим ріжниця між засобом
праці й предметом праці, яка ґрунтується на природі процесу праці,
відбивається в новій формі, як ріжниця між основним капіталом та обіговим.
Лише відтепер річ, що функціонує як засіб праці, робиться основним
капіталом. Якщо вона своїми речовими властивостями може придаватись
і в інших функціях, крім функцій засобів праці, то вона буде
основним капіталом або не буде, залежно від відмінности свого функціонування.
Худоба як робоча худоба, є основний капітал; худоба на заріз
є сировинний матеріял, що, кінець-кінцем, як продукт, входить у
циркуляцію — отже, це не основний капітал, а обіговий.

Простий стан довгочасної фіксованости якогобудь засобу продукції
в повторюваних процесах праці, що між собою зв’язані й являють
безперервний ряд і тому становлять період продукції — тобто ввесь час
продукції, потрібний на те, щоб виготувати продукт, — цей стан довгочасної
фіксованости зумовлює цілком так само, як і основний капітал,
авансування з боку капіталіста на довший або коротший час, але не
перетворює його капіталу на основний капітал. Насіння, напр., зовсім не
є основний капітал, а лише сировинний матеріял, що його майже на
цілий рік фіксується в процесі продукції. Всякий капітал, поки він функціонує
як продуктивний капітал, фіксується в процесі продукції, отже,
фіксуються і всі елементи продуктивного капіталу, хоч яка буде їхня речова
форма, їхня функція та спосіб циркуляції їхньої вартости. Чи це фіксування
триває довший, чи коротший час, залежно від способу продукційного
процесу або бажаного корисного ефекту, не це утворює ріжницю між
основним та обіговим капіталом\footnote{
Що дуже важко дати визначення основного та обігового капіталу, то пан
Льоренц Штайн каже, що ця ріжниця придається лише для популярности викладу.
}.

Частина засобів праці, — куди належать і загальні умови праці — або
прикріплюється до певного місця, коли вона як засіб праці входить у процес
продукції, зглядно, коли її підготовлюється до продуктивної функції, як
напр., машини. Або частину засобів праці з самого початку продукується
в такій нерухомій, зв’язаній з місцем формі, як, напр., земельні
меліорації, фабричні будівлі, домни, канали, залізниці тощо. Постійна
зв’язаність засобів праці з продукційним процесом, що в ньому вони
повинні функціонувати, зумовлюється тут уже речовим способом їхнього
існування. З другого боку, засоби праці можуть фізично завжди переміщуватись,
рухатись, і все ж бути завжди в продукційному процесі, як,
напр., локомотив, судно, робоча худоба і~\abbr{т. ін.} Ні нерухомість не надає
їм у першому випадку характеру основного капіталу, ні рухомість не
\parbreak{}  %% абзац продовжується на наступній сторінці

\parcont{}  %% абзац починається на попередній сторінці
\index{iii1}{0112}  %% посилання на сторінку оригінального видання
Досі на охоронний клапан навішували такий тягар, що він відкривався вже при тисненні пари в 4, 6 або
8 фунтів на квадратний
дюйм; тепер виявили, що підвищенням тиснення до 14 або 20 фунтів\dots{} можна досягти дуже значного
заощадження вугілля; інакше
кажучи, фабрика почала працювати при значно меншому споживанні вугілля\dots{} Ті, що мали для цього
засоби й сміливість, стали
застосовувати систему збільшеного тиснення і розширення в повному її обсягу і застосовували
відповідно до цього збудовані парові казани, які давали пару тисненням в 30, 40, 60 і 70 фунтів на
квадратний дюйм — тиснення, при якому інженер старої школи
від страху зомлів би. Але через те що економічний результат
цього підвищеного тиснення пари\dots{} виявився дуже швидко в цілком недвозначній формі фунтів, шилінгів
і пенсів, парові казани
високого тиснення при конденсаційних машинах стали майже загальним явищем. Ті, що провели реформу
радикально, стали застосовувати вульфові машини, і це мало місце щодо більшості недавно
збудованих машин; вони стали застосовувати особливо вульфові
машини з 2 циліндрами, в одному з яких пара з казана розвиває силу в наслідок перевищення тиснення
над тисненням атмосфери і потім, замість того щоб після кожного підіймання поршня
виходити у повітря, як це було раніш, входить у циліндр
низького тиснення, приблизно вчетверо більший обсягом, і, розширившись там далі, відводиться в
конденсатор. Економічний
результат, одержуваний при таких машинах, полягає в тому, що
одна кінська сила за одну годину добувається при споживанні
3\sfrac{1}{2}—4 фунтів вугілля; тимчасом як при машинах старої системи
для цього потрібно було від 12 до 14 фунтів. За допомогою
майстерного пристрою вульфову систему подвійного циліндра
або комбінованої машини високого й низького тиснення удалось
пристосувати до наявних уже старих машин і таким чином підвищити їх ефективність при одночасному
зменшенні споживання
вугілля. Того самого результату досягнуто протягом останніх
8--10 років за допомогою сполучення машини високого тиснення
з конденсаційною машиною таким чином, що спожита пара першої переходила в другу і пускала її в рух.
Така система корисна
в багатьох випадках“.

„Не легко було б точно встановити, наскільки збільшилась
ефективність праці тих самих колишніх парових машин, до
яких пристосовані деякі або й усі ці нові поліпшення. Але я певен, що на ту саму вагу парової машини
ми одержуємо тепер
пересічно принаймні на 50\% більше корисної роботи і що в багатьох випадках та сама парова машина,
яка в часи обмеженої
швидкості в 220 футів на хвилину давала 50 кінських сил, дає
тепер понад 100. Надзвичайно ефективні щодо економії результати
застосування пари високого тиснення при конденсаційних машинах, так само як і далеко більші вимоги,
які ставляться до старих парових машин з метою розширення підприємств, привели
за останні три роки до введення трубчастих казанів, в наслідок
\parbreak{}  %% абзац продовжується на наступній сторінці

\parcont{}  %% абзац починається на попередній сторінці
\index{ii}{0113}  %% посилання на сторінку оригінального видання
репродукції вартість машини поступінно акумулюється насамперед в формі резервного грошового фонду.

Інші елементи продуктивного капіталу складаються почасти з елементів сталого капіталу, які є в
допоміжних матеріялах та сировинних матеріялах, а почасти із змінного капіталу, витраченого на
робочу силу.

\vtyagnut{}
Аналіза процесу праці й процесу зростання вартости (книга І, розділ V) виявила, що ці різні складові
частини відіграють цілком різну ролю в утворенні продукту і в утворенні вартости. Вартість тієї
частини сталого капіталу, яка складається з допоміжних та сировинних матеріялів — цілком так само,
як і вартість тієї його частини, яка складається з засобів праці — знову з’являється в вартості
продукту, як лише перенесена вартість, тимчасом як робоча сила за посередництвом процесу праці додає
до продукту еквівалент своєї вартости, або дійсно репродукує свою вартість. Далі, одну частину
допоміжних матеріялів, — вугілля на опалення, світильний газ тощо, — зужитковується в процесі праці,
при чому речово вона не увіходить у продукт, тимчасом як друга частина їх своїм тілом увіходить у
продукт і становить матеріял його субстанції. Але всі ці відмінності не мають значення для
циркуляції, а тому й для способу обороту. Коли допоміжні й сировинні матеріяли цілком зужитковується
підчас утворення певного продукту, то вони цілком переносять свою вартість на продукт. Тому вона
через продукт цілком подається в циркуляцію, перетворюється на гроші, а з грошей знову на елементи
продукції товару. Її оборот не переривається, як оборот основного капіталу, але безупинно перебігає
весь кругобіг своїх форм, так що ці елементи продуктивного капіталу постійно відновлюються in
natura.

Щодо змінної складової частини продуктивного капіталу, витрачуваної на робочу силу, то робочу силу
купується на певний час. Коли капіталіст купить її і введе в продуктивний процес, то вона утворює
складову частину його капіталу, а саме — його змінну частину. Вона діє щоденно певний час, що
протягом його вона додає до продукту не лише всю свою денну вартість, а також ще деяку надлишкову
додаткову вартість, яку ми покищо залишаємо осторонь. Після того, як робочу силу куплено, й вона
діяла, напр., протягом тижня, закуп її мусить постійно відновлюватися у певні терміни. Той
еквівалент її вартости, що його робоча сила долучає до продукту протягом свого функціонування і що в
наслідок циркуляції продукту перетворюється на гроші, мусить завжди знову перетворюватись з грошей
на робочу силу або завжди мусить пророблювати повний кругобіг своїх форм, тобто завжди обертатись,
щоб не перервався кругобіг безперервної продукції.

\roztyagnut
Отже, частина вартости продуктивного капіталу, авансована на робочу силу, цілком переходить на
продукт (додаткову вартість ми залишаємо ввесь час осторонь), разом з ним перебігає обидві
метаморфози, що належать до сфери циркуляції, і в наслідок цього постійного відновлення завжди
лишається зв’язана з продукційним процесом. Отже, хоч як у всьому іншому робоча сила відрізняється
щодо утворення вартости від тих складових частин сталого капіталу, які не становлять \emph{основного}
\index{ii}{0114}  %% посилання на сторінку оригінального видання
\emph{капіталу}, спосіб обороту вартости є спільний у робочої сили з цими складовими частинами,
протилежно до основного капіталу. Ці складові частини продуктивного капіталу, — а саме ті частини
його вартости, що їх витрачається на робочу силу й засоби продукції, які не становлять основного
капіталу — в наслідок цієї спільности характеру їхнього обороту, — протистоять основному капіталові,
як \emph{обіговий} або \emph{поточний} капітал.

Як ми бачили раніше, гроші, що їх капіталіст сплачує робітникові за вживання робочої сили, справді є
лише загальна еквівалентна форма доконечних робітникові засобів існування. Остільки й змінний
капітал речово складається з засобів існування. Але тут, розглядаючи оборот, ідеться про форму.
Капіталіст купує не засоби існування робітника, але саму його робочу силу. Змінну частину його
капіталу являють не засоби існування робітника, але його діюща робоча сила. В процесі праці
капіталіст продуктивно споживає саму робочу силу, а не засоби існування робітника. Сам робітник
перетворює на засоби існування ті гроші, що їх він одержав за свою робочу силу, щоб потім
перетвороти знову ці засоби існування на робочу силу й підтримати своє існування, цілком так само,
як, напр., капіталіст перетворює на засоби свого існування деяку частину додаткової вартости, що є в
товарі, який він продає за гроші, і, не зважаючи на це, зовсім не можна сказати, що покупець його
товарів сплачує йому засобами існування. Навіть коли робітникові сплачується частину його заробітної
плати в засобах існування in natura, то це за наших часів є вже друга оборудка. Він продає свою
робочу силу за певну ціну і при цьому умовляється, що частину цієї ціни він одержить в засобах
існування. Цим змінюється лише форма виплати, але не змінюється та обставина, що він дійсно продає
свою робочу силу. Це є друга оборудка, що відбувається вже не між робітником і капіталістом, а між
робітником як покупцем товару і капіталістом як продавцем товару; тимчасом як у першій оборудці
робітник є продавець товару (своєї робочої сили), а капіталіст її покупець. Цілком так само, як коли
б капіталіст, продаючи свій товар, прим., машину на гамарню, захотів мати за неї товар — залізо.
Отже, не засоби існування робітника визначаються як обіговий капітал протилежно до основного. А
також і не робоча сила його, а частина вартости продуктивного капіталу, витрачена на робочу силу,
яка через форму свого обороту набуває цього характеру обігового капіталу, спільно з деякими
складовими частинами сталого капіталу і протилежно до деяких інших складових частин сталого
капіталу.

Вартість поточного капіталу — в робочій силі та засобах продукції — авансується лише на той час, що
протягом його виготовляється продукт, залежно від маштабу продукції, визначуваного розміром
основного капіталу. Ця вартість цілком увіходить у продукт, а тому після продажу продукту цілком
повертається з циркуляції, і можна знову її авансувати. Робоча сила й засоби продукції, що в них
існує поточна складова частина капіталу, вилучається з циркуляції в розмірі, потрібному на
вироблення й продаж готового продукту, але їх завжди треба замінювати
\parbreak{}  %% абзац продовжується на наступній сторінці


\index{iii1}{0115}  %% посилання на сторінку оригінального видання
Шерстяна промисловість була розсудливіша, ніж льонообробна промисловість. „Раніше, звичайно,
вважалося ганебним
збирати відпади шерсті та шерстяні ганчірки для повторного
перероблення, але цей передсуд цілком зник у shoddy trade
(виробництві штучної шерсті), яке стало важливою галуззю
шерстяної промисловості Йоркшірської округи, і немає сумніву,
що й підприємства, що переробляють відпади бавовни, незабаром теж займуть те саме місце, як галузь
промисловості, що
задовольняє визнані потреби. 30 років тому шерстяні ганчірки,
тобто шматки тканини з чистої вовни і~\abbr{т. д.}, коштували пересічно щось 4\pound{ фунти стерлінгів} 4\shil{ шилінги}
за тонну; протягом
останніх кількох років вони стали коштувати 44\pound{ фунти стерлінгів} за тонну. А попит на них так
збільшився, що використовується
навіть мішана тканина з вовни й бавовни, бо знайдено засіб
руйнувати бавовну без пошкодження вовни; і тепер тисячі робітників зайняті у фабрикації shoddy, а
споживач має з того
велику користь, оскільки він тепер може купити сукно доброї
середньої якості за дуже помірну ціну“ („Rep. of Insp. of Fact.,
Oct. 1863“, стор. 107). Поновлювана таким чином штучна шерсть
уже в кінці 1862 року становила третину всього споживання
вовни англійською промисловістю („Rep. of Insp. of Fact., Oct.
1862“, стор. 81). „Велика користь“ для „споживача“ полягає в тому,
що його шерстяний одяг зношується втричі швидше, ніж раніше,
і вшестеро швидше витирається до ниток.

Англійська шовкова промисловість посувалась по тій самій
похилій площині. З 1839 до 1862 року споживання натурального
шовку-сирця трохи зменшилося, тим часом як споживання шовкових відпадів подвоїлося. Поліпшені машини
дали змогу фабрикувати з цього, за інших умов майже нічого не вартого матеріалу, шовк, придатний для
багатьох цілей.

Найразючіший приклад застосування відпадів дає хімічна
промисловість. Вона споживає не тільки свої власні відпади,
знаходячи для них нове застосування, але й відпади найрізнорідніших інших галузей промисловості, і
перетворює, наприклад,
майже некорисний раніше кам’яновугільний дьоготь в анілінові
фарби, в красильну речовину крапу (алізарин), а останнім часом
також у медикаменти.

Від цієї економії на відпадах виробництва внаслідок повторного використання їм треба відрізняти
економію при утворенні
самих відпадів, тобто зведення екскрементів виробництва до
їх мінімуму і безпосереднє максимальне використання всіх сировинних та допоміжних матеріалів, що
входять у виробництво.

Заощадження на відпадах почасти зумовлене якістю застосовуваних машин. Мастило, мило тощо
заощаджуються тим
більше, чим точніше працюють окремі частини машин, і чим краще
вони відполіровані. Це стосується до допоміжних матеріалів.
А почасти, і це найважливіше, від якості застосовуваних машин
і знарядь залежить те, більша чи менша частина сировинного
\parbreak{}  %% абзац продовжується на наступній сторінці

\parcont{}  %% абзац починається на попередній сторінці
\index{iii2}{0116}  %% посилання на сторінку оригінального видання
що їх виплачується під титулом земельної ренти власникові землі за використання
ґрунту чи то з метою виробництва чи то споживання, треба пам’ятати, що
ціна речей, які самі по собі не мають вартости, тобто не є продукти праці, як
земля, або, принаймні, не можуть бути відтворені працею, як старовинні речі,
художні вироби певних майстрів тощо, може визначатися дуже випадковими
комбінаціями.

Щоб продати річ, для цього не треба нічого іншого, як тільки того, щоб
вона могла зробитись об’єктом монополії і відчуження.

\pfbreak{} % see russ. book

Є три головні помилки, що їх при розгляді земельної ренти треба уникати
й що затемнюють аналізу.

1) Сплутування різних форм ренти, відповідних різним ступеням розвитку
суспільного процесу продукції.

Хоч би яка була специфічна форма ренти, всім її типам є спільне те, що привласнення
ренти є економічна форма, що в ній реалізується земельна власність і що
земельна рента в свою чергу має за свою передумову земельну власність, власність
певних індивідуумів на певні дільниці землі, чи власником буде особа, що репрезентує
громаду як в Азії, Єгипті тощо, чи ця земельна власність буде привхідною
обставиною власности певних осіб на особи безпосередніх продуцентів, як за системи
рабства або кріпацтва, чи ж земельна власність буде суто приватною
власністю непродуцентів на природу, простим титулом власности на землю, чи,
нарешті, це буде таке відношення до землі, що як от у колоністів і дрібноселянських
землевласників, за ізольованої і соціально-нерозвиненої праці, виступає,
як відношення безпосередньо дане привласненням і виробництвом продуктів
на певних дільницях землі безпосередніми продуцентами.

Ця \emph{спільність} різних форм ренти — те, що вона являє собою економічну
реалізацію земельної власности, юридичної фікції, в силу якої ріжним індивідуумам
належить виключне володіння певними дільницями землі, — призводить до
того, що ріжниці форм не помічаються.

2) Всяка земельна рента є додаткова вартість, продукт додаткової праці.
У своїй нерозвиненій формі, у формі натуральної ренти, вона ще є безпосередньо
додатковий продукт. Звідси та помилка, ніби та рента, що відповідає капіталістичному
способові продукції і яка завжди становить надлишок над зиском,
тобто над тією частиною вартости товару, що сама складається з додаткової вартости
(додаткової праці), — ніби ця особлива і специфічна складова частина додаткової
вартости буде пояснена тим, що будуть пояснені загальні умови існування додаткової
вартости і зиску взагалі. Ці умови такі: безпосередні продуценти мусять працювати
понад той час, який потрібен для репродукції їхньої власної робочої
сили, для репродукції їх самих. Вони взагалі мусять виконувати додаткову працю.
Це — суб’єктивна умова. А об’єктивна є в тому, щоб у них була і \emph{можливість}
виконувати додаткову працю; щоб природні умови були такі, щоб лише деякої
\emph{частини} робочого часу, яким вони порядкують, було досить для їхньої репродукції
і самозбереження як продуцентів; щоб продукція потрібних засобів їхнього
існування не забирала всієї їхньої робочої сили. Родючість природи становить
тут одну межу, один вихідний пункт, одну основу. З другого боку, розвиток суспільної
продуктивної сили праці становить тут другу межу. Розглядаючи справу
ще ближче, можна сказати: тому що продукція харчових засобів є найперша умова
життя продуцентів і всякої продукції взагалі, — праця застосована до цієї продукції,
отже, хліборобська праця в найширшому економічному розумінні мусить бути
остільки продуктивна, щоб продукцією харчових засобів для безпосередніх продуцентів
забирався не ввесь робочий час, що вони ним порядкують, отже, щоб була
можлива хліборобська додаткова праця, а тому і хліборобський додатковий продукт.
Розвиваючи далі: треба, щоб уся хліборобська праця — потрібна й додаткова
праця — деякої частини суспільства була достатня для того, щоб продукувати
\parbreak{}  %% абзац продовжується на наступній сторінці

\parcont{}  %% абзац починається на попередній сторінці
\index{iii2}{0117}  %% посилання на сторінку оригінального видання
потрібні харчові засоби для всього суспільства, тобто і для нехліборобських робітників;
отже, щоб був можливий цей великий поділ праці між хліборобами
і промисловцями, а також між тими з хліборобів, що продукують харч, і тими,
що продукують сирові матеріяли. Хоч праця безпосередніх продуцентів харчу
щодо них самих поділяється на потрібну і додаткову працю, проте, щодо
суспільства вона становить лише потрібну працю, яка потрібна для продукції
харчових засобів. А втім, це саме має силу і щодо всякого поділу праці
всередині всього суспільства, на відміну від поділу праці всередині окремої майстерні.
Це — праця, що потрібна для продукування особливих речей, для задоволення
особливої потреби суспільства в особливих речах. Коли цей поділ праці
пропорційний, то продукти різних груп продаються по їхніх вартостях (при
дальшому розгляді по цінах їхньої продукції) або ж по цінах, які, визначувані
загальними законами, становлять модифікації цих вартостей, — у відповідних
випадках цін продукції. Це справді, є закон вартости, як він виявляється не
у відношенні до окремих товарів, або речей, а у відношенні до кожноразового
сукупного продукту окремих суспільних сфер продукції, які в наслідок поділу
праці стали самостійними; отже, не тільки так, що на кожен окремий товар вжито
лише потрібний робочий час, але й так, що з усього суспільного робочого часу
на різні групи вжито лише конечну пропорційну кількість. Бо умовою лишається
споживна вартість. Але коли споживна вартість окремого товару залежить
від того, що він сам по собі задовольняє будь-яку потребу, то споживна вартість
суспільної маси продуктів залежить від того, що ця маса адекватна кількісна
певній суспільній потребі в продукті кожного осібного роду, а тому і від того,
що працю розподілено пропорційно між різними сферами продукції відповідно
до цих суспільних потреб, що кількісно визначені. (На цей пункт звернути
увагу в зв’язку з розподілом капіталу між різними сферами виробництва). Суспільна
потреба, тобто споживна вартість у суспільній потенції, вступає тут визначально
для кількостей всього суспільного робочого часу, що припадають на
різні окремі сфери продукції. Але це — лише той самий закон, який виявляється
уже у відношенні до окремого товару, а саме, що споживна вартість товару
є передумова його мінової вартости, а тому і його вартости. Цей пункт має
лише ту дотичність до відношення між потрібного і додатковою працею, що при
порушенні цієї пропорції не може бути реалізована вартість товару, а тому
й додаткова вартість, що міститься в ній. Хай, наприклад, бавовняних тканин
випродуковано пропорційно забагато, хоч у всьому цьому продукті, в цих тканинах
реалізовано лише потрібний для цього в даних умовах робочий час. Але взагалі
на цю осібну галузь витрачено надто багато суспільної праці; тобто частина
продукту некорисна. Тому ввесь продукт продається лише так, як коли б він
був випродукований в потрібній пропорції. Ця кількісна межа тих кількостей
суспільного робочого часу, які можна витратити на різні осібні сфери продукції,
є лише далі розвинений вираз закону вартости взагалі; хоч потрібний робочий
час набуває тут іншого сенсу. Для задоволення певної суспільної потреби треба
стільки от робочого часу. Обмеження тут настає через споживну вартість. Суспільство
в даних умовах продукції на такий от продукт певного роду може
витратити лише стільки й стільки з усього свого робочого часу. Але суб’єктивні
й об'єктивні умови додаткової праці і додаткової вартости взагалі не мають
ніякого чинення до певної форми так зиску, як і ренти. Вони мають силу для
додаткової вартости як такої, хоч би яких особливих форм вона набувала. Тому
вони не пояснюють земельної ренти.

3)~Якраз при економічній реалізації земельної власности, в розвитку земельної
ренти, виявляється дуже своєрідним те, що її величина визначається
зовсім не за допомогою її одержувача, а розвитком суспільної праці, що є незалежний
від його допомоги, і в якому він зовсім не бере участи. Тому легко вважати за
\parbreak{}  %% абзац продовжується на наступній сторінці


\index{i}{0118}  %% посилання на сторінку оригінального видання
Ми вже показали, що додаткова вартість не може виникнути
з циркуляції, отже, при її утворенні за спиною циркуляції мусить
зчинитися щось таке, чого в ній самій не можна помітити\footnote{
«Серед звичайних умов ринку обмін не створює зиску. Коли його
не було раніш, то його не створиться й після цієї оборудки». («Profit, in
the usual condition of the market, is not made by exchanging. Had it not
existed before, neither could it after that transaction»). (\emph{Ramsay}: «An
Essay on the Distribution of Wealth», Edinburgh 1836, p. 184).
}. Але
чи може додаткова вартість виникнути звідкись інакше, опріч
циркуляції? Циркуляція є сума всіх товарових взаємовідносин
посідачів товарів. Поза нею посідач товарів стоїть лише у відношенні
до свого власного товару. Щождо вартости товару, то це
відношення обмежується на тім, що в товарі міститься певна
кількість власної праці товаропосідача, вимірюваної за певними
суспільними законами. Ця кількість праці виражається у величині
вартости його товару, а що величина вартости виражається
в рахункових грошах, то ця кількість праці виражається, приміром,
у ціні в 10\pound{ фунтів стерлінґів}. Але його праця не виражається
у вартости товару і в лишкові понад власну вартість товару,
не виражається в ціні в 10, яка одночасно є ціна в 11, не виражається
у вартості, яка більша за себе саму. Посідач товарів
може своєю працею створювати вартості, але він не може створювати
вартості, що самозростають. Він може підвищити вартість
товару, додаючи новою працею до наявної вартости нову вартість,
приміром, виготовляючи із шкури чоботи. Той самий матеріял
має тепер більшу вартість, бо в ньому міститься більша кількість
праці. Тому чоботи мають більшу вартість, аніж шкура, алеж вартість
шкури лишилась такою, якою вона була. Вона не зросла,
не прилучила до себе додаткової вартости підчас продукції чобіт.
Отже, неможливо, щоб товаропродуцент поза сферою циркуляції,
не стикаючися з іншими посідачами товарів, збільшував вартість,
а тому й неможливо, щоб він поза сферою циркуляції перетворював
гроші або товар на капітал.

Отже, капітал не може виникнути з циркуляції і так само не
може виникнути поза циркуляцією. Він мусить виникнути одночасно
в циркуляції і не в ній.

Таким чином виявився подвійний результат.

\disablefootnotebreak{} 
\looseness=-1
Перетворення грошей на капітал слід розвинути на основі
законів, іманентних товаровій циркуляції, так, щоб обмін еквівалентів
правив за вихідний пункт\footnote{
Після поданих пояснень читач розуміє, що це значить тільки ось
що: утворення капіталу мусить бути можливе й тоді, коли ціни товарів
дорівнюють їхнім вартостям. Утворення капіталу не можна пояснити відхиленням
товарових цін від товарових вартостей. Коли ціни дійсно
відхиляються від вартостей, то треба їх спочатку звести на останні, тобто
залишити цю обставину як випадкову осторонь, щоб мати перед собою в
чистій формі явище утворення капіталу на основі товарового обміну, і
щоб, спостерігаючи його, не заплутатись через побічні обставини, що
ускладнюють самий процес і є чужі для нього. Нарешті, відомо, що ця
редукція ніяким чином не є лише наукова процедура. Постійні коливання
ринкових цін, їхнє піднесення та зниження, урівноважуються,
взаємно
касуються й сами собою зводяться на пересічну ціну як свою внутрішню
норму (Regel). Ця остання є провідна зірка, приміром, для купця або
промисловця в кожному підприємстві, яке функціонує довший час. Отже,
купець або промисловець знає, що, коли розглядати довший період у
цілості, товари дійсно продається не нижче і не вище, а за їхніми пересічними
цінами. Отже, коли б безстороннє мислення було взагалі в його інтересах,
то він мусив би поставити перед собою проблему утворення капіталу
ось як: як може постати капітал за реґулювання цін пересічною ціною,
тобто в останній інстанції вартістю товару? Я кажу «в останній інстанції»,
бо пересічні ціни не збігаються безпосередньо з величиною вартости товарів,
як то гадають А.~Сміт, Рікардо й інші.
}. Наш посідач грошей, який
\index{i}{0119}  %% посилання на сторінку оригінального видання
є покищо тільки гусінню капіталіста, мусить купувати товари
за їхньою вартістю, за їхньою вартістю продавати, і, все ж таки,
наприкінці процесу витягати більше вартости, ніж він авансував.
Його перетворення з гусені на метелика, з лише посідача
грошей на дійсного капіталіста, мусить відбуватися в сфері
циркуляції, і в той самий час воно мусить відбуватись не
в сфері циркуляції. Такі умови цієї проблеми. Ніс Rhodus,
hic salta!\footnote*{
Дослівно: Тут Родос, тут стрибай. — Античне прислів’я з байок
Езопа: відповідь родосців одному хвалькові, що вихвалявся своїми
стрибками. Вживається в значенні: отут покажи свою вмілість. \emph{Ред.}
}
\enablefootnotebreak{}

\subsection{Купівля і продаж робочої сили}

Зміна вартости грошей, що повинні перетворитися на капітал,
не може відбутися в самих цих грошах, бо як засіб купівлі і як
засіб виплати вони реалізують ціну товарів, які за них купують
або за які ними платять, а коли вони залишаються в своїй власній
формі, то вони тверднуть у, так би мовити, скам’янілу, незмінну
величину вартости\footnote{
«У формі грошей\dots{} капітал не продукує зиску» («In the form of
money\dots{} capital is productive of no profit»). (\emph{Ricardo}: «Principles
of Political Economy», 3 rd ed. London 1821, p. 267).
}. Так само не може виникнути ця зміна з
другого акту циркуляції, з перепродажу товару, бо цей акт лише
перетворює товар із натуральної форми назад у грошову форму.
Отже, зміна мусить статися з товаром, що купується в першому
акті $Г — Т$, але не з його вартістю, бо обмінюється еквіваленти,
товар оплачується за його вартістю. Отже, зміна може виникнути
лише з його споживної вартости, як такої, тобто з його споживання.
Щоб здобувати вартість з споживання якогось товару, нашому
посідачеві грошей мусило б пощастити відкрити в межах сфери
циркуляції, на ринку, такий товар, що сама його споживна вартість
мала б своєрідну властивість бути за джерело вартости,
отже, такий товар, що його фактичне споживання саме було б
упредметненням праці, а тому й творенням вартости. І посідач
грошей находить на ринку такий специфічний товар: це здатність
до праці, або робоча сила.

Під робочою силою або здатністю до праці ми розуміємо
сукупність фізичних та інтелектуальних здібностей, які існують
в організмі, живій особистості людини, і що їх вона пускає в рух
щоразу, коли продукує якібудь споживні вартості.
\parbreak{}


\index{i}{0120}  %% посилання на сторінку оригінального видання
\looseness=1
Однак, для того, щоб посідач грошей міг найти на ринку
робочу силу як товар, мусять здійснитись різні передумови\dots{}
Обмін товарів сам по собі не містить у собі жодних інших відносин
залежности, крім тих, що постають з його власної природи.
За цієї передумови робоча сила може з’явитися на ринку як
товар лише тому й остільки, що й оскільки її посідач, особа, що
її робочою силою вона є, подає її як товар на ринок, або продає.
Щоб її посідач міг продавати її як товар, він мусить мати змогу
порядкувати нею, отже, бути вільним власником своєї здатности
до праці, своєї особи\footnote{
В реальних енциклопедіях клясичної старовини можна прочитати
таку нісенітницю, ніби в античному світі капітал був цілком розвинений,
«бракувало лише вільного робітника і кредитових установ». Пан Момзен
у своїй «Римській історії» також робить подібні quid pro quo одне
по одному.
}. Власник робочої сили й посідач грошей
зустрічаються на ринку й увіходять між собою у відношення
як рівноправні посідачі товарів, які різняться лише тим, що
один є покупець, а другий — продавець, отже, обидва юридично
є рівні особи. Щоб це відношення й далі тривало, треба, щоб
власник робочої сили завжди продавав її лише на певний час,
бо, коли б він продав її геть чисто раз назавжди, то він продав би
себе самого й перетворився б із вільної людини на раба, з посідача
товару на товар. Як особа він мусить постійно ставитися до
своєї робочої сили як до своєї власности, і тим то як до свого
власного товару, а це він може робити лише остільки, оскільки
він завжди віддає свою робочу силу до розпорядження покупця
лише тимчасово, на певний період часу, передає лише на вжиток,
отже, відчужуючи її, не зрікається свого права власности
на неї\footnote{
Тому різні законодавства встановлюють певний максимальний реченець
для робочого контракту. У народів, що в них праця вільна, законодавство
реґулює умови відмовлення від контракту. По різних країнах, а особливо
в Мехіко (перед американською громадянською війною також і на територіях,
відірваних від Мехіко, а по суті і в наддунайських провінціях до
перевороту Кузи), рабство ховається під формою peonage’a. Через аванси,
що їх треба сплачувати працею і що переходять від покоління до покоління,
не лише поодинокий робітник, але й родина його фактично стають
власністю інших осіб і їхніх родин. Хуарец скасував peonage. Так званий
цар Максиміліян знов увів його в життя декретом, що його у вашинґтонській
Палаті представників слушно плямували як декрет, що відновлює
рабство в Мехіко. «Свої особливі фізичні й інтелектуальні здібності та
свою дієздатність я можу\dots{} відчужувати іншій особі для користування
на обмежений реченець, бо в наслідок цього обмеження вони набувають
зовнішнього відношення до моєї цілости й загальности. Через відчуження
цілого мого часу, який конкретизується через працю, і цілої моєї продукції
я зробив би власністю іншої особи саму субстанцію її, тобто мою загальну
діяльність і дійсність, мою особу». (\emph{Hegel}: «Philosophie des Rechts», Berlin
1840, S. 104, § 67).
}.

Друга посутня умова, потрібна для того, щоб посідач грошей
міг найти на ринку робочу силу як товар, є та, що посідач робочої
сили, замість мати змогу продавати товари, в яких упредметнилась
його праця, мусить, навпаки, подавати на ринок як товар
саму свою робочу силу, яка існує лише в його живому організмі.

\parcont{}  %% абзац починається на попередній сторінці
\index{iii2}{0121}  %% посилання на сторінку оригінального видання
самим ціну продукції товару. Для промисловця справа стоїть так, що для нього
витрати продукції товару менші. Йому доводиться менше платити за зрічевлену
працю, а також менше платити заробітної плати за живу робочу силу, якої
у нього застосовується менше. А що витрати продукції його товару менші,
то й його індивідуальна ціна продукції менша. Витрати продукції становлять для
нього 90 замість 100. Отже, його індивідуальна ціна продукції була б замість
115 лише 103\sfrac{1}{2} (100: 115 \deq{} 90: 103\sfrac{1}{2)}. Ріжниця між його індивідуальною ціною
продукції і загальною обмежена ріжницею між його індивідуальними витратами
продукції і загальними. Це — одна з величин, що становлять межі його надпродукту\footnote{Терміни Surplusprodukt (надпродукт) і Mehrprodukt (додатковий продукт) Маркс взагалі вживає,
як тотожні. Про специфічне значення терміну «надпродукт» тут і далі, коли йдеться
про рентодайний капітал, див. кінець розд. 41. \Red{Пр.~Ред.}}. Друга — це є величина загальної ціни продукції, в якій бере участь
загальна норма зиску, як один з регуляційних чинників. Коли б вугілля подешевшало,
то ріжниця між його індивідуальними і загальними витратами продукції
зменшилася б, а тому зменшився б і його надзиск. Коли б він мусив продавати
товар по його індивідуальній вартості, або по ціні продукції, визначуваній
його індивідуальною вартістю, то ріжниця відпала б. Вона є наслідок, з одного
боку, того, що товар продається по своїй загальній ринковій ціні, по ціні,
в яку конкуренція вирівнює індивідуальні ціни, а з другого боку — того, що
більша індівидуальна продуктивна сила праці, приведеної ним в рух, іде на користь
не робітникам, а як взагалі всяка продуктивна сила праці, тому, хто їх
застосовує; що вона виступає як продуктивна сила капіталу.

А що однією межею цього надзиску є висота загальної ціни продукції,
а висота загальної норми зиску є один з її чинників, то цей надзиск може
виникнути лише з ріжниці між загальною і індивідуальною ціною продукції,
отже, з ріжниці між індивідуальною і загальною нормою зиску. Надмір над цією
ріжницею має за свою передумову продаж продуктів дорожче, а не по ціні
продукції, регульованій ринком.

\emph{Подруге}. До цього часу надзиск фабриканта, що вживає як рушійну силу
природний водоспад замість пари, аж ніяк не відрізняється від усякого іншого
надзиску. Всякий нормальний надзиск, тобто такий, що виникає не від випадкових
операцій продажу або від коливань ринкової ціни, визначається ріжницею між
індивідуальною ціною продукції товарів цього окремого капіталу і загальною
ціною продукції, яка реґулює ринкові ціни товарів капіталу цієї сфери продукції
взагалі, або що реґулює ринкові ціни товарів усього капіталу, вкладеного в цю
сферу продукції.

Але звідси починається ріжниця.

Якій обставині завдячує фабрикант в даному разі своїм надзиском, тим надміром,
що його дає йому особисто ціна продукції, реґульована загальною нормою зиску?

Насамперед — природній силі, рушійній силі водоспаду, який є даний природою
і який сам не є продукт праці, а тому не має вартости, як от вугілля,
що перетворює воду в пару і яке само є продукт праці, тому має вартість,
та мусить бути оплачене еквівалентом, потребує витрат. Водоспад — такий природний
аґент продукції, що на створення його не треба жодної праці.

Але це не все. Фабрикант, що працює з паровою машиною, теж вживає
природні сили, які нічого не коштують йому, але роблять працю продуктивнішою
і — оскільки вони таким чином здешевлюють виготовлення засобів існування,
потрібних для робітників, — збільшують додаткову вартість, а тому і зиск; які,
отже, цілком так само монополізуються капіталом, як суспільні природні сили
праці, що постають з кооперації, поділу праці тощо. Фабрикант оплачує вугілля,
а не здібність води змінювати свій аґреґатний стан, переходити в пару, не
еластичність пари тощо. Ця монополізація сил природи, тобто спричиненого ними
\parbreak{}  %% абзац продовжується на наступній сторінці

\parcont{}  %% абзац починається на попередній сторінці
\index{ii}{0122}  %% посилання на сторінку оригінального видання
витрачати на медичну допомогу, ніж людині в розквіті сил. Отже, незважаючи на випадковий характер
ремонтних робіт, вони розподіляються нерівномірно на різні життьові періоди основного капіталу.

З цього, а також взагалі випадкового характеру ремонтних робіт, що їх потребує машина, випливає
таке:

З одного боку, справжня витрата на робочу силу й засоби праці для ремонтних робіт є випадкові, як
випадкові й самі обставини, що роблять потрібними ці ремонтні роботи; число потрібних полагоджень
розподіляється нерівномірно на різні життьові періоди основного капіталу. З другого боку, коли
обчислюють пересічний життьовий період основного капіталу, то припускається, що його постійно
підтримується в діяльному
стані, — почасти чищенням (сюди належить і тримання в чистоті приміщень), почасти ремонтом, що його
робиться в разі потреби. Перенесення вартости в наслідок зношування основного капіталу розраховано
на його пересічний життьовий період, але й самий цей пересічний період життя розрахований на те, що
весь час авансуватиметься додатковий капітал, потрібний на його підтримання в доброму стані.

З другого боку, так само зрозуміло, що вартість, долучувана в наслідок цієї додаткової витрати
капіталу й праці, не може входити в ціну товарів одночасно з цими витратами. Коли, напр., у
прядільника на цьому тижні поламалось колесо або розірвався пас, то він не може цього тижня
продавати свою пряжу дорожче, ніж продавав минулого. Загальні витрати прядіння ніяк не змінились в
наслідок такого нещасного випадку на одній фабриці. Тут, як і взагалі при визначенні вартости,
вирішувальне значення має пересічна величина. Досвід виявляє середнє число таких нещасних випадків і
пересічний розмір робіт на підтримання і ремонт, потрібних протягом пересічного життьового періоду
основного капіталу, вкладеного в певну галузь підприємства. Ці пересічні витрати розподіляються на
пересічний життьовий періоді відповідними аліквотними частинами їх долучається до ціни продукту, а
тому й покривається через його продаж.

Додатковий капітал, таким чином заміщуваний, належить до поточного капіталу, хоч спосіб витрат
нереґулярний. А що дуже важливо виправляти кожне ушкодження машини негайно, то при кожній великій
фабриці є, крім власне фабричних робітників, відповідний персонал інженерів, теслярів, механіків,
слюсарів і~\abbr{т. ін.} їхня заробітна плата становить частину змінного капіталу, і вартість їхньої праці
розподіляється на
продукт. З другого боку, потрібні видатки на засоби продукції визначаються за пересічним розрахунком
і відповідно до нього ввесь час входять у продукт, як частина його вартости, хоч фактично їх
авансується нереґулярно, а, значить, нереґулярно входять вони в продукт, зглядно в основний капітал.
Цей капітал, витрачуваний власне на ремонт, з певного погляду є капітал особливого роду, що його не
можна залічити ні до поточного, ні до основного капіталу, але більше до першого, бо він належить до
категорії поточних витрат.

Система бухгальтерії, звичайно, нічого не змінює в дійсному зв’язку речей, що про них ведеться ці
книги. Але важно відзначити, що в багатьох
\index{ii}{0123}  %% посилання на сторінку оригінального видання
галузях підприємств є звичка разом обчислювати витрати на ремонт з справжнім зношуванням
основного капіталу в такий спосіб. Хай авансовий основний капітал буде \num{10.000}\pound{ ф. стерл.}, а його
життьовий період 15 років; річне зношування дорівнює тоді 666\sfrac{2}{3}\pound{ ф. стерл}. Але в дійсності
зношування обчислюють лише на 10 років, тобто до ціни вироблюваних товарів щороку додають на
зношування основного капіталу \num{1.000}\pound{ ф. стерл.} замість 666\sfrac{2}{3}\pound{ ф. стерл.}; інакше кажучи, на ремонтні
роботи тощо утворюється резервний фонд в 333\sfrac{1}{3}\pound{ ф. стерл.} (Числа 10 і 15 беремо лише, як приклад).
Отже, таку суму пересічно витрачається на ремонт для того, щоб основний капітал існував 15 років.
Такий спосіб обчислення, звичайно, не заважає, щоб основний капітал і витрачуваний на ремонт
додатковий капітал становили різні категорії. На ґрунті цього способу обчислення, напр.,
припускається, що мінімальна додача витрат на підтримання й поновлення пароплавів становить річно
15\%, отже, час репродукції дорівнює 6\sfrac{2}{3} рокам. 60-х років управління англійської Peninsular and
Oriental С° обчислювало щорічні витрати на це в 16\%, що відповідає часові репродукції в 6\sfrac{1}{4} років.
На залізницях середній час життя паровоза є 10 років, але, беручи на увагу ремонт, зношування беруть
в 12\sfrac{1}{2}\%, і тому час життя сходить до 8 років. Для пасажирських і товарових вагонів зношування
обчислюється в 9\%, отже, час життя беруть в 11\sfrac{1}{9} років.

Законодавство щодо контрактів про винаймання будинків та інших речей, які є для їхніх власників
основний капітал і здаються ними як такий, визнає повсюди ріжницю між нормальним зношуванням, що
його спричиняють час, вплив природних сил і саме нормальне користання, і випадковими полагодженнями,
які час від часу потрібні в нормальному протязі життя будинку й при нормальному користанні, на
підтримання будинку в нормальному стані. Як загальне правило, ремонт першої відміни покладається на
власників, а другої — на наймачів. Ремонтні роботи далі поділяється на звичайні й капітальні.
Останні є часткове поновлення основного капіталу в його натуральній формі й теж покладається їх на
власників, якщо тільки контракт не вимагає протилежного. Напр., згідно з англійськими законами:

„Наймач лише повинен тримати будівлі рік-у-рік в такому стані, щоб вони не пропускали вітру й води,
оскільки це можливо без капітального ремонту; і взагалі він повинен дбати лише про такі
полагодження, що їх можна назвати звичайними. Але навіть і тут доводиться брати на увагу вік і
загальний стан відповідних частин будівлі в той час, коли наймач їх прийняв, бо він не повинен
старий і зношений матеріял замінювати на новий, ні відшкодовувати зневартнення, що неминуче постає
під впливом часу і нормального користання“ (Holdsworth, „Law of Landlord and Tenant“, p. 90, 91).

Цілком відмінне так від покриття зношування, як і від робіт для зберігання й ремонту, є \emph{страхування},
яке поширюється на руйнацію в наслідок незвичайних природних явищ, пожеж, поводей тощо. Воно
\parbreak{}  %% абзац продовжується на наступній сторінці

\input{ii/_0124.tex}
\parcont{}  %% абзац починається на попередній сторінці
\index{ii}{0125}  %% посилання на сторінку оригінального видання
of Inquiry on Caledonian Railway, передруковано в Money Market Review, 1867)\footnote*{
„Цитоване місце є в нумері з 25 січня 1868 року, і взято його з статті
в „Money Market Rewiew“ — The Caledonian Railway, The Directors Reply, де йде мова про звіт капітана Фіцморіса“. \emph{Ред.}
}.

Практично неможливо й недоцільно розмежовувати заміщення й підтримання основного капіталу в
хліборобстві, принаймні, оскільки воно ще не застосовує сили пари. „Коли є повний, але не надто
великий комплект реманенту (різних хліборобських та інших всякого роду знарядь праці та
господарювання), щорічне зношування та витрати на підтримання реманенту звичайно обчислюється в
15--25\% авансованого капіталу, залежно від різних наявних умов“ (Kirchhof, Handbuch der
landwirtschaftlichen Betriebslehre. Berlin, 1862, p. 137).

Щодо рухомої частини залізниці, то зовсім не можна розмежувати ремонт і заміщення. „Ми підтримуємо
нашу рухому частину у наявних її розмірах. Скільки паровозів є в нас, стільки ми й підтримуємо. Коли
з плином часу паровіз робиться непридатний, так що вигідніше збудувати новий, то ми й будуємо його
на кошти доходів, при чому, звичайно, записуємо на дохід вартість матеріялів, що лишились від старої
машини\dots{} А лишається завжди чимало\dots{} Колеса, осі, казан тощо, коротко кажучи, лишається чимала частина
старого паровозу“. (T.~Gooch, Chairman of Great Western Railway C°, R.~C. № \num{17327}--29). „Ремонтувати
значить відновлювати; для мене немає слова „заміщення“\dots{} Коли залізничне товариство купило вагон
або паротяг, то воно мусить їх так полагодити, щоб вони вічно могли служити (\num{17784}). Ми обчислюємо
витрати на паротяги в 8\sfrac{1}{2}\pens{ пенсів} на англійську милю пробігу. На ці 8\sfrac{1}{2}\pens{ пенсів} ми назавжди
підтримуємо паротяги. Ми поновлюємо наші машини. Коли ви хочете купити машину нову, ви витрачаєте
більше грошей, ніж треба\dots{} В старій машині завжди буде пара коліс, вісь або ще яка придатна
частина, і це дає змогу збудувати дешевше таку саму гарну машину, як і цілком нова (\num{17790}). Тепер я
продукую щотижня новий паротяг, тобто такий самий гарний, як новий, бо в ньому казан, циліндр і рама
нові“ (\num{17823}. Archibald Sturrock, Locomotive Superintendent of Great Northern Railway в R.~C. 1867).

Це стосується й до вагонів: „З плином часу запас паротягів і вагонів постійно поновлюється; одного
разу насаджуються нові колеса, другого разу робиться нову ряму. Частини, що на них ґрунтується рух і
що найбільше зношуються, відновлюються поступінно; таким чином, машини й вагони можуть підлягати
стільком ремонтам, що в багатьох з них не лишиться й сліду старого матеріялу\dots{} Навіть, коли вони
зробляться вже зовсім непридатні для ремонту, з старих вагонів або паротягів перероблюються
поодинокі частини і таким чином вони ніколи не гинуть цілком для залізниці. Тому рухомий капітал
перебуває в стані постійної репродукції:
\index{ii}{0126}  %% посилання на сторінку оригінального видання
те, що для залізничної колії повинно в певний час робити одним заходом, а саме, коли лінію
цілком перекладається наново, — це в рухомій частині робиться поступінно з року на рік. Її існування
вічне, вона завжди омолоджується“ (Lardner, p. 116).

Цей процес, як його описує тут Ларднер щодо залізниць, не підходить до поодинокої фабрики, але
змальовує нам картину постійної, частинної, переплетеної з ремонтом репродукції основного капіталу в
межах якоїсь цілої галузі промисловости, або взагалі в межах сукупної продукції, розглядуваної у
суспільному маштабі.

Наводимо тут ще одну вказівку, що пояснює, в яких широких розмірах спритні управління можуть
орудувати поняттями ремонт і заміщення, щоб здобувані дивіденди. Згідно з вище цитованою доповіддю
Р.~Б.~Вільямса, різні англійські залізничні товариства пересічно за ряд років списували з рахунку
доходів такі суми на ремонт та витрати на підтримання залізничної колії та будівель (на англійську
милю довжини колії щороку):
\begin{table}[H]
\centering
\begin{tabular}{lr@{}r}
London and North Western\dotfill{} & 370 & \pound{ф. стерл.}\\
Midland\dotfill & 225 & \\
London and South Western\dotfill & 257 & \\
Great Northern\dotfill & 360 & \\
Lancashire and Yorkshire  & 377 & \\
South Eastern\dotfill & 263 & \\
Brignton\dotfill & 266 & \\
Manchester and Sheffield & 200 & \\
\end{tabular}
\end{table}

\noindent{}Ці ріжниці лише дуже мало залежать від неоднакового розміру дійсно зроблених витрат: вони походять
майже виключно з неоднаковости в способах обчислення, з того, чи залічується статті видатків на
рахунок капіталу, чи на рахунок доходів. Вільямс прямо каже: «Меншу цифру витрат вибирається тому,
що це потрібно для доброго дивіденда, а більшу цифру подається тому, що є досить високий дохід, який
може витримати це“.

Іноді зношування, отже, і заміщення його стає величиною практично зникомою, так що на увагу береться
лише витрати на ремонт. Те, що Ларднер каже далі про works of art\footnote*{
Works of art — будівельні споруди. \emph{Ред.}
} на залізницях, має силу взагалі
для всіх таких довготривалих споруд, як канали, доки, залізні та кам’яні мости і~\abbr{т. ін.} —
„Зношування, що постає в наслідок повільного впливу часу на солідніших спорудах, діє майже непомітно
протягом невеликих переміжків часу; а коли минає більше часу, прим., століття, то воно мусить
призвести до поновлювання, повного або частинного, навіть для найсолідніших споруд. Це непомітне
зношування, порівняно з помітнішими зношуваннями інших частин залізниці, можна прирівняти до вікових
і періодичних відхилів у русі світових тіл. Вплив часу на масивніші споруди залізниці —
\parbreak{}  %% абзац продовжується на наступній сторінці


\index{iii1}{0127}  %% посилання на сторінку оригінального видання
Припустімо наприклад що спочатку потрібно було 500 фунтів
стерлінгів для того, щоб щотижня приводити в рух 500 робітників, а тепер для цього потрібно тільки
400 фунтів стерлінгів.
Тоді, якщо припустити, що в обох випадках маса виробленої
вартості = 1000 фунтам стерлінгів, маса щотижневої додаткової вартості буде в першому випадку = 500
фунтів стерлінгів, норма додаткової вартості \frac{500}{500} = 100\%; але після зниження заробітної плати
маса додаткової вартості буде 1000 фунтів стерлінгів — 400 фунтів стерлінгів = 600 фунтам
стерлінгів, а її норма \frac{600}{400} = 150\%. І це підвищення норми додаткової вартості є єдиний результат
для того, хто із змінним капіталом в 400 фунтів стерлінгів і з відповідним сталим капіталом починає
нове підприємство в тій самій сфері виробництва. Але в підприємстві, яке вже функціонує, в цьому
випадку в наслідок зниження вартості змінного
капіталу не тільки підвищується маса додаткової вартості з 500
до 600 фунтів стерлінгів і норма додаткової вартості з 100
до 150\%; тут, крім того, звільняється 100 фунтів стерлінгів
змінного капіталу, за допомогою яких знову таки можна експлуатувати працю. Отже, не тільки та сама
кількість праці експлуатується з більшою вигодою, але, в наслідок звільнення цих 100 фунтів
стерлінгів, з тим самим змінним капіталом у 500 фунтів стерлінгів можна експлуатувати при підвищеній
нормі експлуатації більше робітників, ніж раніше.

Припустімо тепер, навпаки, що при 500 занятих робітниках
первісне відношення розподілу продукту є: $400 v + 600 m = 1000$, отже, що норма додаткової вартості є
= 150\% Таким
чином робітник одержує тут щотижня \sfrac{4}{5} фунтів стерлінгів = 16 шилінгів. Якщо тепер в наслідок
підвищення вартості
змінного капіталу 500 робітників коштують щотижня 500 фунтів
стерлінгів, то тижнева заробітна плата кожного робітника = 1 фунтові стерлінгів, і 400 фунтів
стерлінгів можуть привести
в рух тільки 400 робітників. Отже, якщо пускатиметься в рух те
саме число робітників, що й раніш, то ми матимемо $500 v + 500 m = 1000$; норма додаткової вартості
знизиться з 150 до 100\%,
тобто на \sfrac{1}{3}. Для нововкладуваного капіталу єдиним результатом було б це зниження норми додаткової
вартості. При інших
однакових умовах норма зиску відповідно знизилася б, хоч і не
в такій самій пропорції. Якщо, наприклад, c = 2000, то в
першому випадку ми маємо $2000 c + 400 v + 600 m = 3000; m' = 150\%, p' = \frac{600}{2400} = 25\%$. В другому
випадку $2000 c + 500 v + 500 m = 3000; m' = 100\%; р' = \frac{500}{2500} = 20\%$. Навпаки, для вкладеного вже
капіталу результат був би двоякий. 3400 фунтами
стерлінгів змінного капіталу тепер можна привести в рух тільки
400 робітників, до того ж при нормі додаткової вартості в 100\%.
Отже, вся додаткова вартість, яку вони дають, становить тільки
\parbreak{}  %% абзац продовжується на наступній сторінці

\parcont{}  %% абзац починається на попередній сторінці
\index{iii2}{0128}  %% посилання на сторінку оригінального видання
досить, так ціна пшениці почала підноситись доти, поки $C$ не набуло змоги
покрити недостачу подання. Тобто ціна мусила піднестись до 20\shil{ шил.} за
квартер. Скоро тільки ціна пшениці піднеслась до 30\shil{ шил.} за квартер, як
в число оброблюваних земель могла б увійти земля $В$. А якби вона піднеслась
до 60\shil{ шилінґів}, до числа оброблюваних земель могла б увійти й земля $А$, це
не призвело б до того, що на застосований тут капітал довелось би задовольнятися
нормою зиску, нижчою за 20\%. Таким чином, для $D$ створилась би
рента спочатку в 5\shil{ шил.} з квартера \deq{} 20\shil{ шил.} з 4 кв., що тут продукується,
а потім в 45\shil{ шил.} з квартера \deq{} 180\shil{ шил.} з 4 квартерів.

Коли норма зиску з $D$ спочатку також була \deq{} 20\%, то і загальний зиск
з 4 кв. був також лише 10\shil{ шил.}, що проте, при ціні збіжжя в 15\shil{ шил.}, становило
більшу кількість збіжжя, ніж при ціні в 60\shil{ шил.} А що збіжжя входить
у репродукцію робочої сили і частина кожного квартера мусить покривати заробітну
плату, а друга — сталий капітал, то за такого припущення додаткова
вартість була вища, а тому, за інших незміних умов, вища була і норма зиску.
(Справу про норму зиску треба ще дослідити осібно і детальніше).

Коли, навпаки, послідовність була зворотна, коли процес починався з $А$,
то, — якщо довелося б ввести в обробіток нові лани, — ціна квартера спочатку
піднеслась би вище за 60\shil{ шил.}; але тому, що потрібне подання в 2 кварт. постачало
б $В$, то ціна знову понизилась би до 60\shil{ шил.}; хоч $В$ і продукує квартер
за 30\shil{ шилінґів}, проте, продається він за 60, бо його подання вистачало б
якраз тільки для того, щоб покрити попит. Так створилася б рента спочатку в
60\shil{ шил.} для $В$, і таким самим способом для $C$ і $D$, припускаючи завжди, що
ринкова ціна залишається 60\shil{ шил.}, хоч дійсна вартість, по якій $C$ і $D$ дають
квартер пшениці, дорівнює 20 і 15\shil{ шил.}; бо як і давніш потрібно подання одного
квартера, що його постачає $А$, для задоволення загальної потреби. В цьому випадку
підвищення попиту понад ту потребу, яку спочатку задовольняло $А$, потім
$А$ і $В$, могло б привести не до послідовного обробітку $В$, $C$ і $D$, а до поширення
площі обробітку взагалі і можливо, що родючіші землі входили б в обробіток
лише пізніше.

В першому ряді із збільшенням ціни рента стала б підвищуватись, а норма
зиску зменшуватись. Це зменшення могло б цілком або почасти паралізуватися
протидіющими обставинами; на цьому пункті згодом спинимося докладніше.
Не слід забувати, що загальна норма зиску визначається не додатковою вартістю
в усіх сферах продукції рівномірно. Не хліборобський зиск визначає промисловий,
а навпаки. Але про це далі.

У другому ряді норма зиску на витрачений капітал лишилась би та
сама; маса зиску визначилась би в меншій кількості збіжжя; але відносна ціна
його проти інших товарів підвищилась би. Але збільшення зиску, там де воно
відбувається, відокремлюється в формі ренти від зиску, замість того, щоб потрапити
до кишені промислових орендарів і визначитися як зиск, що зростає.
А ціна хліба за такого припущення лишилась би незмінною.

Розвиток і зріст диференційної ренти залишаються однакові так за незмінних
цін, як і за таких, що підвищуються, і так само за безперервного поступу
від гірших земель до кращих, як і за безперервного реґресу від крайніх
до гірших земель.

До цього часу ми вважали: 1)~що ціна при одній послідовності підвищується,
при другій — лишається незмінна, і 2)~що постійно відбувається перехід
від кращих земель до гірших або навпаки — від гірших до кращих.

Але припустімо, що потреба в хлібі піднялась з первісних 10 до 17 кв.;
далі, що найгірша земля $А$ витиснута другою землею $А'$, яка при ціні продукції
в 60\shil{ шил.} (50\shil{ шил.} витрат, плюс 10\shil{ шил.}, що становлять 20\% зиску) дає
1\sfrac{1}{3} кварт., так що ціна продукції одного квартера \deq{} 45\shil{ шил.}; абож припустімо,
\parbreak{}  %% абзац продовжується на наступній сторінці


\index{i}{0129}  %% посилання на сторінку оригінального видання
\chapter{Продукція абсолютної додаткової вартости}

\section[Процес праці і процес зростання вартости]{Процес праці і процес зростання вартости\footnotemarkZ{}}
\subsection{Процес праці}

Споживання\footnotetextZ{У французькому виданні цьому розділові дано заголовок: «Production de valeurs d’usage et production de la plus-value» — «Продукція
споживних вартостей та продукція додаткової вартости», а § 1 цього розділу
дано заголовок: «Production de valeurs d’usage» — «Продукція споживних
вартостей». \emph{Ред.}}
робочої сили — це сама праця. Покупець робочої
сили споживає її, примушуючи продавця її працювати. Останній
стає через те робочою силою, що виявляється в дії (actu),
стає робітником, чим раніш він був лише потенціяльно. Щоб
утілити свою працю в товарах, він мусить насамперед утілити
її у споживних вартостях, у речах, що служать для задоволення
тих або інших потреб. Отже, капіталіст примушує робітника
виготовляти якусь споживну вартість, якийсь певний товар.
Продукція споживних вартостей або дібр не змінює своєї загальної
природи від того, що вона відбувається для капіталіста та
під його контролем. Отже, процес праці треба розглянути насамперед
незалежно від усякої певної суспільної форми\footnote*{
У французькому виданні це речення подано так: «Отже, спочатку
нам треба розглянути процес корисної праці взагалі, абстрагуючись
від усякої осібної форми, що йому може надати та або інша фаза економічного
розвитку суспільства» («Le Capital etc.», vol. I, ch. VII, p. 76). \emph{Ред.}
}.

Праця є насамперед процес між людиною й природою, процес,
що в ньому людина своєю власною діяльністю упосереднює, реґулює
й контролює обмін речовин між собою й природою. Речовині
природи людина сама протистоїть як сила природи. Щоб
присвоїти собі природну речовину у формі, придатній для її власного
життя, вона пускає в рух належні до її тіла природні сили:
рамена й ноги, голову й руки. Впливаючи цим рухом на зовнішню
природу і змінюючи її, вона змінює одночасно і свою власну природу.
Вона розвиває дрімотні в її власній природі здібності (Potenzen)
і гру її сил підбиває під свою власну владу. Ми не будемо
тут розглядати первісних твариноподібних інстинктивних форм
праці. Порівняно з тим станом, коли робітник виступає на ринку
\parbreak{}  %% абзац продовжується на наступній сторінці

\parcont{}  %% абзац починається на попередній сторінці
\index{iii1}{0130}  %% посилання на сторінку оригінального видання
певним органічним законам, зв’язаним з певними природними
строками, не можуть бути раптом збільшені в такій мірі, як,
наприклад, машини та інший основний капітал, вугілля, руда
та ін., збільшення яких, якщо припустити незмінність природних
умов, може відбуватися в промислово розвиненій країні
в найкоротші строки. Тим то можливо, а при розвиненому капіталістичному
виробництві навіть неминуче, що виробництво і
збільшення тієї частини сталого капіталу, яка складається з
основного капіталу, машин і~\abbr{т. д.}, значно випереджає виробництво
і збільшення тієї його частини, яка складається з органічних
сировинних матеріалів, так що попит на ці сировинні матеріали
зростає швидше, ніж подання їх, в наслідок чого ціна їх підвищується.
Це підвищення ціни приводить в дійсності до того:
1)~що ці сировинні матеріали починають довозитися з дальших
місцевостей, бо підвищена ціна покриває збільшені витрати транспорту;
2)~що виробництво цих сировинних матеріалів збільшується
— обставина, яка, однак, залежно від природних умов,
можливо, тільки через рік зможе дійсно збільшити масу продукту,
і 3)~що використовуються всякі раніш невикористовувані
сурогати і починають економніше поводитися з відпадами. Коли
підвищення цін починає дуже помітно впливати на розширення
виробництва й подання, то здебільшого вже настав поворотний
пункт, після якого в наслідок триваючого збільшення кількості
сировинного матеріалу і всіх тих товарів, в які він входить як
елемент, попит падає, і тому настає також реакція в русі ціни
сировинного матеріалу. Незалежно від тих конвульсій, які викликає
ця реакція в наслідок знецінення капіталу в різних формах,
сюди долучаються ще й інші обставини, про які треба
зараз згадати.

Але, насамперед, уже з досі сказаного ясно таке: чим розвиненіше
капіталістичне виробництво і чим більше тому засобів
для раптового й безперервного збільшення тієї частини
сталого капіталу, яка складається з машин і~\abbr{т. д.}, чим швидше нагромадження
(як, особливо, в часи процвітання), тим більша відносна
перепродукція машин та іншого основного капіталу, і тим
частіша відносна недопродукція рослинних і тваринних сировинних
матеріалів, тим чіткіше вищеописане підвищення їх ціни і відповідна
цьому останньому реакція. Отже, тим частіші ті потрясіння
(Revulsionen), які виникають в наслідок цього сильного коливання
ціни одного з головних елементів процесу репродукції.

Але якщо настає крах цих високих цін, через те що їх підвищення
викликало почасти зменшення попиту, а почасти розширення
виробництва і подання з віддалених і досі мало або
й зовсім невикористовуваних виробничих місцевостей, при чому
обидві ці обставини приводять до перевищення подання сировинних
матеріалів над попитом на них — перевищення саме при старих
високих цінах, — то результат цього слід розглядати з різних
точок зору. Раптовий крах ціни сировинного продукту гальмує
\parbreak{}  %% абзац продовжується на наступній сторінці


\index{ii}{0131}  %% посилання на сторінку оригінального видання
\begin{table}[H]
\centering
\begin{tabular}{l@{ }l@{ }r@{ }r@{ }l@{ }l}
\num{50.000} $:$ 2 \deq{} \num{25.000}\usd{ дол.} на & 10 & років & \deq{} & \phantom{0}\num{2.500}\usd{ дол.} на 1 рік \\

\num{50.000} $:$ 4 \deq{} \num{12.500} & \phantom{0}2 & & \deq{} & \phantom{0}\num{6.250} \\

\num{50.000} $:$ 4 \deq{} \num{12.500} & \tbfrac{1}{2}& & \deq{} & \num{25.000} \\
\cmidrule{2-5}
& \multicolumn{2}{@{ }r@{ }}{На 1 рік} & \deq{} & \num{33.750}
\end{tabular}
\end{table}

\noindent{}Отже, пересічний час, що протягом його ввесь капітал обертається
один раз, становить 16 місяців\footnote*{
В обчисленні є помилка. Пересічний час, що протягом його обертається ввесь
капітал, становить не 16 місяців, а 17,76 місяців. \Red{Ред.}
}. Візьмімо другий приклад. Хай чверть
усього капіталу в \num{50.000}\usd{ долярів} обертається протягом 10 років; друга
чверть — протягом року; і решта — половина — двічі на рік. Тоді річні витрати
будуть такі:

\begin{table}[H]
\centering
\noindent\begin{tabular}{r@{ }c@{ }r@{ \deq{} }l}
\num{12.500} & $:$ & 10 & \phantom{0}\num{1.250}\usd{ долярів} \\
\num{12.500} &     &    & \num{12.500}\ditto{\usd{ долярів}} \\
\num{25.000} & ×   & 2  & \num{50.000}\ditto{\usd{ долярів}} \\
\cmidrule(rl){1-4}
& & \hang{r}{Протягом 1 року обернулось} & \num{63.750}\ditto{\usd{ долярів}}
\end{tabular}
\end{table}
% \multicolumn{4}{@{ }r@{ \deq{} }}}
\noindent{}(Scrope „Pol. Econ.“, edit. Alonzo Potter. New-York, 1841, р. 141, 142).

6) Справжні й позірні відмінності в обороті різних частин капіталу. —
Той самий Скроп каже там само: „Капітал, що його фабрикант, сільський
господар або купець витрачає на видачу заробітної плати, циркулює якнайшвидше,
бо він, коли робітникам платиться раз на тиждень, обертається,
може, раз на тиждень в наслідок щотижневих надходжень за
продані товари або оплачені рахунки. Капітал, вкладений в сировинний
матеріял або готові запаси, циркулює з меншою швидкістю; він може
обернутись два або чотири рази на рік, залежно від того, скільки
часу минає між закупові матеріялів і продажем товарів, — ми припускаємо,
що кредит на закуп і продаж дається на однаковий термін. Капітал, вкладений
в знаряддя й машини, циркулює ще повільніше, бо він протягом
5 або 10 років пересічно, може, зробить один оборот, тобто — його зуживеться
і поновиться, хоч деякі знаряддя вже зуживеться й по
небагатьох операціях. Капітал, вкладений в споруди, напр., в фабрики,
крамниці, склади, амбари, брук, зрошувальні споруди, тощо, як здається,
взагалі не циркулює. А в дійсності й ці споруди, відіграючи
свою ролю в продукції, зношуються цілком так само, як і вище згадані,
і їх треба репродукувати, щоб продуцент міг далі продовжувати
свої операції. Ріжниця лише в тому, що їх зуживається й репродукується
повільніше, ніж інші\dots{} Вкладений в них капітал робить, може,
один оборот протягом 20 або 50 років“.

Скроп сплутує тут ту ріжницю в русі певних частин поточного
капіталу, до якої призводять — щодо поодинокого капіталіста — терміни
виплат і кредитові відносини, з тією ріжницею оборотів, яка випливає
\parbreak{}  %% абзац продовжується на наступній сторінці

\input{ii/_0132.tex}

\index{ii}{0133}  %% посилання на сторінку оригінального видання
\section[%
Теорії про основний та обіговий капітал. Фізіократи~і~Адам Сміс
][%
Теорії про основний та обіговий капітал. Фізіократи~і~А. Сміс
]{Теорії про основний та обіговий капітал. Фізіократи~і~Адам Сміс}

У Кене ріжниця між основним і обіговим капіталом з’являється як
ріжниця між avances primitives\footnote*{
Аванси первинні. \emph{Ред.}
} і avances annuelles\footnote*{
Аванси річні. \emph{Ред.}
}. Він правильно визначає
цю ріжницю як ріжницю в межах продуктивного капіталу, тобто
капіталу, вкладеного в безпосередній процес продукції. А що для нього
єдиним справді продуктивним капіталом є капітал, застосовуваний в хліборобстві,
тобто капітал фармера, то й ці ріжниці подає він тільки для
капіталу фармера. Цим самим пояснюється, чому він для однієї частини
капіталу бере річний період обороту, для другої — довший (десятирічний).
В дальшому розвитку свого вчення фізіократи почали мимохідь переносити
ці ріжниці й на інші відміни капіталу, на промисловий капітал взагалі.
Для суспільства ріжниця між авансуваннями щорічними й багаторічними
така важлива, що багато економістів, навіть після Адама Сміса,
повертались до цього визначення.

Ріжниця між обома відмінами авансів постає лише тоді, коли авансовані
гроші перетворено на елементи продуктивного капіталу. Ця ріжниця
існує виключно в рамцях продуктивного капіталу. Тому Кене й не спадає
на думку залічувати гроші до первинних або щорічних авансів. Як аванси
для продукції, тобто як продуктивний капітал, обидві ці категорії протистоять
так само й грошам, як і наявним на ринку товарам. Далі у Кене
ріжниця між цими двома елементами продуктивного капіталу правильно
сходить на ріжницю між способами, що ними ці елементи входять у вартість
готового продукту, а значить, на ріжницю між способами циркуляції
їхньої вартости разом з вартістю продукту, а тому й на ріжницю
між способами їхнього заміщення або їхньої репродукції, коли вартість
одного елемента щорічно заміщується цілком, а вартість другого — частинами
протягом довших періодів\footnote{
Порівн. Quesnay, Analyse du Tableau Economique. (Physiocrates, éd. Daire,
I.~Partie, Paris. 1846). Ми читаємо там, напр., „Щорічні аванси складаються з витрат,
що їх робиться щороку на обробіток землі; ці аванси треба відрізняти від первинних
авансів, що становлять фонд організації сільського господарства“). Les
avances annuelles consistent dans les dépenses qui se font annuellement pour le
travail de la culture; ces avances doivent être distinguées des avances primitives,
qui forment les fonds de l’établissement de la culture. P. 59). У пізніших фізіократів
аванси часто зветься вже просто капіталом: „Capital ou avances“ Dupont de
Nemours, „Origine et Progrès d’une science nouvelle“, 1767 (Daire, I, p. 291 \footnote*{
Цитоване місце є не в статті „Origine et Progrès“, 1767 (Daire, I, p. 291),
a в статті „Maximes du docteur Quesnay“ (Daire, I, p. 391). \emph{Ред.}
},
далі Le Trosne пише: „В наслідок більшої або меншої довготривалости продуктів
праці, нація має чималий фонд багатств, незалежний від його щорічної
репродукції; фонд, що становить \so{капітал}, нагромаджений протягом довгого
часу, первинно оплачений продуктами, постійно поновлюваний і збільшуваний“
(„Au moyen de la durée plus ou moins grande des ouvrages demain d'œuvre, une
nation possède un fonds considérable de richesses, indépendant de sa reproduction
annuelle, qui forme un \so{capital} accumulé de longue main, et originairement payé
avec des productions, qui s'entretient et s’augmente toujours“ (Daire, I, p. 928).
Тюрґо вже систематично вживає слова капітал замість аванси і ще повніше ототожнює
аванси мануфактуристів з авансами фармерів (Turgot, „Rétlexions sur la
Formation et la Distribution des Richesses“, 1766).
}.

Єдиний успіх, що його зробив А.~Сміс, — це узагальнення зазначених
категорій. Він прикладає їх уже не лише до спеціяльної форми капіталу,
до капіталу фармера, але взагалі до всякої форми продуктивного капіталу.
Відси само собою зрозуміло, що замість ріжниці між однорічним і багаторічним
оборотом, різниці, запозиченої від хліборобства, виступає взагалі
ріжниця різночасних оборотів, так що оборот основного капіталу завжди
\index{ii}{0134}  %% посилання на сторінку оригінального видання
охоплює більше, ніж один оборот капіталу обігового, хоч який буде
протяг цих оборотів обігового капіталу: річний, більш ніж річний або
менш ніж річний. Таким чином, у Сміса avances annuelles перетворюються
на обіговий, a avances primitives — на основний капітал. Але цим узагальненням
категорій і обмежується його крок наперед. Щодо виконання він
лишається далеко позаду Кене.

Вже той грубий емпіричний спосіб, що ним він розпочинає свій дослід,
породжує плутанину: „Є два способи застосувати капітал так, щоб
він давав своєму власникові дохід або зиск“\footnote*{
„There are two different ways in which a capital may be employed so as to
yield a revenue or profit to its employer“. (Wealth of Nations. Book II, ch. 1, p. 189.
Edit. Aberdeen, 1848).
}.

Способи приміщувати вартість так, щоб вона функціонувала як капітал,
щоб давала своєму власникові додаткову вартість, так само різні
і так само різноманітні, як і сфери приміщення капіталу. Це є питання
про різні галузі продукції, куди можна вкласти капітал. Але питання, так
зформульоване, поширюється далі. Воно захоплює й питання про те, як
вартість, навіть, коли вона не вкладена в продуктивний капітал, може для
її власника функціонувати, напр., як процентодайний капітал, купецький
капітал тощо. Отже, тут ми безмежно віддалились від справжнього предмету
аналізи, від питання: як розподіл \so{продуктивного} капіталу на
його різні елементи впливає на його оборот, незалежно від різних сфер
його приміщення.

А.~Сміс безпосередньо по цьому каже: „Насамперед його можна застосувати
в сільському господарстві, мануфактурі або на закуп благ і дальший
продаж з зиском“\footnote*{
„First, it may be employed in raising, manufacturing, or purchasing goods and
selling them again with a profit“.
}. А.~Сміс каже тут лише те, що капітал можна
застосувати в сільському господарстві, мануфактурі й торговлі. Отже, він
каже лише про різні сфери приміщення капіталу, і між іншим про такі, де,
як у торговлі, капітал не ввіходить у безпосередній процес продукції, тобто
не функціонує як продуктивний капітал. Тим самим він покидає той ґрунт,
що на ньому стояли фізіократи, визначаючи відмінності різних частин
продуктивного капіталу та їхній вплив на характер обороту. Ба навіть
він одразу наводить, як приклад, купецький капітал у такому питанні,
де йдеться виключно про ріжниці частин продуктивного капіталу
в процесі утворення продукту й вартости — ріжниці, що й собі утворюють
ріжниці в обороті й репродукції капіталу.
\noclub[1]

\parcont{}  %% абзац починається на попередній сторінці
\index{iii1}{0135}  %% посилання на сторінку оригінального видання
було б 1 432 080 000 фунтів бавовни на рік. Але довіз бавовни,
коли відняти вивіз, становив в 1856 і 1857 роках тільки
1 022 576 832 фунти; отже, неминуче мусив постати дефіцит
у 409 503 168 фунтів. Пан Бейнс, який люб’язно згодився обговорити
зі мною цю справу, гадає, що обчислення річного споживання
бавовни, основане на споживанні блекбернської округи,
було б перебільшеним не тільки в наслідок ріжниці нумерів, що
випрядаються, але й в наслідок вищої якості машин. Він оцінює загальне
річне споживання бавовни в Сполученому Королівстві в
1000 мільйонів фунтів. Але якщо він і має рацію, якщо дійсно
є надмір подання в 22\sfrac{1}{2}  мільйони, то вже тепер попит і подання,
як видно, майже урівноважуються, навіть якщо не брати до
уваги додаткові веретена і ткацькі верстати, які, за паном
Бейнсом, встановлюються в його окрузі і, отже, напевно і
в інших округах“ (стор. 59, 60).

\subsection{Загальна ілюстрація: бавовняна криза 1861—1865 рр.}

\subsubsection{Попередній період 1845-1860 рр.}

1845 рік. Час розквіту бавовняної промисловості. Дуже низькі
ціни на бавовну. Л. Горнер каже про це: „Протягом останніх
8 років я не бачив жодного періоду такого пожвавленого стану
справ, як минулим літом і осінню. Особливо у бавовнопрядільництві.
Протягом цілого півроку я щотижня одержував заяви
про нові капіталовкладення у фабрики; це були або нові фабрики,
які будувалися, або ті нечисленні фабрики, які пустували, знаходили
нових орендарів, абож ті фабрики, які працювали, розширювались
і устатковувались новими потужнішими паровими машинами
та збільшеною кількістю робочих машин“ („Rep. of Insp.
of Fact., Oct. 1845“, стор. 13).

1846 рік. Починаються нарікання. „Уже протягом досить довгого
часу я чую від дуже багатьох бавовняних фабрикантів нарікання
на пригнічений стан їх справ\dots{} протягом останніх 6 тижнів
різні фабрики почали працювати неповний час, звичайно
8 годин на день замість 12; це, як видно, поширюється\dots{} дуже
підвищились ціни бавовни і\dots{} не тільки не підвищились ціни
фабрикатів, але\dots{} вони стоять ще нижче, ніж перед підвищенням
цін бавовни. Значне збільшення числа бавовняних фабрик
протягом останніх 4 років мусило мати своїм наслідком,
з одного боку, дуже збільшений попит на сировинний матеріал
і, з другого боку, дуже збільшене подання фабрикатів на ринку;
обидві причини мусили спільно сприяти зниженню зиску, поки
лишались незмінними подання сировинного матеріалу і попит
на фабрикати; але вони вплинули ще далеко дужче, бо, з одного
боку, за останній час було недостатнє подання бавовни,
і, з другого боку, зменшився попит на фабрикати на різних
внутрішніх і закордонних ринках“ („Rep. of Insp. of Fact., Oct.
1846“ стор. 10).

\parcont{}  %% абзац починається на попередній сторінці
\index{ii}{0136}  %% посилання на сторінку оригінального видання
капітал протилежно до його форми, належної до процесу продукції,
тобто протилежно до форми продуктивного капіталу. Це не різні
відміни, що на них поділяє промисловий капіталіст свій капітал, а різні
форми, що їх поступінно завжди набирає й скидає та сама авансована
капітальна вартість протягом свого curriculum vitae\footnote*{
Curriculum vitae (лат.) — перебіг життя. — \emph{Ред.}
}. А.~Сміс сплутує це —
роблячи великий крок назад порівняно з фізіократами — з тими відмінностями
форми, що постають у межах циркуляції капітальної вартости, в її кругобігу
через ряд її послідовних форм тоді, коли капітальна вартість перебуває
в формі продуктивного капіталу; і постають вони саме в наслідок
різних способів, що ними різні елементи продуктивного капіталу беруть
участь в процесі утворення вартости й переносять свою вартість на продукт.
Ми розглянемо далі наслідки цього основного сплутування капіталу
продуктивного й капіталу, що перебуває в сфері циркуляції (товарового
капіталу й грошового капіталу), з одного боку, і основного та поточного
капіталу, з другого. Капітальна вартість, авансована на основний капітал,
так само циркулює разом з продуктом, як і вартість, авансована на поточний
капітал, і через циркуляцію товарового капіталу перша так само перетворюється
на грошовий капітал, як і друга. Ріжниця виникає лише з того,
що вартість, авансована на основний капітал, циркулює частинами, а тому
й мусить вона також частинами, протягом довших або коротших періодів,
заміщуватись, репродукуватися в натуральній формі.

Що А.~Сміс розуміє тут під обіговим капіталом не що інше, як капітал
циркуляції, тобто капітальну вартість в її формах, належних до процесу
циркуляції (товаровий капітал і грошовий капітал), це доводить приклад,
обраний ним особливо невлучно. Він бере як приклад відміну капіталу, що
зовсім не належить до процесу продукції, а існує лише в сфері циркуляції,
складається лише з капіталу циркуляції: він бере купецький капітал.

Як безглуздо починати прикладом, де капітал взагалі фігурує не як
продуктивний капітал, він сам каже про це зараз же далі: „Капітал торговця
складається цілком з обігового капіталу“. („The capital of a merchant
is altogether a circulating capital“). Але ріжниця між обіговим і основним
капіталом постає, як нам далі скажуть, з посутніх ріжниць в середині
самого продуктивного капіталу. З одного боку, А.~Сміс має на
увазі визначену в фізіократів ріжницю, з другого боку, — відмінності форми
що їх пророблює капітальна вартість у процесі свого кругобігу. І те
й друге сплутує він в одну строкату купу.

Але як може утворюватись зиск в наслідок зміни форми грошей і
товару, в наслідок простого перетворення вартости з однієї з цих форм
на другу, це лишається цілком незрозуміло. Та й не можна зовсім цього
пояснити, бо він починає тут з купецького капіталу, що функціонує
лише в сфері циркуляції. Ми ще повернемось до цього, а покищо послухаймо,
що каже А.~Сміс про основний капітал:

„Подруге, його (капітал) можна застосовувати на поліпшення ґрунту,
на закуп корисних машин і знарядь праці та подібні речі, що дають
\parbreak{}  %% абзац продовжується на наступній сторінці

\parcont{}  %% абзац починається на попередній сторінці
\index{ii}{0137}  %% посилання на сторінку оригінального видання
дохід або зиск, не змінюючи власника, не циркулюючи далі. Тому такі
капітали можна назвати основними капіталами у власному значенні цього
слова. Різні підприємства потребують поділу вкладеного в них капіталу
на основний та обіговий в дуже різних пропорціях\dots{} Кожен ремісник
або фабрикант мусить деяку частину свого капіталу зв’язати в формі
засобів праці своєї галузі. Ця частина однак іноді дуже мала, іноді дуже
велика\dots{} Куди більша частина капіталу всіх цих ремісників (кравців, шевців,
ткачів) перебуває однак в циркуляції, то як заробітна плата їхніх
робітників, то як ціна їхнього сировинного матеріялу, і її треба оплатити
з зиском в ціні їхніх продуктів“\footnote*{
„Secondly, it (capital) may be employed in the improvement of land, in the
purchase of useful machines and instruments of trade, or in such like things as
yield a revenue or profit without changing masters, or circulating any further. Such
capitals therefore, may very properly be called fixed capitals. Different occupations
require very different proportions between the fixed and circulating capitals employed
in them\dots{} Some part of the capital of every master artificer or manufacturer
must be fixed in the instruments of his trade. This part, however, is very small in
some, and very great in others\dots{} The far greater part of the capital of all such master
artificers however is circulated, either in the wages of their workmen, or in the
price of their materials, and to be repaid with a profit by the price of the work“.
}.

Не кажучи вже про дитяче визначення джерела зиску, хибність і заплутаність
видно ось з чого: для фабриканта-машинобудівника, напр.,
машина є продукт, що циркулює як товаровий капітал, або, кажучи словами
А.~Сміта: „is parted with, changes masters, circulates further“ (відокремлюється,
змінює власника, циркулює далі). Отже, машина згідно з
його власним визначенням була б не основним, а обіговим капіталом. Ця
плутанина знову таки постає тому, що Сміт сплутує ріжницю між основним
і поточним капіталом, яка постає в наслідок неоднакових способів
циркуляції різних елементів продуктивного капіталу, з відмінностями
форми, що їх перебігає той самий капітал, оскільки він функціонує
в продукційному процесі як продуктивний капітал, а в сфері
циркуляції, навпаки, як капітал циркуляції, тобто як товаровий капітал
або як грошовий капітал. Тому ті самі речі, залежно від того місця, що
його вони мають у життьовому процесі капіталу, можуть, за А.~Смітом,
функціонувати і як основний капітал (як засоби праці, елементи продуктивного
капіталу) і як „обіговий“ капітал, товаровий капітал (як продукт,
виштовхнутий з сфери продукції в сферу циркуляції).

Але А.~Сміт сплутує разом з тим самі основи цього розподілу й суперечить
тому, з чого він почав кількома рядками вище цілий свій
дослід. Це саме сталось у реченні: „Є два способи застосувати капітал
так, щоб він давав своєму власникові дохід або зиск“, а саме — застосувати
його або як обіговий або як основний капітал. Тут мають на увазі,
очевидно, різні способи застосування різних і незалежних один від одного
капіталів, як, напр., капітали, що їх можна вкласти або в промисловість,
або в хліборобство. Але далі ми читаємо: „Різні підприємства
потребують поділу вкладеного в них капіталу на основний та обіговий
в дуже різних пропорціях“. Тепер основний та обіговий капітал є вже
\parbreak{}  %% абзац продовжується на наступній сторінці

\parcont{}  %% абзац починається на попередній сторінці
\index{ii}{0138}  %% посилання на сторінку оригінального видання
не різні самостійні вкладання капіталу, а різні частини того самого продуктивного
капіталу, які в різних сферах приміщення становлять різні
частини сукупної вартости цього капіталу. Отже, це — ріжниці, що випливають
з розподілу самого продуктивного капіталу відповідно до
обставин і тому мають силу лише для цього останнього. Але цьому знову
суперечить та обставина, що торговельний капітал, як виключно обіговий,
протиставиться основному, бо сам Сміт каже: „капітал торговця є
цілком обіговий капітал“. А справді це — капітал, що функціонує лише
в межах сфери циркуляції, і як такий він взагалі протистоїть продуктивному
капіталові, вкладеному в процес продукції, але саме тому його не
можна протиставляти як поточну (обігову) складову частину продуктивного
капіталу основній складовій частині продуктивного капіталу.

В прикладах, що їх наводить Сміт, він визначає, як основний капітал,
знаряддя праці; як обіговий капітал — ту частину капіталу, що витрачена
на заробітну плату й сировинний матеріял, зараховуючи сюди й допоміжні
матеріяли, „оплачувані із зиском в ціні продуктів“ („repaid with а
profit by the price of work“).

Отже, насамперед, за вихідний пункт тут є лише різні елементи процесу
праці: робоча сила (праця) й сировинний матеріял на одному боці,
знаряддя праці — на другому. Але все це є складові частини капіталу, бо
в них вкладено суму вартости, що має функціонувати як капітал. Остільки
це є речові елементи, способи буття продуктивного капіталу, тобто
капіталу, що функціонує в продукційному процесі. Чому ж одна частина
зветься основною? Тому що „деякі частини капіталу мусять бути фіксовані
в засобах праці“ („some parts of the capital must be fixed in the
instruments of trade“). Але друга частина теж є фіксована в заробітній
платі й сировинному матеріялі. Тимчасом, машини і „знаряддя праці\dots{} та
подібні речі\dots{} дають дохід або зиск, не змінюючи власника, не циркулюючи
далі. Тому такі капітали можна назвати основними капіталами у
власному значенні цього слова“.

Візьмімо, напр., гірничу справу. Сировинного матеріялу тут зовсім не
застосовується, бо предмет праці, прим., мідь, є продукт природи, що
його треба лише видобути за допомогою праці. Мідь, що її лише треба
видобути, — продукт процесу, що потім циркулює як товар, зглядно як
товаровий капітал, не становить жодного елемента продуктивного капіталу.
Жодна частина його вартости не вкладена в нього. З другого боку,
інші елементи продукційного процесу, робоча сила й допоміжні матеріяли,
як от вугілля, вода і~\abbr{т. ін.} так само не входять речово в продукт.
Вугілля зуживається цілком, і тільки його вартість ввіходить у продукт,
цілком так само, як частина вартости машин тощо ввіходить у продукт.
Нарешті, робітник лишається так само самостійним проти продукту, міді,
як і машина. Тільки вартість, спродукована його працею, є тепер складова
частина вартости міді. Отже, в цьому прикладі жодна з складових
частин продуктивного капіталу не змінює власника (masters) і жодна з
них не циркулює далі, бо жодна з них не ввіходить речово в продукт.
Де ж тут обіговий капітал? Згідно з власним визначенням А.~Сміта довелось
\index{ii}{0139}  %% посилання на сторінку оригінального видання
би визнати, що ввесь капітал, застосований на розробку мідних
копалень, є лише основний капітал.

Візьмімо, навпаки, іншу промисловість, яка застосовує сировинний
матеріял, що становить субстанцію продукту, і допоміжні матеріяли, які
своєю речовиною, а не лише вартістю — як от вугілля, що йде на опалення
— увіходять у продукт. Разом з продуктом, напр., пряжею, сировинний
матеріял, що з нього складається продукт, напр., бавовна, змінює
власника й переходить з процесу продукції в процес споживання. Але
поки бавовна функціонує як елемент продуктивного капіталу, власник не
продає її, а обробляє, наказує робити з неї пряжу. Власник не випускає
її з рук, або, вживаючи грубофалшивого тривіяльного вислову Смітового,
власник не здобуває жодного зиску, „коли продукт відокремлюється від
нього, коли змінюється його хазяїн або коли він циркулює“ (by parting
with it, by its changing masters, or by circulating it). Він так само мало
пускає в циркуляцію свої матеріяли, як і свої машини. Вони фіксовані
в продукційному процесі цілком так само, як прядільні машини та фабричні
будівлі. Частина продуктивного капіталу мусить навіть бути завжди
фіксована в формі вугілля, вовни тощо, так само, як і в формі засобів
праці.

Ріжниця лише та, що бавовну, вугілля та ін., потрібні, напр., для
щотижневої продукції пряжі, завжди цілком зужитковується на продукцію
тижневого продукту, і, значить, їх треба замінювати на нові екземпляри
бавовни, вугілля, тощо; отже, ці елементи продуктивного капіталу,
хоч вони лишаються тотожні своїм родом, постійно складаються з нових
екземплярів того самого роду, тимчасом як та сама поодинока прядільна
машина, та сама поодинока фабрична будівля й далі беруть участь
у цілому ряді повторюваних тижневих процесів продукції, не заміщуючись
на інші екземпляри того самого роду. Як елементи продуктивного
капіталу, всі його складові частини завжди фіксовані в процесі
продукції, бо без них він не може відбуватися. І всі елементи продуктивного
капіталу, основні й поточні, як продуктивний капітал, однаково
протистоять капіталові циркуляції, тобто товаровому капіталові й грошовому
капіталові.

Так само стоїть справа й щодо робочої сили. Частина продуктивного
капіталу завжди мусить бути фіксована в ній, і той самий капіталіст
протягом більш або менш довгого часу застосовує ті самі тотожні
поміж себе робочі сили, на зразок того, як застосовує ті самі машини.
Ріжниця між ними й машинами тут не в тому, що машину купується раз
назавжди (хоч цього не буває, коли, напр., за неї сплачують рати), а
робітника не на завжди, а в тому, що праця, яку витрачає робітник,
цілком входить у вартість продукту, тимчасом як вартість машини — лише
частинами.

Сміт плутає різні визначення, коли він про обіговий капітал, протилежно
до основного, каже таке: „Капітал, застосований таким способом,
не дає своєму власникові доходу або зиску, поки лишається в його посіданні
або зберігає ту саму форму“. Він ставить на один рівень ту лише
\index{ii}{0140}  %% посилання на сторінку оригінального видання
формальну метаморфозу товару, що її пророблює продукт, товаровий
капітал, в сфері циркуляції, та що упосереднює переміщення товару,
з рук до рук, і ту речову метаморфозу, що її пророблюють різні елементи
продуктивного капіталу протягом процесу продукції. Перетворення
товару на гроші й грошей на товар, купівлю й продаж, він, не довго
думаючи, сплутує тут з перетворенням елементів продукції на продукт.
Його приклад для обігового капіталу є купецький капітал, що перетворюється
з товару на гроші, з грошей на товар; це зміна форми $Т — Г — Т$,
належна до товарової циркуляції. Але ця зміна форми в межах циркуляції
має для діющого промислового капіталу те значення, що товари, на
які зворотно перетворюються гроші, є елементи продукції, засоби праці
й робоча сила; отже, те значення, що вона упосереднює безперервність
функціонування промислового капіталу, процес продукції, як безперервний
процес або як процес репродукції. Вся ця зміна форм відбувається
в циркуляції; саме вона упосереднює справжній перехід товарів з
одних рук до інших. Навпаки, метаморфози, що їх перебігає продуктивний
капітал в своєму процесі продукції, є метаморфози, належні до
процесу праці, доконечні, щоб перетворити елементи продукції на
згаданий продукт. А.~Сміт спиняється на тому, що одна частина засобів
продукції (власне засоби праці) функціонує в процесі праці (що він
неправильно висловлює, кажучи: „дає зиск їхньому власникові“ — yields а
profit to their master), не змінюючи своєї натуральної форми, зношується
лише поступінно, тимчасом як друга частина, матеріяли, змінюється, і
саме в наслідок своєї зміни виконує вона своє призначення як засіб продукції.
Але це різне поводження елементів продуктивного капіталу в
процесі праці становить лише вихідний пункт ріжниці між основним і неосновним
капіталом, а не саму цю ріжницю, і це видно вже з того, що
таке поводження однаково існує за всіх способів продукції, капіталістичних
і некапіталістичних. Але цьому різному речовому поводженню відповідає
віддача вартости продуктові, а цій віддачі знову таки відповідає
заміщення вартости за допомогою продажу продукту; і лише це
заміщення утворює ту ріжницю. Капітал, отже, є основний не тому, що
його зафіксовано в засобах праці, а тому, що частина його вартости,
вкладеної в засоби праці, лишається фіксована в них, тимчасом як друга
частина циркулює як складова частина вартости продукту.

„Коли він (капітал) застосовується для того, щоб утворити в майбутньому
зиск, то він мусить утворити цей зиск, або перебуваючи у
нього (власника), або одходячи від нього. В першому разі це основний,
в другому — обіговий капітал“\footnote*{
„If it (the stock) is employed in procuring future profit, it must procure this
profit by staying with him (the employer), or by going from him. In the one case
it is a fixed, in the other it is a circulating capital“ (p. 189).
}.

Тут насамперед впадає в очі грубо емпіричне — перейняте з уявлення
звичайного капіталіста — уявлення про зиск, яке цілком суперечить ліпшому
езотеричному поглядові А.~Сміта. В ціні продукту заміщується і ціну
матеріялів, і ціну робочої сили, але так само і ту частину вартости знарядь
\index{ii}{0141}  %% посилання на сторінку оригінального видання
праці, що переходить на продукт в наслідок зношування знарядь
праці. З цього заміщення ще зовсім не утворюється зиск. Залежно від
того, чи заміщується в наслідок продажу продукту авансовану на його
продукцію вартість цілком чи частинами, разом чи поступінно, може змінитися
лише спосіб і час заміщення; але це ніяк не може перетворити
спільне обом випадкам заміщення вартости на утворення додаткової вартости.
В основі маємо тут звичайне уявлення, що додаткова вартість —
тому що її реалізується лише через продаж продукту, через його циркуляцію,
— виникає лише з продажу, з циркуляції. А справді різні
способи постання зиску є тут лише неправильні вислови того, що різні
елементи продуктивного капіталу відіграють різну ролю, неоднаково
функціонують в процесі праці як продуктивні елементи. Нарешті, ця
ріжниця висновується в нього не з процесу праці, зглядно процесу зростання
вартости, не з функції самого продуктивного капіталу, а повинна
мати лише суб’єктивне значення для поодинокого капіталіста, що для
нього одна частина капіталу корисна цим, а друга — тим способом.

Навпаки, Кене висновує ці ріжниці з самого процесу репродукції
та його доконечности. Для того, щоб цей процес був безперервний,
вартість річних авансів мусить цілком заміщуватись з вартости річного
продукту, і навпаки — вартість основного капіталу мусить заміщуватись
лише частинами, так що лише протягом ряду років, прим., десятиліття,
її цілком заміщується, а тому й репродукується цілком (заміщується новими
екземплярами того самого роду). А.~Сміт, отже, робить великий
крок назад порівняно з Кене.

Таким чином, для визначення основного капіталу А.~Смітові не лишається
зовсім нічого іншого, як сказати, що це — засоби праці, які, протилежно
до продуктів, що їх утворенню вони допомагають, не змінюють
своєї форми в процесі продукції та функціонують далі в продукції, поки
зносяться. При цьому забувають, що всі елементи продуктивного капіталу
в своїй натуральній формі (як засоби праці, матеріяли й робоча сила)
постійно протистоять продуктові, і продуктові, що циркулює як товар,
і що ріжниця між частиною, яка складається з матеріялів та робочої
сили, і частиною, яка складається з засобів праці, лише в тому, що
робочу силу завжди купується наново (а не на ввесь час її існування,
як купується засоби праці), тимчасом як у процесі праці функціонують
не ті самі тотожні, а завжди нові екземпляри матеріялів того самого роду.
Разом з тим постає ілюзія, ніби вартість основного капіталу не циркулює,
хоч А.~Сміт звичайно зазначав раніше, що зношування основного
капіталу ввіходить як частина в ціну продуктів.

При визначенні обігового капіталу як протилежности до основного,
не пояснюється, що обіговий капітал має цю протилежність лише як та
складова частина продуктивного капіталу, яка мусить цілком заміститись
з вартости продукту й тому мусить цілком брати участь у його
метаморфозах, тимчасом як із основним капіталом цього немає. Замість
пояснити це, А.~Сміт сплутує обіговий капітал з тими формами, що їх
набирає капітал, переходячи зі сфери продукції в сферу циркуляції, —
\parbreak{}  %% абзац продовжується на наступній сторінці

\parcont{}  %% абзац починається на попередній сторінці
\index{ii}{0142}  %% посилання на сторінку оригінального видання
з формами товарового капіталу і грошового капіталу. Але обидві ці форми,
товаровий капітал і грошовий капітал, є носії вартости так само
основної, як і поточної частини продуктивного капіталу. Обидві вони є
капітал циркуляції протилежно до капіталу продуктивного, а не обіговий
(поточний) капітал протилежно до основного.

Нарешті, цілком неправильне пояснення утворення зиску основним
та обіговим капіталом, а саме, що перший ніби утворює зиск, лишаючись
у процесі продукції, а другий — виходячи з нього та циркулюючи далі, —
призводить до того, що через однаковість форми, що її в обороті
мають змінний капітал і поточна складова частина сталого капіталу, приховується
їхня посутня ріжниця в процесі зростання вартости й
утворення додаткової вартости, отже, уся таємниця капіталістичної продукції
ще більше затемнюється. Через загальне означення: „обіговий капітал“,
знімається (wird aufgehoben) цю посутню ріжницю; це повело до
того, що пізніші економісти пішли ще далі: за посутню й єдино відмінну
вони визнавали протилежність не між змінним і сталим капіталом, а
протилежність між основним і обіговим капіталом.

Визначивши основний і обіговий капітал як два різні способи приміщувати
капітал, що з них кожен, сам по собі розглядуваний, дає зиск,
А.~Сміс каже: „Жодний основний капітал не може давати зиску без допомоги
обігового капіталу. Найкорисніші машини та знаряддя праці нічого
не випродукують без обігового капіталу, що дає матеріяли, ними
оброблювані, і дає утримання робітникам, які їх застосовують\footnote*{
„No fixed capital can yield any revenue but by means of a circulating capital.
The most useful machines and instruments of trade will produce nothing wit
hout the circulating capital which affords the materials they are employed upon, and
the maintenance of the workmen who employ them“ (p. 188).
}.

Тут виявляється, що значать попередні вирази: yield a revenue make
a profit\footnote*{
Давати дохід, творити зиск.
} і~\abbr{т. ін.}, а саме — це значить, що обидві частини капіталу служать
як продуктотворчі.

А.~Сміс наводить потім такий приклад: „Частина капіталу фармерового,
вкладена в господарські знаряддя, є основний капітал, а частина,
вкладена в заробітну плату і утримання слуг-робітників, є обіговий капітал“.
(Тут ріжницю між основним й обіговим капіталом правильно зведено
тільки до різної циркуляції, до різного обороту різних складових
частин продуктивного капіталу). „Фармер має зиск від першого, поки має
його в своєму розпорядженні, і від другого, віддаючи його від себе.
Ціна або вартість його робочої худоби є основний капітал“ (тут знову таки
правильно те, що за основу різниці береться вартість, а не речовий
елемент) „так само, як і ціна або вартість господарських знарядь; засоби
утримання її (робочої худоби) є обіговий капітал так само, як і засоби
утримання слуг-робітників. Фармер одержує зиск, лишаючи в своєму
розпорядженні робочу худобу й віддаючи продукти, що є худобі за засіб
існування“. (Фармер лишає корм худобі, не продає його. Він зуживає
його як корм худобі, зуживаючи саму худобу як знаряддя праці.
Ріжниця лише ось у чому: корм для худоби, що зуживається на утримання
\index{ii}{0143}  %% посилання на сторінку оригінального видання
робочої худоби, споживається цілком і мусить постійно заміщуватись
новим кормом безпосередньо з продукту хліборобства або з продажу
його; тимчасом як саму худобу заміщується лише в міру того, як
поодинокі екземпляри її по черзі стають непрацездатні). „І ціна й утримання
худоби, купленої не для роботи, а на відгодовування, є обіговий капітал.
Фармер одержує зиск, віддаючи його“. (Кожен товаропродуцент,
а, значить, і капіталістичний товаропродуцент, продає свій продукт,
результат свого процесу продукції, але через це цей продукт
ще не стає ні основною, ні поточною складовою частиною його
продуктивного капіталу. Навпаки, продукт має тепер ту форму, що
в ній він виштовхується з процесу продукції й мусить функціонувати як товаровий
капітал. Відгодовувана худоба функціонує в процесі продукції як
сировинний матеріял, а не як знаряддя праці, не як робоча худоба. Вона,
отже, входить речово в продукт, і вся її вартість переходить цілком
на цей продукт, як і вартість допоміжних матеріялів (її корму). Саме тому
вона й є поточна частина продуктивного капіталу, а зовсім не тому,
що проданий продукт — відгодована худоба — має тут ту саму натуральну
форму, що й сировинний матеріял, тобто ще не відгодована худоба. Це —
цілком випадкова обставина. Але разом з тим Сміс міг би побачити з
цього прикладу, що не речова форма елемента продукції, а лише його
функція в межах продукційного процесу надають вартості, що міститься
в ньому, характеру основної або поточної частини). „Вся вартість засівного
зерна є теж основний капітал. Хоч воно завжди переходить з землі
в комори й назад, але воно ніколи не змінює власника, а тому в
дійсності й не циркулює. Фармер одержує свій зиск не тому, що продає
його, а тому, що кількість його більшає\footnote*{
„That part of the capital of the farmer which is employed in the implements
of agriculture is a fixed, that which is employed in the wages and maintenance of
his labouring servants is a circulating capital. He makes a profit of the one by
keeping it in his own possession, and of the other by parting with it. The price or
value of his labouring cattle is a fixed capital, in the same manner as that of the
instruments of husbandry; their maintenance is a circulating capital, in the same way
as that of the labouring servants. The farmer makes his profit by keeping the labouring
cattle, and by parting with their maintenance. Both the price and the maintenance
of the cattle which are bought in and fattened, not for labour but for sale,
are a circulating capital. The farmer makes his profit by parting with them. The whole
value of the seed, too is a fixed capital. Though it goes backwards and
forwards between the ground and the granary, it never changes masters, and therefore
it does not properly circulate. The farmer makes his profit not by its sale, but
by its increase“.
}.

Тут особливо яскраво виявляється вся безглуздість Смісового відрізнення.
За його теорією, засівне зерно було б основним капіталом, якби не
відбулося change of masters\footnote*{
Зміна власника. \emph{Ред.}
}, тобто, коли засівне зерно безпосередньо
заміщується з річного продукту, береться з нього. Навпаки,
воно було б обіговим капіталом, коли б продавалось увесь продукт і на
частину вартости його купувалось засівне зерно в другого власника. В
одному разі відбувається change of masters, в другому ні. Сміс тут знову
\index{ii}{0144}  %% посилання на сторінку оригінального видання
таки сплутує поточний капітал і товаровий капітал. Продукт є речовий
носій товарового капіталу. Але, звичайно, лише та частина продукту,
яка справді входить в циркуляцію і не входить знову безпосередньо в
той самий процес продукції, відки вона вийшла, як продукт.

Чи береться зерно, як частина, безпосередньо з продукту, чи продається
ввесь продукт і частину його вартости за допомогою купівлі перетворюється
на чуже зерно, і в тому і в другому випадку маємо лише заміщення
вартости, і цим заміщенням не утворюється жодного зиску. В
першому випадку зерно разом з рештою продукту входить як товар в
циркуляцію, а в другому випадку воно фігурує лише в бухгальтерії як
складова частина вартости авансованого капіталу. Але в обох випадках
воно лишається поточною складовою частиною продуктивного капіталу.
Його зуживається цілком, щоб виготовити продукт, і воно мусить цілком
заміститися з нього, щоб уможливити репродукцію.

„Сировинні й допоміжні матеріяли втрачають ту самостійну форму, в
якій вони ввійшли в процес праці як споживні вартості. Інша справа
з власне засобами праці. Інструмент, машина, фабричний будинок, посуд
і~\abbr{т. ін.} служать у процесі праці лише доти, доки зберігають вони
свою первісну форму, доки й завтра можуть вони входити в процес праці
у тій самій формі, що й учора. І як за свого життя, тобто протягом
процесу праці, вони зберігають проти продукту свою самостійну форму,
так само зберігають її вони й після своєї смерти. Трупи машин, майстерень,
фабричних будівель і далі все ще існують самостійно, окремо
від продуктів, творенню яких вони допомагали“. (Капітал, кн. І, розд. VI).

Ці різні способи застосування засобів продукції для створення продукту,
— при чому одні засоби продукції зберігають свою самостійну форму
проти продукту, а інші змінюють або цілком втрачають її, — цю ріжницю,
властиву процесові праці, як такому, отже, ріжницю, що так само
властива й такому процесові праці, що має на меті задовольнити лише
власні потреби, прим., патріярхальної сім’ї, без якогобудь обміну, без
товарової продукції, — А.~Сміс освітлює неправильно, бо: 1) він притягує
зовсім неналежне сюди визначення зиску: що одні засоби продукції дають
своєму власникові зиск, зберігаючи свою форму, а інші дають зиск,
втрачаючи її; 2) зміни частини елементів продукції в процесі праці він
сплутує з тією зміною форми, яка властива обмінові продуктів, товаровій
циркуляції (купівлі та продажеві), і яка разом з тим включає зміну
власности на товари, що циркулюють.

Оборот має собі за передумову репродукцію, упосереднювану циркуляцією,
тобто продажем продукту, перетворенням його на гроші і зворотним
перетворенням з грошей на елементи його продукції. Але оскільки
капіталістичному підприємцеві частина його власного продукту безпосередньо
сама знову служить як засіб продукції, продуцент виступає як
продавець того самого продукту самому собі і саме так фігурує ця операція
в його бухгальтерії. Отже, ця частина репродукції не упосереднюється
циркуляцією, а відбувається безпосередньо. Але та частина продукту,
що таким чином знову служить як засіб продукції, заміщує
\parbreak{}  %% абзац продовжується на наступній сторінці

\parcont{}  %% абзац починається на попередній сторінці
\index{iii2}{0145}  %% посилання на сторінку оригінального видання
надпродукт з земель $А$, $В$ або якогось іншого розряду можна було б одержати
лише по вищій ціні, ніж 3\pound{ ф. стерл.}, — тільки в цьому випадку зі зменшенням
продукту від додаткової витрати капіталу на якийсь з розрядів $А$, $В$, $C$, $D$
було б зв’язане підвищення ціни продукції і реґуляційної ринкової ціни.
Коли б це усталилось на тривалий час і не спричинило б оброблення додаткової
землі $А$ (принаймні, якости $А$) і взагалі ніякі інші впливи не призвели б
до дешевшого подання, то, за інших незмінних умов, заробітна плата підвищилася б
в наслідок подорожчання хліба і відповідно до цього знизилась би норма зиску.
В цьому випадку було б байдуже, чи задовольнявся б підвищений попит втягненими
до обробітку гіршої, ніж $А$, землі, чи додатковим приміщенням капіталу
у будь-яку з чотирьох родів землі. Диференційна рента підвищилася б в зв’язку
з пониженням норми зиску.

Цей випадок, коли низхідна продуктивність додаткових капіталів, вкладуваних
в землі, що вже перебувають під культурою, може призвести до підвищення
ціни продукції, пониження норми зиску і створення вищої диференційної
ренти, — бо ця остання за даних умов підвищилася б на всіх родах землі
цілком так само, як коли б гірша, ніж $А$, земля стала тепер реґулювати ринкову
ціну, — цей випадок Рікардо перетворює в єдиний випадок, в нормальний
випадок, до якого він зводить все створення диференційної ренти II.

Так воно і було б, коли б оброблювалось лише рід землі $А$ і коли б послідовні
вкладення капіталу на ній не були зв’язані з пропорційним приростом
продукту.

Отже, тут у випадку з диференційною рентою II цілком губиться з пам’яти
диференційна рента І.

За винятком цього випадку, коли або подання з оброблюваних земель
недостатнє, і тому ринкова ціна довгочасно перевищує ціну продукції, поки не
почнеться оброблення нових додаткових земель гіршої якости, або поки ввесь продукт
додаткового капіталу, вкладеного в землю різних розрядів, не буде можливости
збувати по вищій ціні продукції, ніж суща до того часу, — за винятком
цього випадку пропорційне зменшення продуктивности додаткових капіталів не
зачіпає реґуляційної ціни продукції і норми зиску. А втім, можливі ще три
такі випадки:

а) Коли додатковий капітал, вкладений у землю якогось роду $А$, $В$, $C$, $D$,
дає лише норму зиску, визначувану ціною продукції на $А$, то через це не
створюється жодного надзиску, отже, і жодної можливої (евентуальної) ренти;
так само, як коли б почала оброблятися додаткова земля $А$.

в) Коли додатковий капітал дасть більшу кількість продукту, то, як само
собою зрозуміло, створюється новий надзиск (потенціяльна рента), якщо реґуляційна
ціна лишається колишня. Останнє не завжди так буває, саме не буває
тоді, коли ця додаткова продукція виключає землю $А$ з числа оброблюваних, а
разом з тим з числа конкурентних родів землі. В цьому випадку реґуляційна ціна
продукції знижується. Норма зиску підвищилась би, коли б з цим було зв’язане
зниження заробітної плати, або коли б дешевший продукт увійшов елементом
в сталий капітал. Коли б додаткові капітали дали вищу продуктивність на
землях кращих родів $C$ і $D$, то тільки від ступеня підвищення продуктивности
і від маси нововкладених капіталів залежало б, наскільки створення збільшеного
надзиску (отже, і збільшеної ренти) сполучалося б з пониженням ціни і підвищенням
норми зиску. Ця остання може підвищуватися і без пониження заробітної
плати, в наслідок здешевлення елементів сталого капіталу.

c) Якщо додаткове приміщення капіталу відбувається за зменшуваного
надзиску, але все-ж так, що його продукт дає надмір проти продукту такого ж
самого капіталу на землі $А$, то, якщо тільки збільшене подання не виключить
землі $А$ з числа оброблюваних земель, за всяких умов відбудеться створення
\parbreak{}  %% абзац продовжується на наступній сторінці

\parcont{}  %% абзац починається на попередній сторінці
\index{iii1}{0146}  %% посилання на сторінку оригінального видання
часто міняються, і заробіток робітників підвищується чи падає
залежно від якості бавовняної мішанки. Іноді лишалося тільки
15\% від колишнього заробітку, і за один чи два тижні він
падав на 50 або 60\%“. Інспектор Редгрев, що його ми тут
цитуємо, наводить взяті з практики дані про заробітну плату,
з яких тут досить буде таких прикладів:

$А$, ткач, сім’я з 6 осіб, занятий 4 дні на тиждень, 6 шилінгів
8\sfrac{1}{2}, пенсів; $В$, twister [присукальник], 4\sfrac{1}{2} дні на тиждень, 6 шилінгів;
$C$, ткач, сім’я з 4 осіб, 5 днів на тиждень, 5 шилінгів
1 пенс; $D$, slubber [тростильник], сім’я з 6 осіб, 4 дні на тиждень,
7 шилінгів 10 пенсів; $Е$, ткач, сім’я з 7 осіб, 3 дні на
тиждень, 5 шилінгів і т. д. Редгрев каже далі: „Вищенаведені
дані заслуговують уваги, бо вони показують, що для деяких
сімей робота була б нещастям, тому що вона не тільки скоротила
б їх дохід, але й знизила б його настільки, що його вистачило
б тільки на задоволення незначної частини абсолютно
необхідних потреб, якщо не давалося б додаткової допомоги
в тих випадках, коли заробіток сім’ї не досягає тієї суми, яку
вона одержувала б як допомогу, коли б усі члени сім’ї були
без роботи“ („Rep. of Insp. of Fact., Oct. 1863“, стор. 50--53).

„Починаючи з 5 червня 1863 року, не було жодного тижня,
на протязі якого весь робочий час усіх робітників становив би
пересічно більше двох днів 7 годин і кількох хвилин“ (там же,
стор. 121).

З початку кризи до 25 березня 1863 року майже три мільйони
фунтів стерлінгів було витрачено установами піклування
про бідних, центральним комітетом допомоги і лондонським
Mansion-House [муніципальним] комітетом (стор. 13).

„В одній окрузі, де випрядається найтонша пряжа\dots{} прядільники
підпали посередньому зниженню заробітної плати на 15\%
в наслідок переходу від Sea Island до єгіпетської бавовни\dots{}
В одній великій окрузі, де бавовняні відпади застосовуються
великими масами для домішки до індійської бавовни, заробітна
плата прядільників була знижена на 5\% і, крім того, вони
втратили ще 20--30\% в наслідок перероблення сурату й відпадів.
Ткачі, які працювали раніш коло чотирьох верстатів, перейшли
тепер на два верстати. В 1860 році вони виробляли на кожному
верстаті 5 шилінгів 7 пенсів, в 1863 році — тільки 3 шилінги
4 пенси\dots{} Грошові штрафи, які раніш, при застосуванні американської
бавовни, коливалися від 3 до 6 пенсів“ [для прядільників],
„доходять тепер до 1 шилінга — 3 шилінгів 6 пенсів“.
В одній окрузі, де вживалась єгіпетська бавовна, змішана
з ост-індською, „пересічна заробітна плата прядільника на мюлях
в 1860 році становила 18--25 шилінгів, а тепер 10--18 шилінгів.
Це викликано не самим тільки погіршенням бавовни,
але і зменшенням швидкості мюлів для того, щоб надати
пряжі дужчого крутіння, — за що в звичайні часи згідно з умовою
про заробітну плату платилося додатково“ (стор. 43, 44,
\parbreak{}  %% абзац продовжується на наступній сторінці

\parcont{}  %% абзац починається на попередній сторінці
\index{iii1}{0147}  %% посилання на сторінку оригінального видання
45--50). „Хоч подекуди ост-індська бавовна, можливо, перероблялася
з зиском для фабрикантів, проте, ми бачимо (див. розрахунковий
лист заробітків, стор. 53), що робітники порівняно
з 1861 роком терплять від цього. Якщо вживання сурату укріпиться,
то робітники вимагатимуть такого самого заробітку, як
в 1861 році; але таке підвищення заробітної плати серйозно
відбилося б на зиску фабриканта, якщо тільки воно не скомпенсується
ціною або бавовни, або фабрикатів“ (стор. 105).

\emph{Квартирна плата}. „Квартирна плата робітників, в тих випадках,
коли котеджі, в яких вони живуть, належать фабрикантові,
часто відраховується фабрикантом із заробітної плати,
навіть коли робітники працюють неповний час. Не зважаючи
на це, вартість цих будівель знизилась, і хатки тепер можна
мати на 25--50\% дешевше, ніж раніше; котедж, який раніше
коштував 3\shil{ шилінги} 6\pens{ пенсів} за тиждень, тепер можна мати за шилінги 4\pens{ пенси}, а іноді ще дешевше“ (стор. 57).

\emph{Еміграція}. Фабриканти, звичайно, були проти еміграції робітників,
почасти тому, що вони, „чекаючи кращих часів для бавовняної
промисловості, хотіли зберегти в своїх руках засоби
для того, щоб провадити виробництво на своїх фабриках якнайвигідиішим
способом“. Але, крім того, „багато фабрикантів
є власники будинків, в яких живуть заняті ними робітники,
і принаймні деякі з них безумовно розраховують на те, що
пізніше одержать частину квартирної плати, яку їм заборгували
робітники“ (стор. 96).

Пан Берноль Осборн каже в одній з промов до своїх виборців
у парламент від 22 жовтня 1864 року, що робітники
Ланкашіра поводились як античні філософи (стоіки). Чи не як вівці?

\section{Додатки}

Припустімо, як ми це робимо в цьому відділі, що маса
зиску, привласнювана в кожній окремій сфері виробництва, дорівнює
сумі додаткової вартості, яку створює весь капітал, вкладений
у цю сферу. Все ж буржуа не сприйматиме зиск як тотожний
з додатковою вартістю, тобто з неоплаченою додатковою
працею, і саме з таких причин:

1) В процесі циркуляції він забуває процес виробництва.
Реалізація вартості товарів — яка включає і реалізацію вміщеної
в них додаткової вартості — йому здається утворенням додаткової
вартості. [Незаповнена прогалина в рукопису вказує на те,
що Маркс мав намір докладніше розвинути цей пункт. — \emph{Ф. Е.}]

2) Якщо припустити незмінний ступінь експлуатації праці, то,
як уже виявилось, норма зиску, незалежно від усіх викликаних
кредитною системою модифікацій, від усякого взаємного ошуканства
і шахрайства капіталістів, незалежно, далі, від усякого
\parbreak{}  %% абзац продовжується на наступній сторінці

\parcont{}  %% абзац починається на попередній сторінці
\index{ii}{0148}  %% посилання на сторінку оригінального видання
він вийшов, як, напр., вугілля в продукції вугілля, — навіть і тут саме
частина продукту — вугілля, призначена на продаж, не являє ані поточного,
ані основного капіталу, а є вона товаровий капітал.

З другого боку, продукт може мати таку споживну форму, яка робить
його цілком непридатним для того, щоб становити якийнебудь елемент
продуктивного капіталу, чи то як матеріял праці, чи то як засіб
праці. Так, напр., деякі засоби існування. А проте, для своїх продуцентів
він все ж є товаровий капітал, носій вартости так основного, як поточного
капіталу, — першого або другого залежно від того, чи цілком,
чи частинами треба заміщувати вжитий для його продукції капітал, чи
цілком чи частинами цей капітал переносить свою вартість на продукт.

У Сміта в пункті 3 фігурує сировинний матеріял (сировина, напівфабрикат,
допоміжний матеріял), з одного боку, не як складова частина,
що вже ввійшла в продуктивний капітал, а справді лише як особливий ґатунок
споживних вартостей, що з них взагалі складається суспільний
продукт, як особливий ґатунок товарової маси, поряд з переліченими в
пунктах 2 і 4 іншими речовими складовими частинами, засобами існування
тощо. З другого боку, і самі сировинні матеріяли наведено в нього, в
усякому разі, як елементи, що ввійшли в продуктивний капітал, а тому й
як елементи останнього, які є в руках продуцента. Плутанина виявляється в
тому, що, почасти, ці сировинні матеріяли розглядається як діющі в руках
продуцента („в руках торговців худобою, мануфактуристів“ тощо), з
другого боку, як перейшлі до рук торговців („дрібних крамарів, торговців
сукном, лісоматеріялами“ і~\abbr{т. ін.}), де вони є тільки товаровий капітал, а
не складові частини продуктивного капіталу.

А.~Сміт, перелічуючи тут елементи обігового капіталу, в дійсності цілком
забуває про ріжницю між основним і поточним капіталом, яка має
силу тільки для продуктивного капіталу. Навпаки, товаровий капітал і
грошовий капітал, тобто обидві форми капіталу, що належать до процесу
циркуляції, він протиставить продуктивному капіталові, та й то несвідомо.
Варто, нарешті, уваги, що А.~Сміт, перелічуючи складові частини обігового
капіталу, забуває про робочу силу. У цьому є дві причини.

Ми щойно бачили, що, коли облишити осторонь грошовий капітал,
обіговий капітал є в нього лише друга назва товарового капіталу. Але
оскільки робоча сила циркулює на ринку, вона не є капітал, не є будь-яка
форма товарового капіталу. Вона взагалі не капітал; робітник не є
капіталіст, хоч він і виносить на ринок товар, а саме — свою власну шкуру.
Лише після того, як робочу силу вже продано і введено в продукційний
процес, — отже, лише після того, як вона перестала циркулювати
як товар, — вона стає складовою частиною продуктивного капіталу: змінним
капіталом як джерело додаткової вартости, поточною складовою
частиною продуктивного капіталу щодо обороту витраченої на неї капітальної
вартости Сміт сплутує тут поточний капітал з товаровим капіталом,
а тому й не може він підвести робочу силу під свою рубрику обігового
капіталу. Тому змінний капітал виступає в нього в формі тих
\parbreak{}  %% абзац продовжується на наступній сторінці

\parcont{}  %% абзац починається на попередній сторінці
\index{iii1}{0149}  %% посилання на сторінку оригінального видання
Якщо вартість золота падає чи підвищується
% REMOVED \footnote*{
% В першому німецькому виданні тут сказано: „підвищується чи падає“;
% виправлено на підставі рукопису Маркса. \Red{Примітка ред. нім. вид. ІМЕЛ.}
% }
на 100\%, то
в першому випадку той самий капітал, який раніше був вартий
100\pound{ фунтів стерлінгів}, буде вартий 200\pound{ фунтів стерлінгів}, а зиск
матиме вартість в 40\pound{ фунтів стерлінгів}, тобто виражатиметься
в цій грошовій сумі замість колишніх 20\pound{ фунтів стерлінгів}.
В другому випадку капітал падає до вартості в 50\pound{ фунтів
стерлінгів}, і зиск виражається в продукті вартістю в 10\pound{ фунтів
стерлінгів}. Але в обох випадках 200 : 40 \deq{} 50 : 10 \deq{} 100 : 20 \deq{} 20\%.
Однак, в усіх цих випадках в дійсності не сталося б ніякої
зміни у величині капітальної вартості, сталася б зміна тільки
в грошовому виразі тієї самої вартості і тієї самої додаткової
вартості. Отже, це не могло б вплинути і на \frac{m}{K}, або на норму
зиску.

Другий випадок — той, коли має місце дійсна зміна величини
вартості, але зміна ця не супроводиться зміною відношення
$v : c$, тобто коли при незмінній нормі додаткової вартості відношення
капіталу, витраченого на робочу силу (розглядаючи
змінний капітал як показник приведеної в рух робочої сили), до
капіталу, витраченого на засоби виробництва, лишається те
саме. При таких умовах, якщо ми маємо $К$, чи $nK$, чи \frac{K}{n}, наприклад,
1000, чи 2000, чи 500, зиск, при нормі зиску в 20\%,
буде в першому випадку \deq{} 200, в другому \deq{} 400, в третьому \deq{} 100;
але \frac{200}{1000} \deq{} \frac{400}{2000} \deq{} \frac{100}{500} \deq{} 20\%. Тобто норма зиску лишається тут
незмінною, бо склад капіталу лишається той самий і не зачіпається
зміною величини капіталу. Тому збільшення чи зменшення
маси зиску вказує тут тільки на збільшення чи зменшення
величини застосовуваного капіталу.

Отже, в першому випадку має місце тільки позірна зміна
величини застосовуваного капіталу, в другому випадку відбувається
дійсна зміна величини, але не відбувається ніякої зміни
в органічному складі капіталу, не відбувається ніякої зміни
відношення змінної частини капіталу до його сталої частини.
Але, за винятком цих обох випадків, зміна величини застосовуваного
капіталу є або \emph{наслідок} попередньої зміни вартості
однієї з його складових частин, а тому (оскільки із зміною змінного
капіталу не змінюється сама додаткова вартість) і зміни
у відносній величині його складових частин; або ця зміна величини
капіталу (як при роботах у великому масштабі, введенні нових
машин і~\abbr{т. д.}) є \emph{причина} зміни у відносній величині обох його
органічних складових частин. Тому в усіх цих випадках при
інших однакових умовах зміна величини застосовуваного капіталу
мусить супроводитись одночасною зміною норми зиску.

\parcont{}  %% абзац починається на попередній сторінці
\index{iii2}{0150}  %% посилання на сторінку оригінального видання
додаткові земельні дільниці, що їх різні ступені родючости розподіляються між
$D$ і $А$, між родючістю кращої і гіршої землі. Коли послідовні вкладення капіталу
відбуваються виключно на землі $D$, то вони можуть включати ріжниці, що
існують між $D$ і $А$, далі ріжниці між $D$ і $C$ так само як і ріжниці між $D$ і $В$.
Коли ж усі вони відбуваються на землі $C$, то лише ріжниці між $C$ і $А$ або $В$;
коли на $В$ — то лише ріжниці між $В$ і $А$.

Але закон такий: рента на землях усіх цих родів абсолютно зростає, хоч
і не пропорційно додатково вкладеному капіталові.

Норма надзиску зменшується так у відношенні до додаткового капіталу,
як і у відношенні до всього вкладеного в землю капіталу; але абсолютна величина
надзиску збільшується; цілком так само як зменшення норми надзиску
на капітал взагалі здебільша зв’язане зі збільшенням абсолютної маси зиску.
Так, пересічний надзиск з капіталу вкладеного в $В = 90\%$ на капітал, тимчасом
як при першому вкладенні капіталу він = 120\%. Але загальний надзиск
збільшується з 1 квартера до 1\sfrac{1}{2} кв. і з 3\pound{ ф. стерл.} до 4\sfrac{1}{2} . Вся рента,
розглядувана сама по собі, — а не в відношенні до подвоєного розміру авансованого
капіталу — абсолютно зросла. Рiжниці між рентами різних родів землі і
їхнє відношення одна до однієї можуть тут змінюватися; але ця зміна ріжниці,
є тут наслідок, а не причина збільшення рент однієї проти однієї.

ІV. Випадок, коли додаткові вкладення капіталу на кращих землях породжують
більшу кількість продукту, ніж первісні не потребує дальшої аналізи.
Само собою зрозуміло, що за такого припущення ренти з кожного
акра підвищуються і при тому в більшій пропорції, ніж додатковий капітал,
хоч би в який рід землі він був вкладений. В цьому випадку додаткове
капіталовкладення зв’язано з поліпшенням. Де буває тоді, коли додаткове
вкладення меншого капіталу впливає так само або більш продуктивно, ніж зроблене
давніш додаткове вкладення більшого капіталу. Випадок цей не зовсім тотожній
з давнішим, і ріжниця між ними має важливе значення при всіх
вкладеннях капіталу. Коли, наприклад, 100 дають зиску 10, а 200 при певній
формі вживання — зиск в 40, то зиск збільшується з 10\% до 20\%, і остільки
це є те саме, як коли б 50, при ефективнішій формі вживання, дали зиск в 10
замість 5. Ми припускаємо тут, що зиск є зв’язаний з відповідним збільшенням
продукту. Але ріжниця є в тому, що в одному випадку я мушу подвоїти капітал,
тоді як в другому досягаю подвоєного ефекту при колишньому капіталі.
Зовсім не є те саме, чи продукую я: 1) колиши й продукт, витрачаючи половину
колишньої кількости живої й зрічевленої праці, чи 2) подвоєний продукт
за колишньої кількости праці, чи 3) почвірний продукт за подвійної кількости
праці. В першому випадку праця — в живій або зрічевленій формі — стає вільна
і може бути вжита якось інакше; зростає можливість порядкувати працею
і капіталом. Звільнення капіталу (і праці) само по собі є збільшення багатства;
воно впливає цілком так само, як коли б цей додатковий капітал був здобутий
з допомогою акумуляції, але воно заощаджує працю акумуляції.

Припустімо, що капітал в 100 випродукував продукт в 10 метрів. В 100 є так
сталий капітал, як жива праця й зиск. Таким чином, метр коштує 10. Коли я тепер з
таким самим капіталом в 100 можу випродукувати 20 метрів, то метр коштуватиме 5.
Коли навпаки, я можу з капіталом в 50 випродукувати 10 метрів, то метр також
коштуватиме 5, але в цьому випадку звільняється капітал в 50, якщо колишнє подання
товару достатнє. Коли я мушу вкласти капітал в 200, щоб випродукувати
40 метрів, то метр також коштуватиме 5. Визначення вартості або також ціни так
само мало дозволяє помітити тут будь-яку ріжницю, як і маса продукту, що є пропорційна авансованому
капіталові. Але в першому випадку звільняється капітал; у
другому — заощаджується додатковий капітал, коли б треба було приблизно подвоїти
продукцію; в третьому випадку збільшений продукт можна одержати лише тоді, коли
\parbreak{}  %% абзац продовжується на наступній сторінці

\parcont{}  %% абзац починається на попередній сторінці
\index{iii2}{0151}  %% посилання на сторінку оригінального видання
зростає авансований капітал, хоч не в такому самому відношенні, яке потрібне
було б, коли б більшу кількість продукту довелося виготовляти за колишньої
продуктивної сили. (Стосується до відділу I).

З погляду капіталістичної продукції, у відношенні не до збільшення додаткової
вартости, а до зменшення витрат продукції, — а заощадження витрат
навіть на елемент, що створює додаткову вартість, на працю, робить капіталістові
цю послугу і створює для нього зиск, доки регуляційна ціна продукції
лишається та сама, — вживання сталого капіталу завжди дешевше, ніж
вживання змінного. Справді це має за свою передумову відповідний капіталістичному
способові продукції розвиток кредиту і багатість позикового капіталу. З одного
боку, я вживаю 100\pound{ ф. стерл.} додаткового сталого капіталу, коли 100\pound{ ф.
стерл.} становлять продукт, випродукований 5 робітниками протягом року; з
другого боку — 100\pound{ ф. стерл.} як змінний капітал. Коли норма додаткової вартости
\deq{} 100\%, то вартість випродукована 5 робітниками \deq{} 200\pound{ ф. стерл.};
навпаки, вартість 100\pound{ ф. стерл.} сталого капіталу \deq{} 100\pound{ ф. стерл.}, а як капіталу,
можливо, \deq{} 105\pound{ ф. стерл.}, коли рівень проценту \deq{} 5\%. Ті самі грошові суми,
залежно від того чи авансовано їх для продукції як вартісні величини сталого,
чи змінного капіталу, виражають, розглядувані в їхньому продукті, дуже неоднакові
вартості. Далі, щодо витрат продукції товарів, з погляду капіталіста,
то ріжниця є ще в тому, що з цих 100\pound{ ф. стерл.} сталого капіталу, оскільки
вони вкладені в основний капітал, в вартість товару входить лише спрацьовування,
тоді як ці 100\pound{ ф. стерл.}, витрачені на заробітну плату, мусять бути
цілком репродуковані у вартості товару.

У колоністів і взагалі самостійних дрібних продуцентів, які зовсім не
порядкують капіталом, або можуть ним порядкувати тільки за високі проценти,
частина продукту, відповідна заробітній платі, становить їхній дохід, тоді як
для капіталістів вона є авансування капіталу. Тому перший дивиться на цю
витрату праці як на доконечну передумову трудового продукту, про який насамперед
і йдеться. Щодо надмірної праці, витрачуваної ним понад цю потрібну
працю, то вона в усякому разі реалізується в надмірному продукті; і оскільки
він може продати або сам застосувати його, цей продукт розглядає він як
щось, що йому нічого не коштувало, бо він не коштував зрічевленої праці.
Тільки витрата такої праці має для нього значіння відчуження багатства. Звичайно,
він намагається продавати якомога дорожче; але навіть продаж нижче
вартости і нижче капіталістичної ціни продукції все ще має для нього значіння
зиску, оскільки цей зиск не антиципований заборгованістю, гіпотеками тощо.
Навпаки, для капіталістів витрата так змінного капіталу, як і сталого, однаково
є авансування капіталу. Відносно більше авансування сталого капіталу
зменшує за інших незмінних обставин витрати продукції, а також в дійсності і
вартість товарів. Тому, хоч зиск походить лише з додаткової праці, отже, лише
з вживання змінного капіталу, проте поодинокому капіталістові може здаватися,
що жива праця є найдорожчий елемент витрат продукції, який найбільш слід звести
до мінімуму. Це лише капіталістично перекручена форма тієї істини, що відносно
більше вживання зрічевленої праці порівняно з живою, свідчить про підвищення
продуктивности суспільної праці і збільшення суспільного багатства. От в якому фалшивому
вигляді, яким поставленим шкереберть здається все з погляду конкуренції.

Коли припустити незмінні ціни продукції, то додаткові капітали можуть
вкладатися з незмінною, висхідною або низхідною продуктивністю на кращих
землях, тобто на всіх землях, починаючи з $В$ і вище. На самій $А$ це було б
можливо при нашому припущенні або тільки за незмінної продуктивності, за
якої земля, як і давніш, не дає ренти, абож і тоді, коли продуктивність зростає;
одна частина вкладеного в землю $А$ капіталу давала б тоді ренту,
друга — ні. Але це було б неможливо, коли припустити, що продуктивна сила
\parbreak{}  %% абзац продовжується на наступній сторінці

\parcont{}  %% абзац починається на попередній сторінці
\index{iii2}{0152}  %% посилання на сторінку оригінального видання
на А зменшується, бо в такому разі ціна продукції не лишилася б сталою,
а підвищилась би. Але в усіх цих обставинах, тобто, чи буде, надпродукт, що
його дають додаткові капітали, пропорційний їхній величині, чи буде він
вищий чи нижчий від цієї пропорції, — чи лишається, отже, норма надзиску на
капітал, з його зростом, незмінною, чи підвищується вона чи понижується, —
надпродукт і відповідний йому надзиск з акра зростає, отже, зростає і
евентуальна рента, у збіжжі і в грошах. Зростання просто маси надзиску зглядно
ренти, обчислених на акр, тобто збільшення маси, обчисленої на будь-яку постійну
одиницю, отже, в даному разі, на будь-яку певну кількість землі, акр або гектар,
виражається тут, як ростуча пропорція. Тому висота ренти, обчисленої з акра,
вростає за цих умов просто в наслідок збільшення капіталу, вкладеного в землю.
І до того ж це відбувається за незмінних цін продукції, і навпаки при цьому
не має ваги, чи лишається продуктивність додаткового капіталу незмінною, чи
вменшується вона, чи збільшується. Ці останні обставини модифікують розмір,
в якому зростає висота ренти на акр, але нічого не змінюють у факті цього
зросту. Це є явище, що властиве диференційній ренті II і яке відрізняв її від
диференційної ренти І.~Коли б додаткові капітали вкладалося не один по
одному послідовно в часі на тій самій землі, а послідовно в просторі один
поряд одного на нових додаткових землях відповідної якости, то збільшилася б
загальна маса ренти і також, як це показано давніш, пересічна рента з усієї
оброблюваної площі, але не висота ренти з акра. За незмінних наслідків щодо
маси вартости усієї продукції і додаткового продукту, концентрація капіталу на
земельній площі меншого розміру підвищує розмір ренти з акра там, де за тих
самих обставин, його розпорошення на більшій земельній площі, за інших незмінних
умов, не справляє такого впливу. Але що більше розвивається капіталістичний
спосіб продукції, то більше концентрується капітал на тій самій земельній
площі, то більше, отже, підвищується рента, обчислена на акр. Тому в
двох країнах, де ціни продукції були б тотожні, ріжниці між різними родами
землі тотожні, і де вкладено було б однакову масу капіталу, але в одній країні
переважно у формі послідовних вкладень на обмеженій земельній площі, в другій
переважно у формі координованих вкладень на ширшій площі, — рента з акра, а
тому і ціна землі, була б вища в першій і нижча в другій країні, хоч маса
ренти в обох країнах була б та сама. Отже, ріжницю в висоті ренти можна
було б пояснити тут не ріжницею природної родючости різних земель і не
кількістю вжитої праці, а виключно різним способом вкладення капіталу.

Кажучи тут про надпродукт, ми завжди маємо на увазі відповідну
частину продукту, в якій репрезентовано надзиск. Взагалі-ж, під додатковим
продуктом або надпродуктом ми розуміємо ту частину продукту, яка становить
всю додаткову вартість, а в поодиноких випадках ту частину продукту, в якій
репрезентовано пересічний зиск. Специфічне значення, надаване цьому слову,
коли мова йде про капітал, що дає ренту, є за привід до непорозумінь, як це
зазначено давніш.

\section{Диференційна рента II. Другий випадок: низхідна ціна продукції}
\chaptermark{Диференційна рента II. Другий~випадок}

Ціна продукції може падати, коли додаткові капіталовкладення відбуваються
за незмінної, низхідної або висхідної норми продуктивности.

\paragraph{За незмінної продуктивности додаткових капіталовкладень}

Отже, це означає, що на різних землях відповідно до їхньої відносної
якости, продукт зростає в тій самій мірі, в якій зростає вкладений в них
\parbreak{}  %% абзац продовжується на наступній сторінці

\parcont{}  %% абзац починається на попередній сторінці
\index{ii}{0153}  %% посилання на сторінку оригінального видання
сталого капіталу), вважається лише за рівну вартості тих засобів існування,
що їх виплачено робітникам, і що їх вони повинні зужити, щоб
підтримувати своє існування як робочої сили. Виявити ріжницю між сталим
і змінним капіталом заважає фізіократам саме їхня доктрина. Якщо
праця продукує додаткову вартість (крім репродукції своєї власної ціни),
то вона продукує її в промисловості так само, як і в хліборобстві. А
що згідно з системою фізіократів, праця продукує додаткову вартість
тільки в одній галузі продукції, в хліборобстві, то вона постає не з праці,
а з особливої діяльности (співучасти) природи в цій галузі. І тільки
тому, на їхню думку, хліборобство є продуктивна праця протилежно до
всіх інших відмін праці.

А.~Сміт визначає засоби існування робітників, як обіговий капітал
протилежно до основного:

1) Бо він поточний капітал, у протилежність до основного, сплутує
з формами капіталу, що належать до сфери циркуляції, з капіталом
циркуляції; це сплутування, некритично ставлячись, перейняли від нього
пізніші економісти. Тому він сплутує товаровий капітал із поточною
складовою частиною продуктивного капіталу; і само собою зрозуміло,
що там, де суспільний продукт набирає форми товару, засоби існування
робітників, як і не робітників, матеріяли, як і сами засоби праці, мусять
постачатись із товарового капіталу.

2) Але й уявлення фізіократів прозирають у Сміта, хоч вони й суперечать
езотеричній — справді науковій — частині його власного викладу.

Авансований капітал взагалі перетворюється на продуктивний капітал,
тобто набирає форму елементів продукції, які й собі є продукт попередньої
праці. (Сюди належить і робоча сила). Тільки в цій формі він
може функціонувати в процесі продукції. І коли на місце самої робочої
сили, що на неї перетворилась змінна частина капіталу, підставити засоби
існування робітника, то очевидно, що ці засоби існування як такі,
щодо утворення вартости не відрізняються від інших елементів продуктивного
капіталу, від сировинних матеріялів та засобів існування робочої
худоби. Тим самим Сміт у цитованому вище місці ставить їх, за прикладом
фізіократів, на один рівень. Засоби існування сами собою не можуть
збільшити свою вартість або долучити до неї додаткову вартість. Їхня
вартість, як і вартість інших елементів продуктивного капіталу, може
знову з’явитись лише у вартості продукту. Засоби існування не можуть
прилучити до продукту більше вартости, ніж вони сами мають. Від основного
капіталу, який складається з засобів праці, вони, як і сировинний
матеріял, напівфабрикати і~\abbr{т. ін.}, відрізняються лише тим, що вони
(принаймні для капіталіста, який їх оплачує) цілком зуживаються в продукті,
в утворення якого вони входять, і, значить, вартість їхню треба
покрити цілком, тимчасом як для основного капіталу це відбувається
лише поступінно, частинами. Отже, частина продуктивного капіталу,
авансована на робочу силу (зглядно засоби існування) робітника відрізняється
тепер від інших речових елементів продуктивного капіталу лише речово, а
не своєю ролею в процесі праці та процесі зростання вартости. Вона
\parbreak{}  %% абзац продовжується на наступній сторінці

\parcont{}  %% абзац починається на попередній сторінці
\index{ii}{0154}  %% посилання на сторінку оригінального видання
відрізняється лише остільки, оскільки вона разом з однією частиною об’єктивних
продуктотворців („матеріялів“, за загальним позначенням Сміта)
належить до категорії обігового капіталу протилежно до другої частини
об’єктивних продуктотворців, що належить до категорії основного капіталу.

Та обставина, що частина капіталу, витрачена на заробітну плату, належить
до поточної частини продуктивного капіталу, має властивість поточности
— протилежно до основної частини продуктивного капіталу —
спільно з частиною речових продуктотворців, як от сировинний матеріял
тощо, ця обставина абсолютно не має ніякого чинення до тієї ролі, що
її в процесі зростання вартости відіграє ця змінна частина капіталу протилежно
до сталої. Тут річ лише в тім, яким чином ця частина авансованої
капітальної вартости мусить покритись, відновитись, тобто репродукуватись
з вартости продукту за допомогою циркуляції. Повторювані
акти купівлі робочої сили належать до процесу циркуляції. Але тільки
в процесі продукції вартість, витрачена на робочу силу, з певної, сталої
величини, перетворюється (не для робітника, а для капіталіста) на змінну —
і, значить, взагалі тільки в наслідок цього авансована вартість перетворюється
на капітальну вартість, на капітал, на вартість, що сама з себе
зростає. Але через те, що — як це маємо в Сміта — не на робочу силу
витрачену вартість визначається як поточну складову частину продуктивного
капіталу, а вартість, витрачену на засоби існування робітника, то й
не можна через це зрозуміти ріжницю між змінним і сталим капіталом, а
значить, зрозуміти процес капіталістичної продукції взагалі. Визначення
цієї частини капіталу як капіталу змінного, протилежно до сталого капіталу,
витраченого на речових продуктотворців, цілком поховано тут
визначенням, що, згідно з ним, частина капіталу, витрачена на робочу силу,
щодо своєї ролі в обороті належить до поточної частини продуктивного
капіталу. Таке поховання стає тим певніше, що, замість робочої
сили, як елемент продуктивного капіталу наводиться засоби існування
робітника. Чи авансується вартість робочої сили в грошах, чи безпосередньо
в засобах існування, це не має значення. Хоч, звичайно, останнє
на основі капіталістичної продукції може бути лише винятком\footnote{
До якої міри сам А.~Сміт закриває собі шлях розуміння ролі робочої
сили в процесі зростання вартости, доводить таке речення, де працю робітників,
цілком на зразок фізіократів, прирівнюється до праці худоби: „Не лише його
(фармера) слуги, що працюють, а й його робоча худоба є продуктивні робітники“.
(Not only his (the farmer’s) labouring servants, but his labouring cattle are
productive labourers“. — Book II, chap. V, p. 243).
}.

В наслідок того, що визначення обігового капіталу А.~Сміт таким
чином закріпив як вирішувальне для капітальної вартости, вкладеної в робочу
силу — визначення фізіократів без засновків фізіократів — він щасливо
довів своїх наслідувачів до неспроможности зрозуміти частину капіталу,
витрачену на робочу силу, як змінний капітал. Глибші й правильніші
думки, розкидані в А.~Сміта в інших місцях, не перемогли, перемогла
саме ця помилка. Більше за те, пізніші письменники пішли ще
\parbreak{}  %% абзац продовжується на наступній сторінці

\parcont{}  %% абзац починається на попередній сторінці
\index{ii}{0155}  %% посилання на сторінку оригінального видання
далі; для них вирішальне визначення тієї частини капіталу, яку витрачено
на робочу силу, не лише в тому, що вона є обіговий капітал протилежно
до основного, — вони основне визначення обігового капіталу зводять
на те, що його витрачається на засоби існування для робітників.
Відси природно випливає вчення про робочий фонд, складений з доконечних
засобів існування, як про дану величину, яка, з одного боку,
фізично обмежує пайку робітників у суспільному продукті, а, з другого
боку, цілком мусить бути витрачена на закуп робочої сили.

\section{Теорії про основний та обіговий капітал. Рікардо}

Рікардо наводить ріжницю між основним і обіговим капіталом тільки
для того, щоб показати винятки з правила вартости, а саме ті випадки,
коли норма заробітної плати впливає на ціни. Про це ми говоритимемо
лише в III томі.

Однак основна неясність вже з самого початку виявляється в такому
зіставленні. „Ця ріжниця в ступені довготривалости основного капіталу,
і ця змінність відношень, що в них обидві відміни капіталу можуть бути
комбіновані“\footnote{
„This difference in the degree of durability of fixed capital, and this variety
in the proportions in which the two sorts of capital may be combined.“ — Principles,
p. 25.
}.

Коли ми запитаємо, які саме ці дві відміни капіталу, то виявиться ось
що: „Так само різно можуть комбінуватись відношення між капіталом,
призначеним на утримання праці, і капіталом, витраченим на знаряддя,
машини і будівлі“\footnote{
„The proportions, too, in which the capital that is to support labour, and the
capital that is invested in tools, machinery and buildings, may be various by combined“
— 1. c.
}. Отже, основний капітал \deq{} засобам праці, і обіговий
капітал \deq{} капіталові, витраченому на працю. Капітал, призначений на
утримання праці, — уже це є плаский вислів, запозичений в А.~Сміта.
Обіговий капітал сплутується тут, з одного боку, із змінним капіталом,
тобто частиною продуктивного капіталу, витраченою на працю. Але,
з другого боку, тому що цю протилежність узято не з процесу зростання
вартости — сталий і змінний капітал, — а з процесу циркуляції (стара Смітова
плутанина), то постають визначення, подвійно хибні.

Поперше. Ріжниці в ступені довготривали основного капіталу і
ріжниці в складі капіталу, який складається із сталої та змінної частини,
тут розглядається як рівнозначні. Але остання ріжниця визначає ріжницю
в продукції додаткової вартости, навпаки, перша, оскільки зважається на
процес самозростання вартости, стосується лише до того способу, в який
дану вартість переноситься із засобів продукції на продукт; а оскільки
береться до уваги процес циркуляції, вона стосується лише до періоду
\parbreak{}  %% абзац продовжується на наступній сторінці

\noindent{}Загальна сума грошової ренти становила б якраз половину того, що було
в таблиці II, де додаткові капітали були вкладені за незмінних цін продукції.

Найважливіше є порівняти вищенаведені таблиці з таблицею І.

\enlargethispage{\baselineskip}
\looseness=-1
Ми бачимо, що з пониженням ціни продукції на половину, з 60\shil{ шил.} до
30\shil{ шил.} за квартер, загальна сума грошової ренти залишилась та сама \deq{} 18\pound{ ф.
ст.} і відповідно до цього збіжжева рента подвоїлась, саме зросла з 6 кварт. до
12 кварт. Рента з $В$ відпала; з $C$ грошова рента в ІVd збільшилась на половину,
але на половину зменшилась в ІVс; з $D$ вона лишилась та сама \deq{} 9\pound{ ф.
стерл.} у таблиці ІVс, і піднеслась з 9\pound{ ф. стерл.} до 15\pound{ ф. стерл.} у таблиції ІVd.
Продукція піднеслась з 10 квартерів до 34 в ІVс, і до 30 квартер в в IVd;
зиск підвищився з 2\pound{ ф. стерл.} до 5\sfrac{1}{2} в ІVс і до 4\sfrac{1}{2} в IVd.~Загальна сума
вкладеного капіталу зросла в одному випадку з 10\pound{ ф. стерл.} до 27\sfrac{1}{2}\pound{ ф. стерл.},
в другому — з 10 до 22\sfrac{1}{2}\pound{ ф. стерл.}; отже, обидва рази більше, ніж удвоє. Норма
ренти, рента, обчислена у відношенні до авансованого капіталу, в усіх таблицях
від IV до IVd для кожного роду землі всюди та сама, що вже було дано тим припущенням,
що норма продуктивности обох послідовних витрат капіталу на землях
усіх родів не змінюється. Проти таблиці І вона, проте, понизилась пересічно
щодо всіх родів землі і для кожного окремого роду землі. В таблиці І вона \deq{}
180\% пересічно, в таблиці ІVс вона$ \deq{} \frac{18}{27\sfrac{1}{2}} × 100 \deq{} 65\sfrac{5}{11}\%$ і
IVd \deq{} $\frac{18}{22\sfrac{1}{2}} × 100 \deq{} 80\%$. Пересічна грошова рента з акра підвищилась. Її пересічна
величина давніш в таблиці І була 4\sfrac{1}{2}\pound{ ф. стерл.} з акра для всіх 4 акрів,
а тепер у таблицях IVс і d вона дорівнює 6\pound{ ф. стерл.} з акра для 3 акрів.
Її пересічна величина для землі, що дає ренту, була раніш 6\pound{ ф. стерл.}, а тепер
вона дорівнює 9\pound{ ф. стерл.} з акра. Отже, грошова вартість ренти з акра підвищилась
і репрезентує тепер удвоє більше продукту в збіжжі, ніж давніш, але
12 квартерів збіжжевої ренти тепер становлять менше, ніж половину всього продукту
в 34, зглядно 30
% REMOVED  footnote*{В німецькому тексті стоїть: усього «продукту в 33, зглядно 27 квартерів» Явна помилка,
%як це можна бачити з таблиць ІVс і IVd. \emph{Прим. Ред.}}
квартерів, тимчасом як у таблиці І 6 квартерів становлять
\sfrac{3}{5}  усього продукту в 10 квартерів. Отже, хоч рента, коли розглядати
її як відповідну частину всього продукту, а також коли обчислити її у відношенні
до витраченого капіталу, і знизилась, одначе її грошова вартість,
обчислена на акр, збільшилась, а її вартість в продукті збільшилась ще дужче.
Коли ми візьмемо землю $D$ в таблиці IVd, то ціна продукції тут дорівнює
15\pound{ ф. стерл.}, що з них витрачений капітал \deq{} 12\sfrac{1}{2}\pound{ ф. стерл}. Грошова рента \deq{} 15\pound{ ф. стер}. У таблиці І на тій самій землі $D$ ціна продукції була 3\pound{ ф. стерл.}, витрачений
капітал \deq{} 2\sfrac{1}{2}\pound{ ф. стерл.}, грошова рента \deq{} 9\pound{ ф. стерл.}, отже, остання
утроє більша за ціну продукції й майже у чотири рази більша за витрачений
капітал. У таблиці IVd для $D$ грошова рента в 15\pound{ ф. стерл.} якраз дорівнює ціні
продукції і лише на \sfrac{1}{5}  більша за витрачений капітал. А все ж грошова рента
з акра на \sfrac{2}{3}  більша, 15\pound{ ф. стерл.} замість 9\pound{ ф. стерл}. В таблиці І збіжжева
рента в 3 квартери \deq{} \sfrac{3}{4}  усього продукту, що становить 4 квартери, в таблиці
IVd вона \deq{} 10 квартерам, половині всього продукту (20 квартерів) з акра
землі $D$. Це показує, що грошова і збіжжева рента з акра може зрости, хоч
вона і становить відносно меншу частину всього здобутку і знизилась у відношенні
до авансованого капіталу.

Вартість всього продукту в таблиці І \deq{} 30\pound{ ф. стерл.}; рента \deq{} 18\pound{ ф.
стерл.}, більше від половини цієї вартости. Вартість усього продукту в таблиці
IVd \deq{} 45\pound{ ф. стерл.}, що з них 18\pound{ ф. стерл.}, менш від половини, становлять
ренту.

\parcont{}  %% абзац починається на попередній сторінці
\index{iii1}{0157}  %% посилання на сторінку оригінального видання
дорожчі, ніж в $А$. В цьому випадку на 100 фунтів стерлінгів
змінного капіталу в $А$ припадало б, наприклад, 200 фунтів
стерлінгів сталого капіталу, а в $В$ 400. Тоді при нормі додаткової вартості в 100\% вироблена
додаткова вартість в обох випадках дорівнює 100 фунтам стерлінгів; отже й зиск в обох випадках
дорівнює 100 фунтам стерлінгів. Але в $А$ $\frac{100}{200 c + 100 v} =
\sfrac{1}{3} = 33\sfrac{1}{3}\%$, тимчасом як в $В$ $\frac{100}{400 c + 100 v }= \sfrac{1}{5}=20\%$. Дійсно,
якщо ми в обох випадках візьмемо певні відповідні частини
всього капіталу, то в $В$ з кожних 100 фунтів стерлінгів тільки
20 фунтів стерлінгів, або \sfrac{1}{5}, становить змінний капітал, тимчасом як в $А$ з кожних 100 фунтів
стерлінгів 33\sfrac{1}{3} фунтів стерлінгів, або \sfrac{1}{3} становить змінний капітал. $В$ виробляє на кожні 100
фунтів стерлінгів менше зиску, бо приводить в рух менше
живої праці, ніж $А$. Отже, ріжниця норм зиску зводиться тут
знов таки до ріжниці мас зиску, вироблених на кожні 100 одиниць вкладеного капіталу, бо маси зиску
тотожні з масами додаткової вартості.

Ріжниця цього другого прикладу від попереднього є тільки
така: в другому випадку вирівнення між $А$ і $В$ вимагало б тільки
зміни вартості сталого капіталу, чи то в $А$, чи то в $В$, при незмінній технічній базі; навпаки, в
першому випадку сам технічний склад в обох сферах виробництва є різний і для вирівнення він мусив би
зазнати перетворення.

Отже, різний органічний склад капіталів не залежить від їх
абсолютної величини. Питання завжди тільки в тому, скільки
з кожних 100 одиниць є змінного капіталу і скільки сталого.

Отож, капітали різної в процентному обчисленні величини
або, що зводиться до того самого, капітали однакової величини
створюють при однаковому робочому дні і однаковому ступені
експлуатації праці дуже різні кількості зиску, бо створюють
дуже різні кількості додаткової вартості, і це саме тому, що
залежно від різного органічного складу капіталів у різних сферах виробництва їх змінна частина є
різна, отже, різні й кількості живої праці, яку вони приводять в рух, отже й кількості
привласнюваної ними додаткової праці, — субстанції додаткової
вартості, а тому й зиску. Рівновеликі частини всього капіталу
в різних сферах виробництва містять у собі нерівновеликі джерела додаткової вартості, а єдиним
джерелом додаткової вартості є жива праця. При однаковому ступені експлуатації праці маса праці,
приведеної в рух капіталом, рівним 100, а тому й
маса привласнюваної ним додаткової праці, залежить від величини
його змінної складової частини. Коли б капітал, який в процентах складається з $90 c + 10 v$, при
однаковому ступені експлуатації праці виробляв стільки ж додаткової вартості або зиску, як капітал,
що складається з $10 c + 90 v$, то було б ясно, як день,
що додаткова вартість, а тому й вартість взагалі мусять мати
\parbreak{}  %% абзац продовжується на наступній сторінці

\parcont{}  %% абзац починається на попередній сторінці
\index{iii2}{0158}  %% посилання на сторінку оригінального видання
щоб грошова рента лишилась та сама або підвищилась, мусить бути випродукована
певна додаткова кількість надпродукту, а для цього треба то менш
капіталу, що більша родючість земель; які дають надпродукт. Коли б ріжниця
між $В$ і $C$, $C$ і $D$ була ще більша, то потрібно було б ще менш додаткового
капіталу. Певне відношення залежить: 1) від відношення, в якому понижується
ціна, отже, від ріжниці між землею $В$, що тепер не дає ренти, і $А$, яка давніш
не давала ренти; 2) від відношення ріжниць між кращими, ніж $В$ землями; 3) від
маси нововкладуваного додаткового капіталу і 4) від його розподілу між землями
різної якости.

В дійсності бачимо, що закон не виражає нічого іншого, як те, що вже
було розвинено при дослідженні першого випадку: саме, що, коли ціна продукції
є дана, хоч би яка була її величина, рента може підвищуватися в наслідок
додаткового вкладення капіталу. Бо в наслідок вилучення $А$ тепер дана нова диференційна
рента І, за якої земля $В$ є тепер найгірша земля, і 1\sfrac{1}{2}\pound{ ф. стерл.} за
квартер становлять нову ціну продукції. Це однаково має силу так щодо таблиці
IV, як і щодо таблиці II.~Це той самий закон, але за вихідний
пункт береться землю $В$ замість $А$, і ціну продукції в 1\sfrac{1}{2}\pound{ ф. стерл.} замість.
3\pound{ ф. стерл}.

Справа важлива тут лише от чим: оскільки така кількість додаткового
капіталу потрібна була для того, щоб капітал з $А$ відтягти від землі, і обслугувати
постачання без його участи, то й виявляється, що це може супроводитись
незмінною, висхідною або низхідною рентою з акра, якщо не на всіх землях, то
принаймні на деяких, і пересічно для всіх оброблюваних земель. Ми бачили, що
збіжжева рента і грошова рента не співрозмірні. Тільки за традицією збіжжева
рента все ще продовжує відігравати ролю в економії. З однаковим успіхом можна
було б довести, що, наприклад, фабрикант на свій зиск в 5\pound{ ф. стерл.} може купити
геть більшу кількість своєї власної пряжі, ніж давніше на зиск в 10\pound{ ф. стерл}.
Але в усякому разі це доводить, що панове земельні власники, коли вони одночасно
власники або учасники мануфактур, цукроварень, гуралень то що, з пониженням
грошової ренти все таки можуть дуже значно вигравати, як продуценти
свого власного сирового матеріялу\footnote{
У вищенаведених таблицях від IVа до ІVb треба було б виправити в розрахунку помилку, що
проходить через них. Хоч це не зачіпає теоретичних засад, виведених з даних таблиць, але іноді
приводить до неймовірних числових відношень продукції з акра. Але й це по суті не має значіння. У
всіх мапах, що змальовують рельєф і висоту профілю місцевості, беруть значно більший маштаб для
вертикалей, ніж для горизонталей. А хто все таки почуватиме себе ображеним у своїх аграрних
почуттях, тому дається на волю помножити число акрів на перше-ліпше число. Можна також у таблиці І
замінити 1, 2, 3, 4 квартери з акра 10, 12, 15, 16 бушелями (8 бушелів = 1 квартер), з тим
розрахунком, щоб виведені з цього числа інших таблиць не виходили з меж імовірностп; тоді виявиться,
що наслідок — відношення підвищення ренти до збільшення капіталу — зводиться цілком до того самого.
Це й зроблено в тих таблицях, що їх редактор додає до найближчого розділу. —\emph{ Ф.~Е.}
}.

\subsubsection{За низхідної норми продуктивности додаткових капіталів}

Це не викликає нічого нового остільки, оскільки ціна продукції і тут, як
в щойно розгляненому випадку може лише понизитись, коли в наслідок додаткових
вкладень капіталу на землях кращої якости, ніж $А$, продукт з $А$ зробиться
зайвий і тому капітал буде вилучений з $А$, або земля $А$ буде застосована до вироблення
іншого продукту. Випадок цей ми вже вичерпно дослідили. Ми показали,
що збіжжева і грошова ренти з акра можуть при цьому випадку зрости,
зменшитися або лишитися без зміни.


\index{ii}{0159}  %% посилання на сторінку оригінального видання
Ця некритично запозичена в А.~Сміта плутанина заважає Рікардо
не тільки більше, ніж пізнішим апологетам — останнім плутанина понять
не тільки не заважає, а скорше допомагає — а й більше, ніж самому
А.~Смітові, бо Рікардо, дотримуючись на ділі езотеричного вчення
А.~Сміта проти екзотеричного А.~Сміта, протилежно до нього, послідовніше
і гостріше розвинув вчення про вартість і додаткову вартість.

У фізіократів немає й сліду цієї плутанини. Ріжниця між avances
annuelles і avances primitives стосується лише до різних періодів репродукції
різних складових частин капіталу, спеціяльно хліборобського капіталу,
тимчасом як їхні погляди на продукцію додаткової вартости становлять
незалежну від цих ріжниць частину їхньої теорії, а саме частину,
що її вони виставляють як основу теорії. Утворення додаткової вартости
пояснюється в них не з капіталу, як такого, а визнається як властивість
лише певної продукційної сфери капіталу — хліборобства.

2) Найпосутніше для визначення змінного капіталу — а тому й для
перетворення будь-якої суми вартости на капітал — в тому, що капіталіст
обмінює певну, дану (і в цьому розумінні сталу) величину вартости на
силу, яка творить вартість; певну кількість вартости обмінюється на продукцію
вартости, на процес її самозростання. Чи платить капіталіст робітникові
грішми, чи засобами існування, — це нічого не змінює в цьому
найпосутнішому визначенні. Від цього змінюється тільки спосіб існування
авансованої капіталістом вартости, яка в одному разі існує у формі грошей,
що на них робітник сам собі купує на ринку засоби свого існування,
а в другому разі — у формі засобів існування, що їх робітник
споживає безпосередньо. На ділі розвинена капіталістична продукція
припускає, що робітника оплачується грішми, як вона взагалі має собі
за передумову процес продукції, упосереднюваний процесом циркуляції,
тобто має за передумову грошове господарство. Але творення додаткової
вартости — і, значить, капіталізація авансованої суми вартости — не випливає
ні з грошової, ні з натуральної форми заробітної плати, або капіталу,
витраченого на закуп робочої сили. Воно випливає з обміну вартости
на вартостетворчу силу, — з перетворення сталої величини на змінну.

Більша або менша закріпленість засобів праці залежить від ступеня
їхньої довготривалости, тобто від фізичної властивости. Залежно від
ступеня довготривалости вони, за інших незмінних умов, зношуються
швидше або повільніше, отже, функціонують як основний капітал довший
або коротший час. Але вони функціонують як основний капітал зовсім
не в наслідок самої цієї, фізичної властивости — довготривалости. Сировинний
матеріял на металевих фабриках так само довготривалий, як і машини,
що його обробляють, і довготриваліший, ніж деякі складові частини цих
машин: шкіра, дерево тощо. А проте, металь, що служить як сировинний
матеріял, становить частину обігового капіталу, а засіб праці, що функціонує,
зроблений, може, з того самого металю, становить частину основного
капіталу. Отже, не в наслідок фізичної природи речовини, не
в наслідок більшої або меншої незнищуваности той самий металь одного
разу заводиться під рубрику основного, а другого — під рубрику обігового
\parbreak{}  %% абзац продовжується на наступній сторінці

\parcont{}  %% абзац починається на попередній сторінці
\index{ii}{0160}  %% посилання на сторінку оригінального видання
капіталу. Навпаки, ця ріжниця випливає з тієї ролі, що її він відіграє
в процесі продукції, в одному разі як предмет праці, в другому — як засіб
праці.

Функція засобу праці в процесі продукції потребує, беручи пересічно,
щоб цей засіб протягом довшого або коротшого часу знову й знову
служив у повторюваних процесах праці. Тому вже його функцією диктується
більша або менша довготривалість його матеріялу. Але довготривалість
матеріялу, що з нього його виготовлено, сама по собі не
робить його основним капіталом. Той самий матеріял, як сировинний
матеріял, є обіговий капітал, і в економістів, що сплутують ріжницю між
товаровим капіталом і продуктивним капіталом із ріжницею між обіговим
капіталом і основним капіталом, та сама речовина, та сама машина як
продукт є обіговий капітал, а як засіб праці — основний капітал.

\vtyagnut{}
Але хоч основним капіталом засіб праці робиться не в наслідок
довготривалости матеріялу, що з нього його зроблено, а проте, його
роля як засобу праці, потребує, щоб він був з порівняно довготривалого
матеріялу. Отже, довготривалість його матеріялу є умова його функціонування
як засобу праці, а тому й матеріяльна основа того способу
циркуляції, що робить його основним капіталом. За інших незмінних обставин,
більша або менша нетривалість його матеріялу накладає на нього
в меншій або більшій мірі печать закріплености (Fixität) і, значить, має
посутній зв’язок з його якістю як основного (fixes) капіталу.

А коли частину капіталу, витрачену на робочу силу, розглядається
з погляду обігового капіталу, отже, як протилежність до основного
капіталу; коли в наслідок цього і ріжницю між сталим і змінним капіталом
сплутується з ріжницею між основним і обіговим капіталом, то
цілком природно, подібно до того, як речову реальність засобу праці
вважається за посутню основу його характеру як основного капіталу, висновувати
протилежно до цього, з речової реальности капіталу, витраченого
на робочу силу, його характер як обігового капіталу, а потім
знову визначити обіговий капітал за допомогою речової реальности змінного
капіталу.

Справжня речовина капіталу, витраченого на заробітну плату, є сама
праця, діюща, вартостетворча робоча сила, жива праця, що її капіталіст
обміняв на мертву зречевлену працю і ввів у свій капітал, — і в наслідок
лише цього вартість, що є в його руках, перетворюється на вартість, що
сама з себе зростає. Але цієї здібности до самозростання капіталіст не
продає. Вона завжди становить лише складову частину його продуктивного
капіталу, на зразок його засобів праці, але зовсім не становить
складової частини його товарового капіталу, як от, напр., готові продукти,
що їх він продає. В процесі продукції засоби праці як складова
частина продуктивного капіталу не протистоять робочій силі як основний
капітал, так само матеріял праці та допоміжні матеріяли як обіговий капітал
не збігаються з нею; тому й другому робоча сила протистоїть як
особистий чинник, тимчасом як і те й друге є речові чинники, — це
з погляду процесу праці. Те й друге протистоїть робочій силі, змінному
\parbreak{}  %% абзац продовжується на наступній сторінці

\parcont{}  %% абзац починається на попередній сторінці
\index{iii2}{0161}  %% посилання на сторінку оригінального видання
Воно може бути в тому, що на акр взагалі вживається більше капіталу (більше
добрива, більше механічної праці тощо) або також в тому, що взагалі лише
додатковий капітал дає змогу перевести відзначну, якісною стороною продуктивнішу,
витрату капіталу. В обох випадках при витраті 5\pound{ ф. стерл.} на акр одержується
продукт в 2\sfrac{1}{2}  квартери, тоді як при витраті половини цього капіталу,
2\sfrac{1}{2}\pound{ ф. стерл.}, одержується продукт лише в один квартер. Продукт землі $A$,
залишаючи осторонь минущі ринкові відносини, можна було б і далі продавати
по вищій ціні продукції, замість продавати його по новій пересічній
ціні, лише доти, доки значна площа земель розряду $A$ і далі оброблялася б
з капіталом лише в 2\sfrac{1}{2}\pound{ ф. стерл.} на акр. Але скоро нове відношення
в 5\pound{ ф. стерл.} капіталу на акр, а разом з тим, поліпшене господарство набудуть
загального поширення, реґуляційна ціна продукції мусить понизитися до 2\sfrac{8}{11}\pound{ ф.
стерл}. Ріжниця між обома частинами капіталу зникла б, і тоді дійсно акр землі $A$.
оброблюваний лише з капіталом в 2\sfrac{1}{2}  ф. стерл, оброблявся б ненормально,
невідповідно до нових умов продукції. Це вже було б ріжницею не між здобутком від
різних частин капіталу, вкладених у той самий акр, а між достатньою й недостатньою
загальною витратою капіталу на акр. Звідси видно, \emph{поперше}, що недостатність
капіталу в руках більшости орендарів (це мусить бути більшість, бо коли б це
була меншість, їй довелося б лише продавати нижче від своєї ціни продукції)
впливає цілком так само, як диференціювання самих земель в низхідному порядку.
Гірший спосіб обробітку на гіршій землі збільшує ренту з кращої землі;
він може навіть створити ренту з краще оброблюваної землі такої самої кепської
якости, яка взагалі ренти не дає. Звідси видно, \emph{подруге}, що диференційна рента,
оскільки вона виникає з послідовного капіталовкладення на тій самій земельній
площі, в дійсності перетворюється на пересічну величину, в якій уже не можна
розпізнати і відрізнити впливів різних капіталовкладень, і які тому не породжують
ренти на найгіршій землі, а 1) пересічну ціну всього продукту, скажімо
з одного акра $A$, перетворюють на нову регуляційну ціну і 2) виявляються, як
зміна загальної кількости капіталу на акр, що в нових умовах потрібна для
задовільного обробітку землі, і в якій так окремі послідовні капіталовкладення, як і
їхні відповідні впливи так поєднані, що їх не можна відрізнити. Так само стоїть
справа з поодинокими диференційними рентами кращих земель. Вони визначаються
в кожному випадку ріжницею пересічного продукту відповідного роду землі порівняно
з продуктом найгіршої землі за підвищеної витрати капіталу, що тепер
стала нормальною.

Жодна земля не дає будь-якого продукту без витрати капіталу. Отже,
навіть при звичайній диференційній ренті, при диференційній ренті І; коли говорять,
що 1 акр землі $А$, що реґулює ціну продукції, дає стільки й стільки продукту,
по такій-от ціні, і що кращі землі $B$, $C$, $D$ дають стільки й стільки диференційного
продукту, а тому за даної реґуляційної ціни стільки от грошової ренти, то
тут завжди припускається, що вжито певний капітал, який в даних умовах продукції
вважається за нормальний. Цілком так само, як у промисловості для
кожної галузі підприємств потрібен певний мінімум капіталу для того, щоб
можна було виготовляти товари по їхній ціні продукції.

Якщо цей мінімум змінюється в наслідок сполучених з поліпшеннями послідовних
капіталовкладень на тій самій землі, то це відбувається поступово.
Поки в певну кількість акрів, наприклад, землі $A$ не буде вкладено такого додаткового
капіталу, доти рента з краще оброблюваних акрів землі $A$ породжуватиметься
ціною продукції, яка лишилася незмінною, а рента з усіх кращих родів землі
$B$, $C$, $D$, підвищиться. Проте, скоро новий спосіб продукції так пошириться, що
зробиться нормальним, — ціна продукції понизиться; рента з найкращих дільниць
землі знову понизиться, і та частина землі $A$, в яку капітал вкладено в розмірі,
\parbreak{}  %% абзац продовжується на наступній сторінці


\roztyagnut
Рікардо забуває при цьому будинок, де живе робітник, його меблі,
знаряддя його споживання, напр., ножі, виделки, посуд і~\abbr{т. ін.}, що всі
своєю довготривалістю мають той самий характер, як і засоби праці. Ті
самі речі, ті самі кляси речей виступають тут як засоби споживання,
там — як засоби праці.

Ріжниця, як її висловлює Рікардо, ось у чому: „Відповідно до того,
чи зношується капітал швидко й потребує частої репродукції, чи зуживається
його повільно, його клясифікують як обіговий, або як основний
капітал“\footnote{
„According as capital is rapidly perishable and requires to be frequently
reproduced, or is of slow consumption it is classed under the heads of circulating,
or fixed capital“ (Ricardo, 1. c.).
}.

До цього він робить помітку: „Розподіл непосутній, що в ньому, крім
того, немає змоги точно провести розмежувальну лінію“\footnote{
„А division not essential, and in which the line ot demarcation cannot be
accurately drawn“ (Ricardo,~1.~c.).
}.

Таким чином ми щасливо дійшли знову до фізіократів, що в них
ріжниця між avances annuelles і avances primitives була ріжницею в часі
споживання, а, значить, і в часі репродукції ужитого капіталу. Тільки те,
що в них виражає важливий для суспільної продукції феномен і в Tableau
économique подано також у зв’язку з процесом циркуляції, тут стає
суб’єктивним і, як каже сам Рікардо, зайвим відрізненням.

Якщо частина капіталу, витрачена на працю, відрізняється від частини
капіталу, витраченої на засоби праці, лише періодом своєї репродукції,
а тому й часом своєї циркуляції; якщо одна частина складається з засобів
існування так само, як друга з засобів праці, так що останні відрізняються
від перших лише ступенем швидкости зношування й при цьому
перші й собі мають різні ступені тривалости, — коли це так, то differentia
specifica\footnote*{
Характеристичні, відзначні риси. \emph{Ред.}
} між капіталом, витраченим на робочу силу, і капіталом, витраченим
на засоби продукції, звичайно, стирається.

Це цілком суперечить Рікардовій теорії вартости так само, як і його
теорії зиску, що фактично є теорія додаткової вартости. Він розглядає
ріжницю між основним і обіговим капіталом взагалі лише остільки
оскільки різні пропорції обох, при рівновеликих капіталах, впливають
в різних галузях підприємств на закон вартости, а саме, він розглядає,
як, в наслідок цих обставин, підвищення або зниження заробітної плати
впливає на ціни. Але навіть в обмежених рямцях цього досліду він, сплутуючи
основний та обіговий капітал із сталим та змінним, робить величезні
помилки і справді будує свій дослід на цілком хибній основі. А
саме: 1) оскільки частину капітальної вартости, витрачену на робочу
силу, підводиться під рубрику обігового капіталу, неправильно висновується
визначення самого обігового капіталу і особливо ті обставини,
що підводять частину капіталу, витрачену на працю, під цю рубрику; 2)
сплутується те визначення, що, згідно з ним, частина капіталу, витрачена
\index{ii}{0163}  %% посилання на сторінку оригінального видання
на працю, є змінний капітал, і те, що, згідно з ним, вона є обіговий
капітал протилежно до основного.

Вже з самого початку очевидно, що визначення капіталу, витраченого
на робочу силу, як обігового або поточного, є другорядне визначення,
в якому зникають його differentia specifica в продукційному процесі;
бо, з одного боку, при такому визначенні капітали, витрачений на працю
й витрачений на сировинний матеріял і~\abbr{т. ін.}, вважається за рівнозначні;
рубрика, що ототожнює частину сталого капіталу зі змінним, цілком ігнорує
differentia specifica змінного капіталу протилежно до сталого. З другого
боку, частини капіталу, витрачені на працю і на засоби праці, хоч
і протиставиться одна одній, але зовсім не в тому розумінні, що вони
цілком різним способом входять у продукцію вартости, а лише в тому
розумінні, що обидві вони переносять на продукт свою дану вартість
тільки в різні переміжки часу.

В усіх цих випадках ідеться тільки про те, як дана вартість, що її
витрачається на процес продукції товару — хоч то буде заробітна плата,
ціна сировинного матеріялу або ціна засобів праці, — переноситься на
продукт, а, значить, і як вона циркулює за допомогою продукту і в наслідок
продажу його повертається до свого вихідного пункту, або як
покривається її. Єдина ріжниця тут у цьому „\so{як}“, в особливому способі
перенесення, а, значить, і циркуляції цієї вартости.

Чи виплачується кожного разу заздалегідь визначену контрактом ціну
робочої сили грішми, чи засобами існування — це нічого не змінює в її
характері, а саме в тому, що вона є певна дана ціна. А проте, при виплаті
заробітної плати грішми цілком очевидно, що самі гроші не входять
у процес продукції, як входять засоби продукції, що в них не лише
вартість, а й сама речовина входить у продукційний процес. А коли ж,
навпаки, засоби існування, що їх робітник купує на свою заробітну плату,
безпосередньо підводиться як речову форму обігового капіталу під
одну рубрику з сировинними матеріялами тощо й протиставиться засобам
праці, то це надає справі іншого вигляду. Коли вартість цих речей,
тобто засобів продукції, в процесі праці переноситься на продукт, то
вартість тих других речей, тобто засобів існування, знову з’являється в
робочій силі, що їх спожила, і через функціонування робочої сили її знову
таки переноситься на продукт. І в тому, і в другому разі однаково
йдеться про просту повторну появу в продукті вартостей, авансованих
підчас продукції. (Фізіократи брали це серйозно, а тому не визнавали,
що промислова праця створює додаткову вартість). Напр., Вейленд
пише в цитованому вже місці: „Байдуже, в якій саме формі з’являється
знову капітал\dots{} різні відміни харчу, одягу й житла, потрібні для існування
й добробуту людей, також змінюються. Їх зуживається з плином
часу, і вартість їхня з’являється знову і~\abbr{т. ін.}“. (Elements of Political
Economy, ст. 31, 32). Капітальні вартості, авансовані в формі
засобів продукції й засобів існування для продукції тут однаково знову
з’являються в вартості продукту. Так щасливо досягається перетворення
капіталістичного процесу продукції на цілковиту містерію, а походження
\index{ii}{0164}  %% посилання на сторінку оригінального видання
додаткової вартости, що є в продукті, лишається цілком поза
межами поля зору.

Далі, тут завершується властивий буржуазній політичній економії фетишизм,
що перетворює суспільний, економічний характер, накладуваний
на речі суспільним процесом продукції, на природний, з самої речової
природи цих речей посталий характер. Напр., „засоби праці є основний
капітал“ — схоластичне визначення, що призводить до суперечностей і плутанини.
Цілком так само, як при вивченні процесу праці („Капітал“, книга
І, розділ V) показано, що від тієї ролі, яку в кожному окремому
випадку відіграють речові складові частини у певному процесі праці,
від їхньої функції, цілком залежить те, чи будуть вони виступати як
засіб праці, чи як матеріял праці, або як продукт; цілком так само засоби
праці тільки тоді є основний капітал, коли процес продукції є взагалі
капіталістичний продукційний процес і, значить, засоби продукції
взагалі є капітал, коли вони мають економічну визначеність, суспільний
характер капіталу; і, подруге, вони є основний капітал лише тоді, коли
вони свою вартість переносять на продукт певним способом. Коли цього
немає, вони лишаються засобами праці, не являючи основного капіталу.
Так само допоміжні матеріяли, напр., добриво, коли вони передають
свою вартість тим самим особливим способом, що й більша частина засобів
праці, стають основним капіталом, хоч вони й не є засоби праці.
Тут ідеться не про визначення, що під нього можна підводити речі. Тут
ідеться про певні функції, що їх виражається в певних категоріях.

Коли засобам існування самим по собі при всяких обставинах приписується
властивість бути капіталом, витраченим на заробітну плату, то
ця властивість „підтримувати працю“, to support labour (Рікардо, ст. 25),
стає також характеристичною властивістю цього „обігового“ капіталу.
Отже, виходить, що коли б засоби існування не були „капіталом“, то
вони не підтримували б робочої сили; тимчасом характер капіталу надає
їм саме властивости підтримувати \so{капітал} за допомогою чужої праці.

Якщо засоби існування сами по собі є обіговий капітал — після того
як цей останній перетворився на заробітну плату, — то відси випливає
далі, що величина заробітної плати залежить від відношення між числом
робітників і даною масою обігового капіталу — улюблена засада економістів;
тимчасом як у дійсності маса засобів існування, що її робітник
бере з ринку, і маса засобів існування, що її має капіталіст для власного
споживання, залежить від відношення між додатковою вартістю й ціною
праці.

Рікардо, як і Бартон\footnote{
„Observations on the Circumstances which influences the Condition of the
Labouring Classes of Society. London, 1817“. Відповідне, сюди належне, місце
подано в I кн. „Капіталу“, розд. XXIII, 3, примітка 79.
}, сплутують усюди відношення між змінним і сталим
капіталом із відношенням між обіговим і основним капіталом. Ми побачимо
далі, якої хибности набирає в наслідок цього дослід над нормою зиску.

\vtyagnut{}
Далі, Рікардо ототожнює ріжницю між основним і обіговим капіталом із
ріжницями, що постають в процесі обороту підо впливом інших причин. Він
\parbreak{}  %% абзац продовжується на наступній сторінці

\parcont{}  %% абзац починається на попередній сторінці
\index{ii}{0165}  %% посилання на сторінку оригінального видання
пише: „Далі треба зазначити, що обіговий капітал може циркулювати
або повертатися до свого власника в дуже різні періоди часу. Пшениця,
куплена фармером на засів, є основний капітал, порівняно з пшеницею,
що її купив пекар, щоб перетворити її на хліб. Один лишає пшеницю
в землі й може одержати її знову лише за рік, другий може віддати її
змолоти на борошно й продати як хліб своїм покупцям, так що протягом
одного тижня капітал його знову стає вільний, і може він знову почати
з ним ту саму або будь-яку іншу операцію“\footnote{
„It is also to be observed that the circulating capital may circulate, or be
returned to its employer, in very unequal times. The wheat bought by a farmer to
sow is comparatively a fixed capital to the wheat purchased by a baker to make
into loaves. The one leaves it in the ground, and can obtain no return for a year; the
other can get it ground into flour, sell it as bread to his customers, and have his capital
free, to renew the same, or commence any other employement in a week“. (Ricardo,
1. c., p. 26, 27).
}.

Тут характерно те, що пшениця — хоч вона як зерно служить не як
засіб існування, а як сировинний матеріял, є, поперше, обіговий капітал,
бо вона сама по собі є засіб існування, і, подруге, основний капітал, бо
її зворотний приплив відбувається через рік. Але не повільніший або
швидший зворотний приплив робить даний засіб продукції основним капіталом,
основним капіталом його робить певний спосіб перенесення його
вартости на продукт.

Плутанина понять, що походить від А.~Сміта, призводить до таких
наслідків:

1) Ріжницю між основним і поточним капіталом сплутується з ріжницею
між продуктивним капіталом і товаровим капіталом. Так, напр.,
та сама машина є обіговий капітал, коли вона як товар перебуває на
ринку, і основний капітал, коли її введено в процес продукції. При
цьому лишається абсолютно незрозуміле, чому певний рід капіталу треба
вважати більше за основний або більше за обіговий, ніж інший, рід капіталу.

2) Всякий обіговий капітал ототожнюється з капіталом, що його витрачено
або треба витратити на заробітну плату. Так каже Дж.~Ст.~Мілл та ін.

3) Ріжницю між змінним і сталим капіталом, що її вже Бартон, Рікардо
та ін. сплутують з ріжницею між обіговим і основним капіталом, зводиться,
кінець-кінцем, цілком на цю останню, як, напр., у Рамсея, що в нього
всі засоби продукції, сировинні матеріяли і~\abbr{т. ін.}, так само і засоби праці,
є основний капітал, і лише капітал, витрачений на заробітну плату, є
обіговий капітал. А що це зведення робиться саме в такій формі, то й
лишається незрозуміла справжня ріжниця між сталим і змінним капіталом.

4) У новітніх англійських, особливо шотляндських економістів, що розглядають
усе з невимовно обмеженого погляду банкірського прикажчика,
у Маклеода, Патерсона та ін. ріжниця між основним і обіговим капіталом
перетворюється на ріжницю між money at call і money not at call
(грошові вклади, що їх одержується назад без попереднього повідомлення,
і грошові вклади, що їх одержується назад лише після попереднього
повідомлення).
\pfbreak

\input{ii/_0166.tex}
\parcont{}  %% абзац починається на попередній сторінці
\index{iii1}{0167}  %% посилання на сторінку оригінального видання
виробництва містять у собі зиски від $B, C, D,$ так само як витрати виробництва $B, C, D$ і т. д., в
свою чергу, містять у собі
зиск від $A$. Отже, якщо зробимо підрахунок, то зиску від $А$ не
буде в його власних витратах виробництва, і так само зисків
від $В, C, D$ і т. д. не буде в їх власних витратах виробництва.
Ніхто не залічує свого власного зиску до своїх витрат виробництва. Отже, якщо є, наприклад, $n$ сфер
виробництва, і в кожній з них добувається зиск, рівний p, то витрати виробництва в усіх них разом
$= k — np$. Отже, розглядаючи весь обрахунок
в цілому, ми бачимо, що оскільки зиски однієї сфери виробництва входять у витрати виробництва іншої,
остільки ці зиски
введені вже в обрахунок для загальної ціни остаточного кінцевого
продукту і не можуть вдруге з’явитись у графі зиску. Якщо ж
вони з’являються в цій графі, то тільки тому, що сам даний
товар був остаточним продуктом, отже, ціна його виробництва
не входить у витрати виробництва іншого товару.

Якщо у витрати виробництва якогось товару входить сума $= p$, яка становить зиск виробників засобів
виробництва, і якщо
на ці витрати виробництва набавляється зиск, $= p_1$, то весь зиск
$P = p + p_1$. Загальні витрати виробництва товару, якщо абстрагуватись від усіх частин ціни, що
входять у склад зиску, дорівнюють таким чином його власним витратам виробництва мінус $P$.
Якщо ці витрати виробництва назвемо $k$, то, очевидно, $k + P = k + p + p_1$. При дослідженні додаткової
вартості в книзі I,
розд. VII, 2, стор. 229\footnote*{Стор. 153--154 рос. вид. 1935 р. \emph{Ред. укр. перекладу.}}, ми бачили, що продукт кожного капіталу можна розглядати таким чином, ніби
одна його частина
тільки заміщає капітал, а друга тільки виражає додаткову вартість. Застосовуючи це обчислення до
сукупного продукту
суспільства, ми повинні зробити певні поправки, бо, якщо
розглядати суспільство в цілому, зиск, вміщений, наприклад,
у ціні льону, не може фігурувати двічі — як частина ціни полотна
і разом з тим як частина зиску виробника льону.

Між зиском і додатковою вартістю немає ріжниці, оскільки
додаткова вартість, наприклад, капіталіста $А$ входить у сталий
капітал $В$. Адже для вартості товарів зовсім не має значення, чи
складається вміщена в них праця з оплаченої чи неоплаченої праці.
Це показує тільки, що $В$ оплачує додаткову вартість $А$. В загальному підсумку додаткову вартість $А$ не
можна рахувати двічі.

Але ріжниця полягає ось у чому. Крім того, що ціна продукту,
виробленого, наприклад, капіталом $В$, відхиляється від його вартості, бо реалізована в $В$ додаткова
вартість може бути більша
або менша, ніж зиск, доданий в ціні продуктів $В$, ця сама обставина
знов таки має силу й для тих товарів, що становлять сталу
частину капіталу $В$, а посередньо — як засоби існування робітників — і його змінну частину. Щодо
сталої частини, то вона
\parbreak{}  %% абзац продовжується на наступній сторінці

\parcont{}  %% абзац починається на попередній сторінці
\index{iii2}{0168}  %% посилання на сторінку оригінального видання
вони спричинюють зовсім фалшиве уявлення. Коли для ступенів родючости, що
стосуються один до одного, як $1: 2 : 3 : 4$ тощо, виникають ренти ряду $0 : 1 : 2 : 3$
тощо, то зараз же постає спокуса вивести другий ряд з першого і пояснити
подвоєння, потроєння тощо рент, подвоєнням потроєнням тощо всього здобутку.
Але це було б цілком помилково. Ренти стосуються як $0 : 1 : 2 : 3 : 4$ навіть
тоді, коли ступені родючості стосуються як $n : n \dplus{} 1 : n \dplus{} 2 : n \dplus{} 3 : n \dplus{} 4$;
ренти стосуються одна до однієї, не як ступені родючости, а як ріжниці родючости,
виходячи з землі, що не дає ренти, як нулевої точки.

Таблиці оригіналу потрібно було навести для пояснення тексту. Але щоб
здобути наочну основу для наведених нижче наслідків дослідження, я далі
даю новий ряд таблиць, що в них здобуток показано в бушелях (\sfrac{1}{8}  квартера,
або 36, 35 літра) і шилінґах (= марці).

Перша таблиця (XI) відповідає давнішій таблиці І.~Вона дає здобутки
і ренти для земель п’ятьох якостей $A$ — $E$, при \emph{першій} витраті капіталу в 50\shil{ шил.}, що разом з 10\shil{ шил.} зиску \deq{} 60\shil{ шил.} усієї ціни продукції на акр. Здобутки
збіжжя взято низькі: 10, 12, 14, 16, 18 бушелів з акра. Регуляційна
ціна продукції, яка тут складається, є 6\shil{ шил.} за бушель.

Дальші 13 таблиць відповідають трьом випадкам диференційної ренти II,
розгляненим в цьому і в обох попередніх розділах, при чому припускається, що
\emph{додаткова} витрата капіталу на тій самій землі рівна 50\shil{ шил.} на акр за сталої,
низхідної і висхідної ціни продукції. Кожен з цих випадків знову таки
подається так, як він складається 1)~за сталої, 2)~за низхідної, 3)~за висхідної
продуктивности другої витрати капіталу проти першої. При цьому постають ще
деякі особливо наочні варіянти.

В випадку І: стала ціна продукції, ми маємо:

Варіянт 1: незмінна продуктивність другої витрати капіталу (таблиця XII).

Варіянт 2: низхідна продуктивність. Це може статися лише тоді, коли на землі
А не робиться жодної другої витрати. А саме або:

а) так, що земля $В$ теж не дає ренти (таблиця XIII), або

б) так, що земля $В$ не стає землею, що зовсім не дає ренти (таблиця XIV).

Варіянт 3: висхідна продуктивність (таблиця ХV). І цей випадок виключає
другу витрату капіталу на землю $А$.

В випадку II: низхідна ціна продукції, ми маємо:

Варіянт 1: незмінна продуктивність другої витрати (таблиця ХVI).

Варіянт 2: низхідна продуктивність (таблиця XVII). Обидва варіянти призводять
до того, що земля $А$ вилучається з числа конкурентних земель, земля
$В$ перестає давати ренту і регулює ціну продукції.

Варіянт 3: висхідна продуктивність (таблиця XVIII). Тут земля $А$ лишається
регуляційною.

У випадку III: висхідна ціна продукції, можливі дві видозміни: земля
$А$ може лишитися землею, що не дає ренти і яка реґулює ціни, або ж в конкуренцію
вступає земля гіршої якости, ніж $А$, і починає регулювати ціну, так що $А$
тоді дає ренту.

Перша видозміна: земля $А$ залишається реґуляційною.

Варіянт 1: незмінна продуктивність другої витрати (таблиця XIX). Це припустиме
лише за тієї передумови, що продуктивність першої витрати
зменшується.

Варіянт 2: низхідна продуктивність другої витрати (таблиця XX); це не виключає
того, що продуктивність першої витрати не зміниться.

\parcont{}  %% абзац починається на попередній сторінці
\index{i}{0169}  %% посилання на сторінку оригінального видання
продукту лежать одна поруч однієї, у час, де вони йдуть одна по одній. Алеж цю формулу можуть
супроводити і дуже варварські ідеї, особливо в головах, які практично так само заінтересовані
в процесі зростання вартости, як і в тому, щоб теоретично зрозуміти його хибно. Так, можна собі
уявити, що, приміром, наш прядун за перші 8 годин свого робочого дня продукує або покриває вартість
бавовни, за дальші 1 годину й 36 хвилин — вартість зужиткованих засобів праці, за дальші 1 годину 12
хвилин — вартість заробітної плати й лише преславетну «останню
годину» присвячує фабрикантові, продукції додаткової вартости. Таким чином прядунові накидають
подвійне диво, а саме, що він нібито продукує бавовну, веретена, парову машину, вугілля,
олію і~\abbr{т. д.} в той самий момент, коли він ними пряде, і з одного робочого дня даного ступеня
інтенсивности робить п’ять
таких днів. Саме в нашому випадку продукція сировинного матеріялу й засобів праці потребує \sfrac{24}{6} \deq{} 4
дванадцятигодинних робочих днів, а їхнє перетворення на пряжу — ще одного дванадцятигодинного
робочого дня. Що хижацтво вірить у такі дива й що йому ніколи не важко знайти доктринера-сикофанта,
який їх довів би, про це свідчить один славнозвісний в історії приклад.

\vspace{\bigskipamount}

\subsection{«Остання година» Сеніора}

\vspace{\bigskipamount}
\vspace{\medskipamount}

Одного чудового ранку 1836~\abbr{р.} Нассав В.~Сеніор, відомий своїми економічними знаннями і своїм чудовим
стилем, цей, сказати б, Кльорен серед англійських економістів, був запрошений
з Оксфорду до Менчестеру, щоб учитися тут політичної економії замість навчати її в Оксфорді.
Фабриканти обрали його на борця проти недавно виданого Factory Act\footnote*{
фабричного закону. \emph{Ред.}
} і проти аґітації за
десятигодинний
робочий день, яка тривала ще далі. Із звичною практичною дотепністю вони розпізнали, що пан професор
«wanted а
good deal offinishing»\footnote*{
потребує ще порядного остаточного оброблення. \emph{Ред.}
}. Тим то вони й виписали його до Менчестеру. Пан професор з свого боку
устилізував лекцію, дану йому в Менчестері фабрикантами, у памфлеті «Letters on the Factory Act, as
it affects the cotton manufacture. London. 1837». Тут можна вичитати, між іншим, такі повчальні
місця:

«За теперішнього закону жодна фабрика, що на ній роблять особи, молодші за 18 років, не може
працювати більш як 11\sfrac{1}{2} годин на день, тобто по 12 годин перших п’ять днів тижня й 9 годин
суботами. Дальша аналіза (!) доводить нам, що на такій фабриці ввесь чистий прибуток походить від
останньої години. Фабрикант
витрачає \num{100.000}\pound{ фунтів стерлінґів}: \num{80.000}\pound{ фунтів стерлінґів} на фабричні будівлі й машини, \num{20.000}\pound{ фунтів стерлінґів} — на сировинний матеріял і заробітну плату. Припускаючи, що капітал обертається
один раз на рік і що гуртовий прибуток становить
\parbreak{}  %% абзац продовжується на наступній сторінці

\input{ii/_0170.tex}
\parcont{}  %% абзац починається на попередній сторінці
\index{iii1}{0171}  %% посилання на сторінку оригінального видання
капіталів тієї галузі виробництва, в якій склад капіталу випадково є суспільно-пересічний, вартість
і ціна виробництва були б
рівні. А втім, прикладаючи ці означення до певних випадків,
треба, звичайно, брати до уваги, наскільки не ріжниця в технічному складі, а проста зміна вартості
елементів сталого капіталу
відхиляє відношення між $c$ і $v$ від загального пересічного.

Те, що ми тут розвинули, безперечно, модифікує визначення
витрат виробництва товарів. Первісно ми припускали, що витрати виробництва товару дорівнюють
\emph{вартості} товарів, спожитих на його виробництво. Але ціна виробництва якогось товару для покупця
товару є витрати виробництва цього товару
і може таким чином увійти як витрати виробництва в утворення
ціни іншого товару. Через те що ціна виробництва товару може
відхилятись від його вартості, то й витрати виробництва товару,
в яких включена ця ціна виробництва іншого товару, також
можуть бути вищі або нижчі тієї частини всієї його вартості,
яка утворюється вартістю засобів виробництва, що входять в
товар. Треба пам’ятати про це модифіковане значення витрат
виробництва, отже, пам’ятати, що завжди можлива помилка,
якщо в якійсь окремій сфері виробництва витрати виробництва товару прирівнюються до вартості
спожитих на його виробництво
засобів виробництва. Для цього нашого дослідження немає потреби докладніше розглядати цей пункт. При
цьому завжди лишається правильним положення, що витрати виробництва товарів
завжди менші, ніж їх вартість. Справді, як би не відхилялись
витрати виробництва товару від вартості спожитих на нього
засобів виробництва, для капіталіста ця минула помилка не має
ніякого значення. Витрати виробництва товару є дані, вони є
незалежна від його, капіталіста, виробництва передумова, тимчасом як результат його виробництва є
товар, який містить у
собі додаткову вартість, отже, певний надлишок вартості понад
витрати виробництва товару. А втім, положення, що витрати
виробництва є менші, ніж вартість товару, практично перетворилось тепер у положення, що витрати
виробництва є менші,
ніж ціна виробництва. Для сукупного суспільного капіталу, для
якого ціна виробництва дорівнює вартості, це положення є
тотожне з попереднім: що витрати виробництва менші, ніж вартість. Хоч для окремих сфер виробництва
воно має мінливе значення, проте, основою його завжди лишається той факт, що при розгляді сукупного
суспільного капіталу витрати виробництва
вироблених ним товарів є менші, ніж вартість, або в даному разі
для сукупної маси вироблених товарів, менші, ніж тотожна з цією
вартістю ціна виробництва. Витрати виробництва товару відповідають тільки кількості вміщеної в ньому
оплаченої праці,
вартість же — всій кількості вміщеної в ньому оплаченої і неоплаченої праці; ціна виробництва — сумі
оплаченої праці плюс
певна, для кожної окремої сфери виробництва від неї самої незалежна, кількість неоплаченої праці.


Формула, згідно з якою ціна виробництва товару $= k \dplus{} p$,
дорівнює витратам виробництва плюс зиск, визначилась тепер
ближче таким чином, що $p \deq{} kp'$ (де $p'$ є загальна норма зиску),
і, отже, ціна виробництва $= k \dplus{} kp'$. Якщо $k \deq{} 300$, а $p' \deq{} 15\%$,
то ціна виробництва $k \dplus{} kp' \deq{} 300 \dplus{} 300$. $\frac{15}{100} \deq{} 345$.

Ціна виробництва товарів у кожній окремій сфері виробництва може змінювати свою величину:

1)~при незмінній вартості товарів (тобто при умові, що у виробництво товару після зміни ціни
виробництва входить та сама
кількість мертвої і живої праці, як і до зміни) в наслідок незалежної від даної окремої сфери зміни
загальної норми зиску;

2)~при незмінній загальній нормі зиску в наслідок зміни вартості — чи то в самій даній сфері
виробництва, в результаті
технічних змін, чи в наслідок зміни вартості тих товарів, які
входять у сталий капітал цієї сфери як його складові елементи;

3)~нарешті, в наслідок спільного впливу обох цих обставин.
Не зважаючи на великі зміни, які постійно — як це виявиться
далі — відбуваються у фактичних нормах зиску окремих сфер
виробництва, дійсна зміна в загальній нормі зиску, оскільки
вона викликається не винятковими, надзвичайними економічними
подіями, є дуже пізній результат ряду коливань, які охоплюють
дуже довгі періоди часу, тобто коливань, що потребують багато часу, поки вони сконсолідуються і
вирівняються у зміну
загальної норми зиску. Тому при всіх коротших періодах (цілком незалежно від коливань ринкових цін)
зміну цін виробництва
треба завжди пояснювати prima facie [очевидно] дійсною зміною
вартості товарів, тобто зміною всієї суми робочого часу, потрібного для їх виробництва. Проста зміна
грошового виразу тих самих вартостей тут, само собою зрозуміло, зовсім не береться до уваги\footnote{
\emph{Corbett} [„An Inquiry into the Causes and Modes of the Wealth of Individuals“.
London 1841], стор. [33 і далі] 174.
}.

З другого боку, очевидно, що коли розглядати сукупний
суспільний капітал, то сума вартості вироблених ним товарів
(або, в грошовому виразі, їхня ціна), \deq{} вартості сталого капіталу \dplus{} вартість змінного капіталу \dplus{}
додаткова вартість. Якщо припустити, що ступінь експлуатації праці є незмінний, то норма зиску
при незмінній масі додаткової вартості може змінюватись тут
тільки в тому випадку, коли вартість сталого капіталу змінюється,
або коли вартість змінного капіталу змінюється, абож коли змінюються обидві ці вартості, так що
змінюється $K$, а тому й $\frac{m}{K}$, загальна норма зиску. Отже, в кожному випадку зміна загальної
норми зиску передбачає зміну вартості товарів, що входять як
складові елементи в сталий капітал, або в змінний капітал, або
одночасно в той і в другий.

\parcont{}  %% абзац починається на попередній сторінці
\index{ii}{0173}  %% посилання на сторінку оригінального видання
швидше припливає назад в грошовій формі еквівалент зношеної її частини.
Інша справа з обіговим капіталом. Не тільки капітал треба вкладати на
довший час відповідно до протягу робочого періоду, але треба також
повсякчас авансувати новий капітал на заробітну плату, сировинні та
допоміжні матеріяли. Отже, уповільнений зворотний приплив впливає
неоднаково на основний і обіговий капітал. Хоч буде зворотний приплив
повільніший, хоч швидший, основний капітал і далі діє. Навпаки, обіговий
капітал при уповільненому зворотному припливі стає нездатний до функціонування,
якщо його закріплено в формі непроданого або неготового,
ще непридатного до продажу продукту, і якщо немає наявного додаткового
капіталу, щоб відновити його in natura. — „Тимчасом як селянин
голодує, худоба його росте й гладшає. Було досить дощів, і паша стала
буйна. Індійський селянин умре з голоду біля свого жирного бика. Приписи
забобонів суворі проти поодиноких людей, але вони підтримують
суспільство; зберігання робочої худоби забезпечує поступ хліборобства,
а тим самим і джерела майбутніх засобів існування й майбутнього багатства.
Можливо, це звучить жорстоко й сумно, але це так: в Індії легше
замінити людину, ніж бика“. (Return, East India. Madras and Orissa
Famine. № 4, p. 4). Порівняйте з цим таке речення Манара-Дарма-Сестри,
розділ X, стор. 862: „Жертва життям без нагороди, щоб зберегти
життя жерцеві або корові\dots{} може забезпечити блаженство цих
родів низького походження“.

Звичайно неможливо подати на ринок п’ятилітню тварину, раніше
ніж їй буде п’ять років. Але в певних межах можна, змінюючи догляд
за худобою, підгодувати її протягом коротшого часу до її призначення.
Саме це й зробив Беквел. Раніше англійські вівці, як і французькі ще
1855 року, не були готові на заріз до четвертого або п’ятого року. За
системою Беквела, вівцю можна відгодувати протягом одного року і в
усякому разі вона цілком достигає до двох років. Старанно добираючи вівці,
Беквел, фармер з Дішлей Ґренджа, довів кістяк овець до мінімуму, потрібного
для їхнього існування. Ці його вівці зветься ньюлейстерські.
„Вівчар може тепер подати на ринок три вівці за той самий час, за який
раніше давав одну, і ці його вівці товщі, кругліші й розвиненіші
в тих частинах, що дають найбільше м’яса. Майже ціла вага їхня є чисте
м’ясо.“ (Lavergne, The Rural Economy of England etc. 1855, p. 20)

Методи, що скорочують робочий період, в різних галузях продукції
можна застосовувати дуже неоднаковою мірою, й вони не вирівнюють
ріжниці в часі різних робочих періодів. Щоб залишитись при нашому
прикладі, хай через застосування нових робочих машин абсолютно скорочується
робочий період, потрібний на виготовлення одного паровоза.
Але коли в наслідок удосконалення процесу прядіння кількість щоденно
й щотижнево вироблюваного готового продукту збільшиться ще швидше,
ніж у машинобудівництві, то відносно, порівняно з прядінням, довжина
робочого періоду в машинобудівництві збільшиться.


\index{ii}{0174}  %% посилання на сторінку оригінального видання
\section{Час продукції}

Робочий час завжди є час продукції, тобто час, що протягом його
капітал зв’язано в сфері продукції. Але, навпаки, не увесь час, що протягом
його капітал перебуває в процесі продукції, є в наслідок цього
доконечно також робочий час.

Тут ідеться не про ті перерви у процесі праці, що зумовлені природними
межами самої робочої сили, хоч вже й виявилось, якою поважною
спонукою незвичайного подовження процесу праці та заведення
денної й нічної роботи є та лише обставина, що основний капітал
фабричні будівлі, машини тощо — стоять без ужитку підчас перерв у процесі\footnote*{
Див. „Капітал“, кн. І, розд. VIII, 4 та розд. XIII, 3 b. \Red{Ред.}
}.
Тут ідеться про перерву, незалежну від протягу процесу праці,
зумовлену самою природою продукту та способом його виготовлення,
про перерву, що протягом її предмет праці підпадає більш-менш протяжним
природним процесам, мусить зазнати фізичних, хемічних і фізіологічних
змін, перерву, що протягом її процес праці цілком або почасти
припиняється.

Напр., щойно видавлене вино мусить деякий час шумувати, а потім
протягом деякого часу стояти, щоб набути певного ступеня досконалости.
В багатьох галузях промисловости продукт мусить сушитись, напр., у
ганчарстві, або підпадати певним впливам, що змінюють його хемічні
властивості, як от у білильнях. Озимим хлібам треба аж дев’ять місяців
вистигати. Між посівом і жнивами процес праці майже цілком припиняється.
В лісівництві після посіву та потрібних для нього підготовчих
робіт треба, може, сто років, щоб насіння перетворилося на готовий продукт;
а протягом усього цього часу потрібно прикладати відносно лише
дуже мало праці.

В усіх таких випадках протягом більшої частини часу продукції новододаваної
праці потрібно прикладати лише зрідка. Описані в попередньому
розділі умови, що за них до капіталу, вже зв’язаного в процесі
продукції, треба долучити новий додатковий капітал і новододавану працю,
здійснюються тут лише з більшими або меншими перервами.

Отже, в усіх цих випадках час продукції авансованого капіталу складається
з двох періодів: перший період, коли капітал перебуває в процесі
праці; другий період, коли форма існування капіталу — форма ще неготового
продукту — підпадає впливові природних процесів, не перебуваючи
в процесі праці. Справа ані трохи не змінюється від того, що обидва ці
періоди можуть почасти перехрещуватись та вклинюватись один в один.
Робочий період і період продукції тут не збігаються. Період продукції
є довший, ніж робочий період. Але тільки по закінченні періоду продукції
продукт є готовий, достиглий, отже, тільки тоді його можна перетворити
з форми продуктивного капіталу на форму товарового капіталу.
\index{ii}{0175}  %% посилання на сторінку оригінального видання
Отже, залежно від протягу тієї частини часу продукції, яка не є
робочий час, подовжується й період обороту капіталу. Оскільки час продукції,
надмірний порівняно з робочим часом, не визначено раз назавжди
даними законами природи, як от при достиганні хліба, рості дуба тощо,
період обороту часто можна більш-менш скоротити, штучно скорочуючи
час продукції. Напр., коли заводиться хемічне біління замість біління
на полі, — чинніші сушні апарати у процесах сушіння. Так в чинбарстві,
де за старими методами треба було від 6 до 18 місяців, щоб
чинбарська кислота пройняла шкіри, ці операції, за нової методи, коли
почали застосовувати повітряну помпу, скоротились до 1\sfrac{1}{2}--2 місяців.
(I.~G.~Courcelle-Saneuil. Traité théorique et pratique des Entreprises industrielles
etc. Paris, 1857, 2 éd.).

Найяскравіший приклад штучного скорочення часу продукції, заповненого
виключно природними процесами, подає історія залізоробної
продукції і особливо перероблення чавуна на сталь за останні 100 років,
починаючи з відкритого 1780 року пудлінґування й до сучасного бессемерівського
процесу та інших, заведених з того часу найновіших методів.
Час продукції скорочено надзвичайно, але такою самою мірою збільшились
і вкладення основного капіталу.

Своєрідний приклад того, як відхиляється час продукції від робочого
часу, подає американська фабрикація копил на чоботи. Тут чимала частина
затрат постає тому, що дерево має сохнути до 18 місяців, щоб
готове копило не дубилось і не змінювало своєї форми. Протягом цього
часу дерево не підпадає жодному іншому процесові праці. Період обороту
вкладеного тут капіталу визначається, отже, не лише часом, потрібним
на виготовлення самих копил, а й часом, що протягом його капітал
лежить без діла в дереві, що висихає. Дерево перебуває 18 місяців
у процесі продукції, поки воно, нарешті, ввійде у власне робочий процес.
Разом з тим цей приклад показує, які різні можуть бути періоди обороту
різних частин цілого обігового капіталу в наслідок обставин, що постають
не в сфері циркуляції, а в продукційному процесі.

Особливо виразно виступає ріжниця між часом продукції і робочим
часом у сільському господарстві. В нашому помірному підсонні земля дає
збіжжя раз на рік. Скорочення або продовження періоду продукції (пересічно
дев’ятимісячного для озимого засіву) і собі залежить від зміни
сприятливих і несприятливих років, а тому не можна його точно наперед
визначити й контролювати, як у власне промисловості. Лише бічні продукти,
напр., молоко, сир і~\abbr{т. ін.} завжди можна продукувати й продавати
протягом більш-менш коротких періодів. Щождо робочого часу, то тут
справа така: „В різних місцевостях Німеччини, залежно від кліматичних
та інших чинних умов, число робочих днів для трьох головних періодів
праці буде таке: у весняному періоді, з середини березня або початку
квітня до середини травня 50--60 робочих днів; в літньому періоді, з початку
червня до кінця серпня 65--80; в осінньому періоді, з початку
вересня до кінця жовтня або середини або кінця листопада 55--75 робочих
днів. На зиму припадають лише такі роботи, що їх можна виконати
\index{ii}{0176}  %% посилання на сторінку оригінального видання
в цей час, прим., вивіз добрива, дрів, продуктів на ринок, будівельних
матеріялів тощо“ (F.~Kirchhof, Handbuch der landwirtschaftlichen
Betriebslehre. Dresden, 1852, S. 160).

Отже, що несприятливіше підсоння, то коротший робочий період у
сільському господарстві, а тому коротший і той час, коли вкладається
капітал і працю. Візьмімо, напр., Росію. Там у деяких північних місцевостях
польові роботи можливі тільки протягом 130--150 днів на рік.
Легко зрозуміти, якою втратою було б для Росії, коли б 50 мільйонів
із 65 мільйонів її європейської людности лишалось без роботи протягом
шости або восьми зимових місяців, коли мусять припинитись всі польові
роботи. Крім \num{200.000} селян, що роблять на \num{10.500} фабриках Росії, по
селах там скрізь розвинулась своя хатня промисловість. Там є села, де
всі селяни з роду в рід ткачі, чинбарі, шевці, слюсарі, ножівники і~\abbr{т.
ін.}; особливо це стосується до губерень Московської, Владимирської, Калузької,
Костромської та Петербурзької. До речі буде зауважити, що цю
хатню промисловість дедалі більше примушується служити капіталістичній
продукції, напр., ткачам основу та піткання постачають торговці або
безпосередньо або через факторів. (Скорочено за Reports by Н.~М.~Secretaries
of Embassy and Legation, on the Manufactures, Commerce etc.,
№ 8, p. 86, 87). Відси видно, як розходження періоду продукції і робочого
періоду — а останній становить лише частину періоду продукції, —
утворює природну основу для об’єднання хліборобства з сільськими підсобними
промислами і як, з другого боку, ці останні дають точку опори
капіталістам, які спочатку протискуються сюди, як торговці. Рівнобіжно
з тим, як пізніше капіталістична продукція відокремлює мануфактуру від
хліборобства, сільський робітник підпадає під дедалі більшу залежність
від суто-випадкового, підсобного промисла, і його стан через це погіршується.
Для капіталу, як ми побачимо далі, всі ріжниці в обороті вирівнюються.
Для робітника вони не вирівнюються.

Тимчасом як у більшості галузей власне промисловости, як от у гірництві,
транспорті і~\abbr{т. ін.}, процес продукції відбувається рівномірно, рік-у-рік
рівномірно витрачається робочий час, і, лишаючи осторонь, як ненормальні
перерви, коливання цін, розлади в перебігу справ і~\abbr{т. ін.}, рівномірно
розподіляються витрати на капітал, що входять у повсякденний процес
циркуляції, тимчасом як за інших незмінних ринкових відносин
зворотний приплив обігового капіталу або відновлення його так само
розподіляється на рівномірні переміжки часу — в тих галузях приміщення
капіталу, де робочий час становить лише частину часу продукції, обіговий
капітал витрачається в різні періоди року дуже нерівномірно, а назад
він припливає разом, в момент, визначуваний природними умовами. Отже,
тут, при однаковому маштабі підприємства, тобто при однаковій величині
авансованого обігового капіталу, цей останній треба авансувати одним
заходом більшими масами й на довший час, ніж у підприємствах з безперервними
робочими періодами. Життьова тривалість основного капіталу
тут також більше відрізняється від того часу, що протягом його він
справді функціонує продуктивно. Звичайно, коли є ріжниця між робочим
\parbreak{}  %% абзац продовжується на наступній сторінці

\parcont{}  %% абзац починається на попередній сторінці
\index{iii2}{0177}  %% посилання на сторінку оригінального видання
дукції з землі В це становить округло 1\sfrac{1}{2} квартера. Надзиск з В визначається,
отже, у відповідній частині продукту з В, в цих 1\sfrac{1}{2} квартерах, які
становлять ренту, визначену в збіжжі, і які продаються по загальній ціні продукції
за 4\sfrac{1}{2} ф. стерл. Але, навпаки, надмірний продукт з акра землі В, надмірний
проти продукту з акра землі А, не можна просто вважати за надзиск,
а тому й за надпродукт. Згідно з припущенням акр землі В продукує 3\sfrac{1}{2} квартери,
акр землі А лише 1 квартер. Надмірний продукт з землі В є, отже,
2\sfrac{1}{2} квартери, але надпродукт є лише 1\sfrac{1}{2} квартери; бо в землю
В вкладено удвоє більший капітал, ніж у землю А, і тому вся ціна продукції тут удвоє
більша. Коли б у землю А також було вкладено 5 ф стерл. і норма продуктивности
лишилася б без зміни, то продукт становив би 2 квартери замість одного,
і таким чином виявилося б, що дійсний надпродукт можна знайти порівнянням
не 3\sfrac{1}{2} і 1, а 3\sfrac{1}{2} і 2; що, отже, він дорівнює не 2\sfrac{1}{2},
а лише 1\sfrac{1}{2} квартерам.
Але далі, якби в землю В було вкладено третю порцію капіталу в 2\sfrac{1}{2} ф. стерл.,
що дала б лише 1 квартер, так що він коштував би 3 ф. стерл., як на землі А, то
його продажна ціна в 3 ф. ст. покрила б тільки ціну продукції, дала б лише
пересічний зиск, але не дала б надзиску, а отже і нічого, що могло б перетворитися
на ренту. Продукт з акра будь-якого роду землі, порівняно з продуктом
з акра землі А, не показує ані того, чи є він продукт однакової або більшої
витрати капіталу, ані того, чи надмірний продукт покриває тільки ціну продукції,
чи завдячує він своїм виникненням вищій продуктивності додаткового капіталу.

\emph{Друге}: З щойно викладеного випливає, що при низхідній нормі продуктивности
додаткових витрат капіталу, за межу котрих, — оскільки мова йде
про створення нового надзиску, — є така витрата капіталу, що покриває лише
ціну продукції, тобто що продукує квартер так само дорого, як рівна витрата
капіталу на землі А, отже, згідно з припущенням, за 3 ф. стерл., — випливає,
що за межу, на якій загальна витрата капіталу на акр землі В перестала б
давати ренту, є та, коли індивідуальна пересічна ціна продукції продукту з
акра землі В підвищилася б до рівня ціни продукції з акра землі А.

Коли на В робляться лише такі додаткові витрати капіталу, що оплачують
ціну продукції і, отже, не створюють надзиску, а тому й нової ренти, то хоч
це й підвищує індивідуальну пересічну ціну продукції квартера, проте, не зачіпає
надзиску, що створився від попередніх витрат капіталу, евентуально ренти. Бо
пересічна ціна продукції завжди лишається нижча від ціни продукції на А, а коли
надмір ціни з квартера і зменшується, то кількість квартерів збільшується
у тому самому відношенні, так що загальний надмір ціни лишається без зміни.

В наведеному випадку дві перші витрати капіталу в 5 ф. стерл. на землі В
продукують 3\sfrac{1}{2} квартери, отже, згідно з припущенням, 1\sfrac{1}{2}
квартери ренти = 4\sfrac{1}{2} ф. стерл. Коли сюди прилучиться третя витрата
капіталу в 2\sfrac{1}{2} ф. стерл.,
що продукує лише 1 додатковий квартер, то вся ціна продукції (включаючи 20\%
зиску) 4\sfrac{1}{2} квартерів = 9 ф. стерл.; отже, пересічна ціна за
квартер = 2 ф. стерл. Отже, пересічна ціна продукції за квартер на землі В
піднеслась з 1\sfrac{5}{7} ф. стерл.
до 2 ф. стерл., надзиск з квартера порівняно з регуляційною ціною А упав
з 1\sfrac{2}{7} ф. стерл. до 1 ф. стерл. Але 1×4\sfrac{1}{2} = 4\sfrac{1}{2} ф.
стерл., цілком так само,
як раніш $1\sfrac{2}{7} × 3\sfrac{1}{2} = 4\sfrac{1}{2}$ ф. стерл.

Коли ми припустимо, що на В було б зроблено ще четверту і п’яту додаткові
витрати капіталу по 2\sfrac{1}{2} ф. стерл., які продукують квартер лише по його
загальній ціні продукції, то весь продукт з акра становив би тепер 6\sfrac{1}{2} квартерів,
а ціна його продукції була б 15 ф. стерл. Пересічна ціна продукції
квартера для В знову підвищилась би з 2
\footnote*{В німецькому тексті тут стоїть «з 1 ф. стерл.» Очевидна помилка,
бо у вищенаведеному прикладі пересічна ціна продукції квартера для В
становила не 1 ф. стерл., а 2 ф. стерл. \emph{Прим. Ред.}}
до 2\sfrac{4}{13} ф. стерл., а надзиск з квартера
\index{iii2}{0178}  %% посилання на сторінку оригінального видання
порівняно з реґуляційною ціною продукції на землі А знову зменшився
б з 1 ф. стерл, до \sfrac{9}{13} ф. стерл. Але ці \sfrac{9}{13} ф. стерл. тут слід
помножити на 6\sfrac{1}{2} квартерів замість колишніх 4\sfrac{1}{2}
А $\sfrac{9}{13}×6\sfrac{1}{2} = 1× 4\sfrac{1}{2} = 4\sfrac{1}{2}$ ф. стерл.

Звідси насамперед випливає, що за цих обставин не потрібно жодного підвищення
реґуляційної ціни продукції для того, щоб уможливити додаткові витрати
капіталу на рентодайних землях, навіть в такому розмірі, що додатковий
капітал зовсім перестає давати надзиск і дає ще лише пересічний зиск. З
цього випливає далі, що тут сума надзиску на акр лишається без зміни,
хоч би як дуже зменшувався надзиск з квартера; це зменшення завжди урівноважується
відповідним збільшенням квартерів, продукованих на акрі. Для того,
щоб пересічна ціна продукції піднеслась до рівня загальної ціни продукції (отже,
тут досягла б 3 ф. стерл. на землі В), мусять бути зроблені такі додаткові витрати
капіталу, продукт яких мав би вищу ціну продукції, ніж реґуляційна ціна
в 3 ф. стерл. Але ми побачимо, що тільки цього ще не досить, щоб підвищити
пересічну ціну продукції квартера на землі В до рівня загальної ціни продукції
в 3 ф. стерл.

Припустімо, що на землі В було випродуковано:

1) 3\sfrac{1}{2} квартери, що їхня ціна продукції, як і давніш, 6 ф. стерл.; отже,
дві витрати капіталу по  2\sfrac{1}{2} ф. стерл. кожна, при чому обидві дають надзиски,
але низхідної висоти.

2) 1 квартер за 3 ф. стерл.; витрата капіталу, при якій індивідуальна
ціна продукції дорівнювала б реґуляційній ціні продукції.

3) 1 квартер за 4 ф. стерл.; витрата капіталу, при якій індивідуальна
ціна продукції на 25\% вища за реґуляційну ціну.

Ми мали б тоді 5\sfrac{1}{2} квартерів з акра за 13 ф. стерл. при витраті капіталу
в 10 ф. стерл.; первісна витрата капіталу зросла б учетверо, але продукт
першої витрати капіталу не збільшився б і втроє.

5\sfrac{1}{2} квартерів за 13 ф. дають пересічну ціну продукції в 2\sfrac{4}{11} ф. стерл.
за квартер, отже, за реґуляційної ціни продукції в 3 ф. стерл. надмір
в \sfrac{7}{11} ф. стерл. з квартера, який може перетворитися на ренту.
5\sfrac{1}{2} квартерів, продані по реґуляційній ціні в 3 ф. стерл. дають
16\sfrac{1}{2} ф. стерл. За вирахуванням
ціни продукції в 13 ф. стерл. залишається 3\sfrac{1}{2} ф. стерл. надзиску, або
ренти, так що ці 3\sfrac{1}{2} ф. стерл., рахуючи по теперішній пересічній ціні
продукції квартера з землі В, тобто по 2\sfrac{4}{11} ф. стерл. за квартер,
репрезентують 1\sfrac{25}{52}\footnote*{
В німецькому тексті тут стоїть: «1\sfrac{5}{72}». Очевидна помилка. \emph{Прим. Ред.}
} квартера. Грошова рента понизилася б на 1 ф. стерл., збіжжева
рента приблизно на \sfrac{1}{2} квартера, але, не зважаючи на те, що четверта додаткова
витрата капіталу на В не тільки не створює надзиску, але дає менше, ніж
пересічний зиск, — як і давніш, існує надзиск і рента. Коли ми припустимо, що,
крім витрати капіталу 3), і витрата 2) продукує по ціні, що перебільшує реґуляційну
ціну продукції, то вся продукція становитиме: 3\sfrac{1}{2} квартери за
6 ф. ст. + 2 квартери за 8 ф. ст., разом 5\sfrac{1}{2} квартерів за 14 ф. ст.
ціни продукції. Пересічна ціна продукції квартера була б 2\sfrac{6}{11}  ф. ст.,
що давало б надмір в \sfrac{5}{11} ф. ст. Ці 5\sfrac{1}{2}  квартери, продані по
3 ф. ст., дають 16\sfrac{1}{2} ф. ст.; за вирахуванням
14 ф. ст. ціни продукції, лишається 2\sfrac{1}{2} ф. ст. на ренту. За теперішньої
пересічної ціни продукції на землі В це становило б \sfrac{55}{56} квартера.
Отже, рента все ще одержується, хоч і в меншому розмірі, ніж давніш.

В усякому разі це показує, що на кращих земельних дільницях при додаткових
витратах капіталу, що їхній продукт коштує дорожче, ніж реґуляційна
ціна продукції, рента, принаймні в межах допустимих практикою, мусить не
зникнути, а лише зменшитися, і саме відповідно до того, з одного боку, яку
\parbreak{}  %% абзац продовжується на наступній сторінці


\index{ii}{0179}  %% посилання на сторінку оригінального видання
Те саме і в скотарстві. Частина стада (запасу худоби) лишається в продукційному процесі, тимчасом як
другу частину його щорічно продається, як продукт. Лише одна частина капіталу обертається тут
щорічно, цілком так само, як це є з основним капіталом, машинами, робочою худобою тощо. Хоч цей
капітал є капітал закріплений у процесі продукції на довший час, і таким чином він уповільнює оборот
цілого капіталу, все ж він не є основний капітал у категоричному значенні.

Те, що зветься тут запасом — певна кількість живого дерева або худоби — відносно перебуває в процесі
продукції (одночасно як засоби праці й матеріял праці); згідно з природними умовами його
репродукції, при правильному господарстві, чимала частина його мусить завжди перебувати в цій формі.

Подібно впливає на оборот друга відміна запасу, що становить лише потенціяльний продуктивний
капітал, але в наслідок природи господарства мусить нагромаджуватись більшими або меншими масами, а
значить, і авансуватись для продукції на довший час, хоча вона лише помалу ввіходить в активний
процес продукції. Сюди належить, напр., добриво, поки його не вивезено на поле, також зерно, сіно
тощо та інші запаси засобів існування, що увіходять у продукцію худоби. „Чимала частина капіталу
продукції (Betriebskapital) є в господарських запасах. Але останні можуть втратити більше або менше
своєї вартости, скоро не додержано належних запобіжних заходів, потрібних, щоб їх зберегти в доброму
стані; можливо навіть, що в наслідок недостатнього догляду частина запасів продукту й зовсім марно
пропаде для господарства. Ось чому тут потрібен особливо пильний догляд за інбарами, клунями,
пашенними коморами й льохами; так само треба добре замикати приміщення, де зберігаються запаси, а
крім того тримати їх чисто, провітрювати і~\abbr{т. ін.}; збіжжя та овочі, що зберігаються в інбарах, треба
час від часу пересипати, картоплю й буряк захищати від холоду, а також від води й огню“ (Kirchhof,
р. 292). „Обчисляючи власні потреби, особливо те, що потрібно для утримання худоби — при цьому
розподіл цей роблять залежно від урожаю та визначеної мети, — треба мати на увазі не лише
задоволення даної потреби, а також подбати про те, щоб лишався й відповідний запас про непередбачені
випадки. Скоро при цьому виявляється, що цих потреб не можна цілком покрити виробами власного
господарства, то треба насамперед подумати про те, чи не можна цю недостачу покрити іншими
продуктами (сурогатами) або принаймні придбати їх дешево, щоб покрити цю недостачу. Коли, напр.,
виявиться, що не вистачає сіна, то його можна замінити корінниками, додавши соломи. Взагалі при цьому
треба завжди зважувати матеріяльну цінність і ринкову ціну різних продуктів і відповідно до цього
визначати споживання; коли, напр., овес відносно дорогий, а ціни на горох і жито порівняно низькі,
то можна вигідно замінити в кормі коням частину овса на горох або жито, а заощаджений таким чином
овес продати“. (Там же, стор. 300).

Вище, розглядаючи питання про утворення запасів, ми вже зауважили,
\parbreak{}  %% абзац продовжується на наступній сторінці

\parcont{}  %% абзац починається на попередній сторінці
\index{ii}{0180}  %% посилання на сторінку оригінального видання
що потрібна певна більша або менша кількість потенціяльного продуктивного капіталу, тобто певна
кількість призначених для продукції засобів продукції, що їх треба мати у запасі в більших або
менших масах, щоб могли вони помалу входити в процес продукції. При цьому ми зазначили, що в даному
підприємстві або в капіталістичному підприємстві даних розмірів величина цього продукційного запасу
залежить від більшої або меншої важкості його поновлення, від відносної близькости ринків набування
його, розвитку засобів транспорту й комунікації і~\abbr{т. ін.} Всі ці обставини впливають на мінімум
капіталу, що мусить бути наявний в формі продуктивного запасу, отже, і на протяг часу, що на нього
треба авансувати капітал, і на розмір капіталу, що його треба авансувати одним заходом. Цей розмір,
що впливає, отже, на оборот, зумовлено більшим або меншим протягом часу, що на нього закріплюється
обіговий капітал в формі продуктивного запасу, як лише потенціяльний продуктивний капітал. З другого
боку, оскільки це закріплення залежить від більшої або меншої можливости швидкого заміщення, від
ринкових умов тощо, воно саме і собі зумовлюється часом обігу, обставинами, що належать до сфери
циркуляції. „Далі всі предмети реманенту або прилади, як ручний струмент, решета, кошівниці,
вірьовки, дьоготь, гвіздки тощо, на випадок негайного заміщення мають бути в запасі то більшому, що
менша змога швидко дістати їх поблизу. Нарешті, щорічно протягом зими ввесь реманент треба пильно
переглянути й подбати про те, щоб його відповідно поповнити й полагодити. Оскільки великі мають бути
взагалі запаси щодо реманенту, це залежить, головним чином, від місцевих умов. Там, де близько немає
ремісників і крамниць, треба мати більший запас, ніж там, де вони є на місці або близько. А коли при
інших однакових умовах потрібні запаси закуповується разом чималими масами, то звичайно мають ту
перевагу, що купують дешевше, особливо, коли для цього обирають влучний час; правда, при цьому з
обігового капіталу підприємства воднораз береться чималу суму, а без неї не завжди може обійтись
господарство“ (Kirchhof, p. 301).

Ріжниця між часом продукції і робочим часом може, як ми бачили, поставати в дуже різних випадках.
Обіговий капітал може бути в періоді продукції раніш, ніж він увійде в процес праці у власному
значенні слова (виробництво копил); або він перебуває в періоді продукції після того як проробив
власне процес праці (вино, засівне зерно), або час продукції деколи переривається робочим часом
(хліборобство, лісівництво); чимала частина обігоздатного продукту лишається втіленою в
продукційному процесі, тимчасом як куди менша частина ввіходить у річну циркуляцію (лісівництво й
скотарство); довший або коротший час, що на нього треба витратити обіговий капітал в формі
потенціяльного продуктивного капіталу, отже, більша або менша маса капіталу, що його треба витратити
воднораз — це зумовлюється почасти родом продукційного процесу (хліборобство), а почасти залежить
від близькости ринків і~\abbr{т. ін.}, коротко кажучи, від обставин, які належать до сфери циркуляції.

Далі (книга III) ми бачимо, до яких безглуздих теорій призвела Мак-Куллоха,
\index{ii}{0181}  %% посилання на сторінку оригінального видання
Джемса Мілла та інших спроба ототожнити час продукції, що відхиляється від робочого часу, з
цим останнім, — спроба, що сама й собі походить від неправильного застосування теорії вартости.

\pfbreak

Цикл обороту, що ми розглянули вище, визначається тривалістю основного капіталу, авансованого на
процес продукції. А що цей цикл охоплює більший або менший ряд років, то охоплює він і ряд річних,
тобто повторюваних протягом кожного року оборотів основного капіталу.

В хліборобстві такий цикл обороту зумовлюється системою сівозміни. Протяг оренди в усякому разі не
повинен бути менший, ніж час обороту при заведеній сівозміні, тому при трипільному господарстві його
завжди беруть у 3, 6, 9 років. Коли заведено трипільне господарство з чистим паром, то кожне поле
протягом шости років обробляється тільки чотири рази, при цьому в ті роки, як його обробляється, на
ньому сіють озимину або ярину і, коли того потребує або дозволяє властивість ґрунту, послідовно —
пшеницю і жито, ячмінь і овес. Кожна відміна зернівців дає на тому самому ґрунті більші або менші
врожаї, ніж інші відміни, кожна має свою вартість і продається за свою ціну. Тому, коли прибуток з
поля змінюється кожного року обробки, то й за першу половину обороту (за перші три роки) він буде не
той, що за другу. Навіть пересічний прибуток за першу й другу половину часу обороту буде
неоднаковий, бо родючість залежить не лише від якости ґрунту, а також і від погоди, так само, як і
ціни залежать від багатьох умов. Коли ми обчислимо прибуток з поля, беручи на увагу середню
родючість і пересічні ціни за ввесь шестилітній період часу обороту, то знайдемо загальну цифру
щорічного прибутку і для першого й для другого періоду часу обороту. Цього однак не буде, коли ми
обчислимо прибуток лише за половину часу обороту, тобто тільки за три роки, бо тоді загальні цифри
прибутку не будуть однакові. Відси випливає, що при трипільній системі протяг оренди треба визначити
принаймні в шість років. Але куди бажаніше завжди орендареві й землевласникові, щоб час оренди
становив кількаразовий час оренди (sic!), отже, при трипільній системі замість 6 років — 12, 18 і
більш років, а при семипільній замість 7--14, 28 років“. (Kirchhof, S. 117, 118).

(В рукопису тут стоїть: „Англійське сівозмінне господарство. Тут зробити примітку“).

\section{Час обігу}

Всі досі розглянуті обставини, що зумовлюють ріжниці в періодах обігу\footnote{
Тут, очевидно, термін „період обігу“ („Umlaufsperiode“) вжито в широкому розумінні слова — як
період, що охоплює час продукції та час власне обігу, тобто в розумінні періода обороту капіталу.
\emph{Ред.}
} різних капіталів, вкладених у
різні галузі підприємств, а тому
\parbreak{}  %% абзац продовжується на наступній сторінці

\parcont{}  %% абзац починається на попередній сторінці
\index{ii}{0182}  %% посилання на сторінку оригінального видання
й ріжниці в часі, що на нього треба авансувати капітал, постають в самому процесі
продукції, як ріжниця між основним і поточним капіталом, ріжниця
робочих періодів тощо. Однак час обороту капіталу дорівнює сумі часу
його продукції і часу його обігу або циркуляції. Відси зрозуміло само
собою, що різний протяг часу обігу робить різним час обороту, а значить,
і протяг періоду обороту. Найнаочніше це буде або тоді, коли
порівняти два різні капіталовкладення, при чому різні тільки часи обігу,
а всі інші обставини, що модифікують оборот, однакові, або коли взяти
певний капітал певного складу щодо основного й поточного капіталу при
певному робочому періоді і~\abbr{т. ін.}, і гіпотетично зміняти тільки час
його обігу.

Один відділ часу обігу — і порівняно найважливіший — складається з
часу продажу, з того періоду, коли капітал перебуває в стані товарового
капіталу. Відповідно до відносної величини цього періоду подовжується
або скорочується час обігу, а тому й період обороту взагалі. В наслідок
витрат на зберігання тощо може бути потрібна й додаткова витрата капіталу.
Само собою зрозуміло, що час, потрібний для продажу готових
товарів, може бути дуже різний у різних капіталістів у тій самій галузі
підприємств; отже, цей час може бути різний не лише для мас капіталів,
вкладених у різні галузі продукції, а й для різних самостійних капіталів,
що в дійсності є лише усамостійнені частини сукупного капіталу, вкладеного
в ту саму продукційну сферу. За інших незмінних обставин період
продажу для того самого індивідуального капіталу буде змінюватись
разом із загальними коливаннями ринкових відносин, або разом із
коливаннями цих відносин в поодинокій галузі продукції. На цьому ми
не будемо тут більше зупинятись. Ми лише констатуємо простий факт:
всі обставини, що взагалі зумовлюють ріжницю в періодах обороту капіталів,
вкладених у різні галузі підприємств, мають наслідком, якщо ці
обставини впливають індивідуально (коли, напр., один капіталіст має змогу
продавати швидше, ніж його конкурент, коли один більш, ніж інший,
вживає методів, що скорочують робочі періоди тощо), так само ріжницю
в обороті різних індивідуальних капіталів, що перебувають в тій самій
галузі підприємств.

Одна з причин, що завжди зумовлюють ріжницю в часі продажу, а
тому і в часі обороту взагалі, є віддаленість ринку, де продається товар,
від місця, де його виготовлюється.\footnote*{
В нім. тексті тут стоїть: „von ihrem Verkaufsplatz“, тобто: „від місця його продажу“.
Очевидна помилка. \Red{Ред.}
} Протягом цілого часу своєї подорожі
до ринку, капітал лишається зв’язаний в стані товарового капіталу;
коли товар продукують на замовлення, то — до часу здачі; коли не на замовлення,
то до часу подорожі його на ринок долучається ще той час,
що протягом його товар перебуває на ринку, чекаючи на продаж. Поліпшення
засобів зв’язку й транспорту скорочує мандрування товарів абсолютно,
але не знищує зумовлюваної цим мандруванням відносної ріжниці
в часі різних товарових капіталів або й різних частин того самого товарового
\index{ii}{0183}  %% посилання на сторінку оригінального видання
капіталу, що мандрують до різних ринків. Поліпшені вітрильні
судна та пароплави, напр., що скорочують шлях, однаково скорочують
його так до близьких, як і до далеких портів. Відносна ріжниця лишається,
хоч часто зменшена. Але відносні ріжниці можуть у наслідок розвитку
засобів транспорту й зв’язку змінюватись таким способом, який не відповідає
природним віддаленням. Напр., залізниця, що веде від місця продукції до
головного внутрішнього залюдненого центру, може зробити ближчий
унутрішній пункт, що до нього немає залізниці, абсолютно або відносно
більш віддаленим порівняно з пунктом, куди віддаленішим географічно; так
само, в наслідок цієї самої обставини, може змінюватись навіть відносна
віддаленість місць продукції від більших ринків збуту, і цим пояснюється
занепад старих і постання нових центрів продукції рівнобіжно з зміною
засобів транспорту й зв’язку. (До цього ще долучається відносно більша
дешевина транспорту на великі дистанції порівняно з невеликими). Разом
з розвитком засобів транспорту не тільки збільшується швидкість
переміщення, і в наслідок цього просторова віддаль зменшується в часі.
Розвивається не лише маса засобів комунікації, так що, напр., одночасно
багато суден виходять до того самого порту, кілька поїздів одночасно
йдуть різними залізницями між тими самими двома пунктами, але, напр.,
у різні послідовні дні тижня товарові судна виходять з Ліверпулу на
Нью-Йорк, або товарові поїзди в різні години доби йдуть з Менчестера
до Лондону. Правда, абсолютна швидкість — отже, і відповідна частина
часу обігу — в наслідок цієї останньої обставини, за даної провізної спроможности
засобів транспорту, не змінюється. Але все ж послідовні партії товарів
можна відправляти через коротші переміжки часу, що йдуть один по
одному, і таким чином вони можуть послідовно надходити на ринок, не
нагромаджуючись великими масами як потенціяльний товаровий капітал,
поки їх дійсно відправиться. Тому й зворотний приплив розподіляється
на коротші послідовні періоди часу, так що одна частина постійно перетворюється
на грошовий капітал, тимчасом як друга частина циркулює
як товаровий капітал. В наслідок такого розподілу зворотного
припливу на кілька послідовних періодів скорочується весь час обігу, а
тому скорочується й оборот. Насамперед розвивається більша чи менша
частість функціонування засобів транспорту, — напр., численність поїздів
на залізниці розвивається, з одного боку, разом із тим, як осередок
продукції продукує дедалі більше, стає більшим центром продукції, і
розвивається вона в напрямку до вже наявних ринків збуту, отже, в напрямку
до великих центрів продукції та залюднення, до вивізних портів
тощо. Але, з другого боку, ця особлива легкість сполучень і зумовлений
нею прискорений оборот капіталу (оскільки його зумовлює час обігу)
призводить, навпаки, до прискореної концентрації, з одного боку, центру
продукції, а з другого — його ринку збуту. Разом із прискореною
таким чином концентрацією в певних пунктах маси людей та капіталів,
розвивається концентрація цих мас капіталів у небагатьох руках. Разом
з тим знову пересуваються й переміщуються осередки продукції та ринки
в наслідок їх зміненого відносного положення, зумовленого зміною
\parbreak{}  %% абзац продовжується на наступній сторінці

\parcont{}  %% абзац починається на попередній сторінці
\index{ii}{0184}  %% посилання на сторінку оригінального видання
засобів комунікації. Який-будь осередок продукції, що мав особливі переваги,
в наслідок того, що він містився на великому шляху або каналі,
тепер опиняється близько однісінького залізничного рукава, який функціонує
з порівняно великими перервами, тимчасом як інший осередок,
що був раніш зовсім осторонь від головних шляхів сполучення, тепер
опиняється у вузловому пункті кількох залізниць. Другий осередок розвивається,
перший занепадає. Отже, зміна в засобах транспорту зумовлює
місцеві відмінності в часі обігу товарів, в умовах купівлі, продажу
тощо, або вона інакше розподіляє вже наявні місцеві відмінності. Важливість
цієї обставини для обороту капіталу виявляється в суперечках між
представниками купців та промисловців різних місцевостей з управлінням
залізниць. (Див., напр., вище цитовану Синю книгу Railway Committee).

Тому всі галузі продукції, що, відповідно до природи своїх продуктів,
розраховані переважно на місцевий збут, як напр., броварні, розвиваються
до велетенських розмірів у головних залюднених центрах.
Швидший оборот капіталу почасти урівноважує тут більше подорожчання
деяких умов продукції, місця під будівлю тощо.

Коли, з одного боку, з поступом капіталістичної продукції розвиток
засобів транспорту й комунікації скорочує час обігу для даної кількости
товарів, то той самий поступ і дана з розвитком засобів транспорту
й комунікації можливість, навпаки, зумовлює доконечність роботи
на чимраз віддаленіші ринки, коротко кажучи, на світовий ринок. Маса
товарів, що перебувають у дорозі, відправлених до віддалених пунктів,
надзвичайно зростає, а тому абсолютно й відносно зростає і та частина
суспільного капіталу, яка постійно протягом довшого часу перебуває в
стадії товарового капіталу, перебуває в періоді обігу. Одночасно зростає
в наслідок цього й та частина суспільного багатства, що, замість безпосередньо
служити засобом продукції, витрачається на засоби транспорту
й зв’язку та на основний і обіговий капітал, потрібний для їхньої
роботи.

Відносний протяг подорожі товару від місця продукції до місця
збуту зумовлює ріжницю не лише в першій частині часу обігу, в часі
продажу, а і в другій частині, в зворотному перетворенні грошей
на елементи продуктивного капіталу, в часі купівлі. Напр., товар відправляють
в Індію. Це триває, припустімо, чотири місяці. Хай час продажу
дорівнює нулеві, тобто товар надсилається на замовлення й гроші
виплачується аґентові продуцента підчас здачі товару. Зворотна пересилка
грошей (форма, з якій їх пересилається, тут не має значення) триває
знову таки чотири місяці. Отже, минає загалом вісім місяців, раніш
ніж той самий капітал має змогу знову функціонувати як продуктивний капітал,
— раніш ніж з ним можна знову розпочати ту саму операцію.
Спричинені таким чином відмінності в обороті становлять одну з матеріяльних
основ для різних кредитових строків, подібно до того, як
морська торговля, напр., у Венеції та Ґенуї взагалі становить одно з
джерел кредитової системи у власному розумінні слова. „Криза 1847~\abbr{р.}
дала банкам і торговим підприємствам того часу змогу скоротити індійські
\index{ii}{0185}  %% посилання на сторінку оригінального видання
та китайські узанції\footnote*{
Узанції (Usance) — строки оплати векселів, що визначаються згідно з місцевими
купецькими звичаями. \emph{Ред.}
} (для часу, потрібного на подорож векселів
між цими країнами та Европою) з десятьох місяців по написанні до
шістьох місяців по поданні; тепер, по двадцятьох роках, коли прискорено
зв’язки й заведено телеграф, постала потреба в дальшому скороченні
з шости місяців по поданні до чотирьох місяців по написанні як
перший крок до чотирьох місяців по поданні. Плавба вітрильного судна
з Калькути до Лондону повз ріг Доброї Надії триває пересічно мало
не 90 днів. Узанція в чотири місяці по поданні дорівнювала б приблизно
150 дням плавби. А теперішня узанція в шість місяців по поданні
дорівнює 210 дням дороги“. („London Economist“, 16 червня 1866). —
Навпаки, „Бразільська узанція все ще визначається в два й три місяці
по поданні, векселі з Антверпену (на Лондон) видається на 3 місяці по
написанні й навіть Менчестер і Бредфорд видають векселі на Лондон на
три місяці й довший час. За мовчазною згодою купцеві дається достатню
змогу реалізувати свій товар, якщо й не раніше, то все ж
близько того часу, коли кінчається строк виданих за товар векселів.
Тому узанція індійських векселів не надмірна. Індійські продукти, що
їх продається в Лондоні здебільша строком платежу через три місяці,
не можна реалізувати, коли зарахувати сюди деякий час на продаж, за
значно коротший час, ніж п’ять місяців, тимчасом як ще п’ять місяців
пересічно минає між закупом їх в Індії та здачею на англійські склади.
Ми маємо тут період в десять місяців, тимчасом як строк виданих за
товар векселів не перевищує семи місяців“. (Там же, 30 червня 1866).
„2 липня 1866 п’ять великих лондонських банків, що переважно мають
зв’язок з Індією та Китаєм, а також паризька Comptoir d’Escompte заявили,
що з 1 січня 1867 року їхні філії та аґентства на Сході будуть
купувати й продавати лише векселі, видані не більш як на чотири місяці
по поданні“. (Там же 7 липня 1865~\abbr{р.}). Це зниження однак не мало
успіху й довелось його скасувати (з того часу Суецький канал революціонізував
усе це).

Зрозуміло, коли довшає час обігу товарів, то більшає ризик, що
зміняться ціни на ринку продажу, бо довшає період, що протягом нього
можуть змінитись ціни.

Ріжниця в часі обігу — почасти індивідуальна між різними поодинокими
капіталами тієї самої галузі підприємств, почасти між різними галузями
підприємств залежно від різних узанцій, там, де не виплачується
одразу готівкою, — випливає з різних строків виплати при купівлі та
продажу. Ми не будемо тут докладніше зупинятись на цьому пункті,
важливому для кредитової справи.

Від розміру угод на поставки, а він зростає разом із зростом розмірів
і маштабу капіталістичної продукції, залежать також і ріжниці в часі
обороту. Угода на поставку, як оборудка між продавцем і покупцем, є
операція, що належить до ринку, до сфери циркуляції. Ріжниці, що випливають
\index{ii}{0186}  %% посилання на сторінку оригінального видання
відси, в часі обороту, випливають, отже, з сфери циркуляції,
але вони безпосередньо відбиваються на сфері продукції, і до того ж
незалежно від строків виплат і кредитових відносин, а значить, і при виплаті
готівкою. Напр., вугілля, бавовна, пряжа, тощо є продукти подільні.
Кожний день дає певну кількість готового продукту. Але коли прядун
або власник копалень береться поставити таку масу продуктів, що для
неї потрібен, приміром, чотиритижневий або шеститижневий період послідовних
робочих днів, то відносно до протягу часу, що на нього треба
авансувати капітал, це все одно, якби в цьому процесі праці був заведений
безперервний робочий період в чотири або шість тижнів. Тут звичайно
припускається, що всю замовлену масу продуктів треба доставити
одним заходом, або що її оплатиться лише після того, як її всю доставиться.
Отже, кожен день, взятий окремо, дав свою певну кількість готового
продукту. Але ця готова маса завжди є лише частина тієї маси,
що її треба доставити згідно з угодою. В цьому разі, якщо виготовлена
вже частина замовленого товару не перебуває в процесі продукції, то
вона в усякому разі лежить на складі лише як потенціяльний капітал.

Перейдімо тепер до другого відділу часу обігу — до часу купівлі або
до періоду, що протягом його капітал з грошової форми знову перетворюється
на елементи продуктивного капіталу. Протягом цього періоду
він мусить довший або коротший час лежати в стані грошового капіталу,
а значить, певна частина цілого авансованого капіталу має перебувати
безупинно в стані грошового капіталу, хоч ця частина складається з елементів,
що постійно змінюються. В якомубудь певному підприємстві з
усього авансованого капіталу мусить бути в формі грошового капіталу,
прим., 100\pound{ ф. стерл.} × n, і тимчасом як усі складові частини цих 100\pound{ ф.
стерл.} × n безупинно перетворюються на продуктивний капітал, ця сума
все ж так само завжди знову поповнюється припливом із циркуляції, з реалізованого
товарового капіталу. Отже, певна частина вартости авансованого
капіталу постійно перебуває в стані грошового капіталу, отже, в формі,
що належить не до сфери його продукції, а до сфери його циркуляції.

Ми вже бачили, що подовження часу, зумовлене віддаленістю ринку,
подовження, що протягом його капітал є зв’язаний в формі товарового
капіталу, безпосередньо призводить до запізнення зворотного припливу
грошей, отже, затримує перетворення капіталу з грошового капіталу на
продуктивний.

Далі, щодо закупу товарів, ми бачили (розд. VI), як час купівлі,
більша або менша віддаленість від головних джерел придбання сировинного
матеріялу примушує купувати сировинний матеріял на довші періоди
й зберігати його придатним до вжитку у формі продуктивного запасу, лятентного
або потенціяльного продуктивного капіталу; отже, що така
віддаленість, при тому самому зрештою маштабі продукції, збільшує масу
капіталу, що його доводиться авансувати одним заходом, і час, що на
нього доводиться авансувати його.

Подібно впливають в різних галузях підприємства періоди — більш
або менш протяжні — що в них на ринок подається чималі маси сировинного
\index{ii}{0187}  %% посилання на сторінку оригінального видання
матеріялу. Так, напр., у Лондоні що три місяці бувають великі
авкціони шерсти, які реґулюють шерстяний ринок, тимчасом як ринок
бавовни від урожаю до врожаю поновлюється в цілому безупинно, хоч
і не завжди рівномірно. Такі періоди визначають головні строки закупу
цих сировинних матеріялів і особливо впливають на спекулятивні закупи,
що зумовлюють більш або менш протяжні авансування на ці елементи
продукції, — впливають цілком так само, як природа випродукуваних
товарів впливає на спекулятивне, навмисне, довше або коротше затримування
продукту в формі потенціяльного товарового капіталу. „Отже,
сільський господар теж мусить до певної міри бути спекулянтом і тому
утримуватись від продажу своїх продуктів, зважаючи на обставини часу“\dots{}
Далі ідуть деякі загальні правила\dots{} „Тимчасом при збуті продуктів найголовніше
все ж таки залежить від особи, від самого продукту й місцевости.
Коли людина, крім кмітливости та вдачі (!), має достатній капітал для продукції
(Betriebskapital), їй не можна докоряти, якщо при незвичайно низьких цінах
вона залишить лежати свій зібраний хліб ще цілий рік; навпаки, кому
бракує обігового капіталу або взагалі (!) духа спекуляції, той дбатиме
про те, щоб взяти звичайну пересічну ціну і, значить, муситиме продавати,
скоро матиме нагоду до цього. Коли вовну зберігати довше, ніж протягом
одного року, то це майже завжди зробить тільки шкоду; тимчасом як
зернові хліба та олійне насіння можна зберігати кілька років, і при цьому
не псуються їхні властивості й добротність. Ті продукти, що зазнають
звичайно протягом короткого часу великого піднесення та падіння цін як
от, прим., олійне насіння, хміль, ворсувальні шишки тощо, небезпідставно
залишають лежати в ті роки, коли ціни на них нижчі від цін їхньої продукції.
Найменше слід відкладати продаж таких продуктів, що потребують
щоденних витрат на їхнє утримання, як от відгодована худоба, або таких,
що псуються, як от фрукти, картопля і~\abbr{т. ін.} В деяких місцевостях у певну
добу року ціна продукту пересічно є найнижча, а іншого часу, навпаки,
найвища. Так, напр., пересічно ціна на зерно на Мартіна в деяких місцевостях
нижча, ніж між різдвом і великоднем. Далі, в деяких місцевостях
деякі продукти можна добре продати тільки певного часу, як, напр.,
вовну на вовняних ярмарках у таких місцевостях, де, крім ярмарок, звичайно
дуже мало торгують вовною“. (Kirchhof, стор. 302).

Розглядаючи другу половину часу обігу, що протягом його гроші
знову перетворюються на елементи продуктивного капіталу, треба взяти
на увагу не лише це перетворення само собою; не лише час, що протягом
його гроші припливають назад, залежно від віддалености того
ринку, де продається продукт; треба взяти насамперед на увагу й розміри
тієї частини авансованого капіталу, яка постійно мусить перебувати в
грошовій формі, в стані грошового капіталу.

Лишаючи осторонь усяку спекуляцію, розмір закупів тих товарів, які
постійио мають бути наявні як продуктивний запас, залежить від строків
поновлення цього запасу, отже, від обставин, що й собі залежать від
ринкових відносин, і які тому є різні для різних сировинних матеріялів
тощо; отже, тут доводиться час від часу одним заходом авансовувати
\parbreak{}  %% абзац продовжується на наступній сторінці

\parcont{}  %% абзац починається на попередній сторінці
\index{ii}{0188}  %% посилання на сторінку оригінального видання
гроші більшими масами. Вони припливають назад хоч швидше, хоч повільніше
— залежно від обороту капіталу, — але завжди лише частинами.
Частину їх так само постійно витрачається знову в невеликі переміжки
часу, а саме ту частину, що знову перетворюється на заробітну плату.
Але другу частину, ту, що її треба знову перетворити на сировинний
матеріял тощо, треба нагромаджувати протягом довшого часу як запасний
фонд, хоч для закупів, хоч для виплат. Отже, ця частина існує в формі
грошового капіталу, хоч змінюється розмір, що в ньому вона існує в
такій формі.

З наступного розділу ми побачимо, як інші обставини, — хоч випливають
вони з процесу продукції, хоч з процесу циркуляції, — неминуче
зумовлюють отаке перебування певної частини авансованого капіталу в
грошовій формі. Взагалі ж треба зазначити, що економісти мають великий
нахил забувати, що частина потрібного в підприємстві капіталу не лише
постійно перебігає послідовно три форми: грошового капіталу, продуктивного
капіталу й товарового капіталу, але що різні частини його постійно
перебувають одна поряд однієї в цих трьох формах, хоч відносна
величина цих частин постійно змінюється. Вони забувають саме про ту
частину, яка постійно існує як грошовий капітал, хоч саме ця обставина
дуже важлива для розуміння буржуазного господарства, а тому має значення
також і на практиці.

\section{Вплив часу обороту на величину авансованого капіталу}

В цьому та наступному шістнадцятому розділі ми досліджуємо вплив
часу обороту на самозростання вартости капіталу.

Візьмімо товаровий капітал, що є продукт робочого періоду, напр.,
де\-в’я\-тьох тижнів. Лишімо покищо осторонь частину вартости продукту,
долучену до нього в наслідок пересічного зношування основного капіталу,
а також і додаткову вартість, долучену до нього підчас продукційного
процесу; тоді вартість цього продукту дорівнюватиме вартості
авансованого на його продукцію поточного капіталу, тобто заробітної плати
й зужиткованих на його продукцію сировинних і допоміжних матеріялів.
Припустімо, що ця вартість дорівнює 900\pound{ ф. стерл.}, так що тижнева
витрата становить 100\pound{ ф. стерл}. Отже, періодичний час продукції, що
збігається тут з робочим періодом, становить 9 тижнів. При цьому байдуже,
чи припускається, що тут ідеться про робочий період для неподільного
продукту, чи про безперервний робочий період для продукту
подільного, скоро тільки кількість подільного продукту, що його воднораз
подається на ринок, коштує 9 тижнів праці. Припустімо, що час
обігу триває 3 тижні. Отже, весь період обороту триває 12 тижнів.
По 9 тижнях авансований продуктивний капітал перетворюється на товаровий
\index{ii}{0189}  %% посилання на сторінку оригінального видання
капітал, але потім він ще три тижні перебуває в періоді циркуляції.
Отже, новий період продукції може початись знову тільки на початку
13-го тижня, і продукція мала б припинитись на три тижні, або на
четверту частину цілого періоду обороту. Тут знову таки байдуже, чи
припускаємо ми, що це припинення пересічно триває доти, доки товар
буде проданий, чи припускаємо, що воно зумовлене віддаленістю ринку
або строками виплат за проданий товар. Що три місяці продукція
припиняється на три тижні, отже, протягом року вона припиняється на
$4×3 \deq{} 12$ тижнів \deq{} 3 місяцям \deq{} \sfrac{1}{4} річного періоду обороту. Тому
провадити продукцію безперервно тиждень у тиждень у тому самому
маштабі можна лише двома способами.

Або треба скоротити маштаб продукції так, щоб 900\pound{ ф. стерл.} вистачало
на те, щоб тримати роботу в русі так протягом робочого періоду,
як і протягом часу обігу першого обороту. Тоді з початком
10-го тижня відкривається другий робочий період, отже, й другий період
обороту, — відкривається раніше, ніж закінчиться перший період обороту,
бо період обороту дванадцятитижневий, а робочий період дев’ятитижневий.
900\pound{ ф. стерл.}, розподілені на 12 тижнів, дають 75\pound{ ф. стерл.} на тиждень.
Насамперед очевидно, що такий скорочений маштаб підприємства має
собі за передумову зміну розмірів основного капіталу, а значить, і взагалі
скорочення розмірів підприємства. Подруге, сумнівно, чи можна взагалі
провести таке скорочення, бо відповідно до розвитку продукції в різних
підприємствах є певний нормальний мінімум капіталовкладення, і коли
воно нижче від цього мінімуму, то підприємство не може витримати конкуренції.
Самий цей нормальний мінімум з розвитком капіталістичної
продукції теж раз-у-раз зростає і, значить, не є сталий. Але між даним
кожного разу нормальним мінімумом і дедалі більшим нормальним максимумом
є численні проміжні щаблі, — середина, що припускає дуже різні ступені
капіталовкладень. В межах цієї середини, отже, також можна провести
скорочення, що межа його є самий кожноразовий нормальний мінімум.
При затриманнях у продукції, переповненні ринку, подорожчанні сировинного
матеріялу тощо, скорочення нормальних витрат обігового капіталу,
за даної величини основного капіталу, постає через обмеження
робочого часу, через те, що роблять, приміром, тільки півдня; так само
в часи розцвіту за даної величини основного капіталу надмірне збільшення
обігового капіталу постає почасти через подовження робочого часу,
почасти через його інтенсифікацію. В підприємствах, заздалегідь розрахованих
на такі коливання, дають собі раду почасти вищезазначеними
способами, почасти одночасним уживанням більшого числа робітників,
а це сполучається з застосуванням запасного основного капіталу, напр.,
запасних паровозів на залізницях тощо. Але тут, припускаючи нормальні
відношення, ми не будемо брати на увагу таких ненормальних коливань.

Отже, тут, щоб зробити продукцію безперервною, витрату того самого
обігового капіталу розподіляється на довший час, на 12 тижнів замість 9.
Отже, в кожний даний переміжок часу функціонує вменшений продуктивний
капітал; поточна частина продуктивного капіталу зменшується
\parbreak{}  %% абзац продовжується на наступній сторінці

\parcont{}  %% абзац починається на попередній сторінці
\index{iii2}{0190}  %% посилання на сторінку оригінального видання
по якій може бути приставлено ввесь продукт, і в цьому розумінні вона реґулює
ціну цього всього продукту.

Проте, \emph{подруге}, хоч у цьому випадку загальна ціна продукту землі
істотно модифікувалася б, цим зовсім не був би скасований закон диференційної
ренти. Бо коли ціна продукту кляси $А$, а разом з тим і загальна ринкова
ціна $= Р \dplus{} r$, то ціна кляс $B$, $C$, $D$ і~\abbr{т. ін.} теж була б
$= P \dplus{} r$. А що
$Р - Р'$ для кляси $B$ $= d$, то $(Р \dplus{} r) - (Р' \dplus{} r)$ теж було б $= d$, а для $C$
$P - Р'' \deq{} (Р \dplus{} r) - (Р'' \dplus{} r)$ було б $= 2d$, як для $D$, нарешті,
$Р - Р''' \deq{} (Р \dplus{} r) - (Р''' \dplus{} r) \deq{} 3d$ і~\abbr{т. д.} Отже, диференційна рента лишилася б та сама, що
давніш і реґулювалася б тим самим законом, хоч рента мала б у собі елемент незалежний
від цього закону, і хоч вона взагалі підвищилась би одночасно з ціною продукту
землі. Звідси випливає: хоч би як завжди стояла справа з рентою з найменш
родючих родів землі, закон диференційної ренти від цього не тільки не залежить,
але навіть єдиний спосіб зрозуміти саму диференційну ренту відповідно до її
характеру є в тому, що рента кляси землі $А$ припускається рівною нулеві. Чи
вона дійсно $= 0$, чи $> 0$, це байдуже, оскільки справа йде про диференційну
ренту, і насправді не береться на увагу.

Отже, закон диференційної ренти, не залежить від наслідку дальшого
дослідження.

Тепер, коли ставити далі питання про підставу того припущення, щопродукт
землі найгіршого роду $А$ не дає ренти, то відповідь неминуче така:
коли ринкова ціна продукту землі, скажімо, збіжжя, досягає такої висоти, що
додатково авансований капітал, укладений в землю кляси $А$, оплачує звичайну
ціну продукції, тобто дає капіталові звичайний пересічний зиск, то цієї умови
досить для приміщення додаткового капіталу в землю кляси $А$. Тобто, капітаталістові
досить цієї умови для того, щоб укладати новий капітал з звичайним
зиском і використовувати його нормальним способом.

Тут слід зауважити, що і в цьому випадку ринкова ціна мусить стояти
вище, ніж ціна продукції на $А$. Бо скоро створюється додаткове подання, відношення
попиту і подання очевидно зміниться. Давніш подання було недостатнє,
тепер воно достатнє. Отже, ціна мусить понизитись. Але для того, щоб вона могла
понизитись, вона давніш мусила стояти на вищому рівні, ніж ціна продукції на $А$.
Але те, що кляса $А$, яка наново вступає в обробіток, менш родюча, призводить до
того, що ціна не впаде знову до такого низького рівня, як в той час, коли ринок
реґулювала ціна продукції кляси $В$. Ціна продукції на $А$ становить межу не для
тимчасового, а для відносно перманентного підвищення ринкової ціни. — Навпаки,
коли новооброблювана земля родючіша, ніж кляса $А$, яка до того часу була за
реґуляційну, і проте, її досить лише для покриття додаткового попиту, то ринкова
ціна залишається без зміни. Але дослідження того, чи дає ренту нижча
кляса землі, і в цьому випадку збігається з тим, яким ми зайняті тепер, бо
і тут припущення, що кляса землі $А$ не дає ренти, з’ясовувалося б тим, що
капіталістичному орендареві досить ринкової ціни, щоб нею точно покрити
зужиткований капітал плюс пересічний зиск; коротко кажучи, тим, що ринкова
ціна дає йому ціну продукції його товару.

В усякому разі капіталістичний орендар, може за цих відношень обробляти
землю кляси $А$, оскільки він вирішує справи як капіталіст. Умова для
нормального збільшення вартости капіталу на землі роду $А$ є тепер в наявності.
Але з тієї передумови, що орендар міг би вкладати тепер капітал у землі
роду $А$, за умов відповідних пересічним відношенням зростання вартости капіталу,
хоч він і не мав би можливости платити ренту, — зовсім не випливає
висновок, що ця земля, належна до кляси $А$, так і буде без дальших околичностей
віддана орендареві. Та обставина, що орендар міг би використати свій
капітал з звичайним зиском, коли йому не доводиться платити ренти, для
\parbreak{}  %% абзац продовжується на наступній сторінці

\parcont{}  %% абзац починається на попередній сторінці
\index{ii}{0191}  %% посилання на сторінку оригінального видання
першого, ці функції за першого періоду обороту точно відмежовані одна
від однієї, або принаймні їх можна точно відмежувати, тимчасом як протягом
другого періоду обороту вони, навпаки, переплітаються одна з
однією.

Уявімо собі справу наочніше:

Перший період обороту триває 12 тижнів. Перший робочий період —
9 тижнів; оборот авансованого на нього капіталу закінчується на початку
13-го тижня. Протягом останніх 3 тижнів функціонує додатковий капітал
в 300\pound{ ф. стерл.}, який починає другий дев’ятитижневий робочий
період.

Другий період обороту. На початку 13-го тижня 900\pound{ ф. стерл.} припливають
назад і можуть почати новий оборот. Але другий робочий
період уже на десятому тижні почато за допомогою додаткових 300\pound{ ф.
стерл.}; на початку 13-го тижня за допомогою тих самих 300\pound{ ф. стерл.}
уже закінчено третину робочого періоду, 300\pound{ ф. стерл.} з продуктивного
капіталу перетворено на продукт. А що для закінчення другого робочого
періоду треба ще лише 6 тижнів, то в процес продукції другого робочого
періоду можуть ввійти лише дві третини капіталу в 900\pound{ ф. стерл.},
який повернувся назад, а саме 600\pound{ ф. стерл}. З первісних 900\pound{ ф. стерл.}
звільнилося 300\pound{ ф. стерл.}, щоб відігравати ту саму ролю, яку відігравав
у першому робочому періоді додатковий капітал в 300\pound{ ф. стерл}. Наприкінці
6-го тижня другого періоду обороту закінчено другий робочий
період. Витрачений на нього капітал в 900\pound{ ф. стерл.} повертається за три
тижні, отже, наприкінці 9-го тижня другого дванадцятитижневого періоду
обороту. Протягом 3 тижнів його часу обігу ввіходить у робочий період
звільнений капітал в 300\pound{ ф. стерл}. З ним починається на 7-й тиждень
другого періоду обороту або на 19-й тиждень року третій робочий
період капіталу в 900\pound{ ф. стерл}.

Третій період обороту. Наприкінці 9-го тижня другого періоду обороту
знову зворотно припливають 900\pound{ ф. стерл}. Але третій робочий
період почався вже на сьомому тижні попереднього періоду обороту й
6 тижнів його вже минуло. Тому він триває тільки три тижні. Отже,
з 900\pound{ ф. стерл.}, що повернулись назад, у процес продукції ввіходять
лише 300\pound{ ф. стерл}. Четвертий робочий період заповнює дев’ятитижневу
решту цього періоду обороту, і таким чином з 37-го тижня року починається
одночасно четвертий період обороту й п’ятий робочий період.

Щоб спростити обчислення, ми припустимо робочий період в 5 тижнів,
час обігу в 5 тижнів, отже, період обороту в 10 тижнів; рік рахуватимемо
в 50 тижнів, а щотижневу витрату капіталу рахуватимемо в 100\pound{ ф.
стерл}. Отже, робочий період потребує поточного капіталу в 500\pound{ ф. стерл.},
а час обігу потребує додаткового капіталу — нових 500\pound{ ф. стерл}. Робочі
періоди й періоди оборотів позначиться тоді так:

\noindent{}
{\settablefont{}1-й робочий період: тижні 1\textendash{}5 (500\pound{ ф. стерл.} товару повертаються
наприкінці 10 тижня).}

\noindent{}
{\settablefont{}2-й робочий період: тижні 6--10 (500\pound{ ф. стерл.} товару повертаються
наприкінці 15 тижня).}

\noindent{}
{\settablefont{}3-й робочий період: тижні 11--15 (500\pound{ ф. стерл.} товару повертаються
наприкінці 20 тижня).}

\noindent{}
{\settablefont{}4-й робочий період: тижні 16--20 (500\pound{ ф. стерл.} товару повертаються
наприкінці 25 тижня).}

\noindent{}
{\settablefont{}5-й робочий період: тижні 21--25 (500\pound{ ф. стерл.} товару повертаються
наприкінці 30 тижня)

\noindent{}і~\abbr{т. д.}}
\parcont{}  %% абзац починається на попередній сторінці
\index{iii1}{0192}  %% посилання на сторінку оригінального видання
репродукції в наслідок нагромадження капіталу, то, при інших
незмінних умовах, потрібна додаткова кількість бавовни. Те саме
і щодо засобів існування. Робітничий клас, для того, щоб і далі
жити при звичайних пересічних умовах, мусить діставати принаймні
попередню кількість необхідних засобів існування, хоч,
може, і розподілених дещо інакше між різними сортами товарів;
якщо ж узяти до уваги щорічний ріст населення, то потрібна
ще певна додаткова кількість засобів існування; те саме
з більшими чи меншими модифікаціями можна сказати і щодо
інших класів.

Отже, здається, ніби на стороні попиту є певна, даної величини
суспільна потреба, яка для свого задоволення вимагає
певної кількості товару на ринку. Але кількісна визначеність
цієї потреби цілком еластична й хитка. Вона тільки здається
фіксованою. Якби засоби існування були дешевші або грошова
заробітна плата була вища, то робітники купували б більше
засобів існування і виявилася б більша „суспільна потреба“ на
ці сорти товарів, — при чому ми зовсім залишаємо осторонь
пауперів і~\abbr{т. д.}, „попит“ яких стоїть нижче найвужчих меж їх
фізичної потреби. Коли б, з другого боку, подешевшала, наприклад,
бавовна, то попит капіталістів на бавовну виріс би, в бавовняну
промисловість було б вкладено більше додаткового
капіталу і~\abbr{т. д.} При цьому взагалі не слід забувати, що попит
на продуктивне споживання при нашому припущенні є попит
з боку капіталіста і що справжня мета капіталіста є виробництво
додаткової вартості, так що він тільки з цією метою
виробляє певний сорт товарів. З другого боку, це не перешкоджає
тому, що капіталіст, оскільки він виступає на ринку як
покупець, наприклад, бавовни, репрезентує потребу в бавовні,
адже і для продавця бавовни байдуже, чи перетворює покупець
цю бавовну в сорочки, в бавовняний порох, чи має намір
затикати нею вуха собі і всьому світові. Але в усякому разі це
справляє великий вплив на те, якого роду покупець він є. Його
потреба в бавовні істотно модифікується тією обставиною, що
в дійсності вона тільки приховує його потребу добувати зиск. —
Межі, в яких репрезентована на \emph{ринку} потреба в товарах —
попит — кількісно відрізняється від \emph{дійсної суспільної} потреби,
звичайно, дуже різні для різних товарів; я маю на увазі ріжницю
між кількістю товарів, на яку є попит, і тією кількістю
їх, на яку був би попит при інших грошових цінах товарів або
при інших грошових або життьових умовах покупців.

Нема нічого легшого, як зрозуміти нерівномірності попиту
й подання та відхилення, що випливають звідси, ринкових цін
від ринкових вартостей. Справжня трудність полягає у визначенні
того, що слід розуміти під висловом: попит і подання
покриваються.

Попит і подання покриваються, якщо вони стоять у такому
відношенні одне до одного, що товарна маса певної галузі виробництва
\index{iii1}{0193}  %% посилання на сторінку оригінального видання
може бути продана по її ринковій вартості, — ні вище,
ні нижче. Ось перше, що нам кажуть.

Подруге: якщо товари можуть бути продані по їх ринковій
вартості, то попит і подання покриваються.

Якщо попит і подання взаємно покриваються, то вони перестають
діяти, і саме тому товари продаються по їх ринковій вартості.
Якщо дві сили рівномірно діють у протилежних напрямах,
то вони одна одну знищують, зовсім не діють назовні, і явища,
які відбуваються при цій умові, мусять бути пояснені якось
інакше, а не діянням цих двох сил. Якщо попит і подання взаємно
знищуються, то вони перестають щонебудь пояснювати,
не діють на ринкову вартість і залишають нас у цілковитому
невіданні того, чому ринкова вартість виражається саме в цій
сумі грошей, а не в будьякій іншій. Дійсні внутрішні закони
капіталістичного виробництва, очевидно, не можуть бути пояснені
з взаємодіяння попиту й подання (цілком незалежно від
глибшого аналізу цих двох суспільних рушійних сил, який сюди
не стосується), бо ці закони тільки тоді виявляються здійсненими
в чистому вигляді, коли попит і подання перестають
діяти, тобто взаємно покриваються. В дійсності попит і подання
ніколи не покриваються або, якщо і покриваються, то тільки
випадково, — отже, з наукового погляду такі випадки слід прирівняти
до нуля і розглядати як неіснуючі. Але в політичній
економії припускається, що вони покриваються. Чому? Це робиться
для того, щоб розглядати явища в їх закономірному, відповідному
їх поняттю вигляді, тобто розглядати їх незалежно від
того, якими вони здаються в наслідок руху попиту й подання.
З другого боку, для того, щоб знайти дійсну тенденцію їх руху, так
би мовити, фіксувати її. Бо відхилення від рівності мають протилежний
характер і, через те що вони завжди йдуть одне за одним,
вони урівноважуються завдяки своїм протилежним напрямам, завдяки
своїй суперечності. Отже, якщо попит і подання не покриваються
ні в одному випадку, то їх відхилення від рівності йдуть одне
за одним таким чином, — результат відхилення в одному напрямі
є той, що воно викликає відхилення в протилежному напрямі, —
що, коли розглядати підсумок руху за більш-менш довгий період
часу, подання і попит постійно покриваються; однак, вони покриваються
тільки як пересічне минулих уже коливань, тільки як
постійний рух їх суперечності. В наслідок цього ринкові ціни,
що відхиляються від ринкових вартостей, розглядувані щодо
їх пересічної, вирівнюються в ринкові вартості, при чому відхилення
від цих останніх взаємно знищуються як плюс і мінус.
І ця пересічна має не тільки теоретичне значення, вона має
також і практичну важливість для капіталу, вкладення якого розраховане
на коливання й вирівнювання протягом більш-менш певного
періоду часу.

Тому відношення між попитом і поданням пояснює, з одного
боку, тільки відхилення ринкових цін від ринкових вартостей
\parbreak{}  %% абзац продовжується на наступній сторінці

\parcont{}  %% абзац починається на попередній сторінці
\index{iii2}{0194}  %% посилання на сторінку оригінального видання
(радше, в наслідок визначення конкуренцією ціни продукції, яка регулює ринкову
ціну) частини ціни товару, яка зводиться до надзиску, — за при чину перенесення
цієї частини ціни від однієї особи до іншої, від капіталіста до
земельного власника. Але земельна власність тут не є причина, яка \emph{створює}
цю складову частину ціни, або те підвищення ціни, яке є передумовою цієї
частини ціни. Навпаки, коли найгірша земля кляси $А$ не може оброблятись, —
хоч оброблення її дало б ціну продукції, — поки вона не дає надміру над цією
ціною продукції, ренти, то земельна власність є творчою основою \emph{цього} підвищення
ціни. \emph{Сама земельна власність створила ренту}. Це анітрохи не
зміниться від того, що, як у другому розгляненому випадку, рента, виплачувана
тепер з землі $А$, становить диференційну ренту порівняно з тими останніми
додатковими приміщеннями капіталу на старих заорендованих дільницях, які
виплачують лише ціну продукції. Бо та обставина, що оброблення землі $А$
не може початися, поки регуляційна ринкова ціна не підійметься остільки високо,
що земля $А$ зможе давати ренту, — тільки ця обставина є тут причиною
того, що ринкова ціна підвищується до такого пункту, на якому вона для
останніх приміщень капіталу на старих заорендованих дільницях виплачує,
правда, лише їхню ціну продукції, але таку ціну продукції, яка одночасно
дає ренту для землі $А$. Та обставина, що остання взагалі мусить виплачувати
ренту, є тут причиною створення диференційної ренти між землею $А$ і останніми
приміщеннями капіталу на старих заорендованих дільницях.

Коли ми взагалі кажемо, що — припускаючи реґулювання збіжжевої ціни
ціною продукції — земля кляси $А$ не виплачує ренти, то ми маємо на увазі
ренту в категоричному значінні слова. Коли орендар виплачує орендну плату,
яка становить вирахування або з нормальної заробітної плати його робітників,
або з його власного нормального пересічного зиску, то він не виплачує жодної
ренти, жодної самостійної складової частини ціни його товару, яка відрізнялася б
від заробітної плати і зиску. Вже давніш ми відзначали, що на практиці це
завжди трапляється. Коли заробітну плату хліборобських робітників у певній
країні взагалі знижують поза нормальний пересічний рівень заробітної плати,
і тому вирахування з заробітної плати, частина заробітної плати, входить, як
загальне правило, до складу ренти, то це не становить жодного винятку для
орендаря найгіршої землі. В тій самій ціні продукції, яка уможливлює оброблення
найгіршої землі, вже ураховується, як складова стаття, ця низька заробітна
плата, і тому продаж продукту по ціні продукції не дає змоги орендареві
цієї землі виплачувати ренту. Земельний власник може також здати свою землю
в оренду робітникові, який буде готовий усе те, або більшу частину того, що
продажна ціна залишає йому поверх заробітної плати, виплатити у формі
ренти другій особі. Проте, в усіх цих випадках зовсім не виплачується дійсної
ренти, хоч виплачується орендна плата. Але там, де існують відносини, відповідні
капіталістичному способові продукції, рента і орендна плата мусили б
збігатися. Отут ми й повинні дослідити якраз це нормальне відношення.

Якщо навіть розглянуті вище випадки, коли за капіталістичного способу
продукції дійсно можуть вкладатися у землю капітали, не даючи при цьому
ренти, — якщо навіть ці випадки нічого не дають для розв’язання нашої проблеми,
то ще значно менше дасть посилання на колоніяльні відносини. Що робить
колонію колонією, — ми говоримо тут лише про власне хліборобські колонії,
— так це не тільки маса родючих земель, що перебувають у природному
стані. Ні, колоніями робить їх радше та обставина, що ці землі не привласнені,
не підлеглі земельній власності. Саме це і зумовлює таку колосальну ріжницю
між старими землями і колоніями, оскільки справа йде про землю:
юридична або фактична відсутність земельної власности, як слушно відзначив
\parbreak{}  %% абзац продовжується на наступній сторінці

\parcont{}  %% абзац починається на попередній сторінці
\index{ii}{0195}  %% посилання на сторінку оригінального видання
що зменшується розміри продукції. Порівняно з маштабом продукції, капітал,
закріплений в грошовій формі, тут ще більше зростає.

Таким поділом капіталу на первісний продуктивний і додатковий капітал
взагалі досягається безперервна послідовність робочих періодів, постійне
функціонування однаково великої частини авансованого капіталу,
як продуктивного капіталу.

Придивімось до прикладу II.~Капітал, що постійно перебуває в процесі
продукції, є 500\pound{ ф. стерл}. А що робочий період дорівнює 5 тижням,
то протягом 50 тижнів (а їх ми беремо, як рік) цей капітал буде в
роботі 10 разів. Тому й продукт, — лишаючи осторонь додаткову вартість
— дорівнює $500 × 10 \deq{} 5000$\pound{ ф.  стерл}. Отже, з погляду капіталу,
безпосередньо і безупинно діющого в продукційному процесі, — з погляду
капітальної вартости в 500\pound{ ф. стерл.}, — час обігу, здається, цілком
знищується. Період обороту збігається з робочим періодом; час обігу прирівнюється
нулеві.

Коли б, навпаки, продуктивну діяльність капіталу в 500\pound{ ф. стерл.} реґулярно
перепинялося п’ятитижневим періодом обігу, так що він ставав
би знову продукційноздатним лише по закінченні цілого десятитижневого
періоду обороту, то протягом 50 тижнів року ми мали б 5 десятитижневих
оборотів; в них було б 5 п’ятитижневих періодів продукції, отже,
разом 25 тижнів продукції з загальною кількістю продукту на $500 × 5 \deq{} 2500$\pound{ ф. стерл.}; 5 п’ятитижневих періодів обігу, отже, цілого часу обігу теж
25 тижнів. Коли ми тут кажемо, що капітал в 500\pound{ ф. стерл.} обернувся
5 разів протягом року, то очевидно й зрозуміло, що протягом половини
кожного періоду обороту цей капітал в 500\pound{ ф. стерл.} зовсім не функціонував
як продуктивний капітал, і що в підсумку він функціонував тільки
протягом півроку, а другу половину року зовсім не функціонував.

В нашому прикладі на час цих п’ятьох періодів обігу входить у роботу
додатковий капітал в 500\pound{ ф. стерл.}, і в наслідок цього оборот підвищується
з 2500\pound{ ф. стерл.} до 5000\pound{ ф. стерл}. Але й авансований капітал
тепер є 1000\pound{ ф. стерл.} замість 500\pound{ ф. стерл.} 5000 поділені на 1000
дорівнює 5. Отже, замість 10 оборотів маємо 5. Так справді й рахують.
Однак, коли кажуть, що капітал 1000\pound{ ф. стерл.} обернувся 5 разів протягом
року, то в пустій голові капіталіста зникає спогад про час обігу,
і постає сплутане уявлення, ніби цей капітал протягом 5 послідовних
оборотів постійно функціонував у процесі продукції. Але, коли ми кажемо,
що капітал 1000\pound{ ф. стерл.} обернувся п’ять разів, то сюди ввіходить
і час обігу й час продукції. Справді, коли б 1000\pound{ ф. стерл.} безперервно
функціонували в процесі продукції, то при наших припущеннях продукт
мусив би бути \num{10.000}\pound{ ф. стерл.} замість 5000. Але для того, щоб завжди
мати в процесі продукції 1000\pound{ ф. стерл.}, довелось би взагалі авансувати
2000\pound{ ф. стерл}. Економісти, що в них взагалі не знайти нічого виразного
про механізм обороту, завжди недобачають той головний момент, що
продукція може відбуватися безперервно лише тоді, коли в процесі продукції
завжди буде фактично зайнята тільки частина промислового капіталу.
Тимчасом як одна частина перебуває в періоді продукції, друга частина
\index{ii}{0196}  %% посилання на сторінку оригінального видання
весь час мусить бути в періоді циркуляції. Або, інакше кажучи, одна
частина може функціонувати як продуктивний капітал лише з тією умовою,
що другу частину, в формі товарового або грошового капіталу, вилучено
з власне продукції. Недобачати це — значить взагалі недобачати значення
й ролі грошового капіталу.

Нам треба тепер дослідити, яка ріжниця буде в обороті залежно від
того, чи будуть обидва відділи періоду обороту — робочий період і період
циркуляції — рівні один одному, чи робочий період буде більший
або менший, ніж період циркуляції, а потім дослідити, як це впливає на
закріплення капіталу в формі грошового капіталу.

Припустімо, що авансовуваний щотижня капітал в усіх випадках дорівнює
100\pound{ ф. стерл.}, а період обороту — 9 тижням; отже, капітал, який
треба авансувати на кожен період обороту, дорівнює 900\pound{ ф. стерл}.

\subsection{Робочий період дорівнює періодові циркуляції}

Цей випадок, хоч, насправді, він трапляється тільки як рідкісний виняток,
мусить бути за вихідний пункт у дослідженні, бо відношення тут виступають
якнайпростіше та якнайнаочніше.

Два капітали (капітал І, авансований на перший робочий період, і додатковий
капітал II, що функціонує протягом періоду циркуляції капіталу І)
чергуються один по одному в своєму русі, не сплітаючись один з одним.
Тому, за винятком першого періоду, кожний із обох капіталів авансується
лише на свій власний період обороту. Період обороту хай буде,
як у наступних прикладах, 9 тижнів; отже, робочий період і період обігу
буде по 4\sfrac{1}{2} тижні. Тоді ми маємо таку схему року:

\begin{table}[H]
\centering

{\bfseries Таблиця І}

\caption*{Капітал І}

\bigskip

  \begin{tabular}{r r@{~}c r@{~}c c r@{~}c}
    \toprule
    & \multicolumn{2}{c}{Періоди обороту} & \multicolumn{2}{c}{Робочі періоди} & Авансовано & \multicolumn{2}{c}{Періоди циркуляції}\\
    \cmidrule(lr){2-3}
    \cmidrule(lr){4-5}
    \cmidrule(lr){6-6}
    \cmidrule(lr){7-8}
І. & Тижні & \hang{r}{1}\textendash{}\hang{l}{9} & Тижні 
   & \phantom{0}1\textendash{}4\sfrac{1}{2}\phantom{0}
   & 450\pound{ ф. ст.} & Тижні & 4\sfrac{1}{2}\textendash{}9\\
II. & \ditto{Тижні} & 10\textendash{}18 & \ditto{Тижні} & 10\textendash{}13\sfrac{1}{2} 
   & 450\ditto{\pound{ ф. ст.}} & \ditto{Тижні} & 13\sfrac{1}{2}\textendash{}18\\
III. & \ditto{Тижні} & 19\textendash{}27 & \ditto{Тижні} & 19\textendash{}22\sfrac{1}{2} 
   & 450\ditto{\pound{ ф. ст.}} & \ditto{Тижні} & 22\sfrac{1}{2}\textendash{}27\\
IV. & \ditto{Тижні} & 28\textendash{}36 & \ditto{Тижні} & 28\textendash{}31\sfrac{1}{2} 
   & 450\ditto{\pound{ ф. ст.}} & \ditto{Тижні} & 31\sfrac{1}{2}\textendash{}36\\
V. & \ditto{Тижні} & 37\textendash{}45 & \ditto{Тижні} & 37\textendash{}40\sfrac{1}{2} 
   & 450\ditto{\pound{ ф. ст.}} & \ditto{Тижні} & 40\sfrac{1}{2}\textendash{}45\\
VI. & \ditto{Тижні} & \hang{r}{46}\textendash{}\hang{l}{[54]} & \ditto{Тижні} & 46\textendash{}49\sfrac{1}{2} 
   & 450\ditto{\pound{ ф. ст.}} & \ditto{Тижні} 
   & 49\sfrac{1}{2}\textendash{}\hang{l}{[54]\footnotemark{}}\phantom{00} \\
  \end{tabular}
\end{table}
\footnotetext{Тижні, що припадають на другий рік обороту, взято в дужки.}


\index{ii}{0197}  %% посилання на сторінку оригінального видання
\begin{table}[H]
\centering

\caption*{Капітал ІI}

  \begin{tabular}{r r@{~}c r@{~}c c r@{~}c}
    \toprule
    & \multicolumn{2}{c}{Періоди обороту} & \multicolumn{2}{c}{Робочі періоди} & Авансовано & \multicolumn{2}{c}{Періоди циркуляції}\\
    \cmidrule(lr){2-3}
    \cmidrule(lr){4-5}
    \cmidrule(lr){6-6}
    \cmidrule(lr){7-8}

І. & Тижні & \phantom{0}4\tbfrac{1}{2}\textendash{}13\tbfrac{1}{2} & Тижні 
  & 4\tbfrac{1}{2}\textendash{}9 
  & 450\pound{ ф. ст.} & Тижні & 10\textendash{}13\tbfrac{1}{2}\\

II. & \ditto{Тижні} & 13\tbfrac{1}{2}\textendash{}22\tbfrac{1}{2} & \ditto{Тижні} 
  & 13\tbfrac{1}{2}\textendash{}18 
  & 450\ditto{\pound{ ф. ст.}} & \ditto{Тижні} & 19\textendash{}22\tbfrac{1}{2}\\
III. & \ditto{Тижні} & 22\tbfrac{1}{2}\textendash{}31\tbfrac{1}{2} & \ditto{Тижні}
  & 22\tbfrac{1}{2}\textendash{}27 
  & 450\ditto{\pound{ ф. ст.}} & \ditto{Тижні} & 28\textendash{}31\tbfrac{1}{2}\\

IV. & \ditto{Тижні} & 31\tbfrac{1}{2}\textendash{}40\tbfrac{1}{2} & \ditto{Тижні} 
  & 31\tbfrac{1}{2}\textendash{}36
  & 450\ditto{\pound{ ф. ст.}} & \ditto{Тижні} & 37\textendash{}40\tbfrac{1}{2}\\

V. & \ditto{Тижні} & 40\tbfrac{1}{2}\textendash{}49\tbfrac{1}{2} & \ditto{Тижні} 
   & 40\tbfrac{1}{2}\textendash{}45
   & 450\ditto{\pound{ ф. ст.}} & \ditto{Тижні} & 46\textendash{}49\tbfrac{1}{2}\\
VI. & \ditto{Тижні} 
   & 49\tbfrac{1}{2}\textendash{}\hang{l}{[58\tbfrac{1}{2}]}\phantom{00\tbfrac{1}{2}} & \ditto{Тижні}
   & 49\tbfrac{1}{2}\textendash{}\hang{l}{[54]}\phantom{00}
   & 450\ditto{\pound{ ф. ст.}} & \ditto{Тижні} 
   & [55\footnotemarkZ{}\textendash{}58\tbfrac{1}{2}]\\
  \end{tabular}

\end{table}

\noindent{}Протягом
\footnotetextZ{В нім. тексті тут, очевидно, помилково стоїть „54“. \emph{Ред.}}
50 тижнів, що їх ми тут беремо за рік, капітал І закінчив
шість повних робочих періодів, отже, випродукував товарів на $450 × 6
\deq{} 2700$\pound{ ф. стерл.}, а капітал II в п’ять повних робочих періодів — на
$450 × 5 \deq{} 2250$\pound{ ф. стерл}. Крім того, капітал II в останні 1\sfrac{1}{2} тижні року
(з середини 50-го до кінця 51-го тижня\footnote*{
Дальше обчислення побудовано на припущенні 51 тижня в році. \emph{Ред.}
} випродукував ще на 150\pound{ ф.
стерл.} — всього продукту за 51 тиждень випродукувано на 5100\pound{ ф. ст}.
Отже, щодо безпосередньої продукції додаткової вартости — а її продукується
лише протягом робочого періоду — цілий капітал в 900\pound{ ф. стерл.} обернувся
б 5\sfrac{2}{3} раза ($900 × 5\sfrac{2}{3} \deq{} 5100$\pound{ ф. стерл.}). Але коли ми розглянемо
справжній оборот, то побачимо, що капітал І обернувся 5\sfrac{2}{3} раза, бо
наприкінці 51 тижня йому треба ще протягом 3 тижнів закінчувати
свій шостий період обороту; $450 × 5\sfrac{2}{3} \deq{} 2550$\pound{ ф. стерл.}; а капітал
II обернувся 5\sfrac{1}{6} раза, бо він проробив тільки 1\sfrac{1}{2} тижні свого
шостого періоду обороту, значить, ще 7\sfrac{1}{2} тижнів його припадуть на
наступний рік; $450 × 5\sfrac{1}{6} \deq{} 2325$\pound{ ф. стерл.}, ввесь дійсний оборот дорівнює
4875\pound{ ф. стерл}.

Розгляньмо капітал І й капітал II, як два цілком самостійні один проти
одного капітали. В своїх рухах вони цілком самостійні; ці рухи доповнюють
один одного тільки тому, що їхні робочі періоди та періоди
циркуляції безпосередньо чергуються один по одному. Їх можна розглядати
як два цілком незалежні капітали, що належать різним капіталістам.

Капітал І проробив п’ять повних періодів обороту і дві третини свого
шостого періоду обороту. Наприкінці року він перебуває в формі товарового
капіталу, що йому треба ще 3 тижні для своєї нормальної реалізації.
Протягом цього часу він не може ввійти в процес продукції.
Він функціонує як товаровий капітал: він циркулює. З свого останнього
періоду обороту він проробив лише \sfrac{2}{3}. Це можна висловити так: він
обернувся лише \sfrac{2}{3} раза, лише \sfrac{2}{3} цілої вартости його зробили повний оборот.
Ми кажемо: 450\pound{ ф. стерл.} пророблюють свій оборот у 9 тижнів, отже,
300\pound{ ф. стерл.} — у 6 тижнів. При такому способі вислову нехтується органічні
відношення між обома специфічно різними складовими частинами часу обороту.
\index{ii}{0198}  %% посилання на сторінку оригінального видання
Точний зміст вислову, що авансований капітал в 450\pound{ ф. стерл.} зробив
5\sfrac{2}{3} обороту лише той, що він зробив п’ять повних оборотів і тільки
\sfrac{2}{3} шостого. Навпаки, в вислові: капітал, що обернувся, дорівнює
авансованому капіталові, взятому 5\sfrac{2}{3} раза, тобто в наведеному вище прикладі
дорівнює 450\pound{ ф. стерл.} × 5\sfrac{2}{3} \deq{} 2550\pound{ ф. стерл.}, — правильне те, що
коли б цей капітал в 450\pound{ ф. стерл.} не доповнювався б другим капіталом
в 450\pound{ ф. стерл.}, то в дійсності одна частина його мусила б бути в процесі
продукції, а друга — в процесі циркуляції. Коли ми хочемо час обороту
виразити в масі капіталу, що обернувся, то можемо виразити його
виключно в масі наявної вартости (в дійсності — в масі готового продукту).
Та обставина, що авансований капітал не перебуває в такому
стані, в якому він знову може почати процес продукції, виражається в
тому, що лише частина його перебуває в стані, придатному для продукції,
або в тому, що капітал, коли він має бути в стані безперервної продукції,
треба поділити на частини, що з них одна постійно була б у періоді
продукції, а друга — постійно в періоді циркуляції, залежно від взаємного
відношення цих періодів. Це той самий закон, що згідно з ним
масу постійно діющого продуктивного капіталу визначається відношенням
часу обігу до часу обороту.

Наприкінці 51-го тижня — а ми його беремо тут як кінець року
150\pound{ ф. стерл.} з капіталу II авансовано на продукцію недоробленого ще
продукту. Ще деяка частина перебуває в формі поточного сталого капіталу
— сировинного матеріялу тощо — тобто в такій формі, що в ній
вона може функціонувати в процесі продукції як продуктивний капітал.
Але третя частина перебуває в грошовій формі, а саме, принаймні, сума
заробітної плати за решту робочого періоду (3 тижні), що оплачується
лише наприкінці кожного тижня. Хоч на початку нового року, отже, нового
циклу оборотів, ця частина капіталу перебуває не в формі продуктивного
капіталу, а в формі грошового капіталу, що в ній вона не може
ввійти в процес продукції, все ж, коли починається новий оборот, поточний,
змінний капітал, тобто жива робоча сила, уже діє в процесі продукції.
Це явище випливає з того, що хоч робочу силу купується на початку
робочого періоду, напр., щотижня, і так само зуживається, але
оплачується її лише наприкінці тижня. Гроші функціонують тут як засіб
виплати. Тому вони, з одного боку, як гроші перебувають ще в руках
капіталіста, тимчасом як, з другого боку, робоча сила, товар, що на
нього їх перетворюється, вже діє в продукційному процесі; отже, та сама
капітальна вартість з’являється тут двічі.

Коли ми розглядаємо лише робочі періоди, то:

\begin{center}
  \medskip
  \begin{tabular}{l@{~}l@{~}l}
Капітал \phantom{І}І випродукував & 450 × 6 & \deq{} 2700\pound{ф. стерл.}\\

\ditto{Капітал} II \ditto{випродукував} & 450 × 5\sfrac{1}{3} & \deq{} 2400\pound{ф. стерл.}\\
\cmidrule{1-3}
Отже, разом\dotfill & 900 × 5\sfrac{2}{3} & \deq{}  5100\pound{ф. стерл.}\\
  \end{tabular}
\end{center}

\noindent{}Отже, увесь авансований капітал в 900\pound{ ф. стерл.} за рік функціонував
5\sfrac{2}{3} раза як продуктивний капітал. Для продукції додаткової вартости
байдуже, чи функціонують навперемінку 450\pound{ ф. стерл.} ввесь час у процесі
\index{ii}{0199}  %% посилання на сторінку оригінального видання
продукції і 450\pound{ ф. стерл.} ввесь час у процесі циркуляції, чи 900\pound{ ф.
стерл.} функціонують протягом 4\sfrac{1}{2} тижнів у процесі продукції, а протягом
наступних 4\sfrac{1}{2} тижнів — у процесі циркуляції.
\noclub[1]

Навпаки, коли ми розглядаємо періоди обороту, то:

\begin{table}[H]
  \centering
  \begin{tabular}{r@{~}l@{~}l}
    Капітал \phantom{І}І & 450 × 5\tbfrac{2}{3} & \deq{} 2550\pound{ ф. стерл.}\\

    \ditto{Капітал} II & 450 × 5\tbfrac{1}{6} & \deq{} 2325\pound{ ф. стерл.}\\
    \cmidrule{1-3}
    Отже, оборот цілого капіталу & 900 × 5\tbfrac{5}{12} & \deq{} 4875\pound{ ф. стерл.}\\

  \end{tabular}
\end{table}
\noindent{}Бо число оборотів цілого капіталу дорівнює сумі підсумків оборотів капіталів
І і II, поділеній на суму капіталу І і II.

Треба зазначити, що капітали І і II, коли б були вони самостійні
один проти одного, все ж становили б лише різні самостійні частини суспільного
капіталу, авансованого в тій самій сфері продукції. Отже, коли б
суспільний капітал у цій сфері продукції складався лише з І і II, то для
обороту суспільного капіталу в цій сфері мало б силу те саме обчислення,
що тут має силу для обох складових частин, І і II, того самого
приватного капіталу. Йдучи далі, можна зробити таке обчислення для
кожної частини цілого суспільного капіталу, вкладеної в будь-яку особливу
сферу продукції. Нарешті, число оборотів цілого суспільного капіталу
дорівнює сумі капіталу, що обернувся в різних сферах продукції,
поділеній на суму капіталу, авансованого в цих сферах продукції.

Далі треба зауважити, що так само, як тут у тому самому приватному
підприємстві капітали І і II, точно кажучи, мають різні роки обороту
(що цикл оборотів капіталу II починається на 4\sfrac{1}{2} тижні пізніше, ніж
цикл оборотів капіталу І, то рік капіталу І закінчується на 4\sfrac{1}{2} тижні
раніше, ніж рік капіталу II), так і різні приватні капітали в тій самій
сфері продукції починають свою роботу в цілком різні моменти часу,
а тому й закінчують свій річний оборот в різні часи року. Тут досить
зробити таке саме пересічне обчислення, що його ми вище застосували
до капіталів І і II, щоб роки обороту різних самостійних частин суспільного
капіталу звести до одного загального року обороту.

\subsection{Робочий період більший, ніж період циркуляції}

Замість чергуватися один по одному, робочі періоди й періоди обороту
капіталу І і II перехрещуються один з одним. Разом з тим постає
тут звільнення капіталу, чого не було в вище розглянутому випадку.

Але від цього нічого не змінюється в тому, що тепер, як і раніше,
1) число робочих періодів цілого авансованого капіталу дорівнює сумі
вартости річного продукту обох авансованих частин капіталу, поділеній
на весь авансований капітал, і 2) число оборотів цілого капіталу дорівнює
сумі підсумків обох оборотів, поділеній на суму обох авансованих
капіталів. Ми мусимо й тут розглядати обидві частини капіталу так, ніби
вони пророблювали цілком незалежні один від одного рухи обороту.

Отже, ми знову припускаємо, що на процес праці треба щотижня
авансовувати 100\pound{ ф. стерл}. Робочий період триває 6 тижнів, отже, кожного
\parbreak{}  %% абзац продовжується на наступній сторінці

\parcont{}  %% абзац починається на попередній сторінці
\index{ii}{0200}  %% посилання на сторінку оригінального видання
разу він потребує авансування в 600\pound{ ф. стерл.} (капітал І). Період циркуляції
3 тижні; отже, період обороту, як і раніш, 9 тижнів. Капітал II
в 300\pound{ ф. стерл.} ввіходить у роботу протягом тритижневого періоду циркуляції
капіталу І.~Коли розглядати їх обидва, як капітали, незалежні
один від одного, то схема річного обороту матиме такий вигляд:

\begin{table}[H]
\centering
{\bfseries Таблиця II}
\caption*{Капітал І. 600\pound{ ф. стерл.}}
\bigskip
  \begin{tabular}{r r@{~}c r@{~}c c r@{~}c}
    \toprule
    & \multicolumn{2}{c}{Періоди обороту} & \multicolumn{2}{c}{Робочі періоди} & Авансовано & \multicolumn{2}{c}{Періоди циркуляції}\\
    \cmidrule(lr){2-3}
    \cmidrule(lr){4-5}
    \cmidrule(lr){6-6}
    \cmidrule(lr){7-8}

І.  & Тижні & 1\textendash{}9 & Тижні
    & 1\textendash{}6 & 600\pound{ ф. ст.}
    & Тижні & 7\textendash{}9\\

II. & \ditto{Тижні} & 10\textendash{}18 & \ditto{Тижні}
    & 10\textendash{}15 & 600\ditto{\pound{ ф. ст.}}
    & \ditto{Тижні} & 16\textendash{}18\\

III.& \ditto{Тижні} & 19\textendash{}27 & \ditto{Тижні} 
    & 19\textendash{}24 & 600 \ditto{\pound{ ф. ст.}}
    & \ditto{Тижні} & 25\textendash{}27\\

IV. & \ditto{Тижні} & 28\textendash{}36 & \ditto{Тижні}
    & 28\textendash{}33 & 600\ditto{\pound{ ф. ст.}}
    & \ditto{Тижні} & 34\textendash{}36\\

V.  & \ditto{Тижні} & 37\textendash{}45 & \ditto{Тижні} 
    & 37\textendash{}42 & 600\ditto{\pound{ ф. ст.}}
    & \ditto{Тижні} & 43\textendash{}45\\
VI. & \ditto{Тижні} & \hang{r}{46}\textendash{}\hang{l}{[54]} & \ditto{Тижні} 
    & 46\textendash{}51 & 600\ditto{\pound{ ф. ст.}}
    & \ditto{Тижні} & [52\textendash{}54]\\
  \end{tabular}

\caption*{Додатковий капітал II. 300\pound{ ф. стерл.}}
\bigskip
  \begin{tabular}{r r@{~}c r@{~}c c r@{~}c}
    \toprule
    & \multicolumn{2}{c}{Періоди обороту} & \multicolumn{2}{c}{Робочі періоди} & Авансовано & \multicolumn{2}{c}{Періоди циркуляції}\\
    \cmidrule(lr){2-3}
    \cmidrule(lr){4-5}
    \cmidrule(lr){6-6}
    \cmidrule(lr){7-8}

І.  & Тижні & \phantom{0}7\textendash{}15 & Тижні
    & 7\textendash{}9 & 300\pound{ ф. ст.}
    & Тижні & 10\textendash{}15\\

II. & \ditto{Тижні} & 16\textendash{}24 & \ditto{Тижні} 
    & 16\textendash{}18 & 300\ditto{\pound{ ф. ст.}}
    & \ditto{Тижні} & 19\textendash{}24\\

III.& \ditto{Тижні} & 25\textendash{}33 & \ditto{Тижні}
    & 25\textendash{}27 & 300\ditto{\pound{ ф. ст.}} 
    & \ditto{Тижні} & 28\textendash{}33\\

IV. & \ditto{Тижні} & 34\textendash{}42 & \ditto{Тижні} 
    & 34\textendash{}36 & 300\ditto{\pound{ ф. ст.}}
    & \ditto{Тижні} & 37\textendash{}42\\

V.  & \ditto{Тижні} & 43\textendash{}51 & \ditto{Тижні} 
    & 43\textendash{}45 & 300\ditto{\pound{ ф. ст.}}
    & \ditto{Тижні} & 45\textendash{}51\\
  \end{tabular}
\end{table}

\noindent{}Процес продукції відбувається цілий рік безперервно в однакових
розмірах. Обидва капітали І і II лишаються цілком відокремлені. Але
для того, щоб подати їх так відокремленими, нам довелось роз’єднати
їхні справжні схрещування й переплітання, а через це змінити й число
оборотів. А саме, згідно з вище наведеною таблицею, обертається:

\begin{center}
  
  \begin{tabular}{r@{~}l@{~}l}
    Капітал \phantom{І}І & 600 × 5\sfrac{2}{3} & \deq{} 3400\pound{ф. стерл.}\\

    \ditto{Капітал} II & 300 × 5 & \deq{} 1500\pound{ф. стерл.} \\
    \midrule
    отже, ввесь капітал & 900 × 5\sfrac{4}{9} & \deq{} 4900\pound{ф. стерл.}\\
  \end{tabular}
\end{center}

\noindent{}Але це неправильно, бо, як ми побачимо, справжні періоди продукції
та циркуляції не абсолютно збігаються з цими періодами вище наведеної
схеми, де головне було в тому, щоб подати обидва капітали, І і II, незалежними
один від одного.

В дійсності саме капітал II не має ані особливого робочого періоду, ані особливого
періоду циркуляції, відокремлених від цих періодів капіталу І.~Робочий
період триває 6 тижнів, період циркуляції 3 тижні. Що капітал II дорівнює
тільки 300\pound{ ф. стерл.}, то він може виповнити лише частину робочого
\parbreak{}  %% абзац продовжується на наступній сторінці

\parcont{}  %% абзац починається на попередній сторінці
\index{ii}{0201}  %% посилання на сторінку оригінального видання
періоду. Так в дійсності й є. Наприкінці 6-го тижня продукт вартістю
в 600\pound{ ф. стерл.} входить у циркуляцію і наприкінці 9-го тижня повертається
назад в грошовій формі. Разом з тим на початку 7-го тижня
входить у роботу капітал II і покриває потреби наступного робочого
періоду протягом тижнів 7--9. Але, згідно з нашим припущенням, наприкінці
9 тижня робочий період пророблено лише на половину. Отже, на
початку 10-го тижня знову входить у роботу капітал І в 600\pound{ ф. стерл.},
який шойно повернувся назад, і своїми 300\pound{ ф. стерл.} він покриває авансування,
потрібні для тижнів 10--12. Цим завершується другий робочий
період. В циркуляції є продукт вартістю в 600\pound{ ф. стерл.}, і повертаються
вони назад наприкінці 15-го тижня; але, крім того, є 300\pound{ ф. стерл.}
вільних — величина первісного капіталу II, і можуть вони функціонувати
в першу половину наступного робочого періоду, отже, протягом тижнів
13--15. Коли минуть вони, знову повертаються назад 600\pound{ ф. стерл.};
з них 300\pound{ ф. стерл.} вистачить до кінця цього робочого періоду, а
300\pound{ ф. стерл.} лишаються вільні для наступного.

Отже, справа перебігає так:

І.~Період обороту: тижні 1--9.

1-й робочий період: тижні 1--6. Функціонує капітал І, 600\pound{ ф. стерл}.

1-й період циркуляції: тижні 7--9. Наприкінці 9-го тижня повертаються
назад 600\pound{ ф. стерл}.

II.~Період обороту: тижні 7--15.

2-й робочий період: тижні 7--12.

Перша половина: тижні 7--9. Функціонує капітал II, 300\pound{ ф. стерл}.
Наприкінці 9 тижня повертаються назад 600\pound{ ф. стерл.} в грошовій формі
(капітал І).

Друга половина: тижні 10--12. Функціонують 300\pound{ ф. стерл.} капіталу
І.~Решта 300\pound{ ф. стерл.} капіталу І лишаються вільні.

2-й період циркуляції: тижні 13--15. Наприкінці 15-го тижня повертаються
назад в грошовій формі 600\pound{ ф. стерл.} (складені наполовину
з капіталу І, наполовину з капіталу II).

III.~Період обороту: тижні 13--21.

3-й робочий період: тижні 13--18.

Перша половина: тижні 13--15. Вільні 300\pound{ ф. стерл.} входять у
роботу. Наприкінці 15-го тижня повертаються назад 600\pound{ ф. стерл.} в грошовій
формі.

Друга половина: тижні 16--18. З 600\pound{ ф. стерл.}, що повернулись,
функціонують 300\pound{ ф. стерл.}, а решта 300\pound{ ф. стерл.} знову лишаються
вільні.

3-й період циркуляції: тижні 19--21, що наприкінці їх знову зворотно
припливають 600\pound{ ф. стерл.} в грошовій формі; в цих 600\pound{ ф. стерл.}
капітал І і капітал II тепер злито так, що їх годі відрізнити один од одного.

Таким чином, до кінця 51-го тижня відбувається вісім повних оборотів
капіталу в 600\pound{ ф. стерл.} (І: тижні 1--9; II: 7--15; III: 13--21;
IV: 19--27; V: 25--33; VI: 31--39; VII: 37--45; VIII: тижні
43--51). А що тижні 49--51 припадають на восьмий період циркуляції,
\index{ii}{0202}  %% посилання на сторінку оригінального видання
то протягом цього періоду мусять ввійти в роботу й підтримувати
продукцію в русі 300\pound{ ф. стерл.} звільненого капіталу. Разом з тим наприкінці
року оборот має такий вигляд: 600\pound{ ф. стерл.} вісім разів зробили
свій кругобіг, що дає 4800\pound{ ф. стерл}. До цього долучається продукт
останніх 3 тижнів (49--51), який проробив лише третину свого дев’ятитижневого
кругобігу, отже, у суму обороту він увіходить лише третиною
своєї величини, 100\pound{ ф. стерл}. Отже, коли річний продукт, рахуючи рік в
51 тиждень, дорівнює 5100\pound{ ф. стерл.}, то капітал, що обернувся, становитиме
тільки 4800 \dplus{} 100 \deq{} 4900\pound{ ф. стерл.}; отже, ввесь авансований капітал
в 900\pound{ ф. стерл.} обернувся 5\sfrac{4}{9} раза, тобто на незначну величину більше,
ніж у випадку І.

В цьому прикладі припускався такий випадок, коли робочий час \deq{} \sfrac{2}{3},
а час обігу \deq{} \sfrac{1}{3} періоду обороту, отже, робочий час є просте кратне
часу обігу. Треба з’ясувати, чи констатоване вище звільнення капіталу
буде й в інших умовах.

Припустімо, що робочий період дорівнює 5 тижням, час обігу \deq{} 4 тижням,
щотижнево авансовуваний капітал \deq{} 100\pound{ ф. стерл}.

І.~Період обороту: тижні 1--9.

1-й робочий період: тижні 1--5. Функціонує капітал I \deq{} 500\pound{ ф. стерл}.

1-й період циркуляції: тижні 6--9. Наприкінці 9 тижня припливають
назад в грошовій формі 500\pound{ ф. стерл}.

ІІ.~Період обороту: тижні 6--14.

2-й робочий період: тижні 6--10.

Перший відділ: тижні 6--9. Функціонує капітал II \deq{} 400\pound{ ф. стерл}.
Наприкінці 9 тижня зворотно припливає капітал I \deq{} 500\pound{ ф. стерл.} в грошовій
формі.

Другий відділ: 10 тиждень. \num{3500}\pound{ ф. стерл.}, що повернулися, функціонують
100\pound{ ф. стерл}. Решта 400\pound{ ф. стерл.} лишаються вільні для наступного
робочого періоду.

2-й період циркуляції: тижні 11--14. Наприкінці 14 тижня 500\pound{ ф.
стерл.} зворотно припливають у грошовій формі.

\vtyagnut{}
До кінця 14-го тижня (11--14) функціонують раніш звільнені 400\pound{ ф.
стерл.}; із 500\pound{ ф. стерл.}, що потім повернулись, 100\pound{ ф. стерл.} поповнюють
недостачу для потреб третього робочого періоду (тижні 11--15),
так що знову звільняються 400\pound{ ф. стерл.} для четвертого робочого періоду.
Те саме явище повторюється в кожному робочому періоді; на
початку його є 400\pound{ ф. стерл.}, і їх досить на перші 4 тижні. Наприкінці
4-го тижня припливають назад 500\pound{ ф. стерл.} в грошовій формі, що з
них тільки 100\pound{ ф. стерл.} потрібні для останнього тижня, а решта 400\pound{ ф.
стерл.} лишаються вільні для наступного робочого періоду.

Припустімо далі робочий період в 7 тижнів з капіталом І в 700\pound{ ф.
стерл.}; час обігу в два тижні з капіталом II в 200\pound{ ф. стерл}.

В такому разі перший період обороту триває протягом тижнів 1--9,
з них перший робочий період протягом тижнів 1--7, з авансуванням
в 700\pound{ ф. стерл.}, і перший період циркуляції протягом тижнів 8--9. Наприкінці
9-го тижня 700\pound{ ф. стерл.} зворотно припливають у грошовій формі.


\index{ii}{0203}  %% посилання на сторінку оригінального видання
Другий період обороту, тижні 8--16, має в собі другий робочий
період, тижні 8--14. З них потреби 8-го й 9-го тижнів покривається
капіталом II.~Наприкінці 9-го тижня повертаються давніші 700\pound{ ф. стерл.};
з них пускається в роботу до кінця робочого періоду (тижні 10--14)
500\pound{ ф. стерл.}, 200\pound{ ф. стерл.} лишаються вільні для ближчого наступного
робочого періоду. Другий період обігу триває протягом 15-го й 16 тижнів;
наприкінці 16-го тижня знову повертаються назад 700\pound{ ф. стерл}.
З цього моменту в кожному робочому періоді повторюється те саме
явище. Потреба в капіталі протягом перших двох тижнів покривається
за допомогою 200\pound{ ф. стерл.}, що звільнились наприкінці попереднього
робочого періоду; наприкінці 2-го тижня повертаються назад 700\pound{ ф.
стерл.}; але робочий період налічує ще тільки 5 тижнів, так що на нього
можна авансувати лише 500\pound{ ф. стерл.}; отже, 200\pound{ ф. стерл.} завжди лишаються
вільні для наступного робочого періоду.

Отже, виявляється, що в нашому випадку, де ми припускали, що робочий
період більший, ніж період обігу, наприкінці кожного робочого
періоду при всяких обставинах є звільнений грошовий капітал, такої
саме величини, як капітал II, авансований на період циркуляції. В наших
трьох прикладах капітал II дорівнював: в першому — 300\pound{ ф. стерл.}, в
другому — 400\pound{ ф. стерл.}, в третьому — 200\pound{ ф. стерл.}; відповідно до
цього капітал, що звільнявся наприкінці кожного робочого періоду, був
послідовно 300, 400, 200\pound{ ф. стерл}.

\subsection{Робочий період менший від часу обігу}

Спочатку ми знову припустимо період обороту в 9 тижнів: з них
3 тижні становлять робочий період, що для нього є в розпорядженні
капітал  І \deq{} 300\pound{ ф. стерл}. Період обігу хай буде 6 тижнів. Для цих
6 тижнів потрібен додатковий капітал в 600\pound{ ф. стерл.}, який ми знову
можемо розподілити на два капітали по 300\pound{ ф. стерл.}, що з них кожен
заповнює один робочий період. Тоді ми маємо три капітали по 300\pound{ ф.
стерл.}, з них 300\pound{ ф. стерл.} завжди зайнято в продукції, тимчасом як
600\pound{ ф. стерл.} циркулюють.

\begin{table}[H]
\centering
{\bfseries Таблиця III}
\caption*{Капітал І}
\bigskip
  \begin{tabular}{r r@{~}c r@{~}c r@{~}c}
    \toprule
    & \multicolumn{2}{c}{Періоди обороту} & \multicolumn{2}{c}{Робочі періоди}
    & \multicolumn{2}{c}{Періоди обігу}\\
    \cmidrule(lr){2-3}
    \cmidrule(lr){4-5}
    \cmidrule(lr){6-7}

І.  & Тижні & 1\textendash{}9 & Тижні
    & 1\textendash{}3 & Тижні & 4\textendash{}9\\

ІІ. & \ditto{Тижні} & 10\textendash{}18 & \ditto{Тижні} 
    & 10\textendash{}12 & \ditto{Тижні} & 13\textendash{}18\\

III.& \ditto{Тижні} & 19\textendash{}27 & \ditto{Тижні}
    & 19\textendash{}21 & \ditto{Тижні} & 22\textendash{}27\\

IV. & \ditto{Тижні} & 28\textendash{}36 & \ditto{Тижні}
    & 28\textendash{}30 & \ditto{Тижні} & 31\textendash{}36\\
V.  & \ditto{Тижні} & 37\textendash{}45 & \ditto{Тижні} 
    & 37\textendash{}39 & \ditto{Тижні} & 40\textendash{}45\\

VI. & \ditto{Тижні} & \hang{r}{46}\textendash{}\hang{l}{[54]} & \ditto{Тижні}
    & 46\textendash{}48 & \ditto{Тижні} & \hang{r}{49}\textendash{}\hang{l}{[54]}\\
  \end{tabular}
\end{table}


\begin{table}[H]
\index{ii}{0204}  %% посилання на сторінку оригінального видання
\centering

  \caption*{Капітал II}
  \begin{tabular}{r r@{~}c r@{~}c r@{~}c}
    \toprule
    & \multicolumn{2}{c}{Періоди обороту} & \multicolumn{2}{c}{Робочі періоди}
    & \multicolumn{2}{c}{Періоди обігу}\\
    \cmidrule(lr){2-3}
    \cmidrule(lr){4-5}
    \cmidrule(lr){6-7}

І.  & Тижні & \phantom{0}4\textendash{}12 & Тижні
    & 4\textendash{}6 & Тижні & 7\textendash{}12\\

ІІ. & \ditto{Тижні} & 13\textendash{}21 & \ditto{Тижні} 
    & 13\textendash{}15 & \ditto{Тижні} & 16\textendash{}21\\

III.& \ditto{Тижні} & 22\textendash{}30 & \ditto{Тижні}
    & 22\textendash{}24 & \ditto{Тижні} & 25\textendash{}30\\

IV. & \ditto{Тижні} & 31\textendash{}39 & \ditto{Тижні}
    & 31\textendash{}33 & \ditto{Тижні} & 34\textendash{}39\\

V.  & \ditto{Тижні} & 40\textendash{}48 & \ditto{Тижні} 
    & 40\textendash{}42 & \ditto{Тижні} & 43\textendash{}48\\

VI. & \ditto{Тижні} & \hang{r}{49}\textendash{}\hang{l}{[57]} & \ditto{Тижні}
    & 49\textendash{}51 & \ditto{Тижні} & [52\textendash{}57]\\
  \end{tabular}
\end{table}

\begin{table}[H]
\centering

  \caption*{Капітал III}
  \begin{tabular}{r r@{~}c r@{~}c r@{~}c}
    \toprule
    & \multicolumn{2}{c}{Періоди обороту} & \multicolumn{2}{c}{Робочі періоди}
    & \multicolumn{2}{c}{Періоди обігу}\\
    \cmidrule(lr){2-3}
    \cmidrule(lr){4-5}
    \cmidrule(lr){6-7}

І.  & Тижні & \phantom{0}7\textendash{}15   & Тижні 
    & 7\textendash{}9  & Тижні & 10\textendash{}15\\
ІІ. & \ditto{Тижні} & 16\textendash{}24 & \ditto{Тижні} 
    & 16\textendash{}18 & \ditto{Тижні} & 19\textendash{}24\\
III.& \ditto{Тижні} & 25\textendash{}33 & \ditto{Тижні} 
    & 25\textendash{}27 & \ditto{Тижні} & 28\textendash{}33\\
IV. & \ditto{Тижні} & 34\textendash{}42 & \ditto{Тижні} 
    & 34\textendash{}36 & \ditto{Тижні} & 37\textendash{}42\\
V.  & \ditto{Тижні} & 43\textendash{}51 & \ditto{Тижні} 
    & 43\textendash{}45 & \ditto{Тижні} & 46\textendash{}51\\
  \end{tabular}
\end{table}

\noindent{}Тут ми маємо точну подобу випадку І, з тією лише ріжницею, що
тепер чергуються три капітали, замість двох. Схрещування або переплітання
капіталів немає; кожен поодинокий капітал можна простежити
окремо до кінця року. Отже, тут, так само, як і в випадку І, наприкінці
робочого періоду, не постає звільнення капіталу. Капітал І, що його цілком
витрачено на кінець 3-го тижня, припливає цілком назад наприкінці 9-го
тижня і знову починає функціонувати на початку 10-го тижня. Так само і з
капіталами II і III.~Правильне й повне чергування капіталів виключає
будь-яке звільнення.

Весь оборот обчисляється так:
\begin{table}[H]
  \centering
  \begin{tabular}{r@{~}l@{~}l@{~}l}
  Капітал \phantom{II}І & \deq{} 300\pound{ф. стерл.} & × 5\sfrac{2}{3} 
  & \deq{} 1700\pound{ ф. стерл.} \\

  \ditto{Капітал} \phantom{I}II & \deq{} 300 \ditto{\pound{ф. стерл.}} & × 5\sfrac{1}{3} & \deq{} 1600\pound{ ф. стерл.} \\
 
  \ditto{Капітал} III & \deq{} 300\ditto{\pound{ф. стерл.}} & × 5 & \deq{} 1500\pound{ ф. стерл.} \\
  \midrule
  Ввесь капітал & \phantom{\deq{}} 900\ditto{\pound{ф. стерл.}} & × 5\sfrac{1}{3} & \deq{} 4800\pound{ ф. стерл.}\\
  \end{tabular}
\end{table}

\noindent{}Візьмімо тепер ще один приклад, де період обігу не є точне кратне
робочому періоду. Напр., робочий період 4 тижні, період циркуляції
5 тижнів; отже, в такому разі відповідні розміри капіталу були б:
капітал І \deq{} 400\pound{ ф. стерл.}, капітал II \deq{} 400\pound{ ф. стерл.}, капітал III \deq{} 100\pound{ ф. стерл.}
\begin{table}[H]
\centering
\begin{tabular}{r@{ }l}
Капітал \phantom{II}I & \deq{} 400\pound{ф. стерл.}\\
\ditto{Капітал} II & \deq{} 400\pound{ф. стерл.}\\
\ditto{Капітал} III & \deq{} 100\pound{ф. стерл.}\\
\end{tabular}
\end{table}


\noindent{}Подаємо лише перші три обороти.
\begin{table}[H]
\centering
{\bfseries Таблиця IV}
\caption*{Капітал І}
  \begin{tabular}{r r@{~}c r@{~}c r@{~}c}
    \toprule
    & \multicolumn{2}{c}{Періоди обороту} & \multicolumn{2}{c}{Робочі періоди}
    & \multicolumn{2}{c}{Періоди обігу}\\
    \cmidrule(lr){2-3}
    \cmidrule(lr){4-5}
    \cmidrule(lr){6-7}

І.  & Тижні & 1\textendash{}9 & Тижні 
    & \phantom{17~.~1}1\textendash{}4\phantom{0} & Тижні & 5\textendash{}9\\
ІІ. & \ditto{Тижні} & \phantom{0}9\textendash{}17 & \ditto{Тижні}
    & \phantom{0}9~.~10\textendash{}12 & \ditto{Тижні} & 13\textendash{}17\\
III.& \ditto{Тижні} & 17\textendash{}25 & \ditto{Тижні}
    & 17~.~18\textendash{}20 & \ditto{Тижні} & 21\textendash{}25\\
  \end{tabular}
\end{table}

\begin{table}[H]
\centering
  \caption*{Капітал ІI}
  \begin{tabular}{r r@{~}c r@{~}c r@{~}c}
    \toprule
    & \multicolumn{2}{c}{Періоди обороту} & \multicolumn{2}{c}{Робочі періоди}
    & \multicolumn{2}{c}{Періоди обігу}\\
    \cmidrule(lr){2-3}
    \cmidrule(lr){4-5}
    \cmidrule(lr){6-7}

І.  & Тижні & \phantom{0}5\textendash{}13 & Тижні
    & \phantom{17~.~1}5\textendash{}8\phantom{0} 
    & Тижні & \phantom{0}9\textendash{}13\\
ІІ. & \ditto{Тижні} & 13\textendash{}21 & \ditto{Тижні}
    & 13~.~14\textendash{}16 & \ditto{Тижні} & 17\textendash{}21\\
III.& \ditto{Тижні} & 21\textendash{}29 & \ditto{Тижні}
    & 21~.~22\textendash{}24 & \ditto{Тижні} & 25\textendash{}29\\
  \end{tabular}
\end{table}

\begin{table}[H]
\centering
  \caption*{Капітал ІII}
  \begin{tabular}{r r@{~}c r@{~}c r@{~}c}
    \toprule
    & \multicolumn{2}{c}{Періоди обороту} & \multicolumn{2}{c}{Робочі періоди}
    & \multicolumn{2}{c}{Періоди обігу}\\
    \cmidrule(lr){2-3}
    \cmidrule(lr){4-5}
    \cmidrule(lr){6-7}

І.  & Тижні  & 9\textendash{}17 & Тижні
    & \samewidth{17~.~15\textendash{}8}{9} & Тижні & 10\textendash{}17\\
ІІ. & \ditto{Тижні} & 17\textendash{}25 & \ditto{Тижні}
    & 17 & \ditto{Тижні} & 18\textendash{}25\\

III.& \ditto{Тижні} & 25\textendash{}33 & \ditto{Тижні}
    & 25 & \ditto{Тижні} & 26\textendash{}33\\
  \end{tabular}
\end{table}

\noindent{}Переплітання капіталів тут є остільки, оскільки робочий період капіталу
III, що не має самостійного робочого періоду, бо його вистачає
тільки на один тиждень, збігається з першим робочим тижнем капіталу І.~Але зате наприкінці робочого періоду і капіталу І й капіталу II
звільняється рівна капіталові III сума в 100\pound{ ф. стерл}. А саме, коли капітал
III виповнює перший тиждень другого та всіх наступних робочих періодів
капіталу І, а наприкінці цього першого тижня назад припливає ввесь
капітал І, 400\pound{ ф. стерл.}, то для решти робочого періоду капіталу І лишається
тільки 3 тижні, і відповідна витрата капіталу буде 300\pound{ ф. стерл}. Звільнених
таким чином 100\pound{ ф. стерл.} буде досить для першого тижня безпосередньо
наступного робочого періоду капіталу II; наприкінці цього тижня повертається
назад увесь капітал II в 400\pound{ ф. стерл.}; а що розпочатий робочий
період може ввібрати ще тільки 300\pound{ ф. стерл.}, то наприкінці його лишаються
\index{ii}{0206}  %% посилання на сторінку оригінального видання
знову вільних 100\pound{ ф. стерл.} і так далі. Отже, постає звільнення капіталу
наприкінці робочого періоду, скоро час обігу не є просте кратне
робочого періоду; і цей вільний капітал дорівнює тій частині капіталу,
що повинна виповнювати час, який становить надлишок періоду циркуляції
проти робочого періоду або проти кратного робочих періодів.
В усіх досліджених випадках припускалося, що й робочий період і
час обігу протягом цілого року лишались однакові в кожному з розглянутих
тут підприємств. Таке припущення було потрібне, коли ми хотіли
встановити вплив часу обігу на оборот і на авансування капіталу. Той
факт, що в дійсності це припущення не здійснюється з такою безумовністю,
а іноді й зовсім не здійснюється, зовсім не змінює справи.

В цілому відділі цьому ми розглядали тільки обороти обігового капіталу,
а не основного. Проста причина цього та, що розглядуване питання
не має жодного чинення до основного капіталу. Вживані в процесі продукції
засоби праці тощо становлять основний капітал лише остільки,
оскільки час вживання їх триває довше, ніж період обороту поточного
капіталу; оскільки час, що протягом його ці засоби праці й далі служать
у постійно повторюваних процесах праці, більший, ніж період обороту
поточного капіталу, оскільки, отже, він дорівнює $n$ періодам обороту
поточного капіталу. Чи ввесь час, що його складають ці $n$ періодів обороту
поточного капіталу, буде довший, чи коротший, все одно частину
продуктивного капіталу, авансовану на цей час на основний капітал, не
авансуватиметься знову в межах цього часу. Вона й далі функціонує в
своїй старій споживній формі. Ріжниця лише ось в чому: відповідно до
різного протягу поодинокого \emph{робочого періоду}, в кожному періоді
обороту поточного капіталу основний капітал віддає більшу або меншу
частину своєї початкової вартости продуктові цього робочого періоду, і,
відповідно до довжини часу циркуляції в кожному періоді обороту, ця
перенесена на продукт частина вартости основного капіталу швидше або
повільніше припливає назад в грошовій формі. Природа предмету, що
його ми розглядаємо в цьому розділі — оборот обігової частини продуктивного
капіталу — випливає з самої природи цієї частини капіталу. Поточний
капітал, вжитий в одному робочому періоді, не можна вжити в
новому робочому періоді, поки не закінчить він свого обороту, поки не
перетвориться на товаровий капітал, з нього — на грошовий капітал, а
з цього останнього знову на продуктивний капітал. Отже, для того, щоб
за першим робочим періодом одразу починався другий, треба знову
авансувати капітал і перетворити його на поточні елементи продуктивного
капіталу, та ще й авансувати його в достатніх розмірах, щоб виповнити
прогалини, посталі в наслідок періоду циркуляції поточного
капіталу, авансованого на перший робочий період. Відси постає вплив
протягу робочого періоду поточного капіталу на маштаб процесу праці
в підприємстві й на поділ авансованого капіталу, зглядно й на розмір
додаткового авансування нових частин капіталу. А це саме й є те, що ми
повинні були дослідити в цьому відділі.

\parcont{}  %% абзац починається на попередній сторінці
\index{iii1}{0207}  %% посилання на сторінку оригінального видання
засоби існування, отже, без зміни вартості товарів, що входять у споживання робітника.

Абож змінюється відношення суми привласнюваної додаткової вартості до сукупного авансованого
капіталу суспільства. Через те що зміна походить тут не від норми додаткової вартості, вона мусить
походити від сукупного капіталу, а саме від його сталої частини. Маса цієї частини, розглядувана з
технічного боку, збільшується або зменшується пропорціонально до кількості робочої сили, купленої
змінним капіталом, а вартість цієї частини зростає або падає таким чином разом із зростанням чи
зменшенням самої її маси; отже, вона так само зростає або падає пропорціонально до маси вартості
змінного капіталу. Якщо та сама кількість праці приводить в рух більше сталого капіталу, то праця
стала продуктивнішою. У зворотному випадку — навпаки. Отже, сталася зміна в продуктивності праці, і
мусить відбутися зміна вартості певних товарів.

Отже, для обох випадків має силу такий закон: якщо змінюється ціна виробництва якогось товару в
наслідок зміни загальної норми зиску, то хоч власна вартість цього товару може лишитись незмінною,
проте мусить відбутися зміна вартості інших товарів.

\emph{Подруге}. Загальна норма зиску лишається незмінною. Тоді ціна виробництва товару може змінитися
тільки тому, що змінилась його власна вартість; що потрібно більше або менше праці для того, щоб
репродукувати самий товар, в наслідок зміни продуктивності або тієї праці, що виробляє даний товар у
його остаточній формі, або тієї, що виробляє товари, які входять у виробництво даного товару.
Бавовняна пряжа може упасти в ціні виробництва або тому, що дешевше виготовляється бавовна-сирець,
або тому, що праця прядіння в наслідок поліпшення машин стала продуктивнішою.

Ціна виробництва, як уже показано раніш, $= k \dplus{} p$, дорівнює витратам виробництва плюс зиск. Але це
$= k \dplus{} kp'$, де $k$, витрати виробництва, невизначена величина, яка для різних сфер виробництва змінюється
і повсюди дорівнює вартості сталого й змінного капіталу, спожитого на виробництво товару, а $p'$ є
обчислена в процентах пересічна норма зиску. Якщо $k \deq{} 200$, а $p' \deq{} 20\%$, то ціна виробництва $k \dplus{} kp' \deq{}
200 \dplus{} 200\cdot\frac{20}{100} \deq{} 200
\dplus{} 40 \deq{} 240$. Очевидно, що ця ціна виробництва може лишатися незмінною, хоч вартість товарів
змінюється.

Всякі зміни в ціні виробництва товарів зводяться в кінцевому рахунку до зміни вартості; але не всяка
зміна вартості товарів виражається в зміні ціни виробництва, бо ця остання визначається не тільки
вартістю даного товару, але й сукупною вартістю всіх товарів. Отже, зміна в товарі \emph{А} може бути
урівноважена протилежною зміною товару \emph{В}, так що загальне відношення лишається незмінним.

\parcont{}  %% абзац починається на попередній сторінці
\index{iii2}{0208}  %% посилання на сторінку оригінального видання
яка перетворюється на ренту, до авансованого капіталу, який продукує продукт
землі. Це відношення відрізняється від відношення додаткового продукту до
всього продукту, бо весь продукт має в собі не весь авансований капітал, саме
не має в собі основного капіталу, який продовжує існувати поряд з продуктом.
Навпаки, воно припускає, що на тих родах землі, які дають диференційну
ренту, дедалі ростуча частина продукту перетворюється в надмірний  надпродукт.
На найгіршій землі підвищення ціни хліборобського продукту вперше створює
ренту, а тому і ціну землі.

Але рента може зростати і без підвищення ціни хліборобського продукту.
Остання може лишитися сталою або навіть понизитися.

Коли вона лишається сталою, то рента може зрости або тільки тому
(залишаючи осторонь монопольні ціни), що при колишньому розмірі капіталу,
вкладеного у старі землі, починають оброблятись нові землі кращої якости, але
їх лише вистачає на те, щоб покрити вирослий попит, так що реґуляційна
ринкова ціна залишається без зміни. В цьому випадку ціна старих земель не
підвищується, але для землі, наново взятої під оброблення, ціна підвищується
понад рівень ціни старої землі.

Або ж рента підвищується тому, що при незмінній відносній продуктивності
і незмінній ринковій ціні зростає маса капіталу, що експлуатує землю.
Тому, хоч рента у відношенні до авансованого капіталу лишається та сама, її
маса, наприклад, подвоюється, бо сам капітал подвоївся. А щоб не сталося
пониження ціни, то друге приміщення капіталу дає, так само як і перше,
надзиск, який по закінчені терміну оренди теж перетворюється на ренту. Маса
ренти тут збільшується тому, що збільшується маса капіталу, який створює
ренту. Твердження, що різні послідовні приміщення капіталу на тій самій дільниці
землі можуть створити ренту лише тоді, коли продукт їхній неоднаковий
і тому постає диференційна рента, сходить на твердження, що, коли два капітали
по \num{1.000}\pound{ ф. стерл.}, вкладено в два лани однакової продуктивности, то
лише один з них може дати ренту, хоч обидва ці лани належать до кращої
кляси землі, яка дає диференційну ренту. (Отже, загальна маса ренти, вся
рента певної країни, збільшується з масою вкладеного капіталу, при чому
необов’язково, щоб тут зростала ціна одиниці земельної площі, або норма
ренти, або навіть маса ренти на одиницю площі; в цьому випадку маса
всієї ренти зростає з просторовим поширенням культури. Це може навіть
бути поєднане з падінням ренти на окремих володіннях). Інакше це твердження
звелося б до другого твердження, а саме, що приміщення капіталуодне
поряд одного у дві різні дільниці землі підлягає іншим законам, ніж послідовне
приміщення капіталу на тій самій дільниці землі, тимчасом як в дійсності
диференційну ренту висновують саме з тотожності закону в обох випадках,
з приросту продуктивности приміщення капіталу на тім самім лані, як і на
різних ланах. Єдина модифікація, що існує тут, і якої не помічають, є в тому,
що послідовні приміщення капіталів, коли їх вживають до просторово різних
земель, наражаються на таку межу, як земельна власність, тим часом як при послідовних
приміщеннях капіталу в ту саму землю цього не буває. Звідси і та
протилежна дія, в наслідок якої ці різні форми приміщення капіталу на практиці
взаємно обмежують одна одну. Тут ніколи не постає ріжниці з самого капіталу.
Коли склад капіталу лишається той самий, так само, як норма додаткової
вартости, то норма зиску лишається незмінна, так що при подвоєнні
капіталу маса зиску подвоюється. Так само за припущених відношень норма
ренти лишається та сама. Коли капітал в \num{1.000}\pound{ ф. стерл.} дає ренту в х, то
капітал в \num{2.000}\pound{ ф. стерл.} за припущених обставин дає ренту в 2х. Але, коли
обчислити ренту у відношенні до земельної площі, яка лишилася без зміни, бо,
згідно з припущенням, подвоєний капітал працює на тому самому лані, то
\parbreak{}  %% абзац продовжується на наступній сторінці

\parcont{}  %% абзац починається на попередній сторінці
\index{ii}{0209}  %% посилання на сторінку оригінального видання
затримає їх цілком або почасти як грошовий капітал. З другого боку,
само собою зрозуміло, що частину, витрачувану на заробітну плату,
рівну 200\pound{ ф. стерл.}, затримується в грошовій формі. Капіталіст, купивши
робочу силу, не може складати її на складах як сировинний матеріял.
Він мусить ввести її в процес продукції та оплачує її наприкінці тижня.
Отже, із звільненого капіталу в 300\pound{ ф. стерл.} в усякому разі ці 100\pound{ ф.
стерл.} матимуть форму звільненого грошового капіталу, тобто непотрібного
для робочого періоду. Отже, капітал, що звільнився в формі грошового
капіталу, мусить дорівнювати принаймні змінній частині капіталу,
витраченій на заробітну плату; в максимумі цей грошовий капітал може
досягти суми всього звільненого капіталу. А в дійсності величина його
постійно коливається між цим мінімумом і максимумом.

Грошовий капітал, що звільнився таким чином у наслідок самого
лише механізму руху оборотів (поряд грошового капіталу, що утворюється
через послідовний зворотний приплив основного капіталу, і поряд грошового
капіталу, потрібного в кожному процесі праці для капіталу змінного),
мусить відігравати чималу ролю, скоро тільки розвивається кредитова
система, і разом з тим мусить бути за одну з основ її.

Припустімо, що в нашому прикладі час циркуляції скорочується
з 3 тижнів до 2. Це не нормальне явище, а лише наслідок сприятливого
моменту для підприємства, скорочених термінів виплат тощо. Капітал
в 600\pound{ ф. стерл.}, витрачений протягом робочого періоду, повертається на
тиждень раніше, ніж треба, отже, він звільняється на цей тиждень. Далі
в середині робочого періоду, як і раніше, звільняється 300\pound{ ф. стерл.}
(частина тих 600\pound{ ф. стерл.}), але звільняється на 4 тижні замість 3.
Отже, на грошовому ринку протягом одного тижня перебуває 600\pound{ ф.
стерл.}, і 300\pound{ ф. стерл.} перебувають протягом 4 тижнів замість 3. А що це
стосується не до одного лише капіталіста, а до багатьох, і відбувається по
різних галузях підприємств, в різні періоди, то в наслідок цього на ринку
стає більше вільного грошового капіталу. Коли такий стан триває порівняно
довго, то продукція поширюється, там, де це можливо; капіталісти,
що роблять позиченим капіталом, ставитимуть менший попит на грошовому
ринку, а це полегшує стан грошового ринку так само, як збільшене
подання; або, нарешті, суми, що стали надлишковими для механізму
обороту, остаточно викинеться на грошовий ринок.

В наслідок скорочення часу циркуляції\footnote*{
В нім. тексті, очевидно, помилково стоїть „час обороту“. \Red{Ред.}
} з 3 до 2 тижнів, а тому й
періоду обороту з 9 до 8 тижнів, \sfrac{1}{9} цілого авансованого капіталу
стає надлишковою; шеститижневий робочий період може тепер так само
безперервно перебігати при 800\pound{ ф. стерл.}, як раніш при 900\pound{ ф. стерл}.
Тому частина вартости товарового капіталу, рівна 100\pound{ ф. стерл.}, зворотно
перетворившись на гроші, залишається в цьому стані, як грошовий
капітал, не функціонуючи далі як частина капіталу, авансованого на
процес продукції. Тимчасом як продукцію й далі провадиться в попередніх
розмірах і в інших незмінних умовах, як от щодо цін та ін.,
\parbreak{}  %% абзац продовжується на наступній сторінці

\parcont{}  %% абзац починається на попередній сторінці
\index{ii}{0210}  %% посилання на сторінку оригінального видання
сума вартости авансованого капіталу зменшується з 900\pound{ ф. стерл.} до
800\pound{ ф. стерл.}; решта 100\pound{ ф. стерл.} первісно авансованої вартости виділяється
в формі грошового капіталу. Як такий, він надходить на грошовий
ринок і утворює додаткову частину діющого тут капіталу.

З цього видно, як може постати плетора (plethora)\footnote*{
Грецьке слово, що йому відповідає німецьке „Überfülle“,
„Überfluss“, або українське „повнява“, „повня“, „багатість“. \emph{Ред.}
} грошового капіталу
і не тільки в тому розумінні, що подання грошового капіталу вище
за попит на нього; це завжди є лише відносна плетора, що, приміром,
постає в „меланхолійному періоді“, який починає новий цикл по
закінченні кризи. Це плетора грошового капіталу в тому розумінні, що
для провадження сукупного суспільного процесу репродукції (який включає
і процес циркуляції) певна частина авансованої капітальної вартости
є зайва, а тому вона й виділяється в формі грошового капіталу; це плетора,
що постає при незмінному маштабі продукції та незмінних цінах,
виключно в наслідок скорочення періоду обороту. Більша чи менша маса
грошей, що перебуває в циркуляції, не справляє на це жодного впливу.

Припустімо, навпаки, що період циркуляції більшає, напр., з трьох
тижнів до п’яти. Тоді вже при наступному обороті авансований капітал
припливає назад на два тижні пізніше. Останню частину процесу продукції
цього робочого періоду не можна провадити далі механізмом обороту
самого авансованого капіталу. Коли такий стан триває порівняно довго,
то може постати скорочення процесу продукції — того розміру, що в
ньому провадять його — так само, як в попередньому випадку постало
його поширення. Але щоб продовжувати процес у попередньому розмірі,
авансований капітал треба збільшити на \sfrac{2}{9} його розміру, \deq{} 200\pound{ ф.
стерл.}, на весь час подовження періоду циркуляції. Цей додатковий капітал
можна взяти лише з грошового ринку. Коли подовження періоду
циркуляції поширюється на одну або кілька великих галузей підприємств,
то воно може в наслідок цього справити тиск на грошовий ринок, якщо
тільки цей вплив не паралізується протилежним впливом з другого боку.
І в цьому випадку ясно й очевидно, що цей тиск, як і раніш та плетора,
не має жодного чинення ні до зміни товарових цін, ні до зміни маси
наявних засобів циркуляції.

[Виготовити до друку цей розділ становило чималі труднощі. Хоч і
добре знав Маркс альґебру, але в аритметичних обчисленнях, особливо
торгових, не був він швидкий (ungeläufig), не зважаючи на те, що є грубий
пак зшитків, де він сам на багатьох прикладах проробив усі види торгового
рахівництва. Але знання окремих видів рахівництва і вправи в
щоденному практичному рахівництві купця зовсім не те саме, і тому
Маркс заплутався в обчисленні оборотів, так що поряд незакінчености
постали деякі неправильності й суперечності. В наведених вище таблицях
я залишив тільки найпростіше і аритметично правильне, головним чином,
з таких міркувань.

Неправильні результати цих клопітних обчислень спричинили те, що
\parbreak{}  %% абзац продовжується на наступній сторінці

\input{ii/_0211.tex}

\index{iii1}{0212}  %% посилання на сторінку оригінального видання

\chapter{Закон тенденції норми зиску до падіння}

\section{Закон як такий}

При даній заробітній платі і при даному робочому дні змінний
капітал, наприклад, в 100, представляє певне число приведених
у рух робітників; він є показник цього числа. Припустімо,
наприклад, що 100\pound{ фунтів стерлінгів} становлять заробітку плату
100 робітників, скажімо, за 1 тиждень. Якщо ці 100 робітників
виконують стільки ж необхідної праці, скільки додаткової праці,
якщо вони, отже, щодня працюють стільки ж часу на себе
самих, тобто для репродукції своєї заробітної плати, скільки
на капіталістів, тобто для виробництва додаткової вартості, то
вся вироблена ними вартість буде = 200\pound{ фунтам стерлінгів},
а вироблена ними додаткова вартість становитиме 100\pound{ фунтів
стерлінгів}. Норма додаткової вартості \frac{m}{v} була б = 100\%. Однак,
ця норма додаткової вартості, як ми бачили, виражалася б у дуже
різних нормах зиску, залежно від різного розміру сталого капіталу
$c$, а тому й усього капіталу $K$, бо норма зиску $ = \frac{m}{K}$. При нормі
додаткової вартості в 100\%,

\begin{center}
якщо $c = \phantom{0}50$, $v = 100$, то $р' = \frac{100}{150} = 66\frac{2}{3}\%$;

якщо $c = 100$, $v = 100$, то $р' = \frac{100}{200} = 50\phantom{\frac{1}{1}}\%$;

якщо $c = 200$, $v = 100$, то $р' = \frac{100}{300} = 33\frac{1}{3}\%$;

якщо $c = 300$, $v = 100$, то $р' = \frac{100}{400} = 25\phantom{\frac{1}{1}}\%$;

якщо $c = 400$, $v = 100$, то $р' = \frac{100}{500} = 20\phantom{\frac{1}{1}}\%$.
\end{center}

Таким чином при незмінному ступені експлуатації праці та
сама норма додаткової вартості виражалася б у падаючій нормі
зиску, бо разом з матеріальним розміром сталого капіталу зростає,
\index{iii1}{0213}  %% посилання на сторінку оригінального видання
хоч і не в тій самій пропорції, і розмір вартості сталого,
а разом з ним і всього капіталу.

Якщо ми далі припустимо, що ця ступнева зміна в складі
капіталу відбувається не тільки в окремих сферах виробництва,
але більш-менш в усіх або, принаймні, у вирішальних сферах
виробництва, так що вона таким чином рівнозначна зміні в пересічному
органічному складі сукупного капіталу, належного певному
суспільству, то таке ступневе наростання сталого капіталу
порівняно з змінним неминуче мусить мати своїм результатом
\emph{ступневе зниження загальної норми зиску} при незмінній нормі
додаткової вартості, або при незмінному ступені експлуатації
праці капіталом. Але виявилось, як закон капіталістичного способу
виробництва, що з розвитком цього способу виробництва
відбувається відносне зменшення змінного капіталу порівняно
з сталим капіталом і, отже, порівняно з усім капіталом, який
приводиться в рух. Це означає тільки те, що те саме число
робітників, та сама кількість робочої сили, якою можна розпоряджатися
при змінному капіталі даного розміру вартості, в наслідок
особливих методів виробництва, що розвиваються в капіталістичному
виробництві, за той самий час приводить в рух,
переробляє, продуктивно споживає постійно зростаючу масу
засобів праці, машин і всякого роду основного капіталу, сировинних
і допоміжних матеріалів, отже і сталий капітал постійно
зростаючого розміру вартості. Це прогресуюче відносне зменшення
змінного капіталу порівняно з сталим і, отже, з усім капіталом,
тотожне з дедалі вищим пересічним органічним складом
суспільного капіталу. Це — так само тільки інший вираз
прогресуючого розвитку суспільної продуктивної сили праці,
який виявляється саме в тому, що за допомогою зростаючого
застосування машин і взагалі основного капіталу при тому самому
числі робітників за той самий час, тобто з меншою кількістю
праці, перетворюється в продукти більша кількість сировинних
і допоміжних матеріалів. Цьому зростаючому розмірові вартості
сталого капіталу — хоч він тільки віддалено представляє зростання
дійсної маси споживних вартостей, з яких речево складається
сталий капітал — відповідає зростаюче здешевлення продукту.
Кожний індивідуальний продукт, розглядуваний сам по
собі, містить у собі меншу суму праці, ніж на нижчому ступені виробництва,
де відношення капіталу, витраченого на працю, до
капіталу, витраченого на засоби виробництва, є незрівняно
більша величина. Отже, гіпотетичний ряд, наведений нами на
початку цього розділу, виражає дійсну тенденцію капіталістичного
виробництва. Це останнє разом з прогресуючим відносним
зменшенням змінного капіталу порівняно з сталим створює дедалі
вищий органічний склад сукупного капіталу, безпосереднім
наслідком чого є те, що норма додаткової вартості при незмінному
і навіть при зростаючому ступені експлуатації праці
виражається в дедалі нижчій загальній нормі зиску. (Далі буде
\parbreak{}  %% абзац продовжується на наступній сторінці

\input{ii/_0214_0215c.tex}
\parcont{}  %% абзац починається на попередній сторінці
\index{ii}{0216}  %% посилання на сторінку оригінального видання
часу обігу, а разом з тим і часу обороту, виділюється в формі грошового
капіталу \sfrac{1}{9} частина авансованого капіталу \deq{} 100\pound{ ф. стерл.} і коли
ці 100\pound{ ф. стерл.} складаються з 20\pound{ ф. стерл.} періодично надлишкового
грошового капіталу, призначеного для виплати щотижневої заробітної
плати, і з 80\pound{ ф. стерл.}, що існують як періодичний надлишковий тижневий
продукційний запас, — то цьому зменшенню у фабриканта надлишкового
продукційного запасу на 80\pound{ ф. стерл.} відповідає збільшення товарового
запасу у торговця бавовною. Та сама бавовна то довше лежить
на його складах як товар, що менше лежить вона на складах у фабриканта
як продукційний запас.

Досі ми припускали, що скорочення часу обігу в підприємстві $X$ випливає
з того, що $X$ швидше продає свої товари або швидше одержує
за них гроші, зглядно, що при кредиті термін виплати скорочується.
Отже, це скорочення часу обігу випливає з швидкого продажу товарів,
швидкого перетворення товарового капіталу на грошовий, з $Т' — Г'$, з
першої фази процесу циркуляції. Воно могло б випливати й з другої фази,
$Г — Т$, а тому й з одночасної зміни, чи то робочого періоду, чи то часу
обігу капіталів $Y$, $Z$, і~\abbr{т. ін.}, що постачають капіталістові $X$ продукційні
елементи його поточного капіталу.

Коли, напр., бавовна, вугілля та ін., в старих умовах транспорту перебувають
8 тижні в дорозі від місця продукції або від складів до місця
підприємства капіталіста $X$, то мінімуму продукційного запасу $X$ мусить
вистачати принаймні на 3 тижні, поки надійдуть нові запаси. Поки
бавовна та вугілля перебувають в дорозі, вони не можуть служити як
засоби продукції. Вони скоріше становлять тоді предмет праці для транспортової
промисловости й приміщеного в ній капіталу, а також товаровий
капітал для вуглепродуцента або для продавця бавовни, товаровий капітал,
що перебуває в своїй циркуляції. При поліпшеному транспорті час
перевозу скорочується до 2 тижнів. Таким чином, продукційний запас може
перетворитися з тритижневого на двотижневий. Разом з тим звільняється
авансований на це додатковий капітал у 80\pound{ ф. стерл.}, а також 20\pound{ ф. стерл.}, призначені на заробітну плату, бо капітал у 600\pound{ ф. стерл.},
що обернувся, повертається на тиждень раніше.

З другого боку, коли, напр., робочий період капіталу, що постачає
сировинний матеріял, скорочується (приклади про це подано в попередніх
розділах), отже, зростає й можливість відновлювати сировинний матеріял,
то продукційний запас може зменшитись, переміжок від одного періоду
відновлення до другого може скоротитись.

Навпаки, коли час обігу, а тому й період обороту довшає, то потрібне
авансування додаткового капіталу — з кишені самого капіталіста,
коли в нього є додатковий капітал. Але цей капітал є в тій
або іншій формі приміщений, як частина грошового ринку; щоб ним
можна було порядкувати, його треба визволити з старої форми, напр.,
продати акції, взяти вклади, так що й тут постає посередній вплив на
грошовий ринок. Або капіталіст мусить десь позичити додатковий капітал.
Щождо частини додаткового капіталу, потрібної для заробітної плати, то
\parbreak{}  %% абзац продовжується на наступній сторінці

\input{ii/_0217c.tex}
\parcont{}  %% абзац починається на попередній сторінці
\index{ii}{0218}  %% посилання на сторінку оригінального видання
продукт, та якщо цей продукт знову входить як елемент продукції у
другу галузь продукції і pro tanto звільняє тут капітал. В обох випадках
капітал, втрачений для $X$ — для заміщення якого $X$ справляє тиск на грошовий
ринок — можуть дати йому товариші в ділових справах як новий
додатковий капітал. В такому разі відбувається лише переміщення.

Навпаки, коли ціна продукту підвищується, то з сфери циркуляції
привлащується частину капіталу, що її не авансовано. Вона не є
органічна частина капіталу, авансованого на процес продукції, а тому,
коли підприємство не поширюється, вона становить виділений капітал.
А що тут припущено, що ціни елементів продукту дано раніш, ніж він
як товаровий капітал вступив на ринок, то до підвищення цін тут могла б
спричинитись справжня зміна вартости, оскільки ця зміна вартости мала б
зворотний вплив, напр., коли б сировинні матеріяли в дальшому
подорожчали. В цьому разі капіталіст $X$ виграв би на своєму продукті,
що циркулює як товаровий капітал, і на продукційному запасі, що є в
нього. Цей виграш дав би йому додатковий капітал, тепер потрібний для
того, щоб провадити далі підприємство при нових підвищених цінах елементів
продукції.

Або підвищення цін є лише тимчасове. Тоді те, що на боці капіталіста
$X$ потрібне як додатковий капітал, виступає на боці другого капіталіста
як звільнений капітал, оскільки його продукт є елемент продукції
для інших галузей підприємств. Що один втратив, те інший виграв.

\sectionextended{Оборот змінного капіталу}{\subsection{Річна норма додаткової вартости}}

Припустімо обіговий капітал в 2500\pound{ ф. стерл.}, а саме \sfrac{4}{5} \deq{} 2000\pound{ ф.
стерл.} сталого капіталу (матеріяли продукції) і \sfrac{1}{5} \deq{} 500\pound{ ф. стерл.} змінного
капіталу, витрачуваного на заробітну плату.

Період обороту хай дорівнює 5 тижням; робочий період \deq{} 4 тижням;
період циркуляції \deq{} 1 тижневі. Тоді капітал І \deq{} 2000\pound{ ф. стерл.} і складається
з 1600\pound{ ф. стерл.} сталого капіталу і 400\pound{ ф. стерл.} змінного; капітал
ІІ \deq{} 500\pound{ ф. стерл.}, з них 400\pound{ ф. стерл.} сталого капіталу і 100\pound{ ф.
стерл.} змінного капіталу. Протягом кожного робочого тижня витрачається
капітал в 500\pound{ ф. стерл}. Протягом року, що складається з 50 тижнів,
виготовлюється річний продукт в 500 × 50 \deq{} \num{25.000}\pound{ ф. стерл}. Отже, капітал
І в 2000\pound{ ф. стерл.}, що весь час застосовується в робочому періоді,
обертається 12,5 разів. 2000 × 12,5 \deq{} \num{25.000}\pound{ ф. стерл}. З цих \num{25.000}\pound{ ф.
стерл.} \sfrac{4}{5} \deq{} \num{20.000}\pound{ ф. стерл.} сталого капіталу, витраченого на засоби
продукції, і \sfrac{1}{5} \deq{} 5000\pound{ ф. стерл.} змінного капіталу, витраченого на за
\index{ii}{0219}  %% посилання на сторінку оригінального видання
робітну плату. Навпаки, увесь капітал в 2500\pound{ ф. стерл.} обертається
\frac{\num{25.000}}{2500} \deq{} 10 разів.

Витрачений протягом продукції змінний обіговий капітал може знову
функціонувати в процесі продукції лише остільки, оскільки продукт, що
в ньому репродуковано його вартість, продано, перетворено з товарового
капіталу на грошовий капітал, щоб знову витрачуватись на оплату робочої
сили. Але так само стоїть справа і з витраченим на продукцію сталим
обіговим капіталом (матеріялами продукції), що його вартість знову з’являється
в продукції як частина вартости продукту. Що ці обидві частини —
змінна та стала частина обігового капіталу — мають спільного, і що відрізняє
їх від основного капіталу, так це не те, що їхня вартість, перенесена
на продукт, циркулює за допомогою товарового капіталу, тобто
через циркуляцію продукту як товару. Деяка частина вартости продукту,
а тому й продукту, що циркулює як товар, тобто деяка частина товарового
капіталу завжди складається з зношуваної частини основного капіталу,
тобто з частини вартости основного капіталу, перенесеної на продукт
в процесі продукції. Але ріжниця ось у чому: основний капітал і
далі функціонує в процесі продукції в своїй старій споживній формі протягом
більш-менш довгого циклу періодів обороту обігового капіталу
(\deq{} обіговому сталому \dplus{} обіговий змінний капітал); тимчасом як кожен
поодинокий оборот має собі за умову заміщення цілого обігового капіталу,
що ввійшов — у вигляді товарового капіталу — із сфери продукції в
сферу циркуляції. Перша фаза циркуляції $Т' — Г'$ є спільна для поточного
сталого й поточного змінного капіталу. В другій фазі вони відокремлюються.
Гроші, що на них перетворився товар, перетворюються деякою
частиною на продукційний запас (обіговий сталий капітал). Відповідно до
різних термінів купівель складових частин цього запасу одна частина
його може утворитись через перетворення грошей на матеріяли продукції
раніше, друга пізніше, але, кінець-кінцем, так складається ввесь запас продукційних
матеріялів. Друга частина грошей, одержаних з продажу товарів,
лишається лежати як грошовий запас, щоб помалу витрачатись на
оплату робочої сили, введеної в процес продукції. Вона становить обіговий
змінний капітал. А проте, повне заміщення тієї або другої частини капіталу
є кожного разу наслідок обороту капіталу, його перетворення на продукт,
з продукту на товар, з товару на гроші. Саме це є причина того,
що в попередньому розділі ми розглядали оборот обігового капіталу — сталого
й змінного — окремо і разом, не звертаючи уваги на основний капітал.

В питанні, що його нам треба тепер дослідити, ми мусимо зробити ще
один крок далі й розглядати змінну частину обігового капіталу так, ніби
тільки вона одна й становить обіговий капітал; інакше кажучи, ми залишаємо
осторонь сталий обіговий капітал, що обертається разом з нею.

Авансовано 2500\pound{ ф. стерл.}, і вартість річного продукту \deq{} \num{25.000}\pound{ ф.
стерл}. Але змінна частина обігового капіталу становить 500\pound{ ф. стерл.},
тому змінний капітал, що міститься в \num{25.000}\pound{фунт, стерлінґів}, дорівнює
\parbreak{}  %% абзац продовжується на наступній сторінці

\parcont{}  %% абзац починається на попередній сторінці
\index{iii2}{0220}  %% посилання на сторінку оригінального видання
земельна рента є нормальною формою додаткової вартости, а тому і додаткової
праці, тобто всієї надмірної праці, яку безпосередній продуцент мусить даром,
отже, на ділі примусово, виконувати на власника найістотнішої умови його
праці, на власника землі, — хоч цей примус уже не протистоїть йому в старій
брутальній формі. Зиск, — коли ми, фалшиво антиципуючи, назвемо так той
дріб надміру його праці над потрібного працею, що він його привласнює самому
собі, — до такої міри не має визначального впливу на ренту продуктами, що
радше можна було б сказати, що він виростає за спиною останньої і має свою
природну межу в розмірі ренти продуктами. Остання може досягати такого
розміру, що є поважною загрозою репродукції умов праці, самих засобів продукції,
більш або менш унеможливлює поширення продукції і знижує задоволення
потреб безпосереднього продуцента до фізичного мінімуму засобів
існування. Так буває саме в тому випадку, коли цю форму знаходить готового
і починає експлуатувати торговельна нація-завойовник, як, наприклад, англійці
в Індії.

\subsubsection{Грошова рента}

Під грошовою рентою ми розуміємо тут — на відзнаку від промислової
або комерційної земельної ренти, що ґрунтується на капіталістичному способі
продукції і становить лише надмір над пересічним зиском, — земельну ренту,
що виникає з простого перетворення форми ренти продуктами, так само, як ця
остання сама була лише перетвореною відробітною рентою. Замість продукту
безпосередній продуцент має тут виплачувати власникові землі (чи то буде
держава, чи приватна особа) ціну продукту. Отже, надміру продукту в його
натуральній формі вже не досить, його мусять перетворити з цієї натуральної
форми на грошову форму. Хоч безпосередній продуцент, як і давніш, продовжує
продукувати сам, принаймні, більшу частину своїх засобів існування, проте,
частина його продукту мусить тепер бути перетворена на товар, продукуватися
як товар. Отже, характер всього способу продукції більш або менш змінюється.
Він втрачає свою незалежність, свою відокремленість від зв'язку з суспільством.
Відношення витрат продукції, в які тепер входять в більшій чи меншій мірі і
грошові витрати, стає за вирішальне; в усякому разі стає вирішальним
надмір тієї частини гуртового продукту, що її треба перетворити на гроші
над тією частиною, яка, з одного боку, мусить стати знову засобом репродукції
і, з другого боку, безпосереднім засобом існування. А проте, база цього
роду ренти, хоч і наближається до свого розпаду, все ще лишається та сама,
що і при ренті продуктами, яка становить вихідний пункт. Безпосередній продуцент
є, як і давніш, спадковий або інакше традиційний посідач землі, який
повинен виплачувати земельному власникові, як власникові цієї найістотнішої
умови його продукції, надмірну примусову працю, тобто неоплачену, виконувану
без еквівалента працю, в формі додаткового продукту, перетвореного на
гроші. Власність на умови праці, відмінні від землі, хліборобське знаряддя та
інше рухоме майно спочатку фактично, а потім й юридично, перетворюється на
власність безпосередніх продуцентів вже за попередніх форм, і ще більше доводиться
припускати це для такої форми, як грошова рента. Спочатку спорадичне,
потім відбуваючись більш або менш у національному маштабі, перетворення
ренти продуктами на грошову ренту, має своєю передумовою вже порівняно
значний розвиток торгівлі, міської промисловости товарової продукції взагалі, а
разом з тим і грошової циркуляції. Далі воно має своєю передумовою ринкову
ціну продуктів, і те, що вони продаються більш або менш близько до
своєї вартости, чого може і не бути за колишніх форм. На Сході Европи ми
можемо почасти ще на власні очі спостерігати процес цього перетворення.
\parbreak{}  %% абзац продовжується на наступній сторінці

\parcont{}  %% абзац починається на попередній сторінці
\index{iii1}{0221}  %% посилання на сторінку оригінального видання
виміряння додаткової вартості, — а це робиться при всякому
обчисленні зиску, — то взагалі відносне падіння додаткової вартості
і її абсолютне падіння є тотожні. Норма зиску в наведених
вище випадках знижується з 40\% до 30\% і до 20\%, бо в дійсності
маса додаткової вартості, а тому й зиску, вироблена тим
самим капіталом, падає абсолютно з 40 до 30 і до 20. Через те
що величина вартості капіталу, відносно якої вимірюється додаткова
вартість, є дана, = 100, то зменшення відношення додаткової
вартості до цієї незмінної величини може бути тільки іншим
виразом зменшення абсолютної величини додаткової вартості
й зиску. Справді, це — тавтологія. Але те, що таке зменшення
настає, випливає, як уже було показано, з природи розвитку
капіталістичного процесу виробництва.

Але, з другого боку, ті самі причини, які викликають абсолютне
зменшення додаткової вартості, а тому й зиску на даний
капітал, а тому також і обчислюваної в процентах норми зиску,
ці самі причини приводять до зростання привласнюваної суспільним
капіталом (тобто сукупністю капіталістів) абсолютної маси
додаткової вартості, а тому й зиску. Як же це мусить виразитись,
як це може виразитись, або які умови передбачаються
і цією позірною суперечністю?

Якщо кожна відповідна частина, = 100, суспільного капіталу,
отже, кожні 100 капіталу пересічного суспільного складу, є величина
дана, і тому для неї зменшення норми зиску збігається
із зменшенням абсолютної величини зиску саме через те, що
тут капітал, яким вони вимірюються, є величина стала, то,
навпаки, величина сукупного суспільного капіталу, як і капіталу,
який знаходиться в руках окремих капіталістів, є змінна
величина, яка, щоб відповідати припущеним умовам, мусить
змінюватись у зворотному відношенні до зменшення своєї змінної
частини.

В попередньому прикладі, при процентному складі капіталу
в $60c + 40v$, додаткова вартість або зиск на капітал був 40,
а тому й норма зиску була 40\%. Припустім, що при цій висоті
складу сукупний капітал становив один мільйон. В такому разі
сукупна додаткова вартість, а тому й сукупний зиск становив
\num{400000}. Якщо потім склад буде = $80c + 20v$, то при незмінному
ступені експлуатації праці додаткова вартість, або зиск, на кожні
100 = 20. Але через те що додаткова вартість, або зиск, як ми
показали, щодо своєї абсолютної маси зростає, незважаючи на цю
падаючу норму зиску або дедалі менше створення додаткової
вартості кожною сотнею капіталу, — наприклад, зростає, скажімо,
з \num{400000} до \num{440000}, — то це можливе тільки тому, що сукупний
капітал, який утворився одночасно з цим новим складом, зріс
до \num{2200000}. Маса приведеного в рух сукупного капіталу зросла
до 220\%, тимчасом як норма зиску впала на 50\%. Коли б капітал
тільки подвоївся, то при нормі зиску в 20\% він міг би
виробити тільки таку саму масу додаткової вартості й зиску,
\parbreak{}  %% абзац продовжується на наступній сторінці

\parcont{}  %% абзац починається на попередній сторінці
\index{ii}{0222}  %% посилання на сторінку оригінального видання
все ж в річній нормі додаткової вартости капіталів $А$ і $В$ є ріжниця в
900\%.

Правда, це явище має такий вигляд, ніби норма додаткової вартости
залежить не лише від маси та ступеня експлуатації робочої сили,
пущеної в рух змінним капіталом, але крім того, від якихось незрозумілих
впливів, що походять з процесу циркуляції; і справді, це явище
освітлювали саме таким способом; і хоч не в цій чистій, а в своїй складнішій
та прихованішій формі (в формі річної норми зиску) воно спричинило
початку 20-х років цілковите замішання в школі Рікардо.

Дивне в цьому явищі зникає одразу, скоро ми не лише позірно, а
й справді поставимо капітал $А$ і капітал $В$ в цілком однакові обставини.
Обставини будуть однакові тільки тоді, коли змінний капітал $В$ в цілому
своєму об’ємі витрачається на оплату робочої сили протягом того
самого часу, що й капітал $А$.

5000\pound{ ф. стерл.} капіталу $В$ витрачається тоді протягом 5 тижнів, по
1000\pound{ ф. стерл.} щотижня; це становить за рік витрату в \num{50.000}\pound{ ф. стерл}.
Додаткова вартість буде тоді, згідно з нашим припущенням, теж \deq{} \num{50.000}\pound{ ф. стерл}. Капітал, що обернувся \deq{} \num{50.000}\pound{ ф. стерл.}, поділений на авансований
капітал \deq{} 5000\pound{ ф. стерл.}, дає число оборотів \deq{} 10. Норма додаткової
вартости \deq{} \frac{5000 m}{5000 v} \deq{} 100\%, помножена на число оборотів \deq{} 10, дає річну норму додаткової
вартости \deq{} \frac{5000 m}{5000 v} \deq{} \frac{10}{1} \deq{} 1000\%. Отже, тепер річні норми додаткової вартости однакові для
$А$ і для $В$, а саме
1000\%, але маси додаткової вартости становлять: для $В$ — \num{50.000}\pound{ ф. стерл.},
для $А$ — 5000\pound{ ф. стерл.}; маси спродукованої додаткової вартости відносяться
тепер, як авансовані капітальні вартості $В$ і $А$, а саме як
$5000 : 500 \deq{} 10 : 1$. Але зате капітал $В$ в той самий час пустив у рух удесятеро
більше робочої сили, ніж капітал $А$.

Тільки капітал, дійсно застосований у процесі праці, утворює додаткову
вартість і тільки для нього мають силу всі закони, що стосуються
до додаткової вартости, а значить, і той закон, що за даної норми маса
додаткової вартости визначається відносною величиною змінного капіталу.

Самий процес праці вимірюється часом. За даної довжини робочого
дня (як тут, де ми ставимо капітал $А$ і капітал $В$ в однакові обставини,
щоб висвітлити краще різницю в річній нормі додаткової вартости)
робочий тиждень складається з певного числа робочих днів. Або ми
можемо розглядати якийбудь робочий період, напр., в даному разі п’ятитижневий,
як суцільний робочий день, що складається з 300 годин, коли
робочий день \deq{} 10 годинам, а тиждень \deq{} 6 робочим дням. Але далі ми
мусимо помножити це число на число робітників, що їх одночасно щодня
вживається разом у тому самому процесі праці. Коли це число
було б, напр., 10, то тижневий підсумок був би \deq{} 60 × 10 \deq{} 600 годинам,
а п’ятитижневий робочий період \deq{} 600 × 5 \deq{} 3000 годинам. Отже,
при однаковій нормі додаткової вартости і при однаковій довжині робочого
\index{ii}{0223}  %% посилання на сторінку оригінального видання
дня застосовується змінні капітали однакової величини, коли протягом
того самого переміжку часу пускається в рух однакові маси робочої
сили (обчислювані помноженням однієї робочої сили тієї самої
ціни на число цих сил).

Повернімось тепер до наших первісних прикладів. В обох випадках
$А$ і $В$ протягом кожного тижня року застосовується змінні капітали
однакової величини, по 100\pound{ ф. ст.} щотижня. Застосовані, справді діющі
в процесі праці змінні капітали тому однакові, але авансовані змінні
капітали зовсім неоднакові. В прикладі $А$ на кожні 5 тижнів авансовано по
500\pound{ ф. стерл.}, що з них щотижня застосовується 100\pound{ ф. стерл}. В прикладі
$В$ на перший п’ятитижневий період треба авансувати 5000\pound{ ф. стерл.}, але
з них застосовується лише по 100\pound{ ф. стерл.} щотижня, отже, протягом
5 тижнів лише 500\pound{ ф. стерл.} \deq{} \sfrac{1}{10} авансованого капіталу. Протягом
другого п’ятитижневого періоду треба авансувати 4500\pound{ ф. стерл.}, але застосовується
тільки 500\pound{ ф. стерл.} і т. далі. Змінний капітал, авансовуваний
на певний період часу, перетворюється на застосовуваний, тобто справді
діющий і чинний змінний капітал лише тією мірою, якою він справді входить
у відділи цього періоду часу, заповнені процесом праці, якою він
дійсно функціонує в процесі праці. В переміжки, що протягом них
частину його авансовано лише для того, щоб її можна було застосувати
пізніше, ця частина мов би зовсім не існує для процесу праці, а тому
не справляє жодного впливу ні на утворення вартости, ні на утворення
додаткової вартости. Так стоїть, приміром, справа з капіталом $А$ в
500\pound{ ф. стерл}. Його авансовано на 5 тижнів, але в процес праці послідовно
входять з нього щотижня лише 100\pound{ ф. стерл}. Протягом першого тижня
застосовується \sfrac{1}{5} його; \sfrac{4}{5} авансовано, але не застосовано, хоч їх і
треба мати в запасі для процесу праці наступних 4 тижнів, і тому їх
доводиться авансувати.

\vtyagnut
Обставини, що зумовлюють ріжницю у відношенні між авансованим і
застосованим капіталом, впливають на продукцію додаткової вартости —
за даної норми додаткової вартости — лише остільки й лише тим, що
вони роблять різною ту кількість змінного капіталу, яку дійсно можна
застосувати протягом певного періоду часу, напр., протягом одного тижня,
протягом п’ятьох тижнів та ін. Авансований змінний капітал функціонує
як змінний капітал лише остільки й лише протягом того часу, оскільки й
коли його справді застосовується; але не протягом того часу, коли він
лишається авансований як запас, і не застосовується його. Однак усі
обставини, що зумовлюють ріжницю у відношенні між авансованим і застосованим
змінним капіталом, сходять на ріжницю періодів обороту
(визначувану ріжницею або робочого періоду, або періоду циркуляції,
або їх обох). Закон продукції додаткової вартости, є в тому, що, при однаковій
нормі додаткової вартости, однакові маси діющого змінного капіталу
утворюють однакові маси додаткової вартости. Отже, коли з капіталів
$А$ і $В$ за однакові переміжки часу при однаковій нормі додаткової
вартости застосовується однакові маси змінного капіталу, то вони мусять
протягом однакових переміжків часу утворити однакові маси додаткової
\parbreak{}  %% абзац продовжується на наступній сторінці

\parcont{}  %% абзац починається на попередній сторінці
\index{iii2}{0224}  %% посилання на сторінку оригінального видання
рента, ціна землі, а тому і її відчужуваність і відчуження, і що тому не тільки
колишні зобов’язані до виплати ренти можуть перетворитись на незалежних
селян-власників, але й міські і інші посідачі грошей можуть купувати дільниці
землі для того, щоб здавати їх в оренду або селянам або капіталістам
і користуватись рентою як формою проценту на свій в такий спосіб приміщений
капітал; отже, що і ця обставина сприяє перетворенню колишнього способу
експлуатації, відносин між власником і дійсним обробником, а також самої ренти.

\subsubsection{Відчастинне (métairie)\footnote*{
Фр. основне значіння — маєток, хутір, фарма, в даному разі мовиться про відчастинне
господарство (рос. издольное). \Red{Пр.~Ред.}
} господарство і селянська парцелярна
власність}

Тут ми підійшли до кінця нашого ряду розвитку форм земельної ренти.

В усіх цих формах земельної ренти: відробітної ренти, ренти продуктами,
грошової ренти (як просто перетвореної форми ренти продуктами) дійсним
обробником і посідачем землі завжди припускається виплатник ренти, що його
неоплачена додаткова праця безпосередньо йде власникові землі. Це не тільки
можливо, але воно дійсно так і е, навіть при останній формі, при грошовій
ренті, — оскільки вона є в чистому вигляді, тобто як просто перетворена форма
ренти продуктами.

Як переходову форму від первісної форми ренти до капіталістичної ренти
можна розглядати métairie système, або систему відчастинного господарства, за якого
обробник (орендар) крім своєї праці (власної або чужої) дає частину капіталу
для господарювання, а земельний власник дає крім землі іншу частину потрібного
для господарювання капіталу (напр., худобу), і продукт ділиться в певних,
різних для різних країн пропорціях поміж орендарем та земельним власником.
З одного боку, в орендаря тут немає достатнього капіталу для цілковитого капіталістичного
господарювання. З другого боку, та частина, яку одержує тут
земельний власник, не є чиста форма ренти. В дійсності в ній може бути процент
на авансований земельним власником капітал і надмірна рента. Вона може
в дійсності також поглинути всю додаткову працю орендаря, або лишити йому
більшу або меншу частину цієї додаткової праці. Але істотне є в тому, що
рента тут уже більш не виступає, як нормальна форма додаткової вартости взагалі.
На одному боці орендар, чи вживає він тільки власної, чи також і чужої праці,
має домагання на певну частину продукту не тому, що він робітник, а тому,
що він посідач частини знарядь праці, капіталіст сам собі. На другому боці
земельний власник домагається своєї частини, ґрунтуючись не виключно на
своїй власності на землю, але як і позикодавець капіталу\footnote{
Порівн. Buret, Tocqueville, Sismondi.
}.

Рештки старовинної громадської власности на землю, що збереглись після
переходу до самостійного селянського господарства, наприклад, у Польщі та
Румунії, були там за привід для того, щоб здійснити перехід до нижчих форм
земельної ренти. Частина землі належить поодиноким селянам і вони обробляють
її самостійно. Друга частина обробляється спільно і створює додатковий
продукт, який придається почасти для покриття витрат громади, почасти як
резерв на випадок неврожаїв тощо. Ці дві останні частини додаткового продукту,
а кінець-кінцем і весь додатковий продукт, разом з землею, на якій він виростає,
помалу узурпується державними урядовцями і приватними особами,
і первісно вільні селяни-землевласники, що для них зберігається повинність
спільного обробітку цієї землі, перетворюються таким чином на панщанних,
або зобов’язаних до виплати ренти продуктами, тимчасом як узурпатори
\parbreak{}  %% абзац продовжується на наступній сторінці

\parcont{}  %% абзац починається на попередній сторінці
\index{i}{0225}  %% посилання на сторінку оригінального видання
точні постанови закону з 1844~\abbr{р.} про час для їжі дають робітникам
лише дозвіл їсти й пити перед їхнім приходом на фабрику
й після їхнього виходу з фабрики, отже, в себе вдома! І чому б
таки робітникам не обідати перед 9 годиною ранку? Однак коронні
юристи вирішили, що приписаний законом час на їжу «треба
давати в перервах дійсного робочого дня, і що це протизаконно
примушувати працювати без перерви одна по одній 10 годин
від 9 години ранку до 7 години вечора»\footnote{
«Reports etc. for 31 st October 1848», p. 130.
}.

Після цих добродушних демонстрацій капітал, щоб, підготовити
бунт, вдався до такого кроку, що відповідав букві закону
з 1844~\abbr{р.}, отже, був леґальний.

Щоправда, закон з 1844~\abbr{р.} забороняв вживати праці дітей 8--13-річного
віку, які працювали перед 12 годиною дня, знову після
1 години по півдні. Але він ніяким чином не реґулював 6\sfrac{1}{2}-годинної
праці дітей, що їхній робочий час починався о 12 годині
дня або пізніш! Отже, восьмилітніх дітей, якщо вони починали
роботу о 12 годині дня, можна було вживати до праці від 12 до
1 години, тобто на одну годину, від 2 до 4 години, тобто на 2 години,
і від 5 до пів на 9 годину вечора, тобто на 3\sfrac{1}{2} години; разом
6\sfrac{1}{2} годин, визначених законом! Або ще краще. Щоб пристосувати
вживання праці дітей до праці дорослих робітників-чоловіків, які
працювали до пів на 9 годину вечора, фабрикантам треба було не
давати дітям жодної роботи перед 2 годиною по півдні, а потім вони
могли їх тримати на фабриці без жодних перерв до пів на 9 годину
вечора! «А тепер уже виразно визнають, що останніми часами,
в наслідок ненажерливости фабрикантів та їхнього бажання тримати
машини в русі більш, ніж 10 годин на добу, в Англії крадькома
встановилась практика примушувати 8--13-літніх дітей
обох статей працювати після того, як підуть усі підлітки й жінки,
з самими дорослими чоловіками до пів на дев’яту вечора»\footnote{
«Reports etc.», там же, стор. 42.
}.
Робітники й фабричні інспектори протестували з гігієнічних і
моральних причин. Але капітал відповідав:
\vspace{-\medskipamount}
\settowidth{\versewidth}{«На голову мою хай вчинки}
\begin{verse}[\versewidth]
«На голову мою хай вчинки \\
Падуть мої! Я вимагаю \\
Лиш права власного! Пені \\
Та векселя мого оплати!»\footnote*{
Цю цитату, як і дальшу, взято з Шекспірового «Шейлока». \emph{Ред.}
}
\end{verse}
\vspace{-\medskipamount}

\noindent{}Справді, за поданими до Палати громад 26 липня 1850~\abbr{р.}
статистичними даними, на 15 липня 1850~\abbr{р.} було, не зважаючи
на всі протести, \num{3.742} дітей на 275 фабриках, підданих під цю
«практику»\footnote{
«Reports etc. for 31 st October 1850», p. 5, 6.
}. Але цього ще не досить! Капітал своїм оком
рися відкрив, що хоч закон 1844~\abbr{р.} і не дозволяє п’ятигодинної
праці в передобідній час без принаймні 30-хвилинної перерви
для відпочинку, але не приписує нічого подібного щодо післяобідньої
праці. Тому він домагався й добився тієї приємности,
\parbreak{}  %% абзац продовжується на наступній сторінці

\parcont{}  %% абзац починається на попередній сторінці
\index{i}{0226}  %% посилання на сторінку оригінального видання
що міг примушувати восьмилітніх робітників-дітей не лише
надмірно працювати без перерви від другої до пів на дев’яту
годину вечора, але й голодувати!
\vspace{-\bigskipamount}
\settowidth{\versewidth}{Як вексель каже!»}
\begin{verse}[\versewidth]
«Так, серце його, \\
Як вексель каже!»\footnote{
Природа капіталу лишається та сама так у його нерозвинених, як
і в розвинених його формах. У збірці законів, подиктованих території
Нової Мехіки впливом рабовласників незадовго перед початком американської
громадянської війни, сказано: Робітник, оскільки капіталіст
купив його робочу силу, «є його (капіталіста) гроші» («The labourer
is his (the capitalist's) money»). Такий самий погляд був поширений і серед
римських патриціїв. Гроші, що визичали вони плебеєві-винуватцеві,
перетворювалися за допомогою його засобів існування на м'ясо й кров
винуватця. Тим то це «м’ясо й кров» було «їхніми грішми». Звідси шейлоківський
закон 10 таблиць! Гіпотеза Ленґе, нібито кредитори-патриції
влаштовували час-від-часу по той бік Тібру бенкети, де подавано варене
м’ясо винуватців, лишається так само невирішеною, як і гіпотеза
Давмера про християнське причастя.
}
\end{verse}
\vspace{-\bigskipamount}

\noindent{}Те, що фабриканти по-шейлоківському вхопилися за букву
закону 1844~\abbr{р.}, оскільки він реґулює працю дітей, повинно було,
однак, тільки підготувати явний бунт проти цього самого закону,
оскільки він реґулює працю «підлітків і жінок». Пригадаймо, що
скасування «фалшивої Relaissystem» становить головну мету
й головний зміст цього закону. Фабриканти розпочали свій бунт
простою заявою, що пункти закону 1844~\abbr{р.}, які забороняють довільно
вживати праці підлітків і жінок у довільно короткі
періоди п’ятнадцятигодинного фабричного дня, «були порівняно
нешкідливі (comparatively harmless) доти, доки робочий час
обмежувано 12 годинами\footnote*{
У французькому виданні цю фразу зредаґовано так: «Фабриканти
розпочали свій бунт простою заявою, що пункти закону 1844~\abbr{р.}, які забороняють
досхочу вживати праці підлітків і жінок, примушуючи їх о якій
завгодно порі дня переривати і знову братися до праці, були порівняно
дрібницею доти, доки мав силу 12-годинний робочий час». («Le Capital,
etc.», v. I, ch. X, p. 124). \emph{Ред.}
}. Але за десятигодинного закону вони
є нестерпна кривда» (hardship)\footnote{
«Reports etc. for 30 th April 1848». p. 28.
}. Тому вони байдужісінько
заявили інспекторам, що не зважатимуть на букву закону і
самовладно знову запровадять стару систему\footnote{
Так, між іншим, заявив філантроп Ешворд у своєму огидному
квакерівському листі до Леонарда Горнера («Reports etc., April 1849»,
р. 4).
}. Це, мовляв,
буде в інтересах самих же робітників, спантеличених дурними
порадами, бо «дасть змогу платити їм вищу заробітну плату».
«Це однісінький можливий плян, щоб за десятигодинного закону
зберегти промислову перевагу Великобрітанії»\footnote{
Там же, стор. 134.
}. «Можливо,
що за системи змін (Relaissystem) трохи важко викривати порушення
закону, але що з того? (what of that?) Хіба ж можна
великі промислові інтереси цієї країни розглядати як другорядну
річ заради того, щоб фабричним інспекторам і підінспекторам
заощадити трохи більше клопоту (some little trouble)»\footnote{
Там же, стор. 140.
}.


\index{iii1}{0227}  %% посилання на сторінку оригінального видання
[Норма зиску обчислюється на весь застосований капітал, але за певний час, фактично за один рік.
Відношення виробленої за рік і реалізованої додаткової вартості або зиску до всього капіталу,
обчислене в процентах, є норма зиску. Отже, вона не неодмінно дорівнює тій нормі зиску, при якій в
основу обчислення кладеться не рік, а період обороту капіталу, про який іде мова; тільки в тому
випадку, коли цей капітал обертається саме один раз за рік, обидві ці норми збігаються.

З другого боку, зиск, одержаний на протязі року, є тільки сума зисків на товари, вироблені і продані
на протязі того самого року. Якщо ж ми обчислюватимем зиск на витрати виробництва товарів, то
одержимо норму зиску $= \frac{p}{k}$, де $р$ становить реалізований на протязі року зиск, а $k$ — суму витрат
виробництва товарів, вироблених і проданих протягом того самого часу. Очевидно, що ця норма зиску
$\frac{p}{k}$ тільки в тому випадку може збігатися з дійсною нормою зиску $\frac{p}{K}$, — маса зиску, поділена на весь
капітал, — коли $k = К$, тобто коли капітал обертається, саме один раз за рік.

Візьмімо три різні стани якогонебудь промислового капіталу.

І. Капітал в 8000\pound{ фунтів стерлінгів} виробляє і продає щороку 5000 штук товару по 30\shil{ шилінгів} за
штуку, отже, має річний оборот в 7500\pound{ фунтів стерлінгів}. На кожну штуку товару
він дає зиск в 10\shil{ шилінгів} = 2500\pound{ фунтам стерлінгів} на рік. Отже, в кожній штуці містяться 20\shil{ шилінгів} авансованого капіталу і 10\shil{ шилінгів} зиску, отже норма зиску на кожну штуку становить
$\frac{10}{20}= 50\%$. На суму в 7500\pound{ фунтів стерлінгів}, що обернулась, припадає 5000\pound{ фунтів стерлінгів}
авансованого капіталу і 2500\pound{ фунтів стерлінгів} зиску; норма зиску на кожний оборот, $\frac{p}{k}$, так само =
50\%. Навпаки, норма зиску, обчислена на весь капітал, $\frac{p}{K} = \frac{2500}{8000} = 31\sfrac{1}{4}\%$.

II. Припустім, що капітал збільшується до \num{10000}\pound{ фунтів стерлінгів}. Припустім, що в наслідок
збільшеної продуктивної сили праці він може виробляти щороку \num{10000} штук товару при витратах
виробництва в 20\shil{ шилінгів} на штуку. Він продає їх із зиском в 4\shil{ шилінги} на штуку, отже, по 24\shil{ шилінги} за штуку. Тоді ціна річного продукту = \num{12000}\pound{ фунтам стерлінгів}, з яких \num{10000}\pound{ фунтів
стерлінгів} авансованого капіталу і 2000\pound{ фунтів стерлінгів} зиску $\frac{p}{k}$ на кожну штуку $= \frac{4}{20}$, для річного
обороту $= \frac{2000}{\num{10000}}$, отже, в обох випадках = 20\%, а через те що весь
\parbreak{}  %% абзац продовжується на наступній сторінці

\parcont{}  %% абзац починається на попередній сторінці
\index{iii1}{0228}  %% посилання на сторінку оригінального видання
капітал дорівнює сумі витрат виробництва, а саме \num{10000}\pound{ фунтам стерлінгів}, то й $\frac{p}{K}$, дійсна норма
зиску, на цей раз \deq{} 20\%.

III.~Припустім, що капітал при постійно зростаючій продуктивній силі праці збільшується до \num{15000}\pound{ фунтів стерлінгів} і виробляє тепер щорічно \num{30000} штук товару при витратах виробництва в 13\shil{ шилінгів}
на штуку, при чому кожна штука продається з зиском в 2\shil{ шилінги}, отже, по 15\shil{ шилінгів.} Отже, річний
оборот \deq{} 15\shil{ шилінгів} × \num{30000} \deq{} \num{22500}\pound{ фунтам стерлінгів}, з яких \num{19500}\pound{ фунтів стерлінгів}
авансованого капіталу і 3000\pound{ фунтів стерлінгів} зиску. Отже,
$\frac{p}{k} \deq{} \frac{2}{13} \deq{} \frac{3000}{\num{19500}} \deq{} 15\sfrac{5}{13}/\%$.
Навпаки, $\frac{p}{K} \deq{} \frac{3000}{\num{15000}} \deq{} 20\%$.

Отже, ми бачимо, що тільки у випадку II, де капітальна вартість, яка обернулася, дорівнює всьому
капіталові, норма зиску на штуку товару або на суму обороту є така сама, як і норма зиску, обчислена
на весь капітал. У випадку І, де сума обороту менша, ніж весь капітал, норма зиску, обчислена на
витрати виробництва товару, є вища; у випадку III, де весь капітал менший, ніж сума обороту, вона
нижча, ніж дійсна норма зиску, обчислена на весь капітал. Це має загальне значення.

В купецькій практиці оборот звичайно обчислюється неточно. Припускається, що капітал обернувся один
раз, коли сума реалізованих товарних цін досягає суми всього застосованого капіталу. Але \emph{капітал}
може тільки тоді завершити повний оборот, коли сума \emph{витрат виробництва} реалізованих товарів
дорівнюватиме сумі всього капіталу. — Ф.~E.].

І тут знову виявляється, як важливо при капіталістичному виробництві розглядати окремий товар або
товарний продукт, вироблений протягом якогось періоду часу, не ізольовано, не сам по собі, не як
простий товар, а як продукт авансованого капіталу і відносно всього капіталу, який виробляє ці
товари.

Хоча \emph{норму} зиску слід обчислювати, вимірюючи масу виробленої і реалізованої додаткової вартості не
тільки відносно спожитої частини капіталу, яка знову з’являється в товарах, але відносно цієї
частини плюс та частина капіталу, яка не спожита, але застосована і продовжує служити у виробництві,
— проте \emph{маса} зиску може дорівнювати тільки тій масі зиску або додаткової вартості, яка міститься в
самих товарах і має бути реалізована через їх продаж.

\looseness=-1
Якщо продуктивність промисловості збільшується, то ціна окремого товару падає. В ньому міститься
менше праці як оплаченої, так і неоплаченої. Припустім, що та сама праця виробляє, наприклад, утроє
більше продукту; тоді на кожний окремий продукт припадає праці на \sfrac{2}{3} менше. А через те що зиск
може становити тільки частину цієї вміщеної в кожному
\parbreak{}  %% абзац продовжується на наступній сторінці


\index{ii}{0229}  %% посилання на сторінку оригінального видання

Далі з цього випливає: річна норма додаткової вартости завжди
$\deq{}  m'n$, тобто дорівнює справжній нормі додаткової вартости, спродукованої
в один період обороту змінним капіталом, зужитим протягом цього
періоду, помноженій на число оборотів цього змінного капіталу протягом
цього року, або помноженій (що те саме) на обернений дріб його часу
обороту, обчисленого на рік, що береться за одиницю. (Коли
змінний капітал обертається 10 разів на рік, то час його обороту \deq{} \sfrac{1}{10} року;
отже, обернений дріб його часу обороту \deq{} \sfrac{10}{1} \deq{} 10).

Далі з наведеного випливає: $М' \deq{} m'$, коли $n \deq{} 1$. $М'$ більше за $m'$,
коли $n$ більше за 1, тобто, коли авансований капітал обертається більше
як один раз на рік, або коли капітал, що обернувся, більший, ніж капітал
авансований.

Нарешті, $М'$ менше за $m'$, коли $n$ менше за 1, тобто коли капітал,
що обернувся протягом року, є лише частина авансованого капіталу, отже,
коли період обороту триває більш як рік.

Зупинімось трохи на цьому останньому випадку.

Ми зберігаємо всі припущення нашого попереднього прикладу, хай
тільки період обороту продовжиться до 55 тижнів. Процес праці потребує
щотижня 100\pound{ ф. стерл.} змінного капіталу, отже, 5500\pound{ ф. стерл.} для періоду
обороту, і продукує щотижня $100 m$; отже, $m$, як і перше, \deq{} 100\%.

Число оборотів $n$ дорівнює тут $\frac{50}{55} \deq{} \frac{10}{11}$, бо час обороту (беручи рік
50 тижнів) $\deq{} 1 \dplus{} \sfrac{1}{10}\text{ року} \deq{} \frac{11}{10}\text{ року}$.
$М' \deq{} \frac{100\% × 5500 × \sfrac{10}{11}}{5500} \deq{} 100\% × \frac{10}{11} \deq{} \frac{1000}{11}\% \deq{} 90\frac{10}{11}\%$,
отже, менше, ніж 100\%. Справді,
коли б річна норма додаткової вартости була 100\%, то $5500 v$ протягом
року мусили б випродукувати $5500 m$, тимчасом як для цього треба \frac{11}{10}
року. Ці $5500 v$ продукують протягом року лише $5000 m$, отже, річна
норма додаткової вартости $\deq{} \frac{5000 m}{5500 v} \deq{} \frac{10}{11} \deq{} 90\frac{10}{11}\%$.

Тому річна норма додаткової вартости, або відношення між додатковою
вартістю, спродукованою протягом року, і взагалі \emph{авансованим}
змінним капіталом (на відміну від змінного капіталу, що \emph{обернувся}
протягом року), не є просте суб’єктивне відношення, а самий
справжній рух капіталу викликає це зіставлення. Наприкінці року до
власника капіталу $А$ повернувся авансований ним змінний капітал, рівний
500\pound{ ф. стерл.}, і крім того 5000\pound{ ф. стерл.} додаткової вартости. Величину
авансованого ним капіталу виражає не та маса капіталу, що її він застосував
протягом року, а та, що періодично до нього повертається. В
розглядуваному питанні не має жодного значення, чи існує капітал наприкінці
року почасти як продукційний запас, чи почасти як товаровий
або грошовий капітал, і в якому відношенні розподіляється він на ці
різні частини. Для власника капіталу В повернулись 5000\pound{ ф. стерл.}, авансований
\index{ii}{0230}  %% посилання на сторінку оригінального видання
ним капітал, та ще 5000\pound{ ф. стерл.} додаткової вартости. Для власника
капіталу $C$ (останнього розглянутого нами капіталу в 5500\pound{ ф. стерл.})
спродуковано протягом року 5000\pound{ ф. стерл.} додаткової вартости (при витраті
5000\pound{ ф. стерл.} і нормі додаткової вартости в 100\%), але авансований
ним капітал, а так само і спродукована додаткова вартість ще не повернулись
до нього.

$М' \deq{} m'n$ виражає, що норма додаткової вартости, яка має силу
для змінного капіталу, застосованого протягом одного періоду обороту:\[
\frac{\text{маса додаткової вартости, створена протягом одного періоду обороту}}{\text{змінний капітал, застосований протягом одного періоду обороту}}
\]
має бути помножена на число періодів обороту або на число періодів
репродукції авансованого змінного капіталу — на число періодів, що протягом
їх він відновлює свій кругобіг.

В книзі І, розділ IV („Перетворення грошей на капітал“), а потім у
книзі І, розділ XXI („Проста репродукція“) ми бачили вже, що капітальну
вартість взагалі авансується, а не витрачається, бо ця вартість,
проробивши різні фази свого кругобігу, знову повертається до свого
вихідного пункту, та ще й збагачена додатковою вартістю. Це характеризує
її, як авансовану вартість. Час, що минає від її вихідного пункту
до пункту її повороту, і є той час, що на нього авансується її. Ввесь
кругобіг, що його перебігає капітальна вартість, вимірюваний часом від її
авансування де її повороту, становить її оборот, а час тривання цього обороту
становить період обороту. Коли цей період закінчився і кругобіг
вивершено, то та сама капітальна вартість може знову почати той самий
кругобіг, отже, і знову зростати в своїй вартості, утворювати додаткову
вартість. Коли змінний капітал, як в $А$, обертається десять разів на рік,
то протягом року тим самим авансованим капіталом десять разів утворюється
таку масу додаткової вартости, яка відповідає одному періодові
обороту.

Треба з’ясувати природу авансування з погляду капіталістичного суспільства.
Капітал $А$, що обертається десять разів протягом року, авансується
протягом року десять разів. На кожний новий період обороту його авансується
знову. Але разом з тим протягом року $А$ ніколи не авансує нічого
більшого, ніж ту саму капітальну вартість в 500\pound{ ф. стерл.}, і дійсно,
для розглядуваного нами продукційного процесу він ніколи не має в
своєму розпорядженні нічого більшого понад ці 500\pound{ ф. стерл}. Скоро тільки
ці 500\pound{ ф. стерл.} закінчують один кругобіг, капіталіст $А$ повертає їх
знову на такий самий кругобіг: капітал з природи своєї зберігає характер
капіталу лише тому, що він завжди в повторюваних процесах продукції
функціонує як капітал. Його тут ніколи не авансується на довший час,
ніж 5 тижнів. Коли оборот триватиме довший час, то капіталу не вистачить.
Коли час обороту скорочується, то частина капіталу стає надлишковою.
Тут авансується не десять капіталів по 500\pound{ ф. стерл.}, а один
\parbreak{}  %% абзац продовжується на наступній сторінці

\parcont{}  %% абзац починається на попередній сторінці
\index{ii}{0231}  %% посилання на сторінку оригінального видання
капітал в 500\pound{ ф. стерл.} авансується десять разів у послідовні періоди часу.
Тому річну норму додаткової вартости обчислюється не на капітал в 500\pound{ ф.
стерл.}, що його авансується десять разів, або на 5000\pound{ ф. стерл.}, а на капітал
в 500\pound{ ф. стерл.}, що його авансовано один раз; цілком так само,
як один таляр, що обертається десять разів, завжди репрезентує лише
одним один таляр, що перебуває в циркуляції, хоч він виконує функцію
10 талярів. Але в тих руках, де він є при кожній оборудці, він завжди
лишається тією самою вартістю в 1 таляр.

Так само капітал $А$ при кожному своєму повороті, а також при своєму
повороті наприкінці року показує, що його власник завжди орудує
лише тією самою капітальною вартістю в 500\pound{ ф. стерл}. Тому до рук його
кожного разу повертається лише 500\pound{ ф. стерл}. Тому авансований ним капітал
ніколи не перевищує 500\pound{ ф. стерл}. Авансований капітал в 500\pound{ ф. стерл.}
становить тому знаменика того дробу, що виражає річну норму додаткової
вартости. Формула для річної норми додаткової вартости в нас
вище була така:

\begin{center}
$М' \deq{} \frac{m'vn}{v} \deq{} m'n\text{.}$
\end{center}

\noindent{}А що справжня норма додаткової вартости $m' \deq{} \frac{m}{v}$ дорівнює масі додаткової
вартости, поділеній на змінний капітал, що продукує її, то в
$m'n$ ми можемо підставити значення $m'$, тобто $\frac{m}{v}$ і тоді матимемо другу
формулу $М' \deq{} \frac{mn}{v}$.

Але в наслідок десятиразового обороту, а тому в наслідок десятиразового
поновлювання його авансування, капітал в 500\pound{ ф. стерл.} виконує
функцію вдесятеро більшого капіталу, капіталу в 5000\pound{ ф. стерл.}, цілком так
само, як 500 монет, кожна в 1 таляр, обертаючись десять разів на рік,
виконують ту саму функцію, що й 5000 таких самих монет, які обертаються
лише один раз.

\subsection{Кругобіг поодинокого змінного капіталу}

„Хоч яка буде суспільна форма процесу продукції, процес мусить
бути безупинний або мусить періодично знову й знову перебігати ті самі
стадії\dots{} Тому всякий суспільний процес продукції, розглядуваний в
його постійному зв’язку та постійній течії його відновлення, є разом з
тим процес репродукції\dots{} Як періодичний приріст капітальної вартости
або періодичний плід капіталу, додаткова вартість набирає форми доходу,
що походить із капіталу“. (Кн.~І, початок розд. XXI).

Ми маємо 10 п’ятитижневих періодів обороту капіталу $А$; в перший
період обороту авансується 500\pound{ ф. стерл.} змінного капіталу; тобто щотижня
обмінюється на робочу силу 100\pound{ ф. стерл.}, так що наприкінці першого
\parbreak{}  %% абзац продовжується на наступній сторінці


\chapter{Доходи та їхні джерела}

\section{Триєдина формула}

\subsection*{І.}

Капітал\footnote{Дальші три уривки містяться в різних місцях рукопису відділу VІ. — \emph{Ф.~Е.}}
— зиск (підприємницький бариш плюс процент), земля — земельна
рента, праця — заробітна плата, це триєдина форма, яка охоплює всі таємниці
суспільного процесу продукції.

А що далі, як це показано раніш, процент виступає як специфічний,
характеристичний продукт капіталу, а підприємницький бариш протилежно
до цього як незалежна від капіталу заробітна плата, то зазначена триєдина
форма найближче зводиться до такої:

Капітал — процент, земля — земельна рента, праця — заробітна плата, де зиск,
ця форма додаткової вартости, що специфічно характеризує капіталістичний спосіб
продукції, щасливо усувається.

При ближчому розгляді цієї економічної триєдиности ми відкриваємо таке:

Поперше, позірні джерела багатства, що ним можна щороку порядкувати,
належать до цілком різних сфер і не мають найменшої схожости між собою.
Взаємне відношення між ними приблизно таке, як наприклад, між нотаріяльними
оплатами, червоними буряками і музикою.

\looseness=-1
Капітал, земля, праця! Але капітал — це не річ, а певне, суспільне,
належне певній історичній формації суспільства продукційне відношення, яке
виявляється в речі і надає цій речі специфічного суспільного характеру. Капітал
не є сума матеріяльних і випродукованих засобів продукції. Капітал, це —
перетворені на капітал засоби продукції, які сами по собі так само не є капітал,
як золото або срібло сами по собі не є гроші. Це є монополізовані певною
частиною суспільства засоби продукції, усамостійнені проти живої робочої сили
продукти й умови діяльности самої цієї робочої сили, які в наслідок цієї протилежности
персоніфікуються в капіталі. Це не тільки продукти робітників,
перетворені на самостійні сили, продукти як поневільники і покупці своїх продуцентів,
але також і суспільні сили і майбутня\dots{} [? нерозбірливо] форма цієї
праці, — сили, що протистоять робітникам, являючи властивості їхнього продукту.
Отже, ми маємо тут певну, на перший погляд дуже містичну, суспільну форму
одного з чинників певного історично створеного суспільного процесу продукції.

\disablefootnotebreak{}
\looseness=-1
А тепер, поряд з цим земля, неорганічна природа як така, rudis indigestaque
moles\footnote*{
Лат. груба, необроблена груда. \Red{Пр.~Ред.}
} у всій її перевісній дикості. Вартість є праця. Тому додаткова
\parbreak{}  %% абзац продовжується на наступній сторінці

\parcont{}  %% абзац починається на попередній сторінці
\index{iii2}{0233}  %% посилання на сторінку оригінального видання
вартість не може бути землею. Абсолютна родючість землі не призводить ні до
чого іншого, як тільки до того, що певна кількість праці дає певний, зумовлений
природною родючістю землі, продукт. Ріжниця в родючості землі призводить
до того, що ті самі кількості праці й капіталу, отже, та сама вартість, виражається
в різних кількостях хліборобських продуктів; отже, що ці продукти
мають різні індивідуальні вартості. Вирівнювання цих індивідуальних вартостей
за ринковими вартостями призводить до того, що «advantages of fertile over inferior
soil\dots{} are transferred from the cultivator or consumer to the landlord» (Ricardo,
Principles, p. 6)\footnote*{
«Вигоди, одержувані від родючішого ґрунту проти гіршого\dots{} переносяться від обробника або
споживача до лендлорда».
}.
\enablefootnotebreak{}

\looseness=1
І, нарешті, як третій в цій спілці простий привид, — праця «взагалі»,
яка є не що інше, як абстракція і взята сама по собі взагалі не існує, або,
коли ми\dots{} (нерозбірливо) візьмемо, продуктивна діяльність людини взагалі,
з допомогою якої людина упосереджує обмін речовин з природою, не тільки
оголена від усякої суспільної форми і характеристичної визначености, але навіть
і просто в її природному бутті, незалежно від суспільства, абстраговано від
хоч би яких суспільств, і як вияв життя та процес життя, спільна ще несуспільній
людині взагалі з людиною, що має будь-яке суспільне визначення.

\subsection*{ІI.}

\looseness=1
Капітал — процент; земельна власність, приватна власть на землю, до того ж
сучасна, відповідна капіталістичному способові продукції, — рента; наймана праця
— заробітна плата. Отже, в цій формі повинен бути зв’язок між джерелами
доходу. Як капітал, так само й наймана праця і земельна власність є історично
визначені суспільні форми; одна — праці, друга — монополізованої землі, і до
того обидві є форми, відповідні капіталові і належні тій самій економічній
формації суспільства.

Перше, що впадає на очі в цій формулі, є те, що поряд з капіталом,
поряд з цією формою одного елементу продукції, належного певному способові
продукції, певній історичній структурі суспільного процесу продукції, поряд
з елементом продукції, що поєднався з певною соціяльною формою і репрезентований
цією соціяльною формою, без дальших околичностей, ставляться: земля
на одному боці, праця — на другому, два елементи реального процесу праці,
які в цій речовій формі спільні всім способам продукції, є речові елементи
всякого процесу продукції, і не мають ніякого чинення до його суспільної форми.

Подруге. У формулі: капітал — процент, земля — земельна рента, праця —
заробітна плата, капітал, земля, праця виступають відповідно як джерела проценту
(замість зиску), земельної ренти і заробітної плати як своїх продуктів,
витворів; перші — основа, другі — наслідок, перші — причина, другі — дія; і до
того ж таким чином, що кожне окреме джерело стоїть до свого продукту в такому
відношенні, як до чогось від нього відштовхнутого і ним спродукованого.
Усі три доходи, процент (замість зиску), рента, заробітна плата, є три частини
вартости продукту, отже, взагалі частини вартости, або в грошовому виразі
певні частини грошей, частини ціни. Хоч формула: капітал — процент і є найіраціональніша формула
капіталу, а проте це — його формула. Але яким чином
земля може створити вартість, тобто суспільно визначену кількість праці і навіть
ту особливу частину вартости її власних продуктів, яка становить ренту. Земля
діє, наприклад, як аґент продукції при створенні певної споживної вартости,
\parbreak{}  %% абзац продовжується на наступній сторінці

\input{ii/_0234c.tex}

\index{ii}{0235}  %% посилання на сторінку оригінального видання
Ріжниця випливає з неоднаковости періодів обороту, тобто тих періодів,
що в них вартість, яка замішує змінний капітал, застосований протягом
певного часу, знову може функціонувати як капітал, отже, як новий
капітал. У \emph{В}, як і в \emph{А}, однаково заміщується вартість змінного капіталу,
застосованого протягом однакових періодів. Так само протягом однакових
періодів відбувається однаковий приріст додаткової вартости. Але хоч у
\emph{В} й заміщується що п’ять тижнів вартість в 500\pound{ ф. стерл.}, та ще й наростає
500\pound{ ф. стерл.} додаткової вартости, однак ця вартість, що являє собою
заміщення $v$, не є ще новий капітал, бо вона перебуває не в грошовій
формі. У \emph{А} не лише стару капітальну вартість заміщується новою, а її
відновлюється в її грошовій формі, отже, її заміщується як новий, здібний
функціонувати капітал.

Чи раніше, чи пізніше відбувається перетворення вартости, що являє
собою заміщення, на гроші, а тому на форму, що в ній авансується
змінний капітал, — це, очевидно, цілком байдужа обставина для самої
продукції додаткової вартости. Ця продукція залежить від величини застосованого
змінного капіталу й від ступеня експлуатації праці. Але обставина
ця модифікує величину того грошового капіталу, що його треба
авансувати, щоб протягом року пустити в рух певну кількість робочої
сили, а тому вона визначає річну норму додаткової вартости.

\subsection[Оборот змінного капіталу, розглядуваного з суспільного погляду]{Оборот змінного капіталу,\\ розглядуваного з суспільного погляду}

Погляньмо на хвилинку на справу з суспільного погляду. Припустімо,
що один робітник коштує на тиждень 1\pound{ ф. стерл.}, а робочий день \deq{} 10 годинам.
У \emph{А}, як і у \emph{В}, протягом року працюють 100 робітників (100\pound{ ф.
стерл.} на тиждень на 100 робітників становлять за 5 тижнів 500\pound{ ф. стерл.}, а
за 50 тижнів — 5000\pound{ ф. стерл.}); припустімо, що вони працюють 6 днів на
тиждень, по 60 робочих годин кожен. Отже, 100 робітників працюватимуть
протягом тижня 6000 робочих годин, а протягом 50 тижнів, \num{300.000} робочих
годин. І~\emph{А}, і \emph{В} захопили цю робочу силу, отже, суспільство не може витрачати
її на щось інше. Щодо цього, то з суспільного погляду справа
така сама в \emph{А}, як і у \emph{В}. Далі у \emph{А}, як і у \emph{В}, кожні 100 робітників одержують
на рік 5000\pound{ ф. ст.} заробітної плати (отже, всі 200 робітників
одержують разом \num{10.000}\pound{ ф. стерл.}) і беруть у суспільства засобів існування
на цю суму. І щодо цього справа з суспільного погляду така сама в
\emph{А}, як і у \emph{В}. Що робітники в обох випадках одержують заробітну плату
щотижня, то щотижня вони беруть у суспільства й засоби існування, за
які вони в обох випадках щотижня пускають у циркуляцію грошовий
еквівалент. Але відси починається ріжниця.

\so{Поперше}. Гроші, що їх пускає в циркуляцію робітник \emph{А}, є не
тільки грошова форма вартости його робочої сили, як для робітника \emph{В}
(у дійсності — засіб виплати за вже виконану роботу); починаючи з другого
періоду обороту, рахуючи з відкриття підприємства, вони вже є
\index{ii}{0236}  %% посилання на сторінку оригінального видання
грошова форма \so{новоутвореної ним самим} вартости (\deq{} ціні робочої
сили плюс додаткова вартість) першого періоду обороту, що нею
оплачується його працю протягом другого періоду обороту. У \emph{В} справа
інша. Хоч щодо робітника гроші й тут є засіб виплати за вже виконану
ним працю, але цю виконану вже працю оплачується не новоутвореною
нею вартістю, перетвореною на золото (не грошовою формою вартости,
спродукованої самою цією працею). Такий спосіб оплати може постали,
починаючи лише з другого року, коли робітника \emph{В} оплачується спродукованою
ним в минулому році вартістю, перетвореною на золото.

Що коротший період обороту капіталу, — що коротші, отже, переміжки,
що в них протягом року поновлюються терміни його репродукції, —
то швидше змінна частина капіталу, первісно авансована в грошовій формі
капіталістом, перетворюється на грошову форму тієї новоутвореної
вартости (яка, крім того, містить у собі й додаткову вартість), що її
утворив робітник на заміщення цього змінного капіталу; то коротший,
отже, час, на який капіталіст мусить авансувати гроші з свого власного
фонду, то менший, порівняно з даними розмірами маштабу продукції, той
капітал, що його він взагалі авансує; і то більша порівняно та маса додаткової
вартости, що її він за даної норми додаткової вартости видушує
протягом року, бо він то частіше може знову й знову купувати
робітника на грошову форму вартости продукту цього ж таки робітника
й пускати в рух його працю.

За даних розмірів продукції абсолютна величина авансованого змінного
грошового капіталу (як і взагалі обігового капіталу) меншає, а річна
норма додаткової вартости більшає пропорційно до скорочення періоду
обороту. За даної величини авансованого капіталу розміри продукції,
а тому за даної норми додаткової вартости й абсолютна маса додаткової
вартости, утвореної протягом одного періоду обороту, зростають разом
з підвищенням річної норми додаткової вартости, що його зумовлює
скорочення періоду репродукції. Взагалі, з нашого досліду виявилось, що
відповідно до різного протягу періоду обороту доводиться авансовувати грошовий
капітал дуже різної величини для того, щоб при тому самому
ступені експлуатації праці пускати в рух однакову масу продуктивного
обігового капіталу та однакову масу праці.

\so{Подруге} — і це має зв’язок з першою ріжницею — робітник капіталіста
\emph{В}, як і \emph{А}, платить за куповані ним засоби існування змінним
капіталом, що перетворився в його руках на засіб циркуляції. Він не
тільки, напр., бере з ринку пшеницю, а й заміщує її грошовим еквівалентом.
А що гроші, що ними робітник \emph{В} оплачує засоби свого існування,
вилучаючи їх з ринку, не є грошова форма новоутвореної вартости,
подаваної ним на ринок протягом року, як у робітника \emph{А}, то хоч
він і дає гроші продавцеві його засобів існування, але не дає він жодного
товару, — ні засобів продукції, ні засобів існування, — що їх той
міг би купити за вторговані гроші, як це, навпаки, маємо в випадку \emph{А}.
Тому з ринку береться робочу силу, засоби існування для цієї робочої
сили, основний капітал у формі засобів праці й продукційних матеріялів,
\parbreak{}  %% абзац продовжується на наступній сторінці

\parcont{}  %% абзац починається на попередній сторінці
\index{ii}{0237}  %% посилання на сторінку оригінального видання
застосованих у \emph{В}, і на заміщення цього всього подається на ринок еквівалент
в формі грошей; але протягом цього року на ринок не подається
жодного продукту, щоб замістити взяті з ринку речові елементи продуктивного
капіталу. Коли ми уявимо собі не капіталістичне суспільство,
а комуністичне, то, насамперед, зовсім відпадає грошовий капітал, а значить,
і всі ті маскування оборудок, які постають через грошовий капітал.
Справа сходить просто на те, що суспільство мусить наперед обчислити,
скільки праці, засобів продукції та засобів існування воно може без якої-будь
шкоди витрачати на такі галузі продукції, що, як от, напр., будування
залізниць, довгий час, рік або й більше, не дають ні засобів
продукції, ні засобів існування, ні взагалі будь-якого корисного ефекту,
але звичайно відбирають від цілої річної продукції працю, засоби продукції
і засоби існування. Навпаки, в капіталістичному суспільстві, де
суспільний розум завжди виявляє себе тільки post festum\footnote*{
Дослівно: „після свята“, коли справу вже закінчено. \emph{Ред.}
}, можуть і мусять
завжди поставати великі порушення. З одного боку, тиск на грошовий
ринок, тимчасом як гарний стан грошового ринку, навпаки, і собі покликає
до життя багато таких підприємств, отже, призводить саме до таких обставин,
що потім зумовлюють тиск на грошовий ринок. Грошовий ринок
зазнає тиску, бо при цьому треба постійно авансувати великий грошовий
капітал на довгий час. Ми вже зовсім не кажемо про те, що промисловці
й торговці кидають на залізничні спекуляції і таке інше грошовий капітал,
потрібний їм для провадження власних підприємств, і заміщують його позиками
на грошовому ринку. — З другого боку, зазнає тиску продуктивний капітал,
що є в розпорядженні суспільства. А що елементи продуктивного капіталу
постійно вилучається з ринку і натомість на ринок подається лише
грошовий еквівалент, то більшає виплатоспроможний попит, який, із свого
боку, не має в собі жодних елементів подання. Відси підвищення цін
і на засоби існування, і на продукційні матеріяли. До цього долучається
ще й те, що під такий час звичайно розвивається шахрайство і переміщується
чимало капіталу. Зграя спекулянтів, постачальників, інженерів,
адвокатів тощо збагачується. Вони спричиняють на ринку великий попит
на речі споживання, поряд цього підвищується заробітна плата. Щодо
попиту на харчові засоби, то він звичайно підганяє й сільське господарство.
А що цих харчових засобів не можна збільшити одразу, протягом
року, то більшає довіз їх, як і взагалі довіз екзотичних харчових
засобів (кави, цукру, вина тощо) та речей розкошів. Звідси надмірний
довіз і спекуляція в цій галузі імпортної торговлі. З другого боку, в
тих галузях промисловости де продукцію можна швидко збільшити (власне
мануфактура, гірництво тощо), підвищення цін призводить до раптового
поширення, що по ньому скоро настає крах. Такий самий вплив
справляється на робочий ринок, щоб притягти до нових галузей підприємств
великі маси лятентного відносного надміру людности і навіть робітників,
уже зайнятих. Взагалі такі підприємства великого маштабу, як
от залізниці, відтягують від робочого ринку певну кількість сил, що
\parbreak{}  %% абзац продовжується на наступній сторінці

\enablefootnotebreak{}
\parcont{}  %% абзац починається на попередній сторінці
\index{iii1}{0238}  %% посилання на сторінку оригінального видання
І до спожитих на їх виробництво засобів праці дедалі знижується; отже, та обставина, що в цих
товарах упредметнюється дедалі менша кількість додаваної живої праці, бо з розвитком суспільної
продуктивної сили потрібно менше праці для їх виробництва — ця обставина не стосується до того
відношення, в якому уміщена в товарі жива праця ділиться на оплачену і неоплачену. Навпаки. Хоча
загальна кількість уміщеної в товарі додаваної живої праці зменшується, неоплачена частина зростає
порівняно з оплаченою в наслідок або абсолютного, або відносного зниження оплаченої частини; бо той
самий спосіб виробництва, який зменшує загальну масу додаваної живої праці в кожному окремому
товарі, супроводиться зростанням абсолютної і відносної додаткової вартості. Тенденція норми зиску
до зниження зв’язана з тенденцією до підвищення норми додаткової вартості, отже й ступеня
експлуатації праці. Тому немає нічого більш безглуздого, як поясняти зниження норми зиску
підвищенням норми заробітної плати, хоч винятково і це може мати місце. Тільки зрозумівши відносини,
при яких утворюється норма зиску, статистика стає спроможною взятись до дійсного аналізу норми
заробітної плати в різні епохи і в різних країнах. Норма зиску падає не тому, що праця стає менш
продуктивною, а тому, що вона стає більш продуктивною. І те і друге, підвищення норми додаткової
вартості і падіння норми зиску, є тільки особливі форми, в яких капіталістично виражається зростаюча
продуктивність праці.

\subsection{Збільшення акційного капіталу}

До вищенаведених п’яти пунктів можна додати ще один пункт, на якому ми, однак, покищо не можемо
детальніше спинятись. З прогресом капіталістичного виробництва, який іде рука в руку з прискореним
нагромадженням, частина капіталу враховується і застосовується тільки як капітал, що дає процент. Не
в тому розумінні, в якому кожний капіталіст, що позичає комусь капітал, задовольняється процентами,
тимчасом як промисловий капіталіст одержує підприємницький зиск. Це не стосується до висоти
загальної норми зиску, бо для неї зиск \deq{} процентові \dplus{} зиск усякого роду \dplus{} земельна рента, при чому
розподіл на ці особливі категорії для неї не має значення. А в тому
розумінні, що ці капітали, хоч і вкладені у великі продуктивні підприємства, дають після
відрахування всіх витрат тільки великі або малі проценти, так звані дивіденди; наприклад, у
залізничній справі. Отже, вони не беруть участі у вирівнюванні загальної норми зиску, бо вони дають
меншу норму зиску, ніж пересічна норма. Коли б вони брали участь у вирівнюванні, то ця остання упала
б значно нижче. Розглядаючи справу теоретично, їх можна врахувати, і тоді одержимо меншу норму
зиску, ніж та, яка, видимо, існує і дійсно є визначальною для капіталістів, — одержимо меншу норму
зиску, бо саме в цих підприємствах сталий капітал є найбільший порівняно з змінним.

\parcont{}  %% абзац починається на попередній сторінці
\index{iii2}{0239}  %% посилання на сторінку оригінального видання
відчужености й самостійности проти праці, отже, та перетворена форма умов
праці, що в ній випродуковані засоби продукції пертворюються на капітал, а
земля на монополізовану землю, на земельну власність, — ця належна певному
історичному періодові форма збігається, отже, з буттям і функцією випродукованих
засобів продукції і землі в процесі продукції взагалі. Ці засоби продукції
є капіталом сами по собі, з природи; капітал є не що інше, як просто «економічний
термін» для цих засобів продукції; і отак земля є сама по собі, з
природи, землею, монополізованою певного кількістю земельних власників. Як
у капіталі і в капіталістові, — який на ділі є не що інше, як персоніфікований
капітал, — продукти стають самостійною силою проти продуцентів, так і в земельному
власникові персоніфікується земля, яка теж стає дибки, і як самостійна
сила вимагає своєї частини у випродукованому з її допомогою продукті;
так що не земля одержує належну їй частину продукту для відновлення
і підвищення продуктивности, а замість неї земельний власник одержує частину
цього продукту на прогулювання і марнотратство. Ясно, що капітал має
своєю предумовою працю як найману працю. Але так само ясно, що коли виходити
з праці як найманої праці, так що тотожність праці взагалі з найманою
працею уявляється самоочевидною, то тоді капітал і монополізована земля
в свою чергу мусять уявлятись природною формою умов праці проти праці
взагалі. Бути капіталом, — це уявляється тепер природною формою засобів праці,
а тому й як їх суто-речовий характер, що виникає з їхньої функції
в процесі праці взагалі. Таким чином, капітал і випродукований засіб продукції
стають тотожніми виразами. Так само земля і монополізована приватною власністю
земля стають тотожніми виразами. Тому за джерело зиску стають засоби
праці як такі, як капітал з природи, так само як за джерело ренти стає земля
як така.

Працю як таку, в її простій визначеності доцільної продуктивної діяльности,
ставиться в відношення до засобів продукції, взятим не в їхній суспільній
визначеності форми, а в їхній речевій субстанції як матеріялу і засобів праці,
що відрізняються між собою теж лише речево, як споживні вартості: земля —
як невипродукований засіб праці, інші — як випродуковані засоби праці. Отже,
коли праця збігається з найманою працею, то та певна суспільна форма, в
якій умови праці протистоять тепер праці, також збігається з їхнім речевим
буттям. Тоді засоби праці як такі є капітал, і земля як така є земельна
власність. Тоді формальне усамостійнення цих умов праці, проти праці і та
особлива форма цього усамостійнення, яку вони мають проти найманої праці, видається
властивістю невідійманною від них як від речей, як від матеріяльних умов
продукції, видається властивістю, що неодмінно належить та іманентно зрослася
з ними як з елементами продукції. Їхній визначуваний певною історичною
епохою певний соціяльний характер в капіталістичному процесі продукції видається
їхнім речовим характером, природно і так би мовити споконвіку природженим
їм, як елементам процесу продукції. Тому відповідна участь, яку земля
як первісна сфера діяльности праці, як царство природних сил, як наявний
арсенал усіх речей праці, і та друга відповідна участь, яку випродуковані засоби
продукції (знаряддя, сирові матеріяли тощо) беруть у процесі продукції
взагалі — мусять тоді здаватись участю, що знаходить собі вираз у відповідних
частинах, які в формі зиску (проценту) і ренти дістаються їм як капіталові і
земельній власності, тобто їхнім соціяльним представникам, як для робітника
мусить здаватись, що та участь, яку його праця бере в процесі продукції,
виражається в заробітній платі. Таким чином, здається, що рента, зиск, заробітна
плата породжуються тією ролею, яку земля, випродуковані засоби продукції й
праця відіграють у простому процесі праці, навіть і тоді, коли б ми розглядали
цей процес праці як процес просто між людиною й природою і абстрагувались
\parbreak{}  %% абзац продовжується на наступній сторінці

\parcont{}  %% абзац починається на попередній сторінці
\index{iii2}{0240}  %% посилання на сторінку оригінального видання
від усякої історичної його визначености. Ми маємо знову те саме тільки в іншій
формі, коли кажуть: продукт, в якому втілюється праця найманого робітника
на себе самого, як його здобуток, його дохід, це лише заробітна плата, та частина
вартости (а тому й суспільного продукту, вимірюваного цією вартістю),
яка становить його заробітну плату. Отже, коли наймана праця збігається з
працею взагалі, то й заробітна плата збігається з продуктом праці, і та частина
вартости, яка репрезентована заробітною платою, збігається з вартістю, взагалі
створеною працею. Але в наслідок цього і інші частини вартости, зиск і рента,
так само самостійно протиставляться заробітній платі, і їх доводиться виводити
з власних джерел, специфічно відмінних і незалежних від праці; їх доводиться
виводити з співдіющих елементів продукції, що посідачам їх вони припадають,
отже, зиск доводиться виводити з засобів продукції, речевих елементів капіталу,
а ренту з землі або природи, репрезентованої земельним власником (Рошер).

Тому земельна власність, капітал і наймана праця з джерел доходу в тому
розумінні, що капітал притягає в формі зиску до капіталіста ту частину додаткової
вартости, яку він здобуває з праці, монополія на землю притягає до
земельного власника іншу частину в формі ренти, а праця дає робітникові в
формі заробітної плати останню ще вільну частину вартости, з джерел доходу,
що за їх посередництвом одна частина вартости перетворюється на форму зиску,
друга на форму ренти і третя на форму заробітної плати, — перетворюються на
дійсні джерела, що з них виникають ці частини вартости і ті відповідні частини
продукту, що в них вони існують або на які вони можуть бути обмінені —
на джерела, з яких кінець-кінцем виникає сама вартість продукту\footnote{
«Заробітна плата, зиск і рента є три первісні джерела всякого доходу, так само як і всієї
мінової вартости» (А.~Smith). «Таким чином, причини матеріяльної продукції є одночасно джерела всіх
сущих основних доходів» (Storch, І, р. 259).
}.

Розглядаючи простіші категорії капіталістичного способу продукції, і навіть
товарової продукції, товар і гроші, ми вже зазначали той містифікаційний
характер, що перетворює суспільні відносини, що для них при продукції речеві
елементи багатства правлять за носіїв, на властивості самих цих речей (товари)
і ще яскравіше, саме продукційне відношення — на річ (гроші). Всі форми
суспільства, оскільки вони призводять до товарової продукції і грошової циркуляції,
беруть участь у цьому перекрученні. Але за капіталістичного способу
продукції й за капіталу, який є його панівною категорією, його визначальним
відношенням продукції, цей зачарований і перекручений світ розвивається геть
більше. Коли розглядати капітал, насамперед в безпосередньому процесі продукції,
— як висмоктувана додаткової праці, — то це відношення ще дуже просте;
і дійсний внутрішній зв’язок ще нав’язується носіям цього процесу, самим
капіталістам і ще усвідомлюється ними. Це переконливо доводиться упертою
боротьбою за межі робочого дня. Але навіть всередині цієї неускладненої сфери,
сфери безпосереднього процесу між працею й капіталом, справа не лишається
така проста. З розвитком відносної додаткової вартости за власне специфічного
капіталістичного способу продукції, в наслідок чого розвиваються і суспільні
продуктивні сили праці, — ці продуктивні сили і суспільні відносини праці виступають
у безпосередньому процесі праці в такому вигляді, як ніби з праці
вони перенесені в капітал. Тим самим капітал стає дуже таємничою істотою,
бо всі суспільні продуктивні сили праці виступають у такому вигляді,
ніби вони належать йому, а не праці як такій, і як такі сили, що народжуються
в його власних надрах. А потім втручається процес циркуляції, що
в його обмін речовин і зміну форм втягується всі частини капіталу, навіть
хліборобського капіталу, в тій самій мірі, в якій розвивається специфічно
капіталістичний спосіб продукції. Це є така сфера, в якій відносини первісної
\parbreak{}  %% абзац продовжується на наступній сторінці

\parcont{}  %% абзац починається на попередній сторінці
\index{iii2}{0241}  %% посилання на сторінку оригінального видання
продукції вартости відступають цілком на задній плян. Вже в безпосередньому
процесі продукції капіталіст діє одночасно як товаропродуцент, як керівник
товарової продукції. Тому цей процес продукції зовсім не уявляється йому просто
процесом продукції додаткової вартости. Але хоч би яка була та додаткова
вартість, яку капітал у безпосередньому процесі продукції висмоктував і втілював
в товари, вартість і додаткова вартість, що міститься в товарах, мусить реалізуватись
лише в процесі циркуляції. І справа набуває такого вигляду, ніби вартість,
яка покриває вартості, авансовані на продукцію, і особливо додаткова
вартість, що міститься в товарах, не просто реалізуються в циркуляції, але виникають
з неї; цю ілюзію особливо зміцнюють дві обставини: поперше, зиск,
одержуваний при відчуженні, залежить від обману, хитрощів, знання справи,
спритности й тисячі ринкових коньюнктур; а подруге, та обставина, що тут
поряд з робочим часом виступає другий визначальний елемент, час циркуляції.
Хоч він функціонує тільки як негативна межа створення вартости і додаткової
вартости, але має таку подобу, ніби він є так само позитивна причина
їх створення, як сама праця, і ніби він додає незалежне від праці визначення,
що походить з природи капіталу. У книзі II нам, природно, довелось подати цю
сферу циркуляції лише в її відношенні до визначень форм, які вона породжує,
показати дальший розвиток структури капіталу, який відбувається в цій сфері.
Але в дійсності ця сфера є сфера конкуренції, над якою, коли розглядати кожен
окремий випадок, панує випадковість; отже, сфера, в якій внутрішній закон,
що пробивається серед цих випадковостей і регулює їх, стає видимим лише тоді,
коли сполучити ці випадковості в велику масу, в якій, отже, він лишається
невидимим і незрозумілим для самих окремих аґентів продукції. Але далі: дійсний
процес продукції, як єдність безпосереднього процесу продукції і процесу
циркуляції, породжує нові витвори, в яких дедалі більше втрачається нитка
внутрішнього зв’язку, відносини продукції взаємно усамостійнюються, і складові
частини вартости костеніють у самостійних одна проти однієї формах.

Як ми бачили, перетворення додаткової вартости на зиск визначається так
процесом циркуляції, як і процесом продукції. Додаткова вартість, у формі зиску,
відноситься вже не до витраченої на працю частини капіталу, з якої вона виникає,
а до всього капіталу. Норма зиску реґулюється власними законами, що
допускають і навіть зумовлюють її зміну за незмінної норми додаткової
вартости. Все це дедалі більше затушковує справжню природу додаткової вартости,
а тому й дійсний механізм капіталу. Ще в більшій мірі стається це
в наслідок перетворення зиску на пересічний зиск і вартостей на ціни продукції,
на реґуляційні пересічні ринкових цін. Тут втручається складний суспільний
процес, процес вирівнювання капіталів, який відриває відносні пересічні
ціни товарів від їхніх вартостей, і пересічні зиски в різних сферах продукції
(залишаючи цілком осторонь індивідуальні вкладання капіталу в кожній окремій
сфері продукції) від дійсної експлуатації праці окремими капіталами. Тут не
тільки так здається, але й дійсно пересічна ціна товарів відмінна від їхньої
вартости, отже, від реалізованої в них праці, і пересічний зиск окремого капіталу
відмінний від додаткової вартости, яку цей капітал здобув з зайнятих ним
робітників. Вартість товарів виявляється безпосередньо лише в тому впливі, що
його справляють зміни продуктивної сили праці на пониження та підвищення цін
продукції, на їхній рух, а не на їхні кінцеві межі. Зиск, як здається, визначається
безпосередньою експлуатацією праці лише випадково, лише остільки,
оскільки ця експлуатація дає капіталістові можливість за наявности реґуляційних
ринкових цін, які видаються незалежними від цієї експлуатації, реалізувати
зиск, що відхиляється від пересічного зиску. Щодо самих нормальних пересічних
зисків, то вони здаються іманентними капіталові, незалежно від експлуатації;
ненормальна експлуатація, а також пересічна експлуатація за сприятливих
\parbreak{}  %% абзац продовжується на наступній сторінці

\parcont{}  %% абзац починається на попередній сторінці
\index{i}{0242}  %% посилання на сторінку оригінального видання
капітал, є в певних межах незалежне від подання робітників\footnote{
Цей елементарний закон, здається, невідомий вульґарним економістам
які, Архімеди навиворіт, гадають, що у визначенні ринкових цін
праці попитом і поданням вони знайшли пункт опори не для того, щоб
перевернути світ, а щоб спинити його рух.
}.
Навпаки, зменшення норми додаткової вартости лишає масу
випродукованої додаткової вартости незмінною, коли величина
змінного капіталу або число вживаних робітників пропорційно
зростає. [Змінний капітал у 100 талярів, що експлуатує 100 робітників
при нормі додаткової вартости в 100\%, продукує 100 талярів
додаткової вартости. Норма додаткової вартости може зменшитися
вдвоє, але сума її лишається та сама, коли одночасно
подвоюється змінний капітал]\footnote*{
Заведене у прямі дужки ми беремо з французького видання. \Red{Ред.}
}.

Однак, компенсація числа робітників або величини змінного
капіталу збільшенням норми додаткової вартости або здовженням
робочого дня має межі, що їх не сила переступити. Хоч яка
буде вартість робочої сили, отже, все одно, чи робочий час,
доконечний для утримання робітника, становить 2 чи 10 годин,
загальна вартість, яку робітник може продукувати день-у-день,
є завжди менша від вартости, в якій упредметнюються 24 робочі
години, менша від 12\shil{ шилінґів}, або 4 талярів, коли вони є грошовий
вираз 24 упредметнених робочих годин.

[Щодо додаткової вартости, то її межі ще вужчі. Коли частина
робочого дня, доконечна для покриття денної заробітної плати,
становить 6 годин, то від природного дня залишається тільки 18 годин,
що з них, за біологічними законами, частина потрібна для
відпочинку робочої сили. Припустімо, що 6 годин є мінімальна
межа цього відпочинку; коли здовжити робочий день до його максимальної
межі, до 18 годин, то додаткова праця становитиме
лише 12 годин і, отже, спродукує вартість лише в 2 таляри]\footnote*{
Цей абзац ми беремо з французького видання. \Red{Ред.}
}.

За нашої попередньої передумови, за якою потрібно на день
6 робочих годин, щоб репродукувати саму робочу силу, або
покрити капітальну вартість, авансовану на її купівлю, змінний
капітал у 500 талярів, який уживає 500 робітників при нормі
додаткової вартости в 100\%, або за 12-годинного робочого дня,
продукує денно додаткову вартість у 500 талярів, або 6 × 500
робочих годин. А капітал у 100 талярів, який денно вживає
100 робітників при нормі додаткової вартости в 200\%, або за 18-годинного
робочого дня, продукує масу додаткової вартости лише
в 200 талярів, або 12 × 100 робочих годин. Ціла нововипродукована
ним вартість, еквівалент авансованого змінного капіталу
плюс додаткова вартість, ніколи не може досягти суми 400 талярів,
або 24 × 100 робочих годин пересічно на день. Абсолютна
межа пересічного робочого дня, який з природи є завжди менший
від 24 годин, становить абсолютну межу для компенсації зменшення
змінного капіталу підвищенням норми додаткової вартости,
або зменшення числа експлуатованих робітників підвищенням
\parbreak{}  %% абзац продовжується на наступній сторінці

\parcont{}  %% абзац починається на попередній сторінці
\index{i}{0243}  %% посилання на сторінку оригінального видання
ступеня експлуатації робочої сили. Цей цілком очевидний другий
закон є важливий для пояснення багатьох явищ, які випливають
із тенденції капіталу, що її ми маємо розвинути пізніш,
а саме тенденції якомога більше скорочувати число робітників,
що їх він вживає, або його змінну складову частину, перетворену
на робочу силу, — всупереч іншій його тенденції, а саме продукувати
якомога більшу масу додаткової вартости. Навпаки, коли
маса вживаних робочих сил або величина змінного капіталу
зростає, алеж непропорційно до зменшення норми додаткової
вартости, то маса продукованої додаткової вартости меншає\footnote*{
У французькому виданні цей абзац подано так: «Цей абсолютно
ясний закон є важливий для розуміння складних явищ. Ми вже знаємо,
що капітал намагається продукувати якомога більше додаткової вартости;
ми побачимо пізніш, що він разом із цим намагається скоротити до
мінімуму, порівняно з розмірами підприємства, свою змінну частину,
або кількість робітників, що їх він експлуатує. Ці тенденції стають одна
одній суперечними, скоро лише зменшення одного з факторів, що визначають
суму додаткової вартости, вже не може бути компенсоване збільшенням
другого». («Le Capital etc.», v. I, ch. XI, p. 132). \Red{Ред.}
}.

Третій закон випливає з визначення маси продукованої додаткової
вартости двома факторами: нормою додаткової вартости й
величиною авансованого змінного капіталу. Коли дано норму
додаткової вартости, або ступінь експлуатації робочої сили,
і вартість робочої сили, або величину доконечного робочого
часу, то само собою зрозуміло, що чим більший змінний капітал,
тим більша маса продукованої вартости й додаткової вартости.
Коли дано межі робочого дня, а також межі його доконечної
складової частини, то маса вартости й додаткової вартости,
що її продукує поодинокий капіталіст, очевидно, залежить
виключно від тієї маси праці, яку він пускає в рух. Але
маса ця, за даних припущень, залежить від маси робочої сили,
або від числа робітників, яких він експлуатує; а це число, з
свого боку, визначається величиною авансованого ним змінного
капіталу. Отже, за даної норми додаткової вартости й даної вартости
робочої сили маси продукованої додаткової вартости є
просто пропорційні до величин авансованих змінних капіталів.
Та тепер уже відомо, що капіталіст ділить свій капітал на
дві частини. Одну частину він вкладає в засоби продукції. Це —
стала частина його капіталу. Другу частину він перетворює на
живу робочу силу. Ця частина становить його змінний капітал.
На базі того самого способу продукції в різних галузях продукції
відбувається різний поділ капіталу на сталу та змінну складові
частини. В тій самій галузі продукції це відношення змінюється
разом із зміною технічної основи й суспільних комбінацій процесу
продукції. Але хоч як розпадатиметься даний капітал на сталу
й змінну складові частини, чи остання відноситиметься до першої
як $1 : 2$, $1 : 10$, або $1 : х$, — це не порушує щойно встановленого
закону, бо, згідно з попередньою аналізою, вартість сталого капіталу
хоч і з’являється знов у вартості продукту, але не увіходить
у новоутворену вартість. Щоб уживати \num{1.000} прядунів, потрібно,
\parbreak{}  %% абзац продовжується на наступній сторінці

\parcont{}  %% абзац починається на попередній сторінці
\index{iii2}{0244}  %% посилання на сторінку оригінального видання
тим, що норма зиску зростає, коли товар, проданий нижче від його вартости,
становить елемент сталого капіталу, або тим, що зиск і рента втілюються
в більшій кількості продукту, коли товар, проданий нижче від його вартости,
входить як річ особистого споживання в частину вартости, споживану як дохід.
А подруге, це знищується в пересічних коливаннях. В усякому разі, коли навіть
частина додаткової вартости, яка не реалізувалася в ціні товару, не бере участи
в створенні ціни, — сума пересічного зиску плюс рента в її нормальній формі
ніколи не може бути більша за всю додаткову вартість, хоч і може бути
менша за неї.

Її нормальна форма має своєю передумовою заробітну плату, відповідну
до вартости робочої сили. Навіть монопольна рента, оскільки вона не є вирахування
із заробітної плати, отже, не являє собою осібної категорії, посередньо
мусить завжди становити частину додаткової вартости; коли вона і не являє
собою частини надміру ціни над ціною продукції того самого товару, що вона
є його складова частина (як за диференційної ренти), або коли вона і не являє
собою надмірної частини додаткової вартости того самого товару, що вона є його
складова частина, над частиною його власної додаткової вартости, вимірюваної
пересічним зиском (як за абсолютної ренти), то все таки вона становить частину
додаткової вартости інших товарів, тобто товарів, обмінюваних на цей товар,
що має монопольну ціну. — Сума пересічного зиску плюс земельна рента ніколи
не можуть перебільшувати величини, що частинами її є ці зиски і рента, і що її
вже дано до цього поділу. Тому для нашого дослідження байдуже, чи реалізується
в ціні товарів уся додаткова вартість товарів, тобто вся додаткова праця,
що міститься в товарах, чи ні. Додаткова праця вже тому не реалізується цілком,
що при постійній зміні кількости праці, суспільно потрібної для продукції даного
товару, що виникає з постійної зміни продуктивної сили праці, частину
товарів завжди продукується в ненормальних умовах, а тому їх доводиться
продавати нижче від їхньої індивідуальної вартости. В усякому разі зиск
плюс рента дорівнюють усій реалізованій додатковій вартості (додатковій праці)
і для дослідження, про яке тут іде мова, реалізовану додаткову вартість можна
вважати за рівну всій додатковій вартості, бо зиск і рента є реалізована додаткова
вартість, отже, взагалі та додаткова вартість, що входить в ціни товарів,
отже, практично вся та додаткова вартість, яка є складова частина цієї ціни.

З другого боку, заробітна плата, що становить третю своєрідну форму
доходу, завжди дорівнює змінній складовій частині капіталу, тобто тій складовій
частині, яку витрачається не на засоби праці, а на купівлю живої робочої
сили, на виплату робітникам. (Працю, оплачувану при витрачанні доходу, оплачується
з заробітної плати, зиску або ренти і тому вона не становить частини
вартости товарів, що ними її оплачується. Таким чином, її не береться на увагу
при аналізі вартости товарів і складових частин, на які вона розпадається).
Вартість змінного капіталу, отже, і ціна праці репродукується в певній частині
усього зрічевленого робочого дня робітників, в тій частині товарової вартости,
в якій робітник репродукує вартість своєї власної робочої сили або ціну своєї
праці. Весь робочий день робітника розпадається на дві частини. Одна частина
та, підчас якої він виконує кількість праці, потрібну для репродукції вартости
його власних засобів існування: оплачена частина всієї його праці, та частина
його праці, що потрібна для його власного збереження і репродукції. Вся решта
робочого дня, вся надмірна кількість праці, яку він виконує понад працю, реалізовану
в вартості його заробітної плати, є додаткова праця, неоплачена праця,
що втілюється в додатковій вартості усіх випродукованих ним товарів (і тому
в надмірній кількості товару), в додатковій вартості, яка й собі розпадається на
частини з різними назвами, на зиск (підприємницький бариш плюс процент)
та ренту.


\index{ii}{0245}  %% посилання на сторінку оригінального видання
Можливі лише два нормальні випадки репродукції, якщо залишити
осторонь ті порушення, що перешкоджають навіть репродукції в попередньому
маштабі.

Або відбувається репродукція в простому маштабі.

Або відбувається капіталізація додаткової вартости, акумуляція.

\subsection{Проста репродукція}

При простій репродукції додаткова вартість, продукована й реалізовувана
щорічно або — при кількох оборотах — періодично протягом року,
споживається особисто, тобто непродуктивно, її власником, капіталістом.

Та обставина, що вартість продукту складається почасти з додаткової
вартости, почасти з тієї частини вартости, яка складається з репродукованого
в ньому змінного капіталу плюс зужиткований на його продукцію
сталий капітал, — ця обставина абсолютно нічого не змінює ні в кількості,
ні в вартості цілого продукту, що постійно надходить в циркуляцію, як
товаровий капітал, і так само постійно вилучається з неї для продуктивного
або особистого споживання, тобто для того, щоб служити засобом
продукції або засобом споживання. Якщо сталий капітал залишити осторонь,
то ця обставина впливає тільки на розподіл річного продукту між
робітниками й капіталістами.

Тому, навіть коли припустити просту репродукцію, частина додаткової
вартости має постійно перебувати в формі грошей, а не в формі
продукту, бо інакше її не можна перетворити з грошей на продукт для
споживання. Це перетворення додаткової вартости з її первісної товарової
форми на гроші треба тут дослідити далі. Для спрощення справи
візьмімо проблему в її найпростішій формі, а саме припустімо циркуляцію
виключно металевих грошей, грошей, що являють дійсний грошовий еквівалент.

Згідно з законами простої товарової циркуляції (кн. І, розд. III), маси
наявних у країні металевих грошей має вистачити не лише для
циркуляції товарів. Її має вистачити для того, щоб вирівнювати коливання
грошового обігу, що випливають почасти з флюктуацій\footnote*{
Від лат. слова „fluctus“, гра хвиль, хвилювання, почережне
піднесення й спад. \emph{Ред.}
} в швидкості
циркуляції, почасти з змін товарових цін, почасти з різних та
змінних відношень, що в них функціонують гроші як засіб виплати або
як власне засіб циркуляції. Відношення, що в ньому наявна маса грошей
розподіляється на скарб і на гроші в циркуляції, раз-у-раз змінюється, але
маса грошей завжди дорівнює сумі грошей, наявних у формі скарбу та
в формі грошей в циркуляції. Ця маса грошей (маса благородного металю)
є поступінно нагромаджуваний скарб суспільства. Оскільки частина цього
скарбу зужитковується через зношування, її треба щорічно знову заміщувати,
як і всякий інший продукт. Це в дійсності і відбувається через
\parbreak{}  %% абзац продовжується на наступній сторінці


\index{iii2}{0246}  %% посилання на сторінку оригінального видання
Щодо першої трудности: хто повинен оплатити вміщену в продукті сталу
частину вартости і чим? — то припускається, що вартість сталого капіталу, зужиткованого у продукції,
знову з’являється як частина вартости продукту. Це не
суперечить засновкам другої трудности. Бо вже в книзі І, розділ  V (процес праці і
процес зростання вартости) показано, яким чином в наслідок простого долучення
нової праці, хоч вона і не репродукує старої вартости, а створює лише додаток
до неї, створює лише додаткову вартість, все таки разом з тим зберігається
у продукті стара вартість; але одночасно показано, що це відбувається як наслідок
праці не остільки, оскільки вона є вартостетворча, тобто праця взагалі, а в її
функції як певної продуктивної праці. Таким чином, не треба жодної новодолучуваної
праці для того, щоб зберегти вартість сталої частини в тому продукті,
на який витрачається дохід, тобто вся створена протягом року вартість. Але,
зрозуміла річ, потрібна новодолучувана праця для того, щоб покрити вартість
і споживну вартість сталого капіталу, зужиткованого протягом минулого
року. Без такого покриття репродукція взагалі неможлива.

Вся новодолучена праця втілюється у новоствореній протягом року вартості,
яка і собі цілком сходить на три види доходу: заробітну плату,
зиск і ренту. — Отже, з одного боку, не лишається надмірної суспільної праці
для покриття зужиткованого сталого капіталу, що підлягає відновленню почасти
in natura і в його вартості, почасти тільки в його вартості (оскільки справа
йде просто про зношування основного капіталу). З другого боку, вартість, що
створена річною працею і розпадається на форми заробітної плати, зиску і ренти,
і яка в цьому вигляді підлягає витрачанню, є недостатня для того, щоб оплатити
або купити сталу частину капіталу, яка теж мусить міститися в продукті,
крім новоствореної вартости.

Ми бачимо, що поставлену тут проблему вже розв’язано при дослідженні
репродукції сукупного суспільного капіталу, книга II, відділ III.~Тут ми вертаємось
до цього насамперед тому, що там додаткова вартість ще не була розгорнута
в тих її формах, яких вона набуває як дохід: зиск (підприємницький
бариш плюс процент) і рента, а тому і не могла бути досліджена в цих формах; потім також і тому, що
якраз з формою заробітної плати, зиску і ренти
сполучається неймовірний прогріх в аналізі, який проходить через усю політичну
економію, починаючи від А.~Сміта.

Ми поділили там увесь капітал на дві великі кляси: кляса І, що створює
засоби продукції, кляса II, що продукує засоби індивідуального споживання.
Та обставина, що деякі продукти можуть так само правити за речі особистого
користування, як і засоби продукції (кінь, збіжжя тощо) зовсім не знищує абсолютної
правдивости цього поділу. Справді, він не гіпотеза, а лише вираз факту.
Візьмімо річний продукт якоїсь країни. Частина продукту, хоч яка б була здатність
його правити за засіб продукції, входить в індивідуальне споживання.
Це — продукт, на який витрачається заробітну плату, зиск і ренту. Продукт цей
становить продукт певного підрозділу суспільного капіталу. Можливо, що цей
самий капітал продукує також і продукти, що належать до кляси І.~Оскільки
це так, продуктивно спожиті продукти, належні до кляси І, постачаються не
тією частиною цього капіталу, що зужиткована на продукт кляси II, на продукт,
який дійсно дістається індивідуальному споживанню. Весь той продукт
II, що входить в індивідуальне споживання, і на який тому витрачається
дохід, є формою буття зужиткованого на нього капіталу плюс випродукований
надмір. Отже, це — продукт капіталу, вкладеного тільки в продукцію
засобів споживання. І в цьому ж розумінні підрозділ І річного продукту, який
править за засоби репродукції, — сирового матеріялу і знарядь праці, — хоч би
яка взагалі була здатність цього продукту naturaliter правити за засоби споживання,
— є продукт капіталу, вкладеного виключно в продукцію засобів
\parbreak{}  %% абзац продовжується на наступній сторінці

\parcont{}  %% абзац починається на попередній сторінці
\index{ii}{0247}  %% посилання на сторінку оригінального видання
в майбутньому зросла б порівняно з товарами, котрих вартість не змінюється;
тобто ціни товарів знизились би, отже, в майбутньому грошова
сума, витрачувана в акті $Г — Т$, зменшилась би.

Якщо розглядати насамперед тільки обігову частину капіталу, авансовуваного
в $Г$, вихідному пунктові $Г — Т\dots{} П\dots{} Г'$, то виявиться, що певну
грошову суму авансується, пускається в циркуляцію на оплату робочої
сили й на закуп матеріялів продукції. Але через кругобіг \emph{цього} капіталу
її не вилучається знову з циркуляції, щоб знову ж таки подати її
в неї. Продукт є гроші вже в своїй натуральній формі, отже, йому не
доводиться перетворюватись на гроші через обмін, через процес
циркуляції. З процесу продукції в сферу циркуляції він увіходить не в
формі товарового капіталу, що повинен перетворитись на грошовий капітал,
а як грошовий капітал, що повинен перетворитись знову на продуктивний
капітал, тобто знову купувати робочу силу й матеріяли продукції.
Грошову форму обігового капіталу, зужитого на робочу силу й на засоби
продукції, заміщується не через продаж продукту, а натуральною
формою самого продукту; отже, не через зворотне вилучення з циркуляції
вартости продукту в її грошовій формі, а через додаткові новоспродуковані
гроші.

Припустімо, що цей обіговий капітал \deq{} 500\pound{ ф. стерл.}, період обороту
\deq{} 5 тижням, робочий період \deq{} 4 тижням, період циркуляції \deq{} лише 1
тижневі. Гроші з самого початку треба авансувати на 5 тижнів почасти
на продукційний запас, почасти вони мають бути в запасі, щоб можна
було поступінно виплачувати з них заробітну плату. На початку шостого
тижня повертаються назад 400\pound{ ф. стерл.} і звільняються 100\pound{ ф. стерл}. Це
повторюється раз-у-раз. Тут, як і раніш, протягом певного часу обороту
100\pound{ ф. стерл.} завжди будуть в формі вільних грошей. Але вони складаються
з додаткових новоспродукованих грошей, цілком так само, як і останні
400\pound{ ф. стерл}. Ми маємо тут 10 оборотів на рік і спродукований річний
продукт \deq{} 5000\pound{ ф. стерл.} золотом. (Період циркуляції складається тут
не з того часу, що його потребує перетворення товару на гроші, а з
часу, потрібного для перетворення грошей на елементи продукції).

Для всякого іншого капіталу в 500\pound{ ф. стерл.}, що обертається в таких
самих умовах, раз-у-раз відновлювана грошова форма є перетворена форма
спродукованого товарового капіталу, який що чотири тижні подається
в циркуляцію і який у наслідок продажу — отже, в наслідок періодичного
вилучення такої кількости грошей, яка спочатку ввійшла в процес, —
знову й знову набирає цієї грошової форми. Навпаки, тут в кожний період
обороту з самого процесу продукції подається в циркуляцію нову
додаткову суму грошей в 500\pound{ ф. стерл.} для того, щоб постійно вилучати
відти матеріяли продукції та робочу силу. Цих поданих у циркуляцію
грошей не вилучається знову з неї через кругобіг цього капіталу, а їх
дедалі збільшують, додаючи раз-у-раз новоспродуковані маси золота.

Коли ми розглянемо змінну частину цього обігового капіталу й припустимо,
як і перше, що вона дорівнює 100\pound{ ф. стерл.}, то при звичайній
товаровій продукції цих 100\pound{ ф. стерл.} при десятьох оборотах на рік було б
\parbreak{}  %% абзац продовжується на наступній сторінці

\parcont{}  %% абзац починається на попередній сторінці
\index{iii1}{0248}  %% посилання на сторінку оригінального видання
пересічного робочого часу, суспільно-необхідного для виробництва
товарів. І одночасно зростає концентрація, бо за певними
межами великий капітал з невеликою нормою зиску нагромаджує
швидше, ніж невеликий капітал з великою нормою зиску. Ця
зростаюча концентрація, з свого боку, досягнувши певної висоти,
знов таки приводить до нового падіння норми зиску. Маса дрібних
розпорошених капіталів у наслідок цього штовхається на шлях
авантюр: спекуляцій, шахрайських кредитних і акційних підприємств,
криз. Так звана плетора [наддостаток] капіталу завжди
стосується головним чином до плетори такого капіталу, для якого
падіння норми зиску не урівноважується масою зиску, — а такі
завжди є новоутворювані свіжі паростки капіталу, — або до плетори
таких капіталів, які, будучи самі по собі нездатними самостійно
функціонувати, передаються в формі кредиту в розпорядження
керівників великих галузей підприємств. Ця плетора капіталу виростає
з тих самих обставин, які викликають відносне перенаселення,
і тому вона є явище, яке доповнює це останнє, хоч обоє
вони перебувають на протилежних полюсах: на одному боці — незанятий
капітал, на другому боці — незаняте робітниче населення.

Перепродукція капіталу, а не окремих товарів, — хоч перепродукція
капіталу завжди включає перепродукцію товарів, —
означає через це не що інше, як перенагромадження капіталу.
Щоб зрозуміти, що таке є це перенагромадження (докладніше
дослідження його ми подаємо нижче), досить тільки припустити
його абсолютним. Коли перепродукція капіталу була б абсолютною?
І при тому перепродукція, яка поширювалася б не на ту
чи іншу або декілька значних сфер виробництва, а була б абсолютною
в самому своєму об’ємі, отже, охоплювала б усі сфери
виробництва?

Абсолютна перепродукція капіталу була б у наявності в тому
випадку, коли додатковий капітал для цілей капіталістичного
виробництва був би \deq{} 0. Але метою капіталістичного виробництва
є збільшення вартості капіталу, тобто привласнення додаткової
праці, виробництво додаткової вартості, зиску. Отже,
коли б капітал зріс порівняно з робітничим населенням настільки,
що не можна було б ні здовжити абсолютний робочий час, що
його дає це населення, ні розширити відносний додатковий робочий
час (останнє, крім того, було б нездійсниме при таких обставинах,
коли попит на працю є такий значний, отже, коли є тенденція
до підвищення заробітної плати), тобто коли б зрослий
капітал виробляв тільки таку саму або навіть меншу масу додаткової
вартості, ніж до свого зростання, то мала б місце абсолютна
перепродукція капіталу; тобто зрослий капітал $К \dplus{} ΔК$ виробляв
би не більше зиску, або навіть менше зиску, ніж капітал $К$
до свого збільшення на $ΔК$. В обох випадках мало б також
місце значне і раптове падіння загальної норми зиску, але на
цей раз в наслідок переміни у складі капіталу, викликаної не розвитком
продуктивної сили, а підвищенням грошової вартості
\parbreak{}  %% абзац продовжується на наступній сторінці

\parcont{}  %% абзац починається на попередній сторінці
\index{ii}{0249}  %% посилання на сторінку оригінального видання
їхнього існування, другу — $b$, що її вони почасти витрачають на речі
розкошів, а почасти застосовують на поширення продукції; $а$ — в такому
разі репрезентує змінний капітал, $b$ — додаткову вартість. Але такий поділ
не мав би жодного впливу на величину тієї маси грошей, яка потрібна
для циркуляції цілого їхнього продукту. За інших незмінних умов, вартість
товарової маси, що циркулює, була б та сама, а значить, і маса
потрібних для цього грошей була б та сама. Крім того, при однаковому
поділі періодів обороту продуценти мусили б мати такі самі грошові запаси,
тобто постійно мати в грошовій формі таку саму частину свого
капіталу, бо, згідно з нашим припущенням, їхня продукція, як і раніш,
була б товаровою продукцією. Отже, та обставина, що частина товарової
вартости складається з додаткової вартости, абсолютно не змінює маси
грошей доконечних для провадження підприємства.

Один з супротивників Тука, що тримається формули $Г — Т — Г$, запитує
його, як капіталістові вдається постійно вилучати з циркуляції більше
грошей, ніж він подає туди. Це цілком зрозуміло. Тут ідеться не про
\emph{утворення} додаткової вартости. Останнє, являючи єдину таємницю, з
капіталістичного погляду само собою зрозуміле. Застосована бо сума вартости
не була б капіталом, коли б вона не збагачувалась додатковою
вартістю. А що згідно з припущенням вона є капітал, то додаткова вартість
сама собою зрозуміла.

Отже, питання не в тім, відки береться додаткова вартість, а в тім,
відки беруться гроші, що на них вона перетворюється.

Але для буржуазної економії існування додаткової вартости зрозуміле
само собою. Отже, її не лише припускається, але разом з нею припускається
й те, що частина товарової маси, пущеної в циркуляцію, складається
з додаткового продукту, отже, репрезентує таку вартість, що її капіталіст
не кинув у циркуляцію, кидаючи туди свій капітал; що, отже, капіталіст,
разом з своїм продуктом кидає в циркуляцію певний надлишок
порівняно з своїм капіталом, а потім знову вилучає з неї цей надлишок.

Товаровий капітал, що його капіталіст подає в циркуляцію, має більшу
вартість (звідки це постає, не пояснюється або не розуміється, але з
погляду буржуазної економії c’est un fait\footnote*{
Це — факт. \emph{Ред.}
}, ніж продуктивний капітал,
що його він вилучив з циркуляції в формі робочої сили плюс засоби
продукції. Тому при цьому припущенні ясно, чому не лише капіталіст
$А$, але й $В$, $С$, $D$ і~\abbr{т. ін.} можуть постійно вилучати з циркуляції через
обмін своїх товарів більшу вартість, ніж вартість їхнього первісно авансованого
капіталу, що його потім знову й знову авансується. $А$, $В$, $С$,
$D$ і~\abbr{т. ін.} завжди подають в циркуляцію в формі товарового капіталу, —
а ця операція так само багатобічна, як і самостійно діющі капітали, —
більшу товарову вартість, ніж та, що її вони вилучають з циркуляції в
формі продуктивного капіталу. Отже, їм постійно доводиться розподіляти
між собою суму вартости (тобто кожному доводиться вилучати для себе
з циркуляції продуктивний капітал), що дорівнює сумі вартости їхніх
\parbreak{}  %% абзац продовжується на наступній сторінці

\parcont{}  %% абзац починається на попередній сторінці
\index{ii}{0250}  %% посилання на сторінку оригінального видання
відповідно авансованих ними продуктивних капіталів; і так само постійно
доводиться їм розподіляти між собою ту суму вартости, що її вони
з усіх боків подають у циркуляцію в товаровій формі, як відповідний
надлишок товарової вартости проти вартости її елементів продукції.

Але товаровий капітал, перш ніж він перетвориться знову на продуктивний
капітал, і перш ніж витратиться вміщену в ньому додаткову
вартість, треба перетворити на гроші. Відки беруться гроші для цього?
На перший погляд питання це видається складним, і ні Тук, ні хто інший
до цього часу не дали на нього відповіді.

Припустімо, що обіговий капітал в 500\pound{ ф. стерл.} авансований у формі
грошового капіталу, — хоч який буде період його обороту, — є сукупний
обіговий капітал суспільства, тобто кляси капіталістів. Додаткова вартість
хай буде 100\pound{ ф. стерл}. Яким же чином ціла кляса капіталістів може постійно
вилучати з циркуляції 600\pound{ ф. стерл.}, постійно подаючи в неї
лише 500\pound{ ф. стерл.}?

Після того, як грошовий капітал в 500\pound{ ф. стерл.} перетворився на
продуктивний капітал, цей останній у процесі продукції перетворюється
на товарову вартість в 600\pound{ ф. стерл.} і таким чином в циркуляції перебуває
не лише товарова вартість в 500\pound{ ф. стерл.}, рівна первісно авансованому
грошовому капіталові, а й новоспродукована додаткова
вартість в 100\pound{ ф. стерл}.

Цю новододану додаткову вартість в 100\pound{ ф. стерл.} подано в циркуляцію
в товаровій формі. В цьому немає жодного сумніву. Але в наслідок
цієї операції не здобувається додаткових грошей для циркуляції
цієї новододаної товарової вартости.

\manualpagebreak{}
Не слід намагатися обминати цієї трудности за допомогою зовнішньопристойних
викрутів.

Наприклад: щодо сталого обігового капіталу, то очевидно, що його
не всі витрачають одночасно. У той час, коли капіталіст $А$ продає
свій товар, отже, коли авансований ним капітал набирає для нього грошової
форми, для покупця $В$ його капітал, що перебуває в грошовій
формі, набирає, навпаки, форми його засобів продукції, саме тих, що їх
продукує $А$. Тим самим актом, що ним $А$ знову надає грошової форми
своєму спродукованому товаровому капіталові, $В$ знову надає продуктивної
форми своєму капіталові, перетворює його з грошової форми на засоби
продукції та робочу силу; та сама сума грошей функціонує в двобічному
процесі, як при всякій простій купівлі $Т — Г$. З другого боку, коли
$А$ знову перетворює гроші на засоби продукції, він купує їх в $С$, а
цей платить тими самими грішми $В$ і~\abbr{т. ін.} Тоді справу з’ясувалось би. Але:

Всі закони, викладені нами (кн. І, розд. III) щодо кількости грошей,
які циркулюють при товаровій циркуляції, зовсім не змінюються в наслідок
капіталістичного характеру продукційного процесу.

\vtyagnut{}
Отже, коли кажуть, що обіговий капітал суспільства, який треба
авансувати в грошовій формі, становить 500\pound{ ф. стерл.}, то при цьому вже
взято на увагу, що це, з одного боку, є така сума, яку авансовано одночасно,
але що, з другого боку, сума ця пускає в рух більше продуктивного
\index{ii}{0251}  %% посилання на сторінку оригінального видання
капіталу, ніж 500\pound{ ф. стерл.}, бо вона по черзі править за грошовий
фонд різних продуктивних капіталів. Отже, цей спосіб пояснення
припускає, що вже є наявні ті гроші, що їх наявність він має з’ясувати.

Далі можна було б сказати так: капіталіст $А$ продукує такі речі, що
їх капіталіст $В$ споживає особисто, непродуктивно. Отже, гроші $В$ перетворюють
на гроші товаровий капітал $А$, і таким чином та сама грошова
сума служить для перетворення на гроші додаткової вартости $В$ і
обігового сталого капіталу $А$. Але тут ще пряміше припускається розв’язаним
те саме питання, що на нього треба дати відповідь. А саме, відки
$В$ бере гроші на покриття свого доходу? Яким чином він сам перетворив
на гроші цю частину додаткової вартости свого продукту?

Потім можна було б сказати, що частина обігового змінного капіталу,
яку $А$ постійно авансує своїм робітникам, постійно повертається до нього
з циркуляції; і тільки деяка змінна частина її завжди лишається закріплена
в нього самого для видачі заробітної плати. Однак між моментом
витрачання й моментом зворотного припливу минає деякий час, що
протягом нього гроші, витрачені на заробітну плату, можуть, між іншим,
служити і для перетворення на гроші додаткової вартости. — Але, поперше,
ми знаємо, що як довший цей час, то й більша мусить бути маса грошового
запасу, що її капіталіст $А$ постійно мусить зберігати in petto\footnote*{
Дослівно: в серці своєму, тут у значенні: з собою, при собі. \emph{Ред.}
}.
Подруге, робітник витрачає гроші, купує на них товари й тим самим
pro tanto перетворює на гроші додаткову вартість, що міститься в цих
товарах. Отже, ті самі гроші, що їх авансується в формі змінного капіталу,
pro tanto служать і для перетворення на гроші додаткової вартости.
Не вглиблюючись у це питання ще далі, тут досить лише зауважити, що
споживання цілої кляси капіталістів і залежних від неї непродуктивних
осіб відбувається рівнобіжно й одночасно з споживанням робітничої кляси;
отже, одночасно з тим, як робітники подають у циркуляцію гроші,
мусять пускати гроші в циркуляцію і капіталісти, щоб витрачати свою
додаткову вартість як дохід; отже, для цього треба вилучати гроші з
циркуляції. Таким чином, щойно наведене пояснення лише зменшувало б
кількість потрібних грошей, але не усунуло б потреби в них.

Нарешті, можна було б сказати: однак в циркуляцію при першому
вкладенні основного капіталу постійно подається велику кількість грошей,
що їх той, хто пустив їх в циркуляцію, знову вилучає з неї лише поступінно,
частинами, протягом багатьох років. Хіба цієї суми не досить,
щоб перетворити на гроші додаткову вартість? — На це треба відповісти,
що сума в 500\pound{ ф. стерл.} (яка включає й скарботворення для потрібного
резервного фонду) можливо вже включає й застосовування цієї суми як
основного капіталу, якщо не тим, хто пустив її в циркуляцію, то кимось
іншим. Крім того вже припускається, що в сумі, витрачуваній на
придбання продуктів, що служать як основний капітал, оплачено й додаткову
вартість, що міститься в цих товарах, і питання саме в тому, відки
беруться ці гроші.


\index{ii}{0252}  %% посилання на сторінку оригінального видання
Загальну відповідь уже дано: коли має циркулювати маса товарів в
1000\pound{ ф. стерл.} $× X$, то величина грошової суми, потрібної для цієї циркуляції,
абсолютно не змінюється від того, чи є в вартості цієї маси товарів
додаткова вартість, чи немає, чи випродукувано цю товарову масу
капіталістично, чи ні. \emph{Отже, самої проблеми не існує}. За інших
даних умов, швидкости грошової циркуляції та ін., для циркуляції товарової
вартости в 1000\pound{ ф. стерл.} $× Х$, потрібна певна сума грошей, яка
зовсім не залежить від тієї обставини, чи багато, чи мало з цієї вартости
припадає безпосереднім продуцентам цих товарів. Оскільки тут і існує
проблема, вона збігається з загальною проблемою: відки береться сума
грошей, потрібна для циркуляції товарів у даній країні.

А проте, з погляду капіталістичної продукції, існує, звичайно, \emph{подоба}
якоїсь особливої проблеми. А саме за вихідний пункт, відки гроші пускається
в циркуляцію, тут виступає капіталіст. Гроші, що їх витрачає
робітник на оплату засобів свого існування, існують спочатку як грошова
форма змінного капіталу, і тому капіталіст їх спочатку пускає в
циркуляцію як купівельний або виплатний засіб за робочу силу. Крім
того, капіталіст пускає в циркуляцію гроші, що спочатку становили для
нього грошову форму його сталого — основного й поточного — капіталу;
він витрачає їх як купівельний або виплатний засіб на засоби праці та
матеріяли продукції. Але поза цим капіталіст уже не виступає як вихідний
пункт грошової маси, що перебуває в циркуляції.

Але взагалі існують тільки два вихідні пункти: капіталіст і робітник.
Треті особи всіх категорій або мусять одержувати гроші від цих двох
кляс за якібудь послуги, або оскільки вони одержують гроші без якихбудь
послуг з їхнього боку, вони є співвласники додаткової вартости в
формі ренти, проценту й~\abbr{т. ін.} Те, що додаткова вартість не лишається
цілком в кишені промислового капіталіста, й що він мусить поділитися
нею з іншими особами, не має жодного чинення до нашого питання.
Питання в тому, яким чином він перетворює на гроші свою додаткову
вартість, а не в тому, як розподіляються потім здобуті за неї гроші.
Отже, в даному разі ми все ще повинні розглядати капіталіста як єдиного
власника додаткової вартости. Щождо робітника, то вже сказано, що
він є тільки вторинний вихідний пункт, але капіталіст є первинний
вихідний пункт тих грошей, що їх пускає в циркуляцію робітник. Гроші,
спочатку авансовані як змінний капітал, пророблюють уже свій другий
обіг, коли робітник витрачає їх на оплату засобів існування.

Отже, кляса капіталістів лишається єдиним вихідним пунктом грошової
циркуляції. Коли їй треба на оплату засобів продукції 400\pound{ ф. стерл.}
і на оплату робочої сили 100\pound{ ф. стерл.}, то вона пускає в циркуляцію
500\pound{ ф. стерл}. Але додаткова вартість, що міститься в продукті, при нормі
додаткової вартости в 100\%, дорівнює вартості в 100\pound{ ф. стерл}. Як
же вона може постійно вилучати з циркуляції 600\pound{ ф. стерл.}, коли
вона постійно пускає в неї лише 500\pound{ ф. стерл.}? З нічого нічого й не
буде. Ціла кляса капіталістів не може вилучати з циркуляції нічого такого,
чого раніш не було пущено в неї.


\index{ii}{0253}  %% посилання на сторінку оригінального видання
Тут ми лишаємо осторонь ту обставину, що грошової суми в 400\pound{ ф. стерл.} при десятиразовому обороті, може, буде досить для циркуляції
засобів продукції вартістю в 4000\pound{ ф. стерл.} і праці вартістю в 1000\pound{ ф.
стерл.}, а решти 100\pound{ ф. стерл.} так само буде досить для циркуляції додаткової
вартости в 1000\pound{ ф. стерл}. Це відношення грошової суми до товарової
вартості, що циркулює за її допомогою, не має ніякого чинення до
справи. Проблема лишається та сама. Коли б та сама монета не циркулювала
декілька разів, то довелось би пустити в циркуляцію 5000\pound{ ф. стерл.}
як капітал і 1000\pound{ ф. стерл.} були б потрібні для перетворення додаткової
вартости на гроші. Постає питання, відки беруться ці гроші, хоч то
1000\pound{ ф. стерл.}, хоч 100\pound{ ф. стерл}. В усякому разі вони є надлишок понад
грошовий капітал, пущений у циркуляцію.

Справді, хоч як це здається парадоксальним на перший погляд, кляса
капіталістів сама пускає в циркуляцію ті гроші, які служать для реалізації
додаткової вартости, що міститься в товарах. Але nota bene\footnote*{
Добре зауважте. \emph{Ред.}
} — кляса
капіталістів пускає їх в циркуляцію не як авансовані гроші, отже, не як
капітал. Вона витрачає їх як купівельний засіб для свого особистого
споживання. Отже, кляса капіталістів не авансує цих грошей, хоч вона
є вихідний пункт їхньої циркуляції.

\vtyagnut{}
Візьмімо поодинокого капіталіста, що починає справу, приміром,
фармера. Протягом першого року він авансує грошовий капітал, скажімо,
в 5000\pound{ ф. стерл.}, щоб оплатити засоби продукції (4000\pound{ ф. стерл.}) і робочу
силу (1000\pound{ ф. стерл.}). Норма додаткової вартости хай буде 100\%, привлащувана
ним додаткова вартість \deq{} 1000\pound{ ф. стерл}. Вищезазначені 5000\pound{ ф.
стерл.} являють собою всі гроші, що їх він авансує як грошовий капітал.
Однак ця людина мусить також жити, але до кінця року не одержить
вона жодних грошей. Її споживання становить 1000\pound{ ф. стерл}. Вона мусить
мати ці гроші. Правда, вона каже, що мусить авансувати собі ці 1000\pound{ ф. стерл.}
протягом першого року. Однак це авансування — воно має тут лише
суб’єктивне значення — сходить лише на те, що протягом першого року
вона мусить покривати своє особисте споживання з власної кишені, а не
з дармової продукції своїх робітників. Вона не авансує цих грошей як
капітал. Вона витрачає їх, платить їх як еквівалент за ті засоби існування,
що вона споживає. Цю вартість вона витрачає як гроші, подає в
циркуляцію та вилучає з неї як товарові вартості. Ці товарові вартості
вона спожила. Отже, немає тепер будь-якого відношення її до їхньої
вартости. Гроші, що ними вона заплатила за неї, існують тепер як елемент
грошей, що циркулюють. Але вартість цих грошей вона вилучила
в продуктах із циркуляції, а разом з продуктами, що ними вона жила,
знищено й їхню вартість. Вартість ця зникла. Але ось наприкінці року
ця людина пускає в циркуляцію товарову вартість в 6000\pound{ ф. стерл.} і продає її.
В наслідок цього до неї повертається: 1) авансований нею грошовий
капітал в 5000\pound{ ф. стерл.}, 2) перетворена на гроші додаткова вартість в
1000\pound{ ф. стерл}. Вона авансувала 5000\pound{ ф. стерл.} як капітал, пустила їх в
\parbreak{}  %% абзац продовжується на наступній сторінці

\input{ii/_0254_0255.tex}
\parcont{}  %% абзац починається на попередній сторінці
\index{ii}{0256}  %% посилання на сторінку оригінального видання
грошей, що змінними порціями перебуває завжди в руках кляси капіталістів
як грошова форма їхньої додаткової вартости, не є елемент
щорічно продукованого золота, а елемент маси грошей, раніш акумульованих
у країні.

Згідно з нашим припущенням, річної продукції золота в 500\pound{ ф. стерл.}
досить саме лише для того, щоб заміщувати щорічне зношування грошей.
Тому, коли ми матимемо на увазі тільки ці 500\pound{ ф. стерл.} і абстрагуємось
від тієї частини щорічно продукованої маси товарів, що її циркуляцію
обслуговують раніш акумульовані гроші, то додаткова вартість, спродукована
в товаровій формі, уже тому знаходить в циркуляції гроші для
свого перетворення на гроші, що на другому боці щороку продукується
додаткову вартість у формі золота. Це саме має силу й щодо інших частин
продукту-золота в 500\pound{ ф. стерл.}, що заміщують авансований грошовий
капітал.

Тут треба тепер зробити два зауваження.

\enlargethispage{\baselineskip}
З наведеного вище випливає, поперше: додаткова вартість, витрачувана
капіталістами у формі грошей, так само, як і змінний та інший
продуктивний капітал, що його вони авансують в формі грошей, в
дійсності є продукт робітників, а саме робітників, що працюють у
золотопромисловості. Вони знову продукують так ту частину продукту-золота,
що її „авансується“ їм як заробітну плату, як і ту частину
продукту-золота, що безпосередньо репрезентує додаткову вартість
капіталістів-продуцентів золота. Нарешті, щодо тієї частини продукту-золота,
яка лише покриває сталу капітальну вартість, авансовану на
продукцію цього продукту-золота, то вона знову з’являється в грошовій
формі (взагалі в продукті) лише в наслідок річної праці робітників.
При заснуванні підприємства капіталіст віддав її спочатку в вигляді
грошей, не новоспродукованих, а тих, що становили частину маси грошей,
яка циркулювала у суспільстві. Навпаки, оскільки її заміщується новим
продуктом, додатковим золотом, вона вже є річний продукт робітника.
Те, що її авансує капіталіст, — це й тут є лише форма, яка випливає з
того, що робітник не є власник своїх власних засобів продукції й не
має в своєму розпорядженні підчас продукції засобів існування, спродукованих
іншими робітниками.

Але, подруге, щодо тієї маси грошей, яка існує незалежно від цього
річного заміщення в 500\pound{ ф. стерл.} і перебуває почасти в формі скарбу,
почасти в формі грошей, що циркулюють, то з нею справа мусить
стояти, — тобто первісно мусила стояти — цілком так само, як щорічно
стоїть справа з цими 500\pound{ ф. стерл}. До цього пункту ми повернемось
наприкінці цього підвідділу. Але перше зробимо ще кілька зауважень.

\plainbreak{2}

Досліджуючи оборот, ми бачим, що за інших незмінних умов, коли
змінюється величина періодів обороту, змінюється й маса грошового
капіталу, потрібного для того, щоб провадити продукцію в тому самому
маштабі. Отже, елястичність грошової циркуляції мусить бути досить
\parbreak{}  %% абзац продовжується на наступній сторінці

\parcont{}  %% абзац починається на попередній сторінці
\index{ii}{0257}  %% посилання на сторінку оригінального видання
велика, щоб могла вона пристосуватись до такої змінности в подовженні
і скороченні періодів обороту.

Коли далі припустимо інші незмінні умови, — а між ними незмінну
довжину, інтенсивність і продуктивність робочого дня, але \emph{змінний
розподіл новоствореної вартости} між робітниками та додатковою
вартістю, так, що або перша підвищується, а друга меншає,
або, навпаки, то це не справить жодного впливу на масу грошей в
циркуляції. Така зміна може відбуватися без будь-якого збільшення або
зменшення маси грошей, що перебувають в циркуляції. Розгляньмо особливо
той випадок, коли стається загальне підвищення заробітної плати,
а тому — при вищеприпущених умовах — загальне зниження норми додаткової
вартости; при цьому — також згідно з припущенням — не відбувається
жодної зміни в вартості товарової маси, яка циркулює. В цьому
випадку, звичайно, зростає грошовий капітал, що його треба авансувати
як змінний капітал, отже, зростає маса грошей, що служить у цій
функції. Але саме настільки, наскільки зростає маса грошей, потрібних
для функції змінного капіталу, саме на стільки меншає додаткова вартість,
отже, й маса грошей, потрібних для її реалізації. На суму грошей,
потрібних для реалізації товарової вартости, це так само не справляє
жодного впливу, як і на саму цю товарову вартість. Ціна витрат\footnote*{
Про визначення терміну „ціна витрат“ (Kostenpreis або Kostpreis, як Маркс
вживає в книзі III (дивись „Капітал“, т. III, ч. І, розділ 1). \emph{Ред.}
} на
товар підвищується для поодинокого капіталіста, але його суспільна ціна
продукції\footnote*{
Про визначення терміну „ціна продукції“ (Produktionspreis) дивись „Капітал“,
т. III, ч. І, розділ дев’ятий. \emph{Ред.}
} лишається незмінна. Змінюється при цьому тільки те відношення,
що в ньому — лишаючи осторонь сталу частину вартости — ціна
продукції товарів поділяється на заробітну плату й зиск.

Але, можуть сказати, більша витрата змінного грошового капіталу
(вартість грошей, звичайно, припускається за незмінну) значить те саме,
що й збільшення грошових засобів у руках робітників. Звідси випливає
підвищення попиту на товари з боку робітників. Дальший наслідок буде
підвищення цін товарів. Або можуть сказати: коли підвищується заробітна
плата, то капіталісти підвищують ціни на свої товари. В обох випадках
загальне підвищення заробітної плати спричиняється до підвищення ціни
товарів. Тому для циркуляції товарів потрібна більша маса грошей, хоч
у який спосіб пояснюватимуть підвищення цін.

Відповідь на перше міркування: в наслідок підвищення заробітної
плати підвищиться саме попит робітників на доконечні засоби існування.
Куди менше збільшиться попит їхній на речі розкошів або постане попит
на такі речі, що раніш не ввіходили в сферу їхнього споживання. Підвищення
попиту на доконечні засоби існування, що постає раптом та у
великих розмірах, безперечно, зараз же підвищить їхню ціну. Наслідок
цього буде той, що більшу частину суспільного капіталу застосовуватиметься
на продукцію доконечних засобів існування, а меншу — на продукцію
\index{ii}{0258}  %% посилання на сторінку оригінального видання
речей розкошів, бо ці останні дешевшають в наслідок зменшення
додаткової вартости і зумовленого цим зменшення попиту капіталістів на
речі розкошів. Навпаки, оскільки робітники сами купують речі розкошів,
підвищення їхньої заробітної плати не справить — в цих межах — впливу на
підвищення ціни доконечних засобів існування, а лише змінить склад
покупців речей розкошів. Речей розкошів тепер більше йде, ніж раніш,
на споживання робітників і порівняно менш — на споживання капіталістів.
Voilà tout\footnote*{
От і все. \emph{Ред.}
}. Після деяких коливань у циркуляції буде маса товарів такої
самої вартости, як і раніш. — Щождо короткочасних коливань, то наслідок
їх буде лише той, що вільний грошовий капітал, який досі шукав собі
застосування в спекулятивних біржових підприємствах або за кордоном,
тепер надійде в циркуляцію в середині країни.

Відповідь на друге міркування: коли б капіталістичні продуценти
мали змогу з свого бажання підвищувати ціни своїх товарів, то вони
могли б робити це й робили б без усякого підвищення заробітної плати.
Заробітна плата ніколи не підвищувалась би при зниженні цін товарів.
Кляса капіталістів ніколи не ставила б опору тред-юньйонам, бо вона
завжди та за всяких умов могла б робити те, що вона в дійсності робить
тепер, як виняток, в певних особливих, сказати б, місцевих умовах:
а саме, вона могла б використовувати кожне підвищення заробітної плати
для того, щоб куди більше підвищувати ціни товарів, отже, щоб покласти
собі до кишені більший зиск.

Твердження, що капіталісти можуть підвищувати ціни речей розкошів,
бо попит на них меншає (в наслідок зменшеного попиту капіталістів, що
їхні купівельні засоби на це поменшали), це твердження було б цілком
ориґінальним застосуванням закону попиту й подання. Оскільки не постає
простої переміни покупців речей розкошів, заміни капіталістів робітниками, —
а оскільки така заміна постає, попит робітників не зумовлює підвищення
цін доконечних засобів існування, бо робітники не можуть витрачати на
доконечні засоби існування тієї частини додаткового заробітку, яку вони
витрачають на речі розкошів, — остільки ціни речей розкошів знижуються
в наслідок зменшеного попиту. У наслідок цього капітал вилучається з
продукції речей розкошів доти, доки їхнє подання зменшиться до таких
розмірів, що відповідають зміненій ролі їх в суспільному процесі продукції.
При такій скороченій продукції ціни їх, за незмінної вартости, знову
підвищуються до свого нормального рівня. Якщо відбувається таке
скорочення, або такий процес вирівнювання, то протягом його при підвищенні
цін на засоби існування у продукцію цих останніх постійно
подаватиметься стільки ж капіталів, скільки їх вилучатиметься з іншої
галузі продукції, поки насититься попит. Тоді знову постає рівновага, і
кінець цілого процесу той, що суспільний капітал, а тому й грошовий
капітал, розподіляється між продукцією доконечних засобів існування й
продукцією речей розкошів в зміненій пропорції.


\index{ii}{0259}  %% посилання на сторінку оригінального видання
Всі заперечення — це є сліпий вистріл капіталістів та їхніх економістів
сикофантів.

Факти, що дають нагоду для такого сліпого вистрілу, є троякого роду.

1) Загальний закон грошової циркуляції той, що коли сума цін товарів,
що циркулюють, підвищується — все одно чи це збільшення суми
цін постає для тієї самої маси товарів, чи для збільшеної, — то за інших
незмінних обставин зростає маса грошей, що циркулюють.

Тут наслідок сплутують з причиною. Заробітна плата підвищується (хоч
і рідко підвищується, а пропорційно до підвищення цін вона підвищується
тільки в виняткових випадках) із підвищенням цін доконечних засобів існування.
Її підвищення є наслідок, а не причина підвищення цін товарів.

2) При частковому або місцевому підвищенні заробітної плати, тобто
при підвищенні її тільки в поодиноких галузях продукції — може в наслідок
цього постати місцеве підвищення цін на продукти цієї галузі. Але навіть
це залежить від багатьох обставин. Напр., від того, що заробітна плата
тут не була надто низька і норма зиску тому не була надто висока, що
в наслідок підвищення цін ринок для цих товарів не скорочується
(отже, для підвищення їхніх цін не треба попереднього зменшення
подання їх) і~\abbr{т. ін.}

3) При загальному підвищенні заробітної плати підвищується ціна товарів,
продукованих в тих галузях промисловости, де переважає змінний капітал,
але зате спадає в тих, де переважає сталий, зглядно основний капітал.

\pfbreak{}

При дослідженні простої товарової циркуляції (книга І, розд. III, 2) виявилось,
що хоч у процесі циркуляції будь-якої певної кількости товарів її
грошова форма є лише минуща, однак, гроші, зникаючи при метаморфозі
товару в руках однієї особи, неодмінно переходять до рук іншої; отже,
товари насамперед не лише всебічно обмінюються або заміщуються один
одним, але це заміщення упосереднюється й супроводиться всебічним
осіданням грошей. „У наслідок заміщення одного товару іншим товаром
до рук третьої особи одночасно в’язне товар-гроші. Циркуляція постійно
спливає грошовим потом“ (кн. І, розд. III, 2, а). Той самий тотожній
факт на основі капіталістичної товарової продукції виражається в тому,
що частина капіталу постійно існує в формі грошового капіталу, а частина
додаткової вартости так само постійно перебуває в руках її власника
в грошовій формі.

Лишаючи це осторонь, \so{кругобіг грошей} — тобто зворотний приплив
грошей до свого вихідного пункту — оскільки він становить момент
обороту капіталу, є цілком відмінне явище, навіть протилежне \emph{обігові}
грошей\footnote{
Хоч фізіократи ще сплутують обидва ці явища, однак вони перші звернули
увагу на зворотний приплив грошей до свого вихідного пункту, як на важливу
форму циркуляції капіталу, як на форму циркуляції, що упосереднює репродукцію.
„Погляньте на „Tableau Économique“, і ви побачите, що продуктивна кляса дає
гроші на які інші кляси купують у неї продукти, і що вони повертають їй ці
гроші, повертаючись наступного року, щоб знову зробити в неї такі ж закупи\dots{}
Отже, ви не бачите тут іншого кругобігу, крім того, де по витраті
постає репродукція, а по репродукції витрата, — кругобігу, що його перебігає
циркуляція грошей, які є міра витрати й репродукції. („Jetez les yeux sur le
Tableau Economique, vous verrez, que la classe productive d nne l’argent,
avec lequel les autres classes viennent lui acheter des productions, et qu’elles lui
rendent cet argent en revenant l’année suivante faire chez elle les mêmes achats\dots{}
Vous ne v yez donc ici d’autre cercle que celui-ci de la dépense suivie de la reproduction,
et de la réproduktion suivie de la dépense; cercle qui est parcouru par la
circulation de l’argent qui mesure la dépense et la reproduction“ — Quesnay. „Problèmes
économiques, in Daire, Physiocrates, I“, p. 208, 209).

„Саме це постійне авансування й постійний поворот капіталів треба назвати
циркуляцією грошей, тією корисною й плодотворчою циркуляцією, яка оживляє
всю працю суспільства, підтримує рух і життя в політичному організмі і яку
цілком слушно можна порівняти з кровобігом у тваринному організмі“. (C’est
cette avance et cette rentrée continuelle des capitaux qui constituent ce qu’on doit
appeller la circulation de l’argent, cette circulation utile et féconde, qui anime tous
les travaux de la société, qui entretient le mouvement et la vie dans le corps politique,
et qu’on a grande raison de comparer à la circulation du sang dans le corps animal“.
— Turgot, „Reflexions“ etc, Oeuvres, éd. Daire, I, p. 45).}, який виражає постійне \emph{віддалення} їх від вихідного
\index{ii}{0260}  %% посилання на сторінку оригінального видання
пункту в наслідок ряду переміщень. (Кн.~І, розд. III, 2, б). Однак прискорений
оборот ео ipso\footnote*{
Тим самим. \emph{Ред.}
} включає й прискорений обіг.

\vtyagnut{}
Насамперед щодо змінного капіталу: коли, напр., грошовий капітал
в 500\pound{ ф. стерл.} обертається в формі змінного капіталу десять разів на
рік, то очевидно, що ця аліквотна частина грошової маси, яка циркулює,
пускає в циркуляцію вдесятеро більшу суму вартости \deq{} 5000\pound{ ф. стерл}.
Вона обігає між капіталістом і робітником десять разів протягом року.
Протягом року робітника десять разів оплачується, й сам робітник платить
тією самою аліквотною частиною грошової маси циркуляції. Коли
б при однакових розмірах продукції цей змінний капітал обертався
лише один раз протягом року, то тоді відбувся б лише один обіг
в 5000\pound{ ф. стерл}.

Далі, хай стала частина обігового капіталу дорівнює 1000\pound{ ф. стерл}.
Коли капітал обертається десять разів, то капіталіст продає свій товар, а
значить, і сталу обігову частину його вартости десять разів на рік. Та
сама аліквотна частина грошової маси, що циркулює (1000\pound{ ф. стерл.}),
десять разів на рік переходить з рук власників цієї частини до рук капіталіста.
Десять разів переміщуються ці гроші з рук у руки. Подруге,
капіталіст десять разів на рік купує засоби продукції, це знову є десять
обігів грошей з рук до рук. За допомогою грошей на суму 1000\pound{ ф. стерл.}
промисловий капіталіст продає товару на \num{10.000}\pound{ ф. стерл.} і знову купує
товару на \num{10.000}\pound{ ф. стерл}. В наслідок двадцятиразового обігу 1000\pound{ ф. стерл.}
циркулює запас товару в \num{20.000}\pound{ ф. стерл}.

Нарешті, при прискореному обороті швидше циркулює й та частина
грошей, що реалізує додаткову вартість.

Навпаки, швидший обіг грошей і не включає неодмінно швидшого обороту
капіталу, а тому й швидшого обороту грошей, тобто не включає
неодмінно скорочення та швидкого поновлення процесу репродукції.


\index{ii}{0261}  %% посилання на сторінку оригінального видання
Швидший грошовий обіг буває кожного разу, коли за допомогою тієї
самої кількости грошей провадиться більше оборудок. Це може бути і
при однакових періодах репродукції капіталу, в наслідок змінених технічних
пристосовань у грошовому обігу. Далі: може збільшуватися число
оборудок, що в них грошовий обіг не виражає справжнього обміну товарів
(біржові диференційні оборудки і~\abbr{т. ін.}). З другого боку грошового
обігу може й зовсім не бути. Напр., там, де сільський господар сам є
землевласник, немає жодного грошового обігу між орендарем і землевласником;
там, де промисловий капіталіст сам є власник капіталу, немає
жодного обігу між ним і кредитором.

\bigskip
\fancybreak{*\quad*\quad*}
\bigskip
\noindent{}Щодо первісного утворення грошового скарбу в країні та привлащення
його небагатьма особами, то тут немає потреби зупинятися на
цьому докладніше.

Капіталістичний спосіб продукції — а за базу його є так само наймана
праця, як і оплата робітника грішми і взагалі перетворення натуральних
відбутків на грошові — може розвиватись у ширшому й глибшому маштабі
тільки в такій країні, де є досить грошей для циркуляції та для
зумовлюваного нею утворення скарбів (резервного фонду тощо). Така є
історична передумова, хоч не треба розуміти справу так, ніби спочатку
утворюється досить скарбів, а потім починається капіталістична продукція.
Вона розвивається одночасно з розвитком умов для неї, а за одну з
таких умов є достатнє подання благородних металів. Тому посилене подання
благородних металів, починаючи з XVI століття, являє посутній
момент в історії розвитку капіталістичної продукції. Але оскільки йдеться
про потрібне дальше подання грошового матеріялу на базі капіталістичного
способу продукції, то, з одного боку, додаткову вартість
подається в циркуляцію в вигляді продукту, без грошей потрібних для
його перетворення на гроші, а з другого боку, додаткову вартість подається
в циркуляцію в вигляді золота, без попереднього перетворення
продукту на гроші.

Додаткові товари, що мають перетворитися на гроші, знаходять потрібну
грошову суму, бо на другому боці, не через обмін, а самою продукцією
подається в циркуляцію додаткове золото (і срібло), що має
перетворитись на товари.

\subsection{Акумуляція та поширена репродукція}

Оскільки акумуляція відбувається в формі репродукції в поширеному
маштабі, то очевидно, що вона не являє жодної нової проблеми щодо грошової
циркуляції.

Насамперед, щодо додаткового грошового капіталу, потрібного для
функції ростучого продуктивного капіталу, то його дає та частина реалізованої
додаткової вартости, що її капіталіст подає в циркуляцію як грошовий
капітал, а не як грошову форму доходу. Гроші вже є в руках
капіталістів. Тільки застосування їх різне.


\index{ii}{0262}  %% посилання на сторінку оригінального видання
Але в наслідок додаткового продуктивного капіталу в циркуляцію
подається, як продукт його, додаткову товарову масу. Разом з цією
додатковою товаровою масою подається в циркуляцію частину додаткових
грошей, потрібних для реалізації її — а саме подається остільки, оскільки
вартість цієї товарової маси дорівнює вартості продуктивного капіталу,
зужиткованій на її продукцію. Цю додаткову масу грошей авансується
прямо як додатковий грошовий капітал, і тому він зворотно припливає
до капіталіста в наслідок обороту його капіталу. Тут перед нами знову
постає те саме питання, що й раніш. Звідки беруться додаткові гроші на
реалізацію додаткової вартости, що є тепер у товаровій формі в цій
додатковій масі товарів?

Загальна відповідь знову та сама. Сума цін товарової маси, яка циркулює,
збільшилась не тому, що ціна даної товарової маси підвищилась,
а тому, що маса товарів, які тепер циркулюють, більша за масу товарів,
що циркулювали раніше, і при цьому ця ріжниця не вирівнюється зниженням
цін. Додаткові гроші, потрібні для циркуляції цієї більшої товарової
маси більшої вартости, треба здобути або посиленою економією на
масі грошей, що циркулюють, — чи то через взаємне вирівнювання платежів
тощо, чи то засобами, які прискорюють обіг тієї самої монети, —
або їх треба здобути через перетворення грошей з форми скарбу на
обігову форму грошей. Останнє включає не лише те, що бездіяльний
грошовий капітал починає функціонувати як купівельний засіб
або як засіб виплати; або і не лише те, що грошовий капітал, який уже
функціонує як резервний фонд, виконуючи для свого власника функцію
резервного фонду, активно циркулює для суспільства (як от банкові
вклади, що їх завжди дається в позику), отже, виконує двоїсту функцію;
це перетворення включає й те, що заощаджується стаґнаційні монетні
резервні фонди.

„Щоб гроші постійно обігали як монети, монети мусять постійно
осідати як гроші. Постійний обіг монет зумовлено тим, що їх постійно
затримується більшими або меншими кількостями як монетні резервні
фонди, що всюди утворюються в межах циркуляції й зумовлюють її, —
монетні резервні фонди, що їх утворення, розподіл, розпад і нове утворення
завжди чергуються, резервні фонди; що буття їх постійно зникає,
що процес їх зникання ніколи не припиняється. Це безперестанне перетворення
монет на гроші й грошей на монети А.~Сміт висловив таким
чином, що кожен товаровласник поряд з тим особливим товаром, що
його він продає, завжди мусить мати в запасі певну суму загального
товару, що на нього він купує. Ми бачили, що в циркуляції $Т — Г — Т$
другий член $Г — Т$ постійно розпадається на ряд актів купівлі, які відбуваються
не одноразово, а послідовно в часі, так що одна частина $Г$
обігає як монета, тимчасом як друга перебуває в стані спокою як гроші.
Тут гроші справді є лише монети, що їхнє функціонування відкладено,
і окремі складові частини монетної маси, що обігає, завжди з’являються,
чергуючися, то в одній, то в другій формі. Тому, це перше перетворення
засобу циркуляції на гроші являє собою лише технічний момент самого
\parbreak{}  %% абзац продовжується на наступній сторінці

\parcont{}  %% абзац починається на попередній сторінці
\index{ii}{0263}  %% посилання на сторінку оригінального видання
грошового обігу“. (Карл Маркс, „До критики політичної економії“
1859~\abbr{р.}, стор. 105--106. — Вираз „монета“ протилежно до грошей вжито
тут на позначення грошей у їхній функції як простого засобу циркуляції,
протилежно до інших їхніх функцій).

Коли всіх цих засобів не досить, то доводиться додатково продукувати
золото або — що сходить на те саме — частину додаткового продукту
обмінюється безпосередньо або посередньо на золото, на продукт країн,
що продукують благородні металі.

Вся сума робочої сили та суспільних засобів продукції, витрачуваних
на щорічну продукцію золота й срібла як знаряддя циркуляції, становить
чималу частину faux frais капіталістичного способу продукції і взагалі
способу продукції, що ґрунтується на товаровій продукції. Ця продукція
відтягує від суспільного користання відповідну суму можливих,
додаткових засобів продукції та споживання, тобто справжнього багатства.
Оскільки за незмінного даного маштабу продукції або за даного ступеня
її поширення меншають витрати на цей дорогий механізм циркуляції,
остільки ж підвищується в наслідок цього продуктивна сила суспільної
праці. Отже, оскільки так впливають допоміжні засоби, що розвиваються
разом з кредитовою системою, вони безпосередньо збільшують капіталістичне
багатство, або тим, що більшу частину процесу суспільної продукції
та процесу суспільної праці провадиться в наслідок цього без
якоїбудь інтервенції справжніх грошей, або тим, що підвищується
функціональну спроможність грошової маси, яка справді функціонує.

Цим розв’язується безглузде питання про те, чи можлива була б
капіталістична продукція в її теперішніх розмірах без кредитової системи
(коли навіть розглядати її тільки з \emph{цього} погляду), тобто при самій
металевій циркуляції. Очевидно, ні. Навпаки, вона була б обмежена розміром
продукції благородних металів. З другого боку, не треба складати собі
містичних уявлень про продуктивну силу кредитової системи, оскільки
вона дає в розпорядження грошовий капітал або пускає його в рух.
Однак дальший розвиток цього сюди не стосується.
\pfbreak
Тепер ми повинні розглянути той випадок, коли відбувається не
справжня акумуляція, тобто безпосереднє поширення розмірів продукції,
а коли частину реалізованої додаткової вартости на більш-менш довгий
час акумулюється як грошовий резервний фонд, щоб пізніше перетворити
його на продуктивний капітал.

Оскільки гроші, що їх акумулюється таким чином, є додаткові гроші,
справа сама собою зрозуміла. Вони можуть бути лише частиною надлишкового
золота, довезеного з країн, що продукують золото. При цьому
треба зазначити, що в країні немає вже того національного продукту,
що за нього довезено це золото. Його віддано за кордон в обмін на
золото.

Навпаки, коли припустити, що в країні лишається, як і раніш, та сама
маса грошей, то нагромаджувані гроші припливають з циркуляції; змінюється
\index{ii}{0264}  %% посилання на сторінку оригінального видання
лише їхня функція. З грошей, що циркулюють, вони перетворюються
на лятентний грошовий капітал, що поступінно утворюється.

Гроші, нагромаджувані при цьому, є грошова форма проданих товарів,
а саме форма тієї частини їхньої вартости, яка репрезентує для їхніх
власників додаткову вартість. (Тут припускається, що кредитова система
не існує). Капіталіст, що нагромадив ці гроші, pro tanto продавав, не
купуючи.

Коли уявити собі цей процес, як окремий випадок, а не як загальний,
то він не потребує жодних пояснень. Частина капіталістів затримує частину
грошей, вторгованих від продажу своїх продуктів, не купуючи на
них продукту на ринку. Навпаки, друга частина капіталістів перетворює
на продукт усі свої гроші за винятком потрібного для продукції грошового
капіталу, що завжди повертається. Частина продукту, що її, як носія
додаткової вартости, подається на ринок, складається з засобів продукції
або з реальних елементів змінного капіталу, з доконечних засобів існування.
Отже, вона може одразу придатись для поширення продукції.
Ми бо зовсім не припускаємо, що одна частина капіталістів нагромаджує
грошовий капітал, у той час, як друга частина цілком споживає всю свою
додаткову вартість; ми лише припускаємо, що одна частина капіталістів
провадить свою акумуляцію у грошовій формі, утворює лятентний грошовий
капітал, тимчасом як друга справді акумулює, тобто поширює розміри
продукції, справді збільшує свій продуктивний капітал. Маси наявних
грошей завжди досить для потреб циркуляції, коли навіть по черзі одна
частина капіталістів акумулює гроші, тимчасом як друга частина поширює
маштаб продукції, і навпаки. Крім того, нагромадження грошей на одному
боці може відбуватись і без наявних грошей, шляхом самого лише нагромадження
боргових вимог.

Але труднощі постають тоді, коли ми припускаємо акумуляцію грошового
капіталу не як окремий випадок, а як загальну акумуляцію грошового
капіталу в кляси капіталістів. Згідно з нашим припущенням —
загальне й виключне панування капіталістичної продукції — поза цією
клясою взагалі немає жодних інших кляс, крім робітничої кляси. Все,
що купує робітнича кляса, дорівнює сумі її заробітної плати, дорівнює
сумі змінного капіталу, авансованого цілою клясою капіталістів. До цих
останніх ці гроші припливають назад тому, що вони продають свій
продукт робітничій клясі. В наслідок цього їхній змінний капітал знову
набирає грошової форми. Припустімо, що сума цього змінного капіталу,
тобто сума змінного капіталу, не просто авансованого протягом року, а
справді застосованого, дорівнює 100\pound{ ф. стерл.} $× х$; для розглядуваного тут
питання не має жодного значення, чи багато чи мало, залежно від швидкости
обороту, треба грошей для того, щоб авансувати протягом року
змінний капітал такої вартости. Цими 100\pound{ ф. стерл.} $× х$ капіталу кляса капіталістів
купує певну масу робочої сили або сплачує заробітну плату
певному числу робітників — перша оборудка. Робітники на цю саму суму
купують у капіталістів деяку кількість товарів, в наслідок цього сума
100\pound{ ф. стерл.} $× х$ зворотно припливає до капіталістів — друга оборудка.
\parbreak{}  %% абзац продовжується на наступній сторінці

\parcont{}  %% абзац починається на попередній сторінці
\index{iii1}{0265}  %% посилання на сторінку оригінального видання
для нього не тільки як для капіталіста взагалі, а спеціально як
для торговця товарами, само собою очевидно, що його капітал
первісно мусить з’явитись на ринку в формі грошового капіталу,
бо він не виробляє ніяких товарів, а тільки торгує ними, опосереднює їхній рух, а для того, щоб
торгувати ними, він мусить
їх спочатку купити, отже, мусить бути володільцем грошового
капіталу.

Припустімо, що якийсь торговець товарами володіє 3000 фунтами стерлінгів, які він збільшує в їх
вартості як торговельний
капітал. На ці 3000 фунтів стерлінгів він купує у фабриканта,
який виробляє полотно, наприклад, \num{30000} метрів полотна по
2 шилінги за метр. Він продає ці \num{30000} метрів. Якщо пересічна
річна норма зиску = 10\% і якщо він, після відрахування всіх
накладних витрат, одержує 10\% річного зиску, то на кінець року
він перетворить ці 3000 фунтів стерлінгів у 3300 фунтів стерлінгів. Яким чином він одержує цей зиск
— це питання, яке ми
розглянемо тільки пізніше. Тут ми насамперед розглянемо тільки
форму руху його капіталу. На ці 3000 фунтів стерлінгів він
весь час купує полотно і весь час продає це полотно; він постійно повторює цю операцію купівлі для
продажу, $Г — Т — Г'$,
просту форму капіталу, в якій цей капітал цілком зв’язаний у процесі циркуляції, не перериваному
інтервалами процесу виробництва, який лежить поза його власним рухом і функцією.

Яке ж є відношення цього товарно-торговельного капіталу
до товарного капіталу як простої форми існування промислового
капіталу? Щодо фабриканта полотна, то він грішми купця реалізував вартість свого полотна, виконав
першу фазу метаморфози свого товарного капіталу, перетворення його в гроші, і може
тепер, при інших незмінних умовах, знову перетворити гроші
у пряжу, вугілля, заробітну плату і~\abbr{т. д.}, з другого боку — в засоби існування і~\abbr{т. д.} для
споживання свого доходу; отже, залишаючи осторонь витрачання доходу, він може продовжувати процес
репродукції.

Але, хоч для нього, для виробника полотна, вже відбулася
метаморфоза полотна в гроші, його продаж, вона ще не відбулася для самого полотна. Як і раніш,
полотно перебуває на ринку
як товарний капітал і має призначення виконати свою першу
метаморфозу, бути проданим. З цим полотном нічого не сталося,
крім переміни особи його володільця. За своїм призначенням, за
своїм становищем у процесі воно, як і раніше, є товарний капітал, продажний товар; тільки тепер воно
перебуває в руках
купця, а не в руках виробника, як це було раніш. Функція його
продажу, опосереднення першої фази його метаморфози, забрана
від виробника купцем і перетворена в його спеціальне заняття, — тимчасом як раніше це була функція,
яку мав виконувати виробник після виконання функції виробництва полотна.

Припустімо, що купцеві не вдалося продати \num{30000} метрів
протягом того періоду часу, який потрібний виробникові
\parbreak{}  %% абзац продовжується на наступній сторінці

\parcont{}  %% абзац починається на попередній сторінці
\index{iii1}{0266}  %% посилання на сторінку оригінального видання
полотна для того, щоб знову кинути на ринок \num{30000} метрів вартістю в 3000\pound{ фунтів стерлінгів}. Купець
не може їх знову купити, бо він ще має на складі непроданих \num{30000} метрів, які ще
не перетворились для нього в грошовий капітал. Тоді настає
застій, перерив репродукції. Виробник полотна міг би, звичайно,
мати в своєму розпорядженні додатковий грошовий капітал, який
він міг би, незалежно від продажу цих \num{30000} метрів, перетворити
в продуктивний капітал і таким чином продовжувати процес
виробництва. Але таке припущення зовсім не змінює справи.
Оскільки справа йде про капітал, авансований на ці \num{30000} метрів,
процес його репродукції є і лишається перерваним. Отже, тут
дійсно з очевидністю виявляється, що операції купця є не що
інше, як операції, які взагалі мусять бути виконані для того, щоб
перетворити товарний капітал виробника у гроші; операції, які
опосереднюють функції товарного капіталу в процесі циркуляції і репродукції. Якщо замість
незалежного купця цим продажем і, крім того, закупівлею повинен був би займатись як виключною
справою простий прикажчик виробника, то цей зв’язок ні
на одну хвилину не був би прихований.

Отже, товарно-торговельний капітал є безперечно не що
інше, як товарний капітал виробника, капітал, який повинен проробити процес свого перетворення в
гроші, виконати на ринку
свою функцію як товарний капітал; тільки тепер ця функція виступає не як побічна операція виробника,
а як виключна операція особливого роду капіталістів, торговців товарами, усамостійнюється як заняття
в особливій сфері капіталовкладення.

Зрештою, це виявляється і в специфічній формі циркуляції
товарно-торговельного капіталу. Купець купує товари і потім
продає їх: $Г — Т — Г'$. В простій товарній циркуляції або навіть
в циркуляції товарів, якою вона виступає як процес циркуляції
промислового капіталу, $Т' — Г — Т$, циркуляція опосереднюється
тим, що кожний грошовий знак двічі міняє місце. Виробник полотна продає свій товар, полотно,
перетворює його в гроші;
гроші покупця переходять у його руки. На ці самі гроші він
купує пряжу, вугілля, працю і~\abbr{т. д.}, знову витрачає ці самі
гроші, щоб зворотно перетворити вартість полотна в товари, які
становлять елементи виробництва полотна. Товар, який він купує, не той самий товар, товар не того
самого роду, який
він продає. Він продав продукти і купив засоби виробництва. Але
інакше стоїть справа в русі купецького капіталу. На 3000\pound{ фунтів стерлінгів} торговець полотном купує
\num{30000} метрів полотна;
він продає ці самі \num{30000} метрів полотна, щоб одержати назад
з циркуляції грошовий капітал (3000\pound{ фунтів стерлінгів}, крім
зиску). Отже, тут двічі міняє місце не той самий грошовий знак,
а той самий товар; він переходить з рук продавця в руки покупця і з рук покупця, який тепер став
продавцем, в руки
іншого покупця. Він продається двічі і може бути проданий
ще багато разів при втручанні в справу ряду купців; і якраз
\parbreak{}  %% абзац продовжується на наступній сторінці


  \parcont{}  %% абзац починається на попередній сторінці
\index{iii1}{0267}  %% посилання на сторінку оригінального видання
тільки за допомогою цього повторного продажу, за допомогою
дворазової зміни місця того самого товару, перший купець одержує назад гроші, авансовані на купівлю
товару; тільки цим
опосереднюється повернення до нього цих грошей. В одному
випадку, випадку $Т' — Г — Т$, дворазова зміна місця тих самих грошей опосереднює те, що товар
відчужується в одній
формі і привласнюється в другій. В другому випадку, випадку
$Г — Т — Г'$, дворазова зміна місця того самого товару опосереднює те, що авансовані гроші знову
вилучаються з циркуляції.
Саме при цьому й виявляється, що товар ще не проданий остаточно, якщо він перейшов з рук виробника
до рук купця, що
цей останній тільки продовжує операцію продажу, або обслуговування функції товарного капіталу. Але
разом з тим виявляється, що те, що́ для продуктивного капіталіста є $Т — Г$,
проста функція його капіталу в його минущій формі товарного
капіталу, те для купця є $Г — Т — Г'$, особливим процесом збільшення вартості авансованого ним
грошового капіталу. Одна
фаза метаморфози товару виявляється тут, щодо купця, як
$Г — Т — Г'$, отже, як еволюція капіталу особливого роду.

Купець остаточно продає товар, в даному разі полотно,
споживачеві, однаково, чи це буде продуктивний споживач (наприклад, білільник), чи особистий, який
використовує полотно
для свого особистого споживання. В наслідок цього до купця
повертається назад авансований капітал (разом із зиском), і він
може знову почати цю операцію. Коли б при купівлі полотна
гроші функціонували тільки як платіжний засіб, так що купцеві довелося б платити тільки через шість
тижнів після одержання товару, і коли б він його продав раніше, ніж мине цей
час, то він міг би заплатити виробникові полотна за його товар,
не авансувавши особисто ніякого грошового капіталу. Коли б він
не продав товар, то він мусив би авансувати 3000\pound{ фунтів стерлінгів} при настанні строку платежу,
замість того, щоб авансувати їх відразу при здачі йому полотна; а коли б він в наслідок падіння
ринкових цін продав товар нижче купівельної ціни,
то він мусив би замістити недібрану частину з свого власного капіталу.

Що ж надає товарно-торговельному капіталові характеру
самостійно функціонуючого капіталу, тимчасом як у руках виробника, який сам продає свої товари, він,
очевидно, виступає
тільки як особлива форма його капіталу в особливій фазі процесу його репродукції, під час його
перебування в сфері циркуляції?

\emph{Поперше}: Та обставина, що товарний капітал пророблює своє
остаточне перетворення в гроші, отже, свою першу метаморфозу, виконує на ринку властиву йому qua
[як] товарному капіталові функцію, перебуваючи в руках агента, відмінного від
виробника цього товарного капіталу; і та обставина, що ця функція товарного капіталу опосереднюється
операціями купця, його
\parbreak{}  %% абзац продовжується на наступній сторінці

\parcont{}  %% абзац починається на попередній сторінці
\index{ii}{0268}  %% посилання на сторінку оригінального видання
само метаморфоза індивідуального капіталу, його оборот, є ланка в кругобігу
суспільного капіталу.

Цей сукупний процес охоплює так само і продуктивне споживання
(безпосередній процес продукції) разом з перетвореннями форми (обмінами,
коли розглядати справу з речового боку), що упосереднюють його,
і особисте споживання з перетвореннями форми або обмінами, що упосереднюють
це споживання. Він охоплює, з одного боку, перетворення
змінного капіталу на робочу силу, а значить, і введення робочої сили в
капіталістичний процес продукції. Тут робітник виступає як продавець
свого товару, робочої сили, а капіталіст як покупець її. Але, з другого
боку, продаж товару включає й купівлю його робітничою клясою, отже, її
особисте споживання. Тут робітнича кляса виступає як покупець, а капіталісти
— як продавці товарів робітникам.

Циркуляція товарового капіталу включає й циркуляцію додаткової
вартости, а значить, і купівлі й продажі, що ними капіталісти упосереднюють
своє особисте споживання, споживання додаткової вартости.

Кругобіг індивідуальних капіталів, розглядуваних у їхньому з’єднанні
в суспільний капітал, отже, кругобіг, розглядуваний в його цілості, охоплює
не лише циркуляцію капіталу, а й загальну товарову циркуляцію.
Ця остання може первісно складатись лише з двох складових частин:
1) власне кругобігу капіталу і 2) кругобігу товарів, що входять в особисте
споживання, отже, товарів, що на них робітник витрачає свою заробітну
плату, а капіталіст — свою додаткову вартість (або частину своєї
додаткової вартости). В усякому разі кругобіг капіталу охоплює й циркуляцію
додаткової вартости, оскільки вона становить частину товарового
капіталу, а також охоплює і перетворення змінного капіталу на робочу
силу, виплату заробітної плати. Але витрачання цієї додаткової вартости
та заробітної плати на товари не становить жодної ланки циркуляції
капіталу, не зважаючи на те, що принаймні витрачання заробітної плати
зумовлює цю циркуляцію.

В І книзі проаналізовано капіталістичний процес продукції і як окремий
процес і як процес репродукції: продукцію додаткової вартости
і продукцію самого капіталу. Зміни форми та речовин, що їх проробляє
капітал у сфері циркуляції, ми припустили як передумову, що на
ній не зупинялись далі. Отже, ми припускали, що капіталіст, з одного
боку, продає продукт за його вартістю, а з другого, знаходить у сфері
циркуляції речові засоби продукції, потрібні для того, щоб відновити
процес або безупинно провадити його. Єдиним актом у сфері циркуляції,
що на ньому нам довелось там зупинитись, був акт купівлі та продажу
робочої сили як основної умови капіталістичної продукції.

В першому відділі цієї II книги ми розглядали різні форми, що їх
набирає капітал у своєму кругобігу, та різні форми самого цього кругобігу.
До робочого часу, розглянутого в І книзі, тепер долучається час
циркуляції.

В другому відділі ми розглядали кругобіг капіталу як періодичний
процес, тобто як оборот капіталу. Ми показали, з одного боку, як різні
\index{ii}{0269}  %% посилання на сторінку оригінального видання
складові частини капіталу (основний і обіговий) пророблюють кругобіг форм
в різні періоди часу й різним способом; з другого боку, ми дослідили обставини,
що ними зумовлюється різний протяг робочого періоду й періоду
циркуляції. Ми показали, як впливає період кругобігу й різне відношення
його складових частин на розмір самого продукційного процесу і на
річну норму додаткової вартости. В дійсності, коли в першому відділі
розглядалось переважно послідовні форми, що їх у своєму кругобігу капітал
постійно набирає й скидає, то в другому відділі ми розглянули, як
у межах цього руху й послідовности форм капітал даної величини одночасно,
хоч і в змінному розмірі, поділяється на різні форми — на продуктивний
капітал, грошовий капітал і товаровий капітал, так, що ці
форми не лише чергуються одна з однією, але різні частини сукупної
капітальної вартости постійно одна поряд однієї перебувають і функціонують
у цих різних станах. Саме грошовий капітал при цьому виявив
особливість, яка не виявлялась в книзі першій. Ми виявили ті певні закони,
що згідно з ними різні величиною складові частини даного капіталу,
відповідно до умов обороту, постійно мусять авансуватись і відновлюватись
у формі грошового капіталу для того, щоб підтримувати
продуктивний капітал даного розміру в безперервному функціонуванні.

Але і в першому і в другому відділі мова була завжди тільки про індивідуальний
капітал, про рух усамостійненої частини суспільного капіталу.

Але кругобіги індивідуальних капіталів переплітаються один з одним,
являють передумову і зумовлюють один одного і саме в цьому сплетінні
й становлять рух сукупного суспільного капіталу. Як при простій товаровій
циркуляції уся метаморфоза одного товару виступала як ланка
ряду метаморфоз товарового світу, так тепер метаморфоза індивідуального
капіталу виступає як ланка ряду метаморфоз суспільного капіталу. Але
коли проста циркуляція товарів зовсім не включає неодмінно циркуляції
капіталу, — бо товарова циркуляція може відбуватись на основі некапіталістичної
продукції, — то кругобіг сукупного суспільного капіталу включає,
як уже зазначено, і товарову циркуляцію, що не входить в кругобіг
індивідуального капіталу, тобто включає циркуляцію товарів, які не є
капітал.

Тепер ми повинні розглянути процес циркуляції (а він у своїй сукупності
є форма процесу репродукції) індивідуальних капіталів, як складових
частин сукупного суспільного капіталу, отже, розглянути процес
циркуляції \emph{цього суспільного} сукупного капіталу.

\subsection{Роля грошового капіталу}

[Хоч дальший виклад належить до пізнішої частини цього відділу,
все ж ми зараз дослідимо це, тобто грошовий капітал, розглядуваний
як складова частина суспільного сукупного капіталу].

При розгляді обороту індивідуального капіталу грошовий капітал виявив
себе з двох боків.


\index{ii}{0270}  %% посилання на сторінку оригінального видання
Поперше, він являє ту форму, що в ній кожний індивідуальний капітал
виступає на кін, починає свій процес як капітал. Тому він виступає
як primus motor\footnote*{
Перший рушій. \emph{Ред.}
}, що надає руху цілому процесові.

Подруге. Відповідно до різного протягу періоду обороту і різного
відношення між обома складовими частинами його — робочим періодом і
періодом циркуляції — складова частина авансованої капітальної вартости,
що її завжди треба авансувати і відновлювати в грошовій формі, є різна
у відношенні до продуктивного капіталу, що його вона пускає в рух,
тобто у відношенні до безперервного розміру продукції. Але хоч яке
це буде відношення, за всіх обставин та частина капітальної вартости,
що процесує, що може постійно функціонувати як продуктивний капітал,
обмежується тією частиною авансованої капітальної вартости, яка мусить
завжди існувати в грошовій формі поряд продуктивного капіталу. Тут
ідеться лише про нормальний оборот, про абстрактну пересічну величину.
При цьому ми лишаємо осторонь додатковий грошовий капітал,
потрібний, щоб вирівнювати застої циркуляції.

\emph{До першого пункту}. Товарова продукція припускає товарову
циркуляцію, а товарова циркуляція припускає виявлення товару в грошах,
грошову циркуляцію; двоїсте буття товару: як товару, і як грошей,
є закон виявлення продукту як товару. Так само капіталістична товарова
продукція, — хоч суспільно, хоч індивідуально розглядувана —
припускає капітал у грошовій формі або грошовий капітал як primus
motor для кожного новопосталого підприємства, і як постійний рушій. Обіговий
капітал зокрема припускає, що через короткі переміжки постійно
знову й знов з’являється грошовий капітал як рушій. Всю авансовану
капітальну вартість, тобто всі складові частини капіталу, що складаються
з товарів, робочої сили, засобів праці й матеріялів продукції, постійно
доводиться знову й знов купувати на гроші. Те, що тут сказано про індивідуальний
капітал, має силу й щодо суспільного капіталу, який функціонує
лише в формі багатьох індивідуальних капіталів. Але, як уже показано
в І книзі, з цього зовсім не випливає, щоб поле функціонування
капіталу, маштаб продукції, навіть на капіталістичній основі, в своїх
\emph{абсолютних} розмірах залежав від розміру діющого грошового капіталу.

В капітал заведено елементи продукції, що їхня здатність розширюватись,
у певних межах, не залежить від величини авансованого грошового
капіталу. При однаковій оплаті робочої сили її можна екстенсивно
або інтенсивно більше визискувати. Якщо із збільшенням визиску збільшується
грошовий капітал (тобто підвищується заробітну плату), то не
пропорційно до збільшення визиску, отже, pro tanto він зовсім не збільшується.

Продуктивно експлуатований матеріял природи — що зовсім не являє
собою елементу вартости капіталу — земля, море, руди, ліси тощо, при
більшому напруженні тієї самої кількости робочої сили може інтенсивно
або екстенсивно більше експлуатуватись без збільшеного авансування грошового
\index{ii}{0271}  %% посилання на сторінку оригінального видання
капіталу. Таким чином, реальні елементи продуктивного капіталу
збільшуються, не потребуючи додаткового грошового капіталу. А оскільки
його треба буде на додаткові допоміжні матеріяли, то грошовий капітал,
що в ньому авансується капітальну вартість, збільшується не пропорційно
до поширення діяльности продуктивного капіталу, отже, pro
tanto зовсім не збільшується.

Ті самі засоби праці, отже, той самий основний капітал, можна використати
ефективніше так збільшуючи протяг його щоденного вживання,
як і збільшуючи інтенсивність його застосування, не витрачаючи
при цьому додаткових грошей на основний капітал. В такому разі відбувається
лише швидший оборот основного капіталу, але зате елементи
його репродукції постачатиметься швидше.

Лишаючи осторонь матеріяли природи, в процес продукції можуть
заводитись, як чинники більшої або меншої ефективности, сили
природи, що нічого не коштують. Ступінь їхньої ефективности
залежить від методів та поступу науки, що нічого не коштують капіталістові.

Це саме стосується до суспільного сполучення робочої сили в продукційному
процесі та до вмілости, надбаної поодинокими робітниками.
Кері на підставі цього вважає, що власник землі ніколи не одержує досить,
бо йому оплачується не ввесь той капітал, зглядно не всю ту
працю, що її з прадавніх часів вкладалось у землю, щоб надати їй теперішньої
родючости. (Звичайно, про ту родючість, що їй відбирається,
не згадується). Але в такому разі кожен поодинокий робітник мусив
би оплачуватись відповідно до тієї праці, яку витратив увесь рід людський,
щоб перетворити дикуна на сучасного механіка. Тут слід було б
міркувати саме навпаки: коли взяти на увагу всю вкладену в землю
неоплачену, але землевласниками й капіталістами перетворену на гроші
працю, то ввесь вкладений у землю капітал повернуто багато разів та
ще з лихварським процентом, отже суспільство давно вже й багато
разів викупило земельну власність.

Підвищення продуктивних сил праці, оскільки воно не має за передумову
додаткову витрату капітальних вартостей, підвищує, правда, насамперед
лише масу продукту, а не вартість його; останню воно підвищує
лише остільки, оскільки воно дає змогу тією самою працею репродукувати
більше сталого капіталу, отже, зберегти вартість його. Але разом з
тим підвищення продуктивних сил праці утворює новий матеріял для капіталу,
тобто базу для підвищеної акумуляції капіталу.

У першій книзі вже показано, що оскільки сама організація суспільної
праці, а тому й підвищення суспільної продуктивної сили праці потребує,
щоб продукцію провадилось у широкому маштабі, отже, щоб поодинокі
капіталісти авансували великі маси грошового капіталу, — це
стається почасти через централізацію капіталу в небагатьох руках, при
цьому немає потреби в тому, щоб розмір діющих капітальних вартостей,
а тому й розмір того грошового капіталу, що в ньому їх авансується,
абсолютно зростав. Величина поодиноких капіталів може зростати в наслідок
\index{ii}{0272}  %% посилання на сторінку оригінального видання
централізації їх в небагатьох руках, при чому суспільна сума цих
капіталів не зростає. Тут маємо лише змінний розподіл поодиноких капіталів.

Нарешті, в попередньому розділі показано, що скорочення періоду
обороту дозволяє або пускати в рух з меншим грошовим капіталом той
самий продуктивний капітал, або з тим самим грошовим капіталом —
більший продуктивний капітал.

Але все це, очевидно, не стосується власне до питання про грошовий
капітал. Це показує лише, що авансований капітал — дана сума вартости,
що в своїй вільній формі, в своїй формі вартости, складається з
певної суми грошей, — по своєму перетворенні на продуктивний капітал
має в собі продуктивні потенції, що їхні межі визначаються не величиною
його вартости, а можуть, навпаки, до певної міри діяти з різною екстенсивністю
або інтенсивністю. Коли дано ціни елементів продукції — засобів
продукції та робочої сили — то цим визначено величину грошового
капіталу, потрібного на закуп певної кількости цих елементів продукції,
наявних у вигляді товарів. Інакше кажучи, визначено величину вартости
того капіталу, що його треба авансувати. Але розміри, що в них цей
капітал діє як вартостетворець і продуктотворець, елястичні й змінні.

\emph{До другого пункту}. Само собою зрозуміло, що та частина суспільної
праці та засобів продукції, яку доводиться щорічно витрачувати
на продукцію або закуп золота, щоб замістити зужиту монету, є pro
tanto зменшення розміру суспільної продукції. Щождо грошової вартости,
яка функціонує почасти як засіб обігу, а почасти як скарб, то раз
вона вже є, скоро її здобуто, вона перебуває поряд з робочою силою,
спродукованими засобами продукції та природними джерелами багатства.
Її не можна розглядати, як щось, що обмежує все це. Перетворенням її
на елементи продукції, обміном з іншими народами, можна було б розширити
розміри продукції. Але для цього треба, щоб гроші тут, як і
раніше, відігравали ролю світових грошей.

Залежно від величини періоду обороту потрібна більша або менша
маса грошового капіталу, щоб пустити в рух продуктивний капітал. Так
само ми бачили, що поділ періоду обороту на робочий час і час циркуляції
зумовлює збільшення лятентного в грошовій формі капіталу, або
капіталу, що його застосовання відкладається.

Оскільки період обороту визначається протягом робочого періоду, остільки
його за інших незмінних умов, визначається матеріяльною природою
процесу продукції, отже, не специфічним суспільним характером
цього процесу продукції. Однак на основі капіталістичної продукції довготриваліші
широкі операції зумовлюють більші авансування грошового
капіталу на довший час. Отже, продукція в таких галузях залежить від
тих меж, що в них поодинокий капіталіст порядкує грошовим капіталом.
В цих обмеженнях пробиває вилом система кредиту і зв’язані з нею асоціяції,
прим., акційні товариства. Тому порушення на грошовому ринку
припиняють діяльність таких підприємств, тимчасом як ці самі підприємства
і собі зумовлюють порушення на грошовому ринку.


\index{ii}{0273}  %% посилання на сторінку оригінального видання
На основі суспільної продукції треба визначити маштаб, що в ньому
такі операції, які на довгий час відтягують робочу силу й засоби продукції,
не даючи протягом цього часу жодного продукту як корисного
наслідку, можуть провадитись без шкоди для тих галузей продукції, які
постійно або кілька разів на рік не лише відтягують робочу силу й засоби
продукції, а й дають засоби, існування й засоби продукції. За суспільної
продукції, так само, як і за капіталістичної продукції, робітники
в галузях підприємств з короткими робочими періодами, як і раніше, лише
на короткий час відтягуватимуть продукти, не даючи натомість нового
продукту, тимчасом як галузі підприємств з довгими робочими періодами,
перше ніж вони сами почнуть давати продукти, постійно відтягують
продукти на довгий час. Отже, ця обставина випливає з речових
умов відповідного процесу праці, а не з його суспільної форми. За суспільної
продукції грошовий капітал відпадає. Суспільство розподіляє робочу
силу й засоби продукції між різними галузями праці. Продуценти
можуть, правда, одержувати паперові посвідки, що ними вони вилучають
з суспільних споживних запасів ту кількість продуктів, яка відповідає їхньому
робочому часові. Ці посвідки — зовсім не гроші. Вони не циркулюють.

Тепер ми бачимо, що, оскільки потреба в грошовому капіталі випливає
з протягу робочого періоду, її зумовлено двома обставинами: \emph{поперше},
тією, що гроші взагалі є та форма, що в ній мусить виступити
кожен індивідуальний капітал (кредит ми лишаємо осторонь) для того,
щоб перетворитись на продуктивний капітал. Це випливає з суті капіталістичної
продукції, взагалі товарової продукції. — \emph{Подруге}, величину
потрібного грошового авансування зумовлює та обставина, що протягом
порівняно довгого часу суспільству постійно відбирається робочу силу
й засоби продукції, при чому протягом цього часу йому не повертається
жодного продукту, що його можна було б перетворити на гроші.
Першої обставини, а саме того, що авансовуваний капітал треба авансувати
в грошовій формі, не знищує форма самих цих грошей, тобто те,
що вони є або металеві, або кредитові гроші, або знаки вартости й~\abbr{т. ін.} На другу обставину жодного впливу не справляє те, за допомогою
яких грошових засобів або за допомогою якої форми продукції
відтягають працю, засоби існування та засоби продукції, не подаючи
натомість у циркуляцію жодного еквіваленту.
\label{original-273}


\index{ii}{0274}  %% посилання на сторінку оригінального видання

\sectionextended[%
	Давніші уявлення про предмет]{%
	Давніші уявлення про предмет\footnotemark{}}{%
	\subsection{Фізіократи}}

\label{original-274}
Кене%
\footnotetext{Тут починається рукопис VIII.}
в Tableau économique кількома широкими рисами показує, як
річний продукт національної продукції певної вартости розподіляється через
циркуляцію так, що, за інших незмінних обставин, може відбуватися проста
репродукція цього продукту, тобто репродукція в попередньому маштабі.
За вихідний пункт періоду продукції є по суті врожай останнього року.
Незчисленні індивідуальні акти циркуляції тут з самого початку об’єднуються
в характеристично-суспільний масовий рух, — в циркуляцію між
великими, функціонально визначеними економічними клясами суспільства.
Нас тут цікавить ось що: частина сукупного продукту — що, як і всяка
інша частина його, як предмет споживання, являє новий результат праці
минулого року — є разом з тим лише носій старої капітальної вартости,
що знову з’являється в попередній натуральній формі. Вона не циркулює,
а лишається в руках її продуцентів, кляси фармерів, щоб там знову почати
служити їм як капітал. До цієї сталої частини річного продукту
Кене залічує також неналежні сюди елементи, але він схоплює суть
справи завдяки обмеженості свого кругогляду, що за ним хліборобство є
та єдина сфера застосовання людської праці, яка продукує додаткову
вартість, тобто з капіталістичного погляду, єдина справді продуктивна
сфера застосовання праці. Економічний процес репродукції, хоч який
буде її специфічно-суспільний характер, завжди переплітається в цій
галузі (в хліборобстві) з природним процесом репродукції. Цілком очевидні
умови цього останнього пояснюють умови першого й не припускають
до хибних висновків, що до них призводить марево циркуляції.

Етикетка системи відрізняється від етикетки інших товарів, між
іншим, тим, що вона обдурює не лише покупця, а часто і продавця.
Сам Кене та його ближчі учні вірили в свою февдальну вивіску. Так
само вірять у неї досі наші шкільні вчені. А в дійсності система фізіократів
є перша систематична концепція капіталістичної продукції. Представник
промислового капіталу — кляса фармерів — є керівник цілого
економічного руху. Хліборобство провадиться капіталістично, тобто як
підприємство капіталістичного фармера в широкому маштабі; безпосередній
обробник землі є найманий робітник. Продукція створює не лише
предмети споживання, а й вартість їхню; але рушійним мотивом продукції
є здобування додаткової вартости, що місце її зародження є
сфера продукції, а не сфера циркуляції. З тих трьох кляс, що фігурують
як носії суспільного процесу репродукції, упосереднюваного циркуляцією
безпосередній визискувач „продуктивної“ праці, продуцент додаткової
\parbreak{}  %% абзац продовжується на наступній сторінці

\parcont{}  %% абзац починається на попередній сторінці
\index{ii}{0275}  %% посилання на сторінку оригінального видання
вартости, капіталістичний фармер, відрізняється від того, хто її просто
привласнює.

Капіталістичний характер фізіократичної системи ще за доби її розквіту
викликав опозицію, з одного боку, Лінґе й Маблі, а з другого —
захисників дрібного вільного землеволодіння.
\pfbreak
Те, що А.~Сміт у своїй аналізі процесу репродукції робить крок
назад\footnote{
„Капітал“, т. І, розділ XXII, 2, прим. 32.
}, впадає на очі то більше, що він взагалі не лише далі опрацьовує
правильну аналізу Кене, напр., узагальнюючи його „avances primitives“
і „avances annuelles“ в „основний“ капітал та „обіговий“
капітал\footnote{
І тут йому розчистили шлях деякі фізіократи, насамперед Тюрґо. Останній
уже частіше, ніж Кене та інші фізіократи, вживає слово „капітал“ замість avances
і ще більше ототожнює avances або capitaux мануфактуристів із avances або
capitaux фармерів. Напр., „Так само як ці останні (підприємці-мануфактуристи),
вони (фармери, тобто капіталістичні орендарі) повинні одержувати, крім повернених
капіталів“ і~\abbr{т. ін.} („Comme eux les entrepreneurs-manufacturiers), ils (les fermiers)
doivent recueillir outre la rentrée de leurs capitaux etc.“ — Turgot, Oeuvres, éd.
Daire, Paris, 1844. Tome I, p. 40).
}, але подекуди й зовсім допускається фізіократичних помилок.
Напр., щоб довести, що фармер продукує більшу вартість, ніж яка інша
відміна капіталістів, він каже: „Жоден інший капітал однакової величини
не пускає в рух більшої маси продуктивної праці, ніж капітал фармера.
Не лише його челядь, але й робоча худоба його складається з продуктивних
робітників“. (Приємний комплімент для челяді). „В хліборобстві
поряд людей працює також природа; і хоч її \so{праця не коштує
жодних витрат}, все ж її продукт має свою \so{вартість, цілком
так само, як продукт праці найдорожчих робітників.}
Найважливіші операції в хліборобстві, здається, спрямовано не так на те,
щоб підвищити природну родючість, хоч вони спричиняють і це, — як
на те, щоб повернути її на продукцію найкорисніших для людини рослин.
Поле, заросле бур’яном, досить часто дає таку ж саму велику
кількість рослинности, як і найкраще оброблений виноградник або
нива. Насадження рослин і культивування часто мають більший вплив на
реґулювання, ніж на відживлення активної родючости природи. Після
того як усю працю обробітку вивершено, природі припадає ще чимала
частина роботи. Отже, робітники та робоча худоба (!), зайняті в хліборобстві,
не лише репродукують, на зразок мануфактурних робітників,
вартість, рівну власному їхньому споживанню та капіталові, що їх уживає,
до праці, плюс зиск капіталіста: вони репродукують куди більшу вартість.
Крім капіталу фармера та всього його зиску вони реґулярно репродукують
ще й ренту землевласника. Ренту можна розглядати, як продукт
природних сил, що користування ними землевласник позичає орендареві.
Вона більша або менша, залежно від припушуваного рівня цих сил,
тобто залежно від припущеної природної або штучно досягненої родючости
ґрунту. Вона — продукт природи, який лишається по відліченні
\parbreak{}  %% абзац продовжується на наступній сторінці

\parcont{}  %% абзац починається на попередній сторінці
\index{ii}{0276}  %% посилання на сторінку оригінального видання
або покриттю всього того, що можна вважати за продукт рук людських.
Вона рідко коли менша, ніж чверть, і часто більша, ніж третина цілого
продукту. Жодна однакова маса продуктивної праці, застосована в мануфактурі,
ніколи не може зумовити такої великої репродукції. В мануфактурі
природа не робить нічого, людина — все; а репродукція завжди
мусить бути пропорційна потужності аґентів, що її переводять. Тому
капітал, вкладений у хліборобство, не лише пускає в рух більшу масу
продуктивної праці, ніж якийбудь інший, однаковий величиною капітал,
застосований у мануфактурі, але, порівняно з масою зайнятої ним продуктивної
праці, він додає куди більшу вартість до річного продукту
землі та праці даної країни, — до цього справжнього багатства і доходу
її жителів“. (Кн.~II, розд. 5, стор. 242).

А.~Сміт каже в II книзі, розд. І; „Вся вартість засівного матеріялу
теж є власне основний капітал“. Отже, тут капітал \deq{} капітальній вартості;
він існує в „основній“ формі. „Хоч засівний матеріял завжди переходить
з поля до комори й навпаки, він ніколи не змінює свого власника, а тому
в дійсності не циркулює. Фармер здобуває свій зиск не через його продаж,
а через його приріст“, (ст. 186). Обмеженість тут у тому, що Сміт
не бачить, як то вже бачив Кене, що вартість сталого капіталу знову
з’являється в відновленій формі, отже, не бачить важливого моменту
процесу репродукції, а бачить лише ще одну ілюстрацію — та до того ж
і фалшиву — свого відрізнювання між обіговим капіталом і основним
капіталом. Перекладаючи „avances primitives“ і „avances annuelles“
виразами „fixed capital“ і „circulating capital“, Сміт робить крок наперед
щодо вживання слова „капітал“, поняття якого узагальнюється і стає
незалежне від особливого застосування його фізіократами до сфери
„хліборобської“; крок назад у тому, що ріжниці між „основним“ капіталом
і „обіговим“ капіталом розглядається і їх додержується як вирішальних
ріжниць.

\subsection{Адам Сміт}

\subsubsection{Загальні погляди А.~Сміта}

А.~Сміт каже в книзі І, розд. 6, стор. 42; „В усякому суспільстві
ціна кожного товару кінець-кінцем розкладається або на ту або на другу
з цих трьох частин (заробітна плата, зиск, земельна рента), або на всі
три частини; і в усякому розвиненому суспільстві всі вони три, більш
або менш, увіходять як складові частини в ціну переважної більшости
товарів“\footnote{
Щоб читача не ввів у помилку вислів „ціна переважної більшости товарів“,
наведемо витяг про те, як сам А.~Сміт розуміє цей вислів. Напр., в ціну морської
риби рента не входить, а входить лише заробітна плата й зиск; в ціну Scotch
pebbles (шотляндської ріні) входить лише заробітна плата: „В деяких частинах
Шотляндії бідняки промишляють тим, що збирають на морському березї різнокольорові
камінці, так звану шотляндську рінь. Ціна, що її платять їм за ці камінці
різьбарі, складається тільки з їхньої заробітної плати, бо ні земельна рента, ні
зиск не становлять жодної частини її“.
}; або, як сказано далі, стор. 63: „Заробітна плата, зиск і земельна
\index{ii}{0277}  %% посилання на сторінку оригінального видання
рента є \so{три первісні джерела} всякого доходу, так само,
як і всякої \so{мінової вартости}“. Далі ми розглянемо докладніше це
вчення А.~Сміта про „складові частини ціни товарів“, зглядно про „всяку
мінову вартість“. Далі він каже: „Що все це має силу для всякого
поодинокого товару, взятого окремо, то повинно воно мати силу й для
всіх товарів, разом узятих, які становлять \so{увесь річний продукт}
землі та праці кожної країни. \so{Вся ціна або мінова вартість}
цього річного продукту мусить \so{розкладатись} на ці самі три частини
та \so{розподілятись} між різними жителями країни або як \so{плата} за
їхню працю, або як \so{зиск} їхнього капіталу, або як \so{рента} з їхнього
землеволодіння“. (Кн.~II, розд., ст. 190).

Після того, як А.~Сміт і ціну всіх товарів, узятих окремо, і „всю ціну
або мінову вартість\dots{} річного продукту землі та праці кожної країни“
розклав таким чином на три джерела доходів: доходів найманого робітника,
капіталіста й земельного власника, на заробітну плату, зиск і земельну
ренту, він все ж мусить контрабандою ввести обхідним шляхом
четвертий елемент, а саме елемент капіталу. Це робиться через відрізнення
між гуртовим і чистим доходом. „\so{Гуртовий} дохід усіх жителів великої
країни охоплює \so{ввесь річний продукт} їхньої землі та їхньої праці;
чистий \so{дохід} — \so{частину}, що лишається в їхньому розпорядженні,
\so{відлічивши втрати на підтримання}, поперше, їхнього \so{основного},
а подруге, їхнього \so{поточного капіталу}, тобто
частину, що її вони можуть, не порушуючи свого капіталу, залічити
до свого споживного запасу або витратити на своє утримання, комфорт
і втіхи. Справжнє їхнє багатство теж пропорційне не їхньому гуртовому,
а чистому їхньому доходові“. (Там само, ст. 190).

На це ми зауважимо ось що:

1) А.~Сміт тут виразно розглядає тільки просту репродукцію, а не
репродукцію в поширеному маштабі, або акумуляцію; він каже лише про
видатки на підтримання (maintening) діющого капіталу. „Чистий“ дохід
дорівнює тій частині річного продукту — хоч суспільства, хоч індивідуального
капіталіста — яка може ввійти в „фонд споживання“, але розміри
цього фонду не повинні порушити діющого капіталу (encroach upon capital).
Отже, частина вартости, так індивідуального, як і суспільного продукту
не сходить ні на заробітну плату, ні на зиск або земельну ренту,
а сходить на капітал.

2) А.~Сміт ховається від своєї власної теорії за допомогою гри слів,
за допомогою розмежування між gross і net revenue — гуртовим і чистим
доходом. Поодинокий капіталіст, як і ціла кляса капіталістів, або так
звана нація, замість зужиткованого в продукції капіталу, одержує товаровий
продукт, що його вартість — її можна визначити в пропорційних частках
цього самого продукту — з одного боку, покривав витрачену капітальну
вартість, а тому становить дохід або, буквально, revenue (revenue — дієприкметник
від revenir, повертатись), однак, nota bene, являє capital-revenue або
дохід на капітал; з другого боку, маємо складові частини вартости, що
їх „розподіляється між різними жителями країни або як плату за їхню
\parbreak{}  %% абзац продовжується на наступній сторінці

\parcont{}  %% абзац починається на попередній сторінці
\index{ii}{0278}  %% посилання на сторінку оригінального видання
працю, або як зиск з їхнього капіталу, або як ренту з їхньої земельної
власности“, що в звичайному житті й розуміється як дохід. Вартість
цілого продукту, хоч для індивідуального капіталіста, хоч для цілої країни,
являє тому чийсь дохід; але, з одного боку, дохід на капітал, а з другого,
відмінну від цього доходу форму „revenue“ Отже, те, що усувається при
розкладі вартости товару на її складові частини, знову вводиться через
задні двері — через двозначність слова „revenue“\footnote*{
Про двояке значення слова „дохід“ див. „Капітал“, кн. І, розд. XXII, 3,
прим. 33. \Red{Ред.}
}. Але „заприбуткувати“
можна лише такі складові частини вартости продукту, які вже в ньому
існують. Щоб капітал одержувалось як дохід, капітал треба спочатку
витратити.

А.~Сміт каже далі: „Найнижча звичайна норма зиску має завжди
дещо перевищувати те, чого досить для відшкодування випадкових втрат,
що їм підпадає кожне застосування капіталу. Тільки цей надлишок і є
чистий, або нетто-зиск“. (Який же капіталіст розумів би зиск, як
доконечні витрати капіталу?) „В те, що зветься гуртовим зиском, часто
входить не тільки цей надлишок, а й частина, що її зберігається про
такі незвичайні втрати. (Кн.~І, розд. 9, стор. 72). Але це нічого іншого
не значить, а тільки те, що частина додаткової вартости, розглядувана
як частина гуртового зиску, мусить становити страховий фонд для продукції.
Цей страховий фонд утворює частина додаткової праці, яка в
цьому розумінні безпосередньо продукує капітал, тобто фонд, призначений
для репродукції. Щождо до витрат на „підтримання“ основного
капіталу й~\abbr{т. ін.} (див. вище цитовані місця), то заміщення спожитого
основного капіталу новим не становить нового капіталовкладення, а є
лише відновлення старої капітальної вартости в новій формі. Щождо
витрат на ремонт основного капіталу, що їх А.~Сміт теж залічує до витрат
на підтримання, то вони входять у ціну авансованого капіталу.
Та обставина, що капіталіст замість вкладати їх одним заходом, вкладає
їх підчас функціонування капіталу лише поступово та в міру потреби,
і може робити ці вкладання з уже одержаного зиску, зовсім не змінює
джерела цього зиску. Складова частина вартости, що з неї він походить,
показує лише, що робітник дає додаткову працю й для страхового фонду
й для фонду, призначеного на ремонт.

А.~Сміт розповідає нам далі, що з чистого доходу, тобто з доходу в
специфічному значенні, треба вилучити ввесь основний капітал, а також і
всю ту частину обігового капіталу, яка потрібна так для підтримання й
ремонту основного капіталу, як і для поновлення його, — тобто в дійсності
треба вилучити ввесь капітал, що перебуває не в такій натуральній
формі, в якій він призначається для споживного фонду.

„Всі витрати на підтримання основного капіталу, очевидно, треба
виключити з чистого доходу суспільства. Ні сировинні матеріяли, потрібні,
щоб тримати в належному стані корисні машини та промислові
знаряддя, ні продукт праці, потрібний, щоб перетворити ці сировинні
\parbreak{}  %% абзац продовжується на наступній сторінці

\parcont{}  %% абзац починається на попередній сторінці
\index{iii1}{0279}  %% посилання на сторінку оригінального видання
ми припустимо, що, крім цих 900\pound{ фунтів стерлінгів} промислового
капіталу, сюди долучається ще 100\pound{ фунтів стерлінгів} купецького
капіталу, який pro rata [пропорціонально] своїй величині має
таку саму частку в зиску, як і той. Згідно з припущенням,
купецький капітал становить \sfrac{1}{10} сукупного капіталу в 1000.
Отже, з сукупної додаткової вартості в 180 йому припадає \sfrac{1}{10},
і таким чином він одержує зиск нормою в 18\%. Отже, зиск,
який належить поділити між рештою — \sfrac{9}{10} сукупного капіталу,
фактично дорівнює вже тільки 162, або на капітал в 900 він так
само \deq{} 18\%. Отже, ціна, по якій $Т$ продається володільцями промислового
капіталу в 900 торговцям товарами, $= 720c \dplus{} 180v \dplus{} 162m \deq{} 1062$.
Отже, якщо купець накине на свій капітал
в 100 пересічний зиску 18\%, то він продасть товари за $1062 \dplus{} 18 \deq{} 1080$,
тобто по їх ціні виробництва, або — якщо розглядати
сукупний товарний капітал — по їх вартості, хоч він добуває
свій зиск тільки в циркуляції і за допомогою циркуляції, і тільки
в наслідок перевищення його продажної ціни над його купівельною
ціною. Але все ж він продає товари не вище їх вартості
або не вище їх ціни виробництва саме тому, що він купив їх
у промислових капіталістів нижче їх вартості або нижче їх ціни
виробництва.

Таким чином, в утворення загальної норми зиску купецький
капітал входить як визначальний фактор pro rata тій частині,
яку він становить у сукупному капіталі. Отже, якщо в наведеному
випадку ми кажемо: пересічна норма зиску \deq{} 18\%, то вона
була б \deq{} 20\%, якби \sfrac{1}{10} сукупного капіталу не була купецьким
капіталом і якби в наслідок цього загальна норма зиску не знизилася
на \sfrac{1}{10}. Разом з цим з’являється точніше, обмежувальне визначення
ціни виробництва. Під ціною виробництва, як і раніш,
слід розуміти ціну товару \deq{} його витратам (вартості вміщеного
в ньому сталого \dplus{} змінного капіталу) \dplus{} пересічний зиск на них.
Але цей пересічний зиск визначається тепер інакше. Він визначається
сукупним зиском, що його створює сукупний продуктивний
капітал; але обчислюється він не просто на цей сукупний
продуктивний капітал, — так що, коли цей останній, як
припущено вище, \deq{} 900, а зиск \deq{} 180, то пересічна норма зиску
була б $= \frac{180}{900} \deq{} 20\%$, — а на сукупний продуктивний капітал \dplus{} торговельний
капітал, так що, коли продуктивний капітал \deq{} 900, а торговельний \deq{} 100, то пересічна норма зиску \deq{}
$\frac{180}{1000} \deq{} 18\%$.
Отже, ціна виробництва \deq{} $k$ (витратам) \dplus{} 18, замість дорівнювати
$k \dplus{} 20$. В пересічній нормі зиску врахована вже та частина
сукупного зиску, яка припадає на торговельний капітал. Тому
дійсна вартість або ціна виробництва сукупного товарного капіталу
$= k \dplus{} р \dplus{} h$ (де $h$ є торговельний зиск). Отже, ціна виробництва,
або та ціна, по якій продає промисловий капіталіст як
такий, менша, ніж дійсна ціна виробництва товару; або, якщо
\parbreak{}  %% абзац продовжується на наступній сторінці

\parcont{}  %% абзац починається на попередній сторінці
\index{iii1}{0280}  %% посилання на сторінку оригінального видання
розглядати сукупність товарів, то ціни, по яких їх продає клас
промислових капіталістів, менші, ніж їх вартості. Так, у вищенаведеному випадку: 900 (витрати) \dplus{} 18\%
на 900, або 900 \dplus{} 162 \deq{}
1062. Продаючи товар, який коштує йому 100, за 118, купець, звичайно,
накидає 18\%; але тому що товар, який він купив за 100,
вартий 118, то він таким чином продає його не дорожче його
вартості. Ми вживатимем вираз ціна виробництва у вищевикладеному
ближчому його значенні. В такому разі ясно, що зиск
промислового капіталіста дорівнює надлишкові ціни виробництва
товару понад його витрати виробництва і що, в відміну від цього
промислового зиску, торговельний зиск дорівнює надлишкові
продажної ціни понад ціну виробництва товару, яка для купця
є купівельною ціною товару; але ясно, що дійсна ціна товару \deq{}
його ціні виробництва \dplus{} купецький (торговельний) зиск. Подібно
до того, як промисловий капітал тільки реалізує зиск, який міститься
вже у вартості товарів як додаткова вартість, так і торговельний
капітал реалізує зиск тільки тому, що ще не вся
додаткова вартість, або зиск, реалізована в ціні товару, реалізованій
промисловим капіталом\footnote{
\emph{Джон Беллерс.}
}. Таким чином, ціна, по якій купець
продає, стоїть вище його купівельної ціни не тому, що
перша стоїть вище всієї вартості, а тому, що друга стоїть нижче її.

Отже, купецький капітал бере участь у вирівненні додаткової
вартості в пересічний зиск, хоча не бере участі у виробництві
цієї додаткової вартості. Тому загальна норма зиску вже
передбачає відрахування з додаткової вартості, яке припадає купецькому
капіталові, отже, відрахування з зиску промислового
капіталу.

З вищевикладеного випливає:

1) Чим більший купецький капітал порівняно з промисловим
капіталом, тим менша норма промислового зиску, і навпаки;

2) Якщо в першому відділі виявилось, що норма зиску завжди
виражає меншу норму, ніж норма дійсної додаткової вартості,
тобто завжди виражає ступінь експлуатації праці занадто низьким,
— наприклад, у наведеному вище випадку $720 c \dplus{} 180 v \dplus{} 180 m$
норма додаткової вартості в 100\% виражається як норма зиску
тільки в 20\%, — то ці відношення розходяться ще більше,
оскільки тепер сама пересічна норма зиску, якщо врахувати
частку, яка припадає купецькому капіталові, в свою чергу виявляється
меншою, — в даному випадку 18\% замість 20\%. Отже,
пересічна норма зиску безпосередньо експлуатуючого капіталіста
виражає норму зиску меншою, ніж вона є в дійсності.

Якщо припустити всі інші умови незмінними, відносний розмір
купецького капіталу (при чому, однак, одна його різновидність,
капітал роздрібних торговців, становить виняток) стоятиме
у зворотному відношенні до швидкості його обороту, отже,
у зворотному відношенні до енергії процесу репродукції взагалі.
\parbreak{}  %% абзац продовжується на наступній сторінці

\parcont{}  %% абзац починається на попередній сторінці
\index{iii1}{0281}  %% посилання на сторінку оригінального видання
В ході наукового аналізу як вихідний пункт утворення загальної
норми зиску виступають промислові капітали і конкуренція
між ними, і тільки пізніше вноситься поправка, доповнення і модифікація
в наслідок втручання купецького капіталу. В ході історичного
розвитку справа стоїть якраз навпаки. Капітал, який спочатку
визначає ціни товарів більш чи менш по їх вартостях, є торговельний
капітал, а та сфера, в якій вперше утворюється загальна
норма зиску, є сфера циркуляції, яка опосереднює процес репродукції.
Первісно промисловий зиск визначається торговельним
зиском. Тільки після того, як капіталістичний спосіб виробництва
вкорінюється і виробник сам стає купцем, торговельний зиск
зводиться до тієї відповідної частини сукупної додаткової вартості,
яка припадає торговельному капіталові як відповідній
частині сукупного капіталу, занятого в суспільному процесі репродукції.

При додатковому вирівненні зисків в наслідок втручання
купецького капіталу виявилось, що у вартість товару не входить
ніякий додатковий елемент на авансований грошовий капітал
купця, що надбавка до ціни, завдяки якій купець одержує
свій зиск, дорівнює тільки тій частині вартости товару, яку
продуктивний капітал не зараховує в ціну виробництва товару,
поступається нею. З цим грошовим капіталом справа стоїть
саме так, як з основним капіталом промислового капіталіста,
оскільки його не спожито і, отже, його вартість не становить
ніякого елементу вартости товару. Саме в ціні, по якій купець
купує товарний капітал, він заміщає в грошах його ціну виробництва
\deq{} $Г$. Його продажна ціна, як це викладено раніше, \deq{} $Г \dplus{} ΔГ$, при чому $ΔГ$ виражає надбавку до ціни
товару,
визначувану загальною нормою зиску. Отже, якщо він продає
товар, то до нього повертається, крім ΔГ, первісний грошовий
капітал, авансований ним на купівлю товарів. Тут знов таки
виявляється, що його грошовий капітал взагалі є не що інше,
як перетворений у грошовий капітал товарний капітал промислового
капіталіста, який так само мало може впливати на величину
вартості цього товарного капіталу, як коли б цей останній
був проданий не купцеві, а безпосередньо останньому споживачеві.
Фактично він тільки антиципує оплату товару цим
останнім. Однак, це правильно тільки в тому випадку, коли, як
ми це досі припускали, купець не робить ніяких додаткових
витрат, або коли йому, крім грошового капіталу, який він мусить
авансувати на купівлю товару у виробника, не доводиться
в процесі метаморфози товарів, купівлі й продажу, авансувати
ніякого іншого капіталу, обігового чи основного. Однак, як ми
це бачили при розгляді витрат циркуляції (книга II, розд. VI),
це не так. І ці витрати циркуляції виступають почасти як витрати,
які купець може покласти на інших агентів циркуляції,
почасти як витрати, які безпосередньо зв’язані з його специфічним
підприємством.

\parcont{}  %% абзац починається на попередній сторінці
\index{ii}{0282}  %% посилання на сторінку оригінального видання
отже, частини вартости, які теж існують, як аліквотні частини цієї всієї
маси засобів продукції, становлять, правда, разом з тим доходи для
всіх \emph{аґентів, що беруть участь у цій продукції}: заробітну
плату робітників, зиск і ренту капіталістів. Але для \emph{суспільства}
вони становлять не дохід, а \emph{капітал}, хоч річний продукт суспільства
складається лише із суми продуктів індивідуальних капіталістів, що належать
до цього суспільства. Більшість цих продуктів уже з самої природи
своєї може функціонувати лише як засоби продукції, і навіть ті з них,
що в разі потреби могли б функціонувати як засоби споживання, призначені
служити як сировинний або допоміжний матеріял для нової продукції.
Вони функціонують як такий — отже, як капітал — але не в руках
їхніх продуцентів, а в руках тих, хто їх застосовує, а саме:

III.~Капіталістів другого відділу, безпосередніх продуцентів \emph{засобів
споживання}. Ними заміщується капітал, зужиткований на продукцію
засобів споживання (оскільки цей капітал не перетворюється на робочу
силу, тобто оскільки він не становить суми заробітних плат робітників
цього другого відділу), тимчасом як цей зужиткований капітал, що тепер
у формі засобів споживання перебуває в руках капіталістів, які продукують
засоби споживання, і собі, — отже, з суспільного погляду — також
\emph{становить споживний фонд, що в ньому капіталісти
й робітники першого відділу реалізують свої доходи}.

Коли б А.~Сміт продовжив свою аналізу до цього пункту, він мало
не роз\-в’я\-зав би цілої проблеми. Він майже схопив суть справи, бо він
уже помітив, що певні частини вартости одного ґатунку (засобів продукції)
товарового капіталу, з яких складається ввесь річний продукт
суспільства, становлять, правда, дохід для індивідуальних робітників і капіталістів,
зайнятих в їхній продукції, але не становлять складової частини
доходу суспільства; тимчасом як частина вартости другого ґатунку
(засобів споживання), хоч і становить капітальну вартість для індивідуальних
власників цієї частини, — для капіталістів, зайнятих у цій сфері
застосовання капіталу — але все ж становить лише частину суспільного
доходу.

Але з усього наведеного вище випливає ось що:

\emph{Поперше}: хоч суспільний капітал дорівнює лише сумі індивідуальних
капіталів, а тому й річний товаровий продукт (або товаровий капітал)
суспільства дорівнює сумі товарових продуктів цих індивідуальних капіталів;
отже, хоч розклад товарової вартости на її складові частини, який має
силу для кожного індивідуального товарового капіталу, мусить мати силу
і, кінець-кінцем, справді має силу й для капіталу цілого суспільства, все ж
та форма прояву, що в ній цей розклад товарової вартости виявляється
в сукупному суспільному процесі репродукції, є інша.

\emph{Подруге}: навіть на основі простої репродукції відбувається не лише
продукція заробітної плати (змінного капіталу) та додаткової вартости,
а й безпосередня продукція нової сталої капітальної вартости, хоч робочий
день складається лише з двох частин: з однієї, що протягом її
робітних покриває змінний капітал, дійсно продукує еквівалент витрат на
\parbreak{}  %% абзац продовжується на наступній сторінці

\parcont{}  %% абзац починається на попередній сторінці
\index{ii}{0283}  %% посилання на сторінку оригінального видання
закуп його робочої сили, та з другої частини, що протягом її він продукує
додаткову вартість (зиск, ренту й~\abbr{т. ін.}). — Саме та щоденна праця,
яку витрачається на репродукцію засобів продукції, і вартість якої
розкладається на заробітну плату й додаткову вартість, — саме ця праця
реалізується в нових засобах продукції, які заміщують сталу частину
капіталу, витрачену на продукцію засобів споживання.

Головні труднощі, що з них більшу частину уже розв’язано в попередньому
викладі, постають тоді, коли досліджують не акумуляцію, а просту
репродукцію. Тим то А.~Сміт (книга II), як і раніше Кене (Tableau
économique), виходять з простої репродукції, скоро мова йде про рух
річного продукту суспільства та його репродукцію, упосереднену циркуляцією.

\subsubsection[]{Розклад мінової вартости на \emph{v \dplus{} m} у Сміта}

Догму А.~Сміта, ніби ціна або мінова вартість (exchangeable value)
кожного поодинокого товару, — отже, і всіх товарів, сукупність яких
становить річний продукт суспільства (він слушно припускає всюди капіталістичну
продукцію), — складається з трьох складових частин (component
parts) або розкладається на (resolves itself into): заробітну плату,
зиск і ренту, можна звести на те, що товарова вартість — $v \dplus{} m$, тобто
дорівнює вартості авансованого змінного капіталу плюс додаткова вартість.
Це зведення зиску й ренти до того загального й єдиного, що
ми звемо $m$, ми можемо зробити саме з виразного дозволу А.~Сміта, як
це видно з наступних цитат, де ми спочатку не звертаємо уваги на всебічні
пункти, тобто на всі позірні або справжні відхили від догми, що за
нею товарова вартість складається виключно з елементів, які ми позначаємо
як $v \dplus{} m$.

В мануфактурі „вартість, що її робітники долучають до матеріялів,
розкладається на\dots{}\dots{} дві частини, що з них одна оплачує їхню заробітну
плату, а друга — зиск їхньому хазяїнові на ввесь капітал, авансований ним
на матеріял і на заробітну плату“. (Кн.~І, розд. 6, стор. 41). — „Хоч
мануфактуристові“ (мануфактурному робітникові) „його заробітну плату
й авансує підприємець, але в дійсності вона нічого не коштує цьому
останньому, бо звичайно вартість цієї заробітної плати, разом з зиском,
повертається (restored) в збільшеній вартості предмету, що на нього застосовано
працю „мануфактуриста“. (Кн.~II, розд. 3, стор. 221). Частина капіталу
(Stock), витрачена на „утримання продуктивної праці\dots{} після того як вона
служила йому (підприємцеві) в функції капіталу\dots{} становить їх (робітників)
дохід“. (Кн.~II, розд. 3, стор. 223).

А.~Сміт у щойно цитованому розділі виразно каже: „Ввесь річний
продукт землі та праці кожної країни\dots{} сам собою (naturally) розпадається
на дві частини. Одну з цих частин, і часто найбільшу, насамперед
призначається замістити капітал і відновити засоби існування, сировинні
матеріяли й готові продукти, взяті з капіталу. Другу частину призначається
утворити дохід, чи то для власника цього капіталу, як \emph{зиск на
його} капітал, чи то дохід для когобудь іншого, як ренту з його
\parbreak{}  %% абзац продовжується на наступній сторінці

\parcont{}  %% абзац починається на попередній сторінці
\index{iii1}{0284}  %% посилання на сторінку оригінального видання
або більше доходу, чи то в формі заробітної плати, чи в формі
певної частини зиску (провізії, тантьєми), одержуваного при
кожному продажі. В першому випадку купець одержує торговельний
зиск як самостійний капіталіст; у другому випадку прикажчикові,
найманому робітникові промислового капіталіста, виплачується
частина зиску, чи то в формі заробітної плати, чи
в формі пропорціональної участі в зиску промислового капіталіста,
безпосереднім агентом якого він є, а його принципал
у цьому випадку одержує як промисловий, так і торговельний
зиск. Але в усіх цих випадках, хоч самому агентові циркуляції
його дохід може здаватись простою заробітною платою, платою
за виконану ним працю, і хоча — там, де цей дохід не здається
таким — розмір його зиску може дорівнювати тільки заробітній
платі краще оплачуваного робітника, його дохід все ж виникає
тільки з торговельного зиску. Це випливає з того, що його
праця не є вартостетворча праця.

Здовження часу на операцію циркуляції становить собою
для промислового капіталіста: 1) втрату часу особисто для нього,
оскільки це заважає йому виконувати свою функцію керівника
самого процесу виробництва; 2) здовження часу перебування
його продукту, в грошовій або в товарній формі, в процесі циркуляції,
отже, в такому процесі, в якому він не зростає в своїй
вартості і в якому безпосередній процес виробництва переривається.
Щоб цей останній не переривався, доводиться або
скоротити виробництво, або авансувати додатковий грошовий
капітал для того, щоб постійно продовжувати процес виробництва
в тому самому масштабі. Це в кожному разі зводиться
до того, що або при попередньому капіталі одержується менший
зиск, або доводиться авансувати додатковий грошовий
капітал, щоб одержати попередній зиск. Все це ні трохи не
змінюється, коли на місце промислового капіталіста стає купець.
Замість того, щоб першому витрачати більше часу на процес
циркуляції, його витрачає купець; замість того, щоб промисловий
капіталіст авансовував додатковий капітал для циркуляції,
його авансує купець; або — що є те саме — замість того,
щоб більша частина промислового капіталу постійно оберталася
в процесі циркуляції, в ньому цілком замикається капітал
купця; і замість того, щоб промисловий капіталіст одержував
менший зиск, він мусить частину свого зиску цілком відступати
купцеві. Оскільки купецький капітал не перевищує необхідних
меж, ріжниця полягає тільки в тому, що в наслідок
цього поділу функції капіталу вживається менше часу виключно
на процес циркуляції, авансується на це менше додаткового
капіталу, і втрата на сукупному зиску, яка виявляється в формі
торговельного зиску, є менша, ніж вона була б в протилежному
випадку. Якщо в наведеному вище прикладі $720c + 180v + 180m$,
при існуванні поруч з ним купецького капіталу в 100, дає промисловому
капіталістові зиск в 162, або 18\%, отже, спричинює
\parbreak{}  %% абзац продовжується на наступній сторінці

\parcont{}  %% абзац починається на попередній сторінці
\index{iii1}{0285}  %% посилання на сторінку оригінального видання
зменшення зиску на 18, то без такого усамостійнення купецького
капіталу необхідний додатковий капітал становив би, може, 200,
і тоді вся авансована промисловим капіталістом сума була б 1100
замість 900, отже, при додатковій вартості в 180 норма зиску
була б тільки 16\sfrac{4}{11}\%.

Якщо промисловий капіталіст, який разом з тим є своїм
власним купцем, крім додаткового капіталу, на який він купує
новий товар, раніше ніж його продукт, що перебуває в циркуляції,
зворотно перетвориться в гроші, авансував ще, крім того,
капітал (витрати на контору і заробітна плата торговельним
робітникам) для реалізації вартості свого товарного капіталу,
отже, на процес циркуляції, то хоч ці витрати становлять
додатковий капітал, але вони не утворюють додаткової вартості.
Вони мусять бути заміщені з вартості товарів, бо частина
вартості цих товарів мусить знову перетворитися в ці витрати
циркуляції; але цим не утворюється ніякої добавної додаткової
вартості. Щодо сукупного капіталу суспільства це фактично
зводиться до того, що частина його потрібна для другорядних
операцій, які не входять у процес зростання вартості, і що ця
частина суспільного капіталу постійно мусить репродуковуватись
для цих цілей. В наслідок цього зменшується норма зиску
для окремих капіталістів і для всього класу промислових капіталістів,
— результат, який виходить при всякому долученні
додаткового капіталу, оскільки це потрібно для того, щоб привести
в рух ту саму масу змінного капіталу.

Оскільки ці, зв’язані з самою справою циркуляції, додаткові
витрати переймає на себе від промислового капіталіста торговельний
капіталіст, теж відбувається це зменшення норми зиску,
тільки в меншій мірі і іншим шляхом. Справа тепер стоїть так,
що купець авансує більше капіталу, ніж це було б потрібно,
коли б цих витрат не існувало, і що зиск на цей додатковий
капітал підвищує суму торговельного зиску, отже, купецький
капітал в більшому розмірі входить разом з промисловим капіталом
у вирівнення пересічної норми зиску, — тобто пересічний
зиск знижується. Якщо в нашому наведеному вище прикладі крім
100 купецького капіталу авансується ще 50 додаткового капіталу
на ті витрати, про які йде мова, то сукупна додаткова
вартість в 180 тепер розподіляється на продуктивний капітал
в 900 плюс купецький капітал в 150, разом \deq{} 1050. Отже, пересічна
норма зиску знижується до 17\sfrac{1}{7}\%. Промисловий капіталіст
продає купцеві товари за 900 \dplus{} 154\sfrac{2}{7} \deq{} 1054\sfrac{2}{7}, а купець
продає їх за 1130 (1080 \dplus{} 50 за ті витрати, які він мусить знову
замістити). Зрештою, слід визнати, що з розподілом на купецький
і промисловий капітал зв’язана централізація торговельних
витрат і через це скорочення їх.

\looseness=-1
Тепер постає питання: як стоїть справа з торговельними
найманими робітниками, що їх уживає торговельний капіталіст,
в даному випадку торговець товарами?

\parcont{}  %% абзац починається на попередній сторінці
\index{ii}{0286}  %% посилання на сторінку оригінального видання
величини цілком не залежить від її розподілу між трьома групами осіб.
Може здаватися, що потрібна четверта частина, щоб замістити капітал
фармера або замістити зношування його робочої худоби та інших
його хліборобських знарядь. Але треба взяти на увагу, що ціна якогобудь
хліборобського знаряддя, напр., ціна робочого коня, теж складається
з вищезгаданих трьох частин: ренти на землю, де його вирощено,
праці догляду за конем і зиску фармера, що авансує й ренту з цієї
землі й плату за цю працю. Тому, хоч ціна зерна й може покрити так
ціну, як і витрати на утримання коня, все ж ціла ціна безпосередньо
або кінець-кінцем розкладається на ті таки три частини: земельну ренту,
працю (він має на думці заробітну плату) і зиск“. (Кн.~І, розд. 6.,
стор. 42).

Оце буквально все, що подає А.~Сміт, обґрунтовуючи свою дивовижну
доктрину. Його доказ сходить просто на повторення того самого
твердження. Напр., він допускає, що ціна зерна складається не лише з
$v \dplus{} m$, але, крім того, і з ціни засобів продукції, зужиткованих на продукцію
зерна, отже, з капітальної вартости, що її фармер витратив не
на робочу силу. Однак — каже він — ціни всіх цих засобів продукції й
собі розкладаються, як і ціна зерна, на $v \dplus{} m$; А.~Сміт забуває тільки
додати: і крім того, на ціну засобів продукції, зужиткованих на їхню
власну продукцію. Від однієї галузі продукції він відсилає до другої, а
від другої знову відсилає до третьої. Твердження, що вся ціна товарів
„безпосередньо“ або „кінець-кінцем“ (ultimately) розкладається на $v \dplus{} m$,
лише тоді не було б марною викруткою, коли б довести, що товарові
продукти, ціна котрих безпосередньо розкладається на $с$ (ціна зужиткованих
засобів продукції) \dplus{} $v \dplus{} m$ кінець-кінцем, компенсується товаровими
продуктами, які заміщують ці „зужитковані засоби продукції“ в
цілому їх обсязі, і які з свого боку виробляється, протилежно до перших
товарових продуктів, через витрату лише змінного капіталу, тобто
капіталу, витрачуваного на робочу силу. В такому разі ціна останніх
товарових продуктів безпосередньо була б \deq{} $v \dplus{} m$. Тому й ціну перших
товарових продуктів, $c \dplus{} v \dplus{} m$, де $с$ фігурує як стала частина
капіталу, кінець-кінцем, можна було б розкласти на $v \dplus{} m$. А.~Сміт сам
не гадав, що він дав такий доказ, подаючи приклад з збирачами scotch
pebbles, які, проте, за його словами, 1) не дають жодної додаткової вартости,
а продукують лише власну заробітну плату; 2) не вживають жодних
засобів продукції (а все ж і вони мають засоби продукції, як от
кошики, мішки та інші вмістища, щоб забирати камінці).

Ми вже раніше бачили, що А.~Сміт далі сам розбиває свою власну
теорію, не усвідомлюючи однак своїх суперечностей. Однак, джерела їх
треба шукати саме в його наукових вихідних пунктах. Капітал, перетворений
на працю, продукує більшу вартість, ніж його власна вартість.
Яким чином? В наслідок того, каже А.~Сміт, що робітники підчас процесу
продукці втілюють в оброблювані ними речі таку вартість, яка,
крім еквіваленту за їхню власну купівельну ціну, утворює додаткову вартість
(зиск і ренту), що дістається не їм, а тим, хто застосовує їхню працю.
\parbreak{}  %% абзац продовжується на наступній сторінці

\parcont{}  %% абзац починається на попередній сторінці
\index{ii}{0287}  %% посилання на сторінку оригінального видання
Але це й усе, що вони дають і можуть дати. Те, що має силу для одноденної промислової праці, має
силу й для тієї праці, що її вся кляса капіталістів пускає в рух протягом року. Тому, сукупну масу
новоспродукованої суспільної річної вартости можна розкласти лише на $v \dplus{} m$, на еквівалент, що ним
робітники покривають капітальну вартість, витрачену на їхню власну купівельну ціну, і на додаткову
вартість, що її вони поверх цього мусять дати тому, хто застосовує їхню працю. Але ці обидва
елементи вартости товарів становлять разом із тим джерела доходу різних кляс, що беруть участь у
репродукції: перший — заробітну плату, дохід робітників; другий — додаткову вартість, що з неї
промисловий капіталіст залишає собі одну частину в формі зиску, а другу віддає як ренту, як дохід
землевласника. Отже, звідки могла б постати ще одна складова частина вартости, коли новоспродукована
річна вартість не має жодних інших елементів, крім $v \dplus{} m$? Ми стоїмо тут на ґрунті простої
репродукції. Коли вся річна сума праці розкладається на працю, потрібну для репродукції капітальної
вартости, витраченої на робочу силу, і на працю, потрібну для утворення додаткової вартости, то
відки ще взагалі могла б постати праця для продукції капітальної вартости, витраченої не на робочу
силу?

Справа стоїть ось як:

1) А.~Сміс визначає вартість товару тією масою праці, що її найманий робітник додає (adds) до
предмету праці. Він каже буквально „до матеріялів“, бо в нього мова мовиться про мануфактуру, яка
вже сама переробляє продукти праці; але це совсім не змінює справи. Вартість, що її робітник додає
(і це „adds“ є вислів Адама) до предмету, зовсім
не залежить від того, чи мав уже до цього долучення самий предмет, що до нього долучається вартість,
власну вартість, чи ні. Отже, робітник утворює нову вартість у товаровій формі; за А.~Смісом частина
цієї вартости є еквівалент заробітної плати робітника, і цю частину, отже, визначається розміром
вартости його заробітної плати; щоб випродукувати або репродукувати вартість, рівну його заробітній
платі, йому
доводиться додавати більшу або меншу кількість праці, залежно від того, оскільки велика або мала
його заробітна плата. Але, з другого боку, робітник, понад визначувану таким чином межу, додає
дальшу працю, що утворює додаткову вартість капіталістові, що застосовує його. Чи лишається ця
додаткова вартість цілком у руках капіталіста, чи доводиться йому частину її віддати третім особам,
це зовсім нічого не змінює ні в якісному (що це взагалі є додаткова вартість), ні в кількісному
(щодо величини) визначенні додаткової вартости, долученої найманим робітником. Це — вартість, як і
всяка інша частина вартости продукту, але вона відрізняється тим, що робітник не одержав за неї
жодного еквіваленту й потім не одержить його; навпаки, цю вартість капіталіст привласнює без
еквіваленту. Ціла вартість товару визначається кількістю
праці, витраченої робітником на його продукцію; частину цієї цілої вартости визначено тим, що вона
дорівнює вартості заробітної плати, отже, є її еквівалент. Тому другу частину, додаткову вартість,
неодмінно теж
\parbreak{}  %% абзац продовжується на наступній сторінці

\parcont{}  %% абзац починається на попередній сторінці
\index{iii1}{0288}  %% посилання на сторінку оригінального видання
В міру того, як відбувалася б централізація капіталу в сфері
виробництва, відбувалася б його децентралізація у сфері циркуляції.
В наслідок цього чисто купецькі операції промислового
капіталіста, а разом з тим і його чисто купецькі видатки, безмежно
збільшилися б, бо йому б доводилось мати справу, скажемо,
з 1000 купців замість 100. В наслідок цього більша частина
вигоди від усамостійнення купецького капіталу втратилася б;
крім чисто купецьких витрат, зростали б також інші витрати
циркуляції, витрати сортування, відправки і~\abbr{т. д.} Це щодо промислового
капіталу. Розгляньмо тепер купецький капітал. Поперше,
щодо чисто купецьких робіт. В рахівництві обчисляти
більші числа не коштує більше часу, ніж обчисляти малі. Зробити
10 покупок по 100 фунтів стерлінгів коштує вдесятеро більше
часу, ніж зробити \emph{одну} покупку в 1000 фунтів стерлінгів. Кореспонденція,
папір, поштові витрати коштують вдесятеро більше,
коли мати справу с 10 дрібними купцями, ніж коли вести кореспонденцію
з \emph{одним} великим купцем. Обмежений поділ праці в комерційній
майстерні, де один веде книги, другий касу, третій
кореспонденцію, один купує, другий продає, третій роз’їжджає
і~\abbr{т. д.}, заощаджує робочий час у величезних розмірах, так що
число торговельних робітників, вживаних у гуртовій торгівлі,
не стоїть ні в якій відповідності до відносної величини підприємства.
Це тому, що в торгівлі далеко більше, ніж у промисловості,
та сама функція коштує однакової кількості робочого часу
незалежно від того, чи виконується вона у великому чи в малому
масштабі. Тому й концентрація в торговельному підприємстві
історично виявляється раніше, ніж у промисловій майстерні.
Далі, видатки на сталий капітал. 100 дрібних контор коштують
незрівнянно більше, ніж одна велика, 100 дрібних товарних складів
— безмежно більше, ніж один великий, і~\abbr{т. д.} Транспортні витрати,
які входять у купецьке підприємство, принаймні як витрати,
які доводиться авансувати, зростають разом із роздрібненням.

Промисловий капіталіст мусив би витрачати більше праці
і видатків циркуляції в торговельній частині свого підприємства.
Той самий купецький капітал, розподілений між багатьма дрібними
купцями, вимагав би в наслідок такого роздрібнення далеко
більше робітників для опосереднення своїх функцій і, крім
того, потрібний був би більший купецький капітал для того,
щоб обертати той самий товарний капітал.

Якщо ми весь купецький капітал, витрачуваний безпосередньо
в купівлі та продажу товарів, позначимо через $В$, а відповідний
змінний капітал, витрачуваний на оплату допоміжних торговельних
робітників, через $b$, то $B + b$ було б менше, ніж мусив би
бути весь купецький капітал В, коли б кожен купець обходився
без помічників, тобто коли б жодна частина не витрачалась на $b$.
Проте, ми все ще не розв’язали труднощів.

Продажна ціна товарів мусить бути достатньою: 1) для того,
щоб оплатити пересічний зиск на $В + b$. Це пояснюється вже
\parbreak{}  %% абзац продовжується на наступній сторінці

\parcont{}  %% абзац починається на попередній сторінці
\index{ii}{0289}  %% посилання на сторінку оригінального видання
двоїстого характеру самої праці: праці, оскільки вона як витрата робочої сили утворює вартість, і
оскільки вона як конкретна корисна праця утворює предмети споживання (споживну вартість). Загальна
сума виготовлених протягом року товарів, тобто \emph{ввесь річний продукт}, є продукт \emph{корисної} праці, яка
діяла протягом останнього року; всі ці
товари існують лише в наслідок того, що суспільно застосовану працю витрачено в різноманітно
розгалуженій системі різних видів корисної праці; тільки тому в їхній сукупній вартості вартість
засобів продукції, зужиткована на їх продукцію, збереглася, знову з’явившись в новій натуральній
формі. Отже, ввесь \emph{річний продукт є результат корисної} праці, витраченої протягом року. Але протягом
року утворюється знову лише деяка частина \emph{вартости} річного \emph{продукту}; ця частина є \emph{новоспродукована}
річна \emph{вартість}, що в ній втілено суму праці, пущеної в рух протягом даного року.

Отже, коли А.~Сміт у щойно цитованому місці каже: „Річна праця кожної нації є той фонд, що первісно
дає їй усі засоби існування, які вона споживає протягом року і~\abbr{т. ін.}“, то він однобічно стає на
погляд просто корисної праці, яка, щоправда, надала всім цим засобам існування форму придатну для
споживання. Але він забуває при цьому, що це
було б неможливо без участи засобів праці й предметів праці, переданих від минулих років, і що в
наслідок цього „річна праця“, оскільки вона утворювала вартість, ні в якому разі не утворила всієї
вартости виготовленого нею продукту; він забуває, що новопродукована вартість менша, ніж вартість
продукту.

Коли А.~Смітові й не можна закинути, що в цій аналізі він ішов лише до тих меж, як і всі його
наслідувачі (хоч спробу правильно розв’язати питання маємо вже в фізіократів), то все ж треба
сказати, що далі він губиться в хаосі, головним чином, тому, що „езотеричне“ розуміння товарової
вартости в нього взагалі завжди переплітається з екзотеричним, а це останнє в нього здебільша й
переважає, тимчасом як його науковий інстинкт час від часу знову й знов приводить його до
езотеричного погляду.

\subsubsection{Капітал і дохід у А.~Сміта}

Частина вартости кожного товару (а тому й річного продукту), яка становить лише еквівалент
заробітної плати, дорівнює капіталові, авансованому капіталістом на заробітну плату, тобто дорівнює
змінній складовій частині цілого авансованого ним капіталу. Цю складову частину авансованої
капітальної вартости капіталіст одержує назад через новоспродуковану складову частину вартости
товару, виробленого найманими робітниками. Хоч авансується змінний капітал в тому розумінні, що
капіталіст виплачує грішми ту частину ще неготового для продажу продукту, яка припадає робітникові,
або хоч готового, але ще не проданого капіталістом; хоч платить він робітникові грішми, вже
одержаними від продажу виробленого робітниками товару, хоч він за допомогою кредиту антиципував
\index{ii}{0290}  %% посилання на сторінку оригінального видання
ці гроші, — в усіх цих випадках капіталіст витрачає змінний капітал, що допливає робітникам
у вигляді грошей, і володіє, з другого боку, еквівалентом цієї капітальної вартости в вигляді
частини вартости своїх товарів, що в ній робітник знову спродукував ту частину цілої вартости, яка
припадає йому самому, інакше кажучи, ту, що в ній він спродукував вартість своєї власної заробітної
плати. Замість дати робітникові цю частину вартости в натуральній формі його власного продукту,
капіталіст виплачує йому її грішми. Отже, для капіталіста змінна складова частина
авансованої ним капітальної вартости існує тепер у товаровій формі, тимчасом як робітник одержав
еквівалент за продану ним робочу силу в грошовій формі.

Отже, в той час як частина авансованого капіталістом капіталу, перетворена закупом робочої сили на
змінний капітал, функціонує в самому процесі продукції як діюща робоча сила, в той час як
витрачанням цієї сили частину цю знову продукується, тобто репродукується в товаровій формі як нову
вартість, — отже, відбувається репродукція, тобто нова продукція авансованої капітальної вартости! —
робітник витрачає вартість,
зглядно ціну, своєї проданої робочої сили на засоби існування, на засоби репродукції своєї робочої
сили. Сума грошей, рівна змінному капіталові, становить його дохід, отже, дохід, що триває лише
доти, доки він може продавати свою робочу силу капіталістам.

Товар найманого робітника, — сама його робоча сила — функціонує як товар лише остільки, оскільки її
долучається до капіталу капіталіста, оскільки вона функціонує як капітал; з другого боку, капітал
капіталіста, витрачений як грошовий капітал на закуп робочої сили, функціонує як дохід в руках
продавця робочої сили в руках найманого робітника.

Тут переплітаються різні процеси циркуляції та продукції, що їх А.~Сміт не розмежовує.

Поперше. Акти, що належать до процесу \emph{циркуляції}: робітник продає свій товар — робочу силу —
капіталістам; гроші, що на них капіталіст купує її, є для нього гроші, вкладені для збільшення їх
вартости, отже, грошовий капітал; капітал цей не витрачено, а лише авансовано. (В цьому справжнє
значення „авансування“ — avance фізіократів — цілком незалежно від того, відки сам капіталіст бере
гроші. Для капіталіста буде авансованою кожна вартість, що її він сплачує для процесу продукції,
незалежно від того, чи буде це до чи post festum; її авансовано самому процесові продукції). Тут
відбувається лише те, що при всякому продажу товарів: продавець віддає споживну вартість (в даному
разі робочу силу) і одержує її вартість (реалізує її ціну) в грошах; покупець віддає свої гроші й
одержує натомість самий товар — в даному разі робочу силу.

Подруге. В цроцесі продукції куплена робоча сила являє тепер частину діющого капіталу, а сам
робітник функціонує тут лише як особлива натуральна форма цього капіталу, відмінна від тих його
елементів, що існують у натуральній формі засобів продукції. Протягом процесу продукції робітник до
засобів продукції, що їх він перетворює на продукт, долучає витратою своєї робочої сили вартість,
рівну вартості його робочої
\index{ii}{0291}  %% посилання на сторінку оригінального видання
сили (лишаючи осторонь додаткову вартість); отже, він репродукує капіталістові в товаровій
формі ту частину капіталу, що її капіталіст йому авансував або має авансувати як заробітну плату;
продукує йому еквівалент цієї плати; отже, він продукує капіталістові капітал, що його капіталіст
може знову „авансувати“ на закуп робочої сили.

Потрете. При продажу товару частина його продажної ціни повертає капіталістові авансований ним
змінний капітал і цим дає йому змогу знову купувати робочу силу, а робітникові — знову продавати її.

При всіх актах купівлі й продажу товарів — оскільки розглядається лише ці оборудки — цілком байдуже,
що зробить продавець з уторгованими за свій товар грішми, і що зробить покупець з купленими
предметами споживання. Отже, оскільки розглядається лише процес циркуляції, цілком не має значення
також та обставина, що куплена капіталістом робоча сила репродукує йому капітальну вартість, і що, з
другого боку,
гроші, вторговані як купівельна ціна робочої сили, становлять дохід робітника. На величину вартости
предмета торговлі робітника, його робочої сили, не впливає ані те, що вона становить його „дохід“,
ані те, що споживання цього його предмета торговлі покупцем репродукує цьому покупцеві капітальну
вартість.

Через те, що вартість робочої сили, — тобто адекватна продажна ціна цього товару — визначається
кількістю праці, потрібного для її репродукції, а саму цю кількість праці визначається тут тією
кількістю праці, яка потрібна для продукції потрібних засобів існування робітника, отже, кількістю
праці, потрібного для підтримання його життя, заробітна плата стає доходом, що з нього має жити
робітник.

Цілком неправильне твердження А.~Сміта (стор. 223): „\emph{Частина капіталу}, витрачена на утримання
продуктивної праці, після того як вона служила йому [капіталістові] в функції капіталу\dots{} становить
їх (робітників) дохід“. Гроші, що ними капіталіст оплачує куплену ним робочу силу „служать йому в
функції капіталу“, оскільки він за допомогою їх долучав робочу силу до речових складових частин
свого капіталу і тільки цим взагалі ставить свій капітал в умови, що в них він може функціонувати як
продуктивний капітал. Треба відрізняти таке: робоча сила в руках робітника є товар, а не капітал;
вона становить для нього дохід остільки, оскільки він може постійно повторювати її продаж; вона
функціонує як капітал після продажу в руках капіталіста, підчас самого процесу продукції. Робоча
сила служить тут подвійно; в руках робітника, як товар, продаваний за його вартістю; в руках
капіталіста, що купив її, як сила, що продукує вартість і споживну вартість. Але гроші, що їх
одержує робітник від капіталіста, він одержує лише після того, як він дав йому вжиток своєї робочої
сили, після того, як вона вже реалізована в вартості продукту праці. Капіталіст має цю вартість у
своїх руках,
перш ніж оплатить її. Отже, не гроші є те, що двічі функціонує: спочатку як грошова форма змінного
капіталу, а потім як заробітна плата. Двічі функціонує робоча сила: поперше, як \emph{товар} при продажу
робочої сили (при визначенні розміру заробітної плати, що її треба
\parbreak{}  %% абзац продовжується на наступній сторінці

\parcont{}  %% абзац починається на попередній сторінці
\index{ii}{0292}  %% посилання на сторінку оригінального видання
виплатити, гроші відіграють ролю лише ідеальної міри вартости, і при
цьому ще зовсім не потрібно, щоб вони були в руках капіталіста); подруге,
в процесі продукції, де робоча сила функціонує в руках капіталіста
як капітал, тобто як елемент, що утворює споживну вартість і вартість.
Вона вже дала в товаровій формі той еквівалент, що його треба виплатити
робітникові, дала еквівалент цей раніше, ніж капіталіст виплатить
його в грошовій формі робітникові. Отже, робітник сам утворює виплатний
фонд, що з нього капіталіст оплачує його. Та це ще не все.

Робітник витрачає одержувані гроші на утримання своєї робочої
сили, отже, — коли розглядати клясу капіталістів і клясу робітників у
їхній сукупності, — робітник витрачає ці гроші, щоб зберегти капіталістові
те знаряддя, що за допомогою його лише й може він лишатись
капіталістом.

Отже, постійна купівля й продаж робочої сили увічнює, з одного
боку, робочу силу як елемент капіталу; в наслідок цього капітал з’являється
як творець товарів, предметів споживання, що мають вартість;
далі, в наслідок цього ж ту частину капіталу, яка купує робочу силу, постійно
відновлюється продуктом цієї робочої сили, і значить, сам робітник постійно
утворює той фонд капіталу, що з нього йому платять. З другого
боку, постійний продаж робочої сили стає повсякчас поновлюваним джерелом
засобів існування робітника й таким чином його робоча сила
з’являється як здатність, що через неї він одержує дохід, з якого він
живе. Дохід тут значить не що інше, як зумовлюване постійно повторюваним
продажем товару (робочої сили) привласнення вартостей, при
чому самі ці вартості служать лише для постійної репродукції продаваного
товару. І остільки має А.~Сміс рацію казати, що джерелом доходу
робітника стає та частина вартости утвореного самим робітником продукту, за
яку капіталіст дає йому еквівалент у формі заробітної плати. Але це так
само нічого не змінює в природі або величині цієї частини вартости товару,
як нічого не змінює у вартості засобів продукції та обставина, що
вони функціонують як капітальні вартості, або так само, як нічого не
змінюється в природі й величині прямої лінії від того, чи буде вона правити
за основу трикутника чи за діяметр еліпси. Вартість робочої сили,
як і перше, визначається також незалежно від цієї обставини, як і вартість
засобів продукції. Ця частина вартости товару ані складається з доходу,
як одного з її складових самостійних чинників, ані розкладається на
дохід. Хоч ця нова вартість, постійно репродуковувана робітником, і становить
для нього джерело доходу, однак, навпаки, його дохід не становить
складової частини продукованої ним нової вартости. Величина виплачуваної
йому частини, утворюваної ним нової вартости визначає розмір
вартости його доходу, а не навпаки. Та обставина, що ця частина нової
вартости становить для нього дохід, свідчить лише про те, що з нею робиться,
про спосіб вжитку її, але це так само не має жодного чинення до створення
її, як і до створення всякої іншої вартости. Коли я щотижня
одержую десять талерів, то самий факт цього щотижневого одержання
нічого не змінює ні в природі вартости десятьох талерів, ні в величині
\index{ii}{0293}  %% посилання на сторінку оригінального видання
їхньої вартости. Як вартість кожного іншого, товару, вартість робочої
сили визначається кількістю праці, доконечної для її репродукції; те, що
ця кількість праці визначається вартістю доконечних засобів існування
робітника, отже, дорівнює праці, доконечній для репродукції засобів його
власного існування, є характеристичне для цього товару (робочої сили);
але не характеристичніше за те, що вартість в’ючної худоби визначається
вартістю засобів існування, доконечних для її утримання, отже,
масою людської праці, потрібної для того, щоб випродукувати ці засоби
існування.

Саме ця категорія „доходу“ і спричиняє тут усе лихо в А.~Сміса.
Різні відміни доходів становлять у нього „component parts“, складові
частини щорічно продукованої, новоутворюваної товарової вартости, тимчасом
як, навпаки, ті дві частини, що на них розкладається ця товарова
вартість для капіталіста — еквівалент його змінного капіталу, авансованого
в грошовій формі підчас закупу праці, і друга частина вартости, яка також
належить йому, хоч нічого йому й не коштувала, додаткова вартість —
становлять джерела доходів. Еквівалент змінного капіталу знов авансується
на робочу силу і остільки становить дохід для робітника у формі його
заробітної плати. Друга частина — додаткова вартість — не має заміщувати
капіталістові жодного авансування капіталу, і тому він може витратити її
на засоби споживання (доконечні, а також речі розкошів), може спожити її
як дохід, замість перетворювати її на капітальну вартість будь-якого роду.
Передумова цього доходу є сама товарова вартість, і її складові частини
лише остільки відрізняються для капіталіста, оскільки вони являють або
еквівалент за авансовану ним змінну капітальну вартість, або надлишок
над авансованою ним змінною капітальною вартістю. Обидві частини складаються
не з чого іншого, як з робочої сили, витраченої підчас продукції
товару, пущеної в рух у процесі праці. Вони складаються з витрати,
не з надходження або доходу, а з витрати праці.

Після цього qui pro quo, де дохід стає джерелом товарової вартости
замість товаровій вартості бути джерелом доходу, товарова
вартість виступає тепер „складеною“ з різних відмін доходів. Їх
визначається незалежно одну від однієї, і всю вартість товару визначається
доданням величин вартости цих доходів. Але запитаймо тепер,
як визначається вартість кожного з цих доходів, що з них має постати
товарова вартість? Щодо заробітної плати, то її можна визначити, бо вона
є вартість відповідного товару, робочої сили, а цю останню визначається
(як і вартість всякого іншого товару) працею, потрібною на репродукцію
цього товару. Але як визначається додаткову вартість або радше, за А.~Смісом, дві її форми — зиск і земельну ренту? Тут усе сходить на порожню
балаканину. А.~Сміс то подає заробітну плату й додаткову вартість
(зглядно — заробітну плату й зиск) як складові частини, що з них складається
товарова вартість, зглядно ціна, то — і часто майже безпосередньо
по цьому — як частини, що на них „розкладається“ (resolves itself) товарова
ціна; а це значить, навпаки, що товарова вартість є наперед дана,
і що різні частини цієї даної вартости в формі різних доходів дістаються
\parbreak{}  %% абзац продовжується на наступній сторінці

\parcont{}  %% абзац починається на попередній сторінці
\index{ii}{0294}  %% посилання на сторінку оригінального видання
різним особам, які беруть участь у процесі продукції. Але це ні в якому
разі не є тотожне з складанням вартости з цих трьох „складових частин“.
Коли я визначу окремо величину трьох різних прямих ліній, а
потім із цих трьох ліній як „складових частин“ утворю четверту пряму
лінію, рівну величиною сумі трьох ліній, то це зовсім не та сама процедура,
як коли б я, маючи перед собою дану пряму лінію, з тим або
іншим наміром почав ділити, певним способом „розкладати“ її на три
різні частини. В першому разі величина лінії цілком змінюється з зміною
величини трьох ліній, що суму їх вона становить; в останньому разі
величину трьох частин лінії з самого початку обмежено тим, що вони
є частини лінії даної величини.

Але в дійсності, оскільки ми триматимемось того, що є правильного
у викладі А.~Сміта, а саме, що \so{новоутворена річною працею} вартість,
яка є в річному товаровому продукті суспільства (як у кожному поодинокому
товарі, або в денному, тижневому тощо продукті), дорівнює вартості
авансованого змінного капіталу (отже, частині вартости знову призначеної
на закуп робочої сили) плюс додаткова вартість, що її капіталіст може
реалізувати — за простої репродукції та за інших незмінних умов — в засобах
свого особистого споживання, і коли ми пригадаємо далі, що А.~Сміт звалює до однієї купи працю, оскільки вона утворює вартість, є
витрата робочої сили, і працю, оскільки вона утворює споживну вартість,
тобто витрачається в корисній, доцільній формі, — то все уявлення
А.~Сміта зійде ось на що: вартість кожного товару є продукт праці,
отже, і вартість продукту річної праці або вартість річного суспільного
товарового продукту. Але що кожна праця розкладається на: 1) доконечний
робочий час, що протягом його робітник лише репродукує еквівалент
капіталу, авансованого на закуп його робочої сили, і 2) додаткову працю,
що нею він дає капіталістові вартість, за яку той не платить жодного
еквіваленту, отже, додаткову вартість, — то всяка товарова вартість може
розкластися лише на ці дві різні складові частини і становить, кінець-кінцем —
як заробітна плата — дохід робітничої кляси, а як додаткова вартість —
дохід кляси капіталістів. Щождо сталої капітальної вартости, тобто вартости
засобів продукції, зужиткованих у продукції річного продукту, то
хоч і не можна сказати (крім фрази, що капіталіст прираховує її покупцеві,
продаючи свій товар), яким чином ця вартість входить у вартість
нового продукту, однак, кінець-кінцем — ultimately — в наслідок того, що
сами засоби продукції є продукт праці, сама ця частина вартости знову
таки може складатись лише з еквіваленту змінного капіталу та додаткової
вартости: з продукту доконечної праці та додаткової праці. Коли
вартості цих засобів продукції в руках тих, хто застосовує їх, функціонують
як капітальні вартості, то все ж це не заважає тому, що „первісно“,
і якщо дошукатись самої суті їхньої, в інших руках — хоча б і раніше — їх
можна було розкласти на ті самі дві частини вартости, отже, на два
різні джерела доходу.

Правильне у всьому цьому ось що: в русі суспільного капіталу, —
тобто сукупности індивідуальних капіталів — справа виступає інакше, ніж
\parbreak{}  %% абзац продовжується на наступній сторінці

\parcont{}  %% абзац починається на попередній сторінці
\index{ii}{0295}  %% посилання на сторінку оригінального видання
для кожного індивідуального капіталу, розглядуваного окремо, отже, ніж
вона виступає з погляду кожного поодинокого капіталіста. Для останнього
товарова вартість розкладається на 1) сталий елемент (четвертий,
як каже Сміс) і 2) на суму заробітної плати і додаткової вартости, зглядно
заробітної плати, зиску і земельної ренти. Навпаки, з суспільного погляду,
четвертий елемент Сміса, стала капітальна вартість зникає.

\subsubsection{Резюме}

Безглузда формула, що згідно з нею три відміни доходів, заробітна
плата, зиск, рента, становлять три складові частини товарової вартости,
випливає в А.~Сміса з правдоподібнішої формули, за якою товарова
вартість resolves itself, розкладається на ці три складові частини. Однак
і це неправильно, навіть коли припустити, що товарову вартість можна
розподілити лише на еквівалент зужитої робочої сили й на утворену
нею додаткову вартість. Але й ця помилка й собі ґрунтується тут на
глибшій, правильній засаді. Капіталістична продукція ґрунтується на тому,
що продуктивний робітник продає свою власну робочу силу, як свій
товар, капіталістові, в чиїх руках вона потім функціонує просто як елемент
його продуктивного капіталу. Ця, належна до сфери циркуляції,
оборудка — продаж і купівля робочої сили — не лише є вступ до процесу
продукції, але вона й визначає implicite\footnote*{
приховано в собі. \emph{Ред.}
} його специфічний характер.
Продукція споживної вартости і навіть продукція товару (бо її можуть
провадити і не залежні продуктивні робітники) тут є лише засіб для продукції
абсолютної та відносної додаткової вартости для капіталістів. Тому, аналізуючи
процес продукції, ми бачили, як продукцію абсолютної та відносної
додаткової вартости визначає: 1) протяг щоденного процесу праці,
2) ввесь суспільний і технічний устрій капіталістичного процесу продукції.
В ньому самому здійснюється ріжниця між простим збереженням вартости
(сталої капітальної вартости), справжньою репродукцією авансованої вартости
(еквіваленту робочої сили) і продукцією додаткової вартости, тобто
вартости, що за неї капіталіст не авансував жодного еквіваленту раніше,
ані авансує його post festum.

Хоч привласнення додаткової вартости — вартости, яка являє надлишок
над еквівалентом авансованої капіталістом вартости — підготовляється
купівлею й продажем робочої сили, однак воно є акт, що відбувається
в самому процесі продукції й становить істотний елемент його.

Вступний акт, що є акт циркуляції — купівля й продаж робочої сили — і
собі ґрунтується на розподілі елементів продукції, що відбувся перед
розподілом суспільних продуктів і був передумовою його, а саме на
відокремленні робочої сили як товару робітника від засобів продукції
як власности не-робітників.

Але разом з тим це привласнення додаткової вартости або це розмежування
\index{ii}{0296}  %% посилання на сторінку оригінального видання
продукції вартости на репродукцію авансованої вартости й
продукцію нової вартости (додаткової вартости), не заміщуваної жодним
еквівалентом, нічого не змінює в субстанції самої вартости, ні в природі
продукції вартости. Субстанція вартости є й залишається не іншим чим,
як витраченою робочою силою — працею, незалежно від особливого корисного
характеру цієї праці, — а продукція вартости є не що інше, як процес
цього витрачання робочої сили. Так кріпак витрачає протягом шістьох
день свою робочу силу, працює протягом шістьох день, і для самого
факту цього витрачання робочої сили, як такого, не постає ніякої ріжниці,
коли з цих робочих днів кріпак робить, напр., три дні на себе на
своєму власному полі, а три інші дні на свого поміщика на його полі.
Його добровільна праця на себе і його примусова праця на пана, однаково
є праця; хоч розглядатимемо ми її як працю щодо утворених нею
вартостей, хоч щодо утворених нею корисних продуктів, ми не виявимо
жодної відмінности в шостиденній праці кріпака. Відмінність стосується
лише до тих різних відносин, що зумовлюють витрачання його робочої
сили протягом обох половин шеститижневого робочого часу. Цілком так
само стоїть справа з доконечною працею і додатковою працею найманого
робітника.

Продукційний процес згасає в товарі. Те, що на його виготовлення
витрачено робочу силу, видається тепер як речова властивість товару, як
його властивість мати вартість; величину цієї вартости вимірюється
величиною витраченої праці; ні на що інше товарова вартість не розкладається
й не складається ні з чого іншого. Коли я накреслив пряму
лінію певної величини, то я спочатку „спродукував“ пряму лінію (правда,
лише символічно, і це я заздалегідь знаю) за допомогою креслення, що
його роблять згідно з певними, від мене незалежними правилами (законами).
Коли я цю лінію поділю на три відтинки (а вони знову таки
можуть відповідати певному завданню), то кожен з цих трьох відтинків
лишається, як і перше, прямою лінією, а вся лінія, що її частини вони
являють, в наслідок такого поділу не може розкластися на щось відмінне
від прямої лінії, напр., на криву якогобудь роду. Так само я не можу
поділити лінію даної довжини так, щоб сума цих частин була більша,
ніж сама неподілена лінія; отже, величину неподіленої лінії теж не можна
визначити довільно обраними величинами частин лінії. Навпаки, відносні
величини цих останніх з самого початку обмежено межами лінії, що її
частини вони являють.

З цього погляду товар, виготовлений капіталістом, ані трохи не відрізняється
від товарів, виготовлених самостійним робітником, кооперацією
робітників або рабами. Однак у нашому випадку ввесь продукт праці,
як і вся вартість його, належить капіталістові. Як і всякий інший продуцент, він повинен спочатку
через продаж перетворити товар на гроші, щоб
мати змогу дальших маніпуляцій; він мусить перетворити товар на форму
загального еквівалента.

Розгляньмо товаровий продукт до його перетворення на гроші. Він
цілком належить капіталістові. З другого боку, як корисний продукт
\parbreak{}  %% абзац продовжується на наступній сторінці

\parcont{}  %% абзац починається на попередній сторінці
\index{iii1}{0297}  %% посилання на сторінку оригінального видання
попитом, і тому в цих галузях у купців і промисловців справи
йдуть дуже жваво. Криза настає, як тільки затрати купців, які
продають на віддалених ринках (або в яких запаси нагромадились
і всередині країни), починають повертатись до них так
повільно і в таких незначних кількостях, що банки настійливо
вимагають платежів, або строки оплати векселів за куплені
товари настають раніше, ніж відбувається перепродаж цих
товарів. Тоді починаються примусові продажі, продажі для
оплати боргів. І разом з цим вибухає крах, який відразу кладе
кінець позірному процвітанню.

Але зовнішній характер і ірраціональність обороту купецького
капіталу є ще більші в наслідок того, що оборот одного й того ж
купецького капіталу може одночасно або послідовно опосереднювати
обороти дуже різних продуктивних капіталів.

Оборот купецького капіталу може, однак, опосереднювати
не тільки обороти різних промислових капіталів, але й протилежні
фази метаморфози товарного капіталу. Купець купує,
наприклад, полотно у фабриканта і продає його білільникові.
Отже, тут оборот того самого купецького капіталу — в дійсності
те саме $Т — Г$, реалізація полотна — представляє дві протилежні
фази для двох різних промислових капіталів. Оскільки
купець взагалі продає для продуктивного споживання, його
$Т — Г$ завжди представляє $Г — Т$ якогось промислового капіталу,
а його $Г — Т$ завжди представляє $Т — Г$ якогось іншого промислового
капіталу.

Якщо ми, як це зроблено в цьому розділі, оминемо $К$, витрати
циркуляції, ту частину капіталу, яку купець авансує крім
суми, витрачуваної на купівлю товарів, то, звичайно, відпадає
й $ΔК$, додатковий зиск, який він одержує на цей додатковий
капітал. Отже, цей спосіб дослідження є строго логічний і математично
правильний, коли справа йде про те, щоб дізнатись,
як зиск і оборот купецького капіталу впливають на ціни.

Якби ціна виробництва 1 фунта цукру становила 1\pound{ фунт
стерлінгів}, то купець міг би на 100\pound{ фунтів стерлінгів} купити
100 фунтів цукру. Якщо він протягом року купує і продає таку
кількість і якщо річна пересічна норма зиску є 15\%, то він
на 100\pound{ фунтів стерлінгів} накине 15\pound{ фунтів стерлінгів}, а на
1\pound{ фунт стерлінгів} — ціну виробництва 1 фунта цукру — 3\shil{ шилінги.}
Отже, він продавав би 1 фунт цукру за 1\pound{ фунт стерлінгів} 3\shil{ шилінги.}
Навпаки, якби ціна виробництва 1 фунта цукру впала до
1\shil{ шилінга}, то на 100\pound{ фунтів стерлінгів} купець купив би 2000 фунтів
цукру і продавав би його по 1\shil{ шилінгу} і 1\sfrac{4}{5}\pens{ пенса} за фунт.
І в тому і в другому випадку річний зиск на капітал в 100\pound{ фунтів
стерлінгів}, вкладений у цукрову справу, — 15\pound{ фунтам стерлінгів}.
Тільки в одному випадку він мусить продати 100, а в другому
2000 фунтів. Висока чи низька є ціна виробництва, це не має
ніякого значення для норми зиску; але це має дуже велике,
вирішальне значення для того, яка е величина тієї відповідної
\parbreak{}  %% абзац продовжується на наступній сторінці

\parcont{}  %% абзац починається на попередній сторінці
\index{iii1}{0298}  %% посилання на сторінку оригінального видання
частини продажної ціни кожного фунта цукру, яка становить
торговельний зиск, тобто тієї надбавки до ціни, яку робить
купець на певну кількість товару (продукту). Якщо ціна виробництва
товару незначна, то незначна й та сума, яку купець
авансує на купівельну ціну товару, тобто на певну масу його,
а тому, при даній нормі зиску, незначна й та сума зиску, яку
він одержує на цю дану кількість дешевого товару; або, що
зводиться до того самого, він може тоді на даний капітал, наприклад,
в 100, купити велику масу цього дешевого товару,
і загальний зиск у 15, який він одержує на ці 100, розподіляється
маленькими частинами на кожну окрему штуку цієї товарної
маси. І навпаки. Це цілком і повністю залежить від більшої
чи меншої продуктивності того промислового капіталу, товарами
якого він торгує. Якщо виключити випадки, коли купець є монополіст
і разом з тим монополізує виробництво, як, наприклад,
у свій час голландсько-ост-індська компанія, то не може бути
нічого більш безглуздого, ніж ходяче уявлення, ніби від купця
залежить, чи продасть він багато товарів з невеликим зиском,
чи мало товарів з великим зиском на одиницю товару. Дві
межі існують для його продажної ціни: з одного боку, ціна виробництва
товару, яка від нього не залежить; з другого боку, пересічна
норма зиску, яка від нього так само не залежить. Єдине,
що він може вирішувати, — при чому, однак, величина капіталу,
який є в його розпорядженні, та інші обставини також грають
певну роль, — це те, чи хоче він торгувати дорогими чи дешевими
товарами. Тому поведінка купця тут цілком і повністю залежить
від ступеня розвитку капіталістичного способу виробництва,
а не від його бажання. Тільки така торговельна компанія, як стара
голландсько-ост-індська, яка мала монополію виробництва, могла
прийти до думки при цілком змінених відносинах додержувати
методу, який відповідав щонайбільше початкам капіталістичного
виробництва\footnote{
„Profit, on the general principle, is always the same, whatever be price;
keeping its place like an incumbent body on the swelling or sinking tide. As,
therefore, prices rise, a tradesman raises prices; as prices fall, a tradesman lowers
price“. [„Зиск, за загальним правилом, завжди лишається той самий, яка б не була
ціна; він зберігає своє місце подібно до пливучого тіла при припливі й відпливі.
Тому, коли ціни підвищуються, торговець підвищує ціну; коли ціни падають,
торговець знижує ціну“]. (\emph{Corbet}: „An Inquiry into the Causes etc. of the Wealth
of Individuals“. Лондон 1841, стор. 20). — Тут, як і взагалі в тексті, мова йде
тільки про звичайну торгівлю, а не про спекуляцію, дослідження якої, як і взагалі
все те, що стосується до поділу торговельного капіталу, виходить поза
межі нашого дослідження. „The profit of trade is a value added to capital which
is independent of price, the second (speculation) is founded on the variation in
the value of capital or in price itself [„Торговельний зиск є додана до капіталу
вартість, яка не залежить від ціни; друга (спекуляція) основана на зміні вартості
капіталу або самої ціни“] (там же, стор. 12).
}.

Цей популярний передсуд, — який, зрештою, як і всі хибні
уявлення про зиск і~\abbr{т. п.}, випливає із спостерігання самої тільки
торгівлі і з купецького передсуду, — зберігається досі, підтримуваний
між іншим такими обставинами:
\parbreak{}  %% абзац продовжується на наступній сторінці


\emph{Поперше}: явища конкуренції; але вони стосуються тільки до
розподілу торговельного зиску між окремими купцями, володільцями
тієї чи іншої частини сукупного купецького капіталу; якщо,
наприклад, один продає дешевше, щоб вибити з позиції своїх
противників.

\emph{Подруге}: економіст такого калібру, як професор Рошер, все
ще може уявляти собі в Лейпцігу, що зміна в продажних цінах
викликана з міркувань „розсудливості і гуманності“ і що вона
не була результатом перевороту в самому способі виробництва.

\emph{Потретє}: якщо ціни виробництва знижуються в наслідок
підвищення продуктивної сили праці і якщо через це знижуються
і продажні ціни, то попит часто зростає ще швидше, ніж
подання, а разом з ним зростають і ринкові ціни, так що продажні
ціни дають більше, ніж пересічний зиск.

\emph{Почетверте}: який-небудь купець може знизити продажну
ціну (що завжди є не що інше, як зниження звичайного зиску,
який він накидає на ціну), щоб швидше обертати більший
капітал у своєму підприємстві. Все це речі, які стосуються
тільки до конкуренції між самими купцями.

Уже в книзі І було показано, що висота чи низькість товарних
цін не визначає ні маси додаткової вартості, яку
виробляє даний капітал, ні норми додаткової вартості; хоча
відповідно до відносної кількості товару, що його виробляє
дана кількість праці, ціна одиниці товару, а разом з тим
і частина цієї ціни, яка становить додаткову вартість, буде більша
чи менша. Ціни всякої кількості товарів, оскільки вони відповідають
вартостям, визначаються загальною кількістю упредметненої
в цих товарах праці. Якщо невелика кількість праці
упредметнюється у великій кількості товару, то ціна одиниці
товару низька і вміщена в ньому додаткова вартість незначна.
Яким чином праця, втілена в товарі, розпадається на оплачену
і неоплачену працю, отже, яка частина ціни товару представляє
додаткову вартість, це не має ніякого відношення до цієї загальної
кількості праці, отже й до ціни товару. Норма ж додаткової
вартості залежить не від абсолютної величини тієї
додаткової вартості, яка міститься в ціні окремого товару,
а від її відносної величини, від її відношення до заробітної плати,
вміщеної в тому самому товарі. Тому норма може бути висока,
хоч абсолютна величина додаткової вартості в кожній одиниці
товару невелика. Ця абсолютна величина додаткової вартості
в кожній одиниці товару залежить в першу чергу від продуктивності
праці і тільки в другу чергу від поділу праці на оплачену
і неоплачену.

Для торговельної продажної ціни навіть ціна виробництва є
наперед дана зовнішня умова.

Високі торговельні ціни товарів за попередніх часів зумовлювались:
1)~високою ціною виробництва, тобто низькою продуктивністю
праці; 2)~відсутністю загальної норми зиску, при чому
\parbreak{}  %% абзац продовжується на наступній сторінці


\index{ii}{0300}  %% посилання на сторінку оригінального видання
Шторх, який теж у принципі приймає вчення А.~Сміса, вважає однак,
що застосовання цього вчення в Сея не витримує критики. „Коли
допустити, що дохід нації дорівнює її гуртовому продуктові, тобто що з
нього не треба робити якогобудь відрахування капіталу“ (це має значити
сталого капіталу), „то доведеться також допустити, що ця нація може
непродуктивно спожити всю вартість її річного продукту, не зменшивши
ні на крихту свого майбутнього доходу\dots{} Продукти, що становлять“
(сталий) „капітал нації, не можуть споживатись“ (Storck:. Considérations
sur la nature du revenu national, Paris, 1824“, p. 150).

Але Шторх забув сказати, як погодити існування цієї сталої частини
капіталу з аналізою цін, що її він узяв у Сміса, аналізою, що згідно з
нею товарова вартість містить у собі лише заробітну плату й додаткову
вартість, але не містить жодної частини сталого капіталу. Лише завдяки
Сеєві йому стає ясно, що ця аналіза ціни призводить до абсурдних
результатів, і його власне кінцеве слово про це звучить так: „неможливо
розкласти доконечну ціну на її найпростіші елементи“. („Cours d’Economie
Politique“, Petersbourg, 1815. II, p. 140).

Сісмонді, що особливо досліджував відношення між капіталом і доходом
і своє особливе розуміння цього відношення в дійсності перетворив
на differentia specifica своїх „Nouveaux Principes“, не сказав \so{жодного}
наукового слова, не додав \so{жодного} атома для висвітлення проблеми.

Бартон, Рамсай і Шербюльє роблять спроби піднестись понад Смісове
розуміння. Але це їм не вдається, бо вони з самого початку ставлять
проблему однобічно, не відмежовуючи виразно ріжниці між сталою
та змінною капітальною вартістю від ріжниці між основним капіталом та
капіталом обіговим.

Також і Джон Стюарт Мілл із звичайною повагою відтворює доктрину,
що перейшла в спадщину від А.~Сміса до його наслідувачів.

Результат: Смісова плутанина понять існує й далі до нашого часу, і
догма Смісова є ортодоксальний символ віри політичної економії.
\label{original-300-1}

\section{Проста репродукція}

\subsection{Постава питання}

\label{original-300-2}
Коли ми розглянемо\footnote{
З рукопису II.
} річне функціонування суспільного капіталу щодо
його результату, — отже, функціонування сукупного капіталу, що в ньому
індивідуальні капітали становлять лише частини, рух яких є так їхній
індивідуальний рух, як і разом з тим складова ланка руху цілого капіталу,
— тобто, коли ми розглянемо товаровий продукт, що його дає суспільство
протягом року, то мусить виявитись, як відбувається процес
репродукції суспільного капіталу, які риси відрізняють цей процес репродукції
\index{ii}{0301}  %% посилання на сторінку оригінального видання
від процесу репродукції індивідуального капіталу, і які риси
спільні їм обом. Річний продукт охоплює так ті частини суспільного
продукту, які заміщують капітал, суспільну репродукцію, як і ті частини,
що входять у фонд споживання, що їх споживають робітники й капіталісти,
отже, охоплює так продуктивне, як і особисте споживання. Воно
охоплює також і репродукцію (тобто зберігання) кляси капіталістів і
робітничої кляси, а тому й репродукцію капіталістичного характеру сукупного
процесу продукції.

Зрозуміло, що нам треба аналізувати формулу \so{циркуляції}
\[Т' — \left\{
  \begin{array}{c}
    Г — Т\dots{} П\dots{} Т'\\
    г — т
  \end{array}
\right.,
\] при чому споживання неодмінно відіграє в ній
певну ролю; бо вихідний пункт $Т' \deq{} Т \dplus{} т$, товаровий капітал, має в
собі так сталу і змінну капітальну вартість, як і додаткову вартість.
Тому його рух охоплює й особисте, й продуктивне споживання. В кругобігах
$Г — Т\dots{} П\dots{} Т' — Г'$ і $П\dots{} Т' — Г' — Т\dots{} П$ вихідний і кінцевий
пункт є рух \so{капіталу}. Правда, це включає і споживання, бо товар,
продукт, треба продати. Але коли припускається, що цього вже досягнуто,
то для руху поодинокого капіталу буде байдуже, що далі зробиться
з цим товаром. Навпаки, в русі $Т'\dots{} Т'$ умови суспільної репродукції
виявляються саме в тому, що тут треба показати, що зробиться з кожною
частиною вартости цього сукупного продукту $Т'$. Сукупний процес
репродукції тут так само включає процес споживання, упосереднюваний
циркуляцією, як і власне процес репродукції капіталу.

Маючи на увазі мету нашу, ми повинні розглянути процес репродукції
з погляду заміщення так вартости, як і речовини поодиноких складових
частин $Т'$. Тепер нам уже не досить, як то було при аналізі вартости
продукту поодинокого капіталу, \so{припустити}, що поодинокий
капіталіст, через продаж свого товарового продукту, може спочатку перетворити
складові частини свого капіталу на гроші, а потім, знову купуючи
на товаровому ринку елементи продукції, перетворити знову ці
складові частини на продуктивний капітал. Ці елементи продукції, оскільки
вони мають речовий характер, так само становлять складову частину
суспільного капіталу, як і індивідуальний готовий продукт, обмінюваний
на них і заміщуваний ними. З другого боку, рух тієї частини суспільного
товарового продукту, що її споживає робітник, витрачаючи свою
заробітну плату, і капіталіст, витрачаючи додаткову вартість, становить не
лише складову ланку руху цілого продукту, а й переплітається з рухом
індивідуальних капіталів, і тому цього процесу не можна пояснити тим,
що його просто припускають.

Питання, що безпосередньо постає перед нами, таке: як \so{капітал},
спожитий в продукції, заміщується щодо вартости своєї з річного продукту
й як процес цього заміщення переплітається із споживанням додаткової
вартости капіталістами й заробітної плати робітниками? Отже,
насамперед йдеться про репродукцію в простому маштабі. Далі припускається
не лише те, що продукти обмінюється за їхньою вартістю, а й
\parbreak{}  %% абзац продовжується на наступній сторінці

\parcont{}  %% абзац починається на попередній сторінці
\index{ii}{0302}  %% посилання на сторінку оригінального видання
те, що не відбувається жодних переворотів у вартості складових частин
продуктивного капіталу. А втім, щодо відхилення цін від вартостей, то
ця обставина не може справити будь-якого впливу на рух суспільного
капіталу. При цьому в цілому обмінювалось би ті самі маси продуктів,
що й раніше, хоч поодиноким капіталістам при цьому дістались би пайки
вартости, вже непропорціональні їхнім відповідним авансуванням і тим масам
додаткової вартости, що кожен із них випродукував. Щодо переворотів
у вартості, то коли вони мають загальний характер і розподіляються
рівномірно, вони не спричиняють жодних змін у відношеннях між
складовими частинами вартости сукупного річного продукту. Навпаки, коли
вони мають частинний характер і розподіляються нерівномірно, то являють
собою розлади, що їх, \emph{поперше}, можна зрозуміти як такі, лише розглядаючи
їх, як відхили від незмінних відношень вартости; але, \emph{подруге},
коли визначено закон, що згідно з ним одна частина річного продукту заміщує
сталий, а друга — змінний капітал, то в цьому законі нічого не
змінила б революція в вартости хоч сталого, хоч змінного капіталу. Вона
змінила б лише відносну величину тих частин вартости, що функціонують
у тій або іншій якості, бо на місце первісних вартостей виступили
б вартості іншої величини.

Поки ми розглядали продукцію вартости та вартість продукту капіталу,
як індивідуального капіталу, для нашої аналізи натуральна форма
товарового продукту була цілком байдужа, — було цілком байдуже, напр.,
чи складається він з машин, чи з хліба, чи з дзеркал. Всі ці натуральні
форми були б просто прикладом для нас, і перша-ліпша галузь продукції
однаково була б придатна, як ілюстрація. Нам доводилось мати
справу безпосередньо з самим процесом продукції, що в кожному пункті
виступав як процес індивідуального капіталу. Оскільки ми розглядали
репродукцію капіталу, нам досить було того припущення, що частина
товарового продукту, яка являє капітальну вартість, має в сфері циркуляції
змогу зворотно перетворитись на елементи її продукції, і значить,
на форму продуктивного капіталу; цілком так само, як досить було нам
того припущення, що робітник і капіталіст знаходять на ринку товари,
на які вони витрачають заробітну плату й додаткову вартість. Але цей
суто-формальний спосіб викладу вже недостатній, коли ми розглядаємо
сукупний суспільний капітал і вартість його продукту. Зворотне перетворення
однієї частини вартости продукту на капітал, перехід другої частини в
сферу особистого споживання кляси капіталістів і кляси робітників, становить
рух у межах самої вартости продукту, яка є результат сукупного
капіталу; і цей рух є не лише заміщення вартости, а й заміщення речовини,
а тому він так само зумовлюється співвідношенням складових
частин вартости суспільного продукту, як і споживною їхньою вартістю,
їхньою речовою формою.
\label{original-302-1}

\label{original-302-2}
Проста репродукція\footnote{
З рукопису VIII.
} в незмінному маштабі являє абстракцію в тому
розумінні, що, з одного боку, на базі капіталістичної продукції відсутність
\index{ii}{0303}  %% посилання на сторінку оригінального видання
будь-якої акумуляції або репродукції в поширеному маштабі є
неймовірне припущення, а з другого боку, відношення, що в них відбувається
продукція, в різні роки не лишаються абсолютно незмінні (а таке
є наше припущення). Наше припущення те, що суспільний капітал даної
вартости, як минулого року, так і цього року знов дає таку саму масу товарових вартостей і
задовольняє таку саму кількість потреб, хоча б форми товарів і змінилися в процесі репродукції. А
проте, оскільки відбувається акумуляція, проста репродукція завжди становить частину останньої,
отже, її можна розглядати окремо, вона — реальний чинник акумуляції. Вартість річного продукту може
зменшитись, хоч маса споживних вартостей лишається та сама, вартість може лишатись та сама, хоч маса
споживних вартостей меншає; маса вартости й маса репродукованих
споживних вартостей можуть одночасно меншати. Все це залежить
від того, що репродукція відбувається або при сприятливіших умовах,
ніж були раніше, або при гірших умовах, а останні можуть призвести до
неповної — недостатньої — репродукції. Однак усе це стосується лише до
кількісного боку різних елементів репродукції, а не до тієї ролі, що її
вони відіграють в цілому процесі як капітал, що його репродукується,
або як дохід, уже репродукований.

\subsection[Два підрозділи суспільної продукції]{Два підрозділи суспільної продукції\footnotemark{}}

\label{original-303}
Цілий\footnotetext{
В головному з рукопису II.~Схема з рукопису VIII.
}
продукт, отже, і вся продукція суспільства, розпадається на
два великі підрозділи:

I.~\emph{Засоби продукції}, товари, які мають таку форму, що в ній
вони мусять ввійти або принаймні можуть ввійти в продуктивне споживання.

II.~\emph{Засоби споживання}, товари, які мають таку форму, що в
ній вони входять в особисте споживання кляси капіталістів і кляси робітників.

В кожному з цих підрозділів усі різні галузі продукції, належні до
того або того підрозділу, становлять єдину велику галузь продукції,
в одному разі — продукції засобів продукції, в другому — засобів споживання.
Ввесь капітал, застосований в кожній з цих двох галузей продукції,
становить окремий великий підрозділ суспільного капіталу.

У кожному підрозділі капітал розпадається на дві складові частини:

1) \emph{Змінний капітал}. Розглядуваний щодо \emph{вартости} він дорівнює
вартості суспільної робочої сили, застосованої в цій галузі продукції,
отже, дорівнює сумі заробітної плати, сплаченої за цю робочу силу.
Розглядуваний з речового боку, він складається з самої діючої робочої
сили, тобто з живої праці, пущеної в рух цією капітальною вартістю.

2) \emph{Сталий капітал}, тобто вартість усіх засобів продукції, застосованих
для продукції в цій галузі. Вони й собі розпадаються на \emph{основний}
капітал: машини, знаряддя праці, будівлі, робочу худобу і~\abbr{т. ін.},
\parbreak{}  %% абзац продовжується на наступній сторінці

\input{ii/_0304c.tex}
\parcont{}  %% абзац починається на попередній сторінці
\index{ii}{0305}  %% посилання на сторінку оригінального видання

II.~Продукція засобів споживання:

Капітал\dots{} $2000с \dplus{} 500v$ \deq{} 2500.

Товаровий продукт\dots{} $2000с \dplus{} 500v \dplus{} 500m$ \deq{} 3000, що існує в
засобах споживання.

В підсумку ввесь річний товаровий продукт:

I.    $4000с \dplus{} 1000v \dplus{} 1000m$ \deq{} 6000 засобів продукції,

II.    $2000с \dplus{} 500v \dplus{} 500m$ \deq{} 3000 засобів споживання.

Вся вартість \deq{} 9000; звідси, згідно з припущенням, виключено основний
капітал, що й далі функціонує в своїй натуральній формі.

Коли ми тепер дослідимо перетворення, доконечні на основі простої
репродукції, тобто репродукції, що за неї всю додаткову вартість
споживається непродуктивно, і при цьому спочатку не звертатимемо уваги
на грошову циркуляцію, яка їх упосереднює, то матимемо насамперед
три основні точки опори.

1) $500v$, заробітна плата робітників, і $500m$, додаткова вартість капіталістів
підрозділу II, мусять витрачатись на засоби споживання. Але
їхня вартість існує в засобах споживання вартістю в 1000, що для капіталістів
підрозділу II заміщують авансовані $500v$ і репрезентують $500m$.
Отже, заробітну плату й додаткову вартість підрозділу II обмінюється
в межах підрозділу II, на продукт підрозділу II.~Разом з тим з цілого
продукту II зникає ($500v \dplus{} 500m$) II \deq{} 1000 в засобах споживання.

2) $1000v \dplus{} 1000m$ підрозділу І теж мусять витрачатись на засоби
споживання, отже, на продукт підрозділу II.~Отже, вони мусять бути обмінені
на решту цього продукту, що своїм розміром дорівнює сталій
частині капіталу $2000с$. Зате підрозділ II одержує рівну суму засобів
продукції, продукт підрозділу І, що в ньому втілено вартість $1000v \dplus{}
1000m$ підрозділу І.~Разом з тим з обчислення зникають 2000 ІІ~$с$ і
($1000v \dplus{} 1000m$) І.

3) Лишається ще 4000 І~$с$. Вони складаються з засобів продукції, які
можуть бути використані лише в підрозділі І, служать для заміщення
зужиткованого в ньому сталого капіталу, а тому справа з ними розв’язується
взаємним обміном між поодинокими капіталістами І, так само, як
з ($500v \dplus{} 500m$) II вона розв’язується обміном між робітниками й капіталістами
або між поодинокими капіталістами II.

Це покищо лише для того, щоб краще зрозуміти дальший виклад.
\label{original-305-1}

\subsection[Обмін між двома підрозділами]{Обмін
між двома підрозділами: І~\emph{(v \dplus{} m)} на ІІ~\emph{с}\footnotemark{}}

\label{original-305-2}
Ми%
\footnotetext{Відси знову рукопис VIII.}
починаємо з великого обміну між двома клясами. ($1000v \dplus{}
1000m$) І — ці вартості, що в руках їхніх продуцентів існують у натуральній
формі засобів продукції, обмінюються на 2000 ІІ~$с$, на вартості,
що існують у натуральній формі засобів споживання. У наслідок цього
\parbreak{}  %% абзац продовжується на наступній сторінці

\parcont{}  %% абзац починається на попередній сторінці
\index{ii}{0306}  %% посилання на сторінку оригінального видання
кляса капіталістів II знову перетворила свій сталий капітал, рівний 2000,
з форми засобів споживання на форму засобів продукції засобів споживання,
на форму, що в ній він може знову функціонувати як чинник
процесу праці і як стала капітальна вартість для зростання вартости. З
другого боку, в наслідок цього еквівалент робочої сили в І (1000 I~$v$)
і додаткова вартість капіталістів I (1000 І~$m$) реалізувались у засобах
споживання; і те й друге з своєї натуральної форми засобів продукції перетворилися
на таку натуральну форму, що в ній їх можна спожити як дохід.

Але таке взаємне перетворення здійснюється за допомогою грошової
циркуляції, що так само упосереднює його, як і утруднює його розуміння;
однак, вона відіграє вирішально важливу ролю, бо змінна частина капіталу
знову й знов мусить виступати в грошовій формі, як грошовий капітал,
що з грошової форми перетворюється на робочу силу. В усіх галузях
підприємств, що одночасно працюють одне біля одного на периферії
суспільства, все одно, чи належать вони до категорії I чи до II, змінний
капітал доводиться авансувати в грошовій формі. Капіталіст купує
робочу силу раніш ніж вона входить у процес продукції, але оплачує її
лише в строки, визначені умовою, після того, як її вже витрачено на
продукцію споживної вартости. Так само, як і інші частини вартости
продукту, йому належить і та частина її, що є лише еквівалент грошей
витрачених на оплату робочої сили, тобто та частина вартости продукту,
що репрезентує змінну капітальну вартість. Цією частиною вартости продукту
робітник уже дав капіталістові еквівалент своєї заробітної плати. Але
лише зворотне перетворення товару на гроші, продаж товару, відновлює
капіталістові його змінний капітал як грошовий капітал, що його він
знову може авансувати на закуп робочої сили.

Отже, в підрозділі I капіталіст, розглядуваний як збірний капіталіст,
виплатив робітникам 1000\pound{ ф. стерл.} (я кажу ф. стерл. лише для того,
щоб зазначити, що це — вартість у \emph{грошовій формі}) \deq{} $1000v$ за
ту вартість, яка вже існує як частина $v$ вартости продукту I, тобто спродукованих
робітниками засобів продукції. На ці 1000\pound{ ф. стерл.} робітники
купують у капіталістів II засобів споживання на таку саму вартість і таким
чином перетворюють половину сталого капіталу II на гроші; капіталісти II
із свого боку купують на ці 1000\pound{ ф. стерл.} засоби продукції вартістю
на 1000 у капіталістів I; тим самим змінну капітальну вартість останніх
\deq{} $1000 v$, що існує як частина їхнього продукту в натуральній формі
засобів продукції, знову перетворено на гроші, і тепер в руках капіталістів
І знову може вона функціонувати як грошовий капітал, що перетворюється
на робочу силу, отже, на найпосутніший елемент продуктивного
капіталу. Таким чином, в наслідок реалізації частини їхнього товарового
капіталу, до них зворотно припливає їхній змінний капітал у
грошовій формі.

Щодо грошей, потрібних для обміну частини $m$ товарового капіталу
І на другу половину сталої частини капіталу II, то їх можна авансувати
різними способами. На ділі ця циркуляція охоплює незчисленну силу поодиноких
актів купівлі й продажу, що їх переводять індивідуальні капіталісти
\index{ii}{0307}  %% посилання на сторінку оригінального видання
обох категорій, але при цьому гроші в усякому разі мають походити
від цих капіталістів, бо ми вже закінчили обчислення з тими
грішми, що їх подали в циркуляцію робітники. Тут буває або так, що
капіталіст категорії II із свого грошового капіталу, який існує поряд капіталу
продуктивного, купує засоби продукції в капіталістів категорії І;
або, навпаки, капіталіст категорії І із свого грошового фонду, призначеного
на особисті витрати, не на витрачання як капітал, купує засоби
споживання в капіталістів категорії II.~Як ми вже показали в відділі I і
II, ми мусимо в усякому разі припустити, що в руках капіталістів поряд
продуктивного капіталу є певні грошові запаси, — чи то для авансування
капіталу, чи то для витрачання доходу. Припустімо — пропорція не має
жодного значення для нашої цілі, — що капіталісти II авансують половину
грошей на закуп засобів продукції, щоб замістити свій сталий капітал,
а другу половину капіталісти І витрачають на споживання, а саме: підрозділ
II авансує 500\pound{ ф. стерл.} і купує на них у І засоби продукції,
тим самим він заміщує in natura (разом з вищезгаданими 1000\pound{ ф. стерл.},
то походять від робітників) \sfrac{3}{4} свого сталого капіталу; підрозділ І на
одержані таким чином 500\pound{ ф. стерл.} купує у II засоби споживання
і закінчує разом з тим циркуляцію $т — г — т$ для половини тієї частини
свого товарового капіталу, яка складається з $т$, реалізує цей продукт
свій в фонді споживання. В наслідок цього другого процесу 500\pound{ ф. стерл.}
повертаються назад до рук II як грошовий капітал, що його капіталісти
II мають поряд свого продуктивного капіталу. З другого боку, І для
половини тієї частини $т$ свого товарового капіталу, що лежить ще у
нього як продукт, антиципує — раніш, ніж цю частину продано — витрачання
грошей в розмірі 500\pound{ ф. стерл.} на закуп засобів споживання у II.~На ці самі 500\pound{ ф. стерл}. II купує засоби продукції в І і таким чином
заміщує in natura ввесь свій сталий капітал ($1000 \dplus{} 500 \dplus{} 500$ \deq{} 2000),
тимчасом як І реалізував у засобах споживання всю свою додаткову
вартість. В загальному підсумку обмін товарів на суму в 4000\pound{ ф. стерл.}
відбувся б за грошової циркуляції в 2000\pound{ ф. стерл.} і при цьому вона
досягає такої величини лише тому, що, як подано у нас, увесь річний
продукт обмінюється разом, небагатьма великими порціями. Важлива
при цьому лише та обставина, що II підрозділ не лише знову перетворив
на форму засобів продукції свій сталий капітал, репродукований
у формі засобів споживання, але що до нього, крім того, повертаються
ті 500\pound{ ф. стерл.}, що їх він авансував для циркуляції на закуп засобів
продукції; і що І підрозділ так само не лише знову одержав у грошовій
формі, як грошовий капітал, що його можна безпосередньо перетворити
на робочу силу, свій змінний капітал, репродукований ним у формі засобів
продукції, але що до нього, крім того, повертаються знову ті
500\pound{ ф. стерл.}, що їх він витратив на закуп засобів споживання, перед продажем
частини додаткової вартости від свого капіталу, антиципуючи цей
продаж. Але вони повертаються до нього назад не в наслідок витрати,
що вже відбулася, а в наслідок дальшого продажу частини його товарового
продукту, що є носій половини його додаткової вартости.


\index{ii}{0308}  %% посилання на сторінку оригінального видання
В обох випадках не лише сталий капітал II з форми продукту знову
перетворюється на натуральну форму засобів продукції, що в ній лише
й може він функціонувати як капітал; і так само не лише змінна частина
капіталу І перетворюється на грошову форму, а частина додаткової вартости
у формі засобів продукції І — на таку форму, в якій її можна спожити
як дохід. Крім цього, до II повертаються знову ті 500\pound{ ф. стерл.} грошового
капіталу, що їх він авансував на закуп засобів продукції, раніше ніж він
продав відповідну частину вартости сталого капіталу, яка існує у формі
засобів споживання і яка компенсує ці 500\pound{ ф. стерл.}; далі, до І повертаються
ті 500\pound{ ф. стерл.}, що їх він, антиципуючи продаж, витратив на
закуп засобів споживання. Коли до II повертаються назад гроші, авансовані
ним коштом сталої частини його товарового продукту, а до І — гроші,
авансовані коштом частини його товарового продукту, яка являє додаткозу
вартість, то лише тому, ще й та й друга категорія капіталістів пустили
в циркуляцію ще по 500\pound{ ф. стерл.} грошей: одна — крім наявного
в товаровій формі II сталого капіталу, друга — крім наявної в товаровій
формі І додаткової вартости. Кінець-кінцем, вони цілком поквитались одна
з однією, обмінявши відповідні товарові еквіваленти. Гроші, що їх вони
пустили в циркуляцію понад суму вартости їхніх товарів — як засоби обміну
цих товарів — повертаються до кожного з них частинами, пропорційно
тому, що кожен з них пустив у циркуляцію. В наслідок цього вони не
стали ані на шеляг багатші. II підрозділ мав сталий капітал \deq{} 2000 в формі
засобів споживання плюс 500 в грошах; тепер він має, як і раніш, 2000
в засобах продукції і 500 в грошах; так само І, як і раніш, має додаткову
вартість в 1000 (з товарів, засобів продукції, перетворених тепер
на споживний фонд) плюс 500 в грошах. Загальний висновок такий:
з тих грошей, що їх промислові капіталісти подають у циркуляцію, на
упосереднення своєї власної товарової циркуляції, — хоч їх подається
коштом сталої частини вартости товару, хоч коштом додаткової вартости,
яка існує в товарах, оскільки її витрачається як дохід — з цих грошей до
рук відповідних капіталістів повертається стільки, скільки вони авансували
на грошову циркуляцію.

Щодо зворотного перетворення на грошову форму змінного капіталу
кляси І, то він для капіталістів І, після того як вони витратили його на
заробітну плату, спочатку існує в тій товаровій формі, що в ній робітники
дали його їм. Капіталісти виплатили його робітникам у грошовій
формі як ціну робочої сили їх. В цьому розумінні вони сплатили ту
складову частину вартости їхнього товарового продукту, яка дорівнює
цьому змінному капіталові, витраченому в грошах. Тому вони — власники також
і цієї частини товарового продукту. Але застосована ними частина робітничої
кляси зовсім не є покупець засобів продукції, що їх вона сама
спродукувала. Вона — покупець засобів споживання, випродукуваних в II.~Отже, змінний капітал, авансований в грошах на оплату робочої сили, не
безпосередньо повертається до капіталістів І.~В наслідок актів купівлі,
що походить від робітників, він переходить до рук капіталістичних продуцентів
товарів, доконечних і взагалі приступних для робітничих кіл,
\parbreak{}  %% абзац продовжується на наступній сторінці

\parcont{}  %% абзац починається на попередній сторінці
\index{ii}{0309}  %% посилання на сторінку оригінального видання
отже, до рук капіталістів II, і лише в наслідок того, що капіталісти II застосовують
ці гроші на закуп засобів продукції, лише таким обкружним
шляхом повертаються вони знову до рук капіталістів І.

\roztyagnut{}
Виявляється, що при простій репродукції сума вартостей $v \dplus{} m$ товарового
капіталу І (отже, і відповідна пропорційна частина всього товарового
продукту І) мусить дорівнювати сталому капіталові II~$с$, виділеному
так само як пропорційна частина цілого товарового продукту кляси II;
або І ($v \dplus{} m$) \deq{} II~$с$.

\subsection[Обмін в межах підрозділу II. Доконечні засоби існування і речі розкошів]{Обмін в межах підрозділу II. Доконечні засоби існування~і~речі~розкошів}

З вартости товарового продукту підрозділу II нам лишається дослідити
ще складові частини $v \dplus{} m$. Розгляд їх не має ніякого чинення до
найголовнішого питання, що цікавить нас тепер, а саме, в якій мірі
розпад вартости всякого поодинокого капіталістичного товарового продукту
на $c \dplus{} v \dplus{} m$, хоча б і упосереднюваний різними формами виявлення,
має силу й для вартости цілого річного продукту. Це питання
розв’язується, з одного боку, через обмін І ($v \dplus{} m$) на II~$с$, а з другого
— через відкладений надалі дослід того, як І~$с$ репродукується в
річному товаровому продукті І.~Що II ($v \dplus{} m$) існує в натуральній формі
предметів споживання; що змінний капітал, авансований робітникам на
оплату робочої сили, взагалі та в цілому мусить витрачатися ними на
засоби споживання, і що частину товару, яка є $m$, припускаючи просту
репродукцію, фактично витрачається як дохід на засоби споживання, то
prima facie очевидно, що на заробітну плату, одержану від капіталістів
II, робітники II викупають частину свого власного продукту, яка відповідає
розмірам грошової вартости, одержаної ними як заробітна плата. Цим
самим кляса капіталістів II перетворює знову на грошову форму свій
грошовий капітал, авансований на оплату робочої сили; справа цілком
така сама, ніби ці капіталісти оплатили робітників простими знаками вартости.
Скоро робітники реалізують ці знаки вартости, купуючи частину
спродукованого ними й належного капіталістам товарового продукту, ці
знаки вартости повернуться знову до капіталістів, але лише тому, що тут
ці знаки не тільки репрезентують вартість, а й мають її в її золотій
або срібній тілесності. Далі ми дослідимо ближче цей рід зворотного
припливу змінного капіталу, авансованого в грошовій формі, здійснюваний
через процес, що в ньому робітнича кляса виступає як покупець, а
кляса капіталістів — як продавець. А тут ідеться про інше питання,
що його треба розглянути при цьому зворотному припливі змінного
капіталу до його вихідного пункту.

Категорія II річної товарової продукції складається з найрізноманітніших
галузей промисловости, які — щодо їхнього продукту — можна поділити
на два великі підвідділи:

а) Засоби споживання, що входять у споживання робітничої кляси, і —
оскільки це є доконечні засоби існування — становлять також частину
\parbreak{}  %% абзац продовжується на наступній сторінці

\parcont{}  %% абзац починається на попередній сторінці
\index{ii}{0310}  %% посилання на сторінку оригінального видання
споживання кляси капіталістів, хоч у цьому разі вони часто відрізняються
якістю й вартістю від засобів споживання робітників. Для нашої мети
ми можемо охопити весь цей підвідділ однією рубрикою: \so{доконечні}
засоби споживання, і при цьому тут цілком байдуже, чи є такий продукт,
як наприклад, тютюн, доконечний засіб споживання з фізіологічного
погляду чи ні; досить того, що він є звично доконечний засіб споживання.

b) Засоби споживання \deq{} \so{речі розкошів}, що входять лише в
споживання кляси капіталістів, а значить, їх можна обміняти лише на
витрачувану додаткову вартість, що ніколи не дістається робітникам.
Щодо першої рубрики, то очевидно, що змінний капітал, авансований у
грошовій формі на продукцію належних сюди ґатунків товару, мусить
безпосередньо повернутись до тієї частини кляси капіталістів II (тобто
до капіталістів II~\emph{а}), яка продукує ці доконечні засоби існування.
Капіталісти продають їх своїм власним робітникам на суму змінного капіталу,
сплаченого їм як заробітна плата. Цей зворотний приплив щодо цілого
цього підвідділу \emph{а} кляси капіталістів II відбувається \so{безпосередньо},
хоч які б численні були ті оборудки між капіталістами різних належних
сюди галузей промисловости, через що розподіляється pro rata
цей змінний капітал, який зворотно припливає. Це процеси циркуляції,
що для них засоби циркуляції є безпосередньо ті гроші, що їх витрачають
робітники. Інакше стоїть справа з підвідділом II~\emph{b}. Вся та частина
новоспродукованої вартости, що з нею ми маємо тут справу, II~\emph{b} ($v \dplus{} m$)
існує в натуральній формі речей розкошів, тобто речей, що їх робітнича
кляса так само не може купити, як і товарової вартости I~$v$, яка існує
в формі засобів продукції; хоч і ці речі розкошів і ті засоби продукції
є продукти цих робітників. Отже, зворотний приплив, за допомогою
якого змінний капітал, авансований у цьому підвідділі, повертається до
капіталістичного продуцента в своїй грошовій формі, не може бути
прямим, а тільки посереднім, як у I~$v$.

\label{original-310}
Припустімо, напр., як раніше, для всієї кляси II: $v \deq{} 500$; $m \deq{} 500$;
але змінний капітал і відповідна йому додаткова вартість хай розподіляються
так:

\noindent{}Підвідділ \emph{а}, доконечні засоби існування: $v \deq{} 400$, $m \deq{} 400$; отже,
маса товарів в доконечних засобах споживання вартістю в $400 v \dplus{} 400 \deq{} 800$,
або II~\emph{а} ($400 v \dplus{} 400 m$).

\noindent{}Підвідділ \emph{b}: речі розкошів вартістю в $100 v \dplus{} 100 m \deq{} 200$, або
II~\emph{b} ($100 v \dplus{} 100 m$).

Робітники підвідділу II~\emph{b} в оплату за свою робочу силу одержали
100 грішми, напр., 100\pound{ ф. стерл.}; на них робітники купують у капіталістів
II~\emph{а} засоби споживання на суму 100. Ця кляса капіталістів купує
тоді на 100 товару II~\emph{b}, і в наслідок цього до капіталістів II~\emph{b}
зворотно припливає в грошовій формі їхній змінний капітал.

В руках капіталістів II~\emph{а} в наслідок обміну з їхніми власними робітниками
вже є 400 в грошовій формі; крім того, четверта частина їхнього
продукту, яка репрезентує додаткову вартість, відійшла до робітників
II~\emph{b}, і за неї одержано в товарах розкошів II~\emph{b} ($100 v$).

\disablefootnotebreak{}

\index{ii}{0311}  %% посилання на сторінку оригінального видання
Коли ми тепер припустимо пропорційно однаковий поділ витрат
доходу на доконечні засоби існування та засоби розкошів у капіталістів
II~$а$ і II~$b$ — припустимо, що й ті й ці витрачають по \sfrac{3}{5} на доконечні
засоби існування, по \sfrac{2}{5} на речі розкошів, то капіталісти підкляси II~$а$
\sfrac{3}{5} своєї додаткової вартости, свого доходу в $400 m$, отже, 240, витрачатимуть
на свої власні продукти, на доконечні засоби існування, і
\sfrac{2}{5} \deq{} 160 — на речі розкошів. Капіталісти підкляси II~$b$ розподілятимуть
свою додаткову вартість $100 m$ таким самим способом: \sfrac{3}{5} \deq{} 60 на доконечні
засоби і \sfrac{2}{5} \deq{} 40 на речі розкошів; ці останні продукується й
обмінюється в межах її власної підкляси.

Ті 160 в засобах розкошів, що їх одержує (II~$а$) $m$, припливають до
капіталістів II~$а$ таким чином: з (II~$а$) $400 m$, як ми бачили, 100, що є у
формі доконечних засобів існування, обмінюється на рівну суму (II~$b$) $v$,
яка існує в засобах розкошів, а дальші 60 в доконечних засобах існування
обмінюється на (II~$b$) $60 m$ в засобах розкошів. Отже, загальний
підсумок буде такий:

II~$а: 400 v \dplus{} 400 m$; II~$b : 100 v \dplus{} 100 m$.

1) $400 v$ ($а$) споживають робітники II~$a$, що частину їхнього продукту
(доконечні засоби існування) становлять ці $400 v$ ($a$); робітники купують
їх у капіталістичних продуцентів свого власного підрозділу. У наслідок
цього до цих капіталістичних продуцентів повертаються 400\pound{ ф. стерл.}
грішми, повертається їхня змінна капітальна вартість в 400, сплачена як
заробітна плата цим самим робітникам; на цю вартість капіталісти можуть
знову купити робочу силу.

2) Частину $400 m$ ($a$), рівну $100 v$ ($b$), отже, \sfrac{1}{4} додаткової вартости ($а$),
реалізується в речах розкошів таким чином: робітники ($b$) одержують від
капіталістів свого підрозділу ($b$) 100\pound{ ф. стерл.} як заробітну плату; на
цю суму вони купують \sfrac{1}{4}$m$ ($a$), тобто товари, що складаються з доконечних
засобів існування; капіталісти $а$ купують на ці гроші речей
розкошів на таку саму суму вартости \deq{} $100 v$ ($b$), тобто половину всієї
продукції речей розкошів. У наслідок цього до капіталістів $b$ повертається
в грошовій формі їхній змінний капітал, і вони, відновивши закуп
робочої сили, можуть знову почати свою репродукцію, бо ввесь сталий
капітал всієї кляси II вже заміщено через обмін І ($v \dplus{} m$) на ІІ~$с$. Отже,
робочу силу робітників, що продукують речі розкошів, тільки тому
можна продати знову, що частину їхнього власного продукту, утворену
як еквівалент їхньої власної заробітної плати, взяли капіталісти II~$а$ в
свій споживний фонд, продано їм. (Це саме має силу й для продажу робочої
сили підрозділу І: бо те ІІ~$с$, що на нього обмінюється І ($v \dplus{} m$), складається
і з речей розкошів і з доконечних засобів існування, а те, що
відновлюється через І ($v \dplus{} m$), складається з засобів продукції так речей
розкошів, як і доконечних (засобів існування).

3) Переходимо до обміну між $а$ і $b$, оскільки він є обмін лише між
капіталістами обох підвідділів. До цього часу закінчено справу з змінним
\parbreak{}  %% абзац продовжується на наступній сторінці

\parcont{}  %% абзац починається на попередній сторінці
\index{ii}{0312}  %% посилання на сторінку оригінального видання
капіталом ($400 v$) і частиною додаткової вартости ($100 m$) в підвідділі $а$
і з змінним капіталом ($100 v$) у підвідділі $b$. В дальшому ми припускаємо,
що пересічна пропорція між витратами доходу в капіталістів обох кляс
є така: \sfrac{2}{5} на речі розкошів і \sfrac{3}{5} на доконечні засоби існування.
Тому крім 100, уже витрачених на речі розкошів, усій підклясі $а$ припадає
ще 60 на речі розкошів і в такій самій пропорції, тобто 40 припадає
підклясі $b$.

Отже, (II~$а$) $m$ розподіляється так: 240 на засоби існування і 160 на
речі розкошів \deq{} $240 \dplus{} 160 \deq{} 400 m$ (II~$а$).

(II~$b$) $m$ розподіляється так: 60 на засоби існування й 40 на речі
розкошів: $60 \dplus{} 40 \deq{} 100 m$ (II~$b$). Останні 40 ця кляса бере для споживання
з свого власного продукту (\sfrac{2}{5} своєї додаткової вартости); 60 в
засобах існування вона одержує, обмінюючи 60 свого додаткового продукту
на $60 m$ ($а$).

Отже, для цілої кляси капіталістів II ми маємо (при цьому $v \dplus{} m$ в
підвідділі $а$ існують у доконечних засобах існування, в підвідділі $b$ — в
речах розкошів):

II~$а$ ($400 v \dplus{} 400 m$) \dplus{} II~$b$ ($100 v \dplus{} 100 m$) \deq{} \num{1.000}; через рух усе
це реалізується так: $500 v$ ($а \dplus{} b$) [реалізуються в $400 v$ ($а$) і $100 m$ ($а$)] \dplus{}
$500m$ ($а \dplus{} b$) [реалізуються в $300 m$ ($а$) \dplus{} $100 v$ ($b$) \dplus{} $100 m$ ($b$)] \deq{}
1000.

Для $а$ і $b$, розглядуваних окремо, реалізація відбувається таким
чином:

\[\begin{array}{r@{}c@{}r@{}r}
а) \overbrace{400 v (a)}^{v} \dplus{} &
  \overbrace{240 m (a) \dplus{} 100 v (b) \dplus{} 60 m (b)}^{m}
& \dots{} \deq{} & 800 \\
b) \overbrace{100 m (a)}^{v} \dplus{} &
  \overbrace{60 m (a) \dplus{} 40 m (b)}^{m}
&\dots{} \deq{} & 200 \\
& & & \overline{1000}
\end{array}
\]
Коли ми спрощення ради додержуватимемось для обох підвідділів однакового
відношення між змінним і сталим капіталом (що, до речі, зовсім не неодмінно),
то на $400 v$ ($а$) припаде сталий капітал \deq{} 1600, а на 100 ($b$)
сталий капітал \deq{} 400, і для II будуть такі два підрозділи $а$ і $b$:
\begin{align*}
\text{II} а\text{) } 1600 c \dplus{} 400 v \dplus{} 400 m & \deq{} 2400 \\
\text{II} b\text{) } 400 c \dplus{} 100 v \dplus{} 100 m & \deq{} 600 \\
\intertext{a разом}
2000 c \dplus{} 500 v \dplus{} 500 m & \deq{} 3000.
\end{align*}

\noindent{}Відповідно до цього з 2000 II~$с$ в засобах споживання, які обмінюють
на 2000 І ($v \dplus{} m$), 1600 обмінюються на засоби продукції доконечних
засобів існування і 400 — на засоби продукції речей розкошів.

Отже, ці 2000 І ($v \dplus{} m$) і собі поділяться на ($800 v \dplus{} 800 m$) І,
призначених для $а \deq{} 1600$ засобів продукції доконечних засобів існування,
\index{ii}{0313}  %% посилання на сторінку оригінального видання
і на ($200 v \dplus{} 200 m$) І, призначених для $b$, \deq{} 400 засобів продукції
речей розкошів.

Чимала частина не лише власне засобів праці, а й сировинних та
допоміжних матеріялів тощо, однорідна в обох підрозділах. Але щодо
обміну різних частин вартости цілого продукту I ($v \dplus{} m$), то цей поділ
на підрозділи не має жодного значення. Так згадані вище 800 I~$v$, як і
200 I~$v$ реалізується в наслідок того, що заробітну плату витрачається
на засоби споживання 1000 ІІ~$с$, отже, грошовий капітал, авансований на
неї, повертаючись, розподіляється рівномірно між капіталістами продуцентами
I і pro rata заміщує в грошах авансований ними змінний капітал:
з другого боку, щодо реалізації 1000 I~$m$, то і тут капіталісти рівномірно
(пропорційно величині їхнього $m$) візьмуть засоби споживання з
усієї другої половини ІІ~$с \deq{} 1000: 600$ II~$а$ і 400 II~$b$; отже, ті, що заміщують
сталий капітал II~$а$:

480 (\sfrac{3}{5}) з 600$с$ (II~$а$) і 320 (\sfrac{2}{5}\footnote*{
Зазначені тут у дужках дроби \sfrac{3}{5} і \sfrac{2}{5} є частина від усієї другої половини
сталого капіталу II~$а$, тобто від 800, так само, як при II~$b$ вони є частини від усієї
другої половини сталого капіталу II~$b$, тобто від 200. \Red
}) з $400с$ (II~$b$) \deq{} 800; ті, що заміщують
сталий капітал II~$b$:

120 (\sfrac{3}{5}) від $600 с$ (II~$а$) і 80 (\sfrac{2}{5}) від $400 с$ (ІІ~$b$) \deq{} 200. Сума \deq{} 1000.

Що тут узято довільно і для І і для II, так це — відношення змінного
капіталу до сталого, а також однаковість цього відношення в
І і в II і в їхніх підвідділах. Щодо цієї однаковости, то її тут припущено
лише для спрощення; припущення різних пропорцій абсолютно нічого
не змінило б в умовах проблеми та в її розв’язанні. Але, коли припустити
просту репродукцію, то як доконечний результат з цього випливає таке:

1) Нова вартість, утворена річною працею в натуральній формі засобів
продукції (яка розпадається на $v \dplus{} m$), дорівнює репродукованій
у формі засобів споживання сталій капітальній вартості $с$ від вартости
продукту, утвореного другою частиною річної праці. Коли б ця нова вартість
була менша за II~$с$, то II не міг би повнотою замістити свій сталий капітал;
коли б вона була більша, то надлишок не використовувалось би. В
обох випадках порушувалось би наше припущення простої репродукції.

2) Щодо річного продукту, репродукованого в формі засобів споживання,
то змінний капітал $v$, авансований в грошовій формі, можуть реалізувати
його одержувачі — оскільки вони є робітники, що продукують
речі розкошів — лише в тій частині доконечних засобів існування, що в
ній prima facie втілено додаткову вартість капіталістичних продуцентів
цих засобів; отже, $v$, витрачене на продукцію речей розкошів, розмірами
своєї вартости дорівнює відповідній частині $m$, спродукованій у
формі доконечних засобів існування, отже, воно мусить бути менше, ніж
усе це $m$ — а саме (II~$а$) $m$ — і лише через реалізацію цього $v$ в тій частині
$m$ до капіталістичних продуцентів речей розкошів повертається в грошовій
формі авансований ними змінний капітал. Це явище цілком аналогічне
до реалізації I ($v \dplus{} m$) в ІІ~$с$; ріжниця лише та, що в другому випадку
(IІ~$b$) $v$ реалізується в частині (II~$а$) $m$, яка величиною вартости дорівнює
\index{ii}{0314}  %% посилання на сторінку оригінального видання
(II~$b$) $v$. Ці відношення лишаються якісно вирішальні при всякому
розподілі всього річного продукту, оскільки він дійсно входить у процес
річної репродукції, упосереднюваної циркуляцією I ($v \dplus{} m$) можна реалізувати
лише в II~$с$, так само, як II~$с$ в його функції складової частини
продуктивного капіталу можна відновити лише за допомогою цієї реалізації;
так само (II~$b$) $v$ можна реалізувати лише в частині (II~$а$) $m$, і лише
таким способом (II~$b$) $v$ можна знову перетворити на його форму грошового
капіталу. Звичайно, це має силу лише за тієї умови, що все це дійсно
є результат самого процесу репродукції, отже, коли, напр., капіталісти
II~$b$ не одержують грошового капіталу для $v$ за допомогою кредиту з
якихось інших джерел. Навпаки, щодо кількісного боку, то обміни різних
частин річного продукту можуть відбуватися з такою пропорційністю, як
подано вище, лише остільки, оскільки маштаб та відношення вартости
продукції лишаються незмінні, і оскільки ці точно визначені відношення
не зазнають змін в наслідок зовнішньої торговлі.

Коли, за прикладом А.~Сміса, казали, що I ($v \dplus{} m$), розкладається на
II~$с$, II~$с$ розкладається на I ($v \dplus{} m$), або, як він часто каже своїм
звичаєм іще недоладніше, що I ($v \dplus{} m$) становлять складові частини ціни,
зглядно вартости, він каже value in exchange II~$с$, а II~$с$ становить усю
складову частину вартости I ($v \dplus{} m$), то можна й треба було б сказати
також, що (II~$b$) $v$ розкладається на (II~$а$) $m$, або (II~$а$) $m$ на (II~$b$) $v$,
або що (II~$b$) $v$ становить складову частину додаткової вартости II~$а$, і
навпаки: додаткова вартість розкладалась би таким чином на заробітну
плату, зглядно на змінний капітал, а змінний капітал становив би „складову
частину“ додаткової вартости. І справді, така недоречність дійсно
є у А.~Сміса, бо заробітна плата визначається в нього вартістю доконечних
засобів існування, і ці товарові вартості знову таки визначаються
вартістю вміщених у них заробітної плати (змінного капіталу) і додаткової
вартости. Він до того захопився тими частинами, на які при капіталістичній
основі продукції можна розкласти вартість, спродуковану протягом
одного робочого дня, — а саме $v \dplus{} m$, — що цілком забуває про те, що при
простому товаровому обміні цілком байдуже, чи складаються еквіваленти,
які існують в різних натуральних формах, з оплаченої чи неоплаченої праці:
бо в обох випадках вони коштують однакову кількість праці, витраченої
на їхню продукцію; і що так само байдуже, чи є товар якогось \emph{А} засоби
продукції, а товар якогось \emph{В} — засоби споживання, чи має один
товар функціонувати після продажу як складова частина капіталу, а
другий, навпаки, входить у фонд споживання його, і, за Адамом, споживається
як дохід. В який спосіб індивідуальний покупець вживає свій
товар, це не має жодного чинення до обміну товарів, до сфери циркуляції,
і не стосується вартости товару. Це ані трохи не змінюється
від того, що при аналізі циркуляції всього річного суспільного продукту
треба взяти на увагу певний характер вживання, момент споживання
різних складових частин цього продукту.

При вище констатованому обміні (II~$b$) $v$ на рівновартісну частину
(II~$a$) $m$ і при дальших обмінах між (II~$a$) $m$ і (II~$b$) $v$ зовсім не припускається,
\index{ii}{0315}  %% посилання на сторінку оригінального видання
що капіталісти — хоч поодинокі капіталісти II~$а$ і II~$b$, хоч відповідні
категорії капіталістів в їхній сукупності — в однаковому відношенні
розподіляють свою додаткову вартість між доконечними предметами споживання
й засобами розкошів. Один може більше витрачати на одні
предмети споживання, другий — на другі. Лишаючись на ґрунті простої
репродукції, ми припускаємо тільки, що суму вартости, рівну всій додатковій
вартості, реалізується в фонді споживання. Отже, межі тут
дано. В межах кожного підрозділу один може більше витрачати на $а$,
другий на $b$; тут можлива взаємна компенсація, так що кляси капіталістів
$а$ і $b$, взяті кожна як ціле, будуть в однаковій мірі брати участь
в $а$ і $b$. Але відношення вартостей — пропорційна участь в цілій вартості
продукту II обох категорій продуцентів $а$ і $b$ — а значить і певне кількісне
відношення між галузями продукції, що дають ці продукти — ці
відношення неодмінно є дані для кожного конкретного випадку: гіпотетичне
є лише відношення, що фігурує в прикладі; коли припустити
інше відношення, то від цього ніщо не зміниться в якісних моментах;
змінилися б лише кількісні визначення. Але коли б в наслідок тих або
інших обставин постала справжня зміна у відносних величинах $а$ й $b$, то
відповідно змінились би й умови простої репродукції.

З тієї обставини, що (II~$b$) $v$ реалізується в еквівалентній частині
(ІІ~$а$) $m$, випливає, що тією самою мірою, як більшає частина річного
продукту, яка припадає на речі розкошів, отже, тією самою мірою, як
більшає маса робочої сили, що її поглинає продукція засобів розкошів,
такою самою мірою зворотне перетворення авансованого на (II~$b$) $v$
змінного капіталу в грошовий капітал, що знову функціонує як грошова
форма змінного капіталу, а в наслідок цього й існування і репродукція
частини робітничої кляси, занятої в II~$b$ — одержання цією частиною робітничої
кляси доконечних засобів споживання — зумовлюється марнотратством
кляси капіталістів, перетворенням значної частини їхньої додаткової
вартости на речі розкошів.

Кожна криза моментально зменшує споживання речей розкошів; вона
уповільнює, затримує зворотне перетворення (II~$b$) $v$ на грошовий капітал,
лише почасти допускає це перетворення й тим самим викидає частину робітників,
які виробляють речі розкошів, на брук, а з другого боку, саме через
це вона призводить до застою й скорочення продажу доконечних засобів
споживання. Ми залишаємо цілком осторонь звільнених разом з цим
непродуктивних робітників, які за свої послуги одержують від капіталістів
частину їхніх витрат на розкоші (самі ці робітники pro tanto є
предмети розкошів) і беруть також дуже велику участь у споживанні
доконечних засобів існування тощо. Протилежне маємо в періоди процвітання
і особливо підчас спекулятивного процвітання, коли відносна,
виражена в товарах вартість грошей, падає вже з інших причин (при
йому не відбувається дійсного перевороту в вартості), а тому ціна товарів,
незалежно від їхньої власної вартости, підвищується. При цьому
\parbreak{}  %% абзац продовжується на наступній сторінці

\parcont{}  %% абзац починається на попередній сторінці
\index{ii}{0316}  %% посилання на сторінку оригінального видання
підвищується не тільки споживання доконечних засобів існування; робітнича
кляса (куди тепер увіходить, як її активна частина, вся резервна
армія) на малу часину бере участь у споживанні речей розкошів, що іншим
часом для неї неприступні, і, крім того, вона бере участь у споживанні
тієї категорії доконечних засобів споживання, яка іншим часом, здебільшого,
становить „доконечні“ засоби споживання лише для кляси капіталістів;
це також із свого боку зумовлює підвищення цін.

Було б простою тавтологією сказати, що кризи випливають з недостачі
платоспроможного споживання або платоспроможних споживачів.
Капіталістична система не знає інших видів споживання, крім оплачуваного,
за винятком видів sub forma pauperis\footnote*{
У формі одержання милостині. \Red{Ред.}
} або „шахраїв“. Що
товари несила продати, не значить нічого іншого, як те, що на них не
знаходиться платоспроможних покупців, отже, споживачів (припускаючи,
що товари, кінець-кінцем, купується для продуктивного або особистого
споживання). А коли цій тавтології намагаються надати вигляд
глибшого обґрунтовання, кажучи, що робітнича кляса одержує дуже малу
частину свого власного продукту і що цьому лихові можна запобігти,
коли вона одержуватиме більшу частину свого продукту, тобто, коли її
заробітна плата збільшиться, то треба тільки зауважити, що кожна криза
підготовляється саме таким періодом, коли повсюди підвищується заробітна
плата, і робітнича кляса справді одержує більшу пайку тієї частини
річного продукту, що призначена для споживання. Такий період — з погляду
цих лицарів здорового й „простого“ (!) розуму — мусив би, навпаки,
віддалити кризу. Отже, бачимо, що капіталістична продукція включає
незалежні від доброї або злої волі умови, які допускають відносний
добробут робітничої кляси тільки на малу частину, та й це лише завжди
як буровісник кризи\footnote{
Ad notam (до відома) деяких прихильників Ротбертусової теорії криз. \emph{Ф.~Е.}
}.

Ми бачили раніше, як пропорційне відношення між продукцією
доконечних засобів споживання і продукцією засобів розкошів зумовлює
поділ II ($v \dplus{} m$) між II~\emph{а} і II~\emph{b}, a значить, і поділ II $с$ між (II~\emph{а}) $с$ і (II~\emph{b}) $с$.
Отже, цей поділ стосується до самого кореня характеру й кількісних відношень
продукції і є момент, посутньо визначальний для цілого її ладу.

Проста репродукція по суті має на меті споживання, хоч здобування
додаткової вартости й тут являє рушійний чинник для індивідуальних
капіталістів; але додаткова вартість, — хоч яка буде її відносна величина —
кінець-кінцем, повинна тут служити лише для особистого споживання
капіталіста.

Оскільки проста репродукція є також частина, і до того найзначніша
частина, кожної річної репродукції в поширеному маштабі, цей мотив —
особисте споживання — лишається поряд з мотивом збагачення і протилежно
до нього як такого. В дійсності справа видається заплутанішою,
бо спільники (partners) здобичі — додаткової вартости капіталіста — виступають
як незалежні від нього споживачі.


\index{ii}{0317}  %% посилання на сторінку оригінального видання
\subsection{Упосереднення обмінів грошовою циркуляцією}

Оскільки з’ясовано до цього часу, циркуляція між різними клясами
продуцентів перебігала за такою схемою:

1) Між клясою І і клясою II:
\[
\begin{array}{r@{ }l@{ }c@{ }l}
\text{І. } & 4000с \dplus{} & \underbrace{1000 v \dplus{} 1000 m}^{} \\
\text{II. } & \dotfill{} & \dotfill{} 2000 с \dotfill{} & \:\dplus{}\:500 v \dplus{} 500
\end{array}
\]
Отже, закінчилась циркуляція II~$с$ \deq{} 2000, і обмінено його на
І ($1000 v \dplus{} 1000 m$).

А що ми лишаємо на деякий час осторонь 4000 І~$с$, то лишається ще
циркуляція $v \dplus{} m$ в межах кляси II.~Ці II ($v \dplus{} m$) розподіляються між
підклясами II~$а$ і II~$b$ таким чином:

\[
\text{2) II. }500 v \dplus{} 500 m \deq{} а (400 v \dplus{} 400 m) \dplus{} b (100 v \dplus{} 100 m).
\]
$400 v$ ($а$) циркулюють в межах своєї власної підкляси; робітники,
оплачені цими $400 v$ ($а$), купують на них спродуковані ними самими доконечні
засоби існування в своїх наймачів, у капіталістів II~$а$.

А що капіталісти обох підкляс витрачають свою додаткову вартість
по \sfrac{3}{5} на продукти II~$a$ (доконечні засоби існування) і по \sfrac{2}{5} на продукти
II~$b$ (речі розкошів), то \sfrac{3}{5} додаткової вартости $а$, тобто 240, споживається
в межах самої підкляси II~$а$; так само \sfrac{2}{5} додаткової вартости $b$ (що спродукована)
й існує в засобах розкошів — в межах підкляси II~$b$.

Лишається, отже, для обміну між II~$a$ і II~$b$:

на боці II~$а$: $160 m$

на боці II~$b$: $100 v \dplus{} 60 m$. Ці продукти навзаєм покриваються. Робітники
II~$b$ на свої 100, одержані як заробітна плата в грошовій формі,
купують в II~$а$ доконечні засоби існування на суму в 100. Капіталісти
II~$b$ на суму в \sfrac{2}{5} своєї додаткової вартости \deq{} 60 так само купують
доконечні засоби існування в II~$а$. В наслідок цього капіталісти II~$а$ одержують
гроші, потрібні для того, щоб, як припущено вище, \sfrac{2}{5} своєї
додаткової вартости \deq{} $160 m$ витратити на речі розкошів, спродуковані
в ІІ~$b$ ($100 v$, що є в руках капіталістів II~$b$ як продукт, який заміщує
видану заробітну плату, і $60 m$). Отже, маємо для цього таку схему:
\[\begin{array}{r@{~}l}
\text{3) II} а & (400 v) \dplus{} (240 m) \dplus{} 160 m \\
b & \makebox[65pt]{\dotfill{}} \overbrace{100 v \dplus{} 60 m} \dplus{} (40 m)\text{,}
\end{array}
\]
де в дужки взято ті величини, які циркулюють і споживаються лише в
межах своєї власної підкляси.

Безпосередній зворотний приплив грошового капіталу, авансованого у
формі змінного капіталу, який відбудеться лише для підрозділу капіталістів
II~$а$, де продукується доконечні засоби існування, є лише модифіковане
особливими умовами виявлення того згаданого вище
загального закону, що гроші при нормальному перебігу товарової
циркуляції повертаються назад до тих товаропродуцентів, які авансують
їх на циркуляцію. З цього між іншим випливає, що коли за
товаропродуцентом взагалі стоїть грошовий капіталіст, який знову
\parbreak{}  %% абзац продовжується на наступній сторінці

\parcont{}  %% абзац починається на попередній сторінці
\index{ii}{0318}  %% посилання на сторінку оригінального видання
авансує промисловому капіталістові грошовий капітал (в найточнішому
значенні цього слова, тобто капітальну вартість у грошовій формі), то
справжнім пунктом повороту цих грошей є кишеня цього грошового
капіталіста. Таким чином, хоч гроші в своїй циркуляції більш або менш
переходять через усякі руки, маса грошей, що циркулюють, належить
підрозділові грошового капіталу, організованому і сконцентрованому в
формі банків тощо; спосіб, що ним цей підрозділ авансує свій капітал,
зумовлює постійний кінцевий, зворотний приплив до нього цього капіталу
в грошовій формі, хоч це знову таки упосереднюється зворотним перетворенням
промислового капіталу на грошовий капітал.

Для товарової циркуляції завжди потрібні дві умови; товари, подавані
в циркуляцію, і гроші, подавані в циркуляцію. „Процес циркуляції\dots{} не
закінчується, як безпосередній обмін продуктами, після того як споживні
вартості перемінили місця або посідачів. Гроші не зникають тому, що
вони, кінець-кінцем, випали з ряду метаморфоз якогось товару. Вони раз-у-раз
осідають у тих пунктах циркуляції, що їх звільняють ті або інші
товари“. (Книга І, розд. III, 2 п. а).

Напр., розглядаючи циркуляцію між II $с$ і І ($v \dplus{} m$), ми припустили,
що II підрозділ авансував для цієї циркуляції 500\pound{ ф. стерл.} грішми. При
безмежному числі тих процесів циркуляції, що на них сходить циркуляція
між великими суспільними групами продуцентів, продуцент то
однієї, то другої групи спершу виступає як покупець, отже, подає
гроші в циркуляцію. Цілком лишаючи осторонь індивідуальні обставини,
це зумовлено вже неоднаковістю періодів продукції, а тому й оборотів
різних товарових капіталів. Отже, II на 500\pound{ ф. стерл.} купує у І засобів
продукції на таку саму суму вартости, а І купує у II засобів споживання на
500\pound{ ф. стерл.}; отже, гроші припливають назад до II; останній ані трохи
не збагачується таким зворотним припливом. Спочатку він подав у циркуляцію
500\pound{ ф. стерл.} грішми і вилучив звідти товарів на ту саму суму вартости,
потім він продає товарів на 500\pound{ ф. стерл.} і вилучає з циркуляції
таку саму суму вартости в грошах; таким чином 500\pound{ ф. стерл.} припливають
назад. В дійсності II підрозділ подав таким чином у циркуляцію
на 500\pound{ ф. стерл.} грошей і на 500\pound{ ф. стерл.} товарів \deq{} 1000\pound{ ф. стерл.};
він вилучає з циркуляції на 500\pound{ ф. стерл.} товарів і на 500\pound{ ф. стерл.}
грошей. Для обміну 500\pound{ ф. стерл.} товарами (І) і 500\pound{ ф. стерл.} товарами
(II) циркуляція потребує лише 500\pound{ ф. стерл.} грішми; отже,
хто на закуп чужого товару авансував гроші, той одержує їх
назад, продаючи власний товар. Тому, коли б спочатку І купив у II
товару на 500\pound{ ф. стерл.}, а потім продав би підрозділові II товару на
500\pound{ ф. стерл.}, то 500\pound{ ф. стерл.} повернулись би до І, а не до II.

Гроші, витрачені на заробітну плату, тобто змінний капітал, авансований
у грошовій формі, в клясі І повертаються в цій формі не безпосередньо,
а посередньо, обкружним шляхом. Навпаки, в клясі II 500\pound{ ф.
стерл.} заробітної плати повертаються безпосередньо від робітників до
капіталістів, як і взагалі цей зворотний приплив завжди є безпосередній
у всіх тих випадках, коли купівля та продаж між тими самими особами
\parbreak{}  %% абзац продовжується на наступній сторінці

\parcont{}  %% абзац починається на попередній сторінці
\index{ii}{0319}  %% посилання на сторінку оригінального видання
повторюються так, що вони по черзі протистоять один одному раз як
купці, раз як продавці товарів. Капіталіст II оплачує робочу силу грішми;
таким чином він вводить робочу силу в склад свого капіталу і лише
в наслідок цього акту циркуляції, що є для нього тільки перетвір грошового
капіталу на продуктивний капітал, він протистоїть як промисловий
капіталіст робітникові як своєму найманому робітникові. Але
потім робітник, що на першій стадії виступав як продавець, торговець
своєю власною робочою силою, на другій стадії протистоїть як покупець,
як власник грошей, капіталістові як продавцеві товару; в наслідок цього
до капіталіста повертаються гроші, витрачені на заробітну плату. Оскільки
продаж цих товарів не сполучається з шахрайством і~\abbr{т. ін.}, отже,
оскільки в товарах і грошах тут обмінюється еквіваленти, він не є процес,
що за допомогою його збагачується капіталіст. Він не оплачує
робітника двічі: спочатку грішми, а потім товаром; його гроші повертаються
до нього, скоро робітник купить в нього на ці гроші товар.

Але грошовий капітал, перетворений на змінний капітал, — тобто гроші,
авансовані на заробітну плату, — відіграє головну ролю в самій грошовій
циркуляції, бо — через те, що робітнича кляса мусить перебиватися з
дня на день і тому не може кредитувати промислових капіталістів на
довгий час — в незчисленних, територіально різних пунктах суспільства
змінний капітал одночасно мусить авансуватися в грошах через певні
короткі строки, приміром, щотижня тощо, через порівняно швидко
повторювані переміжки часу (що коротші ці переміжки, то порівняно
менша може бути вся сума грошей, кожного разу подаваних у
циркуляцію через цей канал), — хоч які різні будуть періоди обороту
капіталів у різних галузях промисловости. Авансований таким чином грошовий
капітал становить у кожній країні капіталістичної продукціі відносно
вирішальну частину цілої циркуляції, то більше, що ці самі гроші,
перш ніж повернутись до вихідного пункту, циркулюють у найрізноманітніших
каналах і функціонують як засіб циркуляції у безмежному числі
інших операцій.
\pfbreak
Розгляньмо тепер циркуляцію між І ($v \dplus{} m$) і II~$c$ з іншого погляду.

\vtyagnut{}
Капіталісти І авансують на видачу заробітної плати 1000\pound{ ф. стерл.},
що на них робітники купують на 1000\pound{ ф. стерл.} засоби існування в капіталістів
II, а ті теж купують на ті самі гроші засоби продукції в капіталістів
І.~Тепер до останніх повернувся в грошовій формі їхній змінний
капітал, тимчасом як капіталісти II перетворили половину свого сталого
капіталу з форми товарового капіталу знову на продуктивний капітал.
Капіталісти II авансують дальші 500\pound{ ф. стерл.} грішми, щоб дістати засоби
продукції в І; капіталісти І витрачають ці гроші на засоби споживання
II; таким чином ці 500\pound{ ф. стерл.} припливають назад до капіталістів II;
вони знову авансують ці гроші, щоб перетворити знову на продуктивну
натуральну форму останню чверть свого сталого капіталу, перетвореного
на товар. Ці гроші знову повертаються до І і знову забирають у II
\parbreak{}  %% абзац продовжується на наступній сторінці

\parcont{}  %% абзац починається на попередній сторінці
\index{ii}{0320}  %% посилання на сторінку оригінального видання
засобів споживання на таку саму суму; в наслідок цього ці 500\pound{ ф. стерл.}
припливають назад до II; капіталісти цього підрозділу тепер, як і раніше,
мають 500\pound{ ф. стерл.} в грошах і 2000\pound{ ф. стерл.} в сталому капіталі,
однак, останній знову перетворено з форми товарового капіталу на продуктивний
капітал. Циркуляція маси товарів на 5000\pound{ ф. стерл.} відбулась
за посередництвом 1500\pound{ ф. стерл.} грошей, а саме: 1) І виплачує робітникам
1000\pound{ ф. стерл.} за робочу силу такої самої величини вартости;
2) робітники на ці 1000\pound{ ф. стерл.} купують у II засоби існування;
3) II на ті самі гроші купує засоби продукції в І, що в нього таким
чином відновлюється в грошовій формі змінний капітал в 1000\pound{ ф. стерл.};
4) II купує на 500\pound{ ф. стерл.} засоби продукції у І;
5) І купує на ці самі 500\pound{ ф. стерл.} засоби споживання у II;
6) II купує на ті самі 500\pound{ ф. стерл.} засоби продукції у І;
7) І купує на ті самі 500\pound{ ф. стерл.} засоби
існування у II.~До II повернулись назад 500\pound{ ф. стерл.}, що їх він подав у
циркуляцію понад 2000\pound{ ф. стерл.} у своєму товарі й що за них він не
вилучив з циркуляції жодного еквіваленту в товарі\footnote{
Тут виклад трохи відхиляється від вище поданого (стор. 306--307). Там і
І підрозділ подав в циркуляцію додаткову суму в 500. Тут тільки II підрозділ дає
додатковий грошовий матеріял для циркуляції. Однак, це нічого не змінює в
кінцевому наслідку. — \emph{Ф.~Е.}}.

Отже, обмін відбувається так:

1) І платить 1000\pound{ ф. стерл.} грішми за робочу силу, отже, за товар \deq{} 1000\pound{ ф. стерл}.

2) Робітники на свою заробітну плату в сумі 1000\pound{ ф. стерл.} грішми
купують засоби споживання в II; отже, товар \deq{} 1000\pound{ ф. стерл}.

3) II на вторговані від робітників 1000\pound{ ф. стерл.} купує в І засоби
продукції такої ж вартости; отже, товар \deq{} 1000\pound{ ф. стерл}.

В наслідок цього до І повернулись 1000\pound{ ф. стерл.} в грошах як грошова
форма змінного капіталу.

4) II купує в І на 500\pound{ ф. стерл.} засоби продукції, тобто товар \deq{}
500\pound{ ф. стерл}.

5) І купує на ці самі 500\pound{ ф. стерл.} засоби споживання у II; отже,
товар \deq{} 500\pound{ ф. стерл}.

6) II купує на ці самі 500\pound{ ф. стерл.} засоби продукції в І, отже,
товар \deq{} 500\pound{ ф. стерл}.

7) І купує на ті самі 500\pound{ ф. стерл.} засоби споживання в II; отже,
товар \deq{} 500\pound{ ф. стерл}.

Сума обмінених товарових вартостей \deq{} 5000\pound{ ф. стерл}.

500\pound{ ф. стерл.}, що їх II підрозділ авансував на купівлю, повернулись
до нього назад.

Результат такий:

1) І підрозділ має змінний капітал в грошовій формі величиною в 1000\pound{ ф.
стерл.}, що їх він первісно авансував для циркуляції; крім того, він витратив
на своє особисте споживання 1000\pound{ ф. стерл.} у своєму власному
товаровому продукті; тобто витратив ті гроші, що їх він одержав від продажу
засобів продукції вартістю в 1000\pound{ ф. стерл.}.


\index{ii}{0321}  %% посилання на сторінку оригінального видання
З другого боку, та натуральна форма, що на неї мусить перетворитись
змінний капітал, який існує в грошовій формі, — тобто робоча сила, —
в наслідок споживання зберігається, репродукується і знову є наявна,
як той єдиний предмет торговлі її посідачів, що його вони мусять продавати,
коли хочуть жити. Отже, репродукується й відношення між найманими
робітниками й капіталістами.

2) Сталий капітал II заміщено in natura, і 500\pound{ ф. стерл.}, авансовані
цим підрозділом II для циркуляції, повернулися до нього назад.

Для робітників І циркуляція є проста циркуляція $Т — Г — Т$. $T_1$ (робоча
сила) — $Г_2$ (1000\pound{ ф. стерл.}, грошова форма змінного капіталу І) — $T_3$
(доконечні засоби існування в сумі 1000\pound{ ф. стерл.}); ці 1000\pound{ ф. стерл.}
перетворюють на гроші на таку саму величину вартости сталий капітал II,
який існує у формі товару — засобів існування.

Для капіталістів II цей процес є $Т — Г$, перетворення частини їхнього
товарового продукту на грошову форму, що з неї він перетворюється
знову на елементи продуктивного капіталу, а саме на частину потрібних
цим капіталістам засобів продукції.

Авансуючи $Г$ (500\pound{ ф. стерл.}) на закуп другої частини засобів продукції,
капіталісти II антиципують грошову форму тієї частини II~$с$, яка
ще існує в товаровій формі (у формі засобів споживання); в акті $Г — Т$,
коли II купує за $Г$, а І продає $Т$, гроші (II) перетворюються на частину
продуктивного капіталу, тимчасом як $Т$ (І) пророблює акт $Т — Г$, перетворюється
на гроші; але ці гроші репрезентують для І не складову частину
капітальної вартости, а перетворену на гроші додаткову вартість, що її
витрачається лише на засоби споживання.

В циркуляції $Г — Т\dots{} П\dots{} Т' — Г'$ перший акт, $Г — Т$, є акт одного капіталіста,
останній, $Т — Г'$, є акт (або частина акту) іншого. Чи репрезентує
це $Т$, що за допомогою його $Г$ перетворюється на продуктивний капітал,
для продавця $Т$ (який, отже, перетворює це $Т$ на гроші) складову частину
сталого капіталу, чи складову частину змінного капіталу, чи додаткову
вартість, це не має жодного значення для самої товарової циркуляції.

Що стосується до кляси І, щодо складової частини $v \dplus{} m$ її товарового
продукту, то ця кляса вилучила з циркуляції більше грошей, ніж подала
в неї. Поперше, до неї повертаються 1000\pound{ ф. стерл.} її змінного капіталу;
подруге, вона продає (див. вище, обмін під № 4) засобів продукції
на 500\pound{ ф. стерл.}, у наслідок цього перетворюється на гроші половина
її додаткової вартости; потім (обмін під №~6) вона знову продає засобів
продукції на 500\pound{ ф. стерл.}, другу половину своєї додаткової вартости,
і в наслідок цього всю додаткову вартість вилучено з циркуляції в грошовій
формі. Отже, маємо послідовно: 1) змінний капітал перетворюється
знову на гроші \deq{} 1000\pound{ ф. стерл.}; 2) половина додаткової вартости перетворюється
на гроші \deq{} 500\pound{ ф. стерл.}; 3) друга половина додаткової
вартости \deq{} 500\pound{ ф. стерл.}; отже, перетворено на гроші суму
$1000 v \dplus{} 1000 m$ \deq{} 2000\pound{ ф. стерл}. Хоч І (лишаючи осторонь обміни,
що їх розглянеться потім, і що упосереднюють репродукцію І~$с$) подав
\parbreak{}  %% абзац продовжується на наступній сторінці

\parcont{}  %% абзац починається на попередній сторінці
\index{iii1}{0322}  %% посилання на сторінку оригінального видання
становлять споконвічну інтегральну частину цієї єдності промислово-землеробського
виробництва, і таким чином розривають
общину. Навіть тут це діло розкладу вдається їм тільки дуже
повільно. Ще менше воно вдається їм у Китаї, де безпосередня
політична влада не дає допомоги. Велика економія і заощадження
часу, які виникають з безпосереднього поєднання землеробства
і мануфактури, чинять тут якнайупертіший опір продуктам
великої промисловості, в ціну яких входять faux frais [непродуктивні
витрати] процесу циркуляції, який повсюди пронизує
їх. В протилежність до англійської, російська торгівля, навпаки,
лишає непорушною економічну основу азіатського виробництва\footnote{
З того часу, як Росія робить конвульсивні зусилля, щоб розвинути
власне капіталістичне виробництво, розраховане виключно на внутрішній та
на прикордонний азіатський ринок, це теж починає мінятися. — \emph{Ф.~Е.}
}.

Перехід від феодального способу виробництва відбувається
двояким чином. Виробник стає купцем і капіталістом протилежно
до землеробського натурального господарства і зв’язаного цехами
ремесла середньовічної міської промисловості. Це — дійсно
революціонізуючий шлях. Абож купець безпосередньо підпорядковує
собі виробництво. Як би дуже не впливав цей останній
шлях історично як перехідний ступінь — як, наприклад, англійський
clothier [сукняр] XVII століття, який ставить під свій контроль
ткачів, що все ж лишаються самостійними, продає їм вовну
та скуповує в них сукно, — однак, сам по собі він не приводить
до перевороту в старому способі виробництва, який він скоріше
консервує і зберігає як свою передумову. Так, наприклад, ще
до половини цього століття фабрикант у французькій шовковій
промисловості, в англійській панчішній та мережівній промисловості
здебільшого був фабрикантом тільки номінально, в дійсності
він був простим купцем, який полишав ткачам працювати
і далі їх старим роздрібненим способом і який панував над ними
тільки як купець, на якого вони фактично працювали\footnote{
Те саме стосується і до рейнської стрічкової і позументної промисловості
та шовкоткацтва. Біля Крефельда була навіть збудована спеціальна залізниця
для зносин цих сільських ручних ткачів з міськими „фабрикантами“, але
з того часу механічне ткацтво привело до бездіяльності цієї залізниці разом
з ручними ткачами. — \emph{Ф.~Е.}
}. Такі
відносини повсюди являють собою перешкоду для дійсного капіталістичного
способу виробництва і гинуть в міру розвитку останнього.
Не роблячи перевороту в способі виробництва, вони тільки
погіршують становище безпосередніх виробників, перетворюють
їх у простих найманих робітників і пролетарів при гірших умовах,
ніж для робітників, безпосередньо підпорядкованих капіталові,
і привласнення їх додаткової праці відбувається тут на базі старого
способу виробництва. Такі самі відносини, тільки дещо модифіковані,
існують у частині лондонського виробництва меблів,
яке провадиться ремісницьким способом. В Tower Hamlets воно
провадиться дуже широко. Все виробництво поділене на дуже
\parbreak{}  %% абзац продовжується на наступній сторінці

\parcont{}  %% абзац починається на попередній сторінці
\index{ii}{0323}  %% посилання на сторінку оригінального видання
вартости І відбувається лише через продаж товарів І~$m$, що в них
міститься ця додаткова вартість, і перебування додаткової вартости у
формі грошей триває кожного разу лише доти, доки гроші, вторговані
від продажу товару, не витратиться знову на засоби споживання.

Підрозділ І на додаткові гроші (500\pound{ ф. стерл.}) купує в II засоби
споживання; ці гроші І витратив і одержав за них еквівалент в товарах II;
першого разу гроші припливають назад тому, що II купує в І товару
на 500\pound{ ф. стерл.}; отже, вони повертаються як еквівалент товару, проданого
цим І, але цей товар нічого не коштує для І, отже, становить додаткову
вартість для І, і таким чином \emph{гроші, що їх він сам
подає в циркуляцію, перетворюють на гроші його
власну додаткову вартість}; так само при своїй другій купівлі
(№ 6) І одержує еквівалент в товарах II.~Коли припустити, що II не
купує в І засобів продукції (№ 7), то І справді заплатив би за 1000\pound{ ф.
стерл.} засобів споживання — спожив би всю свою додаткову вартість як
дохід, — а саме 500 своїми товарами І (засобами продукції) і 500 грішми;
при цьому в нього не лишилось би на складах 500\pound{ ф. стерл.} у товарах І
(в засобах продукції), і він витратив би грішми 500\pound{ ф. стерл}.

Навпаки, II перетворив би лише три чверті свого сталого капіталу
з форми товарового капіталу знову на продуктивний капітал; а четверту
частину — на форму грошового капіталу (500\pound{ ф. стерл.}), і в дійсності на
гроші, що лежать без діла, або на гроші, що припинили свою функцію
й вичікують. Коли б такий стан тривав довго, то II мусів би скоротити
на одну чверть маштаб репродукції. Але ті 500 в засобах продукції,
які лишаються на шиї в І, не є додаткова вартість, що існує в товаровій
формі; вони з’явились замість авансованих 500\pound{ ф. стерл.} грішми, що І мав
поряд своїх 1000\pound{ ф. стерл.} додаткової вартости в товаровій формі. Як
гроші вони перебувають у формі, що в ній їх завжди можна реалізувати;
як товар їх у даний момент не сила продати. Відси ясно, що проста
репродукція — а за неї кожен елемент продуктивного капіталу мусить бути
заміщений так в II, як і в І, — тут можлива й далі тільки тоді, коли 500 золотих
птахів повернуться до того під розділу І, що спочатку випустив їх.

Коли капіталіст (тут ми все ще маємо перед собою промислового капіталіста,
що є разом з тим представник усіх інших) витратить гроші на
засоби споживання, то ці гроші для нього остаточно зникли, вони пішли
шляхом усього живого. Коли вони знову повертаються до нього, то це
може постати лише в тому разі, якщо він з циркуляції виловить їх за допомогою
товарів, тобто за допомогою свого товарового капіталу. Так само
як вартість його цілого річного товарового продукту (а він для нього \deq{} товаровому
капіталові), так і вартість кожного елемента цього останнього, тобто
вартість кожного поодинокого товару, розпадається для нього на сталу капітальну
вартість, змінну капітальну вартість і додаткову вартість. Отже,
перетворення на гроші одиниці з товарів (що з них як з елементів складається
товаровий продукт) є разом з тим перетворення на гроші певної частини
додаткової вартости, яка міститься в цілому товаровому продукті. Отже,
для даного випадку цілком правильно, що капіталіст сам подав у циркуляцію
\index{ii}{0324}  %% посилання на сторінку оригінального видання
ті гроші — а саме, витрачаючи їх на засоби споживання — що
ними перетворюється на гроші або, інакше кажучи, реалізується його
додаткова вартість. Звичайно, справа тут не в тих самих монетах, а
в сумі дзвінкої монети, рівній тій сумі (або рівній частині тієї суми),
що її він подав у циркуляцію на задоволення особистих потреб.

На практиці це стається двома способами: коли підприємство відкрито
лише поточного року, то мине чимало часу, в кращому разі кілька місяців,
перш ніж капіталіст матиме змогу витрачати на своє особисте споживання
гроші з доходів самого підприємства. Але через це він ні на хвилину
не відкладає свого споживання. „Він сам собі авансує (чи з своєї
власної кишені, чи з чужої в кредит, тут ця обставина зовсім не має
значення) гроші під додаткову вартість, яку він іще лише має здобути;
але цим самим він авансує і засоби циркуляції для реалізації додаткової
вартости, що її треба буде реалізувати пізніше. Навпаки, коли підприємство
вже давно йде правильним ходом, то виплати й надходження розподіляються
на різні строки протягом року. Що відбувається безперервно,
так це споживання капіталіста, яке антиципується і своїми розмірами
розраховується в певній пропорції до звичайних або передбачуваних надходжень.
В продажі кожної партії товару реалізується й частину додаткової
вартости, що її треба видобути протягом року. Але коли б протягом
цілого року продали лише стільки спродукованого товару, скільки
треба для заміщення сталої й змінної капітальної вартости, що є в ньому,
або коли б ціни спали так, що, продавши ввесь річний товаровий продукт,
можна було б реалізувати лише авансовану капітальну вартість, що
міститься в ньому, то у витрачанні грошей виразно виступило б антиципування,
надія на майбутню вартість. Коли наш капіталіст збанкротує,
то його кредитори й суд досліджуватимуть, чи були його антициповані
особисті витрати в правильному відношенні до розмірів його підприємства
й до надходжень додаткової вартости, що звично або нормально відповідають
цим розмірам.

Але коли ми візьмемо цілу клясу капіталістів, то теза, що вона сама
мусить подати в циркуляцію гроші для реалізації своєї додаткової вартости
(зглядно й для циркуляції свого капіталу, сталого й змінного), не
лише не є парадоксальна, але є неодмінна умова цілого механізму; тут
бо є лише дві кляси: робітнича кляса, що тільки й має свою робочу силу,
і кляса капіталістів, що в її монопольному володінні є засоби суспільної
продукції, так само, як і гроші. Парадокс був би тоді, коли б робітнича
кляса з самого початку авансувала з власних коштів гроші, потрібні для
реалізації додаткової вартости, що міститься в товарах. Але поодинокий
капіталіст робить це авансування завжди лише в такій формі, що він діє
як покупець, \emph{витрачає} гроші на закуп засобів споживання, або
\emph{авансує} гроші на закуп елементів свого продуктивного капіталу, чи то
робочої сили, чи то засобів продукції. Він завжди віддає гроші лише
за еквівалент. Гроші він авансує циркуляції лише таким самим способом,
яким авансує їй товари. І в тому, і в цьому разі він діє як вихідний пункт
циркуляції товарів та грошей.


\index{ii}{0325}  %% посилання на сторінку оригінального видання
Справжній перебіг справи затемнюється двома обставинами:

1) Поява \emph{торговельного капіталу} (що його першою формою
завжди є гроші, бо торговець як такий не виготовляє жодного
„продукту“ або „товару“) і \emph{грошового капіталу}, як предмету
операцій особливої відміни капіталістів, у процесі циркуляції промислового
капіталу.

2) Розпад додаткової вартости — а вона насамперед завжди мусить
потрапляти до рук промислового капіталіста — на різні категорії, що їхніми
представниками, поряд промислового капіталіста, виступають землевласник
(для земельної ренти), лихвар (для проценту) і~\abbr{т. ін.}, потім уряд з своїми
урядовцями, рантьє і~\abbr{т. ін.} Ці молодці виступають проти промислового
капіталіста як покупці й остільки як перетворювачі його товарів на гроші;
pro parte\footnote*{
Відповідно до участи своєї. \emph{Ред.}
} і вони подають „гроші“ в циркуляцію, а капіталіст одержує
ці гроші від них. При цьому завжди забувають, з якого джерела вони
первісно одержали й знову та знову одержують гроші.
\label{original-325-1}

\bigskip{}
\subsection[Сталий капітал підрозділу I]{Сталий капітал підрозділу I\footnotemark{}}

\label{original-325-2}
Нам%
\footnotetext{Відси з рукопису II. \emph{Ф.~Е.}}
лишається ще дослідити сталий капітал підрозділу І \deq{} 4000 І~$с$.
Ця вартість дорівнює вартості зужиткованих на продукцію цієї товарової
маси засобів продукції, що знову з’являється в товаровому продукті І.~Ця новоз’явлена вартість, спродукована не в продукційному процесі І, а
яка на рік раніше ввійшла в нього як стала вартість, як дана вартість
його засобів продукції, існує тепер у формі всієї частини товарової маси І,
що не ввібрана категорією II; вартість цієї товарової маси, що лишається
таким чином у руках капіталістів І, \deq{} \sfrac{2}{3} вартости всього їхнього річного
товарового продукту. Про поодинокого капіталіста, що продукує певну
особливу відміну засобів продукції, ми могли б сказати: він продає свій
товаровий продукт, перетворює його на гроші. Перетворюючи його на
гроші, він зворотно перетворює на гроші й сталу частину вартости свого
продукту. На цю частину вартости, перетворену на гроші, він потім знову
купує собі в інших продавців товарів засоби продукції, або перетворює
сталу частину вартости свого продукту на ту натуральну форму, що
в ній вона може знову функціонувати як продуктивний сталий капітал.
Навпаки, в нашому випадку таке припущення неможливе. Кляса капіталістів
І охоплює всю сукупність капіталістів, що продукують засоби продукції.
Крім того, товаровий продукт в 4000, що лишився в їхніх руках,
є та частина суспільного продукту, що її не можна обміняти на жодну
іншу, бо вже не існує жодної такої іншої частини річного продукту. За
винятком цих 4000 увесь лишок уже приміщено; частину ввібрано суспільним
споживним фондом, а друга частина повинна замістити сталий
капітал підрозділу II, що обміняв уже все, що було в його розпорядженні
для обміну з підрозділом І.
\parbreak{}  %% абзац продовжується на наступній сторінці

\parcont{}  %% абзац починається на попередній сторінці
\index{i}{0326}  %% посилання на сторінку оригінального видання
фабрикант домагається таких хлопчаків, які мають вигляд, наче
їм уже минуло 13 років. Раптове — іноді — зменшення числа
дітей, молодших за 13 років, експлуатованих фабрикантами, яке
так дивує в англійській статистиці за останні 20 років, було, за
словами самих фабричних інспекторів, здебільша справою тих
certifying surgeons, що перекручували вік дітей відповідно до
експлуататорської жадоби капіталістів та баришницьких потреб
батьків. В Bethnal Green, цій ославленій окрузі Лондону, щопонеділка
й щовівтірка влаштовують одкритий базар, на якому
діти обох статей, починаючи від 9-літнього віку, сами себе наймають
на лондонські шовкові мануфактури. «Звичайні умови —
1\shil{ шилінґ} 8\pens{ пенсів} на тиждень (це належить батькам) і 2\pens{ пенси} —
для мене самого, та чай». Контракти мають силу тільки на тиждень.
Сцени й мова підчас цього базару справді обурливі\footnote{
«Children’s Employment Commission. 5 th Report», London 1866,
p. 81. n.31. [До 4 видання. Шовкову промисловість у Bethnal Green тепер
майже знищено. — \emph{Ф.~Е.}].
}. Ще й досі
трапляється в Англії, що жінки беруть «хлопчаків із робітного
дому й наймають їх якомубудь покупцеві за 2\shil{ шилінґи}
6\pens{ пенсів} на тиждень»\footnote{
«Children’s Employment Commission. 3 rd Report», London 1864,
p. 53, n. 15.
}. Всупереч законодавству ще й досі щонайменше
\num{2.000} хлопчаків продається у Великобрітанії їхніми
власними батьками як живі сажотрусні машини (дарма що існують
машини для заміни їх)\footnote{
Там же, 5 th Report, p. XXIII, n. 137.
}. Зумовлена машинами революція у
правних відносинах між покупцем та продавцем робочої сили,
позбавивши всю цю угоду навіть подоби контракту між вільними
особами, дала пізніш англійському парляментові юридичну
підставу виправдуватися за втручання держави у фабричний
режим. Щоразу, коли фабричний закон обмежує дитячу працю
в нереґляментованих досі галузях промисловости 6 годинами,
знову й знову лунає голосіння фабрикантів: частина батьків,
мовляв, забирають своїх дітей із реґляментованих тепер галузей
промисловости, щоб запродати їх на такі, де ще панує «воля
праці», тобто, де діти, молодші за 13 років, примушені працювати
як дорослі, отже, на такі галузі, де за них можна дорожче
взяти. А що капітал із своєї природи левелер, тобто вимагає,
як свого природженого права, рівности в умовах експлуатації
праці в усіх сферах продукції, то й законодавче обмеження дитячої
праці в одній галузі промисловости стає причиною обмеження
її в іншій галузі.

\looseness=-1
Ми вже раніш відзначали фізичний занепад дітей, підлітків
і жінок робітників, що їх машини кидають на експлуатацію капіталу
спочатку безпосередньо по фабриках, що виростають на
основі машин, а потім посередньо, в усіх інших галузях промисловости.
Тому ми тут спинимося лише на одному пункті, на жахливій
смертності робітничих дітей у перші роки їхнього життя.
\parbreak{}  %% абзац продовжується на наступній сторінці

\input{ii/_0327.tex}

\index{iii1}{0328}  %% посилання на сторінку оригінального видання
В русі торговельного капіталу $Г — Т — Г'$ той самий товар
двічі або, коли купець продає купцеві, багато разів переходить
з рук в руки; але кожна така переміна місця одного й того ж
товару вказує на метаморфозу, купівлю або продаж товару,
як би часто не повторювався цей процес, поки товар остаточно
не ввійде в споживання.

З другого боку, в $Т — Г — Т$ відбувається двократна переміна
місця одних і тих самих грошей, але це вказує на повну
метаморфозу товару, який спочатку перетворюється в гроші,
а потім з грошей знову в інший товар.

Навпаки, при капіталі, що дає процент, перша переміна місця
$Г$ аж ніяк не є моментом ні в метаморфозі товару, ні в репродукції
капіталу. Таким моментом вона стає лиш при другому
витрачанні, в руках функціонуючого капіталіста, який з цими
грішми провадить торгівлю або перетворює їх у продуктивний
капітал. Перша переміна місця $Г$ не виражає тут нічого іншого,
як уступку або передачу їх від $А$ до $В$; уступку, яка звичайно
відбувається в певних юридичних формах і на певних умовах.

Цьому двократному витрачанню грошей як капіталу, при
чому перше витрачання є проста передача їх від $А$ до $В$, відповідає
і двократний зворотний приплив їх. Як $Г'$ або $Г + ΔГ$ вони
повертаються з руху назад до функціонуючого капіталіста $В$. Цей останній тоді знову передає їх $А$, але вже з частиною
зиску, як реалізований капітал, як $Г + ΔГ$, де $ΔГ$ становить
не весь зиск, а тільки частину зиску, процент. До $В$ вони повертаються
тільки як те, що він витратив як функціонуючий
капітал, але як власність $А$. Тому, щоб зворотний приплив їх
закінчився, $В$ повинен знову передати їх $А$. Але, крім капітальної
суми, $В$ повинен передати $А$ під назвою процента частину
зиску, який він виробив за допомогою цієї капітальної суми,
бо $А$ дав йому гроші тільки як капітал, тобто як вартість, яка
не тільки зберігається в русі, але й створює своєму власникові
додаткову вартість. Вони лишаються в руках $В$ тільки доти,
поки вони є функціонуючий капітал. А після їх зворотного
припливу — після скінчення строку — вони перестають функціонувати
як капітал. Але як гроші, які вже більше не функціонують
як капітал, вони мусять бути знову передані назад до $А$,
який не перестав бути їх юридичним власником.

Форма позики замість форми продажу, яка властива цьому
товарові, капіталові як товарові, — вона, зрештою, трапляється
й при інших угодах, — випливає вже з того визначення, що
капітал виступає тут як товар, або що гроші як капітал стають
товаром.

Тут необхідно розрізняти таке:

Ми бачили (книга II, розд. І) і коротко нагадуємо тут про те,
що капітал у процесі циркуляції функціонує як товарний капітал
і як грошовий капітал. Але в обох формах капітал стає
товаром не як капітал.

\parcont{}  %% абзац починається на попередній сторінці
\index{ii}{0329}  %% посилання на сторінку оригінального видання
капітальну вартість, тобто ту частину вартости, що на неї робітник купує
засоби репродукції самого себе, і додаткову вартість, що її капіталіст
може витрачати на своє власне особисте споживання, — все ж з суспільного
погляду, частину суспільного робочого дня витрачається виключно на
продукцію свіжого сталого капіталу, а саме — продуктів, призначених
виключно для того, щоб функціонувати в процесі праці як засоби
продукції, отже, і для того, щоб функціонувати як сталий капітал у
процесі зростання вартости, що ним супроводиться цей процес праці.
Згідно з нашим припущенням, ввесь суспільний робочий день виражається в
грошовій вартості в 3000, з чого лише \sfrac{1}{3} \deq{} 1000 продукується в підрозділі
II, який продукує засоби споживання, тобто товари, що в них, кінець-кінцем,
реалізується вся змінна капітальна вартість і вся додаткова вартість
суспільства. Отже, згідно з цим припущенням, \sfrac{2}{3} суспільного робочого
дня вживається на продукцію нового сталого капіталу. Хоч з погляду
індивідуальних капіталістів і робітників підрозділу І ці \sfrac{2}{3} суспільного
робочого дня служать лише для продукції змінної капітальної вартости
плюс додаткова вартість цілком так само, як і остання третина суспільного
робочого дня в підрозділі II, однак з суспільного погляду — а
також розглядувані щодо споживної вартости продукту — ці \sfrac{2}{3} суспільного
робочого дня продукують лише заміщення сталого капіталу, який
перебуває або вже зужиткований в процесі продуктивного споживання.
Також з індивідуального погляду, хоч ціла вартість, що її продукують ці
\sfrac{2}{3} робочого дня, дорівнює лише змінній капітальній вартості плюс додаткова
вартість для її продуцентів, однак вони зовсім не продукують
споживних вартостей такого роду, щоб на них можна було витрачати
заробітну плату або додаткову вартість; продукт цих \sfrac{2}{3} робочого дня є
засоби продукції.

Насамперед треба зазначити, що жодна частина суспільного робочого
дня ні в І, ні в II не служить для продукції вартости сталого капіталу,
застосованого в цих двох великих сферах продукції та діющого в них.
Вони продукують лише новододавану вартість, 2000 І ($v \dplus{} m$) \dplus{} 1000
II ($v \dplus{} m$), додатково до сталої капітальної вартости \deq{} 4000 І $c$ \dplus{} 2000
II~$c$. Нова вартість, спродукована в формі засобів продукції, ще не є
сталий капітал. Вона має лише призначення функціонувати як сталий
капітал в майбутньому.

Цілий продукт підрозділу II — засоби споживання — розглядуваний щодо
його споживної вартости, в його конкретній натуральній формі, є продукт,
спродукований в підрозділі II однією третиною суспільного дня;
це є продукт праці в її конкретних формах, праця ткача, праця пекаря
і~\abbr{т. ін.}, що застосовується в цьому підрозділі, продукт цієї праці, оскільки
вона функціонує як суб’єктивний елемент процесу праці. Навпаки,
щодо сталої частини вартости цього продукту II, то вона лише знову
з’являється в новій споживній вартості, в новій натуральній формі, у
формі засобів споживання, тимчасом як раніше вона існувала в формі засобів
продукції. В наслідок процесу праці вартість цієї частини перенесено
з її попередньої натуральної форми на її нову натуральну форму. Але
\parbreak{}  %% абзац продовжується на наступній сторінці

\parcont{}  %% абзац починається на попередній сторінці
\index{ii}{0330}  %% посилання на сторінку оригінального видання
вартість цих \sfrac{2}{3} вартости річного продукту \deq{} 2000 спродуковано не
в процесі зростання вартости II протягом поточного року.

Подібно до того, як розглядуваний з погляду процесу праці продукт
II є результат новодіющої живої праці та даних припущених при цьому
засобів продукції, що в них, як у своїх речових умовах, ця праця здійснюється,
— цілком так само з погляду процесу зростання вартости вартість
продукту II \deq{} 3000 складається з нової вартости, спродукованої новодолученою
\sfrac{1}{3} суспільного робочого дня ($500 v \dplus{} 500m$ \deq{} 1000) і з сталої
вартости, що в ній зречевлено \sfrac{2}{3} минулого суспільного робочого дня,
що минув до початку розглядуваного тут продукційного процесу II.~Ця
частина вартости продукту II виражається в частині самого продукту.
Вона існує в певній масі засобів споживання вартістю в 2000 \deq{} \sfrac{2}{3} суспільного
робочого дня. Це — та нова споживна форма, що в ній знову
з’являється ця частина вартости. Отже, обмін частини засобів споживання \deq{}
2000 ІІ $с$ на засоби продукції І \deq{} І ($1000 v \dplus{} 1000 m$) в дійсності є обмін
\sfrac{2}{3} цілого робочого дня, що не мають жодної частини праці цього року
й минули до початку цього року, на \sfrac{2}{3} робочого дня новододаного протягом
цього року. \sfrac{2}{3} суспільного робочого дня цього року не могли б
бути вжиті на продукцію сталого капіталу і разом з тим становити змінну
капітальну вартість плюс додаткова вартість для продуцентів цього капіталу,
коли б їх не обмінювалося з тією частиною вартости щорічно
споживаних засобів споживання, що в них міститься \sfrac{2}{3} робочого дня,
витраченого й реалізованого, до цього року, не на протязі цього
року. Це — обмін \sfrac{2}{3} робочого дня цього року на \sfrac{2}{3} робочого дня,
витрачені до цього року, обмін між робочим часом цього року й
торішнім робочим часом. Отже, це пояснює нам загадку, чому вартість,
новоутворена протягом цілого суспільного робочого дня, може розкластись
на змінну капітальну вартість плюс додаткова вартість, хоч
\sfrac{2}{3} цього робочого дня витрачається не на продукцію речей, що в
них міг би реалізуватись змінний капітал або додаткова вартість, а
на продукцію засобів продукції, які заміщують капітал, зужиткований
протягом року. Це пояснюється просто тим, що ті \sfrac{2}{3} вартости продукту
II, що в ньому капіталісти й робітники І реалізують спродуковану
ними змінну капітальну вартість плюс додаткова вартість (а вони
становлять разом \sfrac{2}{3} цілої вартости річного продукту), являють, розглядувані
щодо вартости, продукт \sfrac{2}{3} суспільного робочого дня, що
минув до цього року.

Хоч сума суспільного продукту І і II, засоби продукції та засоби
споживання, розглядувані конкретно з погляду їхньої споживної вартости,
їхньої натуральної форми є продукт праці поточного року, однак це має
силу лише остільки, оскільки цю працю розглядається як корисну конкретну
працю, а не як витрату робочої сили, як вартостетворчу працю.
А проте, навіть продуктом праці поточного року вони є в тому лише розумінні,
що засоби продукції перетворилися на новий продукт, на продукт
цього року тільки за допомогою долученої до них живої праці, яка орудувала
ними. Навпаки, праця поточного року не могла б перетворитись на продукт без
\parbreak{}  %% абзац продовжується на наступній сторінці

\parcont{}  %% абзац починається на попередній сторінці
\index{ii}{0331}  %% посилання на сторінку оригінального видання
засобів продукції, що не залежать від неї, без засобів праці та матеріялів
продукції.

\subsection{Сталий капітал в обох підрозділах}

Щодо цілої вартости продукту в 9000 і тих категорій, що на них
вона розпадається, то аналіза її не являє більших труднощів, ніж аналіза
вартости продукту поодинокого капіталу; вона навіть тотожня з нею.

В цілому річному суспільному продукті тут містяться три річні суспільні
робочі дні. Вираз вартости кожного з цих робочіх днів \deq{} 3000; тому
вираз вартости цілого продукту 3000 × 3 \deq{} 9000.

Далі до початку того однорічного продукційного процесу, що його
продукт ми аналізуємо, минуло: в І підрозділі \sfrac{4}{3} робочого дня (новоутворена
вартість в 4000) і в підрозділі II \sfrac{2}{3} робочого дня (новоутворена
вартість в 2000). Разом 2 суспільні робочі дні, що спродукували
нову вартість \deq{} 6000. Тому 4000 І~$c \dplus{} 2000$ II~$c \deq{} 6000 c$ фігурують
як вартість засобів продукції, або стала капітальна вартість, що
знову являється в цілій вартості суспільного продукту.

Далі в підрозділі І з новодолученого річного робочого дня маємо \sfrac{1}{3}
доконечної праці або праці, яка заміщує вартість змінного капіталу 1000
І~$v$ і оплачує ціну праці, застосованої в І.~Так само в II підрозділі \sfrac{1}{6} суспільного
робочого дня є доконечна праця з вартістю в 500. Отже, 1000
I~$v \dplus{} 500$ II~$v \deq{} 1500 v$, вираз вартости половини суспільного робочого
дня, є вираз вартости тієї першої половини цілого долученого в поточному
році робочого дня, яка складається з доконечної праці.

Нарешті, в І підрозділі \sfrac{1}{3} цілого робочого дня новоутворена вартість \deq{}
1000, є додаткова праця; в II підрозділі \sfrac{1}{6} робочого дня, новоспродукована
вартість \deq{} 500, є додаткова праця; разом вони складають другу
половину цілого новодолученого робочого дня. Тому вся спродукована
додаткова вартість \deq{} 1000 І~$m \dplus{} 500$ II~$m \deq{} 1500 m$.

Отже:

Стала капітальна частина вартости суспільного продукту ($с$):

2 робочі дні, витрачені до розглядуваного продукційного процесу;
вираз вартости \deq{} 6000.

Доконечна праця ($v$), витрачена протягом року:

Половина робочого дня, витраченого на річну продукцію; вираз вартости
— 1500.

Витрачена протягом року додаткова праця ($m$):

Половина робочого дня, витраченого на річну продукцію; вираз вартости
\deq{} 1500.

Вартість, новоспродукована річною працею $(v \dplus{} m) \deq{} 3000$.

Ціла вартість продукту $(с \dplus{} v \dplus{} m) \deq{} 9000$.

Отже, труднощі не в аналізі вартости самого суспільного продукту.
Вони постають при зіставленні складових частин вартости суспільного
продукту з його речовими складовими частинами.


\index{ii}{0332}  %% посилання на сторінку оригінального видання
Стала частина вартости, що лише знову з’являється, дорівнює вартості
тієї частини цього продукту, яка складається з засобів \emph{продукції};
вона втілюється в цій частині продукту.

Новоспродукована протягом року вартість \deq{} $v \dplus{} m$ дорівнює вартості
тієї частини цього продукту, яка складається з \emph{засобів споживання};
вона втілюється в цій частині продукту.

Але, лишаючи осторонь неважливі тут винятки, засоби продукції та
засоби споживання є цілком різні ґатунки товарів, продукти цілком різної
натуральної або споживної форми, отже, продукти цілком різних конкретних
відмін праці. Праця, яка застосовує машини, щоб продукувати
засоби існування, тут цілком відрізняється від праці, яка робить машини.
Здається, ніби ввесь річний робочий день, що його вираз вартости \deq{} 3000,
витрачено на продукцію засобів споживання \deq{} 3000, де не з’являється
знову жодної сталої частини вартости, бо ці $3000 \deq{} 1500 v \dplus{} 1500 m$
розкладаються лише на змінну капітальну вартість \dplus{} додаткова вартість. З
другого боку, стала капітальна вартість \deq{} 6000 знову з’являється в такому
ґатунку продуктів, що цілком відрізняється від засобів споживання,
в засобах продукції, і здається, ніби на продукцію цих нових продуктів
не витрачено жодної частини суспільного робочого дня; навпаки, здається,
ніби ввесь робочий день складається лише з таких ґатунків праці, що
результат їхній не засоби продукції, а тільки засоби споживання. Таємницю
вже розв’язано. Новоспродукована річною працею вартість дорівнює
вартості продукту підрозділу II, цілій вартості новоспродукованих засобів
споживання. Але ця вартість продукту на \sfrac{2}{3} більша, ніж частина річної
праці, витрачена в продукції засобів споживання (підрозділу II). На продукцію
їх витрачено лише \sfrac{1}{3} річної праці. \sfrac{2}{3} цієї річної праці витрачено
на продукцію засобів продукції, отже, в підрозділі І.~Утворена протягом
цього часу в І нова вартість, рівна спродукованій в І змінній капітальній
вартості плюс додаткова вартість дорівнює сталій капітальній вартості II,
що знову з’являється в підрозділі II в засобах споживання. Отже, ці величини
можуть обмінятись одна на одну, можуть in natura замістити одна одну.
Тому вся вартість засобів споживання II дорівнює сумі нової новоспродукованої
вартости І \dplus{} II, або II ($с$ \dplus{} $v \dplus{} m$) \deq{} І ($v \dplus{} m$) \dplus{} II ($v \dplus{} m$), отже, дорівнює
сумі нової вартости, спродукованої річною працею у формі $v \dplus{} m$.

З другого боку, вся вартість засобів продукції (І) дорівнює сумі сталої
капітальної вартости, яка знову з’являється в формі засобів продукції
(І) і в формі засобів споживання (II), отже, дорівнює сумі сталої
капітальної вартости, яка знову з’являється в цілому продукті суспільства.
Вся ця вартість дорівнює виразові вартости \sfrac{4}{3} минулого робочого дня,
витрачених у І підрозділі до початку продукційного процесу, і \sfrac{2}{3}
минулого робочого дня, витрачених в II до початку продукційного процесу,
отже, разом дорівнює виразові вартости двох цілих робочих днів.

Отже, в аналізі суспільного річного продукту труднощі постають тому,
що стала частина вартости виражається в продуктах цілком іншого ґатунку
— в засобах продукції — ніж долучена до цієї сталої частини
вартости нова вартість $v \dplus{} m$, яка виражається в формі засобів споживання.
\parbreak{}  %% абзац продовжується на наступній сторінці

\parcont{}  %% абзац починається на попередній сторінці
\index{ii}{0333}  %% посилання на сторінку оригінального видання
Таким чином, здається, ніби \sfrac{2}{3} зужиткованої маси продуктів — коли
розглядати з погляду вартости — знову з’являються в новій формі, як
новий продукт, що на його продукцію суспільство не витратило
жодної праці. Щодо поодинокого капіталу, то цього не буває. Кожен індивідуальний
капіталіст застосовує певний конкретний ґатунок праці, що перетворює
на продукт відповідні йому засоби продукції. Хай, напр., капіталіст
буде машинобудівник, витрачений протягом року сталий капітал $\deq{} 6000 с$,
змінний $\deq{} 1500 v$, додаткова вартість \deq{} $1500 m$; продукт \deq{} 9000; скажімо,
напр., що цей продукт — 18 машин, що з них кожна \deq{} 500. Ввесь продукт
існує тут в тій самій формі, у формі машин. (Коли машінобудівник продукує
кілька ґатунків машин, то для кожного складається окремий рахунок).
Ввесь товаровий продукт є продукт праці, витраченої протягом
року в машинобудівництві, комбінація того самого конкретного ґатунку
праці з тими самими засобами продукції. Тому різні частини вартости
продукту можна виразити в тій самій натуральній формі: в 12 машинах
міститься $6000 с$, в 3 машинах — $1500 v$, в 3 машинах — $1500 m$. Тут
очевидно, що вартість 12 машин дорівнює $6000 с$ не тому, що в цих
12 машинах втілено тільки працю, яка минула до машинобудівництва й
витрачена не на машинобудівництво. Не сама собою вартість засобів
продукції, потрібних на ці 18 машин, перетворилась на 12 машин, а вартість
цих 12 машин (яка сама складається з $4000 с \dplus{} 1000 v \dplus{} 1000 m$)
просто дорівнює цілій сталій капітальній вартості, що міститься в 18 машинах.
Тому машинобудівник мусить продати 12 з 18 машин, щоб замістити
витрачений ним сталий капітал, потрібний йому для репродукції 18
нових машин. Навпаки, справу не можна було б з’ясувати, коли б, хоч
застосовувана праця складається тільки з машин, результат її був: з
одного боку, 6 машин \deq{} $1500 v \dplus{} 1500 m$, з другого боку, залізо,
мідь, гвинти, паси й~\abbr{т. ін.}, на суму вартости в $6000 с$, тобто засоби продукції
машин в їхній натуральній формі, засоби продукції, що їх, як відомо,
поодинокий капіталіст, машинобудівник, сам не продукує, а мусить
заміщувати їх за допомогою процесу циркуляції. І однак на перший погляд
здається, ніби репродукція суспільного річного продукту відбувається
таким безглуздим способом.

Продукт індивідуального капіталу, тобто кожної самостійно діющої,
обдарованої власним життям частини суспільного капіталу, має якусь певну
натуральну форму. Єдина умова в тому, щоб він справді мав споживну
форму, споживну вартість, яка накладає на нього печать здатного до
циркуляції члена товарового світу. При цьому цілком байдужа й випадкова
та обставина, чи може він як засіб продукції знову ввійти в той
самий продукційний процес, що з нього він вийшов як продукт, отже, чи
має та частина вартости його продукту, що в ній виражається стала частина
капіталу, таку натуральну форму, в якій вона в дійсності може знову
функціонувати як сталий капітал. Коли цього немає, то ця частина вартости
продукту через продаж і купівлю знову перетворюєтьса на форму речових
елементів продукції цього продукту, і таким чином репродукується сталий
капітал у тій його натуральній формі, що в ній він може функціонувати.


\index{ii}{0334}  %% посилання на сторінку оригінального видання
Інша справа з продуктом сукупного суспільного капіталу. Всі речові
елементи репродукції мусять в своїй натуральній формі бути частинами
цього самого продукту. Зужитковану сталу частину капіталу можна замістити
за допомогою цілої продукції лише в тому разі, коли вся новоз’явлена
в продукті стала частина капіталу з’являється знову в натуральній
формі нових засобів продукції, що дійсно можуть функціонувати як сталий
капітал. Тому, — якщо ми припустимо просту репродукцію, — вартість
тієї частини продукту, яка складається з засобів продукції, мусить дорівнювати
сталій частині вартости суспільного капіталу.

Далі, коли розглядати справу з індивідуального погляду, то новодолученою
працею капіталіст продукує в вартості свого продукту лише свій змінний
капітал, плюс додаткова вартість, тимчасом як стала частина вартости
переноситься на продукт в наслідок конкретного характеру новодолученої
праці.

З суспільного погляду, та частина суспільного робочого дня, яка продукує
засоби продукції, отже, долучає до них нову вартість і переносить
на них вартість засобів продукції, зужиткованих на їх продукцію, не продукує
нічого іншого, крім нового сталого капіталу, призначеного
замістити сталий капітал, зужиткований в формі старих засобів продукції,
спожитий так в І, як і в II підрозділах. Ця частина продукує лише
продукт, що має ввійти в продуктивне споживання. Отже, ціла вартість
цього продукту є вартість, що може знову функціонувати лише як сталий
капітал, що на неї можна знову купити тільки сталий капітал у його
натуральній формі і що, отже, коли розглядати справу з суспільного погляду,
не розкладається ні на змінний капітал, ні на додаткову вартість.
— З другого боку, та частина суспільного робочого дня, яка продукує
засоби споживання, не продукує жодної частини для заміщення суспільного
капіталу. Вона продукує тільки продукти, що в їхній натуральній
формі призначені на те, щоб реалізувати вартість змінного капіталу
та додаткову вартість в І і II.

Кажучи про суспільний погляд, отже, розглядаючи ввесь суспільний
продукт, який включає так репродукцію суспільного капіталу, як і особисте
споживання, не треба вдаватись у маніру, запозичену Прудоном
у буржуазної економії, і розглядати справу так, ніби суспільство капіталістичного
способу продукції, розглядуване en bloc, як ціле, втрачає цей
свій специфічний, історично-економічний характер. Навпаки. При суспільному
погляді доводиться мати справу з колективним капіталістом. Увесь
капітал виступає як акційний капітал усіх поодиноких капіталістів, узятих
разом. І таке акційне товариство має те спільне з багатьма іншими
акційними товариствами, що кожен знає, що він вклав, та не знає, що
він одержить назад.

\subsection{Ретроспективний погляд на А.~Сміта, Шторха і Рамсая}

Сукупна вартість суспільного продукту становить 9000 \deq{} $6000 с \dplus{}
1500 v \dplus{} 1500 m$, інакше кажучи: 6000 репродукують вартість засобів
\parbreak{}  %% абзац продовжується на наступній сторінці

\parcont{}  %% абзац починається на попередній сторінці
\index{ii}{0335}  %% посилання на сторінку оригінального видання
продукції, а 3000 — вартість засобів споживання. Отже, вартість суспільного
доходу ($v \dplus{} m$) становить тільки \sfrac{1}{3} вартости сукупного продукту,
і сукупність споживачів — робітники і капіталісти — лише на суму вартости
цієї третини можуть брати з цілого суспільного продукту товари,
продукти, і заводити їх у фонд свого споживання. Навпаки, 6000 \deq{} \sfrac{2}{3}
вартости продукту є вартість сталого капіталу, що його треба замістити
in natura. Отже, засоби продукції на таку суму треба знову ввести в
продукційний фонд. Неминучість цього бачив уже Шторх, хоч і не міг
довести цього: „Очевидно, що вартість річного продукту розкладається
почасти на капітал, почасти на зиск, і що кожна з цих частин вартости річного
продукту реґулярно купує продукти, потрібні нації так для підтримання
свого капіталу, як і для відновлення свого споживного фонду\dots{} Продукти,
що становлять капітал нації, не можуть споживатись“\footnote*{
„Il est clair que la valeur du produit annuel se distribue partie en capitaux
et partie en profits, et que chacune de ces portions de la valeur du produit annuel
va régulièrement acheter les produits dont la nation a besoin, tant pour entretenir
son capital que pour renouveler son fond consommable\dots{} les produits qui constituent
le capital d’une nation, ne sont point consommables“.
}. (Storch:
„Considérations sur la nature du revenu national“. Paris. 1824, p. 150).

Однак А.~Сміт подав цю казкову догму, — а їй і досі йметься
віру — не тільки у тій вже вище згаданій формі, що згідно з нею сукупна
вартість суспільного продукту розкладається на дохід, на заробітну
плату плюс додаткова вартість, або — як він каже — на заробітну плату
плюс зиск (процент) плюс земельна рента. Він подав її ще в популярнішій
формі, ніби споживачі, кінець-кінцем (ultimately), мусять оплатити
продуцентам усю вартість продукту. Це й досі лишається
одним з на віру прийнятих загальників або навіть однією з вічних істин
для так званої науки політичної економії. Цю думку хочуть унаочнити таким
на позір правдоподібним способом. Візьмімо якийбудь предмет, напр.,
полотняні сорочки. Насамперед прядун лянної пряжі повинен оплатити
льонівникові всю вартість льону, тобто насіння, добрива, корму для робочої
худоби і~\abbr{т. ін.}, а також ту частину вартости, що її основний
капітал льонівника, як от будівлі, сільсько-господарський реманент і~\abbr{т. ін.},
„передає продуктові; заробітну плату, виплачену протягом продукції
льону; додаткову вартість (зиск, земельну ренту), яка міститься в льоні;
нарешті, витрати на перевіз льону від місця його продукції до прядільні.
Потім ткач повинен повернути прядунові лянної пряжі не лише цю
ціну льону, а й ту частину вартости машин, будівель тощо, коротко,
основного капіталу, що її перенесено на льон; далі, всі зужитковані
в процесі прядіння допоміжні матеріяли, заробітну плату прядунів, додаткову
вартість і~\abbr{т. ін.} — і так само далі стоїть справа з білильником, з витратами
на транспорт готового полотна, нарешті, з фабрикантом сорочок,
який оплатив усю ціну всіх попередніх продуцентів, які дали йому те,
що для нього є лише сировинний матеріял. В його руках далі відбувається
долучення нової вартости: почасти вартости сталого капіталу, зужиткованого
\index{ii}{0336}  %% посилання на сторінку оригінального видання
в формі засобів праці, допоміжних матеріялів і~\abbr{т. ін.} при фабрикації
сорочок, почасти в наслідок витраченої на цю фабрикацію праці,
що долучає вартість заробітної плати робітників, які роблять сорочки,
плюс додаткова вартість фабриканта сорочок. Припустімо, що ввесь цей
продукт — сорочки коштують, кінець-кінцем, 100\pound{ ф. стерл.}, і що це є
та частина всієї вартости річного продукту, яку суспільство витрачає на
сорочки. Споживачі сорочок оплачують 100\pound{ ф. стерл.}, отже, вартість усіх
засобів продукції, що є в сорочках, а також заробітну плату плюс додаткова
вартість льонівника, прядуна, ткача, білильника, фабриканта сорочок,
а також і всіх транспортерів. Це цілком слушно. Це така справа, що й
дитина зрозуміє її. Але потім сказано: так само стоїть справа й щодо
вартости всіх інших товарів. Треба було б сказати: так само стоїть справа
й щодо вартости всіх засобів споживання, щодо вартости тієї
частини суспільного продукту, яка входить у фонд споживання, отже, з
тією частиною вартости суспільного продукту, яку можна витратити як
дохід. Сума вартости всіх цих товарів справді дорівнює вартості всіх
зужиткованих на них засобів продукції (сталих частин капіталу) плюс
вартість, утворена працею, долученою востаннє (заробітна плата плюс
додаткова вартість). Отже, сукупність споживачів може оплатити всю цю
суму вартости, бо хоч вартість кожного окремого товару складається
з $c \dplus{} v \dplus{} m$, але сума вартости всіх товарів, що входять у фонд
споживання, разом узята, в максимальній величині, може дорівнювати лише
тій частині вартости суспільного продукту, яка розкладається на $v \dplus{} m$,
тобто може дорівнювати лише тій вартості, що її долучила витрачена
протягом року праця до вже наявних засобів продукції, до вартости сталого
капіталу. Але щодо сталої капітальної вартости, то ми бачили, що
її заміщується з маси суспільного продукту двояким способом. Поперше,
через обмін капіталістів II, які продукують засоби споживання, з капіталістами
І, які продукують засоби продукції. Тут і є джерело тієї фрази,
ніби те, що для одних є капітал, для інших є дохід. Але справа в дійсності
стоїть не так. Ті 2000 II $с$, що існують у засобах споживання вартістю
в 2000, становлять для кляси капіталістів II сталу капітальну вартість.
Отже, сами капіталісти II не можуть спожити цю вартість, хоч продукт
за його натуральною формою і призначено для споживання. З другого
боку, 2000 І ($v \dplus{} m$) є спродукована клясою капіталістів і робітників І
заробітна плата плюс додаткова вартість. Вони існують у натуральній
формі засобів продукції, речей, що в них їхню власну вартість не можна
спожити. Отже, ми маємо тут суму вартости в 4000, що з них половина,
— і до обміну й після обміну — заміщує лише сталий капітал,
а друга половина становить лише дохід. Але, подруге, сталий капітал
підрозділу І заміщується in natura, почасти через обмін між капіталістами
І, почасти через заміщення in natura в кожному поодинокому підприємстві.

Фраза, ніби вся вартість річного продукту, кінець-кінцем, має бути
оплачена споживачами, була б правильна тільки тоді, коли б споживачів
мислили, як два цілком різні сорти: індивідуальних споживачів і
\parbreak{}  %% абзац продовжується на наступній сторінці


\index{ii}{0337}  %% посилання на сторінку оригінального видання
продуктивних споживачів. Але та обставина, що частина продукту мусить
бути спожита \emph{продуктивно}, значить лише те, що вона мусить
\emph{функціонувати як капітал} і не може бути \emph{спожита} як дохід.

Навпаки, коли вартість сукупного продукту \deq{} 9000 ми розподілимо
на $6000 с \dplus{} 1500 v \dplus{} 1500 m$ і розглядатимемо 3000 ($v \dplus{} m$) лише
в їхній властивості бути доходом, то здається, ніби змінний капітал зник,
і що капітал, розглядуваний з суспільного погляду, складається виключно
з сталого капіталу. Бо те, що спочатку виступало як $1500 v$, розклалось
на частину суспільного доходу, на заробітну плату, дохід робітничої
кляси, — і разом з тим зник характер капіталу цієї частини. Рамсай
і справді зробив такий висновок. На його думку, капітал, розглядуваний
з суспільного погляду, складається лише з основного капіталу, але під
основним капіталом він розуміє сталий капітал, масу вартости, що є в
засобах продукції, хоч будуть ці засоби продукції лише засобами праці
або матеріялами праці, як от сировинний матеріял, напівфабрикат, допоміжний
матеріял тощо. Змінний капітал він зве обіговим капіталом. „Обіговий
капітал складається виключно з засобів існування та інших доконечних
речей, авансовуваних робітникам, поки вивершиться продукт
їхньої праці\dots{} Тільки основний капітал, а не обіговий є, власне кажучи,
джерело національного багатства\dots{} Обіговий капітал не є безпосередній
чинник продукції, і взагалі він не має для неї посутнього значення; це —
лише умова, що стала доконечною в наслідок гірких злиднів маси народу\dots{}
З національного погляду лише основний капітал є елемент витрат
продукції“\footnote*{
„Circulating capital consists exclusively of subsistence and other necessaries
advanced to the workmen, previous to the completion of the produce of their labour\dots{}
Fixed capital\dots{} alone, not circulating, is properly speaking a source of national
wealth\dots{} Circulating capital is not an inmediate agent in production, nor even essential
to it at all, but merely a convenience rendered necessary by the deplorable poverty
of the mass of the people\dots{} Fixed capital alone constitutes an element of cost of
production in a national point of view“.
}. (Ramsay, 1. с., стор. 23--26 passim). Ближче Рамсай так
пояснює основний капітал, що під ним він розуміє сталий: „Час, що
протягом його частина продукту цієї праці (а саме праці, застосованої на
продукцію якогобудь товару) існує як основний капітал, тобто в такій
формі, що в ній вона, хоч і сприяє продукції майбутнього товару, але
не \emph{утримує робітників}“ (р. 59)\footnote*{
„The length of time during which any portion of the product of that labour
(а саме — labour bestowed on any commodity) has existed as fixed capital; that is
in a form in which, though assisting to raise the future commodity, it \emph{does not
maintain labourers}“.
}.

Тут ми знову бачимо оте лихо, що його наробив А.~Сміт, потопивши
ріжницю між сталим та змінним капіталом у ріжниці між основним та
обіговим капіталом. Сталий капітал Рамсая складається з засобів праці,
його обіговий капітал — з засобів існування; і ті й ці є товари даної
вартости, і ті й ці однаково не можуть продукувати додаткову
вартість.
\label{original-337}

\parcont{}  %% абзац починається на попередній сторінці
\index{i}{0338}  %% посилання на сторінку оригінального видання
дня й інтенсифікація праці виключають одне одного, так що
здовження робочого дня можна узгодити лише з пониженням
ступеня інтенсивности праці і, навпаки, підвищення ступеня інтенсивности
— лише із скороченням робочого дня. Скоро тільки
обурення робітничої кляси, що поступінно зростало, примусило
державу силоміць скоротити робочий час та насамперед фабриці
у власному значенні подиктувати нормальний робочий день,
отже, від того моменту, коли раз назавжди поклали край збільшенню
продукції додаткової вартости через здовження робочого
дня, капітал з усієї сили та з повною свідомістю кинувся до продукції
відносної додаткової вартости за допомогою прискореного
розвитку машинової системи. Одночасно настає зміна в характері
відносної додаткової вартости. Взагалі метода продукції відносної
додаткової вартости є в тому, щоб через збільшення продуктивної
сили праці зробити робітника здатним за тієї самої витрати праці
і протягом того самого часу більше продукувати. Той самий
робочий час, як і раніш, додає до загального продукту ту саму
вартість, хоч ця незмінена мінова вартість виражається тепер у
більшій кількості споживних вартостей, наслідком чого вартість
поодинокого товару падає. Та інша справа, коли насильне скорочення
робочого дня разом із величезним поштовхом, який воно
дає розвиткові продуктивної сили й економізації умов продукції,
примушує робітника витрачати за той самий час більше праці,
напружувати більше робочу силу, щільніше заповнювати пори
робочого часу, тобто конденсувати працю до такого ступеня, якого
можна досягти лише в межах скороченого робочого дня. Цю
згущену в даний період часу більшу масу праці вважається тепер
за те, чим вона є, — за більшу кількість праці. Поряд із мірою
робочого часу як «екстенсивної величини» виступає тепер міра
ступеня його згущення\footnote{
Певна річ, по різних галузях продукції взагалі бувають ріжниці
щодо інтенсивности праці. Ці ріжниці, як це вже показав А.~Сміт, компенсуються
почасти побічними обставинами, властивими кожному родові
праці. Але робочий час як міра вартости зазнає і тут впливу лише остільки,
оскільки інтенсивні та екстенсивні величини являють собою протилежні
вирази тієї самої кількости праці, вирази, що один одного виключають.
}. Інтенсивніша година десятигодинного
робочого дня містить у собі тепер стільки ж або більше праці,
тобто витраченої робочої сили, ніж пористіша година дванадцятигодинного
робочого дня. Тому її продукт має таку саму або більшу
вартість, ніж продукт пористіших 1\sfrac{1}{5} годин. Не кажучи вже про
збільшення відносної додаткової вартости через підвищення продуктивної
сили праці, тепер, приміром, 3\sfrac{1}{3} години додаткової
праці на 6\sfrac{2}{3} години доконечної праці дають капіталістові таку
саму масу вартости, як раніш 4 години додаткової праці на 8 годин
доконечної праці.

Тепер спитаємо, яким чином інтенсифікується працю?

Перший наслідок скороченого робочого дня ґрунтується на тому
цілком очевидному законі, що працездатність робочої сили стоїть
\parbreak{}  %% абзац продовжується на наступній сторінці


\index{iii1}{0339}  %% посилання на сторінку оригінального видання
Покупець звичайного товару купує споживну вартість цього
товару, а сплачує він його вартість. Позичальник грошей так само
купує їх споживну вартість як капіталу; але що він сплачує?
Звичайно, не їх ціну або вартість, як це має місце при купівлі
інших товарів. Між позикодавцем і позичальником не відбувається,
як це має місце між продавцем і покупцем, обміну форми
вартості, при чому та вартість, яка одного разу існує у формі грошей,
другого разу існує у формі товару. Тотожність віддаваної і
одержуваної назад вартості виявляється тут цілком інакшим способом.
Сума вартості, гроші, віддаються без еквіваленту і через
певний час повертаються назад. Позикодавець весь час
лишається власником тієї самої вартості, навіть після того, як вона
перейшла з його рук у руки позичальника. При простому товарному
обміні гроші завжди є на боці покупця; але при позиці гроші
перебувають на боці продавця. Це він віддає гроші на певний
час, а покупець капіталу одержує їх як товар. Але це можливе
лиш остільки, оскільки гроші функціонують як капітал і тому
авансуються. Позичальник бере гроші в позику як капітал, як
вартість, що самозростає. Але спочатку це тільки капітал у
собі, як і всякий капітал у його вихідній точці, в момент його
авансування. Тільки за допомогою споживання їх вони збільшуються
в своїй вартості, реалізуються як капітал. Але позичальник
повинен зворотно сплатити їх як \emph{реалізований} капітал,
отже, як вартість плюс додаткова вартість (процент); а цей
останній може бути тільки частиною реалізованого ним зиску.
Тільки частиною, a не всім зиском. Бо для позичальника споживна
вартість грошей полягає в тому, що вони виробляють
йому зиск. Інакше вийшло б, що з боку позикодавця не відбулося
ніякого відчуження споживної вартості. З другого боку,
весь зиск не може дістатися позичальникові. Інакше вийшло б,
що він нічого не заплатив за відчуження споживної вартості і
повертає позикодавцеві авансовані гроші тільки як прості гроші,
а не як капітал, не як реалізований капітал, бо реалізованим
капіталом вони є тільки як $Г \dplus{} ΔГ$.

Обидва, і позикодавець і позичальник, витрачають ту саму
грошову суму як капітал. Але тільки в руках позичальника вона
функціонує як капітал. Зиск не подвоюється від подвійного
буття однієї й тієї самої грошової суми як капіталу для двох
осіб. Вона може функціонувати як капітал для обох тільки в
наслідок поділу зиску. Та частина, яка дістається позикодавцеві,
зветься процентом.

Припускається, що вся угода відбувається між двома видами
капіталістів, між грошовим капіталістом і промисловим або торговельним
капіталістом.

Ніколи не слід забувати, що тут капітал як капітал є товар,
або що товар, про який тут іде мова, є капітал. Тому всі відносини,
які тут виявляються, були б ірраціональні з точки зору
простого товару, або також з точки зору капіталу, оскільки він
\parbreak{}  %% абзац продовжується на наступній сторінці


\index{ii}{0340}  %% посилання на сторінку оригінального видання
Гроші, що спочатку функціонували для капіталіста як грошова
форма змінного капіталу, тепер функціонують у руках робітника як
грошова форма його заробітної плати, що її він перетворює на засоби
існування; отже, як грошова форма доходу, одержуваного ним від
завжди повторюваного продажу своєї робочої сили.

Тут перед нами лише той простий факт, що гроші покупця, в
даному разі капіталіста, з його рук переходять до рук продавця, в
даному разі продавця робочої сили, робітника. Тут не змінний капітал
двічі функціонує — як капітал для капіталіста і як дохід для робітника, —
і ті самі гроші, що спочатку існували в руках капіталіста як грошова
форма його змінного капіталу, отже, як потенціяльний змінний капітал, і
що потім, після того, як капіталіст перетворив їх на робочу силу, служать
у руках робітника як еквівалент проданої робочої сили. А те, що ті
самі гроші в руках продавця використовується інакше, ніж у руках покупця,
є явище властиве кожній купівлі та продажеві товарів.

Апологети-економісти фалшиво освітлюють справу, і це найкраще
видно, коли ми звернемо увагу виключно, — не турбуючись покищо про
дальші наслідки, — тільки на акт циркуляції $Г — Р$ ($\deq{} Г — Т$), перетворення
грошей на робочу силу на боці капіталістичного покупця, $Р — Г$ ($\deq{} Т — Г$),
перетворення товару робочої сили на гроші на боці продавця, робітника.
Вони кажуть: ті самі гроші реалізують тут два капітали; покупець —
капіталіст — перетворює свій грошовий капітал на живу робочу силу, що
її він долучає до свого продуктивного капіталу; з другого боку, продавець
— робітник — перетворює свій товар — робочу силу — на гроші й
витрачає їх як дохід, через що саме й може він знову й знов продавати
й таким чином зберігати свою робочу силу; отже, сама його робоча
сила є його капітал у товаровій формі і є постійне джерело його доходу.
А справді робоча сила є його здібність (яка постійно відновлюється,
репродукується), а не його капітал. Вона єдиний товар, що його він
постійно може й мусить продавати для того, щоб жити, і що діє як
капітал (змінний) лише в руках покупця, капіталіста. Коли якась людина
постійно мусить знову й знов продавати третій особі свою робочу силу,
тобто самого себе, то це, згідно з згаданими економістами, доводить,
що вона — капіталіст, бо їй завжди доводиться продавати „товар“ (саму
себе). В цьому розумінні й раб, хоч його раз назавжди продає як
товар третя особа, стає капіталістом, бо природа цього товару — робітника-раба
— така, що покупець не тільки примушує його кожного дня робити,
а й дає йому ті засоби існування, що завдяки їм він може знову й
знов робити. — (Порівняй про це Сісмонді та Сея в листах до Малтуса).

2) Отже, те, що в обміні 1000 І~$v \dplus{} 1000$ І~$m$ на 2000 II~$с$ є сталий
капітал для одних (2000 II~$с$), стає змінним капіталом і додатковою
вартістю, тобто взагалі доходом для інших; а те, що є змінний капітал
і додаткова вартість 2000 І ($v \dplus{} m$), тобто, взагалі, доходом для одних,
стає сталим капіталом для інших.

Розгляньмо спочатку обмін I~$v$ на II~$с$, насамперед з погляду робітника.


\index{ii}{0341}  %% посилання на сторінку оригінального видання
Збірний робітник І продав свою робочу силу збірному капіталістові
І на 1000; цю вартість виплачено йому грішми в формі заробітної плати.
На ці гроші він купує в II засоби споживання на ту саму суму
вартости. Капіталіст II протистоїть йому лише як продавець товарів, і
нічого більше, хоч би робітник купував у свого власного капіталіста, як
напр., вище (стор.~\pageref{original-310}) в обміні 500 II~$v$. Форма циркуляції, що її пророблює
його товар, робоча сила, це форма простої циркуляції товарів, спрямованої
виключно на задоволення потреб, на споживання $Т$ (робоча сила) —
$Г — Т$ (засоби споживання, товар II). Результат цього акту циркуляції
той, що робітник зберіг себе як робочу силу для капіталіста І, як таку,
і щоб зберегти себе як робочу силу надалі, робітник мусить знову та
знову повторювати процес $Р (Т) — Г — Т$. Його заробітна плата реалізується
в засобах споживання, її витрачається як дохід, і, беручи
робітничу клясу в цілому, завжди знову й знов витрачається як дохід.

Розгляньмо тепер той самий обмін I~$v$ на II~$c$ з погляду капіталіста. Ввесь
товаровий продукт II складається з засобів споживання, отже, з речей,
призначених на те, щоб увійти в річне споживання, тобто служити комубудь
— в даному разі збірному робітникові І — для реалізації доходу. Але
для збірного капіталіста ІI частина його товарового продукту, \deq{} 2000,
являє тепер перетворену на товар форму сталої капітальної вартости
його продуктивного капіталу, що його з цієї товарової форми треба
знову перетворити на ту натуральну форму, в якій він може знову
функціонувати як стала частина продуктивного капіталу. До цього часу
капіталіст II досяг того, що половину (\deq{} 1000) своєї сталої капітальної
вартости, репродукованої в товаровій формі (в засобах споживання) він
знову перетворив на грошову форму через продаж робітникові І. Отже,
на цю першу половину сталої капітальної вартости II с перетворився не
змінний капітал І~$v$, а гроші, які в обміні на робочу силу функціонували
для І як грошовий капітал і потрапили таким чином у посідання продавця
робочої сили, для якого вони являють зовсім не капітал, а дохід
у грошовій формі, тобто він їх витрачає як купівельний засіб на предмети
споживання. З другого боку, гроші \deq{} 1000, що приплили від робітників
І до капіталістів II, не можуть функціонувати як сталий елемент
продуктивного капіталу II.~Це покищо лише грошова форма його товарового
капіталу, що її ще лише треба перетворити на основні або обігові
складові частини сталого капіталу. Отже, II на гроші, вторговані від
робітників І, покупців його товару, купує в І засоби продукції на 1000.
У наслідок цього стала капітальна вартість II на половину всієї своєї
величини відновлюється в тій натуральній формі, що в ній вона знову
може функціонувати як елемент продуктивного капіталу II.~Формою
циркуляції при цьому було $Т — Г — Т$: засоби споживання вартістю в 1000 —
гроші \deq{} 1000 — засоби продукції вартістю в 1000.

Але $Т — Г — Т$ в даному разі є рух капіталу. $Т$, продане робітникам,
перетворюється на $Г$, а це $Г$ перетворюється на засоби продукції; це —
зворотне перетворення з товару на речові творчі елементи цього товару.
З другого боку, так само, як капіталіст II проти І функціонує лише як покупець
\index{ii}{0342}  %% посилання на сторінку оригінального видання
товару, так і капіталіст І проти II функціонує тут лише як продавець товару. І на 1000
грошей, призначених функціонувати як змінний капітал, спочатку купив робочу силу вартістю в 1000;
отже, він одержав еквівалент за свої $1000 v$, віддані в грошовій формі; тепер гроші належать
робітникові, що витрачає їх на акти купівлі в II; ці гроші, що потрапили таким чином до каси II, І
може одержати знову, лише виловлюючи їх назад через продаж товарів на таку саму суму вартости.

Спочатку І мав певну грошову суму \deq{} 1000, призначену функціонувати
як змінна частина капіталу; вона функціонує як така в наслідок перетворення її на робочу силу такого
ж розміру вартости. Але робітник дав йому, як результат продукційного процесу, певну масу товарів
(засобів продукції) вартістю в 6000, що з них \sfrac{1}{6}, або 1000, своєю вартістю
являє еквівалент авансованої в грошах змінної частини капіталу. Як перше,
в своїй грошовій формі, так і тепер в своїй товаровій формі, змінна
капітальна вартість не функціонує як змінний капітал; вона може так
функціонувати лише після того, як перетвориться на живу робочу силу
і лише протягом того часу, поки ця остання функціонує в продукційному
процесі. В грошовій формі, змінна капітальна вартість була лише потенціяльним
змінним капіталом. Але ця вартість перебувала в такій формі,
що в ній її можна було перетворити безпосередньо на робочу силу.
В товаровій формі, та сама змінна капітальна вартість є покищо лише
потенціяльна грошова вартість; її можна знову відновити в первісній
грошовій формі лише через продаж товару, отже, в даному разі, в наслідок
того, що II купує на 1000 товару в І.~Рух циркуляції тут такий:
$1000 v$ (гроші) — робоча сила вартістю в 1000--1000 в товарі (еквівалент
змінного капіталу) — $1000 v$ (гроші); отже, $Г — Т\dots{} Т — Г$ ($\deq{} Г — Р\dots{}
Т — Г$). Самий процес продукції, що припадає між $Т\dots{} Т$, не належить до
сфери циркуляції; він не з’являється в обміні різних елементів річної
репродукції одних на одні, хоч цей обмін включає репродукцію всіх
елементів продуктивного капіталу, так його сталого елементу, як і змінного,
робочої сили. Всі аґенти цього обміну виступають як лише покупці
або продавці, або як ті й ці; робітники виступають в ньому лише як
покупці товару; капіталісти — навперемінки як покупці й продавці, а в
певних межах — лише однобічно як покупці товару або однобічно як продавці
товару.

Результат такий: І має змінну частину вартости свого капіталу знову
в грошовій формі, що тільки з неї й можна перетворити цю частину
вартости безпосередньо на робочу силу, тобто знову має її в тій єдиній
формі, що в ній її справді можна авансувати як змінний елемент його
продуктивного капіталу. З другого боку, щоб мати змогу знову виступити
як покупець товару, робітник тепер мусить уперед знову виступити
як продавець товару, як продавець своєї робочої сили.

Щодо змінного капіталу категорії II (500 II~$v$) процес циркуляції
між капіталістами й робітниками тієї самої кляси продукції виступає в
безпосередній формі — оскільки ми розглядаємо його як процес, що відбувається
між збірним капіталістом II і збірним робітником II.

\parcont{}  %% абзац починається на попередній сторінці
\index{iii1}{0343}  %% посилання на сторінку оригінального видання
час виробництва і час циркуляції входить у визначення ціни
товарів і яким чином саме цим визначається норма зиску для
даного часу обороту капіталу, а визначенням зиску для даного
часу визначається якраз норма процента. Вся його глибокодумність
тут, як і завжди, полягає тільки в тому, щоб бачити хмари
пилу на поверхні і чванливо говорити про цей пил, як про щось
таємниче й значне.

\section{Поділ зиску. Розмір процента.
„Природна“~норма~процента}

Предмет цього розділу, як і взагалі всі явища кредиту, про
які ми говоритимем далі, тут не можуть бути досліджені в подробицях.
Конкуренція між позикодавцями й позичальниками і,
як результат її, короткочасні коливання грошового ринку виходять
за межі нашого дослідження. Змалювання кругобігу, що
його пророблює норма процента протягом промислового циклу,
передбачає попереднє змалювання самого цього циклу, яке тут
так само не може бути зроблене. Те саме стосується й до більшого
чи меншого приблизного вирівнення розміру процента
на світовому ринку. Тут нам доведеться тільки дослідити самостійну
форму капіталу, що дає процент, та усамостійнення процента
відносно зиску.

Через те що процент є просто частина зиску, яку, як ми
це досі припускали, промисловий капіталіст повинен сплачувати
грошовому капіталістові, то максимальну межу процента
являє собою самий зиск, при чому частина, яка дістається функціонуючому
капіталістові, була б \deq{} 0. Залишаючи осторонь окремі
випадки, коли процент фактично може бути більший за зиск, —
але тоді він не може бути виплачений із зиску, — можна було б,
мабуть, за максимальну межу процента вважати весь зиск мінус
та його частина, яка зводиться до плати за нагляд (wages
of superintendence) і яку нам далі доведеться розглянути. Мінімальна
межа процента ніяк не може бути визначена. Процент
може впасти до якого завгодно низького рівня. Але тоді знову
й знову виступають протидіючі обставини і підвищують його
понад цей відносний мінімум.

„Відношення між сумою, сплаченою за вживання капіталу,
і самим цим капіталом виражав норму процента, вимірену в
грошах“. — „Норма процента залежить 1)~від норми зиску; 2)~від
того відношення, в якому весь зиск ділиться між позикодавцем
і позичальником“ („\emph{Economist}“, 22 January 1853 [стор. 89]). „Тому
що виплачуване як процент за користування тим, що береться в
позику, є частина зиску, що його здатне виробити взяте в позику,
то цей процент завжди мусить реґулюватися цим зиском“
(\emph{Massie}: „An Essay on the Governing Causes of the Natural Rate of
Interest etc.“, London 1750, стор. 49).


\index{ii}{0344}  %% посилання на сторінку оригінального видання
Але ті 500 в грошах, що повернулися до капіталіста II, є разом з
тим відновлений потенціяльний змінний капітал у грошовій формі. Чому
це? Гроші, отже, і грошовий капітал є потенціяльний змінний капітал
лише тому й остільки, що й оскільки їх можна перетворити на робочу
силу. Поворот цих 500\pound{ф. стерл} грішми до капіталіста II супроводиться
поворотом робочої сили II на ринок. Поворот грошей і робочої сили на
протилежні полюси — а значить, і з’явлення знову цих 500 в грошовій
формі, не лише як грошей, а також і як змінного капіталу в грошовій
формі — зумовлено тією самою процедурою. Гроші \deq{} 500 повертаються до
капіталіста II тому, що він продав робітникові II засобів споживання на
суму 500, отже, тому, що робітник витратив свою заробітну плату й таким
чином дістав змогу утримувати себе й родину, а тим самим і свою
робочу силу. Щоб йому можна було й далі існувати, і далі виступати
покупцем товарів, він мусить знову продати свою робочу силу. Отже,
поворот до капіталіста II цих 500 грішми є разом з тим поворот, зглядно
збереження, робочої сили як товару, що його можна купити на ці 500
грішми, а тому це є поворот цих 500 грішми як потенціяльного змінного
капіталу.

Щодо категорії II~$b$, яка продукує речі розкошів, то з її $v$ —
(II~$b$) — справа така сама, як і з I~$v$. Гроші, що відновлюють капіталістам
II~$b$ їхній змінний капітал в грошовій формі, припливають до них
обкружним шляхом, через руки капіталістів II~$а$. А проте, є ріжниця в
тому, чи купують робітники засоби свого існування безпосередньо у тих
капіталістичних продуцентів, що їм вони продають свою робочу силу, чи
купують їх у другої категорії капіталістів, за посередництвом яких гроші
повертаються до перших лише обкружним шляхом. А що робітнича кляса
живе з дня на день, то вона купує, поки може купувати. Інша справа
з капіталістом, прим., при обміні 1000 II~$с$ на 1000 І~$v$. Капіталіст живе
не з дня на день. Рушійний мотив для нього — якомога значніше збільшення
вартости його капіталу. Тому, коли постають якісь обставини, що,
зважаючи на них, капіталістові II здається вигідніше, замість відновити
безпосередньо свій сталий капітал, хоча б почасти затримати його в
грошовій формі на більш-менш довгий час, то зворотний приплив цих
1000 ІІ~$с$ (в грошах) до І уповільнюється; уповільнюється, отже, і відновлення
$1000 v$ в грошовій формі, і капіталіст І може провадити далі
роботу в попередньому маштабі лише тоді, коли в його розпорядженні
є запасні гроші, як і взагалі потрібен запасний капітал в грошовій формі
для того, щоб можна було безперервно провадити роботу незалежно від
швидшого або повільнішого зворотного припливу змінної капітальної вартости
в грошах.

Коли треба дослідити обмін різних елементів поточної річної репродукції,
то при цьому треба дослідити й результат минулої річної праці,
праці вже закінченого року. Продукційний процес, що його результат є
цей річний продукт, лежить позад нас, минув, злився з своїм продуктом;
отже, то більше це має силу для процесу циркуляції, що передує
процесові продукції або відбувається рівнобіжно з ним, — для перетворення
\index{ii}{0345}  %% посилання на сторінку оригінального видання
потенціяльного на справжній змінний капітал, тобто для купівлі
й продажу робочої сили. Робочий ринок уже не становить частини того
товарового ринку, що тут є перед нами. Тут робітник не тільки вже продав
свою робочу силу, а й дав у товарі, крім додаткової вартости,
еквівалент ціни своєї робочої сили; з другого боку, заробітна плата є
вже в його кишені і в обміні він фігурує лише як покупець товару
(засобів споживання). Але далі річний продукт мусить мати в собі всі
елементи репродукції, мусить відновити всі елементи продуктивного капіталу,
— отже, насамперед, найважливіший елемент його — змінний капітал.
І ми справді бачили, що відносно до змінного капіталу результат обміну
такий: робітник як покупець товару, витрачаючи свою заробітну плату
й споживаючи куплений товар, зберігає й репродукує свою робочу силу
як єдиний товар, що його він може продавати; як гроші, авансовані
капіталістом на закуп цієї робочої сили, повертаються до капіталіста, так
і робоча сила, як товар, обмінюваний на ці гроші, повертається на робочий
ринок; в наслідок цього ми тут, а саме для 1000 I~$v$, маємо таке:
на боці капіталістів І — $1000v$ грішми; на протилежному боці, на
боці робітників І — робоча сила вартістю в 1000, отже, ввесь процес
репродукції І може початися знову. Це — один результат процесу
обміну.

З другого боку, витрачення заробітної плати робітників І забрало
в II засобів споживання на суму $1000с$ і таким чином перетворило їх з
товарової форми на грошову форму; II з цієї грошової форми перетворив
їх знову на натуральну форму свого сталого капіталу за допомогою
закупу товарів на суму $1000v$ у І; в наслідок цього до І повертається
його змінна капітальна вартість знову в грошовій формі.

Змінний капітал І пророблює три перетворення, що зовсім не виявляються
при обміні річного продукту, або виявляються лише як натяк.

1) Перша форма, 1000 I~$v$ в грошах, які перетворюються на робочу
силу того ж розміру вартости. Саме це перетворення не виявляється в
товаровому обміні між І і II, але його результат виявляється в тому, що
кляса робітників І з 1000 в грошах протистоїть продавцеві товарів II,
цілком так само, як кляса робітників II з 500 в грошах протистоїть
продавцеві товарів — 500 II~$v$ в товаровій формі.

2) Друга форма, — єдина, що в ній змінний капітал справді змінюється,
функціонує як змінний, що в ній вартостетворча сила виступає замість
обміненої на неї даної вартости, — належить виключно до продукційного
процесу, що лежить позаду нас.

3) Третя форма, що в ній в наслідок продукційного процесу змінний
капітал виявляв себе як такий, є новоспродукована вартість, отже,
в І \deq{} $1000 v \dplus{} 1000 m$ \deq{} 2000 І ($v \dplus{} m$). Замість його первісної вартости \deq{}
1000 грішми виступає вдвоє більша вартість \deq{} 2000 в товарах.
А тому змінна капітальна вартість \deq{} 1000 в товарах становить лише
половину тієї нової вартости, що її утворив змінний капітал як елемент
продуктивного капіталу. Ці 1000 I~$v$ в товарах є точний еквівалент тієї
змінної за її призначенням частини всього капіталу, яку первісно І авансував
\index{ii}{0346}  %% посилання на сторінку оригінального видання
в $1000v$ грішми; але в товаровій формі вони є гроші лише потенціяльно
(дійсними грішми вони стають тільки в наслідок продажу), отже,
ще менше вони є безпосередньо змінний грошовий капітал. Кінець-кінцем,
вони стають ним через продаж товару 1000 I~$v$ покупцеві II~$c$ і через
те, що робоча сила одразу знову з’являється як продажний товар, як
матеріял, що на нього можуть перетворитись $1000v$ грішми.

Підчас усіх цих перетворень капіталіст І постійно має в своїх руках
змінний капітал: 1) спочатку як грошовий капітал; 2) потім як елемент
його продуктивного капіталу; 3) ще пізніше як частину вартости його
товарового капіталу, тобто в товаровій вартості; 4) нарешті, знову в грошах,
що їм знову протистоїть робоча сила, на яку їх можна перетворити.
Протягом процесу праці капіталіст має в своїх руках змінний капітал як
діющу вартостетворчу робочу силу, а не як вартість даної величини;
що капіталіст завжди оплачує робітника лише після того, як сила його
діяла вже певний коротший або довший час, то перш ніж оплатити її,
він уже одержує в свої руки утворену нею вартість як еквівалент її
самої плюс додаткова вартість.

\emph{А що змінний капітал в тій або іншій формі постійно
лишається в руках капіталіста, то ні в якому
разі не можна сказати, що він перетворюється для будь-кого
на дохід}. Навпаки, 1000 I~$v$ в товарі перетворюється на гроші
через продаж покупцеві II, що для нього таким чином заміщується
in natura половина його сталого капіталу.

Не змінний капітал І, $1000v$ в грошах, сходить на дохід. Ці гроші,
скоро вони перетворені на робочу силу, перестають функціонувати як
грошова форма змінного капіталу І, — так само, як і гроші всякого іншого
продавця товарів перестають репрезентувати щось належне йому,
скоро він їх перетворить на товар якогось продавця. Перетворення, що
їх пророблюють в руках робітничої кляси гроші, одержані як заробітна
плата, є перетворення не змінного капіталу, а перетвореної на гроші
вартости робочої сили робітничої кляси; цілком так само, як перетворення
новоутвореної робітником вартости [2000 І ($v \dplus{} m$)] є лише перетворення
належного капіталістові товару, перетворення, яке зовсім
не стосується до робітника. Але капіталіст, — а ще більше його теоретичний
тлумач, політикоеконом силу в силу може визволитись від уявлення,
ніби гроші, виплачені робітникові, все ще є його, капіталіста, гроші.
Коли капіталіст є продуцент золота, то змінна частина вартости, тобто
той еквівалент у товарі, що заміщує йому купівельну ціну праці, сама
безпосередньо з’являється в грошовій формі, а тому знову, без обкружних
шляхів зворотного припливу, може функціонувати як змінний грошовий
капітал. Але щодо робітника в II, — оскільки ми лишаємо осторонь
робітників, що продукують речі розкошів — то саме $500v$ існує в товарах,
призначених на споживання робітникові, і їх він, розглядуваний як
збірний робітник, безпосередньо знову купує в того самого збірного
капіталіста, що йому він продав свою робочу силу. Змінна частина
вартости капіталу II своєю натуральною формою складається з засобів
\parbreak{}  %% абзац продовжується на наступній сторінці

\input{ii/_0347c.tex}
\parcont{}  %% абзац починається на попередній сторінці
\index{iii1}{0348}  %% посилання на сторінку оригінального видання
поверхового уявлення про внутрішній зв’язок економічних відносин,
який виявляється в конкуренції. Це є спосіб для того, щоб від
змін, які супроводять конкуренцію, прийти до границь цих змін.
Його не можна прикласти до пересічного розміру процента.
Немає абсолютно ніякої підстави, чому середні умови конкуренції,
рівновага між позикодавцями й позичальниками, повинні давати
позикодавцеві розмір процента в 3, 4, 5\% і т. д. на його капітал
абож певну процентну частину — 20 чи 50\% — гуртового
зиску. В тих випадках, коли справу-тут вирішує конкуренція як
така, визначення само по собі є випадковим, чисто емпіричним,
і тільки педантство або фантазерство може хотіти зобразити
цю випадковість як щось необхідне.\footnote{
Так, наприклад, \emph{J. G. Opdyke}: „А Treatise on Political Economy“, New
York 1851, робить надзвичайно невдалу спробу пояснити загальність розміру
процента в 5\% вічними законами. Незрівняно наївніший пан \emph{Карл Арнд} в „Die
naturgemässe Volkswirtschaft gegenüber dem Monopoliengeist und dem Kommunismus
etc.“, Hanau 1845. Тут можна прочитати таке: „В природному ході виробництва
благ існує тільки \emph{одно} явище, яке — в цілком культивованих країнах — до
певної міри ніби призначене регулювати розмір процента; це — відношення,
в якому збільшуються маси дерев у європейських лісах в наслідок їх щорічного
приросту. Цей приріст відбувається цілком незалежно від їх мінової вартості“
[як це комічно, що дерева організують свій приріст незалежно від своєї мінової
вартості!] „у відношенні 3--4 до 100. — Отже, згідно з цим“ [тому що приріст
дерев зовсім не залежить від їх мінової вартості, хоч і як дуже їх мінова вартість
може залежати від їх приросту] „не можна було б сподіватись падіння
нижче того рівня, що його він“ [розмір процента] „має в теперішній час у
найбагатших на гроші країнах“ (стор. 124 [125]). — Це заслуговує назви „розмір
процента лісового походження“, а його винахідник у тому самому творі здобуває
ще більшу заслугу перед „нашою наукою“ як „філософ собачого податку“
[стор. 420 і далі].
} У парламентських звітах
1857 і 1858 рр., які стосуються законодавства про банки і торговельної
кризи, немає нічого кумеднішого, як базікання директорів
Англійського банку, лондонських банкірів, провінціальних
банкірів і професіональних теоретиків про „real rate produced“
[фактично утворену норму], яке не йшло далі таких загальних
місць, як, наприклад, що „ціна, яка сплачується позиченим
капіталом, може мінятись із зміною подання цього капіталу“,
що „висока норма процента і низька норма зиску не можуть
довгий час існувати одна поряд одної“ та інші подібні банальності.\footnote{
Англійський банк підвищує і знижує норму свого дисконту залежно від
того, припливає чи відпливає золото, хоч, звичайно, він при цьому завжди бере
до уваги норму, яка панує на відкритому ринку. „By which gambling in discounts,
by anticipation of the alterations in the bank rate, has now become half the trade
ol the great heads of they money centre“ [„В наслідок цього спекуляція на зміні
дисконту, яка передхоплює зміни банкової норми, стала тепер наполовину
заняттям великих фірм грошового центру“], — тобто лондонського грошового
ринку („The Theory of the Exchanges etc.“, стор. 113).
}
Звичка, узаконена традиція і т. д. цілком так само, як
і сама конкуренція, впливають на визначення середнього розміру
процента, оскільки він існує не тільки як пересічне число,
але й як фактична величина. Середній розмір процента мусить
уже бути припущений як законний у багатьох судових справах,
\parbreak{}  %% абзац продовжується на наступній сторінці

\parcont{}  %% абзац починається на попередній сторінці
\index{ii}{0349}  %% посилання на сторінку оригінального видання
Отже, ці гроші є грошова форма частини сталої капітальної вартости,
її основної частини. Отже, це утворення скарбу саме є елемент капіталістичного
процесу репродукції, є репродукція і нагромадження — в грошовій
формі — вартости основного капіталу або його поодиноких елементів,
до того моменту, коли основний капітал відживе свій вік і, значить, передасть
свою вартість спродукованим товарам, після чого його доводиться
замістити in natura. Але ці гроші, скоро їх знову перетворено на нові
елементи основного капіталу, щоб замістити елементи, які віджили свій
вік, втрачають лише свою форму скарбу й тому лише знову активно
входять у процес репродукції капіталу, упосереднюваний циркуляцією.

Як проста товарова циркуляція не тотожня з простим обміном продуктів,
так і перетворення річного товарового продукту не можна звести на
простий, безпосередній, взаємний обмін його різних складових частин.
Гроші відіграють у ньому специфічну ролю, яка виявляється і в способі
репродукції основної капітальної вартости. (Далі треба буде дослідити,
який це мало б інший вигляд, коли припустити, що продукція колективна
й не має форми товарової продукції).

Тепер, повертаючись до основної схеми, ми маємо для кляси II:
$2000 с \dplus{} 500 v \dplus{} 500 m$. Всі засоби споживання, спродуковані протягом
року, дорівнюють тут вартості в 3000; і кожен з різних елементів товару,
що з них складається ця сума товару, розкладається за вартістю своєю
на $\sfrac{2}{3} с \dplus{} \sfrac{1}{6} v \dplus{} \sfrac{1}{6} m$,
або у відсотках на $66\sfrac{2}{3} с \dplus{} 16\sfrac{2}{3} v \dplus{} 16\sfrac{2}{3} m$.
Різні ґатунки товарів кляси II можуть мати в собі сталий капітал у
різних пропорціях; основна частина сталого капіталу в них так само
може бути різна; так само і протяг життя основних частин капіталу, а
значить, і річне зношування або та частина вартости, яку вони pro rata
переносять на товари, вироблювані за їх допомогою. Все це тут не має
значення. Щодо суспільного процесу репродукції, то вся справа лише в
обміні між клясами II і I.~II і І протистоять тут один одному лише в
їхніх суспільних масових відношеннях; тому пропорційна величина частини
вартости с товарового продукту II (а тільки вона й має міродайне
значення для розглядуваного тепер питання) є пересічне відношення, коли
зробити загальний підсумок усіх галузей продукції, що входять у II.

Таким чином, кожен з товарових ґатунків (а це здебільша ті самі
ґатунки товарів), що їхню загальну вартість підсумовано в
$2000 с \dplus{} 500 v \dplus{} 500 m$, однаково дорівнює своєю вартістю
$66\sfrac{2}{3}\% с \dplus{} 16\sfrac{2}{3}\% v \dplus{} 16\sfrac{2}{3}\% m$.
Це має силу для всяких 100 одиниць товарів, хоч фігурують
вони під $с$, хоч під $v$, хоч під $m$.

Товари, що в них втілено $2000 с$, теж можна розкласти за їхньою
вартістю на:

1) $1333\sfrac{1}{3}с \dplus{} 333\sfrac{1}{3} v \dplus{} 333\sfrac{1}{3}m \deq{} 2000c$; так само $500 v$ на:

2) $333\sfrac{1}{3} с \dplus{} 83\sfrac{1}{3} v \dplus{} 83\sfrac{1}{3} m \deq{} 500 v$; нарешті, $500 m$ на:

3) $333\sfrac{1}{3}с \dplus{} 83\sfrac{1}{3} v \dplus{} 83\sfrac{1}{3} m \deq{} 500 m$;

Тепер, коли ми складемо $с$, що є в 1, 2 і 3, то матимемо
$1333\sfrac{1}{3} с \dplus{} 333\sfrac{1}{3}с \dplus{} 333\sfrac{1}{3} с \deq{} 2000$.
Так само $333\sfrac{1}{3}v \dplus{} 83\sfrac{1}{3}v \dplus{} 83\sfrac{1}{3}v \deq{} 500$,
\index{ii}{0350}  %% посилання на сторінку оригінального видання
і те саме з $m$; склавши всі ці величини, матимемо, як і раніше,
сукупну вартість в 3000.

Отже, вся стала капітальна вартість, що міститься в масі товарів II
вартістю в 3000, міститься в $2000 с$, і ні $500 v$, ні $500 m$ не мають
жодного атома цієї вартости. Це саме має силу також і для $v$, і для $m$.

Інакше кажучи: вся та кількість товарової маси II, яка репрезентує
сталу капітальну вартість і тому має знову бути перетворена — хоч
на її натуральну, хоч на грошову форму — існує в $2000 с$. Отже, все, що
стосується до обміну сталої вартости товарів II, обмежується рухом 2000
II~$с$; і цей обмін можливий тільки на І ($1000 v \dplus{} 1000 m$).

Так само для кляси І все, що стосується до обміну належної йому
капітальної вартости, треба обмежити розглядом 4000 І~$с$.

\subsubsection{Заміщення в грошовій формі зношуваної частини вартости}

Тепер, коли ми візьмемо насамперед:
\[
\begin{array}{r@{ }l@{ }c@{ }l}
\text{І.} & 4000с \dplus{} & \underbrace{1000 v \dplus{} 1000 m}^{} \\
\text{II.} & \dotfill{} & \dotfill{} 2000 с \dotfill{} &  \:\dplus{}\:500 v \dplus{} 500 m
\end{array}
\]
то обмін товарів 2000 II~$c$ на товари такої самої вартости І ($1000 v \dplus{}
1000 m$) припускав би, що 2000 II~$с$ in natura цілком знову перетворюється
на спродуковані підрозділом І натуральні складові частини сталого
капіталу II; але товарова вартість 2000, в якій існує цей капітал,
містить у собі елемент, що покриває втрату вартости основного капіталу,
який не одразу треба заміщати in natura, а перетворювати на гроші, поступінно
нагромаджувані в цілу суму, поки надійде час відновити
основний капітал у його натуральній формі. Кожен рік є рік смерти для
основного капіталу й доводиться його заміщувати то в цьому, то в тому
поодинокому підприємстві, або навіть то в тій, то в цій галузі промисловості;
в тому самому індивідуальному капіталі доводиться заміщувати
ту або іншу частину основного капіталу (бо частини його мають різний
протяг життя). Розглядаючи річну репродукцію — хоча б і в незміненому
маштабі, тобто абстрагуючись від усякої акумуляції — ми починаємо не
ab ovo\footnote*{
Ab ovo — латинський вираз, дослівно „від яйця“, тобто з самого виникнення,
з самого початку. \emph{Ред.}
}; ми беремо один рік з ряду багатьох, не перший рік по
народженні капіталістичної продукції. Отже, різні капітали, вкладені
в різні галузі продукції кляси II, мають різний вік, і подібно до того,
як щороку вмирають люди, які функціонують у цих галузях продукції,
так само маси основних капіталів щороку доживають свого віку, й їх доводиться
відновлювати in natura з нагромадженого грошового фонду. В
цьому розумінні обмін 2000 II~$с$ на 2000 І ($v \dplus{} m$) включає перетворення
2000 II~$с$ з його товарової форми (засобів споживання) на натуральні
елементи, що складаються не лише з сировинних та допоміжних елементів,
\parbreak{}  %% абзац продовжується на наступній сторінці

\parcont{}  %% абзац починається на попередній сторінці
\index{ii}{0351}  %% посилання на сторінку оригінального видання
а також з натуральних елементів основного капіталу, — машин,
знарядь, будівель і~\abbr{т. ін.} Тому зношування, яке у вартості 2000 II~$с$ треба
замістити грішми, зовсім не відповідає розмірові діющого основного
капіталу, бо частину його щороку доводиться заміщувати in natura; але
де припускає, що в попередні роки в руках капіталістів кляси II нагромадились
гроші, потрібні для цього заміщення. Але саме це припущення
так само має силу для поточного року, як і для минулих років.

\vtyagnut{}
В обміні між І ($1000 v \dplus{} 1000 m$) і 2000 ІІ~$с$ треба насамперед зазначити,
що в сумі вартости І ($v \dplus{} m$) не міститься елементів сталої
вартости, отже, і не міститься жодного елемента вартости для заміщення
зношування, тобто для заміщення вартости, перенесеної з основної
складової частини сталого капіталу на ті товари, що в їхній натуральній
формі існує $v \dplus{} m$. Навпаки, в ІІ~$с$ цей елемент існує, і це є саме та
частина елемента вартости, що завдячує за своє існування основному капіталові,
що її не доводиться безпосередньо перетворювати з грошової
форми на натуральну, а повинна вона спочатку лишатись у грошовій
формі. Тому, розглядаючи обмін І ($1000 v \dplus{} 1000 m$) на 2000 ІІ~$с$, ми
одразу наражаємось на ті труднощі, що засоби продукції І, в натуральній
формі яких існують 2000 ($v \dplus{} m$), на всю суму їхньої вартости в
2000 треба обміняти на еквівалент у формі засобів споживання II, але,
з другого боку, засоби споживання 2000 II~$с$ не можна обміняти на засоби
продукції І ($1000 v \dplus{} 1000 m$) на всю суму їхньої вартости, бо
деяка частина їхньої вартости — рівна зношуванню, або втраті вартости
основного капіталу, що його треба замістити — спочатку повинна осісти
як гроші, що вже не функціонуватимуть знову як засоби циркуляції
протягом поточного річного періоду репродукції, тільки й розглядуваного
тут. Але гроші, що за їхньою допомогою перетворюється на гроші елемент
зношування, який міститься в товаровій вартості 2000 II~$с$, ці гроші
можуть походити тільки від І, бо II не може оплатити сам себе,
але оплачує себе лише через продаж свого товару, і тому, що згідно
з припущенням, І ($v \dplus{} m$) купує всю суму товарів 2000 ІІ~$с$; отже, цією
купівлею кляса І мусить перетворити на гроші для кляси II зазначене
вище зношування. Але, згідно з раніш викладеним законом, гроші, авансовані
для циркуляції, повертаються до капіталістичного продуцента,
який потім подає в циркуляцію таку саму кількість у товарах. Очевидно,
що І, купуючи ІІ~$с$, не може давати підрозділові II раз назавжди на
2000 товарами, і крім того ще додаткову грошову суму (давати так,
щоб вона не поверталась до нього через операцію обміну). Це взагалі
значило б, що І, купуючи товарову масу II~$с$, оплачує її понад ії вартість.
Коли II в обмін на свої $2000 с$ справді дістає І ($1000 v \dplus{} 1000 m$),
то він не має вимагати від І нічого більше, і гроші, які циркулювали
підчас цього обміну, повертаються до І або II, залежно від того, хто з
них подав ці гроші в циркуляцію, тобто, хто з них раніш виступив як
покупець\dots{} Разом із цим підрозділ II в такому разі перетворив би свій
товаровий капітал, на всю суму його вартости, знову в натуральну форму
засобів продукції, тимчасом як ми припустили, що деяка частина його,
\parbreak{}  %% абзац продовжується на наступній сторінці

\parcont{}  %% абзац починається на попередній сторінці
\index{ii}{0352}  %% посилання на сторінку оригінального видання
бувши продана, не перетворюється знову в поточному річному періоді
репродукції з грошей на натуральну форму основних складових частин
свого сталого капіталу. Отже, до II ріжниця грішми могла б допливати
лише тоді, коли б II продав І на 2000, а купив би в І менше, ніж на
2000, наприклад, тільки на 1800; тоді І повинен би був покрити
ріжницю за допомогою 200 грішми, які не повернулись би до нього, бо
ці гроші, авансовані для циркуляції, він не вилучив би знову з неї поданням
у циркуляцію товарів \deq{} 200. В такому разі для II у нас був би
грошовий фонд на рахунок зношування його основного капіталу; але на
другому боці, на боці І, ми мали б перепродукцію засобів продукції, на
суму 200, і таким чином зникла б вся основна схема, а саме репродукція
в незмінному маштабі, за якої припускається цілковита пропорційність
між різними системами продукції. Усунувши одні труднощі, ми мали б
інші, куди неприємніші.

Що ця проблема являє особливі труднощі й до цього часу її взагалі
не досліджували політикоекономи, то ми послідовно розглянемо всі можливі
(принаймні на позір можливі) розв'язання їх або радше постави
самої проблеми.

Насамперед ми щойно припустили, що II продає І підрозділові на
2000, а купує в нього товарів лише на 1800. В товаровій вартості
2000 II~$с$ було 200 на заміщення зношування, що їх треба зберегти
в грошах як скарб; таким чином, вартість 2000 II~$с$ розклалась на
1800, що їх треба обміняти на засоби продукції І і на 200 для заміщення
зношування, що їх (після продажу $2000с$ підрозділові І) треба
зберегти в грошах. Або щодо своєї вартости 2000 II~$с$ були б $\deq{} 1800 с \dplus{}
200 с (d)$, де $d \deq{} $\emph{déchet} [зношування].

В такому разі нам треба було б дослідити:
\[\begin{array}{rcl}
\text{обмін І.} & \underbrace{1000 v \dplus{} 1000 m}^{} \\
\text{II.} & 1800 с \dplus{} 200 с & (d).
\end{array}
\]
На 1000\pound{ ф. стерл.}, що в формі заробітної плати потрапили до робітників
як плата за їхню робочу силу, І купує засобів споживання
1000 II~$с$; II на ці самі 1000\pound{ ф. стерл.} купує засобів продукції 1000 І~$v$.
Таким чином, до капіталістів І повертається їхній змінний капітал у
грошовій формі, і найближчого року вони можуть купити на нього робочу
силу тієї самої величини вартости, тобто можуть in natura замістити
змінну частину свого продуктивного капіталу. — Далі, II на авансовані
400\pound{ ф. стерл.} купує засоби продукції І~$m$, а І~$m$ на ті самі
400\pound{ ф. стерл.} купує засоби споживання II~$с$. Ті 400\pound{ ф. стерл.}, що їх II
авансував для циркуляції, повернулись таким чином до капіталістів II, але
повернулись лише як еквівалент за проданий товар. І на авансовані
400\pound{ ф. стерл.} купує засоби споживання; II купує в І засобів продукції
на 400\pound{ ф. стерл.}, в наслідок чого ці 400\pound{ ф. стерл.} повертаються до І. Отже, рахунок покищо такий:


\index{iii1}{0353}  %% посилання на сторінку оригінального видання
На грошовому ринку протистоять один одному тільки позикодавець
і позичальник. Товар має одну й ту саму форму —
гроші. Всі особливі види капіталу, які він має залежно від
вкладення його в окремі сфери виробництва або циркуляції,
тут зникли. Він існує тут у нерозрізнимому, самому собі рівному
вигляді самостійної вартості, у вигляді грошей. Конкуренція між
окремими сферами тут припиняється; всі вони звалені в одну купу
як позичальники грошей, а капітал також протистоїть їм усім
у такій формі, в якій він ще індиферентний до певного роду й способу
свого застосування. Тут, в попиті й поданні капіталу, останній,
дійсно, цілковито виступає \emph{сам по собі} як \emph{спільний капітал
усього класу}, тимчасом як промисловий капітал виступає
таким тільки в русі й конкуренції між окремими сферами.
З другого боку, грошовий капітал на грошовому ринку дійсно
має таку форму, в якій він, як спільний елемент, індиферентний до
окремого способу свого застосування, розподіляється між різними
сферами, між класом капіталістів, залежно від потреб
виробництва кожної окремої сфери. Сюди долучається ще й те,
що з розвитком великої промисловості грошовий капітал, оскільки
він з’являється на ринку, все більше й більше представлений не
окремим капіталістом, власником тієї чи іншої частини капіталу,
який перебуває на ринку, а виступає як концентрована,
організована маса, яка цілком інакше, ніж реальне виробництво,
поставлена під контроль банкірів, які є представниками суспільного
капіталу. Таким чином, оскільки справа стосується форми попиту,
капіталові, що дається в позику, протистоїть весь клас;
як і сам він, оскільки справа стосується подання, виступає як
позиковий капітал en masse [всією масою].

\looseness=1
Ось деякі з причин того, чому загальна норма зиску здається
розпливчатим міражем поруч з певним розміром процента, величина
якого, правда, коливається, але через те що вона коливається
рівномірно для всіх позичальників, то завжди протистоїть
їм як фіксований даний розмір. Цілком так само, як
зміна вартості грошей не перешкоджає їм мати відносно всіх
товарів однакову вартість. Цілком так само, як щоденні коливання
ринкових цін товарів не перешкоджають тому, щоб ці ціни
щодня відзначались у бюлетенях. Цілком так само і з розміром
процента, який з такою самою реґулярністю відзначається як
„ціна грошей“. Це тому, що тут як товар пропонується сам
капітал у грошовій формі; тому фіксація його ціни є фіксація
його ринкової ціни, як і в усіх інших товарів; тому розмір процента
завжди виступає як загальний розмір процента, як
стільки то грошей за стільки то грошей, як кількісно визначений
розмір. Навпаки, норма зиску навіть у межах однієї
і тієї ж сфери, при однакових ринкових цінах товару, може
бути різна, залежно від різних умов, при яких окремі капітали
виробляють той самий товар; бо норма зиску для окремого
капіталу визначається не ринковою ціною товару, а ріжницею
\parbreak{}  %% абзац продовжується на наступній сторінці

\parcont{}  %% абзац починається на попередній сторінці
\index{ii}{0354}  %% посилання на сторінку оригінального видання
ролях „великої публіки“ робить політикоекономам ту „послугу“, що
пояснює нез’ясоване ними.

Так само мало допомагає, коли, замість безпосереднього обміну між
І і II — між двома великими підрозділами самих капіталістичних продуцентів,
— притягують торговця як посередника і за допомогою його „грошей“
перемагають усі труднощі. Напр., в даному разі 200 І~$m$, кінець-кінцем,
і остаточно мусять бути продані промисловим капіталістам II.~Хай вони
перейдуть через руки ряду торговців — і все ж останній з них буде, згідно
з гіпотезою, в такому ж відношенні проти II, в якому спочатку були
капіталістичні продуценти І, тобто вони не можуть продати ІІ-му 200 І~$m$,
і заїжджена купівельна сума не може відновити той самий процес для I.

З цього видно, як конче потрібно, незалежно від нашої справжньої
мети, дослідити процес репродукції в його фундаментальній формі, —
де усунуто всі бічні обставини, що затемнюють справу, — як конче
потрібно це для того, щоб відкинути фалшиві викрути, які дають подобу
„наукового“ пояснення, коли за предмет аналізи з самого початку
береться суспільний процес репродукції в його заплутаній конкретній
формі.

Отже, закон, що згідно з ним гроші, авансовані капіталістичним продуцентом
для циркуляції, при нормальному перебігу репродукції (хоч у
попередньому, хоч у поширеному маштабі) мусять повертатися до свого
вихідного пункту (при цьому байдуже, чи належать ці гроші капіталістичним
продуцентам, чи їх позичено), — цей закон раз назавжди виключає
ту гіпотезу, що 200 II~$с$ ($d$) перетворюються на гроші за допомогою
грошей, авансованих підрозділом I.

\subsubsection{Заміщення основного капіталу in natura}

Після того, як ми відкинули щойно розглянуту гіпотезу, лишаються
тільки такі можливості, що, крім заміщення грішми зношеної частини, відбувається
також і заміщення цілком відмерлого основного капіталу in natura.

До цього часу ми припускали:

а) Що 1000\pound{ ф. стерл.}, видані І підрозділом на заробітну плату, витрачають
робітники на II~$с$ тієї самої величини вартости, тобто, що вони
купують на ці 1000\pound{ ф. стерл.} засоби споживання.

Що тут І авансує ці 1000\pound{ ф. стерл.} грішми, це є лише констатування
факту. Відповідні капіталістичні продуценти повинні виплатити заробітну
плату грішми; потім робітники витрачають ці гроші на засоби
існування, а для продавців засобів існування ці гроші знову таки правлять
за засіб циркуляції при перетворенні їхнього сталого капіталу
з товарового капіталу на продуктивний капітал. Вони переходять при
цьому через багато каналів (дрібні крамарі, домовласники, збирачі
податків, непродуктивні робітники, як от лікарі і~\abbr{т. ін.}, потрібні самому
робітникові), і тому лише частина їх безпосередньо з рук робітників І
припливає до рук кляси капіталістів II.~Цей приплив може більш або
менш затриматись, а тому можуть бути потрібні нові грошові резерви
\parbreak{}  %% абзац продовжується на наступній сторінці

\parcont{}  %% абзац починається на попередній сторінці
\index{ii}{0355}  %% посилання на сторінку оригінального видання
на боці капіталістів. При розгляді цієї фундаментальної форми ми все це
лишаємо осторонь.

\medskip{}
b) Ми припускаємо, що одного разу І авансує на закуп у II дальші
400\pound{ ф. стерл.} грішми, що припливають назад до нього, а другого разу
II на закуп у І авансує 400\pound{ ф. стерл.}, що повертаються до нього. Це
припущення доводиться зробити, бо було б довільним зворотне припущення,
що кляса капіталістів І або кляса капіталістів II однобічно авансує
на циркуляцію гроші, потрібні для обміну товарів. А що в попередньому
параграфі 1) показано, що треба відкинути як безглузду ту гіпотезу, за
якою додаткові гроші, потрібні на перетворення 200 II~$с$ ($d$) на гроші,
подає в циркуляцію І, то, очевидно, лишається тільки ще, здається, безглуздіша
гіпотеза, а саме, що II сам подає в циркуляцію гроші, що за допомогою
їх перетворюється на гроші складова частина вартости товару,
яка має замістити зношування основного капіталу. Напр., частина вартости,
втрачена в продукції прядільною машиною пана $X$, знову з’являється
як частина вартости ниток до шиття; те, що на одному боці його прядільна
машина втрачає в вартості або як зношування, повинно нагромаджуватись
у нього на другому боці як гроші. Хай $X$ купує, напр., на
200\pound{ ф. стерл.} бавовни в $V$ і таким чином авансує для циркуляції 200\pound{ ф.
стерл.} грішми; $V$ купує в нього пряжі на ці самі 200\pound{ ф. стерл.}, і ці
200\pound{ ф. стерл.} служать тепер для $X$ як фонд заміщувати зношування прядільної
машини. Це сходило б просто на те, що $X$, незалежно від своєї
продукції, її продукту й продажу його, має in petto 200\pound{ ф. стерл.} для
того, щоб виплатити самому собі вартість, яку втрачає його прядільна
машина, тобто, що він, крім вартости, втрачуваної його прядільною машиною,
а вона доходить 200\pound{ ф. стерл.}, мусить щороку додавати з своєї
кишені ще по 200\pound{ ф. стерл.} грішми для того, щоб, кінець-кінцем, мати
змогу купити нову прядільну машину.

Та це безглуздя лише позірне. Кляса II складається з капіталістів,
що їхній основний капітал перебуває на цілком різних ступенях своєї
репродукції. Для одних уже надійшов час, коли його треба цілком замістити
in natura. У других основний капітал більш або менш далекий
від цієї стадії; для всіх членів останнього підрозділу спільне те, що їхній
основний капітал покищо не репродукується реально, тобто не відновлюється
in natura, не заміщується новим екземпляром того самого роду,
але його вартість послідовно нагромаджується в грошах. Перша частина
капіталістів перебуває цілком (або почасти, — це тут не має значення)
в такому самому стані, як при відкритті свого підприємства, коли капіталісти
з грошовим капіталом виступили на ринок, щоб перетворити його,
з одного боку, на (основний та обіговий) сталий капітал, а з другого
боку — на робочу силу, на змінний капітал. Як і тоді, їм тепер доводиться
знову авансувати цей грошовий капітал для циркуляції, — тобто доводиться
авансувати вартість сталого основного капіталу, цілком так само, як і вартість
обігового і вартість змінного капіталу.

\looseness=-2
Отже, коли припускається, що з 400\pound{ ф. стерл.}, подаваних в циркуляцію
клясою капіталістів II для обміну з І, одна половина походить від
\parbreak{}  %% абзац продовжується на наступній сторінці

\parcont{}  %% абзац починається на попередній сторінці
\index{ii}{0356}  %% посилання на сторінку оригінального видання
таких капіталістів підрозділу II, які мусять не тільки замістити за допомогою
своїх товарів свої засоби продукції, що належать до обігового
капіталу, а й поновити за допомогою своїх грошей свій основний
капітал in natura, тимчасом як друга половина капіталістів II своїми
грішми заміщують  in natura тільки обігову частину свого сталого капіталу,
але не відновлюють свій основний капітал  in natura, то при такому
припущенні немає жодної суперечности в тому, що 400\pound{ ф. стерл.}, які зворотно
припливають (вони припливають, скоро І купує на них засоби
споживання), різно розподіляються між цими двома підрозділами капіталістів
II.~Вони припливають назад до кляси II, але не повертаються в ті
самі руки, а різно розподіляються всередині цієї кляси, переходячи від
однієї частини її до іншої.

Одна частина капіталістів II, крім частини засобів продукції, заміщуваної,
кінець-кінцем, її товарами, перетворила 200\pound{ ф. стерл.} грішми на
нові елементи основного капіталу  in natura. Їхні гроші, таким чином
витрачені, — як і при відкритті підприємства — повертаються до них з циркуляції
лише протягом послідовного ряду років, як заміщення зношуваної
складової частини вартости основного капіталу, що перенесена на товари,
продуковані за допомогою цього основного капіталу.

Навпаки, друга частина капіталістів II на 200\pound{ ф. стерл.} не одержала
жодних товарів від І, але І платить їм тими грішми, що на них перша
частина капіталістів II купила елементи основного капіталу. Одна частина
капіталістів II знову має свою основну капітальну вартість у відновленій
натуральній формі, друга ще дбає про те, щоб нагромадити цю вартість
у грошовій формі для наступного заміщування свого основного капіталу
in natura.

Стан, що з нього нам треба виходити після попередніх, обмінів, це —
решта товарів, що їх треба обміняти з обох боків: $400 m$ у І підрозділі, $400 с$
у II\footnote{
Цифри знову не відповідають попередньому припущенню. Але це тут не має
значення, бо тут мають силу тільки відношення. \emph{Ф.~Е.}
}. Ми припускаємо, що II авансує 400 грішми для обміну цих товарів
на суму в 800. Половину цих 400 (\deq{} 200) в усякому разі мусить подати
та частина II~$с$, що нагромадила 200 грішми як вартість зношування
і що тепер повинна знову перетворити їх на натуральну форму свого
основного капіталу.

Цілком так само, як стала капітальна вартість, змінна капітальна вартість
і додаткова вартість — що на них можна розкласти вартість товарових
капіталів так II, як і І — можуть бути виражені в окремих пропорційних
частинах самих товарів II, зглядно товарів І, — цілком так само
може бути виражена й та частина вартости самої сталої капітальної вартости,
яку ще не доводиться перетворювати на натуральну форму основного
капіталу, але треба покищо поступінно нагромаджувати в грошовій
формі як скарб. Певна кількість товарів II (отже, в нашому прикладі —
половина остачі \deq{} 200) є тут лише носій цієї вартости зношування, що
має в наслідок обміну осісти в грошовій формі. (Перша частина капіталістів
\index{ii}{0357}  %% посилання на сторінку оригінального видання
II, яка відновлює основний капітал in natura, за допомогою відповідної
зношуванню частини тієї товарової маси, що від неї тут фігурує
лише остача, можливо, вже таким чином реалізувала частину його зношеної
вартости; але їм лишається ще реалізувати таким чином
200 в грошах).

Далі, щодо другої половини (\deq{} 200) тих 400\pound{ ф. стерл.}, що їх II
подав у циркуляцію при цій прикінцевій операції, то на неї купується
у I обігові складові частини сталого капіталу. Частину цих 200\pound{ ф. стерл.}
подали в циркуляцію, можливо, обидві частини капіталістів II або тільки
та частина, яка не відновлює in natura основної складової частини
вартости.

Отже, за допомогою 400\pound{ ф. стерл.} з I підрозділу вилучено: 1) на
суму в 200\pound{ ф. стерл.} таких товарів, що складаються лише з елементів
основного капіталу, 2) на суму в 200\pound{ ф. стерл.} таких товарів, що заміщують
in natura лише елементи обігової частини сталого капіталу II.~I продав тепер увесь свій річний товаровий продукт, оскільки його доводиться
продати II підрозділові; але вартість однієї п’ятої цього продукту
400\pound{ ф. стерл.} тепер існує в його руках у грошовій формі. Однак ці
гроші є перетворена на гроші додаткова вартість, яку доводиться витратити
як дохід на засоби споживання. Отже, I на ці 400\pound{ ф. стерл.} купує
в II всю товарову вартість \deq{} 400. Таким чином, гроші допливають назад
до II, вилучаючи його товари.

Припустімо тепер три випадки. При цьому ту частину капіталістів II,
яка заміщує основний капітал in natura, ми називаємо „частина 1“, а
ту, що нагромаджує в грошовій формі вартість зношування основного
капіталу, називаємо „частина 2“. Три випадки такі: a) певна частина тих
400, що як остача існують ще в II підрозділі в товарах, має замістити певну
частину обігових частин сталого капіталу для „частини 1“ і „частини 2“
(наприклад, по \sfrac{1}{2}); b) „частина 1“ уже продала ввесь свій товар, отже,
„частина 2“ ще повинна продати 400; c) „частина 2“ продала все, крім
тих 200, що є носії вартости зношування.

Тоді маємо такі розподіли:

a) З товарової вартости $\deq{} 400 с$, яка ще лишається в руках II, частині
1 належить 100 і частині 2--300; 200 з цих 300 репрезентують
зношування. В цьому разі з тих 400\pound{ ф. стерл.} грішми, що їх I тепер
подає назад, щоб одержати товари II, частина 1 спочатку витратила 300,
— а саме 200 грішми, що ними вона вилучила з I елементи основного
капіталу in natura, і 100 грішми для упосереднення свого обміну товарами
з І; навпаки, частина 2 з цих 400 авансувала тільки \sfrac{1}{4}, тобто
100 — так само для упосереднення свого товарового обміну з I.

Отже, з цих 400 грішми частина 1 авансувала 300 і частина
2--100.

Але з цих 400 повертаються назад:

До частини 1: 100, отже, лише \sfrac{1}{3} авансованих нею грошей. Але
замість решти, \sfrac{2}{3}, вона має відновлений основний капітал вартістю в 200.
За цей основний елемент капіталу вартістю в 200 вона дала І підрозділові
\index{ii}{0358}  %% посилання на сторінку оригінального видання
гроші, але не дала потім жодного товару. Щодо цих \sfrac{2}{3}
авансованих нею грошей, частина 1 виступає проти підрозділу I лише як
покупець, але не виступає ще потім як продавець. Отже, ці гроші
не можуть повернутись до частини 1: інакше сталось би, що вона одержала
елементи основного капіталу в подарунок від I. — Щодо останньої
третини авансованих нею грошей, частина 1 виступає спочатку
як покупець обігових складових частин свого сталого капіталу. На ці
самі гроші підрозділ I купує в частини 1 решту її товару вартістю
в 100. Отже, гроші повертаються до неї (до частини 1 підрозділу II)
назад, бо вона виступає як продавець товарів одразу після того, як
виступала покупцем. Коли б вони не повернулись, то сталося б, що
підрозділ II (частина 1) дав підрозділові I за товари в сумі на 100 спочатку
100 грішми, а потім ще 100 товаром, отже, подарував би йому
свій товар.

Навпаки, до частини 2, яка витратила 100 грішми, повертається
300 грішми: 100 — тому, що вона спочатку як покупець подала в циркуляцію
100 грішми, а потім одержала їх назад як продавець; 200 —
тому, що вона функціонує тільки як продавець товарів, на суму вартости
в 200, але не як покупець. Отже, гроші не можуть повернутись до I.~Отже, зношування основного капіталу покривається грішми, що їх II
(частина 1) подав у циркуляцію на закуп елементів основного капіталу;
але вони потрапляють до рук частини 2 не як гроші частини 1, а як
гроші, що належать підрозділові I.

b) При цьому припущенні решта II~$c$ розподіляється так, що частина
1 має 200 грішми, а частина 2--400 в товарах.

Частина 1 продала всі свої товари, але 200 в грошах є перетворена
форма основної складової частини її сталого капіталу, яку треба відновити
in natura. Отже, частина 1 виступає тут лише як покупець і замість
своїх грошей одержує на ту саму суму вартости товари І в формі натуральних
елементів основного капіталу. Частині 2 доводиться подати в
циркуляцію (коли I не авансував грошей для обміну товарів між I і II)
maximum лише 200\pound{ ф. стерл.}, бо в розмірі половини своєї товарової
вартости вона є лише продавець підрозділові I, а не покупець у
підрозділу I.

З циркуляції повертаються 400\pound{ ф. стерл.} до частини 2; 200 — тому,
що вона їх авансувала як покупець і одержує їх назад як продавець
товарів на 200; 200 — тому, що вона продає підрозділові I товарів вартістю
на 200, не одержуючи за це товарового еквіваленту від I.

c) Частина 1 має 200 в грошах і $200 c$ в товарах; частина 2 —
$200 c$ (d) в товарах.

Частина 2 при цьому припущенні не має авансувати грішми нічого,
бо вона проти підрозділу I взагалі вже функціонує не як покупець, а
лише як продавець, отже, їй треба чекати, поки в неї куплять.

Частина 1 авансує 400\pound{ ф. стерл.} грішми; 200 для взаємного обміну
товарами з I, 200 — просто як покупець у I.~На ці останні 200\pound{ ф. стерл.}
грішми вона купує елементи основного капіталу.


\index{ii}{0359}  %% посилання на сторінку оригінального видання
Підрозділ I на 200\pound{ ф. стерл.} грішми купує в частини 1 товару на
200, і в наслідок цього до неї повертаються її 200\pound{ ф. стерл.} грішми,
авансовані на цей товаровий обмін; на другі 200\pound{ ф. стерл.}, — що їх він
теж одержав від частини 1, — I купує товарів на 200 у частини 2, і в
наслідок цього в останньої зношування її основного капіталу осідає в
формі грошей.

\vtyagnut{}
Справа зовсім не змінилась би, коли б ми припустили, що в випадку
$c$ не кляса II (частина 1), а кляса I авансує 200 грішми на обмін наявних
товарів. Потім, коли I купить спочатку у II, частини 2, товарів на
200, — припускається, що частині 2 лишається ще продати тільки цю
решту товару, — то ці 200\pound{ ф. стерл.} не повертаються до I, бо II, частина
2, не виступає знову як покупець; але II, частина 1, має тоді 200\pound{ ф. стерл.}
грішми для закупу, а також ще на 200 товарів для обміну, отже, всього
400 для обміну з I.~Тоді 200\pound{ ф. стерл.} грішми від II, частини 1, повертаються
назад до I.~Коли I знову витрачає їх, щоб купити товару на
200 у II, частини 1, то вони знову повернуться до нього, коли II,
частина 1, візьме в I другу половину 400 в товарах. Частина 1 (II)
витратила 200\pound{ ф. стерл.} грішми просто як покупець елементів основного
капіталу; тому вони не повертаються до неї, а служать для того,
щоб перетворити на гроші $200 c$, решту товарів II, частини 2, тимчасом
як до I гроші, витрачені ним на товаровий обмін, 200\pound{ ф. стерл.}, повертаються
не через II, частину 2, а через II, частину 1. За його товари
на 400 до нього повернувся товаровий еквівалент розміром в 400;
200\pound{ ф. стерл.} грішми, авансовані ним для обміну товарів на 800, теж
повернулись до нього й таким чином усе гаразд.

\fancybreak{*\quad*\quad*}
\medskip{}
\noindent{}Труднощі, які постали при обміні:

I. $\underbrace{1000 v \dplus{} 1000 m}^{}$

II.\makebox[20pt]{}$2000 с$, зійшли на труднощі при обміні решток:

I.\makebox[25pt]{\dotfill{}}$400 m$

II. (1) 200 грішми \dplus{} $200 c$ товарами \dplus{} (2) $200 c$ товарами, або, щоб
подати справу ще ясніше:

I.  $200 m \dplus{} 200 m$

II. (1) 200 грішми \dplus{} $200 c$ товарами \dplus{} (2) $200 c$ товарами.

А що для II, частини 1, $200 c$ товарами обмінюємся на 200 I~$m$
(товарами) і що всі гроші, які циркулюють між І і II при цьому обміні
товарів на 400, повертаються назад до того, хто їх авансував, — або до
I або до II, — то ці гроші, як елемент обміну між I і II, в суті не є
елемент тієї проблеми, яка цікавить нас тепер. Або в іншому освітленні:
коли ми припустимо, що в обміні між 200 I~$m$ (товарами) і 200 II~$c$
(товарами II, частина 1) гроші функціонують як засіб виплати, а не як
купівельний засіб, а тому й не як „засіб циркуляції“ у вузькому значенні,
то зрозуміло, — бо товари 200 I~$m$ і 200 II~$c$ (частина 1)
рівні величиною вартості, — що засоби продукції вартістю в 200 обмінюються
на засоби споживання вартістю в 200, що гроші тут функціонують
лише ідеально, і що в дійсності зовсім не доводиться ні з того,
\parbreak{}  %% абзац продовжується на наступній сторінці

\parcont{}  %% абзац починається на попередній сторінці
\index{ii}{0360}  %% посилання на сторінку оригінального видання
ні з другого боку подавати гроші в циркуляцію для оплати ріжниці.
Отже, проблема виступає в своєму чистому вигляді лише тоді, коли ми
викреслимо на обох боках, І і II, товар 200 I~$m$ і його еквівалент товар
200 ІІ~$с$ (частини 1).

Отже, усунувши ці дві товарові величини рівної вартости (І і II),
що навзаєм одна одну урівноважують, матимемо решту обміну, в якому
проблема виступає в чистому вигляді, а саме:

I. 200 $m$ товаром.

II. (1) 200$c$ грішми \dplus{} (2) 200$c$ товаром.

Тут очевидно: II, частина 1, на 200 грішми купує складові частини
свого основного капіталу, 200 І~$m$, в наслідок цього основний капітал II,
частини 1, відновлено in natura, а додаткова вартість І, вартістю в 200,
з товарової форми (засоби продукції, — а саме, елементи основного
капіталу) перетворена на грошову форму. На ці гроші І купує засоби
споживання у II, частини 2, а результат для II такий, що для частини 1
відновлено in natura основну складову частину її сталого капіталу; і що
для частини 2 друга складова частина (яка заміщує зношування основного
капіталу) осіла в формі грошей, і це щороку повторюється доти,
доки й цю складову частину треба буде відновити in natura.

Попередня умова тут, очевидно, в тому, щоб ця основна складова частина
сталого капіталу II, яка в розмірі всієї своєї вартости зворотно перетворюється
на гроші, і яку, отже, кожного року треба відновлювати in natura
(частина 1), — щоб вона була рівна річному зношуванню другої основної
складової частини сталого капіталу II, яка все ще й далі функціонує в
своїй старій натуральній формі, і зношування якої — втрату вартости,
переношувану на товари, що в їхній продукції функціонує ця частина —
спочатку треба замістити грішми. Тому така рівновага з’являється як
закон репродукції в незмінному маштабі; це значить, інакше кажучи, що
пропорційний поділ праці в клясі І, яка продукує засоби продукції,
мусить лишатись незмінний, оскільки вона дає, з одного боку, обігові,
а з другого — основні складові частини сталого капіталу підрозділові II.

Перш ніж ближче дослідити це, ми повинні розглянути, який вигляд
матиме справа, коли решта суми II~$с$ (1) не дорівнюватиме решті II~$с$
(2); вона може бути більша або менша за цю останню. Візьмімо один
по одному обидва ці випадки.

\so{Перший випадок:}

I.    200 $m$.

II. (1) 220 $с$ (грішми) \dplus{} (2) 200 $с$ (товаром).

Тут II~$с$ (1) на 220\pound{ ф. стерл.} грішми купує товари 200 I~$m$, а І на
ті самі гроші купує товари 200 ІІ~$с$ (2), тобто ту складову частину
основного капіталу, яка має осісти в грошовій формі; її перетворено
таким чином на гроші. Але 20 II~$с$ (1) грішми не сила перетворити на
основний капітал in natura.

Цьому лихові можна, здається, запобігти, коли ми припустимо, що решта
I~$m$ дорівнює не 200, а 220, так що з суми 2000 І попереднім обміном закінчено
\index{ii}{0361}  %% посилання на сторінку оригінального видання
справу не з 1800, а лише з 1780. Отже, в такому разі матимемо:

I.    220 $m$

II. (1) 220 $с$ (грішми) \dplus{} (2) 200 $с$ (товаром).

ІІ~$с$, частина 1, на 220\pound{ ф. стерл.} грішми купує 220 І~$m$, а І на
200\pound{ ф. стерл.} купує потім 200 II~$с$ (2) товаром. Але тоді на боці І
лишається 20\pound{ ф. стерл.} грішми — така частина додаткової вартости, яку
I може лише затримати в грошах, а не витрачати на засоби споживання.
Таким чином труднощі лише переміщено з II~$с$ (частина 1) на І~$m$.

Припустімо тепер, з другого боку, що ІІ~$с$, частина 1, менше, ніж ІІ~$с$
(частина 2), отже:

\so{Другий випадок:}

I.    200 $m$ (товаром).

II. (1) 180 $с$ (грішми) \dplus{} (2) 200 $с$ (товаром).

II. (частина 1), на 180\pound{ ф. стерл.} грішми купує товари 180 І~$m$; на ці
гроші І купує в II (частини 2) товари такої самої вартости, тобто 180 II~$с$ (2); на одному боці лишається 20 І~$m$, що їх не сила продати, і так
само — 20 II~$с$ (2) на другому боці; товари вартістю в 40 не сила перетворити
на гроші.

Коли б ми припустили, що остача I \deq{} 180, це нам ані трохи не допомогло
б; правда, тоді в І не залишилося б жодного надлишку, але в II~$с$
(частині 2), як і раніш, був би надлишок в 20, що його не сила продати,
перетворити на гроші.

В першому випадку, де II (1) більше, ніж II (2), на боці II~$с$ (1)
лишається надлишок в грошах, що його не сила перетворити знову на
основний капітал, або, коли ми припустимо, що остача І~$m$ \deq{} ІІ~$с$ (1) на
боці І~$m$ буде такий самий надлишок у грошах, не перетворюваний на
засоби споживання.

В другому випадку, де ІІ~$с$ (1) менше, ніж ІІ~$с$ (2), виявляється грошовий
дефіцит на боці 200 І~$m$ і II~$с$ (2) і на обох боках товаровий надлишок
однакової величини, або коли припустити, що остача І~$m$ \deq{} ІІ~$с$ (2), дефіцит
в грошах і надлишок у товарі на боці II~$с$ (2).

Коли б ми припустили, що остачі І завжди дорівнюють ІІ~$с$ (1), — бо
продукція визначається замовленнями і в репродукції нічого не змінюється,
коли поточного року випродукувано більше основних складових частин
капіталу, а другого наступного року більше обігових складових частин
сталого капіталу II і І, — то в першому випадку І~$m$ можна було б знову
перетворити на засоби споживання лише тоді, коли б І купив на І~$m$
частину додаткової вартости у II, отже, коли б І її не споживав, а нагромаджував
як гроші; в другому випадку лихові можна було б запобігти
лише тоді, коли б І сам витратив гроші, — а цю гіпотезу ми
відкинули.

Коли ІІ~$с$ (1) більше, ніж ІІ~$с$ (2), то для реалізації грошового надлишку
в І~$m$ потрібен довіз закордонних товарів. Коли ІІ~$с$ (1) менше, ніж ІІ~$с$
(2), то для того, щоб реалізувати зношену частину II~$с$ в засобах продукції,
\index{ii}{0362}  %% посилання на сторінку оригінального видання
потрібен, навпаки, вивіз товару II (засобів споживання). Отже, в
обох випадках потрібна зовнішня торговля.

Даймо навіть, що при вивчанні репродукції в незмінному маштабі
треба припустити, що продуктивність усіх галузей продукції, а значить,
і пропорційні відношення вартостей товарових продуктів цих галузей,
лишаються незмінні, — все ж обидва останні випадки, де ІІ~$с$ (1) більше
або менше, ніж II~$с$ (2), являли б інтерес при вивчанні продукції в поширеному
маштабі, де, безперечно, можуть настати ці випадки.

\subsubsection{Результати}

Щодо заміщення основного капіталу, то взагалі треба зазначити ось що.

Коли — припускаючи, що всі інші умови, а значить, не лише маштаб
продукції, а зокрема й продуктивність праці лишаються незмінні, —
поточного року відмирає більша частина основного елемента ІІ~$с$, ніж у
попередньому році, а тому й більшу частину треба відновлювати in
natura, то та частина основного капіталу, яка є лише на шляху до своєї
смерти й яку до моменту її смерти треба покищо заміщувати в грошах,
теж мусить зменшитись у такій самій пропорції, бо згідно з припущенням
сума (також і сума вартости) основної частини капіталу, діющої в II
лишається та сама. Але це тягне за собою такі обставини. \emph{Поперше}. Коли
більша частина товарового капіталу І складається з елементів основного
капіталу II~$с$, то відповідно менша частина складається з обігових складових
частин ІІ~$с$, бо вся продукція І для ІІ~$с$ лишається незмінна. Коли одна
частина збільшується, то друга зменшується й навпаки. Але, з другого
боку, величина всієї продукції кляси II такожа лишається незмінна. Як
же можливо це, коли меншає в неї сировинних матеріялів, напівфабрикатів,
допоміжних матеріялів (тобто обігових елементів сталого капіталу II)?
\emph{Подруге}. Більша частина основного капіталу II~$с$, знову відновленого в
грошовій формі, припливає до І, щоб знову перетворитись з грошової
форми на натуральну форму. Отже, до І, крім грошей, що циркулюють
між І і II для простого товарного обміну, припливає більше грошей;
більше таких грошей, які правлять не за посередника у взаємному товаровому
обміні, а однобічно виступають лише в функції купівельного
засобу. Але разом з тим пропорційно зменшилась би товарова маса
II~$с$, що є носій вартости на заміщення зношування, тобто та товарова
маса II, яка мусить бути перетворена не на товари І, а лише на гроші І.~Від II до І приплило б більше грошей як просто купівельних засобів і
було б менше товарів у II, що супроти них І мав би функціонувати як
простий покупець. Більшу частину І~$m$, — бо І~$v$ уже перетворено на товари
II, — не сила було б перетворити на товари II, її довелось би затримати в
грошовій формі.

Зворотний випадок, — коли протягом року репродукція відмерлого
основного капіталу II менша і навпаки, частина на заміщення зношування
більша, — після попереднього не потребує дальшого розгляду.

І, таким чином, настала б криза — криза продукції — не зважаючи на
репродукцію в незмінному маштабі.


\index{ii}{0363}  %% посилання на сторінку оригінального видання
Коротко кажучи, коли б при простій репродукції та інших незмінних
обставинах, отже, при незмінних продуктивній силі, загальній масі та
інтенсивності праці, — ми припустили нестале відношення між відмерлим
(що потребує відновлення) і далі діющим в старій натуральній формі
(що просто долучає вартість до продуктів на заміщення свого зношування)
основним капіталом, то в одному випадку маса обігових складових частин,
що їх треба репродукувати, лишилась би та сама, але збільшилась би
маса основних складових частин, що їх треба репродукувати; отже, вся
продукція І мусила б збільшитись або, навіть лишаючи осторонь грошові
відношення, постав би дефіцит в репродукції.

В другому випадку: коли б відносна величина основного капіталу II,
що його треба репродукувати in natura, зменшилась, а тому збільшилась
би в такому ж відношенні та складова частина основного капіталу II,
яку покищо треба замістити лише в грошах, то маса обігових складових
частин сталого капіталу II, репродукованих І, лишилась би незмінна, а
маса основних частин, що їх треба репродукувати, навпаки, зменшилась
би. Отже, або зменшення всієї продукції І, або надлишок (як раніш був
дефіцит) і до того надлишок, що його не сила перетворити на гроші.

Правда, в першому випадку та сама праця при збільшені продуктивності,
протягу та інтенсивності могла б дати більший продукт, і таким
чином можна було б покрити дефіцит у першому випадку; але така
зміна не могла б статись без переміщення праці й капіталу з однієї галузі
продукції І в іншу, а всяке таке переміщення одразу ж викликало б
розлади. А подруге, І підрозділові довелось би (оскільки збільшуються
протяг та інтенсифікація праці) обміняти більшу вартість на
меншу вартість II, отже, сталось би знецінення продукту І.

Зворотне було б у другому випадку, де підрозділ І мусить скорочувати
свою продукцію, а це означає кризу для зайнятих у ньому робітників
і капіталістів, або він дає надлишок, а це знову таки є криза.
Самі по собі такі надлишки є не лихо, а вигода, але при капіталістичній
продукції вони є лихо.

Зовнішня торговля могла б допомогти в обох випадках; в першому
випадку, — щоб товар І, утримуваний в грошовій формі, перетворити на
засоби споживання; в другому випадку, — щоб збути товаровий надлишок.
Але зовнішня торговля, оскільки вона не просто заміщує елементи (також
і за вартістю), лише відсуває суперечності в ширшу сферу, відкриває їм
більший простір.

Коли усунути капіталістичну форму репродукції, то справа сходить
на те, що розмір частини основного капіталу, яка відмирає й тому
повинна заміщуватись in natura (тут капіталу, що функціонує в продукції
засобів споживання), змінюється в різні послідовні роки. Коли одного
року ця частина дуже велика (перевищує середню смертність, як це
буває з смертністю людей), то в наступному році вона, певно, буде
настільки ж менша.

Але від цього маса сировинних матеріялів, напівфабрикатів і допоміжних
матеріялів, потрібна для річної продукції засобів споживання, — припускаючи,
\index{ii}{0364}  %% посилання на сторінку оригінального видання
що інші умови лишились ті самі, — не змінюється; отже, вся
продукція засобів продукції мусила б в одному випадку поширитись, в
другому скоротитись. Цьому можна було б запобігти лише постійною
відносною перепродукцією; з одного боку, продукується основного
капіталу на певну кількість більше, ніж безпосередньо треба; з другого
боку, продукується такий запас сировинного матеріялу та інш., що
перевищує безпосередні річні потреби (це особливо стосується до засобів
існування). Такий рід перепродукції рівнозначний контролеві суспільства
над речовими засобами його власної репродукції. Але в капіталістичному
суспільстві вона є анархічний елемент.

Цей приклад з основним капіталом — при незмінному маштабі репродукції
— є разючий. Непропорційність у продукції основного та обігового
капіталу це — одна з улюблених економістами причин, що ними вони
пояснюють кризи. А що така непропорційність може й мусить поставати
при простому підтриманні основного капіталу, що вона може й
мусить поставати при припущенні ідеальної нормальної продукції, при
простій репродукції уже діющого суспільного капіталу, це для них — щось
нове.

\subsection{Репродукція грошового матеріялу}

До цього часу ми зовсім не звертали уваги на один момент, а саме
на річну репродукцію золота й срібла. Як простий матеріял для речей
розкошів, позолочування тощо, вони так само, як і всякі інші продукти,
не заслуговували б тут на особливу згадку. Навпаки, як грошовий
матеріял, а тому і як потенціяльні гроші, вони відіграють важливу ролю.
Для спрощення ми будемо вважати тут за грошовий матеріял тільки золото.

За старими даними вся річна продукція золота становила 800--900
тисяч фунтів \deq{} заокруглюючи 1100 або 1250 мільйонів марок. Навпаки,
за Зетбеером\footnote{Ad.~Soetbeer, „Edelmetall-Produktion“. 1879, S. 112.} пересічно за 1871--75 роки лише \num{170.675} кг вартістю
в округлих цифрах 476 мільйонів марок. З цього давали: Австралія
округло 167, Сполучені Штати 166, Росія 93 мільйони марок. Решта
розподіляється між різними країнами на суму меншу, ніж 10 мільйонів
марок на кожну. Річна продукція срібла за той самий період становила
трохи менш, ніж 2 мільйони кілограмів вартістю на 354\sfrac{1}{2} мільйони
марок; з цього Мехіко давало округло 108, Сполучені Штати 102,
Південна Америка 67, Німеччина 26 мільйонів і~\abbr{т. ін.}

З країн, де панує капіталістична продукція, лише Сполучені Штати
є продуценти золота й срібла; європейські капіталістичні країни майже
все своє золото й переважну більшість свого срібла одержують з
Австралії, Сполучених Штатів, Мехіко, Південної Америки та Росії.

Але ми переносимо золоті копальні в ту країну капіталістичної продукції,
що її річну репродукцію ми тут аналізуємо, і робимо так ось з
яких міркувань.

Капіталістична продукція взагалі не існує без зовнішньої торговлі.
Але коли ми припускаємо нормальну річну репродукцію в даному маштабі,
\index{ii}{0365}  %% посилання на сторінку оригінального видання
ми тим самим припускаємо, що зовнішня торговля лише заміщує
тубільні предмети предметами іншої споживної або натуральної форми, не
впливаючи при цьому на відношення вартости, а значить, і на ті відношення
вартости, що в них обмінюються одна на одну дві категорії:
засоби продукції та засоби споживання, і так само не впливаючи на
відношення між сталим капіталом, змінним капіталом та додатковою
вартістю, що на них можна розкласти вартість продукту кожної з цих
двох категорій. Отже, притягнення зовнішньої торговлі до аналізи щорічно
репродукованої вартости продукту може лише заплутати справу, не даючи
жодного нового моменту ні для проблеми, ні для її розв’язання. Отже,
тут треба цілком абстрагуватись від неї; тому золото треба вважати тут
за безпосередній елемент річної репродукції, а не за довожуваний з-зовні
в наслідок обміну товаровий елемент.

Продукція золота, як і взагалі продукція металів, належить до кляси І,
до категорії, яка охоплює продукцію засобів продукції. Припустімо, що
річна продукція золота \deq{} 30 (для зручности; а дійсно цифра ця дуже
висока порівняно з числами нашої схеми); хай ця вартість розпадається
на $20с \dplus{}  5v \dplus{} 5m$; $20с$ треба обміняти на інші елементи І~$с$, і це ми розглянемо
потім; a $5v \dplus{} 5m$ (І) треба обміняти на елементи ІІс, тобто на
засоби споживання.

Щодо $5v$, то кожне підприємство, яке продукує золото, починає з
закупу робочої сили: не на золото, спродуковане в самому цьому підприємстві,
а на деяку масу грошей, наявних у країні. На ці $5v$ робітники
купують засоби споживання в II, а цей на ці гроші купує засоби продукції
в І. Коли II купує, скажімо, на $2v$ І золото як товаровий
матеріял і~\abbr{т. ін.} (складову частину свого сталого капіталу), то до продуцента
золота І повертаються $2v$ в грошах, що вже раніше належали
циркуляції. Коли II не купує в І далі матеріялу, то І купує в II, подаючи
своє золото як гроші в циркуляцію, бо на золото можна купити всякий
товар. Ріжниця тільки в тому, що І виступає тут не як продавець, а
лише як покупець. Золотопромисловці І можуть завжди збути свій товар;
він завжди є в такій формі, що його можна безпосередньо обміняти.

Припустімо, що прядун заплатив своїм робітникам $5 v$, а вони дають
йому за це, — лишаючи осторонь додаткову вартість, — пряжу в продукті \deq{} 5;
робітники на 5 купують у II~$с$, останній купує на 5 грішми пряжу в І, і
таким чином $5v$ грішми повертаються назад до прядуна. Навпаки, в
щойно припущеному випадку І~$з$ (так ми позначатимемо продуцента
золота) авансує своїм робітникам $5v$ грішми, що вже раніш належали
циркуляції; робітники витрачають ці гроші на засоби існування; але з
5 тільки 2 повертаються від II до І~$з$. Однак І~$з$ цілком так само, як і
прядун, може знову почати процес репродукції; бо його робітники дали
йому золотом 5, що з них він продав 2, а решту 3 має в формі золота, —
отже, йому доводиться тільки карбувати з них монету\footnote{
„Значну кількість золотих зливків (gold bullion) приставляють продуценти
золота безпосередньо до карбівниці в Сан-Франціско“. — Reports of Н.~М.~Secretaries
of Embassy and Legation. 1879. Part III, p. 337.
} або перетворити
\index{ii}{0366}  %% посилання на сторінку оригінального видання
їх на банкноти — і тоді ввесь його змінний капітал прямо, без
дальшого посередництва II, знову опиняється в його руках у грошовій
формі.

Але вже при цьому першому процесі річної репродукції сталася зміна
в кількості грошей, що дійсно або віртуально належать циркуляції. Ми
припустили, що II~$с$ купив $2v$ (І~$з$) як матеріял, а 3 як грошову форму
змінного капіталу І~$з$ знову витратив в межах II. Отже, з тієї маси грошей,
що її дано в наслідок нової продукції грошей, 3 лишились в межах
II й не повернулись до І. Згідно з нашим припущенням, II задовольнив
свою потребу в грошовому матеріялі. 3 лишаються в його руках як
золотий скарб. А що вони не можуть становити будь-якого елемента
його сталого капіталу й що, далі, II вже раніш мав достатній грошовий
капітал на закуп робочої сили; що, далі, за винятком елемента зношування,
цими додатковими 3$з$ не доводиться виконувати жодної функції в межах II,
на частину якого їх обмінено (вони могли б служити лише для того,
щоб pro tanto покривати елемент зношування тоді, коли II~$с$ (І) менше,
ніж ІІ~$с$ (2), а це буває випадково); що, з другого боку, саме за винятком
елемента зношування, ввесь товаровий продукт ІІ~$с$ треба обміняти на
засоби продукції I ($v \dplus{} m$), — то ці гроші цілком доводиться перенести
з ІІ~$с$ в II~$m$, хоч це останнє буде в доконечних засобах існування або в
засобах розкошів, і, навпаки, відповідну товарову вартість доводиться
перенести з II~$m$ в II~$с$. Результат: частина додаткової вартости нагромаджується
як грошовий скарб.

На другий рік репродукції, коли таку саму частину щорічно продукованого
золота й далі зуживається як матеріял, 2 знову повернуться
до І~$з$, а 3 заміститься in natura, тобто знову звільняться у II як скарб
і~\abbr{т. ін.}

Взагалі щодо змінного капіталу: капіталістові І~$з$, як і всякому
іншому, завжди доводиться авансувати цей капітал в грошах на закуп
праці. На це $v$ не йому, а його робітникам доводиться купувати в II; отже,
ніколи не може бути такого випадку, щоб він виступав як покупець,
тобто подав гроші в II без ініціятиви II.~Але оскільки II купує в нього
матеріял, оскільки II мусить перетворювати свій сталий капітал II~$с$ на золотий
матеріял, частина (І~$з$) $v$ повертається від II до І з таким самим шляхом,
як і до інших капіталістів І; а оскільки цього не постає, він заміщує своє $v$
золотом безпосередньо з свого продукту. Але в тій самій мірі, що в ній $v$,
авансоване в грошовій формі, не повертається до нього від II, частина
вже наявних засобів циркуляції (гроші, що приплили від І до II й не повернулись
до І) перетворюється в II на скарб, і тому частину додаткової
вартости II не витрачається на засоби споживання. Що постійно відкриваються
нові золоті копальні, або відновлюються роботи на старих,
то певна частина грошей, що їх І~$з$ повинен витрачати на $v$, завжди
становить частину тієї маси грошей, яка була вже до нової продукції
золота, яку І~$з$ за допомогою своїх робітників подає в II, і, оскільки
вона не повертається з II до І~$з$, вона становить там елемент для
утворення скарбів.


\index{ii}{0367}  %% посилання на сторінку оригінального видання
Щодо І~$з$, то І~$з$ завжди може виступати тут як покупець; він подає
в циркуляцію своє $m$ як золото і вилучає за це засоби споживання II~$с$;.
в II золото почасти зуживається як матеріял, а тому функціонує як справжній
елемент сталої складової частини $с$ продуктивного капіталу II; а
оскільки цього не постає, воно знову таки стає елементом для утворення
скарбу як частини II~$m$, яка затримується в грошовій формі. Відси
видно — навіть лишаючи осторонь І~$с$, що його ми розглянемо потім\footnote{
В рукопису немає досліду про обмін новопродукованого золота, який відбувається
в межах сталого капіталу підрозділу І.\emph{ Ф.~Е.}
}, —
як навіть проста репродукція, хоч тут і виключено акумуляцію
у власному значенні слова, тобто репродукцію в поширеному маштабі,
все ж неминуче включає нагромаджування грошей або утворення скарбу.
А що це знову повторюється щороку, то цим пояснюється припущення,
що було нам за вихідний пункт при вивченні капіталістичної продукції,
а саме припущення, що на початку репродукції в руках кляси капіталістів
І і II є відповідна товаровому обмінові маса грошових засобів. Таке
нагромадження відбувається навіть, коли відлічити золото, втрачуване в
наслідок зношування грошей, що циркулюють.

Само собою зрозуміло, що чим старіша капіталістична продукція, тим
більша всюди нагромаджена маса грошей, і значить, тим відносно менша
та частина, що її нова річна продукція золота долучає до цієї маси, хоч
абсолютна величина цієї додачі може бути значна. Повернімося в загальних
рисах ще раз до заперечення, зробленого Тукові: як можливо, щоб кожен капіталіст
вилучав з річного продукту додаткову вартість грішми, тобто вилучав
з циркуляції більше грошей, ніж подав до неї, — як це можливо, якщо,
кінець-кінцем, саму клясу капіталістів доводиться розглядати як те джерело,
що з нього взагалі гроші подається в циркуляцію?

Підсумовуючи вище (розд. XVII) розвинуте, даємо на це таку відповідь:

1) Єдине, потрібне тут припущення: що взагалі є досить грошей для
того, щоб обміняти різні елементи маси річної репродукції, — зовсім не
порушується тією обставиною, що частина товарової вартости складається
з додаткової вартости. Коли б припустити, що вся продукція належить
самим робітникам і додаткова праця є, отже, додаткова праця лише для
них самих, а не для капіталістів, то маса товарової вартости в циркуляції
лишалась би та сама і, при інших незмінних обставинах, потребувала б
тієї самої маси грошей для своєї циркуляції. Отже, в обох випадках питання
ось у чому: відки беруться гроші для обміну всієї цієї товарової
вартости? — А зовсім не в тому, відки беруться гроші для перетворення
додаткової вартости на гроші.

Звичайно, — повертаємось ще раз до цього — кожен поодинокий товар
складається з $с \dplus{} v \dplus{} m$, і значить, для циркуляції всієї товарової маси
потрібна, з одного боку, певна грошова сума для циркуляції капіталу $с \dplus{} v$,
а з другого боку, потрібна друга грошова сума для циркуляції доходу
капіталістів, додаткової вартости $m$. Як для поодиноких капіталістів, так
і для цілої кляси гроші, що в них вона авансує капітал, відрізняються
\parbreak{}  %% абзац продовжується на наступній сторінці

\parcont{}  %% абзац починається на попередній сторінці
\index{ii}{0368}  %% посилання на сторінку оригінального видання
від грошей, що в них вона витрачає дохід. Відки беруться ці останні
гроші? Та просто з тієї маси грошей, що є в руках кляси капіталістів,
отже, взагалі та в цілому з усієї маси грошей, яка є в суспільстві,
деяка частина служить для циркуляції доходу капіталістів. Ми
вже бачили вище, як кожен капіталіст, що відкриває нове підприємство,
витрачає гроші на засоби споживання для власного утримання, потім,
коли підприємство вже працює, знову виловлює їх як гроші, що служать
для перетворення на гроші його додаткової вартости. Але загалом кажучи,
всі труднощі походять з двох джерел.

Поперше, коли ми розглядатимемо тільки циркуляцію та оборот капіталу,
отже, і капіталіста лише як персоніфікацію капіталу, а не як капіталістичного
споживача та розкішника, то, хоч ми побачимо, що він
постійно подає в циркуляцію додаткову вартість як складову частину
свого товарового капіталу, але ми ніколи не побачимо в його руках грошей
як форми доходу; ми ніколи не побачимо, щоб він подавав у циркуляцію
гроші для споживання додаткової вартости.

Подруге, коли кляса капіталістів подає в циркуляцію певну грошову
суму в формі доходу, то здається, ніби вона виплачує еквівалент за цю
частину цілого річного продукту, і тому ця остання перестає вже репрезентувати
додаткову вартість. Але додатковий продукт, що в ньому
втілено додаткову вартість, нічого не коштує клясі капіталістів. Як кляса,
вона має й користається з неї безплатно, і грошова циркуляція нічого
не може змінити в цьому. Зміна, зумовлена циркуляцією, є просто в тому,
що кожен капіталіст, замість споживати свій додатковий продукт in natura,
а це здебільша зовсім неможливо, витягує з усієї маси річного суспільного
додаткового продукту й привлащує різного роду товари на суму
привлащеної ним додаткової вартости. Але механізм циркуляції показав,
що коли кляса капіталістів подає в циркуляцію гроші на витрачання
доходу, то вона знову й вилучає з циркуляції ці гроші, а тому завжди
може знову розпочати той самий процес; що, отже, розглядувана як
кляса капіталістів, вона, як і раніш, має цю грошову суму, потрібну для
перетворення додаткової вартости на гроші. Отже, коли капіталіст не
лише вилучає з товарового ринку для свого споживного фонду додаткову
вартість у формі товарів, але разом з тим до нього повертаються назад
і гроші, що на них він купив ці товари, то, очевидно, що він вилучив
з циркуляції товари, не давши за них жодного еквіваленту. Вони нічого
не коштують йому, хоч він заплатив за них гроші. Коли я купую товарів
на фунт стерлінґів, а продавець товару повертає мені цей фунт за додатковий
продукт, що нічого не коштував мені, то я, очевидно, безплатно
одержав товари. Постійне повторення цієї операції нічого не змінює в
тому, що я постійно вилучаю товари й постійно лишаюсь власником
фунта стерлінґів, хоч, щоб одержати товари, я на деякий час віддаю
його. Капіталіст постійно одержує ці гроші назад як перетворену на
гроші додаткову вартість, яка нічого не коштувала йому.

Ми бачили, що в А.~Сміса сукупна суспільна вартість продукту розкладається
на дохід, на $v \dplus{} m$, отже, що стала капітальна вартість у
\parbreak{}  %% абзац продовжується на наступній сторінці

\parcont{}  %% абзац починається на попередній сторінці
\index{ii}{0369}  %% посилання на сторінку оригінального видання
нього дорівнює нулеві. З цього неодмінно випливає, що грошей, потрібних
для циркуляції річного доходу, досить і для циркуляції сукупного
річного продукту; що, отже, в нашому випадку, грошей, потрібних для
циркуляції засобів споживання вартістю в \num{3.000}, досить і для циркуляції
сукупного річного продукту вартістю в \num{9.000}. Такий справді погляд
А.~Сміта, і Т.~Тук повторює його. Це хибне уявлення про відношення
маси грошей, потрібної, щоб перетворити дохід на гроші, до маси грошей,
потрібної для циркуляції сукупного суспільного продукту, є неминучий
результат незрозумілого, непродуманого уявлення про спосіб, що
ним репродукуються й щороку заміщуються різні речові й вартісні елементи
сукупного річного продукту. Тому його вже й збито.

Послухаймо самого Сміта й Тука.

Сміт каже (книга II, розділ 2): „Циркуляцію кожної країни можна
розподілити на дві частини: циркуляцію між самими торговцями й циркуляцію
між торговцями й споживачами. Хоч ті самі грошові одиниці, — паперові
або металеві, — можуть застосовуватись то в одній, то в другій
циркуляції, однак, і та й друга безупинно відбуваються одночасно одна поряд
однієї, і тому кожна з них потребує певної маси грошей того або
іншого роду, щоб і далі продовжувати свій рух. Вартість товарів, що циркулюють
між різними торговцями, ніколи не може перевищити вартости
товарів, що циркулюють між торговцями й споживачами; бо, хоч що купують
торговці, вони мусять усе це, кінець-кінцем, продати споживачам.
А що циркуляція між торговцями відбувається en gros\footnote*{
Гуртом, оптом. \Red{Ред.}
}, то вона взагалі потребує
досить великих сум для кожного поодинокого обміну. Навпаки, циркуляція
між торговцями й споживачами відбувається здебільша en détail\footnote*{
На роздріб. \Red{Ред.}
}
і часто потребує лише дуже незначних грошових сум; часто досить одного
шилінґа або навіть половини пенні. Але невеличкі суми циркулюють
куди швидше, ніж великі\dots{} Тому, хоч річні закупи всіх споживачів принаймні
(чудове це „принаймні“) дорівнюють вартістю закупам усіх
торговців, однак, їх звичайно можна переводити куди меншою масою
грошей“ і~\abbr{т. ін.}

До цього місця Адама Т.~Тук („An Inquiry into the Currency Principle.
London, 1844“, стор. 34--36 passim) зауважує: „Не викликає жодного
сумніву, що ця подана тут ріжниця в суті правильна\dots{} Обмін між торговцями
й споживачами охоплює також і виплату заробітної плати, яка
являє головний дохід (the principal means) споживачів\dots{} Всі обміни між
торговцями, тобто всі продажі, починаючи від продуцента або імпортера,
переходячи всі щаблі посередніх процесів мануфактури й т. інш. і закінчуючи
роздрібним торговцем або купцем-експортером, можна звести
до рухів переміщення капіталу. Але переміщення капіталу не мають собі
за неодмінну передумову й на практиці справді при більшості обмінів не
призводять до того, щоб підчас переміщення справді передавалось банкноти
\index{ii}{0370}  %% посилання на сторінку оригінального видання
або монети, — я маю на увазі матеріяльну, а не фіктивну передачу\dots{}
Загальна сума взаємних обмінів між торговцями мусить, кінець-кінцем,
визначатись і обмежуватись сумою обмінів між торговцями й споживачами“.

Коли б у Тука остання теза була висловлена відокремлено, то можна
було б думати, що він просто констатує, що є співвідношення між обмінами
поміж самими торговцями і обмінами поміж торговцями й споживачами, —
інакше кажучи, співвідношення між вартістю сукупного річного доходу й
вартістю капіталу, що за допомогою його продукується дохід. Однак це
не так. Він прямо пристає на погляд А.~Сміта. Тому зайве було б
критикувати зокрема його теорію циркуляції.

2) Кожен промисловий капітал при відкритті підприємства одним заходом
подає в циркуляцію гроші на всю свою основну складову частину, яку
він знову вилучає лише поступінно, протягом ряду років, продаючи свій
річний продукт. Отже, спочатку він подає в циркуляцію більше грошей,
ніж вилучає з неї. Це повторюється кожного разу при відновленні цілого
капіталу in natura; не повторюється щороку для певного числа підприємств,
що їхні основні капітали доводиться відновлювати in natura; частинно
це повторюється при кожному ремонті, при кожному лише частинному
відновленні основного капіталу. Отже, коли одна сторона
вилучає з циркуляції більше грошей, ніж подає в неї, то друга сторона —
навпаки.

В усіх галузях промисловости, де період продукції (як величина
відмінна від робочого періоду) охоплює порівняно довгий час, капіталістичні
продуценти протягом цього періоду ввесь час подають гроші в циркуляцію,
— почасти на оплату застосованої робочої сили, почасти на закуп
засобів продукції, що їх треба застосувати; таким чином, засоби продукції
безпосередньо вилучаються з ринку, а засоби споживання почасти посередньо
через робітників, які витрачають свою заробітну плату, почасти
безпосередньо самими капіталістами, які зовсім не відкладають свого
споживання, і при цьому ці капіталісти спочатку не подають на ринок
жодного еквіваленту товарами. Гроші, що їх вони подають в циркуляцію,
протягом цього періоду служать для перетворення на гроші товарової
вартости, а втім і вміщеної в ній додаткової вартости. Дуже важливий
стає цей момент при розвиненій капіталістичній продукції, в довгочасних
підприємствах, що їх засновують акційні товариства і~\abbr{т. ін.}, як
от будування залізниць, каналів, доків, великих міських споруд, залізних
пароплавів, дренування ґрунту в широких розмірах і~\abbr{т. ін.}

3) Тимчасом як інші капіталісти, — лишаючи осторонь витрати на
основний капітал, — вилучають з циркуляції більше грошей, ніж подали в
неї, купуючи робочу силу та обігові елементи, капіталісти, що продукують
золото й срібло, — лишаючи осторонь благородний металь, що служить як
сировинний матеріял, — подають в циркуляцію тільки гроші, а вилучають
з неї тільки товари. Сталий капітал, за винятком зношеної частини,
більшу частину змінного капіталу й усю додаткову вартість, за винятком
скарбу, який, можливо, нагромаджується в їхніх власних руках, — усе це
як гроші подається в циркуляцію.


\index{iii1}{0371}  %% посилання на сторінку оригінального видання
Плата за управління, яку одержують управителі, як торговельних,
так і промислових підприємств, є цілком відокремленою від
підприємницького доходу як у кооперативних фабриках робітників,
так і в капіталістичних акційних підприємствах. Відокремлення
плати за управління від підприємницького доходу, яке в інших
випадках є випадковим, тут є постійне. У кооперативній фабриці
антагоністичний характер праці нагляду відпадає, бо управитель
оплачується робітниками, а не протистоїть їм як представник
капіталу. Акційні підприємства — які розвиваються разом
з кредитною справою — взагалі мають тенденцію все більше й
більше відокремлювати цю працю управління як функцію від
володіння капіталом, своїм власним, чи взятим у позику; цілком
так само, як з розвитком буржуазного суспільства функції суду
й управління відокремлюються від землеволодіння, атрибутами
якого вони були за феодальних часів. Але коли, з одного
боку, простому власникові капіталу, грошовому капіталістові,
протистоїть функціонуючий капіталіст, і з розвитком кредиту
цей грошовий капітал сам набирає суспільного характеру, концентрується
в банках і віддається в позику ними, а не його безпосередніми
власниками; коли, з другого боку, простий
управитель, який не володіє капіталом ні під яким титулом,
ні позиково, ні якнебудь інакше, виконує всі реальні функції,
які припадають функціонуючому капіталістові як такому, — тоді
лишається тільки службовець, а капіталіст, як зайва особа,
зникає з процесу виробництва.

З опублікованих звітів\footnote{
Наведені тут дані звітів доходять щонайбільше до 1864 року, бо вищесказане
було написано в 1865 році. — \emph{Ф.~Е.}
} кооперативних фабрик в Англії видно,
що — після відрахування плати управителя, яка становить частину
витраченого змінного капіталу цілком так само, як і плата
всіх інших робітників — зиск був більший, ніж пересічний зиск,
не зважаючи на те, що кооперативні фабрики подекуди платили
далеко вищий процент, ніж приватні фабриканти. Причиною
вищого зиску в усіх цих випадках була більша економія
в застосуванні сталого капіталу. Але нас при цьому цікавить те,
що тут пересічний зиск (= процент + підприємницький дохід)
фактично і наочно виступає як величина, цілком незалежна від
плати за управління. Тому що зиск був тут більший за пересічний
зиск, то й підприємницький дохід був більший, ніж взагалі.

Те саме явище спостерігається і в деяких капіталістичних акційних
підприємствах, наприклад, в акційних банках (Joint Stock
Banks). London and Westminster Bank у 1863 році виплатив 30\%
річного дивіденду, Union Bank of London та інші — 15\%. З гуртового
зиску тут, крім плати управителям, відходить процент,
який сплачується за вклади. Високий зиск пояснюється тут тим,
що вкладений в підприємства капітал становить незначну величину
порівняно з вкладами. Наприклад, в London and Westminster
\parbreak{}  %% абзац продовжується на наступній сторінці

\parcont{}  %% абзац починається на попередній сторінці
\index{iii1}{0372}  %% посилання на сторінку оригінального видання
Bank в 1863 році: вкладений капітал \num{1000000} фунтів стерлінгів,
вклади \num{14540275} фунтів стерлінгів. В Union Bank of London в
1863 році: вкладений капітал \num{600000} фунтів стерлінгів, вклади
\num{12384173} фунтів стерлінгів.

Змішання підприємницького доходу з платою за нагляд
або управління первісно виникло з тієї антагоністичної форми,
якої надлишок зиску понад процент набирає в протилежність
до проценту. Воно розвинулося далі з апологетичного наміру
зобразити зиск не як додаткову вартість, тобто неоплачену працю,
а як заробітну плату самого капіталіста за виконувану ним працю.
Цьому з боку соціалістів була протиставлена вимога звести зиск
фактично до того, чим його виставлялося в теорії, а саме до простої
плати за нагляд. І ця вимога виступала проти теоретичного
прикрашування тим неприємніше, чим більше ця плата за нагляд,
з утворенням численного класу промислових і торговельних
управителів\footnote{
„Masters are labourers as well as their journeymen. In this character their
interest is precisely the same as that of their men. But they are also either capitalists,
or the agents of capitalists, and in this respect their interest is decidedly
opposed to the interest of the workmen“. [„Майстри — такі ж робітники, як і їх
поденники. В цьому відношенні їх інтереси цілком ті самі, що й інтереси їх людей.
Але, крім того, вони є або капіталісти або агенти капіталістів, і в цьому відношенні
їх інтереси рішуче протилежні інтересам робітників“] (стор. 27). „The
wide spread of education among the journeymen mechanics of this country diminishes
daily the value of the labour and skill of almost all masters and employers
by increasing the number of persons who possess their peculiar knowledge“ ' [„Значне
поширення освіти серед промислових робітників цієї країни з кожним днем
зменшує вартість праці і вправності майже всіх майстрів і підприємців, збільшуючи
число осіб, які мають такі ж спеціальні знання“] (стор 30. \emph{Hodgskin}:
„Labour defended against the Claims of Capital etc.“ London 1825).
}, з одного боку, знаходила, як і всяка інша заробітна
плата, свій певний рівень і свою певну ринкову ціну; і чим
нижче, з другого боку, вона падала, як і всяка плата за вправну
працю, разом із загальним розвитком, що знижував витрати виробництва
спеціально навченої робочої сили.\footnote{
„The general relaxation of conventional barriers, the increased facilities of education
tend to bring down the wages of skilled labour Instead of raising those of
the unskilled“ („Загальне ослаблення умовних перепон і збільшення можливості
дістати освіту діють в напрямі зниження плати кваліфікованої праці замість
того, щоб підвищувати плату некваліфікованої“] (\emph{J. St. Mill}: „Principles of Political
Economy“. 2 вид., Лондон 1849, І, стор. 463).
} З розвитком кооперації
серед робітників, акційних підприємств серед буржуазії,
був знищений і останній привід для змішання підприємницького
доходу з платою за управління, і зиск і на практиці
виступив як те, чим він незаперечно був теоретично — як проста
додаткова вартість, як вартість, за яку не сплачено ніякого
еквіваленту, як реалізована неоплачена праця; так що функціонуючий
капіталіст дійсно експлуатує працю, і плід його експлуатації,
якщо він працює з узятим в позику капіталом, ділиться на процент
і підприємницький дохід, — надлишок зиску понад процент.

На базі капіталістичного виробництва в акційних підприємствах
розвивається нове шахрайство з платою за управління,
\parbreak{}  %% абзац продовжується на наступній сторінці

\parcont{}  %% абзац починається на попередній сторінці
\index{ii}{0373}  %% посилання на сторінку оригінального видання
стану, зглядно від відносної величини продукційних запасів у різних
підприємствах та різних поодиноких капіталістів тієї самої галузі підприємства,
отже, ріжниці в строках закупу елементів сталого капіталу —
все це на протязі року репродукції: досить лише на досвіді помітити
всі ці різноманітні моменти стихійного руху та звернути на них увагу, —
щоб з’явився імпульс до планомірного використовування їх так для механічних
допоміжних засобів кредитової системи, як і для справжнього
виловлювання наявних капіталів, що їх можна дати в позику.

До цього долучається ще ріжниця між такими підприємствами, що
їхня продукція в нормальних, загалом беручи, умовах відбувається
безперервно в тих самих розмірах, і такими, що в різні періоди року
застосовують робочу силу в неоднаковому розмірі, як, напр., сільське
господарство.
\label{original-373-1}

\subsection[Теорія репродукції Детю де-Трасі]{Теорія репродукції Детю де-Трасі\footnotemark{}}

\label{original-373-2}
За%
\footnotetext{З рукопису II.}
приклад плутаної й разом з тим бундючної безтямности політикоекономів
при розгляді суспільної репродукції є великий логік Детю
де-Трасі (пор. кн. І, розд. IV, 2, прим. 30), до якого навіть Рікардо
ставиться серйозно, називаючи його a very distinguished writer\footnote*{
Дуже видатним письменником. \emph{Ред.}
}. (Principles,
p. 333).

Цей „видатний письменник“ дає такі пояснення щодо сукупного суспільного
процесу репродукції та циркуляції.

„Мене запитають, як одержують ці промислові підприємці такі великі
зиски, і від кого вони можуть їх брати. Я відповідаю, що вони досягають
цього тому, що продають все продуковане ними дорожче, ніж коштує
їм продукція; і тому що вони це продають:

1) один одному в розмірі всієї частини свого споживання, призначеної
на задоволення їхніх потреб, яку вони оплачують частиною їхнього
зиску;

2) найманим робітникам, так тим, що їх оплачують вони самі, як і тим, що їх
оплачують капіталісти-нероби; таким способом вони одержують назад від цих
робітників всю їхню заробітну плату, за винятком хіба невеликих заощаджень;

3) капіталістам-неробам, які платять їм частиною свого доходу, ще
не витраченою на наймання робітників, що роблять безпосередньо для
них, так що уся рента, щорічно виплачувана промисловими підприємцями
капіталістам-неробам, тим або іншим способом знову припливає
назад до промисловців“. (Destutt de Tracy. Fraité de la volonté et de ses
effets. Paris. 1826“, p. 239).

Отже, капіталісти збагачуються, поперше, обдурюючи один одного при
обміні тієї частини додаткової вартости, яку вони призначають для особистого
споживання або споживають як дохід. Отже, коли ця частина
\parbreak{}  %% абзац продовжується на наступній сторінці

\parcont{}  %% абзац починається на попередній сторінці
\index{ii}{0374}  %% посилання на сторінку оригінального видання
їхньої додаткової вартости, зглядно їхнього зиску \deq{} 400\pound{ ф. стерл.}, то
ці 400\pound{ ф. стерл.} перетворюються, напр., на 500\pound{ ф. стерл.} в наслідок
того, що кожен співвласник цих 400\pound{ ф. стерл.} продає свою частину другому
дорожче на 25\%. Що всі роблять так, то наслідок такий самий,
як коли б вони навзаєм продавали один одному за дійсною вартістю.
Тільки для циркуляції товарової вартости в 400\pound{ ф. стерл.} їм потрібна
маса грошей в 500\pound{ ф. стерл.}, а це є, здається, скорше метод збіднення,
ніж збагачення, бо їм доводиться чималу частину всього свого майна
непродуктивно зберігати в некорисній формі засобів циркуляції. Все
сходить на те, що кляса капіталістів, не зважаючи на всебічне номінальне
підвищення цін їхніх товарів, може розподіляти поміж себе для свого
особистого споживання лише запас товарів вартістю в 400\pound{ ф.
стерл.}, але вони роблять один одному приємність, пускаючи в циркуляцію
400\pound{ ф. стерл.} товарової вартости за допомогою такої маси грошей,
яка потрібна для 500\pound{ ф. стерл.} товарової вартости.

Ми зовсім лишаємо осторонь, що тут припускається „частину їхнього
зиску“, і значить, взагалі запас товарів, що в ньому виражається зиск.
А проте Детю хотів саме з’ясувати нам, відки походить цей зиск. Маса
грошей, потрібна для його циркуляції, це питання цілком другорядне. Та маса
товарів, яка репрезентує зиск, здається, походить від того, що капіталісти
не лише продають її один одному, хоч уже й це дуже добре й
глибоко розумно, але що вони продають один одному дуже дорого.
Отже, ми знаємо тепер одне джерело збагачення капіталістів. Воно сходить
до таємниці „ентспектора Брезіґа“, що великі злидні походять з великої
pauvreté\footnote*{
Бідности. \emph{Ред.}
}.

2) Далі, ті самі капіталісти продають „найманим робітникам, так тим,
що їх оплачують вони сами, як і тим, що їх оплачують капіталісти-нероби;
таким чином, вони одержують назад від цих робітників всю їхню заробітну
плату, за винятком хіба невеликих заощаджень“.

Зворотний приплив до капіталістів того грошового капіталу, що
в формі його вони авансували заробітну плату робітникові, є, за паном
Детю, друге джерело збагачення цих капіталістів.

Отже, коли кляса капіталістів виплатить робітникам, напр., 100\pound{ ф. стерл.},
як заробітну плату, а потім ті самі робітники купують товари такої самої
вартости в 100\pound{ ф. стерл.} у тієї самої кляси капіталістів, і тому сума
в 100\pound{ ф. стерл.}, авансована капіталістами як покупцями робочої сили,
припливає до них назад при продажу цим робітникам товарів на 100\pound{ ф.
стерл.}, то капіталісти в наслідок цього збагачуються. З погляду
доброго розуму виходить, що капіталісти за допомогою цієї процедури
знову мають ті 100\pound{ ф. стерл.}, що були в них до цієї процедури. На
початку процедури в них було 100\pound{ ф. стерл.} грішми, на ці 100\pound{ ф. стерл.}
вони купили робочу силу. За ці 100\pound{ ф. стерл.} грішми куплена праця
продукує товари вартістю, оскільки ми знаємо до цього часу, в 100\pound{ ф.
стерл}. В наслідок того, що робітникам продано ці 100\pound{ ф. стерл.} в товарах,
\index{ii}{0375}  %% посилання на сторінку оригінального видання
капіталісти одержують знову 100\pound{ ф. стерл.} грішми. Отже, у капіталістів
знову є 100\pound{ ф. стерл.} грішми, а в робітників — на 100\pound{ ф. стерл.}
товару, що його вони сами спродукували. Важко зрозуміти, як могли б
капіталісти збагатитись на цьому. Коли б 100\pound{ ф. стерл.} грішми не припливали
до них назад, то їм довелось би, поперше, заплатити робітникам
за їхню працю 100\pound{ ф. стерл.} грішми, і, подруге, безплатно віддати
їм продукт цієї праці, засоби споживання на 100\pound{ ф. стерл.}. Отже, зворотний
приплив грошей міг би, щонайбільш, пояснити, чому капіталісти
не збіднюються в наслідок цієї операції, але ні в якому разі не міг би
пояснити, чому вони з неї збагачуються.

Звичайно, друге питання, звідки капіталісти беруть ці 100\pound{ ф. стерл.}
грішми, і чому робітники мусять обмінювати свою робочу силу на ці 100\pound{ ф.
стерл.}, замість самим продукувати товари власним коштом. Але це є щось
само собою зрозуміле для мислителів типу Детю.

Детю сам не цілком задоволений з такого розв’язання. Він бо не
сказав нам, що збагачення постає тому, що витрачають грошову суму
в 100\pound{ ф. стерл.} і потім знову одержують грошову суму в 100\pound{ ф. стерл.},
отже, не в наслідок зворотного припливу 100\pound{ ф. стерл.} грішми, який
з’ясовує лише, чому ці 100\pound{ ф. стерл.} грішми не втрачається. Він сказав
нам, що капіталісти збагачуються, „продаючи все продуковане ними
дорожче, ніж коштував їм закуп цього“.

Отже, капіталісти в своїй оборудці з робітниками мусять збагачуватись
тому, що вони продають робітникам дуже дорого. Чудово! „Вони
виплачують заробітну плату\dots{} і все це зворотно припливає до них в наслідок
витрат всіх цих людей, що платять за них“ (за продукти) „дорожче,
ніж вони коштували їм“ (капіталістам) „при такій заробітній
платі“ (стор. 240). Отже, капіталісти платять робітникам 100\pound{ ф.
стерл.} заробітної плати, а потім продають робітникам власний продукт
останніх за 120\pound{ ф. стерл.}, так що до капіталістів не лише припливають
назад ці 100\pound{ ф. стерл.}, а ще виграється 20\pound{ ф. стерл.}? Це неможливо.
Робітники можуть заплатити лише тими грішми, що їх вони одержали
в формі заробітної плати. Коли вони одержали від капіталістів 100\pound{ ф.
стерл.} заробітної плати, то вони можуть купити лише на 100\pound{ ф. стерл.}
а не на 120\pound{ ф. стерл}. Отже, цим способом питання не розв’язується. Але
є ще один спосіб. Робітники купують у капіталістів товару на 100\pound{ ф.
стерл.}, а в дійсності одержують товар вартістю лише на 80\pound{ ф. стерл}.
Тому їх, безперечно, обшахраяли на 20\pound{ ф. стерл}. А капіталіст, безперечно,
збагатився на 20\pound{ ф. стерл.}, бо він фактично оплатив робочу силу
на 20\% нижче від її вартости або обкружним шляхом зробив одрахування
в 20\% з номінальної заробітної плати.

Кляса капіталістів досягла б цього самого, якби вона з самого початку
виплатила робітникам заробітної плати лише 80\pound{ ф. стерл.}, а потім
дала б їм за ці 80\pound{ ф. стерл.} грішми товарову вартість дійсно на 80\pound{ ф.
стерл}. Ось такий, — коли взяти цілу клясу, — здається, нормальний спосіб,
бо, за висловом самого пана Детю, робітнича кляса мусить одержувати
„достатню заробітну плату“ (ст. 219), бо цієї заробітної плати
\parbreak{}  %% абзац продовжується на наступній сторінці

\parcont{}  %% абзац починається на попередній сторінці
\index{ii}{0376}  %% посилання на сторінку оригінального видання
мусить вистачити принаймні на те, щоб підтримати її існування й
працездатність, „придбати найпотрібніші засоби існування“ (стор. 180).
Коли робітники не одержують такої достатньої плати, то, за тим самим
Детю, це — „смерть для промисловости“ (стор. 208), отже, здавалось би,
зовсім не засіб збагачення для капіталістів. Але хоч яка буде висота заробітної
плати, виплачуваної клясою капіталістів робітничій клясі, ця плата
має певну вартість, напр., 80\pound{ ф. стерл}. Отже, коли кляса капіталістів
виплачує робітникам 80\pound{ ф. стерл.}, то вона повинна давати їм за ці 80\pound{ ф.
стерл.} товарову вартість в 80\pound{ ф. стерл.}, а тому зворотний приплив цих
80\pound{ ф. стерл.} не збагачує її. А коли вона виплачує їм грішми 100\pound{ ф. стерл.}
і продає їм за 100\pound{ ф. стерл.} товарову вартість в 80\pound{ ф. стерл.}, то вона
виплачує їм грішми на 25\% більше за їхню нормальну заробітну плату
і за те дає їм товарами на 25\% менше.

Інакше кажучи, фонд, що з нього взагалі кляса капіталістів бере свій
зиск, утворювався б з одрахувань з нормальної заробітної плати, в наслідок
оплати робочої сили нижче за її вартість, тобто нижче за вартість засобів
існування, доконечних для нормальної репродукції робочої сили як
найманих робітників. Отже, коли б виплачувалось нормальну заробітну
плату, а так повинно бути за Детю, то не існувало б жодного фонду
зиску ні для промисловців, ні для капіталістів-нероб.

Отже, усю таємницю, як збагачується кляса капіталістів, панові Детю
довелось би звести ось на що: в наслідок одрахувань з заробітної плати.
В такому разі не існує інших фондів додаткової вартости, що про них
він каже в пунктах 1 і 3.

Отже, в усіх країнах, де грошову заробітну плату робітників зведено
на вартість засобів споживання, потрібних на їхнє існування як кляси,
не існувало б ні фонду споживання, ні фонду нагромадження для капіталістів,
а значить, не існувало б і фонду існування кляси капіталістів, а
тому й кляси капіталістів. І за Детю, це саме так було б у всіх багатих,
розвинених країнах старої цивілізації, бо тут в „наших суспільствах
старого походження фонд, звідки покривається заробітну плату,
є\dots{} майже стала величина“ (стор. 202).

І коли урізується заробітну плату, збагачення капіталістів випливає
не з того, що вони спочатку виплачують робітникові 100\pound{ ф. стерл.}
грішми, а потім дають йому на ці 100\pound{ ф. стерл.} грішми 80\pound{ ф. стерл.}
товарами, — отже, в дійсності 80\pound{ ф. стерл.} товару пускають у циркуляцію
грошовою сумою в 100\pound{ ф. стерл.}, на 25\% більшою, — але з того, що капіталіст
привлащує собі з продукту робітника, крім додаткової вартости — тієї
частини продукту, що в ній втілюється додаткова вартість, — ще й
25\% тієї частини продукту, яка в формі заробітної плати повинна була
б дістатися робітникові. Таким безглуздим способом, як це уявляє собі Детю,
кляса капіталістів абсолютно нічого не виграла б. Вона платить робітникам
100\pound{ ф. стерл.} як заробітну плату і за ці 100\pound{ ф. стерл.} повертає робітникові
з власного його продукту 80\pound{ ф. стерл.} товарової вартости. Але
при наступній операції вона мусить знову авансувати на ту саму процедуру
100\pound{ ф. стерл}. Отже, вона вдається лише до марної гри, авансуючи
\parbreak{}  %% абзац продовжується на наступній сторінці

\parcont{}  %% абзац починається на попередній сторінці
\index{ii}{0377}  %% посилання на сторінку оригінального видання
100\pound{ ф. стерл.} грішми й даючи за них 80\pound{ ф. стерл.} товаром, замість авансувати 80\pound{ ф. стерл.} грішми й
дати за них 80\pound{ ф. стерл.} товаром. Інакше кажучи, вона без якоїбудь користи постійно авансує на 25\%
більший грошовий капітал для циркуляції свого змінного капіталу, а це є цілком ориґінальний метод
збагачення.

3) Нарешті, кляса капіталістів продає „капіталістам-неробам, які платять, їм частиною свого доходу,
ще невитраченою на наймання робітників, що роблять безпосередньо для них, так що уся рента, щорічно
виплачувана промисловими підприємцями капіталістам-неробам, тим або іншим способом знову припливає
назад до промисловців“.

Раніше ми бачили, що промислові капіталісти, „частиною свого зиску оплачують усю ту частину їхнього
споживання, яка призначена на задоволення їхніх потреб“. Отже, припустімо, що їхній зиск \deq{} 200\pound{ ф.
стерл}. Припустімо, прим., що 100\pound{ ф. стерл.} вони витрачають на своє особисте споживання. Але друга
половина \deq{} 100\pound{ ф. стерл.} належить не їм, а капіталістам-неробам, тобто одержувачам земельної ренти й
капіталістам-позикодавцям
за проценти. Отже, промислові капіталісти повинні виплачувати цим людям 100\pound{ ф. стерл.} грішми.
Скажімо, що з цих грошей капіталістам-неробам треба 80\pound{ ф. стерл.} на їхнє власне споживання і 20\pound{ ф.
стерл.} на наймання слуг і~\abbr{т. ін.} Отже, на ці 80\pound{ ф. стерл.} вони купують засоби споживання у
промислових капіталістів. В наслідок цього до промислових капіталістів, тимчасом як від них
відходить продукт в 80\pound{ ф. стерл.}, повертається назад 80\pound{ ф. стерл.} грішми, або \sfrac{4}{5} тих 100\pound{ ф. стерл.},
що їх вони заплатили капіталістам-неробам під назвою ренти, проценту й~\abbr{т. ін.} Далі, кляса слуг,
безпосередні наймані робітники капіталістів-нероб, одержали від своїх панів 20\pound{ ф. стерл}. Вони
купують на них — теж у промислових капіталістів — засоби споживання на 20\pound{ ф. стерл}. В наслідок цього
до промислових капіталістів, тимчасом як від них відходить продукт на 20\pound{ ф. стерл.}, повертається
назад 20\pound{ ф. стерл.} грішми, або остання п’ята частина тих 100\pound{ ф. стерл.} грішми, що їх вони заплатили
капіталістам-неробам як ренту, процент та ін.

По закінченні оборудки до промислових капіталістів зворотно припливають ті 100\pound{ ф. стерл.} грішми, що
їх вони віддали капіталістам-неробам, сплачуючи ренту, процент і~\abbr{т. ін.}, тимчасом як половина їхнього
додаткового продукту \deq{} 100\pound{ ф. стерл.} з їхніх рук перейшла до фонду споживання капіталістів-нероб.

Отже, для питання, що про нього тут ідеться, очевидно, було б цілком зайве в тому або іншому вигляді
притягати до справи розподіл цих 100\pound{ ф. стерл.} між капіталістами-неробами та їхніми безпосередніми
найманими робітниками. Справа проста: їхні ренти, проценти, коротше, ту пайку, що їм припадає з
додаткової вартости \deq{} 200\pound{ ф. стерл.}, виплачують їм промислові капіталісти грішми, 100\pound{ ф. стерл}. На
ці 100\pound{ ф. стерл.} вони безпосередньо або посередньо купують засоби споживання у промислових
капіталістів. Отже, вони виплачують, їм назад 100\pound{ ф. стерл.} грішми й беруть у них на 100\pound{ ф. стерл.}
засобів споживання.

\parcont{}  %% абзац починається на попередній сторінці
\index{iii1}{0378}  %% посилання на сторінку оригінального видання
беручи до уваги умов репродукції і праці, як самодіяльний
автомат, як просте число, що само собою збільшується (цілком
так само як Мальтус розглядав людей у своїй геометричній
прогресії), то він уявив, що відкрив закон зростання капіталу
в формулі $s \deq{} c (1 \dplus{} z)^n$, де $s$ \deq{} сумі капіталу \dplus{} проценти на
проценти, $c$ \deq{} авансованому капіталові, $z$ \deq{} розмірові процента
(вираженому у відповідних частинах 100), а $n$ — ряд років, протягом
яких відбувається процес.

Пітт цілком серйозно приймає містифікацію д-ра Прайса.
В 1786 році палата громад ухвалила зібрати 1 мільйон фунтів стерлінгів
на громадські потреби. За Прайсом, в якого вірував Пітт,
не було нічого кращого, як оподаткувати народ, щоб „нагромадити“
одержану таким способом суму і таким чином за допомогою
таїнства складних процентів чарами позбутись державного
боргу. Після цій резолюції палати громад незабаром з ініціативи
Пітта був виданий закон, який приписував нагромаджувати по
\num{250000}\pound{ фунтів стерлінгів} доти, „поки фонд разом з відмерлими
пожиттьовими рентами зросте до \num{4000000}\pound{ фунтів стерлінгів}
на рік“ (Act 26, Georg III, Кар. 31\footnote*{
Тобто 31-й закон від 26-го року королювання Георга III.
\emph{Примітка ред. нім. вид. ІМЕЛ.}
}). Пітт у своїй промові 1792 року,
в якій він пропонував збільшити суму, призначену для фонду
оплати боргів, серед причин торговельної переваги Англії наводить:
машини, кредит і~\abbr{т. д.}, але як „найпоширенішу і найтривкішу
причину — нагромадження. Принцип цей цілком розвинутий
і досить пояснений у творах Сміта, цього генія\dots{} Це нагромадження
капіталів було б викликане, коли б відкладали принаймні
частину річного зиску для того, щоб збільшити основну
суму, яка, при такому самому застосуванні її в наступному році,
давала б постійний зиск“. Таким чином за допомогою д-ра Прайса
Пітт перетворює теорію нагромадження Сміта в теорію збагачення
народу шляхом нагромадження боргів і приходить до
приємного безконечно прогресуючого збільшення позик, позик
для оплати позик.

Вже в Josias Child’a, батька сучасних банкірів, ми знаходимо,
що „100\pound{ фунтів стерлінгів} з 10\% виробили б за 70 років, при
процентах на проценти, \num{102400}\pound{ фунтів стерлінгів}“ („Traité sur le
commerce etc. par \emph{J.~Child}, traduit etc.“ Amsterdam et Berlin
1754, стор. 115. Написано в 1669 році).

Як погляд д-ра Прайса непродумано прохоплюється у сучасних
економістів, показує таке місце з „Economist’a“: „Capital,
with compound interes on every portion of capital saved, is so
all-engrossing that all the wealth in the world from which income
is derived, has long ago become the interest of capital\dots{} all rent is
now the payement of interest on capital previously invested in the
land“ [„Капітал із складними процентами на кожну частину заощадженого
капіталу є до такої міри всезахоплюючим, що все
\parbreak{}  %% абзац продовжується на наступній сторінці

\parcont{}  %% абзац починається на попередній сторінці
\index{ii}{0379}  %% посилання на сторінку оригінального видання
Вони продають їм усі товари дорожче, напр., на 20\%. Тут можливі два випадки. Нероби, крім
тих 100\pound{ ф. стерл.}, що їх вони одержують щорічно від промисловців, мають ще інші грошові засоби, або
не мають їх. В першому разі промисловці продають їм свої товари вартістю в 100\pound{ ф. стерл.} за ціну,
прим., в 120\pound{ ф. стерл}. Отже, через продаж їхніх товарів до них повертаються назад не лише ті 100\pound{ ф.
стерл.}, що їх вони заплатили неробам, але, крім того, ще 20\pound{ ф. стерл.}, які справді являють для них
нову вартість. Як же тепер стоїть справа з розрахунком? Вони дурно віддали товарів на 100\pound{ ф. стерл.},
бо ті 100\pound{ ф. стерл.} грішми, що ними їм виплатили частину суми, були їхні власні гроші. Отже, їхній
власний товар їм оплачено їхніми власними грішми. Отже, 100\pound{ ф. стерл.} втрати. Але вони одержали,
крім того, 20\pound{ ф. стерл.} як надлишок ціни над вартістю. Отже, 20\pound{ ф. стерл.} бариша; при 100\pound{ ф. стерл.}
втрат, це становить 80\pound{ ф. стерл.} втрат; завжди лишається мінус, ніколи не буде плюса. Шахрайство,
спричинене неробам, зменшило втрати промисловців, але від цього втрата багатства не перетворилась
для них на засіб до збагачення. Але такий метод не можна вживати протягом довгого часу, бо нероби не
можуть щороку платити 120\pound{ ф. стерл.} грішми, одержуючи щороку лише 100\pound{ ф. стерл.} грішми.

Тоді маємо другий спосіб: промисловці продають товарів вартістю в 80\pound{ ф. стерл.} за ті 100\pound{ ф. стерл.}
грішми, що їх вони заплатили неробам. В цьому разі, як і раніш, вони дурно віддають 80\pound{ ф. стерл.}
грішми у формі ренти, проценту й~\abbr{т. ін.} Таким шахрайством вони зменшили свою данину неробам, але
вона лишається, як і раніше, і, згідно з тією самою теорією, що ціни залежать від доброї волі
продавців, нероби можуть у майбутньому вимагати 120\pound{ ф. стерл.} ренти, процентів і~\abbr{т. ін.} за свою
землю та капітал, а не 100\pound{ ф. стерл.}, як до цього часу.

Цей блискучий дослід цілком гідний глибокого мислителя, який на одній сторінці списує в А.~Сміcа що
„праця є джерело всякого багатства“ (стор. 242), що промислові капіталісти „застосовують свій
капітал, щоб оплачувати працю, яка репродукує капітал з зиском“ (стор. 246), а на другій сторінці
робить висновок, що ці промислові капіталісти „годують
усіх інших людей, що тільки вони збільшують суспільне майно і утворюють усі засоби для нашого
споживання“ (стор. 242), що не робітники годують капіталістів, а капіталісти робітників, і це з тієї
чудової причини, що гроші, якими оплачується робітників, не лишаються в їхніх руках, а постійно
повертаються назад до капіталістів як плата за спродуковані робітниками товари. „Вони одержують лише
однією рукою, а другою віддають назад. Отже, їх споживання треба розглядати як породжене тими, хто
платить їм утримання“\dots{} (стор. 235).

Після такого вичерпного опису суспільної репродукції та споживання, як вони відбуваються за
посередництвом грошової циркуляції, Детю каже далі: „Ось чим поповнюється це perpetuum mobile\footnote*{Прилад, що безупинно рухається без впливу зовнішньої движної сили. \emph{Ред.}}
багатства, рух.
\parbreak{}  %% абзац продовжується на наступній сторінці

\parcont{}  %% абзац починається на попередній сторінці
\index{ii}{0380}  %% посилання на сторінку оригінального видання
що хоч його й неправильно розуміють“ (mal connu — певно!), „але слушно звуть циркуляцією; бо він
справді є кругобіг і завжди повертається назад до свого вихідного пункту. Цей пункт є той, де
відбувається продукція“ (р. 239, 240).

Детю, цей very distinguished writer (видатний письменник), membre de l’Institut de France et de la
Société Philosophique de Philadelphie (член Інституту Франції та Філософського т-ва Філадельфії) і
справді до певної міри світило серед вульґарних економістів, наприкінці прохає читачів дивуватися з
тієї дивовижної ясности, що з нею він виклав перебіг суспільного процесу, з того потоку світла, що
його він пролив на предмет, і робить навіть таку ласку, що розкриває читачеві, відки походить все це
світло. Це треба навести в ориґіналі: „On remarquera, j’espère, combien cette manière de considérer
la consommation de nos richesses est concordante avec tout ce que nous avons dit à propos de leur
production et de leur distribution, et en même temps quelle clarté elle répand sur toute la marche
de la société. D'où viennent cet accord et cette lucidité? De ce que nous avons rencontré la vérité.
Cela rappelle l’effet de ces miroirs où les objets se peignent nettement et dans leurs justes
proportions, quand on est placé dans leur vrai point de vue, et où tout paraît confus et désuni,
quand on en est trop près ou trop loin“ (p. 242, 243)\footnote*{
„Сподіваюсь, що звернуть увагу на те, наскільки такий спосіб розглядати споживання наших багатств
узгоджується з усім сказаним нами про їхню продукцію та їхній розподіл, і яким світом він разом з
тим осяює ввесь перебіг суспільного розвитку. Відки походить ця гармонія і ця ясність? З того, що ми
відкрили істину. Це нагадує нам ті дзеркала, що точно й зберігаючи дійсні пропорції між частинами,
відображають усе, що ставиться перед ними в справжньому їхньому фокусі, і де все розпливається, коли
дуже наближатись або віддалятись“.
}.

Ось такий буржуазний кретинізм в його блискучому самозадоволенні\footnote*{
Voilà le crétinisme bourgeois dans toute sa béatitude!
}.
\label{original-380-1}

\section[Акумуляція та поширена репродукція]{Акумуляція та поширена репродукція\footnotemark{}}

\label{original-380-2}
В книзі І%
\footnotetext{Відси до кінця рукопис VIII.}
показано, як перебігає акумуляція для поодинокого капіталіста. В наслідок перетворення на
гроші товарового капіталу перетворюється на гроші й додатковий продукт, що в ньому втілюється
додаткова вартість. Цю додаткову вартість, перетворену таким чином на гроші, капіталіст знову
перетворює на додаткові натуральні елементи свого продуктивного капіталу. При наступному кругобігу
продукції збільшений капітал дає більшу кількість продукту. Але те, що відбувається з індивідуальним
капіталом, мусить також виявитись і в сукупній річній продукції, цілком подібно до того, що ми
бачили, розглядаючи просту репродукцію, де —
\parbreak{}  %% абзац продовжується на наступній сторінці

\parcont{}  %% абзац починається на попередній сторінці
\index{iii1}{0381}  %% посилання на сторінку оригінального видання
обмежує число робочих днів, що їх можна одночасно експлуатувати.
Якщо ж, навпаки, додаткова вартість береться в ірраціональній
формі процента, то межа нагромадження є тільки кількісна
і перевищує всяку фантазію.

Але в капіталі, що дає процент, уявлення про капітал-фетиш
є завершене, уявлення, яке нагромадженому продуктові праці,
і при тому ще й фіксованому у формі грошей, приписує силу за
допомогою природженої таємної властивості, подібно до справжнього
автомата, виробляти додаткову вартість у геометричній
прогресії, так що цей нагромаджений продукт праці, як гадає
„Economist“, давно вже дисконтував усе багатство світу за всі
часи, як таке, що по праву належить і дістається йому. Продукт
минулої праці, сама минула праця тут сама по собі вагітна частиною
сучасної чи майбутньої живої додаткової праці. Ми знаємо,
навпаки, що в дійсності збереження, а остільки й репродукція
вартості продуктів минулої праці є \emph{тільки} результат їх
контакту з живою працею; і, подруге, що панування продуктів
минулої праці над живою додатковою працею триває якраз тільки
доти, поки триває капіталістичне відношення, певне соціальне
відношення, при якому минула праця самостійно протистоїть
живій праці та підкорює її собі.

\section{Кредит і фіктивний капітал}

Детальний аналіз кредиту і тих знарядь, які він собі створює
(кредитні гроші і~\abbr{т. д.}), не входить у наш план. Тут слід відзначити
лиш деякі окремі пункти, необхідні для характеристики
капіталістичного спосббу виробництва взагалі. При цьому ми
матимем справу тільки з комерційним і банкірським кредитом.
Зв’язок між розвитком цього кредиту і розвитком громадського
кредиту лишається поза нашим розглядом.

Я вже раніше (книга І, розд. III, З, b) показав, яким чином
з простої товарної циркуляції розвивається функція грошей як
засобу платежу і разом з тим відношення кредитора і боржника
між виробниками товарів і торговцями товарами. З розвитком
торгівлі і капіталістичного способу виробництва, який
виробляє, розраховуючи тільки на циркуляцію, ця природно виросла
основа кредитної системи розширюється, стає загальною,
виробляється. Загалом і в цілому гроші функціонують тут тільки
як засіб платежу, тобто товар продається не за гроші, а за
писану обіцянку платежу в певний строк. Всі ці платіжні обіцянки
ми можемо для короткості підвести під загальну категорію
векселів. Такі векселі до дня скінчення їх строку і настання
дня платежу в свою чергу самі циркулюють як засіб платежу;
і саме вони становлять власне торговельні гроші. Оскільки вони,
кінець-кінцем, в наслідок вирівнення вимог і боргів взаємно
\parbreak{}  %% абзац продовжується на наступній сторінці

\parcont{}  %% абзац починається на попередній сторінці
\index{i}{0382}  %% посилання на сторінку оригінального видання
цілих 6\shil{ шилінґів} 11\pens{ пенсів.} Тижнева плата ткачів наприкінці
1862~\abbr{р.} починалась з 2\shil{ шилінґів} 6\pens{ пенсів}»\footnote{
«Reports of Insp. of Fact, for 31 st October 1863», p. 41, 42.
}. Плату за помешкання
часто відраховували від заробітної плати навіть і тоді, коли
руки працювали лише короткий час\footnote{
Там же, стор. 51.
}. Не диво, що в деяких
частинах Ланкашіру почалось щось ніби голодна чума! Але найхарактеристичніше
було те, що революціонізування процесу продукції
відбувалося коштом робітника. Це були справжні experimenta
in corpora vili\footnote*{
експерименти на нічого не вартих тілах. \Red{Ред.}
}, як от експерименти анатомів на жабах.
«Хоч я, — каже фабричний інспектор Редґрев, — подав дійсні
доходи робітників по багатьох фабриках, але з цього не можна
зробити такого висновку, ніби вони тиждень-у-тиждень дістають
ту саму суму. Доходи робітників зазнають якнайбільших
коливань через постійне експериментування («experimentalizing»)
з боку фабрикантів\dots{} їхні заробітки зростають або падають залежно
від якости бавовняної сумішки; вони то наближаються
до їхніх попередніх заробітків, відхиляючися від них лише на
15\%, то в найближчий або другий тиждень падають на 50 —
60\%»\footnote{
Там же, стор. 50, 51.
}. Ці експерименти роблено не тільки коштом засобів
існування робітників. Робітники мусили поплатитись усіма своїми
п’ятьма почуттями. «Ті робітники, що працювали коло відкривання
паків бавовни, оповідали мені, що нестерпний сморід
доводить їх до непритомности. Тим робітникам, що їх уживають
до праці по відділах мішання та чухрання бавовни, порох та бруд,
що вилітають із бавовни, подразнюють дишні шляхи, викликають
кашель та стиснене дихання\dots{} Через те, що волокна короткі,
при шліхтуванні додають до пряжі багато всякого матеріялу, а
саме всяких суроґатів замість борошна, що його вживали раніш.
Звідси млості та диспепсія в ткачів. Через порох панує бронхіт,
а так само запал горла, далі шкіряні недуги через подразнення
шкури брудом, що є в сураті». З другого боку, суроґати борошна
були для панів фабрикантів за джерело наживи, бо збільшували
вагу бавовни. Ці суроґати давали те, що «15 фунтів перепряденого
сировинного матеріялу важили 26 фунтів»\footnote{
Там же, стор. 62, 63.
}. У звіті фабричних
інспекторів з 30 квітня 1864~\abbr{р.} ми читаємо: «Промисловість
користується тепер цим допоміжним джерелом у справді таки
непристойній мірі. Від дуже авторитетної особи я знаю, що восьмифунтову
тканину виготовляють із 5\sfrac{1}{4} фунтів бавовни та 2\sfrac{3}{4} фунтів
шліхти. В іншій 5\sfrac{1}{4}-фунтовій тканині було 2 фунти шліхти.
Це були звичайні шертинґи для вивозу. До інших сортів іноді
додають 50\% шліхти, так що фабриканти можуть вихвалятись
та дійсно вихваляються, що вони багатіють, «продаючи тканини
дешевше, ніж номінально коштує вміщена в них пряжа»\footnote{
«Reports etc. for 30 th April 1864», p. 27.
}.
\parbreak{}  %% абзац продовжується на наступній сторінці

\parcont{}  %% абзац починається на попередній сторінці
\index{ii}{0383}  %% посилання на сторінку оригінального видання
не дали б нам, крім можливости пояснити, яким чином може одночасно
утворюватись повсюдно скарб, і щоб при цьому сама репродукція, за
винятком репродукції у золотопромисловців, не посунулась ані на
крок далі.

Раніше, ніж розв’язати ці позірні труднощі, ми повинні відрізняти:
акумуляцію в підрозділі І (продукція засобів продукції) і акумуляцію
в підрозділі ІІ (продукція засобів споживання). Ми почнемо з підрозділу
І.

\subsection{Акумуляція в підрозділі І}

\subsubsection{Утворення скарбу}

Очевидно, що так капіталовкладення в численних галузях промисловости,
з яких складається кляса І, як і різні індивідуальні капіталовкладення
в кожній з цих галузей промисловости, залежно від їхнього віку,
тобто від протягу їхнього минулого вже функціонування, — ми цілком
лишаємо осторонь їхню величину, технічні умови, ринкові відносини
і~\abbr{т. ін.}, — перебувають на різних ступенях процесу послідовного перетворення
додаткової вартости на потенціяльний грошовий капітал, все одно
для якої з двох форм поширення продукції має служити цей грошовий
капітал: чи для збільшення діющого капіталу, чи для закладення нових підприємств.
Тому одна частина капіталістів постійно перетворює свій потенціяльний
грошовий капітал, вирослий до відповідної величини, на продуктивний
капітал, тобто за гроші, нагромаджені через перетворення на
золото додаткової вартости, купує засоби продукції, додаткові елементи
сталого капіталу, тимчасом як друга частина капіталістів ще нагромаджує
свій потенціяльний грошовий капітал. Отже, капіталісти цих двох категорій
протистоять один одному: одні — як покупці, інші — як продавці, і
кожний з цих двох категорій виступає виключно в одній з цих ролей.

\looseness=2
Наприклад, $А$ продає $В$ (що може репрезентувати й кількох покупців)
600 ($\deq{} 400 с \dplus{} 100 v \dplus{} 100 m$). Він продав товарів на 600, за
600 грішми, що з них 100 репрезентують додаткову вартість; ці
100 він вилучає з циркуляції, нагромаджує їх як гроші; але ці 100 грішми
є лише грошова форма додаткового продукту, що був носієм вартости
величиною в 100. Утворення скарбу взагалі не є продукція, отже, певна
річ, і не приріст продукції. Діяльність капіталіста сходить при цьому
лише на те, що він вилучає з циркуляції, затримує в себе й зберігає
гроші, здобуті через продаж додаткового продукту в 100. Ця операція
відбувається не тільки на боці $А$, вона відбувається в численних пунктах
на периферії циркуляції в інших капіталістів $А'$, $А''$, $А'''$, і всі вони однаково
старанно працюють коло такого утворення скарбу. Ці численні
пункти, де гроші вилучається з циркуляції та нагромаджуються в численні
індивідуальні скарби, зглядно потенціяльно грошові капітали, являють
так само численні перешкоди для циркуляції, бо вони унерухомлюють
золото й на більш або менш довгий час позбавляють їх циркуляційної
здібности. Але треба взяти на увагу, що скарби утворюються
\parbreak{}  %% абзац продовжується на наступній сторінці

\parcont{}  %% абзац починається на попередній сторінці
\index{ii}{0384}  %% посилання на сторінку оригінального видання
вже за простої товарової циркуляції, за довгий час до того, як ця циркуляція
уґрунтується на капіталістичній товаровій продукції; кількість
наявних у суспільстві грошей завжди більша за ту частину їх, яка є
в активній циркуляції, хоч ця частина, залежно від обставин більшає
або меншає. Тут ми знову маємо такі самі скарби й таке саме утворення
скарбів, але тепер уже як момент, іманентний капіталістичному процесові
продукції.

Можна зрозуміти таку приємність, коли при системі кредиту всі ці
потенціяльні капітали, концентруючись у банках і~\abbr{т. ін.}, стають капіталом,
що ним можна порядкувати, „loanable capital“\footnote*{
Позиковий капітал. \Red{Ред.}
}, грошовим капіталом,
і саме вже не пасивним, не музикою майбутнього, а активним,
Wucher-капіталом (тут слово Wucher\footnote*{
В німецькій мові слово Wucher має значення, крім звичайного — лихварство,
ще й значення — зростання. \Red{Ред.}
} — в розумінні зростання).

Але $А$ переводить таке утворення скарбу лише остільки, оскільки він —
щодо свого додаткового продукту — виступає тільки як продавець, не виступаючи
потім, як покупець. Отже, послідовна продукція додаткового
продукту, — носія його додаткової вартости, що її треба перетворити
на золото, — є для нього передумова утворення скарбу. В даному разі,
коли ми розглядаємо циркуляцію лише в межах категорії І, натуральна
форма додаткового продукту, як і всього продукту, що з нього додатковий
продукт являє частину, є натуральна форма одного з елементів
сталого капіталу І, тобто належить до категорії засобів продукції для
засобів продукції. Що з нього стає, тобто для якої функції він служить
в руках покупців $В$, $В'$, $В''$ і~\abbr{т. ін.} це ми зараз побачимо.

Але тут ми насамперед повинні пам’ятати ось що: хоч $А$ на свою
додаткову вартість вилучає гроші з циркуляції й нагромаджує їх як
скарб, він, з другого боку, подає в циркуляцію товари, не вилучаючи
з неї за них інших товарів, в наслідок чого $В$, $В'$,  $В''$ і~\abbr{т. ін.} і собі
можуть подавати в циркуляцію гроші і натомість вилучати з неї лише
товари. В даному разі цей товар і своєю натуральною формою і своїм
призначенням входить як основний або поточний елемент в сталий капітал
$В$, $В'$, і~\abbr{т. ін.} Про останній ми скажемо докладніше, коли матимемо
справу з покупцями додаткового продукту, $В$, $В'$ і~\abbr{т. ін.}

\pfbreak

Між іншим, зазначмо тут ось що: як і раніше, при досліді простої
репродукції, ми тут знову бачимо, що перетворення різних складових
частин річного продукту, тобто їхня циркуляція (а вона мусить разом
з тим охоплювати й репродукцію капіталу, а саме його відновлення
в різних його визначенностях: сталого, змінного, основного, обігового,
грошового капіталу й товарового капіталу) зовсім не має за передумову
простої купівлі товару, доповненої наступним продажем, або продажу,
доповненого наступною купівлею, так щоб в дійсності тільки обмінювалось
товари один на один, як це припускає політична економія, а саме фритредерська
\parbreak{}  %% абзац продовжується на наступній сторінці

\parcont{}  %% абзац починається на попередній сторінці
\index{ii}{0385}  %% посилання на сторінку оригінального видання
школа від часів фізіократів і Адама Сміса. Ми знаємо, що основний
капітал, після того як витрату на нього вже раз зроблено, не відновлюється
протягом усього часу свого функціонування, а функціонує й
далі в старій формі, тммчасом як вартість його поступінно осаджується
в формі грошей. Ми бачили, що періодичне відновлення основного капіталу
II~$с$ [а вся капітальна вартість II~$с$ обмінюється на елементи вартістю
в І ($v \dplus{} m$)] має за передумову, з одного боку, просту купівлю
основної частини II~$с$, яка зворотно перетворюється з грошової форми
на натуральну, при чому цій купівлі відповідає простий продаж І~$m$;
з другого боку, воно має за передумову простий продаж з боку
II~$с$, продаж тієї основної (зношеної) частини його вартости, яка осаджується
в формі грошей, при чому цьому продажеві відповідає проста
купівля І~$m$. Для того, щоб обмін відбувався тут нормально, треба припустити,
що проста купівля з боку II~$с$ величиною вартости дорівнює
простому продажеві з боку II~$с$, і так само, що простий продаж І~$m$
1-ій частині II~$с$ дорівнює простій купівлі II~$с$, частини 2 (стор. 360).
Інакше просту репродукцію порушиться; проста купівля тут мусить
покриватись простим продажем там. Так само тут треба припустити,
що простий продаж частини І~$m$, яка утворює скарб для $А$, $А'$, $А''$,
урівноважується простою купівлею частини І~$m$ з боку $В$, $В'$, $В''$
і~\abbr{т. д.}, які перетворюють свій скарб на елементи додаткового продуктивного
капіталу.

Оскільки рівновага відновлюється через те, що покупець потім виступає
як продавець на таку саму суму вартости, і навпаки, остільки відбувається
зворотний приплив грошей до тієї сторони, яка авансувала їх
підчас купівлі, яка продала раніше, ніж знову купила. Але дійсна рівновага,
щодо самого товарового обміну, обміну різних частин річного продукту,
зумовлюється рівністю величини вартости обмінюваних один проти
одного товарів.

Але оскільки відбуваються просто однобічні перетворення, прості
купівлі, з одного боку, і прості продажі, з другого, — а ми бачили, що
нормальне перетворення річного продукту на капіталістичній основі зумовлює
такі однобічні метаморфози, — остільки рівновага буде лише при
тому припущенні, що сума вартости однобічних купівель і сума вартости
однобічних продажів покривають одна одну. Та обставина, що товарова
продукція є загальна форма капіталістичної продукції, включає вже й ту
ролю, що її відіграють у ній гроші не лише як засіб циркуляції, а й
як грошовий капітал, і утворює певні, властиві цьому способові продукції
умови нормального обміну, отже, нормального перебігу репродукції,
усе одно, чи в простому, чи в поширеному маштабі, — умови,
що перетворюються на так само численні умови ненормального перебігу
репродукції, на так само численні можливості криз, бо рівновага за стихійного
ладу (naturwüchsigen Gestaltung) цієї продукції — сама є випадок.

Так само ми бачили, що при обміні І~$v$ на відповідну суму вартости
II~$с$, саме для II~$с$, кінець-кінцем, відбувається заміщення товару II
рівною сумою вартости товару І, отже, що з боку збірного капіталіста
\index{ii}{0386}  %% посилання на сторінку оригінального видання
II тут продаж власного товару доповнюється купівлею товару І на
таку саму суму вартости. Таке заміщення відбувається; але не відбувається
обміну між самими капіталістами І і II при цьому перетворенні
їхніх товарів. II~$с$ продає свої товари робітничій клясі І; ця остання
протистоїть йому однобічно як покупець товарів, а II~$с$ протистоїть робітничій
клясі І однобічно як продавець товарів; з грішми, вторгованими
таким чином, II~$с$ протистоїть збірному капіталістові І однобічно як покупець
товарів, а збірний капіталіст І однобічно протистоїть йому як продавець
товарів на суму І~$v$. Тільки через цей продаж товарів підрозділ І, кінець-кінцем,
репродукує свій змінний капітал знову в формі грошового капіталу.
Коли капітал І протистоїть капіталові II однобічно як продавець
товару на суму І~$v$, то своїй робітничій клясі він протистоїть як покупець
товарів, що купує її робочу силу; і коли робітнича кляса І протистоїть
капіталістові II однобічно як покупець товару (а саме як покупець
засобів існування), то капіталістові І вона протистоїть однобічно як продавець
товару, а саме як продавець своєї робочої сили.

Постійне подання робочої сили з боку робітничої кляси в І, зворотне
перетворення частини товарового капіталу І на грошову форму змінного
капіталу, заміщення частини товарового капіталу II натуральними елементами
сталого капіталу II~$с$ — всі ці доконечні передумови навзаєм зумовлюють
одна одну, але їх упосереднює дуже складний процес, який
має в собі три процеси циркуляції, що перебігають незалежно один від
одного, але в той самий час переплітаються один з одним. Складність самого
цього процесу дає так само численні нагоди до ненормального перебігу.

\subsubsection{Додатковий сталий капітал}

Додатковий продукт, носій додаткової вартости, нічого не коштує
капіталістам І, його привлащувачам. Їм не доводиться в жодній формі
авансувати гроші або товари, щоб його одержати. Аванс (avance)
уже у фізіократів є загальна форма вартости, реалізованої в елементах
продуктивного капіталу. Вони, отже, нічого не авансують, крім свого сталого
й змінного капіталу. Своєю працею робітник не лише зберігає їм
їхній сталий капітал; він не тільки заміщує їм змінну капітальну вартість,
утворюючи відповідну нову частину вартости в формі товару; своєю додатковою
працею він, крім того, дає їм додаткову вартість, що існує
в формі додаткового продукту. Послідовно продаючи цей додатковий
продукт, воии утворюють скарб, додатковий потенціяльний грошовий капітал.
В розглядуваному тут випадку цей додатковий продукт складається
з самого початку із засобів продукції засобів продукції. Тільки в руках
$В$, $В'$, $В''$ і~\abbr{т. д.} (І) цей додатковий продукт функціонує як додатковий
сталий капітал; але віртуально він є ним раніше, ніж його продасться,
уже в руках утворювачів скарбу $А$, $А'$, $А''$ (І). Коли ми розглядаємо
тільки розмір вартости репродукції на боці І, ми перебуваємо
ще в межах простої репродукції, бо жодного додаткового капіталу
не пущено в рух, щоб утворити цей віртуальний додатковий сталий
\parbreak{}  %% абзац продовжується на наступній сторінці

\parcont{}  %% абзац починається на попередній сторінці
\index{ii}{0387}  %% посилання на сторінку оригінального видання
капітал (додатковий продукт), не пущено в рух і більше додаткової праці,
ніж та, яку витрачалось на основі простої репродукції. Ріжниця тут
тільки в формі застосовуваної додаткової праці, в конкретній природі її
особливого корисного характеру. Її витрачено на засоби продукції для
І~$с$ замість II~$с$, на засоби продукції засобів продукції, а не на засоби
продукції засобів споживання. При простій репродукції припускалось,
що всю додаткову вартість І витрачається як дохід, отже, на товари II;
отже, вона складалась лише з таких засобів продукції, які мали знову
замістити сталий капітал II~$с$, в його натуральній формі. Отже, для того,
щоб відбувся перехід від простої репродукції до поширеної, продукція підрозділу
І мусить мати змогу виробити менше елементів сталого капіталу
для II, але остільки ж більше для І.~Цей перехід, що не завжди відбувається
без труднощів, полегшує та обставина, що деякі продукти І
можуть служити як засоби продукції в обох підрозділах.

З цього випливає, що, — коли дивитись на справу лише з погляду величини
вартости, — в межах простої репродукції продукується матеріяльний
субстрат поширеної репродукції. Це — просто додаткова праця робітничої
кляси І, витрачена безпосередньо на продукцію засобів продукції, на утворення
віртуального додаткового капіталу І.

Отже, утворення віртуального додаткового грошового капіталу з боку
$А$, $А'$, $А''$ (І) — через послідовний продаж їхнього додаткового продукту,
який утворюється без якоїбудь капіталістичної витрати грошей, — є тут
лише грошова форма додатково спродукованих засобів продукції І.

Отже, продукція віртуального додаткового капіталу в нашому випадку
(бо, як побачимо далі, він може утворитись цілком інакше) виражає не
що інше, як явище самого процесу продукції, продукції, в певній формі,
елементів продуктивного капіталу.

Отже, продукція додаткового віртуального грошового капіталу в широкому
маштабі — в багатьох пунктах на периферії циркуляції — є не що
інше, як результат і вираз багатобічної продукції віртуального додаткового
продуктивного капіталу, що саме постання його не має собі за передумову
жодних додаягкових грошових витрат з боку промислових капіталістів.

Послідовне перетворення з боку $А$, $А'$, $А''$ і~\abbr{т. д.} (І) цього віртуально
додаткового продуктивного капіталу на віртуальний грошовий
капітал (на скарб), перетворення, що зумовлюється послідовним продажем
їхнього додаткового продукту, — отже, повторюваним однобічним продажем
товару без доповнення купівлею, — відбувається через повторюване
вилучення з циркуляції грошей і відповідне йому утворення скарбу. Таке
утворення скарбу — за винятком того випадку, коли покупцем є золотопромисловець,
— зовсім не має собі за передумову додаткового багатства
в благородних металях, а лише зміну функцій обігових до цього часу
грошей. До цього часу вони функціонували як засоби циркуляції; тепер
вони функіонують як скарб, як утворюваний, віртуально новий грошовий
капітал. Отже, утворення додаткового грошового капіталу й маса
наявного в країні благородного металю не мають жодного причинного
зв'язку між собою.


\index{ii}{0388}  %% посилання на сторінку оригінального видання
З цього далі випливає: чим більший продуктивний капітал, уже
діющий в даній країні (зараховуючи долучувану до нього робочу силу,
витворця додаткового продукту), чим більше розвинена продуктивна сила
праці, а разом з тим і технічні засоби для швидкого поширення продукції
засобів продукції, чим більша, отже, маса додаткового продукту
так своєю вартістю, як і кількістю споживних вартостей, що в них
втілюється додаткова вартість, тим більший:

1) віртуально додатковий продуктивний капітал в формі додаткового
продукту в руках $А$, $А'$, $А''$ і~\abbr{т. д.} і

2) тим більша маса цього додаткового продукту, перетвореного на
гроші, отже, віртуально додаткового грошового капіталу в руках $А$, $А'$, $А''$.
Отже, коли, напр., Фулартон нічого не хоче знати про перепродукцію
в звичайному розумінні цього слова, але визнає перепродукцію капіталу,
а саме грошового капіталу, то це ще раз доводить, що навіть найкращі
буржуазні економісти абсолютно нічого не розуміють у механізмі своєї
системи.

Коли додатковий продукт, безпосередньо продукований і привлашуваний
капіталістами $А$, $А'$, $А''$ (І), є реальна база акумуляції капіталу,
тобто поширеної репродукції, хоч він в цій властивості активно функціонує
лише в руках $В$, $В'$, $В''$ і~\abbr{т. д.} (І), — то, навпаки, в своїй грошовій
лялечці, — як скарб і як лише поступінно утворюваний віртуальний
грошовий капітал, — він абсолютно непродуктивний, рухається
в цій формі паралельно з процесом продукції, але перебуває поза ним.
Він є мертвий тягар (dead wight) капіталістичної продукції. Жадоба використати
для одержання зиску й доходу цю додаткову вартість, нагромаджувану
як скарб у формі віртуального грошового капіталу, знаходить
мету своїх прагнень в кредитовій системі і в „папірцях“. Тому грошовий
капітал в іншій формі набирає величезного впливу на перебіг і потужний
розвиток капіталістичної системи продукції.

Додатковий продукт, перетворений на віртуальний грошовий капітал,
буде своєю масою тим більший, чим більша була вся сума вже діющого
капіталу, що в наслідок його функціонування постав цей додатковий
продукт. Але при абсолютному збільшенні розміру щорічно репродуковуваного
віртуального грошового капіталу полегшується і його сегментація,
так що його швидше можна вкласти в особливе підприємство, хоч буде
воно у руках того капіталіста, хоч в інших руках (напр., членів його родини,
при поділі спадщини і~\abbr{т. ін.}). Сегментація грошового капіталу значить
тут, що він цілком відокремлюється від первісного капіталу, щоб як
новий грошовий капітал приміститись у новому самостійному підприємстві.

Коли продавці додаткового продукту $А$, $А'$, $А''$ і~\abbr{т. д.} (І) одержали
його як такий безпосередній наслідок процесу продукції, що, крім авансування
на сталий І змінний капітал, потрібного й за простої репродукції,
не має собі за передумову дальших актів циркуляції, коли вони
далі цим самим утворюють реальну базу для репродукції в поширеному
маштабі, дійсно фабрикують віртуально додатковий капітал, то
\parbreak{}  %% абзац продовжується на наступній сторінці

\parcont{}  %% абзац починається на попередній сторінці
\index{ii}{0389}  %% посилання на сторінку оригінального видання
з $В$, $В'$, $В''$ і~\abbr{т. д.} (І) справа стоїть інакше. 1) Тільки в їхніх руках
додатковий продукт $А$, $А'$, $А''$ і~\abbr{т. д.} буде функціонувати активно як
додатковий сталий капітал (другий елемент продуктивного капіталу, додаткову
робочу силу, отже, додатковий змінний капітал, ми покищо
лишаємо осторонь); 2) для того, щоб додатковий продукт потрапив до
їхніх рук, потрібен акт циркуляції, вони повинні купити цей додатковий
продукт.

До пункту (І) тут треба зауважити, що чимала частина додаткового
продукту (віртуально додаткового сталого капіталу), продукованого
$А$, $А'$, $А''$ (І), хоч її спродуковано поточного року, може активно функціонувати
як промисловий капітал в руках $В$, $В'$, $В''$ (І) тільки наступного
року або навіть пізніше; до пункту 2) постає питання, відки беруться
гроші, потрібні для процесу циркуляції?

Оскільки продукти, що їх продукують $В$, $В'$, $В''$ і~\abbr{т. д.} (І), сами
знову входять in natura в їхній процес продукції, то само собою зрозуміло,
що pro tanto частина їхнього власного додаткового продукту безпосередньо
(без посередництва циркуляції) переміщується в їхній продуктивний
капітал і входить в нього як додатковий елемент сталого капіталу.
Але pro tanto вони й не перетворюють на золото додаткового
продукту $А$, $А'$ і~\abbr{т. д.} (І). Лишаючи це осторонь, відки ж беруться
гроші? Ми знаємо, що $В$, $В'$, $В''$ і~\abbr{т. д.} (І) утворили свій скарб, як і
$А$, $А'$ і~\abbr{т. д.}, через продаж їхніх відповідних додаткових продуктів і
тепер досягли тієї мети, коли їхній нагромаджений як скарб, лише віртуально
грошовий капітал, повинен тепер справді функціонувати як
додатковий грошовий капітал. Але так ми лише блукаємо в зачарованому
колі. Все ж лишається питання, відки беруться гроші, що їх раніше
вилучили з циркуляції та нагромадили капіталісти $В$, $В'$, $В''$ і~\abbr{т. д.} (І)?

Однак уже з досліду простої репродукції ми знаємо, що в руках
капіталістів І і II мусить бути певна маса грошей для того, щоб перетворювати
їхній додатковий продукт. Гроші, що служили лише для витрат
як дохід на засоби споживання, повертались там назад до капіталістів
в міру того, як вони авансували їх для обміну своїх власних товарів;
тут знову з’являються ті самі гроші, але функція їхня змінилась.
Капіталісти $А$, $А'$ і~\abbr{т. д.} І~$В$, $В'$ і~\abbr{т. д.} (І) навперемінку подають гроші
для перетворення додаткового продукту на додатковий віртуальний грошовий
капітал, і навперемінку знову подають у циркуляцію новоутворений
грошовий капітал як купівельний засіб.

Єдине припущення при цьому те, що наявної в країні маси грошей
(швидкість обігу та ін. припускається однакові) досить так для активної
циркуляції, як і для утворення запасного скарбу; отже — те саме припущення,
що, як ми бачили, мусить бути здійснене і при простій товаровій
циркуляції. Тільки функція скарбу тут інша. Крім того, маса наявних
грошей мусить бути більша: 1) бо при капіталістичній продукції
кожен продукт (за винятком новоспродукованих благородних металів та
небагатьох продуктів, що їх споживає сам продуцент) продукується як
товар, отже, мусить проробити грошове залялькування; 2) бо на капіталістичній
\index{ii}{0390}  %% посилання на сторінку оригінального видання
основі маса товарового капіталу й величина його вартости не
лише абсолютно більша, але й зростає з куди більшою швидкістю;
3) дедалі більший змінний капітал завжди мусить перетворюватись на
грошовий капітал; 4) бо рівнобіжно з поширенням продукції утворюються
нові капітали, отже, мусить бути наявний і матеріял для їхнього нагромадження
в формі скарбу. — Якщо це має силу для першої фази капіталістичної
продукції, коли і кредитова система супроводиться переважно
металевою циркуляцією, то й для найрозвиненішої фази кредитової
системи це має силу остільки, оскільки її базою лишається металева
циркуляція. З одного боку, додаткова продукція благородних металів
може тут, оскільки вона навперемінку буває буйніша або бідніша,
викликати порушення в товарових цінах не лише протягом довгих, а й
в межах дуже коротких періодів часу; з другого боку, ввесь кредитовий
механізм постійно дбає про те, щоб всілякими операціями, методами,
технічними засобами обмежити справжню металеву циркуляцію відносно
дедалі меншим мінімумом, наслідком чого відповідно збільшується також
штучність цілого механізму й шанси на порушення нормального його
перебігу.

Різні $В$, $В'$, $В''$ і~\abbr{т. д.} (І), що їхній віртуальний новий грошовий
капітал вступає в операції як активний капітал, можуть купувати один в одного
і продавати один одному свої продукти (частини свого додаткового
продукту). При нормальному перебігу справ гроші, авансовані на циркуляцію
додаткового продукту, pro tanto повертаються назад до різних
$В$, $В'$ і т. д, в такій самій пропорції, в якій кожен з них авансував
ці гроші на циркуляцію своїх відповідних товарів. Коли гроші циркулюють
як виплатний засіб, то тут доводиться виплачувати лише ріжницю,
оскільки взаємні купівлі й продажі не покривають одна одну. Але важливо
всюди, як ми це робимо тут, припустити спочатку металеву циркуляцію
в її найпростішій, найпервіснішій формі, бо тоді приплив
і відплив грошей, вирівнювання ріжниць, коротко кажучи, всі моменти,
які з’являються в кредитовій системі, як свідомо уреґульовані
процеси, виступлять як наявні незалежно від кредитової системи, і
вся справа виявиться в своїй природній формі, а не в пізнішій,
відображеній.

\subsubsection{Додатковий змінний капітал}

Що до цього часу мова йшла тільки про додатковий сталий капітал,
то тепер маємо перейти до розгляду додаткового змінного
капіталу.

В книзі І\footnote*{
Див. „Капітал“, т. I, розділ 23, § 3. — \Red{Ред.}
} докладно з’ясовано, як на основі капіталістичної продукції
завжди є в запасі робоча сила і як, в разі потреби, можна пустити
в рух більше праці, не збільшуючи числа вживаних робітників або маси
робочої сили. Тому покищо не треба далі зупинятись на цьому, навпаки
\parbreak{}  %% абзац продовжується на наступній сторінці

\parcont{}  %% абзац починається на попередній сторінці
\index{ii}{0391}  %% посилання на сторінку оригінального видання
ми повинні припустити, що частина новоутвореного грошового капіталу,
яку можна перетворити на змінний капітал, завжди знаходить робочу
силу, на яку вона має перетворитись. Так само в книзі І\footnote*{
Див. „Капітал“, т. I, розділ 22, §4. — \emph{Ред.}
} з’ясовано,
як даний капітал може без акумуляції поширювати в певних межах розміри
своєї продукції. Але тут ідеться про акумуляцію капіталу в специфічному
значенні, так що поширення продукції зумовлюється перетворенням
додаткової вартости на додатковий капітал, отже, зумовлюється
поширенням капітальної бази продукції.

Золотопромисловець може акумулювати деяку частину своєї золотої
додаткової вартости як віртуальний грошовий капітал; скоро він досягає
потрібних розмірів, золотопромисловець може безпосередньо перетворити
його на новий змінний капітал, для цього йому не доводиться спочатку
продавати свій додатковий продукт; так само він може перетворити свій
додатковий продукт на елементи сталого капіталу. Однак у цьому останньому
випадку він мусить знайти ці речові елементи свого сталого капіталу,
все одно, чи робить кожен продуцент, як ми припускали в попередньому
викладі, про запас, а потім виносить свої готові товари на ринок,
чи він робить на замовлення. В обох випадках припускається реальне
поширення продукції, тобто припускається додатковий продукт, в
одному випадку як справді наявний, а в другому як віртуально наявний,
що може бути поданий.

\subsection{Акумуляція в підрозділі II}

До цього часу ми припускали, що $А$, $А'$, $А'''$ (І) продають свій додатковий
продукт $В$, $В'$, $В'''$ і т. д., що належать до того самого підрозділу
І.~Але припустімо, що $А$ (І), перетворює на золото свій додатковий
продукт, продаючи його $В$ з підрозділу II.~Це може статися
лише в наслідок того, що $А$ (І), продавши $В$ (II) засоби продукції,
не купує потім засобів споживання; отже, лише в наслідок однобічного
продажу з його боку. Оскільки II~$с$ з форми товарового капіталу можна
перетворити на натуральну форму продуктивного сталого капіталу лише
таким способом, що не тільки I~$v$, але принаймні, і деяка частина I~$m$
обмінюється на деяку частину II~$с$, яке існує у формі засобів споживання;
але тепер А перетворює на золото своє І~$m$ в такий спосіб, що такого
обміну не постає; навпаки, наш $А$ вилучає з циркуляції гроші, вторговані
від II через продаж свого І~$m$, замість повертати їх на закуп засобів
споживання II~$с$, — то хоч на боці $А$ (І) утворюється додатковий
віртуальний грошовий капітал, але на другому боці закріплюється в формі
товарового капіталу рівна величиною вартости частина сталого капіталу
$В$ (II), яка не може перетворитись на натуральну форму продуктивного
сталого капіталу. Інакше кажучи, не сила продати частину товарів
$В$ (II), а саме prima facie ту частину, що без її продажу $В$ (II) не може
знову перетворити ввесь свій сталий капітал на продуктивну форму;
\parbreak{}  %% абзац продовжується на наступній сторінці

\parcont{}  %% абзац починається на попередній сторінці
\index{ii}{0392}  %% посилання на сторінку оригінального видання
тому щодо цієї частини постає перепродукція, яка, знову таки щодо цієї
частини, гальмує репродукцію — навіть у незмінному маштабі.

Отже, у цьому випадку додатковий віртуальний грошовий капітал на
боці $А$ (І) є, правда, перетворена на золото форма додаткового продукту
(додаткової вартости); але додатковий продукт (додаткова вартість), розглядуваний
як такий, є тут явище простої репродукції, але ще не явище
репродукції в поширеному маштабі. І ($v \dplus{} m$) — а для нього це в усякому
разі має силу щодо деякої частини $m$ — кінець-кінцем, мусить обмінятись
на II~$с$, щоб репродукція II~$c$ могла відбуватися в незмінному маштабі.
$А$ (І), продавши $В$ (II) свій додатковий продукт, дав йому відповідну
частину вартости сталого капіталу в натуральній формі, але разом з тим,
вилучаючи гроші з циркуляції й не доповнюючи свого продажу наступною
купівлею, унеможливив продаж частини товарів $В$ (II) такої самої
вартости. Отже, коли ми маємо на увазі всю суспільну репродукцію, яка
однаково охоплює капіталістів І і II, то перетворення додаткового продукту
$А$ (І) на віртуальний грошовий капітал виражає неможливість зворотного
перетворення рівного величиною вартости товарового капіталу
$В$ (II) на продуктивний (сталий) капітал; отже, виражає не віртуальну
продукцію в поширеному маштабі, а гальмування простої репродукції,
тобто дефіцит у простій репродукції. Що утворення й продаж додаткового
продукту $А$ (І) сами собою є нормальні явища простої репродукції,
то ми маємо тут уже на основі простої репродукції такі навзаєм зумовлювані
явища: утворення віртуально додаткового грошового капіталу у
кляси І (тому недостатнє споживання з погляду II); затримка у кляси II
товарових запасів, що їх не можна знову перетворити на продуктивний
капітал (отже, відносна перепродукція в II); надмірний грошовий капітал
в І і дефіцит у репродукції в II.

Не зупиняючись далі на цьому пункті, зауважмо лише ось що.
Коли описувалось просту репродукцію, то припускалось, що всю додаткову
вартість І і II витрачається як дохід. Але в дійсності одну частину
додаткової вартости витрачається як дохід, а другу частину перетворюється
на капітал. Тільки при такому припущенні відбувається дійсна
акумуляція. Вважати, що акумуляція відбувається коштом споживання, —
коли це розуміти в такій загальній формі, — є ілюзія, яка суперечить суті
капіталістичної продукції, бо це припускає, що мета й движний мотив
її є споживання, а не здобування додаткової вартости й капіталізація її,
тобто акумуляція.

\pfbreak

Розгляньмо тепер трохи ближче акумуляцію в підрозділі II.

Перша трудність щодо II~$с$, тобто щодо його зворотного перетворення
з складової частини товарового капіталу II на натуральну форму
сталого капіталу II, стосується до простої репродукції. Візьмімо попередню
схему:
\begin{center}
  ($1000 v \dplus{} 1000 m$) І обмінюються на
  2000 II~$с$.
\end{center}



\noindent{}Коли, напр., половину додаткового продукту І, отже, $\frac{1000}{2} m$ або
500 І~$m$ знову вводиться як сталий капітал у підрозділ І, то ця частина
додаткового продукту, затримана в І, не може замістити жодної частини
II~$с$. Замість перетворитись на засоби споживання (а тут у цьому підрозділі
циркуляції — на відміну від заміщення 1000 II~$с$ через 1000 І~$v$,
що його упосереднюють робітники, — між І і II відбувається справжній
взаємний обмін, тобто двобічне переміщення товарів) ця частина мусить
служити за додаткові засоби продукції в самому І.~Вона не може виконувати
таку функцію одночасно в І і II.~Капіталіст не може витрачати
вартість свого додаткового продукту на засоби споживання й разом
з тим продуктивно споживати самий додатковий продукт, тобто долучати
його до свого продуктивного капіталу. Отже, замість 2000 І ($v \dplus{} m$)
лише 1500, а саме ($1000 v \dplus{} 500 m$) І можна обміняти на 2000 II~$с$;
отже, 500 II~$с$ не можуть знову перетворитися з своєї товарової форми
на продуктивний (сталий) капітал II.~Отже, в II сталася б перепродукція,
яка своїми розмірами точно відповідала б розмірам поширення продукції
в І. Можливо, що перепродукція в II так дуже впливала б на І,
що навіть зворотний приплив 1000, витрачених робітниками І на засоби
споживання II, відбувся б лише почасти, отже, ці 1000 не повернулися б
в формі змінного грошового капіталу до рук капіталістів І.~Таким чином,
для цих останніх утруднилось би навіть репродукцію в попередніх
розмірах і саме в наслідок простої лише спроби поширити її. І при цьому
треба взяти на увагу, що фактично в І відбулась лише проста репродукція,
і що елементи її, як їх подано в нашій схемі, лише згруповано інакше
задля майбутнього поширення її, напр., у наступному році.
\index{ii}{0393}  %% посилання на сторінку оригінального видання

Можна було б зробити спробу обійти цю трудність так: 500 II~$с$,
що лежать на складі у капіталістів і що їх не можна безпосередньо
перетворити на продуктивний капітал, зовсім не являють перепродукції,
а, навпаки, являють доконечний елемент репродукції, що його ми досі
не брали на увагу. Ми бачили, що грошовий запас мусить нагромаджуватись
у багатьох пунктах, отже, гроші доводиться вилучати з циркуляції
почасти для того, щоб уможливити утворення нового грошового
капіталу в самому І, почасти для того, щоб вартість поступінно
зужиткованого основного капіталу деякий час затримувати у грошовій
формі. А що за нашою схемою всі гроші та всі товари з самого
початку перебувають виключно в руках капіталістів І і II і що
при цьому не існує ні купців, ні торговців грішми, ні банкірів, ні таких
кляс, що лише споживають і не беруть безпосередньої участи в продукції
товарів, — то тут для того, щоб підтримувати в русі механізм репродукції,
також конче потрібне постійне утворення товарових запасів
у руках самих відповідних продуцентів їх. Отже, 500 II~$с$, що лежать
на складах у капіталістів II, репрезентують товаровий запас засобів споживання,
що забезпечує безперервність процесу споживання, включеного
в репродукцію, і значить, в даному разі забезпечує перехід від одного
року до наступного. Фонд споживання, який при цьому є ще в руках
\parbreak{}  %% абзац продовжується на наступній сторінці

\input{ii/_0394c.tex}
\parcont{}  %% абзац починається на попередній сторінці
\index{ii}{0395}  %% посилання на сторінку оригінального видання
Ми взяли суму меншу, ніж у схемі І, саме для того, щоб унаочнити,
що репродукція в поширеному маштабі (а ми її розуміємо тут як продукцію,
проваджувану з більшим капіталовкладенням) не має жодного
чинення до абсолютної величини продукту, що для даної маси товарів
вона має за передумову тільки інше розміщення або інше функціональне
призначення різних елементів даного продукту, отже, за величиною вартости,
вона є спочатку лише проста репродукція. Змінюється не кількість,
а якісне призначення даних елементів простої репродукції, і ця зміна є
матеріяльна передумова пізніше посталої репродукції в поширеному
маштабі\footnote{
Це раз назавжди кладе край суперечці про акумуляцію капіталу між Джемсом
Міллом і S.~Bailey’єм, розглянутій в кн. І (розділ XXII, 5) з іншого
погляду; а саме — суперечкі про те, в якій мірі може розширюватися діяння промислового
капіталу при незмінній величині його. Пізніше слід до цього повернутись.
}.

Ми могли б подати цю схему інакше, при іншому відношенні між
змінним і сталим капіталом; напр., так:
\[
 \text{Схема b)} \qquad \left.\begin{array}{r@{~}r@{~}r@{~}r@{~}l}
        \text{I. }&4000 с \dplus{}& 875 v \dplus{}& 875 m & \deq{} 5750\\
        \text{II. }&1750 с \dplus{}& 376 v \dplus{}& 376 m & \deq{} 2502
       \end{array}
 \right\}
 \qquad \text{Сума \deq{} 8252.}
\]
В такому вигляді вона виступала б як побудована для простої репродукції,
так що всю додаткову вартість витрачалося б як дохід, а не нагромаджувалось.
В обох випадках, і при а) і при b), ми маємо річний
продукт однакової величини вартости, тільки в схемі b) його елементи
групуються за своїми функціями так, що знову починається репродукція
в попередньому маштабі, тимчасом як в схемі а) утворюється матеріяльна
основа для репродукції в поширеному маштабі. А саме в схемі b)
($875 v \dplus{} 875 m$) I \deq{} 1750 І ($v \dplus{} m$) обмінюються без остачі на
1750 II~$c$, тимчасом як в схемі а) ($1000 v \dplus{} 1000 m$) I \deq{} 2000 І
($v \dplus{} m$) при обміні на 1500 II~$с$ дають остачу в 500 І~$m$ для акумуляції
в клясі І.

Тепер до ближчої аналізи схеми а). Припустімо, що і в І і в II половину
додаткової вартости, замість витрачати як дохід, акумулюється, тобто
перетворюється на елемент додаткового капіталу. А що половину 1000
І~$m$ \deq{} 500 повинно акумулювати в тій або іншій формі, вкласти як додатковий
грошовий капітал, тобто перетворити на додатковий продуктивний
капітал, то як дохід витрачається тільки ($1000 v \dplus{} 500 m$) І.~Тому
як нормальна величина II~$с$ тут фігурують теж лише 1500. Обмін між
1500 І ($v \dplus{} m$) і 1500 II~$с$ не треба досліджувати далі, бо ми вже описали
його як процес простої репродукції; так само не треба розглядати
4000 І~$с$, бо поновне розміщення його для репродукції, що знову починається
(а вона тепер відбувається в поширеному маштабі), ми теж розглянули
як процес простої репродукції.

Отже, нам лишається дослідити тут тільки одне, а саме 500 І~$m$ і
($376 v \dplus{} 376 m$) II, оскільки розглядається, з одного боку, внутрішнє відношення
так в І, як і в II, а з другого боку — рух між ними обома. А що припускається,
\parbreak{}  %% абзац продовжується на наступній сторінці

\parcont{}  %% абзац починається на попередній сторінці
\index{ii}{0396}  %% посилання на сторінку оригінального видання
що в II теж повинна акумулюватись половина додаткової вартости,
то тут на капітал треба перетворити 188, з них на змінний капітал
1/4 \deq{} 47, беручи заокруглено, 48; лишається 140, що їх треба
перетворити на сталий капітал.

Тут ми натрапляємо на нову проблему, що саме існування її мусить
здаватись чимсь дивним з того загального погляду, за яким
товари одного ґатунку звичайно обмінюються на товари іншого ґатунку,
або товари обмінюються на гроші, а ці гроші знов обмінюються на
товари іншого ґатунку. Ці 140 II~$m$ тільки тому можуть перетворитись
на продуктивний капітал, що їх заміщується частиною товарів І~$m$ на
ту саму суму вартости. Зрозуміло само собою, що частина І~$m$, обмінювана
на II~$m$, мусить складатися з засобів продукції, які можуть
увійти так у продукцію І, як і в продукцію II, або тільки виключно
в продукцію II. Це заміщення може статися лише через
однобічну купівлю з боку II, бо ввесь додатковий продукт 500 І~$m$, що
його нам ще треба дослідити, повинен служити для акумуляції в межах І,
отже, його не можна обміняти на товари II; інакше кажучи, І не може
одночасно і акумулювати його й з’їдати. Отже, II мусить купити 140 І~$m$
за готівку, при чому ці гроші не повертаються до нього через наступний
продаж його товару підрозділові І.~І такий процес повторюється постійно,
при кожній новій річній продукції, оскільки вона є репродукція в поширеному
маштабі. Відки ж у II походить джерело грошей для цього?

Навпаки, II підрозділ, здається, є цілком неплідне поле для утворення
нового грошового капіталу, яке супроводить справжню акумуляцію і
зумовлює її при капіталістичній продукції, — утворення нового грошового
капіталу, що фактично спочатку виступає як просте утворення
скарбу.

Спочатку маємо 376 II~$v$; грошовий капітал в 376, авансований на
робочу силу, через закуп товарів II постійно повертається назад до
капіталіста II як змінний капітал у грошовій формі. Це постійно повторюване
віддалення від вихідного пункту — з кишені капіталіста — і поворот
до нього ні в якому разі не збільшує кількости грошей, що
циркулюють в цьому кругобігу. Отже, воно не є джерело акумуляції
грошей; цих грошей не можна також вилучити з цієї циркуляції для
того, щоб утворити нагромаджуваний як скарб віртуально новий грошовий
капітал.

Але почекайте! Чи не можна з цього здобути якийсь маленький
баришик?

Ми не повинні забувати, що кляса II має тут перевагу проти кляси І,
що вживані нею робітники знову повинні купувати в неї товари, спродуковані
ними самими. Кляса II є покупець робочої сили і разом з тим
продавець товарів власникам застосовуваної нею робочої сили. Отже,
кляса II може:

1) і це в неї спільне з капіталістами кляси І, — просто знижувати
заробітну плату нижче від її нормального пересічного рівня. В наслідок цього
звільняється частина грошей, які функціонують як грошова форма змін
\index{ii}{0397}  %% посилання на сторінку оригінального видання
ного капіталу, і при постійному повторюванні цього процесу це могло б
стати нормальним джерелом утворення скарбів, а значить, і джерелом
утворення віртуального додаткового грошового капіталу кляси II.~Звичайно,
тут, де йдеться про нормальне утворення капіталу, ми лишаємо
осторонь випадковий зиск з шахрайства. Але не треба забувати, що
справді виплачувану нормальну заробітну плату (а вона ceteris paribus\footnote*{
В інших однакових умовах. \emph{Ред.}
}
визначає величину змінного капіталу) виплачується зовсім не з ласки
капіталістів; її виплачується тому, що при даних відношеннях вона
мусить бути виплачена. Таким чином, цей спосіб пояснення усувається.
Коли ми припускаємо, що змінний капітал, який має витратити кляса II,
становить $376 v$, то для того, щоб розв’язати новопосталу проблему, ми
не можемо одразу висунути гіпотезу, що кляса II авансує, напр., тільки
$350 v$, а не $376 v$.

2) Але, з другого боку, як ми вже сказали, кляса II, розглядувана як
ціле, має ту перевагу проти кляси І, що як покупець робочої сили вона
разом з тим є продавець своїх товарів своїм власним робітникам. Як це
можна визискувати, — яким чином можна номінально виплачувати нормальну
заробітну плату, а в дійсності частину її загарбати собі без відповідного
еквіваленту, інакше кажучи, украсти в робітників; як це можна робити
почасти за допомогою truck system, а почасти через фалшування засобів
циркуляції (хоч його не завжди дається виявити юридично), — про
це є цілком наочні дані в кожній промисловій країні, напр., в Англії та
Сполучених Штатах. (З цього приводу треба це розвинути на влучно
обраних прикладах). Це — та сама операція, що в 1), тільки замаскована
і пророблювана обкружним шляхом. Отже, її треба тут відкинути так
само, як і ту. Тут ідеться не про номінальну, а про справді виплачувану
заробітну плату.

Ми бачимо, що при об’єктивній аналізі капіталістичного механізму не
можна скористатися з деяких ганебних плям, які екстраординарно ще
гніздяться в ньому, для того, щоб викрутом усунути теоретичні труднощі.
Але більшість моїх буржуазних критиків якось чудно здіймає галас, ніби
я, прим., в І книзі „Капіталу“ своїм припущенням, що капіталіст виплачує
дійсну вартість робочої сили — а цього він здебільша не робить —
зробив велику кривду цим самим капіталістам! (Тут можна з тією самою
великодушністю, яку приписується мені, цитувати Шефле).

Отже, з 376 II~$v$ для згаданої мети нічого не вдієш.

Але ще більші труднощі, здається, постають щодо 376 II~$m$. Тут протистоять
один одному лише капіталісти тієї самої кляси, що продають
один одному й купують один в одного продуковані ними засоби споживання.
Гроші, потрібні для цього обміну, функціонують тільки як засоби
циркуляції, і при нормальному перебігу мусять повертатись назад
до учасників у тій мірі, в якій вони їх авансували для циркуляції, а потім
знову й знов переходити той самий шлях.

\input{ii/_0398c.tex}
\input{ii/_0399c.tex}
\parcont{}  %% абзац починається на попередній сторінці
\index{ii}{0400}  %% посилання на сторінку оригінального видання
тут покищо лишимо осторонь гроші, то як результат цієї оборудки
матимемо:

I.    $4400 с \dplus{} 50 m$ (що їх треба капіталізувати); крім того, в споживному
фонді капіталістів і робітників 1650 ($v \dplus{} m$) реалізовані в
товарах II~$с$.

II.    $1650 с$ (а саме 50, як сказано вище, додано з II~$m$) $\dplus{} 800 v \dplus{}
750 m$ (споживний фонд капіталістів).

Але коли в II між $v$ і $с$ зберігається попередня пропорція, то на $50 с$
доведеться витратити ще $25 v$; їх можна взяти з $750 m$; отже, маємо:
\[
\text{II. }1650 с \dplus{} 825 v \dplus{} 725 m
\]
В І треба капіталізувати $550 m$; коли зберігається попередня пропорція,
то 440 з них становлять сталий капітал, а 110 — змінний капітал.
Ці 110 можна взяти з 725 II~$m$, тобто засоби споживання вартістю в
110 споживуть робітники І, а не капіталісти II; отже, останні примушені
капіталізувати ці $110 m$, що їх вони не можуть спожити. Таким чином,
з 725 II~$m$ лишається 615 II~$m$. Але коли II таким чином перетворює
ці 110 на додатковий сталий капітал, то йому потрібно ще 55 додаткового
змінного капіталу; їх він мусить взяти знову таки з своєї додаткової
вартости; коли їх відлічити з 615 II~$m$, то для споживання капіталістів II
лишиться 560; зробивши всі ці справжні й потенціяльні переміщення,
матимемо таку капітальну вартість:
\[
 \left.\begin{array}{@{}r@{~}c@{~\dplus{}~}c@{~\deq{}~}r@{~}r@{~}l@{}}
        \text{I.}&
            (4400 с \dplus{} 440 с)
            &(1100 v \dplus{} 110 v)
            &4840 c \dplus{} & 1210 v & \deq{} 6050\\
        \text{II.}&
            (1600 с \dplus{} 50 с \dplus{} 110 с)
            &(800 v \dplus{} 25 v \dplus{} 55 v)
            &1760 c \dplus{} & 880 v & \deq{} 2640\\
       \end{array}
 \right\}
 \text{\deq{} 8690.}
\]
Для того, щоб справа йшла нормально, акумуляція II мусить відбуватись
швидше, ніж у І, бо інакше частина І ($v \dplus{} m$), яку треба обміняти
на товари II~$с$, зростала б швидше, ніж II~$с$, що на нього тільки й можна
її обміняти.

Коли на цій основі та в інших незмінних умовах репродукцію провадитиметься
й далі, то наприкінці наступного року матимемо:

\noindent{}
\[
 \left.\begin{array}{r@{~}r@{~}r@{~}r@{~}l}
        \text{I. }&4840 с \dplus{}& 1210 v \dplus{}& 1210 m & \deq{} 7260\\
        \text{II. }&1760 с \dplus{}& 880 v \dplus{}& 880 m & \deq{} 3520
       \end{array}
 \right\}
 \text{\deq{} \num{10780}.}
\]
Насамперед при незмінній нормі розподілу додаткової вартости в І
треба витратити як дохід: $1210 v$ й половину $m$ \deq{} 605, разом 1815.
Цей фонд споживання знову на 55 більший, ніж II~$с$. Ці 55 слід відлічити
з $880 m$, лишається 825. Якщо 55 II~$m$ перетворюються на ІІ~$с$, то це
має собі за передумову дальше відлічення з II~$m$ для відповідного змінного
капіталу \deq{} 27\sfrac{1}{2}; лишається для споживання 797\sfrac{1}{2} II~$m$.

Тепер треба в І капіталізувати $605 m$; з них 484 сталого капіталу й
121 змінного; ці останні треба відлічити з II~$m$, що тепер дорівнює
\parbreak{}  %% абзац продовжується на наступній сторінці

\parcont{}  %% абзац починається на попередній сторінці
\index{ii}{0401}  %% посилання на сторінку оригінального видання
797\sfrac{1}{2}; лишається $676\sfrac{1}{2} \text{II} m$. Отже, II перетворює на сталий капітал
ще 121, і для цього треба йому 60\sfrac{1}{2} нового змінного капіталу: його
так само береться з 676\sfrac{1}{2}; для споживання лишається 616.

Тоді матимемо капіталу:
\[
\begin{array}{r@{~}l@{~}r@{~}r@{~}l}
  \text{I. }&\text{Сталого} & 4840 \dplus{} & 484 & \deq{} 5324\\
            &\text{Змінного}& 1210 \dplus{} & 121 & \deq{} 1331
\end{array}
\]
\[
\begin{array}{r@{~}l@{~}r@{~}r@{~}r@{~}l}
  \text{II. }&\text{Сталого} & 1760 \dplus{} 
    & 55\phantom{\sfrac{1}{2}} \dplus{} & 121\phantom{\sfrac{1}{2}} & \deq{} 1936\\
            &\text{Змінного}& 880 \dplus{} 
    & 27\sfrac{1}{2} \dplus{} & 60\sfrac{1}{2} &\deq{} 1331
\end{array}
\]
\[
 \text{Разом: }\left.\begin{array}{r@{~}r@{~}r@{~}l}
        \text{I. }&5324 с \dplus{}& 1331 v & \deq{} 6655\\
        \text{II. }&1936 с \dplus{}& 968 v & \deq{} 2904
       \end{array}
 \right\}
 \text{\deq{} \num{9559},}
\]
а наприкінці року матимемо продукту:
\[
 \left.\begin{array}{r@{~}r@{~}r@{~}r@{~}l}
        \text{I. }&5324 с \dplus{}& 1331 v \dplus{}& 1331 m & \deq{} 7986\\
        \text{II. }&1936 с \dplus{}& 968 v \dplus{}& 968 m & \deq{} 3872
       \end{array}
 \right\}
 \text{\deq{} \num{11858}.}
\]
Повторюючи це обчислення й заокруглюючи дроби, матимемо наприкінці
наступного року продукту:
\[
 \left.\begin{array}{r@{~}r@{~}r@{~}r@{~}l}
        \text{I. }&5856 с \dplus{}& 1464 v \dplus{}& 1464 m & \deq{} 8784\\
        \text{II. }&2129 с \dplus{}& 1065 v \dplus{}& 1065 m & \deq{} 4259
       \end{array}
 \right\}
 \text{\deq{} \num{13043}\footnotemarkZ{}.}
\]
\footnotetextZ{В нім. тексті тут, як і подекуди далі, є аритметичні помилки. Ці помилки ми виправили. \Red{Ред}.} % текст примітки прямо під заголовком
А наприкінці наступного року:
\[
 \left.\begin{array}{r@{~}r@{~}r@{~}r@{~}l}
        \text{I. }&6442 с \dplus{}& 1610 v \dplus{}& 1610 m & \deq{} 9662\\
        \text{II. }&2342 с \dplus{}& 1171 v \dplus{}& 1170 m & \deq{} 4684
       \end{array}
 \right\}
 \text{\deq{} \num{14346}.}
\]
Протягом чотирилітньої репродукції в поширеному маштабі ввесь
капітал І і II збільшився з $5500 с \dplus{} 1750 v \deq{} 7250$ до $8784 с \dplus{} 2781 v \deq{}
\num{11565}$, отже, у відношенні 100: 160. Вся додаткова вартість спочатку
становила 1750, тепер вона становить 2781. Спожита додаткова вартість
спочатку була 500 для І і 600 для II, а разом 1100; вона була в
\parbreak{}  %% абзац продовжується на наступній сторінці

\parcont{}  %% абзац починається на попередній сторінці
\index{ii}{0402}  %% посилання на сторінку оригінального видання
останньому році 732 для І і 746 для II, разом 1478. Отже, вона
зросла у відношенні $100: 134$.

\subsubsection{Другий приклад}

Візьмім тепер річний продукт в 9000, що цілком перебуває в руках
кляси промислових капіталістів як товаровий капітал, у формі, де загальне
пересічне відношення змінного й сталого капіталу становить $1: 5$.
Це має за передумову: уже значний розвиток капіталістичної продукції
й відповідний цьому розвиток продуктивної сили суспільної праці; далі
це має за передумову значне, вже раніш постале поширення маштабу
продукції, нарешті, розвиток усіх умов, що спричиняють відносне перелюднення
в робітничій клясі. Річний продукт буде тоді розподілятись по
заокругленні дробів так:
\[
 \left.\begin{array}{r@{~}r@{~}r@{~}r@{~}l}
        \text{I. }&5000 с \dplus{}& 1000 v \dplus{}& 1000 m & \deq{} 7000\\
        \text{II. }&1430 с \dplus{}& 285 v \dplus{}& 285 m & \deq{} 2000
       \end{array}
 \right\}
 \text{\deq{} \num{9000}.}
\]
Припустімо тепер, що кляса капіталістів І половину додаткової вартости
\deq{} 500 споживає, а другу половину акумулює. Тоді ($1000 v \dplus{}
500 m) \text{I} \deq{} 1500$ треба було б замістити через $1500 \text{ II} с$. А що $\text{ II} с$
дорівнює тут лише 1430, то 70 треба додати з додаткової вартости;
відлічуючи їх з $285 \text{ II} m$, маємо остачу $215 \text{ II} m$.

Отже, маємо:

I.  $5000 с \dplus{} 500 m\text{ (що їх треба капіталізувати)} \dplus{} 1500 (v \dplus{} m)$ в споживному
фонді капіталістів і робітників.

II.  $1430 c \dplus{} 70 m\text{ (що їх треба капіталізувати)} \dplus{} 285 v \dplus{} 215 m$.

А що при цьому $70 \text{ II} m$ безпосередньо долучаються до $\text{ II} c$, то для
того, щоб пустити в рух цей додатковий сталий капітал, треба змінного
капіталу в \frac{70}{5} \deq{} 14; ці 14 знову береться з $215 \text{ II} с$; лишається $201 \text{ II} m$,
і ми маємо:
\[
\text{II. }(1430с \dplus{} 70с) \dplus{} (285 v \dplus{} 14 v) \dplus{} 201 m.
\]
Обмін 1500 І ($v \dplus{} \sfrac{1}{2} m$) на $1500 \text{ ІІ} с$ є процес простої репродукції,
і тому з ним закінчено. Однак ми повинні тут зазначити ще деякі особливості,
які випливають з того, що при репродукції сполученій з акумуляцією,
$\text{І} (v \dplus{} \sfrac{1}{2} m)$ заміщується не самим лише $\text{ II} с$, а $\text{ ІІ} с$ плюс частина
$\text{ II} m$.

Зрозуміло само собою, що коли припускається акумуляцію, то $\text{І} (v \dplus{} m)$
більше за $\text{ ІІ} с$, а не дорівнює $\text{ ІІ} с$, як при простій репродукції, бо
1) І заводить частину свого додаткового продукту в свій власний
продуктивний капітал і перетворює \sfrac{5}{6} цієї частини на сталий капітал,
отже, він не може разом з тим замістити ці \sfrac{5}{6} засобами споживання II;
2) І з свого додаткового продукту повинен дати матеріял для сталого
\parbreak{}  %% абзац продовжується на наступній сторінці

\parcont{}  %% абзац починається на попередній сторінці
\index{ii}{0403}  %% посилання на сторінку оригінального видання
капіталу, потрібного для акумуляції в межах II, цілком так само, як II
повинен дати матеріял для змінного капіталу, що має пустити в рух ту
частину додаткового продукту І, яку сам І застосовує як додатковий
сталий капітал. Ми знаємо, що справжній змінний капітал, а значить, і
додатковий, складається з робочої сили. Тепер не капіталіст І купує в II
доконечні засоби існування про запас і зберігає їх для додаткової робочої
сили, що йому її треба буде застосувати в майбутньому, як це
мусив робити рабовласник. Сами робітники мають справу з II.~Та це не
заважає тому, що з погляду капіталіста засоби споживання додаткової
робочої сили є лише засоби продукції та зберігання його евентуальної
додаткової робочої сили, отже, натуральна форма його змінного капіталу.
Його власна найближча операція, в даному разі виконувана І, сходить
лише на те, що він нагромаджує новий грошовий капітал, потрібний для
закупу додаткової робочої сили. Скоро тільки він долучає її до свого
капіталу, гроші стають для цієї робочої сили засобом до купівлі
товарів II, отже, вона мусить знайти засоби свого споживання.

Між іншим. Пан капіталіст, як і його преса, часто бувають незадоволені
з того способу, яким робоча сила витрачає свої гроші, і з
тих товарів II, що в них вона реалізує ці гроші. Він філософує з цього
приводу, розводить теревені про культуру та удає філантропа, як це,
прим., робить п. Друммонд, секретар англійського посольства в Вашінґтоні.
„The Nation“ (газета) наприкінці жовтня 1879~\abbr{р.} вмістила цікаву
статтю, де, між іншим, сказано: „Робітники відстали в культурі від поступу
в винаходах; для них стала приступна маса речей, що їх вживати
вони не вміють, і що для них вони, отже, не являють ринку“.
(Кожен капіталіст, звичайно, хоче, щоб робітник купував його товар).
„Немає жодної підстави для того, щоб робітник не хотів жити з таким
самим комфортом, як піп, адвокат або лікар, що одержує стільки ж, як
і він“. (Справді багатенько комфорту можуть дозволити собі з свого бажання
такі адвокати, попи й лікарі!). „Але він цього не робить. Питання
все ще в тому, якими раціональними здоровими заходами підвищити його
рівень як споживача; це питання не легке, бо все його шанолюбство не
йде далі скорочення робочих годин, і демагоги радше підбурюють його
до цього, ніж до поліпшення його стану через удосконалення його розумових
і моральних здібностей“. (Reports of Н.~M-s Secretaries of Embassy
and Legation on the Manufactures, Commerce etc. of the Countries in
which they reside. London 1879, p. 404).

Довгий робочий день являє, певно, таємницю раціональних і здорових
заходів, що повинні поліпшити стан робітника, удосконалюючи його розумові
й моральні здібності, та зробити з нього раціонального споживача.
Щоб стати раціональним споживачем товарів капіталістів він мусить
насамперед почати — але цьому заважає демагог! — з того, щоб дозволити
своєму власному капіталістові споживати його робочу силу
нераціональним і шкідливим для здоров’я способом. Як розуміє
капіталіст раціональне споживання, це виявляється там, де ласка капіталіста
доходить того, що він безпосередньо береться до торговлі
\index{ii}{0404}  %% посилання на сторінку оригінального видання
засобами споживання своїх робітників, — в truck-system’i, що одною з
багатьох галузей її є винаймання помешкань робітникам, так що капіталіст
є разом з тим квартировласник своїх робітників.

Той самий Друммонд, що його прекрасна душа мріє про капіталістичні
спроби поліпшити стан робітничої кляси, в тому самому звіті розповідає,
між іншим, про зразкові бавовнопрядільні Lowell and Lawrence Mills,
їдальні й помешкання робітниць належать акційному товариству, якому
належать і сами фабрики; завідательки цих будинків служать у того
самого товариства, яке приписує їм правила поведінки; жодна робітниця
не сміє повертатись додому пізніше, ніж о 10-ій годині вечора. Але
ось перл: патрулі спеціяльної поліції товариства доглядають у навкольності,
щоб не порушувалось цей житловий порядок. Після 10-ої години
вечора жодну робітницю не випускають з будинку й не впускають туди.
Жодна робітниця не сміє мешкати десь інде, крім території, що належить
товариству; кожний будинок дає товариству щотижня приблизно 10\usd{ долярів}
плати за помешкання; і тут ми бачимо раціональних споживачів у всій
славі. „Що в багатьох кращих будинках для робітниць є повсюди піяніно,
то музика, співи й танці відіграють значну ролю, принаймні для тих з
них, кому після нудної безперервної десятигодинної праці біля ткацького
варстату, більш потрібна переміна, ніж справжній відпочинок“ (стор. 412).
Але головна таємниця, як з робітника зробити раціонального споживача,
ще далі. Пан Друммонд завітав на фабрику ножів Turners Falls (Connecticut
River), і пан Окмен, скарбник акційного товариства, розповівши
йому, що американські столові ножі якістю переважають англійські, додав:
„Ми поб’ємо Англію і щодо цін; ми вже й тепер перевищуємо її якістю,
це визнано; але ми мусимо мати й дешевші ціни, і ми досягнемо цього
так само, як ми одержали дешевше нашу сталь і знизили плату за нашу
працю!“ (стор. 427). Зниження заробітної плати й довгий робочий день —
у цьому вся суть раціональних і здорових заходів, що повинні піднести
робітника до ранґу раціонального споживача, щоб він утворив ринок
для маси предметів, зроблених для нього приступними культурою та поступом
у винаходах.

\pfbreak

Отже, як І повинен дати з свого додаткового продукту додатковий
сталий капітал для II, так II дає в цьому розумінні додатковий змінний
капітал для І.~Оскільки ходить про змінний капітал, II акумулює для І і
для себе самого, репродукуючи більшу частину всього свого продукту,
отже, і свого додаткового продукту, в формі доконечних засобів споживання.
При продукції на дедалі більшій капітальній базі І ($v \dplus{} m$) мусить
дорівнювати $\text{ II} с$ плюс та частина додаткового продукту, яка знову долучається
до капіталу, плюс додаткова частина сталого капіталу, потрібна
для поширення продукції в II; а мінімум цього поширення такий, що
без нього не можлива справжня акумуляція, тобто справжнє поширення
продукції в самому І.

\input{ii/_0405.tex}

\index{ii}{0406}  %% посилання на сторінку оригінального видання
I. $5000 с \dplus{} 500 m \text{ (що їх треба капіталізувати)} \dplus{} 1500 (v \dplus{} m)$ споживного
фонду \deq{} 7000 в товарах.

II. $1500 с \dplus{} 299 v \dplus{} 201 m \deq{} 2000$ в товарах. Загальна сума 9000 в
товаровому продукті.

Капіталізація тепер відбувається так:

В І підрозділі $500 m$, що їх капіталізується, поділяються на $\sfrac{5}{6} \deq{}
417 с \dplus{} \sfrac{1}{6} \deq{} 83 v$\footnote*{
\label{note-406}
Обчислено з закругленням дробів. \Red{Ред.}
}. Ці $83 v$ вилучають таку саму суму з $\text{II} m$, і на
неї купується елементи сталого капіталу, отже, їх долучається до $\text{II} с$.
Збільшення $\text{II} с$ на 83 зумовлює збільшення 
$\text{II} v$ на \sfrac{1}{5} від $83 \deq{} 17$\footref{note-406}.
Отже, після обміну ми маємо:
\[
 \left.\begin{array}{r@{~}r@{~}r@{~}r@{~}r@{~}r@{~}r@{~}l}
        \text{I. } & (5000 с \dplus{} 
          & 417 m) с \dplus{} & (1000 v & \dplus{} 83 m) v \deq{} & 
             5417 c \dplus{} & 1083 v \deq{} & 6500 \\

        \text{II. }& (1500 с \dplus{} 
          & 83 m) c \dplus{} & (299 v & \dplus{} 17 m) v \deq{} &
            1583 с \dplus{} & 316 v \deq{} & 1899
       \end{array}
 \right\}
 \text{\deq{} \num{8399}.}
\]
Капітал в І зріс з 6000 до 6500, отже, на \sfrac{1}{12}. В II з 1715 до 1899,
отже, майже на \sfrac{1}{9}.

На другий рік репродукція на такій основі дає наприкінці року
капітал:
\[
 \begin{array}{@{}r@{~}c@{~\dplus{}~}c@{~\deq{}~}r@{~}r@{~}l@{}}
  \text{I.}&
      (5417 \dplus{} 452 m) c
      &(1083 v \dplus{} 90 m) v
      &5869 с \dplus{} & 1173 v & \deq{} 7042\\
  \text{II.}&
     (1583 с \dplus{} 42 m \dplus{} 90 m) c
      &(316 v \dplus{} 8m \dplus{} 18 m) v
      &1715 c \dplus{} & 342 v& \deq{} 2057\text{,}\\
  \end{array}
\]
а наприкінці третього року дає продукт:
\[
 \begin{array}{r@{~}r@{~}r@{~}r@{~}r}
  \text{I.}& 5869 c \dplus{}& 1173 v \dplus{}& 1173 m \\
  \text{II.}& 1715 с \dplus{}& 342 v \dplus{}& 342 m& \text{.}
  \end{array}
\]
Коли І акумулює при цьому, як і раніше, половину додаткової вартости,
то І ($v \dplus{} \sfrac{1}{2} m$) дає $1173 v \dplus{} 587 (\sfrac{1}{2} m) \deq{} 1760$, отже, більше,
ніж усі $1715 \text{ ІІ} с$, а саме більше на 45. Отже, цю ріжницю знову треба
покрити переміщенням до II~$с$ засобів продукції на таку саму суму. Отже,
ІІ~$с$ зростає на 45, що зумовлює приріст в II~$v$ на \sfrac{1}{5} \deq{} 9. Далі капіталізовані
$587 \text{ I} m$ поділяються на \sfrac{5}{6} і \sfrac{1}{6} на $489 с$ і $98 v$; ці 98 зумовлюють
в II нову додачу 98 до сталого капіталу, а це теж зумовлює
збільшення змінного капіталу II на \sfrac{1}{5} \deq{} 20. Ми маємо тоді:
\[
 \left.\begin{array}{@{}r@{~}c@{~\dplus{}~}c@{~\deq{}~}r@{~}r@{~}l@{}}
        \text{I.}&
            (5869 с \dplus{} 489 m) c
            &(1173 v \dplus{} 98 m) v
            &6358 с \dplus{} & 1271 v& \deq{} 7629\\
        \text{II.}&
            (1715 с \dplus{} 45 m \dplus{} 98 m) c
            &(342 v \dplus{} 9 m \dplus{} 20 m) v
            &1858 c \dplus{} & 371 v & \deq{} 2229\\
       \end{array}
 \right\}
 \text{\deq{} 9858.}
\]
Отже, при репродукції, що протягом трьох років зростала,
весь капітал підрозділу І зріс з 6000 до 7629, а ввесь капітал підрозділу
II з 1715 до 2229, сукупний суспільний капітал з 7715 до 9858.

\parcont{}  %% абзац починається на попередній сторінці
\index{iii1}{0407}  %% посилання на сторінку оригінального видання
фонди без покриття тільки в найрідкіших випадках (він був
банкіром моєї фірми в Манчестері), то так само ясно й те, що
його прекрасні описи тих мас капіталу, які великодушні банкіри
позичають фабрикантам, що потребують капіталу, є лукава
брехня.

Зрештою, в розділі XXXII Маркс каже по суті те саме:
„Попит на засоби платежу є просто попит на \emph{перетворення в
гроші}, оскільки купці й виробники можуть дати добре забезпечення; він є попит на \emph{грошовий капітал},
оскільки вони не
дають такого забезпечення, оскільки, отже, позика засобів платежу дає їм не тільки \emph{грошову форму}, а
\emph{еквівалент}, — в будь-якій формі, — якого їм не вистачає для платежу“. — Далі,
в розділі XXXIII: „При розвиненому кредиті, коли гроші концентруються в руках банків, банки,
\emph{принаймні номінально}, авансують гроші. Це авансування стосується тільки тих грошей\footnote*{
В першому німецькому виданні: „не стосується тих грошей"; виправлено
на підставі цитованого тексту і рукопису Маркса. \emph{Примітка ред. нім. вид. ІМЕЛ.}
}, що перебувають
в циркуляції. Це — авансування \emph{засобів
циркуляції}, а не авансування, капіталів, які циркулюють за допомогою цих засобів“. — Пан Chapmann,
який мусить це знати,
також підтверджує вищенаведене розуміння дисконтної справи:
„Bank Committee“ 1857: „У банкіра є вексель, банкір \emph{купив вексель}“. Evidence. Запитання 5139.

А втім, ми ще повернемось до цієї теми в розділі
XXVIII — \emph{Ф. Е.}]

„3744. Чи не будете ласкаві описати, що ви в дійсності розумієте під висловом капітал? — (Відповідь
Оверстона:) Капітал
складається з різних товарів, за допомогою яких підтримується
хід підприємства (capital consists of various commodities, by the
means of which trade is carried on); буває капітал основний і буває капітал обіговий. Ваші кораблі,
ваші доки, ваші верфі —
основний капітал; ваші харчові продукти, ваш одяг і т. д. — обіговий капітал“.

„3745. Чи має відплив золота за кордон шкідливі наслідки для
Англії? — Ні, якщо з цим словом зв’язується раціональне розуміння“. [Тепер з’являється стара теорія
грошей Рікардо]\dots{} „При
природному стані речей гроші всього світу розподіляються між
різними країнами світу в певних пропорціях; пропорції ці такого
роду, що при подібному розподілі“ [грошей] „зносини між якою-небудь країною, з одного боку, і всіма
іншими країнами світу, з другого боку, є простий обмін; але існують впливи, які час
від часу порушують цей розподіл, і якщо такі впливи виникають,
то частина грошей даної країни відпливає в інші країни“. — „3746.
Ви уживаєте тепер вислову: гроші. Якщо я раніше зрозумів вас
правильно, ви називали це втратою капіталу. — Що я називав
втратою капіталу?“ — „3747. Відплив золота. — Ні, я цього не
казав. Якщо ви золото вважаєте за капітал, тоді це, без сумніву,
\parbreak{}  %% абзац продовжується на наступній сторінці

\parcont{}  %% абзац починається на попередній сторінці
\index{ii}{0408}  %% посилання на сторінку оригінального видання
вживання в хліборобстві зерна власної продукції. При обміні між І і II
цю частину ІІ~$с$ теж не доводиться брати на увагу, як і І~$с$. Справа зовсім
не змінюється й тоді, коли частина продуктів II теж може ввійти в І
як засоби продукції. Їх покривається частиною засобів продукції, поданих
підрозділом І, і цю частину треба з самого початку виключити на обох
сторонах, коли ми хочемо дослідити в чистому, незатемненому вигляді
обмін між двома великими клясами суспільної продукції, між продуцентами
засобів продукції та продуцентами засобів споживання.

Отже, при капіталістичній продукції І~$(v \dplus{} m)$ не може дорівнювати
II~$с$, або обидва вони при обміні не можуть покривати один одного.
Навпаки, коли І~$ \sfrac{m}{x}$ є та частина І~$m$, що її як дохід витрачають капіталісти І,
то I~$(v \dplus{} \sfrac{m}{x})$ може бути рівне, більше або менше, ніж II~$c$; але I~$(v \dplus{} \sfrac{m}{x})$
завжди мусить бути менше, ніж II~$(с \dplus{} m)$, а саме — менше на ту частину
II~$m$, що її кляса капіталістів II сама мусить споживати за всяких обставин.
Треба відзначити, що при такому зображенні акумуляції не точно
зображено вартість сталого капіталу, оскільки він є частина вартости
товарового капіталу, що в його продукції цей сталий капітал бере участь.
Основна частина новоакумульованого сталого капіталу входить у товаровий
капітал лише поступінно й періодично, відповідно до різної природи
цих основних елементів; тому там, де сировинний матеріял, напівфабрикат
і~\abbr{т. ін.} масами входять у товарову продукцію, найбільша частина
цього товарового капіталу складається з заміщення обігової сталої
складової частини та змінного капіталу. (Однак так можна було робити,
зважаючи на оборот обігових складових частин; таким чином, припускається,
що обігова частина разом з переміщеною на неї частиною вартости основного
капіталу обертається протягом року так часто, що вся сума вартости
виготовлених товарів дорівнює вартості цілого капіталу, який входить у
річну продукцію). Але там, де при машинній продукції входить не сировинний
матеріял, а лише допоміжні матеріяли, — елемент праці, рівний $v$,
мусить знов з’явитися в товаровому капіталі як його найбільша складова
частина. Тимчасом як у нормі зиску додаткову вартість обчислюється
на ввесь капітал, незалежно від того, багато чи мало вартости періодично
передають продуктові основні складові частини, — до вартости кожного
періодично вироблюваного товарового капіталу основну частину сталого
капіталу треба прираховувати лише остільки, оскільки вона в наслідок
пересічного зношування передає вартість самому продуктові.

\subsection{Додаткові зауваження}

За первісне грошове джерело для II є $v \dplus{} m$ золотопромисловости І,
обмінюване на частину II~$с$; лише оскільки продуцент золота нагромаджує
додаткову вартість або перетворює її на засоби продукції І, отже,
поширює свою продукцію, його $v \dplus{} m$ не входить в II; з другого боку,
оскільки акумуляція грошей самим продуцентом золота, кінець-кінцем,
\parbreak{}  %% абзац продовжується на наступній сторінці

\parcont{}  %% абзац починається на попередній сторінці
\index{ii}{0409}  %% посилання на сторінку оригінального видання
призводить до поширеної репродукції, частина додаткової вартости золотопромисловости,
витрачувана не як дохід, а як додатковий змінний
капітал, входить в II, сприяє тут новому утворенню скарбів або дає нові
засоби купувати в І, безпосередньо знову не продаючи йому. З грошей,
які походять від цього І~$(v \dplus{} m)$ золотопромисловости, відходить частина
золота, яку деякі галузі продукції II потребують як сировинний матеріял
тощо, коротко кажучи, як елемент, що заміщує їхній сталий капітал.
Елемент для попереднього утворення скарбу — з метою майбутнього розширення
репродукції — при обміні між І і II є: для І тільки в тому разі,
коли частину І~$m$ продається підрозділові II однобічно, без зустрічної
купівлі, і служить вона тут для додаткового сталого капіталу II; для II
в тому разі, коли те саме маємо в І для додаткового змінного капіталу;
далі, в тому разі, коли частину додаткової вартости, витраченої підрозділом
І як дохід, не покривається за допомогою II~$с$, отже, на неї купується
частина II~$m$, яка в наслідок цього перетворюється на гроші. Коли
I~$(v \dplus{} \frac{m}{x})$ більше, ніж II~$с$, то для своєї простої репродукції II~$с$ не
доводиться заміщувати товарами з І те, що І взяв для споживання з II~$m$.
Постає питання, до якої міри може утворитись скарб в межах обміну
капіталістів II між собою, — обміну, що може бути лише взаємним
обміном II~$m$. Ми знаємо, що в межах II безпосередня акумуляція відбувається
тому, що частину II~$m$ безпосередньо перетворюється на змінний
капітал (цілком так само, як в І частину I~$m$ безпосередньо перетворюється
на сталий капітал). При різному часі пливу акумуляції в різних галузях
підприємств підрозділу II і в межах кожної поодинокої галузі для поодиноких
капіталістів, справа пояснюється, mutatis mutandis, цілком так само,
як і в підрозділі І.~Одні перебувають ще на стадії утворення скарбів,
продають, не купуючи, інші вже досягли пункту справжнього розширення
репродукції, купують, не продаючи. Звичайно, додатковий змінний грошовий
капітал спочатку витрачають на додаткову робочу силу; але робітники
купують засоби існування в тих власників додаткових засобів
споживання, — засобів, що входять в споживання робітників, — які ще
утворюють скарб. Від цих власників засобів споживання гроші, відповідно
до утворення скарбів у них, не повертаються до свого вихідного
пункту; вони нагромаджують їх.
\label{original-409}

  \input{i/part.4.5.tex}
  \addtocontents{toc}{\protect\newpage}
  \parcont{}  %% абзац починається на попередній сторінці
\index{iii1}{0103}  %% посилання на сторінку оригінального видання
й нервів і мозку. Справді, тільки шляхом найпотворнішого марнотратства індивідуального розвитку
забезпечується і здійснюється
розвиток людства взагалі в цю історичну епоху, яка безпосередньо передує свідомій перебудові
людського суспільства. Через те що вся економія, про яку тут іде мова, виникає з суспільного
характеру праці, то фактично саме цей безпосередньо
суспільний характер праці породжує це марнотратство життям і
здоров’ям робітників. Характерним у цьому відношенні є питання,
поставлене вже фабричним інспектором Б.~Бекером: „Все питання потребує серйозного обміркування того,
яким способом
можна найкраще відвернути це \emph{жертвування дитячим життям,
яке спричинює праця тісно скупченими масами} (\emph{congregational
labour})“ („Rep. of Insp. of Fact., Oct. 1863“, стор. 157).

\emph{Фабрики}. Сюди належить відсутність будь-яких охоронних
заходів для безпеки, комфорту й здоров’я робітників також і на
фабриках у власному значенні слова. Більша частина бюлетенів
убою, які перелічують ранених і вбитих промислової армії (дивись щорічні фабричні звіти), виходить
звідси. Так само недостача місця, повітря і~\abbr{т. д.}

Ще в жовтні 1855 року Леонард Горнер скаржився на опір
дуже значного числа фабрикантів вимогам закону про захисні
пристрої до горизонтальних валів, не зважаючи на те, що небезпека постійно доводиться нещасними,
часто смертельними
випадками, і що такі захисні пристрої і не дорогі і ніяк не заважають виробництву („Rep. of Insp. of
Fact., Oct. 1855“, стор. 6 [7]).
В цьому опорі цим та іншим постановам закону фабрикантів
відкрито підтримували неплатні мирові судді, які, самі здебільшого фабриканти або друзі'
фабрикантів, мали розв’язувати ці судові справи. Якого роду були вироки цих панів,
видно з слів вищого судді Кемпбеля з приводу одного з таких
вироків, на який йому подано було апеляційну скаргу: „Це —
не тлумачення парламентського акта, це — просто скасування
його“ (там же, стор. 11). В тому самому звіті Горнер оповідає, що на багатьох фабриках машини
пускають у рух без
попередження про це робітників. Через те що й на спиненій
машині завжди є що робити, при чому ця робота завжди виконується руками й пальцями, нещасні випадки
виникають просто
внаслідок неподачі сигналу (там же, стор. 44). Для опору
фабричному законодавству фабриканти утворили в ті часи
у Манчестері тред-юніон, так звану „National Association for the
Amendment of the Factory Laws“ („Національна асоціація для поліпшення фабричних законів“), який у
березні 1855~\abbr{р.} внесками
по 2\shil{ шилінги} від кінської сили зібрав суму понад \num{50000}\pound{ фунтів
стерлінгів}, щоб з неї оплачувати судові витрати своїх членів
по судових скаргах фабричних інспекторів і вести процеси від
імени асоціації. Завдання полягало в тому, щоб довести, що killing no murder [умертвіння не є
вбивство], коли це робиться
задля зиску. Шотландський фабричний інспектор, сер Джон
\parbreak{}  %% абзац продовжується на наступній сторінці


\index{ii}{0104}  %% посилання на сторінку оригінального видання
\chapter{Оборот капіталу}

\section{Час обороту й число оборотів}

\label{original-104}
Ми бачили: сукупний час циркуляції даного\footnote*{
Термін „сукупний час циркуляції“ тут Маркс вживає в тому самому розумінні,
в якому він далі в цьому ж розділі вживає термін „час обороту“, тимчасом
як взагалі він в цій книзі термін „час циркуляції“ вживає в тому самому
розумінні, що і „час обігу“, тобто в розумінні того часу, що протягом його капітал
перебуває в сфері циркуляції. (Дивись розділ V). \Red{Ред.}
} капіталу дорівнює сумі
часу його обігу та часу його продукції. Це є відтинок часу від моменту
авансування капітальної вартости в певній формі до моменту, коли капітальна
вартість, що процесує, повертається в тій самій формі.

Мета, що визначає капіталістичну продукцію, завжди є зростання
авансованої вартости, чи авансовано цю вартість в її самостійній формі,
тобто в грошовій формі, чи в формі товару, так що його форма вартости
має лише ідеальну самостійність у ціні авансованих товарів.
В обох випадках ця капітальна вартість перебігає протягом свого кругобігу
різні форми існування. Її тотожність з самою собою констатується
в книгах капіталіста або в формі рахункових грошей.

Хоч візьмемо ми форму $Г\dots{} Г'$, хоч форму $П\dots{} П$, обидві форми
значать: 1) що авансована вартість функціонувала як капітальна вартість
і зросла своєю вартістю; 2) що по закінченні процесу вона повернулась
до тієї форми, в якій почала його. Зростання авансованої вартости $Г$ і
разом з тим поворот капіталу до цієї форми (до грошової форми) виразно
помітно в $Г\dots{} Г'$. Але те саме відбувається і в другій формі. Бо
вихідний пункт для $П$ є наявність елементів продукції, товарів даної
вартости. Ця форма має в собі зростання цієї вартости ($Т'$ і $Г'$) і поворот
до первісної форми, бо в другому $П$ авансована вартість
знову має форму елементів продукції, що в ній її первісно авансовано.

Раніше ми бачили: „Якщо продукція має капіталістичну форму, то
і репродукція має ту саму форму. Як процес праці за капіталістичного
способу продукції є лише засіб для процесу зростання вартости, так
\parbreak{}  %% абзац продовжується на наступній сторінці

\parcont{}  %% абзац починається на попередній сторінці
\index{iii2}{0105}  %% посилання на сторінку оригінального видання
собі за передмову, з одного боку, визволення безпосередного продуцента з стану
простої приналежности до землі (в формі підвладного, кріпака, невільника і~\abbr{т. ін.})
а, з другого боку, експропріяцію землі в маси народу.

В цьому розумінні монополія на земельну власність є історична передумова й
лишається постійною основою капіталістичного способу продукції, як і всіх попередніх
способів продукції, що спираються на визиск мас в тій або іншій
формі. Але та форма земельної власности, що її знаходить капіталістичний
спосіб продукції на початку свого розвитку, не відповідає йому. Форму, що йому
відповідає, утворює лише він сам, підпорядковуючи хліборобство капіталові:
а тому й февдальна земельна власність, власність клану, або дрібна селянська
власність з громадою марки, хоч і які різні їхні юридичні форми,
перетворюються на економічну форму, що відповідає цьому способові продукції.
Одним з великих результатів капіталістичного способу продукції є те, що, з одного
боку, він перетворює хліборобство з простої емпіричної та механічної
традиційної методи найнерозвинутішої частини суспільства на свідомий науковий
ужиток аґрономії, оскільки це взагалі можливо серед умов, даних приватною
власністю\footnote{
Цілком консервативні аґрикультурні хеміки, як от, напр., В.~Johnston, визнають, що дійсно
раціональне хліборобство скрізь надибує непереможні межі в приватній власності. Те саме визнають
письменники, оборонці ex professe монополії приватної власности на землю, як от напр., пан Charles
Comte у двотомній праці, що має собі за спеціальну мету боронити приватну власність. «Народ» — каже
він — «не може досягнути того ступеня добробуту та сили, що визначається його природою, якщо
кожна частина тієї землі, що його годує, не одержить призначення, найбільш згідного з загальним
інтересом. Щоб значно розвинути свої багатства, мусила б по змозі єдина та передусім освічена воля
взяти до своїх рук розпорядок над кожним окремим кавалком своєї території та кожний кавалок зуживати
так, щоб тим допомагати поспіхові всіх інших. Але існування такої волі\dots{} не сила було б погодити
з поділом землі на приватні земельні ділянки\dots{} та з даною кожному власникові змогою майже абсолютно
порядкувати своїм майном». — Johnston, Comte і~\abbr{т. д.}, розглядаючи суперечність між власністю та
раціональною аґрономією, мають на оці тільки потребу обробляти землю певної країни як одну цілість.
Але залежність культури окремих продуктів землі від коливань ринкових цін, та невпинна зміна цієї
культури разом з тими коливаннями цін, увесь дух капіталістичної продукції, що простує до
безпосереднього
найближчого грошового зиску, — це все стоїть у суперечності до хліборобства, що йому
доводиться господарювати серед сукупних постійних життьових умов послідовних різних людських
ґенерацій.
Яскравий приклад цього є ліси, що ними іноді господарюють до певної міри в дусі громадських
інтересів
тільки там, де ті ліси не становлять приватної власности, а підлягають державному управлінню.
}; що, з одного боку, він цілком звільняє земельну власність від
відносин панування та підлеглости, а з другого боку, цілком відлучає землю
як умову праці від земельної власности та земельного власника, для якого та
земля не становить нічого більше, крім певного грошового податку, що його
він бере від промислового капіталіста фармера за посередництвом своєї монополії:
остільки рве цей зв’язок земельного власника з землею, що земельний власник
цілий свій вік може прожити в Костянтинополі, дарма що його земельна власність
буде в Шотландії. Оттак земельна власність одержує свою суто-економічну
форму, скидаючи з себе всі свої попередні політичні й соціяльні лямівки та
зв’язки, коротко — всі ті традиційні додатки, що їх, як ми пізніше побачимо, сами
капіталісти й їхні теоретичні проводирі в запалі своєї боротьби з земельною
власністю проголосили некорисною та безглуздою надмірністю. З одного боку,
раціоналізація хліборобства, що вперше дає змогу провадити його на суспільних
основах, з другого боку, доведення земельної власности до абсурду, — це великі
заслуги капіталістичного способу продукції. Як і всі свої інші історичні кроки
поступу, так само й ці, купив він передусім ціною повного зубожіння безпосередніх
продуцентів.

Раніше, ніж перейти до самої теми, треба зробити ще кілька попередніх
уваг, щоб уникнути непорозумінь.

Отже, передумова капіталістичного способу продукції така: дійсні хлібороби
— то наймані робітники, що мають працю від капіталіста, фармера, який
\parbreak{}  %% абзац продовжується на наступній сторінці

\parcont{}  %% абзац починається на попередній сторінці
\index{iii2}{0106}  %% посилання на сторінку оригінального видання
провадить сільське господарство, тільки як осібне поле експлуатації капіталу, як
приміщення свого капіталу в осібній сфері продукції. Цей фармер-капіталіст у
певні реченці, напр., щороку, платить земельному власникові, власникові визискуваної
ним землі, певну, контрактом усталену грошову суму (цілком так, як позикоємець
грошового капіталу платить певний процент) за дозвіл уживати свій капітал
на цьому осібному полі продукції. Ця грошова сума зветься земельною рентою,
все одно, чи платиться її від орної землі, будівельної ділянки, копалень, рибальства,
лісів і~\abbr{т. ін.} Її платиться протягом всього часу, що на нього земельний
власник за контрактом визичив, винайняв землю орендареві. Отже, земельна
рента становить тут ту економічну форму, що в ній земельна власність економічно
реалізується, даючи вартість. Далі, ми маємо тут усі три кляси — найманого
робітника, промислового капіталіста, земельного власника, — що всі разом
та одна проти однієї являють кістяк новітнього суспільства.

Капітал може бути зафіксований в землі, долучений до неї, почасти
більше тимчасово, як от при поліпшеннях хемічної натури, удобреннях і~\abbr{т. ін.},
почасти більше постійно, як от при дренажі, зрошувальних спорудах, нівелюваннях,
господарчих будівлях і~\abbr{т. ін.} В іншому місці я назвав капітал, що
отак долучається до землі, la terre-capital\footnote{
Misère de la Philosophie р. 165. Там я розрізняю terre-matière і terre-capital. «Досить лише
примістити до ділянок землі, вже перетворених на засоби продукції, нові суми капіталу, — і ми
збільшуємо la terre-capital, ані трохи не збільшуючи la terre-matière, тобто простору землі\dots{} La
terre-capital так само не є вічний, як і всякий інший капітал. t. La terre-capital e основний
капітал, але основний капітал зужитковується так само, яві оборотні капітали».
}. Він належить до категорії основного
капіталу. Процент за долучений до землі капітал та за поліпшення, що їх вона
тим способом одержує, як знаряддя продукції, може становити частину тієї
ренти, що її платить фармер земельному власникові\footnote{
Я кажу «може», бо в певних обставинах цей процент регулюється законом земельної ренти, а тому як
от при конкуренції нових земель, що мають велику природну родючість, може зникнути.
}, однак ця частина не
являє собою власне земельної ренти, що її платиться за користування землею
як такою, однаково, чи перебуває та земля в природному стані, — чи її культивується.
Коли б ми систематично — що до нашого плану не належить — розглядали
земельну власність, то треба було б докладно з’ясувати цю частину
доходу земельного власника. Тут досить буде сказати кілька слів про це.
Капіталовкладання більш тимчасового характеру, що їх викликають звичайні
процеси продукції в хліборобстві, всі без винятку переводить фармер. Ці вкладання,
як і простий обробіток землі взагалі, коли його проводять до певної
міри раціонально, отже коли він не сходить до брутального виснажування
ґрунту, як от, напр., в колишніх американських рабовласників, — проти чого
однак панове земельні власники забезпечують себе в контракті, — ці вкладання
поліпшують ґрунт\footnote{
Дивись James Anderson і Сагеу.
}, збільшують кількість продуктів землі та перетворюють
землю з простої матерії на землю — капітал. Оброблене поле більш варте, ніж
необроблене тієї самої природної якости. І основні капітали, що долучені до
землі на довший час, зужитковуються протягом довгого часу, витрачає, здебільшого,
фармер, а в деяких сферах іноді тільки сам фармер. Коли ж усталений в
договорі час оренди мине — і це одна з причин, чому з розвитком капіталістичного
способу продукції земельний власник силкується по змозі дужче скоротити
час оренди, — то пороблені в землі поліпшення як приналежність невідійманна
від субстанції, від землі, припадають як власність власникові тієї землі.
Роблячи новий орендний контракт, земельний власник додає до власне земельної
ренти процент на капітал, долучений до землі; однаково, чи винаймає він землю
тому фармерові, що ті поліпшення поробив, — чи якомусь іншому фармерові.
Таким чином його рента бубнявіє; або, коли він хоче продати землю — ми далі
\parbreak{}  %% абзац продовжується на наступній сторінці

\parcont{}  %% абзац починається на попередній сторінці
\index{iii2}{0107}  %% посилання на сторінку оригінального видання
побачимо, як визначається її ціну, — її вартість тепер уже стала вища. Він продає
не тільки землю, але поліпшену землю, долучений до землі капітал, що йому
нічого не коштував. Це одна з таємниць — цілком не вважаючи на рух власне
земельної ренти — чимраз більшого збагачування земельних власників, невпинного бубнявіння їхніх рент
та зросту грошової вартости їхніх земель з поступом
економічного розвитку. Так земельні власники ховають до своєї кешені цей
результат суспільного розвитку, що склався без їхньої допомоги, — fruges consumere nati\footnote*{
Спороджені, щоб споживати плоди. Прим. Ред.
}. Але
одночасно це є одна з найбільших перешкод для раціонального хліборобства, бо фармер уникає всяких
поліпшень та витрат, що їхнього
повного повороту не можна сподіватися протягом часу його оренди; і ми
бачимо, що цю обставину чим далі більш проголошують за таку перешкоду
так само в минулому віці James Anderson, що власне винайшов новітню теорію
ренти, та одночасно був практиком-фармером і видатним для свого часу агрономом, як в наші дні
противники сучасної побудови земельної власности в Англії.

A. A. Walton в «History of the Landed Tenures of Greath Britain and
Ireland» 1865, на стор. 96, 97 говорить про це так: «Всі намагання численних
сільсько-господарських установ нашої країни не в стані дати дуже значних або
дійсно помітних результатів щодо дійсного поступу поліпшеного обробітку землі,
поки такі поліпшення збільшують вартість земельної власности та висоту ренти
земельного власника до далеко вищого ступеня, ніж поліпшують стан фармера
або сільського робітника. Загалом кажучи, фармери точнісінько так само, як
і земельний власник, або його скарбник-управитель або навіть сам президент
сільсько-господарського товариства, знають, що добрий дренаж, добре угноєння та добре
господарювання, разом з збільшеним ужитком праці для
ґрунтовного очищення й оброблення землі, даватимуть дивовижні результати
як щодо поліпшення ґрунту, так і щодо піднесення продукції. Але все це
потребує значних витрат, а фармери так само добре знають, що хоч і як
вони поліпшуватимуть землю та підвищуватимуть її вартість, однаково головну користь від цього
пізніше пожнуть земельні власники в формі підвищеної ренти та збільшеної вартости землі\dots{} Вони
досить мудрі, щоб
примітити те, що ті промовці [землевласники та їхні управителі на сільськогосподарських
бенкетах] надиво завжди забувають їм сказати, а власне, що
левова пайка від усіх пороблених фармером поліпшень завжди мусить іти,
кінець-кінцем, до кешені земельного власника\dots{} Хоч і як попередній фармер
поліпшив орендовану землю, його наступник завжди бачитиме, що земельний
власник підвищить ренту відповідно до піднесеної попередніми поліпшеннями
вартости землі».

У власне хліборобстві цей процес виявляється ще не так ясно, як
от при використуванні землі для будівництва. Переважну частину землі, що її
продають в Англії для будівельних цілей, а не як freehold\footnote*{
Freehold, т. зв. білий маєток, тобто вільний від залежности. Прим. Ред.
} земельні власники
винаймають на 99 років або, коли можна, на коротший час. Коли мине цей
час, будівлі з самою землею дістаються земельному власникові. «Вони [орендарі]
зобов’язуються по закінченні контракту наймів — по тому, як вони аж до цього
моменту платили прибільшену земельну ренту — передати дім великому земельному
власникові в доброму для житла стані. Ледве закінчився цей контракт,
як от приходить аґент або інспектор того земельного власника, оглядає ваш
дім, дбаючи про те, щоб ви довели його до доброго стану, потім забирає його
у володіння землевласника, анексуючи його до земель останнього. Факт той, що,
коли ця система в своїй повній силі лишиться ще протягом довшого часу; то
вся домовласність в королівстві, тав само як і сільська землевластність, буде
\parbreak{}  %% абзац продовжується на наступній сторінці

\parcont{}  %% абзац починається на попередній сторінці
\index{iii1}{0108}  %% посилання на сторінку оригінального видання
лондонські норми смертності для осіб цього віку зовсім позбавлені значення як показники
антисанітарного стану промисловості (стор. 30).

Такий самий стан, як у кравців, маємо і в складачів, у яких
до відсутності вентиляції, до отруєного повітря і т. ін. долучається ще нічна праця. Їх звичайний
робочий час триває
12—13 годин, іноді 15—16. „Страшенний жар і задушливе повітря, як тільки запалюють газ\dots{} Нерідко
буває, що випари
з словолитні або сморід від машин чи стокових ям підіймаються
з нижчого поверху і ще більше погіршують антисанітарний стан
верхніх приміщень. Нагріте повітря нижчих приміщень нагріває
вищі вже самим тільки нагріванням їх підлоги, і коли при великому споживанні газу приміщення низькі,
то це — велике лихо.
Ще гірше стоїть справа там, де парові казани стоять у нижчому
поверсі і наповнюють весь будинок незносним жаром\dots{} Загалом можна сказати, що вентиляція абсолютно
незадовільна
і зовсім недостатня для того, щоб після заходу сонця усунути
жар та продукти згорання газу, і що в багатьох майстернях,
особливо там, де раніше були житлові приміщення, становище
надзвичайно сумне“. „У деяких майстернях, особливо в тих,
де друкуються тижневі видання і де зайняті також хлопці 12—16 років, працюють майже без перерв два
дні і одну ніч; а в інших складальних майстернях, в яких виконують „негайні“ роботи,
робітник не має відпочинку навіть у неділю, і його робочий
тиждень становить 7 днів замість 6“ (стор. 26, 28).

Про швачок і модисток (milliners and dressmakers) ми вже
казали в книзі І, розд. VIII, 3, стор. 263\footnote*{Стор. 181 рос. вид. 1935 р. \emph{Ред. укр. перекладу.}}, коли мова йшла про
надмірну працю. Їх робочі приміщення у нашому звіті описані
доктором Ордом. Навіть якщо вдень вони кращі, то в години,
коли горить газ, в них надзвичайний жар, повітря зіпсоване (foul)
і нездорове. В 34 кращих майстернях доктор Орд знайшов, що
пересічна кількість кубічних футів повітря на кожну робітницю була: „В 4 випадках більша, ніж 500; в
4 інших випадках —
400—500; в 5 — від 200 до 250; в 4 — від 150 до 200; і, нарешті,
в 9 — всього 100—150. Навіть у найбільш сприятливому з цих
випадків повітря ледве-ледве вистачає для довгої праці, якщо
приміщення недостатньо провітрюється\dots{} Навіть при добрій вентиляції увечері в майстернях стає дуже
жарко й душно через
те, що в них потрібно багато газових ріжків“. А ось зауваження
доктора Орда про одну з відвіданих ним майстерень нижчої
категорії, де робота провадиться коштом посередника (middleman):
„Кімната має 1280 кубічних футів; в ній знаходяться
14 осіб; на кожну з них припадає 91,5 кубічних футів. Робітниці мали тут спрацьований і виснажений
вигляд. Їх заробіток
визначається в 7—15 шилінгів на тиждень, крім того чай\dots{} Робочі
години — від 8 до 8. Маленька кімната, в якій скупчені ці 14 осіб,
\parbreak{}  %% абзац продовжується на наступній сторінці

\parcont{}  %% абзац починається на попередній сторінці
\index{iii1}{0109}  %% посилання на сторінку оригінального видання
провітрюється погано. В ній є два вікна, що відчиняються, і камін,
але забитий; будь-яких спеціальних вентиляційних пристроїв немає“ (стор. 27).

Той самий звіт зауважує щодо надмірної праці модисток:
„Надмірна праця молодих жінок панує у фешенебельних модних майстернях тільки протягом 4 приблизно
місяців на рік,
але в такій потворній мірі, що це в багатьох випадках викликало хвилинне здивування й незадоволення
публіки; протягом
цих місяців у майстерні звичайно працюють повних 14 годин
щодня, а при скупченні спішних замовлень 17--18 годин на день.
В інші пори року в майстерні працюють, мабуть, 10--14 годин;
ті, що працюють дома, працюють реґулярно 12 або 13 годин.
У виробництві дамських мантильок, комірців, сорочок і~\abbr{т. д.}
число годин праці в спільній майстерні, включаючи й працю на
швацькій машині, менше і не перевищує здебільшого 10--12“;
але, каже доктор Орд, „в деяких майстернях реґулярний робочий час у певні періоди здовжується окремо
оплачуваними надурочними годинами, а в інших майстернях беруть роботу додому, щоб закінчити її після
звичайного робочого часу: і та
і друга форма надмірної праці, можемо додати, часто є примусова“ (стор. 28). Джон Сімон зауважує в
примітці до цієї сторінки: „Пан Редкліф, секретар епідеміологічного товариства,
який мав особливо багато нагод досліджувати здоров’я модисток
у майстернях першого типу, знайшов на кожних 20 дівчат, які
самі вважали себе „цілком здоровими“, тільки одну здорову;
у решти виявились різні ступені фізичної перевтоми, нервового
виснаження і численних викликаних цим функціональних розладів“.
За причини цього він вважає: насамперед довжину робочого
дня, яку він визначає мінімум у 12 годин на день навіть для
тихого сезону; подруге, „переповнення і погане провітрювання
майстерень, зіпсоване газовими ріжками повітря, недостатнє
або погане харчування і недостатнє піклування про домашній
комфорт“.

Висновок, до якого приходить голова англійського санітарного відомства, такий: „Для робітників
практично неможливо
настояти на тому, що теоретично є їх найелементарнішим правом
на здоров’я, а саме настояти, щоб підприємець, який збирає
їх для виконання будь-яких робіт, своїм власним коштом усував,
оскільки це від нього залежить, всі не необхідні в цій спільній
роботі умови, які шкідливо впливають на здоров’я; і що в той
час, як самі робітники фактично неспроможні добитись для себе
цієї санітарної справедливості, вони так само мало можуть —
не зважаючи на гаданий намір законодавця — сподіватись будь-якої ефективної допомоги від тих
урядовців, які повинні проводити в життя Nuisances Removal Acts [закони для усунення антисанітарного
стану]“ (стор. 29). — „Без сумніву, визначення точних меж, поза якими підприємці мусять підлягати
реґулюванню, становитиме деякі дрібні технічні труднощі. Але\dots{} в принципі вимога
\parbreak{}  %% абзац продовжується на наступній сторінці

\input{iii.2/_0110.tex}
\parcont{}  %% абзац починається на попередній сторінці
\index{iii2}{0111}  %% посилання на сторінку оригінального видання
принаймні, своїм власним капіталом. Таке постійне пограбування становить об’єкт
боротьби за ірляндське земельне законодавство, яке пропонують звести до того,
щоб примусити земельного власника, який відмовляє орендареві, винагородити
його за зроблені ним поліпшення ґрунту або за долучений до землі капітал.
Пальмерстон звичайно давав на це цинічну відповідь: «Палата громад — палата
земельних власників».

Ми не говоримо також про ті виключні відносини, коли навіть в країнах
капіталістичної продукції земельний власник може вичавлювати високу оредну
плату, що аніяк не відповідає продуктові землі, як наприклад здача в оренду
в англійських промислових округах дрібних клаптиків землі фабричним робітникам
чи то під малесенькі садочки чи то для аматорського обробітку землі на
дозвіллі (Reports of Inspectors of Factories).

Ми говоримо про хліборобську ренту в країнах розвиненої капіталістичної
продукції. Наприклад, серед англійських орендарів є певна кількість дрібних
капіталістів, які вихованням, освітою, традиціями, конкуренцією та іншими обставинами
призначені й примушені до того, щоб вкладати свій капітал у
хліборобство, як орендарі. Вони примушені задовольнятися зиском меншим,
ніж пересічний і віддавати частину його у формі ренти землевласникові. Це —
однісінька умова, за якої їм тільки й може бути дозволено вкладати свій
капітал у ґрунт, у хліборобство. А що земельні власники всюди мають значний,
в Англії навіть переважний, вплив на законодавство, то й можуть вони використати
цей вплив для того, щоб ошукувати цілу клясу орендарів. Наприклад,
хлібні закони 1816 року — податок на хліб, як відомо, накладений на
країну з тією метою, щоб забезпечити для неробів землевласників дальше існування
збільшених рент, що надзвичайно зросли підчас анти-якобінської війни —
за винятком окремих виключно урожайних років, впливали правда так, що тримали
ціну сільсько-господарських продуктів вище від того рівня, до якого вони
упали б при вільному довозі хліба. Проте, вони не мали таких наслідків, щоб
утримати хлібні ціни на такій висоті, яку землевласники-законодавці декретували,
як свого роду нормальну ціну, так щоб вони становили законну межу довозу
закордонного збіжжя, але орендні договори складалося під вражінням цих
нормальних цін. Коли ілюзії зникли, складено новий закон з новими нормальними
цінами, які, проте, були так само простим безсилим виразом загребущої
землевласницької фантазії, як і старі. Таким способом орендарів ошукували
від 1815 до 30-х р. р. Звідси agricultural distress\footnote*{
Пригнічений стан хліборобства Пр.~Ред.
} як постійна тема протягом
усього цього часу. Звідси — протягом цього періоду експропріяція і руйнування
цілого покоління орендарів, заміщення їх новою клясою капіталістів\footnote{
Див. Anti-Corn-Law Prize-Essays. Тимчасом хлібні закони все-таки тримали ціни на штучно
підвищеному рівні. Це сприяло кращим орендарям. Вони вигравали від інертности, до якої охоронні
мита схиляли переважну масу орендарів, що покладались — підставно чи ні, — на виключну
пересічну ціну.
}.

Але далеко загальніший і важливий факт являє собою пониження заробітної
плати власне хліборобських робітників нижче за її нормальний пересічний
рівень, так що частина заробітної плати віднімається у робітника, становить
собою складову частину орендної плати і таким чином під маскою земельної ренти
дістається землевласникові замість робітника. Наприклад, в Англії і Шотландії,
за винятком небагатьох графств, що перебувають у сприятливім становищі,
це — загальне явище. Праці уряджених перед запровадженням хлібних законів
в Англії парламентських слідчих комісій про висоту заробітної плати — до цього
часу найцінніші і майже цілком невикористані матеріяли з історії заробітної
плати XIX століття і одночасно ганебний пам’ятник, поставлений англійською
\parbreak{}  %% абзац продовжується на наступній сторінці

\parcont{}  %% абзац починається на попередній сторінці
\index{iii1}{0112}  %% посилання на сторінку оригінального видання
Досі на охоронний клапан навішували такий тягар, що він відкривався вже при тисненні пари в 4, 6 або
8 фунтів на квадратний
дюйм; тепер виявили, що підвищенням тиснення до 14 або 20 фунтів\dots{} можна досягти дуже значного
заощадження вугілля; інакше
кажучи, фабрика почала працювати при значно меншому споживанні вугілля\dots{} Ті, що мали для цього
засоби й сміливість, стали
застосовувати систему збільшеного тиснення і розширення в повному її обсягу і застосовували
відповідно до цього збудовані парові казани, які давали пару тисненням в 30, 40, 60 і 70 фунтів на
квадратний дюйм — тиснення, при якому інженер старої школи
від страху зомлів би. Але через те що економічний результат
цього підвищеного тиснення пари\dots{} виявився дуже швидко в цілком недвозначній формі фунтів, шилінгів
і пенсів, парові казани
високого тиснення при конденсаційних машинах стали майже загальним явищем. Ті, що провели реформу
радикально, стали застосовувати вульфові машини, і це мало місце щодо більшості недавно
збудованих машин; вони стали застосовувати особливо вульфові
машини з 2 циліндрами, в одному з яких пара з казана розвиває силу в наслідок перевищення тиснення
над тисненням атмосфери і потім, замість того щоб після кожного підіймання поршня
виходити у повітря, як це було раніш, входить у циліндр
низького тиснення, приблизно вчетверо більший обсягом, і, розширившись там далі, відводиться в
конденсатор. Економічний
результат, одержуваний при таких машинах, полягає в тому, що
одна кінська сила за одну годину добувається при споживанні
3\sfrac{1}{2}—4 фунтів вугілля; тимчасом як при машинах старої системи
для цього потрібно було від 12 до 14 фунтів. За допомогою
майстерного пристрою вульфову систему подвійного циліндра
або комбінованої машини високого й низького тиснення удалось
пристосувати до наявних уже старих машин і таким чином підвищити їх ефективність при одночасному
зменшенні споживання
вугілля. Того самого результату досягнуто протягом останніх
8--10 років за допомогою сполучення машини високого тиснення
з конденсаційною машиною таким чином, що спожита пара першої переходила в другу і пускала її в рух.
Така система корисна
в багатьох випадках“.

„Не легко було б точно встановити, наскільки збільшилась
ефективність праці тих самих колишніх парових машин, до
яких пристосовані деякі або й усі ці нові поліпшення. Але я певен, що на ту саму вагу парової машини
ми одержуємо тепер
пересічно принаймні на 50\% більше корисної роботи і що в багатьох випадках та сама парова машина,
яка в часи обмеженої
швидкості в 220 футів на хвилину давала 50 кінських сил, дає
тепер понад 100. Надзвичайно ефективні щодо економії результати
застосування пари високого тиснення при конденсаційних машинах, так само як і далеко більші вимоги,
які ставляться до старих парових машин з метою розширення підприємств, привели
за останні три роки до введення трубчастих казанів, в наслідок
\parbreak{}  %% абзац продовжується на наступній сторінці

\parcont{}  %% абзац починається на попередній сторінці
\index{iii2}{0113}  %% посилання на сторінку оригінального видання
скаржитись, що вони не зможуть платити таких високих рент, як звичайно
платили, бо в наслідок еміґрації праця дорожчає». Отже, тут висока земельна
рента прямо ототожнюється з низькою заробітною платою. І оскільки висота
земельної ціни зумовлюється цією обставиною, яка підвищує ренту, остільки підвищення
вартости землі тотожне із знеціненням праці, високий рівень земельної
ціни — з низьким рівнем ціни праці.

Те саме і у Франції. «Орендна плата підвищується, бо на однім боці підвищується
ціна хліба, вина, м’яса, городини і овочів, а на другім боці ціна
праці лишається незмінна. Коли б старі люди порівняли рахунки їхніх батьків, —
що відсунуло б нас назад майже на 100 років, — вони побачили б, що тоді ціна
робочого дня у сільській Франції була достоту така, як і тепер. Ціна м’яса від того
часу збільшилась утроє\dots{} Хто жертва цього перевороту? Чи багатий, що є власник
здаваної в оренду землі, чи бідняк, що її обробляє\elli{?..} Зріст орендних
цін є доказ суспільного лиха». (Du Mécanisme de la Société en France et en
Angleterre. Par 1. Rubichon, 2-me édit. Paris 1837, p. 101).

Приклади ренти, як наслідку вирахування, з одного боку, з пересічного
зиску, з другого — з пересічної заробітної плати:

Цитований вище Мортон, сільський аґент і сільсько-господарський інженер,
каже, що в багатьох місцевостях зроблено спостереження, що рента
за великі оренди нижча, ніж за дрібні, бо «конкуренція за останні, звичайно,
більша, ніж за перші, і тому що дрібні орендарі, які рідко мають можливість
узятись до якогось іншого діла, крім хліборобства, вимушені пекучою потребою
знайти підхоже діло, часто погоджуються платити таку ренту, про яку вони
сами знають, що вона надто висока». (John C.~Morton, The Resources of Estates.
London 1885. p. 116).

Проте, на його думку, в Англії ця ріжниця поступово згладжується, чому,
як він вважає, дуже сприяє еміґрація кляси саме дрібних орендарів Той самий
Мортон наводить приклад, коли в земельну ренту безперечно входить вирахування
з заробітної плати самого орендаря, а тому ще безперечніше і з заробітної
плати робітників, які в нього працюють. А саме, коли орендні дільниці
менші, ніж 70--80 акрів (30--34 гектари), при яких неможливо держати парокінний
плуг. «Коли орендар не працює своїми власними руками так само
дбало, як будь-який робітник, він не може існувати від своєї оренди. Коли він
виконання роботи покладе на своїх людей, а сам обмежиться виключно наглядом
за ними, то він, найімовірніше, дуже скоро виявить, що не зможе виплачувати
орендної плати» (1. c., р 118). З цього Мортон висновує, що коли орендарі
в краю не дуже бідні, то розміри віддаваних на оренду дільниць не повинні
бути менші за 70 акрів, щоб орендар міг держати двох або трьох коней.

Надзвичайна мудрість пана Léonce de Lavergne, Membre de l’Institut et de
la Société Centrale d’Agriculture. У своїй Economie Rurale de l’Angletterre (цитовано
з англійського перекладу, London 1855), він робить таке порівняння
річних вигід від рогатої худоби, яку у Франції вживається для роботи, а в Англії
не вживається, бо її заміняють коні (р. 42):
\begin{center}
\begin{tabular} {l c c}
    & Франція, міл.\pound{ ф. ст.} & Англія, міл.\pound{ ф. ст.} \\
  молоко & \phantom{0}4 & 16\\
  м'ясо & 16 & 20 \\
  робота & \phantom{0}8 & \textemdash \\
  \midrule
    & 28 & 36
\end{tabular}
\end{center}
\noindent{}Але вищий продукт для Англії тут виходить лише тому, що згідно з його
власними даними молоко в Англії коштує удвоє дорожче, ніж у Франції, тимчасом
як для м’яса він припускає однакові ціни в обох країнах (р. 35); отже, молочний
\parbreak{}  %% абзац продовжується на наступній сторінці

\parcont{}  %% абзац починається на попередній сторінці
\index{iii1}{0114}  %% посилання на сторінку оригінального видання
обміну речовин у людини, почасти ту форму, в якій предмети
споживання лишаються після споживання їх. Отже, покидьки
виробництва в хемічній промисловості є побічні продукти, які
при незначному масштабі виробництва пропадають марно; залізні
стружки, які лишаються при фабрикації машин і знову входять
як сировинний матеріал у виробництво заліза і т. д. Екскременти споживання — це виділювані людиною
природні речовини,
рештки одягу у формі ганчірок і т. д. Екскременти споживання
мають найбільше значення для землеробства. Щодо застосування
їх, капіталістичне господарство відзначається колосальним марнотратством; у Лондоні, наприклад, воно
не знаходить кращого
застосування для екскрементів 4\sfrac{1}{2} мільйонів людей, як з величезними витратами заражати ними Темзу.

Подорожчання сировинних матеріалів є, звичайно, спонукою
до використовування відпадів.

Загалом умовами цього повторного використання є: масовість
цих екскрементів, яка можлива тільки при роботах у великому
масштабі; поліпшення машин, завдяки чому речовини, які раніш
у своїй даній формі були непридатні до вжитку, переходять
у таку форму, в якій вони можуть бути використані в новому
виробництві; прогрес науки, особливо хемії, яка відкриває корисні властивості таких відпадів.
Правда, і в дрібному землеробстві, де поля обробляються як сади, як от у Ломбардії, південному Китаї
та Японії, також має місце значна економія
цього роду. Але, загалом кажучи, при цій системі продуктивність землеробства купується великим
марнотратством людської робочої сили, відтягуваної від інших сфер виробництва.

Так звані відпади відіграють значну роль майже в кожній
галузі промисловості. Так, наприклад, у грудневому фабричному звіті за 1863 рік [стор. 139]
наводиться як одна з головних
причин того, чому в Англії — як і в багатьох частинах Ірландії —
орендарі тільки неохоче й рідко сіють льон, ось що: „Значна
кількість відпадів\dots{} які відходять при обробітку льону в дрібних льонотіпальних фабриках, де
рушійною силою є вода
(scutch mills)\dots{} Відпадів від бавовни порівняно небагато, а при
обробленні льону їх дуже багато. Старанна робота при мочінні і механічному тіпанні льону може значно
обмежити цю
втрату\dots{} В Ірландії льон часто тіпають надзвичайно незадовільним способом, так що 28—30\% його
пропадало марно“; усе це
могло б бути усунене при застосуванні кращих машин. Костриця при цьому відпадає в такій великій
кількості, що фабричний інспектор каже: „З деяких тіпальних фабрик в Ірландії
мене повідомили, що тіпальники часто вживають у себе дома
відпади, які утворюються на цих фабриках, як паливний матеріал для своїх печей, а це ж дуже цінний
матеріал“ („Rep. of
Insp. of Fact. Oct., 1863“, стор. 140). Про відпади бавовни мова
буде далі, там, де ми розглядаємо коливання цін на сировинний
матеріал.


\index{iii1}{0115}  %% посилання на сторінку оригінального видання
Шерстяна промисловість була розсудливіша, ніж льонообробна промисловість. „Раніше, звичайно,
вважалося ганебним
збирати відпади шерсті та шерстяні ганчірки для повторного
перероблення, але цей передсуд цілком зник у shoddy trade
(виробництві штучної шерсті), яке стало важливою галуззю
шерстяної промисловості Йоркшірської округи, і немає сумніву,
що й підприємства, що переробляють відпади бавовни, незабаром теж займуть те саме місце, як галузь
промисловості, що
задовольняє визнані потреби. 30 років тому шерстяні ганчірки,
тобто шматки тканини з чистої вовни і~\abbr{т. д.}, коштували пересічно щось 4\pound{ фунти стерлінгів} 4\shil{ шилінги}
за тонну; протягом
останніх кількох років вони стали коштувати 44\pound{ фунти стерлінгів} за тонну. А попит на них так
збільшився, що використовується
навіть мішана тканина з вовни й бавовни, бо знайдено засіб
руйнувати бавовну без пошкодження вовни; і тепер тисячі робітників зайняті у фабрикації shoddy, а
споживач має з того
велику користь, оскільки він тепер може купити сукно доброї
середньої якості за дуже помірну ціну“ („Rep. of Insp. of Fact.,
Oct. 1863“, стор. 107). Поновлювана таким чином штучна шерсть
уже в кінці 1862 року становила третину всього споживання
вовни англійською промисловістю („Rep. of Insp. of Fact., Oct.
1862“, стор. 81). „Велика користь“ для „споживача“ полягає в тому,
що його шерстяний одяг зношується втричі швидше, ніж раніше,
і вшестеро швидше витирається до ниток.

Англійська шовкова промисловість посувалась по тій самій
похилій площині. З 1839 до 1862 року споживання натурального
шовку-сирця трохи зменшилося, тим часом як споживання шовкових відпадів подвоїлося. Поліпшені машини
дали змогу фабрикувати з цього, за інших умов майже нічого не вартого матеріалу, шовк, придатний для
багатьох цілей.

Найразючіший приклад застосування відпадів дає хімічна
промисловість. Вона споживає не тільки свої власні відпади,
знаходячи для них нове застосування, але й відпади найрізнорідніших інших галузей промисловості, і
перетворює, наприклад,
майже некорисний раніше кам’яновугільний дьоготь в анілінові
фарби, в красильну речовину крапу (алізарин), а останнім часом
також у медикаменти.

Від цієї економії на відпадах виробництва внаслідок повторного використання їм треба відрізняти
економію при утворенні
самих відпадів, тобто зведення екскрементів виробництва до
їх мінімуму і безпосереднє максимальне використання всіх сировинних та допоміжних матеріалів, що
входять у виробництво.

Заощадження на відпадах почасти зумовлене якістю застосовуваних машин. Мастило, мило тощо
заощаджуються тим
більше, чим точніше працюють окремі частини машин, і чим краще
вони відполіровані. Це стосується до допоміжних матеріалів.
А почасти, і це найважливіше, від якості застосовуваних машин
і знарядь залежить те, більша чи менша частина сировинного
\parbreak{}  %% абзац продовжується на наступній сторінці

\parcont{}  %% абзац починається на попередній сторінці
\index{iii2}{0116}  %% посилання на сторінку оригінального видання
що їх виплачується під титулом земельної ренти власникові землі за використання
ґрунту чи то з метою виробництва чи то споживання, треба пам’ятати, що
ціна речей, які самі по собі не мають вартости, тобто не є продукти праці, як
земля, або, принаймні, не можуть бути відтворені працею, як старовинні речі,
художні вироби певних майстрів тощо, може визначатися дуже випадковими
комбінаціями.

Щоб продати річ, для цього не треба нічого іншого, як тільки того, щоб
вона могла зробитись об’єктом монополії і відчуження.

\pfbreak{} % see russ. book

Є три головні помилки, що їх при розгляді земельної ренти треба уникати
й що затемнюють аналізу.

1) Сплутування різних форм ренти, відповідних різним ступеням розвитку
суспільного процесу продукції.

Хоч би яка була специфічна форма ренти, всім її типам є спільне те, що привласнення
ренти є економічна форма, що в ній реалізується земельна власність і що
земельна рента в свою чергу має за свою передумову земельну власність, власність
певних індивідуумів на певні дільниці землі, чи власником буде особа, що репрезентує
громаду як в Азії, Єгипті тощо, чи ця земельна власність буде привхідною
обставиною власности певних осіб на особи безпосередніх продуцентів, як за системи
рабства або кріпацтва, чи ж земельна власність буде суто приватною
власністю непродуцентів на природу, простим титулом власности на землю, чи,
нарешті, це буде таке відношення до землі, що як от у колоністів і дрібноселянських
землевласників, за ізольованої і соціально-нерозвиненої праці, виступає,
як відношення безпосередньо дане привласненням і виробництвом продуктів
на певних дільницях землі безпосередніми продуцентами.

Ця \emph{спільність} різних форм ренти — те, що вона являє собою економічну
реалізацію земельної власности, юридичної фікції, в силу якої ріжним індивідуумам
належить виключне володіння певними дільницями землі, — призводить до
того, що ріжниці форм не помічаються.

2) Всяка земельна рента є додаткова вартість, продукт додаткової праці.
У своїй нерозвиненій формі, у формі натуральної ренти, вона ще є безпосередньо
додатковий продукт. Звідси та помилка, ніби та рента, що відповідає капіталістичному
способові продукції і яка завжди становить надлишок над зиском,
тобто над тією частиною вартости товару, що сама складається з додаткової вартости
(додаткової праці), — ніби ця особлива і специфічна складова частина додаткової
вартости буде пояснена тим, що будуть пояснені загальні умови існування додаткової
вартости і зиску взагалі. Ці умови такі: безпосередні продуценти мусять працювати
понад той час, який потрібен для репродукції їхньої власної робочої
сили, для репродукції їх самих. Вони взагалі мусять виконувати додаткову працю.
Це — суб’єктивна умова. А об’єктивна є в тому, щоб у них була і \emph{можливість}
виконувати додаткову працю; щоб природні умови були такі, щоб лише деякої
\emph{частини} робочого часу, яким вони порядкують, було досить для їхньої репродукції
і самозбереження як продуцентів; щоб продукція потрібних засобів їхнього
існування не забирала всієї їхньої робочої сили. Родючість природи становить
тут одну межу, один вихідний пункт, одну основу. З другого боку, розвиток суспільної
продуктивної сили праці становить тут другу межу. Розглядаючи справу
ще ближче, можна сказати: тому що продукція харчових засобів є найперша умова
життя продуцентів і всякої продукції взагалі, — праця застосована до цієї продукції,
отже, хліборобська праця в найширшому економічному розумінні мусить бути
остільки продуктивна, щоб продукцією харчових засобів для безпосередніх продуцентів
забирався не ввесь робочий час, що вони ним порядкують, отже, щоб була
можлива хліборобська додаткова праця, а тому і хліборобський додатковий продукт.
Розвиваючи далі: треба, щоб уся хліборобська праця — потрібна й додаткова
праця — деякої частини суспільства була достатня для того, щоб продукувати
\parbreak{}  %% абзац продовжується на наступній сторінці

\parcont{}  %% абзац починається на попередній сторінці
\index{iii2}{0117}  %% посилання на сторінку оригінального видання
потрібні харчові засоби для всього суспільства, тобто і для нехліборобських робітників;
отже, щоб був можливий цей великий поділ праці між хліборобами
і промисловцями, а також між тими з хліборобів, що продукують харч, і тими,
що продукують сирові матеріяли. Хоч праця безпосередніх продуцентів харчу
щодо них самих поділяється на потрібну і додаткову працю, проте, щодо
суспільства вона становить лише потрібну працю, яка потрібна для продукції
харчових засобів. А втім, це саме має силу і щодо всякого поділу праці
всередині всього суспільства, на відміну від поділу праці всередині окремої майстерні.
Це — праця, що потрібна для продукування особливих речей, для задоволення
особливої потреби суспільства в особливих речах. Коли цей поділ праці
пропорційний, то продукти різних груп продаються по їхніх вартостях (при
дальшому розгляді по цінах їхньої продукції) або ж по цінах, які, визначувані
загальними законами, становлять модифікації цих вартостей, — у відповідних
випадках цін продукції. Це справді, є закон вартости, як він виявляється не
у відношенні до окремих товарів, або речей, а у відношенні до кожноразового
сукупного продукту окремих суспільних сфер продукції, які в наслідок поділу
праці стали самостійними; отже, не тільки так, що на кожен окремий товар вжито
лише потрібний робочий час, але й так, що з усього суспільного робочого часу
на різні групи вжито лише конечну пропорційну кількість. Бо умовою лишається
споживна вартість. Але коли споживна вартість окремого товару залежить
від того, що він сам по собі задовольняє будь-яку потребу, то споживна вартість
суспільної маси продуктів залежить від того, що ця маса адекватна кількісна
певній суспільній потребі в продукті кожного осібного роду, а тому і від того,
що працю розподілено пропорційно між різними сферами продукції відповідно
до цих суспільних потреб, що кількісно визначені. (На цей пункт звернути
увагу в зв’язку з розподілом капіталу між різними сферами виробництва). Суспільна
потреба, тобто споживна вартість у суспільній потенції, вступає тут визначально
для кількостей всього суспільного робочого часу, що припадають на
різні окремі сфери продукції. Але це — лише той самий закон, який виявляється
уже у відношенні до окремого товару, а саме, що споживна вартість товару
є передумова його мінової вартости, а тому і його вартости. Цей пункт має
лише ту дотичність до відношення між потрібного і додатковою працею, що при
порушенні цієї пропорції не може бути реалізована вартість товару, а тому
й додаткова вартість, що міститься в ній. Хай, наприклад, бавовняних тканин
випродуковано пропорційно забагато, хоч у всьому цьому продукті, в цих тканинах
реалізовано лише потрібний для цього в даних умовах робочий час. Але взагалі
на цю осібну галузь витрачено надто багато суспільної праці; тобто частина
продукту некорисна. Тому ввесь продукт продається лише так, як коли б він
був випродукований в потрібній пропорції. Ця кількісна межа тих кількостей
суспільного робочого часу, які можна витратити на різні осібні сфери продукції,
є лише далі розвинений вираз закону вартости взагалі; хоч потрібний робочий
час набуває тут іншого сенсу. Для задоволення певної суспільної потреби треба
стільки от робочого часу. Обмеження тут настає через споживну вартість. Суспільство
в даних умовах продукції на такий от продукт певного роду може
витратити лише стільки й стільки з усього свого робочого часу. Але суб’єктивні
й об'єктивні умови додаткової праці і додаткової вартости взагалі не мають
ніякого чинення до певної форми так зиску, як і ренти. Вони мають силу для
додаткової вартости як такої, хоч би яких особливих форм вона набувала. Тому
вони не пояснюють земельної ренти.

3)~Якраз при економічній реалізації земельної власности, в розвитку земельної
ренти, виявляється дуже своєрідним те, що її величина визначається
зовсім не за допомогою її одержувача, а розвитком суспільної праці, що є незалежний
від його допомоги, і в якому він зовсім не бере участи. Тому легко вважати за
\parbreak{}  %% абзац продовжується на наступній сторінці

\parcont{}  %% абзац починається на попередній сторінці
\index{iii1}{0118}  %% посилання на сторінку оригінального видання
заробітної плати на норму зиску; підсумок виходить тоді сам
собою.

Але тут, як і в попередньому випадку, треба, загалом кажучи, зауважити таке: якщо відбуваються
зміни, чи то в наслідок економії на сталому капіталі, чи в наслідок коливань цін сировинного
матеріалу, то вони завжди впливають на норму
зиску, навіть і тоді, коли вони зовсім не зачіпають заробітної плати, отже, і норми і маси
додаткової вартості. У формулі $m' \frac{v}{K}$ вони змінюють величину $К$, а тим самим і значення цілого дробу.
Отже, і тут цілком байдуже — в відміну від того, що виявилось при розгляді додаткової вартості — в
яких сферах виробництва відбуваються ці зміни; чи виробляють зачеплені цими змінами галузі
промисловості засоби існування для робітників,
відповідно сталий капітал для виробництва таких засобів існування, чи ні. Розвинене тут в такій
самій мірі стосується й до
тих випадків, коли зміни відбуваються у виробництві предметів
розкоші, а під виробництвом предметів розкоші тут слід розуміти всяке виробництво, яке не є потрібне
для репродукції робочої сили.

Під сировинним матеріалом ми розумітимемо тут і допоміжні
матеріали, як от індиго, вугілля, газ і ін. Далі, оскільки в цій
рубриці розглядаються машини, їх власний сировинний матеріал
складається з заліза, дерева, шкіри та ін. Тому на їх власну
ціну впливають коливання цін сировинного матеріалу, який входить у їх конструкцію. Оскільки їх ціна
підвищується в наслідок коливання цін, — чи сировинного матеріалу, з якого вони складаються, чи то
допоміжних матеріалів, споживаних під час
їх функціонування, — знижується pro tanto [відповідно до цього]
норма зиску. У зворотному випадку — навпаки.

В дальшому дослідженні ми обмежимось коливаннями цін сировинного матеріалу, але не того матеріалу,
що входить у процес виробництва як сировинний матеріал тих машин, що функціонують як засоби праці,
або як допоміжний матеріал при застосуванні машин, а того, що входить як сировинний матеріал
безпосередньо в процес виробництва товару. Тут слід відзначити тільки таке: природне багатство на
залізо, вугілля, дерево
і~\abbr{т. д.}, на головні елементи будування й застосовування машин,
здається тут природною родючістю капіталу і є елементом
у визначенні норми зиску, незалежно від високого чи низького
рівня заробітної плати.

Через те що норма зиску є \frac{m}{K} або \deq{} \frac{m}{c \dplus{} v}, то ясно, що все те, що спричинює
зміну у величині $c$, а тому й $К$, викликає
також зміну в нормі зиску, навіть і тоді, коли $m$ і $v$ і їх взаємне
відношення лишаються незмінними. Але сировинний матеріал
становить головну частину сталого капіталу. Навіть у ті галузі
\parbreak{}  %% абзац продовжується на наступній сторінці

\parcont{}  %% абзац починається на попередній сторінці
\index{iii2}{0119}  %% посилання на сторінку оригінального видання
маса нехліборобських товаропродуцентів і нехліборобської товарової продукції.
Але тому що це відбувається без його участи, то і видається як якась його
специфічна особливість, те, що маса вартости, маса додаткової вартости і перетворення
частини цієї додаткової вартости на земельну ренту залежить від
суспільного процесу продукції, від розвитку товарової продукції взагалі. Тому
Dove наприклад, хоче звідси вивести ренту. Він говорить, що рента залежить
не від маси хліборобського продукту, а від його вартости; а ця вартість залежить
від маси і продуктивности нехліборобської людности. Але ж і для всякого іншого
продукту справедливо, що він як товар розвивається поруч з тим, як розвивається
почасти маса, почасти різноманітність ряду інших товарів, що становлять у відношенні
до нього еквіваленти. Це вже було показано при загальному викладі
вартости. З одного боку, здібність до обміну певного продукту взагалі залежить
від різноманітности товарів, що існують крім нього. З другого боку від цього ж
залежить особливо та кількість, в якій саме цей продукт можна виробити як товар.

Жоден продуцент, — ні промисловий, ні хліборобський, — розглядуваний
ізольовано, не продукує вартости або товару. Його продукт стає вартістю і товаром
лише в певних суспільних відносинах. Поперше, — оскільки він є вираз суспільної
праці, отже, оскільки власний робочий час даного продуцента є частина суспільного
робочого часу взагалі; подруге, в грошовому характері продукту і в його загальній
здібності до обміну, визначуваній ціною, цей суспільний характер праці продуцента
виступає як наданий (aufgepräger) його продуктові суспільний характер.

Отже, коли пояснення ренти заміняють, з одного боку, поясненням додаткової
вартости або, при ще обмеженішому розумінні, поясненням додаткового
продукту взагалі, то тут, з другого боку, роблять ту помилку, що характер, властивий
усім продуктам як товарам і вартостям, приписують виключно хліборобським
продуктам. Пояснення це стає ще більш вульґарним, коли від загального визначення
вартости переходять до \emph{реалізації} певної товарової вартости. Усякий товар
може реалізувати свою вартість лише в процесі циркуляції, а чи реалізує
він її та в якій мірі реалізує, це кожного разу залежить від умов ринку.

Отже, своєрідна особливість земельної ренти є не в тому, що хліборобські
продукти розвиваються в вартості і як вартості, тобто не в тому, що вони як
товари протистоять іншим товарам, і нехліборобські продукти протистоять їм як
товари, або що вони розвиваються як особливі вирази суспільної праці. Своєрідна
особливість є в тому, що разом з умовами, в яких хліборобські продукти
розвиваються як вартості (товари), і разом з умовами реалізації їхніх вартостей
розвивається і сила земельної власности привлащувати собі чим раз більшу
частину цих створюваних без її участи вартостей, в тому, що чим раз більша
частина додаткової вартости перетворюється на земельну ренту.

\section{Диференційна рента. Загальні уваги}

Аналізуючи земельну ренту, ми спочатку виходитимемо з припущення, що
продукти, з яких виплачується таку ренту, що в них частина додаткової вартости,
а тому і частина всієї ціни зводиться до ренти — для нашої цілі досить мати на
увазі хліборобські продукти або також продукти копалень, — отже, що продукти
ґрунту або копалень, як усі інші товари продаються по цінах їхньої продукції.
Тобто їхні продажні ціни дорівнюють елементам витрат їхньої продукції (вартості
спожитого сталого і змінного капіталу) плюс зиск, визначений загальною
нормою зиску, обчислений на весь авансований капітал, спожитий і неспожитий.
Отже, ми припускаємо, що пересічні продажні ціни цих продуктів дорівнюють їхнім
\parbreak{}  %% абзац продовжується на наступній сторінці


\index{i}{0120}  %% посилання на сторінку оригінального видання
\looseness=1
Однак, для того, щоб посідач грошей міг найти на ринку
робочу силу як товар, мусять здійснитись різні передумови\dots{}
Обмін товарів сам по собі не містить у собі жодних інших відносин
залежности, крім тих, що постають з його власної природи.
За цієї передумови робоча сила може з’явитися на ринку як
товар лише тому й остільки, що й оскільки її посідач, особа, що
її робочою силою вона є, подає її як товар на ринок, або продає.
Щоб її посідач міг продавати її як товар, він мусить мати змогу
порядкувати нею, отже, бути вільним власником своєї здатности
до праці, своєї особи\footnote{
В реальних енциклопедіях клясичної старовини можна прочитати
таку нісенітницю, ніби в античному світі капітал був цілком розвинений,
«бракувало лише вільного робітника і кредитових установ». Пан Момзен
у своїй «Римській історії» також робить подібні quid pro quo одне
по одному.
}. Власник робочої сили й посідач грошей
зустрічаються на ринку й увіходять між собою у відношення
як рівноправні посідачі товарів, які різняться лише тим, що
один є покупець, а другий — продавець, отже, обидва юридично
є рівні особи. Щоб це відношення й далі тривало, треба, щоб
власник робочої сили завжди продавав її лише на певний час,
бо, коли б він продав її геть чисто раз назавжди, то він продав би
себе самого й перетворився б із вільної людини на раба, з посідача
товару на товар. Як особа він мусить постійно ставитися до
своєї робочої сили як до своєї власности, і тим то як до свого
власного товару, а це він може робити лише остільки, оскільки
він завжди віддає свою робочу силу до розпорядження покупця
лише тимчасово, на певний період часу, передає лише на вжиток,
отже, відчужуючи її, не зрікається свого права власности
на неї\footnote{
Тому різні законодавства встановлюють певний максимальний реченець
для робочого контракту. У народів, що в них праця вільна, законодавство
реґулює умови відмовлення від контракту. По різних країнах, а особливо
в Мехіко (перед американською громадянською війною також і на територіях,
відірваних від Мехіко, а по суті і в наддунайських провінціях до
перевороту Кузи), рабство ховається під формою peonage’a. Через аванси,
що їх треба сплачувати працею і що переходять від покоління до покоління,
не лише поодинокий робітник, але й родина його фактично стають
власністю інших осіб і їхніх родин. Хуарец скасував peonage. Так званий
цар Максиміліян знов увів його в життя декретом, що його у вашинґтонській
Палаті представників слушно плямували як декрет, що відновлює
рабство в Мехіко. «Свої особливі фізичні й інтелектуальні здібності та
свою дієздатність я можу\dots{} відчужувати іншій особі для користування
на обмежений реченець, бо в наслідок цього обмеження вони набувають
зовнішнього відношення до моєї цілости й загальности. Через відчуження
цілого мого часу, який конкретизується через працю, і цілої моєї продукції
я зробив би власністю іншої особи саму субстанцію її, тобто мою загальну
діяльність і дійсність, мою особу». (\emph{Hegel}: «Philosophie des Rechts», Berlin
1840, S. 104, § 67).
}.

Друга посутня умова, потрібна для того, щоб посідач грошей
міг найти на ринку робочу силу як товар, є та, що посідач робочої
сили, замість мати змогу продавати товари, в яких упредметнилась
його праця, мусить, навпаки, подавати на ринок як товар
саму свою робочу силу, яка існує лише в його живому організмі.

\parcont{}  %% абзац починається на попередній сторінці
\index{iii2}{0121}  %% посилання на сторінку оригінального видання
самим ціну продукції товару. Для промисловця справа стоїть так, що для нього
витрати продукції товару менші. Йому доводиться менше платити за зрічевлену
працю, а також менше платити заробітної плати за живу робочу силу, якої
у нього застосовується менше. А що витрати продукції його товару менші,
то й його індивідуальна ціна продукції менша. Витрати продукції становлять для
нього 90 замість 100. Отже, його індивідуальна ціна продукції була б замість
115 лише 103\sfrac{1}{2} (100: 115 \deq{} 90: 103\sfrac{1}{2)}. Ріжниця між його індивідуальною ціною
продукції і загальною обмежена ріжницею між його індивідуальними витратами
продукції і загальними. Це — одна з величин, що становлять межі його надпродукту\footnote{Терміни Surplusprodukt (надпродукт) і Mehrprodukt (додатковий продукт) Маркс взагалі вживає,
як тотожні. Про специфічне значення терміну «надпродукт» тут і далі, коли йдеться
про рентодайний капітал, див. кінець розд. 41. \Red{Пр.~Ред.}}. Друга — це є величина загальної ціни продукції, в якій бере участь
загальна норма зиску, як один з регуляційних чинників. Коли б вугілля подешевшало,
то ріжниця між його індивідуальними і загальними витратами продукції
зменшилася б, а тому зменшився б і його надзиск. Коли б він мусив продавати
товар по його індивідуальній вартості, або по ціні продукції, визначуваній
його індивідуальною вартістю, то ріжниця відпала б. Вона є наслідок, з одного
боку, того, що товар продається по своїй загальній ринковій ціні, по ціні,
в яку конкуренція вирівнює індивідуальні ціни, а з другого боку — того, що
більша індівидуальна продуктивна сила праці, приведеної ним в рух, іде на користь
не робітникам, а як взагалі всяка продуктивна сила праці, тому, хто їх
застосовує; що вона виступає як продуктивна сила капіталу.

А що однією межею цього надзиску є висота загальної ціни продукції,
а висота загальної норми зиску є один з її чинників, то цей надзиск може
виникнути лише з ріжниці між загальною і індивідуальною ціною продукції,
отже, з ріжниці між індивідуальною і загальною нормою зиску. Надмір над цією
ріжницею має за свою передумову продаж продуктів дорожче, а не по ціні
продукції, регульованій ринком.

\emph{Подруге}. До цього часу надзиск фабриканта, що вживає як рушійну силу
природний водоспад замість пари, аж ніяк не відрізняється від усякого іншого
надзиску. Всякий нормальний надзиск, тобто такий, що виникає не від випадкових
операцій продажу або від коливань ринкової ціни, визначається ріжницею між
індивідуальною ціною продукції товарів цього окремого капіталу і загальною
ціною продукції, яка реґулює ринкові ціни товарів капіталу цієї сфери продукції
взагалі, або що реґулює ринкові ціни товарів усього капіталу, вкладеного в цю
сферу продукції.

Але звідси починається ріжниця.

Якій обставині завдячує фабрикант в даному разі своїм надзиском, тим надміром,
що його дає йому особисто ціна продукції, реґульована загальною нормою зиску?

Насамперед — природній силі, рушійній силі водоспаду, який є даний природою
і який сам не є продукт праці, а тому не має вартости, як от вугілля,
що перетворює воду в пару і яке само є продукт праці, тому має вартість,
та мусить бути оплачене еквівалентом, потребує витрат. Водоспад — такий природний
аґент продукції, що на створення його не треба жодної праці.

Але це не все. Фабрикант, що працює з паровою машиною, теж вживає
природні сили, які нічого не коштують йому, але роблять працю продуктивнішою
і — оскільки вони таким чином здешевлюють виготовлення засобів існування,
потрібних для робітників, — збільшують додаткову вартість, а тому і зиск; які,
отже, цілком так само монополізуються капіталом, як суспільні природні сили
праці, що постають з кооперації, поділу праці тощо. Фабрикант оплачує вугілля,
а не здібність води змінювати свій аґреґатний стан, переходити в пару, не
еластичність пари тощо. Ця монополізація сил природи, тобто спричиненого ними
\parbreak{}  %% абзац продовжується на наступній сторінці

\parcont{}  %% абзац починається на попередній сторінці
\index{iii2}{0122}  %% посилання на сторінку оригінального видання
підвищення робочої сили, спільна всякому капіталові, що працює з паровими
машинами. Вона може збільшити ту частину продукту праці, яка становить
додаткову вартість, проти тієї частини, що перетворюється в заробітну плату.
Оскільки вона це робить, вона підвищує загальну норму зиску, але вона не
утворює надзиску, бо цей надзиск зводиться саме до надміру індивідуального
зиску над пересічним зиском. Те, що застосування певної природної сили — водоспаду
— утворює тут надзиск, не може, отже, постати з того лише факту, що
підвищення продуктивної сили праці завдячує тут застосуванню природної сили.
Для цього мусять постати дальші модифікаційні обставини.

Навпаки. Звичайне застосування сил природи в промисловості може вплинути
на висоту загальної норми зиску, бо воно впливає на масу праці, яку
треба зужити на продукцію потрібних засобів існування. Але воно само по собі не
створює жодного відхилу від загальної норми зиску, а тут справа саме йде
про такий відхил. Далі: надзиск, що його взагалі реалізує індивідуальний капітал
в якійсь окремій сфері продукції, — бо відхили норм зиску між окремими сферами
продукції невпинно вирівнюються в пересічну норму зиску, — постає, коли лишити
осторонь випадкові відхили, від зменшення витрат продукції, отже, видатків на
продукцію; а це зменшення в свою чергу завдячує або тій обставині, що
капітал застосовується в більших масах ніж пересічно і тому faux frais\footnote*{
Французький термін, що значить «фалшиві» витрати, тобто непродуктивні. \Red{Пр.~Ред.}
}
продукції зменшуються, тимчасом як загальні причини підвищення продуктивної
сили праці (кооперація, поділ праці тощо), можуть діяти в підвищеній мірі, з
більшою інтенсивністю, бо вони діють на ширшому полі праці; абож зменшення
витрат продукції завдячує тій обставині, що — коли лишити осторонь той розмір,
в якому функціонує капітал — вживається кращих методів праці, нових винаходів,
удосконалених машин, хемічних таємниць фабрикації тощо, коротко, нових
удосконалених, вищих за пересічний рівень засобів продукції і методів продукції.
Зменшення витрат продукції і надзиск, що випливає з цього, постають тут
з того способу, що ним застосовується капітал, що функціонує. Вони постають
або в наслідок того, що капітал виключно великими масами концентрується в
одних руках, — обставина, яка відпадає, скоро тільки вживаються пересічно
рівновеликі маси капіталу, — або з того, що капітал певної величини функціонує
особливо продуктивним способом, — обставина, яка відпадає, скоро тільки
виключний спосіб продукції набуває загального поширення або випереджується
ще розвиненішим способом.

Отже, причина надзиску постає тут з самого капіталу (залічуючи до
нього і пущену ним у рух працю); чи з ріжниць в розмірі застосованого
капіталу, чи з доцільнішого способу його застосування; само по собі ніщо не
перешкоджає тому, щоб увесь капітал в певній сфері продукції застосувався
однаковим способом. Конкуренція між капіталами намагається навпаки дедалі
більше зрівнювати ці ріжниці; визначення вартости суспільно потрібним робочим
часом здійснюється у здешевленні товарів і в спонукуванні продукувати товари
в однаково сприятливих умовах. Але з надзиском фабриканта, що застосовує
водоспад, справа стоїть інакше. Підвищена продуктивна сила застосовуваної
ним праці постає ані з самого капіталу і праці, ані з простого застосування
природної сили, відмінної від капіталу і праці, та долученої до капіталу. Вона
постає з більшої природної продуктивної сили праці, в зв’язку з використанням
природної сили, але не такої природної сили, що її може мати всякий капітал
в даній сфері продукції, як наприклад, еластичність пари; отже, не такої природної
сили, що її застосування розуміється само собою, скоро тільки капітал
взагалі застосовується в цій сфері. А такої монополізованої природної сили, якою,
як от водоспадом, можуть порядкувати лише ті, що порядкують окремими дільницями
землі, разом з їхніми належностями. Від капіталу ніяк не залежить покликати
\parbreak{}  %% абзац продовжується на наступній сторінці

\parcont{}  %% абзац починається на попередній сторінці
\index{iii2}{0123}  %% посилання на сторінку оригінального видання
до життя цю природну умову підвищеної продуктивної сили праці, в такий
спосіб як кожен капітал може воду перетворити в пару. Ця природна умова
трапляється в природі лише місцями, і там, де її немає, її неможливо створити
певного витратою капіталу. Вона зв’язана не з продуктами, створюваними працею,
як машини, вугілля тощо, а з певними природними відносинами певної частини
землі. Та частина фабрикантів, що їм належать водоспади, усувають ту частину
фабрикантів, у яких немає водоспадів, від застосування цієї природної сили,
бо земля — і тим паче земля, що має водну силу, — обмежена. Це не виключає
того, що хоч кількість природних водоспадів у певній країні обмежена, кількість
водяної сили, яку може використовувати промисловість, може бути збільшена.
Водоспад можна штучно відвести, щоб цілком використати його рушійну силу;
коли вже є водоспад, водяне колесо можна удосконалити, щоб більше використати
силу води; там, де для подачі води звичайне колесо непридатне, можна застосувати
турбіни і~\abbr{т. ін.} Посідання цією природною силою становить монополію
в руках її посідача, таку умову високої продуктивної сили вкладеного капіталу,
яку не можна створити продукційним процесом самого капіталу\footnote{
Див. про надзиск „Inquiry“ (проти Мальтуса).
}; ця природна
сила, яка може бути так монополізована, завжди зв’язана з землею. Така природна
сила не належить ні до числа загальних умов згаданої сфери продукції, ні до
числа таких її умов, що їх можна створити як загальні умови.

Тепер, коли ми собі уявимо що водоспади разом з прилежною до них землею
перебувають в руках осіб, які вважаються власниками цих частин землі,
землевласниками, то ми побачимо, що вони не дозволяють прикладати капітал
до водоспаду, використовувати його з допомогою капіталу. Вони можуть
дозволити і не дозволити використання водоспаду. Але капітал не може створити
водоспад із себе. Тому надзиск, що постає з цього використання водоспаду, постає
не з капіталу, а з застосування капіталом цієї природної сили, яку монополізувати
можна і яка монополізована. В таких обставинах надзиск перетворюється на земельну
ренту, тобто він дістається власникові водоспаду. Коли фабрикант виплачує
йому за його водоспад 10\pound{ ф. ст.} на рік, то його зиск становить 15\pound{ ф. ст.}; 15\% на
ті 100\pound{ ф. ст.}, що їх тепер досягають його витрати продукції; і він опиняється
тепер цілком в такому самому становищі, може в кращому, ніж усі інші капіталісти
його сфери продукції, що працюють з допомогою пари. Справа ані трохи
не відмінилась би від того, коли б капіталіст сам був власником водоспаду. Він,
як і раніш, одержував би надзиск в 10\pound{ ф. ст.} не як капіталіст, а як власник водоспаду,
і саме тому, що цей надмір постає не з його капіталу, як такого, а
з порядкування такою природною силою, що її можна відділити від його капіталу,
що її можна монополізувати, та яка обмежена в своїх розмірах, — саме тому, цей
надмір переворюється на земельну ренту.

\emph{Перше}: Ясно, що ця рента завжди становить диференційну ренту, бо
вона не ввіходить визначально в загальну ціну продукції товару, а навпаки,
має її за передумову. Вона завжди виникає з ріжниці між індивідуальною
ціною продукції, для окремого капіталу, який порядкує монополізованою природною
силою, і загальною ціною продукції для капіталу, взагалі вкладеного у згадану
сферу продукції.

\emph{Друге}: Ця земельна рента постає не з абсолютного підвищення продуктивної
сили застосованого капіталу — зглядно привласненої ним праці, —
що взагалі могло б призвести лише до зменшення вартости товарів; а
з більшої відносної продуктивности певних окремих капіталів, приміщених
в певну сферу продукції, порівняно з тими приміщенями капіталу, які усунені
від цих виключних, створених природою сприятливих умов підвищення
продуктивної сили. Коли б, наприклад, не зважаючи на те, що вугілля має вартість,
\index{iii2}{0124}  %% посилання на сторінку оригінального видання
а сила води не має вартости, користання парою все ж давало б рішучі
переваги, недосяжні при використанні сили води, і коли б ці переваги більше
ніж компенсували силу води, то сила води не мала б застосування і не могла б
породити жодного надзиску, а, отже, і ренти.

\emph{Третє}: Сила природи не є джерело надзиску, а лише його природна
база, бо це є природна база виключно підвищеної продуктивної сили праці. Так
взагалі споживна вартість є носій мінової вартости, а не причина її. Коли б
ту саму споживну вартість можна було створювати без праці, вона б не мала
жодної мінової вартости, але як і давніш, мала б свою природну корисність
як споживна вартість. Але, з другого боку, без споживної вартости, отже,
без такого природного носія праці, річ не має жодної мінової вартости. Коли б
різні вартості не вирівнювались у ціни продукції і різні індивідуальні ціни
продукції не вирівнювались би в загальну ціну продукції, яка реґулює ринок,
то звичайне підвищення продуктивної сили праці в наслідок використання водоспаду,
лише знизило б ціну товарів, продукованих з допомогою водоспаду, але
не підвищило б тієї частини зиску, що міститься в цих товарах; так само, як,
з другого боку, ця підвищена продуктивна сила праці взагалі не перетворювалась
би на додаткову вартість, коли б капітал продуктивну силу вживаної ним
праці, природну і суспільну, не привлащував би як свою власну.

\emph{Четверте}: Земельна власність на водоспад сама по собі не має ніякого
чинення до створення цієї частини додаткової вартости (зиску), а тому і взагалі
ціни товару, який продукується з допомогою водоспаду. Цей надзиск існував
би і тоді коли б не існувало земельної власности, коли б, наприклад,
земля, до якої належить водоспад, використовувалась фабрикантом, як безгосподарна
земля. Отже, земельна власність не створює тієї частини вартости,
яка перетворюється в надзиск, а лише дає земельному власникові, власникові
водоспаду, можливість перекласти цей надзиск з кишені фабриканта у свою
власну. Земельна власність є причина не створення цього надзиску, а його
перетворення у форму земельної ренти, отже, привласнення цієї частини зиску,
зглядно ціни товару, власником землі або водоспаду.

\emph{П’яте}: Ясно, що ціна водоспаду, отже, ціна, яку одержав би земельний
власник, коли б він продав його третій особі, або самому фабрикантові,
спочатку не входить у ціну продукції товарів, хоч входить в індивідуальні
витрати продукції даного фабриканта; бо рента виникає тут з ціни продукції
товарів того самого роду, продукованих з допомогою парових машин,
з ціни продукції, що реґулюється незалежно від водоспаду. Але, далі, ця ціна
водоспаду взагалі є іраціональний вираз, що за ним ховається реальне економічне
відношення. Водоспад, як земля взагалі, як усі сили природи, не має
жодної вартости, бо в ньому не зрічевлено жодної праці, а тому не має він
жодної ціни, яка нормально є не що інше, як виражена в грошах вартість.
Де немає вартости, там ео ipso\footnote*{
Тим самим. \emph{Пр.~Ред.}
} нічого виражати в грошах. Ця ціна є не
що інше, як капіталізована рента. Земельна власність дає власникові можливість
захоплювати ріжницю між індивідуальним зиском і пересічним зиском,
захоплюваний в такий спосіб зиск, що відновляється щорічно, може бути капіталізований
і тоді виступає як ціна самої сили природи. Коли надзиск, що
його дає фабрикантові використання водоспаду, становить 10\pound{ ф. ст.} на рік, а
пересічний процент 5\%, то ці 10\pound{ ф. ст.} на рік становлять проценти з капіталу в
200\pound{ ф. ст.} і ця капіталізація річних 10\pound{ ф. ст.}, що водоспад дає змогу власникові
його захоплювати їх у фабриканта, виступає тоді, яв капітальна вартість самого
водоспаду. Те, що водоспад не має вартости і що ціна його є звичайний
відбиток захоплюваного надзиску, капіталістично обчисленого, це одразу
\parbreak{}  %% абзац продовжується на наступній сторінці

\parcont{}  %% абзац починається на попередній сторінці
\index{iii2}{0125}  %% посилання на сторінку оригінального видання
виявляється в тому, що ціна в 200\pound{ ф. ст.} становить лише продукт надзиску
в 10\pound{ ф. ст.} на 20 років, тимчасом як цей самий водоспад в інших однакових
умовах дає власникові можливість щорічно захоплювати ці 10\pound{ ф. ст.} протягом
невизначеного часу, 30, 100, х років, і тимчасом як з другого боку, коли нова
метода продукції, що її не можна застосувати до водяної сили, знизить витрати
продукції товарів, вироблюваних з допомогою парової машини з 100 до 90\pound{ ф. ст.},
то зникне надзиск, а разом з ним і рента, а разом з нею і ціна водоспаду.

Встановивши таким чином загальне поняття диференційної ренти, ми переходимо
тепер до розгляду її власне у хліборобстві. Що буде сказано про нього,
взагалі стосується і до гірництва.

\section{Перша форма диференційної ренти (диференційна рента І)}

Рікардо цілком має слушність в таких засадах:

«Рента» [тобто диференційна рента; він гадає, що взагалі не існує якоїсь
іншої ренти, крім диференційної] «завжди становить ріжницю між продуктом,
одержаним в наслідок рівновеликих витрат капіталу і праці» (Principles р. 59).
«На однакових величиною земельних дільницях» треба було б йому додати,
оскільки справа йде про земельну ренту, а не про надзиск взагалі.

Іншими словами: надзиск, коли він створюються нормально, а не в наслідок
випадкових обставин в процесі циркуляції, завжди продукується як ріжниця
між продуктом двох однакових кількостей капіталу і праці, і цей надзиск перетворюється
на земельну ренту, коли дві однакові кількості капіталу і праці з
неоднаковими наслідками зайняті на однакових величиною земельних дільницях.
Проте, немає безумовної доконечности в тому, щоб цей надзиск виникав з
неоднакових наслідків однакових кількостей зайнятого капіталу. В різних
підприємствах можуть бути зайняті і різної величини капітали; здебільша справа
навіть так і стоїть; але рівні пропорційні частини, отже, наприклад, 100\pound{ ф. ст.} кожного
капіталу, дають неоднакові наслідки; тобто норма зиску різна. Це — загальна
передумова існування надзиску в усякій сфері приміщення капіталу взагалі.
Друга — є перетворення цього надзиску на форму земельної ренти (взагалі ренти,
як форми відмінної від зиску); в усякому разі треба дослідити, коли, як і в
яких обставинах відбувається це перетворення.

Далі Рікардо, висловлюючи таку засаду, має слушність, лише оскільки вона
обмежується диференційною рентою:

«Усе, що зменшує ріжницю в продукті, одержаному з тієї самої або з
нової землі, має тенденцію зменшити ренту; а все, що збільшує цю ріжницю,
неодмінно призводить до протилежного ефекту і має тенденцію її збільшити».
%(р.~74). ТУДУ: не влазить

До числа цих причин належать не тільки загальні (родючість і положення),
але й 1) розподіл податків залежно від того, чи впливають вони
рівномірно, чи ні; останнє завжди буває, коли вони не централізовані, як наприклад,
в Англії, і коли податок береться з землі, а не з ренти; 2) ріжниці,
що походить з неоднакового розвитку хліборобства в різних частинах країни,
а ця галузь промисловости, за своїм традиційним характером, важче нівелюється,
ніж мануфактура і 3) нерівномірний розподіл капіталу між орендарями.
А що завойовання хліборобства капіталістичним способом продукції, перетворення
селянина з самостійного господаря в найманого робітника є в дійсності останнім
завойованням цього способу продукції взагалі, то ці ріжниці тут більші,
ніж в будь-якій іншій галузі промисловости.


\index{iii2}{0126}  %% посилання на сторінку оригінального видання
Після цих попередніх зауважень я хочу коротко подати особливості мого
дослідження в відміну від Рікардо та ін.

\pfbreak

Ми розглянемо спочатку неоднакові наслідки, що їх дають однакові маси
капіталу, застосовані на різних земельних дільницях однакової величини: або,
при земельних дільницях неоднакової величини, наслідки, обчислені щодо однакової
земельної площі.

Дві незалежні від капіталу загальні причини цієї неоднаковости наслідків
є: 1)~\emph{Родючість}. (До цього пункту (1) слід вияснити, що взагалі і які
різні моменти розуміються під природною родючістю земель). 2)~\emph{Положення}
земельних дільниць. Остання причина є вирішальна для колоній, і взагалі — для
послідовности, в якій можуть іти під обробіток земельні дільниці одна по одній.
Далі ясно, що ці дві різні основи диференційної ренти, родючість і положення,
можуть впливати в протилежному напрямку. Земля може бути добре розташована
і мало родюча, і навпаки.

Ця обставина є важлива, бо вона пояснює нам, чому, обробляючи землі
даної країни, можна поступово переходити від кращої землі до гіршої, так само,
як і навпаки. Нарешті ясно, що прогрес суспільної продукції взагалі, з одного
боку, нівелює вплив положення, як основу диференційної ренти, бо він створює
місцеві ринки і, створюючи засоби сполучення й транспорту, змінює умови
положення; з другого боку, цей проґрес збільшує ріжниці в місцевому положенні
земельних дільниць, відокремлюючи хліборобство від мануфактури і створюючи
великі промислові центри, з одного боку, і відносне відокремлення села, з другого.

Але спочатку ми залишимо цей пункт, положення, не будемо звертати на
нього уваги і розглянемо лише природну родючість. Лишаючи осторонь кліматичні
та інші моменти, ріжниця у природній родючості сходить на ріжницю
хемічного складу верхнього шару ґрунту, тобто на ріжницю в кількості потрібних
для виростання рослин поживних речовин, що містяться в ньому. Проте, коли
припустити дві земельні дільниці з однаковим хемічним складом ґрунту і в
цьому розумінні однакової природної родючости, то дійсна ефективна родючість
буде різна залежно від тієї форми, в якій перебувають ці поживні речовини і в якій
вони більш-менш засвоюються, більш або менш безпосередньо йдуть на живлення
рослин. Отже, почасти від розвитку хліборобської хемії, почасти від розвитку
хліборобської механіки залежить те, якою мірою на земельних дільницях однакової
природної родючости можна дійсно використати цю природну родючість.
Отже, хоч родючість і є об’єктивна властивість ґрунту, проте, економічно вона
постійно має в собі певне відношення, відношення до даного рівня розвитку
хліборобської хемії і механіки, і змінюється разом з цим рівнем розвитку.
Як з допомогою хемічних засобів (наприклад, застосуванням певного текучого
гною на щільнім глинястім ґрунті, абож обпалюванням важкого глинястого
ґрунту), так і з допомогою механічних засобів (наприклад, особливих
плугів для важких ґрунтів) можна усунути перешкоди, що робили такі самі
родючі ґрунти фактично менше родючими (сюди ж належить і дренування ґрунту).
Це може змінити і саму послідовність в обробітку різних родів землі, як це
було, наприклад, щодо легкого піщаного і важкого глинястого ґрунтів за одного
з періодів розвитку англійського хліборобства. Це знов таки показує, яким
чином історично — в послідовному перебізі обробітку — перехід може однаково відбуватися
так від родючіших земель до менш родючих, як і навпаки. Те саме
може статись і в умовах штучно переведених поліпшень у складі ґрунту, або
в умовах простої зміни в методах хліборобства. Нарешті, такий самий результат
може постати з зміни в ієрархії щодо родів ґрунту в наслідок різних умов
підґрунтя, скоро тільки підґрунтя теж починає оброблятися й перетворюється
на зорану землю. Це зумовлюється почасти застосуванням нових хліборобських
\parbreak{}  %% абзац продовжується на наступній сторінці

\parcont{}  %% абзац починається на попередній сторінці
\index{iii2}{0127}  %% посилання на сторінку оригінального видання
методів (кормові трави), почасти механічними засобами, які перетворюють
підґрунтя в верхній шар ґрунту, або змішують його з ним, або обробляють підґрунтя,
не переміщуючи його на поверхню.

Всі ці впливи на диференційну родючість різних земель сходять на те, що для
економічної родючости стан продуктивної сили праці, в даному разі здібність хліборобства
одразу використовувати природну родючість ґрунту, — здібність, яка різна
на різних ступенях розвитку, — становить так само момент так званої природної
родючости ґрунту, як і його хемічний склад і інші природні властивості.

Отже, ми припускаемо певний ступінь розвитку хліборобства. Ми припускаємо
далі, що ієрархія щодо родів ґрунту відповідає цьому ступеневі розвитку,
як це звичайно завжди буває щодо одночасних приміщень капіталу на різних
землях. В такому разі диференційна рента може бути представлена у висхід ій
або низхідній послідовності, бо, хоч певна послідовність дана для всієї сукупности
дійсно оброблюваних земель, проте завжди відбувається послідовний рух,
в якому складалась ця послідовність.

Припустімо землю чотирьох родів: $А$, $В$, $C$, $D$. Припустимо далі, що ціна
квартера пшениці = 3\pound{ ф. стерл.} або 60 шил. А що рента є просто диференційна рента,
то ця ціна в 60 шил. за квартер з найгіршої землі дорівнює ціні продукції,
тобто дорівнює капіталові плюс пересічний зиск.

Хай $А$ буде ця найгірша земля, що на 50 шил. витрат дає 1 квартер = 60
шил.; отже, 10 шил. зиску, або 20\%.

Хай $В$ при цій самій витраті дає 2 кварт. = 120 шил. Це дало б 70 шил.
зиску, або 60 шил. надзиску.

Хай $C$ при такій самій витраті дає 3 кварт — 180 шил.; загальний
зиск = 130 шил.; надзиск = 120 шил.

Хай $D$ дає 4 кварт. = 240 шил. = 180 шил. надзиску.

Ми мали б тоді тоді таку послідовність:

Відповідні ренти були б для $D = 190 - 10$ шил. або ріжниця між $D$ та
$А$; для $C = 130 - 10$ шил. або ріжниця між $C$ та $А$; для $В = 70 - 10$ шил. або ріжниця
між $В$ та $А$; а загальна рента для $В$, $C$, $D$ = 6 кв. = 360 шил., дорівнювала б сумі ріжниць між
$D$ і $А$, $C$ і $А$, $В$ та $А$.

\begin{table}[h]
  \begin{center}

    \emph{Таблиця I}

  \begin{tabular}{ccccccсс}
    \toprule
      \multirowcell{2}{\makecell{Рід \\землі}} &
      \multicolumn{2}{c}{Продукт} &
      \multirowcell{2}{\makecell{Авансова-\\ний капітал}} &
      \multicolumn{2}{c}{Зиск} &
      \multicolumn{2}{c}{Рента}
      \\
    \cmidrule(rl){2-3}
    \cmidrule(l){5-6}
    \cmidrule(l){7-8}
    &
    \makecell{Квар-\\тери} &
    \makecell{Ши-\\лінги} &
    &
    \makecell{Квар-\\тери} &
    \makecell{Ши-\\лінги} &
    \makecell{Квар-\\тери} &
    \makecell{Ши-\\лінги} &
    \\
    \midrule
     A  &  1  &  \phantom{0}60 & 50 & \phantom{0}\sfrac{1}{6}   &  \phantom{0}10  &   \textemdash & \textemdash \\
     B  &  2  &  120           & 50 & 1\sfrac{1}{6}  &  \phantom{0}70  &   1           & \phantom{0}60 \\
     C  &  3  &  180           & 50 & 2\sfrac{1}{6}  &  130 &   2           & 120 \\
     D  &  4  &  240           & 50 & 3\sfrac{1}{6}  &  190 &   3           & 180 \\
     \cmidrule(rl){2-3}
     \cmidrule(l){7-8}
     Разом & 10 квар. & 600 ш. &    &       &      &   6 квар. &     360 ш. \\
  \end{tabular}
  \end{center}
\end{table}

Ця послідовність,
що становить за даних умов даний продукт, коли справу розглядати
абстрактно (а ми вже показали ті причини, що в наслідок їх така послідовність
може бути і в дійсності), може бути і в низхідному порядку (низхідному
від $D$ до $А$, від родючої землі до менш і менш родючої (так само як і в висхідному
порядку (висхідному від $А$ до $D$, від відносно неродючої до чимраз родючішої землі)
і, нарешті, перемінно, то в низхідному, то в висхідному порядку, наприклад,
від $D$ до $C$, від $C$ до $А$, від $А$ до $В$.

Процес, що відбувався при низхідній послідовності, був такий: ціна квартера
поступово підвищується, скажемо, з 15 шил. до 60. Скоро виявилося, що
випродукованих на $D$ 4 кв. (під ними можпа розуміти мільйони) уже не
\parbreak{}  %% абзац продовжується на наступній сторінці

\parcont{}  %% абзац починається на попередній сторінці
\index{iii2}{0128}  %% посилання на сторінку оригінального видання
досить, так ціна пшениці почала підноситись доти, поки $C$ не набуло змоги
покрити недостачу подання. Тобто ціна мусила піднестись до 20\shil{ шил.} за
квартер. Скоро тільки ціна пшениці піднеслась до 30\shil{ шил.} за квартер, як
в число оброблюваних земель могла б увійти земля $В$. А якби вона піднеслась
до 60\shil{ шилінґів}, до числа оброблюваних земель могла б увійти й земля $А$, це
не призвело б до того, що на застосований тут капітал довелось би задовольнятися
нормою зиску, нижчою за 20\%. Таким чином, для $D$ створилась би
рента спочатку в 5\shil{ шил.} з квартера \deq{} 20\shil{ шил.} з 4 кв., що тут продукується,
а потім в 45\shil{ шил.} з квартера \deq{} 180\shil{ шил.} з 4 квартерів.

Коли норма зиску з $D$ спочатку також була \deq{} 20\%, то і загальний зиск
з 4 кв. був також лише 10\shil{ шил.}, що проте, при ціні збіжжя в 15\shil{ шил.}, становило
більшу кількість збіжжя, ніж при ціні в 60\shil{ шил.} А що збіжжя входить
у репродукцію робочої сили і частина кожного квартера мусить покривати заробітну
плату, а друга — сталий капітал, то за такого припущення додаткова
вартість була вища, а тому, за інших незміних умов, вища була і норма зиску.
(Справу про норму зиску треба ще дослідити осібно і детальніше).

Коли, навпаки, послідовність була зворотна, коли процес починався з $А$,
то, — якщо довелося б ввести в обробіток нові лани, — ціна квартера спочатку
піднеслась би вище за 60\shil{ шил.}; але тому, що потрібне подання в 2 кварт. постачало
б $В$, то ціна знову понизилась би до 60\shil{ шил.}; хоч $В$ і продукує квартер
за 30\shil{ шилінґів}, проте, продається він за 60, бо його подання вистачало б
якраз тільки для того, щоб покрити попит. Так створилася б рента спочатку в
60\shil{ шил.} для $В$, і таким самим способом для $C$ і $D$, припускаючи завжди, що
ринкова ціна залишається 60\shil{ шил.}, хоч дійсна вартість, по якій $C$ і $D$ дають
квартер пшениці, дорівнює 20 і 15\shil{ шил.}; бо як і давніш потрібно подання одного
квартера, що його постачає $А$, для задоволення загальної потреби. В цьому випадку
підвищення попиту понад ту потребу, яку спочатку задовольняло $А$, потім
$А$ і $В$, могло б привести не до послідовного обробітку $В$, $C$ і $D$, а до поширення
площі обробітку взагалі і можливо, що родючіші землі входили б в обробіток
лише пізніше.

В першому ряді із збільшенням ціни рента стала б підвищуватись, а норма
зиску зменшуватись. Це зменшення могло б цілком або почасти паралізуватися
протидіющими обставинами; на цьому пункті згодом спинимося докладніше.
Не слід забувати, що загальна норма зиску визначається не додатковою вартістю
в усіх сферах продукції рівномірно. Не хліборобський зиск визначає промисловий,
а навпаки. Але про це далі.

У другому ряді норма зиску на витрачений капітал лишилась би та
сама; маса зиску визначилась би в меншій кількості збіжжя; але відносна ціна
його проти інших товарів підвищилась би. Але збільшення зиску, там де воно
відбувається, відокремлюється в формі ренти від зиску, замість того, щоб потрапити
до кишені промислових орендарів і визначитися як зиск, що зростає.
А ціна хліба за такого припущення лишилась би незмінною.

Розвиток і зріст диференційної ренти залишаються однакові так за незмінних
цін, як і за таких, що підвищуються, і так само за безперервного поступу
від гірших земель до кращих, як і за безперервного реґресу від крайніх
до гірших земель.

До цього часу ми вважали: 1)~що ціна при одній послідовності підвищується,
при другій — лишається незмінна, і 2)~що постійно відбувається перехід
від кращих земель до гірших або навпаки — від гірших до кращих.

Але припустімо, що потреба в хлібі піднялась з первісних 10 до 17 кв.;
далі, що найгірша земля $А$ витиснута другою землею $А'$, яка при ціні продукції
в 60\shil{ шил.} (50\shil{ шил.} витрат, плюс 10\shil{ шил.}, що становлять 20\% зиску) дає
1\sfrac{1}{3} кварт., так що ціна продукції одного квартера \deq{} 45\shil{ шил.}; абож припустімо,
\parbreak{}  %% абзац продовжується на наступній сторінці

\parcont{}  %% абзац починається на попередній сторінці
\index{iii2}{0129}  %% посилання на сторінку оригінального видання
що перша земля $А$ поліпшилась в наслідок постійного раціонального обробітку,
або що вона при незмінності витрат стала продуктивніше оброблятися,
наприклад, в наслідок заведення конюшини тощо, так що її продукт, за незмінної
величини авансованого капіталу, збільшився до 1\sfrac{1}{3} кварт. Припустімо
далі, що землі $В$, $C$, $D$, як і давніш, дають ту саму кількість продукту, але що
почали оброблятися нові землі $А'$ пересічної між $А$ і $В$ родючости, далі $В'$, $В''$, що
містяться своєю родючістю між $В$ і $C$; в цьому випадку постали б такі явища:

\emph{Перше}: ціна продукції квартера пшениці, або її реґуляційна ринкова
ціна, зменшилась би з 60 до 45\shil{ шил.}, або па 25\%.

\emph{Друге}: відбувся б одночасний перехід від родючішої землі до менш
родючої, і від менш родючої землі до родючішої. Земля $А'$ родючіша, ніж $А$, але
менш родюча, ніж $В$, $C$, $D$, що оброблялись до цього часу; а $В'$, $В''$ родючіші, ніж
$А$, $А'$ і $В$, але менш родючі, ніж $C$ і $D$. Отже, перехід від однієї землі до другої
відбувався б у всіх напрямках; відбувся б перехід не до абсолютно
менш родючої землі проти $А$ тощо, а до відносно менш родючої, порівняно
з землями $C$ і $D$, які до цього часу були найродючіші; з другого боку, перехід
відбувався б не до абсолютно родючішої землі, а до відносно родючішої проти
земель $А$, — або $А$ і $В$, — які до цього часу були найменш родючі.

\emph{Третє}: Рента з $В$ знизилася б; а також рента з $C$ і $D$; але загальна
сума ренти, визначена в збіжжі, піднеслась би з 6 до 7\sfrac{2}{3} кв.; маса землі, що
обробляється і дає ренту, збільшилася б, а також збільшилася б і маса продукту
з 10 до 17 квар. Зиск, хоч він і лишився без перемін для $А$, визначений у
збіжжі, підвищився б; але можливо, що навіть норма зиску підвищилася б, бо
підвищилася б відносна додаткова вартість. В цьому випадку в наслідок здешевлення
засобів існування зменшилася б заробітна плата, отже, витрата на змінний капітал,
отже, і загальні витрати. Вся сума ренти, визначена в грошах, знизилась би з 360 до 345\shil{ шил.}

Подаємо нову послідовність переходу
(див. табл. II).

\begin{table}[h]
  \begin{center}
  \footnotesize
    \emph{Таблиця II}

  \begin{tabular}{c c c c c c c c c}
    \toprule
      \multirowcell{2}{\makecell{Рід \\землі}} &
      \multicolumn{2}{c}{Продукт} &
      \multirowcell{2}{\makecell{Витрата \\капіталу}} &
      \multicolumn{2}{c}{Зиск} &
      \multicolumn{2}{c}{Рента} &
      \multirowcell{2}{\makecell{Ціна про-\\дукції \\квартера}}
      \\
    \cmidrule(rl){2-3}
    \cmidrule(l){5-6}
    \cmidrule(l){7-8}
    &
    \makecell{Квар-\\тери} &
    \makecell{Ши-\\лінґи} &
    &
    \makecell{Квар-\\тери} &
    \makecell{Ши-\\лінґи} &
    \makecell{Квар-\\тери} &
    \makecell{Ши-\\лінґи} &
    \\
    \midrule
     А\phantom{''}   &  1\sfrac{1}{3}            & \phantom{0}60 & 50  &  \phantom{0}\sfrac{2}{9} & \phantom{0}10  &  \textemdash             & \textemdash    & 45\phantom{\sfrac{1}{1}} шил. \\
     А'\phantom{'}   &  1\sfrac{2}{3}            & \phantom{0}75 & 50  &  \phantom{0}\sfrac{5}{9} & \phantom{0}25  &  \phantom{0}\sfrac{1}{3} & \phantom{0}15  & 36\phantom{\sfrac{1}{1}} \ditto{шил.} \\
     B\phantom{''}   &  2\phantom{\sfrac{1}{1}}  & \phantom{0}90 & 50  &  \phantom{0}\sfrac{8}{9} & \phantom{0}40  &  \phantom{0}\sfrac{2}{3} & \phantom{0}30  & 30\phantom{\sfrac{1}{1}} \ditto{шил.} \\
     В'\phantom{'}   &   2\sfrac{1}{2}           & 105           & 50  &  1\sfrac{2}{9}           & \phantom{0}55  &  1\phantom{\sfrac{1}{1}}                       & \phantom{0}45  & 25\sfrac{2}{7} \ditto{шил.} \\
     В''             &   2\sfrac{2}{3}           & 120           & 50  &  1\sfrac{5}{9}           & \phantom{0}70  &  1\sfrac{1}{3}           & \phantom{0}60  & 22\sfrac{1}{2} \ditto{шил.} \\
     C\phantom{''}   &  3\phantom{\sfrac{1}{1}}  & 135           & 50  &  1\sfrac{8}{9}           & \phantom{0}85  &  1\sfrac{2}{3}           & \phantom{0}75  & 20\phantom{\sfrac{1}{1}} \ditto{шил.} \\
     D\phantom{''}   &  4\phantom{\sfrac{1}{1}}  & 180           & 50  &  2\sfrac{8}{9}           & 130            &  2\sfrac{2}{3}           & 120            & 15\phantom{\sfrac{1}{1}} \ditto{шил.} \\
     \cmidrule(rl){2-2}
     \cmidrule(l){7-8}
     Разом & 17 & &    &       &      &   7\sfrac{2}{3} &     345 \\
  \end{tabular}
  \end{center}
\end{table}

Нарешті, коли б, як і давніш, оброблялись тільки землі $А$, $В$, $C$, $D$, але продуктивність їхня зросла б
остільки, що земля $А$ замість 1 квартера давала б 2, $В$ замість 2 квартерів — 4,
$C$ замість 3 квартерів — 7 і $D$ замість 4 квартерів — 10, отже, коли б ті самі
причини по-різному вплинули б на різні землі, то вся продукція підвищилася
б з 10 до 23 квартерів. Припустімо, що попит в наслідок приросту
людности і пониження ціни поглинув би ці 23 квартери, в такому разі ми
мали б такий результат (див. табл. III).

Числові відношення тут, як і в попередніх таблицях, довільні, але припущення
цілком раціональні.


\index{iii2}{0130}  %% посилання на сторінку оригінального видання
Перше і головне припущення є, що поліпшення в хліборобстві нерівномірно
впливає на землі різних родів, і тут воно більше впливає на кращі землі
$C$ і $D$, ніж на $А$ і $В$. Досвід довів, що, звичайно, справа так і стоїть, хоч може
статись і зворотне. Коли б поліпшення більше впливало на гірші землі, ніж на
кращі, то рента з останніх понизилася б замість підвищитись. — Але з абсолютним
зростом родючости всіх родів землі у таблиці одночасно припускається зріст
вищої відносної родючості кращих родів землі $C$ і $D$, а тому зріст ріжниці в продукті за однакової
величини застосованого капіталу, а тому і зріст диференційної ренти.
Друге припущення є в тому, що з зростанням всього продукту відповідно зростає і загальна потреба в
ньому. \emph{Поперше}, не слід уявляти собі це зростання раптовим; воно відбувається поступово, доти, доки
не встановиться ряд III. \emph{Подруге}, невірно, нібито споживання потрібних засобів існування не зростає разом з їхнім
здешевленням. Скасування хлібних законів в Англії (дивись Newman) довело зворотне, і протилежне
уявлення постало лише тому, що великі і раптові ріжниці в урожаях, які пояснюються тільки погодою,
спричиняють то неспіврозмірне пониження, то неспіврозмірне підвищення цін збіжжя.
Коли в цьому разі раптове і скороминуще здешевлення не встигає справити повного впливу на поширення
споживання, то зворотне явище спостерігається в тому випадку, коли здешевлення випливає із зменшення
самої регуляційної ціни продукції, отже, коли воно має тривалий характер. \emph{Потрете}, частина збіжжя
може бути спожита у вигляді горілки або пива. А зростаюуче споживання обох цих продуктів ніяк не
обмежено вузькими межами. \emph{Почетверте}, тут справа залежить почасти від приросту людности, почасти
від експорту збіжжя в тих країнах, що вивозять збіжжя — як от Англія, до і
пізніше половини XYIII століття, і де тому потребу реґулюється межами не самого
тільки національного споживання. \emph{Нарешті}, збільшення і здешевлення
продукції пшениці може мати своїм наслідком, що замість жита або вівса за
головний засіб харчування маси народу стане пшениця, так що вже в наслідок
самого цього ринок для неї зросте подібно до того, як при зменшенні кількості
продукту і збільшенні його ціни може постати зворотне явище. — При цих припущеннях, отже, і при
взятих числових відношеннях, ряд III дає той наслідок,
що ціна знижується з 60 до ЗО шил. за квартер, отже на 50\%; що продукція проти ряду І зростає з 10
до 23 квартерів, отже, на 130\%; що рента з землі $В$ лишається незмінною, рента з землі $C$
подвоюється, а з $D$ більше, ніж подвоюється, і що загальна сума ренти підвищується з 18 до 22\pound{ ф.
стерл.}, отже, на 22\sfrac{1}{9}\%.

З порівняння цих трьох таблиць (при чому ряд I треба брати подвійно:
у висхідному напрямку від $А$ до $D$ і в низхідному від $D$ до $А$), що їх можна
розглядати або як дані ступені хліборобства, за даного стану суспільства, наприклад,

\begin{table}[h]
  \begin{center}
  \footnotesize
    \emph{Таблиця III}

  \begin{tabular}{c c c c c c c c c}
    \toprule
      \multirowcell{2}{\makecell{Рід \\землі}} &
      \multicolumn{2}{c}{Продукт} &
      \multirowcell{2}{\makecell{Витрата \\капіталу}} &
      \multicolumn{2}{c}{Зиск} &
      \multicolumn{2}{c}{Рента} &
      \multirowcell{2}{\makecell{Ціна про-\\дукції \\квартера}}
      \\
    \cmidrule(rl){2-3}
    \cmidrule(l){5-6}
    \cmidrule(l){7-8}
    &
    \makecell{Квар-\\тери} &
    \makecell{Ши-\\лінґи} &
    &
    \makecell{Квар-\\тери} &
    \makecell{Ши-\\лінґи} &
    \makecell{Квар-\\тери} &
    \makecell{Ши-\\лінґи} &
    \\
    \midrule
      А  &  \phantom{0}2  &  \phantom{0}60  & 50 & \phantom{0}\sfrac{1}{3}  & \phantom{0}10  & \phantom{0}0 & \phantom{00}0  &  30\\
      B  &  \phantom{0}4  &  120            & 50 & 2\sfrac{1}{3}            & \phantom{0}70  & \phantom{0}2 & \phantom{0}60  &  15\\
      C  &  \phantom{0}7  &  210            & 50 & 5\sfrac{1}{3}            & 160            & \phantom{0}5 & 150            &  8\sfrac{4}{7} \\
      D  &  10            &  300            & 50 & 8\sfrac{1}{3}            & 250            & \phantom{0}8 & 240            &  6\phantom{0} \\
      \cmidrule(rl){2-2}
      \cmidrule(l){7-8}
      Разом & 23          &                 &    &                          &                & 15           & 450           & \\
  \end{tabular}
  \end{center}
\end{table}

\parcont{}  %% абзац починається на попередній сторінці
\index{iii2}{0131}  %% посилання на сторінку оригінального видання
один поряд одного в трьох різних країнах або як такі, що йдуть одна по одній в різних періодах
розвитку тієї самої країни, — з цього порівняння випливає:

1) Що ряд у своєму закінченному вигляді, — хоч би який завжди був перебіг процесу його складання, —
завжди виступає як низхідний; бо при розгляді ренти завжди виходять спочатку від землі, яка дає
максимум ренти/і лише нарешті переходять до тієї, що не дає жодної ренти.

2) Ціна продукції на найгіршій землі, що не дає ренти, завжди становить реґуляційну ринкову ціну,
хоч ціна ця в таблиці 1, коли вона склалася у висхідному порядку, тільки через те лишається
незмінна, що всю кращу землю оброблено. В цьому випадку ціна збіжжя, випродукованого на кращій
землі, є реґуляційною остільки, оскільки від кількости продукту, випродукованого на ній, залежить, в
якій мірі земля А лишається реґуляційною. Коли б на землях В, C, В продукувалось понад потребу, то
земля А перестала б бути реґуляційною. Це й штовхнуло Шторха на те, що він за реґуляційпі визнав
найкращі землі. В цьому розумінні англійські ціпи хліба реґулюються американськими.

3) Диференційна рента постає з ріжниці в природній родючості різного ґрунту (тут положення землі ще
не береться на увагу), — ріжниці даної для кожного даного ступеня розвитку культури, отже, з
обмежености кількости кращих земель, і з тієї обставини, що однакові капітали доводиться вкладати в
неоднакові землі, які, отже, при витраті однакового капіталу дають неоднакову кількість продукту.

4) Диференційна рента і ґрадація диференційної ренти можуть однаково виникнути так у низхідному
порядку в наслідок переходу від кращої землі до гіршої, як і навпаки, від гіршої до кращої, або в
наслідок переходу впереміжку в усіх напрямках. (Ряд І може скластися в наслідок переходу так від І)
до А, як і від А до Б. Ряд II охоплює обидва види руху).

5) Залежно від способу виникнення, диференційна рента може постати за сталої, висхідної і низхідної
ціни хліборобського продукту. За низхідної ціни загальна продукція і загальна сума ренти може
підвищитись, і земельні дільниці, що не давали до цього часу ренти, можуть почати давати ренту, не
зважаючи на те, що гірша земля А витиснена кращою або сама поліпшилась, і що рента з інших кращих і
навіть найкращих земель понижується (таблиця II);
цей процес може бути також зв’язаний з пониженням загальної суми ренти (в грошах). Нарешті, за
низхідних цін, зумовлених загальним поліпшенням обробітку, в зв’язку з чим кількість і ціна продукту
з найгіршої землі понижується, — рента з частини кращих земель може лишитись незмінною або
понизитися, але рента з найліпших земель може зрости. Коли ріжниця мас продукту дана, то
диференційна рента з усякої землі проти найгіршої землі без сумніву залежить від ціни, наприклад,
квартера пшениці. Але коли ціна є дана, то диференційна рента залежить від розміру ріжниці між
масами продукту, і коли з підвищенням абсолютної родючости всіх земель родючість земель відносно
більше підвищується, ніж родючість гірших, то разом з цим зростає і величина цієї ріжниці. Так
(таблиця І) при ціні в 60 шил. рента з Б визначається ріжницею в продукті проти А, тобто, надміром в
3 квартери; тому рента = ЗХX 60 = 180 шил. Але в таблиці III, де ціна = 30 шил., вона визначається
масою надмірного продукту з землі Б порівняно з А = 8 кварт.; але 8X30 = 240 шил.

Разом з тим відпадає та перша фалшива передумова диференційної ренти, — яка ще панує в Веста,
Мальтуса, Рікардо, нібито диференційна рента неодмінно має за передумову перехід до земель дедалі
гіршої якости, або постійно зменшувану родючість ґрунту. Як ми вже бачили, диференційна рента може
постати при переході до земель дедалі ліпшої якости; вона може постати, коли нижчий ступінь займе
краща земля, замість колишньої гіршої; вона може постати в зв’язку з ростучим поступом хліборобства.
Умовою її виникнення є лише неоднаковість родів землі. Оскільки береться на увагу розвиток
продуктивності,
\parbreak{}  %% абзац продовжується на наступній сторінці

\parcont{}  %% абзац починається на попередній сторінці
\index{iii2}{0132}  %% посилання на сторінку оригінального видання
умова її виникнення є в тому, що підвищення абсолютної родючості всієї земельної площі не знищує
тієї неоднаковості, але збільшує її, або залишає без зміни, абож лише зменшує.

Від початку і до половини XVIII століття панувало в Англії, не зважаючи на пониження ціни золота і
срібла, безперервне падіння цін збіжжя одночасно (коли розглядати весь період) з ростом ренти,
загальної суми ренти, розміру оброблюваної земельної площі, хліборобської продукції і людности. Це
відповідає таблиці І, комбінованій з таблицею II, у висхідному напрямку, але так, що гірша земля $А$
або поліпшується, або виключається з числа земель, оброблюваних під збіжжя; це, звичайно, не значить,
що вона не буде використана для інших сільськогосподарських або промислових цілей.

Від початку XIX століття (треба точніше подати час) до 1815 року безперервне підвищення цін збіжжя
одночасно з постійним зростом ренти, загальної суми ренти, розміру оброблюваної земельної площі,
хліборобської продукції і людності. Це відповідає таблиці І у низхідному напрямку. (Тут слід навести
цитату щодо обробітку гірших земель за того часу).

За доби Петті і Давенанта скарги сільської людности і земельних власників на поліпшення і поширення
обробітку; пониження ренти на кращих землях, підвищення загальної суми ренти в наслідок поширення
площі землі, що дає ренту.

(До цих трьох пунктів навести потім дальші цитати; також щодо ріжниці у родючості різних частин
обробленої землі в країні).

Щодо диференційної ренти слід взагалі зауважити, що ринкова вартість завжди перевищує загальну ціну
продукції даної маси продуктів. Для прикладу візьмімо таблицю І. 10 кватерів всього продукту
продаються за 600\shil{ шил.}, бо ринкова ціна визначається ціною продукції на $А$, яка становить 60\shil{ шил.} за
квартер. Але дійсна ціна продукції є:

\begin{table}[H]
  \centering
  \small
  \begin{tabular}{l r@{~}r l l}
    А & 1 кварт. \deq{} & 60\shil{шил.} & & 1 кварт. \deq{} 60\shil{шил.} \\
    В & 2 кварт. \deq{} & 60\shil{шил.} & & 1 кварт. \deq{} 30\shil{шил.} \\
    C & 3 кварт. \deq{} & 60\shil{шил.} & & 1 кварт. \deq{} 20\shil{шил.} \\
    D & 4 кварт. \deq{} & 60\shil{шил.} & & 1 кварт. \deq{} 15\shil{шил.} \\
    \midrule
      &10 кварт. \deq{} &240\shil{шил.} & ~пересічно & 1 кварт. \deq{} 24\shil{шил.} \\
  \end{tabular}
\end{table}

\noindent{}Дійсна ціна продукції 10 квартерів є 240\shil{ шил.}; вони продаються за 600, тобто на 250\% дорожче.
Дійсна пересічна ціна 1 квартера є 24\shil{ шил.}: ринкова ціна — 60\shil{ шил.}, тобто теж на 250\% дорожча.

Тут маємо визначення за посередництвом ринкової вартости в тому її вигляді, як вона на базі
капіталістичного способу продукції пробивається за посередництвом конкуренції; ця остання породжує
фалшиву соціяльну вартість. Це постає з закону ринкової вартости, якому підпорядковані продукти
хліборобства.
Визначення ринкової вартости продуктів, отже, і хліборобських продуктів, є суспільний акт, хоч і акт
суспільно несвідомий і ненавмисний, акт, що неминуче ґрунтується на міновій вартості продукту, а не
на землі і ріжницях її родючості. Коли уявити собі, що капіталістична форма суспільства знищена і
суспільство організоване як свідома і плянова асоціяція, то ці 10 квартерів являтимуть собою
кількість самостійного робочого часу, рівну тому, що міститься в 240\shil{ шил.} Отже, суспільство не
купувало б цього хліборобського продукту за таку кількість робочого часу, яка в 2\sfrac{1}{2}, раза більша
за робочий час, який дійсно міститься в цьому продукті. Тим самим відпала б база класу власників
землі. Це впливало б цілком так само, як здешевлення продукту на таку суму в наслідок чужоземного
довозу. Тому, оскільки справедливо було б сказати, що — в умовах збереження сучасного способу
продукції, але припускаючи, що диференційна рента діставатиметься державі — ціни земельних
продуктів, за інших незміних умов, залишились би тими самими, так само помилково було б
\parbreak{}  %% абзац продовжується на наступній сторінці

\parcont{}  %% абзац починається на попередній сторінці
\index{iii2}{0133}  %% посилання на сторінку оригінального видання
казати, що вартість продуктів залишилась би та сама в умовах заміни капіталістичної
продукції асоціацією. Тотожність ринкової ціни однорідних товарів є
спосіб, у який, на базі капіталістичного способу продукції і взагалі продукції,
що ґрунтується на обміні товарів між поодинокими продуцентами, пробивається
суспільний характер вартости. Те, що суспільство, розглядуване як споживач,
переплачує за хліборобські продукти, і що становить мінус у реалізації його
робочого часу в хліборобській продукції, це становить тепер плюс для однієї
частини суспільства, для земельних власників.

Друга обставина, важлива для того, про що говориться в дальшім розділі
під рубрикою II, така:

Справа не тільки в ренті з акра або з гектара, взагалі не тільки в ріжниці
між ціною продукції і ринковою ціною, або між індивідуальною й загальною
ціною продукції з акра, але також і в тому, скільки акрів кожного
роду землі обробляється. Тут важлива безпосередньо лише величина загальної
суми ренти, тобто сукупної ренти з усієї оброблюваної площі; але це дає
нам одночасно можливість перейти до з’ясовування того, як підвищується норма
ренти, хоч ціни не збільшуються, і хоч за низхідних цін не збільшуються
ріжниці у відносній родючості різних родів землі. Вище ми мали:
% (див. табл. I).

\begin{table}[H]
  \small
  \centering
  \caption*{Таблиця I}

  \begin{tabular}{l c c c c c}
    \toprule
      \makecell[l]{Рід\\землі} &
      \makecell{Акри} &
      \makecell{Ціна\\продукції,\pound{}} &
      \makecell{Продукт,\\кварт.} &
      \makecell{Рента\\в збіжжі, кварт.} &
      \makecell{Грошова\\рента,\pound{}}
      \\
     \midrule
     A & 1 & 3 & 1 & 0 & 0 \\
     B & 1 & 3 & 2 & 1 & 3 \\
     C & 1 & 3 & 3 & 2 & 6 \\
     D & 1 & 3 & 4 & 3 & 9 \\
     \midrule
     Сума & 4 & \textendash{} & \hang{r}{1}0 & 6 & \hang{r}{1}8 \\
  \end{tabular}
\end{table}

\noindent{}Припустімо тепер, що число оброблюваних акрів кожного розряду подвоїлося.
В такому разі ми матимемо:
% (див. табл. Iа).

\begin{table}[H]
  \small
  \centering
  \caption*{Таблиця Iа}

  \begin{tabular}{l c c c c c}
    \toprule
      \makecell[l]{Рід\\землі} &
      \makecell{Акри} &
      \makecell{Ціна\\продукції,\pound{}} &
      \makecell{Продукт,\\кварт.} &
      \makecell{Рента\\в збіжжі, кварт.} &
      \makecell{Грошова\\рента,\pound{}}
      \\
     \midrule
     A & 2 & 6 & 2 & 0 & \phantom{0}0 \\
     B & 2 & 6 & 4 & 2 & \phantom{0}6 \\
     C & 2 & 6 & 6 & 4 & 12 \\
     D & 2 & 6 & 8 & 6 & 18 \\
     \midrule
     Сума 
       & 8 & \textendash{} & \hang{r}{2}0 & \hang{r}{1}2 & 36 \\
  \end{tabular}
\end{table}

\noindent{}Ми припустимо ще 2 випадки; перший, коли продукція розширюється
на обох гірших родах землі. Отже, тоді матимемо:
%(див. табл. Іb).

\begin{table}[H]
  \small
  \centering
  \caption*{Таблиця Ib}
  \begin{tabular}{l c c c c c c}
    \toprule
      \multirowcell{2}[0ex][l]{Рід\\землі} &
      \multirowcell{2}{Акри} &
      \multicolumn{2}{c}{Ціна продукції,\pound{ ф. ст.}} &
      \multirowcell{2}{Продукт} &
      \multirowcell{2}{Рента\\в збіжжі, кварт.} &
      \multirowcell{2}{Грошова\\рента,\pound{}} \\
      \cmidrule(rl){3-4}

      &  &  на акр. & в сумі & &                    &  \\
      \midrule
      A & 4 & 3 & 12 & 4 & 0 &  \pZ{}0 \\
      B & 4 & 3 & 12 & 8 & 4 & 12 \\
      C & 2 & 3 & \pZ{}6 & 6 & 4 & 12 \\
      D & 2 & 3 & \pZ{}6  & 8 & 6 & 18 \\
     \midrule
     Сума & \hang{r}{1}2 & \textendash{} & 36 & \hang{r}{2}6 & \hang{r}{1}4 & 42 \\
  \end{tabular}
\end{table}

\noindent{}І, нарешті, коли маємо неоднакове поширення продукції і оброблюваної площі в чотирьох розрядах:
% (див. табл. Iс).

\begin{table}[H]
  \small
  \centering
  \caption*{Таблиця Iс}
  \begin{tabular}{l c c c c c c}
    \toprule
      \multirowcell{2}[0ex][l]{Рід\\землі} &
      \multirowcell{2}{Акри} &
      \multicolumn{2}{c}{Ціна продукції,\pound{ ф. ст.}} &
      \multirowcell{2}{Продукт} &
      \multirowcell{2}{Рента\\в збіжжі, кварт.} &
      \multirowcell{2}{Грошова\\рента,\pound{}} \\
      \cmidrule(rl){3-4}

      &  &  на акр. & в сумі & &                    &  \\
      \midrule
      A & 1 & 3 & \pZ{}3 & \pZ{}1 & \pZ{}0 &  \pZ{}0 \\
      B & 2 & 3 & \pZ{}6 & \pZ{}4 & \pZ{}2 & \pZ{}6 \\
      C & 5 & 3 & 15      & 15     & 10 & 30 \\
      D & 4 & 3 & 12      & 16     & 12 & 36 \\
     \midrule
     Сума & \hang{r}{1}2 & \textendash{} & 36 & 36 & 24 & 72 \\
  \end{tabular}
\end{table}

\noindent{}Насамперед, в усіх цих випадках І, І$а$, І$b$, І$с$ рента з одного акра лишається та сама; бо в
дійсності продукт однакової маси капіталу на кожному
\parbreak{}  %% абзац продовжується на наступній сторінці

\parcont{}  %% абзац починається на попередній сторінці
\index{iii2}{0134}  %% посилання на сторінку оригінального видання
акрі землі того самого роду лишився незмінний; припущено тільки, — і це кожного
даного моменту відбувається в усякій країні, — що землі різних родів перебувають
у певному відношенні до всієї оброблюваної землі; і що відношення, — а це постійно
відбувається в двох країнах при порівнянні їх одна з однією, або в тій самій
країні за різних часів, — в якому вся оброблювана земельна площа розподіляється
між різними родами землі, змінюється.

Порівнюючи Іа з І, ми бачимо, що коли площа оброблюваних земель усіх
чотирьох розрядів зростає в однаковій пропорції, то з подвоєнням кількосте
оброблюваних акрів подвоюється вся продукція, а також рента в збіжжі і грошах.

\begin{table}[h]
  \begin{center}
    \emph{Таблиця Iс.}
    \footnotesize

  \begin{tabular}{c c c c c c c}
    \toprule
      \multirowcell{2}{\makecell{Рід \\землі}} &
      \multirowcell{2}{\makecell{Акри}} &
      \multicolumn{2}{c}{Ціна продукції} &
      \multirowcell{2}{\makecell{Продукт}} &
      \multirowcell{2}{\makecell{Рента \\ в збіжжі}} &
      \multirowcell{2}{\makecell{Грошова \\рента}} \\
      \cmidrule(rl){3-4}

      &  &  На акр. & В сумі & &                    &  \\
      \midrule

      A & 1\phantom{акр.} &  3\pound{ ф. ст.}                 & 3\pound{ ф. ст.}         & 1  кварт.         & 0  кварт.         & 0\pound{ ф. ст.}\\
      B & 2\phantom{акр.} &  3  \ditto{ф.} \ditto{ст.} & 6  \ditto{ф.} \ditto{ст.} & 4  \ditto{кварт.} & 2  \ditto{кварт.} & 6  \ditto{ф.} \ditto{ст.}\\
      C & 5\phantom{акр.} &  3  \ditto{ф.} \ditto{ст.} & 15  \ditto{ф.} \ditto{ст.}  & 15  \ditto{кварт.} & 10  \ditto{кварт.} & 30  \ditto{ф.} \ditto{ст.}\\
      D & 4\phantom{акр.} &  3  \ditto{ф.} \ditto{ст.} & 12  \ditto{ф.} \ditto{ст.}  & 16  \ditto{кварт.} & 12  \ditto{кварт.} & 36  \ditto{ф.} \ditto{ст.}\\
     \cmidrule(rl){1-1}
     \cmidrule(rl){2-2}
     \cmidrule(rl){4-4}
     \cmidrule(rl){5-5}
     \cmidrule(rl){6-6}
     \cmidrule(rl){7-7}
     Сума & 12 акр. &                 & 36\pound{ ф. ст.}  & 36 кварт.        & 24  кварт.         & 72\pound{ ф. ст.} \\
  \end{tabular}
  \end{center}
\end{table}

Але порівнюючи послідовно І$b$ з І$с$, ми знайдемо, що в обох випадках площа
оброблюваної землі збільшується утроє. В обох випадках вона збільшується з
4 до 12 акрів, але найбільше збільшення в I$b$ відбувається в розрядах
$а$ і $b$, в яких $а$ не дає жодної ренти, а $b$ дає найменшу
диференційну ренту, а саме з 8 новооброблюваннх акрів на землю $а$ і $b$
припадає по 3, разом 6, тимчасом,
як на $c$ і $d$ припадає лише по 1 акрові, разом 2. Іншими словами: \sfrac{3}{4} приросту
припадають на $а$ і $b$ і лише \sfrac{1}{4} на $c$ і $d$. Коли таке припустити, то в
I$b$ порівняно з І, потроєному збільшенню площі обробленої землі не відповідає
таке саме потроєне збільшення продукту, бо кількість його збільшилась з 10
не до 30, а лише до 26. З другого боку, тому, що значна частина усього приросту
постала на землі $А$, що не дає ренти, а більша частина приросту на
кращих землях постала на розряді $В$, то рента в збіжжі збільшується лише
з 6 до 14 кварт., а грошова рента — з 18 до 42\pound{ ф. стерл}.

Коли ми, навпаки, порівняємо І$с$ з І, де земля, яка не дає ренти, зовсім
не збільшується в розмірі, а земля, що дає мінімальну ренту, збільшується
лише незначно, тоді як найбільший приріст припадає на $C$ і $D$, то ми побачимо,
що при потроєному збільшенні обробленої землі продукція зросла з 10 до
36 квартерів, тобто більш, ніж у три рази; рента в збіжжі збільшилася з 6 до
24 квартерів, або в чотири рази; і так само збільшилась грошова рента
з 18 до 72\pound{ ф. стерл}.

В усіх цих випадках по самій суті справи ціна хліборобського продукту
лишається незмінною; в усіх випадках загальна сума ренти зростає з поширенням
культури, оскільки воно відбувається не виключно на гіршій землі, що
не дає жодної ренти. Але зріст цей різний. В тій самій мірі, в якій поширення
відбувається на кращих землях і в якій, отже, маса продукту зростає не тільки
пропорційно поширенню земельної площи, але швидше, — зростає і рента в
збіжжі і в грошах. В тій самій мірі, в якій поширення відбувається переважно
на найгіршій землі і на близьких до неї родах землі (при чому припускається,
що розряд найгіршої землі лишається той самий), в цій самій мірі загальна
сума ренти збільшується непропорційно поширенню культивованої площі. Отже,
в двох країнах, де земля $А$, що не дає ренти, однакова якістю, сума ренти
буде перебувати у зворотному відношенні до тієї відповідної частини, яку в загальній
площі обробленої землі становлять найгірші і менш родючі землі, а тому

\parbreak{}  %% абзац продовжується на наступній сторінці

\parcont{}  %% абзац починається на попередній сторінці
\index{iii1}{0135}  %% посилання на сторінку оригінального видання
було б 1 432 080 000 фунтів бавовни на рік. Але довіз бавовни,
коли відняти вивіз, становив в 1856 і 1857 роках тільки
1 022 576 832 фунти; отже, неминуче мусив постати дефіцит
у 409 503 168 фунтів. Пан Бейнс, який люб’язно згодився обговорити
зі мною цю справу, гадає, що обчислення річного споживання
бавовни, основане на споживанні блекбернської округи,
було б перебільшеним не тільки в наслідок ріжниці нумерів, що
випрядаються, але й в наслідок вищої якості машин. Він оцінює загальне
річне споживання бавовни в Сполученому Королівстві в
1000 мільйонів фунтів. Але якщо він і має рацію, якщо дійсно
є надмір подання в 22\sfrac{1}{2}  мільйони, то вже тепер попит і подання,
як видно, майже урівноважуються, навіть якщо не брати до
уваги додаткові веретена і ткацькі верстати, які, за паном
Бейнсом, встановлюються в його окрузі і, отже, напевно і
в інших округах“ (стор. 59, 60).

\subsection{Загальна ілюстрація: бавовняна криза 1861—1865 рр.}

\subsubsection{Попередній період 1845-1860 рр.}

1845 рік. Час розквіту бавовняної промисловості. Дуже низькі
ціни на бавовну. Л. Горнер каже про це: „Протягом останніх
8 років я не бачив жодного періоду такого пожвавленого стану
справ, як минулим літом і осінню. Особливо у бавовнопрядільництві.
Протягом цілого півроку я щотижня одержував заяви
про нові капіталовкладення у фабрики; це були або нові фабрики,
які будувалися, або ті нечисленні фабрики, які пустували, знаходили
нових орендарів, абож ті фабрики, які працювали, розширювались
і устатковувались новими потужнішими паровими машинами
та збільшеною кількістю робочих машин“ („Rep. of Insp.
of Fact., Oct. 1845“, стор. 13).

1846 рік. Починаються нарікання. „Уже протягом досить довгого
часу я чую від дуже багатьох бавовняних фабрикантів нарікання
на пригнічений стан їх справ\dots{} протягом останніх 6 тижнів
різні фабрики почали працювати неповний час, звичайно
8 годин на день замість 12; це, як видно, поширюється\dots{} дуже
підвищились ціни бавовни і\dots{} не тільки не підвищились ціни
фабрикатів, але\dots{} вони стоять ще нижче, ніж перед підвищенням
цін бавовни. Значне збільшення числа бавовняних фабрик
протягом останніх 4 років мусило мати своїм наслідком,
з одного боку, дуже збільшений попит на сировинний матеріал
і, з другого боку, дуже збільшене подання фабрикатів на ринку;
обидві причини мусили спільно сприяти зниженню зиску, поки
лишались незмінними подання сировинного матеріалу і попит
на фабрикати; але вони вплинули ще далеко дужче, бо, з одного
боку, за останній час було недостатнє подання бавовни,
і, з другого боку, зменшився попит на фабрикати на різних
внутрішніх і закордонних ринках“ („Rep. of Insp. of Fact., Oct.
1846“ стор. 10).

\parcont{}  %% абзац починається на попередній сторінці
\index{iii2}{0136}  %% посилання на сторінку оригінального видання
становить 240\% замість 180\%. Увесь продукт збільшився з 10 до
36 квартерів.

Порівняно з І$b$, де загальне число оброблених акрів, застосований капітал
і ріжниці між обробленими родами землі лишились ті самі, але розподіл
їх інший, продукт становить 36 квартерів замість 26 кварт., пересічна рента
з акра становить 6\pound{ ф. стерл.} замість 3\sfrac{1}{2} і норма ренти щодо всього авансового
капіталу тієї самої величини — 240\% замість 140\%.

Хоч як би ми стали розглядати різні становища, подані в таблицях І$а$,
І$b$, І$с$, чи як становища, що одночасно існують одно біля одного в різних країнах,
чи як послідовні становища в тій самій країні, — однаково, виявиться таке:
за сталої ціни збіжжя, сталої тому, що продукт з найгіршої землі, яка не дає
ренти, лишається той самий; за незмінної різниці у родючости різних розрядів
оброблюваної землі; при відносно однаковій кількості продукту, а, отже,
при однаковій витраті капіталу на відповідно однакові частини (акри) земельної
площі, оброблюваної в кожному розряді; при сталому в наслідок цього
відношенні між рентами з акра кожного роду землі і при однаковій нормі ренти
на капітал, вкладений у кожну дільницю землі того самого роду — виявиться,
\emph{поперше}, що сума ренти завжди зростає, разом з поширенням оброблюваної
площі, а тому із збільшенням витрати капіталу, за винятком того випадку, коли
весь приріст припав би на землю, що не дає ренти. \emph{Подруге}, так пересічна
рента на акр (загальна сума ренти, поділена на все число оброблюваних
акрів), як і пересічна норма ренти (загальна сума ренти, поділена на ввесь
витрачений капітал) можуть значно варіювати, і хоч обидві в одному напрямку,
але в різних пропорціях у відношенні одна до однієї. Якщо не брати
на увагу того випадку, коли приріст відбувається лише на землі $А$,
яка не дає ренти, то виявляється, що пересічна рента на акр і пересічна,
норма ренти на капітал, вкладений у хліборобство, залежить від того, які пропорційні
частини всієї оброблюваної землі становлять землі різних розрядів;
або, що схопіть на те саме, від розподілу всього застосованого капіталу між
землями різної родючости. Чи багато, чи мало землі обробляється і тому (за
винятком того випадку, коли приріст припадає лише на $А$) чи більша, чи
менша є загальна сума ренти, пересічна рента на акр або пересічна рента на
застосований капітал лишається та сама, доки відношення різних родів оброблюваної
землі до всієї її площі лишається те саме. Дарма, що з поширенням
культури і збільшенням застосованого капіталу відбувається підвищення і
навіть значне підвищення загальної суми ренти, пересічна рента на акр і
пересічна норма ренти на капітал понижується, коли поширення земельних
дільниць, що не дають ренти, або дають лише незначну диференційну ренту,
зростає швидше, ніж поширення кращих земельних дільниць, що дають більшу
ренту. Навпаки, пересічна рента на акр і пересічна норма ренти на капітал підвищується в міру того,
як кращі землі починають становити відносно більшу
частину всієї площі, і тому на них припадає відносно більше застосованого
капіталу.

Таким чином, коли розглядати пересічну ренту на акр або гектар усієї
оброблюваної землі, як це звичайно робиться в статистичних працях при порівнянні
різних країн за тієї самої доби, або різних діб у тій самій країні, та
виявляється, що пересічна висота ренти на акр, а тому і загальна сума ренти
в певних пропорціях (хоч зовсім не в тих самих, а в швидше ростучих) відповідає
не відносній, а абсолютній родючості хліборобства в країні, тобто відповідає,
масі продуктів, одержуваній пересічно з однакової земельної площі. Бо що
більшу частину із загальної площі становлять кращі землі, то більша маса
продуктів, одержувана з земельної площі однакової величини за однакової величини
застосованого капіталу, і то більша пересічна рента на акр. Зворотне в зворотному
\index{iii2}{0137}  %% посилання на сторінку оригінального видання
випадку. Тому здається, що рента визначається не відношенням диференційної
родючости, а абсолютною родючістю, і що таким чином закон диференційної
ренти знищується. Тому деякі явища заперечуються, або їх намагаються
пояснити несущими різницями пересічних цін хліба і ріжницями диференційної
родючости оброблюваних дільниць землі, тимчасом, як ці явища ґрунтуються
просто на тому, що відношення загальної суми ренти так до всієї площі оброблюваної
землі, як і до всього капіталу, вкладеного в землю, за однакової родючости
землі, що не дає ренти, а тому і за однакових цін продукції і за
однакової ріжниці між землями різних родів визначаються не тільки рентою
на акр, або нормою ренти на капітал, але в такій же мірі відношенням числа
акрів кожного роду до загального числа оброблюваних акрів; або, що сходить
на те саме, розподілом усього застосованого капіталу між різними родами землі.
До цього часу на цю обставину, дивовижно, зовсім не звертали уваги. В усякому
разі виявляється, і це є важливе для нашого дальшого досліду, що відносна
висота пересічної ренти на акр і пересічна норма ренти, або відношення
загальної суми ренти до всього вкладеного в землю капіталу, може збільшуватися
або зменшуватися просто в наслідок екстенсивного поширення культури,
за незмінних цін, незмінної ріжниці в родючості оброблюваних дільниць землі
і незмінної ренти з акра, або норми ренти на капітал, вкладений в акр
кожного розряду землі, що дійсно дає ренту, або на ввесь капітал, що дійсно
дає ренту.

\pfbreak

Треба зробити ще такі доповнення щодо тієї форми диференційної ренти,
яка досліджена в нас під рубрикою І, і що почасти мають також значення і
для диференційної ренти II.

\emph{Перше:} Ми бачили, як пересічна рента з акра або пересічна норма
ренти на капітал може підвищитись з поширенням культури, за сталих цін і
незмінної ріжниці в родючості оброблюваних земельних дільниць. Скоро вся
земля в будь-якій країні буде привласнена, вкладення капіталу в землю, культура
і людність досягнуть певної висоти — обставини, наявність яких доводиться
припускати, скоро капіталістичний спосіб продукції став панівним і упідлеглив
собі і хліборобство, — ціна необроблюваної землі різної якости (просто припускаючи
існування диференційної ренти) визначається ціною оброблюваних дільниць
землі однакової якости і рівноцінного положення. Ціна є така сама — за вирахуванням
витрат на обробіток, що приєднується до неї — хоч ця земля і не
дає ренти. Ціна землі, звичайно, є не що інше, як капіталізована рента. Але
і в ціні оброблених земельних дільниць оплачуються лише майбутні ренти, наприклад,
одним заходом виплачується наперед ренти за 20 років, коли міродайний
розмір проценту є 5\%. Коли продається земля, вона продається як така, що дає
ренту, і перспективний характер ренти (яку розглядається тут як витвір землі, чим
вона є тільки з видимости) призводить до того, що необроблювана земля не
відрізняється від оброблюваної. Ціна необроблюваних дільниць землі, як і рента
з них, — а ціна становить лише зосереджену формулу ренти — має суто ілюзорний
характер, поки ці дільниці не будуть дійсно використані. Але вона, таким
чином, визначається a priori і реалізується, скоро знаходяться покупці. Тому,
коли дійсна пересічна рента в певній країні визначається дійсною пересічною
річною сумою ренти і відношенням цієї останньої до всієї оброблюваної площі,
то ціна необроблюваної частини земельної площі визначається ціною оброблюваної
і є тому лише відбиток вкладення капіталу та його наслідків на оброблюваних
земельних дільницях. А що за винятком найгіршої землі, землі усіх родів
дають ренту (а ця рента, як ми побачимо в рубриці II, зростає з масою капіталу
а з відповідною до цієї маси інтенсивністю культури), то і створюється таким чином
\parbreak{}  %% абзац продовжується на наступній сторінці

\parcont{}  %% абзац починається на попередній сторінці
\index{iii1}{0138}  %% посилання на сторінку оригінального видання
для грубих нумерів пряжі і важких тканин\dots{} Є побоювання, що
збільшена кількість машин, недавно установлених у камвольній
промисловості, приведе до подібної реакції і в цій галузі промисловості.
Пан Бекер обчислює, що в самому лиш 1849 році в цій
галузі промисловості продукт ткацьких верстатів збільшився на
40\%, продукт веретен на 25--30\%, а розширення промисловості
все ще продовжується в тих самих розмірах“ („Rep. of Insp.
of Fact., April 1850', стор. 54).

1850 рік. Жовтень. „Ціна бавовни продовжує\dots{} викликати
значну пригніченість в цій галузі промисловості, особливо для
таких товарів, для яких сировинний матеріал становить значну
частину витрат виробництва. Значний ріст ціни на шовк-сирець
часто приводив до пригнічення і в цій галузі“ („Rep. of Insp. of
Fact., Oct. 1850“, стор. 14). — За цитованим тут звітом комітету
королівського товариства культури льону в Ірландії, висока
ціна льону при низьких цінах інших сільськогосподарських продуктів
забезпечила тут значне розширення виробництва льону
для наступного року (стор. [31] 33).

\looseness=-1
1853 рік. Квітень. Великий розквіт. „Ніколи ще за ті 17 років,
протягом яких мені офіціально доводилось знайомитися з станом
фабричної округи Ланкашіра, я не спостерігав такого загального
процвітання; діяльність по всіх галузях надзвичайна“, —
каже Л.~Горнер („Rep. of Insp. of Fact., April 1853“, стор. 19).

1853 рік. Жовтень. Депресія в бавовняній промисловості.
„Перепродукція“ („Rep. of Insp. of Fact., Oct. 1853“, стор. [13] 15).

1854 рік. Квітень. „Шерстяна промисловість, хоч справи в ній
йшли не жваво, повністю завантажила всі фабрики; так само
й бавовняна промисловість. Камвольна промисловість протягом
цілого минулого півріччя всюди працювала нереґулярно\dots{} В лляній
промисловості відбувалися порушення в наслідок зменшеного
подання льону й коноплі з Росії в зв’язку з кримською війною“
(„Rep. of Insp. of Fact., [April] 1854“, стор. 37).

1859 рік. „Справи в шотландській лляній промисловості все
ще в пригніченому стані\dots{} бо сировинний матеріал рідкий і дорогий;
погана якість торішнього урожаю в прибалтійських країнах,
звідки йде до нас головний довіз, справлятиме шкідливий вплив
на промисловість цієї округи; навпаки, джут, який в багатьох
грубих товарах помалу витискує льон, не є ні надзвичайно дорогий,
ні рідкий\dots{} приблизно половина машин в Денді пряде
тепер джут“ („Rep. of Insp. of Fact., April 1859“, стор. 19). —
„В наслідок високої ціни сировинного матеріалу льонопрядіння
все ще зовсім невигідне, і в той час, як усі інші фабрики працюють
повний час, ми маємо ряд прикладів спинення машин,
що перероблюють льон\dots{} Прядіння джуту\dots{} перебуває в більш
задовільному стані, бо за останній час ціна на цей матеріал
стала помірнішою“ (Rep. of Insp. of Fact., Oct. 1859“, стор. 20).

\parcont{}  %% абзац починається на попередній сторінці
\index{iii2}{0139}  %% посилання на сторінку оригінального видання
ринок в готовому вигляді ті продукти, от-як одіж, знаряддя тощо, які в інших
обставинах їм довелося б виробляти самим. Тільки на такій базі південні штати
Союзу і могли зробити бавовну своїм головним продуктом. Поділ праці на світовому
ринку дає їм цю можливість. Коли тому здається, що вони, беручи
на увагу їхню молодість і відносну нечисленність людности, продукують дуже
великий надмірний продукт, то це не в наслідок родючости їхнього ґрунту і не
в наслідок продуктивности їхньої праці, а однобічної форми їхньої праці, отже,
і того надмірного продукту, в якому ця праця виявляється.

Але далі, відносно менш родюча орна земля, що почала оброблятись уперше
і ще не була зачеплена будь-якою культурою, за більш-менш сприятливих
кліматичних умов має, принаймні у верхніх шарах, так багато легко розчинюваних
речовин, живних для рослин, що вона довгий час дає урожай, без будьякого
добрива, до того ж при найповерховішім обробітку. Щодо західніх прерій,
то сюди приєднується ще те, що на їх обробіток майже не треба підготовчих
витрат, бо вони придатні для оброблення вже з природи\footnote{
[Саме швидкий зріст оброблення таких прерій і степових місцевостей за останнього часу і звів до
рівня дитячого жарту славнозвісну засаду Мальтуса: «Людність тисне на засоби існування», і в
протилежність цьому породив скарги аграріїв на те, що хліборобство, а разом з ним і Німеччина
загинуть, коли насильницькими заходами не усунути засобів існування, що тиснуть на людність. Але
обробіток цих степів, прерій, пампасів і льяносів тощо ще тільки починається; тому його
революціонізаційний вплив на европейське сільське господарство згодом дасться в знаки цілком інакше,
ніж до цього часу. — \emph{Ф.~Е.}].
}.. В менш родючих
краях цього роду надмір походить не з високої родючости ґрунту, тобто не з
високого продукту на акр, а з великої кількости акрів, які можуть бути поверхово
оброблені, бо сама ця земля або нічого не коштує хліборобові, або проти
старих країн коштує надзвичайно мало. Наприклад, там, де існує відчастинна
(Métairie) оренда, як в декотрих частинах Ныо-Йорку, Мічіґену, Канади, тощо.
Одна родина поверхово обробляє, скажімо, 100 акрів, і хоч кількість продукту,
що дає акр, невелика, з 100 акрів це дає значний надмір для продажу. До
цього приєднується ще майже дарове утримання худоби на природних пасовиськах,
без штучних лук. Переважне значіння має тут не якість, а кількість землі.
Можливість такого поверхового обробітку, природно, більш або менш швидко
вичерпується, в зворотному відношенні до родючости нової земли і в прямому відношенні до вивозу її
продукту. «А проте така земля дає чудові перші врожаї, навіть
пшениці; той, хто бере перший взяток з землі, може послати на ринок великий
надмір пшениці» (1. c, р. 224). В країнах старої культури відносини власности,
ціна необроблюваної землі, визначувана ціною оброблюваної тощо, унеможливлюють
таке екстенсивне господарство.

Але, що — всупереч думці Рікардо — ця земля не повинна неодмінно бути
дуже родюча, а також не має потреби в тому, щоб оброблювались лише землі,
однакові своєю родючістю, це видно з такого: в штаті Мічіґені 1848 року засіяно
пшеницею \num{463.900} акрів і випродуковано \num{4.739.300} бушелів, або пересічно по
10\sfrac{1}{5} бушеля на акр; по вирахуванні насіння це дає менш за 9 бушелів на
акр. З 29 округ штату 2 продукували пересічно 7 бушелів, 3--8, 2--9, 7--10, 6--11, 3--12, 4--13 бушелів, і
лише одна округа — 16 бушелів і ще одна — 18 бушелів на акр (1. c., р. 226).

Для хліборобської практики більша родючість ґрунту збігається з можливістю
більшого негайного використання цієї родючости. Можливість ця може бути
більша щодо бідного з природи ґрунту, ніж щодо ґрунту з природи багатого;
але це той сорт землі, що до нього колоніст візьметься насамперед і за браком
капіталу мусить до нього взятись.

\emph{Нарешті}, поширення культури на більші площі — лишаючи осторонь щойно розглянений випадок, коли
доводиться звертатися до землі гіршої якости,
\parbreak{}  %% абзац продовжується на наступній сторінці

\parcont{}  %% абзац починається на попередній сторінці
\index{iii2}{0140}  %% посилання на сторінку оригінального видання
ніж та, яка оброблялася до того часу, — на різних родах землі, починаючи з А
і до D, отже, наприклад, обробіток більших площ В і C, зовсім не має за свою передумову
попереднього підвищення цін хліба, подібно до того, як щорічне поширення,
наприклад, бавовнопрядіння не потребує постійного підвищення цін пряжі. Хоч
значне підвищення або пониження ринкових цін впливає на розмір продукції,
проте, залишаючи це осторонь, і при пересічних цінах, що своїм рівнем не справляють
на продукцію ні пригнобного, ні особливо оживного впливу, у хліборобстві
(як і в усіх інших галузях продукції, проваджених капіталістично) постійно
відбувається та відносна перепродукція, яка сама по собі тотожня з
акумуляцією і яка, за інших способів продукції, безпосередньо спричинюється
зростом людности, а в колоніях — постійною іміграцією. Попит постійно зростає,
і передбачаючи це, постійно вкладають в нові землі все нові й нові капітали;
хоч, залежно від обставин, капітали вкладають на створення різних хліборобських
продуктів. До цього само по собі призводить наростання нових капіталів.
Щодо окремого капіталіста, то розміри своєї продукції він припасовує до розміру
капіталу, що він ним порядкує, оскільки сам він може ще його контролювати.
Він прагне лише того, щоб захопити якомога більше місця на ринку.
Коли настає перепродукція, то він обвинувачує в цьому не себе, а своїх конкурентів.
Окремий капіталіст може розширювати свою продукцію так привласнюючи
собі порівняно більшу відповідну частину даного ринку, як і розширюючисамий
ринок.

\section{Друга форма диференційної ренти.
(Диференційна рента II)}

До цього часу ми розглядали диференційну ренту лише як наслідок різної
продуктивности однакових капіталовкладень на однакових площах землі з різною
родючістю, так що диференційна рента визначалась ріжницею між продуктом
капіталу, вкладеного в найгіршу землю, що не дає ренти, і продуктом капіталу,
вкладеного в кращу землю. При цьому ми мали одночасне приміщення капіталів
в різні дільниці землі, так що кожному новому приміщенню капіталу відповідало
поширення обробітку землі, збільшення оброблюваної площі. Але, кінець-кінцем,
диференційна рента по суті справи була лише наслідком різної
продуктивности рівних капіталів, вкладених в землю. Чи буде будь-яка ріжниця,
коли капітали різної продуктивности вкладаються один після одного в ту
саму дільницю землі, і коли вони вкладаються один поряд одного в різні дільниці
землі — чи буде якась ріжниця, коли тільки припустити, що наслідки ті самі?

Насамперед, не можна заперечувати, що, оскільки справа йде про створення
надзиску, цілком байдуже, чи дадуть 3 ф. ст. ціни продукції, вкладені в акр
землі А, продукт в 1 квартер, так що 3 ф. ст. будуть ціною продукції і регуляційною
ринковою ціною одного квартера, тоді як 3 ф. ст. ціни продукції на акрі землі
В дадуть 2 квартери і, таким чином, надзиск в 3 ф. ст., а 3 ф. ст. ціни продукції
на акрі землі C дадуть 3 квартери і 6 ф. ст. надзиску і, нарешті, 3 ф. ст. ціни
продукції на акрі землі D дадуть 4 квартери і 9 ф. ст. надзиску; чи такий самий
наслідок буде досягнутий тим, що ці 12 ф. ст. ціни продукції, або 10 ф. ст.,
капіталу будуть вкладені з таким самим успіхом, в такій самій послідовності,
в один і той самий акр. І в тому, і в тому випадку це капітал в 10 ф. ст.
що частини його вартости, послідовно вкладені по 2\sfrac{1}{2} ф. ст., — однаково, чи вкладаються
вони один поряд одного на 4 акрах землі різної родючости, чи послідовно на
тому самому акрі, — в наслідок того, що продукт їхній різний, однією своєю частиною
\parbreak{}  %% абзац продовжується на наступній сторінці

\parcont{}  %% абзац починається на попередній сторінці
\index{iii2}{0141}  %% посилання на сторінку оригінального видання
не дають надзиску, тимчасом як їхні інші частини дають надзиск, відповідний
ріжниці між їхнім продуктом і продуктом того вкладення, що не дає ренти.

Надзиски і різні норми надзиску з різних частин вартости капіталу створюються
в обох випадках рівномірно. А рента є не що інше, як форма цього
надзиску, який становить її субстанцію. Але в усякому разі другий спосіб являє
собою труднощі щодо перетворення надзиску в ренту, цієї зміни форми, яка містить
в собі перенесення надзиску від капіталістичного орендаря до земельного власника.
Звідси упертий опір офіційній хліборобській статистиці з боку англійських
орендарів. Звідси боротьба між ними й землевласниками за встановлення дійсних
наслідків приміщення їхніх капіталів (Morton). Рента встановлюється саме
при оренді земель, і після цього надзиски, що постають з послідовного приміщення
капіталу, потрапляють в кишеню орендаря весь час, поки триває орендний
договір. Звідси боротьба орендарів за тривалі орендні договори і навпаки,
в наслідок переваги сили лендлордів, збільшення числа контрактів, які можна
щороку скасовувати (tenancies at will).

Тому ясно з самого початку: хоч для закону створення надзиску нічого не
змінюється від того, чи вкладено рівні капітали з різними наслідками один поряд
одного в рівновеликі земельні дільниці, чи вкладено їх послідовно один за одним в ту
саму дільницю землі, — проте, це становить значну ріжницю для перетворення
надзиску в земельну ренту. Останній спосіб замикає це перетворення, з одного
боку, у вужчі, з другого — у мінливіші межі. Тому в країнах інтенсивної культури
(а економічно під інтенсивною культурою ми розуміємо не що інше, як
концентрацію капіталу на тій самій земельній площі, замість розподілу його
між земельними дільницями, що лежать одна біля однієї) праця таксатора, як
це зазначає Morton у своїх «Resources of States», стає дуже важливою, складною
і важкою професією. При триваліших поліпшеннях землі, коли минає термін
орендного договору, штучно підвищена диференційна родючість землі збігається
з її природною, а тому і оцінка розміру ренти збігається з оцінкою розміру
ренти від земель різної родючости взагалі. Навпаки, оскільки створення надзиску
визначається висотою капіталу, вкладеного в продукцію, висота ренти, одержуваної
при певній величині цього капіталу, приєднується до пересічної ренти
країни і тому дбають про те, щоб новий орендар порядкував капіталом, достатнім
для продовження культури з колишнім ступенем інтенсивности.

\pfbreak

При розгляді диференційної ренти II треба відзначити ще такі пункти:

\emph{Поперше}: Її база і вихідний пункт, не тільки історично, але й оскільки
справа йде про рух за всякого даною моменту, є диференційна рента I, тобто
одночасний обробіток розміщених одна поряд однієї земельних дільниць, різних
своєю родючістю і положенням; отже одночасне вживання одної поряд однієї
різних складових частин усього хліборобського капіталу на земельних дільницях
різної якости.

Історично це само собою зрозуміло. В колоніях колоністам доводиться прикладати
лише незначний капітал; за головних аґентів продукції є праця і земля.
Кожен окремий голова родини намагається добитися для себе і своїх самостійного
поля дії, поряд з товаришами-колоністами. У власне хліборобстві це
взагалі мусило так відбуватися вже за докапіталістичних способів продукції. При
вівчарстві і взагалі скотарстві як самостійних галузях продукції земля експлуатується
більш або менш спільно, і з самого початку експлуатація має екстенсивний
характер. Капіталістичний спосіб продукції походить з давніших способів
продукції, за яких засоби продукції, фактично або юридично, становлять
власність самого обробника, словом, з ремісничої продукції в хліборобств. По
суті справи з ремісничої продукції лише поступово розвивається концентрація
засобів продукції і перетворення їх на капітал, що протистоїть безпосереднім
\parbreak{}  %% абзац продовжується на наступній сторінці

\parcont{}  %% абзац починається на попередній сторінці
\index{iii2}{0142}  %% посилання на сторінку оригінального видання
продуцентам, перетвореним в найманих робітників. Оскільки капіталістичний
спосіб продукції виступає тут з своїми характеристичними особливостями,
відбувається це спочатку поперше і особливо в галузі вівчарства і скотарства;
але далі це виявляється не в концентрації капіталу на відносно невеликій
площі землі, а в продукції в більшому маштабі, що дає заощадження на
утриманні коней і інших витратах продукції; але в дійсності тут немає вживання
більшого капіталу на тій самій землі. Далі, в природних законах хліборобства
є те, що за певної висоти культури і відповішого їй виснаження
ґрунту, капітал, що його тут розуміється також, як уже вироблені засоби
продукції, стає вирішальним елементом хліборобства. Поки оброблювана
земля становить відносно невелику площу проти необробленої, і сила землі ще
не виснажена, (а таке становище маємо при перевазі скотарства і м’ясної їжі
за періоду, що передує періодові переваги власне хліборобства і рослинної їжі)
народжуваний новий спосіб продукції протистоїть селянській продукції саме розміром
земельної площі, що обробляється коштом одного капіталіста, отже, знов таки
екстенсивним вживанням капіталу на земельній площі більшого простору. Таким
чином, від самого початку слід установити, що диференційна рента І є тією
історичною основою, яка править за вихідний пункт. З другого боку, рух диференційної
ренти II для кожного даного моменту починається лише в такій царині,
яка сама знову-таки становить мозаїкову основу для диференційної ренти І.

\emph{Подруге}. При диференційній ренті в формі II до ріжниці родючости приєднуються
ріжниці в розподілі капіталу (і кредитоспроможности) між орендарями.
У власне мануфактурі для кожної галузі продукції швидко створюється особливий
мінімум розміру підприємства і відповідно до цього мінімум капіталу,
без якого не можна успішно провадити окремі підприємства. Так само в кожній
галузі продукції створюється нормальний пересічний розмір капіталу, який
перевищує цей мінімум і яким мусять порядкувати і порядкують більшість продуцентів.
Капітал більший понад цей розмір може дати надзиск; менший — не
дає і пересічного зиску. Капіталістичний спосіб продукції лише повільно і нерівномірно
охоплює сільське господарство, як це можна спостерігати в Англії,
клясичній країні капіталістичного способу продукції в хліборобстві. Коли немає
вільного довозу збіжжя, або його дія лише обмежена, бо розмір довозу обмежений,
то ринкову ціну визначають продуценти, що працюють на гіршій землі,
отже, в умовах продукції несприятливіших, ніж пересічні. Більша частина загальної
маси капіталу, який вживається в хліборобстві і яким воно взагалі
порядкує, перебуває в їхніх руках.

Вірно, що селянин, наприклад, витрачає багато праці на свою маленьку
парцелю. Але праці ізольованої і позбавленої об’єктивних, так суспільних, як і
матеріяльних умов продуктивности, — праці від цих умов відлученої.

Ця обставина призводить до того, що справжні капіталістичні орендарі
мають змогу привласнювати собі частину надзиску; це відпало б, принаймні,
оскільки справа торкається цього пункту, тоді, коли капіталістичний спосіб продукції
був би так само рівномірно розвинений у сільському господарстві, як у
мануфактурі.

Розгляньмо насамперед створення надзиску при диференційній ренті II, не
торкаючись умов, за яких може відбутися перетворення цього надзиску в земельну
ренту.

Тоді ясно, що диференційна рента II є лише інший вираз диференційної
ренти І, а по суті збігається з нею. Різна родючість різних земель впливає при
створенні диференційної ренти. І лише остільки, оскільки вона призводить до
того, що вкладений в землю капітал дає неоднакові наслідки, неоднакову кількість
продуктів на капітали однакової величини, або пропорційно до їхньої відносної
величини. Чи постає ця нерівність для різних капіталів, вкладуваних
\parbreak{}  %% абзац продовжується на наступній сторінці

\parcont{}  %% абзац починається на попередній сторінці
\index{iii2}{0143}  %% посилання на сторінку оригінального видання
один по одному на тій самій дільниці землі, чи для капіталів, що застосовуються
до декількох дільниць різнорідної землі, — це нічого не може змінити в ріжниці
родючости або в кількості її продукту, а тому і в створенні диференційної
ренти з частин капіталу, вкладених з більшою родючістю. За рівних капіталовкладень
різну родючість, як і давніш, виявляє земля, але тут та ж сама земля,
за послідовних вкладень різних частин капіталу, дає те, що при диференційній
ренті І дають різні роди землі при вкладенні в них різних частин суспільною
капіталу однакової величини.

Коли той самий капітал в 10\pound{ ф. ст.}, що в таблиці І вкладається різними орендарями
у вигляді самостійних капіталів по 2\sfrac{1}{2}\pound{ ф. ст.} в кожен акр чотирьох родів
землі $А$, $В$, $C$ і $D$, замість цього вкладався б послідовно у той самий акр
$D$, так що перше вкладення дало б 4 квартери, друге — 3, третє — 2 і останнє
— 1 квартер (абож у зворотній послідовності), то ціна одного квартера, одержуваного
від найменш дохідної частини капіталу, рівна 3\pound{ ф. ст.}, не давала б диференційної
ренти, але визначала б ціну продукції до того часу, поки зберігається
потреба в довозі пшениці, ціна продукції якої є 3\pound{ ф. ст}. А що згідно з припущенням
продукція провадиться капіталістично і, отже, ціна в 3\pound{ ф. ст.} має
в собі пересічний зиск, що його взагалі дає капітал в 2\sfrac{1}{2}\pound{ ф. ст.}, то три інші
частини по 2\sfrac{1}{2}\pound{ ф. ст.} даватимуть надзиски залежно від ріжниці в кількості
продукту, бо цей продукт продається не по його ціні продукції, а по ціні продукції
найменш дохідного вкладення капіталу в 2\sfrac{1}{2}\pound{ ф. ст.}, вкладення, яке не дає ренти
і ціна продукту якого реґулюється за загальними законами цін продукції. Створення
надзиску було б таке саме, як у таблиці І.

Тут знову виявляється, що диференційна рента II має за свою передумову
диференційну ренту І.~Мінімум продукту, що дає капітал в 2\sfrac{1}{2}\pound{ ф. ст.},
тобто капітал, вкладений в найгіршу землю, ми взяли тут в 1 квартер.
Отже, ми припускаємо, що орендар землі $D$, крім тих 2\sfrac{1}{2}\pound{ ф. ст.}, які
дають йому 4 квартери, за що він виплачує 3 квартери диференційної ренти, вживає
на тій самій землі 2\sfrac{1}{2}\pound{ ф. ст.}, що дають йому лише 1 квартер, як це дав
би той самий капітал на
найгіршій землі $А$. В такому випадку це було б таке приміщення капіталу,
що не дає ренти, бо воно дало б орендареві тільки пересічний зиск. Тут не
було б жодного надзиску, щоб перетворитись на ренту. Але, з другого боку, це
зменшення продукту від другого вкладення капіталу в $D$ не зробило б жодного
впливу на норму зиску. Це було б те саме, як коли б 2\sfrac{1}{2}\pound{ ф. стерл.} були
знову вкладені в якийсь новий акр землі сорту $А$, — обставина, яка жодним способом
не зачіпає надзиску, а, отже, і диференційної ренти з земель $А$, $В$, $C$, $D$.
Але для орендаря це додаткове вкладення 2\sfrac{1}{2}\pound{ ф. стерл.} у $D$ було б вигідно
в такій самій мірі, як, згідно з нашим припущенням, вкладення первісних
2\sfrac{1}{2}\pound{ ф. стерл.} у акр землі $D$, хоч вона дає 4 квартери. Коли б далі два наступні
вкладення капіталу по 2\sfrac{1}{2}\pound{ ф. стерл.} кожне дали йому: перше 3, а друге
2 квартери додаткової продукції, то тут знову відбулося б зменшення продукту
порівняно з продуктом від першого вкладення в 2\sfrac{1}{2}\pound{ ф. стерл.} у $D$, яке
дало 4 квартери, а тому 3 квартери надзиску. Але це було б лише зменшенням
величини надзиску і не зачепило б ні пересічного зиску, ні регуляційної ціни
продукції. Останнє могло б статися лише в тому випадку, коли б додаткова
продукція, яка дає понижений надзиск, зробила б зайвою продукцію з $А$ і таким
чином виключила б $А$ з числа оброблюваних земель. В цьому випадку
з пониженням родючости додаткового вкладення капіталу в акр землі $D$ було б
зв’язане падіння ціни продукції, наприклад, з 3\pound{ ф. стерл.} до 1\sfrac{1}{2}\pound{ ф. стерл.},
коли б акр землі $В$ зробився землею, що не дає ренти, отже, реґулює ринкову ціну.

Продукт з $D$ був би тепер $= 4 \dplus{} 1 \dplus{} 3 \dplus{} 2 \deq{} 10$ квартерам, тимчасом
як раніш він дорівнював 4 квартерам. Але ціна квартера, регульована $В$, впала б
до 1\sfrac{1}{2}\pound{ ф. стерл}. Ріжниця між $D$ і $В$ була б $= 10 - 2 \deq{} 8$ квартерів, що по
\parbreak{}  %% абзац продовжується на наступній сторінці

\parcont{}  %% абзац починається на попередній сторінці
\index{iii2}{0144}  %% посилання на сторінку оригінального видання
1\sfrac{1}{2}\pound{ ф. стерл.} за квартер становило б 12\pound{ ф. стерл.}, тимчасом як грошова рента
з D давніш дорівнювала 9\pound{ ф. стерл}. Це слід відзначити. Коли зробити розрахунок
на акр, то висота ренти піднеслась би на 33\sfrac{1}{3}\%, не зважаючи на зменшування
норми надзиску з 2 додаткових капіталів по 2\sfrac{1}{2}\pound{ ф. стерл.} кожен.

Звідси видно, до яких надзвичайно складних комбінацій призводить диференційна
рента взагалі, і особливо в її формі II в зв’язку з формою І, тимчасом
як Рікардо, наприклад, трактує її цілком однобічно і як річ просту. Буває, наприклад,
як вище наведено, падіння реґуляційної ринкової ціни і одночасно зріст
ренти на родючих землях, так що зростає як абсолютний продукт, так і
абсолютний надпродукт. (При диференційній ренті І по низхідній лінії може
зростати відносний надпродукт, а тому й рента з акра, хоч абсолютний надпродукт
з акра лишається той самий або навіть зменшується). Але одночасно
зменшується продуктивність капіталовкладень, що їх роблять одне по одному
на тій самій землі, хоч значна частина їх і припадає на родючіші землі. Коли
дивитися з одного погляду — з погляду кількости продукту і цін продукції —
продуктивність праці зросла. Але з другого погляду вона зменшилась, бо норма
надзиску і надпродукт на акр для різних капіталовкладень на тій самій землі
зменшились.

Диференційна рента II за зменшеної продуктивности послідовних приміщень
капіталу тільки тоді була б неодмінно зв’язана з подорожченням ціни продукції
і абсолютним зменшенням продуктивности, коли б ці приміщення капіталу
могли бути зроблені виключно на гіршій землі $А$. Коли акр землі $А$, що
при вкладенні капіталу в 2\sfrac{1}{2}\pound{ ф. стерл.} давав 1 квартер по ціні продукції
в 3\pound{ ф. стерл.}, при дальшому вкладенні в 2\sfrac{1}{2}\pound{ ф. стерл.}, тобто при загальному
вкладенні в 5\pound{ ф. стерл.}, дає сукупно лише 1\sfrac{1}{2} квартера, то ціна продукції цих
1\sfrac{1}{2}, квартерів \deq{} 6\pound{ ф. стерл.}, а тому 1 квартера \deq{} 4\pound{ ф. стерл}. Всяке пониження
продуктивности при ростучому вкладенні капіталу було б тут відносним зменшенням
продукту з акра, тимчасом як на землі кращих сортів воно є лише
зменшення надмірного надпродукту.

Але сама природа справи призводить до того, що з розвитком інтенсивної
культури, тобто послідовних вкладень капіталу в ту саму землю, ці вкладення
відбуваються переважно або в більшій мірі на землях кращих родів. (Ми не говоримо
тут про ті тривалі поліпшення, за допомогою яких землі, що були непридатні,
перетворюються в придатні). Зменшувана продуктивність послідовних
витрат капіталу мусить, отже, діяти переважно вищезазначеним чином. Найкращу
землю вибирається тут тому, що вона дає найбільшу надію на рентабельність
від застосованого на ній капіталу, бо має в собі найбільшу кількість природних
елементів родючости, що їх треба лише використати.

Коли після скасування хлібних законів культура в Англії зробилась ще
інтенсивніша, масу земель, на яких до того сіяли пшеницю, було використано
з іншою метою, а саме, як пасовиська; навпаки, родючі простори землі, найпридатніші
для сіяння пшениці, були дреновані та іншим способом поліпшені.
Таким чином капітал, що вживався у виробленні пшениці, був сконцентрований
на меншій дільниці.

В цьому випадку — а всі можливі додаткові норми, що містяться між найбільшою
кількістю надпродукту з кращої землі і кількістю продукту землі
$А$, що не дає ренти, відповідають тут не відносному, а абсолютному збільшенню
надпродукту на акр — новоутворений надзиск (евентуальна рента) становить не
перетворену на ренту частину колишнього пересічного зиску (частина продукту,
в якій раніше виявлявся пересічний зиск), а додатковий надзиск, який з цієї
форми перетворився на ренту.

Навпаки, тільки в тому випадку, коли попит на збіжжя збільшився
так, що ринкова ціна перевищила б ціну продукції на землі $А$, і тому
\parbreak{}  %% абзац продовжується на наступній сторінці

\parcont{}  %% абзац починається на попередній сторінці
\index{iii2}{0145}  %% посилання на сторінку оригінального видання
надпродукт з земель $А$, $В$ або якогось іншого розряду можна було б одержати
лише по вищій ціні, ніж 3\pound{ ф. стерл.}, — тільки в цьому випадку зі зменшенням
продукту від додаткової витрати капіталу на якийсь з розрядів $А$, $В$, $C$, $D$
було б зв’язане підвищення ціни продукції і реґуляційної ринкової ціни.
Коли б це усталилось на тривалий час і не спричинило б оброблення додаткової
землі $А$ (принаймні, якости $А$) і взагалі ніякі інші впливи не призвели б
до дешевшого подання, то, за інших незмінних умов, заробітна плата підвищилася б
в наслідок подорожчання хліба і відповідно до цього знизилась би норма зиску.
В цьому випадку було б байдуже, чи задовольнявся б підвищений попит втягненими
до обробітку гіршої, ніж $А$, землі, чи додатковим приміщенням капіталу
у будь-яку з чотирьох родів землі. Диференційна рента підвищилася б в зв’язку
з пониженням норми зиску.

Цей випадок, коли низхідна продуктивність додаткових капіталів, вкладуваних
в землі, що вже перебувають під культурою, може призвести до підвищення
ціни продукції, пониження норми зиску і створення вищої диференційної
ренти, — бо ця остання за даних умов підвищилася б на всіх родах землі
цілком так само, як коли б гірша, ніж $А$, земля стала тепер реґулювати ринкову
ціну, — цей випадок Рікардо перетворює в єдиний випадок, в нормальний
випадок, до якого він зводить все створення диференційної ренти II.

Так воно і було б, коли б оброблювалось лише рід землі $А$ і коли б послідовні
вкладення капіталу на ній не були зв’язані з пропорційним приростом
продукту.

Отже, тут у випадку з диференційною рентою II цілком губиться з пам’яти
диференційна рента І.

За винятком цього випадку, коли або подання з оброблюваних земель
недостатнє, і тому ринкова ціна довгочасно перевищує ціну продукції, поки не
почнеться оброблення нових додаткових земель гіршої якости, або поки ввесь продукт
додаткового капіталу, вкладеного в землю різних розрядів, не буде можливости
збувати по вищій ціні продукції, ніж суща до того часу, — за винятком
цього випадку пропорційне зменшення продуктивности додаткових капіталів не
зачіпає реґуляційної ціни продукції і норми зиску. А втім, можливі ще три
такі випадки:

а) Коли додатковий капітал, вкладений у землю якогось роду $А$, $В$, $C$, $D$,
дає лише норму зиску, визначувану ціною продукції на $А$, то через це не
створюється жодного надзиску, отже, і жодної можливої (евентуальної) ренти;
так само, як коли б почала оброблятися додаткова земля $А$.

в) Коли додатковий капітал дасть більшу кількість продукту, то, як само
собою зрозуміло, створюється новий надзиск (потенціяльна рента), якщо реґуляційна
ціна лишається колишня. Останнє не завжди так буває, саме не буває
тоді, коли ця додаткова продукція виключає землю $А$ з числа оброблюваних, а
разом з тим з числа конкурентних родів землі. В цьому випадку реґуляційна ціна
продукції знижується. Норма зиску підвищилась би, коли б з цим було зв’язане
зниження заробітної плати, або коли б дешевший продукт увійшов елементом
в сталий капітал. Коли б додаткові капітали дали вищу продуктивність на
землях кращих родів $C$ і $D$, то тільки від ступеня підвищення продуктивности
і від маси нововкладених капіталів залежало б, наскільки створення збільшеного
надзиску (отже, і збільшеної ренти) сполучалося б з пониженням ціни і підвищенням
норми зиску. Ця остання може підвищуватися і без пониження заробітної
плати, в наслідок здешевлення елементів сталого капіталу.

c) Якщо додаткове приміщення капіталу відбувається за зменшуваного
надзиску, але все-ж так, що його продукт дає надмір проти продукту такого ж
самого капіталу на землі $А$, то, якщо тільки збільшене подання не виключить
землі $А$ з числа оброблюваних земель, за всяких умов відбудеться створення
\parbreak{}  %% абзац продовжується на наступній сторінці

\parcont{}  %% абзац починається на попередній сторінці
\index{iii1}{0146}  %% посилання на сторінку оригінального видання
часто міняються, і заробіток робітників підвищується чи падає
залежно від якості бавовняної мішанки. Іноді лишалося тільки
15\% від колишнього заробітку, і за один чи два тижні він
падав на 50 або 60\%“. Інспектор Редгрев, що його ми тут
цитуємо, наводить взяті з практики дані про заробітну плату,
з яких тут досить буде таких прикладів:

$А$, ткач, сім’я з 6 осіб, занятий 4 дні на тиждень, 6 шилінгів
8\sfrac{1}{2}, пенсів; $В$, twister [присукальник], 4\sfrac{1}{2} дні на тиждень, 6 шилінгів;
$C$, ткач, сім’я з 4 осіб, 5 днів на тиждень, 5 шилінгів
1 пенс; $D$, slubber [тростильник], сім’я з 6 осіб, 4 дні на тиждень,
7 шилінгів 10 пенсів; $Е$, ткач, сім’я з 7 осіб, 3 дні на
тиждень, 5 шилінгів і т. д. Редгрев каже далі: „Вищенаведені
дані заслуговують уваги, бо вони показують, що для деяких
сімей робота була б нещастям, тому що вона не тільки скоротила
б їх дохід, але й знизила б його настільки, що його вистачило
б тільки на задоволення незначної частини абсолютно
необхідних потреб, якщо не давалося б додаткової допомоги
в тих випадках, коли заробіток сім’ї не досягає тієї суми, яку
вона одержувала б як допомогу, коли б усі члени сім’ї були
без роботи“ („Rep. of Insp. of Fact., Oct. 1863“, стор. 50--53).

„Починаючи з 5 червня 1863 року, не було жодного тижня,
на протязі якого весь робочий час усіх робітників становив би
пересічно більше двох днів 7 годин і кількох хвилин“ (там же,
стор. 121).

З початку кризи до 25 березня 1863 року майже три мільйони
фунтів стерлінгів було витрачено установами піклування
про бідних, центральним комітетом допомоги і лондонським
Mansion-House [муніципальним] комітетом (стор. 13).

„В одній окрузі, де випрядається найтонша пряжа\dots{} прядільники
підпали посередньому зниженню заробітної плати на 15\%
в наслідок переходу від Sea Island до єгіпетської бавовни\dots{}
В одній великій окрузі, де бавовняні відпади застосовуються
великими масами для домішки до індійської бавовни, заробітна
плата прядільників була знижена на 5\% і, крім того, вони
втратили ще 20--30\% в наслідок перероблення сурату й відпадів.
Ткачі, які працювали раніш коло чотирьох верстатів, перейшли
тепер на два верстати. В 1860 році вони виробляли на кожному
верстаті 5 шилінгів 7 пенсів, в 1863 році — тільки 3 шилінги
4 пенси\dots{} Грошові штрафи, які раніш, при застосуванні американської
бавовни, коливалися від 3 до 6 пенсів“ [для прядільників],
„доходять тепер до 1 шилінга — 3 шилінгів 6 пенсів“.
В одній окрузі, де вживалась єгіпетська бавовна, змішана
з ост-індською, „пересічна заробітна плата прядільника на мюлях
в 1860 році становила 18--25 шилінгів, а тепер 10--18 шилінгів.
Це викликано не самим тільки погіршенням бавовни,
але і зменшенням швидкості мюлів для того, щоб надати
пряжі дужчого крутіння, — за що в звичайні часи згідно з умовою
про заробітну плату платилося додатково“ (стор. 43, 44,
\parbreak{}  %% абзац продовжується на наступній сторінці

\parcont{}  %% абзац починається на попередній сторінці
\index{iii2}{0147}  %% посилання на сторінку оригінального видання
давніш 6\pound{ ф. стерл.} ренти до 2\sfrac{1}{2}\pound{ ф. стерл.} капіталу. Тут не виникає нових
ріжниць між витраченими капіталами, але виникають нові надзиски тільки
тому, що додатковий капітал вкладається в якусь з земель, що дають ренту,
або в усі землі, даючи при цьому пропорційно своїй величині той самий
продукт. Коли б подвійна витрата капіталу була зроблена, наприклад, лише
на $C$, то диференційна рента між $C$, $В$ і $D$, обчислена на капітал, залишалася б
та сама; бо хоч її маса з $C$ і подвоїлася б, але подвоївся б і вкладений
капітал.

Звідси видно, що за незмінної ціни продукції, незмінної норми зиску і
незмінних ріжниць (а тому і за незмінної норми надзиску або ренти, обчислених
на капітал), висота ренти, визначеної в продукті і в грошах, може підвищитись
з акра, а тому може підвищитись і ціна землі.

Те саме може статися при зменшуваних нормах надзиску, отже, і ренти,
тобто при зменшуваній продуктивності додаткових вкладень капіталу, що все
ще дають ренту. Коли б другі вкладення капіталу в 2\sfrac{1}{2}\pound{ ф. стерл.} не дали
подвоєного продукту, а дали б на $В$ лише З\sfrac{1}{2} квартери, на $C$ — 5 і на $D$ —
6 квартерів, то диференційна рента на $В$ для другого вкладення капіталу в 2\sfrac{1}{2} ф.
cтерл. була б лише \sfrac{1}{2} квартера замість 1, на C — 1 замість 2 і на $D$ — 2 замість
3 квартерів. Відношення між рентою і капіталом для обох послідовних
витрат було б таке:

\begin{table}[H]
  \centering
  \small
  \begin{tabular}{l l l}
   
  & Перша витрата & Друга витрата \\

$В$: & Рента 3\pound{ ф. стерл.}, капітал 2\sfrac{1}{2}\pound{ ф. стерл.} 
      & Рента 1\sfrac{1}{2}\pound{ ф. стерл.}, капітал 2\sfrac{1}{2}\pound{ ф. стерл.} \\

$C$: & \ditto{Рента} 6\ditto{\pound{ ф. стерл.}, капітал} 2\sfrac{1}{2} 
      & \ditto{Рента} З\phantom{\sfrac{1}{2}}\ditto{\pound{ ф. стерл.}, капітал} 2\sfrac{1}{2} \\

$D$: & \ditto{Рента} 9\ditto{\pound{ ф. стерл.}, капітал} 2\sfrac{1}{2}
      & \ditto{Рента} 6\phantom{\sfrac{1}{2}}\ditto{\pound{ ф. стерл.}, капітал} 2\sfrac{1}{2} \\
  \end{tabular}
\end{table}

\noindent{}Не зважаючи на таку понижену норму відносної продуктивности капіталу,
а тому і надзиску, обчисленого на капітал, збіжжева і грошова рента підвищилася
б для $В$ з 1 до 1\sfrac{1}{2} квартерів (з 3 до 4\sfrac{1}{2}\pound{ ф. стерл.}), для $C$ з 2 до 3 квартерів (з 6 до 9\pound{ ф. стерл.}) і для $D$ з 3 до 5 квартерів (з 9 до 15\pound{ ф. стерл.})
В цьому випадку ріжниці для додаткових капіталів порівняно з капіталом,
вкладеним в $А$, зменшилися б, ціна продукції лишилася б та сама, але рента
на акр, а тому і ціна землі на акр підвищилася б.

Щодо комбінацій диференційної ренти II, що має за свою передумову, як
свою базу диференційну ренту І, то вони такі.

\section{Диференційна рента II.~Перший випадок: стала ціна продукції}

Таке припущення включає й те, що ринкова ціна, як і давніше, регулюється
капіталом, вкладеним в найгіршу землю $А$.

I.~Коли додатковий капітал, вкладений в якусь із земель $В$, $C$, $D$, що
дають ренту, продукує лише стільки, скільки продукує такий самий капітал на
землі $А$, тобто коли при регуляційній ціні продукції він дає лише пересічний
зиск, не даючи, отже, жодного надзиску, то вплив справлений ним на ренту, дорівнює
нулеві. Все лишається, як було давніш. Це те саме, як коли б перше-ліпше
число акрів якости $А$, найгіршої землі, було приєднано до вже оброблюваної
площі.

II.~Додаткові капітали дають на землях усіх родів додаткові продукти в кількості,
пропорційній величині цих капіталів; тобто — величина продукції зростає,
\parbreak{}  %% абзац продовжується на наступній сторінці

\parcont{}  %% абзац починається на попередній сторінці
\index{iii2}{0148}  %% посилання на сторінку оригінального видання
залежно від специфічної родючости землі кожного типу, пропорційно величині
додаткового капіталу. В XXXIX розділі ми виходили з такої таблиці І:

\begin{table}[h]
  \begin{center}
    \emph{Таблиця І}
    \footnotesize

  \begin{tabular}{c c c c c c c c c c c}
    \toprule
      \multirowcell{2}{\makecell{Рід \\землі}} &
      \multirowcell{2}{\rotatebox[origin=c]{90}{Акри}} &
      \rotatebox[origin=c]{90}{Капітал} &
      \rotatebox[origin=c]{90}{Зиск} &
      \rotatebox[origin=c]{90}{\makecell{Ціна про- \\ дукції}} &
      \multirowcell{2}{\rotatebox[origin=c]{90}{\makecell{Продукт \\ в кварт.}}} &
      \rotatebox[origin=c]{90}{\makecell{Продажна \\ ціна}} &
      \rotatebox[origin=c]{90}{Здобуток} &
      \multicolumn{2}{c}{Рента} &
      \multirowcell{2}{\makecell{Норма \\надзиску}} \\

      \cmidrule(rl){3-3}
      \cmidrule(rl){4-4}
      \cmidrule(rl){5-5}
      \cmidrule(rl){7-7}
      \cmidrule(rl){8-8}
      \cmidrule(rl){9-10}

       &  &  ф. ст. & ф. ст. & ф. ст. & & ф. ст. & ф. ст. & Кварт. & ф. ст. &  \\
      \midrule

      A & 1 &  \phantom{0}2\sfrac{1}{2} & \sfrac{1}{2} & \phantom{0}3 & \phantom{0}1 & 3 & \phantom{0}3 & 0 & \phantom{0}0 & \phantom{00}0\\
      B & 1 &  \phantom{0}2\sfrac{1}{2} & \sfrac{1}{2} & \phantom{0}3 & \phantom{0}2 & 3 & \phantom{0}6 & 1 & \phantom{0}3 & 120\% \footnotemarkZ{}\\ % ця мітка у заголовку \\
      C & 1 &  \phantom{0}2\sfrac{1}{2} & \sfrac{1}{2} & \phantom{0}3 & \phantom{0}3 & 3 & \phantom{0}9 & 2 & \phantom{0}6 & 240\%\\
      D & 1 &  \phantom{0}2\sfrac{1}{2} & \sfrac{1}{2} & \phantom{0}3 & \phantom{0}4 & 3 & 12           & 3 & \phantom{0}9 & 360\%\\
     \cmidrule(rl){1-1}
     \cmidrule(rl){2-2}
     \cmidrule(rl){3-3}
     \cmidrule(rl){5-5}
     \cmidrule(rl){6-6}
     \cmidrule(rl){8-8}
     \cmidrule(rl){9-9}
     \cmidrule(rl){10-10}

     Разом & 4 & 10 & & 12 & 10 & & 30 & 6 & 18 &\\
  \end{tabular}

  \end{center}
\end{table}
\footnotetextZ{В німецькому тексті тут стоїть «12\%, 24\%, 36\%». Очевидна помилка. \Red{Прим. Ред.}} % текст примітки прямо під заголовком

Тепер ця таблиця перетворюється на:
\begin{table}[h]
  \begin{center}
    \emph{Таблиця ІI}
    \footnotesize

  \begin{tabular}{c c c c c c c c c c c}
    \toprule
      \multirowcell{2}{\makecell{Рід \\землі}} &
      \multirowcell{2}{\rotatebox[origin=c]{90}{Акри}} &
      Капітал &
      \rotatebox[origin=c]{90}{Зиск} &
      \rotatebox[origin=c]{90}{\makecell{Ціна про- \\ дукції}} &
      \multirowcell{2}{\rotatebox[origin=c]{90}{\makecell{Продукт \\ в кварт.}}} &
      \rotatebox[origin=c]{90}{\makecell{Продажна \\ ціна}} &
      \rotatebox[origin=c]{90}{Здобуток} &
      \multicolumn{2}{c}{Рента} &
      \multirowcell{2}{\rotatebox[origin=c]{90}{\makecell{Норма \\ надзиску}}} \\

      \cmidrule(rl){3-3}
      \cmidrule(rl){4-4}
      \cmidrule(rl){5-5}
      \cmidrule(rl){7-7}
      \cmidrule(rl){8-8}
      \cmidrule(rl){9-10}

       &  &  ф. ст. & ф. ст. & ф. ст. & & ф. ст. & ф. ст. & Кварт. & ф. ст. &  \\
      \midrule

      A & 1 & 2\sfrac{1}{2} \dplus{} 2\sfrac{1}{2} \deq{} 5 & 1 & 6 & \phantom{0}2 & 3 & \phantom{0}6 & \phantom{0}0 & \phantom{0}0 & \phantom{00}0\phantom{\%}\\
      B & 1 & 2\sfrac{1}{2} \dplus{} 2\sfrac{1}{2} \deq{} 5 & 1 & 6 & \phantom{0}4 & 3 & 12           & \phantom{0}2 & \phantom{0}6 & 120\% \\ % ця мітка у заголовку \\
      C & 1 & 2\sfrac{1}{2} \dplus{} 2\sfrac{1}{2} \deq{} 5 & 1 & 6 & \phantom{0}6 & 3 & 18           & \phantom{0}4 & 12 & 240\%\\
      D & 1 & 2\sfrac{1}{2} \dplus{} 2\sfrac{1}{2} \deq{} 5 & 1 & 6 & \phantom{0}8 & 3 & 25           & \phantom{0}6 & 18 & 360\%\\
     \cmidrule(rl){1-1}
     \cmidrule(rl){2-2}
     \cmidrule(rl){3-3}
     \cmidrule(rl){6-6}
     \cmidrule(rl){8-8}
     \cmidrule(rl){9-9}
     \cmidrule(rl){10-10}

     Разом & 4 & \phantom{2\sfrac{1}{2} \dplus{} 2\sfrac{1}{2} \deq{}}20 & & & 20 & & 60 & 12 & 36 &\\
  \end{tabular}

  \end{center}
\end{table}

Тут немає потреби в тому, щоб капітал вкладати у кожний з типів землі
в подвоєному розмірі, як це є в таблиці. Закон лишається той самий, скоро
тільки на якийсь один або декілька родів землі, що дають ренту, вжито додатковий капітал, хоч би в
якому розмірі. Треба лише, щоб продукція на землях
кожного роду збільшувалася в тому самому відношенні, в якому збільшується
капітал. Рента підвищується тут виключно в наслідок збільшення вкладеного
в землю капіталу і відповідно до цього збільшення капіталу. Це збільшення
продукту і ренти, в наслідок збільшення вкладеного капіталу і пропорційно
йому, є,  щодо кількости продукту і ренти, цілком таке саме, як у тому випадку,
коли оброблювана площа рівних за якістю дільниць землі, що дають ренту,
збільшилася б, оброблючись з такою самою витратою капіталу, з якою давніш оброблялись земельні
дільниці тієї самої якости. В випадку, поданому в таблиці II, наприклад, наслідок був би той самий,
коли б додатковий капітал,
в 2\sfrac{1}{2}\pound{ ф. стерл.} на акр було вкладено в другі акри земель $B$, $C$ і $D$.


\index{iii2}{0149}  %% посилання на сторінку оригінального видання
Цей випадок не припускає далі жодного продуктивнішого застосування
капіталу, а лише застосування більшого капіталу до тієї самої площі і з тими
самими наслідками, як і до того часу.

Усі відносні величини тут лишаються ті самі. Звичайно, коли розглядати
не відносні ріжниці, а суто аритметичні, то диференційна рента з різних земельможе
змінитися. Припустімо, наприклад, що додатковий капітал вкладено лише
в $В$ і $D$. Тоді ріжниця між $D$ і $А$ \deq{} 7 квартерам, давніш вона \deq{} 3; ріжниця
між $В$ і $А$ \deq{} 3 кварт., давніш вона \deq{} 1; ріжниця між $C$ і В $\deq{} - 1$, давніш
вона $= \dplus{} 1$ і~\abbr{т. ін.} Але ця аритметична ріжниця, вирішальна щодо диференційної
ренти І, оскільки в ній виражається ріжниця в продуктивності за однакового
розміру вкладеного капіталу, тут цілком не має ваги, бо вона є лише
наслідок того, чи вкладено, чи ні різні додаткові капітали, за незмінної ріжниці
для кожної рівної частини капіталу на ріжних дільницях.

IIІ.~Додаткові капітали дають надмірний продукт і створюють тому надзиски,
але при понижуваній нормі, не пропорційно їхньому збільшенню.

\begin{table}[H]
  \centering
  \caption*{Таблиця ІІІ}
  \footnotesize

  \settowidth\rotheadsize{\theadfont Продажна}
  \begin{tabular}{l c r c c c c c c c c}
    \toprule
      \thead[tl]{Рід\\землі} &
      &
      \thead[t]{Капітал} &
      \rothead{Зиск} &
      \rothead{Ціна\\продукції} &
      \rothead{Продукт} & % \\ в кварт.}}} \\ в кварт.}}}
      \rothead{Продажна\\ціна} &
      \rothead{Здобуток} &
      \multicolumn{2}{c}{Рента} &
      \rothead{Норма\\надзиску} \\

      \cmidrule(rl){2-11}

       & акри  & \poundsign{} & \poundsign{} & \poundsign{} & кв. & \poundsign{} & \poundsign{} & кв. & \poundsign{}  & \% \\
      \midrule

      A & 1 & \phantom{2\sfrac{1}{2} \dplus{} 2\sfrac{1}{2} \deq{}} 2\sfrac{1}{2} & \phantom{0}\sfrac{1}{2} & \phantom{0}3 & \phantom{2 \dplus{} 1\sfrac{1}{2} \deq{}} 1\phantom{\sfrac{1}{2}} & 3 & \phantom{0}3\phantom{\sfrac{1}{2}} &\phantom{0} 0\phantom{\sfrac{1}{2}} & \phantom{0}0\phantom{\sfrac{1}{2}} & \phantom{00}0\phantom{\%} \\
      B & 1 & 2\sfrac{1}{2} \dplus{} 2\sfrac{1}{2} \deq{} 5\phantom{\sfrac{1}{2}} & 1\phantom{\sfrac{1}{2}} & \phantom{0}6 & 2 \dplus{} 1\sfrac{1}{2} \deq{} 3\sfrac{1}{2}           & 3           & 10\sfrac{1}{2}                     & \phantom{0}1\sfrac{1}{2}           & \phantom{0}4\sfrac{1}{2}           & 90\% \\
      C & 1 & 2\sfrac{1}{2} \dplus{} 2\sfrac{1}{2} \deq{} 5\phantom{\sfrac{1}{2}} & 1\phantom{\sfrac{1}{2}} & \phantom{0}6 & 3 \dplus{} 2\phantom{\sfrac{1}{2}} \deq{} 5\phantom{\sfrac{1}{2}} & 3 & 15\phantom{\sfrac{1}{2}}           & \phantom{0}3\phantom{\sfrac{1}{2}} & \phantom{0}9\phantom{\sfrac{1}{2}} & 180\%\\
      D & 1 & 2\sfrac{1}{2} \dplus{} 2\sfrac{1}{2} \deq{} 5\phantom{\sfrac{1}{2}} & 1\phantom{\sfrac{1}{2}} & \phantom{0}6 & 4 \dplus{} 3\sfrac{1}{2} \deq{} 7\sfrac{1}{2}           & 3           & 22\sfrac{1}{2}                     & \phantom{0}5\sfrac{1}{2}           & 16\sfrac{1}{2}                     & 330\%\\
     \midrule

     Разом &  & \phantom{2\sfrac{1}{2} \dplus{} 2\sfrac{1}{2} \deq{}} 17\sfrac{1}{2} & 3\sfrac{1}{2} & 21 & \phantom{2 \dplus{} 1\sfrac{1}{2} \deq{}}17\phantom{\sfrac{1}{2}} & & 51\phantom{\sfrac{1}{2}}  & 10 & 30\phantom{\sfrac{1}{2}} &\\
  \end{tabular}
\end{table}

\noindent{}При цьому третьому припущені знов таки байдуже, чи повторні додаткові
капітали вкладаються рівномірно або нерівномірно на землі різних родів або ні
в однакових чи неоднакових відношеннях відбувається зменшення продукції
надзиску; чи всі додаткові капітали вкладаються в той самий сорт землі, щодає
ренту, чи розподіляються вони рівномірно або нерівномірно, між землями
різної якости, що дають ренту. Всі ці обставини байдужі для закону, що його тут
розвиваємо. Єдине наше припущення є в тому, що додатковий капітал,
вкладений в будь-який сорт землі, що дає ренту, дає надзиск, але в зменшуваній
пропорції проти розміру збільшення капіталу. Межі цього зменшення
коливаються в прикладах вищенаведеної таблиці, між 4 квартерами \deq{} 12\pound{ ф. стерл.},
продуктом першого капіталовкладення на найкращій землі $В$ і 1 квартером
\deq{} 3\pound{ ф. стерл.}, продуктом такого самого вкладення капіталу на найгіршій
землі $А$. Продукт з найкращої землі при витраті капіталу і становить максимальну
межу, а продукт з найгіршої землі $А$, що не дає ні ренти, ні надзиску,
становить, за однакового вкладення капіталу, мінімальну межу продукту,
який дають послідовні вкладення капіталу па будь-якого роду землях, що дають надзиск за зменшуваної
продуктивности послідовних вкладень капіталу. Як
припущення ІІ відповідає тому, що нові однакові якістю дільниці землі кращих
родів приєднується до оброблюваної площі, так що кількість якогось роду обробленої
землі збільшується, так припущення ІІІ відповідає тому, що оброблюються
\parbreak{}  %% абзац продовжується на наступній сторінці

\parcont{}  %% абзац починається на попередній сторінці
\index{iii2}{0150}  %% посилання на сторінку оригінального видання
додаткові земельні дільниці, що їх різні ступені родючости розподіляються між
$D$ і $А$, між родючістю кращої і гіршої землі. Коли послідовні вкладення капіталу
відбуваються виключно на землі $D$, то вони можуть включати ріжниці, що
існують між $D$ і $А$, далі ріжниці між $D$ і $C$ так само як і ріжниці між $D$ і $В$.
Коли ж усі вони відбуваються на землі $C$, то лише ріжниці між $C$ і $А$ або $В$;
коли на $В$ — то лише ріжниці між $В$ і $А$.

Але закон такий: рента на землях усіх цих родів абсолютно зростає, хоч
і не пропорційно додатково вкладеному капіталові.

Норма надзиску зменшується так у відношенні до додаткового капіталу,
як і у відношенні до всього вкладеного в землю капіталу; але абсолютна величина
надзиску збільшується; цілком так само як зменшення норми надзиску
на капітал взагалі здебільша зв’язане зі збільшенням абсолютної маси зиску.
Так, пересічний надзиск з капіталу вкладеного в $В = 90\%$ на капітал, тимчасом
як при першому вкладенні капіталу він = 120\%. Але загальний надзиск
збільшується з 1 квартера до 1\sfrac{1}{2} кв. і з 3\pound{ ф. стерл.} до 4\sfrac{1}{2} . Вся рента,
розглядувана сама по собі, — а не в відношенні до подвоєного розміру авансованого
капіталу — абсолютно зросла. Рiжниці між рентами різних родів землі і
їхнє відношення одна до однієї можуть тут змінюватися; але ця зміна ріжниці,
є тут наслідок, а не причина збільшення рент однієї проти однієї.

ІV. Випадок, коли додаткові вкладення капіталу на кращих землях породжують
більшу кількість продукту, ніж первісні не потребує дальшої аналізи.
Само собою зрозуміло, що за такого припущення ренти з кожного
акра підвищуються і при тому в більшій пропорції, ніж додатковий капітал,
хоч би в який рід землі він був вкладений. В цьому випадку додаткове
капіталовкладення зв’язано з поліпшенням. Де буває тоді, коли додаткове
вкладення меншого капіталу впливає так само або більш продуктивно, ніж зроблене
давніш додаткове вкладення більшого капіталу. Випадок цей не зовсім тотожній
з давнішим, і ріжниця між ними має важливе значення при всіх
вкладеннях капіталу. Коли, наприклад, 100 дають зиску 10, а 200 при певній
формі вживання — зиск в 40, то зиск збільшується з 10\% до 20\%, і остільки
це є те саме, як коли б 50, при ефективнішій формі вживання, дали зиск в 10
замість 5. Ми припускаємо тут, що зиск є зв’язаний з відповідним збільшенням
продукту. Але ріжниця є в тому, що в одному випадку я мушу подвоїти капітал,
тоді як в другому досягаю подвоєного ефекту при колишньому капіталі.
Зовсім не є те саме, чи продукую я: 1) колиши й продукт, витрачаючи половину
колишньої кількости живої й зрічевленої праці, чи 2) подвоєний продукт
за колишньої кількости праці, чи 3) почвірний продукт за подвійної кількости
праці. В першому випадку праця — в живій або зрічевленій формі — стає вільна
і може бути вжита якось інакше; зростає можливість порядкувати працею
і капіталом. Звільнення капіталу (і праці) само по собі є збільшення багатства;
воно впливає цілком так само, як коли б цей додатковий капітал був здобутий
з допомогою акумуляції, але воно заощаджує працю акумуляції.

Припустімо, що капітал в 100 випродукував продукт в 10 метрів. В 100 є так
сталий капітал, як жива праця й зиск. Таким чином, метр коштує 10. Коли я тепер з
таким самим капіталом в 100 можу випродукувати 20 метрів, то метр коштуватиме 5.
Коли навпаки, я можу з капіталом в 50 випродукувати 10 метрів, то метр також
коштуватиме 5, але в цьому випадку звільняється капітал в 50, якщо колишнє подання
товару достатнє. Коли я мушу вкласти капітал в 200, щоб випродукувати
40 метрів, то метр також коштуватиме 5. Визначення вартості або також ціни так
само мало дозволяє помітити тут будь-яку ріжницю, як і маса продукту, що є пропорційна авансованому
капіталові. Але в першому випадку звільняється капітал; у
другому — заощаджується додатковий капітал, коли б треба було приблизно подвоїти
продукцію; в третьому випадку збільшений продукт можна одержати лише тоді, коли
\parbreak{}  %% абзац продовжується на наступній сторінці

\parcont{}  %% абзац починається на попередній сторінці
\index{iii2}{0151}  %% посилання на сторінку оригінального видання
зростає авансований капітал, хоч не в такому самому відношенні, яке потрібне
було б, коли б більшу кількість продукту довелося виготовляти за колишньої
продуктивної сили. (Стосується до відділу I).

З погляду капіталістичної продукції, у відношенні не до збільшення додаткової
вартости, а до зменшення витрат продукції, — а заощадження витрат
навіть на елемент, що створює додаткову вартість, на працю, робить капіталістові
цю послугу і створює для нього зиск, доки регуляційна ціна продукції
лишається та сама, — вживання сталого капіталу завжди дешевше, ніж
вживання змінного. Справді це має за свою передумову відповідний капіталістичному
способові продукції розвиток кредиту і багатість позикового капіталу. З одного
боку, я вживаю 100\pound{ ф. стерл.} додаткового сталого капіталу, коли 100\pound{ ф.
стерл.} становлять продукт, випродукований 5 робітниками протягом року; з
другого боку — 100\pound{ ф. стерл.} як змінний капітал. Коли норма додаткової вартости
\deq{} 100\%, то вартість випродукована 5 робітниками \deq{} 200\pound{ ф. стерл.};
навпаки, вартість 100\pound{ ф. стерл.} сталого капіталу \deq{} 100\pound{ ф. стерл.}, а як капіталу,
можливо, \deq{} 105\pound{ ф. стерл.}, коли рівень проценту \deq{} 5\%. Ті самі грошові суми,
залежно від того чи авансовано їх для продукції як вартісні величини сталого,
чи змінного капіталу, виражають, розглядувані в їхньому продукті, дуже неоднакові
вартості. Далі, щодо витрат продукції товарів, з погляду капіталіста,
то ріжниця є ще в тому, що з цих 100\pound{ ф. стерл.} сталого капіталу, оскільки
вони вкладені в основний капітал, в вартість товару входить лише спрацьовування,
тоді як ці 100\pound{ ф. стерл.}, витрачені на заробітну плату, мусять бути
цілком репродуковані у вартості товару.

У колоністів і взагалі самостійних дрібних продуцентів, які зовсім не
порядкують капіталом, або можуть ним порядкувати тільки за високі проценти,
частина продукту, відповідна заробітній платі, становить їхній дохід, тоді як
для капіталістів вона є авансування капіталу. Тому перший дивиться на цю
витрату праці як на доконечну передумову трудового продукту, про який насамперед
і йдеться. Щодо надмірної праці, витрачуваної ним понад цю потрібну
працю, то вона в усякому разі реалізується в надмірному продукті; і оскільки
він може продати або сам застосувати його, цей продукт розглядає він як
щось, що йому нічого не коштувало, бо він не коштував зрічевленої праці.
Тільки витрата такої праці має для нього значіння відчуження багатства. Звичайно,
він намагається продавати якомога дорожче; але навіть продаж нижче
вартости і нижче капіталістичної ціни продукції все ще має для нього значіння
зиску, оскільки цей зиск не антиципований заборгованістю, гіпотеками тощо.
Навпаки, для капіталістів витрата так змінного капіталу, як і сталого, однаково
є авансування капіталу. Відносно більше авансування сталого капіталу
зменшує за інших незмінних обставин витрати продукції, а також в дійсності і
вартість товарів. Тому, хоч зиск походить лише з додаткової праці, отже, лише
з вживання змінного капіталу, проте поодинокому капіталістові може здаватися,
що жива праця є найдорожчий елемент витрат продукції, який найбільш слід звести
до мінімуму. Це лише капіталістично перекручена форма тієї істини, що відносно
більше вживання зрічевленої праці порівняно з живою, свідчить про підвищення
продуктивности суспільної праці і збільшення суспільного багатства. От в якому фалшивому
вигляді, яким поставленим шкереберть здається все з погляду конкуренції.

Коли припустити незмінні ціни продукції, то додаткові капітали можуть
вкладатися з незмінною, висхідною або низхідною продуктивністю на кращих
землях, тобто на всіх землях, починаючи з $В$ і вище. На самій $А$ це було б
можливо при нашому припущенні або тільки за незмінної продуктивності, за
якої земля, як і давніш, не дає ренти, абож і тоді, коли продуктивність зростає;
одна частина вкладеного в землю $А$ капіталу давала б тоді ренту,
друга — ні. Але це було б неможливо, коли припустити, що продуктивна сила
\parbreak{}  %% абзац продовжується на наступній сторінці

\parcont{}  %% абзац починається на попередній сторінці
\index{iii2}{0152}  %% посилання на сторінку оригінального видання
на А зменшується, бо в такому разі ціна продукції не лишилася б сталою,
а підвищилась би. Але в усіх цих обставинах, тобто, чи буде, надпродукт, що
його дають додаткові капітали, пропорційний їхній величині, чи буде він
вищий чи нижчий від цієї пропорції, — чи лишається, отже, норма надзиску на
капітал, з його зростом, незмінною, чи підвищується вона чи понижується, —
надпродукт і відповідний йому надзиск з акра зростає, отже, зростає і
евентуальна рента, у збіжжі і в грошах. Зростання просто маси надзиску зглядно
ренти, обчислених на акр, тобто збільшення маси, обчисленої на будь-яку постійну
одиницю, отже, в даному разі, на будь-яку певну кількість землі, акр або гектар,
виражається тут, як ростуча пропорція. Тому висота ренти, обчисленої з акра,
вростає за цих умов просто в наслідок збільшення капіталу, вкладеного в землю.
І до того ж це відбувається за незмінних цін продукції, і навпаки при цьому
не має ваги, чи лишається продуктивність додаткового капіталу незмінною, чи
вменшується вона, чи збільшується. Ці останні обставини модифікують розмір,
в якому зростає висота ренти на акр, але нічого не змінюють у факті цього
зросту. Це є явище, що властиве диференційній ренті II і яке відрізняв її від
диференційної ренти І.~Коли б додаткові капітали вкладалося не один по
одному послідовно в часі на тій самій землі, а послідовно в просторі один
поряд одного на нових додаткових землях відповідної якости, то збільшилася б
загальна маса ренти і також, як це показано давніш, пересічна рента з усієї
оброблюваної площі, але не висота ренти з акра. За незмінних наслідків щодо
маси вартости усієї продукції і додаткового продукту, концентрація капіталу на
земельній площі меншого розміру підвищує розмір ренти з акра там, де за тих
самих обставин, його розпорошення на більшій земельній площі, за інших незмінних
умов, не справляє такого впливу. Але що більше розвивається капіталістичний
спосіб продукції, то більше концентрується капітал на тій самій земельній
площі, то більше, отже, підвищується рента, обчислена на акр. Тому в
двох країнах, де ціни продукції були б тотожні, ріжниці між різними родами
землі тотожні, і де вкладено було б однакову масу капіталу, але в одній країні
переважно у формі послідовних вкладень на обмеженій земельній площі, в другій
переважно у формі координованих вкладень на ширшій площі, — рента з акра, а
тому і ціна землі, була б вища в першій і нижча в другій країні, хоч маса
ренти в обох країнах була б та сама. Отже, ріжницю в висоті ренти можна
було б пояснити тут не ріжницею природної родючости різних земель і не
кількістю вжитої праці, а виключно різним способом вкладення капіталу.

Кажучи тут про надпродукт, ми завжди маємо на увазі відповідну
частину продукту, в якій репрезентовано надзиск. Взагалі-ж, під додатковим
продуктом або надпродуктом ми розуміємо ту частину продукту, яка становить
всю додаткову вартість, а в поодиноких випадках ту частину продукту, в якій
репрезентовано пересічний зиск. Специфічне значення, надаване цьому слову,
коли мова йде про капітал, що дає ренту, є за привід до непорозумінь, як це
зазначено давніш.

\section{Диференційна рента II. Другий випадок: низхідна ціна продукції}
\chaptermark{Диференційна рента II. Другий~випадок}

Ціна продукції може падати, коли додаткові капіталовкладення відбуваються
за незмінної, низхідної або висхідної норми продуктивности.

\paragraph{За незмінної продуктивности додаткових капіталовкладень}

Отже, це означає, що на різних землях відповідно до їхньої відносної
якости, продукт зростає в тій самій мірі, в якій зростає вкладений в них
\parbreak{}  %% абзац продовжується на наступній сторінці

\parcont{}  %% абзац починається на попередній сторінці
\index{iii2}{0153}  %% посилання на сторінку оригінального видання
капітал. Це включає, за незмінних ріжниць між родами землі, зріст надпродукту,
пропорційний зростові вкладеного капіталу. Отже, випадок цей виключає всяку
додаткову витрату на землю $А$, яка вплинула б на диференційну ренту. На цій землі
норма надзиску \deq{} 0; отже, вона лишається \deq{} 0, бо ми припустили, що продуктивна
сила додаткового капіталу, а тому і норма надзиску лишаються сталими.

Але реґуляційна ціна продукції може за цих умов лише знизитися, бо замість
ціни продукції з $А$ реґуляційною стає ціна продукції ближчої якістю землі
$В$ або взагалі з будь-якої землі, кращої, ніж $А$; отже, коли б ціна продукції
на землі $C$ зробилась регуляційною, то капітал був би вилучений з $А$,
або навіть з $А$ і $В$, і таким чином всі землі, гірші, ніж $C$, випали б з конкуренції
земель, на яких сіють пшеницю. Умова, потрібна для цього за даних
припущень, є в тому, щоб надпродукт з додаткових капіталовкладень задовольняв
потребам, і щоб тому продукція на гіршій землі $А$ тощо зробилася
зайвою для поновлення подання.

Отже, візьмімо, наприклад, таблицю II, але змінимо її так, щоб замість 20
квартерів, потребу задовольняли 18 квартерів. Земля $А$ відпала б; $D$, а
разом з нею ціна продукції в 30\shil{ шил.} за кв. стала б реґуляційною. Диференційна
рента набуває тоді такої форми:

\begin{table}[h]
  \begin{center}
    \emph{Таблиця ІV}
    \footnotesize

  \begin{tabular}{c c c c c c c c c c c}
    \toprule
      \multirowcell{2}{\makecell{Рід \\землі}} &
      \multirowcell{2}{\rotatebox[origin=c]{90}{Акри}} &
      \rotatebox[origin=c]{90}{Капітал} &
      \rotatebox[origin=c]{90}{Зиск} &
      \rotatebox[origin=c]{90}{\makecell{Ціна про- \\ дукції}} &
      \multirowcell{2}{\rotatebox[origin=c]{90}{\makecell{Продукт \\ в кварт.}}} &
      \rotatebox[origin=c]{90}{\makecell{Продажна \\ ціна}} &
      \rotatebox[origin=c]{90}{Здобуток} &
      \multicolumn{2}{c}{Рента} &
      \multirowcell{2}{\makecell{Норма \\надзиску}} \\

      \cmidrule(rl){3-3}
      \cmidrule(rl){4-4}
      \cmidrule(rl){5-5}
      \cmidrule(rl){7-7}
      \cmidrule(rl){8-8}
      \cmidrule(rl){9-10}

       &  &  ф. ст. & ф. ст. & ф. ст. & & ф. ст. & ф. ст. & Кварт. & ф. ст. &  \\
      \midrule

      B & 1 &  \phantom{0}5 & 1 & \phantom{0}6 & \phantom{0}4 & 1\sfrac{1}{2} & \phantom{0}6 & 0 & \phantom{0}0 & \phantom{00}0\% \\ % ця мітка у заголовку \\
      C & 1 &  \phantom{0}5 & 1 & \phantom{0}6 & \phantom{0}6 & 1\sfrac{1}{2} & \phantom{0}9 & 2 & \phantom{0}3 & \phantom{0}60\%\\
      D & 1 &  \phantom{0}5 & 1 & \phantom{0}6 & \phantom{0}8 & 1\sfrac{1}{2} & 12           & 4 & \phantom{0}6 & 120\%\\
     \cmidrule(rl){1-1}
     \cmidrule(rl){2-2}
     \cmidrule(rl){3-3}
     \cmidrule(rl){4-4}
     \cmidrule(rl){5-5}
     \cmidrule(rl){6-6}
     \cmidrule(rl){8-8}
     \cmidrule(rl){9-9}
     \cmidrule(rl){10-10}

     Разом & 3 & 15 & 3 & 18 & 18 & & 27 & 6 & 9 &\\
  \end{tabular}

  \end{center}
\end{table}

Отже, вся рента порівняно з таблицею II знизилась би з 36\pound{ ф. стерл.}
до 9, а в збіжжі з 12 кварт, до 6; вся продукція знизилася б лише на 2
квартери, з 20 до 18. Норма надзиску, обчислена у відношенні до капіталу,
знизилася б наполовину, з 180 до 90\%\footnote*{
До 60\%, тобто знизилася б втроє, бо в таблиці II вона \deq{} \frac{36}{20} × 100 \deq{} 180\%, а в тaблиці
IV вона $= \frac{9}{15} × 100 \deq{} 60\%$. \emph{Прим. Ред.}
}. Отже, пониженню ціни продукції
тут відповідає зменшення збіжжевої і грошової ренти.

Порівняно з таблицею І, відбувається лише зменшення грошової ренти;
збіжжева рента в обох випадках дорівнює 6 квартерам; але тільки в одному
випадку вона \deq{} 18\pound{ ф. стерл.}, а в другому \deq{} 9\pound{ ф. стерл}. Для земель $C$ і $D$
збіжжева рента проти таблиці І лишилась та сама\footnote*{
Те, що сказано тут, правильне лише для землі $C$, але неправильне для землі $D$, бо в табл. І земля
$D$ дає 3 кв. ренти, а в табл. IV земля $D$ дає 4 кв. ренти. Те, що тут сказано, було б правильне, коли
взяти загальну ренту з земель $B$, $C$ і $D$. \emph{Прим. Ред.}
}. В дійсності, в наслідок
того, що додаткова продукція, досягнена з допомогою додаткового капіталу рівної
продуктивности, витиснула з ринку продукт $А$ і разом з тим усунула землю $А$
з числа конкурентних аґентів продукції — в наслідок цього в дійсності створилася
нова диференційна рента І, в якій краща земля $В$ від грає ту саму ролю,
яку давніш відігравала гірша земля $А$. В наслідок цього, з одного боку, відпадає
рента з $В$; з другого боку, згідно з припущенням, вкладення додаткового
\parbreak{}  %% абзац продовжується на наступній сторінці

\parcont{}  %% абзац починається на попередній сторінці
\index{iii2}{0154}  %% посилання на сторінку оригінального видання
капіталу нічого не змінило в ріжницях між $В$, $C$ і $D$. Тому частина продукту,
що перетворюється на ренту, зменшується.

Коли б вищенаведений наслідок — задоволення попиту при виключенні
землі $А$ — був спричинений тич, що більше, ніж подвійна кількість капіталу
вкладалася б в землю $C$ або $D$ або в обидві разом, то справа набула б іншого
вигляду. Наприклад, коли б третє вкладення капіталу було зроблено на $C$:

\begin{table}[h]
  \begin{center}
    \emph{Таблиця ІVa}
    \footnotesize

  \begin{tabular}{c c c c c c c c c c c}
    \toprule
      \multirowcell{2}{\makecell{Рід \\землі}} &
      \multirowcell{2}{\rotatebox[origin=c]{90}{Акри}} &
      \rotatebox[origin=c]{90}{Капітал} &
      \rotatebox[origin=c]{90}{Зиск} &
      \rotatebox[origin=c]{90}{\makecell{Ціна про- \\ дукції}} &
      \multirowcell{2}{\rotatebox[origin=c]{90}{\makecell{Продукт \\ в кварт.}}} &
      \rotatebox[origin=c]{90}{\makecell{Продажна \\ ціна}} &
      \rotatebox[origin=c]{90}{Здобуток} &
      \multicolumn{2}{c}{Рента} &
      \multirowcell{2}{\makecell{Норма \\надзиску}} \\

      \cmidrule(rl){3-3}
      \cmidrule(rl){4-4}
      \cmidrule(rl){5-5}
      \cmidrule(rl){7-7}
      \cmidrule(rl){8-8}
      \cmidrule(rl){9-10}

       &  &  ф. ст. & ф. ст. & ф. ст. & & ф. ст. & ф. ст. & Кварт. & ф. ст. &  \\
      \midrule

      B & 1 &  \phantom{0}5\phantom{\sfrac{1}{2}} & 1\phantom{\sfrac{1}{2}} & \phantom{0}6 & \phantom{0}4 & 1\sfrac{1}{2} & \phantom{0}6\phantom{\sfrac{1}{2}} & 0 & \phantom{0}0\phantom{\sfrac{1}{2}}   & \phantom{00}0\% \\
      C & 1 &  \phantom{0}7\sfrac{1}{2}           & 1\sfrac{1}{2}           & \phantom{0}9 & \phantom{0}9 & 1\sfrac{1}{2} & 13\sfrac{1}{2}                     & 3 & \phantom{0}4\sfrac{1}{2}            & \phantom{0}60\%\\
      D & 1 &  \phantom{0}5\phantom{\sfrac{1}{2}} & 1\phantom{\sfrac{1}{2}} & \phantom{0}6 & \phantom{0}8 & 1\sfrac{1}{2} & 12\phantom{\sfrac{1}{2}}           & 4 & \phantom{0}6\phantom{\sfrac{1}{2}}  & 120\%\\
     \cmidrule(rl){1-1}
     \cmidrule(rl){2-2}
     \cmidrule(rl){3-3}
     \cmidrule(rl){4-4}
     \cmidrule(rl){5-5}
     \cmidrule(rl){6-6}
     \cmidrule(rl){8-8}
     \cmidrule(rl){9-9}
     \cmidrule(rl){10-10}

     Разом & 3 & 17\sfrac{1}{2} & 3\sfrac{1}{2} & 21 & 21 & & 30\sfrac{1}{2} & 7 & 10\sfrac{1}{2} &\\
  \end{tabular}

  \end{center}
\end{table}

Продукт з $C$ збільшився тут проти таблиці ІV з 6 кватерів до 9, надпродукт
— з 2 квартерів до 3, грошова рента зросла з 3\pound{ ф. стерл.} до 4\sfrac{1}{2}\pound{ ф.
стерл}. Проти таблиці II, де грошова рента була 12\pound{ ф. стерл.} і таблиці І,
де вона була 6\pound{ ф. стерл.}, вона навпаки зменшилась. Загальна сума ренти визначена
в збіжжі = 7 квартерів, зменшилась проти таблиці II (12 квартерів),
збільшилась проти таблиці І (6 квартерів); визначена в грошах (10\sfrac{1}{2}\pound{ ф. стерл.})
зменшилася проти обох (18\pound{ ф. стерл.} і 36\pound{ ф. стерл.}).

Коли б у землю $В$ було вкладено третій капітал в 2\sfrac{1}{2}\pound{ ф. стерл.}, то хоч
це й змінило б масу продукції, але не зачепило б ренти, бо згідно з припущенням
послідовні вкладення капіталу не вносять жодної ріжниці в землю того
самого роду, а земля $В$ ренти не дає.

Навпаки, коли ми припустимо, що третій капітал вкладається в землю $D$,
замість $C$, то ми матимемо:

\begin{table}[h]
  \begin{center}
    \emph{Таблиця ІVb}
    \footnotesize

  \begin{tabular}{c c c c c c c c c c c}
    \toprule
      \multirowcell{2}{\makecell{Рід \\землі}} &
      \multirowcell{2}{\rotatebox[origin=c]{90}{Акри}} &
      \rotatebox[origin=c]{90}{Капітал} &
      \rotatebox[origin=c]{90}{Зиск} &
      \rotatebox[origin=c]{90}{\makecell{Ціна про- \\ дукції}} &
      \multirowcell{2}{\rotatebox[origin=c]{90}{\makecell{Продукт \\ в кварт.}}} &
      \rotatebox[origin=c]{90}{\makecell{Продажна \\ ціна}} &
      \rotatebox[origin=c]{90}{Здобуток} &
      \multicolumn{2}{c}{Рента} &
      \multirowcell{2}{\makecell{Норма \\надзиску}} \\

      \cmidrule(rl){3-3}
      \cmidrule(rl){4-4}
      \cmidrule(rl){5-5}
      \cmidrule(rl){7-7}
      \cmidrule(rl){8-8}
      \cmidrule(rl){9-10}

       &  &  ф. ст. & ф. ст. & ф. ст. & & ф. ст. & ф. ст. & Кварт. & ф. ст. &  \\
      \midrule

      B & 1 &  \phantom{0}5\phantom{\sfrac{1}{2}} & 1\phantom{\sfrac{1}{2}} & \phantom{0}6 & \phantom{0}4 & 1\sfrac{1}{2}  & \phantom{0}6 & 0 & \phantom{0}0 & \phantom{00}0\% \\
      C & 1 &  \phantom{0}5\phantom{\sfrac{1}{2}} & 1\phantom{\sfrac{1}{2}} & \phantom{0}6 & \phantom{0}6 & 1\sfrac{1}{2}  & \phantom{0}9 & 2 & \phantom{0}3 & \phantom{0}60\%\\
      D & 1 &  \phantom{0}7\sfrac{1}{2}           & 1\sfrac{1}{2}           & \phantom{0}9 & \phantom{0}12 & 1\sfrac{1}{2} & 18           & 6 & \phantom{0}9 & 120\%\\
     \cmidrule(rl){1-1}
     \cmidrule(rl){2-2}
     \cmidrule(rl){3-3}
     \cmidrule(rl){4-4}
     \cmidrule(rl){5-5}
     \cmidrule(rl){6-6}
     \cmidrule(rl){8-8}
     \cmidrule(rl){9-9}
     \cmidrule(rl){10-10}

     Разом & 3 & 17\sfrac{1}{2} & 3\sfrac{1}{2} & 21 & 22 & & 33 & 8 & 12 &\\
  \end{tabular}

  \end{center}
\end{table}

Тут загальна кількість продукту = 22 кварт., більша ніж удвоє проти
загальної кількости продукту таблиці І, хоч авансований капітал є лише 17\sfrac{1}{2}\pound{ ф. стерл.} проти 10\pound{ ф. стерл.}, отже, не подвоївся. Далі, загальна кількість продукту
на 2 квартерн більша, ніж загальна кількість продукту у таблиці II, хоч
в останній авансований капітал більший, а саме 20\pound{ ф. стерл.}


\index{iii2}{0155}  %% посилання на сторінку оригінального видання
На землі $D$ збіжжева рента проти таблиці I зросла з 3\footnote*{
В німецькому тексті стоїть: «з 2 квартерів». Явна помилка, як це можна бачити з таблиці І \emph{Прим. Ред.}
} квартерів до 6
тимчасом як грошова рента лишилася, як і давніш, 9\pound{ ф. стерл}. Проти таблиці II
збіжжева рента з $D$ лишилася колишня, 6 квартерів, але грошова рента знизилась
з 18\pound{ ф. стер.} до 9\pound{ ф. стерл}.

Коли розглядати загальні суми ренти, то збіжева рента таблиці IVb = 8
квартерам, більша, ніж рента в таблиці І, що дорівнює 6 квартерам, і більша,
ніж рента в таблиці IVа, що дорівнює 7 квартерам; і навпаки, вона менша, ніж
рента в таблиці II = 12 кварт. Грошова рента в таблиці IVb = 12\pound{ ф. стерл.},
більша, ніж грошова рента в таблиці ІVа = 10\sfrac{1}{2}\pound{ ф. стерл.}, і менша від грошової
ренти таблиці І = 18\pound{ ф. стерл.} і таблиці II = 36\pound{ ф. стерл}.

Щоб по відпаданні ренти з $В$ в умовах таблиці IVb загальна сума ренти
дорівнювала такій у таблиці I, ми мусимо одержати ще на 6\pound{ ф. стерл.}
надпродукту, тобто 4 квартери по 1\sfrac{1}{2}\pound{ ф. стерл.}, що є новою ціною продукції.
Тоді ми знову маємо загальну суму ренти в 18\pound{ ф. стерл.}, як у таблиці І.~Величина
потрібного на це додаткового капіталу буде різна залежно від того, чи
вкладемо ми його в $C$ або $D$, чи розподілимо його між обома родами землі.

На $C$ капітал в 5\pound{ ф. стерл.} дає 2 квартери надпродукту, отже, 10\pound{ ф. ст.}
додаткового капіталу дадуть 4 квартери додаткового надпродукту. На $D$ було б
досить додаткової витрати в 5\pound{ ф. стерл.}, щоб випродукувати 4 квартери додаткової
збіжжевої ренти при зробленому тут основному припущенні, що продуктивність
додаткових капіталовкладень лишається та сама. Тому здобуваємо
такі наслідки.

\begin{table}[h]
  \begin{center}
    \emph{Таблиця ІVc}
    \footnotesize

  \begin{tabular}{c c c c c c c c c c c}
    \toprule
      \multirowcell{2}{\makecell{Рід \\землі}} &
      \multirowcell{2}{\rotatebox[origin=c]{90}{Акри}} &
      \rotatebox[origin=c]{90}{Капітал} &
      \rotatebox[origin=c]{90}{Зиск} &
      \rotatebox[origin=c]{90}{\makecell{Ціна про- \\ дукції}} &
      \multirowcell{2}{\rotatebox[origin=c]{90}{\makecell{Продукт \\ в кварт.}}} &
      \rotatebox[origin=c]{90}{\makecell{Продажна \\ ціна}} &
      \rotatebox[origin=c]{90}{Здобуток} &
      \multicolumn{2}{c}{Рента} &
      \multirowcell{2}{\makecell{Норма \\надзиску}} \\

      \cmidrule(rl){3-3}
      \cmidrule(rl){4-4}
      \cmidrule(rl){5-5}
      \cmidrule(rl){7-7}
      \cmidrule(rl){8-8}
      \cmidrule(rl){9-10}

       &  &  ф. ст. & ф. ст. & ф. ст. & & ф. ст. & ф. ст. & Кварт. & ф. ст. &  \\
      \midrule

      B & 1 &  \phantom{0}5\phantom{\sfrac{1}{2}} & 1\phantom{\sfrac{1}{2}} & \phantom{0}6 & \phantom{0}4 & 1\sfrac{1}{2} & \phantom{0}6 & 0 & \phantom{0}0 & \phantom{00}0\% \\
      C & 1 & 15\phantom{\sfrac{1}{2}}            & 3\phantom{\sfrac{1}{2}} & 18           & 18           & 1\sfrac{1}{2} & 27           & 6 & \phantom{0}9 & \phantom{0}60\%\\
      D & 1 &  \phantom{0}7\sfrac{1}{2}           & 1\sfrac{1}{2}           & \phantom{0}9 & 12           & 1\sfrac{1}{2} & 18           & 6 & \phantom{0}9 & 120\%\\
     \cmidrule(rl){1-1}
     \cmidrule(rl){2-2}
     \cmidrule(rl){3-3}
     \cmidrule(rl){4-4}
     \cmidrule(rl){5-5}
     \cmidrule(rl){6-6}
     \cmidrule(rl){8-8}
     \cmidrule(rl){9-9}
     \cmidrule(rl){10-10}

     Разом & 3 & 27\sfrac{1}{2} & 5\sfrac{1}{2} & 33 & 34 & & 51 & 12 & 18 &\\
  \end{tabular}

  \end{center}
\end{table}

\begin{table}[h]
  \begin{center}
    \emph{Таблиця ІVd}
    \footnotesize

  \begin{tabular}{c c c c c c c c c c c}
    \toprule
      \multirowcell{2}{\makecell{Рід \\землі}} &
      \multirowcell{2}{\rotatebox[origin=c]{90}{Акри}} &
      \rotatebox[origin=c]{90}{Капітал} &
      \rotatebox[origin=c]{90}{Зиск} &
      \rotatebox[origin=c]{90}{\makecell{Ціна про- \\ дукції}} &
      \multirowcell{2}{\rotatebox[origin=c]{90}{\makecell{Продукт \\ в кварт.}}} &
      \rotatebox[origin=c]{90}{\makecell{Продажна \\ ціна}} &
      \rotatebox[origin=c]{90}{Здобуток} &
      \multicolumn{2}{c}{Рента} &
      \multirowcell{2}{\makecell{Норма \\надзиску}} \\

      \cmidrule(rl){3-3}
      \cmidrule(rl){4-4}
      \cmidrule(rl){5-5}
      \cmidrule(rl){7-7}
      \cmidrule(rl){8-8}
      \cmidrule(rl){9-10}

       &  &  ф. ст. & ф. ст. & ф. ст. & & ф. ст. & ф. ст. & Кварт. & ф. ст. &  \\
      \midrule

      B & 1 & \phantom{0}5\phantom{\sfrac{1}{2}} & 1\phantom{\sfrac{1}{2}} & \phantom{0}6 & \phantom{0}4 & 1\sfrac{1}{2} & \phantom{0}6 & \phantom{0}0 & \phantom{0}0 & \phantom{00}0\% \\
      C & 1 & \phantom{0}5\phantom{\sfrac{1}{2}} & 1\phantom{\sfrac{1}{2}} & \phantom{0}6 & \phantom{0}6 & 1\sfrac{1}{2} & \phantom{0}9 & \phantom{0}2 & \phantom{0}3 & \phantom{0}60\%\\
      D & 1 & 12\sfrac{1}{2}                     & 2\sfrac{1}{2}           & 15           & 20           & 1\sfrac{1}{2} & 30           & 10           & 15           & 120\%\\
     \cmidrule(rl){1-1}
     \cmidrule(rl){2-2}
     \cmidrule(rl){3-3}
     \cmidrule(rl){4-4}
     \cmidrule(rl){5-5}
     \cmidrule(rl){6-6}
     \cmidrule(rl){8-8}
     \cmidrule(rl){9-9}
     \cmidrule(rl){10-10}

     Разом & 3 & 22\sfrac{1}{2} & 4\sfrac{1}{2} & 27 & 30 & & 45 & 12 & 18 &\\
  \end{tabular}

  \end{center}
\end{table}


\noindent{}Загальна сума грошової ренти становила б якраз половину того, що було
в таблиці II, де додаткові капітали були вкладені за незмінних цін продукції.

Найважливіше є порівняти вищенаведені таблиці з таблицею І.

\enlargethispage{\baselineskip}
\looseness=-1
Ми бачимо, що з пониженням ціни продукції на половину, з 60\shil{ шил.} до
30\shil{ шил.} за квартер, загальна сума грошової ренти залишилась та сама \deq{} 18\pound{ ф.
ст.} і відповідно до цього збіжжева рента подвоїлась, саме зросла з 6 кварт. до
12 кварт. Рента з $В$ відпала; з $C$ грошова рента в ІVd збільшилась на половину,
але на половину зменшилась в ІVс; з $D$ вона лишилась та сама \deq{} 9\pound{ ф.
стерл.} у таблиці ІVс, і піднеслась з 9\pound{ ф. стерл.} до 15\pound{ ф. стерл.} у таблиції ІVd.
Продукція піднеслась з 10 квартерів до 34 в ІVс, і до 30 квартер в в IVd;
зиск підвищився з 2\pound{ ф. стерл.} до 5\sfrac{1}{2} в ІVс і до 4\sfrac{1}{2} в IVd.~Загальна сума
вкладеного капіталу зросла в одному випадку з 10\pound{ ф. стерл.} до 27\sfrac{1}{2}\pound{ ф. стерл.},
в другому — з 10 до 22\sfrac{1}{2}\pound{ ф. стерл.}; отже, обидва рази більше, ніж удвоє. Норма
ренти, рента, обчислена у відношенні до авансованого капіталу, в усіх таблицях
від IV до IVd для кожного роду землі всюди та сама, що вже було дано тим припущенням,
що норма продуктивности обох послідовних витрат капіталу на землях
усіх родів не змінюється. Проти таблиці І вона, проте, понизилась пересічно
щодо всіх родів землі і для кожного окремого роду землі. В таблиці І вона \deq{}
180\% пересічно, в таблиці ІVс вона$ \deq{} \frac{18}{27\sfrac{1}{2}} × 100 \deq{} 65\sfrac{5}{11}\%$ і
IVd \deq{} $\frac{18}{22\sfrac{1}{2}} × 100 \deq{} 80\%$. Пересічна грошова рента з акра підвищилась. Її пересічна
величина давніш в таблиці І була 4\sfrac{1}{2}\pound{ ф. стерл.} з акра для всіх 4 акрів,
а тепер у таблицях IVс і d вона дорівнює 6\pound{ ф. стерл.} з акра для 3 акрів.
Її пересічна величина для землі, що дає ренту, була раніш 6\pound{ ф. стерл.}, а тепер
вона дорівнює 9\pound{ ф. стерл.} з акра. Отже, грошова вартість ренти з акра підвищилась
і репрезентує тепер удвоє більше продукту в збіжжі, ніж давніш, але
12 квартерів збіжжевої ренти тепер становлять менше, ніж половину всього продукту
в 34, зглядно 30
% REMOVED  footnote*{В німецькому тексті стоїть: усього «продукту в 33, зглядно 27 квартерів» Явна помилка,
%як це можна бачити з таблиць ІVс і IVd. \emph{Прим. Ред.}}
квартерів, тимчасом як у таблиці І 6 квартерів становлять
\sfrac{3}{5}  усього продукту в 10 квартерів. Отже, хоч рента, коли розглядати
її як відповідну частину всього продукту, а також коли обчислити її у відношенні
до витраченого капіталу, і знизилась, одначе її грошова вартість,
обчислена на акр, збільшилась, а її вартість в продукті збільшилась ще дужче.
Коли ми візьмемо землю $D$ в таблиці IVd, то ціна продукції тут дорівнює
15\pound{ ф. стерл.}, що з них витрачений капітал \deq{} 12\sfrac{1}{2}\pound{ ф. стерл}. Грошова рента \deq{} 15\pound{ ф. стер}. У таблиці І на тій самій землі $D$ ціна продукції була 3\pound{ ф. стерл.}, витрачений
капітал \deq{} 2\sfrac{1}{2}\pound{ ф. стерл.}, грошова рента \deq{} 9\pound{ ф. стерл.}, отже, остання
утроє більша за ціну продукції й майже у чотири рази більша за витрачений
капітал. У таблиці IVd для $D$ грошова рента в 15\pound{ ф. стерл.} якраз дорівнює ціні
продукції і лише на \sfrac{1}{5}  більша за витрачений капітал. А все ж грошова рента
з акра на \sfrac{2}{3}  більша, 15\pound{ ф. стерл.} замість 9\pound{ ф. стерл}. В таблиці І збіжжева
рента в 3 квартери \deq{} \sfrac{3}{4}  усього продукту, що становить 4 квартери, в таблиці
IVd вона \deq{} 10 квартерам, половині всього продукту (20 квартерів) з акра
землі $D$. Це показує, що грошова і збіжжева рента з акра може зрости, хоч
вона і становить відносно меншу частину всього здобутку і знизилась у відношенні
до авансованого капіталу.

Вартість всього продукту в таблиці І \deq{} 30\pound{ ф. стерл.}; рента \deq{} 18\pound{ ф.
стерл.}, більше від половини цієї вартости. Вартість усього продукту в таблиці
IVd \deq{} 45\pound{ ф. стерл.}, що з них 18\pound{ ф. стерл.}, менш від половини, становлять
ренту.


\index{iii2}{0157}  %% посилання на сторінку оригінального видання
Причина ж того, що не зважаючи на пониження ціни на 1\sfrac{1}{2}\pound{ ф. стерл.}
за квартер, отже на 50\%, і не зважаючи на зменшення площі конкурентної
землі з 4 до 3 акрів, загальна сума грошової ренти лишається та сама, а збіжжева
рента подвоюється, тимчасом як збіжжева й грошова рента, обчислена на акр, підвищується,
— причина цього в тому, що вироблено більше квартерів надпродукту.
Ціна збіжжя знижується на 50\%, надпродукт зростає на 100\%.
Але для досягнення такого наслідку вся продукція, згідно з нашими умовами,
мусить збільшитись утроє, а капітал, вкладений у кращу землю, мусить більше
ніж подвоїтись. В якому відношенні він мусить збільшуватись, залежить насамперед
від того, як розподіляється додаткові вкладення капіталу між кращими та
найкращими землями, припускаючи завжди, що продуктивність капіталу на
кожній категорії землі зростає пропорційно його величині.

Коли б пониження ціни продукції було менш значне, то потрібно було б
менше додаткового капіталу, щоб випродукувати ту саму грошову ренту. Коли б
подання збіжжя потрібне для того, щоб вилучити $А$ з числа оброблюваних земель,
— а це залежить не тільки від кількости продукту з акра землі $А$, але
також і від того, яку частину всієї оброблюваної земельної площі становить $А$, —
отже, коли б потрібне для цього подання було більше, отже, коли б також
потрібно було і більшої маси додаткового капіталу на кращій, ніж $А$, землі, то,
за інших незмінних відношень грошова і збіжжева ренти зросли б ще більше,
не зважаючи на те, що земля $В$ перестала б давати грошову і збіжжеву ренту.

Коли б капітал, що перестав функціонувати на землі $А$, дорівнював 5\pound{ ф.
стерл.}, то для цього випадку треба було б взяти для порівняння обидві таблиці:
II і ІV$d$. Весь продукт збільшився б з 20 до 30 квартерів. Грошова рента
зменшилася б удвоє, вона дорівнювала б 18\pound{ ф. стерл.} замість 36\pound{ ф. стерл.},
збіжжева рента залишилась би та сама \deq{} 12 квартерів.

Коли б можна було випродукувати на землі $D$ 44 квартери загального
продукту \deq{} 66\pound{ ф. стерл.}, вкладаючи капітал в 27\sfrac{1}{2}\pound{ ф. стерл.}, — що відповідало б
колишньому припущенню для $D$: 4 квартери на 2\sfrac{1}{2}\pound{ ф. стерл.} капіталу, —
то загальна сума\footnote*{
Тут очевидно справа йде про загальну грошову ренту. \Red{Прим. Ред.}
} ренти знову досягла б тієї висоти, яку вона мала в таблиці
II, і таблиця набула б такого вигляду:

\vspace{\bigskipamount}
\begin{table}[H]
  \centering
  \small
  \begin{tabular}{l c c c c}
  \toprule
  Рід землі & Капітал, \poundsign{} & Продукт, кв. & \makecell{Збіжжева \\ рента, кв.} & \makecell{Грошова\\рента, \poundsign{}} \\
  \midrule
  B &    \phantom{0}5\phantom{\tbfrac{1}{2}} & \phantom{0}4  & \phantom{0}0  & \phantom{0}0\\
  C &    \phantom{0}5\phantom{\tbfrac{1}{2}} & \phantom{0}6  & \phantom{0}2  & \phantom{0}3\\
  D &   27\tbfrac{1}{2}                      & 44            & 22            & 33\\
  \midrule
  Разом & 37\tbfrac{1}{2} &      54  &  24  &  36\\
  \end{tabular}
\end{table}

\looseness=1
\noindent{}Уся продукція була б 54 квартери проти 20 квартерів у таблиці II, грошова рента була
б та сама \deq{} 36\pound{ ф. стерл}. Але весь капітал був би 37\sfrac{1}{2}\pound{ ф. стерл.}, тимчасом як у таблиці II
він був \deq{} 20\pound{ ф. стерл}. Весь авансований капітал майже подвоївся б, тимчасом як продукція майже
потроїлася б; збіжжева рента збільшилася б удвоє, грошова рента залишилася б та сама.
Отже, коли ціна, за незмінної продуктивности, знижується в наслідок приміщення
додаткового грошового капіталу у кращі землі, що дають ренту, отже
в усі землі кращі від $А$, то весь капітал має тенденцію зростати не в такій
самій пропорції, як продукція і збіжжева рента; так що зростання збіжжевої ренти
може урівноважити падіння грошової ренти, яке постає в наслідок пониження ціни.
Той самий закон виявляється і в тому, що авансований капітал мусить бути більший
відповідно до того, як його вживається більше на землі $C$, ніж на $D$, — на землі,
що дає менше ренти, ніж на тій, яка дає більше ренти. Це визначає лише ось що:
\parbreak{}  %% абзац продовжується на наступній сторінці

\parcont{}  %% абзац починається на попередній сторінці
\index{iii2}{0158}  %% посилання на сторінку оригінального видання
щоб грошова рента лишилась та сама або підвищилась, мусить бути випродукована
певна додаткова кількість надпродукту, а для цього треба то менш
капіталу, що більша родючість земель; які дають надпродукт. Коли б ріжниця
між $В$ і $C$, $C$ і $D$ була ще більша, то потрібно було б ще менш додаткового
капіталу. Певне відношення залежить: 1) від відношення, в якому понижується
ціна, отже, від ріжниці між землею $В$, що тепер не дає ренти, і $А$, яка давніш
не давала ренти; 2) від відношення ріжниць між кращими, ніж $В$ землями; 3) від
маси нововкладуваного додаткового капіталу і 4) від його розподілу між землями
різної якости.

В дійсності бачимо, що закон не виражає нічого іншого, як те, що вже
було розвинено при дослідженні першого випадку: саме, що, коли ціна продукції
є дана, хоч би яка була її величина, рента може підвищуватися в наслідок
додаткового вкладення капіталу. Бо в наслідок вилучення $А$ тепер дана нова диференційна
рента І, за якої земля $В$ є тепер найгірша земля, і 1\sfrac{1}{2}\pound{ ф. стерл.} за
квартер становлять нову ціну продукції. Це однаково має силу так щодо таблиці
IV, як і щодо таблиці II.~Це той самий закон, але за вихідний
пункт береться землю $В$ замість $А$, і ціну продукції в 1\sfrac{1}{2}\pound{ ф. стерл.} замість.
3\pound{ ф. стерл}.

Справа важлива тут лише от чим: оскільки така кількість додаткового
капіталу потрібна була для того, щоб капітал з $А$ відтягти від землі, і обслугувати
постачання без його участи, то й виявляється, що це може супроводитись
незмінною, висхідною або низхідною рентою з акра, якщо не на всіх землях, то
принаймні на деяких, і пересічно для всіх оброблюваних земель. Ми бачили, що
збіжжева рента і грошова рента не співрозмірні. Тільки за традицією збіжжева
рента все ще продовжує відігравати ролю в економії. З однаковим успіхом можна
було б довести, що, наприклад, фабрикант на свій зиск в 5\pound{ ф. стерл.} може купити
геть більшу кількість своєї власної пряжі, ніж давніше на зиск в 10\pound{ ф. стерл}.
Але в усякому разі це доводить, що панове земельні власники, коли вони одночасно
власники або учасники мануфактур, цукроварень, гуралень то що, з пониженням
грошової ренти все таки можуть дуже значно вигравати, як продуценти
свого власного сирового матеріялу\footnote{
У вищенаведених таблицях від IVа до ІVb треба було б виправити в розрахунку помилку, що
проходить через них. Хоч це не зачіпає теоретичних засад, виведених з даних таблиць, але іноді
приводить до неймовірних числових відношень продукції з акра. Але й це по суті не має значіння. У
всіх мапах, що змальовують рельєф і висоту профілю місцевості, беруть значно більший маштаб для
вертикалей, ніж для горизонталей. А хто все таки почуватиме себе ображеним у своїх аграрних
почуттях, тому дається на волю помножити число акрів на перше-ліпше число. Можна також у таблиці І
замінити 1, 2, 3, 4 квартери з акра 10, 12, 15, 16 бушелями (8 бушелів = 1 квартер), з тим
розрахунком, щоб виведені з цього числа інших таблиць не виходили з меж імовірностп; тоді виявиться,
що наслідок — відношення підвищення ренти до збільшення капіталу — зводиться цілком до того самого.
Це й зроблено в тих таблицях, що їх редактор додає до найближчого розділу. —\emph{ Ф.~Е.}
}.

\subsubsection{За низхідної норми продуктивности додаткових капіталів}

Це не викликає нічого нового остільки, оскільки ціна продукції і тут, як
в щойно розгляненому випадку може лише понизитись, коли в наслідок додаткових
вкладень капіталу на землях кращої якости, ніж $А$, продукт з $А$ зробиться
зайвий і тому капітал буде вилучений з $А$, або земля $А$ буде застосована до вироблення
іншого продукту. Випадок цей ми вже вичерпно дослідили. Ми показали,
що збіжжева і грошова ренти з акра можуть при цьому випадку зрости,
зменшитися або лишитися без зміни.


Для зручности порівняння поновимо насамперед таку таблицю:

\begin{table}[H]
  \centering
  \caption*{Таблиця І}
  \footnotesize

  \settowidth\rotheadsize{\theadfont Продажна}
  \begin{tabular}{l c r c c c c c c}
    \toprule
      \thead[tl]{Земля} &
      &
      \rothead{Капітал} &
      \rothead{Зиск} &
      \rothead{Ціна\\продукції} &
      \rothead{Продукт} &
      \multicolumn{2}{c}{Рента} &
      \rothead{Норма\\надзиску} \\

      \cmidrule(rl){2-9}

       & акри  & \makecell{\poundsign{}} & \poundsign{} & кв. & кв. & кв. & \poundsign{}  & \% \\
      \midrule

       A & 1 & 2\tbfrac{1}{2} & \tbfrac{1}{2} & 3\phantom{\tbfrac{1}{2}} & \phantom{0}1 & 0 & \phantom{0}0 & \phantom{00}0 \\
       B & 1 & 2\tbfrac{1}{2} & \tbfrac{1}{2} & 1\tbfrac{1}{2}           & \phantom{0}2 & 1 & \phantom{0}3 & 120 \\
       C & 1 & 2\tbfrac{1}{2} & \tbfrac{1}{2} & 1\phantom{\tbfrac{1}{2}} & \phantom{0}3 & 2 & \phantom{0}6 & 240\\
       D & 1 & 2\tbfrac{1}{2} & \tbfrac{1}{2} & \phantom{0}\tbfrac{3}{4} & \phantom{0}4 & 3 & \phantom{0}9 & 360\\
    \midrule
      Разом & 4 & \hang{r}{1}0\pF & & & 10 & 6 & 18 & \makecell[t]{180 \\ пересічно}\\
  \end{tabular}

\end{table}

\noindent{}Коли ми тепер припустимо, що цифра 16 квартерів, що їх даватимуть землі
$В$, $C$, $D$ за низхідної норми продуктивности, достатня для того, щоб вилучити
$А$ з числа оброблюваних земель, то таблиця III перетворюється на таку:

\begin{table}[H]
  \centering
  \caption*{Таблиця V}
  \footnotesize

  \settowidth\rotheadsize{\theadfont Продажна}
  \begin{tabular}{l c r c r c c c c c}
  \toprule

\thead[tl]{Земля} &
&
\thead[t]{Вкладення \\ капіталу} &
\rothead{Зиск} &
\thead[t]{Продукт} &
\rothead{Продажна\\ціна} &
\rothead{Здобуток} &
\multicolumn{2}{c}{Рента} &
\rothead{Норма\\надзиску} \\

  \cmidrule(rl){2-10}
  & акри  & \poundsign{} & \poundsign{} & кв. & \poundsign{} & \poundsign{} & кв. & \poundsign{}  & \% \\
  \midrule


       B & 1 & 2\tbfrac{1}{2} \dplus{} 2\tbfrac{1}{2} & 1 & 2 \dplus{} 1\tbfrac{1}{2} \deq{} 3\tbfrac{1}{2} 
            & 1\tbfrac{5}{7} & \phantom{0}6\phantom{\tbfrac{1}{2}} & 0\phantom{\tbfrac{1}{2}} & 0\phantom{\tbfrac{1}{2}} & \phantom{00}0\\
       C & 1 & 2\tbfrac{1}{2} \dplus{} 2\tbfrac{1}{2} & 1 & 3 \dplus{} 2\phantom{\tbfrac{1}{2}} \deq{} 5\phantom{\tbfrac{1}{2}} 
            & 1\tbfrac{5}{7} & \phantom{0}8\tbfrac{4}{7}           & 1\tbfrac{1}{2}           & 2\tbfrac{4}{7}           & \phantom{0}51\hang{l}{\tbfrac{2}{5}}\\
       D & 1 & 2\tbfrac{1}{2} \dplus{} 2\tbfrac{1}{2} & 1 & 4 \dplus{} 3\tbfrac{1}{2} \deq{} 7\tbfrac{1}{2}
            & 1\tbfrac{5}{7} & 12\tbfrac{6}{7}                     & 4\phantom{\tbfrac{1}{2}} & 6\tbfrac{6}{7}           & 137\hang{l}{\tbfrac{1}{5}}\\  

      \midrule

      Разом & 3 & 15 & &  \phantom{2 \dplus{} 1\tbfrac{1}{2} \deq{}}16\phantom{\tbfrac{1}{2}} & & 27\tbfrac{3}{7} & 5\tbfrac{1}{2} & 9\tbfrac{3}{7} & \makecell[t]{94\hang{l}{\tbfrac{3}{10}} \\ пересічно\footnotemarkZ{}}\\
  \end{tabular}
\end{table}
\footnotetextZ{Тут пересічну норму надзиску обчислено не до всього вкладеного капіталу, а тільки до капіталу, вкладеного в рентодайні дільниці $C$ і $D$. \Red{Прим. Ред.}}

\noindent{}Тут за низхідної норми продуктивности додаткових капіталів і за різного
ступеня цього зменшення на різних землях, реґуляційна ціна продукції знизилася
з 3\pound{ ф. стерл.} до 1\sfrac{5}{7}\pound{ ф. стерл}. Вкладення капіталу збільшилося наполовину з 10\pound{ ф.
стерл.} до 15\pound{ ф. стерл}. Грошова рента зменшилася майже удвоє, з 18 до 9\sfrac{3}{7}\pound{ ф.
стерл.}, але збіжжева рента лише на \sfrac{1}{2},
% REMOVED \footnote*{
% В німецькому тексті тут стоїть «\sfrac{1}{22}». Очевидна помилка. \emph{Прим. Ред.}}
з 6 квартерів до 5\sfrac{1}{2}. Весь продукт
збільшився з 10 до 16, або на 60\%.
% REMOVED \footnote*{
% В німецькому тексті тут помилково стоїть: «160\%». \emph{Прим. Ред.}}
Збіжжева рента становить небагато більше
від третини всього продукту. Авансований капітал відноситься до грошової ренти
як $15 : 9\sfrac{3}{7}$, тимчасом як давніш це відношення було $10:18$.

\paragraph{За висхідної норми продуктивности додаткових капіталів.}

Цей випадок відрізняється від варіянту І, наведеного на початку цього
розділу, де ціна продукції за незмінної норми продуктивности знижується, тільки
тим, що коли потрібна додаткова кількість продукту для того, щоб вилучити
землю $А$, то це відбувається тут швидше.

Так за низхідної, як і за висхідної продуктивности додаткових вкладень
капіталу може це різно впливати, залежно від того, як ці вкладення розподіляються
між різними родами землі. В міру того, як цей різний вплив
\parbreak{}  %% абзац продовжується на наступній сторінці

\parcont{}  %% абзац починається на попередній сторінці
\index{iii2}{0160}  %% посилання на сторінку оригінального видання
буде вирівнювати або загострювати ріжниці, диференційна рента з кращих земель,
а разом з тим і загальна сума ренти знизиться або підвищиться, як це було
вже в випадку з диференційною рентою І.~В решті, це залежить від величини земельної
площі й капіталу, вилучених разом з $А$, і від відносного розміру авансованого
капіталу, потрібного за висхідної продуктивности для того, щоб дати
додаткову кількість продукту для покриття попиту.

Єдиний пункт, на дослідженні якого тут варто спинитися, і який взагалі
вертає нас до дослідження того, як цей диференційний зиск перетворюється
на диференційну ренту, є такий:

У першому випадку, коли ціна продукції лишається та сама, додатковий
капітал, вкладений в землю $А$, не справляє впливу на диференційну ренту, як
таку, бо земля $А$, як і давніш, не дає ренти, ціна її продукту лишається та
сама, і продовжує реґулювати ринок.

У другому випадку, варіянт І, коли ціна продукції за незмінної норми продуктивности
понижується, земля $А$ неодмінно відпадає, і ще в більшій мірі це
відбувається у варіянті II (низхідна ціна продукції за низхідної норми продуктивности),
бо в противному разі додатковий капітал, вкладений у землю $А$,
мусив би підвищити ціну продукції. Але тут, у варіянті III другого випадку,
коли ціна продукції понижується, бо продуктивність додаткового капіталу підвищується,
цей додатковий капітал за певних умов може бути вкладений так
в землю $А$, як і в землі кращої якости.

Припустімо, що додатковий капітал в 2\sfrac{1}{2}\pound{ ф. стерл.}, вкладений в землю
$А$, продукує 1\sfrac{1}{5} кварт. замість 1 квартера.

\begin{table}[h]
  \begin{center}
    \emph{Таблиця VI}
    \footnotesize

  \begin{tabular}{c@{  } c@{  } c@{  } c@{  } c@{  } c@{  } c@{  } c@{  } c@{  } c@{  } c}
    \toprule
      \multirowcell{2}{\makecell{Рід\\ землі}} &
      \multirowcell{2}{Акри} &
      Капітал &
      Зиск &
      \makecell{Ціна\\ продук.} &
      \multirowcell{2}{\makecell{Продукт в\\ квартерах}} &
      \makecell{Продажна \\ ціна} &
      \makecell{Здо-\\буток} &
      \multicolumn{2}{c}{Рента} &
      \multirowcell{2}{\makecell{Норма \\надзиску}} \\

      \cmidrule(r){3-3}
      \cmidrule(r){4-4}
      \cmidrule(r){5-5}

      \cmidrule(r){7-7}
      \cmidrule(r){8-8}
      \cmidrule(r){9-9}
      \cmidrule(r){10-10}

       &  & ф. ст. & ф. ст. & ф. ст. & & ф. ст. & ф. ст. & Кварт. & ф. ст. &   \\
      \midrule
       A & 1 & 2\sfrac{1}{2} \dplus{} 2\sfrac{1}{2} \deq{} 5 & 1 & 6 & 1 \dplus{} 1\sfrac{1}{5} \deq{} 2\sfrac{1}{5} & 2\sfrac{8}{11} & \phantom{0}6 & 0\phantom{\sfrac{1}{2}} & \phantom{0}0 & \phantom{00}0\% \\
       B & 1 & 2\sfrac{1}{2} \dplus{} 2\sfrac{1}{2} \deq{} 5 & 1 & 6 & 2 \dplus{} 2\sfrac{2}{5} \deq{} 4\sfrac{2}{5} & 2\sfrac{8}{11} & 12           & 2\sfrac{1}{5}           & \phantom{0}6 & 120\% \\
       C & 1 & 2\sfrac{1}{2} \dplus{} 2\sfrac{1}{2} \deq{} 5 & 1 & 6 & 3 \dplus{} 3\sfrac{3}{5} \deq{} 6\sfrac{3}{5} & 2\sfrac{8}{11} & 18           & 4\sfrac{2}{5}           & 12           & 240\%\\
       D & 1 & 2\sfrac{1}{2} \dplus{} 2\sfrac{1}{2} \deq{} 5 & 1 & 6 & 4 \dplus{} 4\sfrac{4}{5} \deq{} 8\sfrac{4}{5} & 2\sfrac{8}{11} & 24           & 6\sfrac{3}{5}           & 18           & 360\%\\
     \cmidrule(r){1-1}
     \cmidrule(r){2-2}
     \cmidrule(r){3-3}
     \cmidrule(r){4-4}
     \cmidrule(r){5-5}
     \cmidrule(r){6-6}

     \cmidrule(r){8-8}
     \cmidrule(r){9-9}
     \cmidrule(r){10-10}
     \cmidrule(r){11-11}

      Разом & 4 & \phantom{2\sfrac{1}{2} \dplus{} 2\sfrac{1}{2} \deq{}}20 & 4 & 24 & \phantom{2 \dplus{} 1\sfrac{1}{2} \deq{}}22\phantom{\sfrac{1}{2}} & & 60 & 13\sfrac{1}{5} & 36 & 240\%\footnotemarkZ{}\\
  \end{tabular}

  \end{center}
\end{table}
\footnotetextZ{Тут пересічну норму надзиску обчислено не до всього вкладеного капіталу, а тільки до капіталу, вкладеного в рентодайні дільниці $В$, $C$ і $D$. \emph{Прим. Ред.}} % текст примітки прямо під заголовком

Цю таблицю слід порівняти, крім основної таблиці І, і з таблицею II, в якій
подвоєне вкладення капіталу сполучається з сталою продутивністю, пропорційною
капіталовкладенню.

Згідно з припущенням, регуляційна ціна продукції понижується. Коли б
вона залишалася сталою, 3\pound{ ф. стерл.}, то найгірша земля $А$, що давніш, при
капіталовкладенні лише в 2\sfrac{1}{2}\pound{ ф. стерл.}, не давала ренти, тепер почала б давати
ренту, хоч ніякої нової найгіршої землі не було б притягнено до оброблення;
це сталося б саме в наслідок того, що продуктивність на ній збільшилася б, але
лише для частини капіталу, а не для первісно вкладеного капіталу. Перші 3\pound{ ф.
стерл.} ціни продукції дають 1 квартер; другі — 1\sfrac{1}{5} квартера; але ввесь продукт в
2\sfrac{1}{5} квартери продається тепер по його пересічній ціні. А що норма продуктивности
зростає з додатковим капіталовкладенням, то це включає й поліпшення.

\parcont{}  %% абзац починається на попередній сторінці
\index{iii2}{0161}  %% посилання на сторінку оригінального видання
Воно може бути в тому, що на акр взагалі вживається більше капіталу (більше
добрива, більше механічної праці тощо) або також в тому, що взагалі лише
додатковий капітал дає змогу перевести відзначну, якісною стороною продуктивнішу,
витрату капіталу. В обох випадках при витраті 5\pound{ ф. стерл.} на акр одержується
продукт в 2\sfrac{1}{2}  квартери, тоді як при витраті половини цього капіталу,
2\sfrac{1}{2}\pound{ ф. стерл.}, одержується продукт лише в один квартер. Продукт землі $A$,
залишаючи осторонь минущі ринкові відносини, можна було б і далі продавати
по вищій ціні продукції, замість продавати його по новій пересічній
ціні, лише доти, доки значна площа земель розряду $A$ і далі оброблялася б
з капіталом лише в 2\sfrac{1}{2}\pound{ ф. стерл.} на акр. Але скоро нове відношення
в 5\pound{ ф. стерл.} капіталу на акр, а разом з тим, поліпшене господарство набудуть
загального поширення, реґуляційна ціна продукції мусить понизитися до 2\sfrac{8}{11}\pound{ ф.
стерл}. Ріжниця між обома частинами капіталу зникла б, і тоді дійсно акр землі $A$.
оброблюваний лише з капіталом в 2\sfrac{1}{2}  ф. стерл, оброблявся б ненормально,
невідповідно до нових умов продукції. Це вже було б ріжницею не між здобутком від
різних частин капіталу, вкладених у той самий акр, а між достатньою й недостатньою
загальною витратою капіталу на акр. Звідси видно, \emph{поперше}, що недостатність
капіталу в руках більшости орендарів (це мусить бути більшість, бо коли б це
була меншість, їй довелося б лише продавати нижче від своєї ціни продукції)
впливає цілком так само, як диференціювання самих земель в низхідному порядку.
Гірший спосіб обробітку на гіршій землі збільшує ренту з кращої землі;
він може навіть створити ренту з краще оброблюваної землі такої самої кепської
якости, яка взагалі ренти не дає. Звідси видно, \emph{подруге}, що диференційна рента,
оскільки вона виникає з послідовного капіталовкладення на тій самій земельній
площі, в дійсності перетворюється на пересічну величину, в якій уже не можна
розпізнати і відрізнити впливів різних капіталовкладень, і які тому не породжують
ренти на найгіршій землі, а 1) пересічну ціну всього продукту, скажімо
з одного акра $A$, перетворюють на нову регуляційну ціну і 2) виявляються, як
зміна загальної кількости капіталу на акр, що в нових умовах потрібна для
задовільного обробітку землі, і в якій так окремі послідовні капіталовкладення, як і
їхні відповідні впливи так поєднані, що їх не можна відрізнити. Так само стоїть
справа з поодинокими диференційними рентами кращих земель. Вони визначаються
в кожному випадку ріжницею пересічного продукту відповідного роду землі порівняно
з продуктом найгіршої землі за підвищеної витрати капіталу, що тепер
стала нормальною.

Жодна земля не дає будь-якого продукту без витрати капіталу. Отже,
навіть при звичайній диференційній ренті, при диференційній ренті І; коли говорять,
що 1 акр землі $А$, що реґулює ціну продукції, дає стільки й стільки продукту,
по такій-от ціні, і що кращі землі $B$, $C$, $D$ дають стільки й стільки диференційного
продукту, а тому за даної реґуляційної ціни стільки от грошової ренти, то
тут завжди припускається, що вжито певний капітал, який в даних умовах продукції
вважається за нормальний. Цілком так само, як у промисловості для
кожної галузі підприємств потрібен певний мінімум капіталу для того, щоб
можна було виготовляти товари по їхній ціні продукції.

Якщо цей мінімум змінюється в наслідок сполучених з поліпшеннями послідовних
капіталовкладень на тій самій землі, то це відбувається поступово.
Поки в певну кількість акрів, наприклад, землі $A$ не буде вкладено такого додаткового
капіталу, доти рента з краще оброблюваних акрів землі $A$ породжуватиметься
ціною продукції, яка лишилася незмінною, а рента з усіх кращих родів землі
$B$, $C$, $D$, підвищиться. Проте, скоро новий спосіб продукції так пошириться, що
зробиться нормальним, — ціна продукції понизиться; рента з найкращих дільниць
землі знову понизиться, і та частина землі $A$, в яку капітал вкладено в розмірі,
\parbreak{}  %% абзац продовжується на наступній сторінці

\parcont{}  %% абзац починається на попередній сторінці
\index{iii2}{0162}  %% посилання на сторінку оригінального видання
меншім від того, що став тепер пересічним, муситиме продавати продукти
нижче від своєї індивідуальної ціни продукції, отже, з зиском нижчим за
пересічний.

Це саме відбувається і за низхідної ціни продукції, навіть за низхідної продуктивности
додаткового капіталу, скоро лише в наслідок збільшеної витрати капіталу
весь потрібний продукт постачатимуть кращі землі, і, отже, застосовуваний капітал
буде вилучений, наприклад, з землі $А$, так що $А$ перестає конкурувати в продукції
цього певного продукту, наприклад, пшениці. Та кількість капіталу, яка
тепер пересічно вживається на кращій землі $В$, що зробилася реґуляційною
землею, стає тепер нормальною; і коли говориться про різну родючість земельних
дільниць, то припускається, що на акр вживається така нова нормальна
кількість капіталу.

З другого боку ясно, що цей пересічний розмір вкладуваного капіталу, як,
наприклад, в Англії 8\pound{ ф. стерл.} на акр до 1848 року і 12\pound{ ф. стерл.} після 1848 року, —
становить маштаб при складанні орендних договорів. Для орендаря, що витрачає
більше, надзиск, поки триває орендний договір, не перетворюється на ренту. Чи
станеться це по закінченні орендного договору, залежатиме від конкуренції орендарів,
які можуть робити таке саме надзвичайне авансування. При цьому не мається
на увазі перманентних поліпшень ґрунту, що за однакової або навіть зменшуваної
витрати капіталу продовжують забезпечувати збільшений продукт. Ці поліпшення,
хоч вони і є продуктом капіталу, проте, діють цілком так само, як
ріжниця в природних якостях землі.

Отже, ми бачимо, що при диференційній ренті II значення має такий момент,
який при диференційній ренті І як такій не виявляється, бо остання може
і далі існувати незалежно від будь-якої зміни нормальної витрати капіталу
на акр. Це є, з одного боку, згладжування наслідків різних витрат капіталу
на реґуляційній землі $А$, що продукт з неї виступає тепер просто як нормальний
пересічний продукт з акра. З другого боку, це зміна в нормальному мінімумі
або пересічній величині витрати капіталу на акр, так що ця зміна виступає як
властивість землі. Нарешті, це — ріжниця способу перетворення надзиску на форму
ренти.

Далі таблиця VI порівняно з таблицею І і II показує, що збіжжева рента
проти І більше ніж подвоїлась, проти II збільшилась на 1\sfrac{1}{5}  квартера; тимчасом
як грошова рента проти І подвоїлась, а проти II не змінилась. Вона значно
зросла б, коли б (за інших рівних умов) більша частина додаткового капіталу
припала на землю кращих родів, або коли б з, другого боку, дія додаткового
капіталу на $А$ була б менш значна і, отже, реґуляційна пересічна ціна квартера
з $А$ була б вища.

Коли б збільшення родючости, що відбувається в наслідок додаткової витрати
капіталу, різно впливало на різних родах землі, то це призвело б до зміни
диференційних рент з цих земель.

В усякому разі доведено, що коли нижчає ціна продукції, в наслідок
підвищення норми продуктивности додаткової витрати капіталу, —
отже, коли ця продуктивність зростає у більшому відношенні, ніж авансований
капітал, — рента з акра, наприклад, при подвоєній витрати капіталу може не
тільки подвоїтись, але й більше, ніж подвоїтись. Але вона може і знизитись,
коли в наслідок швидкого зростання продуктивности землі $А$ ціна продукції
зменшиться ще в значно більшій мірі.

Коли б ми припустили, що додаткові витрати капіталу, наприклад, на землях
$В$ і $C$ збільшили продуктивність не в такій мірі, як на землі $А$, так що для земель
$В$ і $C$ відносні ріжниці зменшуються і приріст продукту не компенсує пониження
ціни, то проти таблиці II рента на $D$ підвищилася б, на $В$ і $C$ знизилася
б.


\begin{table}[H]
  \begin{center}
    \emph{Таблиця VIa}
    \footnotesize

  \begin{tabular}{c@{  } c@{  } c@{  } c@{  } c@{  } c@{  } c@{  } c@{  } c}
    \toprule
      \multirowcell{2}{Земля} &
      \multirowcell{2}{Акри} &
      Капітал &
      Зиск &
      \multirowcell{2}{\makecell{Продукт з акра\\ в квартерах}} &
      \makecell{Продажна \\ ціна} &
      %% TODO: Якась проблема з роздільними лініями починаючи зі здобуток
      \makecell{Здобу-\\ток} &
      \multicolumn{2}{c}{Рента} \\

      \cmidrule(r){3-3}
      \cmidrule(r){4-4}
      \cmidrule(r){6-6}

      \cmidrule(r){7-7}
      \cmidrule(r){8-8}
      \cmidrule(r){9-9}

       &  & ф. ст. & ф. ст. & & ф. ст. & ф. ст. & Кварт. & ф. ст.   \\
      \midrule
       A & 1 & 2\sfrac{1}{2} \dplus{} 2\sfrac{1}{2} \deq{} 5 & 1 & 1 \dplus{} \phantom{0}3\phantom{\sfrac{1}{2}} \deq{} \phantom{0}4\phantom{\sfrac{1}{2}}   & 1\sfrac{1}{2} & \phantom{0}6\phantom{\sfrac{3}{4}} & \phantom{0}0\phantom{\sfrac{1}{2}}\footnotemarkZ{}  & \phantom{0}0\phantom{\sfrac{1}{2}} \\
       B & 1 & 2\sfrac{1}{2} \dplus{} 2\sfrac{1}{2} \deq{} 5 & 1 & 2 \dplus{} \phantom{0}2\sfrac{1}{2} \deq{} \phantom{0}4\sfrac{1}{2}                       & 1\sfrac{1}{2} & \phantom{0}6\sfrac{3}{4}           & \phantom{00}\sfrac{1}{2}                            & \phantom{00}\sfrac{3}{4}           \\
       C & 1 & 2\sfrac{1}{2} \dplus{} 2\sfrac{1}{2} \deq{} 5 & 1 & 3 \dplus{} \phantom{0}5\phantom{\sfrac{1}{2}} \deq{} \phantom{0}8\phantom{\sfrac{1}{2}}   & 1\sfrac{1}{2} & 12\phantom{\sfrac{3}{4}}           & \phantom{0}4\phantom{\sfrac{1}{2}}                  & \phantom{0}6\phantom{\sfrac{1}{2}} \\
       D & 1 & 2\sfrac{1}{2} \dplus{} 2\sfrac{1}{2} \deq{} 5 & 1 & 4 \dplus{} 12\phantom{\sfrac{1}{2}} \deq{} 16\phantom{\sfrac{1}{2}}                       & 1\sfrac{1}{2} & 24\phantom{\sfrac{3}{4}}           & 12\phantom{\sfrac{1}{2}}                            & 18\phantom{\sfrac{1}{2}}           \\
     \cmidrule(r){1-1}
     \cmidrule(r){2-2}
     \cmidrule(r){3-3}
     \cmidrule(r){4-4}
     \cmidrule(r){5-5}
     \cmidrule(r){6-6}

     \cmidrule(r){8-8}
     \cmidrule(r){9-9}

      Разом & 4 & \phantom{2\sfrac{1}{2} \dplus{} 2\sfrac{1}{2} \deq{}}20 & & \phantom{2 \dplus{} 12\sfrac{1}{2} \deq{}}32\sfrac{1}{2} & & & 16\sfrac{1}{2} & 24\sfrac{3}{4}\\
  \end{tabular}

  \end{center}
\end{table}
\footnotetextZ{В німецькому тексті тут очевидно помилково стоїть «6» \emph{Прим. Ред.}} % текст примітки прямо під заголовком

Нарешті, грошова рента підвищилася б, коли б у кращі земельні дільниці,
при тому самому відносному підвищенні родючости, вкладено було більше
додаткового капіталу, ніж у землю $А$, або коли б додаткові вкладання капіталу в кращі
земельні дільниці впливали, підвищуючи норму продуктивности. В обох випадках
ріжниці зростали б.

Грошова рента понижується, коли поліпшенння, що сталося в наслідок
додаткової витрати капіталу, зменшує всі ріжниці, або частину їх, впливаючи
більше на $А$, ніж на $В$ і $C$. Вона понижується то більше, що незначніше
підвищення продуктивности кращих земельних дільниць. Від відносної неоднаковости
впливу залежить, чи підвищиться збіжжева рента, чи понизиться або
залишиться без зміни.

Грошова рента підвищується, а також і збіжжева рента, або тоді, коли за
незмінної відносної ріжниці в додатковій родючості різних земель більше вкладається
додаткового капіталу в землю, що дає ренту, ніж у землю $А$, що не дає
ренти, і більше у землю, що дає вищу, ніж у землю, що дає нижчу ренту;
абож тоді, коли родючість, при однаковому додатковому капіталі, більше зростає
на кращій і найкращій землі, ніж на землі $А$, причому грошова і збіжжева
рента підвищується саме у такому відношенні, в якому це збільшення родючости
на вищих розрядах землі вище, ніж на нижчих.

Але за всяких обставин рента відносно підвищується, коли підвищена продуктивність
є наслідок додаткової витрати капіталу, а не просто наслідок
збільшеної родючости за незмінної витрати капіталу. Це є абсолютний погляд,
який показує, що тут, як і в усіх давніших випадках, рента і збільшена рента з акра
(подібно до того, як при диференційній ренті І висота пересічної ренти на всю
оброблювану площу) є наслідок збільшеної витрати капіталу на землю, при
чому байдуже, чи функціонує ця витрата з сталою нормою продуктивности за
сталих або понижених цін, чи з низхідною нормою продуктивности за сталих або
за понижених цін, чи з висхідною нормою продуктивности за понижених цін.
Бо наше припущення: стала ціна за сталої, низхідної або висхідної норми продуктивности додаткового
капіталу, і низхідна ціна, за сталої, низхідної і висхідної
норми продуктивности, зводиться ось до чого: стала норма продуктивности додаткового капіталу при
сталій або низхідній ціні, низхідна норма продуктивности
при сталій або низхідній ціні, висхідна норма продуктивности за сталої
\parbreak{}  %% абзац продовжується на наступній сторінці

\parcont{}  %% абзац починається на попередній сторінці
\index{iii2}{0164}  %% посилання на сторінку оригінального видання
і низхідної ціни. Хоч в усіх цих випадках рента може залишитися без зміни
і може понизитись, вона понизилася б значніше, коли б додаткове вживання
капіталу, за інших незмінних обставин не зумовлювало збільшення родючости. Додаткове вкладення
капіталу тоді завжди є за причину відносної висоти ренти,
хоча б абсолютно вона й понизилась.

\section{Дифереційна рента II. — Третій випадок:
висхідна ціна продукції}

[Підвищення ціни продукції має за свою передумову, що продуктивність землі
найгіршої якости, що не дає ренти, зменшується. Ціна продукції, взята нами за
реґуляційну, може піднестися вище від 3\pound{ ф. ст.} за кв., лише тоді, коли 2\sfrac{1}{2}\pound{ ф. ст.},
витрачені на $А$, продукуватимуть менш за 1 квартер, або 5\pound{ ф. ст.} менш за
2 квартери, або коли б довелося обробляти землю ще гіршої якости, ніж $А$.

За незмінної або навіть висхідної продуктивности другого вкладення капіталу
це було б можливе лише тоді, коли б продуктивність першого вкладення в 2\sfrac{1}{2} ф. cт.
зменшилась. Цей випадок трапляється досить часто. Наприклад, коли виснажений
при поверховій оранці зверхній шар ґрунту дає при старій системі обробітку
дедалі менші врожаї, то витягнений на поверхню з допомогою глибшої
оранки нижній шар за раціонального обробітку починає давати вищі
урожаї, ніж давніш. Але цей сцеціяльний випадок, точно кажучи, сюди не
стосується. Пониження продуктивности першої витрати капіталу в 2\sfrac{1}{2}\pound{ ф. ст.} зумовлює для кращих
земель, навіть коли там припустити аналогічні відношення,
пониження диференційної ренти І; проте тут ми розглядаємо лише диференційну
ренту II. Але тому, що даний спеціяльний випадок не може статися, коли не
припускається існування диференційної ренти II і тому, що він в дійсності
становить відбитий вплив модифікації диференційної ренти І на диференційну
ренту II, то ми наведемо приклад цього випадку.

\begin{table}[H]
  \begin{center}
    \emph{Таблиця VII}
    \footnotesize

  \begin{tabular}{c@{  } c@{  } c@{  } c@{  } c@{  } c@{  } c@{  } c@{  } c@{  } c@{  } c}
    \toprule
      \multirowcell{2}{\makecell{Рід\\ землі}} &
      \multirowcell{2}{Акри} &
      Капітал &
      Зиск &
      \makecell{Ціна\\ продук.} &
      \multirowcell{2}{\makecell{Продукт в\\ квартерах}} &
      \makecell{Продажна \\ ціна} &
      \makecell{Здо-\\буток} &
      \multicolumn{2}{c}{Рента} &
      \multirowcell{2}{\makecell{Норма \\ренти}} \\

      \cmidrule(r){3-3}
      \cmidrule(r){4-4}
      \cmidrule(r){5-5}
      \cmidrule(r){7-7}
      \cmidrule(r){8-8}
      \cmidrule(r){9-9}
      \cmidrule(r){10-10}

       &  & ф. ст. & ф. ст. & ф. ст. & & ф. ст. & ф. ст. & Кварт. & ф. ст. &   \\
      \midrule
      A & 1 & 2\sfrac{1}{2} + 2\sfrac{1}{2} = 5 & 1 & 6 & \phantom{0}\sfrac{1}{2} + 1\sfrac{1}{4} = 1\sfrac{3}{4}                      & 3\sfrac{3}{7} & \phantom{0}6 & 0\phantom{\sfrac{1}{2}} & \phantom{0}0 & \phantom{00}0\% \\
      B & 1 & 2\sfrac{1}{2} + 2\sfrac{1}{2} = 5 & 1 & 6 & 1\phantom{\sfrac{0}{0}} + 2\sfrac{1}{2} = 3\sfrac{1}{2}                     & 3\sfrac{3}{7} & 12           & 1\sfrac{3}{4}           & \phantom{0}6 & 120\% \\
      C & 1 & 2\sfrac{1}{2} + 2\sfrac{1}{2} = 5 & 1 & 6 & 1\sfrac{1}{2} + 3\sfrac{3}{4} = 5\sfrac{1}{4}                               & 3\sfrac{3}{7} & 18           & 3\sfrac{1}{2}           & 12           & 240\%\\
      D & 1 & 2\sfrac{1}{2} + 2\sfrac{1}{2} = 5 & 1 & 6 & 2\phantom{\sfrac{0}{0}} + 5\phantom{\sfrac{0}{0}} = 7\phantom{\sfrac{0}{0}} & 3\sfrac{3}{7} & 24           & 5\sfrac{1}{4}           & 18           & 360\%\\

     \cmidrule(r){3-3}
     \cmidrule(l){6-6}
     \cmidrule(r){8-8}
     \cmidrule(r){9-9}
     \cmidrule(r){10-10}
     \cmidrule(r){11-11}

      Разом & & \phantom{2\sfrac{1}{2} + 2\sfrac{1}{2} =}20 & & & \phantom{2 + 1\sfrac{1}{2} =}17\sfrac{1}{2} & & 60 & 10\sfrac{1}{2} & 36 & 240\%\footnotemarkZ{}\\
  \end{tabular}

  \end{center}
\end{table}
\footnotetextZ{Тут, як і далі в таблицях VIII, IX, і X пересічну норму ренти обчислено не до всього
вкладеного капіталу, а тільки до капіталу, вкладеного в рентодайні дільниці. \emph{Пр. Ред.}} % текст примітки прямо під заголовком

Грошова рента, як і грошовий здобуток, лишаються ті самі, що і в таблиці II.
Підвищена реґуляційна ціна продукції точнісінько покриває те, що втрачено
на кількості продукту; а що ця ціна продукції і кількість продукту змінюються
в зворотному відношенні, то само собою зрозуміло, що здобуток їх лишається
той самий.


\index{iii2}{0165}  %% посилання на сторінку оригінального видання
У вищенаведеному випадку ми припускали, що продуктивна сила другого
капіталовкладення вища, ніж первісна продуктивність першого вкладення. Справа
не зміниться, коли ми припустимо для другого капіталовкладення лише таку саму
продуктивність, що її мала первісна продуктивність першого вкладення, як от у
таблиці VIII.

\begin{table}[h]
  \begin{center}
    \emph{Таблиця VIII}
    \footnotesize

  \begin{tabular}{c@{  } c@{  } c@{  } c@{  } c@{  } c@{  } c@{  } c@{  } c@{  } c@{  } c}
    \toprule
      \multirowcell{2}{\makecell{Рід\\ землі}} &
      \multirowcell{2}{Акри} &
      Капітал &
      Зиск &
      \makecell{Ціна\\ продук.} &
      \multirowcell{2}{\makecell{Продукт в\\ квартерах}} &
      \makecell{Продажна \\ ціна} &
      \makecell{Здо-\\буток} &
      \multicolumn{2}{c}{Рента} &
      \multirowcell{2}{\makecell{Норма \\надзиску}} \\

      \cmidrule(r){3-3}
      \cmidrule(r){4-4}
      \cmidrule(r){5-5}
      \cmidrule(r){7-7}
      \cmidrule(r){8-8}
      \cmidrule(r){9-9}
      \cmidrule(r){10-10}

       &  & ф. ст. & ф. ст. & ф. ст. & & ф. ст. & ф. ст. & Кварт. & ф. ст. &   \\
      \midrule
      A & 1 & 2\sfrac{1}{2} + 2\sfrac{1}{2} = 5 & 1 & 6 & \phantom{0}\sfrac{1}{2} + 1 = 1\sfrac{1}{2}                                 & 4 & \phantom{0}6 & 0\phantom{\sfrac{1}{2}} & \phantom{0}0 & \phantom{00}0\% \\
      B & 1 & 2\sfrac{1}{2} + 2\sfrac{1}{2} = 5 & 1 & 6 & 1\phantom{\sfrac{0}{0}} + 2 = 3\phantom{\sfrac{0}{0}}                       & 4 & 12           & 1\sfrac{1}{2}           & \phantom{0}6 & 120\% \\
      C & 1 & 2\sfrac{1}{2} + 2\sfrac{1}{2} = 5 & 1 & 6 & 1\sfrac{1}{2} + 3 = 4\sfrac{1}{4}                                           & 4 & 18           & 3\phantom{\sfrac{1}{2}} & 12           & 240\%\\
      D & 1 & 2\sfrac{1}{2} + 2\sfrac{1}{2} = 5 & 1 & 6 & 2\phantom{\sfrac{0}{0}} + 4 = 6\phantom{\sfrac{0}{0}} & 4 & 24           & 4\sfrac{1}{2}           & 18           & 360\%\\

     \cmidrule(r){3-3}
     \cmidrule(l){6-6}
     \cmidrule(r){8-8}
     \cmidrule(r){9-9}
     \cmidrule(r){10-10}
     \cmidrule(r){11-11}

      Разом & & \phantom{2\sfrac{1}{2} + 2\sfrac{1}{2} =}20 & & & \phantom{2 + 1\sfrac{1}{2} =}15 & & 60 & 9 & 36 & 240\%\\
  \end{tabular}

  \end{center}
\end{table}

І тут ціна продукції, яка підвищується в тому самому відношенні, зумовлює
те, що зменшення продуктивности цілком урівноважується так щодо здобутку,
як і щодо грошової ренти.

У своєму чистому вигляді третій випадок виступає лише за низхідної продуктивности
другого капіталовкладення, тимчасом як продуктивність першого вкладення,
як це всюди припускалось для першого і другого випадків, лишається сталою.
Диференційна рента І тут не зачіпається, зміна відбувається лише з тією частиною,
що походить з диференційної ренти II.~Ми подаємо два приклади: в
першому продуктивність другого капіталовкладення зводиться до \sfrac{1}{2}, у другому
— до \sfrac{1}{4} продуктивности першого вкладення.

\begin{table}[h]
  \begin{center}
    \emph{Таблиця IX}
    \footnotesize

  \begin{tabular}{c@{  } c@{  } c@{  } c@{  } c@{  } c@{  } c@{  } c@{  } c@{  } c@{  } c}
    \toprule
      \multirowcell{2}{\makecell{Рід\\ землі}} &
      \multirowcell{2}{Акри} &
      Капітал &
      Зиск &
      \makecell{Ціна\\ продук.} &
      \multirowcell{2}{\makecell{Продукт в\\ квартерах}} &
      \makecell{Продажна \\ ціна} &
      \makecell{Здо-\\буток} &
      \multicolumn{2}{c}{Рента} &
      \multirowcell{2}{\makecell{Норма \\ренти}} \\

      \cmidrule(r){3-3}
      \cmidrule(r){4-4}
      \cmidrule(r){5-5}
      \cmidrule(r){7-7}
      \cmidrule(r){8-8}
      \cmidrule(r){9-9}
      \cmidrule(r){10-10}

       &  & ф. ст. & ф. ст. & ф. ст. & & ф. ст. & ф. ст. & Кварт. & ф. ст. &   \\
      \midrule
      A & 1 & 2\sfrac{1}{2} + 2\sfrac{1}{2} = 5 & 1 & 6 & 1 + \phantom{0}\sfrac{1}{2} = 1\sfrac{1}{2}                                 & 4 & \phantom{0}6 & 0\phantom{\sfrac{1}{2}} & \phantom{0}0 & \phantom{00}0\% \\
      B & 1 & 2\sfrac{1}{2} + 2\sfrac{1}{2} = 5 & 1 & 6 & 2 + 1\phantom{\sfrac{0}{0}} = 3\phantom{\sfrac{0}{0}}                       & 4 & 12           & 1\sfrac{1}{2}           & \phantom{0}6 & 120\% \\
      C & 1 & 2\sfrac{1}{2} + 2\sfrac{1}{2} = 5 & 1 & 6 & 3 + 1\sfrac{1}{2} = 4\sfrac{1}{2}                                           & 4 & 18           & 3\phantom{\sfrac{1}{2}} & 12           & 240\%\\
      D & 1 & 2\sfrac{1}{2} + 2\sfrac{1}{2} = 5 & 1 & 6 & 4 + 2\phantom{\sfrac{0}{0}} = 6\phantom{\sfrac{0}{0}} & 4 & 24           & 4\sfrac{1}{2}           & 18           & 360\%\\

     \cmidrule(r){3-3}
     \cmidrule(l){6-6}
     \cmidrule(r){8-8}
     \cmidrule(r){9-9}
     \cmidrule(r){10-10}
     \cmidrule(r){11-11}

      Разом & & \phantom{2\sfrac{1}{2} + 2\sfrac{1}{2} =}20 & & & \phantom{2 + 1\sfrac{1}{2} =}15 & & 60 & 9 & 36 & 240\%\\
  \end{tabular}

  \end{center}
\end{table}

Таблиця IX та сама, що й таблиця VIII, тільки в таблиці VIII зменшення
продуктивности припадає на перше, в таблиці IX — на друге капіталовкладення.


\begin{table}[H]
  \centering
  \caption*{Таблиця X}

  \footnotesize
  \setlength{\tabcolsep}{4.5pt}
  \settowidth\rotheadsize{\theadfont Продажна}

  \begin{tabular}{l c r c c r c c c c c}
    \toprule
      \thead[tl]{Рід\\землі} &
      &
      \thead[t]{Капітал} &
      \rothead{Зиск} &
      \rothead{Ціна\\продукції} &
      \thead[t]{Продукт} & % \\ в кварт.}}}
      \rothead{Продажна\\ціна} &
      \rothead{Здобуток} &
      \multicolumn{2}{c}{Рента} &
      \rothead{Норма\\надзиску} \\

    \cmidrule(rl){2-11}
      & акри  & \poundsign{} & \poundsign{} & \poundsign{} & кв. & \poundsign{} & \poundsign{} & кв. & \poundsign{} & \% \\

    \midrule
      A & 1 & 2\tbfrac{1}{2} \dplus{} 2\tbfrac{1}{2} \deq{} 5 & 1 & 6 & 1 \dplus{} \phantom{1}\tbfrac{1}{4} \deq{} 1\tbfrac{1}{4}            & 4\tbfrac{4}{5} & \phantom{0}6 & 0\phantom{\tbfrac{1}{2}} & \phantom{0}0 & \phantom{00}0\\
      B & 1 & 2\tbfrac{1}{2} \dplus{} 2\tbfrac{1}{2} \deq{} 5 & 1 & 6 & 2 \dplus{} \phantom{1}\tbfrac{1}{2} \deq{} 2\tbfrac{1}{2}            & 4\tbfrac{4}{5} & 12           & 1\tbfrac{1}{4}           & \phantom{0}6 & 120\\
      C & 1 & 2\tbfrac{1}{2} \dplus{} 2\tbfrac{1}{2} \deq{} 5 & 1 & 6 & 3 \dplus{} \phantom{1}\tbfrac{3}{4} \deq{} 3\tbfrac{3}{4}            & 4\tbfrac{4}{5} & 18           & 2\tbfrac{1}{2}           & 12           & 240\\
      D & 1 & 2\tbfrac{1}{2} \dplus{} 2\tbfrac{1}{2} \deq{} 5 & 1 & 6 & 4 \dplus{} 1\phantom{\tbfrac{1}{1}} \deq{} 5\phantom{\tbfrac{1}{1}}  & 4\tbfrac{4}{5} & 24           & 3\tbfrac{3}{4}           & 18           & 360\\

    \midrule
      Разом & & 20 & & \hang{r}{2}4 & 12\tbfrac{1}{2} & & 60 & 7\tbfrac{1}{2} & 36 & 240\\
  \end{tabular}
\end{table}

\noindent{}В цій таблиці загальний здобуток, сума грошової ренти і норма ренти
теж лишаються такі самі, як у таблицях, II, VII і VIII, бо продукт і продажна
ціна знов таки змінились у зворотному відношенні, а капіталовкладення лишилось
те саме.

Але як стоїть справа в іншому випадку, можливому за висхідної ціни
продукції, а саме в тому випадку, коли гірша земля, яку до цього часу не
варто було обробляти, тепер починає оброблятись.

Припустімо, що така земля, яку ми позначимо \emph{а}, вступає в конкуренцію.
Тоді земля $А$, що не давала до цього часу ренти, почала б давати ренту, і
вищенаведені таблиці VII, VIII і X набули б такого вигляду:

\begin{table}[H]
  \centering
  \caption*{Таблиця VIIa}

  \footnotesize
  \setlength{\tabcolsep}{4.5pt}
  \settowidth\rotheadsize{\theadfont Продажна}

  \begin{tabular}{l c r c c r c c c c c}
    \toprule
      \thead[tl]{Рід\\землі} &
      &
      \thead[t]{Капітал} &
      \rothead{Зиск} &
      \rothead{Ціна\\продукції} &
      \thead[t]{Продукт} & % \\ в кварт.}}}
      \rothead{Продажна\\ціна} &
      \rothead{Здобуток} &
      \multicolumn{2}{c}{Рента} &
      \thead[t]{Підвищення} \\

    \cmidrule(rl){2-11}
      & акри  & \poundsign{} & \poundsign{} & \poundsign{} & кв. & \poundsign{} & \poundsign{} & кв. & \poundsign{} & \\

    \midrule
      a & 1 & \phantom{2\tbfrac{1}{2} \dplus{} }5\phantom{\tbfrac{1}{2}} & 1 & 6 & \phantom{1\tbfrac{1}{2} \dplus{} 3\tbfrac{3}{4} \deq{} }1\tbfrac{1}{2}                     & 4 & \phantom{0}6 & 0\phantom{\tbfrac{1}{2}} & \phantom{0}0 & 0\phantom{+ 3 × 7} \\
      A & 1 & 2\tbfrac{1}{2} \dplus{} 2\tbfrac{1}{2}                     & 1 & 6 & \phantom{0}\tbfrac{1}{2} \dplus{} 1\tbfrac{1}{4} \deq{} 1\tbfrac{3}{4}                     & 4 & \phantom{0}7 & \phantom{}\tbfrac{1}{4}  & \phantom{0}1 & 1\phantom{+ 3 × 7} \\
      B & 1 & 2\tbfrac{1}{2} \dplus{} 2\tbfrac{1}{2}                     & 1 & 6 & 1\phantom{\tbfrac{1}{2}} \dplus{} 2\tbfrac{1}{2} \deq{} 3\tbfrac{1}{2}                     & 4 & 14           & 2\phantom{\tbfrac{1}{2}} & \phantom{0}8 & 1 \dplus{} 7\phantom{ × 7} \\
      C & 1 & 2\tbfrac{1}{2} \dplus{} 2\tbfrac{1}{2}                     & 1 & 6 & 1\tbfrac{1}{2} \dplus{} 3\tbfrac{3}{4} \deq{} 5\tbfrac{1}{4}                               & 4 & 21           & 3\tbfrac{3}{4}           & 15           & 1 \dplus{} 2 × 7\\
      D & 1 & 2\tbfrac{1}{2} \dplus{} 2\tbfrac{1}{2}                     & 1 & 6 & 2\phantom{\tbfrac{1}{2}} \dplus{} 5\phantom{\tbfrac{1}{2}} \deq{} 7\phantom{\tbfrac{1}{2}} & 4 & 28           & 5\tbfrac{1}{2}           & 22           & 1 \dplus{} 3 × 7\\

    \midrule
      Разом & & & & \hang{r}{3}0 & \phantom{2 \dplus{} 1\tbfrac{1}{2} \deq{}}19\phantom{\tbfrac{1}{2}} & & 76 & 11\tbfrac{1}{2} & 46 & \\
  \end{tabular}
\end{table}

\begin{table}[H]
  \centering
  \caption*{Таблиця VIIIa}

  \footnotesize
  \setlength{\tabcolsep}{4.5pt}
  \settowidth\rotheadsize{\theadfont Продажна}

  \begin{tabular}{l c r c c r c c c c c}
    \toprule
      \thead[tl]{Рід\\землі} &
      &
      \thead[t]{Капітал} &
      \rothead{Зиск} &
      \rothead{Ціна\\продукції} &
      \thead[t]{Продукт} & % \\ в кварт.}}}
      \rothead{Продажна\\ціна} &
      \rothead{Здобуток} &
      \multicolumn{2}{c}{Рента} &
      \thead[t]{Підвищення} \\

    \cmidrule(rl){2-11}
      & акри  & \poundsign{} & \poundsign{} & \poundsign{} & кв. & \poundsign{} & \poundsign{} & кв. & \poundsign{} & \\

    \midrule
      a & 1 & \phantom{2\tbfrac{1}{2} \dplus{} }5\phantom{\tbfrac{1}{2}} & 1 & 6 & \phantom{1\tbfrac{1}{2} \dplus{} 3 \deq{} }1\tbfrac{1}{4}           & 4\tbfrac{4}{5} & \phantom{0}6\phantom{\tbfrac{1}{5}} & 0\phantom{\tbfrac{1}{2}} & \phantom{0}0             & 0\phantom{\tbfrac{1}{5} \dplus{} 3 × 7\tbfrac{1}{5}} \\
      A & 1 & 2\tbfrac{1}{2} \dplus{} 2\tbfrac{1}{2}                     & 1 & 6 & \phantom{0}\tbfrac{1}{2} \dplus{} 1 \deq{} 1\tbfrac{1}{2}           & 4\tbfrac{4}{5} & \phantom{0}7\tbfrac{1}{5}           & \phantom{}\tbfrac{1}{4}  & \phantom{0}1\tbfrac{1}{5} & 1\tbfrac{1}{5}\phantom{ \dplus{} 3 × 7\tbfrac{1}{5}} \\
      B & 1 & 2\tbfrac{1}{2} \dplus{} 2\tbfrac{1}{2}                     & 1 & 6 & 1\phantom{\tbfrac{1}{2}} \dplus{} 2 \deq{} 3\phantom{\tbfrac{1}{2}} & 4\tbfrac{4}{5} & 14\tbfrac{2}{5}                     & 1\phantom{\tbfrac{3}{4}} & \phantom{0}8\tbfrac{2}{5} & 1\tbfrac{1}{5} \dplus{} 7\tbfrac{1}{5}\phantom{ × 7} \\
      C & 1 & 2\tbfrac{1}{2} \dplus{} 2\tbfrac{1}{2}                     & 1 & 6 & 1\tbfrac{1}{2} \dplus{} 3 \deq{} 4\tbfrac{1}{2}                     & 4\tbfrac{4}{5} & 21\tbfrac{3}{5}                     & 2\tbfrac{1}{4}           & 15\tbfrac{3}{5}           & 1\tbfrac{1}{5} \dplus{} 2 × 7\tbfrac{1}{5}\\
      D & 1 & 2\tbfrac{1}{2} \dplus{} 2\tbfrac{1}{2}                     & 1 & 6 & 2\phantom{\tbfrac{1}{2}} \dplus{} 4 \deq{} 6\phantom{\tbfrac{1}{2}} & 4\tbfrac{4}{5} & 28\tbfrac{4}{5}                     & 4\tbfrac{3}{4}           & 22\tbfrac{4}{5}           & 1\tbfrac{1}{5} \dplus{} 3 × 7\tbfrac{1}{5}\\

    \midrule
      Разом & 5 & & & \hang{r}{3}0 & \phantom{2 \dplus{} 1\tbfrac{1}{2} \deq{}}16\tbfrac{1}{4} & & 78\phantom{\tbfrac{1}{5}} & 9\phantom{\tbfrac{1}{2}} & 48 & \\
  \end{tabular}
\end{table}


\begin{table}[H]
  \begin{center}
    \emph{Таблиця Xa}
    \footnotesize

  \begin{tabular}{c@{  } c@{  } c@{  } c@{  } c@{  } c@{  } c@{  } c@{  } c@{  } c@{  } c}
    \toprule
      \multirowcell{2}{\makecell{Рід\\ землі}} &
      \multirowcell{2}{Акри} &
      Капітал &
      Зиск &
      \makecell{Ціна\\ продук.} &
      \multirowcell{2}{\makecell{Продукт в\\ квартерах}} &
      \makecell{Продажна \\ ціна} &
      \makecell{Здо-\\буток} &
      \multicolumn{2}{c}{Рента} &
      \multirowcell{2}{Підвищення} \\

      \cmidrule(r){3-3}
      \cmidrule(r){4-4}
      \cmidrule(r){5-5}
      \cmidrule(r){7-7}
      \cmidrule(r){8-8}
      \cmidrule(r){9-9}
      \cmidrule(r){10-10}

       &  & ф. ст. & ф. ст. & ф. ст. & & ф. ст. & ф. ст. & Кварт. & ф. ст. &   \\
      \midrule
      a & 1 & \phantom{2\sfrac{1}{2} \dplus{} }5\phantom{\sfrac{1}{2}} & 1 & 6 & \phantom{1\sfrac{1}{2} \dplus{} 3 \deq{} }1\sfrac{1}{8}           & 5\sfrac{1}{3} & \phantom{0}6\phantom{\sfrac{1}{5}} & 0\phantom{\sfrac{1}{2}}  & \phantom{0}0\phantom{\sfrac{1}{1}} & 0\phantom{\sfrac{1}{5} \dplus{} 3 × 7\sfrac{1}{5}} \\
      A & 1 & 2\sfrac{1}{2} \dplus{} 2\sfrac{1}{2}                     & 1 & 6 & 1 \dplus{} \phantom{0}\sfrac{1}{4} \deq{} 1\sfrac{1}{4}           & 5\sfrac{1}{3} & \phantom{0}6\sfrac{2}{3}           & \phantom{0}\sfrac{1}{8}  & \phantom{00}\sfrac{2}{3}           & \sfrac{2}{3}\phantom{ \dplus{} 3 × 7\sfrac{1}{5}} \\
      B & 1 & 2\sfrac{1}{2} \dplus{} 2\sfrac{1}{2}                     & 1 & 6 & 2 \dplus{} \phantom{0}\sfrac{1}{2} \deq{} 2\sfrac{1}{2}           & 5\sfrac{1}{3} & 13\sfrac{1}{3}                     & 1\sfrac{3}{8}            & \phantom{0}7\sfrac{1}{3}           & \sfrac{2}{3} \dplus{} 6\sfrac{2}{3}\phantom{ 1 ×} \\
      C & 1 & 2\sfrac{1}{2} \dplus{} 2\sfrac{1}{2}                     & 1 & 6 & 3 \dplus{} \phantom{0}\sfrac{3}{4} \deq{} 3\sfrac{3}{4}           & 5\sfrac{1}{3} & 20\phantom{\sfrac{3}{5}}           & 2\sfrac{5}{8}            & 14\phantom{\sfrac{3}{5}}           & \sfrac{2}{3} \dplus{} 2 × 6\sfrac{2}{3}\\
      D & 1 & 2\sfrac{1}{2} \dplus{} 2\sfrac{1}{2}                     & 1 & 6 & 4 \dplus{} 1\phantom{\sfrac{0}{0}} \deq{} 5\phantom{\sfrac{0}{0}} & 5\sfrac{1}{3} & 26\sfrac{2}{3}                     & 3\sfrac{7}{8}            & 20\sfrac{2}{3}                     & \sfrac{2}{3} \dplus{} 3 × 6\sfrac{2}{3}\\

     \cmidrule(r){5-5}
     \cmidrule(r){6-6}
     \cmidrule(r){8-8}
     \cmidrule(r){9-9}
     \cmidrule(r){10-10}

      Разом & & & & 30 & \phantom{2 \dplus{} 1\sfrac{1}{2} \deq{}}13\sfrac{5}{8} & & 72\sfrac{2}{3} & 8\phantom{\sfrac{1}{2}} & 42\sfrac{2}{3} & \\
  \end{tabular}

  \end{center}
\end{table}

Приєднанням землі \emph{а} породжується нову диференційну ренту І; на цій
новій основі розвивається потім диференційна рента II теж у зміненому вигляді.
Земля \emph{а} має в кожній з трьох вищенаведених таблиць ріжну родючість; ряд
відповідно висхідних ступенів родючости починається лише з $А$. Відповідно до
цього розміщується і ряд висхідних рент. Рента з найгіршої рентодайної землі,
що раніш ренти не давала, становить постійну величину, яка просто приєднується
до всіх вищих рент; лише за вирахуванням цієї сталої величини ясно виступає
при порівнянні вищих рент ряд ріжниць і його паралелізм з рядом, що
визначає родючість різних земель. У всіх таблицях різні ступені родючости, починаючи
з $А$ до $D$, стосуються один до одного, як $1: 2 : 3 : 4$, і відповідно до
цього ренти стосуються одна до однієї:

\begin{tabular}{l}
в VIIa, як $1 : 1 \dplus{} 7 : 1 \dplus{} 2 × 7 : 1 \dplus{} 3 × 7$,\\
в VIIIa, як $1\sfrac{1}{5}:1\sfrac{1}{5} \dplus{} 7\sfrac{1}{5} : 1\sfrac{1}{5} \dplus{} 2 × 7\sfrac{1}{5} : 1\sfrac{1}{5} \dplus{} 3 × 7\sfrac{1}{5}$,\\
в Xa, як $\sfrac{2}{3} : \sfrac{2}{3} \dplus{} 6\sfrac{2}{3} : \sfrac{2}{3} \dplus{} 2 × 6\sfrac{2}{3} : \sfrac{2}{3} \dplus{} 3 × 6\sfrac{2}{3}$.\\
\end{tabular}

\noindent{}Коротко: коли рента з $А \deq{} n$, а рента з землі безпосередньо вищої родючости
$= n \dplus{} m$, то ряд буде такий: $n: n \dplus{} m: n \dplus{} 2m : n \dplus{} З m$ і~\abbr{т. д.} —~Ф.~Е.]

\pfbreak

[А що вищенаведений третій випадок в рукопису не був опрацьований —
там є лише його заголовок, — то завдання редактора було по змозі доповнити
це, як зроблено вище. Але йому лишається ще зробити загальні висновки, що
випливають з усього попереднього дослідження диференційної ренти II в її трьох
головних випадках і дев’ятьох похідних випадках. Але для цієї мети наведені
в рукопису випадки придаються лише дуже мало. Поперше, в них порівнюються
дільниці землі, що з них здобутки для площ однакової величини стосуються
як $1: 2 : 3 : 4$; отже, беруться ріжниці, що вже від самого початку дуже перебільшені,
і які в дальшому розвитку зроблених на цій основі припущень і обчислень
призводять до цілком насильницьких числових відношень. Але подруге,
\parbreak{}  %% абзац продовжується на наступній сторінці

\parcont{}  %% абзац починається на попередній сторінці
\index{iii2}{0168}  %% посилання на сторінку оригінального видання
вони спричинюють зовсім фалшиве уявлення. Коли для ступенів родючости, що
стосуються один до одного, як $1: 2 : 3 : 4$ тощо, виникають ренти ряду $0 : 1 : 2 : 3$
тощо, то зараз же постає спокуса вивести другий ряд з першого і пояснити
подвоєння, потроєння тощо рент, подвоєнням потроєнням тощо всього здобутку.
Але це було б цілком помилково. Ренти стосуються як $0 : 1 : 2 : 3 : 4$ навіть
тоді, коли ступені родючості стосуються як $n : n \dplus{} 1 : n \dplus{} 2 : n \dplus{} 3 : n \dplus{} 4$;
ренти стосуються одна до однієї, не як ступені родючости, а як ріжниці родючости,
виходячи з землі, що не дає ренти, як нулевої точки.

Таблиці оригіналу потрібно було навести для пояснення тексту. Але щоб
здобути наочну основу для наведених нижче наслідків дослідження, я далі
даю новий ряд таблиць, що в них здобуток показано в бушелях (\sfrac{1}{8}  квартера,
або 36, 35 літра) і шилінґах (= марці).

Перша таблиця (XI) відповідає давнішій таблиці І.~Вона дає здобутки
і ренти для земель п’ятьох якостей $A$ — $E$, при \emph{першій} витраті капіталу в 50\shil{ шил.}, що разом з 10\shil{ шил.} зиску \deq{} 60\shil{ шил.} усієї ціни продукції на акр. Здобутки
збіжжя взято низькі: 10, 12, 14, 16, 18 бушелів з акра. Регуляційна
ціна продукції, яка тут складається, є 6\shil{ шил.} за бушель.

Дальші 13 таблиць відповідають трьом випадкам диференційної ренти II,
розгляненим в цьому і в обох попередніх розділах, при чому припускається, що
\emph{додаткова} витрата капіталу на тій самій землі рівна 50\shil{ шил.} на акр за сталої,
низхідної і висхідної ціни продукції. Кожен з цих випадків знову таки
подається так, як він складається 1)~за сталої, 2)~за низхідної, 3)~за висхідної
продуктивности другої витрати капіталу проти першої. При цьому постають ще
деякі особливо наочні варіянти.

В випадку І: стала ціна продукції, ми маємо:

Варіянт 1: незмінна продуктивність другої витрати капіталу (таблиця XII).

Варіянт 2: низхідна продуктивність. Це може статися лише тоді, коли на землі
А не робиться жодної другої витрати. А саме або:

а) так, що земля $В$ теж не дає ренти (таблиця XIII), або

б) так, що земля $В$ не стає землею, що зовсім не дає ренти (таблиця XIV).

Варіянт 3: висхідна продуктивність (таблиця ХV). І цей випадок виключає
другу витрату капіталу на землю $А$.

В випадку II: низхідна ціна продукції, ми маємо:

Варіянт 1: незмінна продуктивність другої витрати (таблиця ХVI).

Варіянт 2: низхідна продуктивність (таблиця XVII). Обидва варіянти призводять
до того, що земля $А$ вилучається з числа конкурентних земель, земля
$В$ перестає давати ренту і регулює ціну продукції.

Варіянт 3: висхідна продуктивність (таблиця XVIII). Тут земля $А$ лишається
регуляційною.

У випадку III: висхідна ціна продукції, можливі дві видозміни: земля
$А$ може лишитися землею, що не дає ренти і яка реґулює ціни, або ж в конкуренцію
вступає земля гіршої якости, ніж $А$, і починає регулювати ціну, так що $А$
тоді дає ренту.

Перша видозміна: земля $А$ залишається реґуляційною.

Варіянт 1: незмінна продуктивність другої витрати (таблиця XIX). Це припустиме
лише за тієї передумови, що продуктивність першої витрати
зменшується.

Варіянт 2: низхідна продуктивність другої витрати (таблиця XX); це не виключає
того, що продуктивність першої витрати не зміниться.


\index{iii2}{0169}  %% посилання на сторінку оригінального видання
Варіянт III: Висхідна продуктивність другої витрати (таблиця XXI); це знов
таки зумовлює низхідну продуктивність першої витрати.

Друга видозміна: земля гіршої якости, (позначувана: літерою а)
вступає в конкуренцію; земля $А$ дає ренту.

Варіянт 1: Незмінна продуктивність другої витрати (таблиця XXII).

Варіант 2: Низхідна продуктивність (таблиця XXIII).

Варіант 3: Висхідна продуктивність (таблиця XXIV).

Ці три варіянти відповідають загальним умовам проблеми і не дають
приводу до будь-яких зауважень.

Тепер ми наведемо таблиці:

\begin{table}[H]
  \begin{center}
    \emph{Таблиця XI}
    \footnotesize

  \begin{tabular}{c@{  } c@{  } c@{  } c@{  } c@{  } c@{  } c}
    \toprule
      \multirowcell{2}{\makecell{Рід\\ землі}} &
      Ціна продукції &
      Продукт &
      \makecell{Продажна \\ ціна} &
      \makecell{Здо-\\буток} &
      Рента &
      \multirowcell{2}{Підвищення ренти} \\

      \cmidrule(r){2-2}
      \cmidrule(r){3-3}
      \cmidrule(r){4-4}
      \cmidrule(r){5-5}
      \cmidrule(r){6-6}

       & Шил. & Бушелі & Шил. & Шил. & Шил. & \\
      \midrule
      A & 60 & 10 & 6 & 60  & \phantom{00}0 & \phantom{00 × 0}0 \\
      B & 60 & 12 & 6 & 72  & \phantom{0}12 & \phantom{01 × }12 \\
      C & 60 & 14 & 6 & 84  & \phantom{0}24 & \phantom{0}2 × 12           \\
      D & 60 & 16 & 6 & 96  & \phantom{0}36 & \phantom{0}3 × 12           \\
      E & 60 & 18 & 6 & 108 & \phantom{0}48 & \phantom{0}4 × 12           \\

     \cmidrule(r){6-6}
     \cmidrule(r){7-7}

      & & & & & 120 & 10 × 12 \\
  \end{tabular}

  \end{center}
\end{table}

За другої витрати капіталу на тій самій землі.

Перший випадок: за незмінної ціни продукції.

Варіянт 1: за незмінної продуктивности другої витрати капіталу.

\begin{table}[H]
  \begin{center}
    \emph{Таблиця XII}
    \footnotesize

  \begin{tabular}{c@{  } c@{  } c@{  } c@{  } c@{  } c@{  } c}
    \toprule
      \multirowcell{2}{\makecell{Рід\\ землі}} &
      Ціна продукції &
      Продукт &
      \makecell{Продажна \\ ціна} &
      \makecell{Здо-\\буток} &
      Рента &
      \multirowcell{2}{Підвищення ренти} \\

      \cmidrule(r){2-2}
      \cmidrule(r){3-3}
      \cmidrule(r){4-4}
      \cmidrule(r){5-5}
      \cmidrule(r){6-6}

       & Шил. & Бушелі & Шил. & Шил. & Шил. &  \\
      \midrule
      A & 60 \dplus{} 60 \deq{} 120 & 10 \dplus{} 10 \deq{} 20 & 6 & 120  & \phantom{00}0 & \phantom{00 × 0}0 \\
      B & 60 \dplus{} 60 \deq{} 120 & 12 \dplus{} 12 \deq{} 24 & 6 & 144  & \phantom{0}24 & \phantom{01 × }24 \\
      C & 60 \dplus{} 60 \deq{} 120 & 14 \dplus{} 14 \deq{} 28 & 6 & 168  & \phantom{0}48 & \phantom{0}2 × 24 \\
      D & 60 \dplus{} 60 \deq{} 120 & 16 \dplus{} 16 \deq{} 32 & 6 & 192  & \phantom{0}72 & \phantom{0}3 × 24 \\
      E & 60 \dplus{} 60 \deq{} 120 & 18 \dplus{} 18 \deq{} 36 & 6 & 216  & \phantom{0}96 & \phantom{0}4 × 24 \\

     \cmidrule(r){6-6}
     \cmidrule(r){7-7}

      & & & & & 240 & 10 × 24 \\
  \end{tabular}

  \end{center}
\end{table}

Варіянт 2: за низхідної продуктивности другої витрати капіталу: на землі
$А$ не зроблено другої витрати.

1) Коли земля $В$ стає землею, що не дає ренти.


\begin{table}[H]
  \begin{center}
    \emph{Таблиця XIII}
    \footnotesize

  \begin{tabular}{c@{  } c@{  } c@{  } c@{  } c@{  } c@{  } c}
    \toprule
      \multirowcell{2}{\makecell{Рід\\ землі}} &
      Ціна продукції &
      Продукт &
      \makecell{Продажна \\ ціна} &
      \makecell{Здо-\\буток} &
      Рента &
      \multirowcell{2}{Підвищення ренти} \\

      \cmidrule(r){2-2}
      \cmidrule(r){3-3}
      \cmidrule(r){4-4}
      \cmidrule(r){5-5}
      \cmidrule(r){6-6}

       & Шил. & Бушелі & Шил. & Шил. & Шил. & \\
      \midrule
      A & \phantom{60 \dplus{} 60 \deq{} 0}60 & \phantom{12 \dplus{} 10\tbfrac{1}{3} \deq{}} 10\phantom{\tbfrac{2}{3}}           & 6 & \phantom{0}60 & \phantom{00}0 & \phantom{0 × 0}0 \\
      B & 60 \dplus{} 60 \deq{} 120           & 12 \dplus{} \phantom{0}8\phantom{\tbfrac{1}{3}} \deq{} 20\phantom{\tbfrac{2}{3}} & 6 & 120           & \phantom{00}0 & \phantom{0 × 0}0 \\
      C & 60 \dplus{} 60 \deq{} 120           & 14 \dplus{} \phantom{0}9\tbfrac{1}{3} \deq{} 23\tbfrac{1}{3}                     & 6 & 140           & \phantom{0}20 & \phantom{1 × }20 \\
      D & 60 \dplus{} 60 \deq{} 120           & 16 \dplus{} 10\tbfrac{2}{3} \deq{} 26\tbfrac{2}{3}                               & 6 & 160           & \phantom{0}40 & 2 × 20 \\
      E & 60 \dplus{} 60 \deq{} 120           & 18 \dplus{} 12\phantom{/}= 30\phantom{\tbfrac{2}{3}}                  & 6 & 180           & \phantom{0}60 & 3 × 20 \\

     \cmidrule(r){6-6}
     \cmidrule(r){7-7}

      & & & & & 120 & 6 × 20 \\
  \end{tabular}

  \end{center}
\end{table}
% REMOVED
% \footnotemarkZ{}
% \footnotetextZ{В німецькому тексті тут стоїть «20». Очевидна помилка. \Red{Прим. Ред.}}

2)~Кола земля $В$ не стає землею, що зовсім не дає ренти.

\begin{table}[H]
  \begin{center}
    \emph{Таблиця XIV}
    \footnotesize

  \begin{tabular}{c@{  } c@{  } c@{  } c@{  } c@{  } c@{  } c}
    \toprule
      \multirowcell{2}{\makecell{Рід\\ землі}} &
      Ціна продукції &
      Продукт &
      \makecell{Продажна \\ ціна} &
      \makecell{Здо-\\буток} &
      Рента &
      \multirowcell{2}{Підвищення ренти} \\

      \cmidrule(r){2-2}
      \cmidrule(r){3-3}
      \cmidrule(r){4-4}
      \cmidrule(r){5-5}
      \cmidrule(r){6-6}

       & Шил. & Бушелі & Шил. & Шил. & Шил. & \\
      \midrule
      A & \phantom{60 \dplus{} 60 \deq{} 0}60 & \phantom{12 \dplus{} 10\tbfrac{1}{3} \deq{}} 10\phantom{\tbfrac{2}{3}}           & 6 & \phantom{0}60 & \phantom{00}0 & \phantom{4 ×}0\phantom{ \dplus{} 3 × 21}\\
      B & 60 \dplus{} 60 \deq{} 120           & 12 \dplus{} \phantom{0}9\phantom{\tbfrac{1}{3}} \deq{} 21\phantom{\tbfrac{2}{3}} & 6 & 126           & \phantom{00}6 & \phantom{4 ×}6\phantom{ \dplus{} 3 × 21}\\
      C & 60 \dplus{} 60 \deq{} 120           & 14 \dplus{} 10\tbfrac{1}{2} \deq{} 24\tbfrac{1}{2}                               & 6 & 147           & \phantom{0}27 & \phantom{4 ×}6 \dplus{} 21\phantom{1 × } \\
      D & 60 \dplus{} 60 \deq{} 120           & 16 \dplus{} 12\phantom{\tbfrac{2}{3}} \deq{} 28\phantom{\tbfrac{2}{3}}           & 6 & 168           & \phantom{0}48 & \phantom{4 ×}6 \dplus{} 2 × 21 \\
      E & 60 \dplus{} 60 \deq{} 120           & 18 \dplus{} 13\tbfrac{1}{2}= 31\tbfrac{1}{2}                                & 6 & 189           & \phantom{0}69 & \phantom{4 ×}6 \dplus{} 3 × 21 \\

     \cmidrule(r){6-6}
     \cmidrule(r){7-7}

      & & & & & 150 & 4 × 6 \dplus{} 6 × 21 \\
  \end{tabular}

  \end{center}
\end{table}

Варіянт 3: за висхідної продуктивности другої витрати капіталу; на землі
$А$ тут теж не робиться другої витрати.

\begin{table}[H]
  \begin{center}
    \emph{Таблиця XV}
    \footnotesize

  \begin{tabular}{c@{  } c@{  } c@{  } c@{  } c@{  } c@{  } c}
    \toprule
      \multirowcell{2}{\makecell{Рід\\ землі}} &
      Ціна продукції &
      Продукт &
      \makecell{Продажна \\ ціна} &
      \makecell{Здо-\\буток} &
      Рента &
      \multirowcell{2}{Підвищення ренти} \\

      \cmidrule(r){2-2}
      \cmidrule(r){3-3}
      \cmidrule(r){4-4}
      \cmidrule(r){5-5}
      \cmidrule(r){6-6}

       & Шил. & Бушелі & Шил. & Шил. & Шил. & \\
      \midrule
      A & \phantom{60 \dplus{} 60 \deq{} 0}60 & \phantom{12 \dplus{} 10\tbfrac{1}{3} \deq{}} 10\phantom{\tbfrac{2}{3}}           & 6 & \phantom{0}60 & \phantom{00}0 & \phantom{4 ×0}0\phantom{ \dplus{} 3 × 27}\\
      B & 60 \dplus{} 60 \deq{} 120           & 12 \dplus{} 15\phantom{\tbfrac{1}{3}} \deq{} 27\phantom{\tbfrac{2}{3}}           & 6 & 162           & \phantom{0}42 & \phantom{4 ×}42\phantom{ \dplus{} 3 × 27}\\
      C & 60 \dplus{} 60 \deq{} 120           & 14 \dplus{} 17\tbfrac{1}{2} \deq{} 31\tbfrac{1}{2}                               & 6 & 189           & \phantom{0}69 & \phantom{4 ×}42 \dplus{} 27\phantom{1 × } \\
      D & 60 \dplus{} 60 \deq{} 120           & 16 \dplus{} 20\phantom{\tbfrac{2}{3}} \deq{} 36\phantom{\tbfrac{2}{3}}           & 6 & 216           & \phantom{0}96 & \phantom{4 ×}42 \dplus{} 2 × 27 \\
      E & 60 \dplus{} 60 \deq{} 120           & 18 \dplus{} 22\tbfrac{1}{2}= 40\tbfrac{1}{2}                                & 6 & 243           & 123           & \phantom{4 ×}42 \dplus{} 3 × 27 \\

     \cmidrule(r){6-6}
     \cmidrule(r){7-7}

      & & & & & 330 & 4 × 42 \dplus{} 6 × 27 \\
  \end{tabular}

  \end{center}
\end{table}


\index{iii2}{0171}  %% посилання на сторінку оригінального видання
Другий випадок за низхідної ціни продукції.

Варіянт 1: за незмінної продуктивности другої витрати капіталу: земля
$А$ випадає з конкуренції, земля $В$ стає землею, що не дає ренти.

\begin{table}[H]
  \begin{center}
    \emph{Таблиця XVI}
    \footnotesize

  \begin{tabular}{c@{  } c@{  } c@{  } c@{  } c@{  } c@{  } c}
    \toprule
      \multirowcell{2}{\makecell{Рід\\ землі}} &
      Ціна продукції &
      Продукт &
      \makecell{Продажна \\ ціна} &
      \makecell{Здо-\\буток} &
      Рента &
      \multirowcell{2}{Підвищення ренти} \\

      \cmidrule(r){2-2}
      \cmidrule(r){3-3}
      \cmidrule(r){4-4}
      \cmidrule(r){5-5}
      \cmidrule(r){6-6}

       & Шил. & Бушелі & Шил. & Шил. & Шил. &  \\
      \midrule
      B & 60 \dplus{} 60 \deq{} 120 & 12 \dplus{} 12 \deq{} 24 & 5 & 120  & \phantom{00}0 & \phantom{01 × }0 \\
      C & 60 \dplus{} 60 \deq{} 120 & 14 \dplus{} 14 \deq{} 28 & 5 & 140  & \phantom{0}20 & \phantom{1 ×} 20 \\
      D & 60 \dplus{} 60 \deq{} 120 & 16 \dplus{} 16 \deq{} 32 & 5 & 160  & \phantom{0}40 & 2 × 20 \\
      E & 60 \dplus{} 60 \deq{} 120 & 18 \dplus{} 18 \deq{} 36 & 5 & 180  & \phantom{0}60 & 3 × 20 \\

     \cmidrule(r){6-6}
     \cmidrule(r){7-7}

      & & & & & 120 & 6 × 20 \\
  \end{tabular}

  \end{center}
\end{table}

Варіант 2: за низхідної продуктивности другої витрати капіталу; земля
$А$ випадає з конкуренції, земля $В$ стає землею, що не дає ренти.

\begin{table}[H]
  \begin{center}
    \emph{Таблиця XVII}
    \footnotesize

  \begin{tabular}{c@{  } c@{  } c@{  } c@{  } c@{  } c@{  } c}
    \toprule
      \multirowcell{2}{\makecell{Рід\\ землі}} &
      Ціна продукції &
      Продукт &
      \makecell{Продажна \\ ціна} &
      \makecell{Здо-\\буток} &
      Рента &
      \multirowcell{2}{Підвищення ренти} \\

      \cmidrule(r){2-2}
      \cmidrule(r){3-3}
      \cmidrule(r){4-4}
      \cmidrule(r){5-5}
      \cmidrule(r){6-6}

       & Шил. & Бушелі & Шил. & Шил. & Шил. &  \\
      \midrule
      B & 60 \dplus{} 60 \deq{} 120 & 12 \dplus{} \phantom{0}9\phantom{\sfrac{1}{2}} \deq{} 21\phantom{\sfrac{1}{2}} & 5\sfrac{5}{7} & 120  & \phantom{00}0 & \phantom{01 × }0 \\
      C & 60 \dplus{} 60 \deq{} 120 & 14 \dplus{} 10\sfrac{1}{2} \deq{} 24\sfrac{1}{2}                               & 5\sfrac{5}{7} & 140  & \phantom{0}20 & \phantom{1 ×} 20 \\
      D & 60 \dplus{} 60 \deq{} 120 & 16 \dplus{} 12\phantom{\sfrac{1}{2}} \deq{} 28\phantom{\sfrac{1}{2}}           & 5\sfrac{5}{7} & 160  & \phantom{0}40 & 2 × 20 \\
      E & 60 \dplus{} 60 \deq{} 120 & 18 \dplus{} 13\sfrac{1}{2} \deq{} 31\sfrac{1}{2}                               & 5\sfrac{5}{7} & 180  & \phantom{0}60 & 3 × 20 \\

     \cmidrule(r){6-6}
     \cmidrule(r){7-7}

      & & & & & 120 & 6 × 20 \\
  \end{tabular}

  \end{center}
\end{table}

Варіант 3: за висхідної продуктивности другої витрати капіталу; земля
$А$ залишається конкурентною. Земля $В$ дає ренту.

\begin{table}[H]
  \begin{center}
    \emph{Таблиця XVIII}
    \footnotesize

  \begin{tabular}{c@{  } c@{  } c@{  } c@{  } c@{  } c@{  } c}
    \toprule
      \multirowcell{2}{\makecell{Рід\\ землі}} &
      Ціна продукції &
      Продукт &
      \makecell{Продажна \\ ціна} &
      \makecell{Здо-\\буток} &
      Рента &
      \multirowcell{2}{Підвищення ренти} \\

      \cmidrule(r){2-2}
      \cmidrule(r){3-3}
      \cmidrule(r){4-4}
      \cmidrule(r){5-5}
      \cmidrule(r){6-6}

       & Шил. & Бушелі & Шил. & Шил. & Шил. &   \\
      \midrule
      A & 60 \dplus{} 60 \deq{} 120 & 10 \dplus{} 15 \deq{} 25 & 4\sfrac{4}{5} & 120  & \phantom{00}0 & \phantom{00 × 0}0 \\
      B & 60 \dplus{} 60 \deq{} 120 & 12 \dplus{} 18 \deq{} 30 & 4\sfrac{4}{5} & 144  & \phantom{0}24 & \phantom{01 × }24 \\
      C & 60 \dplus{} 60 \deq{} 120 & 14 \dplus{} 21 \deq{} 35 & 4\sfrac{4}{5} & 168  & \phantom{0}48 & \phantom{0}2 × 24 \\
      D & 60 \dplus{} 60 \deq{} 120 & 16 \dplus{} 24 \deq{} 40 & 4\sfrac{4}{5} & 192  & \phantom{0}72 & \phantom{0}3 × 24 \\
      E & 60 \dplus{} 60 \deq{} 120 & 18 \dplus{} 27 \deq{} 45 & 4\sfrac{4}{5} & 216  & \phantom{0}96 & \phantom{0}4 × 24 \\

     \cmidrule(r){6-6}
     \cmidrule(r){7-7}

      & & & & & 240 & 10 × 24 \\
  \end{tabular}

  \end{center}
\end{table}


Третій випадок: за висхідної ціни продукції.

А.~Коли земля $А$ не дає ренти й продовжує реґулювати ціну.

Варіянт 1: за незмінної продуктивности другої витрати капіталу, що зумовлює
низхідну продуктивність першої витрати.

\begin{table}[H]
  \centering
  \footnotesize
  \caption*{Таблиця XIX}

  \begin{tabular}{lcccccc}
    \toprule
      \thead[tl]{Рід\\землі} &
      Ціна продукції &
      Продукт &
      \thead[t]{Продажна\\ціна} &
      \thead[t]{Здо-\\буток} &
      Рента &
      \thead[t]{Підвищення\\ренти} \\

    \cmidrule(r){2-6}
      & \shil{Шил.} & бушелі & \shil{Шил.} & \shil{Шил.} & \shil{Шил.} & \\

    \midrule
      A & 60 \dplus{} 60 \deq{} 120 & \pZ{}7\tbfrac{1}{2} \dplus{} 10 \deq{} 17\tbfrac{1}{2}    & 6\tbfrac{6}{7} & 120  & \phantom{00}0 & \phantom{01 × }0 \\
      B & 60 \dplus{} 60 \deq{} 120 & \pZ{}9\pF{}         \dplus{} 12 \deq{} 21\pF{}            & 6\tbfrac{6}{7} & 144  & \phantom{0}24 & \phantom{1 ×} 24 \\
      C & 60 \dplus{} 60 \deq{} 120 &     10\tbfrac{1}{2} \dplus{} 14 \deq{} 24\tbfrac{1}{2}    & 6\tbfrac{6}{7} & 168  & \phantom{0}48 & 2 × 24 \\
      D & 60 \dplus{} 60 \deq{} 120 &     12\pF{}         \dplus{} 16 \deq{} 28\pF{}            & 6\tbfrac{6}{7} & 192  & \phantom{0}72 & 3 × 24 \\
      E & 60 \dplus{} 60 \deq{} 120 &     13\tbfrac{1}{2} \dplus{} 18 \deq{} 31\tbfrac{1}{2}    & 6\tbfrac{6}{7} & 216  & \phantom{0}96 & 4 × 24 \\

    \cmidrule(r){6-7}
      & & & & & 240 & \hang{r}{1}0 × 24 \\
  \end{tabular}
\end{table}
% REMOVED. + Fixed numbers in column 3 (Продукт)
% \footnotetextZ{Це є таблиця висхідної продуктивности другої витрати капіталу. Порівн. табл. XXI. \Red{Прим. Ред}}

Варіянт 2: за низхідної продуктивности другої витрати капіталу, що не виключає
незмінюваної продуктивности першої витрати.

\begin{table}[H]
  \centering
  \footnotesize
  \caption*{Таблиця XX}

  \begin{tabular}{lcccccc}
    \toprule
      \thead[tl]{Рід\\землі} &
      Ціна продукції &
      Продукт &
      \thead[t]{Продажна\\ціна} &
      \thead[t]{Здо-\\буток} &
      Рента &
      \thead[t]{Підвищення\\ренти} \\

    \cmidrule(r){2-6}
      & \shil{Шил.} & бушелі & \shil{Шил.} & \shil{Шил.} & \shil{Шил.} & \\

    \midrule
      A & 60 \dplus{} 60 \deq{} 120 & 10 \dplus{} 5 \deq{} 15  & 8 & 120  & \phantom{00}0 & \phantom{01 × }0 \\
      B & 60 \dplus{} 60 \deq{} 120 & 12 \dplus{} 6 \deq{} 18  & 8 & 144  & \phantom{0}24 & \phantom{1 ×} 24 \\
      C & 60 \dplus{} 60 \deq{} 120 & 14 \dplus{} 7 \deq{} 21  & 8 & 168  & \phantom{0}48 & 2 × 24 \\
      D & 60 \dplus{} 60 \deq{} 120 & 16 \dplus{} 8 \deq{} 24  & 8 & 192  & \phantom{0}72 & 3 × 24 \\
      E & 60 \dplus{} 60 \deq{} 120 & 18 \dplus{} 9 \deq{} 27  & 8 & 216  & \phantom{0}96 & 4 × 24 \\

    \cmidrule(r){6-7}
      & & & & & 240 & \hang{r}{1}0 × 24 \\
  \end{tabular}
\end{table}

Варіянт 3: за висхідної продуктивности другої витрати капіталу, що, за даних
припущень, обумовлює низхідну продуктивність першої витрати.

\begin{table}[H]
  \centering
  \footnotesize
  \caption*{Таблиця XXI}

  \begin{tabular}{lcccccc}
    \toprule
      \thead[tl]{Рід\\землі} &
      Ціна продукції &
      Продукт &
      \thead[t]{Продажна\\ціна} &
      \thead[t]{Здо-\\буток} &
      Рента &
      \thead[t]{Підвищення\\ренти} \\

    \cmidrule(r){2-6}
      & \shil{Шил.} & бушелі & \shil{Шил.} & \shil{Шил.} & \shil{Шил.} & \\

    \midrule
      A & 60 \dplus{} 60 \deq{} 120 & 5 \dplus{} 12\tbfrac{1}{2} \deq{} 17\tbfrac{1}{2}                      & 6\tbfrac{6}{7} & 120  & \phantom{00}0 & \phantom{01 × }0 \\
      B & 60 \dplus{} 60 \deq{} 120 & 6 \dplus{} 15\phantom{\tbfrac{1}{2}} \deq{} 21\phantom{\tbfrac{1}{2}}  & 6\tbfrac{6}{7} & 144  & \phantom{0}24 & \phantom{1 ×} 24 \\
      C & 60 \dplus{} 60 \deq{} 120 & 7 \dplus{} 17\tbfrac{1}{2} \deq{} 24\tbfrac{1}{2}                      & 6\tbfrac{6}{7} & 168  & \phantom{0}48 & 2 × 24 \\
      D & 60 \dplus{} 60 \deq{} 120 & 8 \dplus{} 20\phantom{\tbfrac{1}{2}} \deq{} 28\phantom{\tbfrac{1}{2}}  & 6\tbfrac{6}{7} & 192  & \phantom{0}72 & 3 × 24 \\
      E & 60 \dplus{} 60 \deq{} 120 & 9 \dplus{} 22\tbfrac{1}{2} \deq{} 31\tbfrac{1}{2}                      & 6\tbfrac{6}{7} & 216  & \phantom{0}96 & 4 × 24 \\

    \cmidrule(r){6-7}
      & & & & & 240 & \hang{r}{1}0 × 24 \\
  \end{tabular}
\end{table}


\index{iii2}{0173}  %% посилання на сторінку оригінального видання
В.~Коли гірша (позначувана літерою а) земля стає землею, яка реґулює
ціну і через те $А$ починає давати ренту. Де не виключає можливости незмінюваної
продуктивности другої витрати для всіх варіянтів.

Варіянт 1: Незмінювана продуктивність другої витрати капіталу.

\begin{table}[H]
  \begin{center}
    \emph{Таблиця XXII}
    \footnotesize

  \begin{tabular}{c@{  } c@{  } c@{  } c@{  } c@{  } c@{  } c}
    \toprule
      \multirowcell{2}{\makecell{Рід\\ землі}} &
      Ціна продукції &
      Продукт &
      \makecell{Продажна \\ ціна} &
      \makecell{Здо-\\буток} &
      Рента &
      \multirowcell{2}{Підвищення ренти} \\

      \cmidrule(r){2-2}
      \cmidrule(r){3-3}
      \cmidrule(r){4-4}
      \cmidrule(r){5-5}
      \cmidrule(r){6-6}

       & Шил. & Бушелі & Шил. & Шил. & Шил. &  \\
      \midrule
      a & \phantom{60 \dplus{} 60 \deq{} }120 & \phantom{10 \dplus{} 10 \deq{} }16 & 7\sfrac{1}{2} & 120  & \phantom{00}0  & \phantom{01 × }0 \\
      A & 60 \dplus{} 60 \deq{} 120           & 10 \dplus{} 10 \deq{} 20            & 7\sfrac{1}{2} & 150  & \phantom{0}30 & \phantom{1 ×} 30 \\
      B & 60 \dplus{} 60 \deq{} 120           & 12 \dplus{} 12 \deq{} 24            & 7\sfrac{1}{2} & 180  & \phantom{0}60 & 2 × 30 \\
      C & 60 \dplus{} 60 \deq{} 120           & 14 \dplus{} 14 \deq{} 28            & 7\sfrac{1}{2} & 210  & \phantom{0}90 & 3 × 30 \\
      D & 60 \dplus{} 60 \deq{} 120           & 16 \dplus{} 16 \deq{} 32            & 7\sfrac{1}{2} & 240  & 120           & 4 × 30 \\
      E & 60 \dplus{} 60 \deq{} 120           & 18 \dplus{} 18 \deq{} 36            & 7\sfrac{1}{2} & 270  & 150           & 5 × 30 \\

     \cmidrule(r){6-6}
     \cmidrule(r){7-7}

      & & & & & 450 & 15 × 30 \\
  \end{tabular}

  \end{center}
\end{table}

Варіянт 2: Низхідна продуктивність другої витрати капіталу.

\begin{table}[H]
  \begin{center}
    \emph{Таблиця XXIII}
    \footnotesize

  \begin{tabular}{c@{  } c@{  } c@{  } c@{  } c@{  } c@{  } c}
    \toprule
      \multirowcell{2}{\makecell{Рід\\ землі}} &
      Ціна продукції &
      Продукт &
      \makecell{Продажна \\ ціна} &
      \makecell{Здо-\\буток} &
      Рента &
      \multirowcell{2}{Підвищення ренти} \\

      \cmidrule(r){2-2}
      \cmidrule(r){3-3}
      \cmidrule(r){4-4}
      \cmidrule(r){5-5}
      \cmidrule(r){6-6}

       & Шил. & Бушелі & Шил. & Шил. & Шил. &  \\
      \midrule
      a & \phantom{60 \dplus{} 60 \deq{} }120 & \phantom{10 \dplus{} 10\sfrac{1}{2} \deq{} }15\phantom{\sfrac{1}{2}}  & 8 & 120 & \phantom{00}0 & \phantom{5 × 0}0 \phantom{+ 01 × 28} \\
      A & 60 \dplus{} 60 \deq{} 120           & 10 \dplus{} \phantom{0}7\sfrac{1}{2} \deq{} 17\sfrac{1}{2}                       & 8 & 140 & \phantom{0}20 & \phantom{5 × }20 \phantom{+ 01 × 28} \\
      B & 60 \dplus{} 60 \deq{} 120           & 12 \dplus{} \phantom{0}9\phantom{\sfrac{1}{2}} \deq{} 21\phantom{\sfrac{1}{2}}   & 8 & 168 & \phantom{0}48 & \phantom{5 × }20 \dplus{} \phantom{01 × }28\\
      C & 60 \dplus{} 60 \deq{} 120           & 14 \dplus{} 10\sfrac{1}{2} \deq{} 24\sfrac{1}{2}                      & 8 & 194 & \phantom{0}76 & \phantom{5 × }20 \dplus{} \phantom{0}2 × 28 \\
      D & 60 \dplus{} 60 \deq{} 120           & 16 \dplus{} 12\phantom{\sfrac{1}{2}} \deq{} 28\phantom{\sfrac{1}{2}}  & 8 & 224 & 104           & \phantom{5 × }20 \dplus{} \phantom{0}3 × 28 \\
      E & 60 \dplus{} 60 \deq{} 120           & 18 \dplus{} 13\sfrac{1}{2} \deq{} 31\sfrac{1}{2}                      & 8 & 252 & 132           & \phantom{5 × }20 \dplus{} \phantom{0}4 × 28 \\

     \cmidrule(r){6-6}
     \cmidrule(r){7-7}

      & & & & & 380 & 5 × 20 \dplus{} 10 × 28 \\
  \end{tabular}

  \end{center}
\end{table}

Варіянт 3: Висхідна продуктивність другої витрати капіталу.

\begin{table}[H]
  \begin{center}
    \emph{Таблиця XXIV}
    \footnotesize

  \begin{tabular}{c@{  } c@{  } c@{  } c@{  } c@{  } c@{  } c}
    \toprule
      \multirowcell{2}{\makecell{Рід\\ землі}} &
      Ціна продукції &
      Продукт &
      \makecell{Продажна \\ ціна} &
      \makecell{Здо-\\буток} &
      Рента &
      \multirowcell{2}{Підвищення ренти} \\

      \cmidrule(r){2-2}
      \cmidrule(r){3-3}
      \cmidrule(r){4-4}
      \cmidrule(r){5-5}
      \cmidrule(r){6-6}

       & Шил. & Бушелі & Шил. & Шил. & Шил. &  \\
      \midrule
      a & \phantom{60 \dplus{} 60 \deq{} }120 & \phantom{10 \dplus{} 10\sfrac{1}{2} \deq{} }16\phantom{\sfrac{1}{2}}  & 7\sfrac{1}{2} & 120\phantom{\sfrac{1}{2}} & \phantom{00}0\phantom{\sfrac{1}{2}} & \phantom{5 × 15 \dplus{} 15 × }0\phantom{\sfrac{3}{4}} \\
      A & 60 \dplus{} 60 \deq{} 120           & 10 \dplus{} 12\sfrac{1}{2} \deq{} 22\sfrac{1}{2}                      & 7\sfrac{1}{2} & 168\sfrac{3}{4}           & \phantom{0}48\sfrac{3}{4}           & \phantom{5 × }15 \dplus{} \phantom{1 × }33\sfrac{3}{4} \\
      B & 60 \dplus{} 60 \deq{} 120           & 12 \dplus{} 15\phantom{\sfrac{1}{2}} \deq{} 27\phantom{\sfrac{1}{2}}  & 7\sfrac{1}{2} & 202\sfrac{1}{2}           & \phantom{0}82\sfrac{1}{2}           & \phantom{5 × }15 \dplus{} 2 × 33\sfrac{3}{4} \\
      C & 60 \dplus{} 60 \deq{} 120           & 14 \dplus{} 17\sfrac{1}{2} \deq{} 31\sfrac{1}{2}                      & 7\sfrac{1}{2} & 236\sfrac{1}{4}           & 116\sfrac{1}{4}                     & \phantom{5 × }15 \dplus{} 3 × 33\sfrac{3}{4} \\
      D & 60 \dplus{} 60 \deq{} 120           & 16 \dplus{} 20\phantom{\sfrac{1}{2}} \deq{} 36\phantom{\sfrac{1}{2}}  & 7\sfrac{1}{2} & 270\phantom{\sfrac{1}{2}} & 150\phantom{\sfrac{1}{2}}           & \phantom{5 × }15 \dplus{} 4 × 33\sfrac{3}{4} \\
      E & 60 \dplus{} 60 \deq{} 120           & 18 \dplus{} 22\sfrac{1}{2} \deq{} 40\sfrac{1}{2}                      & 7\sfrac{1}{2} & 303\sfrac{3}{4}           & 183\sfrac{3}{4}                     & \phantom{5 × }15 \dplus{} 5 × 33\sfrac{3}{4} \\

     \cmidrule(r){6-6}
     \cmidrule(r){7-7}

      & & & & & 581\sfrac{3}{4} & 5 × 15 \dplus{} 15 × 33\sfrac{3}{4} \\
  \end{tabular}

  \end{center}
\end{table}


\index{iii2}{0174}  %% посилання на сторінку оригінального видання

Отож з цих таблиць випливає таке.

Насамперед, що в ряду рент відношення точно такі самі, як у ряду ріжниць
родючости, виходячи з реґуляційної землі, що не дає ренти, як нулевого
пункту. Рента визначається не абсолютним здобутком, а лише ріжницями здобутку.
Чи дають різного роду землі 1, 2, 3, 4, 5 бушелів, чи 11, 12, 13, 14,
15 бушелів здобутку з акра, ренти в обох випадках становлять ряд: 0, 1, 2,
З, 4 бушелі, або відповідний цьому грошовий здобуток.

Але куди важливіший наслідок є у відношенні до загальної суми, ренти
при повторній витраті капіталу на тій самій землі.

В п’ятьох випадках з досліджених тринадцятьох \emph{подвоюється} разом з витратою
капіталу і загальна сума ренти; замість $10 × 12$ шил. вона стає
$10 × 24$ шил. = 240\shil{ шил.} Випадки ці такі:

Випадок І, стала ціна, варіянт І: незмінне підвищення продукції (таблиця
XII).

Випадок II, низхідна ціна, варіянт III: ростуче підвищення продукції
(таблиця XVIII).

Випадок III, висхідна ціна, перша видозміна, коли земля $А$ залишається
реґуляційною у всіх трьох варіянтах (таблиці XIX, XX, XXI).

У чотирьох випадках рента підвищується \emph{більш, ніж удвоє}, а саме:

Випадок І, варіянт III: стала ціна, але ростуче підвищення продукції
(таблиця XV). Сума ренти підвищується до 330\shil{ шил.}

Випадок III, друга видозміна, коли земля $А$ дає ренту в усіх трьох варіянтах
(таблиця XXII, рента $= 15 × 30 = 450$\shil{ шил.}; таблиця ХХІІІ, рента $= 5 ×
20 + 10 × 28 = 380$ шил.; таблиця XXIV, рента
$= 5 × 15 + 15 × 33\sfrac{3}{4} = 581\sfrac{1}{4}$\shil{ шил.}).

В одному випадку вона \emph{підвищується}, але не до подвійної суми проти
ренти, одержуваної при першій витраті капіталу.

Випадок І, стала ціна, варіянт II: низхідна продуктивність другої витрати
за умов, коли $В$ не стає землею, що зовсім не дає ренти (таблиця XIV, рента
$= 4 × 6 + 6 × 21 = 150$\shil{ шил.}).

Нарешті, тільки в трьох випадках загальна рента при другій витраті
капіталу, для всіх родів землі разом, лишається в тому самому становищі, як при
першій витраті (таблиця XI); це ті випадки, коли земля $А$ перестає брати участь
у конкуренції, а земля $В$ стає реґуляційною, і тому землею, що не дає ренти.
Отже, рента відпадає не тільки з $В$, але вона зменшується в кожному наступному
члені ряду рент; цим зумовлюється наслідок. Випадки ці такі:

Випадок І, варіянт II, коли умови такі, що земля $А$ випадає (таблиця XIII).
Сума рент дорівнює $6 × 20$, отже $10 × 12 = 120$, як у таблиці XI.

Випадок II, варіянт І і II.~Тут, згідно з припущеннями, неодмінно випадає
земля $А$ (таблиці XVI і XVII), і сума ренти є знову $6 × 20 = 10 × 12 =
120$\shil{ шил.}

Таким чином це значить: в переважній більшості всіх можливих випадків
в наслідок збільшеного приміщення капіталу в землю рента підвищується як
з акра землі, що дає ренту, так і в її загальній сумі. Лише в трьох випадках
з досліджених тринадцятьох загальна сума її лишається без зміни. Це ті випадки,
коли найгіршої якости земля, що до того часу не давала ренти й відігравала
ролю реґуляційної землі, перестає брати участь у конкуренції, і земля безпосередньо
краща якістю стає на її місце, отже, перестає давати ренту. Але і в
цих випадках ренти з кращих земель підвищуються проти тих рент, які завдячують
своїм походженням першій витраті капіталу; коли рента з $C$ понижується
з 24 до 20, то для $D$ і $Е$ вона підвищується з 36 і 48 до 40 і 60\shil{ шил.}

Пониження загальної суми рент нижче від того рівня, що вона мала за
першої витрати капіталу (таблиця XI) було б можливе лише тоді, коли б, крім
\parbreak{}  %% абзац продовжується на наступній сторінці

\parcont{}  %% абзац починається на попередній сторінці
\index{iii2}{0175}  %% посилання на сторінку оригінального видання
землі $А$, перестала брати участь у конкуренції і земля $В$, і земля $C$ зробилася б
регуляційною землею, що не дає ренти.

Отже, що більше капіталу вживається на землі, що вищого розвитку досягли
у країні хліборобство і цивілізація взагалі, то вище підносяться ренти
з акра, так само як і загальна сума рент, то колосальніший стає податок, що
його виплачує суспільство великим земельним власникам у вигляді надзисків, — доки
всі роди землі, що вже підлягли обробленню, зберігають здатність до конкуренції.

Цей закон пояснює дивовижну живучість кляси великих землевласників.
Жодна інша кляса суспільства не живе так марнотратно, як ця; жодна інша
не заявляє такої претенсії на звичну «відповідну станові» розкіш, хоч би звідки
одержувано для цього гроші; жодна інша кляса не нагромаджує з таким легким
серцем боргів за боргами. А проте, вона завжди хоч і вскочить, а вискочить —
завдяки капіталові, що його інші люди вклали в землю і що дає їй ренти позавсяким
співвідношенням з зисками, що їх одержує з нього капіталіст. Але той самий закон
пояснює також, чому ця живучість великого землевласника поволі вичерпується.

Коли 1846 року скасовано було в Англії збіжжеві мита, англійські фабриканти
думали, що цим вони перетворили землевласницьку аристократію на
павперів. Замість цього вона забагатіла більше, ніж будь-коли раніш. Яким
чином це сталося? Дуже просто. Поперше, від цього часу до орендарів почали
ставити закріплену контрактом вимогу, за якою вони зобов’язувалися витрачати
щорічно по 12\pound{ ф. ст.} замість 8\pound{ ф. ст.} на акр, і по-друге, землевласники, що мали і в
нижній палаті дуже численних представників, асигнували собі велику державну
допомогу для дренування та інших перманентних поліпшень своїх земель. А що
цілковитого витиснення найгіршої землі не сталося, а відбулося, щонайбільше,
застосування її для іншої мети, та й то здебільша тимчасове, то ренти підвищились
відповідно до підвищенної витрати капіталу, і земельна аристократія
виграла від цього більше, ніж будь-коли раніш.

Але все минає. Трансатлантійські пароплави, а також північно південноамериканські
та індійські залізниці дали змогу цілком особливим країнам конкурувати
на європейських збіжжевих ринках. Це були, з одного боку, північноамериканські
прерії, арґентінські пампаси, степи вже від природи придатні для
обробітку плугом, незайманий ґрунт, що багато років давав багаті врожаї навіть
за примітивної культури і без добрива. Далі це були землі російських та
індійських комуністичних громад, які мусили продавати частину свого продукту,
до того ж дедалі більшу, щоб одержати гроші для виплати податків, що їх виплачувати
примушував, досить часто з допомогою катування, нещадний деспотизм
держави. Ці продукти продавалося безвідносно до цін продукції, продавалося
за ціну, яку пропонував торговець, бо селянин на строк виплати мусив
мати гроші хоч би за яку ціну. І з цією конкуренцією, — незайманої степової
землі, а також російських та індійських селян, що знемагають під податковим пресом,
— європейський орендар і селянин не міг упоратись при старих рентах. Частина
землі в Европі остаточно стала щодо продукції збіжжя, конкурентно неспроможною,
ренти всюди занепали; другий наш випадок, варіянт II: низхідна ціна
і низхідна продуктивність додаткових витрат капіталу зробився загальним
правилом для Европи, звідси лемент аґраріїв від Шотландії до Італії, від Південної
Франції до Східньої Прусії. На щастя, ще геть не всі степові землі
оброблено; їх ще надто досить для того, щоб зруйнувати все європейське велике
землеволодіння та крім того і дрібне. — Ф.~Е.].

\pfbreak

Рубрики, під якими треба дослідити ренту, такі:

А.~Диференційна рента.

1) Поняття диференційної ренти. Ілюстрація силою води. Перехід до власне
хліборобської ренти.
\parbreak{}  %% абзац продовжується на наступній сторінці


2)~Диференційна рента І, що походить з ріжниці в родючості різних земельних
дільниць.

3)~Диференційна рента II, що походить з послідовної витрати капіталу
на тій самій землі. Диференційна рента II підлягає дослідженню:

a) при сталій,

b) при низхідній,

c) при висхідній ціні продукції.

І далі:

d) перетворення надзиску в ренту.

4)~Вплив цієї ренти на норму зиску.

B.~Абсолютна рента.

C.~Ціна землі.

D.~Кінцеві зауваження про земельну ренту.

\pfbreak

Як загальний наслідок розгляду диференційної ренти, виходить таке:

\emph{Перше:} Створення надзиску може відбуватися різними шляхами. З одного
боку, на базі диференційної ренти І, тобто на базі витрати всього
хліборобського капіталу на земельній площі, що складається з земель різної
родючости. Далі, як диференційна рента II, на базі різної диференційної продуктивности
послідовних витрат капіталу на тій самій землі, тобто на базі
більшої продуктивности, визначеної, наприклад, у квартерах пшениці, ніж та,
що постає при тій самій витраті капіталу на найгіршій землі, що не дає ренти,
але реґулює ціну продукції. Але хоч би як виникали ці надзиски, перетворення
їх у ренту, отже, їх перехід від орендаря до землевласника, завжди припускав
як попередню умову, що різні дійсні індивідуальні ціни продукції (тобто незалежно
від загальної ціни продукції, що реґулює ринок) окремих продуктів, окремих
послідовних витрат капіталу попередньо вирівнюються в індивідуальну
пересічну ціну продукції. Надмір загальної регуляційної ціни продукції продукту
з акра над цією його індивідуальною пересічною ціною продукції становить
і визначає величину ренти на акр. При диференційній ренті І диференційні
наслідки розпізнаються сами по собі, бо вони постають на різних дільницях
землі, що лежать одна поза однією і одна біля однієї, — за такої витрати капіталу
на акр, яку береться за нормальну, і при відповідному до цієї витрати
нормальному обробленні. При диференційній ренті II їх спершу треба зробити
розрізнюваними; справді, вони мусять бути перетворені зворотно у диференційну
ренту І, а це можна зробити лише зазначеним способом. Візьмімо, наприклад,
ренту І, а це можна зробити лише зазначеним способом.
таблицю III, ст. 149.
%% TODO: add link

Земля $В$ дає в наслідок першої витрати капіталу в 2\sfrac{1}{2}\pound{ ф. стерл.} 2 квартери
з акра, а в наслідок другої витрати, однакової розміром, — 1\sfrac{1}{2} квартери; разом
3\sfrac{1}{2} квартери з того самого акра. За цими 3\sfrac{1}{2} квартерами, що виросли
на тій самій землі, не можна побачити, яка частина з них є продуктом витрати
капіталу І і яка витрати капіталу II.~Вони в дійсності становлять продукт
усього капіталу в 5\pound{ ф. стерл.}; і дійсний факт є лише в тому, що капітал
в 2\sfrac{1}{2}\pound{ ф. стерл.} дав 2 квартери, а капітал в 5\pound{ ф. стерл.} — не 4, а 3\sfrac{1}{2} квартери.
Справа ані трохи не змінилася б, якби ці 5\pound{ ф. стерл.} дали 4 квартери, так що
продукти обох витрат капіталу були б однакові, або навіть 5 квартерів, так
що друга витрата капіталу дала б надмір в 1 квартер. Ціна продукції перших
двох квартерів дорівнює 1\sfrac{1}{2}\pound{ ф. стерл.} за квартер, ціна продукції других 1\sfrac{1}{2} квартерів є 2\pound{ ф.
стерл.} за квартер. Ці 3\sfrac{1}{2} квартери разом коштують тому 6\pound{ ф. стерл}.
Це є індивідуальна ціна продукції всього продукту, а пересічно вона становить
1\pound{ ф. стерл.} 14\sfrac{2}{7}\shil{ шил.} за квартер, округло, скажімо, 1\sfrac{3}{4}\pound{ ф. стерл}. За загальної ціни
продукції в 3\pound{ ф. стерл.}, що визначається землею $А$, це дає надзиск в 1\sfrac{1}{4}\pound{ ф. стерл.}
на квартер і, отже, для 3\sfrac{1}{2} квартерів разом. — 4\sfrac{3}{8}\pound{ ф. стерл}. За пересічної ціни
\parbreak{}  %% абзац продовжується на наступній сторінці

\parcont{}  %% абзац починається на попередній сторінці
\index{iii2}{0177}  %% посилання на сторінку оригінального видання
дукції з землі В це становить округло 1\sfrac{1}{2} квартера. Надзиск з В визначається,
отже, у відповідній частині продукту з В, в цих 1\sfrac{1}{2} квартерах, які
становлять ренту, визначену в збіжжі, і які продаються по загальній ціні продукції
за 4\sfrac{1}{2} ф. стерл. Але, навпаки, надмірний продукт з акра землі В, надмірний
проти продукту з акра землі А, не можна просто вважати за надзиск,
а тому й за надпродукт. Згідно з припущенням акр землі В продукує 3\sfrac{1}{2} квартери,
акр землі А лише 1 квартер. Надмірний продукт з землі В є, отже,
2\sfrac{1}{2} квартери, але надпродукт є лише 1\sfrac{1}{2} квартери; бо в землю
В вкладено удвоє більший капітал, ніж у землю А, і тому вся ціна продукції тут удвоє
більша. Коли б у землю А також було вкладено 5 ф стерл. і норма продуктивности
лишилася б без зміни, то продукт становив би 2 квартери замість одного,
і таким чином виявилося б, що дійсний надпродукт можна знайти порівнянням
не 3\sfrac{1}{2} і 1, а 3\sfrac{1}{2} і 2; що, отже, він дорівнює не 2\sfrac{1}{2},
а лише 1\sfrac{1}{2} квартерам.
Але далі, якби в землю В було вкладено третю порцію капіталу в 2\sfrac{1}{2} ф. стерл.,
що дала б лише 1 квартер, так що він коштував би 3 ф. стерл., як на землі А, то
його продажна ціна в 3 ф. ст. покрила б тільки ціну продукції, дала б лише
пересічний зиск, але не дала б надзиску, а отже і нічого, що могло б перетворитися
на ренту. Продукт з акра будь-якого роду землі, порівняно з продуктом
з акра землі А, не показує ані того, чи є він продукт однакової або більшої
витрати капіталу, ані того, чи надмірний продукт покриває тільки ціну продукції,
чи завдячує він своїм виникненням вищій продуктивності додаткового капіталу.

\emph{Друге}: З щойно викладеного випливає, що при низхідній нормі продуктивности
додаткових витрат капіталу, за межу котрих, — оскільки мова йде
про створення нового надзиску, — є така витрата капіталу, що покриває лише
ціну продукції, тобто що продукує квартер так само дорого, як рівна витрата
капіталу на землі А, отже, згідно з припущенням, за 3 ф. стерл., — випливає,
що за межу, на якій загальна витрата капіталу на акр землі В перестала б
давати ренту, є та, коли індивідуальна пересічна ціна продукції продукту з
акра землі В підвищилася б до рівня ціни продукції з акра землі А.

Коли на В робляться лише такі додаткові витрати капіталу, що оплачують
ціну продукції і, отже, не створюють надзиску, а тому й нової ренти, то хоч
це й підвищує індивідуальну пересічну ціну продукції квартера, проте, не зачіпає
надзиску, що створився від попередніх витрат капіталу, евентуально ренти. Бо
пересічна ціна продукції завжди лишається нижча від ціни продукції на А, а коли
надмір ціни з квартера і зменшується, то кількість квартерів збільшується
у тому самому відношенні, так що загальний надмір ціни лишається без зміни.

В наведеному випадку дві перші витрати капіталу в 5 ф. стерл. на землі В
продукують 3\sfrac{1}{2} квартери, отже, згідно з припущенням, 1\sfrac{1}{2}
квартери ренти = 4\sfrac{1}{2} ф. стерл. Коли сюди прилучиться третя витрата
капіталу в 2\sfrac{1}{2} ф. стерл.,
що продукує лише 1 додатковий квартер, то вся ціна продукції (включаючи 20\%
зиску) 4\sfrac{1}{2} квартерів = 9 ф. стерл.; отже, пересічна ціна за
квартер = 2 ф. стерл. Отже, пересічна ціна продукції за квартер на землі В
піднеслась з 1\sfrac{5}{7} ф. стерл.
до 2 ф. стерл., надзиск з квартера порівняно з регуляційною ціною А упав
з 1\sfrac{2}{7} ф. стерл. до 1 ф. стерл. Але 1×4\sfrac{1}{2} = 4\sfrac{1}{2} ф.
стерл., цілком так само,
як раніш $1\sfrac{2}{7} × 3\sfrac{1}{2} = 4\sfrac{1}{2}$ ф. стерл.

Коли ми припустимо, що на В було б зроблено ще четверту і п’яту додаткові
витрати капіталу по 2\sfrac{1}{2} ф. стерл., які продукують квартер лише по його
загальній ціні продукції, то весь продукт з акра становив би тепер 6\sfrac{1}{2} квартерів,
а ціна його продукції була б 15 ф. стерл. Пересічна ціна продукції
квартера для В знову підвищилась би з 2
\footnote*{В німецькому тексті тут стоїть «з 1 ф. стерл.» Очевидна помилка,
бо у вищенаведеному прикладі пересічна ціна продукції квартера для В
становила не 1 ф. стерл., а 2 ф. стерл. \emph{Прим. Ред.}}
до 2\sfrac{4}{13} ф. стерл., а надзиск з квартера
\index{iii2}{0178}  %% посилання на сторінку оригінального видання
порівняно з реґуляційною ціною продукції на землі А знову зменшився
б з 1 ф. стерл, до \sfrac{9}{13} ф. стерл. Але ці \sfrac{9}{13} ф. стерл. тут слід
помножити на 6\sfrac{1}{2} квартерів замість колишніх 4\sfrac{1}{2}
А $\sfrac{9}{13}×6\sfrac{1}{2} = 1× 4\sfrac{1}{2} = 4\sfrac{1}{2}$ ф. стерл.

Звідси насамперед випливає, що за цих обставин не потрібно жодного підвищення
реґуляційної ціни продукції для того, щоб уможливити додаткові витрати
капіталу на рентодайних землях, навіть в такому розмірі, що додатковий
капітал зовсім перестає давати надзиск і дає ще лише пересічний зиск. З
цього випливає далі, що тут сума надзиску на акр лишається без зміни,
хоч би як дуже зменшувався надзиск з квартера; це зменшення завжди урівноважується
відповідним збільшенням квартерів, продукованих на акрі. Для того,
щоб пересічна ціна продукції піднеслась до рівня загальної ціни продукції (отже,
тут досягла б 3 ф. стерл. на землі В), мусять бути зроблені такі додаткові витрати
капіталу, продукт яких мав би вищу ціну продукції, ніж реґуляційна ціна
в 3 ф. стерл. Але ми побачимо, що тільки цього ще не досить, щоб підвищити
пересічну ціну продукції квартера на землі В до рівня загальної ціни продукції
в 3 ф. стерл.

Припустімо, що на землі В було випродуковано:

1) 3\sfrac{1}{2} квартери, що їхня ціна продукції, як і давніш, 6 ф. стерл.; отже,
дві витрати капіталу по  2\sfrac{1}{2} ф. стерл. кожна, при чому обидві дають надзиски,
але низхідної висоти.

2) 1 квартер за 3 ф. стерл.; витрата капіталу, при якій індивідуальна
ціна продукції дорівнювала б реґуляційній ціні продукції.

3) 1 квартер за 4 ф. стерл.; витрата капіталу, при якій індивідуальна
ціна продукції на 25\% вища за реґуляційну ціну.

Ми мали б тоді 5\sfrac{1}{2} квартерів з акра за 13 ф. стерл. при витраті капіталу
в 10 ф. стерл.; первісна витрата капіталу зросла б учетверо, але продукт
першої витрати капіталу не збільшився б і втроє.

5\sfrac{1}{2} квартерів за 13 ф. дають пересічну ціну продукції в 2\sfrac{4}{11} ф. стерл.
за квартер, отже, за реґуляційної ціни продукції в 3 ф. стерл. надмір
в \sfrac{7}{11} ф. стерл. з квартера, який може перетворитися на ренту.
5\sfrac{1}{2} квартерів, продані по реґуляційній ціні в 3 ф. стерл. дають
16\sfrac{1}{2} ф. стерл. За вирахуванням
ціни продукції в 13 ф. стерл. залишається 3\sfrac{1}{2} ф. стерл. надзиску, або
ренти, так що ці 3\sfrac{1}{2} ф. стерл., рахуючи по теперішній пересічній ціні
продукції квартера з землі В, тобто по 2\sfrac{4}{11} ф. стерл. за квартер,
репрезентують 1\sfrac{25}{52}\footnote*{
В німецькому тексті тут стоїть: «1\sfrac{5}{72}». Очевидна помилка. \emph{Прим. Ред.}
} квартера. Грошова рента понизилася б на 1 ф. стерл., збіжжева
рента приблизно на \sfrac{1}{2} квартера, але, не зважаючи на те, що четверта додаткова
витрата капіталу на В не тільки не створює надзиску, але дає менше, ніж
пересічний зиск, — як і давніш, існує надзиск і рента. Коли ми припустимо, що,
крім витрати капіталу 3), і витрата 2) продукує по ціні, що перебільшує реґуляційну
ціну продукції, то вся продукція становитиме: 3\sfrac{1}{2} квартери за
6 ф. ст. + 2 квартери за 8 ф. ст., разом 5\sfrac{1}{2} квартерів за 14 ф. ст.
ціни продукції. Пересічна ціна продукції квартера була б 2\sfrac{6}{11}  ф. ст.,
що давало б надмір в \sfrac{5}{11} ф. ст. Ці 5\sfrac{1}{2}  квартери, продані по
3 ф. ст., дають 16\sfrac{1}{2} ф. ст.; за вирахуванням
14 ф. ст. ціни продукції, лишається 2\sfrac{1}{2} ф. ст. на ренту. За теперішньої
пересічної ціни продукції на землі В це становило б \sfrac{55}{56} квартера.
Отже, рента все ще одержується, хоч і в меншому розмірі, ніж давніш.

В усякому разі це показує, що на кращих земельних дільницях при додаткових
витратах капіталу, що їхній продукт коштує дорожче, ніж реґуляційна
ціна продукції, рента, принаймні в межах допустимих практикою, мусить не
зникнути, а лише зменшитися, і саме відповідно до того, з одного боку, яку
\parbreak{}  %% абзац продовжується на наступній сторінці

\parcont{}  %% абзац починається на попередній сторінці
\index{iii2}{0179}  %% посилання на сторінку оригінального видання
частку всього витраченого капіталу становить цей менш продуктивний капітал,
а з другого боку, відповідно до зменшення його продуктивности. Пересічна
ціна продукту цього менш продуктивного капіталу все ще була б нижча
від регуляційної ціни, і тому все ще лишався б надзиск, який міг би перетворитися
на ренту.

Припустімо тепер, що пересічна ціна квартера з землі $В$ збігається з загальною
ціною продукції, в наслідок чотирьох послідовних витрат капіталу
(2\sfrac{1}{2}, 2\sfrac{1}{2}, 5 і 5\pound{ ф. ст.}) з низхідною продуктивністю:

\begin{table}[H]
  \begin{center}
    \footnotesize

  \begin{tabular}{c@{  } c@{  } c@{  } c@{  } c@{  } c@{  } c@{  } c@{  } c@{  } c@{  } c}
    \toprule
      \multicolumn{2}{c}{Капітал} &
      Зиск &
      Здобуток &
      \multicolumn{2}{c}{Ціна продукції} &
      \makecell{Продажна \\ ціна} &
      Здобуток &
      \multicolumn{2}{c}{Додаток для ренти} \\

      \cmidrule(r){1-2}
      \cmidrule(r){3-3}
      \cmidrule(r){4-4}
      \cmidrule(r){5-6}
      \cmidrule(r){7-7}
      \cmidrule(r){8-8}
      \cmidrule(r){9-10}

        & ф. ст. & ф. ст. & Кварт & \makecell{за квартер \\ ф. ст.} & \makecell{разом \\ ф. ст.} & ф. ст. & ф. ст. & Кварт. & ф. ст.   \\
      \midrule
      1) & \phantom{0}2\sfrac{1}{2}           & \phantom{1}\sfrac{1}{2} & 2\phantom{\sfrac{1}{2}} & 1\sfrac{1}{2}           & \phantom{0}3 & 3 & \phantom{0}6\phantom{\sfrac{1}{2}} & \phantom{-}1\phantom{\sfrac{1}{2}} & \phantom{-}3\phantom{\sfrac{1}{2}} \\
      2) & \phantom{0}2\sfrac{1}{2}           & \phantom{1}\sfrac{1}{2} & 1\sfrac{1}{2}           & 2\phantom{\sfrac{1}{2}} & \phantom{0}3 & 3 & \phantom{0}4\sfrac{1}{2}           & \phantom{-1}\sfrac{1}{2}           & \phantom{-}1\sfrac{1}{2} \\
      3) & \phantom{0}5\phantom{\sfrac{1}{2}} & 1\phantom{\sfrac{1}{2}} & 1\sfrac{1}{2}           & 4\phantom{\sfrac{1}{2}} & \phantom{0}6 & 3 & \phantom{0}4\sfrac{1}{2}           & -\phantom{1}\sfrac{1}{2}           & -1\sfrac{1}{2}           \\
      4) & \phantom{0}5\phantom{\sfrac{1}{2}} & 1\phantom{\sfrac{1}{2}} & 1\phantom{\sfrac{1}{2}} & 6\phantom{\sfrac{1}{2}} & \phantom{0}6 & 3 & \phantom{0}3\phantom{\sfrac{1}{2}} & -1\phantom{\sfrac{1}{2}}           & -3\phantom{\sfrac{1}{2}} \\
     \cmidrule(r){2-2}
     \cmidrule(r){3-3}
     \cmidrule(r){4-4}
     \cmidrule(r){6-6}
     \cmidrule(r){8-8}
     \cmidrule(r){9-9}
     \cmidrule(r){10-10}

       & 15\phantom{\sfrac{1}{2}} & 3\phantom{\sfrac{1}{2}} & 6\phantom{\sfrac{1}{2}} & & 18 & & 18\phantom{\sfrac{1}{1}} & \phantom{-}0\phantom{\sfrac{1}{2}} & \phantom{-}0\phantom{\sfrac{1}{2}} \\
  \end{tabular}

  \end{center}
\end{table}

Тут орендар продає кожен квартер по його індивідуальній ціні продукції,
і тому все число квартерів продає він по їхній пересічній ціні продукції квартера,
яка збігається з регуляційною ціною в 3\pound{ ф. стерл}. Він одержує тому на свій
капітал в 15\pound{ ф. стерл.}, як і давніш, 20\% зиску = 3\pound{ ф. стерл}. Але рента зникла.
Куди ж дівся надмір при цьому вирівнянні індивідуальних цін продукції кожного
квартера з загальною ціною продукції?

Надзиск з перших 2\sfrac{1}{2}\pound{ ф. стерл.} був 3\pound{ ф. стерл.}; з других 2\sfrac{1}{2}\pound{ ф. стерл.}
він був 1\sfrac{1}{2}\pound{ ф. стерл.}; разом надзиск на  \sfrac{1}{3} авансованого капіталу, тобто на
5\pound{ ф. стерл.} = 4\sfrac{1}{2}\pound{ ф. стерл.} = 90\%.

При витраті капіталу 3) 5\pound{ ф. стерл.} не тільки не дають надзиску, але
продукт їхній в 1\sfrac{1}{2} квартера, проданий по загальній ціні продукції, дає мінус в
1\sfrac{1}{2}\pound{ ф. стерл}. Нарешті, при витраті капіталу 4) теж в 5\pound{ ф. стерл.}, продукт
їхній в 1 кв., проданий по загальній ціні продукції, дає мінус в 3\pound{ ф. стерл}. Отже,
обидві витрати капіталу, взяті разом, дають мінус в 4\sfrac{1}{2}\pound{ ф. стерл.}, рівний надзискові
в 4\sfrac{1}{2}\pound{ ф. стерл.}, який постав від витрат капіталу 1) і 2).

Надзиск і мінус-зиск урівноважуються. Тому рента зникає. Але в дійсності
це можливе тому, що елементи додаткової вартости, які раніш становили
надзиск або ренту, входять тепер в створення пересічного зиску. Орендар одержує
цей пересічний зиск в розмірі 3\pound{ ф. стерл.} на 15\pound{ ф. стерл.}, або в розмірі
20\% коштом ренти.

Вирівняння індивідуальної пересічної ціни продукції з землі $В$ з загальною
ціною продукції $А$, яка регулює ринкову ціну, має за передумову, що ріжниця,
на яку індивідуальна ціна продукту перших витрат капіталу нижча,
ніж регуляційна ціна, дедалі більше зрівноважується і нарешті знищується
ріжницею, на яку продукт пізніших витрат капіталу починає перебільшувати
регуляційну ціну. Те, що являє собою надзиск, поки продукт перших витрат
капіталу продається сам по собі, в такий спосіб поступово стає частиною їхньої
пересічної ціни продукції, і разом з тим входить в створення пересічного зиску,
аж поки, нарешті, не буде зовсім поглинуте цим пересічним зиском.

Коли б замість вкладати в землю $В$ 15\pound{ ф. стерл.} капіталу, в неї було вкладено
лише 5\pound{ ф. стерл.} і додаткові 2\sfrac{1}{2} квартери останньої таблиці були випродуковані
\parbreak{}  %% абзац продовжується на наступній сторінці

\parcont{}  %% абзац починається на попередній сторінці
\index{iii2}{0180}  %% посилання на сторінку оригінального видання
в такий спосіб, що 2\sfrac{1}{2} акри землі $А$ були б оброблені наново з витратою капіталу
в 2\sfrac{1}{2}\pound{ ф. стерл.} на акр, то витрачений додатковий капітал становив би
лише 6\sfrac{1}{4} ф. стерл, отже, вся витрата на $А$ і $В$ для продукції цих 6 квартерів
становила б лише 11\sfrac{1}{4}\pound{ ф. стерл.} замість 15\pound{ ф. стерл.}, і вся їхня ціна
продукції, включаючи й зиск, становила б 13\sfrac{1}{2}\pound{ ф. стерл}. Ці 6 квартерів, як
і давніш, разом продавалося б за 18\pound{ ф. стерл.}, але витрата капіталу зменшилася
б на 3\sfrac{3}{4}\pound{ ф. стерл.}, і рента з $В$ становила б, як і давніше, 4\sfrac{1}{2}\pound{ ф. стерл.}
на акр. Інакше стояла б справа, коли б для продукції додаткових 2\sfrac{1}{2} квартерів
довелося вжити гіршої, ніж $А$, землі, $А_{-1}$, $А_{-2}$; так що ціна продукції
квартера для 1\sfrac{1}{2} квартера на землі $А_{-1} \deq{} 4$\pound{ ф. стерл.}, а для останнього
квартера на землі $А_{-2} \deq{} 6$\pound{ ф. стерл}. В цьому випадку 6\pound{ ф. стерл.}
зробилися б реґуляційною ціною продукції квартера. 3\sfrac{1}{2} квартери з землі $В$
були б продані за 21\pound{ ф. стерл.} замість 10\sfrac{1}{2}\pound{ ф. стерл.}, що дало б ренту в 15\pound{ ф.
стерл.} замість 4\sfrac{1}{2}\pound{ ф. стерл.}, а в збіжжі ренту в 2\sfrac{1}{2} кв. замість 1\sfrac{1}{2} квартера.
Так само квартер з землі $А$ тепер репрезентував би ренту в 3\pound{ ф. стерл.} \deq{}
\sfrac{1}{2} квартера.

Перше, ніж дослідити цей пункт далі, зробимо ще одне зауваження.

Пересічна ціна квартера з $В$ вирівнюється, збігається з загальною ціною
продукції в 3\pound{ ф. стерл.} з квартера, регульованою землею $А$, скоро частина
всього капіталу, що продукує надмірні 1\sfrac{1}{2} квартера зрівноважиться тією частиною
всього капіталу, яка недопродуковує 1\sfrac{1}{2} квартера. Як скоро вирівнюються
ці ціни, або скільки для цього треба витратити на $В$ капіталу з недостатньою
продуктивною силою, — це залежить, коли припустити додаткову
продуктивність перших витрат капіталу за дану, від відносно недостатньої продуктивности
наступних витрат капіталу проти продуктивности рівновеликої витрати
капіталу на найгіршій регуляційній землі $А$, або від індивідуальної ціни
продукції їхнього продукту порівняно з реґуляційною ціною.

\pfbreak

З попереднього насамперед випливає:

\emph{Перше.} Доти, доки додаткові капітали витрачаються на тій самій землі
з додатковою продуктивністю, хоча б і низхідною, абсолютно рента з акра
так збіжжева, як і грошова, зростає, хоч відносно проти авансованого капіталу
(отже, норма надзиску або ренти) вона зменшується. За межу тут є той додатковий
капітал, що дає лише пересічний зиск, або для продукту якого індивідуальна
ціна продукції збігається з загальною ціною продукції. Ціна продукції
за цих умов залишається та сама, коли тільки продукція на гірших землях
не стає зайвою в наслідок збільшеного подання. Навіть за низхідних цін ці
додаткові капітали можуть у певних межах все ще продукувати надзиск, хоч
і менш значний.

\emph{Друге.} Витрата додаткового капіталу, що продукує тільки пересічний
зиск і додаткова продуктивність якого, отже \deq{} 0, нічого не змінює у висоті
створеного надзиску, а тому і ренти. Індивідуальна пересічна ціна квартера на
кращих землях в наслідок цього зростає. Надмір з квартера зменшується, але
число квартерів, що дають такий зменшений надмір, збільшується, так що здобуток
лишається той самий.

\emph{Третє.} Додаткові витрати капіталу, які дають продукт, що його індивідуальна
ціна продукції перевищує регуляційну ціну, отже, додаткова продуктивність
яких не тільки \deq{} 0, але менша нуля, деякий мінус, тобто нижча,
ніж продуктивність рівної витрати капіталу на регуляційній землі $А$, дедалі
більше наближають індивідуальну пересічну ціну всього продукту з кращих
земель до загальної ціни продукції, отже, дедалі більше зменшують ріжницю
між обома, ту ріжницю, яка створює надзиск, зглядно ренту. Дедалі
більша частина того, що раніше становило надзиск або ренту, входить в
\parbreak{}  %% абзац продовжується на наступній сторінці

\parcont{}  %% абзац починається на попередній сторінці
\index{iii1}{0181}  %% посилання на сторінку оригінального видання
плату плюс додаткову вартість, додаткову працю понад їх необхідні
потреби, при чому, однак, результати її належали б їм
самим. Висловлюючись капіталістичною мовою, обидва робітники
одержують рівну заробітну плату плюс рівний зиск, але разом
з тим і вартість, виражену, наприклад, у продукті десятигодинного
робочого дня. Але, поперше, вартості їх товарів були б
різні. Нехай, наприклад, з уміщеної в товарі І вартості на спожиті
засоби виробництва припадає більша частина вартості, ніж у товарі
II, і — щоб урахувати тут усі можливі ріжниці — припустімо,
що товар І вбирає більше живої праці, отже, потребує довшого
робочого часу для свого виготовлення, ніж товар II.~Отже, вартість
цих товарів І і II дуже різна. Так само різні й суми товарних вартостей,
які є продуктом праці, виконаної за даний час робітником
І і робітником II.~Норми зиску теж дуже різні для І і II,
якщо ми назвемо тут нормою зиску відношення додаткової вартості
до всієї вартості витрачених засобів виробництва. Засоби
існування, які щодня споживаються робітниками І і II протягом
виробництва і які представляють заробітну плату, становлять
тут ту частину авансованих засобів виробництва, яку ми в інших
випадках звемо змінним капіталом. Але додаткові вартості за
однаковий робочий час були б для І і II однакові, або ще точніше:
через те що І і II одержують кожний вартість продукту
одного робочого дня, вони одержують — якщо відрахувати вартість,
авансованих „сталих“ елементів — однакові вартості, одну
частину яких можна розглядати як заміщення спожитих у виробництві
засобів споживання, а другу — як додаткову вартість,
яка лишається понад це. Якщо І зробив більше витрат, то вони
заміщаються більшою частиною вартості його товару, яка заміщає
цю „сталу“ частину, і тому він повинен також більшу
частину всієї вартості свого продукту перетворити знову в речові
елементи цієї сталої частини, тимчасом як II, якщо він
менше одержав як заміщення, повинен зате настільки ж менше
знову перетворити в елементи сталої частини. Отже, при цьому
припущенні ріжниця в нормах зиску була б байдужою обставиною,
цілком так само, як нині для найманого робітника байдуже,
в якій нормі зиску виражається видушена з нього кількість
додаткової вартості, і цілком так само, як у міжнародній торгівлі
ріжниця норм зиску у різних націй є байдужа обставина
для їх товарообміну.

Отже, для обміну товарів по їх вартостях, або приблизно
по їх вартостях, потрібен значно нижчий ступінь, ніж для обміну
по цінах виробництва, для якого потрібна певна висота капіталістичного
розвитку.

Яким би чином не встановлювались або реґулювались первісно
ціни різних товарів одного відносно одного, закон вартості
керує їх рухом. Де зменшується робочий час, потрібний для
виробництва товарів, там падають і ціни; де він збільшується,
там підвищуються, при інших незмінних умовах, і ціни.

\parcont{}  %% абзац починається на попередній сторінці
\index{iii2}{0182}  %% посилання на сторінку оригінального видання
капіталу, яка продукувала б квартер дорожче за 3\pound{ ф. стерл.}, призвела б до
скорочення його зиску. Це, за недостатньої продуктивности, перешкоджає вирівнянню індивідуальної
пересічної ціни.

Візьмімо цей випадок у колишньому прикладі, коли ціна продукції на
землі $А$ в 3\pound{ ф. стерл.} за квартер реґулює ціну для землі $В$.

\begin{table}[h]
  \begin{center}
    \footnotesize

  \begin{tabular}{c@{  } c@{  } c@{  } c@{  } c@{  } c@{  } c@{  } c@{  } c@{  } c@{  } c}
    \toprule
      Капітал &
      Зиск &
      \makecell{Ціна \\ продукції} &
      Здобуток &
      \makecell{Ціна \\ продукції \\ за кварт.} &
      \multicolumn{2}{c}{Продажна ціна} &
      Надзиск &
      Витрата \\

      \cmidrule(r){1-1}
      \cmidrule(r){2-2}
      \cmidrule(r){3-3}
      \cmidrule(r){4-4}
      \cmidrule(r){5-5}
      \cmidrule(r){6-7}
      \cmidrule(r){8-8}
      \cmidrule(r){9-9}\pound{ ф. ст.} & ф. ст. & ф. ст. & Кварт & ф. ст. & \makecell{за квартер \\ ф. ст.} & \makecell{разом \\ ф. ст.} & ф. ст. & ф. ст.   \\
      \midrule
      \phantom{0}2\sfrac{1}{2}           & \phantom{1}\sfrac{1}{2} & \phantom{0}3 & 2\phantom{\sfrac{1}{2}} & 1\sfrac{1}{2}           & 3 & \phantom{0}6\phantom{\sfrac{1}{2}} & 3\phantom{\sfrac{1}{2}} & \textemdash \\
      \phantom{0}2\sfrac{1}{2}           & \phantom{1}\sfrac{1}{2} & \phantom{0}3 & 1\sfrac{1}{2}           & 2\phantom{\sfrac{1}{2}} & 3 & \phantom{0}4\sfrac{1}{2}           & 1\sfrac{1}{2}           & \textemdash \\
      \phantom{0}5\phantom{\sfrac{1}{2}} & 1\phantom{\sfrac{1}{2}} & \phantom{0}6 & 1\sfrac{1}{2}           & 4\footnotemarkZ{}\phantom{1}  & 3 & \phantom{0}4\sfrac{1}{2}           & \textemdash             & 1\sfrac{1}{2}           \\
      \phantom{0}5\phantom{\sfrac{1}{2}} & 1\phantom{\sfrac{1}{2}} & \phantom{0}6 & 1\phantom{\sfrac{1}{2}} & 6\phantom{\sfrac{1}{2}} & 3 & \phantom{0}3\phantom{\sfrac{1}{2}} & \textemdash             & 3\phantom{\sfrac{1}{2}} \\

     \cmidrule(r){1-1}
     \cmidrule(r){2-2}
     \cmidrule(r){3-3}
     \cmidrule(r){7-7}
     \cmidrule(r){8-8}
     \cmidrule(r){9-9}

       15\phantom{\sfrac{1}{2}} & 3\phantom{\sfrac{1}{2}} & 18 & & & & 18\phantom{\sfrac{1}{1}} & 4\sfrac{1}{2} & 4\sfrac{1}{2} \\
  \end{tabular}

  \end{center}
\end{table}
\footnotetextZ{В німецькому тексті тут стоїть «3». Очевидна помилка. \Red{Пр.~Ред}}

Ціна продукції 3\sfrac{1}{2} квартерів з перших двох витрат капіталу так само
становить для орендаря 3\pound{ ф. стерл.} за квартер, бо йому доводиться виплачувати
ренту в 4\sfrac{1}{2}\pound{ ф. стерл.}, при чому ріжниця між його індивідуальною ціною продукції
і загальною ціною продукції іде, таким чином, не в його кишеню. Отже,
надмір ціни продукту перших двох витрат не може йому покрити дефіциту
в продуктах третьої і четвертої витрати капіталу.

1\sfrac{1}{2} квартера від витрати капіталу 3) коштують орендареві, включаючи
і зиск, 6\pound{ ф. стерл.}; але за реґуляційної ціни в 3\pound{ ф. стерл.} за квартер він може
продати їх лише за 4\sfrac{1}{2}\pound{ ф. стерл}. Отже, він втратив би не тільки ввесь
зиск, але й понад нього \sfrac{1}{2}\pound{ ф. стерл.} або  10\% вкладеного капіталу в 5\pound{ ф. стерл}.
Його втрата з зиску і капіталу при витраті капіталу 3) дорівнювала б  1\sfrac{1}{2}\pound{ ф. стерл.}, а при витраті капіталу 4) — 3\pound{ ф. стерл.}, разом 4\sfrac{1}{2}\pound{ ф. стерл.}, якраз
стільки, скільки становить рента від продуктивніших витрат капіталу, що їхня
індивідуальна ціна продукції саме тому не може справити вирівнювального
впливу на індивідуальну пересічну ціну продукції всього продукту землі $В$, що
надмір його доводиться виплатити як ренту третій особі.

Коли б для задоволення попиту довелося випродукувати додаткові 1\sfrac{1}{2}
квартери з допомогою третьої витрати капіталу, то регуляційна ринкова ціна
мусила б піднестись до 4\pound{ ф. стерл.} за квартер. В наслідок цього підвищення
реґуляційної ринкової ціни рента на землі $В$ для першої і другої витрати капіталу
підвищилась би, а на землі $А$ створилася б рента.

Отже, хоч диференційна рента є лише формальне перетворення надзиску
в ренту, і хоч власність на землю дає тут власникові лише можливість перемістити
надзиск з рук орендаря в свої, проте виявляється, що послідовна витрата
капіталу на ту саму земельну площу або, що сходить на те саме, збільшення
капіталу, витраченого на тій самій земельній площі, за низхідної норми
продуктивности капіталу і незмінної реґуляційної ціни геть швидше доходить
до своєї межі, отже, досягає в дійсності більш або менш штучної межі, в наслідок
просто формального перетворення надзиску в земельну ренту, що є наслідком
земельної власности. Отже підвищення загальної ціни продукції, яке стає
тут доконечним за вужчих меж, ніж в інших умовах, є тут не тільки за причину
\parbreak{}  %% абзац продовжується на наступній сторінці

\parcont{}  %% абзац починається на попередній сторінці
\index{iii2}{0183}  %% посилання на сторінку оригінального видання
підвищення диференційної ренти, але саме існування диференційної ренти як
ренти є разом з тим причина ранішого й швидшого підвищення загальної ціни
продукції, щоб таким чином забезпечити збільшене подання продукту, яке стало доконечним.

Треба зауважити далі таке:

Через додаткове капіталовкладення в землю $В$ не могла б підвищитися
регуляційна ціна до 4\pound{ ф. стерл.}, як це наведено вище, коли б земля $А$, в наслідок
другої витрати капіталу, давала додаткову продукцію дешевше від 4\pound{ ф.
стерл.}, або коли б вступила в конкуренцію нова гірша, ніж $А$, земля, що на
ній ціна продукції була б хоч і вища 3, але нижча 4\pound{ ф. стерл}. Таким чином,
ми бачимо, як диференційна рента І і диференційна рента II, тимчасом як
перша є за базу для другої, одночасно правлять одна для однієї за межу, що
спричинює то послідовні витрати капіталу на тій самій земельній дільниці, то
витрати капіталу одну біля однієї на новій додатковій землі. Так само вони
обмежують одна одну і в інших випадках, коли, наприклад, черга доходить до
кращих земель.

\section{Диференційна рента і з найгіршої з оброблюваних земель}

Припустімо, що попит на збіжжя підвищується, і що подання може бути
задоволене лише через послідовні витрати капіталу з недостатньою продуктивністю
на землях, що дають ренту, або через додаткову витрату капіталу
теж з низхідною продуктивністю на землі $А$, або через витрату капіталу на
нових землях гіршої якости, ніж $А$.

Візьмімо землю $В$ як представницю земель, що дають ренту.

Щоб уможливити додаткову продукцію 1 квартера на землі $В$ (який
тут може становити 1 мільйон квартерів, як кожен акр — мільйон акрів), додаткове
капіталовкладення вимагає підвищення ринкової ціни понад 3\pound{ ф. стерл.}
за квартер, що були до цього часу за регуляційну ціну. На землях $C$ і $D$ і~\abbr{т.
ін.} родах землі з найвищою рентою, теж може бути випродуковано додатковий
продукт, але лише з низхідною додатковою продуктивною силою; проте,
припускається, що 1 квартер землі $В$ потрібен для задоволення попиту.
Коли цей один квартер можна дешевше випродукувати з допомогою додаткового
капіталу на $В$, ніж з допомогою рівної витрати додаткового капіталу
на $А$, або спускаючись до землі $А_{-1}$, яка може випродукувати квартер, наприклад,
лише за 4\pound{ ф. стерл.}, тимчасом як додатковий капітал на $А$ міг би випродукувати
квартер уже за 3\sfrac{3}{4}\pound{ ф. стерл.}, то додатковий капітал, витрачений на
$В$, почав би регулювати ринкову ціну.

Земля $А$, як і давніш, випродукувала 1 квартер за 3\pound{ ф. стерл.} $В$ теж,
як і давніш, випродукувала в цілому 3\sfrac{1}{2} квартера, що їхня індивідуальна ціна
продукції становить разом 6\pound{ ф. стерл}. Тепер, коли б на землі $В$ потрібно було
додаткової витрати в 4\pound{ ф. стерл.} ціни продукції (включаючи і зиск), щоб випродукувати ще 1 квартер,
тимчасом як на $А$ його можна випродукувати за
3\sfrac{3}{4}\pound{ ф. стерл.}, то, зрозуміла річ, він був би випродукований на $А$, а не на $В$.
Отже, припустімо, що він може бути випродукований на $В$ з 3\sfrac{1}{2}\pound{ ф. стерл.}
додаткової ціни продукції. В цьому випадку 3\sfrac{1}{2}\pound{ ф. стерл.} були б регуляційною ціною
для всієї продукції. Тоді $В$ продав би свій теперішній продукт в 4\sfrac{1}{2}  квартери за
15\sfrac{3}{4}\pound{ ф. стерл}. З цього па ціну продукції перших 3\sfrac{1}{2}  квартерів припадає
6\pound{ ф. стерл.} і на останній квартер З\sfrac{1}{2}\pound{ ф. стерл.}, разом 9\sfrac{1}{2}\pound{ ф. стерл}. Лишається
надзиск, для ренти \deq{} 6\sfrac{1}{4}  ф. стерл, проти лише 4\sfrac{1}{2}\pound{ ф. стерл.} колишніх.
В цьому випадку акр землі $А$ також дав би ренту в  \sfrac{1}{2}\pound{ ф. стерл.};
\parbreak{}  %% абзац продовжується на наступній сторінці

\parcont{}  %% абзац починається на попередній сторінці
\index{iii2}{0184}  %% посилання на сторінку оригінального видання
але ціну продукції в 3\sfrac{1}{2}\pound{ ф. стерл.} реґулювала б не найгірша земля $А$, а краща
земля $В$. Звичайно, при цьому припускається, що нова земля якости $А$ такого
ж зручного положення, як оброблювана до цього часу, є неприступна, і що довелося
би зробити другу витрату капіталу на вже оброблюваній дільниці $А$, але
з більшою ціною продукції, або довелось би притягнути до обробітку ще гіршу
землю $А_{-1}$. Коли в наслідок послідовних витрат капіталу починає діяти днференційна
рента II, то може статися, що межі підвищуваної ціни продукції реґулюватимуться
кращою землею і гірша земля, база диференційної ренти І, тоді
теж може давати ренту. Таким чином, при наявності самої лише диференційної
ренти всі оброблювані землі почали б тоді давати ренту. Ми мали б тоді такі
дві таблиці, в яких під ціною продукції розуміється суму авансованого капіталу
плюс 20\%  зиску, отже на кожні 2\sfrac{1}{2}\pound{ ф. стерл.} капіталу по \sfrac{1}{2}\pound{ ф. стерл.}
зиску, разом 3\pound{ ф. стерл.} (див. табл. І).

\begin{table}[h]
  \begin{center}
    \footnotesize

  \begin{tabular}{c c c c c c c c}
    \toprule
      \multirowcell{2}{\makecell{Рід \\землі}} &
      \multirowcell{2}{\rotatebox[origin=c]{90}{Акри}} &
      \rotatebox[origin=c]{90}{\makecell{Ціна про- \\ дукції}} &
      \multirowcell{2}{\rotatebox[origin=c]{90}{\makecell{Продукт \\ в кварт.}}} &
      \rotatebox[origin=c]{90}{\makecell{Продажна \\ ціна}} &
      \rotatebox[origin=c]{90}{\makecell{Грошовий \\ здобуток}} &
      \rotatebox[origin=c]{90}{\makecell{Збіжжева \\ рента}} &
      \rotatebox[origin=c]{90}{\makecell{Грошова \\ рента}} \\

      \cmidrule(rl){3-3}
      \cmidrule(rl){5-5}
      \cmidrule(rl){6-6}
      \cmidrule(rl){7-7}
      \cmidrule(rl){8-8}

       &  &  ф. ст. & & ф. ст. & ф. ст. & ф. ст. & ф. ст.  \\
      \midrule

      A & 1 &  \phantom{0}3 & \phantom{0}1\phantom{\sfrac{1}{2}} & 3 & \phantom{0}3\phantom{\sfrac{1}{2}} & \phantom{0}0\phantom{\sfrac{1}{2}} & \phantom{0}0\phantom{\sfrac{1}{2}} \\
      B & 1 &  \phantom{0}6 & \phantom{0}3\sfrac{1}{2}           & 3 & 10\sfrac{1}{2}                     & \phantom{0}1\sfrac{1}{2}           & \phantom{0}4\sfrac{1}{2} \\
      C & 1 &  \phantom{0}6 & \phantom{0}5\sfrac{1}{2}           & 3 & 16\sfrac{1}{2}                     & \phantom{0}3\sfrac{1}{2}           & 10\sfrac{1}{2} \\
      D & 1 &  \phantom{0}6 & \phantom{0}7\sfrac{1}{2}           & 3 & 22\sfrac{1}{2}                     & \phantom{0}5\sfrac{1}{2}           & 16\sfrac{1}{2}           \\

     \cmidrule(rl){1-1}
     \cmidrule(rl){2-2}
     \cmidrule(rl){3-3}
     \cmidrule(rl){4-4}
     \cmidrule(rl){6-6}
     \cmidrule(rl){7-7}
     \cmidrule(rl){8-8}

     Разом & 4 & 21 & 17\sfrac{1}{2} & & 52\sfrac{1}{2} & 10\sfrac{1}{2} & 31\sfrac{1}{2} \\
  \end{tabular}

  \end{center}
\end{table}

Таке становище речей перед новою витратою капіталу в
3\sfrac{1}{2}\pound{ ф. стерл.} на $В$, що дає тільки 1 квартер. Після цієї витрати
капіталу справа стоїть так: (див. табл. II).

\begin{table}[h]
  \begin{center}
    \footnotesize

  \begin{tabular}{c c c c c c c c}
    \toprule
      \multirowcell{2}{\makecell{Рід \\землі}} &
      \multirowcell{2}{\rotatebox[origin=c]{90}{Акри}} &
      \rotatebox[origin=c]{90}{\makecell{Ціна про- \\ дукції}} &
      \multirowcell{2}{\rotatebox[origin=c]{90}{\makecell{Продукт \\ в кварт.}}} &
      \rotatebox[origin=c]{90}{\makecell{Продажна \\ ціна}} &
      \rotatebox[origin=c]{90}{\makecell{Грошовий \\ здобуток}} &
      \rotatebox[origin=c]{90}{\makecell{Збіжжева \\ рента}} &
      \rotatebox[origin=c]{90}{\makecell{Грошова \\ рента}} \\

      \cmidrule(rl){3-3}
      \cmidrule(rl){5-5}
      \cmidrule(rl){6-6}
      \cmidrule(rl){7-7}
      \cmidrule(rl){8-8}

       &  &  ф. ст. & & ф. ст. & ф. ст. & ф. ст. & ф. ст.  \\
      \midrule

      A & 1 &  \phantom{0}3\phantom{\sfrac{1}{2}} & \phantom{0}1\phantom{\sfrac{1}{2}} & 3\sfrac{1}{2} & \phantom{0}3\sfrac{1}{2} & \phantom{00}\sfrac{1}{7}   & \phantom{00}\sfrac{1}{2} \\
      B & 1 &  \phantom{0}9\sfrac{1}{2}           & \phantom{0}4\sfrac{1}{2}           & 3\sfrac{1}{2} & 15\sfrac{3}{4}           & \phantom{0}1\sfrac{11}{14} & \phantom{0}6\sfrac{1}{4} \\
      C & 1 &  \phantom{0}6\phantom{\sfrac{1}{2}} & \phantom{0}5\sfrac{1}{2}           & 3\sfrac{1}{2} & 19\sfrac{1}{4}           & \phantom{0}3\sfrac{11}{14} & 13\sfrac{1}{4} \\
      D & 1 &  \phantom{0}6\phantom{\sfrac{1}{2}} & \phantom{0}7\sfrac{1}{2}           & 3\sfrac{1}{2} & 26\sfrac{1}{4}           & \phantom{0}5\sfrac{11}{14} & 20\sfrac{1}{4}           \\

     \cmidrule(rl){1-1}
     \cmidrule(rl){2-2}
     \cmidrule(rl){3-3}
     \cmidrule(rl){4-4}
     \cmidrule(rl){6-6}
     \cmidrule(rl){7-7}
     \cmidrule(rl){8-8}

     Разом & 4 & 24\sfrac{1}{2} & 18\sfrac{1}{2} & & 64\sfrac{3}{4} & 11\sfrac{1}{2} & 40\sfrac{1}{4} \\
  \end{tabular}

  \end{center}
\end{table}

[Це знов не зовсім вірно обчислено. Орендареві $В$ продукція цих 4\sfrac{1}{2} квартерів
коштує, по-перше, 9\sfrac{1}{2}\pound{ ф. стерл.} ціни продукції і, по-друге,
4\sfrac{1}{2}\pound{ ф. стерл.} ренти, разом 14\pound{ ф. стерл.}; пересічно за квартер 3\sfrac{1}{9}\pound{ ф. стерл}.
Ця пересічна ціна всієї його продукції стає через це за реґуляційну ринкову ціну. Тому рента на
$А$ становила б \sfrac{1}{9}\pound{ ф. стерл.} замість \sfrac{1}{2}\pound{ ф. стерл.}, а рента на $В$ лишалася б, як і давніш, 4\sfrac{1}{2}\pound{ ф.
стерл.}: 4\sfrac{1}{2} квартерн по 3\sfrac{1}{2}\pound{ ф. стерл.} \deq{} 14\pound{ ф. стерл.}, звідси вирахувати
9\sfrac{1}{2}\pound{ ф. стерл.} ціни продукції, лишається надзиск в 4\sfrac{1}{2}\pound{ ф. стерл}. Бачимо: не
зважаючи на змінені числа, приклад показує, як з допомогою диференційноі
ренти II краща земля, що вже дає ренту, може стати за регуляційну щодо ціни,
і через це вся земля, також і та, що до того часу не давала ренти, може перетворитися
в рентодайну. — \emph{Ф.~Е.}].

Збіжжева рента мусить підвищитись, скоро підвищується реґуляційна
ціна продукції збіжжя, отже, скоро підвищується ціна продукції квартера збіжжя
на реґуляційній землі, або реґуляційна витрата капіталу на одному з родів
землі. Це все одно, як коли б усі роди землі стали менш плодючі і продукували
б, наприклад, на кожні 2\sfrac{1}{2}\pound{ ф. стерл.} нових витрат капіталу лише по \sfrac{5}{7}
квартера замість 1 квартера. Весь надмір збіжжя, що його вони продукують
\parbreak{}  %% абзац продовжується на наступній сторінці
% ВИДАТИЛИ, не знаю до чого це
% Рід землі    Акри    Ціна продукції    Продукт в кварт. Продажна  ціна    Грошовий  здобуток
% Збіжжева рента    Грошова  рента
%         ф. стер. ф. стер. ф. стер. ф. стер. ф. стер
% А                    1    3             1            31/2       31/2        1/7            1/2
% В                    1    9 1/2       41/2    31/2       153/4        111/14    61/4
% C                    1    6             51/2    31/2        191/2        311/14    131/4
% D                    1    6             71/2    31/2        261/2        511/14     201/4
% Разом           4    241/2    181/2           643/4    111/2            401/4
\parcont{}  %% абзац починається на попередній сторінці
\index{iii2}{0185}  %% посилання на сторінку оригінального видання
з тією самою витратою капіталу, перетворюється на надпродукт, який репрезентує
надзиск, а тому й ренту. Коли припустити, що норма зиску лишається
та сама, то орендар міг би купити на свій зиск меншу кількість збіжжя.
Норма зиску може лишитись та сама, коли заробітна плата не підвищиться,
або тому, що її понижено до фізичного мінімуму, отже, нижче нормальної вартости
робочої сили; або тому, що інші речі споживання робітників, давані мануфактурою,
стали порівняно дешевші; або тому, що робочий день став довший
або зробився інтенсивніший, і тому норма зиску в нехліборобських галузях
продукції, яка проте, реґулює хліборобський зиск, лишилась незмінна, якщо
тільки не підвищилась; абож тому, що хоч у хліборобстві й витрачається такий
самий капітал, але більш сталого і менше змінного.

Ми тут розглянули перший спосіб, у який може постати рента на землі
$А$, що до цього часу була найгірша, без того, щоб притягалось до оброблення
ще гіршу землю; а саме, коли вона постає в наслідок ріжниці індивідуальної
ціни продукції на цій землі, — ціни продукції, що до цього часу була за
реґуляційну проти тієї нової, вищої ціни продукції, по якій останній додатковий
капітал, витрачений з недостатною продуктивною силою на кращій землі,
дав потрібну додаткову кількість продукту.

Коли додаткова продукція мусила б постачатись землею $А_{-1}$, яка може дати
квартер лише за 4\pound{ ф. стерл.}, то рента з акра на $А$ підвищилася б до 1\pound{ ф. стерл}. Але в цьому випадку
земля $А_{-1}$ пересунулася б на місце $А$, як
найгірша з культивованих земель, а земля $А$ вступила б як нижчий член в
ряд родів землі, що дають ренту. Диференційна рента I змінилася б. Отже,
цей випадок лежить поза аналізою диференційної ренти II, яка виникає з різної
продуктивности послідовних витрат капіталу на тій самій дільниці землі.

Але, крім того, диференційна рента на землі $А$, може постати ще двояким
способом:

Коли за незмінної ціни, — будь-якої ціни, хоч би вона і була знижена
проти колишньої, — додаткова витрата капіталу породжує додаткову продуктивність,
що prima facie до певної межі завжди мусить статися якраз на найгіршій
землі.

Подруге, тоді, коли навпаки, продуктивність послідовних витрат капіталу
на землі $А$ понижується.

В обох випадках припускається, що стан попиту потребує збільшення
продукції.

Але, з погляду диференційної ренти, тут виступає специфічна трудність
в зв’язку з раніш викладеним законом, що за ним визначальною для всієї продукції
(або для всієї витрати капіталу) завжди є індивідуальна пересічна ціна
продукції одного квартера. Але для землі $А$, у протилежність кращим родам
землі, ціна продукції, яка обмежує для нових витрат капіталу вирівняння індивідуальної
з загальною ціною продукції, дана не поза нею. Бо індивідуальна ціна
продукції на $А$ і є та сама загальна ціна продукції, що реґулює ринкову ціну.

Припустімо:

1)~За висхідної продуктивної сили послідовних витрат
капіталу на одному акрі землі $А$, з авансованим капіталом в 5\pound{ ф. стерл.},
відповідно 6\pound{ ф. стерл.} ціни продукції, можна випродукувати замість 2 квартерів
3 квартери. Перша витрата капіталу в 2\sfrac{1}{2}\pound{ ф. стерл.} дає 1 квартер, друга — 2 квартери. В цьому
випадку 6\pound{ ф. стерл.} ціни продукції дають 3 квартери,
отже, квартер коштуватиме пересічно 2\pound{ ф. стерл.}; отже, коли 3 квартери
будуть продані по 2\pound{ ф. стерл.}, то $А$, як і давніш, не дасть ренти, але зміниться
лише основа диференційної ренти II; за реґуляційну ціну продукції стали
2\pound{ ф. стерл.} замість 3\pound{ ф. стерл.}; на найгіршій землі капітал в 2\sfrac{1}{2}\pound{ ф. стерл.}
продукує тепер пересічно 1\sfrac{1}{2}  замість 1 квартера, і це тепер офіційна родючість
\parbreak{}  %% абзац продовжується на наступній сторінці

\parcont{}  %% абзац починається на попередній сторінці
\index{iii2}{0186}  %% посилання на сторінку оригінального видання
для всіх кращих земель при витраті в 2\sfrac{1}{2}\pound{ ф. стерл}. Частина їхнього колишнього
надпродукту входить тепер в створення їхнього потрібного продукту, так
само, як частина їхнього колишнього надзиску — в створення пересічного зиску.

Коли, навпаки, обчислити так само, як на кращих землях, де пересічне обчислення
нічого не змінює в абсолютній величині надпродукту, зглядно надзиску, бо для
них межа витрати капіталу дана загальною ціною продукції, то квартер від першої
витрати капіталу коштує 3\pound{ ф. стерл.}, а 2 квартери від кожної другої витрати
лише по 1\sfrac{1}{2}\pound{ ф. стерл}. Отже, на $А$ постала б збіжжева рента в 1 квартер
і грошова рента в 3\pound{ ф. стерл.}, але ці 3 квартери продавалися б по старій
ціні, разом за 9\pound{ ф. стерл}. Коли б сталася третя витрата капіталу в 2\sfrac{1}{2}\pound{ ф.
стерл.} з такою самою продуктивністю, як друга, то були б випродуковані тепер
разом 5 квартерів за 9\pound{ ф. стерл.} ціни продукції. Коли б індивідуальна
пересічна ціна продукції на $А$ лишилась реґуляційною, то квартер довелося б
тепер продавати за 1\sfrac{4}{5}\pound{ ф. стерл}. Пересічна ціна знову понизилася б не в
наслідок нового підвищення продуктивности третьої витрати капіталу, а лише в
наслідок нової додаткової витрати капіталу з такою самою додатковою продуктивностю
як друга. Замість підвищити ренту, як це було б на землях, що дають
ренту, послідовні витрати капіталу вищої, але відносно незмінної продуктивности
на землі $А$, відповідно понизили б ціну продукції, а разом з тим, в інших
рівних умовах, і диференційну ренту на всіх інших родах землі. Навпаки,
коли б перша витрата капіталу, яка продукує 1 квартер за 3\pound{ ф. стерл.} ціни
продукції, залишилася сама по собі міродайною, то 5 квартерів були б продані
за 15\pound{ ф. стерл.}, і диференційна рента від пізніших витрат капіталу на землі
$А$ становила б 6\pound{ ф. стерл}. Приєднання додаткового капіталу до акра землі $А$,
хоч би в якій формі відбулося воно, було б тут поліпшенням, і додатковий
капітал зробив би продуктивнішою і первісну частину капіталу. Було б безглуздям
сказати, що \sfrac{1}{3}  капіталу випродукувала 1 квартер, а інші \sfrac{2}{3} випродукували
4 квартери. 9\pound{ ф. стерл.} на акр завжди продукували б 5 квартерів, тимчасом
як 3\pound{ ф. стерл.} продукували б тільки 1 квартер. Чи постала б тут рента, надзиск,
чи ні, це цілком залежало б від обставин. Нормально, реґуляційна ціна
продукції мусула б понизитись. Так буде в тому випадку, коли це поліпшене,
а тому сполучене із збільшеними витратами оброблення відбувається на землі $А$
лише тому, що воно також відбувається і на кращих родах землі, що, отже,
відбувається загальна революція в хліборобстві; так що тепер, коли мова йде
про природну родючість землі $А$, то припускається, що на неї витрачено 6, зглядно
9\pound{ ф. стерл.} замість 3\pound{ ф. стерл}. Це особливо мало б силу тоді, коли більшість
оброблених акрів землі $А$, які постачають головну масу подання в даній країні,
підпали б цій новій методі. Але коли б поліпшення охопило спочатку лише
невелику частку площі $А$, то ця краще оброблювана частка давала б надзиск,
що його землевласник швидко подбав би перетворити цілком або почасти
на ренту і фіксувати як ренту. Таким чином, коли б попит розвивався рівнобіжно
з ростучим поданням, то поступово, в міру того, як земля $А$ в усій своїй площі
помалу підпадала б під нову методу обробітку, могла б створитися рента
на всій землі якости $А$, і додаткова продуктивність цілком або почасти, залежно
від умов ринку, була б конфіскована. Таким чином, вирівнянню ціни продукції
з землі $А$ у пересічну ціну її продукту, одержуваного з неї при збільшеній витраті
капіталу, могло б перешкодити фіксування надзиску від цієї збільшеної витрати
капіталу у формі ренти. В цьому випадку це знову було б, як ми бачили це
давніш на кращих землях за низхідної продуктивної сили додаткових капіталів,
перетворення надзиску в земельну ренту, тобто втручання земельної власности,
яке підвищило б ціну продукції, замість того, щоб диференційна рента була
просто наслідком ріжниць між індивідуальною і загальною ціною продукції. Це
перешкодило б для землі $А$ збігові обох цін, бо перешкодило б реґулюванню
\parbreak{}  %% абзац продовжується на наступній сторінці

\parcont{}  %% абзац починається на попередній сторінці
\index{iii2}{0187}  %% посилання на сторінку оригінального видання
ціни продукції пересічною ціною продукції з $А$; отже, це тримало б ціну продукції
на вищому рівні, ніж це потрібно, і таким чином створило б ренту.
Навіть при вільному довозі хліба з-за кордону такий результат міг би скластись
або триматись, бо орендарі вимушені були б для землі, яка при зовні
визначеній ціні продукції могла б конкурувати у продукції збіжжя, не даючи
ренти, дати інше призначення, наприклад, призначити її під пасовисько, і таким
чином лише рентодайні землі були б зайняті під збіжжя, тобто лише землі,
на яких індивідуальна пересічна ціна продуктції за квартер була б нижча від
ціни продукції, визначуваної зовні. В цілому можна визнати, що в даному випадку
ціна продукції понизиться, але не до рівня пересічної ціни, і буде стояти
вище від неї, але нижче ціни продукції на гірше оброблюваній землі $А$, так
що конкуренцію нової землі $А$ буде обмежено.

\emph{2) За низхідної продуктивної сили додаткових капіталів.}

Припустімо, що земля $А_{-1}$ могла б випродукувати додатковий квартер
лише за 4\pound{ ф. стерл.}, а земля $А$ за 3\sfrac{3}{4}, отже дешевше ніж $А_{-1}$ але на \sfrac{3}{4}\pound{ ф. стерл.} дорожче, ніж квартер, випродукований першою витратою капіталу на
$А$. В цьому випадку вся ціна двох випродукованих на $А$ квартерів була б \deq{}
6\sfrac{3}{4}\pound{ ф. стерл.}; отже, пересічна ціна за квартер \deq{} 3\sfrac{3}{8}\pound{ ф. стерл}. Ціна продукції
пидвищилася б, але лише на \sfrac{3}{8}\pound{ ф. стерл.}, тимчасом як коли б додатковий
капітал був витрачений на новій землі, яка продукує квартер за 3\sfrac{3}{4}\pound{ ф. стерл.}, вона підвищилася б на дальші \sfrac{3}{8}\pound{ ф. стерл.} до 3\sfrac{3}{4}\pound{ ф. стерл.}, і цим
було б спричинене відповідне підвищення усіх інших диференційних рент.

Ціна продукції в 3\sfrac{3}{8}\pound{ ф. стерл} за квартер на землі $А$ таким чином
вирівнялася б за пересічною ціною продукції на тій самій землі за збільшеної
витрати капіталу і стала б реґуляційною; отже, вона не дала б ренти, бо не
було б надзиску.

Але коли б цей квартер, випродукований другою витратою капіталу, був проданий
за 3\sfrac{3}{4}\pound{ ф. стерл.}, то земля $А$ дала б тепер ренту в \sfrac{3}{4}\pound{ ф. стерл.},
дала б її також і на всі акри $А$, на яких не зроблено додаткової витрати і
які, отже, як і давніш, продукують квартер за 3\pound{ ф. стерл}. Поки існують ще
необроблені дільниці землі $А$, ціна могла б лише тимчасове підвищитись до
3\sfrac{3}{4}\pound{ ф. стерл}. Конкуренція нових дільниць $А$ підтримувала б ціну продукції
на 3\pound{ ф. стерл.}, поки не були б вичерпані всі землі $А$, що їхнє сприятливе положення
дає їм можливість продукувати квартер дешевше, ніж за 3\sfrac{3}{4}\pound{ ф. стерл}.
Отже, доводиться припустити це, хоч власник землі, коли один акр землі дає ренту,
не відступить орендареві другого акра без ренти.

Чи вирівняється ціна продукції відповідно до пересічної ціни, чи за реґуляційну
зробиться індивідуальна ціна продукції другої витрати капіталу в 3\sfrac{3}{4}\pound{ ф. стерл.},
це залежить знов таки від того, більшого чи меншого загального поширення набула
друга витрата капіталу на наявній землі $А$. За реґуляційну ціну стає 3\sfrac{3}{4}\pound{ ф. стерл.} тільки в тому випадку, коли у землевласника є досить часу для того,
щоб фіксувати як ренту той надзиск, який одержувано б при ціні в 3\sfrac{3}{4}\pound{ ф.
стерл.} за квартер, поки не задовольниться попиту.

\pfbreak

Щодо низхідної продуктивности землі за послідовних витрат капіталу,
слід подивитися Лібіха. Ми бачили, що послідовне зменшення додаткової продуктивної
сили витрат капіталу постійно збільшує ренту з акра, коли ціна
продукції не змінюється, і що воно може призвести до цього навіть за низхідної
ціни продукції.

Але взагалі треба відзначити таке:

З погляду капіталістичного способу продукції відносне подорожчання
продукту відбувається завжди, коли для одержання того самого продукту
\parbreak{}  %% абзац продовжується на наступній сторінці

\parcont{}  %% абзац починається на попередній сторінці
\index{iii2}{0188}  %% посилання на сторінку оригінального видання
доводиться зробити певну витрату, коли доводиться оплатити щось, що давніше
не оплачувалося. Бо під покриттям капіталу, зужиткованого в продукції, слід
розуміти покриття тільки вартостей, що являють собою певні засоби продукції,
Елементи природи, що входять як аґенти, у продукцію, без жодних витрат,
хоч би яку ролю вони завжди відігравали в продукції, входять в неї не як складові
частини капіталу, а як дарова природна сила капіталу, тобто як дарова природна
продуктивна сила праці, яка на базі капіталістичного способу продукції
виступає, подібно до всякої продуктивної сили, як продуктивна сила капіталу.
Отже, коли в продукції бере участь така природна сила, яка первісно
нічого не коштує, то вона не входить в розрахунок при визначенні ціни, поки
продукт, виготовлюваний з її допомогою, достатній для задоволення потреби.
Але коли в перебізі розвитку потрібно більше продукту, ніж можна виготувати
з допомогою цієї природної сили, тобто коли доведеться випродукувати
цей додатковий продукт без допомоги цієї природної сили, або з допомогою
людини, людської праці, то в капітал ввійде новий додатковий елемент. Отже,
для одержання колишнього продукту потрібно буде відносно більшої витрати
капіталу. За інших рівних умов відбудеться подорожчання продукції.

\pfbreak

(З зошиту «Початого в половині лютого 1876 року»).

\noindent{}\emph{Диференційна рента і рента як просто процент на капітал,
долучений до землі.}

\noindent{}Так звані сталі меліорації, — які змінюють фізичні, почасти й хемічні
властивості ґрунту через операції, що коштують витрат капіталу, і що можуть
розглядатися, як долучення капіталу до землі, — майже всі сходять на те,
щоб певній дільниці землі, ґрунтові в певному обмеженому місці надати таких
властивостей, що їх інший ґрунт в другому місці, часто зовсім близько, має
з природи. Одна земля нівельована з природи, іншу ще доводиться нівелювати:
одна має природні водозбіги, інша потребує штучного дренажу; одна з природи
має глибокий орний шар, на іншій його треба поглибити штучно; один глинястий
ґрунт з природи змішаний з належною кількістю піску, у іншого ще
треба штучно створити це відношення, одна лука зрошується з природи, або
вкривається шаром намулу, на іншій цього доводиться досягати працею, або,
кажучи мовою буржуазної економії, капіталом.

Справді чудна теорія, за якою тут на одній землі, що її відносні вигоди
придбані, рента є процент, а на іншій землі, яка має ці вигоди з природи, не
є процент. (На ділі, при застосуванні цієї теорії, справа перекручується так, що в
наслідок того, що в одному випадку рента дійсно збігається з процентом, її і в інших
випадках, коли фактично цього немає, мусять називати процентом, перебріхують
в процент). Але після того, як зроблено витрату капіталу, земля дає ренту не тому,
що в неї вкладено капітал, а тому, що витрата капіталу зробила землю продуктивнішою
ділянкою приміщення капіталу проти давнішого. Припустімо,
що вся земля певної країни потребує такої витрати капіталу; в такому разі
кожна дільниця землі, на якій вона ще не була зроблена, муситиме пройти цю
стадію і рента (процент, який в даному випадку дає земля), що її дає земля, на
якій вже було зроблено таку витрату капіталу, так само становить диференційну
ренту, як коли б ця земля з природи мала цю перевагу, а інша земля
мусила б її набувати лише штучним шляхом.

І ця зводжувана до проценту рента стає чистою диференційною рентою,
скоро витрачений капітал буде амортизовано. Інакше той самий капітал як
капітал мусив би існувати подвійно.

\pfbreak


\index{iii2}{0189}  %% посилання на сторінку оригінального видання
Одно з найкумедніших явищ є в тому, що всі противники Рікардо, які
заперечують визначення вартости виключно працею, в справі з диференційною
рентою, що випливає з ріжниць землі, надають ваги тому, що тут вартість
визначається природою, а не працею; і одночасно приписують це визначення
положенню, або, і ще більше, процентові на капітал, вкладений в землю при
обробітку. Та сама праця дає однакову вартість для продукту, створеного
протягом даного часу; але величина або кількість цього продукту, отже, і та
частина вартости, яка припадає на відповідну частину цього продукту за даної
кількости праці, залежить єдино від кількости продукту, а це знову від продуктивности
даної кількости праці, не від величини цієї кількости. Чи завдячує
ця продуктивність своїм походженням природі, чи суспільству — цілком байдуже.
Тільки в тому разі, коли вона сама коштує праці, отже, капіталу, вона
збільшує ціну продукції новою складовою частиною, чого природа сама по собі
не робить.

\section{Абсолютна земельна рента}

Аналізуючи диференційну ренту, ми виходили з припущення, що найгірша
земля не виплачує земельної ренти, або, висловлюючись загальніше, що земельну
ренту виплачує тільки така земля, для продукту якої індивідуальна ціна продукції
нижча від ціни продукції, що реґулює ринок, так що в такий спосіб
виникає надзиск, що перетворюється на ренту. Потрібно насамперед зауважити,
що закон диференційної ренти, як диференційної ренти, зовсім не залежить від
правильности чи неправильности того припущення.

Коли загальну ціну продукції, що реґулює ринок, ми назвемо $Р$, то $Р$ для
продукту найгіршого роду землі $А$ збігається з індивідуальною ціною продукції
на цій землі; тобто вона оплачує зужиткований у продукції сталий і змінний капітал
плюс пересічній зиск (= підприємницькому баришеві плюс процент).

Рента тут дорівнює нулеві. Індивідуальна ціна продукції найближчого
кращого роду землі $В \deq{} Р'$, і $Р>Р'$; тобто $Р$ оплачує більше, ніж дійсну
ціну продукції продукту на клясі землі $В$. Хай тепер $Р — Р' \deq{} d$; тому
$d$, надмір $Р$ над $Р'$, є той надзиск, що його добуває орендар з цієї кляси $В$.
Це $d$ перетворюється на ренту, яку доводиться виплачувати власникові землі.
Хай для третьої кляси землі $C$ за дійсну ціну продукції буде $Р''$, і хай
$Р - Р'' \deq{} 2d$; отже, ці $2d$ перетворюються на ренту; так само для четвертої кляси
$D$ індивідуальна ціна продукції хай буде $Р'''$, а $Р - Р''' \deq{} 3d$, які перетворюються
на земельну ренту і~\abbr{т. д.} Даймо тепер, що припущення, ніби для
кляси землі $А$ рента $= 0$, а тому ціна її продукту $= Р \dplus{} 0$, помилкове. Хай,
навпаки, і вона дає ренту $= r$. В цьому випадку маємо двоякі наслідки.

\emph{Поперше}: ціна продукту землі кляси $А$ не реґулювалася б ціною продукції
на цій землі, а мала б деякий надмір над цією ціною, вона була б
$= P - r$. Бо, коли припускається, нормальний перебіг капіталістичного способу
продукції, отже, коли припускається, що надмір $r$, виплачуваний від орендаря
земельному власникові, не становить вирахування ані з заробітної плати, ані
з пересічного зиску на капітал, то орендар може виплачувати його лише тому,
що його продукт продається понад ціну продукції, що він, отже, дав би йому
надзиск, коли б не доводилося відступати цей надмір у формі ренти земельному
власникові. Реґуляційна ринкова ціна всього наявного на ринку продукту
всіх родів землі була б тоді не та ціна продукції, яку дає капітал взагалі
у всіх сферах продукції, тобто не ціна рівна витратам плюс пересічний
зиск, а була б ціною продукції плюс рента, $Р \dplus{} r$, не $Р$. Бо ціна продукту
кляси $А$ визначає взагалі межу реґуляційної загальної ринкової ціни, тієї ціни,
\parbreak{}  %% абзац продовжується на наступній сторінці

\parcont{}  %% абзац починається на попередній сторінці
\index{iii2}{0190}  %% посилання на сторінку оригінального видання
по якій може бути приставлено ввесь продукт, і в цьому розумінні вона реґулює
ціну цього всього продукту.

Проте, \emph{подруге}, хоч у цьому випадку загальна ціна продукту землі
істотно модифікувалася б, цим зовсім не був би скасований закон диференційної
ренти. Бо коли ціна продукту кляси $А$, а разом з тим і загальна ринкова
ціна $= Р \dplus{} r$, то ціна кляс $B$, $C$, $D$ і~\abbr{т. ін.} теж була б
$= P \dplus{} r$. А що
$Р - Р'$ для кляси $B$ $= d$, то $(Р \dplus{} r) - (Р' \dplus{} r)$ теж було б $= d$, а для $C$
$P - Р'' \deq{} (Р \dplus{} r) - (Р'' \dplus{} r)$ було б $= 2d$, як для $D$, нарешті,
$Р - Р''' \deq{} (Р \dplus{} r) - (Р''' \dplus{} r) \deq{} 3d$ і~\abbr{т. д.} Отже, диференційна рента лишилася б та сама, що
давніш і реґулювалася б тим самим законом, хоч рента мала б у собі елемент незалежний
від цього закону, і хоч вона взагалі підвищилась би одночасно з ціною продукту
землі. Звідси випливає: хоч би як завжди стояла справа з рентою з найменш
родючих родів землі, закон диференційної ренти від цього не тільки не залежить,
але навіть єдиний спосіб зрозуміти саму диференційну ренту відповідно до її
характеру є в тому, що рента кляси землі $А$ припускається рівною нулеві. Чи
вона дійсно $= 0$, чи $> 0$, це байдуже, оскільки справа йде про диференційну
ренту, і насправді не береться на увагу.

Отже, закон диференційної ренти, не залежить від наслідку дальшого
дослідження.

Тепер, коли ставити далі питання про підставу того припущення, щопродукт
землі найгіршого роду $А$ не дає ренти, то відповідь неминуче така:
коли ринкова ціна продукту землі, скажімо, збіжжя, досягає такої висоти, що
додатково авансований капітал, укладений в землю кляси $А$, оплачує звичайну
ціну продукції, тобто дає капіталові звичайний пересічний зиск, то цієї умови
досить для приміщення додаткового капіталу в землю кляси $А$. Тобто, капітаталістові
досить цієї умови для того, щоб укладати новий капітал з звичайним
зиском і використовувати його нормальним способом.

Тут слід зауважити, що і в цьому випадку ринкова ціна мусить стояти
вище, ніж ціна продукції на $А$. Бо скоро створюється додаткове подання, відношення
попиту і подання очевидно зміниться. Давніш подання було недостатнє,
тепер воно достатнє. Отже, ціна мусить понизитись. Але для того, щоб вона могла
понизитись, вона давніш мусила стояти на вищому рівні, ніж ціна продукції на $А$.
Але те, що кляса $А$, яка наново вступає в обробіток, менш родюча, призводить до
того, що ціна не впаде знову до такого низького рівня, як в той час, коли ринок
реґулювала ціна продукції кляси $В$. Ціна продукції на $А$ становить межу не для
тимчасового, а для відносно перманентного підвищення ринкової ціни. — Навпаки,
коли новооброблювана земля родючіша, ніж кляса $А$, яка до того часу була за
реґуляційну, і проте, її досить лише для покриття додаткового попиту, то ринкова
ціна залишається без зміни. Але дослідження того, чи дає ренту нижча
кляса землі, і в цьому випадку збігається з тим, яким ми зайняті тепер, бо
і тут припущення, що кляса землі $А$ не дає ренти, з’ясовувалося б тим, що
капіталістичному орендареві досить ринкової ціни, щоб нею точно покрити
зужиткований капітал плюс пересічний зиск; коротко кажучи, тим, що ринкова
ціна дає йому ціну продукції його товару.

В усякому разі капіталістичний орендар, може за цих відношень обробляти
землю кляси $А$, оскільки він вирішує справи як капіталіст. Умова для
нормального збільшення вартости капіталу на землі роду $А$ є тепер в наявності.
Але з тієї передумови, що орендар міг би вкладати тепер капітал у землі
роду $А$, за умов відповідних пересічним відношенням зростання вартости капіталу,
хоч він і не мав би можливости платити ренту, — зовсім не випливає
висновок, що ця земля, належна до кляси $А$, так і буде без дальших околичностей
віддана орендареві. Та обставина, що орендар міг би використати свій
капітал з звичайним зиском, коли йому не доводиться платити ренти, для
\parbreak{}  %% абзац продовжується на наступній сторінці

\parcont{}  %% абзац починається на попередній сторінці
\index{ii}{0191}  %% посилання на сторінку оригінального видання
першого, ці функції за першого періоду обороту точно відмежовані одна
від однієї, або принаймні їх можна точно відмежувати, тимчасом як протягом
другого періоду обороту вони, навпаки, переплітаються одна з
однією.

Уявімо собі справу наочніше:

Перший період обороту триває 12 тижнів. Перший робочий період —
9 тижнів; оборот авансованого на нього капіталу закінчується на початку
13-го тижня. Протягом останніх 3 тижнів функціонує додатковий капітал
в 300\pound{ ф. стерл.}, який починає другий дев’ятитижневий робочий
період.

Другий період обороту. На початку 13-го тижня 900\pound{ ф. стерл.} припливають
назад і можуть почати новий оборот. Але другий робочий
період уже на десятому тижні почато за допомогою додаткових 300\pound{ ф.
стерл.}; на початку 13-го тижня за допомогою тих самих 300\pound{ ф. стерл.}
уже закінчено третину робочого періоду, 300\pound{ ф. стерл.} з продуктивного
капіталу перетворено на продукт. А що для закінчення другого робочого
періоду треба ще лише 6 тижнів, то в процес продукції другого робочого
періоду можуть ввійти лише дві третини капіталу в 900\pound{ ф. стерл.},
який повернувся назад, а саме 600\pound{ ф. стерл}. З первісних 900\pound{ ф. стерл.}
звільнилося 300\pound{ ф. стерл.}, щоб відігравати ту саму ролю, яку відігравав
у першому робочому періоді додатковий капітал в 300\pound{ ф. стерл}. Наприкінці
6-го тижня другого періоду обороту закінчено другий робочий
період. Витрачений на нього капітал в 900\pound{ ф. стерл.} повертається за три
тижні, отже, наприкінці 9-го тижня другого дванадцятитижневого періоду
обороту. Протягом 3 тижнів його часу обігу ввіходить у робочий період
звільнений капітал в 300\pound{ ф. стерл}. З ним починається на 7-й тиждень
другого періоду обороту або на 19-й тиждень року третій робочий
період капіталу в 900\pound{ ф. стерл}.

Третій період обороту. Наприкінці 9-го тижня другого періоду обороту
знову зворотно припливають 900\pound{ ф. стерл}. Але третій робочий
період почався вже на сьомому тижні попереднього періоду обороту й
6 тижнів його вже минуло. Тому він триває тільки три тижні. Отже,
з 900\pound{ ф. стерл.}, що повернулись назад, у процес продукції ввіходять
лише 300\pound{ ф. стерл}. Четвертий робочий період заповнює дев’ятитижневу
решту цього періоду обороту, і таким чином з 37-го тижня року починається
одночасно четвертий період обороту й п’ятий робочий період.

Щоб спростити обчислення, ми припустимо робочий період в 5 тижнів,
час обігу в 5 тижнів, отже, період обороту в 10 тижнів; рік рахуватимемо
в 50 тижнів, а щотижневу витрату капіталу рахуватимемо в 100\pound{ ф.
стерл}. Отже, робочий період потребує поточного капіталу в 500\pound{ ф. стерл.},
а час обігу потребує додаткового капіталу — нових 500\pound{ ф. стерл}. Робочі
періоди й періоди оборотів позначиться тоді так:

\noindent{}
{\settablefont{}1-й робочий період: тижні 1\textendash{}5 (500\pound{ ф. стерл.} товару повертаються
наприкінці 10 тижня).}

\noindent{}
{\settablefont{}2-й робочий період: тижні 6--10 (500\pound{ ф. стерл.} товару повертаються
наприкінці 15 тижня).}

\noindent{}
{\settablefont{}3-й робочий період: тижні 11--15 (500\pound{ ф. стерл.} товару повертаються
наприкінці 20 тижня).}

\noindent{}
{\settablefont{}4-й робочий період: тижні 16--20 (500\pound{ ф. стерл.} товару повертаються
наприкінці 25 тижня).}

\noindent{}
{\settablefont{}5-й робочий період: тижні 21--25 (500\pound{ ф. стерл.} товару повертаються
наприкінці 30 тижня)

\noindent{}і~\abbr{т. д.}}
\parcont{}  %% абзац починається на попередній сторінці
\index{iii2}{0192}  %% посилання на сторінку оригінального видання
капіталу відпадає, і саме через договір з самим земельним власником. Але він не платить
ренти за ці дільниці тільки тому, що він платить ренту за землю, до
якої вони належать. Тут припускається якраз така комбінація, коли доводиться
звернуть до гіршого роду землі $А$ не як до самостійного нового поля продукції,
яке покрило б недостатнє подання, а як до такого, що становить лише
неподільну смугу в кращій землі. А випадок, який ми маємо дослідити, є якраз той,
коли доводиться самостійно провадити господарство на дільницях землі роду $А$,
отже, коли вони мусять за наявности загальних передумов капіталістичного способу
продукції здаватися в оренду як самостійні дільниці.

\emph{Третє}: Орендар може вкласти додатковий капітал у ту саму орендовану
дільницю, хоч за сущих ринкових цін одержувана в такий спосіб додаткова
продукція дає йому лише ціну продукції, звичайний зиск, але не дає йому
змоги платити додаткову ренту. Таким чином, на одну частину капіталу, вкладеного
в землю, він виплачує земельну ренту, на другу — ні. Як мало це припущення
розв’язує проблему, видно ось з чого: коли ринкова ціна (і разом
з цим родючість землі) дає йому можливість на додатковий капітал одержувати
додатковий здобуток, який подібно до старого капіталу дає йому, крім ціни продукції,
певний надзиск, то він бо скінчення терміну орендного договору залишає
його в себе. Але чому? Тому, що поки триває термін орендного договору, відпадає
та межа для примінення його капіталу у землю, яку створює земельна
власність. Проте, та звичайна обставина, що для забезпечення йому цього надзиску
мусить розпочатися самостійний обробіток додаткової гіршої землі і її самостійне
заорендування, незаперечно доводить, що приміщення додаткового капіталу
у стару землю не досить для створення потрібного підвищеного подання.
Одно припущення виключає друге. Правда, тепер можна було б сказані: сама
рента з найгіршого роду землі $А$ є диференційна рента, чи то порівняно з землею,
яка обробляється самим власником (проте, це трапляється у чистому вигляді
лише як випадковий виняток), чи то порівняно з додатковим приміщенням
капіталу на тих старих заорендованих дільницях землі, що не дають ренти.
Але це була б 1)~така диференційна рента, що виникала б не з ріжниці родючости
різних родів землі, а тому не мала б за свою передумову того, що земля
роду $А$ не дає ренти, і що продукти її продається по ціні продукції; і 2)~та обставина,
чи дають ренту додаткові приміщення капіталу на тій самій заорендованій
дільниці, чи ні, цілком також байдужа щодо того, чи виплачує ренту новооброблювана
земля кляси $А$, чи ні, так само як наприклад, для заснування нового самостійного
фабричного підприємства байдуже, чи вкладе інший фабрикант тієї
самої галузі підприємств у процентні папери частину свого капіталу, не бувши в
стані її цілком використати у своєму підприємстві, чи він зробить ряд таких окремих
розширень, що не дають йому повного зиску, а проте дають більше за процент.
Це для нього справа другорядна. Навпаки, додаткові нові підприємства мусять
давати пересічний зиск і споруджуються в надії на пересічний зиск. В усякому
разі, додаткові приміщення капіталу на старих заорендованих дільницях
землі і додаткове оброблення нової землі роду $А$ становлять межі одне для одного.
Межа, до якої в ту саму заорендовану дільницю може вкладатись додатковий
капітал за менш сприятливих умов продукції, визначається конкурентними
новими приміщеннями у землю кляси $А$; з другого боку, рента, яку
може давати земля цієї кляси, обмежується конкурентними додатковими приміщеннями
капіталу на старих заорендованих землях.

Проте, всі ці фалшиві викрути не розв’язують проблеми, яка в простій
поставі така: припустімо, що ринкова ціна збіжжя (яке в цьому дослідженні
є для нас за представника всякого продукту землі) достатня для того, щоб
можна було почати оброблення частин землі кляси $А$, і щоб капітал, вкладений
у ці нові лани, здобув ціну продукції продукту, тобто покриття капіталу плюс
\parbreak{}  %% абзац продовжується на наступній сторінці

\parcont{}  %% абзац починається на попередній сторінці
\index{iii2}{0193}  %% посилання на сторінку оригінального видання
пересічний зиск. Отже, припустімо, що в наявності є умови для нормального
використування капіталу на землі кляси $А$. Чи досить цього? Чи дійсно можна
буде тоді вкласти цей капітал? Чи, може, ринкова ціна мусить підвищитись
до такого ступеня, щоб ренту давала й найгірша земля $А$? Чи покладає, отже,
монополія земельного власника приміщенню капіталу таку межу, якої не було б
з суто-капіталістичного погляду без існування цієї монополії? Вже з умов поставленого
питання випливає, що коли, наприклад, на старих заорендованих
дільницях вкладено додаткові капітали, які при даній ринковій ціні не дають
жодної ренти, а дають лише пересічний зиск, то ця обставина ніяк не розв'язує
того питання, чи можна тепер в дійсності вкласти капітал у землю кляси $А$,
яка теж стала б давати пересічний зиск, але жодної ренти. В цьому якраз і є
питання. Що додаткові вкладання капіталу, які не дають ренти, не задовольняють
попиту, — це доводиться доконечністю притягнення нової землі кляси $А$. Якщо
додаткове оброблення землі $А$ відбувається тільки тоді, коли вона дає ренту,
отже, більше, ніж ціну продукції, то можливі тільки два випадки. Або ринкова
ціна мусить стояти на такому рівні, щоб навіть останні додаткові вкладання
капіталу на старих заорендованих дільницях давали надзиск, чи потрапляє він
в кишеню орендаря, чи власника. Це підвищення ціни і цей надзиск від останніх
додаткових вкладень капіталу були б тоді наслідком того, що земля $А$ не може
бути оброблювана, коли вона не дає ренти. Бо якби для оброблення було б
досить ціни продукції, тобто одержувати просто пересічний зиск, то ціна не
підвищилася б до такої міри, і конкуренція нових дільниць землі вже почалася
б, скоро вони стали б давати тільки ці ціни продукції. З додатковими
приміщеннями капіталу на старих заорендованих дільницях, що не дають ренти,
тоді почали б конкурувати приміщення капіталу на землі $А$, що так само не
дають ренти. — Абож останні приміщення капіталу на старих заорендованих
дільницях не дають ренти, але ринкова ціна піднеслась однак досить високо
для того, щоб земля $А$ почала оброблятися і давати ренту. В цьому випадку
додаткове вкладання капіталу, що не дає ренти, було можливе лише тому, що
земля $А$ не може оброблятись, поки ринкова ціна не дозволить їй давати ренту.
Без цієї умови культура її почалася б уже при нижчому рівні ціни; і ті пізніші
вкладання капіталу на старих заорендованих дільницях, які для того,
щоб давати звичайний зиск без ренти, потребують високої ринкової ціни, не
могли б статись. Адже і при високій ринковій ціні вони дають лише пересічний
зиск Отже при нижчій ціні, яка при культурі землі $А$ стала б реґуляційною,
як ціна продукції на ній, вони не давали б цього зиску, отже, при цьому
припущенні вони взагалі не могли б статись. Щоправда, рента з землі
$А$ була б таким чином диференційною рентою порівняно з цими приміщеннями
капіталу на старих заорендованих дільницях, що не дають ренти. Але
що дільниці землі $А$ створюють таку диференційну ренту, це є лише наслідок
того, що вони взагалі неприступні для оброблення, хіба тільки тоді, коли
даватимуть ренту; отже, наслідок того, що виникає доконечність цієї ренти, яка
сама по собі не зумовлюється хоч би якою ріжницею між родами землі, і яка
створює межу для можливого приміщення додаткових капіталів на старих заорендованих
дільницях. В обох випадках рента з землі $А$ була б не звичайним
наслідком підвищення ціни збіжжя, а навпаки: та обставина, що найгірша
земля мусить давати ренту для того, щоб її взагалі дозволили обробляти, була б
за причину підвищення ціни збіжжя до такого пункту, на якому постане змога
здійснити цю умову.

\looseness=1
Диференційна рента має ту особливість, що земельна власність тут лише
уловлює той надзиск, що його інакше захопив би орендар, і за певних обставин,
поки не скінчиться термін його орендного договору, дійсно захоплює. Земельна
власність є тут лише за причину перенесення певної, виниклої без її допомоги
\parbreak{}  %% абзац продовжується на наступній сторінці

\parcont{}  %% абзац починається на попередній сторінці
\index{iii2}{0194}  %% посилання на сторінку оригінального видання
(радше, в наслідок визначення конкуренцією ціни продукції, яка регулює ринкову
ціну) частини ціни товару, яка зводиться до надзиску, — за при чину перенесення
цієї частини ціни від однієї особи до іншої, від капіталіста до
земельного власника. Але земельна власність тут не є причина, яка \emph{створює}
цю складову частину ціни, або те підвищення ціни, яке є передумовою цієї
частини ціни. Навпаки, коли найгірша земля кляси $А$ не може оброблятись, —
хоч оброблення її дало б ціну продукції, — поки вона не дає надміру над цією
ціною продукції, ренти, то земельна власність є творчою основою \emph{цього} підвищення
ціни. \emph{Сама земельна власність створила ренту}. Це анітрохи не
зміниться від того, що, як у другому розгляненому випадку, рента, виплачувана
тепер з землі $А$, становить диференційну ренту порівняно з тими останніми
додатковими приміщеннями капіталу на старих заорендованих дільницях, які
виплачують лише ціну продукції. Бо та обставина, що оброблення землі $А$
не може початися, поки регуляційна ринкова ціна не підійметься остільки високо,
що земля $А$ зможе давати ренту, — тільки ця обставина є тут причиною
того, що ринкова ціна підвищується до такого пункту, на якому вона для
останніх приміщень капіталу на старих заорендованих дільницях виплачує,
правда, лише їхню ціну продукції, але таку ціну продукції, яка одночасно
дає ренту для землі $А$. Та обставина, що остання взагалі мусить виплачувати
ренту, є тут причиною створення диференційної ренти між землею $А$ і останніми
приміщеннями капіталу на старих заорендованих дільницях.

Коли ми взагалі кажемо, що — припускаючи реґулювання збіжжевої ціни
ціною продукції — земля кляси $А$ не виплачує ренти, то ми маємо на увазі
ренту в категоричному значінні слова. Коли орендар виплачує орендну плату,
яка становить вирахування або з нормальної заробітної плати його робітників,
або з його власного нормального пересічного зиску, то він не виплачує жодної
ренти, жодної самостійної складової частини ціни його товару, яка відрізнялася б
від заробітної плати і зиску. Вже давніш ми відзначали, що на практиці це
завжди трапляється. Коли заробітну плату хліборобських робітників у певній
країні взагалі знижують поза нормальний пересічний рівень заробітної плати,
і тому вирахування з заробітної плати, частина заробітної плати, входить, як
загальне правило, до складу ренти, то це не становить жодного винятку для
орендаря найгіршої землі. В тій самій ціні продукції, яка уможливлює оброблення
найгіршої землі, вже ураховується, як складова стаття, ця низька заробітна
плата, і тому продаж продукту по ціні продукції не дає змоги орендареві
цієї землі виплачувати ренту. Земельний власник може також здати свою землю
в оренду робітникові, який буде готовий усе те, або більшу частину того, що
продажна ціна залишає йому поверх заробітної плати, виплатити у формі
ренти другій особі. Проте, в усіх цих випадках зовсім не виплачується дійсної
ренти, хоч виплачується орендна плата. Але там, де існують відносини, відповідні
капіталістичному способові продукції, рента і орендна плата мусили б
збігатися. Отут ми й повинні дослідити якраз це нормальне відношення.

Якщо навіть розглянуті вище випадки, коли за капіталістичного способу
продукції дійсно можуть вкладатися у землю капітали, не даючи при цьому
ренти, — якщо навіть ці випадки нічого не дають для розв’язання нашої проблеми,
то ще значно менше дасть посилання на колоніяльні відносини. Що робить
колонію колонією, — ми говоримо тут лише про власне хліборобські колонії,
— так це не тільки маса родючих земель, що перебувають у природному
стані. Ні, колоніями робить їх радше та обставина, що ці землі не привласнені,
не підлеглі земельній власності. Саме це і зумовлює таку колосальну ріжницю
між старими землями і колоніями, оскільки справа йде про землю:
юридична або фактична відсутність земельної власности, як слушно відзначив
\parbreak{}  %% абзац продовжується на наступній сторінці

\parcont{}  %% абзац починається на попередній сторінці
\index{iii2}{0195}  %% посилання на сторінку оригінального видання
Wakefield\footnote{
Wakefeld: England and America. London 1833. Порів, також книгу 1, розд. XXV.
} і вже задовго до нього відкрили фізіократ Мірабо-батько та інші
старі економісти. Тут цілком байдуже, чи привласнюють колоністи собі землю
просто, чи вони під виглядом номінальної ціни землі в дійсності виплачують державі
лише податок за правний юридичний титул на землю. Байдуже також і те, що вже
осілі колоністи є юридичні власники землі. Фактично земельна власність не
становить тут межі для приміщення капіталу або також праці без капіталу;
захоплення частини землі вже осілими колоністами не виключає для нових
прихідців можливости зробити нову землю сферою приміщення їхнього капіталу
або їхньої праці. Тому тоді, коли доводиться досліджувати як земельна власність
впливає на ціни продуктів землі і на ренту там, де ця власність обмежує
землю як сферу приміщення капіталу, було б в найбільшій мірі недоладним
посилатися на вільні буржуазні колонії, де немає ані капіталістичного способу
продукції в хліборобстві, ані відповідної йому форми земельної власности, і де
остання взагалі фактично не існує. Так робить, наприклад, Рікардо, в розділі
про земельну ренту. Спочатку він говорить, що хоче дослідити вплив привласнення
землі на вартість продуктів землі і безпосередньо після цього бере, як
ілюстрацію, колонії, при чому припускає, що земля існує там в порівняно
первісних умовах і експлуатація її не обмежується монополією земельної
власности.

Сама юридична власність на землю не створює земельної ренти для власника.
Але дає йому, певно, силу усувати свою землю від експлуатації доти,
доки економічні відносини дозволять таке використання її, яке дасть йому
певний надмір, при чому байдуже, чи застосовуватиметься землю для власне
хліборобства, чи для інших продукційних цілей, як будівлі тощо. Він не може
збільшити або зменшити абсолютного розміру цієї сфери підприємств, але, певна
річ, може зробити це щодо тієї кількости її, яка перебуває на ринку. Звідси,
як відзначив уже Фур’є, той характеристичний факт, що в усіх цивілізованих
країнах порівняно значна частина землі завжди усунена від оброблення.

Отже, припускаючи такий випадок, що попит потребує обробітку нових
земель, скажімо, менш родючих, ніж оброблювані до того часу, то чи стане
тоді земельний власник здавати ці землі в оренду даром, тому що ринкова
ціна продукту землі піднеслась досить високо, так що приміщення капіталу
в ту землю дає орендареві ціну продукції, а тому й звичайний зиск? Ні в якому
разі. Вкладання капіталу мусить дати йому ренту. Він здає в оренду лише тоді,
коли йому може бути виплачена орендна плата. Отже, щоб можна було виплачувати
земельному власникові ренту, ринкова ціна мусить піднестись вище
ціни продукції, до $Р \dplus{} r$. А що, згідно з припущенням, земельна власність
без здачі в оренду нічого не дає, економічно є безвартісна, то невеликого підвищення
ринкової ціни над ціною продукції досить для того, щоб дати на ринок
нову землю найгіршого роду.

Тепер постає таке питання: чи випливає з земельної ренти з найгіршої
землі, ренти, яка не може бути виведена з ріжниці родючости, те, що ціна
продукту землі неминуче є монопольною ціною в звичайному значінні, або
ціною, до складу якої рента входить в такій самій формі, як податок, з тією
тільки ріжницею, що цей податок стягає земельний власник замість держави?
Що такий податок має свої певні економічні межі, це зрозуміло само собою.
Він обмежується додатковими приміщеннями капіталу на старих заорендованих
дільницях, конкуренцією закордонних продуктів землі — припускаючи вільний
довіз їх — конкуренцією земельних власників між собою, нарешті, потребою
і платоспроможністю споживачів. Але тут мова йде не про те. Мова йде про
те, чи входить рента, виплачувана найгіршою землею, в ціну її продукту, яка
\parbreak{}  %% абзац продовжується на наступній сторінці

\parcont{}  %% абзац починається на попередній сторінці
\index{iii1}{0196}  %% посилання на сторінку оригінального видання
під „попитом“ і „природною ціною“ те, що ми досі розуміли під
цим, покликаючись на А.~Сміта, завжди мусить бути відношенням
рівності, бо тільки тоді, коли подання дорівнює дійсному
попитові, тобто попитові, який не хоче платити ні більше,
ні менше природної ціни, — тільки тоді дійсно сплачується природна
ціна; отже, в різний час той самий товар може мати дві
дуже різні природні ціни, і все ж відношення між поданням
і попитом, в обох випадках може бути однаковим, а саме
відношенням рівності“.] Отже, тут допускається, що при двох
різних natural prices [природних цінах] одного й того самого
товару в різний час попит і подання кожного разу можуть взаємно
покриватись і мусять покриватись для того, щоб товар
в обох випадках був проданий по його natural price. Але через
те що в обох випадках немає ніякої ріжниці у відношенні між
попитом і поданням, але є ріжниця у величині самої natural
price, то ця остання, очевидно, визначається незалежно від попиту
й подання і, отже, менш за все може бути ними визначена.

Для того, щоб товар продавався по його ринковій вартості,
тобто пропорціонально до вміщеної в ньому суспільно-необхідної
праці, сукупна кількість суспільної праці, вживана для
виробництва сукупної маси цього роду товарів, мусить відповідати
величині суспільної потреби в цих товарах, тобто платоспроможної
суспільної потреби. Конкуренція, коливання ринкових
цін, які відповідають коливанням відношення між попитом
і поданням, постійно намагаються звести до цієї міри сукупну
кількість праці, вжитої на кожний рід товарів.

У відношенні між попитом і поданням товарів повторюється,
поперше, відношення між споживною вартістю і міновою вартістю,
між товаром і грішми, між покупцем і продавцем; подруге,
відношення між виробником і споживачем, хоч обидва
вони можуть бути представлені третіми особами, торговцями.
При дослідженні відношення між покупцем і продавцем досить
протиставити їх, кожного окремо, один одному, щоб розвинути
це відношення. Трьох осіб досить для повної метаморфози
товару і, отже, для процесу продажу-купівлі, взятого в цілому.
$А$ перетворює свій товар у гроші $В$, якому він продає товар,
і знову перетворює свої гроші в товар, який він купує на ці
гроші в $C$; весь процес відбувається між ними трьома. Далі:
при дослідженні грошей ми припускали, що товари продаються
по їх вартості, бо не було ніякої підстави розглядати ціни, що
відхиляються від вартості, оскільки йшлося тільки про ті зміни
форми, які пророблює товар, стаючи грішми і знову перетворюючись
з грошей у товар. Раз товар взагалі продається і на
виручені гроші купується новий товар, то ми маємо перед
собою цілу метаморфозу, і для неї як такої однаково, чи стоїть
ціна товару нижче чи вище його вартості. Вартість товару зберігає
своє значення як основа, бо тільки з цієї основи можуть бути
раціонально виведені гроші, і ціна за своїм загальним поняттям
\parbreak{}  %% абзац продовжується на наступній сторінці

\parcont{}  %% абзац починається на попередній сторінці
\index{iii2}{0197}  %% посилання на сторінку оригінального видання
значну ролю. Однак, в гірничій промисловості друга частина сталого капіталу, основний капітал,
відіграє значну ролю. Проте, і тут перебіг розвитку може вимірятися відносним зростом сталого
капіталу порівняно з змінним.

Коли склад капіталу у власне хліборобстві нижчий, ніж склад пересічного
суспільного капіталу, то це prima facie було б виразом того, що в країнах
розвиненої продукції хліборобство не досягло такого ступеня розвитку, як обробна промисловість.
Такий факт, залишаючи осторонь всі інші і до того ж почасти вирішені економічні обставини, мав би
для себе пояснення вже в давнішому і швидшому розвитку механічних наук і особливо в їхньому
застосуванні порівняно з пізнішим і почасти зовсім недавнім розвитком хімії, геології, фізіології і
особливо знов таки в їхньому застосуванні до хліборобства. Проте, це безперечний і давно відомий\footnote{Див. Dombasle і R.Jones}
факт, що прогрес самого хліборобства постійно визначається у відносному зрості сталої частини
капіталу проти змінної. Чи в певній країні капіталістичної продукції, наприклад, в Англії, склад
хліборобського капіталу є нижчий, ніж склад пересічного суспільного капіталу, — це питання, що його
можна розв’язати лише статистично, і яке детально розглядати було б зайвим з огляду на нашу мету. В
усякому разі теоретично усталено, що тільки при цьому припущенні вартість хліборобських продуктів
може бути вища від їхньої ціни продукції; тобто, що додаткова вартість, породжувана в хліборобстві
капіталом певної величини, або, що сходить на те саме, додаткова праця (отже, і вжита жива праця
взагалі), пущена ним в рух і упідлеглена йому, більша, ніж при рівновеликому капіталі пересічного
суспільного складу.

Отже, для форми ренти, що її ми досліджуємо тут, і яка може постати лише
при цьому припущенні, досить, коли ми зробимо це припущення. Де ця гіпотеза
відпадає, там відпадає і відповідна їй форма ренти.

Проте, простий факт надміру вартости хліборобських продуктів над їхньою ціною продукції сам по
собі ні в якому разі недостатній для того, щоб пояснити існування земельної ренти, незалежної від
різниці у родючості різних родів землі, або послідовних приміщень капіталу на тій самій землі,
коротко, такої ренти, яка в понятті відмінна від диференційної ренти і яку ми можемо тому позначити
як \emph{абсолютну ренту}. Цілий ряд мануфактурних продуктів має ту властивість, що їхня вартість вища від
їхньої ціни продукції і, не зважаючи на це, вони не дають такого надміру над пересічним зиском або
такого надзиску, що міг би перетворитись на ренту. Навпаки. Існування і поняття ціни продукції і
загальної норми зиску, яку вона включає, ґрунтуються на тому, що окремі товари продаються не по
їхній вартості. Ціни продукції виникають з вирівняння товарових вартостей, яке по покритті
відповідних капітальних вартостей, з ужиткованих в різних сферах продукції, розподіляє всю додаткову
вартість не в тій пропорції, що в ній її створено в окремих сферах продукції, і скільки її тому є в
продуктах останніх, а пропорційно величині авансованих капіталів. Тільки таким чином виникає
пересічний зиск і ціна продукції товарів, для якої пересічний зиск є характеристичним елементом. Це
є постійна тенденція капіталів, через конкуренцію здійснювати це вирівнювання в розподілі додаткової
вартости, створеної усім капіталом, і перемагати всі перешкоди цьому
вирівнюванню. Звідси і тенденція їхня допускати тільки такі надзиски, як вони виникають за всяких
обставин, не з різниці між вартостями і цінами продукції товарів, а радше, з різниці між загальною
ціною продукції, що реґулює ринок, і відмінними від неї індивідуальними цінами продукції; такі
надзиски, які тому, постають не з різниці між двома різними сферами продукції, а в межах кожної
сфери продукції, які, отже, не зачіпають загальних цін продукції різних сфер, тобто, загальної норми
зиску, а радше мають своєю
\parbreak{}  %% абзац продовжується на наступній сторінці

\parcont{}  %% абзац починається на попередній сторінці
\index{iii2}{0198}  %% посилання на сторінку оригінального видання
передумовою перетворення вартостей на ціни продукції і загальну норму зиску. Проте, як з’ясовано
давніше, ця передумова ґрунтується на раз-у-раз змінюваному пропорційному розподілі всього
суспільного капіталу між різними сферами продукції, на невпинній іміграції і еміграції капіталів, на
можливості переносити їх з одної сфери в іншу, коротко на вільному русі їх між цими різними сферами
продукції, як між відповідними вільними
царинами для приміщення самостійних частин усього суспільного капіталу. При цьому припускається, що
жодне — за винятком хіба лише випадкового і тимчасового — обмеження не перешкоджає конкуренції
капіталів зводити вартість до розміру ціни продукції, напр., у такій сфері продукції, в якій
вартість товарів вища від їхньої ціни продукції, або в якій створена додаткова вартість більша, ніж
пересічний зиск, зводити тут вартість до розмірів ціни продукції і тим самим розподіляти над мірну
додаткову вартість цієї сфери продукції пропорційно між усіма сферами,
що їх експлуатує капітал. Але коли настає протилежне цьому, коли капітал
наражається на чужу силу, яку він зовсім не може подолати, або може подолати
лише почасти, і яка обмежує його вкладення в окремих сферах продукції,
допускає його лише на умовах, що цілком або почасти виключають
згадане загальне вирівнювання додаткової вартості на пересічний зиск, то очевидно, що в таких сферах
продукції через надмір товарової вартости над ціною продукції їхніх товарів постав би надзиск, який
міг би перетворитися на ренту, а вона як така, могла б усамостійнитися проти зиску. Але як така чужа
сила і обмеження при приміщенні у землю капіталові протистоїть земельна власність, або капіталістові
— земельний власник.

Земельна власність є тут за бар’єр, що не дозволяє вкладати нових капіталів
у необроблену до того часу або не здану в оренду землю, не стягуючи
при цьому мита, тобто не вимагаючи ренти, хоч ця новопритягнена до обробітку
земля належить до такого роду, який не дає диференційної ренти, і який, коли б не існувало земельної
власності, міг би оброблятися вже при такому незначному підвищенні ринкової ціни, коли реґуляційна
ринкова ціна виплачувала б обробникові цієї найгіршої землі лише ціну продукції. Проте, в наслідок
межі, що її ставить земельна власність, ринкова ціна мусить підвищитись до такого пункту, коли земля
може виплачувати надмірну понад ціну продукції, тобто ренту. А що, згідно з припущенням, вартість
товарів, продукованих хліборобським капіталом, вища від їхньої ціни продукції, то ця рента (за
винятком того випадку, який буде зараз досліджено), становить надмір вартости над ціною продукції,
або частину цього надміру. Чи рівна рента всій різниці між вартістю і ціною продукції, чи тільки
більшій або меншій частині цієї різниці, це цілком залежить від стану попиту та подання і від
розміру площі, новопритягненої до обробітку. Доки рента не дорівнює надмірної вартости хліборобських
продуктів над їхньою ціною продукції, частина цього надміру завжди братиме участь у загальному
вирівнюванні і пропорційному розподілі всієї додаткової вартости між різними поодинокими капіталами.
Скоро б рента стала рівною надмірові вартости над ціною продукції, то вся ця частина додаткової
вартости, що становить залишок над пересічним зиском, була б відтягнена від цього вирівнювання. Але
чи дорівнює ця абсолютна рента всьому надмірові вартости над ціною продукції, чи дорівнює лише
частині його, все
одно, хліборобські продукти продавалося б по монопольній ціні не тому, що
їхня ціна вища, ніж їхня вартість, а тому, що вона дорівнює їхній вартості,
або тому, що вона нижча, ніж їхня вартість, але вища, ніж їхня ціна продукції, їхнє монопольне
становище було б у тому, що вони у
протилежність іншим промисловим продуктам, вартість яких вища від загальної ціни продукції, не
нівелювались би на ціну продукції. А що частина вартости, як і ціни продукції, є фактично дана стала
величина, а саме, витрати продукції, зужиткований у
\parbreak{}  %% абзац продовжується на наступній сторінці


\index{i}{0199}  %% посилання на сторінку оригінального видання
\looseness=-1
Не кажучи вже про загальні шкідливі впливи нічної праці\footnote{
«Цілком природно, — зауважує один фабрикант сталі, який
уживає до нічної праці дітей, — що молодь, яка працює вночі, не має
змоги спати вдень і не може користуватися путнім відпочинком, а лише
без перестанку тиняється на другий день («Children’s Employment Commission.
Fourth Report», 63, p. XIII). Про значення сонячного світла для
збереження й розвитку організму один лікар каже, між іншим: «Світло
безпосередньо впливає і на тканини тіла, яким дає міць і елястичність.
Мускули тварин, позбавлених нормальної кількости світла, стають як губка
і втрачають свою елястичність, сила нервів через недостачу побудливих
спонук втрачає свій тонус, і розвиток усього, що перебуває в процесі
зростання, занепадає\dots{} Щождо дітей, то для їхнього здоров’я є вельми
важливий постійний рясний приплив денного світла й безпосередній вплив
сонячного проміння протягом якоїсь частини дня. Світло помагає перетворювати
харч у добру плястичну кров і зміцнює новоутворені фібри. Воно
побуджує й органи зору і через те викликає інтенсивнішу діяльність різних
мозкових функцій». Пан В.~Стрендж, старший лікар «General Hospital»
у Worcester’i, що з його твору «Про здоров’я» (1864) запозичено це
місце, пише в одному листі до члена слідчої комісії пана Вайта: «Я мав
давніше нагоду стежити в Ланкашірі за впливом нічної праці на фабричних
дітей і, всупереч улюбленому запевненню деяких працедавців, я
рішуче заявляю, що така праця швидко підтинає здоров’я дітей». («Children’s
Employment Commission. 4 th Report», 284, p. 55). Що такі речі
можуть взагалі бути предметом серйозних суперечок, найкраще показує
те, як впливає капіталістична продукція на «мозкові функції» капіталістів
та їхніх retainers\footnote*{
прихильників. \emph{Ред.}
}.
},
безперервний процес продукції, що триває протягом двадцяти чотирьох
годин, дає незвичайно бажану нагоду для того, щоб переступати
межі номінального робочого дня. Приміром, у згаданих
вище галузях промисловости, де працюють з дуже великим напруженням,
офіціяльний робочий день становить для кожного робітника
здебільшого 12 годин нічних або денних. Але наднормова
праця, яка виходить поза ці межі, в багатьох випадках, уживаючи
слів англійського офіційного звіту, «справді повна жаху»
(«truly fearful»)\footnote{
Там же, 57, p. XII.
}. Ніякий людський розум, — каже звіт, — не
може уявити собі тієї маси праці, яку, за даними свідків, виконують
хлопці 9--12 років, і не дійти при цьому неминуче
до висновку, що такого зловживання владою батьків та працедавців
надалі не можна дозволяти»\footnote{
Там же (4 th Report, 1865), 58, p. XII.
}.

\looseness=-1
«Вже та метода, що хлопчаків взагалі примушують працювати
навпереміну то вдень то вночі, — вже це приводить так підчас
оживлення справ, як і за звичайного стану речей до ганебного здовження
робочого дня. Це здовження в багатьох випадках є не лише
жорстоке, але просто неймовірне. Часто-густо буває, що з тієї
або іншої причини іноді не прийде якийсь із хлопчаків на зміну.
Тоді один або декілька з присутніх хлопчаків, що вже скінчили
свій робочий день, мусять заступити відсутнього. Ця система
така загальновідома, що управитель однієї вальцювальні на мій запит,
як заповнюється місця відсутніх хлопчаків, відповів: «Аджеж
я знаю, що вам це так само добре відомо, як і мені», — і, ні трохи
\parbreak{}  %% абзац продовжується на наступній сторінці


\index{iii2}{0200}  %% посилання на сторінку оригінального видання
Коли б уся земля певної країни, придатна для хліборобства, була вже
здана в оренду, — при чому припускається, як загальне явище, капіталістичний
спосіб продукції і нормальні відносини, — то не було б такої землі, яка не
давала б ренти, але могли б існувати такі приміщення капіталу, окремі частини
капіталу вкладеного в землю, які не давали б ренти; бо, скоро земля здана в
оренду, земельна власність перестає діяти, як абсолютна межа потрібного вкладення
капіталу. Як відносна межа, вона продовжує ще діяти і після цього
в такій мірі, в якій перехід до земельного власника долученного до землі капіталу
ставить тут перед орендарем дуже визначені межі. Тільки в цьому випадку
вся рента перетворилася б на диференційну ренту, на диференційну ренту, яка
визначається не ріжницями в якості землі, а ріжницями між надзисками, що
постають від останніх приміщень капіталу на певній землі, і рентою, яка виплачувалася
б за оренду землі найгіршої кляси. Як межа земельна власність
діє абсолютно лише остільки, оскільки допущення до землі взагалі, як до сфери
приміщення капіталу, зумовлює данину земельному власникові. Коли це допущення
сталося, останній уже не може протиставити ніяких абсолютних меж
кількісному розмірові приміщення капіталу на даній дільниці землі. Будуванню
будинків взагалі кладеться межу земельною власністю третьої особи на ту дільницю
землі, на якій мається збудувати будинок. Але скоро лише ця земля
орендована під будівлю будинків, то вже від орендаря залежить, чи бажає він
збудувати на ній високий чи низький будинок.

Коли б пересічний склад хліборобського капіталу був такий самий або
вищий, ніж пересічний склад суспільного капіталу, то абсолютна рента, знов
таки в щойно дослідженому розумінні, відпала б; тобто відпала б рента, яка
відрізняється так від диференційної ренти, як і від ренти, що ґрунтується на
власне монопольній ціні. Тоді вартість хліборобського продукту не була б
вища від його ціни продукції, і хліборобський капітал пускав би в рух не
більшу кількість праці, отже, реалізував би також не більшу кількість додаткової
праці, ніж нехліборобський капітал. Те саме сталося б тоді, коли б з
проґресом культури склад хліборобського капіталу зрівнявся із пересічним
складом суспільного капіталу.

На перший погляд здається за суперечність припускати, що, з одного
боку, склад хліборобського капіталу підвищується, отже, зростає його стала частина
проти змінної, а з другого боку, що ціна хліборобського продукту має
піднестися остільки високо, щоб нова і гірша, ніж колишня, земля могла виплачувати
ренту, яка в цьому випадку могла б виникнути лише з надміру ринкової
ціни над вартістю і ціною продукції, коротко, лише з монопольної ціни
продукту.

Тут треба відрізняти таке.

Розглядаючи створення норм зиску, ми, насамперед, бачили, що капітали
які, технологічно розглядувані, складені однаково, тобто порівняно з кількістю
машин і сирового матеріялу пускають в рух однакову кількість праці,
можуть, проте, бути різного складу в наслідок того, що сталі частини цих капіталів
мають різну вартість. Сировий матеріял або машини можуть бути в
одному випадку дорожчі, ніж у другому. Щоб пустити в рух таку саму масу
праці (а це, згідно з припущенням, було б потрібне для перероблення такої ж
самої маси сирового матеріялу), в одному випадку довелося б авансувати більший
капітал, ніж у другому, тому що, наприклад, з капіталом 100 я не можу пустити
в рух однакової кількости праці, коли сировий матеріял, який теж доводиться
оплачувати з цих 100, в одному випадку коштує 40, в другому 20.
Але те, що технологічно ці капітали все ж складені однаково, негайно виявилося
б, скоро ціна дорожчого сирового матеріялу знизилася б до рівня дешевшого.
Відношення вартости змінного і сталого капіталу тоді стали б однакові,
\parbreak{}  %% абзац продовжується на наступній сторінці

\parcont{}  %% абзац починається на попередній сторінці
\index{iii2}{0201}  %% посилання на сторінку оригінального видання
хоч у технічному відношенні між ужитою живою працею та масою і природою
застосованих умов праці не сталося жодної зміни. З другого боку, коли розглядати
справу виключно з погляду складу за вартістю, капітал порівняно низького
органічного складу в наслідок простого підвищення вартостей його сталих частин
міг би справити таке вражіння, ніби він піднісся на один ступінь з капіталом
вищого органічного складу. Хай дано капітал $= 60c \dplus{} 40v$, тому що він
застосовує багато машин і сирового матеріялу, порівняно з живою працею, і
другий капітал $= 40c \dplus{} 60v$, тому що він вживав багато живої праці (60\%),
мало машин (скажімо, 10\%) і відносно до робочої сили мало, до того ж
ще і дешевого, сирового матеріялу (скажімо, 30\%); таким чином в наслідок простого
підвищення вартости сирових і допоміжних матеріялів з 30 до 80, склад
міг би зрівнятися так, що в другому капіталі на 10 в машинах припадало б 80
в сировому матеріялі і 60 робочої сили, тобто $90c \dplus{} 60v$, що, визначене в процентах,
теж дорівнювало б $60c \dplus{} 40v$, при чому не сталося б жодної зміни в технічному
складі. Отже, капітали однакового органічного складу можуть мати
різний вартісний склад, і капітали однакового процентного вартісного складу
можуть стояти на різних ступенях органічного складу отже, виражати різні ступені
розвитку суспільної продуктивної сили праці. Отже, сама лише обставина,
що за вартісним складом хліборобський капітал стояв би на загальному рівні, ще
не доводила б того, що суспільна продуктивна сила праці досягла у нього
такого самого рівня. Вона могла б лише показувати, що власний продукт цього
капіталу, який знову таки становить частину умов його продукції, є дорожчий,
або що допоміжні матеріяли, от як добриво, котрі давніш були просто під руками,
тепер доводиться довозити здалека тощо.

Але, залишаючи це осторонь, треба взяти на увагу своєрідний характер
хліборобства.

Припустімо, що вживання в хліборобстві машин, які зберігають працю,
хемічних допоміжних засобів тощо, набирають тут ширшого розміру, отже, що сталий
капітал технічно зростає не тільки щодо вартости, але й щодо маси, порівняно
я масою ужитої робочої сили; в такому разі в хліборобстві (як і в гірничій
промисловості) справа йде не тільки про суспільну, але і про природну продуктивність
праці, яка залежить від природних умов праці. Можливо, що збільшення
суспільної продуктивної сили в хліборобстві лише компенсує, або навіть
не зовсім компенсує зменшення природної сили — ця компенсація завжди може
впливати лише протягом деякого часу, — так що, не зважаючи на технічний
розвиток, продукт не здешевлюється, а лише гальмується його ще більше подорожчання.
Можливо також, що при висхідній ціні збіжжя абсолютна маса
продукту зменшується, тимчасом як відносний надпродукт зростає; це можливо
саме при відносному збільшенні сталого капіталу, що складається переважно
я машин або худоби, при чому доводиться покривати тільки його зношування,
і при відповідному зменшенні змінної частини капіталу, яка витрачається на
заробітну плату, що її постійно доводиться покривати з продукту цілком.

Але можливо також, що з поступом хліборобства потрібно буде лише помірне
підвищення ринкової ціни над пересічною для того, щоб могла оброблятись
і одночасно давати ренту така земля гіршої якости, яка при нижчому рівні
технічних допоміжних засобів потребувала б вищого підвищення ринкової ціни.

Може здатися, що та обставина, що, наприклад, у скотарстві, коли воно провадиться
в великих розмірах, маса вжитої робочої сили дуже мала проти сталого
капіталу, який є в самій худобі, може здатися, що ця обставина цілком суперечить
тому, що хліборобський капітал, обчислений у процентах, пускає в рух робочої
сили більше, ніж нехліборобський пересічний суспільний капітал. Але тут слід
відзначити, що при розгляді ренти ми виходимо, як з вирішної, з тієї частини
хліборобського капіталу, яка продукує основний рослинний засіб харчування,
\parbreak{}  %% абзац продовжується на наступній сторінці

\parcont{}  %% абзац починається на попередній сторінці
\index{iii2}{0202}  %% посилання на сторінку оригінального видання
отже, для цивілізованих народів взагалі головний засіб існування. Вже А.~Сміт
показав, — і це одна з його заслуг, — що в скотарстві і взагалі пересічно в
усіх капіталах, вкладених в землю не для продукції головних засобів існування,
наприклад, збіжжя, ціну визначається зовсім інакше. Саме її визначається
тут таким чином, що ціна продукту землі, яку, скажімо, як штучні луки використовується
для скотарства, але яку з такою самою зручністю можна було б
повернути на орну землю певної гідности, — що ціна продукту такої землі мусить,
підвищитись остільки, щоб давала вона таку саму ренту, як орна земля такої
самої якости; отже, рента з орної землі бере тут участь у визначенні ціни худоби,
чому Рамзей слушно відзначив, що ціна худоби в такий спосіб штучно
підвищується рентою, економічним виразом земельної власности, отже, земельною
власністю.

«В наслідок поширення культури, необроблюваних пустирів уже не вистачає
для того, щоб поповняти подання худоби, яка йде на заріз. Значну частину
оброблюваних земель доводиться повертати під розведення та відгодовування
худоби, що ціна її тому мусить бути така висока, щоб оплатити не
тільки вжиту на це працю, але й ренту, яку міг би здобувати землевласник,
і зиск, який міг би здобувати орендар з цієї землі, коли б вона була
оброблена як орна земля. Худоба вигодована на найнепридатніших до оброблення
торфовищах, буде продана, відповідно до ваги і якости, по тій самій ціні, поякій
на тому самому ринку продається худобу, вигодовану на землі якнайкраще
культивованій. Власники цих торфовищ виграють від цього і підвищують ренту
своїх земель відповідно до цін худоби». (A.~Smith, Book 1, chap. XI, part. I).
Отже, і тут на відміну від збіжжевої ренти диференційна рента йде на користь
гіршій землі.

Абсолютна рента пояснює деякі явища, які з першого погляду справляють
таке вражіння, ніби рента є наслідок просто монопольної ціни. Щоб почати
з прикладу А.~Сміта, уявімо, наприклад, власника ростучого без усякої
людської допомоги лісу, отже, такого лісу, що існує не як продукт лісництва,
наприклад, в Норвегії. Коли ренту виплачує йому капіталіст, що займається
рубанням лісу, бо на нього є попит в Англії, або коли власник сам як капіталіст
теж береться до рубання, то в дереві йому, крім зиску на авансований
капітал, виплачується більшу або меншу ренту. У відношенні до цього чисто
природного продукту це видається чисто монопольною надвишкою. Але в дійсності
капітал складається тут майже тільки з змінного капіталу, витрачуваного
на працю, і тому він пускає в рух більшу кількість додаткової праці, ніж інший
капітал рівної величини. Отже, у вартості дерева міститься більший надмір
неоплаченої праці або додаткової вартости, ніж у продукті капіталів вищого
складу. Тому з дерева може виплачуватись пересічний зиск, і значний надмір у
формі ренти може припадати власникові лісу. Навпаки, доводиться визнати, що
при тій легкості, з якою може поширюватися рубання лісу, тобто при тій
швидкості, з якою може збільшуватися ця продукція, треба дуже значного збільшення
попиту для того, щоб ціна дерева зрівнялась з його вартістю, і щоб тому
ввесь надмір неоплаченої праці (над тією її частиною, яка дістається капіталістові
як пересічний зиск) дістався б власникові у формі ренти.

Ми припускали, що новопритягнена до обробітку земля своює якістю ще
гірша, ніж та найгірша, що оброблялась останнього часу. Коли вона краща, то
вона дає диференційну ренту. Але тут ми досліджуємо саме той випадок, коли
рента являє собою не диференційну ренту. Тут можливі тільки два випадки.
Або новопритягнена до обробітку земля гірша, абож вона такої самої якости,
як остання з оброблюваних земель. Коли вона гірша, то цей випадок ми вже
дослідили. Отже, лишається дослідити ще тільки той випадок, коли вона такої
самої якости.


\index{iii2}{0203}  %% посилання на сторінку оригінального видання
З поступом культури до нового оброблення може притягатись, — як ми це
вже покатали, досліджуючи диференційну ренту — землю такої самої і навіть
кращої якости, так само як і землю гіршої якости.

\emph{Поперше}, тому, що при диференційній ренті (і при ренті взагалі, бо
навіть при не диференційній ренті завжди постає питання, чи дозволяє, з одного
боку, родючість землі взагалі, а з другого її положення, обробляти її,
одержуючи при регуляційній ринковій ціні зиск і ренту), діють в зворотному
напрямі дві умови, які то взаємно паралізують одна одну, то беруть
перевагу одна над однією. Підвищення ринкової ціни — припускаючи, що витрати
продукції потрібні для оброблення не понизились, іншими словами, що
не завоювання технічного характеру становлять новий момент, який зумовлює
нове оброблення, — може призвести до оброблення родючішої землі, яка давніш
через своє положення виключалась з числа конкурентних земель. Або ж для
менш родючої землі це може остільки підвищити вигоди положення, що ними
вирівнюється меншу родючість. Або, і без підвищення ринкової ціни, в наслідок
поліпшення засобів комунікації, положення може змінитись так, що кращі землі
ввійдуть у конкуренцію, що спостерігається у великому маштабі в степових
штатах Північної Америки. Та і в країнах старої цивілізації це відбувається
постійно, хоч і не в такому самому маштабі. як у колоніях, де, як справедливо
відзначів Wakefield, вирішна роля належить положенню. Отже, поперше,
протилежні дії положення і родючости, і змінливість чинника положення, який
постійно вирівнюється, постійно проробляє прогресивні зміни, що спрямовуються
до вирівнювання, — це призводить до того, що в конкуренцію з уже оброблюваними
землями навперемінку вступають дільниці землі однакової, кращої й
гіршої якости.

\emph{Друге}. З розвитком природничих наук і агрономії змінюється і родючість
землі, бо змінюються засоби, що з допомогою їх уможливлюється негайне
використання елементів ґрунту. Таким чином, у Франції і східніх графствах
Англії легкі ґрунти, які раніш вважалися за кепські, ще зовсім недавно були
перетворені в першорядні (див. Passy). З другого боку, земля, що її за хемічним
складом не вважалось за погану, але яка лише ставила певні механічнофізичні
перешкоди обробіткові, перетворюється на добру землю, скоро знаходять
засоби для того, щоб подолати ці перешкоди.

\emph{Третє}. У всіх країнах старої цивілізації старі історичні і традиційні
відносини, наприклад, у формі державних земель, громадських земель тощо,
цілком випадково відволікли великі дільниці землі від обробітку, до якого їх
притягають лише поступово. Порядок, в якому їх притягається до оброблення,
не залежить ані від їхньої якости, ані від їхнього положення, а лише від цілком
зовнішніх обставин. Коли простежити історію англійських громадських земель,
простежити, як вони законами про обгороджування (Enclosure Bills) поступово перетворювалися
на приватну власність і оброблялися, то виявилося б, що не може
бути нічого безглуздішого за те фантастичне припущення, ніби вибором цього
порядку керував якийсь сучасний хліборобський хемік, наприклад, Лібіх, призначаючи
певні лани через їхні хемічні якості під культуру та виключаючи
інші. Тут, радше, переважне значіння відігравала нагода, яка робить з людини
злодія; більш або менш зовнішньо пристойні юридичні зачіпки для привласнення,
що їх могли використати великі лендлорди.

\emph{Четверте}. Залишаючи осторонь той факт, що досягнутий в кожний
певний момент ступінь розвитку в зрості людности і капіталу кладе поширенню
хліборобської культури певну, хоча б і еластичну межу; залишаючи осторонь
діяння таких випадковостей, які справляють тимчасовий вплив на ринкову
ціну, от як ряд сприятливих або несприятливих діб року, — просторове поширеная
хліборобської культури залежить від загального стану ринку капіталів і
\parbreak{}  %% абзац продовжується на наступній сторінці

\parcont{}  %% абзац починається на попередній сторінці
\index{iii2}{0204}  %% посилання на сторінку оригінального видання
становища справ у країні. Для того, щоб додатковий капітал спрямувати до
хліборобства, за періодів скрути не досить того, що необроблений ґрунт може
дати орендареві пересічний зиск, — чи виплачує він ренту, чи ні. В інші періоди,
періоди достатку (der Plethora) капіталу він прямує до оброблення землі, навіть
без підвищення ринкової ціни, аби тільки взагалі були в наявності нормальні
умови. Тоді краща земля, ніж оброблювана до того часу, в дійсності могла б
бути виключена з конкуренції тільки або моментом її положення, або все ще
непоборними межами її винятковости, або випадковістю. Тому ми можемо зайнятися
тільки тими родами землі, що якістю однакові з останніми з оброблених
земель. Але між новою землею і останньою з оброблених все ще лишається
ріжниця витрат на оброблення нової землі, і від стану ринкових цін і відносин
кредиту залежить, чи будуть вони зроблені чи ні. Потім, скоро лише ця земля
дійсно візьме участь у конкуренції, ринкова ціна за інших незмінних відносин
знову понизиться до свого колишнього рівня, при чому новооброблювана земля
буде давати таку саму ренту, як відповідна їй стара. Засаду, що вона не даватиме
ренти, прихильники цієї засади доводять припущенням того, що треба ще довести,
а саме: що остання земля не дала ренти. Таким самим способом можна було б довести,
що останні з побудованих будинків, крім власне плати за наймання (Miethzins)\footnote*{
Звичайно плата за наймання чогось; тут — комірна плата, як процент та амортизація
вкладеного у будівлю капіталу, на відміну від власне ренти. \Red{Прим. Ред.}
}
будівель, не дають жодної ренти, хоч і винаймається їх. Але факт такий,
що коли вони протягом довгого часу лишаються незайняті, вони дають ренту, ще
до того, як починають давати плату за наймання (Miethzins). Подібно до того,
як послідовні приміщення капіталу на певній дільниці землі можуть давати
відповідний додатковий здобуток, а тому і таку саму ренту, як перші, — цілком
так само лани такої якости, як останні з оброблених, можуть при рівних витратах
давати рівний здобуток. Інакше взагалі було б незрозуміло, яким чином
можна було б лани однакової якости брати під оброблення послідовно, а не
всі разом або, радше, не брати жодного, щоб не викликати конкуренції всіх
інших. Земельний власник завжди готовий здобувати ренту, тобто одержувати
щось даром; але щоб задовольнити його бажання, капітал потребує певних
умов. Тому взаємна конкуренція між землями залежить не від того, що землевласник
хоче їхньої конкуренції, але від того, чи знайдеться капітал, який
схоче на нових ланах конкурувати з іншими.

Оскільки власне хліборобська рента є просто наслідок монопольної ціни,
вона може бути лише незначна, як і абсолютна рента може бути тут за нормальних
умов лише незначною, хоч би який був надмір вартости продукту
над його ціною продукції. Отже, суть абсолютної ренти є ось у чому: рівновеликі
капітали в різних сферах продукції, при рівній нормі додаткової вартости
або рівній експлуатації праці, продукують, залежно від їхнього різного пересічного
складу, різні маси додаткової вартости. В промисловості ці різні маси
додаткової вартости вирівнюються в пересічний зиск і рівномірно розподіляються
між окремими капіталами, як між відповідними частинами суспільного капіталу.
Земельна власність, коли для продукції потрібна земля чи то для хліборобства,
чи то для здобування сирових матеріялів, перешкоджає цьому вирівнюванню для
капіталів, приміщених у землю, і уловлює певну частину додаткової вартости,
яка інакше взяла б участь у вирівнюванні на загальну норму зиску. Отож,
рента становить частину вартости, точніше додаткової вартости товарів, алеж
тільки таку, що замість дістатися клясі капіталістів, яка здобула її з робітників,
дістається земельним власникам, які здобувають її з капіталістів. При
цьому припускається, що хліборобський капітал пускає в рух більше праці,
ніж рівновелика частина нехліборобського капіталу. До якої міри велике
\parbreak{}  %% абзац продовжується на наступній сторінці

\parcont{}  %% абзац починається на попередній сторінці
\index{iii1}{0205}  %% посилання на сторінку оригінального видання
Рікардо не досліджує), досить тільки перевернути щойно наведені
міркування.

I.~Пересічний капітал $= 80 c \dplus{} 20 v \deq{} 100$; норма додаткової
вартості \deq{} 100\%; ціна виробництва \deq{} товарній вартості $= 80 c \dplus{}
20 v \dplus{} 20 p \deq{} 120$; норма зиску \deq{} 20\%. Нехай заробітна плата
впаде на одну чверть, тоді той самий сталий капітал приводитиметься
в рух 15-ма $v$ замість $20 v$. Отже, ми маємо товарну
вартість $ \deq{} 80 c \dplus{} 15 v \dplus{} 25 p \deq{} 120$. Кількість праці, вироблена $v$,
лишається  незмінною, і тільки створена ним нова вартість інакше
розподіляється між капіталістом і робітниками. Додаткова вартість
підвищилась з 20 до 25, і норма додаткової вартості
підвищилась з \frac{20}{25} до \frac{25}{15}, отже, з 100\% до 166\sfrac{2}{3}\%.
Зиск на 95 тепер \deq{} 25, отже, норма зиску на 100 \deq{} 26\sfrac{6}{19}. Новий
процентний склад капіталу тепер є $84\sfrac{4}{19}c \dplus{} 15\sfrac{15}{19}v \deq{} 100$.

II.~Нижчий склад. Первісно $50 c \dplus{} 50 v$, як вище. В наслідок
падіння заробітної плати на \sfrac{1}{4}, $v$ зводиться до 37\sfrac{1}{2}, і тим самим
весь авансований капітал зводиться до $50 c \dplus{} 37\sfrac{1}{2}v \deq{} 87\sfrac{1}{2}$. Якщо
ми застосуємо до цього нову норму зиску в 26\sfrac{6}{19}\%, то
$100 : 26\sfrac{6}{19} \deq{} 87\sfrac{1}{2} : 23\sfrac{1}{38}$. Та сама товарна маса,
яка раніш коштувала 120, коштує
тепер $87\sfrac{1}{2} \dplus{} 23\sfrac{1}{38} \deq{} 110\sfrac{10}{19}$; падіння ціни
майже на 8\%\footnote*{
В першому німецькому виданні тут стоїть: „майже на 10\%“. В рукопису
Маркса в цьому місці сказано: „Знижується майже на 10“, тобто дається абсолютне
число. Точно воно дорівнює 9\sfrac{9}{19} і становить 7\sfrac{7}{19}\%. \emph{Примітка ред. нім. вид. ІМЕЛ.}
}.

III.~Вищий склад. Первісно $92 c \dplus{} 8 v \deq{} 100$. Падіння заробітної
плати на \sfrac{1}{4} знижує $8 v$ до $6 v$, весь капітал до 98. Отже,
$100 : 26\sfrac{6}{19} \deq{} 98 : 25\sfrac{15}{19}$. Ціна виробництва товару,
раніш $100 \dplus{} 20 \deq{} 120$, тепер, після падіння заробітної плати, є
$98 \dplus{} 25\sfrac{15}{19} \deq{} 123\sfrac{15}{19}$;
отже, вона підвищилась більше ніж на 3\%\footnote*{
В першому німецькому виданні тут стоїть: „майже на 4\%“. В рукопису
Маркса тут так само дається абсолютне число (3\sfrac{15}{19}). В процентах воно
становить 3\sfrac{3}{19}\%. \emph{Примітка ред. нім. вид. ІМЕЛ.}
}.

Отже, ми бачимо, що досить тільки повторити попередні міркування в зворотному
напрямі і з відповідними змінами: загальне падіння заробітної плати має своїм
наслідком загальне підвищення додаткової вартості, норми додаткової вартості, а
при інших незмінних умовах і норми зиску, хоч і в
іншій пропорції; далі воно має своїм наслідком падіння цін виробництва для
товарних продуктів капіталів нижчого складу і підвищення цін виробництва для
товарних продуктів капіталів вищого складу. Результат якраз протилежний до того,
що  виявився при загальному підвищенні заробітної плати\footnote{
Надзвичайно дивно, що Рікардо (який, звичайно, застосовує іншого
методу, ніж це зроблено тут, бо не розуміє процесу вирівнення вартостей в ціни
виробництва) навіть не приходить до цієї думки, а розглядає тільки перший
випадок, підвищення заробітної плати і вплив його на ціни виробництва товарів
(„Principles etc.“, Лондон 1852, стор. 26 і далі]. A servum pecus
imitatorum [рабське стадо наслідувачів] не додумалось навіть до того, щоб
зробити це, само собою зрозуміле, по суті тавтологічне застосування.
}.
\parbreak{}  %% абзац продовжується на наступній сторінці

\parcont{}  %% абзац починається на попередній сторінці
\index{iii2}{0206}  %% посилання на сторінку оригінального видання
й розвиток основного капіталу, який або долучається до землі, або укорінюється
в ній, спирається на ній, як усі промислові будівлі, залізниці, товарові склепи,
фабричні будівлі, доки тощо. Сплутування комірної плати, оскільки вона становить
процент та амортизацію капіталу, вкладеного в будинок, з рентою просто за
землю, неможливе тут навіть при всій добрій волі, як у Кері, особливо тоді
коли, як в Англії, земельний власник і будівельний спекулянт є цілком різні
особи. Тут береться на увагу два елементи: з одного боку, експлуатація землі
з метою репродукції або видобувної промисловости; з другого, — простір, який
потрібен як елемент усякої продукції й усякої людської діяльности. І за те і за
друге земельна власність вимагає своєї дані. Попит на будівельні дільниці підвищує
вартість землі як простору і основи, тимчасом як через це і разом
з цим зростає попит на елементи землі, що правлять за будівельний матеріял\footnote{
«Забрукування лондонських вулиць дало можливість власникам деяких голих скель на шотляндському
березі здобувати ренту з абсолютно некорисного до того часу кам’янистого ґрунту». A. Smith, Book I,
chap. XI. 2.
}.

У книзі II, розд. XII в свідченнях Едварда Каппа, великого лондонського
будівельного спекулянта, перед банковою комісією 1857 року, ми бачили приклад
того, яким чином в швидко ростучих містах, особливо коли будівлю провадиться,
як у Лондоні, фабричним способом, за головний об’єкт будівельної
спекуляції є власне не будинок, а земельна рента. Він говорить там № 5435:

«Я вважаю, що людина, яка бажає поступувати на світі, навряд чи може
розраховувати, що вона буде поступувати, провадячи тільки солідну справу (fair
trade)\dots{} вона неминуче мусить, крім того, будувати з метою спекуляції і до
того ж у великому маштабі; бо підприємець здобуває дуже мало зиску з самих
будівель, свій головний зиск він здобуває з підвищених земельних рент. Припустімо,
що він орендує дільницю землі і виплачує за неї 300\pound{ ф. стерл.} на рік;
коли він, дбало опрацювавши будівельний плян, збудує на цій дільниці будинки
належного розряду, то йому може пощастити здобути за це 400 або 450\pound{ ф. стерл.}
на рік, і його зиск у багато більшій мірі був би у збільшеній земельній ренті
на 100 або 150\pound{ ф. стерл.} на рік, ніж у зиску від будівель, який він у багатьох
випадках взагалі навряд чи бере на увагу». До того не слід забувати, що
по закінченні договору про винаймання, який найчастіше складається на 99 років,
земля з усіма будівлями на ній і з земельною рентою, яка за цей час
здебільша підвищується більше, ніж у два-три рази, знову повертається від
будівельного спекулянта або його правонаступника до первісного останнього
земельного власника.

Власне рента з копалень визначається цілком так само, як хліборобська
рента. «Бувають такі копальні, що їхній продукт навряд чи достатній для того,
щоб оплатити працю і покрити вкладений туди капітал разом з звичайним
зиском. Вони дають деякий зиск підприємцеві, але жодної ренти для земельного
власника. Їх міг би з вигодою обробляти тільки земельний власник, який сам,
бувши підприємцем, здобуває звичайний зиск на свій вкладений капітал. Багато
вугільних шахт у Шотландії розробляються в такий спосіб, і не могли б
розроблятися якось інакше. Земельний власник нікому не дозволяє розробляти
їх, коли йому не виплачують ренти, але ніхто не може виплачувати за них
ренти». (A. Smith, Book I, chap. XI, 2).

Треба відрізняти, чи випливає рента з монопольної ціни тому, що існує
незалежна від неї монопольна ціна продуктів або самої землі, або ж чи продаються
продукти по монопольній ціні тому, що існує рента. Коли ми говоримо
про монопольну ціну, ми взагалі маємо на думці таку ціну, яка визначається
тільки прагненням купити і платоспроможністю покупців, незалежно так від
тієї ціни, що визначається загальною ціною продукції, як і від тієї, що визначається
вартістю продукту. Виноградник, що продукує вино цілком виключної
\parbreak{}  %% абзац продовжується на наступній сторінці

\parcont{}  %% абзац починається на попередній сторінці
\index{iii2}{0207}  %% посилання на сторінку оригінального видання
якости, вино, яке взагалі може продукуватися лише в порівняно невеликій кількості,
має монопольну ціну. В наслідок цієї монопольної ціни, надмір якої над вартістю
продукту визначається тільки багатством і смаком вельможних споживачів вина,
винороб реалізував значний надзиск. Цей надзиск, який тут випливає з монопольної
ціни, перетворюється на ренту і дістається в цій формі земельному власникові, в
наслідок його титулу на цю дільницю землі, що має особливі властивості. Отже,
ренту тут створює монопольна ціна. Навпаки, рента створила б монопольну ціну,
коли б в наслідок тієї межі, яку покладає земельна власність нерентодайному приміщенню
капіталу на необробленій землі, коли б, у наслідок цієї межі, збіжжя продавалось
не тільки вище його ціни продукції, але й вище його вартости. Що самий
тільки титул власности певного числа осіб на земну кулю дає їм можливість привласнювати
собі частину додаткової праці суспільства, як дань, до того ж привласнювати
її собі з розвитком продукції в постійно ростучому маштабі, — це затушковується
тією обставиною, що капіталізована рента, отже, саме ця капіталізована
дань виступає, як ціна землі, і тому земля може продаватись, як усякий інший
об’єкт торговлі. Тому покупцеві здається, що він одержав своє домагання на ренту
не даром, і не даром одержав до розпорядку працю, риск і підприємецький дух
капіталу, а заплатив за це відповідний еквівалент. Рента, як відзначено вже
давніш, здається йому тільки процентом на капітал, за який він купив землю,
а тому і домагання на ренту. Цілком так само рабовласникові, що купив негра,
здається, що він придбав свою власність на негра не через інститут рабства
як такий, а через купівлю та продаж товарів. Але ж самий титул актом продажу
не породжується, а лише переноситься. Титул мусить існувати до того,
як його можна продати, і продаж, так само як і ряд продажів і їхнє постійне
повторення, не можуть створити цього титулу. Що взагалі створило його, так
це продукційні відносини. Скоро вони досягають такого пункту, де вони мусять
змінити свою шкуру, відпадає матеріяльне джерело титулу, економічно
та історично виправдуваного й виниклого з процесу суспільної продукції
життя, а разом з ним відпадають і засновані на ньому операції. З погляду
вищої економічної формації суспільства, приватна власність окремих індивідуумів
на земну кулю буде здаватись цілком такою самою безглуздою, як приватна
власність однієї людини на іншу людину. Навіть ціле суспільство, нація
і навіть усі одночасно сущі суспільства, узяті разом, не є власники землі. Вони є
лише її посідачі, лише користувачі з неї і як boni patres familias\footnote*{
У римлян — голова родини; поняття це ширше, ніж в українській мові — бо
«родина» у римлян складалась з членів власне сем’ї, родини разом з усією челяддю, рабами тощо. Boni
patres familias — порядні голови родин. \Red{Пр.~Ред.}
} вони мусять залишити її наступним поколінням поліпшеною.

\pfbreak

В дальшому дослідженні ціни землі ми залишаємо осторонь усі коливання
конкуренції, всяку спекуляцію землею, а також дрібну земельну власність, за
якої земля становить головне знаряддя продуцентів, бо вони вимушені купувати її
за всяку ціну.

I.~Ціна землі може підвищитись, хоч рента її не підвіщується; саме:

1) в наслідок просто пониження розміру проценту, яке впливає так, що
ренту продається дорожче, а тому капіталізована рента, ціна землі зростає;

2) тому що зростає процент на долучений до землі капітал.

II.~Ціна землі може підвіщитися тому, що зростає рента.

Рента може зростати тому, що підвищується ціна продукту землі; в цьому
випадку завжди підвищується норма диференційної ренти, незалежно від того,
чи буде рента з найгіршої з оброблюваних земель велика, мала, чи її зовсім
не буде. Під нормою ми розуміємо відношення тієї частини додаткової вартости,
\parbreak{}  %% абзац продовжується на наступній сторінці

\parcont{}  %% абзац починається на попередній сторінці
\index{iii2}{0208}  %% посилання на сторінку оригінального видання
яка перетворюється на ренту, до авансованого капіталу, який продукує продукт
землі. Це відношення відрізняється від відношення додаткового продукту до
всього продукту, бо весь продукт має в собі не весь авансований капітал, саме
не має в собі основного капіталу, який продовжує існувати поряд з продуктом.
Навпаки, воно припускає, що на тих родах землі, які дають диференційну
ренту, дедалі ростуча частина продукту перетворюється в надмірний  надпродукт.
На найгіршій землі підвищення ціни хліборобського продукту вперше створює
ренту, а тому і ціну землі.

Але рента може зростати і без підвищення ціни хліборобського продукту.
Остання може лишитися сталою або навіть понизитися.

Коли вона лишається сталою, то рента може зрости або тільки тому
(залишаючи осторонь монопольні ціни), що при колишньому розмірі капіталу,
вкладеного у старі землі, починають оброблятись нові землі кращої якости, але
їх лише вистачає на те, щоб покрити вирослий попит, так що реґуляційна
ринкова ціна залишається без зміни. В цьому випадку ціна старих земель не
підвищується, але для землі, наново взятої під оброблення, ціна підвищується
понад рівень ціни старої землі.

Або ж рента підвищується тому, що при незмінній відносній продуктивності
і незмінній ринковій ціні зростає маса капіталу, що експлуатує землю.
Тому, хоч рента у відношенні до авансованого капіталу лишається та сама, її
маса, наприклад, подвоюється, бо сам капітал подвоївся. А щоб не сталося
пониження ціни, то друге приміщення капіталу дає, так само як і перше,
надзиск, який по закінчені терміну оренди теж перетворюється на ренту. Маса
ренти тут збільшується тому, що збільшується маса капіталу, який створює
ренту. Твердження, що різні послідовні приміщення капіталу на тій самій дільниці
землі можуть створити ренту лише тоді, коли продукт їхній неоднаковий
і тому постає диференційна рента, сходить на твердження, що, коли два капітали
по \num{1.000}\pound{ ф. стерл.}, вкладено в два лани однакової продуктивности, то
лише один з них може дати ренту, хоч обидва ці лани належать до кращої
кляси землі, яка дає диференційну ренту. (Отже, загальна маса ренти, вся
рента певної країни, збільшується з масою вкладеного капіталу, при чому
необов’язково, щоб тут зростала ціна одиниці земельної площі, або норма
ренти, або навіть маса ренти на одиницю площі; в цьому випадку маса
всієї ренти зростає з просторовим поширенням культури. Це може навіть
бути поєднане з падінням ренти на окремих володіннях). Інакше це твердження
звелося б до другого твердження, а саме, що приміщення капіталуодне
поряд одного у дві різні дільниці землі підлягає іншим законам, ніж послідовне
приміщення капіталу на тій самій дільниці землі, тимчасом як в дійсності
диференційну ренту висновують саме з тотожності закону в обох випадках,
з приросту продуктивности приміщення капіталу на тім самім лані, як і на
різних ланах. Єдина модифікація, що існує тут, і якої не помічають, є в тому,
що послідовні приміщення капіталів, коли їх вживають до просторово різних
земель, наражаються на таку межу, як земельна власність, тим часом як при послідовних
приміщеннях капіталу в ту саму землю цього не буває. Звідси і та
протилежна дія, в наслідок якої ці різні форми приміщення капіталу на практиці
взаємно обмежують одна одну. Тут ніколи не постає ріжниці з самого капіталу.
Коли склад капіталу лишається той самий, так само, як норма додаткової
вартости, то норма зиску лишається незмінна, так що при подвоєнні
капіталу маса зиску подвоюється. Так само за припущених відношень норма
ренти лишається та сама. Коли капітал в \num{1.000}\pound{ ф. стерл.} дає ренту в х, то
капітал в \num{2.000}\pound{ ф. стерл.} за припущених обставин дає ренту в 2х. Але, коли
обчислити ренту у відношенні до земельної площі, яка лишилася без зміни, бо,
згідно з припущенням, подвоєний капітал працює на тому самому лані, то
\parbreak{}  %% абзац продовжується на наступній сторінці

\parcont{}  %% абзац починається на попередній сторінці
\index{iii2}{0209}  %% посилання на сторінку оригінального видання
виявиться, що в наслідок збільшення маси ренти підвищився і її рівень. Той
самий акр, що давав 2\pound{ ф. стерл.} ренти, дає тепер 4\pound{ ф. стерл.}\footnote{
Одна з заслуг Родбертуса, що до його важливої праці про ренту ми вернемося в книзі IV,
є в тому, що він розвинув цей пункт. Але, поперше, він помиляється, припускаючи, ніби для капіталу
зріст зиску завжди виявляється як і зріст капіталу, так що при збільшенні маси зиску відношення
лишається
те саме. А проте, це невірно, бо коли склад капіталу змінюється, норма зиску, не зважаючи на
незмінну експлуатацію праці, може підвищитись саме тому, що відносна вартість сталої частини
капіталу
проти змінної знизилася. — Подруге, він помиляється, трактуючи це відношення грошової ренти до
кількісно певної дільниці землі, наприклад, до одного акра, як щось таке, що взагалі припускає
тисячна
економія в її дослідженнях про підвищення або пониження ренти. Це знов невірно. Вона постійно
розглядає норму ренти у відношенні до продукту, — оскільки вона розглядає ренту в її натуральній
формі, — і у відношенні до авансованого капіталу, — оскільки вона розглядає ренту як грошову ренту,
—
бо це в дійсності є раціональні вирази.
}.

Відношення певної частини додаткової вартости, грошової ренти, — бо гроші
є самостійний вираз вартости, — до землі само по собі є безглузде й іраціональне;
бо це не співмірні величини, що тут виміряються одна одною, певна
споживна вартість, дільниця землі на стільки-от квадратових футів з одного
боку, і вартість, точніше, додаткова вартість — з другого. В дійсності це не виражає
нічого іншого, а тільки те, що в даних відносинах власність на стільки-от квадратових
футів землі дає земельному власникові можливість уловлювати певну кількість
неоплаченої праці, реалізованої капіталом, який риється на цих квадратових футах,
як свиня у картоплі (в рукопису тут стоїть в дужках, але закреслене: Лібіх). Але
prima facie цей вираз є те саме, як коли б ми здумали говорити про відношення
п’ятифунтової банкноти до діяметра землі. Однак, до посередництва тих іраціональних
форм, в яких виступають і на практиці резюмуються певні економічні
відносини, практичним носіям цих відносин у їхньому житті-бутті немає
жодного діла; а що вони привикли рухатися в цих посередницьких відносинах,
то їхній розум ані трохи не спотикається на них. Цілковита суперечність для них
не має рішуче нічого таємничого. У формах проявлення, відчужених від внутрішнього
зв’язку і безглуздих, коли їх узяти самих по собі, вони почувають себе
так само вдома, як риба у воді. Тут справедливе те, що Геґель сказав про
відомі математичні формули: те, що звичайний людський розум вважає за
і раціональне, є раціональне, а раціональне для нього є сама і раціональність.

Отже, коли розглядати справу у відношенні до самої площі землі, то підвищення
маси ренти виражається цілком так само, як підвищення норми ренти;
а звідси труднощі, що постають, коли умови, які пояснювали б один випадок,
відсутні в іншому випадку.

Але ціна землі може підвищитись навіть тоді, коли ціна продукту землі
зменшується.

В цьому випадку в наслідок дальшого диференціювання може збільшитися
диференційна рента, а тому й ціна кращих земель. Або ж коли цього немає, то
при збільшеній продуктивній силі праці ціна хліборобського продукту може понизитись,
але так, що це буде більш, ніж урівноважено збільшенням продукції. Припустімо,
що квартер коштував 60\shil{ шил.} Коли на тім самім акрі при тому самому
капіталі будуть випродуковані 2 квартери замість одного, і квартер понизиться
до 40 шил, то 2 квартери дадуть 80 шил, так що вартість продукту того самого
капіталу на тому самому акрі підвищиться на одну третину, хоч ціна акра
понизилась на одну третину. Як це можливо без того, щоб продукт продавався
вище його ціни продукції або вартости, було показано при дослідженні диференційної
ренти. В дійсності це можливо тільки в два способи. Або гіршу
землю вилучається з конкуренції, але ціна кращої землі зростає, коли
диференційна рента зростає, отже, коли загальне поліпшення діє нерівномірно
на різні роди землі. Або ж на найгіршій землі та сама ціна продукції
(і та сама вартість, коли виплачується абсолютну ренту) в наслідок підвищення
\index{iii2}{0210}  %% посилання на сторінку оригінального видання
продуктивности праці виражається у збільшеній масі продукту. Продукт
становить тепер ту саму вартість, що й давніш, але ціна його складових
частин понизилася, тимчасом, як число цих частин збільшилося. Коли вживається
той самий капітал, це неможливе, бо в цьому випадку та сама вартість
виражається в якій завгодно масі продукту. Але це можливе, коли витрачено
додатковий капітал на гіпс, гуано тощо, коротко кажучи, на такі поліпшення,
що вплив їхній триває багато років. Умова цього є в тому, щоб ціна одного квартера
хоч і знизилась, але не в такому самому відношенні, як зростає число квартерів.

III.~Ці різні умови підвищення ренти, а тому і ціни землі взагалі або окремих
родів землі, можуть почасти конкурувати між собою, почасти вони виключають
одна одну і можуть діяти лише навперемінки. Але з вище розвинутого, випливає,
що з підвищення ціни землі не можна без дальших околичностей робити
висновку, що рента підвищилась, і з підвищення ренти, яке завжди спричинює
підвищення ціни землі, не можна без дальших околичностей робити висновку,
що продукт землі збільшився\footnote{
Про падіння земельних цін при підвищенні ренти як про факт дивись Passy.
}.

\pfbreak

Замість звернутися до дійсних природних причин виснаження ґрунту, які,
проте, в наслідок стану хліборобської хемії в той час були невідомі усім економістам,
що писали про диферецційну ренту, — по допомогу звернулися до того
поверхового погляду, що в просторово обмежений лан не можна вкласти необмежену
масу капіталу; паприклад, Westminster Rewiew заперечує Річардові
Джонсові, що не можливо було б прогодувати цілу Англію обробітком Soho Square.
Хоч це вважається за особливу невигоду хліборобства, але справедливе як раз
зворотне. У хліборобстві можна продуктивно провадити послідовні приміщення
капіталу тому, що сама земля діє як знаряддя продукції, тимчасом як цього зовсім
немає, або є лише в дуже вузьких межах у випадку з фабрикою, де земля
функціонує лише як фундамент, як місце, як просторова операційна база. Правда,
можна — так і робить велика промисловість — саме на відносно невеликім, проти
парцельованого ремесла, просторі концентрувати велику продукційну споруду.
Але за даного ступеня розвитку продуктивної сили завжди потрібен певний простір,
і будування в висоту теж має свої певні практичні межі. Поширення продукції
за ці межі потребує і поширення простору землі. Основний капітал, вкладений
у машини тощо, не поліпшується споживанням, а навпаки, зношується. Внаслідок
нових винаходів і тут можуть статися окремі поліпшення, але, припускаючи
даний ступінь розвитку продуктивної сили, машина при споживанні може
лише погіршуватись. При швидкому розвитку продуктивної сили всю сукупність
старих машин доводиться заміняти вигіднішими, отже, вони гинуть. Навпаки,
земля, коли вона правильно обробляється, дедалі поліпшується. Та перевага
землі, що послідовні приміщення капіталу можуть дати вигоду без втрати колишніх,
одночасно має в собі можливість різної продуктивности цих послідовних
приміщень капіталу.

\section{Генеза капіталістичної земельної ренти}

\subsection{Вступ.}

Треба з’ясувати собі, в чому власне є труднощі трактування земельної
ренти з погляду сучасної економії, як теоретичного виразу капіталістичного
способу продукції. Цього ще не розуміє навіть величезне число новітніх письменників,
про що свідчить всяка нова спроба з’ясувати земельну ренту «по
новому». Новіша тут майже завжди є в повороті до давно вже побореного погляду.
\index{iii2}{0211}  %% посилання на сторінку оригінального видання
Трудність не в тому, щоб взагалі з’ясувати створений хліборобським
капіталом додатковий продукт і відповідну до нього додаткову вартість. Це
питання, радше, є вже розв’язане аналізою додаткової вартости, створюваної
всяким продуктивним капіталом, хоч би в яку сферу він був вкладений. Трудність
є в тому, що треба показати, звідки після того, як додаткова вартість
вирівнялась між різними капіталами на пересічний зиск, на відповідну до
їхніх відносних величин пропорційну частину всієї додаткової вартости, створеної
всім суспільним капіталом у всіх сферах продукції, — звідки після цього вирівняння,
після того як розподіл усієї додаткової вартости, яка взагалі може
бути розподілена, вже очевидно стався — звідки ж тут після цього береться
ще й та надмірна частина цієї додаткової вартости, яку капітал, вкладений
в землю, виплачує в формі земельної ренти земельному власникові.
Цілком лишаючи осторонь практичні мотиви, які спонукали сучасних економістів
як оборонців промислового капіталу проти земельної власности
досліджувати це питання, — мотиви, які ми накреслимо ближче в розділі
про історію земельної ренти, — це питання становило для них, як для теоретиків,
переважний інтерес. Визнати, що появлення ренти на капітал, вкладений
в хліборобство, завдячує особливій дії самої сфери приміщення, властивостям,
належним земній корі, як такій, це значило б відмовитись від самого
поняття вартости, отже, відмовитися від усякої можливости наукового
пізнання в цій галузі. Саме звичайне спостереження, що ренту виплачується
з ціни продукту землі, а це так і є навіть в тому випадку, коли її виплачується
в натуральній формі, скоро тільки орендар здобуває свою ціну продукції,
— показує, оскільки безглуздо надмір цієї ціни над звичайною ціною
продукції, отже, відносну дорожнечу хліборобського продукту, пояснювати надміром
природної продуктивности хліборобської промисловости над продуктивністю
інших галузей промисловости; бо, навпаки, що продуктивніша праця, то
дешевша кожна складова частина її продукту, тому що тим більша маса споживних
вартостей, в якій репрезентована та сама кількість праці, отже, та сама вартість.

Отже, при аналізі ренти вся трудність була в тому, що треба було
пояснити надмір хліборобського зиску над пересічним зиском, з’ясувати не додаткову
вартість, а властиву цій сфері продукції надмірну додаткову вартість,
отже, знов таки не «чистий продукт», а надмір цього чистого продукту над
чистим продуктом інших галузей промисловости. Сам пересічний зиск є продукт,
витвір процесу соціяльного життя, що відбувається в цілком певних історичних
продукційних відносинах, продукт, що має своєю передумовою, як ми
бачили, дуже широкосяжні посередницькі ланки. Для того, щоб взагалі можна було
говорити про надмір над пересічним зиском, сам цей пересічний зиск мусить
взагалі скластися як маштаб і — як це відбувається за капіталістичного способу
продукції, — як регулятор продукції. Отже, в таких суспільних формах, де ще
немає капіталу, який виконує ту функцію, що вимушує всю додаткову працю
і привласнює в першу чергу собі всю додаткову вартість, отже, де капітал ще
не упідлеглив собі суспільної праці, або упідлеглив її лише місцями, — взагалі
не може бути мови про ренту в сучасному значенні, про ренту як надмір
над пересічним зиском, тобто над пропорційною частиною всякого індивідуального
капіталу в додатковій вартості, спродукованій усім суспільним капіталом. Те
що, наприклад, пан Passy (дивись далі) говорить вже про ренту в первісному стані
як про надмір над зиском, як про надмір над історично-певного суспільною
формою додаткової вартости, так що за п. Passy ця форма могла б, мабуть,
існувати і без суспільства, — свідчить лише про його наївність.

Для колишніх економістів, які взагалі лише починали аналізу капіталістичного
способу продукції, ще нерозвиненого за їхнього часу, аналіза ренти
або взагалі не становила жодних труднощів, або становила лише труднощі цілком
\index{iii2}{0212}  %% посилання на сторінку оригінального видання
іншого характеру. Петті, Кантільйон, взагалі письменники, що ближче стоять
до доби февдалізму, беруть земельну ренту як нормальну форму додаткової
вартости взагалі, тимчасом як зиск для них ще не визначився і поєднується
з заробітною платою, або, щонайбільше, виступає як та частина цієї
додаткової вартости, що її капіталіст витискує з земельного власника. Отже, вони
виходять з такого стану, коли, поперше. хліборобська людність становить ще рішучу
переважну частину нації, і коли, подруге, земельний власник ще є тією
особою, яка, користуючись монополією земельної власности, у першу чергу привласнює
надмірну працю беспосередніх продуцентів, коли, отже, земельна власність
все ще є головна умова продукції. Для них ще не могло існувати такої
постави питання, що, навпаки, з погляду капіталістичного способу продукції,
намагається дослідити, яким чином земельна власність досягає того, що
віднімає від капіталу частину спродукованої ним (тобто вичавленої з безпосередніх
продуцентів) і в першу чергу привласненої вже ним додаткової вартости.

\emph{У фізіократів} труднощі вже іншого характеру. Як дійсно перші систематичні
тлумачі капіталу, вони намагалися аналізувати природу додаткової вартости
взагалі. Для них ця аналіза збігається з аналізою ренти, однісінької
форми, в якій для них існує додаткова вартість. Капітал, що дає ренту, або
хліборобський капітал, є для них однісінький капітал, що продукує додаткову
вартість, і пущена ним в рух хліборобська праця є однісінька, що створює додаткову
вартість, отже, з капіталістичного погляду цілком послідовно однісінька
продуктивна праця. Продукцію додаткової вартости вони цілком слушно вважають
за визначальний момент. Їм, залишаючи осторонь інші заслуги, про які мова
буде в книзі ІV, належить насамперед та велика заслуга, що від торговельного
капіталу, який функціонує тільки в сфері циркуляції, вони звернулись до продуктивного
капіталу, протилежно до меркантильної системи, яка за своїм грубим
реалізмом була справжньою вульґарною економією тієї доби, що її практичними
інтересами було відсунуто цілком на задній плян початки наукової аналізи
у Петті та його послідовників. Між іншим, тут, при критиці меркантильної системи,
мова йде лише про її погляди на капітал та додаткову вартість. Вже
давніш ми відзначали, що продукцію на світовий ринок і перетворення продукту
на товар, а тому і на гроші, монетарна система справедливо проголосила за передумову
і умову капіталістичної продукції. В її продовженні, в меркантильній системі,
переважну ролю відіграє вже не перетворення товарової вартости на гроші, а створення
додаткової вартости, але розглядається воно з іраціонального погляду сфери
циркуляції, до того ж так, що ця додаткова вартість виступає в формі додаткових
грошей, в надмірі торговельного балансу. Разом з тим справді характеристичне
для заінтересованих купців і фабрикантів того часу і адекватне тому періодові
капіталістичного розвитку, який вони репрезентують, є те, що при перетворенні
хліборобських февдальних громад на промислові, і при відповідній промисловій
боротьбі націй на світовому ринку, справа залежить від прискореного розвитку
капіталу, що досягається не так званим природним шляхом, а примусовими заходами.
Величезна ріжниця є в тому, чи перетворюється національний капітал на промисловий
поступово і повільно, чи це перетворення прискорюється в часі, в наслідок податків,
що ними вони в формі охоронних мит оподатковували переважно земельних
власників, середніх і дрібних селян і ремесло, в наслідок прискореної експропріяції
самостійних безпосередній продуцентів, в наслідок насильницької прискореної
акумуляції і концентрації капіталів, коротко, в наслідок прискореного
створення умов капіталістичного способу продукції. Разом з тим це становить
величезну ріжницю в капіталістичній і промисловій експлуатації природної національної
продуктивної сили. Тому національний характер меркантильної системи
в устах її оборонців є не просто фраза. З тієї притоки, що їх ніби цікавить тільки
багатство нації та допоміжні ресурси держави, вони в дійсності проголошують
\parbreak{}  %% абзац продовжується на наступній сторінці

\parcont{}  %% абзац починається на попередній сторінці
\index{iii2}{0213}  %% посилання на сторінку оригінального видання
інтереси кляси капіталістів і збагачення взагалі за конечну мету держави і прокламують
буржуазне суспільство протилежно старій надземній державі. Але разом
з цим виявляється свідомість того, що розвиток інтересів капіталу і кляси капіталістів,
капіталістичної продукції, зробився за базу національної сили і національної
переваги в сучасному суспільстві.

Далі, у фізіократів справедливе те, що на ділі вся продукція додаткової вартости,
отже, і ввесь розвиток капіталу, розглядуваний з боку природної бази,
ґрунтується на продуктивності хліборобської праці. Коли б люди взагалі не могли
продукувати протягом одного робочого дня більше засобів існування, отже, в вузькому
розумінні, більше хліборобських продуктів, ніж потрібно кожному робітникові
для його власної репродукції, — коли б денної витрати всієї його робочої сили
було досить лише для того, щоб випродукувати засоби існування, потрібні для
його особистого споживання, то взагалі не могло б бути мови ані про додатковий
продукт, ані про додаткову вартість. Продуктивність хліборобської праці, що
перебільшує індивідуальну потребу робітника, є база всякого суспільства, і насамперед
база капіталістичної продукції, яка дедалі більшу частину суспільства
відриває від продукції безпосередніх засобів існування і перетворює її, за висловом
Стюарта, в free heads\footnote*{
Дослівно вільні голови, тобто вільні робочі руки. \emph{Прим. Ред.}
}, дає можливість користатися нею в інших сферах.

Але, що сказати про тих новіших письменників-економістів, котрі як
Daire, Passy та інші, на схилі життя всієї клясичної економії, навіть на її смертельній
постелі, повторюють найпервісніші уявлення про природні умови додаткової
праці, і, отже, додаткової вартости взагалі, і гадають, ніби вони цим дають
щось нове і переконливе про земельну ренту після того як цю земельну ренту
вже давно описано як осібну форму і специфічну частину додаткової вартости?
Саме для вульгарної економії характеристичне таке: те, що на певнім пережитім ступені
розвитку було нове, оригінальне, глибоке і слушне, вона повторює в
такий час, коли воно є тривіяльне, відстале і фалшиве. Вона визнає таким
чином, що в неї не має навіть передчуття про проблеми, які цікавили клясичну
економію. Вона сплутує їх з питаннями, що могли ставитись лише на нижчому
ступені розвитку буржуазного суспільства. Так само стоїть справа з її безнастанним
та самозадоволеним пережовуванням фізіократичних засад про вільну
торгівлю. Ці засади давно втратили всякий теоретичний інтерес, хоч би як
практично вони цікавили ту або іншу державу.

У власне натуральному господарстві, де хліборобський продукт зовсім не
вступає в процес циркуляці, або вступає в нього лише дуже незначна частина
цього продукту, і навіть лише порівняно незначна частка тієї частини продукту,
яка становить дохід земельного власника, — як наприклад, в багатьох
староримських лятифундіях, в віллах Карла Великого, а також (дивись Vincard,
Histoire du travail) в більшій чи меншій мірі протягом усього середньовіччя, —
продукт і додатковий продукт великих маєтків зовсім не складається тільки
з продуктів хліборобської праці. Він охоплює також і продукти промислової
праці. Домашня реміснича і мануфактурна праця, як допоміжна продукція
при хліборобстві, що становить базу, є умова того способу продукції, на якому
ґрунтується це натуральне господарство так у давній і середньовічній Европі, якще
до нашого часу — і в індійській громаді, де її традиційна організація ще не
зруйнована. Капіталістичний спосіб продукції цілком знищує це сполучення:
процес, який у великому маштабі можна вивчити особливо на прикладі Англії
за останню третину XVIII століття. Голови, що виросли у більш чи менш напівфевдальних
суспільствах, Гереншванд, наприклад, ще в кінці XVIII століття
вбачають у цьому відокремленні хліборобства від мануфактури одчайдушно сміливий
суспільний експеримент, незрозуміло ризикований спосіб існування. І навіть
\parbreak{}  %% абзац продовжується на наступній сторінці

\parcont{}  %% абзац починається на попередній сторінці
\index{iii2}{0214}  %% посилання на сторінку оригінального видання
в тих хліборобських господарствах старовини, в яких виявляється найбільша
аналогія з капіталістичним сільським господарством, у Картагені та Римі,
навіть в них більше схожости з господарством плянтацій, ніж з формою, відповідною
до дійсного капіталістичного способу експлуатації\footnote{
А. Сміт показує, до якої міри за його часу (та й для нашого часу це має силу щодо
плантаторського господарства в тропічних та субтропічних країнах) рента й зиск ще не відокремились,
бо земельний власник є одночасно і капіталіст, як був, наприклад, Катон у своїх маєтках. Але таке
відокремлення є саме передумова капіталістичного способу продукції, що його поняттю до того ж
взагалі суперечить така база, як рабство.
}. Формальної аналогії,
— яка однак в усіх істотних пунктах виступає цілком облудною для того,
хто зрозумів капіталістичний спосіб продукції і хто не відкриває, як пан Момзен\footnote{
У своїй римській історії п. Момзен бере слово капіталіст зовсім не в розумінні сучасної економії
і сучасного суспільства, а в дусі популярної уяви, яка все ще розповсюджується не в Англії або в
Америці, а на континенті, яв старовинна традиція зниклих відносин.
}
капіталістичного способу продукції вже в усякому грошовому господарстві —
ми взагалі не знайдемо в старовину в континентальній Італії, хіба тільки в Сіцілії,
бо остання існувала як хліборобська країна, що виплачувала дань Римові,
і де тому хліборобство по суті провадилось на експорт. Тут трапляються орендарі
в сучасному розумінні.

Неправильне розуміння природи ренти ґрунтується на тій обставині, що
рента в натуральній формі з натурального господарства середньовіччя, і в цілковитій
суперечності умовам капіталістичного способу продукції, перейшла в новітній
час почасти у вигляді церковної десятини, почасти як дивовижність, увічнена
старовинними договорами. Через це здається, що рента виникає не з ціни
хліборобського продукту, а з маси продукту, тобто не з суспільних відносин,
а з землі. Вже давніш ми показали, що хоч додаткова вартість втілюється у
надпродукті, проте додатковий продукт в розумінні звичайного збільшення маси
продукту не втілює, навпаки, додаткової вартости. Він може являти собою мінус
вартости. Інакше бавовняна промисловість 1860 року проти 1840 мусила б
втілювати величезну додаткову вартість, тимчасом як ціна пряжі, навпаки, понизилась. В наслідок ряду
неврожайних років рента може зрости надзвичайно,
бо ціна збіжжя підвищується, хоч ця додаткова вартість втілюється в абсолютно
зменшеній масі подорожалої пшениці. Навпаки, в наслідок ряду урожайних
років рента може понизитись, бо ціна падає, хоч зменшена рента втілюється в
більшій масі порівняно дешевої пшениці. Тепер щодо ренти продуктами слід
насамперед зазначити, що вона являє собою просто традицію, перетягнуту з віджилого
способу продукції, що скніє як руїна останнього, і суперечність цієї
традиції з капіталістичним способом продукції виявляється в тому, що вона
сама собою зникає з приватних договорів, і що там, де могло втручитися законодавство,
як у випадку з церковними десятинами в Англії, вона була ґвалтовно
знесена як безглуздя. Але, подруге, там, де вона на базі капіталістичного
способу продукції і далі існує, вона була й могла бути не чим іншим, як
середньовічно замаскованим виразом грошової ренти. Хай, наприклад квартер
пшениці доходить до 40\shil{ шил.} Частина цього квартера мусить покрити заробітну
плату, що міститься в ньому, та мусить бути продана, щоб можна було знову
його витрачати; друга частина квартера мусить бути продана для того, щоб виплатити
частину податків, яка припадає на нього. Там, де капіталістичний
спосіб продукції розвинений, а з ним і поділ суспільної праці, насіння і навіть
частина добрива входять у репродукцію, як товари, отже, мусять бути куплені
для покриття; щоб здобути грошей на це, знов таки частина квартера мусить
бути продана. Оскільки ж їх в дійсності не доводиться купувати як товари,
а можуть бути взяті вони з самого продукту in natura, щоб знову
ввійти як умови продукції в його репродукцію, — як це трапляється не тільки
в хліборобстві, але і в багатьох галузях продукції, що продукують сталий
\parbreak{}  %% абзац продовжується на наступній сторінці

\parcont{}  %% абзац починається на попередній сторінці
\index{iii2}{0215}  %% посилання на сторінку оригінального видання
капітал, — то вони входять в рахунок, виражені як рахункові гроші, і віднімаються
як складові частини витрат продукції. Зношування машин і взагалі
основного капіталу доводиться покривати грішми. Нарешті, маємо зиск, який
обчислюється на суму цих витрат, що виражені в дійсних грошах або в рахункових.
Цей зиск втілюється в певній частині гуртового продукту, яка визначається
його ціною. А та частина, що залишається після цього, становить ренту.
Коли рента продуктами, встановлена контрактом, більша за цю. визначену ціною
рештку, то це вже буде не рента, а вирахування з зиску. Вже в наслідок самої
цієї можливості рента продуктами, що не відповідає ціні продукту, що, отже,
може становити і більше і менше, ніж дійсна рента і яка тому може становити
вирахування не тільки з зиску, але і з тих складових елементів, що ними покривається
капітал, — вже з самої цієї можливості вона становить архаїчну
форму. В дійсності ця рента продуктами, оскільки вона є рента не тільки з
назви, але й по суті, визначається виключно надміром ціни продукту над його
ціною продукції. Річ тільки в тому, що ця змінна величина припускається нею
як стала. Але ж це є таке узяте з минулого уявлення, що продукту in natura,
поперше, досить для того, щоб прохарчувати робітників, далі, дати капіталістичному
орендареві більше їжі, ніж йому потрібно, і що надмір над цим становить
природну ренту. Цілком так само, як коли фабрикант фабрикує \num{200.000}
ліктів ситцю. Цих ліктів досить для того, щоб не тільки одягти його робітників,
але й більше, ніж одягти його дружину і всіх його нащадків і його самого, залишити
крім того ситець на продаж і, нарешті, виплачувати ситцем величезну
ренту. Така собі звичайна річ! Досить тільки з \num{200.000} ліктів ситцю вирахувати
ціну їхньої продукції, і тоді мусить залишитися надмір ситцю, що становить
ренту. Наприклад, з \num{200.000} ліктів ситцю вирахувати ціну їхньої продукції
в \num{10.000}\pound{ ф. ст.}, не знаючи продажної ціни ситцю, з ситцю вирахувати
гроші, з споживної вартости як такої, вирахувати мінову вартість, і потім визначити
надмір ліктів ситцю над фунтами стерлінґів, — це дійсно наївна уява.
Це гірше, ніж квадратура круга, в основі якої принаймні лежить уява про
межі, що в них зливаються пряма лінія і крива. Але саме такий є рецепт п.
Passy. Вирахуйте гроші з ситцю, перше ніж у голові або в дійсності! ситець
перетворився на гроші! Надмір становить ренту, але вона має стати обмацальною
naturaliter\footnote*{
Лат., з самої своєї природи, природно. \emph{Пр.~Ред.}
} (див. напр. Карла Арнда), а не через «софістичну» чортівню!
До цього безглуздя, до вирахування ціни продукції з стількох-от шефелів пшениці,
до вирахування грошової суми з міри об’єму зводиться вся реставрація
натуральної ренти.

\subsubsection{Відробітна рента}

Коли розглядати земельну ренту в її найпростішій формі, у формі \emph{відробітної
ренти}, коли безпосередній продуцент частину тижня обробляє фактично
належну йому землю знаряддями праці (плуг, худоба, тощо), що фактично
або юридично належать йому ж, а інші дні тижня працює в маєтку землевласника,
для землевласника, задурно, то тут справа ще цілком ясна, рента і додаткова вартість
тут тотожні. Рента, а не зиск, — ось та форма, що в ній тут виражається неоплачена
додаткова праця. В якій мірі робітник (self sustaining serf)\footnote*{
Англ. раб, що сам себе утримує. \emph{Пр.~Ред.}
} може одержати тут
надмір над доконечними засобами свого існування, тобто надмір понад те, що
при капіталістичному способі продукції ми назвали б заробітною платою, це
залежить за інших незмінних умов від того відношення, в якому його робочий
час ділиться на робочий час для нього самого і панщизняний робочий час для
\parbreak{}  %% абзац продовжується на наступній сторінці

\parcont{}  %% абзац починається на попередній сторінці
\index{ii}{0216}  %% посилання на сторінку оригінального видання
часу обігу, а разом з тим і часу обороту, виділюється в формі грошового
капіталу \sfrac{1}{9} частина авансованого капіталу \deq{} 100\pound{ ф. стерл.} і коли
ці 100\pound{ ф. стерл.} складаються з 20\pound{ ф. стерл.} періодично надлишкового
грошового капіталу, призначеного для виплати щотижневої заробітної
плати, і з 80\pound{ ф. стерл.}, що існують як періодичний надлишковий тижневий
продукційний запас, — то цьому зменшенню у фабриканта надлишкового
продукційного запасу на 80\pound{ ф. стерл.} відповідає збільшення товарового
запасу у торговця бавовною. Та сама бавовна то довше лежить
на його складах як товар, що менше лежить вона на складах у фабриканта
як продукційний запас.

Досі ми припускали, що скорочення часу обігу в підприємстві $X$ випливає
з того, що $X$ швидше продає свої товари або швидше одержує
за них гроші, зглядно, що при кредиті термін виплати скорочується.
Отже, це скорочення часу обігу випливає з швидкого продажу товарів,
швидкого перетворення товарового капіталу на грошовий, з $Т' — Г'$, з
першої фази процесу циркуляції. Воно могло б випливати й з другої фази,
$Г — Т$, а тому й з одночасної зміни, чи то робочого періоду, чи то часу
обігу капіталів $Y$, $Z$, і~\abbr{т. ін.}, що постачають капіталістові $X$ продукційні
елементи його поточного капіталу.

Коли, напр., бавовна, вугілля та ін., в старих умовах транспорту перебувають
8 тижні в дорозі від місця продукції або від складів до місця
підприємства капіталіста $X$, то мінімуму продукційного запасу $X$ мусить
вистачати принаймні на 3 тижні, поки надійдуть нові запаси. Поки
бавовна та вугілля перебувають в дорозі, вони не можуть служити як
засоби продукції. Вони скоріше становлять тоді предмет праці для транспортової
промисловости й приміщеного в ній капіталу, а також товаровий
капітал для вуглепродуцента або для продавця бавовни, товаровий капітал,
що перебуває в своїй циркуляції. При поліпшеному транспорті час
перевозу скорочується до 2 тижнів. Таким чином, продукційний запас може
перетворитися з тритижневого на двотижневий. Разом з тим звільняється
авансований на це додатковий капітал у 80\pound{ ф. стерл.}, а також 20\pound{ ф. стерл.}, призначені на заробітну плату, бо капітал у 600\pound{ ф. стерл.},
що обернувся, повертається на тиждень раніше.

З другого боку, коли, напр., робочий період капіталу, що постачає
сировинний матеріял, скорочується (приклади про це подано в попередніх
розділах), отже, зростає й можливість відновлювати сировинний матеріял,
то продукційний запас може зменшитись, переміжок від одного періоду
відновлення до другого може скоротитись.

Навпаки, коли час обігу, а тому й період обороту довшає, то потрібне
авансування додаткового капіталу — з кишені самого капіталіста,
коли в нього є додатковий капітал. Але цей капітал є в тій
або іншій формі приміщений, як частина грошового ринку; щоб ним
можна було порядкувати, його треба визволити з старої форми, напр.,
продати акції, взяти вклади, так що й тут постає посередній вплив на
грошовий ринок. Або капіталіст мусить десь позичити додатковий капітал.
Щождо частини додаткового капіталу, потрібної для заробітної плати, то
\parbreak{}  %% абзац продовжується на наступній сторінці

\parcont{}  %% абзац починається на попередній сторінці
\index{iii2}{0217}  %% посилання на сторінку оригінального видання
специфічної форми держави. Це не перешкоджає тому, що та сама економічна
база — та сама з боку головних умов — в наслідок безконечно різних емпіричних
обставин, природних умов, расових відносин, історичних впливів, що діють
зовні тощо, може показувати у своєму вияві безконечні варіації і ґрадації, що
їх зрозуміти можливо лише з допомогою аналізи цих емпірично даних обставин.

Щодо відробітної ренти, найпростішої і найпервісної форми ренти, то очевидно
таке: рента є тут первісною формою додаткової вартости й збігається з
нею. Але збіг додаткової вартости з неоплаченою чужою працею не потребує
тут дальшої аналізи, тому що вона існує тут ще в своїй очевидній, обмацальній
формі, бо праця безпосереднього продуцента на самого себе тут ще відокремлена
в просторі і часі від його праці на земельного власника, а остання виступає
безпосередньо в грубій формі примусової праці на другу особу. Так само «властивість»
землі давати ренту сходить тут до обмацально розкриваної таємниці, бо до
природи, що дає ренту, належить також прикріплена до землі людська робоча сила,
і відносини власности, які примушують власника робочої сили напружувати і витрачати
її поза межами того, що потрібно було б для задоволення його власних
доконечних потреб. Рента становить беспосереднє привласнення земельним
власником цієї надмірної витрати робочої сили, бо крім цього безпосередній
продуцент не виплачує йому жодної ренти. Тут, де не тільки тотожні додаткова
вартість і рента, але додаткова вартість ще обмацально має форму додаткової праці,
цілком ясно виступають і природні умови або межі ренти, бо це є природні
умови і межі додаткової праці взагалі. Беспосередній продуцент
мусить 1)~мати достатню робочу силу і 2)~природні умови його праці, отже,
в першу чергу оброблюваної землі, мусять бути досить сприятливі, одним
словом, природна продуктивність його праці мусить бути досить велика для
того, щоб у нього лишалася можливість витрачати надмірну працю понад працю
потрібну для задоволення його власних доконечних потреб. Ця можливість не
створює ренти, її створює лише примус, що перетворює можливість на дійсність.
Але сама можливість зв’язана з суб’єктивними й об’єктивними природними
умовами. В цьому теж немає рішуче нічого таємничого. Коли робоча сила незначна
і природні умови праці мізерні, то додаткова праця незначна, але в
такому випадку незначні, з одного боку, потреби продуцентів, з другого боку,
відносна кількість визискувачів додаткової праці, і, нарешті, незначний додатковий
продукт, що в ньому реалізується ця мало продуктивна додаткова праця для
цього відносно незначного числа визискувачів-власників.

Нарешті, при відробітній ренті ясно само собою, що, за інших незмінних
умов, від відносних розмірів додаткової або панщинної праці цілком залежить,
в якій мірі в безпосереднього продуцента з’явиться можливість поліпшувати
своє власне становище, збагачуватися, продукувати певний надмір понад доконечні
засоби існування, або, коли ми антиципуємо капіталістичний спосіб виразу,
чи з’явиться у нього і в якій мірі можливість продукувати хоч би якийсь
зиск для себе самого, тобто надмір над його заробітною платою, продукуваною
ним самим. Рента тут нормальна, всежеруща, так би мовити, законна форма
додаткової праці: вона далека від того, щоб становити надмір над зиском, тобто
далека від того, щоб бути тут за надмір над якимось іншим надміром понад
заробітну плату; тут не тільки розмір такого зиску, але й саме його існування
залежить, за інших незмінних умов, від розміру ренти, тобто додаткової праці,
примусово виконуваної для власника.

Деякі історики висловили своє здивування перед тим, що хоч беспосередній
продуцент не є власник, а лише посідач, і вся його додаткова праця de jure
дійсно належить земельному власникові, — що за цих умов взагалі може відбуватись
самостійний розвиток майна і, кажучи відносно, багатства у зобов’язаних
\parbreak{}  %% абзац продовжується на наступній сторінці

\parcont{}  %% абзац починається на попередній сторінці
\index{iii2}{0218}  %% посилання на сторінку оригінального видання
панщиною або крепаків. Тимчасом ясно, що при тому примітивному і нерозвиненому
стані, на якому ґрунтується це суспільне продукційне відношення і
відповідний йому спосіб продукції, традиція мусить відігравати переважну ролю.
Далі ясно, що тут, як і всюди, переважна частина суспільства заінтересована
в тому, щоб усвячувати суще, як закон, і ті його межі, які дано звичаєм і
традицією, фіксувати як законні. Проте, лишаючи все інше осторонь, це стається
само собою, скоро постійна репродукція бази сущого стану, відношення,
що лежить в його основі, набуває з перебігом часу уреґульованої і упорядкованої
форми; і ця уреґульованість і цей порядок сами є доконечний момент всякого
способу продукції, коли він має набути суспільної сталости і незалежности від
звичайного випадку або сваволі. Уреґульованість і порядок є саме форма суспільного
зміцнення даного способу продукції, і тому його відносної емансипації
від просто сваволі і звичайного випадку. Він досягає цієї форми при застійному
стані так процесу продукції, як і відповідних до нього суспільних відносин
через просту повторну репродукцію їх самих. Коли ця форма проіснувала протягом
певного часу, вона зміцнюється, як звичай і традиція, і нарешті усвячується
як виразний закон. А що форма цієї додаткової праці, панщинна праця,
ґрунтується на нерозвиненості всіх суспільних продуктивних сил праці, на
примітивності самого способу праці, то і мусить вона природно віднімати у
безпосереднього продуцента незрівняно меншу відповідну частину всієї праці,
ніж за розвинених способів продукції й особливо за капіталістичної продукції.
Припустімо, наприклад, що панщинна праця на земельного власника первісно становила
два дні на тиждень. Ці два дні панщинної праці на тиждень таким чином
усталились, вони є стала величина, законно уреґульована звичаєвим або писаним
правом. Але продуктивність решти днів тижня, що ними може порядкувати
сам безпосередній продуцент, є величина змінна, яка мусить розвиватися в
процесі його досвіду, — цілком так само, як нові потреби, з якими він знайомиться,
цілком так само як поширення ринку для його продукту, ростуча забезпеченість
порядкування для самого себе цією частиною своєї робочої сили,
підганятиме його до підвищеного напруження робочої сили, при чому не слід
забувати, що вживання цієї робочої сили зовсім не обмежується хліборобством,
але охоплює й сільську домашню промисловість. Тут дана можливість певного
економічного розвитку, зрозуміла річ, залежно від більш або менш сприятливих
обставин, від природженого расового характеру тощо.

\subsection{Рента продуктами}

Перетворення відробітної ренти на ренту продуктами, економічною мовою
висловлюючись, нічого не змінює в суті земельної ренти. Суть земельної ренти при
таких умовах, які ми розглядаємо тут, в тому, що земельна рента є однісінька
панівна і нормальна форма додаткової вартости, або додаткової праці; а це в
свою чергу виражається в тому, що вона становить однісіньку додаткову працю
або однісінький додатковий продукт, що його безпосередній продуцент, який
\emph{посідає} умови праці, що потрібні для його власної репродукції, повинен
дати \emph{власникові} такої умови праці, яка в цьому стані охоплює все, тобто
власникові землі; і що з другого боку тільки земля і протистоїть йому, як
умова праці, що перебуває в чужій власності, відокремлена проти нього і
персоніфікована у земельному власникові. Коли рента продуктами становить
панівну і найрозвиненішу форму земельної ренти, вона все ж постійно в більшій
або меншій мірі супроводиться рештками попередньої форми, тобто ренти, що
її виплачується безпосередньо працею, отже, панщинною працею, і це однаково,
чи є земельним власником приватна особа чи держава. Рента продуктами має
своєю передумовою вищий культурний рівень безпосереднього продуцента, отже
\parbreak{}  %% абзац продовжується на наступній сторінці

\parcont{}  %% абзац починається на попередній сторінці
\index{iii2}{0219}  %% посилання на сторінку оригінального видання
вищий ступінь розвитку його праці і суспільства взагалі; і відрізняється вона
від попередньої форми тим, що додаткову працю доводиться виконувати вже
не в її натуральному вигляді, а тому вже не під безпосереднім наглядом і
примусом земельного власника, або його представника; навпаки, безпосередній
продуцент повинен виконувати її на свою власну відповідальність, примушуваний
силою відносин замість безпосереднього примусу, і постановою закону замість
нагая. Додаткова продукція, в розумінні продукції понад доконечні потреби
безпосереднього продуцента, і продукція на фактично йому самому належному
полі продукції, ним самим експлуатованій землі, замість продукції в панському
маєтку біля і поза своїм, як було давніш, стали тут уже само собою зрозумілим
правилом. При цих відносинах безпосередній продуцент більш або менш
порядкує застосуванням усього свого робочого часу, хоч частина цього робочого
часу, первісно майже вся надмірна частина його, як і давніш даром належить
земельному власникові; ріжниця тільки в тому, що останній уже не
одержує його безпосередньо в його власній і натуральній формі, а одержує в
натуральній формі того продукту, в якому цей час реалізується. Обтяжливі і
залежно від реґулювання панщинної праці більш або менш перешкідні перерви,
зумовлювані працею на земельного власника (порівн. книга перша, розд. VIII, 2,
фабрикант і маґнат) відпадають, коли рента продуктами є в чистому вигляді
або зводиться, принаймні, до нечисленних коротких перерв протягом року,
коли поряд з рентою продуктами й далі тривають певні панщини. Праця продуцента
на самого себе і його праця на земельного власника обмацально
вже більше не відокремлюються в часі і просторі. Ця рента продуктами в її
чистому вигляді, хоч її уламки можуть доходити до розвиненіших способів
продукції і продукційних відносин, як і давніш, має своєю передумовою натуральне
господарство, тобто припускає, що умови господарювання цілком або в
переважній частині продукуються в самому господарстві, покриваються і репродукуються
безпосередньо з його гуртового продукту. Далі, вона має своєю передумовою
сполучення сільської домашньої промисловости з хліборобством; додатковий
продукт, що створює ренту, є продукт цієї об’єднаної хліборобсько-промислової
родинної праці, однаково, чи має в собі рента продуктами в більшій або
меншій мірі промислові продукти, як це часто було за середньовіччя, чи вона
виплачується лише в формі власне хліборобського продукту. При цій формі
ренти, рента продуктами, що в ній втілюється додаткова праця, ніяк не потребує
того, щоб вичерпувалось всю надмірну працю сільської родини. Навпаки, продуцентові
дається тут, порівняно з відробітною рентою, більшу волю для того, щоб
здобути час для надмірної праці, продукт якої належить йому самому, цілком так
само, як продукт його праці, що задовольняє його доконечні потреби. Так само
разом з цією формою постають більші ріжниці в економічному становищі окремих
безпосередніх продуцентів. Принаймні, є можливість для цього, а також та
можливість, що цей безпосередній продуцент здобуде засоби для того, щоб і
собі безпосередньо визискувати чужу працю. Проте, тут, де ми розглядаємо
чисту форму ренти продуктами, це нас не стосується; як і взагалі ми не можемо
розглядати безконечно різних комбінацій, в яких різні форми ренти
можуть сполучатися, фалшуватися і з’єднуватися. Через те, що ця форма
ренти, рента продуктами, зв’язана з певним характером продукту і самої продукції,
через доконечне для неї сполучення сільського господарства і домашньої
промисловости, через те, що з нею сільська родина набуває майже цілком
самодостатьного характеру, через її незалежність від ринку, від продукційного
і історичного руху частини суспільства, що стоїть поза нею, коротко кажучи,
через характер натурального господарства взагалі, ця форма цілком придатна
для того, щоб бути за базу застійних станів суспільства, як це ми спостерігаємо,
наприклад, в Азії. Тут, як і при найдавнішій формі відробітної ренти,
\parbreak{}  %% абзац продовжується на наступній сторінці

\parcont{}  %% абзац починається на попередній сторінці
\index{iii2}{0220}  %% посилання на сторінку оригінального видання
земельна рента є нормальною формою додаткової вартости, а тому і додаткової
праці, тобто всієї надмірної праці, яку безпосередній продуцент мусить даром,
отже, на ділі примусово, виконувати на власника найістотнішої умови його
праці, на власника землі, — хоч цей примус уже не протистоїть йому в старій
брутальній формі. Зиск, — коли ми, фалшиво антиципуючи, назвемо так той
дріб надміру його праці над потрібного працею, що він його привласнює самому
собі, — до такої міри не має визначального впливу на ренту продуктами, що
радше можна було б сказати, що він виростає за спиною останньої і має свою
природну межу в розмірі ренти продуктами. Остання може досягати такого
розміру, що є поважною загрозою репродукції умов праці, самих засобів продукції,
більш або менш унеможливлює поширення продукції і знижує задоволення
потреб безпосереднього продуцента до фізичного мінімуму засобів
існування. Так буває саме в тому випадку, коли цю форму знаходить готового
і починає експлуатувати торговельна нація-завойовник, як, наприклад, англійці
в Індії.

\subsubsection{Грошова рента}

Під грошовою рентою ми розуміємо тут — на відзнаку від промислової
або комерційної земельної ренти, що ґрунтується на капіталістичному способі
продукції і становить лише надмір над пересічним зиском, — земельну ренту,
що виникає з простого перетворення форми ренти продуктами, так само, як ця
остання сама була лише перетвореною відробітною рентою. Замість продукту
безпосередній продуцент має тут виплачувати власникові землі (чи то буде
держава, чи приватна особа) ціну продукту. Отже, надміру продукту в його
натуральній формі вже не досить, його мусять перетворити з цієї натуральної
форми на грошову форму. Хоч безпосередній продуцент, як і давніш, продовжує
продукувати сам, принаймні, більшу частину своїх засобів існування, проте,
частина його продукту мусить тепер бути перетворена на товар, продукуватися
як товар. Отже, характер всього способу продукції більш або менш змінюється.
Він втрачає свою незалежність, свою відокремленість від зв'язку з суспільством.
Відношення витрат продукції, в які тепер входять в більшій чи меншій мірі і
грошові витрати, стає за вирішальне; в усякому разі стає вирішальним
надмір тієї частини гуртового продукту, що її треба перетворити на гроші
над тією частиною, яка, з одного боку, мусить стати знову засобом репродукції
і, з другого боку, безпосереднім засобом існування. А проте, база цього
роду ренти, хоч і наближається до свого розпаду, все ще лишається та сама,
що і при ренті продуктами, яка становить вихідний пункт. Безпосередній продуцент
є, як і давніш, спадковий або інакше традиційний посідач землі, який
повинен виплачувати земельному власникові, як власникові цієї найістотнішої
умови його продукції, надмірну примусову працю, тобто неоплачену, виконувану
без еквівалента працю, в формі додаткового продукту, перетвореного на
гроші. Власність на умови праці, відмінні від землі, хліборобське знаряддя та
інше рухоме майно спочатку фактично, а потім й юридично, перетворюється на
власність безпосередніх продуцентів вже за попередніх форм, і ще більше доводиться
припускати це для такої форми, як грошова рента. Спочатку спорадичне,
потім відбуваючись більш або менш у національному маштабі, перетворення
ренти продуктами на грошову ренту, має своєю передумовою вже порівняно
значний розвиток торгівлі, міської промисловости товарової продукції взагалі, а
разом з тим і грошової циркуляції. Далі воно має своєю передумовою ринкову
ціну продуктів, і те, що вони продаються більш або менш близько до
своєї вартости, чого може і не бути за колишніх форм. На Сході Европи ми
можемо почасти ще на власні очі спостерігати процес цього перетворення.
\parbreak{}  %% абзац продовжується на наступній сторінці

\parcont{}  %% абзац починається на попередній сторінці
\index{iii1}{0221}  %% посилання на сторінку оригінального видання
виміряння додаткової вартості, — а це робиться при всякому
обчисленні зиску, — то взагалі відносне падіння додаткової вартості
і її абсолютне падіння є тотожні. Норма зиску в наведених
вище випадках знижується з 40\% до 30\% і до 20\%, бо в дійсності
маса додаткової вартості, а тому й зиску, вироблена тим
самим капіталом, падає абсолютно з 40 до 30 і до 20. Через те
що величина вартості капіталу, відносно якої вимірюється додаткова
вартість, є дана, = 100, то зменшення відношення додаткової
вартості до цієї незмінної величини може бути тільки іншим
виразом зменшення абсолютної величини додаткової вартості
й зиску. Справді, це — тавтологія. Але те, що таке зменшення
настає, випливає, як уже було показано, з природи розвитку
капіталістичного процесу виробництва.

Але, з другого боку, ті самі причини, які викликають абсолютне
зменшення додаткової вартості, а тому й зиску на даний
капітал, а тому також і обчислюваної в процентах норми зиску,
ці самі причини приводять до зростання привласнюваної суспільним
капіталом (тобто сукупністю капіталістів) абсолютної маси
додаткової вартості, а тому й зиску. Як же це мусить виразитись,
як це може виразитись, або які умови передбачаються
і цією позірною суперечністю?

Якщо кожна відповідна частина, = 100, суспільного капіталу,
отже, кожні 100 капіталу пересічного суспільного складу, є величина
дана, і тому для неї зменшення норми зиску збігається
із зменшенням абсолютної величини зиску саме через те, що
тут капітал, яким вони вимірюються, є величина стала, то,
навпаки, величина сукупного суспільного капіталу, як і капіталу,
який знаходиться в руках окремих капіталістів, є змінна
величина, яка, щоб відповідати припущеним умовам, мусить
змінюватись у зворотному відношенні до зменшення своєї змінної
частини.

В попередньому прикладі, при процентному складі капіталу
в $60c + 40v$, додаткова вартість або зиск на капітал був 40,
а тому й норма зиску була 40\%. Припустім, що при цій висоті
складу сукупний капітал становив один мільйон. В такому разі
сукупна додаткова вартість, а тому й сукупний зиск становив
\num{400000}. Якщо потім склад буде = $80c + 20v$, то при незмінному
ступені експлуатації праці додаткова вартість, або зиск, на кожні
100 = 20. Але через те що додаткова вартість, або зиск, як ми
показали, щодо своєї абсолютної маси зростає, незважаючи на цю
падаючу норму зиску або дедалі менше створення додаткової
вартості кожною сотнею капіталу, — наприклад, зростає, скажімо,
з \num{400000} до \num{440000}, — то це можливе тільки тому, що сукупний
капітал, який утворився одночасно з цим новим складом, зріс
до \num{2200000}. Маса приведеного в рух сукупного капіталу зросла
до 220\%, тимчасом як норма зиску впала на 50\%. Коли б капітал
тільки подвоївся, то при нормі зиску в 20\% він міг би
виробити тільки таку саму масу додаткової вартості й зиску,
\parbreak{}  %% абзац продовжується на наступній сторінці


\index{i}{0222}  %% посилання на сторінку оригінального видання
Однак фабриканти дозволили такий «проґрес» не без компенсації
його «реґресом». За їхньою спонукою Палата громад скоротила
мінімальний вік дітей, яких можна було експлуатувати,
з 9 до 8 років, щоб забезпечити для капіталу «додаткове постачання
фабричних дітей»\footnote{
«Що скорочення годин їхньої праці спричиниться до збільшення
числа потрібних для праці (дітей), то вирішено, що додаткове постачання
дітей од 8 до 9 років завстаршки може покрити збільшений
попит» («As a reduction in their hours of work would cause a large number
(of children) to be employed, it was thought that the additional supply of
children from eight to nine years of age, would meet the increased demand»)
(1. c., p. 13).
} — річ, належна йому на основі всіх
божих і людських прав.

\looseness=-1
Роки 1846 і 47 становлять епоху в економічній історії Англії.
Скасовано хлібні закони, скасовано мито на довіз бавовни й
інших сировинних матеріялів, проголошено волю торговлі за
провідну зірку законодавства! Словом, наставало тисячолітнє
царство. З другого боку, чартистський рух і агітація за десятигодинний
робочий день дійшли цими роками свого найвищого
пункту. Вони знайшли собі спільників у торі, що палали помстою.
Не зважаючи на фанатичний опір зрадливої армії вільної торговлі
з Брайтом і Кобденом на чолі, біла про десятигодинний робочий
день, об’єкт такої довгочасної боротьби, парлямент ухвалив.

Новий фабричний закон з 8 червня 1847~\abbr{р.} ухвалив, що з
1 липня 1847~\abbr{р.} набирає сили попереднє скорочення робочого дня
до 11 годин для «підлітків» (від 13 до 18 років) і для всіх робітниць,
а 1 травня 1848~\abbr{р.} — остаточне обмеження робочого
дня 10 годинами для тих самих категорій. Щодо решти, то цей
закон був лише виправленим додатком до законів 1833~\abbr{р.} і 1844~\abbr{р.}

Капітал розпочав попередній похід з тим, щоб не допустити
до повного проведення в життя закону з 1 травня 1848~\abbr{р.} І власне
сами робітники, нібито навчені досвідом, повинні були допомогти
знову зруйнувати свою власну справу. Момент було обрано влучний.
«Треба собі пригадати, що в наслідок страшної кризи 1846 —
1847~\abbr{рр.} серед фабричних робітників панували великі злидні,
бо багато фабрик працювало лише неповний час, інші зовсім не
працювали. Тому значне число робітників перебувало в найскрутнішому
стані, багато було в боргах. Тим то з досить великою
певністю можна було припустити, що вони ладні будуть згодитися
на довший робочий час, щоб поповнити колишні втрати, сплатити,
може, борги, або викупити з заставничих домів свої меблі, або
змінити на нове продане майно, або придбати нову одежу собі й
своїй родині»\footnote{
«Reports of Insp. of Fact, for 31 st October 1848», p. 16.
}. Пани фабриканти старалися збільшити природний
вплив цих обставин загальним зниженням заробітної плати
на 10\%. Це мало бути, так би мовити, посвятинами нової доби
вільної торговлі. Потім, скоро тільки робочий день скорочено
до 11 годин, наступило дальше пониження заробітної плати на
8\sfrac{1}{3}\% і нове подвійне зниження, скоро тільки робочий день остаточно
скорочено було до 10 годин. Тому повсюди, де це дозволяли
\parbreak{}  %% абзац продовжується на наступній сторінці

\parcont{}  %% абзац починається на попередній сторінці
\index{iii2}{0223}  %% посилання на сторінку оригінального видання
цього зиску визначає ренту, а навпаки, сам він визначається рентою як своєю
межею. Висока норма зиску за середньовіччя завдячує своєю висотою не тільки
низькому складові капіталу, що в ньому переважає змінний, витрачуваний на
заробітну плату елемент. Вона завдячує своєю висотою гнобленню села, привласненню
частини ренти земельного власника і доходу його підлеглих. Коли за
середньовіччя село визискує місто політично, всюди де февдалізм не був зламаний
виключним розвитком міст, як в Італії, то місто всюди і без винятків
визискує село економічно своїми монопольними цінами, своєю системою податків,
своїм цеховим ладом, своїм безпосереднім купецьким обманом і своїм
лихварством.

\looseness=1
Можна було б думати, що проста поява капіталістичного орендаря в сільсько\dash{}господарській
продукції дає доказ того, що ціна хліборобських продуктів,
які здавна в тій чи іншій формі виплачували ренту, мусить стояти вище, ніж
ціни продукції мануфактури, принаймні, за доби цієї появи; чи тому, що
вона досягла рівня монопольної ціни, чи тому, що вона підвищилась до
рівня вартости хліборобських продуктів, а їхня вартість в дійсності вища за
ціну продукції, реґульовану пересічним зиском. Бо, коли б цього не було, то
капіталістичний орендар за наявних цін хліборобських продуктів не міг би
спочатку реалізувати з ціни цих продуктів пересічний зиск, а потім з цієї
самої ціни ще виплатити в формі ренти надмір над цим зиском. З цього можна
було б зробити той висновок, що загальна норма зиску, яка характеризує капіталістичного
орендаря в його контракті з земельним власником, створилась без
долучення ренти і тому вона, починаючи відігравати реґуляційну ролю в сільському
господарстві, знаходить цей надмір готовим і виплачує його земельному
власникові. Таким традиційним способом пояснює собі справу, наприклад,
п. Родбертус. Але:

\emph{Поперше}. Цей вступ капіталу як самостійної і керівної сили в хліборобство
відбувається не разом і не всюди, а поступово і в окремих галузях продукції.
Він захоплює спочатку не власне хліборобство, а такі галузі продукції,
як скотарство, особливо вівчарство, що його головний продукт, вовна, з піднесенням
промисловости дає спочатку сталий надмір ринкової ціни над ціною
продукції, причому ці ціни лише згодом вирівнюються. Так було в Англії протягом
XVI століття.

\emph{Подруге}. Тому що ця капіталістична продукція спочатку постає лише
спорадично, то нічого не можна заперечити проти припущення, що вона спочатку
опановує лише такі комплекси земель, які в наслідок своєї специфічної
родючости, або в наслідок особливо сприятливого положення, в цілому можуть
виплачувати диференційну ренту.

\emph{Потретє}. Припустімо навіть, що ціни хліборобського продукту при
появі цього способу продукції, — що в дійсності припускає зріст значення міського
попиту, — були вищі, ніж ціна продукції, як це без усякого сумніву було,
наприклад, за останньої третини XVII століття в Англії, то тоді, — скоро цей
спосіб продукції до певної міри виб’ється з простого упідлеглення хліборобства
капіталові, і скоро постане доконечне зв’язане з його розвитком поліпшення
в хліборобстві і пониження витрат продукції, — відбудеться процес вирівняння
в наслідок реакції, певного пониження ціни хліборобських продуктів, як це
було в першій половині XVIII століття в Англії.

Отже, цим традиційним способом не можна пояснити ренту, як надмір
над пересічним зиском. Хоч би в яких історично даних умовах рента з’явилась
спочатку, — скоро тільки вона пустила коріння, — вона може існувати вже
лише в вище викладених сучасних умовах.

На закінчення, щодо перетворення ренти продуктами на грошову ренту,
слід ще зауважити, що разом з цим стає істотним моментом капіталізована
\parbreak{}  %% абзац продовжується на наступній сторінці

\parcont{}  %% абзац починається на попередній сторінці
\index{iii2}{0224}  %% посилання на сторінку оригінального видання
рента, ціна землі, а тому і її відчужуваність і відчуження, і що тому не тільки
колишні зобов’язані до виплати ренти можуть перетворитись на незалежних
селян-власників, але й міські і інші посідачі грошей можуть купувати дільниці
землі для того, щоб здавати їх в оренду або селянам або капіталістам
і користуватись рентою як формою проценту на свій в такий спосіб приміщений
капітал; отже, що і ця обставина сприяє перетворенню колишнього способу
експлуатації, відносин між власником і дійсним обробником, а також самої ренти.

\subsubsection{Відчастинне (métairie)\footnote*{
Фр. основне значіння — маєток, хутір, фарма, в даному разі мовиться про відчастинне
господарство (рос. издольное). \Red{Пр.~Ред.}
} господарство і селянська парцелярна
власність}

Тут ми підійшли до кінця нашого ряду розвитку форм земельної ренти.

В усіх цих формах земельної ренти: відробітної ренти, ренти продуктами,
грошової ренти (як просто перетвореної форми ренти продуктами) дійсним
обробником і посідачем землі завжди припускається виплатник ренти, що його
неоплачена додаткова праця безпосередньо йде власникові землі. Це не тільки
можливо, але воно дійсно так і е, навіть при останній формі, при грошовій
ренті, — оскільки вона є в чистому вигляді, тобто як просто перетворена форма
ренти продуктами.

Як переходову форму від первісної форми ренти до капіталістичної ренти
можна розглядати métairie système, або систему відчастинного господарства, за якого
обробник (орендар) крім своєї праці (власної або чужої) дає частину капіталу
для господарювання, а земельний власник дає крім землі іншу частину потрібного
для господарювання капіталу (напр., худобу), і продукт ділиться в певних,
різних для різних країн пропорціях поміж орендарем та земельним власником.
З одного боку, в орендаря тут немає достатнього капіталу для цілковитого капіталістичного
господарювання. З другого боку, та частина, яку одержує тут
земельний власник, не є чиста форма ренти. В дійсності в ній може бути процент
на авансований земельним власником капітал і надмірна рента. Вона може
в дійсності також поглинути всю додаткову працю орендаря, або лишити йому
більшу або меншу частину цієї додаткової праці. Але істотне є в тому, що
рента тут уже більш не виступає, як нормальна форма додаткової вартости взагалі.
На одному боці орендар, чи вживає він тільки власної, чи також і чужої праці,
має домагання на певну частину продукту не тому, що він робітник, а тому,
що він посідач частини знарядь праці, капіталіст сам собі. На другому боці
земельний власник домагається своєї частини, ґрунтуючись не виключно на
своїй власності на землю, але як і позикодавець капіталу\footnote{
Порівн. Buret, Tocqueville, Sismondi.
}.

Рештки старовинної громадської власности на землю, що збереглись після
переходу до самостійного селянського господарства, наприклад, у Польщі та
Румунії, були там за привід для того, щоб здійснити перехід до нижчих форм
земельної ренти. Частина землі належить поодиноким селянам і вони обробляють
її самостійно. Друга частина обробляється спільно і створює додатковий
продукт, який придається почасти для покриття витрат громади, почасти як
резерв на випадок неврожаїв тощо. Ці дві останні частини додаткового продукту,
а кінець-кінцем і весь додатковий продукт, разом з землею, на якій він виростає,
помалу узурпується державними урядовцями і приватними особами,
і первісно вільні селяни-землевласники, що для них зберігається повинність
спільного обробітку цієї землі, перетворюються таким чином на панщанних,
або зобов’язаних до виплати ренти продуктами, тимчасом як узурпатори
\parbreak{}  %% абзац продовжується на наступній сторінці

\parcont{}  %% абзац починається на попередній сторінці
\index{iii2}{0225}  %% посилання на сторінку оригінального видання
громадської землі перетворюються на власників не тільки узурпованої ними
громадської землі, але й самих селянських дільниць.

Тут нам немає потреби докладніше спинятись на власне рабовласницькому
господарстві (яке теж проходить ряд ступенів від патріархальної системи, розрахованої
переважно на власне споживання, до власне плантаторської системи,
що працює на світовий ринок) і на поміщицькому господарстві, в якому земельний
власник провадить обробіток власним коштом, посідає всі знаряддя
продукції і визискує працю наймитів, невільних чи вільних, оплачуваних натурою
чи грішми. Земельний власник і власник знарядь продукції, а тому й
безпосередній визискувач робітників, що належать до числа цих елементів продукції,
тут збігаються. Так само збігаються рента і зиск, поділу різних форм
додаткової вартости не відбувається. Всю додаткову працю, яка тут втілюється
в додатковому продукті, безпосередньо здобуває з робітників власник усіх знарядь
продукції, до яких належить земля, а при первісній формі рабства і сами
безпосередні продуценти. Там, де панує капіталістичний спосіб уявлення, як
на американських плянтаціях, всю цю додаткову вартість розглядається як зиск;
там, де не існує самого капіталістичного способу продукції і куди ще не перенесено
відповідного йому способу уявлення з капіталістичних країн, вона
виступає як рента. В усякому разі, ця форма не являє жодних труднощів. Дохід
земельного власника, хоч би яку назву давали йому, привласнюваний ним
додатковий продукт, що ним можна порядкувати, є тут тією нормальною і панівною
формою, в якій безпосередньо привласнюється всю неоплачену додаткову
працю, і земельна власність створює базу цього привласнення.

Далі, \emph{парцелярна власність}. Селянин є тут одночасно і вільний власник
своєї землі, яка є головним знаряддям його продукції, конче потрібним полем
для застосування його праці та його капіталу. При цій формі не виплачується
жодної орендної плати: отже, рента не виступає як відокремлена форма додаткової
вартости, хоч в країнах, де в інших сферах розвинувся капіталістичний
спосіб продукції, вона, порівняно з іншими галузями продукції, виступає як
надзиск, але як такий надзиск, що припадає селянинові, як і взагалі ввесь здобуток
від його праці.

Ця форма земельної власности має своєю передумовою, що, як і при її
давніших старих формах, сільська людність має величезну чисельну перевагу
над міською, що, отже, хоч капіталістичний спосіб продукції взагалі й панує,
але відносно він лише мало розвинений, а тому і в інших галузях продукції
концентрація капіталів рухається в вузьких межах, переважає розпорошення
капіталів. По суті справи переважна частина сільсько-господарського продукту
тут мусить споживатись самими продуцентами його, селянами, як беспосередній
засіб існування, і лише надмір над цим може як товар брати участь у торгівлі
з містами. Хоч би як реґулювалась тут пересічна ринкова ціна хліборобського
продукту, диференційна рента, надмірна частина ціни товарів з кращих або
краще розташованих земель очевидно мусить тут існувати так само, як за
капіталістичного способу продукції. Навіть тоді, коли ця форма існує за таких
становищ суспільства, коли ще взагалі не розвинулась загальна ринкова ціна,
існує ця диференційна рента; вона виступає тоді у вигляді надмірного додаткового
продукту. Але потрапляє вона до кишені того селянина, що його праця
реалізується в сприятливіших природних умовах. Якраз за цієї форми, де ціна
землі для селянина входить як елемент в фактичні витрати продукції, чи тому,
що з дальшим розвитком цієї форми земля при поділі спадщини дістається замість
певної грошової вартости, чи тому, що при постійних переходах з рук
у руки всієї власности або її складових елементів, сам обробник купує землю,
причому гроші він здобуває здебільша під гіпотеку; де, отже, ціна землі є не
що інше, як капіталізована рента, наперед узятий елемент, і де тому здається,
\parbreak{}  %% абзац продовжується на наступній сторінці

\parcont{}  %% абзац починається на попередній сторінці
\index{iii2}{0226}  %% посилання на сторінку оригінального видання
що рента існує, незалежно від будь-яких ріжниць у родючості і положенні
землі, — якраз тут, у пересічному доводиться припускати, що абсолютної ренти
не існує, і що, отже, найгірша земля не дає жодної ренти; бо абсолютна рента
має своєю передумовою або реалізований надмір вартости продукту над його ціною
продукції, або надмірну монопольну ціну, що перевищує вартість продукту. А що
сільське господарство провадиться тут переважно як хліборобство за-для безпосередніх
засобів існування, і що земля становить для більшости людности
конче потрібне поле для застосування її праці і капіталу, то регуляційна ринкова
ціна продукту лише за виключних обставин досягне розмірів його вартости;
а вартість ця, в наслідок переваги елементу живої праці, буде, взагалі кажучи,
стояти вище, ніж ціна продукції, хоч цей надмір вартости над ціною продукції
в свою чергу буде обмежуватись тим, що в країнах парцелярного господарства
і нехліборобський капітал має низький склад. За межу експлуатації для парцелярного
селянина, з одного боку, не є пересічний зиск на капітал, оскільки
сам він є дрібний капіталіст, ані доконечність ренти, з другого боку, оскільки
він сам є земельний власник. За абсолютну межу для нього, як для дрібного
капіталіста, є лише заробітна плата, яку він виплачує сам собі по вирахуванні
власне витрат. Поки ціна продукту покриває заробітну плату для нього, він
оброблятиме свою землю, при чому він часто спускається до фізичного мінімуму
заробітної плати. Щодо нього як земельного власника, то для нього відпадає
межа, яку кладе власність, бо вона може виявитись лише в протилежність
відокремленому від неї капіталові (включаючи і працю), ставлячи перешкоду для
застосування його. Певна річ, що процент на ціну землі, який до того ж здебільша
доводиться виплачувати третій особі, гіпотечному кредиторові, становить
межу. Але цей процент може виплачуватися саме з тієї частини додаткової
праці, яка за капіталістичних відносин становила б зиск. Отже, рента, антиципована
в земельній ціні і в виплачуваному на неї проценті, не може бути чимось
іншим, як лише частиною капіталізованої додаткової праці селянина, понад
працею, потрібною для його існування, причому ця додаткова праця не реалізується
в частині вартости товару, рівній усьому пересічному зискові, і тим
паче вона не реалізується в надмірі над додатковою працею, в пересічному зиску
реалізованою, тобто в надзиску. Рента може становити вирахування з пересічного
зиску, або навіть однісіньку частину його, яка тільки й реалізується. Отже,
для того, щоб парцелярний селянин міг обробляти свою землю, або купити
землю для оброблення, немає потреби, як за нормального капіталістичного
способу продукції, в тому, щоб ринкова ціна хліборобського продукту піднеслась
так високо, щоб давати йому пересічний зиск, а тим паче надмір над цим
пересічним зиском, фіксований у формі ренти. Отже, немає потреби в тому,
щоб ринкова ціна підвищилась, або до рівня вартости, або до рівня ціни продукції
продукту селянина. Це є одна з причин, що пояснює, чому в країнах,
де панує парцелярна власність, ціна збіжжя стоїть нижче, ніж у країнах капіталістичного
способу продукції. Частину додаткової праці селян, що працюють
в найнейсприятливіших умовах, даром дається суспільству і не бере вона участи
в регулюванні цін продукції або в створенні вартости взагалі. Отже, ця порівняно
низька ціна є наслідок убозтва продуцентів, але ніяк не продуктивности
їхньої праці.

Ця форма вільної парцелярної власности селян, що сами господарюють, як
панівна, нормальна форма створює, з одного боку, економічну основу суспільства
за найкращих часів класичної давнини, а, з другого боку, ми подибуємо
її в сучасних народів як одну з форм, що постають з розпаду февдально,
земельної власности. Такі yeomanry\footnote*{
Селяни-землевласники. \Red{Пр.~Ред.}
} в Англії, селянський стан у Швеції
\parbreak{}  %% абзац продовжується на наступній сторінці


\index{i}{0227}  %% посилання на сторінку оригінального видання
Звичайно, всі ці викрути нічого не помогли. Фабричні інспектори
вдалися до суду. Але незабаром на міністра внутрішніх
справ сера Джорджа Грея спала така хмара петицій від фабрикантів,
що в обіжнику з 5 серпня 1848~\abbr{р.} він наказав інспекторам
«не позивати взагалі за порушення букви закону, поки не буде
доведене зловживання Relaissystem’ою з метою примусити підлітків
і жінок працювати понад десять годин». Після цього фабричний
інспектор Ф.~Стюарт дозволив так звану систему змін протягом
п’ятнадцятигодинного періоду фабричного дня для цілої Шотляндії,
де вона незабаром знов розцвіла, як колись. Навпаки,
англійські фабричні інспектори заявили, що міністер не має
жодної диктаторської влади припинити чинність закону, і далі
провадили судові переслідування проти Proslavery rebels.

Алеж нащо було притягати до суду, коли суди, county magistrates\footnote{
Ці «county magistrates», «great unpaid»\footnote*{величні неоплачувані.  \Red{Ред.} },
як їх називає В.~Кобе — це щось наче безплатні мирові судді, що їх обирають із почесних
осіб графства. В дійсності вони являють собою патримоніяльні суди панівних
кляс.
},
виправдували притягуваних до права? По цих судах
засідали пани фабриканти, щоб самих себе судити. Ось приклад.
Якийсь Іскрідж із бавовнопрядної фірми Кершоу, Лізе і К°
подав був фабричному інспекторові своєї округи схему Relaissystem,
призначену для його фабрики. Одержавши відмову,
він спочатку тримався пасивно. Декілька місяців пізніш якийсь
індивід, на ім’я Робінзон, теж бавовняник, і коли не П’ятниця,
то в усякому разі родич Іскріджа, став перед Borough Justices\footnote*{мировими суддями. \Red{Ред.}}
у Стокпорті, обвинувачуваний у тому, що завів у себе таку
систему змін, яку вигадав Іскрідж. Засідало четверо суддів,
серед них три бавовняні фабриканти, з тим самим неминучим
Іскріджем на чолі. Іскрідж виправдав Робінзона й заявив: що є
законоправне для Робінзона, те справедливе й для Іскріджа.
Покликаючись на свій власний судовий присуд, що набрав правної
сили, він зараз же завів цю систему й на своїй власній фабриці\footnote{
«Reports etc. for 30 th April 1849», p. 21, 22. Порівн. подібні
приклади там же, \stor{}4, 5.
}.
Певна річ, уже самий склад таких суддів був явним порушенням
закону\footnote{
Законом 1 і 2 Вільяма IV, с. 24, s. 10, відомим під назвою фабричного
закону сера Джона Гобговза, забороняється кожному посідачеві
бавовнопрядної або ткацької фабрики, а також і батькові, синові або
братові такого посідача виконувати обов'язки мирового судді в питаннях,
які стосуються до фабричного закону.
}. «Такі судові фарси, — каже інспектор Хоуелл, —
аж волають по ліки\dots{} або пристосуйте закон до таких присудів,
або віддайте вирішення справ не такому вже порочному трибуналові,
який свої присуди пристосував би до закону\dots{} в усіх таких
випадках. Дуже бажано, щоб посада судді була платна!»\footnote{
«Reports etc. for 30 th April 1849».
}

Коронні юристи проголосили фабрикантську інтерпретацію
закону 1848~\abbr{р.} за недоладну, але рятівники суспільства не дали
\parbreak{}  %% абзац продовжується на наступній сторінці

\parcont{}  %% абзац починається на попередній сторінці
\index{i}{0228}  %% посилання на сторінку оригінального видання
себе збити з пантелику. «Після того, — оповідає Леонард Горнер, —
як я спробував примусити виконувати закон, розпочавши 10 процесів
у 7 різних судових округах, і лише в одному випадку найшов
підтримку в суддів\dots{} я вважаю за некорисні дальші переслідування
за оминання закону. Та частина закону, що її укладено
з метою створити одностайність у робочих годинах\dots{} вже не існує
більше в Ланкашірі. Так само я абсолютно не маю, як і мої помічники,
ніяких засобів, щоб упевнитися, що по тих фабриках, де
панує так звана Relaissystem, підлітків і жінок не примушують
працювати більш як 10 годин. Наприкінці квітня 1849~\abbr{р.} вже
114 фабрик у моїй окрузі працювали за цією методою, і число
їх останніми часами швидко зростає. Загалом же вони працюють
тепер 13\sfrac{1}{2} годин, від шостої години ранку до пів на восьму вечора;
в деяких випадках вони працюють 15 годин, від пів на шосту
ранку до пів на дев’яту вечора»\footnote{
«Reports etc. for 30 th April 1849», p. 5.
}. Вже у грудні 1848~\abbr{р.} Леонард
Горнер мав список 65 фабрикантів і 29 фабричних доглядачів,
які одноголосно заявляли, що жодна система контролю не може
за такої системи змін перешкодити поширенню якнайінтенсивнішої
надмірної праці\footnote{
«Reports etc. for 31 st October 1849», p. 6.
}. То тих самих дітей і підлітків переводять
із прядільні до ткальні й~\abbr{т. д.}, то протягом 15 годин їх
кидають (shifted) з однієї фабрики до однієї\footnote{
«Reports etc. for 30 th April 1849», p. 21.
}. Як можна контролювати
таку систему змін, «яка зловживає словом зміна, щоб із
безмежною різноманітністю перемішувати робочі руки, як карти,
і день-у-день так пересовувати години праці й відпочинку
різних осіб, що один і той самий повний асортимент рук ніколи
не працює разом на тому самому місці в той самий час»!\footnote{
«Reports etc. for 1 st December 1848», p. 95.
}

\looseness=-1
Але й залишаючи цілком осторонь дійсну надмірну працю,
ця так звана система змін була таким витвором фантазії капіталу,
що його ніколи не перевищив Фур’є у своїх гумористичних
нарисах «courtes séances»\footnote*{
коротких сеансів. \emph{Ред.}
}, з тією лише ріжницею, що притягання
праці тут перетворилося на притягання капіталу. Подивімось
на ці схеми, утворені фабрикантами і прославлені добрячою
пресою як зразок того, «що можна зробити з розумною мірою
дбайливости й методичности» («what a reasonable degree of care
and method can accomplish»). Робочий персонал розділювано
іноді на 12--15 категорій, що знову раз-у-раз зміняли свої
складові частини. Протягом п’ятнадцятигодинного періоду фабричного
дня капітал притягав робітника то на 30 хвилин, то на
годину, потім знову відштовхував його, щоб знову притягти
його на фабрику й знов одштовхнути, ганяючи його то туди, то сюди
розрізненими шматками часу, але постійно не випускаючи його
із своїх рук, доки десятигодинну працю не буде цілком закінчено.
Як на театральній сцені, мали виступати ті самі особи навпереміну
в різних явах різних дій. Але, як актор належить до
\parbreak{}  %% абзац продовжується на наступній сторінці

\parcont{}  %% абзац починається на попередній сторінці
\index{iii2}{0229}  %% посилання на сторінку оригінального видання
рента з землі, або як процент на державну позику, обчислюється ним як процент
на гроші, що їх коштувала йому купівля титулу на цей дохід. Як капітал
він може реалізувати їх тільки через перепродаж. Але тоді інший, новий покупець,
потрапляє в таке саме становище, в якому був перший, і жодними переміщеннями
з рук в руки витрачені в такий спосіб гроші не можуть перетворитись
на дійсний капітал для того, хто їх витратив.

При дрібній земельній власності ще дужче зміцнюється та ілюзія, ніби
сама земля має вартість і тому, як капітал, входить в ціну продукції продукту,
цілком так само як машина або сировий матеріял. Але ми бачили, що тільки
в двох випадках рента, а тому і капіталізована рента, ціна землі, може ввійти
визначально в ціну хліборобського продукту. Поперше, коли в наслідок складу
хліборобського капіталу, — капіталу, який не має нічого спільного з капіталом,
витраченим на купівлю землі, — вартість хліборобського продукту стоїть вище за
його ціну продукції, і ринкові відносини дають можливість земельному власникові
використати цю ріжницю. Подруге, коли є монопольна ціна. І те і те
найрідше трапляється при парцелярному господарстві і дрібній земельній
власності, бо саме тут продукція в дуже великій частині своїй задовольняє
власні потреби, і відбувається незалежно від регулювання загальною нормою
зиску. Навіть там, де парцелярне господарство провадиться на орендованій землі,
орендна плата в незрівняно більшій мірі, ніж за будь-яких інших відносин,
має в собі частину зиску і навіть вирахування з заробітної плати; в такому
випадку це є рента лише номінально, а не рента як самостійна категорія
протилежно до заробітної плати і зиску.

Отже, витрата грошового капіталу на купівлю землі не є приміщення хліборобського
капіталу. Вона є pro tanto скорочення того капіталу, що ним могли б
порядкувати дрібні селяни в сфері самої продукції. Вона зменшує pro tanto розмір
їхніх засобів продукції і тому звужує економічну базу репродукції. Вона
упідлеглює дрібного селянина лихвареві, бо в цій сфері власне кредит взагалі
трапляється рідше. Вона є за гальмо для хліборобства, навіть тоді, коли цю
купівлю роблять великі поміщицькі господарства. Вона дійсно суперечить капіталістичному
способові продукції, що йому взагалі байдужа заборгованість
земельного власника, — байдуже те, чи одержав він свій маєток в спадщину, чи
купив. Чи сам він забирає ренту, чи мусить і він її віддати гіпотечному кредиторові,
— це само по собі нічого не змінює в господарюванні на орендованому
маєтку.

Ми бачили, що за даної земельної ренти ціна землі реґулюється розміром проценту.
Коли процент низький, то ціна землі висока, і навпаки. Отже, нормально висока
ціна землі і низький розмір проценту мусили б іти поруч, так, що коли б селянин
в наслідок низького розміру проценту дорого заплатив за землю, то цей самий низький
розмір проценту мусив би призвести до того, що він на сприятливих умовах здобув би в кредит капітал
для господарювання. В дійсності при пануванні парцелярної
власности справа стоїть інакше. Насамперед, до селян не стосуються загальні закони
кредиту, бо вони мають за свою передумову, що продуцент є капіталіст.
Подруге, там, де панує парцелярна власність, — про колонії тут немає мови —
і парцелярний селянин являє собою головний стовбур нації, створення капіталу,
тобто суспільна репродукція, є порівняно мала, і ще менше створення позикового
грошового капіталу в раніш викладеному розумінні. Воно має своєю передумовою
концентрацію і наявність кляси багатих нероб капіталістів (Massie). Потретє, тут,
де власність на землю становить життєву умову для більшої частини продуцентів
і неодмінну сферу для приміщення їхнього капіталу, ціна землі підвищується,
незалежно від розміру проценту і часто в зворотному відношенні до
нього, в наслідок переваги попиту на земельну власність над поданням. Земля,
продавана парцелями часто дає тут геть вищі ціни, ніж при продажу великими
\parbreak{}  %% абзац продовжується на наступній сторінці

\parcont{}  %% абзац починається на попередній сторінці
\index{iii1}{0230}  %% посилання на сторінку оригінального видання
весь авансований капітал і на скільки більше метрів він виробляє за даний час.

Те випливаюче з природи капіталістичного способу виробництва явище, що при зростаючій продуктивності
праці ціна окремого товару або даної кількості товарів знижується, число товарів збільшується, маса
зиску на окремий товар і норма зиску на суму товарів знижується, а маса зиску на всю суму товарів
підвищується, — це явище, при поверховому розгляді, виражає тільки падіння маси зиску, яка припадає
на окремий товар, падіння його ціни, зростання маси зиску на збільшене загальне
число товарів, які виробляє сукупний капітал суспільства абож і окремий капіталіст. Це далі
сприймається так, ніби капіталіст з доброї волі бере менше зиску на окремому товарі, але відшкодовує
себе більшою кількістю товарів, які він виробляє. Таке розуміння базується на уявленні зиску, який
виникає з продажу (profit upon alienation), — уявленні, яке, з свого боку, знов таки абстраговане з
способу розуміння, властивого купецькому капіталові.

Раніше, в четвертому й сьомому відділах першої книги, ми бачили, що маса товарів, яка зростає разом
з продуктивною силою праці, і здешевлення окремого товару, як таке (оскільки ці товари не входять як
визначальні фактори в ціну робочої сили), не зачіпають відношення оплаченої і неоплаченої праці в
окремому товарі, не зважаючи на зниження ціни.

Через те що в конкуренції все виступає у фальшивому, в перекрученому вигляді, то окремий капіталіст
може уявляти: 1)~що він зменшує свій зиск на окремий товар, знижуючи його ціну, але одержує більший
зиск в наслідок збільшення маси товарів, які він продає; 2)~що він установлює ціну окремого товару і
за допомогою множення визначає ціну сукупного продукту, тимчасом як первісний процес є ділення (див.
книгу І, розд. X), а множення є правильним тільки в другу чергу, при умові такого ділення.
Вульгарний економіст в дійсності не робить нічого іншого, як тільки перекладає дивовижні уявлення
капіталістів, захоплених конкуренцією, на позірно більш теоретичну, узагальнюючу мову, і мучиться
над тим, щоб сконструювати правильність цих уявлень.

В дійсності падіння товарних цін і зростання маси зиску на вирослу масу здешевлених товарів є тільки
інший вираз закону падіння норми зиску при одночасному збільшенні маси зиску.

Дослідження того, наскільки падаюча норма зиску може збігатися із зростаючими цінами, так само мало
стосується сюди, як і той пункт, який ми розглянули раніш, книга I, X, при розгляді відносної
додаткової вартості. Капіталіст, який застосовує поліпшені способи виробництва, що не стали ще,
однак, загальнопоширеними, продає нижче ринкової ціни, але вище своєї індивідуальної ціни
виробництва; таким чином норма зиску для нього зростає, поки її не вирівняє конкуренція; поки
\parbreak{}  %% абзац продовжується на наступній сторінці

\parcont{}  %% абзац починається на попередній сторінці
\index{iii1}{0231}  %% посилання на сторінку оригінального видання
триває цей період вирівнення, виникає друга потреба — збільшити вкладуваний капітал; залежно від
ступеня цього зростання капіталу, капіталіст буде спроможний ужити до праці при нових умовах частину
занятого раніше числа робітників, а, може, навіть усіх або й більше, ніж раніш, число робітників,
отже виробляти ту саму або більшу масу зиску.

\section{Протидіючі причини}

\looseness=-1
Якщо взяти до уваги величезний розвиток продуктивних сил суспільної праці навіть тільки за останні
30 років, порівняно з усіма попередніми періодами, особливо якщо взяти до уваги величезну масу
основного капіталу, який, крім власне машин, входить у сукупність суспільного процесу виробництва,
то замість тієї трудности, якою досі займались економісти, а саме — трудности пояснити падіння норми
зиску, виникає протилежна трудність, а саме — трудність пояснити, чому це падіння не є більшим або
швидшим. Тут мусять діяти протилежні впливи, які перехрещують і знищують вплив загального закону і
надають йому характеру тільки тенденції, через що ми й назвали
падіння загальної норми зиску тенденцією до падіння. Найзагальніші з цих причин такі:

\subsubsection{Підвищення ступеня експлуатації праці}

\looseness=-1
Ступінь експлуатації праці, привласнення додаткової праці І додаткової вартості підвищується зокрема
за допомогою здовження робочого дня та інтенсифікації праці. Ці обидва пункти докладно висвітлені в
книзі І при дослідженні виробництва абсолютної й відносної додаткової вартості. Існує багато таких
моментів інтенсифікації праці, які передбачають зростання сталого капіталу порівняно із змінним,
отже, падіння норми зиску — наприклад, коли робітникові доводиться доглядати за більшим числом
машин. Тут — як і при більшості способів, що служать для виробництва відносної додаткової вартості —
ті самі причини, які викликають зростання норми додаткової вартості, можуть, — якщо розглядати дані
величини всього застосовуваного капіталу, — викликати падіння маси додаткової вартості. Але існують
інші моменти інтенсифікації, як, наприклад, прискорена швидкість машин; останні при цьому за той
самий час споживають, правда, більше сировинного матеріалу, але щодо основного капіталу, то, хоч
машини й зношуються швидше, однак це ніяк не зачіпає відношення їх вартості до ціни тієї праці, яка
приводить їх у рух. Але особливо збільшує масу привласнюваної
\parbreak{}  %% абзац продовжується на наступній сторінці


  
\chapter{Доходи та їхні джерела}

\section{Триєдина формула}

\subsection*{І.}

Капітал\footnote{Дальші три уривки містяться в різних місцях рукопису відділу VІ. — \emph{Ф.~Е.}}
— зиск (підприємницький бариш плюс процент), земля — земельна
рента, праця — заробітна плата, це триєдина форма, яка охоплює всі таємниці
суспільного процесу продукції.

А що далі, як це показано раніш, процент виступає як специфічний,
характеристичний продукт капіталу, а підприємницький бариш протилежно
до цього як незалежна від капіталу заробітна плата, то зазначена триєдина
форма найближче зводиться до такої:

Капітал — процент, земля — земельна рента, праця — заробітна плата, де зиск,
ця форма додаткової вартости, що специфічно характеризує капіталістичний спосіб
продукції, щасливо усувається.

При ближчому розгляді цієї економічної триєдиности ми відкриваємо таке:

Поперше, позірні джерела багатства, що ним можна щороку порядкувати,
належать до цілком різних сфер і не мають найменшої схожости між собою.
Взаємне відношення між ними приблизно таке, як наприклад, між нотаріяльними
оплатами, червоними буряками і музикою.

\looseness=-1
Капітал, земля, праця! Але капітал — це не річ, а певне, суспільне,
належне певній історичній формації суспільства продукційне відношення, яке
виявляється в речі і надає цій речі специфічного суспільного характеру. Капітал
не є сума матеріяльних і випродукованих засобів продукції. Капітал, це —
перетворені на капітал засоби продукції, які сами по собі так само не є капітал,
як золото або срібло сами по собі не є гроші. Це є монополізовані певною
частиною суспільства засоби продукції, усамостійнені проти живої робочої сили
продукти й умови діяльности самої цієї робочої сили, які в наслідок цієї протилежности
персоніфікуються в капіталі. Це не тільки продукти робітників,
перетворені на самостійні сили, продукти як поневільники і покупці своїх продуцентів,
але також і суспільні сили і майбутня\dots{} [? нерозбірливо] форма цієї
праці, — сили, що протистоять робітникам, являючи властивості їхнього продукту.
Отже, ми маємо тут певну, на перший погляд дуже містичну, суспільну форму
одного з чинників певного історично створеного суспільного процесу продукції.

\disablefootnotebreak{}
\looseness=-1
А тепер, поряд з цим земля, неорганічна природа як така, rudis indigestaque
moles\footnote*{
Лат. груба, необроблена груда. \Red{Пр.~Ред.}
} у всій її перевісній дикості. Вартість є праця. Тому додаткова
\parbreak{}  %% абзац продовжується на наступній сторінці

\parcont{}  %% абзац починається на попередній сторінці
\index{iii2}{0233}  %% посилання на сторінку оригінального видання
вартість не може бути землею. Абсолютна родючість землі не призводить ні до
чого іншого, як тільки до того, що певна кількість праці дає певний, зумовлений
природною родючістю землі, продукт. Ріжниця в родючості землі призводить
до того, що ті самі кількості праці й капіталу, отже, та сама вартість, виражається
в різних кількостях хліборобських продуктів; отже, що ці продукти
мають різні індивідуальні вартості. Вирівнювання цих індивідуальних вартостей
за ринковими вартостями призводить до того, що «advantages of fertile over inferior
soil\dots{} are transferred from the cultivator or consumer to the landlord» (Ricardo,
Principles, p. 6)\footnote*{
«Вигоди, одержувані від родючішого ґрунту проти гіршого\dots{} переносяться від обробника або
споживача до лендлорда».
}.
\enablefootnotebreak{}

\looseness=1
І, нарешті, як третій в цій спілці простий привид, — праця «взагалі»,
яка є не що інше, як абстракція і взята сама по собі взагалі не існує, або,
коли ми\dots{} (нерозбірливо) візьмемо, продуктивна діяльність людини взагалі,
з допомогою якої людина упосереджує обмін речовин з природою, не тільки
оголена від усякої суспільної форми і характеристичної визначености, але навіть
і просто в її природному бутті, незалежно від суспільства, абстраговано від
хоч би яких суспільств, і як вияв життя та процес життя, спільна ще несуспільній
людині взагалі з людиною, що має будь-яке суспільне визначення.

\subsection*{ІI.}

\looseness=1
Капітал — процент; земельна власність, приватна власть на землю, до того ж
сучасна, відповідна капіталістичному способові продукції, — рента; наймана праця
— заробітна плата. Отже, в цій формі повинен бути зв’язок між джерелами
доходу. Як капітал, так само й наймана праця і земельна власність є історично
визначені суспільні форми; одна — праці, друга — монополізованої землі, і до
того обидві є форми, відповідні капіталові і належні тій самій економічній
формації суспільства.

Перше, що впадає на очі в цій формулі, є те, що поряд з капіталом,
поряд з цією формою одного елементу продукції, належного певному способові
продукції, певній історичній структурі суспільного процесу продукції, поряд
з елементом продукції, що поєднався з певною соціяльною формою і репрезентований
цією соціяльною формою, без дальших околичностей, ставляться: земля
на одному боці, праця — на другому, два елементи реального процесу праці,
які в цій речовій формі спільні всім способам продукції, є речові елементи
всякого процесу продукції, і не мають ніякого чинення до його суспільної форми.

Подруге. У формулі: капітал — процент, земля — земельна рента, праця —
заробітна плата, капітал, земля, праця виступають відповідно як джерела проценту
(замість зиску), земельної ренти і заробітної плати як своїх продуктів,
витворів; перші — основа, другі — наслідок, перші — причина, другі — дія; і до
того ж таким чином, що кожне окреме джерело стоїть до свого продукту в такому
відношенні, як до чогось від нього відштовхнутого і ним спродукованого.
Усі три доходи, процент (замість зиску), рента, заробітна плата, є три частини
вартости продукту, отже, взагалі частини вартости, або в грошовому виразі
певні частини грошей, частини ціни. Хоч формула: капітал — процент і є найіраціональніша формула
капіталу, а проте це — його формула. Але яким чином
земля може створити вартість, тобто суспільно визначену кількість праці і навіть
ту особливу частину вартости її власних продуктів, яка становить ренту. Земля
діє, наприклад, як аґент продукції при створенні певної споживної вартости,
\parbreak{}  %% абзац продовжується на наступній сторінці

\parcont{}  %% абзац починається на попередній сторінці
\index{iii1}{0234}  %% посилання на сторінку оригінального видання
самим. Ще більше підвищується вона при зростаючому населенні; і хоч це зв’язане з відносним
зменшенням числа занятих робітників порівняно з величиною всього капіталу, проте це зменшення
уміряється або затримується в наслідок підвищення норми додаткової вартості.

Раніше ніж перейти до дальшого пункту, слід ще раз підкреслити, що при даній величині капіталу \emph{норма}
додаткової вартості може зростати, хоч \emph{маса} її падає, і навпаки. Маса додаткової вартості дорівнює
її нормі, помноженій на число робітників; але норма додаткової вартості ніколи не обчислюється на
весь капітал, а тільки на змінний, в дійсності тільки на один робочий день. Навпаки, при даній
величині капітальної вартості \emph{норма зиску} ніколи не може підвищитись або впасти без того, щоб \emph{маса
додаткової вартості} так само не підвищилась або не впала.

\subsubsection{Зниження заробітної плати нижче її вартості}

Ми подаємо це тут тільки емпірично, бо в дійсності, воно, як і багато чого іншого, що тут слід би
було навести, не має ніякого відношення до загального аналізу капіталу, а стосується до дослідження
конкуренції, яке не входить в завдання цієї праці. Проте, воно є одною з найзначніших причин, які
затримують тенденцію норми зиску до падіння.

\subsubsection{Здешевлення елементів сталого капіталу}

\looseness=-1
Сюди стосується все, що було сказано в першому відділі цієї книги про причини, які підвищують норму
зиску при незмінній нормі додаткової вартості або незалежно від норми додаткової вартості, отже, і
той випадок, коли, — якщо розглядати весь капітал, — вартість сталого капіталу зростає не в такій
пропорції, як його матеріальний розмір. Наприклад, маса бавовни, яку переробляє окремий європейський
робітник-прядільник на сучасній фабриці, зросла в найколосальнішій мірі порівняно з тією масою, яку
раніше переробляв європейський прядільник за допомогою прядки. Але вартість перероблюваної бавовни
зросла не в такій пропорції, як її маса. Так само стоїть справа
з машинами та іншим основним капіталом. Коротко кажучи, той самий розвиток, який збільшує масу
сталого капіталу порівняно з змінним, зменшує вартість його елементів в наслідок підвищення
продуктивної сили праці і, значить, перешкоджає тому, щоб вартість сталого капіталу, хоч вона
постійно зростає, зростала в такій самій пропорції, як його матеріальний розмір, тобто матеріальний
розмір засобів виробництва, які приводяться в рух тією самою кількістю робочої сили. В окремих
випадках маса елементів сталого капіталу може навіть збільшитись, тимчасом як його вартість
лишається та сама або навіть падає.

\parcont{}  %% абзац починається на попередній сторінці
\index{iii2}{0235}  %% посилання на сторінку оригінального видання
в книзі І, є вираз, що prima facie суперечить поняттю вартости, а також
і поняттю ціни, яка взагалі є лише певний вираз вартости; і «ціна праці»
є так само іраціональна, як жовтий логаритм. Але саме тут вульґарний
економіст лише найбільше і заспокоюється, бо він тут дійшов до глибокого погляду
буржуа, який вважає, що він платить гроші за працю, і що саме суперечність
формули поняттю вартости усуває для нього обов’язок зрозуміти останню.

\pfbreak

Ми\footnote{Початок розділу XLVІІІ за рукописом.} бачили, що капіталістичний процес продукції є історично певна
форма суспільного процесу продукції взагалі. Цей останній є і процес продукції
матеріяльних умов людського життя, і процес, що відбувається в специфічних
історико-економічних відносинах продукції, що продукує і репродукує сами
ці відносини продукції, а разом з тим і носіїв цього процесу, матеріяльні умови їхнього
існування і їхні взаємні відносини, тобто певні суспільно-економічні
форми останніх. Бо сукупність цих відносин, що в них носії цієї продукції перебувають
до природи і один до одного, відносин, що в них вони продукують, ця сукупність
саме і є суспільство, розглядуване з погляду його економічної структури.
Подібно до всіх його попередників капіталістичний процес продукції відбувається
в певних матеріяльних умовах, що є одночасно за носіїв певних суспільних
відносин, в які вступають індивідууми в процесі репродукції свого життя. Як
ті умови, так і ці відносини, є, з одного боку, передумови, з другого — наслідки
і витвори капіталістичного процесу продукції; вони ним продукуються й репродукуються.
Далі ми бачили: капітал, — а капіталіст є лише персоніфікований
капітал, функціонує в процесі продукції лише як носій капіталу — отже, капітал
висмоктує у відповідному йому суспільному процесі продукції певну кількість
додаткової праці з безпосередніх продуцентів або робітників, додаткову працю,
що він її одержує без еквіваленту, і яка за своєю суттю завжди лишається
примусовою працею, хоча б вона і здавалася наслідком вільної договірної угоди.
Ця додаткова праця втілюється у додатковій вартості, і ця додаткова вартість
існує у додатковому продукті. Додаткова праця взагалі, як праця понад міру
даної кількости потреб, мусить завжди існувати. Але в капіталістичній, як і в рабській
системі тощо вона має лише антагоністичну форму і доповнюється цілковитим
неробством певної частини суспільства. Певна кількість додаткової праці потрібна
як страхування проти випадковостей, в наслідок доконечного, відповідного розвиткові
потреб і поступові людности, проґресивного поширення процесу репродукції,
що з капіталістичного погляду називається нагромадженням. Одна з цивілізаторських
сторін капіталу є в тому, що він вимушує цю додаткову працю
таким способом і в таких умовах, які для розвитку продуктивних сил, суспільних
відносин і створення елементів вищої нової формації є вигідніші, ніж за колишніх
форм рабства, крепацтва тощо. Він приводить, таким чином, з одного боку, до
ступеня, на якому відпадає примус і монополізація суспільного розвитку (включаючи
сюди його матеріяльні й інтелектуальні вигоди) однією частиною суспільства
за рахунок іншої; з другого боку, він створює матеріяльні засоби
і зародок для відносин, які в вищій формі суспільства дадуть можливість сполучити
цю додаткову працю з значнішим обмеженням часу, присвяченого матеріяльній
праці взагалі. Бо додаткова праця, залежно від розвитку продуктивної
сили праці, може бути велика при малій загальній довжині робочого дня і відносно
мала при великій загальній довжині робочого дня. Коли потрібний робочий
час \deq{} 3 і додаткова праця \deq{} 3, то весь робочий день \deq{} 6, і норма додаткової
праці \deq{} 100\%. Коли потрібна праця \deq{} 9 і додаткова праця \deq{} 3, то ввесь
робочий день \deq{} 12, і норма додаткової праці \deq{} лише ЗЗ\sfrac{1}{3}\%. Але далі від продуктивности
праці залежить, яка кількість споживної вартости продукується
\parbreak{}  %% абзац продовжується на наступній сторінці

\parcont{}  %% абзац починається на попередній сторінці
\index{iii2}{0236}  %% посилання на сторінку оригінального видання
протягом певного часу, отже, і протягом певного додаткового робочого часу. Отже,
дійсне багатство суспільства і можливість постійно поширювати процес його
репродукції залежить не від протяжности додаткової праці, а від її продуктивности
і від більшого або меншого достатку тих умов продукції, за яких вона відбувається.
Царство волі починається в дійсності лише там, де припиняється праця,
викликана потребою і зовнішньою доцільністю, отже, з природи речей воно
лежить по той бік сфери власне матеріяльної продукції. Як дикун, щоб задовольняти
свої потреби, щоб зберегти і репродукувати своє життя, мусить боротися
з природою, так мусить боротися і цивілізований, і він мусить так боротися
в усіх суспільних формах і при всіх можливих способах продукції. З його
розвитком поширюється це царство природної доконечности, бо його потреби
поширюються; але одночасно поширюються і продуктивні сили, які служать
для їхнього задоволення. Воля в цій царині може бути лише в тому, що соціялізована
людина, асоційовані продуценти раціонально реґулюють цей свій обмін
речовин з природою, становлять його під свій громадський контроль, замість того,
щоб він як сліпа сила панував над ними; — лише в тому, що вони провадять його
з найменшою витратою сили і в найгідніших і найадекватніших їхній людській
природі умовах. Але це все ж таки лишається царством доконечности. По той
бік його починається розвиток людської сили, що є самоціль, справжнє царство
волі, яке, проте, може розцвісти лише на цьому царстві доконечности як на
своїй базі. Скорочення робочого дня — основна умова.

Ця додаткова вартість або цей додатковий продукт розподіляється в капіталістичному
суспільстві між капіталістами, — коли ми лишимо осторонь випадкові
коливання розподілу і розглядатимемо його реґуляційний закон, його нормівні
межі, — як дивіденд пропорційно тій частині, що належить кожному в суспільному
капіталі. В цьому вигляді додаткова вартість виступає як пересічний
зиск, що дістається капіталові, пересічний зиск, який в свою чергу розпадається
на підприємницький бариш і процент, і який в кожній з цих двох категорій
може дістатися різного роду капіталістам. Це присвоювання і розподіл додаткової
вартости, зглядно додаткового продукту, капіталом, має, проте, свою межу в земельній
власності. Як капіталіст, що функціонує, висмоктує з робітника додаткову
працю, а тим самим висмоктує він у формі зиску додаткову вартість
і додатковий продукт, так земельний власник і собі висмоктує з капіталіста
частину цієї додаткової вартости або додаткового продукту у формі ренти, згідно
з раніш викладеними законами.

Отже, коли ми говоримо тут про зиск як про частину додаткової вартости,
що припадає капіталові, то ми маємо на увазі пересічний зиск (дорівнює підприємницькому
баришеві плюс процент), який уже обмежений вирахуванням
ренти з усього зиску (зиску, що у своїй масі тотожний з усією додатковою
вартістю); тобто припускається вирахування ренти. Отже, зиск з капіталу (підприємницький
бариш плюс процент) і земельна рента є не що інше, як окремі
складові частини додаткової вартости, категорії, що в них остання стає різна,
залежно від того, чи дістається вона капіталові, чи земельній власності, рубрики,
які проте нічого не змінюють в її суті. Складені одна з однією, вони становлять
суму суспільної додаткової вартости. Капітал висмоктує додаткову працю, втілену
в додатковій вартості і додатковому продукті, безпосередньо з робітників. Отже,
в цьому розумінні його можна розглядати як продуцента додаткової вартости.
Земельна властність не має жодного чинення до дійсного процесу продукції.
Її роля обмежується тим, що вона переміщує частину випродукованої додаткової
вартости з кишені капіталіста у свою власну. А проте, земельний власник відіграє
певну ролю в капіталістичному процесі продукції не тільки в наслідок
тиснення, яке він справляє на капітал, і не просто в наслідок того, що велика
земельна власність є передумова і умова капіталістичного способу продукції,
\parbreak{}  %% абзац продовжується на наступній сторінці

\parcont{}  %% абзац починається на попередній сторінці
\index{ii}{0237}  %% посилання на сторінку оригінального видання
застосованих у \emph{В}, і на заміщення цього всього подається на ринок еквівалент
в формі грошей; але протягом цього року на ринок не подається
жодного продукту, щоб замістити взяті з ринку речові елементи продуктивного
капіталу. Коли ми уявимо собі не капіталістичне суспільство,
а комуністичне, то, насамперед, зовсім відпадає грошовий капітал, а значить,
і всі ті маскування оборудок, які постають через грошовий капітал.
Справа сходить просто на те, що суспільство мусить наперед обчислити,
скільки праці, засобів продукції та засобів існування воно може без якої-будь
шкоди витрачати на такі галузі продукції, що, як от, напр., будування
залізниць, довгий час, рік або й більше, не дають ні засобів
продукції, ні засобів існування, ні взагалі будь-якого корисного ефекту,
але звичайно відбирають від цілої річної продукції працю, засоби продукції
і засоби існування. Навпаки, в капіталістичному суспільстві, де
суспільний розум завжди виявляє себе тільки post festum\footnote*{
Дослівно: „після свята“, коли справу вже закінчено. \emph{Ред.}
}, можуть і мусять
завжди поставати великі порушення. З одного боку, тиск на грошовий
ринок, тимчасом як гарний стан грошового ринку, навпаки, і собі покликає
до життя багато таких підприємств, отже, призводить саме до таких обставин,
що потім зумовлюють тиск на грошовий ринок. Грошовий ринок
зазнає тиску, бо при цьому треба постійно авансувати великий грошовий
капітал на довгий час. Ми вже зовсім не кажемо про те, що промисловці
й торговці кидають на залізничні спекуляції і таке інше грошовий капітал,
потрібний їм для провадження власних підприємств, і заміщують його позиками
на грошовому ринку. — З другого боку, зазнає тиску продуктивний капітал,
що є в розпорядженні суспільства. А що елементи продуктивного капіталу
постійно вилучається з ринку і натомість на ринок подається лише
грошовий еквівалент, то більшає виплатоспроможний попит, який, із свого
боку, не має в собі жодних елементів подання. Відси підвищення цін
і на засоби існування, і на продукційні матеріяли. До цього долучається
ще й те, що під такий час звичайно розвивається шахрайство і переміщується
чимало капіталу. Зграя спекулянтів, постачальників, інженерів,
адвокатів тощо збагачується. Вони спричиняють на ринку великий попит
на речі споживання, поряд цього підвищується заробітна плата. Щодо
попиту на харчові засоби, то він звичайно підганяє й сільське господарство.
А що цих харчових засобів не можна збільшити одразу, протягом
року, то більшає довіз їх, як і взагалі довіз екзотичних харчових
засобів (кави, цукру, вина тощо) та речей розкошів. Звідси надмірний
довіз і спекуляція в цій галузі імпортної торговлі. З другого боку, в
тих галузях промисловости де продукцію можна швидко збільшити (власне
мануфактура, гірництво тощо), підвищення цін призводить до раптового
поширення, що по ньому скоро настає крах. Такий самий вплив
справляється на робочий ринок, щоб притягти до нових галузей підприємств
великі маси лятентного відносного надміру людности і навіть робітників,
уже зайнятих. Взагалі такі підприємства великого маштабу, як
от залізниці, відтягують від робочого ринку певну кількість сил, що
\parbreak{}  %% абзац продовжується на наступній сторінці

\parcont{}  %% абзац починається на попередній сторінці
\index{iii1}{0238}  %% посилання на сторінку оригінального видання
І до спожитих на їх виробництво засобів праці дедалі знижується; отже, та обставина, що в цих
товарах упредметнюється дедалі менша кількість додаваної живої праці, бо з розвитком суспільної
продуктивної сили потрібно менше праці для їх виробництва — ця обставина не стосується до того
відношення, в якому уміщена в товарі жива праця ділиться на оплачену і неоплачену. Навпаки. Хоча
загальна кількість уміщеної в товарі додаваної живої праці зменшується, неоплачена частина зростає
порівняно з оплаченою в наслідок або абсолютного, або відносного зниження оплаченої частини; бо той
самий спосіб виробництва, який зменшує загальну масу додаваної живої праці в кожному окремому
товарі, супроводиться зростанням абсолютної і відносної додаткової вартості. Тенденція норми зиску
до зниження зв’язана з тенденцією до підвищення норми додаткової вартості, отже й ступеня
експлуатації праці. Тому немає нічого більш безглуздого, як поясняти зниження норми зиску
підвищенням норми заробітної плати, хоч винятково і це може мати місце. Тільки зрозумівши відносини,
при яких утворюється норма зиску, статистика стає спроможною взятись до дійсного аналізу норми
заробітної плати в різні епохи і в різних країнах. Норма зиску падає не тому, що праця стає менш
продуктивною, а тому, що вона стає більш продуктивною. І те і друге, підвищення норми додаткової
вартості і падіння норми зиску, є тільки особливі форми, в яких капіталістично виражається зростаюча
продуктивність праці.

\subsection{Збільшення акційного капіталу}

До вищенаведених п’яти пунктів можна додати ще один пункт, на якому ми, однак, покищо не можемо
детальніше спинятись. З прогресом капіталістичного виробництва, який іде рука в руку з прискореним
нагромадженням, частина капіталу враховується і застосовується тільки як капітал, що дає процент. Не
в тому розумінні, в якому кожний капіталіст, що позичає комусь капітал, задовольняється процентами,
тимчасом як промисловий капіталіст одержує підприємницький зиск. Це не стосується до висоти
загальної норми зиску, бо для неї зиск \deq{} процентові \dplus{} зиск усякого роду \dplus{} земельна рента, при чому
розподіл на ці особливі категорії для неї не має значення. А в тому
розумінні, що ці капітали, хоч і вкладені у великі продуктивні підприємства, дають після
відрахування всіх витрат тільки великі або малі проценти, так звані дивіденди; наприклад, у
залізничній справі. Отже, вони не беруть участі у вирівнюванні загальної норми зиску, бо вони дають
меншу норму зиску, ніж пересічна норма. Коли б вони брали участь у вирівнюванні, то ця остання упала
б значно нижче. Розглядаючи справу теоретично, їх можна врахувати, і тоді одержимо меншу норму
зиску, ніж та, яка, видимо, існує і дійсно є визначальною для капіталістів, — одержимо меншу норму
зиску, бо саме в цих підприємствах сталий капітал є найбільший порівняно з змінним.

\parcont{}  %% абзац починається на попередній сторінці
\index{iii2}{0239}  %% посилання на сторінку оригінального видання
відчужености й самостійности проти праці, отже, та перетворена форма умов
праці, що в ній випродуковані засоби продукції пертворюються на капітал, а
земля на монополізовану землю, на земельну власність, — ця належна певному
історичному періодові форма збігається, отже, з буттям і функцією випродукованих
засобів продукції і землі в процесі продукції взагалі. Ці засоби продукції
є капіталом сами по собі, з природи; капітал є не що інше, як просто «економічний
термін» для цих засобів продукції; і отак земля є сама по собі, з
природи, землею, монополізованою певного кількістю земельних власників. Як
у капіталі і в капіталістові, — який на ділі є не що інше, як персоніфікований
капітал, — продукти стають самостійною силою проти продуцентів, так і в земельному
власникові персоніфікується земля, яка теж стає дибки, і як самостійна
сила вимагає своєї частини у випродукованому з її допомогою продукті;
так що не земля одержує належну їй частину продукту для відновлення
і підвищення продуктивности, а замість неї земельний власник одержує частину
цього продукту на прогулювання і марнотратство. Ясно, що капітал має
своєю предумовою працю як найману працю. Але так само ясно, що коли виходити
з праці як найманої праці, так що тотожність праці взагалі з найманою
працею уявляється самоочевидною, то тоді капітал і монополізована земля
в свою чергу мусять уявлятись природною формою умов праці проти праці
взагалі. Бути капіталом, — це уявляється тепер природною формою засобів праці,
а тому й як їх суто-речовий характер, що виникає з їхньої функції
в процесі праці взагалі. Таким чином, капітал і випродукований засіб продукції
стають тотожніми виразами. Так само земля і монополізована приватною власністю
земля стають тотожніми виразами. Тому за джерело зиску стають засоби
праці як такі, як капітал з природи, так само як за джерело ренти стає земля
як така.

Працю як таку, в її простій визначеності доцільної продуктивної діяльности,
ставиться в відношення до засобів продукції, взятим не в їхній суспільній
визначеності форми, а в їхній речевій субстанції як матеріялу і засобів праці,
що відрізняються між собою теж лише речево, як споживні вартості: земля —
як невипродукований засіб праці, інші — як випродуковані засоби праці. Отже,
коли праця збігається з найманою працею, то та певна суспільна форма, в
якій умови праці протистоять тепер праці, також збігається з їхнім речевим
буттям. Тоді засоби праці як такі є капітал, і земля як така є земельна
власність. Тоді формальне усамостійнення цих умов праці, проти праці і та
особлива форма цього усамостійнення, яку вони мають проти найманої праці, видається
властивістю невідійманною від них як від речей, як від матеріяльних умов
продукції, видається властивістю, що неодмінно належить та іманентно зрослася
з ними як з елементами продукції. Їхній визначуваний певною історичною
епохою певний соціяльний характер в капіталістичному процесі продукції видається
їхнім речовим характером, природно і так би мовити споконвіку природженим
їм, як елементам процесу продукції. Тому відповідна участь, яку земля
як первісна сфера діяльности праці, як царство природних сил, як наявний
арсенал усіх речей праці, і та друга відповідна участь, яку випродуковані засоби
продукції (знаряддя, сирові матеріяли тощо) беруть у процесі продукції
взагалі — мусять тоді здаватись участю, що знаходить собі вираз у відповідних
частинах, які в формі зиску (проценту) і ренти дістаються їм як капіталові і
земельній власності, тобто їхнім соціяльним представникам, як для робітника
мусить здаватись, що та участь, яку його праця бере в процесі продукції,
виражається в заробітній платі. Таким чином, здається, що рента, зиск, заробітна
плата породжуються тією ролею, яку земля, випродуковані засоби продукції й
праця відіграють у простому процесі праці, навіть і тоді, коли б ми розглядали
цей процес праці як процес просто між людиною й природою і абстрагувались
\parbreak{}  %% абзац продовжується на наступній сторінці

\parcont{}  %% абзац починається на попередній сторінці
\index{iii2}{0240}  %% посилання на сторінку оригінального видання
від усякої історичної його визначености. Ми маємо знову те саме тільки в іншій
формі, коли кажуть: продукт, в якому втілюється праця найманого робітника
на себе самого, як його здобуток, його дохід, це лише заробітна плата, та частина
вартости (а тому й суспільного продукту, вимірюваного цією вартістю),
яка становить його заробітну плату. Отже, коли наймана праця збігається з
працею взагалі, то й заробітна плата збігається з продуктом праці, і та частина
вартости, яка репрезентована заробітною платою, збігається з вартістю, взагалі
створеною працею. Але в наслідок цього і інші частини вартости, зиск і рента,
так само самостійно протиставляться заробітній платі, і їх доводиться виводити
з власних джерел, специфічно відмінних і незалежних від праці; їх доводиться
виводити з співдіющих елементів продукції, що посідачам їх вони припадають,
отже, зиск доводиться виводити з засобів продукції, речевих елементів капіталу,
а ренту з землі або природи, репрезентованої земельним власником (Рошер).

Тому земельна власність, капітал і наймана праця з джерел доходу в тому
розумінні, що капітал притягає в формі зиску до капіталіста ту частину додаткової
вартости, яку він здобуває з праці, монополія на землю притягає до
земельного власника іншу частину в формі ренти, а праця дає робітникові в
формі заробітної плати останню ще вільну частину вартости, з джерел доходу,
що за їх посередництвом одна частина вартости перетворюється на форму зиску,
друга на форму ренти і третя на форму заробітної плати, — перетворюються на
дійсні джерела, що з них виникають ці частини вартости і ті відповідні частини
продукту, що в них вони існують або на які вони можуть бути обмінені —
на джерела, з яких кінець-кінцем виникає сама вартість продукту\footnote{
«Заробітна плата, зиск і рента є три первісні джерела всякого доходу, так само як і всієї
мінової вартости» (А.~Smith). «Таким чином, причини матеріяльної продукції є одночасно джерела всіх
сущих основних доходів» (Storch, І, р. 259).
}.

Розглядаючи простіші категорії капіталістичного способу продукції, і навіть
товарової продукції, товар і гроші, ми вже зазначали той містифікаційний
характер, що перетворює суспільні відносини, що для них при продукції речеві
елементи багатства правлять за носіїв, на властивості самих цих речей (товари)
і ще яскравіше, саме продукційне відношення — на річ (гроші). Всі форми
суспільства, оскільки вони призводять до товарової продукції і грошової циркуляції,
беруть участь у цьому перекрученні. Але за капіталістичного способу
продукції й за капіталу, який є його панівною категорією, його визначальним
відношенням продукції, цей зачарований і перекручений світ розвивається геть
більше. Коли розглядати капітал, насамперед в безпосередньому процесі продукції,
— як висмоктувана додаткової праці, — то це відношення ще дуже просте;
і дійсний внутрішній зв’язок ще нав’язується носіям цього процесу, самим
капіталістам і ще усвідомлюється ними. Це переконливо доводиться упертою
боротьбою за межі робочого дня. Але навіть всередині цієї неускладненої сфери,
сфери безпосереднього процесу між працею й капіталом, справа не лишається
така проста. З розвитком відносної додаткової вартости за власне специфічного
капіталістичного способу продукції, в наслідок чого розвиваються і суспільні
продуктивні сили праці, — ці продуктивні сили і суспільні відносини праці виступають
у безпосередньому процесі праці в такому вигляді, як ніби з праці
вони перенесені в капітал. Тим самим капітал стає дуже таємничою істотою,
бо всі суспільні продуктивні сили праці виступають у такому вигляді,
ніби вони належать йому, а не праці як такій, і як такі сили, що народжуються
в його власних надрах. А потім втручається процес циркуляції, що
в його обмін речовин і зміну форм втягується всі частини капіталу, навіть
хліборобського капіталу, в тій самій мірі, в якій розвивається специфічно
капіталістичний спосіб продукції. Це є така сфера, в якій відносини первісної
\parbreak{}  %% абзац продовжується на наступній сторінці

\parcont{}  %% абзац починається на попередній сторінці
\index{iii2}{0241}  %% посилання на сторінку оригінального видання
продукції вартости відступають цілком на задній плян. Вже в безпосередньому
процесі продукції капіталіст діє одночасно як товаропродуцент, як керівник
товарової продукції. Тому цей процес продукції зовсім не уявляється йому просто
процесом продукції додаткової вартости. Але хоч би яка була та додаткова
вартість, яку капітал у безпосередньому процесі продукції висмоктував і втілював
в товари, вартість і додаткова вартість, що міститься в товарах, мусить реалізуватись
лише в процесі циркуляції. І справа набуває такого вигляду, ніби вартість,
яка покриває вартості, авансовані на продукцію, і особливо додаткова
вартість, що міститься в товарах, не просто реалізуються в циркуляції, але виникають
з неї; цю ілюзію особливо зміцнюють дві обставини: поперше, зиск,
одержуваний при відчуженні, залежить від обману, хитрощів, знання справи,
спритности й тисячі ринкових коньюнктур; а подруге, та обставина, що тут
поряд з робочим часом виступає другий визначальний елемент, час циркуляції.
Хоч він функціонує тільки як негативна межа створення вартости і додаткової
вартости, але має таку подобу, ніби він є так само позитивна причина
їх створення, як сама праця, і ніби він додає незалежне від праці визначення,
що походить з природи капіталу. У книзі II нам, природно, довелось подати цю
сферу циркуляції лише в її відношенні до визначень форм, які вона породжує,
показати дальший розвиток структури капіталу, який відбувається в цій сфері.
Але в дійсності ця сфера є сфера конкуренції, над якою, коли розглядати кожен
окремий випадок, панує випадковість; отже, сфера, в якій внутрішній закон,
що пробивається серед цих випадковостей і регулює їх, стає видимим лише тоді,
коли сполучити ці випадковості в велику масу, в якій, отже, він лишається
невидимим і незрозумілим для самих окремих аґентів продукції. Але далі: дійсний
процес продукції, як єдність безпосереднього процесу продукції і процесу
циркуляції, породжує нові витвори, в яких дедалі більше втрачається нитка
внутрішнього зв’язку, відносини продукції взаємно усамостійнюються, і складові
частини вартости костеніють у самостійних одна проти однієї формах.

Як ми бачили, перетворення додаткової вартости на зиск визначається так
процесом циркуляції, як і процесом продукції. Додаткова вартість, у формі зиску,
відноситься вже не до витраченої на працю частини капіталу, з якої вона виникає,
а до всього капіталу. Норма зиску реґулюється власними законами, що
допускають і навіть зумовлюють її зміну за незмінної норми додаткової
вартости. Все це дедалі більше затушковує справжню природу додаткової вартости,
а тому й дійсний механізм капіталу. Ще в більшій мірі стається це
в наслідок перетворення зиску на пересічний зиск і вартостей на ціни продукції,
на реґуляційні пересічні ринкових цін. Тут втручається складний суспільний
процес, процес вирівнювання капіталів, який відриває відносні пересічні
ціни товарів від їхніх вартостей, і пересічні зиски в різних сферах продукції
(залишаючи цілком осторонь індивідуальні вкладання капіталу в кожній окремій
сфері продукції) від дійсної експлуатації праці окремими капіталами. Тут не
тільки так здається, але й дійсно пересічна ціна товарів відмінна від їхньої
вартости, отже, від реалізованої в них праці, і пересічний зиск окремого капіталу
відмінний від додаткової вартости, яку цей капітал здобув з зайнятих ним
робітників. Вартість товарів виявляється безпосередньо лише в тому впливі, що
його справляють зміни продуктивної сили праці на пониження та підвищення цін
продукції, на їхній рух, а не на їхні кінцеві межі. Зиск, як здається, визначається
безпосередньою експлуатацією праці лише випадково, лише остільки,
оскільки ця експлуатація дає капіталістові можливість за наявности реґуляційних
ринкових цін, які видаються незалежними від цієї експлуатації, реалізувати
зиск, що відхиляється від пересічного зиску. Щодо самих нормальних пересічних
зисків, то вони здаються іманентними капіталові, незалежно від експлуатації;
ненормальна експлуатація, а також пересічна експлуатація за сприятливих
\parbreak{}  %% абзац продовжується на наступній сторінці

\parcont{}  %% абзац починається на попередній сторінці
\index{i}{0242}  %% посилання на сторінку оригінального видання
капітал, є в певних межах незалежне від подання робітників\footnote{
Цей елементарний закон, здається, невідомий вульґарним економістам
які, Архімеди навиворіт, гадають, що у визначенні ринкових цін
праці попитом і поданням вони знайшли пункт опори не для того, щоб
перевернути світ, а щоб спинити його рух.
}.
Навпаки, зменшення норми додаткової вартости лишає масу
випродукованої додаткової вартости незмінною, коли величина
змінного капіталу або число вживаних робітників пропорційно
зростає. [Змінний капітал у 100 талярів, що експлуатує 100 робітників
при нормі додаткової вартости в 100\%, продукує 100 талярів
додаткової вартости. Норма додаткової вартости може зменшитися
вдвоє, але сума її лишається та сама, коли одночасно
подвоюється змінний капітал]\footnote*{
Заведене у прямі дужки ми беремо з французького видання. \Red{Ред.}
}.

Однак, компенсація числа робітників або величини змінного
капіталу збільшенням норми додаткової вартости або здовженням
робочого дня має межі, що їх не сила переступити. Хоч яка
буде вартість робочої сили, отже, все одно, чи робочий час,
доконечний для утримання робітника, становить 2 чи 10 годин,
загальна вартість, яку робітник може продукувати день-у-день,
є завжди менша від вартости, в якій упредметнюються 24 робочі
години, менша від 12\shil{ шилінґів}, або 4 талярів, коли вони є грошовий
вираз 24 упредметнених робочих годин.

[Щодо додаткової вартости, то її межі ще вужчі. Коли частина
робочого дня, доконечна для покриття денної заробітної плати,
становить 6 годин, то від природного дня залишається тільки 18 годин,
що з них, за біологічними законами, частина потрібна для
відпочинку робочої сили. Припустімо, що 6 годин є мінімальна
межа цього відпочинку; коли здовжити робочий день до його максимальної
межі, до 18 годин, то додаткова праця становитиме
лише 12 годин і, отже, спродукує вартість лише в 2 таляри]\footnote*{
Цей абзац ми беремо з французького видання. \Red{Ред.}
}.

За нашої попередньої передумови, за якою потрібно на день
6 робочих годин, щоб репродукувати саму робочу силу, або
покрити капітальну вартість, авансовану на її купівлю, змінний
капітал у 500 талярів, який уживає 500 робітників при нормі
додаткової вартости в 100\%, або за 12-годинного робочого дня,
продукує денно додаткову вартість у 500 талярів, або 6 × 500
робочих годин. А капітал у 100 талярів, який денно вживає
100 робітників при нормі додаткової вартости в 200\%, або за 18-годинного
робочого дня, продукує масу додаткової вартости лише
в 200 талярів, або 12 × 100 робочих годин. Ціла нововипродукована
ним вартість, еквівалент авансованого змінного капіталу
плюс додаткова вартість, ніколи не може досягти суми 400 талярів,
або 24 × 100 робочих годин пересічно на день. Абсолютна
межа пересічного робочого дня, який з природи є завжди менший
від 24 годин, становить абсолютну межу для компенсації зменшення
змінного капіталу підвищенням норми додаткової вартости,
або зменшення числа експлуатованих робітників підвищенням
\parbreak{}  %% абзац продовжується на наступній сторінці

\parcont{}  %% абзац починається на попередній сторінці
\index{i}{0243}  %% посилання на сторінку оригінального видання
ступеня експлуатації робочої сили. Цей цілком очевидний другий
закон є важливий для пояснення багатьох явищ, які випливають
із тенденції капіталу, що її ми маємо розвинути пізніш,
а саме тенденції якомога більше скорочувати число робітників,
що їх він вживає, або його змінну складову частину, перетворену
на робочу силу, — всупереч іншій його тенденції, а саме продукувати
якомога більшу масу додаткової вартости. Навпаки, коли
маса вживаних робочих сил або величина змінного капіталу
зростає, алеж непропорційно до зменшення норми додаткової
вартости, то маса продукованої додаткової вартости меншає\footnote*{
У французькому виданні цей абзац подано так: «Цей абсолютно
ясний закон є важливий для розуміння складних явищ. Ми вже знаємо,
що капітал намагається продукувати якомога більше додаткової вартости;
ми побачимо пізніш, що він разом із цим намагається скоротити до
мінімуму, порівняно з розмірами підприємства, свою змінну частину,
або кількість робітників, що їх він експлуатує. Ці тенденції стають одна
одній суперечними, скоро лише зменшення одного з факторів, що визначають
суму додаткової вартости, вже не може бути компенсоване збільшенням
другого». («Le Capital etc.», v. I, ch. XI, p. 132). \Red{Ред.}
}.

Третій закон випливає з визначення маси продукованої додаткової
вартости двома факторами: нормою додаткової вартости й
величиною авансованого змінного капіталу. Коли дано норму
додаткової вартости, або ступінь експлуатації робочої сили,
і вартість робочої сили, або величину доконечного робочого
часу, то само собою зрозуміло, що чим більший змінний капітал,
тим більша маса продукованої вартости й додаткової вартости.
Коли дано межі робочого дня, а також межі його доконечної
складової частини, то маса вартости й додаткової вартости,
що її продукує поодинокий капіталіст, очевидно, залежить
виключно від тієї маси праці, яку він пускає в рух. Але
маса ця, за даних припущень, залежить від маси робочої сили,
або від числа робітників, яких він експлуатує; а це число, з
свого боку, визначається величиною авансованого ним змінного
капіталу. Отже, за даної норми додаткової вартости й даної вартости
робочої сили маси продукованої додаткової вартости є
просто пропорційні до величин авансованих змінних капіталів.
Та тепер уже відомо, що капіталіст ділить свій капітал на
дві частини. Одну частину він вкладає в засоби продукції. Це —
стала частина його капіталу. Другу частину він перетворює на
живу робочу силу. Ця частина становить його змінний капітал.
На базі того самого способу продукції в різних галузях продукції
відбувається різний поділ капіталу на сталу та змінну складові
частини. В тій самій галузі продукції це відношення змінюється
разом із зміною технічної основи й суспільних комбінацій процесу
продукції. Але хоч як розпадатиметься даний капітал на сталу
й змінну складові частини, чи остання відноситиметься до першої
як $1 : 2$, $1 : 10$, або $1 : х$, — це не порушує щойно встановленого
закону, бо, згідно з попередньою аналізою, вартість сталого капіталу
хоч і з’являється знов у вартості продукту, але не увіходить
у новоутворену вартість. Щоб уживати \num{1.000} прядунів, потрібно,
\parbreak{}  %% абзац продовжується на наступній сторінці

\parcont{}  %% абзац починається на попередній сторінці
\index{iii2}{0244}  %% посилання на сторінку оригінального видання
тим, що норма зиску зростає, коли товар, проданий нижче від його вартости,
становить елемент сталого капіталу, або тим, що зиск і рента втілюються
в більшій кількості продукту, коли товар, проданий нижче від його вартости,
входить як річ особистого споживання в частину вартости, споживану як дохід.
А подруге, це знищується в пересічних коливаннях. В усякому разі, коли навіть
частина додаткової вартости, яка не реалізувалася в ціні товару, не бере участи
в створенні ціни, — сума пересічного зиску плюс рента в її нормальній формі
ніколи не може бути більша за всю додаткову вартість, хоч і може бути
менша за неї.

Її нормальна форма має своєю передумовою заробітну плату, відповідну
до вартости робочої сили. Навіть монопольна рента, оскільки вона не є вирахування
із заробітної плати, отже, не являє собою осібної категорії, посередньо
мусить завжди становити частину додаткової вартости; коли вона і не являє
собою частини надміру ціни над ціною продукції того самого товару, що вона
є його складова частина (як за диференційної ренти), або коли вона і не являє
собою надмірної частини додаткової вартости того самого товару, що вона є його
складова частина, над частиною його власної додаткової вартости, вимірюваної
пересічним зиском (як за абсолютної ренти), то все таки вона становить частину
додаткової вартости інших товарів, тобто товарів, обмінюваних на цей товар,
що має монопольну ціну. — Сума пересічного зиску плюс земельна рента ніколи
не можуть перебільшувати величини, що частинами її є ці зиски і рента, і що її
вже дано до цього поділу. Тому для нашого дослідження байдуже, чи реалізується
в ціні товарів уся додаткова вартість товарів, тобто вся додаткова праця,
що міститься в товарах, чи ні. Додаткова праця вже тому не реалізується цілком,
що при постійній зміні кількости праці, суспільно потрібної для продукції даного
товару, що виникає з постійної зміни продуктивної сили праці, частину
товарів завжди продукується в ненормальних умовах, а тому їх доводиться
продавати нижче від їхньої індивідуальної вартости. В усякому разі зиск
плюс рента дорівнюють усій реалізованій додатковій вартості (додатковій праці)
і для дослідження, про яке тут іде мова, реалізовану додаткову вартість можна
вважати за рівну всій додатковій вартості, бо зиск і рента є реалізована додаткова
вартість, отже, взагалі та додаткова вартість, що входить в ціни товарів,
отже, практично вся та додаткова вартість, яка є складова частина цієї ціни.

З другого боку, заробітна плата, що становить третю своєрідну форму
доходу, завжди дорівнює змінній складовій частині капіталу, тобто тій складовій
частині, яку витрачається не на засоби праці, а на купівлю живої робочої
сили, на виплату робітникам. (Працю, оплачувану при витрачанні доходу, оплачується
з заробітної плати, зиску або ренти і тому вона не становить частини
вартости товарів, що ними її оплачується. Таким чином, її не береться на увагу
при аналізі вартости товарів і складових частин, на які вона розпадається).
Вартість змінного капіталу, отже, і ціна праці репродукується в певній частині
усього зрічевленого робочого дня робітників, в тій частині товарової вартости,
в якій робітник репродукує вартість своєї власної робочої сили або ціну своєї
праці. Весь робочий день робітника розпадається на дві частини. Одна частина
та, підчас якої він виконує кількість праці, потрібну для репродукції вартости
його власних засобів існування: оплачена частина всієї його праці, та частина
його праці, що потрібна для його власного збереження і репродукції. Вся решта
робочого дня, вся надмірна кількість праці, яку він виконує понад працю, реалізовану
в вартості його заробітної плати, є додаткова праця, неоплачена праця,
що втілюється в додатковій вартості усіх випродукованих ним товарів (і тому
в надмірній кількості товару), в додатковій вартості, яка й собі розпадається на
частини з різними назвами, на зиск (підприємницький бариш плюс процент)
та ренту.


\index{ii}{0245}  %% посилання на сторінку оригінального видання
Можливі лише два нормальні випадки репродукції, якщо залишити
осторонь ті порушення, що перешкоджають навіть репродукції в попередньому
маштабі.

Або відбувається репродукція в простому маштабі.

Або відбувається капіталізація додаткової вартости, акумуляція.

\subsection{Проста репродукція}

При простій репродукції додаткова вартість, продукована й реалізовувана
щорічно або — при кількох оборотах — періодично протягом року,
споживається особисто, тобто непродуктивно, її власником, капіталістом.

Та обставина, що вартість продукту складається почасти з додаткової
вартости, почасти з тієї частини вартости, яка складається з репродукованого
в ньому змінного капіталу плюс зужиткований на його продукцію
сталий капітал, — ця обставина абсолютно нічого не змінює ні в кількості,
ні в вартості цілого продукту, що постійно надходить в циркуляцію, як
товаровий капітал, і так само постійно вилучається з неї для продуктивного
або особистого споживання, тобто для того, щоб служити засобом
продукції або засобом споживання. Якщо сталий капітал залишити осторонь,
то ця обставина впливає тільки на розподіл річного продукту між
робітниками й капіталістами.

Тому, навіть коли припустити просту репродукцію, частина додаткової
вартости має постійно перебувати в формі грошей, а не в формі
продукту, бо інакше її не можна перетворити з грошей на продукт для
споживання. Це перетворення додаткової вартости з її первісної товарової
форми на гроші треба тут дослідити далі. Для спрощення справи
візьмімо проблему в її найпростішій формі, а саме припустімо циркуляцію
виключно металевих грошей, грошей, що являють дійсний грошовий еквівалент.

Згідно з законами простої товарової циркуляції (кн. І, розд. III), маси
наявних у країні металевих грошей має вистачити не лише для
циркуляції товарів. Її має вистачити для того, щоб вирівнювати коливання
грошового обігу, що випливають почасти з флюктуацій\footnote*{
Від лат. слова „fluctus“, гра хвиль, хвилювання, почережне
піднесення й спад. \emph{Ред.}
} в швидкості
циркуляції, почасти з змін товарових цін, почасти з різних та
змінних відношень, що в них функціонують гроші як засіб виплати або
як власне засіб циркуляції. Відношення, що в ньому наявна маса грошей
розподіляється на скарб і на гроші в циркуляції, раз-у-раз змінюється, але
маса грошей завжди дорівнює сумі грошей, наявних у формі скарбу та
в формі грошей в циркуляції. Ця маса грошей (маса благородного металю)
є поступінно нагромаджуваний скарб суспільства. Оскільки частина цього
скарбу зужитковується через зношування, її треба щорічно знову заміщувати,
як і всякий інший продукт. Це в дійсності і відбувається через
\parbreak{}  %% абзац продовжується на наступній сторінці


\index{iii2}{0246}  %% посилання на сторінку оригінального видання
Щодо першої трудности: хто повинен оплатити вміщену в продукті сталу
частину вартости і чим? — то припускається, що вартість сталого капіталу, зужиткованого у продукції,
знову з’являється як частина вартости продукту. Це не
суперечить засновкам другої трудности. Бо вже в книзі І, розділ  V (процес праці і
процес зростання вартости) показано, яким чином в наслідок простого долучення
нової праці, хоч вона і не репродукує старої вартости, а створює лише додаток
до неї, створює лише додаткову вартість, все таки разом з тим зберігається
у продукті стара вартість; але одночасно показано, що це відбувається як наслідок
праці не остільки, оскільки вона є вартостетворча, тобто праця взагалі, а в її
функції як певної продуктивної праці. Таким чином, не треба жодної новодолучуваної
праці для того, щоб зберегти вартість сталої частини в тому продукті,
на який витрачається дохід, тобто вся створена протягом року вартість. Але,
зрозуміла річ, потрібна новодолучувана праця для того, щоб покрити вартість
і споживну вартість сталого капіталу, зужиткованого протягом минулого
року. Без такого покриття репродукція взагалі неможлива.

Вся новодолучена праця втілюється у новоствореній протягом року вартості,
яка і собі цілком сходить на три види доходу: заробітну плату,
зиск і ренту. — Отже, з одного боку, не лишається надмірної суспільної праці
для покриття зужиткованого сталого капіталу, що підлягає відновленню почасти
in natura і в його вартості, почасти тільки в його вартості (оскільки справа
йде просто про зношування основного капіталу). З другого боку, вартість, що
створена річною працею і розпадається на форми заробітної плати, зиску і ренти,
і яка в цьому вигляді підлягає витрачанню, є недостатня для того, щоб оплатити
або купити сталу частину капіталу, яка теж мусить міститися в продукті,
крім новоствореної вартости.

Ми бачимо, що поставлену тут проблему вже розв’язано при дослідженні
репродукції сукупного суспільного капіталу, книга II, відділ III.~Тут ми вертаємось
до цього насамперед тому, що там додаткова вартість ще не була розгорнута
в тих її формах, яких вона набуває як дохід: зиск (підприємницький
бариш плюс процент) і рента, а тому і не могла бути досліджена в цих формах; потім також і тому, що
якраз з формою заробітної плати, зиску і ренти
сполучається неймовірний прогріх в аналізі, який проходить через усю політичну
економію, починаючи від А.~Сміта.

Ми поділили там увесь капітал на дві великі кляси: кляса І, що створює
засоби продукції, кляса II, що продукує засоби індивідуального споживання.
Та обставина, що деякі продукти можуть так само правити за речі особистого
користування, як і засоби продукції (кінь, збіжжя тощо) зовсім не знищує абсолютної
правдивости цього поділу. Справді, він не гіпотеза, а лише вираз факту.
Візьмімо річний продукт якоїсь країни. Частина продукту, хоч яка б була здатність
його правити за засіб продукції, входить в індивідуальне споживання.
Це — продукт, на який витрачається заробітну плату, зиск і ренту. Продукт цей
становить продукт певного підрозділу суспільного капіталу. Можливо, що цей
самий капітал продукує також і продукти, що належать до кляси І.~Оскільки
це так, продуктивно спожиті продукти, належні до кляси І, постачаються не
тією частиною цього капіталу, що зужиткована на продукт кляси II, на продукт,
який дійсно дістається індивідуальному споживанню. Весь той продукт
II, що входить в індивідуальне споживання, і на який тому витрачається
дохід, є формою буття зужиткованого на нього капіталу плюс випродукований
надмір. Отже, це — продукт капіталу, вкладеного тільки в продукцію
засобів споживання. І в цьому ж розумінні підрозділ І річного продукту, який
править за засоби репродукції, — сирового матеріялу і знарядь праці, — хоч би
яка взагалі була здатність цього продукту naturaliter правити за засоби споживання,
— є продукт капіталу, вкладеного виключно в продукцію засобів
\parbreak{}  %% абзац продовжується на наступній сторінці

\parcont{}  %% абзац починається на попередній сторінці
\index{iii2}{0247}  %% посилання на сторінку оригінального видання
продукції. Незрівняно більша частина продуктів, що становлять сталий капітал
перебуває також у такій речевій формі, в якій вони не можуть увійти
в індивідуальне споживання. Оскільки це для неї можливе, оскільки, наприклад,
селянин може з’їсти своє збіжжя, призначене на насіння, або зарізати свою робочу
худобу, економічне обмеження щодо цієї частини призводить цілком до того
самого, як коли б ця частина існувала в формі непридатній для споживання.

Як уже сказано, при розгляді обох кляс, ми залишаємо осторонь основну
частину сталого капіталу, яка продовжує існувати in natura і за своєю вартістю,
незалежно від річного продукту обох кляс.

У клясі II, що на її продукти витрачається заробітну плату, зиск і ренту,
коротко кажучи, споживаються доходи, продукт за його вартістю в свою чергу
складається з трьох складових частин. Одна складова частина дорівнює вартості
зужиткованої в продукції сталої частини капіталу; друга складова частина дорівнює
вартості тієї авансованої на продукцію змінної частини капіталу, яка
витрачена на заробітну плату; нарешті, третя складова частина дорівнює випродукованій додатковій
вартості, отже, \deq{} зискові \dplus{} рента. Перша складова частина
продукту кляси II, вартість сталої частини капіталу, не може бути спожита
ні капіталістами, ні робітниками кляси II, ні земельними власниками. Вона не
становить частини їхнього доходу, а мусить бути покрита in natura, а щоб це
могло статись, вона мусить бути продана. Навпаки, дві інші складові частини
цього продукту дорівнюють вартості випродукованих у цій клясі доходів, \deq{} заробітній
платі \dplus{} зиск \dplus{} рента.

У клясі І продукт складається за формою з таких самих складових частин.
Але ту частину, яка становить тут дохід, заробітна плата \dplus{} зиск \dplus{} рента, коротко,
змінну частину капіталу \dplus{} додаткову вартість, споживається тут не
в натуральній формі продуктів цієї кляси І, а в продуктах кляси II.~Отже, вартість
доходів кляси І мусить бути спожита в натуральній формі тієї частини
продукту кляси II, яка становить належний покриттю сталий капітал цієї
кляси II.~Частина продукту кляси II, що мусить покрити сталий капітал цієї
кляси II, споживається в її натуральній формі робітниками, капіталістами та
земельними власниками кляси І.~Вони витрачають свої доходи на цей продукт II.~З другого боку, продукт кляси І, оскільки він становить дохід кляси І, продуктивно
споживається в його натуральній формі клясою II, що її сталий капітал
він покриває in natura. Нарешті, зужиткована частина сталого капіталу кляси
І покривається власними продуктами цієї кляси, які складаються саме з засобів
праці, сирових і допоміжних матеріялів тощо, почасти покривається за посередництвом
обміну капіталістів І між собою, почасти тим, що частина цих капіталістів
може безпосередньо застосувати свій власний продукт, як засіб продукції.

Візьмімо давнішу схему (книга II, розділ XX, II) простої репродукції:
\[
 \left.\begin{aligned}
        \text{I. }\num{4.000} c \dplus{} \num{1.000} v \dplus{} \num{1.000} m \deq{} \num{6.000}\\
        \text{II. }\num{2.000} c \dplus{} \phantom{1}500 v \dplus{} \phantom{1}500 m \deq{} \num{3.000}
       \end{aligned}
 \right\}
  \deq{} \num{9.000}
\]
За нею в II продуценти та земельні власники споживають як дохід
$500 v \dplus{} 500 m \deq{} \num{1.000}$; залишається покрити $\num{2.000} c$. Це споживають
робітники, капіталісти та одержувачі ренти кляси І, що їхній дохід \deq{}
$\num{1.000} v \dplus{} \num{1.000} m \deq{} \num{2.000}$. Спожитий продукт кляси II споживається як дохід
клясою І, а частина доходу кляси І, представлена в неспоживному продукті,
споживається як сталий капітал в клясі II.~Отже, лишається дати звіт про
$\num{4.000} c$ кляси І.~Це покривається з власного продукту кляси І \deq{} \num{6.000}, або радше
$= \num{6.000} — \num{2.000}$, бо ці \num{2.000} вже перетворені на сталий капітал для кляси II.~Слід
зауважити, що числа звичайно взято довільно, отже, і відношення між вартістю
доходу І і вартістю сталого капіталу II є довільне. А проте, очевидно, що коли
процес репродукції відбувається нормально і за інших незмінних умов, тобто,
\parbreak{}  %% абзац продовжується на наступній сторінці

\parcont{}  %% абзац починається на попередній сторінці
\index{iii1}{0248}  %% посилання на сторінку оригінального видання
пересічного робочого часу, суспільно-необхідного для виробництва
товарів. І одночасно зростає концентрація, бо за певними
межами великий капітал з невеликою нормою зиску нагромаджує
швидше, ніж невеликий капітал з великою нормою зиску. Ця
зростаюча концентрація, з свого боку, досягнувши певної висоти,
знов таки приводить до нового падіння норми зиску. Маса дрібних
розпорошених капіталів у наслідок цього штовхається на шлях
авантюр: спекуляцій, шахрайських кредитних і акційних підприємств,
криз. Так звана плетора [наддостаток] капіталу завжди
стосується головним чином до плетори такого капіталу, для якого
падіння норми зиску не урівноважується масою зиску, — а такі
завжди є новоутворювані свіжі паростки капіталу, — або до плетори
таких капіталів, які, будучи самі по собі нездатними самостійно
функціонувати, передаються в формі кредиту в розпорядження
керівників великих галузей підприємств. Ця плетора капіталу виростає
з тих самих обставин, які викликають відносне перенаселення,
і тому вона є явище, яке доповнює це останнє, хоч обоє
вони перебувають на протилежних полюсах: на одному боці — незанятий
капітал, на другому боці — незаняте робітниче населення.

Перепродукція капіталу, а не окремих товарів, — хоч перепродукція
капіталу завжди включає перепродукцію товарів, —
означає через це не що інше, як перенагромадження капіталу.
Щоб зрозуміти, що таке є це перенагромадження (докладніше
дослідження його ми подаємо нижче), досить тільки припустити
його абсолютним. Коли перепродукція капіталу була б абсолютною?
І при тому перепродукція, яка поширювалася б не на ту
чи іншу або декілька значних сфер виробництва, а була б абсолютною
в самому своєму об’ємі, отже, охоплювала б усі сфери
виробництва?

Абсолютна перепродукція капіталу була б у наявності в тому
випадку, коли додатковий капітал для цілей капіталістичного
виробництва був би \deq{} 0. Але метою капіталістичного виробництва
є збільшення вартості капіталу, тобто привласнення додаткової
праці, виробництво додаткової вартості, зиску. Отже,
коли б капітал зріс порівняно з робітничим населенням настільки,
що не можна було б ні здовжити абсолютний робочий час, що
його дає це населення, ні розширити відносний додатковий робочий
час (останнє, крім того, було б нездійсниме при таких обставинах,
коли попит на працю є такий значний, отже, коли є тенденція
до підвищення заробітної плати), тобто коли б зрослий
капітал виробляв тільки таку саму або навіть меншу масу додаткової
вартості, ніж до свого зростання, то мала б місце абсолютна
перепродукція капіталу; тобто зрослий капітал $К \dplus{} ΔК$ виробляв
би не більше зиску, або навіть менше зиску, ніж капітал $К$
до свого збільшення на $ΔК$. В обох випадках мало б також
місце значне і раптове падіння загальної норми зиску, але на
цей раз в наслідок переміни у складі капіталу, викликаної не розвитком
продуктивної сили, а підвищенням грошової вартості
\parbreak{}  %% абзац продовжується на наступній сторінці

\parcont{}  %% абзац починається на попередній сторінці
\index{ii}{0249}  %% посилання на сторінку оригінального видання
їхнього існування, другу — $b$, що її вони почасти витрачають на речі
розкошів, а почасти застосовують на поширення продукції; $а$ — в такому
разі репрезентує змінний капітал, $b$ — додаткову вартість. Але такий поділ
не мав би жодного впливу на величину тієї маси грошей, яка потрібна
для циркуляції цілого їхнього продукту. За інших незмінних умов, вартість
товарової маси, що циркулює, була б та сама, а значить, і маса
потрібних для цього грошей була б та сама. Крім того, при однаковому
поділі періодів обороту продуценти мусили б мати такі самі грошові запаси,
тобто постійно мати в грошовій формі таку саму частину свого
капіталу, бо, згідно з нашим припущенням, їхня продукція, як і раніш,
була б товаровою продукцією. Отже, та обставина, що частина товарової
вартости складається з додаткової вартости, абсолютно не змінює маси
грошей доконечних для провадження підприємства.

Один з супротивників Тука, що тримається формули $Г — Т — Г$, запитує
його, як капіталістові вдається постійно вилучати з циркуляції більше
грошей, ніж він подає туди. Це цілком зрозуміло. Тут ідеться не про
\emph{утворення} додаткової вартости. Останнє, являючи єдину таємницю, з
капіталістичного погляду само собою зрозуміле. Застосована бо сума вартости
не була б капіталом, коли б вона не збагачувалась додатковою
вартістю. А що згідно з припущенням вона є капітал, то додаткова вартість
сама собою зрозуміла.

Отже, питання не в тім, відки береться додаткова вартість, а в тім,
відки беруться гроші, що на них вона перетворюється.

Але для буржуазної економії існування додаткової вартости зрозуміле
само собою. Отже, її не лише припускається, але разом з нею припускається
й те, що частина товарової маси, пущеної в циркуляцію, складається
з додаткового продукту, отже, репрезентує таку вартість, що її капіталіст
не кинув у циркуляцію, кидаючи туди свій капітал; що, отже, капіталіст,
разом з своїм продуктом кидає в циркуляцію певний надлишок
порівняно з своїм капіталом, а потім знову вилучає з неї цей надлишок.

Товаровий капітал, що його капіталіст подає в циркуляцію, має більшу
вартість (звідки це постає, не пояснюється або не розуміється, але з
погляду буржуазної економії c’est un fait\footnote*{
Це — факт. \emph{Ред.}
}, ніж продуктивний капітал,
що його він вилучив з циркуляції в формі робочої сили плюс засоби
продукції. Тому при цьому припущенні ясно, чому не лише капіталіст
$А$, але й $В$, $С$, $D$ і~\abbr{т. ін.} можуть постійно вилучати з циркуляції через
обмін своїх товарів більшу вартість, ніж вартість їхнього первісно авансованого
капіталу, що його потім знову й знову авансується. $А$, $В$, $С$,
$D$ і~\abbr{т. ін.} завжди подають в циркуляцію в формі товарового капіталу, —
а ця операція так само багатобічна, як і самостійно діющі капітали, —
більшу товарову вартість, ніж та, що її вони вилучають з циркуляції в
формі продуктивного капіталу. Отже, їм постійно доводиться розподіляти
між собою суму вартости (тобто кожному доводиться вилучати для себе
з циркуляції продуктивний капітал), що дорівнює сумі вартости їхніх
\parbreak{}  %% абзац продовжується на наступній сторінці

\parcont{}  %% абзац починається на попередній сторінці
\index{iii2}{0250}  %% посилання на сторінку оригінального видання
дійсно цілком сходить на суму вартости, складеної з заробітної плати плюс
зиск плюс рента, тобто на всю вартість трьох доходів, хоч вартість цієї частини
продукту цілком так само, як і тієї, що не входить в дохід, містить частину
вартости \deq{} $C$, рівну вартості сталого капіталу, що міститься в цих частинах,
отже, prima facie не може обмежуватись вартістю доходу, — ця обставина є, з одного
боку, практично безперечним фактом, з другого боку, так само безперечною
теоретичною суперечністю. Цю трудність найпростіше оминають, кажучи, що
товарова вартість лише з подоби, з погляду окремого капіталіста, має в собі
якусь іншу частину вартости, відмінну від частини, сущої в формі доходу.
Фраза: те, що для одного є доходом, для другого становить капітал, позбавляє
потреби всякого дальшого думання. Яким чином, коли вартість усього продукту
належить споживанню у формі доходів, може бути покритий старий капітал; і
яким чином вартість продукту кожного індивідуального капіталу може бути
рівна сумі вартости трьох доходів плюс $C$, сталий капітал, а вся разом складена
сума вартости продукту всіх капіталів дорівнює сумі вартости трьох доходів
плюс 0, — все це звичайно виступає при цьому як нерозв’язна загадка, все це
мусить бути пояснено тим, що аналіза взагалі нездібна виявити прості елементи
ціни та мусить задовольнятись обертанням в порочному колі і відсуванням
задачі до безконечности. Таким чином, те, що з’являється як сталий капітал,
може бути розкладене на заробітну плату, зиск, ренту, а товарові вартості,
що в них репрезентовані заробітна плата, зиск, рента, в свою чергу визначаються
заробітною платою, зиском, рентою і так далі до безконечности\footnote{
«В усякому суспільстві ціна кожного товару кінець-кінцем зводиться до однієї, або другої або до
всіх цих трьох частин [тобто заробітної плати, зиску, ренти]\dots{} Четверта частина, як можна було
припустити, потрібна для покриття капіталу орендаря, або зношування і полагодження і для відновлення
робочої худоби та інших знарядь хліборобства. Але слід зауважити, що ціна хоч би якого
хліборобського знаряддя, наприклад, робочого коня, в свою чергу складається з цих самих трьох
частин: ренти
з тієї землі, що на ній він виріс, праці, витраченої на догляд його і на його годівлю, і зиску
фармера, що авансує й ренту за землю і заробітну плату за працю. Тому, хоч ціна збіжжя покриє як
ціну, так і утримання коня, а проте, вся ціна розпадається безпосередньо або кінець-кінцем на ті
самі три частини: ренту, працю [треба сказати заробітну плату] і зиск». (А.~Сміт). Ми покажемо
пізніш, що А.~Сміт сам розуміє всю суперечність і недостатність цього викруту, бо що ж інше, як не
викрут відсилати
нас від Понтія до Пілата, ніде не показуючи нам тієї дійсної витрати капіталу, що при ній ціна
продукту кінець-кінцем без дальшого відсування, без рештки розпадається на ці три частини.
}.

Фалшива в своїй основі догма, що вартість товарів кінець-кінцем може
бути розкладена на заробітну плату \dplus{} зиск \dplus{} рента, набуває ще й такого виразу,
ніби споживач кінець-кінцем мусить оплатити всю вартість сукупного
продукту; або що грошова циркуляція між продуцентами й споживачами кінець-кінцем
мусить дорівнювати грошовій циркуляції між самими продуцентами
(Tooke); засади, що так само неправдиві, як та основна засада, на яку вони
спираються.

Труднощі, які призводять до цієї помилкової і prima facie абсурдної аналізи,
коротко кажучи такі:

1)~Нерозуміння основного відношення між сталим та змінним капіталом,
отже й природи додаткової вартости, а разом з тим і всієї бази капіталістичного
способу продукції. Вартість кожної частини продукту капіталу, кожного
окремого товару має в собі частину вартости \deq{} сталому капіталові, частину
вартости \deq{} змінному капіталові (перетвореному на заробітну плату робітників)
і частину вартости \deq{} додатковій вартості (пізніше поділяється на зиск і ренту).
Отже, яким же чином можливо, щоб робітник на свою заробітну плату, капіталіст
на свій зиск, земельний власник на свою ренту могли купити товари, що
з них кожен має в собі не тільки одну з цих складових частин, але всі три, і
яким чином можливо, щоб сума вартости заробітної плати, зиску, ренти, отже,
всіх трьох джерел доходу, разом узятих, могла купити товари, що становлять
\parbreak{}  %% абзац продовжується на наступній сторінці

\parcont{}  %% абзац починається на попередній сторінці
\index{iii2}{0251}  %% посилання на сторінку оригінального видання
геть усе споживання одержувачів цих доходів, товари, які, крім цих трьох складових
частин вартости, мають у собі ще одну лишню складову частину вартости,
а саме сталий капітал? Яким чином можуть вони на вартість, що має в
собі три складові частини, купити таку, що має в собі чотири складові частини?\footnote{
Прудон виявляє цілковиту нездібність зрозуміти це в своїй обмеженій формулі: l’ouvrier ne peut
pas racheter son propre produit (робітник не може викупити свого власного продукту), бо в продукті
міститься процент, який додається до prix-de-revient (покупної ціни). Але як напоумляє його кращому
п. Eugène Forcade? «Коли б заперечення Прудона було слушне, воно стосувалося б не тільки profits du
capital (зиск з капіталу), але знищило б саму можливість промисловости. Коли робітник мусить платити
100 за річ, за яку він одержав лише 80, коли заробітна плата може викупити в продукті лише вартість,
вкладену в нього нею самою, то це рівнозначно твердженню, що робітник не може нічого викупити, що
заробітна плата нічого не може оплатити. Справді, в покупній ціні завжди є щось більше, ніж
заробітна плата робітника, а в продажній ціні щось більше, ніж зиск підприємця, наприклад, ціна
сирового матеріялу, часто виплачена закордон\dots{} Прудон забув про безупинний зріст національного
капіталу; він забув, що цей зріст відбувається для всіх робітників, для робітників підприємств так
само, як для ремесників» (Revue des deux Mondes, 1848, t. 24, p. 998). Mu маємо тут перед собою
оптимізм буржуазного безглуздя в найвідповіднішій йому формі глибокодумности. Поперше, п. Forcade
вважає, що робітник не міг би жити, коли б не одержував, крім вартости, яку продукує, ще вищої
вартости, тимчасом як, навпаки, капіталістичний спосіб продукції був би неможливий, коли б робітник
дійсно
одержував вартість, яву він продукує. Подруге, він правильно узагальнює трудність, висловлену
Прудоном лише з певного обмеженого погляду. Ціна товару має в собі надмір не тільки над заробітною
платою, але також і над зиском, а саме сталу частину вартости. Таким чином і капіталіст, згідно з
міркуванням Прудона не міг би викупити товари на свій зиск. Як же розв’язує Forcade загадку?
Безглуздою фразою про зріст капіталу. Отже, постійний зріст капіталу повинен між іншим виявлятися і
в тому, що аналізи товарової ціни, неможлива для економіста при капіталі в 100, стає зайвою при
капіталі в \num{10.000}. Що сказали б ми про хеміка, який на запитання: чим пояснюється, що в
хліборобському продукті міститься більше вуглецю, ніж у самому ґрунті, — відповів би: це пояснюється
постійним зростом хліборобської продукції. Добромисне бажання розкрити в буржуазному світі найкращий
з можливих світів заміняє у вульґарній економії усяку доконечність любови до істини і прагнення до
наукового дослідження.
}.

Ми дали аналізу в книзі II, відділ III.

2) Нерозуміння способу, в який праця, долучаючи нову вартість, зберігає
стару вартість у новій формі, не продукуючи цієї останньої вартости наново.

3) Нерозуміння загального зв’язку процесу репродукції, розглядуваного не
з погляду поодинокого капіталу, а з погляду сукупного капіталу; нерозуміння
труднощів, які є в тому, яким чином продукт, що в ньому реалізується заробітна
плата і додаткова вартість, отже, вся вартість, створена всією новодолученою
протягом року працею, може покривати свою сталу частину вартости, і ще одночасно
зводитися до вартости, що обмежена самими лише доходами; яким чином,
далі, зужиткований у продукції сталий капітал може бути речево і за
вартістю покритий новим, хоч загальна сума новодолученої праці реалізується
лише в заробітній платі і додатковій вартості, і вичерпно визначається в сумі
вартости обох. Саме в цьому і є головна трудність, в аналізі репродукції і в відношенні
її різних складових частин, з боку так їхнього речового характеру, як
і відношень їхньої вартости.

4) Але сюди приєднуються дальші труднощі, які ще збільшуються, скоро
різні складові частини додаткової вартости виступають у формі самостійних один
проти одного доходів. Труднощі ці є в тому, що тверді призначення доходу і
капіталу взаємно міняються, міняють своє місце, так що здаються лише відносними
призначеннями з погляду поодинокого капіталіста, які, як здається, зникають,
скоро ми подивимось на сукупний процес продукції. Наприклад, дохід
робітників і капіталістів кляси І, яка продукує сталий капітал, покриває вартість,
і речовину сталого капіталу кляси капіталістів II, яка продукує засоби споживання.
Отже, можна поминути труднощі з допомогою того уявлення, що те, що
для одного — дохід, для другого — капітал, а тому ці визначення не мають жодного
чинення до дійсного відокремлення складових частин вартости товару. Далі:
товари, призначені кінець-кінцем до того, щоб правити за речові елементи, на
\parbreak{}  %% абзац продовжується на наступній сторінці

\parcont{}  %% абзац починається на попередній сторінці
\index{iii2}{0252}  %% посилання на сторінку оригінального видання
які витрачається дохід, тобто правити за засоби споживання, перебігають протягом
року різні ступені, наприклад, вовняна пряжа, сукно. На одному ступені
вони становлять частину сталого капіталу, на другому — їх особисто споживається,
отже, цілком, входять в склад доходу. Можна, отже, уявити собі разом
з А.~Смітом, що сталий капітал є лише позірний елемент товарової вартости,
який в загальному зв’язку зникає. Таким самим чином відбувається далі обмін
змінного капіталу на дохід. Робітник купує на свою заробітну плату частину
товарів, що становить його дохід. Одночасно він покриває цим самим для капіталіста
грошову форму змінного капіталу. Нарешті: частина продуктів, що становлять
сталий капітал, покривається або in natura, або за посередництвом обміну
між самими продуцентами сталого капіталу; процес, до якого споживачі
не мають жодного чинення. Коли спустити це з уваги, то може постати зовнішня
видимість, що дохід споживачів покриває ввесь продукт, отже і сталу частину
вартости.

5) Крім плутанини, яку вносить перетворення вартостей на ціни продукції,
виникає ще й інша в наслідок перетворення додаткової вартости на різні окремі
форми доходу, самостійні одна проти однієї і залічені до різних елементів продукції,
на зиск і ренту. При цьому забувається, що вартості товарів є основою, і що
розпадання цієї товарової вартости на окремі складові частини, і дальший розвиток
цих складових частин вартости у форми доходу, їх перетворення на відносини
різних посідачів різних чинників продукції до цих окремих складових
частин вартости, їх розподіл між цими посідачами згідно з певними категоріями
і титулами, нічого не змінює у самому визначенні вартости й законів її.
Так само мало змінюється закон вартости тією обставиною, що вирівнювання
зиску, тобто розподіл сукупної додаткової вартости між різними капіталами, і
перешкоди, що почасти (в абсолютній ренті) ставляться землеволодінням на
шляху цього вирівнювання, призводять до відхилу регуляційних пересічних цін
товарів від їхніх індивідуальних вартостей. Це впливає знов таки тільки на
добавку додаткової вартости до цін різних товарів, але не знищує самої додаткової
вартости і сукупної вартости товарів як джерела цих різних складових
частин ціни.

Тут перед нами quid pro quo, яке ми розглядаємо в дальшім розділі, і
яке неминуче зв’язане з ілюзією, що нібито вартість виникає з її власних
складових частин. А саме: спершу різні складові частини вартости товару набувають
в доходах самостійних форм, і як такі доходи їх залічують до окремих
речових елементів продукції, як до їхніх джерел, замість залічити їх до
вартости товару як до їхнього джерела. Вони дійсно залічуються до зазначених
окремих джерел, але не як складові частини вартости, а як доходи, як складові
частини вартости, що дістаються цим певним категоріям агентів продукції: робітникові,
капіталістові, земельному власникові. А проте, можна уявити собі, що
ці складові частини вартости замість виникати від розкладу товарової вартости,
навпаки, лише створюють її своїм сполученням; тоді саме й виникає чудове
порочне коло: вартість товарів виникає з суми вартости заробітної плати, зиску,
ренти, а вартість заробітної плати, зиску, ренти в свою чергу визначається
вартістю товарів і~\abbr{т. ін.}\footnote{
«Оборотний капітал, витрачений на матеріяли, сировий матеріял і викінчені вироби, сам
складається з товарів, що їх потрібна ціна складена з тих самих елементів; так що, розглядаючи
сукупність товарів у певній країні, було б зайво зачислювати цю частину оборотного капіталу до
елементів потрібної ціни». (Storch, Cours d’Ec.~Pol., II, p. 140). Під цими елементами оборотного
капіталу Шторх розуміє (основний — це тільки змінена форма оборотного) сталу частину вартости.
«Правда, що заробітна плата робітника так само як і частина зиску підприємця, яка складається з
заробітних плат, коли розглядати їх як частину засобів існування, і собі складається з товарів, що
куплені по ринковій ціні, і теж мають в собі заробітні плати, ренти на капітали, земельні ренти і
підприємницькі зиски\dots{} спостереження це доводить лише неможливість розкласти потрібну ціну на її
простіші елементи» (ib., примітка). — У своїх Considérations sur la nature du revenu national (Paris
1824)
Шторх y своїй полеміці s Сеєм, правда, розуміє все безглуздя, що до нього призводить помилкова
аналіза товарової вартости, що розкладає її без остачі тільки на доходи, і правильно висловлюється
про все безглуздя цих висновків — з погляду не поодиноких капіталістів, а нації, — але сам він не
робить і кроку вперед в аналізі prix nécessaire (потрібної ціни), відносно якої він, замість
відсувати розв’язання
завдання до безконечности, заявляє в своєму «Cours» що її неможливо розкласти на її дійсні елементи.
«Ясно, що вартість річного продукту поділяється почасти на капітали, почасти на зиски, і що кожна з
цих частин вартости річного продукту регулярно купує продукти, потрібні нації так для збереження її
капіталу, як і для відновлення її споживного фонду (р. 134--135)\dots{} Чи зможе вона (селянська родина,
що працює самостійно) жити у своїх клунях і стайнях, живитись тільки насінням
і травою, одягатися з своєї робочої худоби, витрачати свої хліборобські зваряддя? Згідно з
твердженими п. Сея, слід було б відповісти позитивно на всі ці питання (135--136)\dots{} Коли визнати, що
дохід нації дорівнює її гуртовому продуктові, тобто, що з нього не доводиться вираховувати капітали,
то доведеться також визнати, що вона може непродуктивно витратити всю вартість свого річного
продукту,
не роблячи найменшої шкоди своєму майбутньому доходові. (147) Продукти, що складають капітал нації,
не підлягають споживанню», (р. 150)
}.

\index{iii2}{0253}  %% посилання на сторінку оригінального видання
При нормальному стані продукції тільки частина новодолученої праці
вживається на продукцію і тому на покриття сталого капіталу: саме якраз та
частина, що покриває сталий капітал, витрачений у продукції засобів споживання,
речових елементів доходу. Це вирівнюється тим, що ця стала частина,
кляси II не коштує новодолучуваної праці. Але сталий капітал, який (коли розглядати
сукупний процес репродукції, що в ньому, отже, включено і те вирівнювання
кляс І і II), не є продукт новодолученої праці, хоч цей продукт неможливо
було б випродукувати без нього, — цей сталий капітал, розглядуваний з речового
боку, підлягає підчас процесу репродукції, випадковостям і небезпекам,
які можуть його зменшити. (Але далі, коли розглядати його щодо вартости, то
він також може знецінитися в наслідок зміни у продуктивній силі праці;
проте, це стосується лише до поодиноких капіталістів). Відповідно до цього частина
зиску, отже, додаткової вартости, а тому і додаткового продукту, що в ньому
(коли розглядати його з погляду вартости) репрезентується лише новодолучена
праця, служить страховим фондом. При цьому суть справи ані трохи не змінюється
від того, чи порядкує цим страховим фондом страхове товариство як
окреме підприємство, чи ні. Це є однісінька частина доходу, що не споживається
як такий, і не служить неодмінно фондом акумуляції. Чи служить вона фактично
фондом акумуляції, чи лише покриває прогріхи репродукції, це залежить від
випадку. Це також однісінька частина додаткової вартости і додаткового продукту,
отже, додаткової праці, що крім частини, яка служить для акумуляції, отже,
для поширення процесу репродукції, мусить існувати і далі по знищенні
капіталістичного способу продукції. Звичайно, це має своєю передумовою, що
частина, регулярно споживувана безпосередніми продуцентами, не лишиться обмеженою на своєму
теперішньому мінімальному рівні. За винятком додаткової
праці на тих, хто через свій вік ще не може або вже не може брати участи у продукції,
відпаде всяка праця на утримання тих, хто не працює. Коли ми уявимо
собі суспільство при його виниканні, то побачимо, що тут немає ще випродукованих
засобів продукції, отже, немає сталого капіталу, що його вартість увіходить
у продукт, і при репродукції в тому самому маштабі мусить покриватися
in natura з продукту в розмірі, визначуваному його вартістю. Але природа
безпосередньо дає тут засоби існування, їх не доводиться продукувати. Тому
вона залишає також дикунові, що йому доводиться задовольняти лише малі потреби,
час, — не на те, щоб використати ще не сущі в наявності засоби продукції для
нової продукції, а на те, щоб, крім праці, якої коштує привласнення наявних
у природі засобів існування, витрачати працю на перетворення інших продуктів
природи на засоби продукції, лук, кам’яний ніж, човен і~\abbr{т. ін.} Процес цей,
\parbreak{}  %% абзац продовжується на наступній сторінці

\parcont{}  %% абзац починається на попередній сторінці
\index{iii2}{0254}  %% посилання на сторінку оригінального видання
коли розглядати його тільки з речового боку, цілком відповідає у дикуна зворотному
перетворенню додаткової праці на новий капітал. У процесі акумуляції
безупинно відбувається перетворення такого продукту надмірної праці на капітал;
і та обставина, що всякий новий капітал походить з зиску, ренти або інших
форм доходу, тобто з додаткової праці, призводить до помилкового уявлення,
ніби вся вартість товарів виникає з доходу. Це зворотне перетворення зиску
на капітал при ближчій аналізі показує, навпаки, що додаткова праця, яка завжди
виступає в формі доходу, служить не для збереження, або репродукції старої
капітальної вартости, а, — оскільки її не споживається як дохід — для створення
нового додаткового капіталу.

Вся трудність випливає з того, що вся новодолучена праця, оскільки
створена нею вартість не зводиться до заробітної плати, з’являється в формі
зиску — тут зрозумілого як форма додаткової вартости взагалі, — тобто як
вартість, яка нічого не коштувала капіталістові, що нею, отже, йому не доводиться
покривати нічого авансованого, жодного капіталу. Тому вартість ця існує
в формі вільного, додаткового багатства, або, кажучи коротко, з погляду окремого
капіталіста, у формі його зиску. Але цю новостворену вартість однаково
можна спожити    як продуктивно, так і особисто, однаково і як капітал, і як
дохід. Частина її вже за своєю натуральною формою мусить бути спожита    продуктивно. Отже, ясно, що
щорічно долучувана праця створює так капітал, як і
дохід; як це й виявляється в процесі акумуляції. Але ту частину робочої сили,
яку застосовується на те, щоб наново створити капітал (отже, за аналогією
ту частину робочого дня, яку дикун застосовує не на те, щоб привласнювати
їжу, а на те, щоб виготовляти знаряддя, що ним він привласнює їжу), вже тому
не можна розпізнати, що ввесь продукт додаткової праці, насамперед, виступає
в формі зиску; визначення, яке в дійсності не має нічого до діла з самим
цим додатковим продуктом, а стосується тільки приватного відношення капіталіста
до одержаної ним додаткової вартости. В дійсності додаткова вартість,
створювана робітником, розпадається на дохід і капітал, тобто на засоби споживання
і додаткові засоби продукції. Але старий, залишений від минулого
року, сталий капітал, (залишаючи осторонь ту частину, яка ушкоджена і, отже,
pro tanto, знищена, отже, оскільки його не доводиться репродукувати, — а такі
порушення процесу репродукції стосуються до сфери страхування), розглядуваний
з боку його вартости, не репродукується новодолученою працею.

Ми бачимо далі, що частину новодолученої праці постійно поглинається
в процесі репродукції і покриття зужиткованого сталого капіталу, хоч ця новодолучена
праця розпадається тільки на доходи: заробітну плату, зиск і ренту.
Але при цьому спускається з уваги, 1)~що частина вартости продукту цієї
праці являє собою не продукт цієї новодолученої праці, а вже сущий і спожитий
сталий капітал; що тому та частина продуктів, що в ній виявляється
ця частина вартости, теж не перетворюєтьея на дохід, а покриває in natura
засоби продукції цього сталого капіталу; 2)~що та частина вартости, що в ній
дійсно втілюється ця новодолучена праця, не споживається in natura як дохід,
а покриває сталий капітал в іншій сфері, де він перетворюється на таку натуральну
форму, в якій він може бути спожитий як дохід, але останній в свою
чергу знов таки не є виключно продукт новодолученої праці.

Оскільки репродукція відбувається в незмінному маштабі, кожен спожитий
елемент сталого капіталу мусить покриватись in natura новим екземпляром відповідного роду, коли не
такої самої кількости й форми, то такої самої дієздатности.
Якщо продуктивна сила праці залишається та сама, то це покриття in
natura включає покриття тієї самої вартости, яку сталий капітал мав у своїй
старій формі. Якщо ж продуктивна сила праці збільшується так, що ті самі
речові елементи можна репродукувати меншою працею, то менша частина вартости
\parbreak{}  %% абзац продовжується на наступній сторінці

\parcont{}  %% абзац починається на попередній сторінці
\index{iii2}{0255}  %% посилання на сторінку оригінального видання
продукту зможе цілком покрити in natura сталу частину. Надмір може тоді
придатись до створення нового додаткового капіталу, або можна буде значнішій
частині продукту надати форми засобів споживання, абож зменшити додаткову
працю. Навпаки, коли продуктивна сила праці зменшується, то на покриття старого
капіталу мусить піти значніша частина продукту; додатковий продукт зменшується.

Зворотне перетворення зиску або взагалі будь-якої форми додаткової вартости
на капітал показує, — коли ми абстрагуємось від певної історичної форми
і розглядатимемо це перетворення як просте створення нових засобів продукції, —
що все ще зберігається те становище, коли робітник, крім праці для придбання
безпосередніх засобів існування, витрачає працю на продукцію засобів продукції.
Перетворення зиску на капітал визначає не що інше, як застосування
частини надмірної праці на створення нових додаткових засобів продукції. Та
обставина, що це відбувається в формі перетворення зиску на капітал визначає лише,
що не робітник, а капіталіст порядкує надмірною працею. Що ця надмірна праця
мусить спочатку перейти стадію, в якій вона виступає як дохід (тимчасом як,
наприклад, у дикуна вона виступає як надмірна праця, що спрямована безпосередньо
на продукцію засобів продукції), це визначає лише, що ця праця або
її продукт привласнюється не тим, хто працює. Але в дійсності на капітал
перетворюється не зиск як такий. Перетворення додаткової вартости на капітал
визначає лише, що додаткову вартість і додатковий продукт не споживаються
індивідуально капіталістом як дохід. Такого перетворення в дійсності
зазнає вартість, зрічевлена праця зглядно продукт, що в ньому безпосередньо втілюється ця вартість,
або на яку її обмінюється після попереднього перетворення
на гроші. Так само і тоді, коли зиск знову перетворюється на капітал, не ця
певна форма додаткової вартости, не зиск являє собою джерело нового капіталу.
Додаткова вартість при цьому тільки перетворюється з однієї форми на другу.
Але не це перетворення форми робить її капіталом. Як капітал тепер функціонують
товар і його вартість. Але та обставина, що вартість товару не оплачена, —
а тільки в наслідок цього вона стає додатковою вартістю, — немає жодного значення
для зрічевлення праці, для самої вартости.

Непорозуміння виявляться в різних формах. Наприклад, в тому, що товари,
що з них складається сталий капітал, також мають в собі елементи заробітної
плати, зиску й ренти. Або в тому, що те, що становить для одного дохід, для
іншого становить капітал, і тому це є просто суб’єктивні відношення. Приміром,
пряжа прядуна має частину вартости, яка становить для нього зиск. Отже, ткач,
купуючи пряжу, реалізує зиск прядуна, але для нього самого ця пряжа є лише
частина його сталого капіталу.

Крім того, що вже давніш сказано про відношення доходу і капіталу,
тут слід зауважити: те, що розглядуване з боку вартости, входить разом з пряжею,
як складова частина, в капітал ткача, є саме вартість пряжі. Яким чином
частини цієї вартости розклались для самого прядуна на капітал і дохід, іншими
словами на оплачену і неоплачену працю, — це цілком байдуже для визначення
самої вартости товару (коли абстрагуватись від змін, зумовлюваних пересічним
зиском). Але тут на заднім пляні завжди чигає уявлення, ніби зиск, взагалі
додаткова вартість, є такий надмір над вартістю товару, що створюється в наслідок
лише додачі, взаємного шахрайства, баришів від продажу. Коли оплачується
ціну продукції або навіть вартість товару, оплачується звичайно і ті
складові частини вартости товару, що для його продавця виступають у формі
доходу. Про монопольні ціни тут звичайно немає мови.

Подруге, цілком справедливо, що складові частини товарів, що з них складається
сталий капітал, зводяться, подібно до всякої іншої товарової вартости,
до частин вартости, які для продуцентів і власників засобів продукції розкладаються
на заробітну плату, зиск і ренту. Це є лише капіталістична форма
\parbreak{}  %% абзац продовжується на наступній сторінці

\parcont{}  %% абзац починається на попередній сторінці
\index{iii2}{0256}  %% посилання на сторінку оригінального видання
виразу того факту, що всяка товарова вартість є лише міра вміщеної в товарі
суспільно потрібної праці. Але вже в першій книзі показано, що це ані трохи
не перешкоджає розпаданню товарового продукту всякого капіталу на окремі
частини, що з них одна становить виключно сталу частину капіталу, друга —
змінну частину капіталу і третя — тільки додаткову вартість.

Шторх висловлює не тільки свою думку, але й думку багатьох інших,
коли говорить: «Les produits vendables qui constituent le revenu national doivent
être considérés dans l’économie politique de deux manières différentes: relativement aux
individus comme des valeurs; et relativement à la nation comme des biens; car le
revenu d’une nation ne s’apprécie pas comme celui d’un individu, d’après sa valeur,
mais d’après son utilité ou d’après les besoins auxquels il peut satisfaire» (Consid.
sur le revenu national, p. 19)\footnote*{
«Продукти, що належать продажеві й становлять національний дохід, в політичній економії слід
розглядати подвійно: як вартості щодо осіб і як блага щодо нації; бо дохід нації визначається не
так, як дохід окремого індивідуума, не за його вартістю, а за його корисністю або за тими потребами,
що їх він може задовольнити».
}.

Поперше, це цілком помилкова абстракція, коли націю, що її спосіб продукції
ґрунтується на вартості, що (націю) далі, капіталістично організовану,
розглядають як спільний організм, який працює тільки для задоволення національних
потреб.

Подруге, по знищенні капіталістичного способу продукції, але при збереженні
суспільної продукції, визначення вартости лишається панівним в тому
розумінні, що реґулювання робочого часу і розподіл суспільної праці між різними
галузями продукції, нарешті, бухгальтерія щодо всього, цього стають,
важливіші, ніж колибудь.

\section{Позірна роля конкуренції}

Вже показано, що вартість товарів, або реґульована їхньою сукупною вартістю
ціна продукції, розпадається на:

1) Частину вартости, яка покриває сталий капітал, або репрезентує минулу
працю, зужитковану в формі засобів продукції при виготовленні товару; одним
словом вартість або ціну, що її мали ці засоби продукції, які ввійшли в процес
продукції товару. Ми говоримо тут завжди не про окремі товари, а про товаровий
капітал, тобто про ту форму, що її набирає продукт капіталу за певний
період, наприклад, за рік; — про товаровий капітал, що його окремий товар
становить лише один з його елементів і який, проте, врешті теж, за своєю
вартістю, розпадається на такі самі складові частини, що й товаровий капітал.

2) Частину вартости, яка становить змінний капітал, що виміряє собою
дохід робітника і перетворюється для нього на його заробітну плату, отже, на
заробітну плату, яку робітник репродукував у цій змінній частині вартости;
коротко, це — та частина вартости, що в ній втілюється оплачена частина
праці, новодолученої в продукції товару до першої сталої частини вартости.

3) Додаткову вартість, тобто ту частину вартости товарового продукту,
що в ній втілюється неоплачена або додаткова праця. Ця остання частина
вартости набуває в свою чергу самостійних форм, які одночасно є форми
доходу: форму зиску на капітал (процент на капітал як такий, підприємницький
бариш з капіталу як капіталу, що функціонує) і форму земельної ренти.
\parbreak{}  %% абзац продовжується на наступній сторінці

\parcont{}  %% абзац починається на попередній сторінці
\index{iii1}{0257}  %% посилання на сторінку оригінального видання
полягає саме в тому, що частка живої праці зменшується,
а частка минулої праці збільшується, але так, що загальна сума
вміщеної в товарі праці зменшується; отже, так, що жива праця
зменшується дужче, ніж збільшується минула. Минула праця,
втілена у вартості товару — стала частина капіталу — складається
почасти із зношування основного, почасти з обігового
сталого капіталу, який цілком входить у товар, — сировинного
й допоміжного матеріалу. Та частина вартості, що походить
з сировинного й допоміжного матеріалу, мусить з розвитком
продуктивності праці зменшуватись, бо ця продуктивність відносно
зазначених матеріалів виявляється саме в тому, що їх вартість
знижується. Навпаки, найхарактернішим для зростаючої
продуктивної сили праці є саме те, що основна частина сталого
капіталу дуже значно збільшується, а разом з тим так само
збільшується і та частина його вартості, яка в наслідок зношування
переноситься на товари. Для того, щоб новий метод
виробництва виявився як справжнє підвищення продуктивності,
він мусить переносити на окремий товар меншу додаткову частину
вартості, відповідну зношуванню основного капіталу,
ніж та частина вартості, яка віднімається, заощаджується в наслідок
зменшення живої праці, — одним словом він мусить зменшувати
вартість товару. Само собою зрозуміло, що він мусить
зменшувати її навіть і тоді, коли — як це буває в окремих випадках
— в утворення вартості товару входить, крім додатково зношуваної
частини основного капіталу, додаткова частина вартості
відповідно до більшої кількості або до дорожчих сировинних і
допоміжних матеріалів. Всі надбавки до вартості мусять бути
більше ніж урівноважені зменшенням вартості, яке випливає із
зменшення живої праці.

Тому це зменшення загальної кількості праці, яка входить
у товар, здавалося б, мало бути істотною ознакою підвищеної
продуктивної сили праці, незалежно від того, при яких суспільних
умовах відбувається виробництво. В суспільстві, в якому
виробники регулюють своє виробництво за складеним заздалегідь
планом, і навіть при простому товарному виробництві продуктивність
праці безумовно вимірювалась би цим масштабом.
Але як стоїть справа при капіталістичному виробництві?

Припустім, що певна капіталістична галузь виробництва
виробляє нормальну штуку свого товару при таких умовах: зношування
основного капіталу становить на штуку \sfrac{1}{2}\shil{ шилінга} або
марки; сировинного й допоміжного матеріалу входить у кожну
штуку на 17\sfrac{1}{2}\shil{ шилінгів}; заробітної плати 2\shil{ шилінги}, і при нормі
додаткової вартості в 100\% додаткова вартість становить 2\shil{ шилінги.}
Вся вартість \deq{} 22\shil{ шилінгам} або маркам. Для спрощення
ми припускаємо, що в цій галузі виробництва капітал має пересічний
склад суспільного капіталу, що, отже, ціна виробництва
товару збігається з його вартістю, а зиск капіталіста збігається
з виробленою додатковою вартістю. В такому разі витрати
\parbreak{}  %% абзац продовжується на наступній сторінці

\parcont{}  %% абзац починається на попередній сторінці
\index{iii2}{0258}  %% посилання на сторінку оригінального видання
в 400. Тому створена в зазначений спосіб і визначувана кількістю зрічевленою
в ній праці товарова вартість в 250 становить межу тієї суми дивіденду,
яку робітник, капіталіст і земельний власник можуть здобути з цієї вартости
в формі доходу, в формі заробітної плати, зиску й ренти.

Хай капітал з тим самим органічним складом, тобто з тим самим відношенням
між ужитою живою робочою силою і пущеним в рух сталим капіталом,
мусить платити 150\pound{ фунтів стерл}, замість 100 за ту саму робочу силу,
що пускає в рух сталий капітал в 400; хай далі додаткова вартість поділяється
знов таки, в новому відношенні, на зиск і ренту. А що припущено,
що змінний капітал в 150\pound{ фунтів стерл} пускає в рух ту саму масу праці,
яка раніш пускалась у рух капіталом в 100, то новоспродукована вартість, як
і давніш, дорівнювала б 250, і вартість сукупного продукту, як і давніш, дорівнювала
б 650, але ми мали тоді: $400c \dplus{} 150v \dplus{} 100m$; і ці $100m$ розпадаються,
скажімо, на 45 зиску плюс 55 ренти. Пропорція, що в ній сукупна новостворена
вартість розподіляється тепер між заробітною платою, зиском і рентою,
була б цілком відмінна від колишньої; цілком інша була б також величина
сукупного авансованого капіталу, хоч він пускає в рух ту саму сукупну масу
праці. Заробітна плата становила б 27\sfrac{3}{11}\%, зиск — 8\sfrac{2}{11}\%, рента — 10\% на
авансований капітал; отже, сукупна додаткова вартість трохи більша за 18\%.

В наслідок підвищення заробітної плати змінилася б величина доплаченої
частини сукупної праці, отже, і величина додаткової вартости. При десятигодинному
робочому дні робітникові довелося б тепер 6 годин працювати на
себе і лише 4 години на капіталіста. Відношення між зиском і рентою також
змінилося б; зменшена додаткова вартість ділилася б у новій пропорції між
капіталістом і земельним власником. Нарешті, в наслідок того, що вартість
сталого капіталу лишилась незмінна, а вартість авансованого змінного капіталу
зросла, понижена додаткова вартість виражається в ще пониженішій гуртовій
нормі зиску, під якою ми розуміємо тут відношення сукупної додаткової вартости
до всього авансованого капіталу.

Зміна вартости заробітної плати, норми зиску і норми ренти, хоч би яка
була дія законів, що реґулюють взаємовідношення цих частин, могла б рухатися
лише в межах, визначуваних величиною новостворенної товарової вартости
в 250. Виняток міг би бути лише тоді, коли б рента ґрунтувалась на
монопольній ціні. Це не відмінімо б закону, а лише ускладнило б дослідження.
Коли б ми в цьому випадку почали розглядати тільки самий продукт, то зміна
виявилася б лише в розподілі додаткової вартости; коли ж ми стали б розглядати
відносну вартість цього продукту проти інших товарів, то ми знайшли б
лише ту відмінність, що частина додаткової вартости останніх переноситься на
цей специфічний товар.

Отже, підсумуймо:
\begin{center}
\centering{}
\small

\begin{tabular}{@{}l l@{~}c c c@{}}
Вартість продукту & &\makecell{Новостворена \\ вартість}  & \makecell{Норма\\ додаткової\\ вартости}  & \makecell{Норма\\ гуртового\\зиску} \\
\midrule
Перший випадок: & 400c \dplus{} 100v \dplus{} 150m \deq{} 650   &    250  & 150\phantom{\sfrac{2}{3}}\%     &   30\phantom{\sfrac{2}{11}}\% \\
Другий випадок: & 400c \dplus{} 150v \dplus{} 100m \deq{}  650  &    250  & \phantom{0}66\sfrac{2}{3}\%     &  18\sfrac{2}{11}\% \\
\end{tabular}
\end{center}
%REMOVED\footnotetextZ{У німецькому тексті тут помилково стоїть «66\sfrac{1}{3}\%». \emph{Пр. ред}}

\noindent{}Перш за все додаткова вартість понижується на третину своєї колишньої
величини, з 150 до 100. Норма зиску понижується трохи більш, ніж на одну
третину, з 30\% до 18\%, бо зменшена додаткова вартість обчислюється тепер
на вирослий авансований сукупний капітал. Але вона понижується далеко не
\parbreak{}  %% абзац продовжується на наступній сторінці

\parcont{}  %% абзац починається на попередній сторінці
\index{iii2}{0259}  %% посилання на сторінку оригінального видання
втім відношенні, як норма додаткової вартости. Ця остання з \frac{150}{100} понижується
\frac{100}{150}, отже з 150\% до 66\sfrac{2}{3}\%, тимчасом як норму зиску понижується
лише з \frac{150}{500} до \frac{100}{550}, або з 30\% до 18 2\sfrac{2}{11}\%. Таким чином, відносне пониження
норми зиску більше, ніж пониження маси додаткової вартости, але менше, ніж
пониження норми додаткової вартости. Далі виявляється, що вартості і маси продуктів
лишаються незмінні, коли вживається ту саму кількість праці, що і давніш,
хоча б авансований капітал в наслідок збільшення його змінної складової частини
і збільшився. Це збільшення авансованого капіталу позначилося б, звичайно,
дуже чутливо на капіталісті, що починає нове підприємство. Але з погляду
репродукції в цілому збільшення змінного капіталу визначає не що інше, а тільки
те, що значніша частина вартости, новоствореної новодолученою працею, перетворюється на заробітну
плату, тобто насамперед на змінний капітал, замість
перетворюватись на додаткову вартість і додатковий продукт. Отже, вартість
продукту лишається незмінна, бо вона обмежена, з одного боку, вартістю
сталого капіталу = 400, з другого боку — числом 250, що в ньому визначається
новодолучена праця. Але обидві ці величини лишились незмінні. Продукт цей,
оскільки він сам знову входить в сталий капітал, в даній величині вартости
являє ту саму, що й давніш, масу споживної вартости; отже, та сама маса
елементів сталого капіталу зберігає ту саму вартість. Інакше стояла б справа,
коли б заробітна плата підвищилась не тому, що робітник одержував би більшу
частину своєї власної праці, але коли б він одержував більшу частину своєї
власної праці тому, що понизилась продуктивність праці. В цьому випадку сукупна
вартість, що в ній втілюється та сама кількість праці, оплаченої і неоплаченої,
лишилася б незмінна; але маса продукту, що в ній втілюється вся
кількість праці, зменшилася б, і, отже, зросла б ціна кожної даної частини
продукту, бо кожна така частина являла б більшу кількість праці. Підвищена
заробітна плата в 150 являла б не більше продукту, ніж колишня заробітна
плата в 100; понижена додаткова вартість в 100 являла б лише \sfrac{2}{3} колишнього
продукту, 66\sfrac{2}{3}\% тієї маси додаткових вартостей, які давніш визначалися
в 100. В цьому випадку подорожчав би і сталий капітал, оскільки в нього
входить цей продукт. Але це не було б наслідком підвищення заробітної плати;
— навпаки підвищення заробітної плати було б наслідком подорожчання
товарів і наслідком пониженої продуктивности тієї самої кількости праці. Тут
постає ілюзія, ніби підвищення заробітної плати удорожчує продукт; але тут
підвищення це є не причина, а результат зміни вартости товарів в наслідок
пониженої продуктивности праці.

Коли, навпаки, за інших рівних умов, коли, отже, та сама кількість
вжитої праці визначається, як і давніш, в 250, — вартість застосованих нею
засобів продукції підвищиться або знизиться, то й вартість тієї самої маси продуктів
підвищиться або знизиться на ту саму величину. $450 c + 100v + 150  m$
дає вартість продукту $= 700$; навпаки, $350 c + 100 v + 150 m$ дає для вартости
тієї самої маси продукту лише 600, замість колишніх 650. Отже, коли зростає
або зменшується авансований капітал, пущений в рух тією самою кількістю
праці, тоді, за інших рівних умов, зростає або зменшується і вартість продукту,
коли це збільшення або зменшення авансового капіталу походить із зміни
величини вартости сталої частини капіталу. Навпаки, вона не змінюється, коли
збільшення або зменшення авансованого капіталу походить із зміни величини
вартости змінної частини капіталу при незмінній продуктивності праці. Збільшення
або зменшення вартости сталого капіталу не компенсується жодним протилежним
рухом. Але збільшення або зменшення вартости змінного капіталу, при незмінній
\parbreak{}  %% абзац продовжується на наступній сторінці

\parcont{}  %% абзац починається на попередній сторінці
\index{iii2}{0260}  %% посилання на сторінку оригінального видання
продуктивності праці, компенсується зворотним рухом додаткової вартости, так що
вартість змінного капіталу плюс додаткова вартість, отже, вартість новодолучена
працею до засобів продукції і новостворена в продукті лишається незмінна.

Навпаки, коли збільшення або зменшення вартости змінного капіталу або
заробітної плати є наслідок подорожчання або пониження ціни товарів, тобто
наслідок зменшення або збільшення продуктивности праці, вжитої в цій сфері
приміщення капіталу, то це впливає на вартість продукту. Але підвищення
або пониження заробітної плати є тут не причина, а тільки наслідок.

Навпаки, коли б у вищенаведеному прикладі при незмінному сталому
капіталі $= 400 c$, зміна $100 v \dplus{} 150 m$ на $150 v \dplus{} 100 m$, отже, підвищення
змінного капіталу, було наслідком пониження продуктивної сили праці не в даній
окремій галузі продукції, наприклад, у бавовнопрядінні, але, скажімо, в хліборобстві,
що постачає робітникові харчові продукти, — отже, було б наслідком
подорожчання цих харчових продуктів, то вартість продуктів не змінилася б.
Вартість в 650, як і давніш, визначалася б тією самою масою бавовняної пряжі.

З викладеного вище випливає далі таке: коли в наслідок економії тощо,
зменшуються витрати сталого капіталу в тих галузях продукції, що їхні продукти
входять в споживання робітника, то це, так само, як і безпосереднє зростання
продуктивности самої ужитої праці, може призвести до зменшення заробітної
плати, бо це здешевлює засоби існування робітника, а тому це може
призвести до підвищення додаткової вартости; так що норма зиску зростає тут
з двох причин, а саме: з одного боку, тому, що зменшується вартість сталого
капіталу, і, з другого боку, тому, що збільшується додаткова вартість. Розглядаючи
перетворення додаткової вартости на зиск, ми припускали, що заробітна
плата не понижується, а лишається сталою, бо там нам треба було дослідити
коливання норми зиску, незалежно від зміни норми додаткової вартости. Крім
того, розвинуті нами там закони мають загальний характер, вони мають силу
і для тих приміщень капіталу, що їхні продукти не входять в споживання робітника,
і що зміни вартости їхнього продукту не впливають тому на заробітну плату.

\pfbreak

Отже, відокремлення і розпад вартости, яку новоприєднувана праця щорічно
знову долучає до засобів продукції, або до сталої частини капіталу, на різні форми
доходу: на заробітну плату, зиск і ренту — ані трохи не змінює межі самої вартости,
тієї суми вартости, що розподіляється між цими різними категоріями; так
само як зміна відношення між цими окремими частинами не може змінити суми їх,
цієї даної величини вартости. Дане число 100 залишається завжди тим самим, чи
розкладемо ми його на $50 \dplus{} 50$ чи на $20 \dplus{} 70 \dplus{} 10$, чи на $40 \dplus{} 30 \dplus{} 30$. Та
частина вартости продукту, що розпадається на ці доходи, є визначена, як і стала
частина вартости капіталу, вартістю товарів, тобто кількістю праці, зрічевленою
в них в кожному даному випадку. Отже, поперше, дано величину вартости
товарів, яка розподіляється на заробітну плату, зиск і ренту; дано, отже, абсолютну
межу суми окремих частин вартости цих товарів. Подруге, щодо самих
цих категорій, то дано також їхні пересічні і реґуляційні межі. Заробітна плата
становить базу цього останнього обмеження. Вона, з одного боку, реґулюється
природним законом; її мінімальна межа дана фізичним мінімумом засобів існування,
потрібних робітникові для збереження і репродукції його робочої сили;
дана, отже, як певна кількість товарів. Вартість цих товарів визначається робочим
часом, потрібним для їхньої репродукції; отже, тієї частиною праці, новодолученої
до засобів продукції, або тією частиною кожного робочого дня, яку
робітник витрачає на продукцію і репродукцію еквіваленту вартості цих потрібних
засобів існування. Коли, наприклад, пересічна вартість його засобів
існування за день дорівнює 6 годинам пересічної праці, то він мусить пересічно
працювати на себе 6 годин на день. Дійсна вартість його робочої сили
\parbreak{}  %% абзац продовжується на наступній сторінці

\parcont{}  %% абзац починається на попередній сторінці
\index{iii1}{0261}  %% посилання на сторінку оригінального видання
(Звичайно, здешевлення сталого капіталу в усіх цих галузях
може підвищити норму зиску при незмінній експлуатації робітника.)
Як тільки новий метод виробництва починає поширюватись,
і цим фактично дається доказ того, що ці товари можуть
вироблятись дешевше, то капіталісти, які працюють при старих
умовах виробництва, мусять продавати свій продукт нижче
його повної ціни виробництва, бо вартість цього товару впала,
робочий час, потрібний їм для виробництва цього товару, стоїть
вище суспільного. Одним словом, — і це виявляється як наслідок
конкуренції, — вони так само мусять запровадити новий метод
виробництва, при якому відношення змінного капіталу до
сталого є менше.

Всі обставини, які спричиняють те, що застосування машин
здешевлює ціну товарів, вироблюваних цими машинами,
завжди зводяться до зменшення тієї кількості праці, яку вбирає
одиниця товару; а подруге, вони зводяться до зменшення
зношуваної частини машин, вартість якої входить в одиницю
товару. Чим повільнішим є зношування машин, тим на більшу
кількість товарів воно розподіляється, тим більше живої праці
заміщають вони до строку їх репродукції. В обох випадках
збільшується кількість і вартість основного сталого капіталу
порівняно із змінним.

„All other things being equal, the power of a nation to save from
its profits varies with the rate of profits, is great when they are high,
less, when low; but as the rate of profit declines, all other things do
not remain equal\dots{} A low rate of profit is ordinarily accompanied by
a rapid rate of accumulation, relatively to the numbers of the people,
as in England\dots{} a high rate of profit by a slower rate of accumulation,
relatively to the numbers of the people“. [„При всіх інших
однакових умовах спроможність нації робити заощадження з своїх
зисків змінюється із зміною норми зиску; вона більша, коли
норма зиску — висока, менша, коли вона — низька; але якщо
норма зиску знижується, всі інші умови не лишаються однаковими\dots{}
Низька норма зиску звичайно супроводиться швидким
темпом нагромадження порівняно з чисельністю населення,
як в Англії\dots{} висока норма зиску — повільнішим темпом нагромадження
порівняно з чисельністю населення“.] Приклади: Польща,
Росія, Індія і~\abbr{т. д.} (\emph{Richard Jones}: „An Introductory Lecture on Political
Economy“, London 1833, стор. 50 і далі). Джонс правильно відзначає,
що, не зважаючи на падаючу норму зиску, inducements
and faculties to accumulate [спонуки до нагромадження і можливості
нагромаджувати] збільшуються. Поперше, в наслідок зростаючого
відносного перенаселення. Подруге, тому, що з зростанням
продуктивності праці збільшується маса споживних вартостей,
представлених тією самою міновою вартістю, отже, збільшується
маса речових елементів капіталу. Потретє, тому що
збільшується різноманітність галузей виробництва. Почетверте,
в наслідок розвитку кредитної системи, акційних товариств і~\abbr{т. д.}
\parbreak{}  %% абзац продовжується на наступній сторінці


\index{ii}{0262}  %% посилання на сторінку оригінального видання
Але в наслідок додаткового продуктивного капіталу в циркуляцію
подається, як продукт його, додаткову товарову масу. Разом з цією
додатковою товаровою масою подається в циркуляцію частину додаткових
грошей, потрібних для реалізації її — а саме подається остільки, оскільки
вартість цієї товарової маси дорівнює вартості продуктивного капіталу,
зужиткованій на її продукцію. Цю додаткову масу грошей авансується
прямо як додатковий грошовий капітал, і тому він зворотно припливає
до капіталіста в наслідок обороту його капіталу. Тут перед нами знову
постає те саме питання, що й раніш. Звідки беруться додаткові гроші на
реалізацію додаткової вартости, що є тепер у товаровій формі в цій
додатковій масі товарів?

Загальна відповідь знову та сама. Сума цін товарової маси, яка циркулює,
збільшилась не тому, що ціна даної товарової маси підвищилась,
а тому, що маса товарів, які тепер циркулюють, більша за масу товарів,
що циркулювали раніше, і при цьому ця ріжниця не вирівнюється зниженням
цін. Додаткові гроші, потрібні для циркуляції цієї більшої товарової
маси більшої вартости, треба здобути або посиленою економією на
масі грошей, що циркулюють, — чи то через взаємне вирівнювання платежів
тощо, чи то засобами, які прискорюють обіг тієї самої монети, —
або їх треба здобути через перетворення грошей з форми скарбу на
обігову форму грошей. Останнє включає не лише те, що бездіяльний
грошовий капітал починає функціонувати як купівельний засіб
або як засіб виплати; або і не лише те, що грошовий капітал, який уже
функціонує як резервний фонд, виконуючи для свого власника функцію
резервного фонду, активно циркулює для суспільства (як от банкові
вклади, що їх завжди дається в позику), отже, виконує двоїсту функцію;
це перетворення включає й те, що заощаджується стаґнаційні монетні
резервні фонди.

„Щоб гроші постійно обігали як монети, монети мусять постійно
осідати як гроші. Постійний обіг монет зумовлено тим, що їх постійно
затримується більшими або меншими кількостями як монетні резервні
фонди, що всюди утворюються в межах циркуляції й зумовлюють її, —
монетні резервні фонди, що їх утворення, розподіл, розпад і нове утворення
завжди чергуються, резервні фонди; що буття їх постійно зникає,
що процес їх зникання ніколи не припиняється. Це безперестанне перетворення
монет на гроші й грошей на монети А.~Сміт висловив таким
чином, що кожен товаровласник поряд з тим особливим товаром, що
його він продає, завжди мусить мати в запасі певну суму загального
товару, що на нього він купує. Ми бачили, що в циркуляції $Т — Г — Т$
другий член $Г — Т$ постійно розпадається на ряд актів купівлі, які відбуваються
не одноразово, а послідовно в часі, так що одна частина $Г$
обігає як монета, тимчасом як друга перебуває в стані спокою як гроші.
Тут гроші справді є лише монети, що їхнє функціонування відкладено,
і окремі складові частини монетної маси, що обігає, завжди з’являються,
чергуючися, то в одній, то в другій формі. Тому, це перше перетворення
засобу циркуляції на гроші являє собою лише технічний момент самого
\parbreak{}  %% абзац продовжується на наступній сторінці

\parcont{}  %% абзац починається на попередній сторінці
\index{iii2}{0263}  %% посилання на сторінку оригінального видання
вартість між двома посідачами цього самого чинника продукції. Але та обставина,
що тут немає певної закономірної межі для поділу пересічного зиску
на окремі частини, не знищує меж його самого як частини товарової вартости,
так само як та обставина, що два спільника якогось підприємства з якихось
зовнішніх обставин ділять поміж собою зиск не нарівно, ніяк не зачіпає меж
цього зиску.

Отже, коли та частина товарової вартости, що в ній визначається праця,
новодолучена до вартости засобів продукції, розпадається на різні частини, що
набувають у формі доходів самостійного вигляду одна проти однієї, то звідси ще
зовсім не випливає, що заробітну плату, зиск і ренту треба розглядати, як
конституційні елементи, з сполучення або підсумовування яких виникає реґуляційна
ціна (natural price, prix nécessaire) самих товарів, — так що при цьому
вже не товарова вартість по вирахуванні з неї сталої частини вартости була б
первісно даною одиницею, яка поділяється на зазначені три частини, а навпаки,
ціна кожної з цих трьох частин являла б собою величину, визначувану самостійно,
і лише з комплексу цих трьох незалежних величин складалася б ціна
товару. В дійсності вартість товару є величина наперед дана, сукупність загальної
вартости заробітної плати, зиску й ренти, хоч би які були відносні величини їх
поміж себе. Навпаки, при зазначеному помилковому погляді заробітна плата,
зиск і рента є три самостійні величини вартости, що їх сукупна величина
створює, обмежує і визначає величину товарової вартости.

Насамперед, ясно, що коли б заробітна плата, зиск і рента конституювали
ціну товарів, то це в однаковій мірі мало б силу так до сталої частини товарової
вартости, як і до другої її частини, що в ній визначається змінний капітал
і додаткова вартість. Отже, ця стала частина може тут бути залишена
осторонь, бо вартість товарів, з яких вона складається, так само зводиться до
суми вартости заробітної плати, зиску й ренти. Як уже зазначено, цей погляд
заперечує також саме існування такої сталої частини вартости.

Ясно далі, що тут відпадає саме розуміння вартости. Лишається тільки
уявлення про ціну, в тому розумінні, що посідачам робочої сили, капіталу
й землі виплачується певну суму грошей. Але що таке гроші? Гроші не річ,
а певна форма вартости, отже в свою чергу мають своєю передумовою вартість.
Отже, припустімо, що певну кількість золота або срібла виплачується за ті елементи
продукції, або що їх у гадці прирівнюється до цієї кількости. Але ж
золото й срібло (і освічений економіст пишається з цього відкриття) сами є
товаром, як усякі інші товари. Отже, ціну золота й срібла також визначається
заробітною платою, зиском і рентою. Отже, ми не можемо заробітну плату, зиск
і ренту визначити тим, що їх можна прирівняти до певної кількости золота
і срібла, бо вартість цього золота і срібла, якою ми хочемо виміряти їх, як їхнім
еквівалентом, має ще й собі лише визначитись саме ними, незалежно від золота
й срібла, тобто незалежно від вартости всякого товару, яка сама є продукт зазначених
трьох чинників. Отже, сказати, що вартість заробітної плати, зиску
й ренти є в тому, що вони дорівнюють певній кількості золота та срібла, значить
лише сказати, що вони дорівнюють певній кількості заробітної плати,
зиску й ренти.

Візьмімо насамперед заробітну плату. Бо і з цього погляду праця мусить
бути за вихідний пункт. Отже, чим визначається реґуляційна ціна заробітної
плати, та ціна, що навколо неї коливаються її ринкові ціни?

Скажімо, попитом і поданням робочої сили. Але про який попит на робочу
силу іде тут мова? Про попит від капіталу. Отже, попит на працю рівнозначний
поданню капіталу. Щоб говорити про подання капіталу, ми мусимо
насамперед знати, що таке капітал. З чого складається капітал? Коли взяти
його найпростішу форму: з грошей і товарів. Але гроші є лише форма товару-
\parbreak{}  %% абзац продовжується на наступній сторінці

\parcont{}  %% абзац починається на попередній сторінці
\index{iii1}{0264}  %% посилання на сторінку оригінального видання
випадки циркуляції товарного капіталу почасти змішують з своєрідними
функціями купецького або товарно-торговельного капіталу;
почасти вони на практиці сполучаються з його своєрідними
специфічними функціями, хоча з розвитком суспільного
поділу праці функція купецького капіталу розвивається і в чистому
вигляді, тобто відокремлюється від цих реальних функцій
і усамостійнюється щодо них. Отже, для нашої мети, де справа
йде про визначення специфічної відмінності цієї особливої форми
капіталу, слід абстрагуватись від згаданих функцій. Оскільки
капітал, який функціонує тільки в процесі циркуляції, спеціально
товарно-торговельний капітал, почасти сполучає ці функції з своїми,
він виступає не в своїй чистій формі. Відкинувши, усунувши
ці функції, ми матимем чисту форму товарно-торговельного капіталу.

Ми бачили, що буття капіталу як товарного капіталу і та
метаморфоза, яку він, як товарний капітал, проробляє в сфері
циркуляції, на ринку, — метаморфоза, яка зводиться до купівлі
й продажу, до перетворення товарного капіталу в грошовий капітал
і грошового капіталу в товарний капітал, — становить фазу
процесу репродукції промислового капіталу, отже, фазу сукупного
процесу виробництва його; але разом з тим ми бачили, що
в цій своїй функції, як капітал циркуляції, він відрізняється
від себе самого, як продуктивного капіталу. Це — дві окремі,
відмінні форми існування одного й того ж капіталу. Частина
сукупного суспільного капіталу постійно перебуває на ринку
в цій формі існування як капітал циркуляції, в процесі цієї метаморфози,
хоч для кожного окремого капіталу його буття як
товарного капіталу і його метаморфоза як такого становить
тільки постійно зникаючий і постійно відновлюваний перехідний
пункт, перехідну стадію безперервності його процесу виробництва,
і хоч через це елементи товарного капіталу, який перебуває
на ринку, постійно змінюються, бо вони постійно витягаються
з товарного ринку і так само постійно повертаються на нього
як новий продукт процесу виробництва.

Отож, товарно-торговельний капітал є не що інше, як перетворена
форма частини цього капіталу циркуляції, який постійно
перебуває на ринку, постійно перебуває в процесі метаморфози
і постійно охоплений сферою циркуляції. Ми кажемо: частини,
бо певна частина продажу й купівлі товарів постійно відбувається
безпосередньо між самими промисловими капіталістами.
Цю частину ми в цьому дослідженні залишаємо цілком осторонь,
бо вона нічого не дає для визначення поняття, для зрозуміння
специфічної природи купецького капіталу, а, з другого
боку, для нашої мети ми вичерпно дослідили її вже в книзі II.

Торговець товарами, як капіталіст взагалі, виступає на ринок
насамперед як представник певної суми грошей, яку він авансує
як капіталіст, тобто яку він хоче перетворити з $x$ (первісної
вартості цієї суми) $x \dplus{} Δx$ (цю суму плюс зиск на неї). Але
\parbreak{}  %% абзац продовжується на наступній сторінці

\parcont{}  %% абзац починається на попередній сторінці
\index{iii1}{0265}  %% посилання на сторінку оригінального видання
для нього не тільки як для капіталіста взагалі, а спеціально як
для торговця товарами, само собою очевидно, що його капітал
первісно мусить з’явитись на ринку в формі грошового капіталу,
бо він не виробляє ніяких товарів, а тільки торгує ними, опосереднює їхній рух, а для того, щоб
торгувати ними, він мусить
їх спочатку купити, отже, мусить бути володільцем грошового
капіталу.

Припустімо, що якийсь торговець товарами володіє 3000 фунтами стерлінгів, які він збільшує в їх
вартості як торговельний
капітал. На ці 3000 фунтів стерлінгів він купує у фабриканта,
який виробляє полотно, наприклад, \num{30000} метрів полотна по
2 шилінги за метр. Він продає ці \num{30000} метрів. Якщо пересічна
річна норма зиску = 10\% і якщо він, після відрахування всіх
накладних витрат, одержує 10\% річного зиску, то на кінець року
він перетворить ці 3000 фунтів стерлінгів у 3300 фунтів стерлінгів. Яким чином він одержує цей зиск
— це питання, яке ми
розглянемо тільки пізніше. Тут ми насамперед розглянемо тільки
форму руху його капіталу. На ці 3000 фунтів стерлінгів він
весь час купує полотно і весь час продає це полотно; він постійно повторює цю операцію купівлі для
продажу, $Г — Т — Г'$,
просту форму капіталу, в якій цей капітал цілком зв’язаний у процесі циркуляції, не перериваному
інтервалами процесу виробництва, який лежить поза його власним рухом і функцією.

Яке ж є відношення цього товарно-торговельного капіталу
до товарного капіталу як простої форми існування промислового
капіталу? Щодо фабриканта полотна, то він грішми купця реалізував вартість свого полотна, виконав
першу фазу метаморфози свого товарного капіталу, перетворення його в гроші, і може
тепер, при інших незмінних умовах, знову перетворити гроші
у пряжу, вугілля, заробітну плату і~\abbr{т. д.}, з другого боку — в засоби існування і~\abbr{т. д.} для
споживання свого доходу; отже, залишаючи осторонь витрачання доходу, він може продовжувати процес
репродукції.

Але, хоч для нього, для виробника полотна, вже відбулася
метаморфоза полотна в гроші, його продаж, вона ще не відбулася для самого полотна. Як і раніш,
полотно перебуває на ринку
як товарний капітал і має призначення виконати свою першу
метаморфозу, бути проданим. З цим полотном нічого не сталося,
крім переміни особи його володільця. За своїм призначенням, за
своїм становищем у процесі воно, як і раніше, є товарний капітал, продажний товар; тільки тепер воно
перебуває в руках
купця, а не в руках виробника, як це було раніш. Функція його
продажу, опосереднення першої фази його метаморфози, забрана
від виробника купцем і перетворена в його спеціальне заняття, — тимчасом як раніше це була функція,
яку мав виконувати виробник після виконання функції виробництва полотна.

Припустімо, що купцеві не вдалося продати \num{30000} метрів
протягом того періоду часу, який потрібний виробникові
\parbreak{}  %% абзац продовжується на наступній сторінці

\parcont{}  %% абзац починається на попередній сторінці
\index{iii1}{0266}  %% посилання на сторінку оригінального видання
полотна для того, щоб знову кинути на ринок \num{30000} метрів вартістю в 3000\pound{ фунтів стерлінгів}. Купець
не може їх знову купити, бо він ще має на складі непроданих \num{30000} метрів, які ще
не перетворились для нього в грошовий капітал. Тоді настає
застій, перерив репродукції. Виробник полотна міг би, звичайно,
мати в своєму розпорядженні додатковий грошовий капітал, який
він міг би, незалежно від продажу цих \num{30000} метрів, перетворити
в продуктивний капітал і таким чином продовжувати процес
виробництва. Але таке припущення зовсім не змінює справи.
Оскільки справа йде про капітал, авансований на ці \num{30000} метрів,
процес його репродукції є і лишається перерваним. Отже, тут
дійсно з очевидністю виявляється, що операції купця є не що
інше, як операції, які взагалі мусять бути виконані для того, щоб
перетворити товарний капітал виробника у гроші; операції, які
опосереднюють функції товарного капіталу в процесі циркуляції і репродукції. Якщо замість
незалежного купця цим продажем і, крім того, закупівлею повинен був би займатись як виключною
справою простий прикажчик виробника, то цей зв’язок ні
на одну хвилину не був би прихований.

Отже, товарно-торговельний капітал є безперечно не що
інше, як товарний капітал виробника, капітал, який повинен проробити процес свого перетворення в
гроші, виконати на ринку
свою функцію як товарний капітал; тільки тепер ця функція виступає не як побічна операція виробника,
а як виключна операція особливого роду капіталістів, торговців товарами, усамостійнюється як заняття
в особливій сфері капіталовкладення.

Зрештою, це виявляється і в специфічній формі циркуляції
товарно-торговельного капіталу. Купець купує товари і потім
продає їх: $Г — Т — Г'$. В простій товарній циркуляції або навіть
в циркуляції товарів, якою вона виступає як процес циркуляції
промислового капіталу, $Т' — Г — Т$, циркуляція опосереднюється
тим, що кожний грошовий знак двічі міняє місце. Виробник полотна продає свій товар, полотно,
перетворює його в гроші;
гроші покупця переходять у його руки. На ці самі гроші він
купує пряжу, вугілля, працю і~\abbr{т. д.}, знову витрачає ці самі
гроші, щоб зворотно перетворити вартість полотна в товари, які
становлять елементи виробництва полотна. Товар, який він купує, не той самий товар, товар не того
самого роду, який
він продає. Він продав продукти і купив засоби виробництва. Але
інакше стоїть справа в русі купецького капіталу. На 3000\pound{ фунтів стерлінгів} торговець полотном купує
\num{30000} метрів полотна;
він продає ці самі \num{30000} метрів полотна, щоб одержати назад
з циркуляції грошовий капітал (3000\pound{ фунтів стерлінгів}, крім
зиску). Отже, тут двічі міняє місце не той самий грошовий знак,
а той самий товар; він переходить з рук продавця в руки покупця і з рук покупця, який тепер став
продавцем, в руки
іншого покупця. Він продається двічі і може бути проданий
ще багато разів при втручанні в справу ряду купців; і якраз
\parbreak{}  %% абзац продовжується на наступній сторінці

\parcont{}  %% абзац починається на попередній сторінці
\index{iii1}{0267}  %% посилання на сторінку оригінального видання
тільки за допомогою цього повторного продажу, за допомогою
дворазової зміни місця того самого товару, перший купець одержує назад гроші, авансовані на купівлю
товару; тільки цим
опосереднюється повернення до нього цих грошей. В одному
випадку, випадку $Т' — Г — Т$, дворазова зміна місця тих самих грошей опосереднює те, що товар
відчужується в одній
формі і привласнюється в другій. В другому випадку, випадку
$Г — Т — Г'$, дворазова зміна місця того самого товару опосереднює те, що авансовані гроші знову
вилучаються з циркуляції.
Саме при цьому й виявляється, що товар ще не проданий остаточно, якщо він перейшов з рук виробника
до рук купця, що
цей останній тільки продовжує операцію продажу, або обслуговування функції товарного капіталу. Але
разом з тим виявляється, що те, що́ для продуктивного капіталіста є $Т — Г$,
проста функція його капіталу в його минущій формі товарного
капіталу, те для купця є $Г — Т — Г'$, особливим процесом збільшення вартості авансованого ним
грошового капіталу. Одна
фаза метаморфози товару виявляється тут, щодо купця, як
$Г — Т — Г'$, отже, як еволюція капіталу особливого роду.

Купець остаточно продає товар, в даному разі полотно,
споживачеві, однаково, чи це буде продуктивний споживач (наприклад, білільник), чи особистий, який
використовує полотно
для свого особистого споживання. В наслідок цього до купця
повертається назад авансований капітал (разом із зиском), і він
може знову почати цю операцію. Коли б при купівлі полотна
гроші функціонували тільки як платіжний засіб, так що купцеві довелося б платити тільки через шість
тижнів після одержання товару, і коли б він його продав раніше, ніж мине цей
час, то він міг би заплатити виробникові полотна за його товар,
не авансувавши особисто ніякого грошового капіталу. Коли б він
не продав товар, то він мусив би авансувати 3000\pound{ фунтів стерлінгів} при настанні строку платежу,
замість того, щоб авансувати їх відразу при здачі йому полотна; а коли б він в наслідок падіння
ринкових цін продав товар нижче купівельної ціни,
то він мусив би замістити недібрану частину з свого власного капіталу.

Що ж надає товарно-торговельному капіталові характеру
самостійно функціонуючого капіталу, тимчасом як у руках виробника, який сам продає свої товари, він,
очевидно, виступає
тільки як особлива форма його капіталу в особливій фазі процесу його репродукції, під час його
перебування в сфері циркуляції?

\emph{Поперше}: Та обставина, що товарний капітал пророблює своє
остаточне перетворення в гроші, отже, свою першу метаморфозу, виконує на ринку властиву йому qua
[як] товарному капіталові функцію, перебуваючи в руках агента, відмінного від
виробника цього товарного капіталу; і та обставина, що ця функція товарного капіталу опосереднюється
операціями купця, його
\parbreak{}  %% абзац продовжується на наступній сторінці


\index{iii2}{0268}  %% посилання на сторінку оригінального видання
Отже, за таких припущень, коли б вартість товарів була і уявлялась величиною
сталою, коли б частина вартости товарового продукту, що розкладається
на доходи, лишалася сталою величиною і саме такою уявлялася, коли б, нарешті,
ця дана і стала частина вартости розпадалася на заробітну плату, зиск і ренту
завжди в незмінній пропорції, — навіть за таких припущень дійсний рух неминуче
мусив би виступати в перекрученому вигляді: не як розпад наперед даної
величини вартости на три частини, що набувають форми незалежних один від
одного доходів, але навпаки, як створення цієї вартости із суми незалежних,
самостійно визначуваних кожен сам по собі, складових її елементів: заробітної
плати, зиску й земельної ренти. Ця ілюзія виникла б неминуче, бо в дійсному
русі окремих капіталів і їхніх товарових продуктів не вартість товару виступає
як передумова її розпаду на складові частини, а навпаки, складові частини, на які
вони розпадаються, виступають як передумова вартости товарів. Насамперед, для
окремого капіталіста, як ми бачили, витрати продукції товару виступають
як дана величина, що як така, завжди визначається в дійсній ціні продукції.
Але витрати продукції дорівнюють вартості сталого капіталу, авансованих засобів
продукції, плюс вартість робочої сили, яка проте, в очах аґентів продукції набуває
іраціональної форми ціни праці, так що заробітна плата одночасно виступає
як дохід робітника. Пересічна ціна праці є величина дана, бо вартість робочої
сили, як і всякого іншого товару, визначається робочим часом, потрібним
для її репродукції. Щодо тієї частини вартости товарів, яка становить заробітну
плату, то вона виникає не з того, що вона набуває цієї форми заробітної плати,
не з того, що капіталіст авансує робітникові його частину в його власному
продукті в зовнішній формі заробітної плати, а в наслідок того, що робітник
створює еквівалент, відповідний до його заробітної плати, тобто протягом певної
частини своєї щоденної або річної праці створює вартість, яка міститься в ціні
його робочої сили. Але заробітну плату встановлюється контрактом раніш, ніж
випродуковано відповідний їй еквівалент вартости. Як елемент ціни, що його
величина дана раніш, ніж випродуковано товар і товарову вартість, як складова
частина витрат продукції, заробітна плата виступає тому не як частина,
що в самостійній формі відривається від усієї вартости товару, а навпаки, як
величина дана, що наперед визначає всю вартість товару, тобто, як чинник,
що створює ціну або вартість. Ролю аналогічну тій, що її заробітна плата
відіграє у витратах продукції товару, пересічний зиск відіграє в ціні продукції
товару, бо ціна продукції дорівнює витратам продукції плюс пересічний
зиск на авансований капітал. Цей пересічний зиск практично входить в
уявлення і розрахунки самого капіталіста, як реґуляційний елемент, — і не
тільки в тому розумінні, що він реґулює перенесення капіталу з однієї сфери
приміщення в другу, але й також при всяких купівлях і контрактах, що охоплюють
процес репродукції за довший період. Але оскільки це так, остільки пересічний
зиск є наперед дана величина, що дійсно не залежить від вартости і додаткової
вартости, створюваної в кожній певній галузі продукції, а тому тим паче —
кожним окремим капіталом, приміщеним у межах кожної такої галузі. Пересічний
зиск у своєму зовнішньому вияві видається не наслідком розпаду вартости,
а радше величиною, незалежною від вартости товарового продукту, наперед даною
в процесі продукції товарів і до того ж визначальною для пересічної ціни
товарів, тобто чинником, що створює вартість. При цьому додаткова вартість
в наслідок розпадання її різних частин, на цілком незалежні одна від однієї
форми, виступає в ще конкретнішій формі як передумова створення вартости
товарів. Частина пересічного зиску, в формі проценту, протистоїть — самостійно
як елемент, що є передумова продукції товарів і їхньої вартости, — капіталістові,
що функціонує. Хоч би як коливалась величина проценту, за кожного
даного моменту і для кожного даного капіталіста вона є величина дана, що
\parbreak{}  %% абзац продовжується на наступній сторінці

\parcont{}  %% абзац починається на попередній сторінці
\index{iii2}{0269}  %% посилання на сторінку оригінального видання
для нього — поодинокого капіталіста — входить в склад витрат продукції, продукованих
ним товарів. Так само стоїть справа для хліборобського капіталіста з
земельною рентою у формі встановленої контрактом орендної плати, і в формі
комірного за промислові будівлі для інших підприємців. А що ці частини, на які
розпадається додаткова вартість, з’являються для кожного окремого капіталіста
як дані елементи його витрат продукції, то й видаються вони, навпаки, чинниками,
що створюють додаткову вартість: чинниками, що створюють одну
частину товарової ціни, подібно до того, як заробітна плата створює її другу
частину. Таємниця того, чому ці продукти розпаду товарової вартости завжди
здаються передумовами самого створення вартости, є просто в тому, що капіталістичний
спосіб продукції, як і всякий інший, невпинно репродукує не тільки
матеріяльний продукт, але й суспільно-економічні відносини, економічно певні
форми його утворення. Тому наслідок цього процесу продукції так само постійно
набуває вигляду його передумов, як його передумови — вигляду його наслідку.
І саме ця невпинна репродукція тих самих відносин антиципується окремим
капіталістом як сам собою зрозумілий факт, що не підлягає жодному
сумнівові. Поки капіталістична продукція як така продовжує існувати, одна частина
новодолученої праці постійно перетворюється на заробітну плату, друга
на зиск (процент і підприємницький бариш), третя — на ренту. При складанні
контрактів між власниками різних елементів продукції це є передумова, і ця
передумова правильна, хоч би як коливалась в кожному окремому випадку відносна
величина зазначених трьох частин. Та певна форма, в якій протистоять
одна одній частини вартости, є передумова, бо вона постійно репродукується, і
вона постійно репродукується, бо вона постійно є передумова.

Правда, досвід і зовнішній вигляд явищ показують також, що ринкові
ціни, вплив яких видається капіталістові дійсно єдиним чинником, що визначає
вартості, — що ці ринкові ціни, розглядувані з боку їхньої величини, зовсім не
залежать від цих антиципацій капіталіста, зовсім не рівняються за тим, високий
чи низький процент, висока чи низька рента, зумовлені контрактом. Але
ринкові ціни є сталі лише в зміні, і їхня пересічна за довші періоди саме і дає
відповідні пересічні для заробітної плати, зиску й ренти як сталі величини,
отже, кінець-кінцем панівні над ринковими цінами.

З другого боку, дуже простою здається така думка: коли заробітна плата,
зиск і рента є вартостетворчі чинники, тому що вони видаються передумовами
продукції вартости, і для окремого капіталіста входять як такі передумови в
витрати продукції і ціни продукції, то і стала частина капіталу, що її вартість
входить в продукцію кожного товару як дана величина, є вартостетворчий чинник.
Але стала частина капіталу є не що інше, як сума товарів, а тому і товарових
вартостей. Отже, це сходить на вульґарну тавтологію, що товарова вартість
є витворець і причина товарової вартости.

\looseness=1
Але коли б капіталіст мав якийсь інтерес поміркувати над цим, —
а його міркування як капіталіста визначається виключно його інтересами
і мотивами, що випливають з цих інтересів, — то досвід покаже йому, що
продукт, який він сам продукує, входить в інші сфери продукції як стала
частина капіталу, а продукти цих інших сфер продукції, входять в його продукт
як стала частина капіталу. А що для нього, оскільки в нього відбувається
нова продукція, новостворена вартість складається, як здається, з трьох
величин — заробітної плати, зиску й ренти, — то це, як здається, має силу
і щодо сталої частини, яка складається з продуктів інших капіталістів:
а тому ціна сталої частини капіталу і тим самим і сукупна вартість товарів,
зводиться кінець-кінцем, що правда, не послідовним шляхом, до суми вартости,
яка постає з складання заробітної плати, зиску й ренти, як самостійних
\parbreak{}  %% абзац продовжується на наступній сторінці

\input{iii.2/_0270.tex}
\parcont{} %% абзац починається на попередній сторінці
\index{iii2}{0271}  %% посилання на сторінку оригінального видання
й ренти входять в розрахунки, як сталі й реґуляційні величини, — сталі не в тому
розумінні, що величини ці не зміняються, а в тому розумінні, що в кожному окремому
випадку вони є дані і становлять сталу межу для ринкових цін, які безупинно
коливаються. Наприклад, при конкуренції на світовому ринку справа
йде виключно про те, чи можна при даній заробітній платі, даному проценті й
даній ренті продати товар по даній загальній ринковій ціні або нижче цієї ціни
з вигодою, тобто реалізуючи при цьому відповідний підприємницький бариш.
Коли в одній країні заробітна плата і ціна землі низькі, а процент на капітал
високий, бо капіталістичний спосіб продукції тут взагалі нерозвинений,
тимчасом як в іншій країні заробітна плата і ціна землі номінально високі, а
процент на капітал низький, то капіталіст у першій країні вживає більше праці
й землі, в другій порівняно більше капіталу. Оцінюючи, в якій мірі можлива
конкуренція між обома цими капіталістами, обидва зазначені чинники треба
взяти на увагу як визначальні елементи. Отже досвід показує тут теоретично, а
заінтересовані розрахунки капіталіста показують практично, що ціни товарів
визначаються заробітною платою, процентом і рентою, ціною праці, капіталу
й землі, і що ці елементи ціни дійсно є реґуляційні, цінотворчі чинники.

Природна річ, при цьому завжди лишається один елемент, який є не
передумова, а наслідок ринкової ціни товарів, — саме, надмір над витратами продукції, що постають з
складання зазначених вище елементів: заробітної плати,
проценту й ренти. Цей четвертий елемент, як здається, визначається в кожному
окремому випадку конкуренцією, а в пересічному з цих випадків — пересічним
зиском, який знов таки реґулюється тією самою конкуренцією, тільки за довший
період.

\emph{Поп’яте}. На базі капіталістичного способу продукції розпад вартости, що
в ній втілюється новодолучена праця, на доходи в формі заробітної плати, зиску
й земельної ренти є остільки сам собою зрозумілий, що цю методу застосовується
також там, де немає навіть умов існування цих форм доходу (ми не говоримо
вже про колишні історичні періоди, що їх ми наводили при дослідженні
земельної ренти). Це значить, що під зазначені форми доходу підводиться за
аналогією все що завгодно.

Коли самостійний робітник — візьмімо дрібного селянина, бо тут є застосовні
всі три форми доходу — працює на себе самого і продає свій власний продукт, то
його розглядається, поперше, як свого власного працедавця (капіталіста), що вживає
самого себе як робітника; подруге, як свого власного земельного власника, що править
для самого себе за орендаря. Як найманому робітникові, він виплачує собі заробітну
плату, як капіталістові дає собі зиск, як земельному власникові платить
собі ренту. Припускаючи, що капіталістичний спосіб продукції і відповідні
йому відносини є загальною соціяльною базою, це підведення є слушне
остільки, оскільки самостійний робітник не своїй праці, а своїй власності на
засоби продукції, — що взагалі кажучи, набули тут форми капіталу, — завдячує
тим, що він має змогу привласнити свою власну додаткову працю. І далі,
оскільки він продукує свій продукт як товар і тому залежить від ціни останнього
(а коли навіть цього й немає, то ціну все таки треба взяти на увагу),
маса додаткової праці, що він її може реалізувати як вартість, залежить не від
її власної величини, а від загальної норми зиску; так само той надмір, що його він
можливо одержує понад певну частку додаткової вартости, визначувану загальною
нормою зиску, залежить знов таки не від кількости витраченої ним праці, але
може бути привласнений ним лише через те, що він є власник землі. А що
такі форми продукції, які цілком не відповідають капіталістичному способові
продукції, можуть бути підведені — і до певної міри не без слушности — під капіталістичні форми
доходу, то тим дужче зміцнюється ілюзія, ніби капіталістичні
відносини є природні відносини всякого способу продукції.
\parbreak{}  %% абзац продовжується на наступній сторінці


\index{iii2}{0272}  %% посилання на сторінку оригінального видання
Коли звести заробітну плату до її загальної основи, тобто до тієї частини
продукту власної праці, що входить в особисте споживання робітника: коли
звільнити цю частину від капіталістичних обмежень і поширити споживання до
такого розміру, який, з одного боку, допускається наявною продуктивною силою
суспільства (тобто суспільною продуктивною силою його власної праці як дійсно
суспільної) і якого, з другого боку, потребує цілковитий розвиток індивідуальности; коли звести далі
додаткову працю, й додатковий продукт до тих розмірів,
які при даних суспільних умовах продукції потрібні, з одного боку, для створення
страхового і резервного фонду, з другого боку, для безупинного поширення
репродукції в тій мірі, що визначається суспільною потребою; коли включити
нарешті, в № 1, в потрібну працю, і в № 2, в додаткову працю, ту кількість
праці, що її мусять завжди виконувати працездатні члени суспільства на ще
непрацездатних або вже непрацездатних членів суспільства; отже, коли таким
чином усунути всі специфічно капіталістичні риси так у заробітній платі, як і в
додатковій вартості, так у потрібній, як і в додатковій праці, — тоді перед
нами залишаться вже не ці форми, а лише їхні основи, спільні всім суспільним
способам продукції.

Проте, треба сказати, що таке підведення було властиве і колишнім панівним
способам продукції, наприклад, февдальному. Продукційні відносини, що
цілком не відповідали йому, стояли цілком поза ним, підводились під февдальні
відносини, наприклад в, Англії tenures in common socage\footnote*{
Володіння на основі панщини. \Red{Пр.~Ред.}
} (протилежно до tenures
оn knight’s service)\footnote*{
Володіння на основі рабськоі праці. \Red{Пр.~Ред.}
}, які мали в собі виключно грошові зобов’язання
лише з назви були февдальними.

\section{Розподільчі відносини й продукційні відносини}

Вартість, новостворювана щорічно нововитрачуваною працею, — отже, і та
частина річного продукту, що в ній визначається ця вартість і яка може бути
вилучена, виділена з сукупного продукту, — розпадається, отже на три частини,
що набувають трьох різних форм доходу, форм, які виражають одну частину
цієї вартости, як належну або взагалі припалу посідачеві робочої
сили, другу — посідачеві капіталу, третю — посідачеві земельної власности.
Отже, це є відносини або форми розподілу, бо вони визначають ті відносини,
що в них сукупна новоспродукована вартість розподіляється між посідачами
різних чинників продукції.

Згідно з звичайним поглядом ці відносини розподілу виступають як природні
відносини, відносини, що виникають з природи всякої суспільної продукції, з
законів людської продукції взагалі. Хоч і немає можливости заперечувати, що
докапіталістичні суспільства виявляють інші способи розподілу, проте, ці останні
тлумачаться, як нерозвинені, недосконалі й замасковані, що не досягли свого
найчистішого виразу і своєї найвищої форми, своєрідно забарвлені різностаті
цих природних розподільчих відносин.

В такому уявленні правильне лише одно: коли дано суспільну продукцію,
хоч би якого роду (наприклад, коли дано суспільну продукцію природно вирослої
індійської громади або більш штучно розвиненого перуанського комунізму),
то завжди можна відрізнити ту частину праці, що її продукт безпосередньо
особисто споживається продуцентами та їхніми    родинами, — лишаючи
осторонь працю, що припадає продуктивному споживанню, — від тієї
\parbreak{}  %% абзац продовжується на наступній сторінці


\index{ii}{0273}  %% посилання на сторінку оригінального видання
На основі суспільної продукції треба визначити маштаб, що в ньому
такі операції, які на довгий час відтягують робочу силу й засоби продукції,
не даючи протягом цього часу жодного продукту як корисного
наслідку, можуть провадитись без шкоди для тих галузей продукції, які
постійно або кілька разів на рік не лише відтягують робочу силу й засоби
продукції, а й дають засоби, існування й засоби продукції. За суспільної
продукції, так само, як і за капіталістичної продукції, робітники
в галузях підприємств з короткими робочими періодами, як і раніше, лише
на короткий час відтягуватимуть продукти, не даючи натомість нового
продукту, тимчасом як галузі підприємств з довгими робочими періодами,
перше ніж вони сами почнуть давати продукти, постійно відтягують
продукти на довгий час. Отже, ця обставина випливає з речових
умов відповідного процесу праці, а не з його суспільної форми. За суспільної
продукції грошовий капітал відпадає. Суспільство розподіляє робочу
силу й засоби продукції між різними галузями праці. Продуценти
можуть, правда, одержувати паперові посвідки, що ними вони вилучають
з суспільних споживних запасів ту кількість продуктів, яка відповідає їхньому
робочому часові. Ці посвідки — зовсім не гроші. Вони не циркулюють.

Тепер ми бачимо, що, оскільки потреба в грошовому капіталі випливає
з протягу робочого періоду, її зумовлено двома обставинами: \emph{поперше},
тією, що гроші взагалі є та форма, що в ній мусить виступити
кожен індивідуальний капітал (кредит ми лишаємо осторонь) для того,
щоб перетворитись на продуктивний капітал. Це випливає з суті капіталістичної
продукції, взагалі товарової продукції. — \emph{Подруге}, величину
потрібного грошового авансування зумовлює та обставина, що протягом
порівняно довгого часу суспільству постійно відбирається робочу силу
й засоби продукції, при чому протягом цього часу йому не повертається
жодного продукту, що його можна було б перетворити на гроші.
Першої обставини, а саме того, що авансовуваний капітал треба авансувати
в грошовій формі, не знищує форма самих цих грошей, тобто те,
що вони є або металеві, або кредитові гроші, або знаки вартости й~\abbr{т. ін.} На другу обставину жодного впливу не справляє те, за допомогою
яких грошових засобів або за допомогою якої форми продукції
відтягають працю, засоби існування та засоби продукції, не подаючи
натомість у циркуляцію жодного еквіваленту.
\label{original-273}

\parcont{}  %% абзац починається на попередній сторінці
\index{iii2}{0274}  %% посилання на сторінку оригінального видання
цьому мають на увазі різні титули на ту частину продукту, що припадає
особистому споживанню. Навпаки, ті розподільчі відносини, про які ми щойно
говорили, є базою окремих суспільних функцій, що припадають в межах самого
продукційного відношення певним діячам його, у протилежність безпосереднім
продуцентам. Вони надають самим умовам продукції та їхнім представникам специфічно
суспільної якости. Вони визначають увесь характер і ввесь рух продукції.

Дві характеристичні риси від самого початку відзначають капіталістичний
спосіб продукції:

\emph{Поперше}. Він продукує свої продукти як товари. Не продукування товарів
відрізняє цей спосіб продукції від інших способів продукції, а те, що для
його продуктів переважною і визначальною рисою є їхній товарний характер. Це
включає насамперед і те, що сам робітник виступає як продавець товару, і тому
як вільний найманий робітник, отже, праця виступає взагалі як наймана праця.
Після всього того, що ми виклали до цього часу, зайво було б знову доводити
тут, як відношення між капіталом і найманою працею визначає ввесь характер даного
способу продукції. Головні діячі самого цього способу продукції, капіталіст і
найманий робітник, являють, як такі, лише втілення, персоніфікацію капіталу і
найманої праці; це — певні суспільні характери, що їх надає індивідуумам суспільний
процес продукції; це — продукти даних певних суспільних продукційних
відносин.

Характер 1)~продукту як товару, і 2)~товару як продукту капіталу вже
включає всю сукупність відносин циркуляції, тобто включає певний суспільний
процес, що його мусять проробити продукти, і в якому вони набирають
певного суспільного характеру; так само він включає певні відносини між
діячами продукції, відносини, що ними визначається вживання їхнього продукту
і його зворотне перетворення чи на засоби існування, чи на засоби
продукції. Але навіть лишаючи це осторонь, з зазначених вище двох характеристичних
особливостей продукту як товару, або товару як капіталістично
випродукованого товару, випливає все визначення вартости і регулювання вартістю
сукупної продукції. В цій цілком специфічній формі вартости праця має
значення, з одного боку, тільки як суспільна праця; з другого боку, розподіл
цієї суспільної праці і її взаємне довершення, обмін речовин її продуктів, її
упідлеглення перебігові суспільного механізму і включення в цей останній, —
все це підпадає випадковим потягам поодиноких капіталістичних продуцентів,
потягам, що взаємно знищуються. А що ці капіталістичні продуценти протистоять
один одному лише як товаропосідачі, при чому кожен намагається
продати свій товар можливо дорожче (і навіть при урегулюванні продукції
керується нібито тільки своєю сваволею), то внутрішній закон пробивається
лише за посередництвом їхньої конкуренції, їхнього взаємного тиснення один
на одного, що ним взаємно знищуються всі відхили. Лише як внутрішній
закон, що виступає проти окремих діячів продукції як сліпий закон природи,
діє тут закон вартости і проводить суспільну рівновагу продукції серед її випадкових
флюктуацій.

Далі, вже в товарі, і ще більше в товарі як продукті капіталу, включено
зрічевлення суспільно-продукційних визначень і уособлення матеріяльних
основ продукції, що характеризує ввесь капіталістичний спосіб продукції.

\emph{Друга} особливість, що спеціяльно визначає капіталістичний спосіб продукції,
це є продукція додаткової вартости як безпосередня мета і визначальний
мотив продукції. Капітал продукує переважно капітал, і він досягає цього
лише остільки, оскільки продукує додаткову вартість. При дослідженні відносної
додаткової вартости і, далі, при дослідженні перетворення додаткової вартости
на зиск, ми бачили, як на цьому ґрунтується характеристичний для капіталістичного
періоду спосіб продукції, — особлива форма розвитку суспільних продук-
\parbreak{}  %% абзац продовжується на наступній сторінці

\parcont{}  %% абзац починається на попередній сторінці
\index{iii2}{0275}  %% посилання на сторінку оригінального видання
продуктивних сил праці, які набувають, проте, проти робітника вигляду самостійних
сил капіталу і перебувають тому в безпосередній суперечності з власним його,
робітника, розвитком. Продукція ради вартости і додаткової вартости включає,
як показали наші дальші досліди, безупинно діющу тенденцію, що намагається
звести робочий час, потрібний для продукції товару, тобто вартість останнього,
до сущої в кожний даний момент суспільної пересічної. Прагнення звести витрати
продукції до їхнього мінімуму стає найсильнішим знаряддям підвищення
суспільної продуктивної сили праці, яка проте, лише тут видається безупинним
підвищенням продуктивної сили капіталу.

Той авторитет, що його набуває капіталіст як персоніфікація капіталу в
безпосередньому процесі продукції, та суспільна функція, яку він має як керівник
і владар продукції, посутньо відмінні від авторитету, що виростає на базі
продукції рабської, крепацької тощо.

Тимчасом як на базі капіталістичної продукції масі безпосередніх продуцентів
протистоїть суспільний характер їхньої продукції, в формі суворого реґуляційного
авторитету й цілком розвиненої ієрархії суспільного механізму їхнього
трудового процесу, — при чому, одначе, цього авторитету його носії набувають
лише як персоніфікація умов праці в протилежність самій праці, а не як політичні
або теократичні владарі, як це було за давніших форм продукції — серед
самих носіїв цього авторитету, серед самих капіталістів, які протистоять один
одному лише як товаропосідачі, панує цілковита анархія, що в її рямцях суспільний
зв’язок продукції здійснюється лише як могутній закон природи наперекір
індивідуальній сваволі.

Тільки в наслідок того, що праця в формі найманої праці і засоби продукції
в формі капіталу дані як передумова — отже, тільки в наслідок цієї специфічно
суспільної форми цих двох істотних чинників продукції — частина вартости
(продукту) виступає як додаткова вартість і ця додаткова вартість як
зиск (рента), як бариш капіталіста, як додаткове, що є в його розпорядженні,
належне йому, багатство. Але тільки тому, що вона виступає таким чином як
\emph{його зиск}, додаткові засоби продукції, що призначені для поширення репродукції
і становлять частину зиску капіталіста, виступають як додатковий капітал,
а поширення процесу репродукції взагалі виступає як процес капіталістичної
акумуляції.

Хоч наймана праця є форма праці, що має вирішне значення для
форми всього процесу і для специфічного характеру самої продукції, проте,
не найманою працею визначається вартість. При визначенні вартости справа
йде про суспільний робочий час взагалі, про кількість праці, що нею взагалі
може порядкувати суспільство, і поглинення якої в різних пропорціях різними
продуктами визначає, так би мовити, їхню питому суспільну вагу. Та певна форма,
що в ній суспільний робочий час визначально здійснюється у вартості товарів,
перебуває звичайно в зв’язку з найманою працею як формою праці, і капіталом
як відповідною формою засобів продукції остільки, оскільки лише на цій базі
товарова продукція стає загальною формою продукції.

Розгляньмо, проте, так звані розподільчі відносини сами по собі. Заробітна
плата має своєю передумовою найману працю, зиск — капітал. Отже, ці певні
форми розподілу мають своєю передумовою певний суспільний характер умов
продукції і певні суспільні відносини діячів продукції. Отже, певні розподільчі
відносини є лише вираз історично певних продукційних відносин.

А тепер візьмімо зиск. Ця певна форма додаткової вартости є передумова
того, що створення нових засобів продукції відбувається в формі капіталістичної
продукції; отже, це є відношення, що панує над репродукцією, хоч окремому
капіталістові і здається, що він міг би власне проїсти ввесь свій зиск як дохід. Він
наражається, проте, при цьому на межі, що постають перед ним уже у формі
\parbreak{}  %% абзац продовжується на наступній сторінці

\parcont{}  %% абзац починається на попередній сторінці
\index{iii2}{0276}  %% посилання на сторінку оригінального видання
фонду страхування і резервного фонду, закону конкуренції і~\abbr{т. ін.} і практично
доводять йому, що зиск не є просто категорія розподілу продукту, призначеного
для особистого споживання. Далі, ввесь капіталістичний процес продукції реґулюється
цінами продукту. Але реґуляційні ціни продукції в свою чергу реґулюється
процесом вирівнювання норми зиску і відповідним їй розподілом капіталу
між різними сферами суспільної продукції. Отже, зиск з’являється тут, як головний
чинник, не розподілу продукту, а самої його продукції, чинником розподілу
капіталів і самої праці між різними сферами продукції. Розпадання зиску на
підприємницький бариш і процент виступає як поділ того самого доходу. Але воно
виникає насамперед з розвитку капіталу як вартости, що сама з себе зростає, створює
додаткову вартість, — з розвитку цієї певної суспільної форми панівного
процесу продукції. Воно розвиває з себе кредит і кредитові заклади, і разом з
тим дану форму продукції. У формі проценту і~\abbr{т. ін.} позірні форми розподілу
входять до складу ціни як визначальні продукційні моменти.

Щодо земельної ренти, то могло б здатися, що вона є просто розподільча
форма, бо земельна власність, як така, не виконує в самому процесі продукції
жодної, принаймні, жодної нормальної функції. Але та обставина, що 1)~рента
обмежується надміром над пересічним зиском, 2)~що земельний власник з керівника
і владаря процесу продукції і всього життєвого суспільного процесу зводиться
до ролі звичайного здавача землі в оренду, земельного лихваря, звичайного
одержувача ренти, — ця обставина є специфічний історичний витвір капіталістичного
способу продукції. Те явище, що земля набула форми земельної
власности, є історична передумова цього способу продукції. Та обставина, що
земельна власність набуває форм, які допускають капіталістичний спосіб провадження
сільського господарства, є витвір специфічного характеру цього способу
продукції. Дохід земельного власника і при інших формах суспільства
можна було б назвати рентою. Але він був би істотно відмінний від ренти,
якою вона є при цьому способі продукції.

Отже, так звані розподільчі відносини, виникають з історично певних,
специфічно суспільних форм процесу продукції і тих відносин, в які вступають
між собою люди в процесі репродукції свого людського життя, — виникають з цих
форм процесу продукції і відносин і їм відповідають. Історичний характер цих
відносин розподілу є історичний характер продукційних відносин, і визначають вони
тільки один бік останніх. Капіталістичний розподіл відрізняється від тих форм
розподілу, що постають з інших способів продукції, і кожна форма розподілу зникає
разом з певного формою продукції, що з неї вона постає і якій вона відповідає.

Погляд, що згідно з ним, як історично дані розглядається лише розподільчі
відносини, а не продукційні відносини, є, з одного боку, лише погляд щойно пробудженої,
але ще не визволеної критики буржуазної економії. Але, з другого боку,
він ґрунтується на сплутуванні і ототожнюванні суспільного процесу продукції з
простим процесом праці, що його мусила б чинити і протиприродно ізольована
людина, поза всякою допомогою суспільства. Оскільки процес праці є тільки процес
між людиною і природою, його прості елементи спільні всім формам суспільного
розвитку. Але кожна певна історична форма цього процесу розвиває далі його матеріяльні
основи і його суспільні форми. Досягнувши певного ступеня достиглости,
дана історична форма усувається і поступається місцем вищій формі. Що момент
такої кризи наступив, виявляється, скоро лише суперечність і протилежність між
розподільчими відносинами, а тому й історично певного формою відповідних їм відносин
продукції, з одного боку, і продуктивними силами, продуктивною здібністю
і розвитком її чинників, з другого боку, набуває широти й глибочини. Тоді вибухає
конфлікт між матеріяльним розвитком продукції та її суспільною формою\footnote{
Дивись працю про Competition and Cooperation (1832?)
}.


\index{iii2}{0277}  %% посилання на сторінку оригінального видання

\section{Кляси}

Власники самої тільки робочої сили, власники капіталу й земельні власники
що їхніми відповідними джерелами доходів є заробітна плата, зиск і земельна рента,
отже, наймані робітники, капіталісти й земельні власники становлять три великі
кляси сучасного суспільства, яке ґрунтується на капіталістичному способі
продукції.

В Англії сучасне суспільство своєю економічною структурою досягло безперечно
найвищого клясичного розвитку. Проте і тут це клясове розчленування не
виступає ще в цілком чистому вигляді. Навіть і тут середні й переходові ступені
всюди затемнюють межові лінії (правда в селі геть менше, ніж у містах).
А втім це не має значіння для нашого досліду. Ми вже бачили, що постійна
тенденція і закон розвитку капіталістичного способу продукції є в тому, що
засоби продукції дедалі більше відокремлюються від праці, і розпорошені засоби
продукції дедалі більше концентруються в значних масах, що, отже, праця
перетворюється на найману працю, а засоби продукції на капітал. І цій тенденції
відповідає на другому боці самостійне відокремлювання земельної власности
від капіталу й праці\footnote{
F. List слушно зауважує: «Перевага самодостатнього господарства у великих маєтках
свідчить тільки про брак цивілізації, засобів комунікації, тубільних промислів та багатих міст. Тому
ми й знаходимо його всюди в Росії, у Польщі, Угорщині, Мекленбурзі. Давніш воно панувало і в Англії;
з розвитком торговлі й промислу на його місці став поділ на господарства середньої величини та
здавання в оренду» (Die Ackerverfassung, die Zwergwirtschaft und die Auswanderung, 1842, p. 10).
}, або перетворення всякої земельної власности на форму
земельної власности, відповідну капіталістичному способові продукції.

Найближче питання, на яке треба відповісти, таке: що утворює клясу?
причому відповідь ця випливає сама собою з відповіді на інше питання: що
робить з найманих робітників, капіталістів і землевласників утворювачів трьох
великих суспільних кляс?

На перший погляд, це є тотожність доходів і джерел доходу. Перед нами
три великі суспільні групи, що їх члени, індивідууми, які утворюють ці групи,
живуть відповідно з заробітної плати, зиску й земельної ренти, використовуючи
свою робочу силу, свій капітал і свою земельну власність.

Але з цього погляду лікарі і урядовці, наприклад, становили б теж дві
кляси, бо вони належать до двох різних суспільних груп, причому члени кожної
з цих двох груп одержують свої доходи з того самого джерела. Те саме мало б
силу щодо безконечної роздрібнености інтересів і становищ, що до неї призводить
суспільний розподіл праці так серед робітників, як і серед капіталістів та
земельних власників, — напр., розчленовуючи останніх на посідачів виноградників,
орної землі, лісів, копалень, риболовель.

\begin{center}
[Тут рукопис уривається].
\end{center}

\parbreak{}  %% абзац продовжується на наступній сторінці


  \setcounter{footnote}{0}% Reset footnote counter

\bookpages{Додаток}{Фрагмент «Капіталу» у~перекладі Івана~Франка}{}
  \addtocontents{toc}{\protect\booktocentry{Додаток}{Фрагмент «Капіталу»\protect\par у~перекладі Івана~Франка}}

\nonumsection{Чи застарів «застарілий» Маркс?}{~}{Іван Дзюба}

Почати з того, що Маркс застарівав уже не один раз. Спершу — ще після 
революцій 1848 року, які розвивалися не за логікою «Комуністичного 
маніфесту». Потім — після невдачі Паризької Комуни. Далі — коли його 
відмодельовували в протилежні боки «ревізіоністи» (від Бернштейна до 
Каутського) і Ленін, більшовики, Сталін\ldots{} А скільки в нього 
застарілих окремих формул і тез! Наприклад, колись знамените про 
«ідіотизм сільського життя». Це ж як воно звучить тепер, коли світ 
потерпає від набагато глибшого і страшнішого ідіотизму мегаполісів?!


Про те, що Карл Маркс застарів, знає у нас навіть той, хто взагалі 
нічого не знає (особливо він). Але, хоч як дивно, всупереч нашому знанню 
і незнанню, у західних університетах його праці поважно вивчають 
(звісно, з певної критичної позиції), про нього пишуть видатні 
соціологи й філософи як про одного з великих мислителів людства, його 
перевидають і шукають у нього стежок до пояснення економічних криз 
сучасного світу тощо, — а 5-го травня цього року широко відзначалося 
його 200-річчя. Але все це — «за бугром». У нас же будь-який 
малоосвічений публіцист може при нагоді поглузувати з «двох 
німецьких гномів» (це довелося зустріти недавно в інтернетному 
дописі). Таке от бачення (власне, небачення) історії, такий рівень 
культури мислення, таке розуміння динаміки інтелектуального розвитку 
людства, за якої насправді нові осягнення виростають із 
«застарілого», а заперечуване відходить, тільки стимулювавши саме 
заперечення, або й повертається невпізнане.


Ще одна біда — коли говорити про «масову людину» — брак історичного 
підходу в поцінуванні культурних явищ та феноменів думки, понятійних 
категорій. Багато хто в простоті душевній гадає, що це Маркс придумав 
класи, класову боротьбу, пролетаріат, революції та інший клопіт, отож 
усі біди від нього, Маркса. Таким чином на нього ніби падає 
відповідальність за століття (або й тисячоліття) соціальних 
конфліктів і майнових битв на нашій планеті. Хай так. Але на 
виправдання Маркса можна сказати, що в нього було багато попередників. 
Не будемо зазирати в біблійні, або античні, чи й середньовічні часи, а 
звернімося до ХIХ ст., в якому й визрівало те, що згодом дістало назву 
марксизму.

\looseness=1
Отож: у межах європейських феодальних монархій народжується і набирає 
сил буржуазія, що здобуває економічні й політичні позиції, 
використовуючи суперечності між монархом і його васалами та свої 
фінансово-майнові важелі; відбувається нагромадження капіталу, 
розвивається фабрична промисловість, змінюється характер виробничих 
і суспільних відносин та способи експлуатації робітника, колишні 
дрібні власники й обезземелені селяни стають знекоріненою і 
безправною «робочою силою», зростає безробіття. Пролетаризація 
охоплює цілі суспільні верстви, збільшуючи зубожіння люду й набираючи 
катастрофічного характеру. Як відповідь на біди, що їх приніс масі 
населення, насамперед трудовому людові, бурхливий розвиток 
жорстокого й хижацького капіталізму; як відповідь на загострення 
соціальних антагонізмів та масових злиднів — виникають, з одного 
боку, стихійні бунти, наприклад, луддитів та інших руйначів машин, 
потім і бунти та масові революційні рухи, а з другого — народжуються 
соціальні міфи й утопії та спроби окремих гуманістично мислячих, з 
чутливою соціальною совістю особистостей, як правило з освічених, 
упривілейованих станів, запропонувати моделі подолання кричущих 
суспільних дисгармоній і шляхи влаштування справедливих відносин, 
бодай у окремих локальних осередках, якщо не взагалі у світі. Так 
народжується утопічний соціалізм, яскрава плеяда теоретиків якого 
(К.~А.~Сен-Симон, Ш.~Фур'є, Р.~Оуен, а також Т.~Мюнцер, Т.~Кампанелла, Мореллі, 
Ж.~Мельє, Дж.~Уїнстлі, Г.~Б.~Маблі, Г.~Бабьйоф, Т.~Дезамі) за всієї відмінності 
поміж собою в конкретних позиціях і національному представництві 
були суголосними в критиці реального стану супільств як 
невідповідного поняттям про гідне життя, справедливість, моральність 
і доцільність, — а тому й неприйнятного для людського розуму й 
совісті. Їхні проекти ідеального суспільства базувалися на 
ідеалістичних уявленнях про рівність, свободу і братерство, про 
нібито добрі від природи моральні засади людини. На зміну приватній 
власності мало прийти велике колективне виробництво із справедливим 
розподілом і забезпеченням потреб кожного; в такому суспільстві буде 
подолано різницю між розумовою і фізичною працею, суперечність між 
містом і селом. (Мрія ця супроводжувала і ще супроводжуватиме чи не всю 
історію світу, вона позачасова!) Щоб прийти до такого суспільства 
справедливості, до здійснення цієї споконвічної мрії людства, треба 
було всього лиш переконати людність у його перевагах. Одначе ця проста 
і зрозуміла справа чомусь не вдавалася, як не вдавалися і спроби 
жертовних мрійників подати власний приклад організацією 
соціалістичних комун або фаланстерів. Побудувати комунізм чи 
соціалізм в окремо взятій громаді виявилося неможливим.


За цих умов і постає необхідність в іншому, неутопічному, 
реалістичному підході, обгрунтованому не моральною риторикою, а 
економічно, ідеологічно, політично, з орієнтацією на докорінну, 
найпевніше силову, перебудову всього суспільства. А на яку соціальну 
силу можна покладатися?


1842 року з'являється праця німецького вченого-юриста Лоренца фон 
Штайна (1815--1890) «Соціалізм і комунізм у сьогоднішній Франції». Це було 
за шість років до європейських революцій 1848-го і до появи 
«Комуністичного маніфесту» Маркса й Енгельса. Лоренц фон Штайн був 
одним із перших, хто розробляв теорію пролетаріату (невдовзі, 1845-го, 
з'являється праця Маркса і Енгельса «Свята родина», присвячена цій 
темі, але вже із розробленням стратегії дій революційного 
пролетаріату). Штайн показав, що клас пролетарів неминуче з'являється 
внаслідок появи і діяльності класу капіталістів. За вільноринкової 
економіки свободу і права мають власники, а не робітники. Учений-юрист 
ліберальних переконань, з лівих молодогегельянців, він гадав, що певні 
правові норми могли б допомогти пролетарям урівнятися з 
капіталістами і, таким чином, соціальної справедливості можна було б 
досягти без революції, шляхом реформ.

\subsection*{Молодий Маркс}

По-іншому розумів справу Карл Маркс. Він також вийшов із 
молодогегельянства, але швидко переріс його (праця Маркса і Енгельса 
«Німецька ідеологія», 1845--1846, містила розгорнуту критику ідеалізму 
Гегеля й непослідовності матеріалізму Феєрбаха). В його особі 
визначилися і рідкісно поєдналися філософ, ідеолог, соціолог, 
політичний діяч, журналіст-пропагандист, а згодом і економіст. Він уже 
був відомий як автор численних журналістських публікацій та наукових 
праць, присвячених обговоренню політичних і філософських проблем, 
полеміці з іншими теоретиками й ідеологами, аналізові тогочасного 
буржуазного суспільства. Досвід практичної роботи, широкий світогляд, 
філософська системність критичного мислення і потужний інтелект дали 
йому можливість узагальнити й переосмислити здобутки німецької 
філософії, французьких і англійських соціалістичних та комуністичних 
теорій, англійської політекономії (як відомо, Енгельс називав три 
джерела марксизму: німецька філософія, англійська політекономія, 
праці французьких істориків) — і прийти до принципово нових 
висновків. Вони чітко викладені в «Комуністичному маніфесті» 
авторства Маркса й Енгельса, по суті полемічному щодо утопічних або 
ретроградних ідей попередників. Не реформи, не регулювання ринку, не 
обмеження приватної власності на засоби виробництва, а повна їх 
націоналізація й одержавлення способів розподілу, що — уявлялося — 
зробить неможливою експлуатацію людини людиною, приведе до 
ліквідації класів та створення безкласового суспільства, в якому 
вільний розвиток кожного буде умовою вільного розвитку всіх. Для 
цього пролетаріат має взяти владу в свої руки революційним шляхом. Хоч 
є у Маркса й неоднозначні думки на цю тему. Революція відбудеться тоді, 
коли пролетаріат стане більшістю в суспільстві, але тоді він може 
прийти до влади й мирним шляхом. Зокрема, припускалося, що в країнах, де 
вже утвердився парламентський лад (Англія, США), пролетаріат може 
прийти до влади, перемігши на виборах. Саме на це згодом орієнтувалися 
соціал-демократичні партії II Інтернаціоналу, але які цього так і не 
дочекалися.


Картина майбутнього соціалістичного суспільства та шляхи його 
творення в «Комуністичному маніфесті» не обговорені скільки-небудь 
конкретно. Це була не так наукова праця, як 
політично-пропагандистський документ узагальнювального характеру. 
До речі, не слід забувати, що «Комуністичний маніфест» Маркс і Енгельс 
написали не з власного задуму, а на прохання міжнародної робітничої 
організації «Союз Комуністів». І точна його назва —  «Маніфест 
Комуністичної Партії». Тобто: вже існував досить організований 
робітничий рух, який потребував ідеологічного осмислення, і цю 
потребу мали задовольнити Маркс і Енгельс, вибір на яких упав, 
звичайно ж, не випадково. Але факт, що від самого початку не вони 
інспірували організований робітничий рух (у чому їх подеколи 
«звинувачували»), а робітничий рух їх «інспірував». Інша річ, що вони 
своїми ідеями надали нової якості й потужної енергії цьому рухові. 


«Маніфест Комуністичної Партії» завершував першу фазу діяльності 
«молодого» Маркса, в якій означилися основні його ідеї, що дістануть 
дальший розвиток, але вже й тоді своєю сукупністю були новим (хоч і не 
беззаперечним, і не беззаперечно новим у всьому) словом у науці й стали 
відомі під назвою історичний матеріалізм. Це, зокрема, погляд на 
історію людства крізь призму класової боротьби, соціальних 
антагонізмів, які і є рушієм розвитку (тут Маркс поглибив поняття 
класової боротьби, введене в обіг французькими істориками). Це 
твердження про неминучість революційних змін у суспільствах 
унаслідок суперечності між зростанням засобів виробництва й 
інерційністю суспільних відносин, боротьби між класом експлуататорів 
і класом експлуатованих. Це погляд на суму економічних відносин у 
суспільстві як на той базис, на якому виростає складна світоглядна, 
юридична, політична, ідеологічна, художня та ін. надбудова, що 
змінюється із зміною базису (теза, яка зазнавала і зазнає спростувань, 
почасти і через її профанацію вульгаризаторами марксизму: сам Маркс 
мав на увазі не пряму підпорядкованість надбудви базисові, а складну й 
багатоетапну опосередкованість зв'язку між базисом і надбудовою, хоча 
точних меж між одним і другим він не визначив, як і не наголосив 
зворотного впливу надбудови на базис). Далі, це важлива думка про те, що 
старий лад не відходить доти, доки не вичерпає своїх можливостей, а 
новий не приходить йому на зміну, доки не визріли передумови для нього. 
Навколо цих та інших Марксових ідей десятиліттями точилися суперечки 
між марксистами й антимарксистами, між ортодоксами й ревізіоністами, 
догматиками й реформаторами тощо.


Критики Маркса здебільше не охоплювали сукупності його поглядів та 
їхньої діалектики, їхньої часом вільної гри в концерті Марксових ідей. 
Так, один із непримиренних його негаторів, видатний мислитель ХХ ст. 
Арнольд Дж.~Тойнбі у «Дослідженні історії» писав: «Німецький єврей 
Карл Маркс намалював у барвах, які запозичив з 
апокаліптичних видінь відкинутої ним традиційної релігії, 
страхітливу картину відокремлення пролетаріату й класової війни, яку 
він розв'яже. Величезне враження, яке справив цей марксистський 
матеріалістичний апокаліпсис на стільки мільйонів умів, почасти 
пояснюється політичною войовничістю Марксової схеми, бо хоч вона й 
становить ядро загальної філософії історії, вона також являє собою 
революційний заклик до збройної боротьби» (Арнольд Дж.~Тойнбі. 
Дослідження історії. Т.~1. -- К., 1995. -- С.~362). 


Мусимо визнати, що ущиплива іронія Тойнбі стосується Марксової 
риторики чи метафорики, але не зачіпає суті, змісту його послання. Так 
само небагато дає і ревний пошук юдаїстських коренів у марксизмі. 
«Маркс поставив богиню «Історична Необхідність» на місце Єгови, а на 
місце євреїв, богообраного народу, — внутрішній пролетаріат 
західного світу. Його Мессіанське Царство — це диктатура 
Пролетаріату, але грандіозна будівля Єврейського Апокаліпсису легко 
вгадується під цим благеньким укриттям» (там само, с.~391). Безперечно! 
Але розпізнавання цієї метафорики не є спростуванням марксизму, бо ця 
метафорика давно вже стала складником європейського мислення, — хіба 
що за всіма ідеалами комунізму доведеться бачити проповіді Христа і 
зводити справу до цього. Не випадково ж існує християнський соціалізм, 
був християнський комунізм, який заперечувано ще в «Комуністичному 
маніфесті». 


Власне, Тойнбі іронізує фактично з «молодого» Маркса, часів до 
написання «Капіталу», і, як історик, бере до уваги насамперед його 
узагальнені історіософські моделі, що не вкладалися в циклопічну 
будову тойнбівского «Дослідження історії», яке охоплювало не одне 
тисячоліття і в масштабі якого марксизм міг здаватися епізодом.


\subsection*{Марксів «Капітал»}


\ldots{}«Молодий» Маркс був філософом, ідеологом, політичним публіцистом, 
але ще не економістом. «Зрілий» Маркс, критично опанувавши досягнення 
сучасної йому економічної науки, насамперед англійської, розпочинає 
фундаментальне дослідження капіталізму як формації, 
капіталістичного способу виробництва, — типологічно, за Марксовим 
визначенням, відмінного від азійського, античного й феодального 
розвитком продуктивних сил та способом експлуатації людини людиною 
(це дуже важлива частина Марксового вчення), — його очевидних та 
прихованих механізмів, його «таємниць» і перспектив та меж. Так 
з'являється перший том його «Капіталу» — праці, що справила 
величезний вплив на розвиток людської думки і на політичну історію 
людства. (Свій задум Маркс не встиг довести до кінця, і другий та третій 
томи «Капіталу» готував Енгельс з Марксових чернеток.)  


Маркс показав, що капіталізм — принципово новий історичний і 
економічний феномен: у тому сенсі, що для нього характерний не обмін 
товарів за допомогою грошей, як це було на докапіталістичних етапах 
історії людства, а обмін грошей за допомогою товарів. Через це метою 
капіталіста є грошовий прибуток, заради якого він готовий на все. А що 
є джерелом прибутку? Як створюється додаткова вартість? Це, сказати б, 
головна «таємниця» капіталізму, без розкриття якої не можна мати 
адекватного бачення його і не можна опрозорити його міфологію. Маркс 
зосереджується на цій «таємниці» і створює теорію вартості, теорію 
заробітної платні і теорію додаткової вартості — найбільшої 
«таємниці» капіталізму, що відтак перестає бути таємницею. 
Скрупульозний Марксів аналіз показує, що робочий день трудівника 
складається з двох частин — праці, яка повернеться в його зарплатню, і 
додану працю. Тобто, додаткова вартість — це неоплачена частина праці 
робітника. Праця робітника — товар, але дивовижний товар, єдиний 
товар, який виробляє вартість, вищу за власну вартість! Звідси — 
прибутки капіталіста, які тим більші, чим вище співвідношення між 
доданою вартістю і заробітною платнею. Це співвідношення є 
\emph{нормою}\emph{ }\emph{експлуатації}.


Можна, мабуть. сказати, що до Маркса категорія \emph{праці} виступала у 
суспільній свідомості (принаймні у вульгарно-матеріалістичному 
мисленні або в моралістичному) узагальнено, нерозчленованою: як 
джерело усякого багатства. На таке уявлення впливала не в останню 
чергу й протестантська трудова мораль. Певні ілюзії існували і в 
німецькому робітничому русі. Так, філософ-робітник Іосиф Дицген 
вважав, що праця — це Рятівник, удосконалення праці зробить те, чого не 
зміг досягти жоден Визволитель. Натомість Маркс не тільки показав, 
кому реально дістаються плоди праці, а й проаналізував економічні 
«складові» праці, її місце в процесі експлуатації робітника. 


Марксова демістифікація капіталізму, розкриття його механізму 
експлуатації мали не тільки наукове й політичне значення, але не в 
останню чергу й етичне, гуманістичне. Вони дали потужний поштовх 
робітничому революційному рухові спочатку в Європі, а потім і в усьому 
світі. Вони змінили світ. Зрештою змінили і самий капіталізм. І коли 
кажуть, що капіталізм давно вже не той, про який писав Маркс, то треба 
додати, шо став він «не тим» (хоч і не зовсім «не тим») завдяки зокрема й 
Марксу: капіталізмові нічого не залишалося, як змінитися під потужним 
тиском робітничого революційного руху, профспілкового руху, впливу на 
суспільства комуністичних і соціал-демократичних партій, — зрештою, 
і, мабуть, не в останню чергу, внаслідок власних внутрішніх 
суперечностей як джерела руху і завдяки невикористаним резервам, про 
можливість яких говорив Маркс (пригадаймо його тезу про те, що старий 
лад ніколи не сходить зі сцени, поки не вичерпає всіх своїх 
можливостей, певна річ, і здатності до змін).


Тут не буду говорити про те, як інтерпретували Маркса його 
послідовники (сам він якось саркастично сказав, що не хотів би бути 
марксистом), як розвивав марксизм В.~І.~Ленін і як на місці марксизму 
утворилося нове вчення — \emph{марксизм-ленінізм}. Це окрема велика тема. 
Але нагадаю про те, що в перше десятиліття радянської влади над 
вивченням Маркса й Енгельса трудилися спеціально створені солідні 
наукові інституції, які публікували свої праці, відбувалися дискусії 
тощо. В московському Інституті Маркса-Енгельса під керівництвом 
філософів-марксистів Д.~Рязанова та І.~Рубіна досліджували 
першоджерела, публікували невідомі твори. Тобто, в автентичному 
марксизмі бачили джерело ідей, що могли допомогти зрозуміти реальні 
суспільно-політичні процеси, орієнтуватися в будівництві нового 
суспільства. Ще жили такі ілюзії. Історична школа М.~Покровського з 
марксистських позицій гостро викривала російський імперіалізм. В 30-і 
роки, коли Сталін утвердив свій спрощений (ще набагато спрощеніший, 
ніж ленінський) варіант марксизму, всі ці структури ліквідовано, 
провідні вчені, дослідники й популяризатори Маркса були репресовані 
то як меншовики, то як троцькісти, а єдиним законним речником 
марксизму зробився сам Сталін.


Не менш цікаве й те, що коїлося з Марксом-Енгельсом і з марксизмом 
після розвалу СРСР у нашій самостійній Україні. Їхні твори опинилися у 
спецфондах. Посилатися на них — моветон, ознака «совковості», 
відсталості мислення й антипатріотизму. Та про це далі. А спочатку про 
те, яке місце посідав Маркс у політичній свідомості видатних 
українців минулого, чи мав він якусь «причетність» до визвольної 
боротьби українців?


\subsection*{Від Франка до української діаспори}

Дивно було б припускати, що Маркс, який став «душею» всіх 
комуністичних і соціалістичних рухів, залишиться «чужим» для України, 
яка шукала вирішення своїх національних проблем, що були водночас і 
соціальними. Марксом поважно цікавилися М.~Драгоманов, М.~Павлик, І.~Франко,
Леся Українка. Іван Франко 1879 року зробив перший український 
переклад частини «Капіталу» (фрагмент друкується у цьому виданні). 
Професор Київського університету Микола Зібер, видатний економіст і 
соціолог, перший в Україні й Росії популяризував ідеї Маркса. Він 
зустрічався з Марксом і Енгельсом у Лондоні. 1885 року опублікував працю 
«Д.~Рикардо и К.~Маркс в их общественно-экономических исследованиях», 
яку Маркс читав і прихильно цитував. Учень М.~Зібера Сергій 
Подолинський також зустрічався з Марксом і Енгельсом та листувався з 
ними; він був автором перших марксистських праць — популярних брошур 
— українською мовою, в яких застосовував Марксові ідеї до аналізу 
проблем українського селянства: «Про хліборобство» (1874), «Парова 
машина» (1875), «Про багатство та бідність» (1876), «Життя і здоров'я людей 
на Україні» (1879), «Ремесла і фабрики на Україні» (1880) та ін. Вони 
друкувалися, зрозуміло, в Галичині, але розповсюджувалися по всій 
Україні зусиллями Драгоманова, Павлика і київської «Громади». У 
Львові й Чернівцях 1892 року друкуються брошурами українські переклади 
з Маркса й Енгельса, а перший український переклад «Комуністичного 
маніфесту» виходить 1902 року у Львові. Цікавий етап у розповсюдженні 
марксистських ідей у Російській імперії — це розквіт т.~зв. 
«легального марксизму», найяскравіше представленого у Києві: В.~Кістяковський,
 С.~улгаков, М.~Ратнер, М.~Туган-Барановський (пізніше 
виступав з критикою Маркса). З «легальним марксизмом» уперто боровся 
Ленін, який бачив у ньому джерело ревізіонізму.


Якщо на перших порах популяризацією марксизму захоплювалися ліберали 
й народники, то з розвитком в Україні соціал-демократичного руху він 
стає елементом партійних програм. До марксизму апелювала створена 1905 
року на основі РУПу (Революційної Української Партії) — УСДРП 
(Українська Соціал-Демократична Робітнича Партія), визначними діячами 
якої були В.~Винниченко, С.~Петлюра, Д.~Антонович, Л.~Юркевич,
М.~Ковальський, М.~Тимченко та~ін. При цьому україноцентричні 
соціал-демократи звертаються до марксизму для висвітлення 
колоніального становища України та обстоювання ідеї національного й 
соціального визволення України. Видатним науковцем і політичним 
діячем цього гатунку був Микола Порш, один із чільних діячів РУПу та 
УСДРП, міністр в урядах УНР, соціолог і статистик, автор праць «Із 
статистики України» (1907), «Пролетаріат на Україні» (1907), «Про автономію 
України» (1907), «Автономія України і соціал-демократія» (1917), «Україна і 
Росія на робітничому ринку» (1918), «Україна в державному бюджеті Росії» 
(1918) та ін. Він же переклав українською мовою перший том «Капіталу» 
Маркса (не був виданий). Напередодні першої світової війни в Україні 
зростає мережа соціал-демократичної преси: «Дзвін» у Києві, «Воля», 
«Вперед», «Робітник», «Наш голос» — у Львові. Одним із організаторів і 
активних публіцистів у них був Володимир Левицький, автор книжок 
«Нарис розвитку українського робітничого руху в Галичині» (1914), 
«Царская Россия и украинский вопрос» (1919), «Соціалістичний 
інтернаціонал і поневолені народи» та ін. Українські марксисти 
зберігали європейське обличчя марксизму і відмежовувалися від 
марксизму ленінського. Про такий «лібералізований» марксизм можна 
говорити і стосовно Володимира Винниченка та інших лідерів УСДРП. 


Під час Світової війни українська соціал-демократична преса Галичини 
(в підросійській Україні вона була заборонена) не просто осуджувала 
варварське кровопролиття, а викривала загарбницький, 
імперіалістичний характер війни. Глибокий аналіз її причин з 
марксистських позицій дали В.~Левицький та М.~Залізняк. 


Тут слід нагадати, що в цей самий час значна частина європейських 
соціал-демократів, як відомо, розбіглася по «національних квартирах» 
і так чи інакше ставала по боці «своїх» урядів. 


Антивоєнна й правдиво інтернаціоналістична позиція українських 
соціал-демократів парадоксальним чином обернулася проти них у 
визвольну добу 1918--1919 рр., коли вони відігравали провідну роль у 
Центральній Раді та в урядах УНР.~Як соціалісти і марксистські 
налаштовані політики, вони сподівалися на розуміння і мирну 
домовленість із соціалістами й марксистами «великого братнього» 
народу. Але виявилося, що то зовсім інакший «соціалізм» і зовсім 
інакший «марксизм». Ставлення до загрози російсько-більшовицької 
агресії в різних колах УСДРП було різне, і це спричинило внутрішню 
боротьбу й розколи, що також додалося до причин поразки. 


Ще у складнішому становищі опинилися українські соціал-демократи 
після приходу більшовиків в Україну. Вони не могли ігнорувати того 
факту, що більшовицькі гасла неабияк впливали на українські маси, а 
Радянська Росія немовби очолила світовий революційний і 
соціалістичний рух, за яким бачилося майбутнє. Невблаганний 
історичний процес диктував необхідність стратегії і тактики, 
відповідної до нових і непередбачуваних умов, здатної забезпечити 
можливість впливати на події і не бути відкинутими на задвірки 
історії. Тут неминучими стали нові незгоди й розколи. Частина 
вчорашніх соціал-революціонерів та соціал-демократів обирає 
співпрацю, на певних умовах, з більшовиками, сподіваючись таким чином 
впливати на характер перетворень і обстоювати українські національні 
інтереси, як вони їх уявляли. На перших порах ці сподівання почасти 
справджувалися, оскільки більшовики потребували підтримки досить 
сильної партії боротьбистів і йшли на деякі поступки. Але в міру 
зміцнення своєї влади вони дедалі більше утискували своїх 
ситуативних союзників. Тим часом і серед більшовиків, у тодішній 
КП(б)У, були, хоч і не переважальні, «націонал-ухильницькі» (на 
офіційному парткерівному жаргоні) сили, пов'язані зі своїм народом і 
відповідальні за його долю принаймні в тому розумінні, що хотіли 
бачити його рівноправним з іншими в уявлюваному комуністичному 
суспільстві, яке мало привести до вільного розвитку всіх народів. 
Зрештою, сума вагомих чинників — спротив українського села, опозиція 
національної інтелігенції, наявність різних ідеологічних елементів 
та різного бачення історичної перспективи у самій партії, слабкість 
її позицій в українському та інших національних суспільствах, — а не в 
останню чергу й претензія на роль маяка антиколоніальних революцій на 
Сході, для яких комуністична Україна мала стати переконливим і 
звабливим прикладом, — змусила більшовицьку партію, на виконання 
нового курсу Леніна, вдатися до політики «українізації», ширше —
«коренізації», оскільки йшлося й про інші колонізовані народи. Тобто, 
це був пошук надійного опертя в неросійських народах. В Україні ця 
політика пов'язувалася з лідерами націонал-комунізму, давніми 
партійними діячами Олександром Шумським та Миколою Скрипником, які 
прагнули дати їй марксистське обгрунтування. Відповідні дослідження 
проваджувано в Українському Інституті Марксизму та Ленінізму (УІМЛ, 
1922--1931). Професійна марксистська методологія з різною мірою успіху 
впроваджувалася в різних галузях суспільних наук. Серед яскравих 
представників цього штибу мислення можна назвати А.~Річицького, 
одного із засновників УКП (Української Комуністичної Партії), 
сподвижника М.~Скрипника, наукового працівника УІМЛ, автора праць на 
літературні й Марксівські теми, редактора першого видання «Капітала» 
Маркса українською мовою (1927--1929); філософа-марксиста, поета, 
публіциста і літературознавця В.~Юринця; історика М.~Яворського, школа 
якого працювала до погрому 30-х років. 


Доля УІМЛ була такою ж, як і московського Інституту Маркса-Енгельса, 
хіба що набагато трагічнішою, бо стала частиною тотальних репресій 
проти українських наукових і культурних установ та їхніх діячів — під 
моторошний акомпанемент голодомору. 


Зрозуміло, що після цього будь-які серйозні роботи в галузі марксизму 
стали неможливими і втратили сенс, черга реорганізацій закінчилася 
створенням Інституту історії партії при ЦК КПУ — як філіалу Інституту 
марксизму-ленінізму при ЦК КПРС. 


Марксистська фразеологія стала способом придушення самостійного 
мислення, і не дивно, що на час розвалу СРСР марксизм був у суспільстві 
остаточно скомпрометований, хоча долинали ще якісь відгомони 
європейського неомарксизму й були спроби створювати нелегальні 
робітничі гуртки з орієнтацією на «справжній марксизм» (про це ми 
могли довідатися з великим запізненням, у 90-і роки, — після того, як СБУ 
опублікувала секретні матеріали провокаційної кагебістської «Справи 
,,Блок``», по якій «проходили» не тільки «українські буржуазні 
націоналісти», а й широкий спектр інших «підривних елементів»). 


Прикметно, що з настанням горбачовської «гласності» та після здобуття 
Україною незалежності Маркса у нас зовсім не стало. Його уникали, мов 
якогось «совєтського» маркера, і офіціоз, і рухівська опозиція. Щодо 
офіціозу зрозуміло: йому з Марксом не було про що говорити, та й 
некомфортно. А \mbox{РУХ}ом він залишився непрочитаний. На мій погляд, велика 
помилка \mbox{РУХ}у й одна з причин його досить швидкого занепаду — 
абсолютизація національного питання й невміння розкрити всю 
конкретність його пов'язаності з соціальним. Це саме те, чого можна 
було повчитися у Маркса. Але Маркс вважався завербованим у офіційну 
совєтчину (пригадую: коли я в своєму самвидавському опусі 
«Інтернаціоналізм чи русифікація» рясно посилався на погляди Маркса 
й Енгельса, як і Леніна, з національного питання, на його листування, в 
якому фактично заперечується його власна, з «Комуністичного 
маніфесту», теза про те, що пролетаріат не має вітчизни, і говориться 
протилежне: щоб успішно вести свою боротьбу, пролетаріат повинен 
насамперед визволити чи об'єднати свою вітчизну, — багато хто навіть 
із прихильних до мене були подивовані, часом і неприємно, або 
сприймали це як курйоз чи риторичний прийом). Можна зрозуміти: Маркс 
такий далекий, а українське національне питання таке пекуче, що 
багатьом воно здавалося самодостатнім. Серед рухівців панувало 
стихійне переконання: національне — головне, соціальне — 
підпорядковане. Вся історія людства, м'яко кажучи, не підтверджувала 
цього, але гіркі розчарування варті того, щоб їх пережити самому. 
Вкотре наочно виявилося, що в дилемі національне-соціальне (до якої, 
власне, і не повинен допускати розумний політик!) національне стає 
пріоритетом для героїчної меншості, а соціальне — для решти. Героїчна 
меншість може творити революції, але парламенти обирає негеролїчна 
більшість. Тож український виборець у масі своїй голосував не за 
безкорисливих патріотів української мови (або й історії), а за хижих 
демагогів, які обіцяли швидке і фантастичне полагодження житейських 
проблем. І в захисники трудящих на політичній арені перевдягалися 
їхні найжорстокіші експлуататори, захребетники — як оті донецькі 
вугільно-металургійні барони, які невеличку частину прибутків, 
здертих з каторжної праці робітників, витрачали на фінансування 
організованих ними експедицій цих робітників під Верховну Раду чи 
Кабмін, для стукання шахтарськими касками, — так, ніби це українські 
урядовці, а не вони, донецькі барони, винні в несплаті заробітку і в 
жахливому занедбанні техніки безпеки та постійних катастрофах. 


Ні, не став РУХ реальним захисником трудового люду. Як не стали ним і 
профспілки, спосіб організації яких, структура і зміст роботи, права і 
можливості залишаються далекими від прозорості. Може, я помиляюся, але 
мені здається, що ні в кого немає ніякої концепції — ані марксистської, 
ані неомарксистської, ані просто немарксистської, ані навіть 
антимарксистської — захисту трудящих за умов нашого дикого 
капіталізму. Ані концепції, ані продуманих ідей, ані якоїсь — 
політичної чи моральної — гуманістичної настанови, Про це останнє 
кажу тому, що автентичний марксизм — насамперед гуманістичне вчення! 
Воно має глибоке коріння в історії людських борінь за справедливість, 
виразно перегукується з етикою шукань істини, пропонує свого роду 
соціологію пізнання. Як філософ і письменник (у широкому значенні 
слова), Маркс не чужий феноменології, в нього знаходять елементи 
екзистенціалізму. Може, найважливіше чи, принаймні, найцікавіше для 
гуманітаристики в «Капіталі» —це дослідження товарного фетишизму й 
відчуження праці, які фундаментальним чином діють у напрямку 
збіднення світу людини, її знелюднення. Це чинники універсальні, яким 
людство ще не знайшло противаги і не знати, чи колись знайде. Тут, може, 
найважливіші з Марсових відкриттів, і вони варті не меншої уваги, ніж 
його суто економічні осягнення. 


Ще\ldots{} Про Маркса часто говорили й писали, говорять і пишуть, що він 
нібито зневажав духовну творчість або ставився до неї догматично. Як 
на мене, це прикре непорозуміння. Маркс добре знав історію культури, 
його твори не бідні на апеляцію до її фактів та на глибокі думки про 
літературу, великих письменників минулого і сучасників. А прочитайте 
його листування, прилучіться до обсягу його естетичних переживань і 
читацьких реакцій! Отут іще один незнаний нам Маркс!


\ldots{}На закінчення варто додати, що в той час, як в Україні «набридлий» 
за радянські часи Маркс для науковців перестав бути актуальним (хоча б 
для цитування, про вивчення й мови немає), а політологам і публіцистам 
було не до Маркса (та й попиту ніякого), — «націоналісти», а власне 
інтелектуали в українській діаспорі про нього не забували. Крім тих, 
хто поважно студіював Маркса (Р.~Роздольський, відомий як один із 
кращих знавців економічних і національних поглядів Маркса, в 
молодості один із організаторів КПЗУ, якийсь час співробітник, під 
керівництвом Д.~Рязанова й І.~Рубіна, московського Інституту 
Маркса-Енгельса, після його ліквідації працював у архівах Відня, 
Львова, Кракова, у 1942--1945 — в'язень німецьких концтаборів, від 1945 — у 
Детройті, автор багатьох досліджень про Маркса, зокрема й 
неомарксистського тлумачення «Капіталу» — от така дивовижна людина, 
варта біографічного роману або кіно- чи телесеріалу; Панас Феденко, 
один із організаторів УСДРП та лідер її в еміграції; історик і 
публіцист В.~Голубничий; були ті, хто розумів його роль у розвитку 
суспільних наук та в політичній історії людства і знаходив йому 
належне місце в системі своїх ідеологічних оцінок постатей і 
феноменів сучасності, як-от Іван Лисяк-Рудницький. Можна говорити про 
певну пов'язаність з марксизмом Івана Багряного та інших ідеологів 
УРДП — Української Революційно-Демократичної Партії, яка за складних 
умов політичного розбрату в еміграції мужньо обстоювала ідею єдності 
українців на основі не «філологічного паріотизму», а спільності 
корінних інтересів соціальної справедливості й прагнення до свободи. 
Ідеї Івана Багряного могли б уберегти український політикум, 
насамперед рухівців та пізніших «правих», від прикрих помилок та 
неуваги до соціальної сторони української проблематики, — але, на 
жаль, вони не знайшли належного відгуку та й просто місця у вузькому 
кругозорі наших «патріотів» (не кажучи вже про байдужих до України). 


Окремо слід сказати про солідну працю видатного українського 
історика в США (власне, й американського історика) Романа Шпорлюка 
«Націоналізм і комунізм» (Оксфорд, 1988; український переклад Георгія~Касьянова — К., «Основи», 1998). Праця має підзаголовок: «Карл Маркс проти 
Фрідріха Ліста», але фактично Марксова полеміка з німецьким 
економістом і теоретиком націоналізму Лістом — це лише сюжетний 
стрижень праці, який обростає великим фактичним матеріалом та 
інтелектуальними розважаннями автора й лектурою на тему взаємодії 
марксизму й націоналізму як двох великих проектів модернізації 
суспільств — проектів принципово суперечних один одному, але й 
суголосних багато в чому та навіть «повчальних» один до одного — аж 
такою мірою, що в процесі суспільного розвитку ХIХ--ХХ ст. марксизм 
помітно «націоналізувався», а націоналізм — почасти «омарксизмився».


Праця Романа Шпорлюка, на мій погляд, особливо важлива для українців 
тим, що виводить уявлення про націоналізм з провінційних вимірів у 
глобальні, знайомить нашого читача з інтерпретацією націоналізму 
широким колом сучасних європейських істориків, соціологів, філософів; 
те ж саме стосується і марксизму, якого українське суспільство — 
виглядає — так і не освоїло, хоч у ньому є ще немало інтелектуальних 
резервів для нас. Вони ждуть свого «будителя». І, може, якимось 
імпульсом стане сподівана публікація українського перекладу 
«Капіталу». 

\begin{flushright}
  \emph{Іван Дзюба}
\end{flushright}

{\small 22 липня 2018 р.}

\parcont{}  %% абзац починається на попередній сторінці
\index{iii1}{0061}  %% посилання на сторінку оригінального видання
частини упредметненої в ньому праці, яку він оплатив. Додаткова
праця, що міститься в товарі, нічого не коштує капіталістові,
хоч робітникові вона цілком так само коштує праці, як
і оплачена, і хоч вона цілком так само, як і оплачена, створює
вартість і входить у товар як вартостетворчий елемент. Зиск
капіталіста постає з того, що він має для продажу щось, чого
він не оплатив. Додаткова вартість, відповідно зиск, складається
саме з надлишку товарної вартості понад витрати її виробництва,
тобто з надлишку всієї суми праці, вміщеної в товарі, понад
вміщену в ньому оплачену суму праці. Таким чином додаткова
вартість, звідки б вона не виникала, є надлишок понад увесь
авансований капітал. Отже, цей надлишок стоїть у такому відношенні
до всього капіталу, яке виражається дробом \frac{m}{K}, де
$К$ означає весь капітал. Таким чином одержуємо \emph{норму зиску}
\frac{m}{K} \deq{} \frac{m}{c+v}, у відміну від норми додаткової вартості \frac{m}{v}.

Величина додаткової вартості у її відношенні до змінного
капіталу зветься нормою додаткової вартості; величина додаткової
вартості у її відношенні до всього капіталу зветься нормою зиску.
Це два різні виміри тієї самої величини, які в наслідок ріжниці в
масштабах виражають одночасно різні пропорції або відношення
однієї і тої самої величини.

З перетворення норми додаткової вартості в норму зиску
слід виводити перетворення додаткової вартості в зиск, а не
навпаки. І справді, вихідним пунктом історично була норма зиску.
Додаткова вартість і норма додаткової вартості є, відносно, те
невидиме і суттєве, що треба розкрити, тимчасом як норма
зиску, а тому й така форма додаткової вартості як зиск виявляються
на поверхні явищ.

Щодо окремого капіталіста, то ясно, що єдине, що його
інтересує, це відношення додаткової вартості або надлишку вартості,
ради якого він продає свої товари, до всього капіталу,
авансованого на виробництво товару; тимчасом як певне відношення
цього надлишку до окремих складових частин капіталу
і його внутрішній зв’язок з цими складовими частинами не тільки
не інтересує його, але він ще й заінтересований в тому, щоб
оповити туманом це певне відношення і цей внутрішній зв’язок.

Хоча надлишок вартості товару понад витрати його виробництва
виникає в безпосередньому процесі виробництва, але реалізується
він тільки в процесі циркуляції, — і він тим легше набуває видимості
виникнення з процесу циркуляції, що в дійсності, серед
конкуренції, на дійсному ринку, від ринкових відносин залежить,
чи реалізується цей надлишок, чи ні, і в якому розмірі. Тут
немає потреби пояснювати, що коли товар продається вище
або нижче його вартості, то має місце тільки інший розподіл
додаткової вартості, і що цей інший розподіл, змінене
відношення, в якому різні особи ділять між собою додаткову вартість,
\index{iii1}{0062}  %% посилання на сторінку оригінального видання
нічого не змінює ні в величині, ні в природі додаткової
вартості. В дійсному процесі циркуляції не тільки відбуваються
перетворення, які ми розглянули в книзі II, але вони збігаються
з дійсною конкуренцією, з купівлею і продажем товарів вище
або нижче їх вартості, так що для окремого капіталіста реалізована
ним самим додаткова вартість залежить так само від
взаємного ошукування, як і від безпосередньої експлуатації
праці.

В процесі циркуляції поряд робочого часу починає діяти час
циркуляції, який цим самим обмежує масу додаткової вартості,
яку можна реалізувати за певний період. На безпосередній
процес виробництва впливають визначально ще й інші моменти,
які виникають з циркуляції. І те і друге, безпосередній
процес виробництва і процес циркуляції, постійно переходять
один в один, пронизують один одного, і тим самим постійно
перекручують свої характерні відмінні ознаки. Виробництво додаткової
вартості, як і вартості взагалі, набуває в процесі циркуляції,
як показано вище, нових визначень; капітал перебігає
круг своїх перетворень; нарешті, він вступає з свого, так би
мовити, внутрішнього органічного життя в зовнішні життьові
відносини, у відносини, де один одному протистоять не капітал
і праця, а з одного боку капітал і капітал, з другого боку
індивіди знов таки просто як покупці і продавці; час циркуляції
і робочий час перехрещуються на своєму шляху, і таким
чином здається, ніби вони обидва в однаковій мірі визначають
додаткову вартість; та первісна форма, в якій протистоять один
одному капітал і наймана праця, замасковується в наслідок втручання
відносин, які, як здається, незалежні від неї; сама додаткова
вартість здається не продуктом привласнення робочого часу,
а надлишком продажної ціни товарів понад витрати їх виробництва,
в наслідок чого витрати виробництва легко можуть здаватися
дійсною вартістю (valeur intrinsèque) товарів, так що зиск
здається надлишком продажної ціни товарів понад їх імманентну
вартість.

Правда, під час безпосереднього процесу виробництва природа
додаткової вартості постійно доходить до свідомості капіталіста,
як це вже при розгляді додаткової вартості показала
нам його жадоба до чужого робочого часу і~\abbr{т. д.} Але: 1)~сам
безпосередній процес виробництва є тільки минущий момент,
який постійно переходить у процес циркуляції, як і цей останній
переходить у нього, так що ясніше чи туманніше проблискуюча
в процесі виробництва догадка про джерело здобутого у ньому
баришу, тобто про природу додаткової вартості, щонайбільше
виступає як момент рівноправний з тим уявленням, ніби реалізований
надлишок походить від руху, який не залежить від
процесу виробництва і виникає з самої циркуляції, отже, руху,
який належить капіталові незалежно від його відношення до
праці. Адже навіть сучасними економістами, як от Рамсей,
\parbreak{}  %% абзац продовжується на наступній сторінці

\parcont{}  %% абзац починається на попередній сторінці
\index{iii1}{0063}  %% посилання на сторінку оригінального видання
Мальтус, Сеніор, Торренс і~\abbr{т. д.}, ці явища наводяться безпосередньо
як докази того, ніби капітал просто в своєму речовому
існуванні, незалежно від того суспільного відношення до
праці, в якому він саме й стає капіталом, є, поряд з працею
і незалежно від праці, самостійним джерелом додаткової вартості.
2)~Під рубрикою витрат, куди належить заробітна плата
цілком так само, як і ціна сировинного матеріалу, зношування
машин і~\abbr{т. д.}, видушування неоплаченої праці здається тільки
заощадженням на оплаті одного з тих предметів, які входять
у витрати, тільки меншою платою за певну кількість праці;
цілком так само, як відбувається заощадження, коли дешевше
купують сировинний матеріал або зменшують зношування машин.
Таким чином видушування додаткової праці втрачає свій
специфічний характер; його специфічне відношення до додаткової
вартості затемнюється; і цьому затемнінню дуже допомагає
і дуже його полегшує, як показано в книзі І, відділ VI,
представлення вартості робочої сили в формі заробітної плати.

Через те що всі частини капіталу однаково здаються джерелами
надлишкової вартості (зиску), то капіталістичне відношення
містифікується.

Той спосіб, яким додаткова вартість за допомогою переходу
через норму зиску перетворюється в форму зиску, є, однак,
тільки дальший розвиток того переплутання суб’єкта і об’єкта,
яке відбувається уже в процесі виробництва. Вже там ми бачили,
як усі суб’єктивні продуктивні сили праці здаються продуктивними
силами капіталу. З одного боку, вартість, минула праця,
яка панує над живою працею, персоніфікується в капіталісті;
з другого боку, навпаки, робітник виступає просто як предметна
робоча сила, як товар. З цього перекрученого відношення неминуче
виникає вже в самому простому відношенні виробництва
відповідне перекручене уявлення, перенесена з цього відношення
свідомість, яка розвивається далі в наслідок перетворень і модифікацій
власне процесу циркуляції.

Спроба представити закони норми зиску безпосередньо як закони
норми додаткової вартості, або навпаки, є цілком хибна, як
у цьому можна пересвідчитися на прикладі школи Рікардо. В голові
капіталіста, звичайно, ці закони не розрізняються. У виразі \frac{m}{K}
додаткова вартість вимірюється вартістю всього капіталу, авансованого
на її виробництво і почасти в цьому виробництві цілком спожитого,
а почасти тільки застосованого в ньому. Відношення \frac{m}{K} в
дійсності виражає ступінь зростання вартості всього авансованого
капіталу, тобто, взяте відповідно до його раціонального, внутрішнього
зв’язку і природи додаткової вартості, воно показує,
яке є відношення величини, на яку змінився змінний капітал, до
величини всього авансованого капіталу.


\index{iii1}{0064}  %% посилання на сторінку оригінального видання
Величина вартості всього капіталу сама по собі не стоїть
у будь-якому внутрішньому відношенні до величини додаткової
вартості, принаймні не стоїть безпосередньо. Щодо своїх речових
елементів весь капітал мінус змінний капітал, отже, сталий капітал,
складається з речових умов здійснення праці, з засобів праці
і матеріалу праці. Для того, щоб певна кількість праці реалізувалась
у товарах і, отже, утворила вартість, потрібна певна
кількість матеріалу праці і засобів праці. Залежно від особливого
характеру додаваної праці існує певне технічне відношення
між масою праці і масою засобів виробництва, до яких повинна
бути додана ця жива праця. Отже, остільки існує також певне
відношення між масою додаткової вартості або додаткової праці
і масою засобів виробництва. Якщо, наприклад, час, необхідний
для виробництва заробітної плати, становить 6 годин на день,
то робітник мусить працювати 12 годин, щоб дати 6 годин додаткової
праці, щоб створити додаткову вартість у 100\%. Він
споживає за ці 12 годин удвоє більше засобів виробництва, ніж
за ці 6 годин. Але від цього додаткова вартість, яку він додає
за 6 годин, зовсім не стає в будь-яке безпосереднє відношення
до вартості засобів виробництва, спожитих за ці 6 чи навіть
за ці 12 годин. Ця вартість тут не має ніякого значення; ідеться
тільки про технічно необхідну масу. Чи сировинний матеріал або
засоби праці дешеві, чи дорогі, це не має ніякого значення;
аби тільки вони мали потрібну споживну вартість і були наявні
в технічно встановленій пропорції до тієї живої праці, яку треба
поглинути. Однак, якщо я знаю, що за одну годину перепрядається
$х$ фунтів бавовни, які коштують $а$ шилінгів, то я, звичайно,
знаю і те, що за 12 годин перепрядається 12 $х$ фунтів
бавовни \deq{} 12 $а$ шилінгам, і тоді я можу обчислити відношення
додаткової вартості до вартості цих 12, так само як і до вартості
цих 6. Але відношення живої праці до \emph{вартості} засобів
виробництва тут привходить лиш остільки, оскільки $а$ шилінгів
служать назвою для $х$ фунтів бавовни; бо певна кількість бавовни
має певну ціну, а тому й навпаки, певна ціна може служити
показником певної кількості бавовни, поки ціна бавовни
не зміниться. Якщо я знаю, що для того, щоб привласнити 6 годин
додаткової праці, я повинен примушувати працювати 12 годин,
отже, мушу мати напоготові бавовни на 12 годин, і якщо я знаю
ціну цієї потрібної для 12 годин кількості бавовни, то посередньо
існує відношення між ціною бавовни (як показником необхідної
кількості) і додатковою вартістю. Навпаки, з ціни сировинного
матеріалу я ніколи не можу зробити висновок про масу сировинного
матеріалу, яка може бути перепрядена, наприклад, за
одну годину, а не за 6. Отже, немає ніякого внутрішнього, необхідного
відношення між вартістю сталого капіталу, — а тому
і між вартістю всього капіталу ($= c \dplus{} v$) і додатковою вартістю.

Якщо норма додаткової вартості відома і величина додаткової
вартості дана, то норма зиску виражає не що інше, як
\parbreak{}  %% абзац продовжується на наступній сторінці

\index{franko}{0065}

\looseness=1
Перший крок перевороту, що поклав основу капталістичній продукції, припадає в послідній третині \RNum{15} і
в першій чверти \RNum{16} віку. Тоді скасовано феодальне дворацтво, котре, як справедливо замічає Джемс
Стюерт, „злягло  всі хати і двори безхосенно“. Через те викинено масу голих пролєтаріїв на
робучий торг. Хоть королівська власть, що й сама виросла з буржуазного розвитку, намагаючи до
неограниченого панованя, силою скасувала те великопанське дворацтво, то прецінь вона не була єдиною
причиною нового перевороту. Ні, в упертім опорі протів королівства та
парляменту витворили великі пани-феодали далеко більшу масу пролєтаріяту, прогонюючи силою
хліборобів з ґрунту і посідлости, хоть хлібороби мали до тих ґрунтів більше право, ніж вони, і
забираючи для себе громадські ґрунти. Беспосередний товчок до того в Англії дав іменно росцвіт
фляндрійської вовняної мануфактури і звязане з ним підскоченє цін вовни. Стара феодальна шляхта
вигибла в великих феодальних війнах, а нова шляхта — се були діти свого часу, для котрих гроші були
силою понад всі сили. З вірного поля пасовиська для овець! — се став тепер їх загальний оклик.
Гаррізен в своїй „Description of England. Prefixed to Holinshed’s Chronicles“ описує, як
вивласнюванє дрібних ґаздів руйнує край. „Але що нашим великим самозванцям до того?“ Мешканя ґаздів
та коттеджі робітників валят вони силою або прогнавши людей лишают пустками. „Коли перездримо
давнійші інвентарі кождої домінії, то побачимо, що незлічимі хати та дрібні ґаздівства пощезали, що
ґрунт годує далеко меньше люда, що богато міст підупало, хоть деякі нові підносятся\dots{} Мож би
чимало наросповідатися про місточка та села, зруйновані для того, щоб було місце на толоки для
овець; тілько самотні панські двори стоят серед тих толок“. Правда, наріканя тих старих літописів
усе пересаджені, але вони досадно малюют те вражінє, яке на самих сучасників робив переворот
обставин продукційних. Порівнанє між письмами канцлєрів Фортеске і Томаса Моруса вказує наглядно
пропасть між \RNum{15} а \RNum{16} віком. „Із золотого віку — каже справедливо Зорнтон — попали англійські
робітники без ніяких перехідних ступнів прямо в зелізну“.

\looseness=1
Праводавство злякалось сего перевороту. Воно не стояло ще на такім високім ступни цівілізації, де
„богацтво народне“, т. є. богацтво капіталістів і безграничне висисанє та зубожінє маси люду
становит верх премудрости
політичної. В своїй історії Генріха VII каже Бекон: „В тім часі (1489) посипалися скарги на то, що
вірне поле перемінюєсь в пасовиська, котрих лехко може дозирати кілька пастухів. Ґрунти, що вперед
виарендовувались на кілька літ, на доживотну або щорічну умову, тепер зіллято разом
\index{franko}{0066}
с панськими. Се підкопало добробуток люду, а через те й міста, церкви, десятини\dots{} Щоб зарадити
тому лиху, проявили король і парлямент дивну на ті часи мудрість\dots{} Вони видали право протів того
обезлюднюючого край загарбуваня громадських ґрунтів (depopulating inclosures) і невідлучної
від него обезлюднюючої ґосподарки толочної (depopulating pastures)“. Оден акт Генріха VII з р.~
1489 заказує руйнувати хліборобські хати, до котрих належит що найменьше 20 екрів ґрунту. Генріх
VIII відновив той самий указ. Говорится там між їншим, що „многі аренди і огромні отари, особливо
овець, нагромаджуются в немногих руках, через що дохід
з ґрунту дуже вбільшився, а рільництво дуже підупало, церкви і хати повалено, дивовижні маси народа
стали неспосібні вдержувати себе і свої родини“. Указ наказує затим відбудовувати повалені хутори,
означує, кілько має бути вірного поля в стосунку до овечих толок і т. д. Їнший акт з р.~1533
жалуєсь, що деякі властивці мают по \num{24000} овець, і ограничує їх число на 2000\footnote{
В своїй „Утопії“ говорит Томас Морус про дивовижний край, де
„вівці їдят людей“.
}. Наріканя народа і
праводавство протів вивласнюваня дрібних арендаторів та хліборобів, що почалось від Генріха VII і
трівало зо 150 літ
— не помогли нічо. Чому не помогли, пояснює нам Бекон, сам того не знаючи. „Акт Генріха~VII, — каже
він в своїх „Essays, civil and moral“, Sect. 20, — був глибоко і дивно обдуманий. Він утворив
сільскі ґаздівства і хліборобські доми певного нормального розміру, т. є. вдержав для них таку
пропорцію ґрунту, котра давала їм змогу плодити на світ підданих доста заможних і не придавлених
нуждою, так що плуг був в руках властивців, а не наємників\footnote{
Бекон пояснює далі звязок між свобідним, заможним селянством
а доброю інфантерією. „Се була дивно важна річ для сили і мужности
королівства — мати аренди достаточного розміру, щоб дільних мужів
забеспечити від нужди і велику часть ґрунту краєвого запевнити в посіданє джоменам, т. є. людім
середної заможности між шляхтою а халупниками (cottagers) та наймитами. Бо се загальна думка
найліпших знавців воєнного діла\dots{} що головна сила армії, се інфантерія або піхота. Але щоб
витворити добру інфантерію, тре людей вихованих не в притиску ані в нужді, але свобідно і в певній
заможности. Коли затим яка держава вросте переважно в шляхту та делікатне панство, а хлібороби та
ратаї зійдут на простих зарібників та наймитів або халупників, т. є. жебраків з власною хатою, то
така держава може мати добру кінницю, але доброї піхоти не буде мати. Се видно в Італії і Франції і
деяких других заграничних краях, де справді все або шляхта або нужденні зарібники\dots{} Дійшло там до
того, що ті краї мусят уживати наємного зброду Швейцарів та др. для своєї піхоти: відти то й пішло,
що ті держави мают богато людий, а мало вояків“. („The Reign of Henry VII“ і т.~д.).
}. А між
\parbreak{}


\index{iii2}{0067}  %% посилання на сторінку оригінального видання
Про розділ банку на два відділи та про надмірне піклування в справі забезпечення
розміну банкнот Тук висловлюється так перед C.~D 1848/57:

Більші коливання рівня проценту в 1847 році проти років 1837 та 1839
завдячували лише розділові банку на два відділи. (3010). — Забезпечености банкнот
не було порушено ані в 1825, ані в 1837 та 1839 роках. (3015). — Попит
на золото в 1825 році мав на меті тільки заповнити порожняву, що утворилася
в наслідок цілковитого дискредитування однофунтівок-банкнот провінціяльних
банків; цю порожняву можна було заповнити тільки золотом, поки Англійський
банк не почав теж видавати однофунтівки-банкноти. — (3022). В листопаді та
грудні 1825 року не було ані найменшого попиту на золото для вивозу. (3023).

«Щодо дискредитування банку всередині країни та закордоном, то припинення
виплати дивідендів та вкладів мало б куди тяжчі наслідки, ніж припинення
оплати банкнот (3028)».

«3035. Чи не сказали б ви, що кожна обставина, яка, кінець-кінцем,
загрожує небезпекою розмінові банкнот, могла б в момент комерційного пригнічення
породити нові та серйозні труднощі? — Аж ніяк».

Протягом 1847 року «збільшене видання банкнот, може бути, допомогло б
знову поповнити золотий скарб банку, як це сталося в 1825 році». (3058).

Newmarch свідчить перед В А. 1857: «1357. Перший лихий вплив\dots{}
цього відокремлення обох відділів [банку] та розділу золотого запасу на дві частини,
розділу, що неминуче випливав з такого відокремлення, був той, що банкові
операції Англійського банку, отже, цілу ту ділянку його операцій, що ставить
його в безпосередній зв’язок з торговлею країни, провадилось далі лише за допомогою
половини суми попереднього запасу. В наслідок цього розділу запасу
дійшло до того, що банк мусив підвищувати норму свого дисконту, скоро запас
банкового відділу зменшувався хоч трохи. Тому цей зменшений запас зумовлював
ряд раптових змін у нормі дисконту. — 1358. Таких змін, починаючи від
1844 року [до червня 1857 року], було, може, з 60, тимчасом коли протягом
такого самого часу перед 1844 роком вони ледве чи становили дюжину». Особливий
інтерес має теж свідчення Palmer’a, що від 1811 року був директором, а
деякий час управителем англійського банку, перед C.~D. комісією лордів (1848--57).

«828. В грудні 1825 року банк ще зберіг приблизно \num{1.100.000}\pound{ ф. ст.}
золота. Він мусив би тоді, безперечно, цілком збанкрутувати, коли б тоді був
цей акт (1844 року). В грудні він видав, на мою думку, 5 або 6 мільйонів
банкнот протягом одного тижня, й це значно полегшило тодішню паніку.

«825. Перший період [від 1 липня 1825 року], коли сучасне банкове
законодавство збанкрутувало б, якщо банк спробував би довести до кінця вже
розпочаті операції, був 28 лютого 1837 року; в ті часи в розпорядженні банку
було \num{3.900.000} до 4 мільйонів ф. ст., і він зберіг би тоді лише \num{650.000}\pound{ ф. ст.}
в запасі. Другий такий період був у 1839 році й тривав від 9 липня до 5 грудня.
— 826. Яка була сума запасу в цьому випадку? — 5 вересня запас
складався з дефіциту в цілому на суму \num{200.000}\pound{ ф. ст.} (the reserve was minus
altogether \num{200.000} ф. ст). На 5 листопада запас зріс приблизно до 1--1\sfrac{1}{2}
мільйонів. — 830. Акт 1844 року заважав би банкові підтримувати торговлю
з Америкою. — 831. Три головні американські фірми збанкрутували\dots{} Майже
кожну фірму, що провадила американські операції, позбавлено кредиту, і
коли б у ті часи банк не прийшов на поміч, то я не думаю, щоб більше, як
1 або 2 фірми, могли витримати, — 836. Скруту 1837 року не можна рівняти
з скрутою 1847 року. В 1837 році вона обмежилася головне на американських
операціях». — 838. (На початку червня 1837 року дирекція банку дискутувала
питання, як зарадити тій скруті). «В цій справі декотрі з панів боронили думку\dots{}
що найправильнішим принципом було б підвищити рівень проценту, через що
товарові ціни впали б; коротко, зробити гроші дорожчими, а товари дешевшими,
\parbreak{}  %% абзац продовжується на наступній сторінці


Але сесі беспосередні наслідки реформації не були
найтривкійші. Церковна власність, се була реліґійна підпора
старосвіцьких порядків ґрунтових. Впала вона, то й їм не
довго було вже встоятись.

Ще в послідних десятилітях \RNum{17} віку було джоменів
(самостійних ґаздів хліборобів) більше ніж арендаторів.
Вони творили головну силу Кромвеля і — як свідчит сам
Маколєй — визначувались дуже корисно супротів роспитих
паничів та їх прислужників — сільских попів. Ще навіть
сільскі наємники були співвластивцями громадського ґрунту.
Аж около 1750 щезли джомени зовсім, а в послідних десятиліттях
\RNum{18} віку щезли послідні сліди громадських ґрунтів
хліборобських. Ми ту не берем на ввагу чисто економічних
двигачів рільничого перевороту, але глядимо тілько на пoсторонні,
насильні товчки.

За реставрації Стюартів перевели великі властивці
ґрунтів правним способом такий самий рабунок, який в прочій
Европі робився і без правних оборотів. Вони знесли
феодальні ґрунтові порядки, т.~є. скасували всі ті повинности,
які припадали державі з ґрунтів, „відшкодували“ державу
тим, що наложили податки на хліборобів та прочу
масу народа, а самі забрали в тісну приватну власність усі
добра, над котрими вперед мали лиш феодальну зверхність,
і накинули вкінци народови такі права осідленя (laws of
settlement), котрі, mutatis mutandis, так само повліяли на
англійських хліборобів, як указ татарина Бориса Ґодунова
на россійських хліборобів.

„Преславна революція“ (glorious Revolution) з Вільгельмом
III Оранським дала панованє в руки ґрунтових та капіталістичних
богатирів. Вони почали нову еру тим, що до
роскраданя державних ґрунтів, котре доси велося скромно
і тайком, взялися тепер на кольосальний розмір. Ті ґрунти
роздаровувано, продавано за песі гроші або й прямо без
даня рації прилучувано до приватних дібр\footnote{
„Безправна рострата коронних дібр чи то через продаж, чи через
роздарованє, становит огидну картку англійської історії\dots{} Се величезне
окраденє народа\dots{}“ (F.~W.~Newmann: „Lectures on Political Economy.
London, 1851“. стор. 129, 130).
}. Все то робилося
без найменьшої вваги на правні формальности. Ті закрадені
добра державні ураз із церковним фурфантєм, яке
\parbreak{}

\parcont{}  %% абзац починається на попередній сторінці
\index{i}{0069}  %% посилання на сторінку оригінального видання
товару іншим разом з тим призводить до того, що в руках третьої
особи лишається товар-гроші\footnote{
Примітка до другого видання. Хоч і як впадає на очі це явище,
однак політико-економи здебільша не помічають його, особливо ж вульґарні
прихильники вільної торговлі.
}. Циркуляція завжди стікає грошовим
потом.

\disablefootnotebreak{}
Нема нічого безглуздішого, як та догма, ніби циркуляція
товарів доконечно зумовлює рівновагу продажів і купівель, з
тієї причини, що кожний продаж є купівля й vice versa\footnote*{
навпаки. \emph{Ред.}
}. Коли
цим хочуть сказати, що число дійсно проведених продажів дорівнює
такому самому числу купівель, то це пласка тавтологія.
Але цією догмою хочуть довести, що продавець веде за собою
на ринок свого покупця. Продаж і купівля є тотожний акт як
взаємне відношення між двома полярно протилежними особами:
посідачем товарів і посідачем грошей. Вони становлять два полярно
протилежні акти як учинки тієї самої особи. Тим то тотожність
продажу й купівлі містить у собі й те, що товар стає некорисним,
коли він, кинутий в альхемічну реторту циркуляції, не
виходить із неї у формі грошей, коли посідач товарів його не продасть,
отже, коли посідач грошей його не купить. Ця тотожність
містить у собі далі те, що цей процес, якщо він удасться, становить
певну павзу, певний період в житті товару, який може тривати
довше або коротше. Через те, що перша метаморфоза товару є
одночасно продаж і купівля, цей частинний процес є разом з
тим самостійний процес. Покупець має товар, продавець має
гроші, тобто товар, який зберігає форму, що робить його здатним
до циркуляції, незалежно від того, чи він раніше або пізніше
знову з’явиться на ринку. Ніхто не може продати без того, щоб
хтось інший не купив. Але ніхто не потребує негайно купувати
через те тільки, що він сам щось продав. Циркуляція товарів
ламає часові, місцеві й індивідуальні межі обміну продуктів
саме тим, що наявну тут безпосередню тотожність між відчуженням
через обмін власного продукту праці й привласненням
чужого вона розколює на два протилежні акти — продаж і купівлю.
А що ці самостійні один проти одного процеси становлять
унутрішню єдність, — то це так само говорить і про те, що їхня
внутрішня єдність рухається в зовнішніх протилежностях\footnote*{
У французькому виданні це речення подано так: «Правда, що купівля
є доконечне доповнення продажу, але не менш справедливо, що
їхня єдність є єдність протилежностей». («Le Capital etc.», v. I, ch. III, p. 47).
\emph{Ред.}
}.
Коли зовнішнє усамостійнення внутрішньо несамостійних актів, —
бож вони доповнюють один одного, — доходить до якогось певного
пункту, то єдність їхня проявляється ґвалтовно — через
кризу. Іманентна товарові протилежність споживної вартости й
вартости, приватної праці, що разом з тим мусить з’являтися
як безпосередньо суспільна праця, осібної конкретної праці,
що разом з тим має значення тільки абстрактної загальної праці,
між персоніфікацією речей і зречевленням осіб, — ця іманентна
\parbreak{}  %% абзац продовжується на наступній сторінці

\parcont{}  %% абзац починається на попередній сторінці
\index{iii2}{0070}  %% посилання на сторінку оригінального видання
за нею в наші часи провадиться значну частину операцій. Такі люди охоче
гублять 20, 30 та 40\% на одній відправі товару кораблем; ближча операція
може вернути їм ті втрати. Коли їм раз-по-раз не щастить, тоді вони гинуть;
і саме такі випадки ми часто бачили останніми часами; торговельні фірми
збанкрутували, не залишивши жодного шилінґа в активі.

«4791. Нижчий рівень проценту [протягом останніх 10 років], звичайно,
має для банкірів несприятливий вплив, але, не подаючи вам для огляду торговельних
книг, мені було б дуже тяжко пояснити вам, оскільки теперішній
зиск [його власний] вищий від попереднього. Коли рівень проценту низький у
наслідок надмірного видання банкнот, то в нас є багато вкладів; коли рівень
проценту високий, то це дає нам безпосередній бариш. — 4794. Коли гроші можна
мати за помірний процент, то ми маємо більший попит на них; ми більше
визичаємо; такий вплив має це [для нас, банкірів] у цьому випадку. Якщо він
підноситься, то ми одержуємо за ті позики більше, ніж то годиться; ми одержуємо
більше, ніж повинні б мати».

Ми бачили, що всі експерти вважають кредит банкнот Англійського банку
за непохитний. А проте, банковий акт покладає цілком точно суму 9--10
мільйонів золотом для забезпечення розміну тих банкнот. Святість та непорушність
цього скарбу здійснюється, отож, цілком інакше, ніж у давніх збирачів
скарбів. W. Brown (Liverpool) свідчить перед C. D. 1847/58, 2311 так: «Щодо
тієї користи, яку ці гроші [металевий скарб в емісійному відділі] давали в ті
часи, так їх так само добре можна було б кинути в море; аджеж не можна було
навіть найменшої частини їх ужити, не ламаючи того парламентського акту».

Підприємець — будівничий Е. Capps, що його ми вже раніше згадували,
той самий, що з його свідчень узято характеристику сучасної лондонської системи
будівництва (Книга II, розд. XII), так резюмує свій погляд на банковий
акт 1844 року (В. А. 1857): «5508. Отже, ви взагалі тієї думки, що сучасна
система [банкового законодавства] дуже зручна установа для того, щоб періодично
кидати зиски промисловости до грошової торби лихваря? — Я такої думки.
Я знаю, що в будівельній справі ця система мала такий вилив».

Як згадано, шотландські банки примушено банковим актом 1845 року до
такої системи, що наближалась до англійської. На них поклали обов’язок тримати
золото в запасі на покриття банкнот, що вони видаватимуть понад суму,
усталену для кожного банку. Який вилив це мало, про це подаємо тут кілька
свідчень перед В. C. 1857.

Kennedy, управитель одного шотландського банку: «3375. Чи перед заведенням
банкового акту 1845 було в Шотландії дещо таке, що можна було б
назвати золотою циркуляцією? — Нічого подібного. — 3376. Чи від того часу
збільшилась кількість золота в циркуляції? — Ані найменше; люди не хочуть
мати золота (the people dislike gold)» — 3450. Ті приблизно 900.000 ф. ст. золота,
що їх шотландські банки мусять тримати, починаючи від 1845 року, на його думку,
тільки шкодять та «непожиточно поглинають рівну собі частину капіталу Шотландії».

Далі, Anderson, управитель Union Bank of Scotland: «3558. Єдиний значний
попит на золото, що його Англійський банк мав з боку шотландських
банків, був з нагоди закордонних вексельних курсів? — Це так; і цей попит
не зменшився від того, що ми тримаємо золото в Едінбурзі. — 3590. Поки ми
тримаємо ту саму суму цінних паперів в Англійському банкові» [або в приватних
банках Англії], «ми маємо ту саму силу, що й раніш, до того, щоб
викликати відплив золота з Англійського банку».

Насамкінець, ще одна стаття з Economist’a (Wilson): «Шотландські банки
тримають у своїх лондонських аґентів вільні суми готівкою; останні тримають
ті суми в англійському банкові. Це дає шотландським банкам змогу порядкувати
металевим скарбом банку в межах цих сум, а той скарб є завжди тут, на тому
\parbreak{}  %% абзац продовжується на наступній сторінці

\parcont{}  %% абзац починається на попередній сторінці
\index{iii2}{0071}  %% посилання на сторінку оригінального видання
місці, де його уживають, коли доводиться робити закордонні платежі». Цю систему
порушив акт 1845 року: «В наслідок акту, виданого для Шотландії
року 1845, останнім часом відбувся значний відплив золотих монет з Англійського
банку, щоб попередити той лише можливий попит на них в Шотландії,
що, може бути, й ніколи не постав би\dots{} Від цього часу одну значну суму
реґулярно закріпляють в Шотландії, а друга теж значна сума раз-у-раз мандрує
туди та сюди між Лондоном та Шотландією. Якщо настає час, коли шотландський
банкір чекає збільшеного попиту на свої банкноти, то відправляється
туди скриньку з золотом з Лондону; а коли цей час минув, то та сама скринька,
здебільша навіть не бувши одкрита, вертається назад до Лондону». (Economist
23 жовтня 1847~\abbr{р.}).

[А що каже з приводу всього цього батько банкового акту, банкір Samuel
Jones Loyd, alias лорд Оверстон?

Вже 1848 року він повторив перед C.~D. комісією лордів, що «грошову
скруту та високий рівень проценту, викликані недостатньою кількістю капіталу,
не можна полегшити збільшеним виданням банкнот» (1514), дарма що
самого лише \emph{дозволу} урядового листа з 25 жовтня 1847 року на збільшене
видання банкнот було досить, щоб збити тій кризі вістря.

Він лишається при тій думці, що «висока норма рівня проценту та пригнічений
стан фабричної промисловости були неминучим наслідком зменшення
\emph{матеріяльного} капіталу, що його можна було ужити для промислових та комерційних
цілей». (1604). А проте, пригнічений стан фабричної промисловости
протягом місяців був у тому, що матеріяльний товаровий капітал понад міру
наповнював комори, але його просто не сила було продати, і саме тому матеріяльний
продуктивний капітал цілком або напів лежав без діла, щоб не продукувати
ще більше того товарового капіталу, що його не сила було продати.

І перед банковою комісією 1857 року він каже: «Через гостре та ретельне
додержування засад акту 1844 року все відбувалося реґулярно та легко, грошова
система певна та непохитна, розцвіт країни безперечний, громадське довір’я
до акту 1844 року щодня зростає. Коли комісія бажає ще дальших
практичних доказів тому, що ті принципи, на які спирається цей акт, здорові,
та доказів тих благодійних наслідків, що їх він забезпечує, то на це ось правдива
та достатня відповідь: подивіться навколо себе; погляньте на сучасний
стан справ в нашій країні, погляньте на задоволення народу; погляньте на
багатства та розцвіт всіх кляс суспільства, і тоді, побачивши все це, комісія
буде в стані вирішити, чи схоче вона повстати проти дальшого існування акту,
що ним досягнуто таких наслідків». (В.~C. 1857, № 4189).

На цей дитирамб, що його Оверстон заспівав перед Комісією 14 липня,
відповідь дано було 12 листопада того самого року, тим листом до дирекції
банку, що ним уряд припиняв чинність чудотворного закону 1844 року, щоб
врятувати бодай те, що можна ще було врятувати. — Ф.~Е.]

\sectionextended{Благородний метал та вексельний курс}{%
\subsection{Рух золотого скарбу}}

Щодо нагромадження банкнот підчас скрути треба зауважити, що тут повторюється
те саме явище збирання скарбів у благородному металі, що в початковому
періоді суспільного розвитку завжди бувало за неспокійних часів. Акт
1844 року в своїй чинності являє інтерес тому, що він воліє перетворити ввесь
наявний у країні благородний метал на засоби циркуляції; він намагається вирівняти
відплив золота скороченням, а прилив золота збільшенням засобів циркуляції. Але
\parbreak{}  %% абзац продовжується на наступній сторінці

\index{franko}{0072}

Послухаймо ще хвильку, що говорит оден защитник „прилучуваня“ а противник Р.~Прайса: „Зовсім
фальшива тота гадка, що край обезлюднів, бо не видно-ді людей працюючих в чистім поли. Коли їх
тепер убуло по селах, то за то прибуло їх по містах\dots{} Коли дрібні ґазди-рільники перемінились в
наємних робітників, то через те сама кількість добутої праці стає більша, а се прецінь користь
пожадана для суспільности (тілько що, розумієся, самі „перемінені“ не належат до тої
суспільности!)\dots{} Добутку буде більше, коли скомбінована праця тих наємників буде ужита в \so{одній}
аренді; таким способом повстане надвишка витворів, котра піде до мануфактур, а через те й
мануфактур, тих жерел нашого богацтва, стане більше в стосунку до витвореної многоти збіжя“.

Незамутимий супокій, з яким суспільний економіст глядит на найзухвальше топтанє „святого права
власности“, на найгидше знущанє над людьми, коли йно все то робится для того, щоб покласти підвалину
капіталістичній продукції, проявляє між їншими торій і „філянтроп“ сер Ф.~М.~Еден. Цілий ряд
рабунків, головництв і притисків народних, серед яких відбувалося вивласнюванє люду від послідної
третини \RNum{15} до кінця \RNum{18} віку, викликає у него тілько сей супокійно-радісний вивід: „Належита
пропорція між вірними полями а толоками мусіла бути встановлена. Ще в цілім \RNum{14} і найбільшій части
\RNum{15} віку на оден екр толоки приходилося 2, 3, а навіть 4 екри вірного поля. В половині \RNum{16} віку
перемінилася пропорція: 3 екри толоки приходили на 2 екри рілі, а пізнійше 2 екри толоки на 1 екр
рілі, аж поки вкінци не вийшла належита пропорція: 3 екри толоки на 1 екр рілі“.

В \RNum{19} віці защезла, розумієся, й память про звязок між хліборобами а власністю громадською. Не
згадую вже зовсім о найпослідних часах, — але чи одержали селяне хоть оден шелюг відплати за тих
\num{3511770} екрів громадського ґрунту, котрі їм зрабовано між роками 1801 а 1831 і котрі с
парляментарними формальностями сільскі льорди подарували собі самим?

Послідний великий процес вивласнюваня хліборобів, се вкінци т. зв. „\textenglish{Clearing of Estates}“ (обчищуванє
дібр, або радше вимітанє з них людей). Тото „обчищуванє“, се вершок усіх англійських способів, які
ми доси бачили. Там, де вже не осталося незалежних ґаздів-хліборобів, доходит до вимітаня коттеджів,
так, що хліборобські робітники не можут уже найти й кусничка місця для замешканя на тім ґрунті,
котрий оброблюют. Властиве „обчищуванє дібр“ відзначуєся нечуваною сістематичністю і огромним
розміром, в якім тота операція нараз виконуєсь (в Шотляндії н.~пр.
вона відбувалася нараз на просторах таких завбільшки, як
\parbreak{}

\parcont{}
\index{franko}{0073}
цілі німецькі князівства), а також окремою формою ґрунтової власности, котру так насильно перемінюют
в приватну власність. Ті ґрунти, то була власність повіту (clan), — начальник або „великий чоловік“
був тілько титулярним властивцем, як представник повіту, так само, як королева англійська є
титулярною властителькою всего ґрунту Англії. Тот переворот, котрий в Шотляндії почався по посліднім
повстаню претендента, мож слідити в перших єго початках у письмах Джемса Стеєрта і Джемса Андерсона\footnote{
Стеєрт каже: „Рента в тих околицях (він хибно називає рентою тоту оплату, яку обивателі повіту
(taksmen) складали начальникови повіту) зовсім незначна в стосунку до обширности піль, але що до
числа осіб, котрих удержує одна аренда, мож сміло твердити, що оден кусник ґрунту в шотлянських
горах виживлює десять раз більше людей, ніж так само заобширний ґрунт в найбогатших рівнинах“.
}. В \RNum{18} віці заборонено притім Ґелям, прогнаним з ґрунтів, виселюватись в чужі краї, щоб їх таким
способом силою попхнути до Ґлязґова і других фабричних міст\footnote{
1860 виводжено тих насильно вивласнених хліборобів до Канади, отуманивши їх фальшивими
обіцянками. Деякі повтікали в гори і на сусідні пусті острови. Поліція пустилася за ними в погоню,
прийшло до бійки і втікачі здужали вирватися та порозбігатись.
}. За примір
методи пануючої в девятнайцятім віці\footnote{
„В шотляндських горах“, каже Бюкенен, коментатор А.~Сміта, 1814, „день в день насильно затираєся
давний власностевий порядок\dots{} Сільский льорд, без згляду на дідичних арендаторів (знов хибно
названі тексмени), винаймає ґрунт тому, хто найбільше платит, а коли той належит до меліораторів
(imprower), то зараз заводит новий спосіб управи поля. Ґрунт, давнійше покритий дрібними
властивцями, був в стосунку до своєї плодовитости досить заселений; при новім сістемі поліпшеної
управи і побільшеної ренти одержуєсь як мож найбільше плодів як мож найменьшим коштом, і для того
віддалюются робітники, котрі стали тепер непотрібними. Ті вигнанці з рідних хат шукают відтак
утриманя в фабричних містах і т. д. (David Buchanan: „Observations on A.~Smith’s Wealth of Nations.
Edinb. 1814“.) „Шотляндські маґнати вивласнили цілі родини, немов хопту випололи: вони так обійшлися
з селами й людністю, як Інди розїдлі пімстою з дикими звірями по норах\dots{} Чоловіка продают
за овече руно, за волове стегно, ба ні, ще за меньшу дрібницю\dots{} Підчас нападу на північні
провінції Хіни була на раді Монголів така думка, щоб усіх мешканців витратити а їх край перемінити
в степ. Тоту раду богато північно-шотляндських маґнатів дословно виповнили в своїм власнім краю і на
своїх власних земляках“. (Джордж Ензер: „An Inquiry concerning the Population of Nations. Lond.
1818“. Стор. 215, 216.)
} досить буде ту навести „обчищуваня“ герцоґині Созерлєнд.
Тота в економії вишколена особа постановила зараз в початку свого панованя взятися до радікального
ліку економічного, і ціле ґрафство, в котрім задля давнійших подібних процесів осталось
\index{franko}{0074}
було всего лиш \num{15000} люда, перемінити в толоку для овець. Від 1814 до 1820 сістематично
прогонювано та нищено тих \num{15000} мешканців, т. є. майже 3000 родин. Всі їх села поруйновано і
попалено, всі їх поля пороблено толоками. Англійських жовнірів викомендерувано там для еґзекуції, і
між ними а мешканцями прийшло до бійки. Одна
стара баба згоріла враз іс хатою, с котрої не хтіла вступитися. І таким способом присвоїла собі
вельможна герцоґиня \num{794000} екрів ґрунту, котрі споконвіку належали до
повіту. Вигнаним мешканцям визначила вона на морськім узберіжю около 6000 екрів, по 2 екри на
родину. Тих 6000 екрів лежали доси пусто і не давали властительці ніякого
доходу. Герцоґиня так далеко зайшла в своїй щедрости, що винаймила екр пересічно по 2\shil{ шілінґи}
6\pens{ пенсів} для тих самих селян, котрі много сот літ проливали кров свою за
вельможну герцоґську родину. Увесь зрабований ґрунт повіту поділила герцоґиня на 29 великих аренд
для випасаня овець; в кождій аренді осіла тілько одна родина, переважно англійські наємні
арендаторі. 1825 р. замісць \num{15000} Ґелів на їх ґрунтах жило вже \num{181000} овець. А родини, вивержені на
морський беріг, старалися жити риболовством. З них поробилися земноводяні, і вони жили, як каже
писатель, на половину в воді, а на половину на березі, тілько що ні ту ні там не могли найти
достаточного прожитку\footnote{
Коли теперішна герцоґиня Созерлєнд витала в Льондоні з великою парадою міссіс Бічер Стоу,
авторку „Хати дядька Томи“, щоб виставити на показ свою прихильність для муринів-невольників в
американській републіці — чого вона і єї співарістократки певно не булиб зробили підчас домашної
війни американської, бо тоді кожде „шляхотне“ англійське серце було прихильне плянтаторам — в той
сам час описав
я в газеті „New-York-Tribune“ побут невольників созерлєндських. (Деякі місця тої статі навів Керей в
своїй „The Slave Trade. London 1853“.). Мою статю перепечатала одна шотляндська ґазета і викликала
дуже чемну перепалку між тою ґазетою а підхлібниками та похвальками герцоґів Созерлєндів.
}.

Але небораки Ґелі мусіли щераз відпокутувати свою романтичну наклінність для „великих мужів“, т. є.
для начальників повітових (Сlanchef). Запах риб, котрими прокормлювались земноводяні Ґелі, ударив
великим мужам в ніс. Вони завітрили тут щось зисковного і заарендували морське узберіжє великим
льондонським гендлярам риб. Ґелів другий раз вигнано на штири вітри\footnote{
Цікаву історію того рибного торгу найде читатель у д. Девіда Оркуарта в єго книжці: „Portofolio.
New Series“. Сеніор в одній іс своїх посмертних статей називає „процедуру в Созерлєндшайрі“ одним з
найблагодатнійших очищень від віків.
}.

Аж вкінци одну часть пасовиськ назад перемінено
\parbreak{}

\parcont{}
\index{franko}{0075}
в місця для польованя. Звісна річ, що в Англії нема правдивих гір. Дичина в маґнацьких парках, се
констітуційна домашна худоба, товста, як льондонські ольдермени. Шотляндія, се затим послідне місце
для „шляхотного занятя“ — стрілецтва. „В шотляндських горах“, каже Сомерс 1848, „ліси дуже стали
обширні. Ось з одного боку від Ґейка маєте новий ліс Ґлєнфшай, а з другого боку також новий ліс
Ордверікай. Поряд з ними бачите Блік-Моунт, огромну пустиню, свіжо заложено. Від сходу до заходу,
від околиць Обердіна аж до урвищ Обена тягнутся тепер без перерви ліси. А й по других частях
Шотляндії находятся нові ліси, як ось в Льоч Орчейґ, Ґлєнджеррі, Ґлєнморістен і др\dots{} Переміна
ґрунтів в овечі толоки прогнала Шотляндців на неврожайні пустарі. Тепер серни та лиси починают
витискати овець, а Шотляндців кидати ще в страшнійшу нужду\dots{} „Ліси для дичини“\footnote{
Шотляндські „ліси для дичини“ (deer forests) не мают і одного деревця. Се звичайні голі толоки,
с котрих вигнано вівці, а на їх місце нагнано оленів, — тай ось і преславетний „ліс для дичини“. Про
засів та плеканє лісів і казки нема!
}, а народ(и) не
можут істнувати побіч себе. Або одні, або другі мусят уступити місця. Нехай тілько число і розмір
польовань в слідуючих 25 роках зможеся так, як в минувших, то певно не здиблете й одного Шотляндця в
єго ріднім краю. А се змаганє між шотляндськими маґнатами походит по части з моди, панської бути,
забагів на польованє і т. д., а по части вони займаются дичиною для зиску. Бо кусень гористого
простору, затичений для польованя, нераз далеко більше дає зиску, ніж колиб був толокою. Той, кому
забаглося польованя і шукає такого обшару, платит за него тілько, накілько статчит єго кішеня,
— (а се вже певно, що бідному чоловікови не до польованя!)\dots{} Відти то сплило на шотляндські гори
тілько недолі, кілько єї сплило на Англію ізза політики норманських
королів. Оленям віддано огромні простори до волі, а людей зігнано в тісні і чим раз тіснійші
закамарки\dots{} Одну вільність за другою видирано народови\dots{} І притиск той ще
день-денно змагаєся. Властивці з засади і загалом вимітают і виганяют народ, мов сіно косят, — мов
ті австральські та американські осадники відвічні ліси витинают, і тота операція поступає чим раз
далі, спокійно, с холодною розвагою та обрахунком\footnote{
Robert Somers: „Letters from the Highlands: or, the Famine of 1847. Lond. 1848“, стор. 12--28. Ті
листи надрукувала зразу ґазета Таймc. Англійські економісти, розумієся, зараз ростолкували, що
Шотляндці бідуют і мрут з голоду задля — перелюдненя. Сяк чи так, а їсти не було що. І в Німеччині
не чужа тота операція „Clearing of Estates“, — єї ту прізвано „Bauernlegen“ (обалюванє мужиків), і
вона особливо далася чути по 30-тилітній війні. Ще 1790 в одній части Саксонії вона викликала хлопські бунти. А найбільше
ширилося „обалюванє мужиків“ в східній Німеччині. В найбільшій части німецьких провінцій аж Фрідріх
II запевнив селянам право власности. По завойованю Шльонська він присилував дідичів до відбудованя
хат, стоділ і т. д., а також до заосмотреня мужицьких осад худобою та ґосподарськими знадобами.
Фрідріх II потребував жовнірів для війська і оподаткованих людей для побільшеня державного
скарбу. Впрочім селянам і під Фрідріхом II жилось далеко не гарно, як се мож побачити с письм єго
головного похвальця, Мірабо.

В цвітню 1872, 18 літ по виданю згаданої ту книжки Сомерса читав проф. Лєоні Лєві в „Society of
Arts“ відчит про переміну овечих пасовиск в ліси для дичини, де вказує дальший розвиток спустошеня в
шотлянських височинах. Між їншим каже він: „Обезлюдненє і переміна ґрунтів в голі толоки, се був
найвигіднійший для панів спосіб — получити доходи без видатків\dots{} Тепер же в височинах звичайно
пороблено с толок „deer forest-и“. Дичина прогнала овець так, як недавно вівці прогнали були людей. Мож вандрувати від
дібр ґрафа Дельгаузі в Форфершайрі аж до Джона o’Ґротса, не виходячи зовсім з лісів. В многих іс
тих лісів замешкуют лиси, дикі коти, куни, тхорі, ласиці та альпейські заяці; від недавна
росплодилися там також крілики, вивірки та щурі. Огромні простори, котрі в шотлянській статистиці
значились „надзвичайно врожайні і розляглі пастівники“, позбавлені тепер всякої управи і поправи і служат виключно для мисливської
забави кількох осіб, тай то лиш короткий час в році!“

Льондонський „Economist“ з 2 червця 1866 каже: „Одна шотлянська ґазета доносит послідного тижня між
їншими новинами ось що: Одна з найкращих овечих аренд в Созерлєндшайрі, за котру недавно, за упливом
біжучого арендового контракту, давано річної ренти 1200\pound{ фунтів штерлінґів}, тепер зістає перемінена в
„deer forest!“. Феодальні інстінкти проявляются й тепер так само, як тоді, коли норманський
завойовник зруйнував 36 сіл, щоб закласти Ню-форс (Новий ліс)\dots{} Два мілійони екрів, самих
найурожайнійших в Шотляндії, опустошуются тепер до крихітки. Природна трава Ґлєн-Тільту належала до
найпоживнійших в ґрафстві Перз; теперішний дір-форст Бен-Альдер був найкращим пасовиском в цілім
лісистім Бедноч; одна часть теперішного Блік-Моунт-форста була найліпшим на всю Шотляндію пасовиском
для чорномордих овець. Про обсяг ґрунтів опустошеннх для стрілецької примхи мож виробити собі
яке-таке понятє, зваживши, що вони обіймают далеко більше простору, ніж ціле ґрафство Перз. Що через
те насильне опустошенє стратив край на жерелах продукції, мож оцінити с того, що форс Бен-Альдер міг
% REMOVED \footnote*{В рукописі: між.}
прокормити \num{15000} овець і що він становит лиш \sfrac{1}{30} всіх „диких лісів“ шотлянських.
Весь той „дикий“ ґрунт зовсім не продуктівний\dots{} На одно б вийшло, як би був запався в фалі
Північного моря. Сильна рука праводавства повинна би прецінь зупинити розріст і творенє таких
самовільних пустинь“.
}.

\index{franko}{0076}
Рабунок дібр церковних, злодійське загарбуванє державних маєтків, крадіж громадських ґрунтів,
безправна \parbreak{}

\parcont{}
\index{franko}{0077}
а з беззглядною жорстокістю переведена переміна феодальної та окружної (Clan-)
власности в новійшу приватну власність, — ось які іділлічні були способи
первісного нагромадженя капіталу. Вони здобули ґрунт для капіталістичного
рільництва, втягли землю в обсяг капіталу, а міському промислови
достатчили потрібних „рук“, т. є. вольного і голого пролєтаріяту.

\subsection{Кроваві устави протів пролєтаріїв при кінци XV віку}

Вольний і голий пролєтаріят, вигнаний с хат і ґрунтів
через скасованє феодальних дворів і через насильне раз-заразом
вивласнюванє, не міг відразу перелятися весь до
новоповстаючих мануфактур так швидко, як швидко сам
повстав. А при тімже се були люде, викинені раптово с привичного
способу житя, — а такі люде не швидко можут
застосоватися до яких небудь нових, непривичних порядків.
На першій порі з них поробилися маси жебраків, розбійників,
волоцюг, — деякі з наклінности, а найбільша часть під гнетом обставин. С
кінцем XV і підчас цілого XVI віку бачимо проте в цілій Західній Европі
кроваві устави протів волоцюгів. Батьки нинішної робітницької верстви мусіли
на самім вступі відбути страшну кару, — за що? За то, що їх перемінено в волоцюг
та голоту. Праводавці вважали їх „добровільними переступцями“ і думали, що
тілько від їх доброї волі залежит — працювати далі серед давних обставин, котрі
між тим зо світа щезли.

В Англії почалось те праводавство під Генріхом VII.

Генріх VIII, 1530: Старі і неспосібні до праці жебраки одержуют дозвіл на
жебрацтво. За то здорові й міцні волоцюги карані будут батогами й арештом. Вони
мают бути привязані ззаду до тачок і бичовані доти, доки не поплине кров з їх
тіла, — відтак мусят зложити присягу, вернути на місце уродженя або там, де
пробули послідні 3 роки  і „засісти до праці“ (to put himself to labour). Що за
безсердечна насмішка! В 27 уст. Генріха VIII повторена попередна устава, але
заострена новими додатками. Як кого другий раз зловят на волоцюгованю, то такого
бичувати ще раз і відтяти му пів вуха. За третим разом непоправного волоцюгу,
як тяжкого злочинця і ворога суспільности — вкарати смертю.

\looseness=-1
Едуард VI: Устава с першого року єго панованя 1547, наказує, що скоро хто
отягаєся від праці, той має бути присуджений на невольника тій особі, котра
донесла урядови о єго неробстві. Пан має годувати невольника хлібом і водою,
слабими напитками і такими обрізками мяса, які му видадутся відповідними. Він
має право всилувати го батогами \index{franko}{0078}
та зелізними ланцами до всякої, хотьби й як гидкої роботи. Коли невольник на 14
день віддалится, то зістає засуджений на віковічну неволю і має бути на чолі
або на лици напятнований буквою S, а коли до трох раз утече, то має бути
вкараний смертю, як зрадник держави. Пан може го продати, передати в наслідство,
визичити другому в неволю, зовсім так, як усяке друге рухоме добро, як худобу.
Коли невольники в чім небудь станут супротів панів, то мают також бути покарані
смертю. Мирові судьї повинні за отриманим остереженєм слідити за волоцюгами.
Коли покажеся, що такий волоцюга три дни волочився без діла, то такого
відставити на місце, де родився, роспеченим зелізом напятнувати на груди буквою
V і тамій в зелізних ланцюхах уживати до замітаня вулиці або до якої небудь
їншої служби. Коли волоцюга подасть фальшиво місце вродженя, то за кару має
бути віковічним невольником тої громади, тих мешканців або того товариства і
напятнований буквою S. Кождий має право відобрати у волоцюги єго діти і яко
помічників та термінаторів держати хлопців до 24, дівчат до 20 літ. Коли вони
втечут, то мают аж до тих літ бути невольниками майстра, а тому вільно їх
заковувати в ланци, бити і пр., як му сподобаєсь. Кождий пан може заложити
зелізну обручку на шию, руку або ногу свого невольника, щоби міг го ліпше
пізнати і бути певним, що му не втече\footnote{
Автор книжки „Essay on Trade and Commerce“ 1770, каже: „Під панованєм Едварда
VI взялись були Англічане зовсім, здаєсь, серйозно до піддвигненя мануфактур і
затрудненя бідних. Се бачимо з одної дивовижної устави, в котрій приписуєсь, що
всі волоцюги мают бути пятновані, і т. д. (Essay on Trade and Commerce, стор.
8).
}. Послідна часть тої устави наказує, щоб
деяких бідних брали на себе громади або поєдинчі люде; ті мают їм давати їсти
й пити і старатись для них о роботу. Тот рід громадських невольників удержувався
в Англії гет ще в \RNum{19} віці під назвою roundsmen (люде, що ходят від хати до
хати).

Єлисавета, 1572: жебраки без дозволу і віком понад 14 літ мают бути без
милосердя бичовані і напятновані на лівім вусі, хіба що їх хто схоче взяти на
два роки на службу; в разі повтореня, коли мают над 18 літ, мают бути — смертю
карані, скоро їх ніхто не схоче взяти на два роки на службу; за третим разом
мают без милосердя як зрадники державні бути покарані смертю. Подібна також \RNum{18}
устава Єлисавети, розділ 13, і устава з р. 1597 \footnote{
Томас Морус каже в своїй „Утопії“: „Так то дієся, що оден захланний і неситий
ненаїсник, правдива чума нашої вітчини, може тисячі екрів ґрунту збити до купи
і обпалькувати, обгородити одним плотом, або силою та кривдою до того довести
єго властивців, що вони будут мусіли все спродувати. Сяким чи таким способом,
чи там гнись чи ломайся, він присилує їх забиратися, — бідні, прості, нещасливі
душі! Мужчини й женщини, чоловіки й жінки, сироти без батьків, удови, плачучі
матері с пеленковими дітьми, і вся челядь, убога добром, а богата
ротами, бо рільництво вимагає богато рук. І волочутся вони, кажу вам,
з знакомих, рідних місць, не находячи пристанівку. Якби при й нетаких
обставинах, то моглиб бодай що то вторгувати за свій, хоть і не дуже
цінний, домашний спряток; але раптово повикидувані, мусят усе продавати
за песій гріш. А коли перебурлачат послідний свій гріш, то щож
тоді мают робити, як не красти, а відтак, боже добрий, по всій формі та
правді згинути на шибеници або пуститися на жебри. А й тоді ще їх
попрут до вязниць як волоцюгів, що-ді плентаются а нічо не робят.
А що там судови до того, що їх ніхто не хоче взяти на роботу, хоть би
й як радо самі на ню напрошувались!“ І таких бідних утікачів, котрих
по словам Томаса Моруса присилувано до крадіжи, „за панованя Генріха
VIII, повішено \num{72000} великих та дрібних злодіїв“. (Ноllingshed, Dеscription
of England, т.~І, стор. 186). За часів Єлисавети „вішано волоцюгів
цілими рядами; а прецінь не було такого року, в котрім би на
однім або другім пляцу не повішено їх 300--400“ (Strype`s Annals, т.~II).
Той сам Страйп свідчит, що в Соммерcетшайрі за оден рік повішено 40
люда, напятновано 35, бито батогами 37, а випущено 183 „непоправних
злочинців“. А такій, каже він, „те велике число оскаржених не становит
ще й пятої части всіх злочинців, дякувати недбальству мирових судів
і глупому милосердю народа“. Він додає: „Прочі англійські ґрафства
зовсім не стояли ліпше від Соммерсетшайра, а богато стояло в тім згляді
ще далеко гірше“.
}.

\index{franko}{0079}
Яков І: Кождий, хто ходит від села до села і жебрає,
узнаєсь волоцюгою. Мирові суді мают право засудити го на
прилюдне бичованє і за першим разом на 6 місяців, за
другим на 2 роки тюрми. Підчас сидженя в тюрмі мают
бути так часто і так богато бичовані, як се мировий судя
узнасть за добре\dots{} Непоправні і небеспечні волоцюги мают
бути на лівім плечи напятновані буквою R і заставлені до
робіт примусових, а як їх ще коли придиблют на жебранині,
то мают бути без милосердя і без сповіди повішені. Ті устави,
% REMOVED \footnote*{В рукописі: уставі.}
правосильні аж до перших літ \RNum{18} віку,
знесені зістали доперва \RNum{12} уст. Анни, розд. 23.

Подібні устави бачимо і в Франції, де в половині \RNum{17}
віку завязалось було ціле царство волоцюгів (truands) в Парижи.
Ще в початку панованя Людовіка XVI (Указ з дня
13 липня 1777) кождий здорово збудований чоловік від 16
до 60 літ віку, скоро був без удержаня і не мав означеного
занятя, мав бути висланий на ґалєри. Подібні також: устава
Карля V для Нідерляндів з 6 жовтня 1537, перший едікт
держав і міст голяндських з 19 марта 1614., оповіщенє Сполучених
провінцій з д. 25 червня 1649 і богато других.
Ось яким способом, — батогами, пятнованєм та тортурами
на підставі нелюдських, кровавих устав увігнано мужиків,
\parbreak{}

\parcont{}
\index{franko}{0080}
насилу обрабованих з ґрунту, хат і майна, насилу пороблених
злодіями та волоцюгами, в ті тверді рами карности,
конечної при сістемі наємної праці.

\vspace{-\medskipamount}
\subsection{Устави для знищеня робучої плати}
\vspace{-\bigskipamount}

\disablefootnotebreak{}
Не досить того, що знадоби продукції розділюются:
на однім боці сам капітал (в руках властивців богатирів),
а на другім боці сама праця, т. є. люде, котрі нічо не мают
на продаж крім своєї праці. Не досить ще присилувати
тих людей до того, щоб добровільно себе самих запродували.
В дальшім ході капіталістичної продукції виростає
вже верства робітників, котра з вихованя, традиції, привички
признає вимоги того способу продукованя природними законами,
чимось таким, що й бути інакше не може. Впорядкованє
видосконаленого капіталістичного процесу продукційного
перемагає всі запори; ненастанне повставанє релятівного
перелюдненя\footnote*{
Звісно, що перелюдненєм звеся то, коли де небудь є забагато
людей, т. є. властиво більш людей, ніж може вижити. А релятівне перелюдненє
значит, що тілько в певнім місци і серед певних обставин є для
певного діла забогато людей, так що всі вони не можут приміститися,
і одна часть з них дармує. Кождий пійме, що вже сама проява такого
релятівного перелюдненя є знаком нездорових економічних обставин.
Між тим, як побачимо далі, ціла капіталістична продукція нерозлучно
звязана с релятівним перелюдненєм, котре змоглося в краях промислових
особливо від заведеня парових машин, через що мілійони рук робітницьких
стратили роботу (\emph{Прим.~перев.}).
} вдержує довіз робучих рук і попит
за працею, значит, і робучу плату на такій висоті, яка кориснійша
для підростаючого капіталу; німий примус економічних
обставин довершує панованя капіталіста над робітником.
Позаекономічна, беспосередна сила входит все ще
в уживанє, але вже лиш виїмково. При звичайнім ході діла
досить є — лишити робітника під властю „природних законів
продукції“, т. є. лишити го в залежности від капіталу,
витвореній і навіки забеспеченій самими вимінками
продукційними. Але сего не мож зробити в тій історичній
хвили, коли капітал і етична продукція інощо зароджуєсь.
Підростаюча буржоазія потребує і уживає власти державної,
щоб „реґулювати“ робучу плату, т. є. втискати єї в такі
границі, які найкориснійші для баришництва, продовжувати
день робучий і вдержувати самого робітника в „належитій“
степени залежности. Се також дуже важний причинок до
т. зв. первісного нагромадженя капіталу.
\enablefootnotebreak{}

\looseness=-1
Верства наємних робітників, що повстала в послідній
половині \RNum{14} віку, становила тоді і в слідуючих столітях
тілько дуже незначну часть людности, котрої становище
\parbreak{}

\parcont{}
\index{franko}{0081}
притім міцно обезпечували самостійні ґаздівства по селах
а цехові звязки по містах. По селах і містах не було великої
суспільної ріжниці між майстрами а робітниками.
Підчиненє праці під капітал було тілько формальне, т. є.
продукція сама не мала ще на собі окремої капіталістичної
ціхи. Попит за наємною працею змагався прото дуже швидко
за кождим нагромадженєм капіталу, — між тим рук готових
найматися до праці прибувало дуже поволи. Велика
часть витворів суспільних, що пізнійше стала фондом вбільшуючим
капітал, тоді переходила ще в руки робітника для
єго власного зужитку.

Праводавство про наємну працю, згори вже вицілене
на визискуванє робітника і в своїм розвитку йому завсігди
однаково неприхильне, почалося в Англії від виданя „Устави
робітницької“ (Statute of Labourers) Едвардом III, 1349.
Рівночасно видано в Франції Указ 1350 р. в імени короля
Жана. Англійські і французькі устави виходят рівнобіжно
і зовсім однакі що до змісту.

Устава робітницька зістала видана за про голосні наріканя
послів. „Давнійше“, каже наівно оден Торі, „жадали
бідні такої великої плати за роботу, що промисл і богацтво
були загрожені. Тепер плата така низька, що знов грозит
промисловії й богацтву і то може ще небеспечнійше ніж
тоді“. Установлено правну тарифу платну для міст і сіл,
за роботу
%REMOVED (в рукоп. „робуту“)
на дни й від штуки. Сільскі робітники
повинні винайматися на рік, міські „с прилюдного
торгу“. Під карою тюрми заборонено платити висшу плату
від означеної в уставі; а хто бере більшу плату, того кара
виносит більше, ніж сама плата. Так само ще в розд. 18
і 19 устави о учениках ремісницьких, виданої за Єлисавети,
грозится карою 10 день тюрми тому, хто платит більше,
а 21 день тюрми тому, хто бере більшу плату від правом
приписаної. Устава з р. 1360 заострила кари і навіть дала
майстрам право силувати робітників мусом до праці за таку
плату, яка означена в тарифі. Всякі звязки, угоди, присяги
і т. д., котрими взаїмно сполучилися теслі з мулярами,
узнані неважними. Стоваришеня робітницькі караются як
тяжка провина від \RNum{14} віку до 1825, в котрім скасовано
устави протів стоваришень. Дух „Робітницької устави“ з р.
1349 і єї потомків просвічує ясно й с тих устав протів стоваришень.
Се тота сама засада: держава приписує, кілько
мож найбільше платити робітникови, але хрань боже, щоб
хоть натякнула на те, кілько мож йому найменьше платити!

В \RNum{16} віці, як звісно, положінє робітників дуже погіршилося.
Правда, грішми плачено більше, тількож що ціна
грошей стала меньша а ціна товарів без міри більша. Наділі
затим і плата вменьшилася. А прецінь устави для єї
зниженя трівают далі порівно з обрізуванєм вух та пятнованєм
\index{franko}{0082}
тих, „котрих ніхто не хоче взяти на службу. Єлисаветина
5 устава про учеників ремісницьких, уст. 3 надає
мировим судям власть становити де в яких реміслах плату
і змінювати її відповідно до пори року і ціни товарів. Яков
I ростягнув ту саму реґуляцію робітницької плати на ткачів,
прядільників і на всі можливі розряди робітників\footnote{
З одної примітки до устави 2 за Якова І, розд. 6 видно, що
деякі суконники позваляли собі самі яко мирові судьї урядово діктувати
платну тарифу в своїх варстатах. — В Німеччині, а іменно по 30-літній
війні, виходит богато устав для знижуваня робучої плати. „Поміщикам
на безлюдних ґрунтах дуже прикро давалась чути недостача слуг і робітників.
Всім мужикам-ґаздам заказано приймати в комірне мужчин та
женщин вільного стану; про всіх таких комірників повинно доноситися
урядови, а той запирає їх в тюрму, скоро не хотят стати слугами, хоть би
й без того мали яке їнше вдержанє, хоть би працювали у  мужиків за поденщину
або навіть торгували грішми та збіжєм. (Цісарські прівілєї та
ухвали для Шльонська, І, стор. 125). Через цілих сто літ роздаются в приписах
князів та поміщиків раз~відразу гіркі наріканя на злосливих
і здуфалих слуг, що не хотят піддатися важким условинам, не хотят вдоволюватися
платою правом приписаною. Виходят накази, щоб поєдинчий
поміщик не смів своїм слугам платити більше, ніж кілько весь краєвий
збір покладе в таксу. А прецінь условини служби по війні нераз ще
бувают ліпші, ніж були 100 літ опісля. В р. 1652 діставали ще слуги на
Шльонську по два рази до тижня мясо; а ще в нашім столітю іменно
там були такі округи, де слуги діставали мясо хіба три рази до року.
І поденщина (плата за день роботи) по 30-літній війні була більша, ніж
в слідуючих столітях“ (Ґустав Фрейтаґ).
}, Джордж
II ростягнув устави протів робітницьких товариств на всі
мануфактури. В властивій порі мануфактуровій капіталістична
продукція була вже досить сильною, щоб правну
реґуляцію робучої плати зробити непотрібною, а то й неможливою,
але все такі ще на всякий злучай не закидувано
того перестарілого оружя. Ще 8 устава Джорджа II заказує
давати кравецьким челядникам в Льондоні і околици більше
понад 2\shil{ шіллінґи} і півосьма пенса денної плати, окрім хіба
в разах загальної жалоби. Ще 13 уст. Джорджа III, розд.
68 повіряє мировим судям реґульованє робучої плати у виробників
шовку. Ще 1796 тре було двох декретів висших
судів для рішеня, чи накази мирових судьїв що до робучої
плати мают вагу і для не-рільничих робітників. Ще 1799
потвердила ухвала парляменту, що плата копальників шотляндських
уреґульована уставою Єлисавети і двома шотляндськими
актами з р. 1661 і 1671. А який між тим переворот
доконався у всіх обставинах, доказала подія нечувана
в англійській палаті панів. Ту, де від звиш 400 літ
фабриковано устави виключно о тім, понад яку міру не
може ніяк переступити робуча плата, — ту поставив 1799
\parbreak{}

\parcont{}
\index{franko}{0083}
Уайтбрід внесок устави, яка може бути найменьша плата
для робітників рільничих\dots{} Хоть Пітт супротивлявся тому
внескови, то прецінь і сам признав, що „положінє вбогих
страшенне (cruel)“. Вкінци 1813 скасовано устави про реґуляцію
плати. Вони стались смішним недоріцтвом, відколи
капіталіст порядив у своїй фабриці після власних приватних
прав, а плата рільничого робітника давно впала понизше
мінімум конечного до прожитку, і мусіла до висоти
того мінімум доповнюватися с „податку на бідних“. Постанови
„Устави робітницької“ що до згоди між майстром
а наємним робітником, що до вимовленя терміну і т. д.,
постанови дозволяючі тілько цівільну скаргу на недодержуючого
умови майстра, а крімінальну скаргу на недодержуючого
умови робітника, — ті постанови стоят ще й доси
в повній силі. Нелюдські ухвали супротів стоваришень
скасовано 1825 з ляку перед грізною поставою пролєтаріяту.
Парлямент зніс їх дуже нерадо\footnote{
Деякі останки устави протів стоваришень знесено аж 1859 р.

(Додаток до 2 вид.) Устава з 29 червня 1871 зносит всі устави
протів стоваришень і урядово признає „Робучі Звязки“ (Trades Unions).
Але в однім додатковім акті с того самого дня, п. н. „An Act to amend
the Criminal Law relating to violence, threats and molestation” — устави
протів стоваришень щасливо воскресли в новій формі. Сесь акт піддає
іменно робітників за вживанє деяких средств воєнних протів майстрів
під окремі устави крімінальні, а судят робітників на підставі тих устав
самі ж майстри, яко мирові судьї. Два роки передтим та сама палата
послів і тот сам Ґлядстон, що 1871 винайшли нові проступки на робітників,
вихвалювали при другім єго читанню один внесок до устави, в котрім
чесним способом роблено конець всяким окремим праводавствам
протів робітників. Вихвалювали, вихвалювали, тай хитро-мудро стали на
другім читанню\footnote*{Звісно, що в Англійськім парляменті кождий внесок,
заким одержит силу права, мусит бути три рази читаний і більшістю голосів
принятий. (\emph{Прим. перев.})}. Цілі два роки відволікано сю справу, аж
поки „велике ліберальне сторонництво“ не звязалось зі своїми противниками
і не почулося задосить сильним, щоб разом стати — протів спільного
ворога — робітників.
},  той сам парлямент, що
сам довгі столітя с цинічним безвстидством виступав як
неустаюче стоваришенє капіталістів супроті робітників.

Сейчас в початках революційної бурі поквапилась французька
буржоазія інощо здобуте право стоваришень знов
видерти робітникам. В декреті с 14 червня 1791 оголосила
вона, що всі робітницькі стоваришеня, се „замах на свободу
і признані права чоловіка“, за котрий накладаєсь кара
500 ліврів і позбавленє на рік актівних прав горожанських.
Се право, котре конкуренційну боротьбу між капіталом
а працею силою поліційно-державною втискає в такі границі,
які вигідні для капіталу, перетрівало революції та зміни
\parbreak{}

\parcont{}  %% абзац починається на попередній сторінці
\index{iii2}{0084}  %% посилання на сторінку оригінального видання
«і тому цією операцією ви мусити порушити вексельний курс, бо закордонний
борг не оплачено в наслідок того, що ваш експорт не має відповідного імпорту.
— Це правило для всіх країн взагалі».

Лекція Вілсона сходить на те, що всякий експорт без відповідного імпорту
становить одночасно імпорт без відповідного експорту; бо в продукцію товарів,
що їх експортують, ввіходять чужоземні, отже, імпортовані товари. Перед
нами припущення, що всякий такий експорт ґрунтується на неоплаченому
імпорті або породжує його, — отже, породжує борг закордонові, або ґрунтується
на ньому. Це — помилкова річ, навіть, коли не вважати на ті дві обставини,
що 1)~Англія має даремний імпорт, не платячи за нього жодного еквівалента;
напр., частину свого індійського імпорту. Індійський імпорт вона може обмінювати
на американський імпорт, експортуючи останній без еквівалентного імпорту;
щож до вартости, то в усякім разі Англія експортувала тільки те, що їй нічого
не коштувало; 2)~Англія може й оплатила імпорт, напр., американський, що
утворює додатковий капітал; коли вона той імпорт споживає непродуктивно,
напр., на військові припаси, то це не утворює боргу проти Америки та не
впливає на вексельний курс з Америкою. Newmarch суперечить сам собі в
посвідченнях 1934 та 1935, й Wood звертає його увагу на це в 1938: «Коли
жодна частина товарів, ужитих на виготовлення речей, що їх ми вивозимо без
зворотного припливу» [військові видатки] «не походить з тієї країни, куди ці
речі експортуються, то яким способом це впливатиме на вексельний курс з цією
країною? Нехай торговля з Турцією перебуває у звичайному стані рівноваги;
яким способом вивіз військових припасів до Криму вплине на вексельний курс
між Англією та Турцією?» — Тут Newmarch втрачає свою рівновагу, забуваючи,
що саме на це просте питання він дав уже слушну відповідь під № 1934, він
каже: «Ми вже, мені здасться, вичерпали практичне питання, а тепер увіходимо
в дуже високу ділянку метафізичної дискусії».

[Вілсон має ще й інше формулювання того свого твердження, що на вексельний
курс впливає всяке перенесення капіталу з однієї країни до іншої, однаково,
чи відбувається воно у формі благородного металу, чи у формі товарів.
Вілсон, природно, знає, що на вексельний курс впливає рівень проценту, а
саме, відношення чинних норм проценту в тих двох країнах, що їхній взаємний
вексельний курс розглядається. Отже, коли він буде в стані довести, що надмір
капіталу взагалі, отже, передусім надмір товарів всякого роду, в тім і благородного
металу, має разом з іншими обставинами вплив на рівень проценту, визначаючи
його, то він буде уже на крок ближче до своєї мети; перенесення
значної частини цього капіталу з однієї країни до іншої мусить змінити рівень
проценту в обох країнах, і то саме в протилежному напрямку а тому другою
чергою мусить воно змінити й вексельний курс між обома країнами. — \emph{Ф.~Е.}].

В Economist’і, що його він тоді редаґував, за рік 1847, на стор. 475,
він пише:

«Очевидно, що такий надмір капіталу, який виявляється у великих запасах
всякого роду, в тім і благородного металу, неминуче мусить привести не тільки
до низьких цін на товари взагалі, але й до нижчого рівня проценту за ужиток
капіталу1). Коли ми маємо запас товарів, достатній для того, щоб обслужити
потреби країни протягом двох наступних років, то порядкування цими
товарами протягом даного періоду можна здобути за далеко нижчу норму, ніж
тоді, коли того запасу вистачить ледви чи на два місяці2). Всякі позики грошей,
хоч і в якій формі їх робитиметься, являють лише передачу порядкування над
товарами від однієї особи до іншої. Тому, коли товарів є понад міру, грошовий
процент мусить бути низький, а коли товарів обмаль, він мусить бути високий3). Коли
\parbreak{}  %% абзац продовжується на наступній сторінці

\parcont{}
\index{franko}{0085}
капітал через ужитє наємних робітників і одну часть надвишки витворів, грішми чи натурою, платят
дідичови яко ренту ґрунтову. Доки в \RNum{15} віці незалежний мужик, а також сільский наймит, що попри
наймитство й сам про себе веде ґосподарство, збогачуются самі власною працею, доти й обставини тай
обсяг продукційний арендатора остаются дуже скромні. Переворот в рільництві, що почався в послідній
третині \RNum{15} віку і трівав через цілий \RNum{16} вік крім єго послідних десятиліть, збогатив го майже так
само прудко, як прудко зубожив мужиків\footnote{
„Арендаторі“, каже Гаррізен в своїй „Description of England“, „котрим давнійше годі було
заплатити 4\pound{ ф. шт.} ренти, платят тепер по 40, 50 та 100\pound{ ф. шт.} і ще кажут, що їм зле повелося, коли
по упливі арендового контракту не зложили бодай тілько готівки, кілько виносит 6--7-милітна рента“.
}. Загарбанє громадських пасовиск і т. д. дозволяє му
богато побільшувати число худоби майже без ніяких видатків, а між тим худоба достатчувала му далеко
більше обірнику для поправи ґрунту. В \RNum{16} віці причинюєсь ще одна рішучо важна обставина. Тоді
арендові контракти були довгі, нераз де з на 99 літ. А ту в \RNum{16} віці вартість золота та срібла, а
разом з ним і вартість грошей раз~у~раз вменьшуєсь, і арендаторам се принесло золоті плоди. Не
зважаючи на прочі, вперед згадані обставини, арендаторі першим ділом вменьшили робучу плату. Те, що
урвано робітникам на платі, побільшувало
арендовий зиск. А з другого боку ціна збіжя, вовни, мяса, — одим словом, всіх плодів рільничих,
раз~у~раз вбільшуєсь, через що змагаєся грошевий капітал арендатора
без єго причинку, — а притім ще ренту ґрунтову дідичови платит він давними, стратившими на вартости,
грішми. Таким способом він збогачуєсь рівночасно на кошт своїх наймитів і свого дідича. Не диво
% REMOVED (в рукоп. „даво“)
затим, що вже с кінцем \RNum{16} віку витворилась в Англії окрема верства як на тодішні
обставини богатих „капіталістичних“ арендаторів\footnote{
В Франції з „Regisseur-ів“, т. є. панських окономів та тивунів середновікових поробилися швидко
т. зв. hommes d'affaires, т. є. люде, що туманництвом та шахрайством подороблялися капіталів. Такі
окономи, то були нераз великі пани. Як в Англії, так і в Франції великі феодалні добра поділені були
на богато дрібних ґосподарств, але з условинами далеко гіршими для мужиків. В~\RNum{14} віці повстают і ту
аренди, звані ту „fermes“ або „terriers“. Число їх раз~у~раз змагалося і дійшло гет понад
\num{100000}. Вони платили чи то грішми чи натурою ренту ґрунтову, котра виносила від 12-тої до 5-тої
части річного здобутку. Ті terriers були цілими або частковими леннами як до вартости і обєму
ґрунтів, котрі нераз виносили заледво кілька прутів. Всі арендаторі мали до певної степені (степенів
було штири) власть судову над мужиками, жиючими
на їх ґрунтах. Лехко поняти, якого притиску мусів дізнавати люд від
усіх тих дрібних тиранів. Монтейль каже, що тоді було в Франції \num{160000}
судів, де тепер вистарчає (враз із мировими судами) 4000 трибуналів.
}.
\index{franko}{0086}

\subsection{Вліянє рільничого перевороту на промисл.
Промисловий капітал~здобуває собі в краю ринок
відбутовий}

Раптове і частими нападами повторюване вивласнюванє
та прогонюванє мужиків достатчило, як ми бачили,
міському промислови раз~за~разом маси пролєтаріїв, не належачих
зовсім до ніяких цехових звязків, — дуже мудра
подія, про котру старший Андерзен (не треба го мішати
з Джемсом Андерзеном) в своїй історії торговлі каже, що
се прямо боже провидініє так зробило. Ще хвилю мусимо
задержатися над тим складником первісного нагромадженя
капіталів. Не тілько що по селах убуло незалежного, самоґосподаруючого
мужицтва, а по містах прибуло промислового
пролєтаріяту, так, як після Жаффроа Сент-Улєра світової
матерії в одних місцях убуває, між тим коли в других
місцях вона згущаєсь. Помимо меньшого числа оброблюючих
рук ґрунт видавав прото однако або й ще більше
плодів, бо разом с переворотом в ґрунтових відносинах
власностевих настали також ліпші способи управи, більша
кооперація, зосередженє средств продукційних і т. д., а з другого боку
сільські наємники не тілько силувані були до тяжшої
праці — на се головно напирає сер Джемс Стеарт, —
а й обсяг їх домашної продукції, де вони працювали самі
на себе, чим раз більше вменьшувався. З освободженєм
одної части мужицтва освободжені зістали також єго давні
средства прожитку. Вони стают тепер матеріяльним складником
\so{змінного} капіталу\footnote*{
Звісно, що Маркс ділит капітал на постійний (constant) і змінний
(variabel), а то після того, чи в довшім протягу продукції вартість
єго зміняєся, чи ні. І так машини, сирий матеріял, будинки фабричні
і т. д., се капітал постійний, бо продукція не змінює в загальній сумі
єго вартости, а то, що убуде вартости на машинах і приладах і пр.,
котрі зуживаются при роботі, прибуває самим витворам, котрі через переробку
зискуют на вартости. Між тим друга часть капіталу, а іменно
тота, котра йде на наймленє і удержанє робітника і містится в понятю
робучої плати, се капітал змінний, бо по кождім процесі продукційнім
капіталіст добуває з него більше, ніж видав. Робітник витворює вартість
більшу, ніж тота, яку одержав в формі робучої плати. (\emph{Прим. перев.})
}. Бездомний та немаючий мужик
мусит окупувати собі ті средства прожитку від свого
нового пана, промислового капіталіста, в формі робучої
плати. Як зі средствами прожитку, так само сталося й з домашним
рільничим сирим матеріялом, котрого переробкою
займався промисл. Той сирий матеріял став частиною \so{постійного}
\index{franko}{0087}
капіталу. Се бачимо не тілько в Англії. За часів
Фрідріха II бачимо н. пр., що часть вестфальських мужиків,
котрі всі прядут лен, — хоть ще не шовк, — насилу
вивласнено і прогнано з хат і ґрунтів, а прочу часть перемінено
в наймитів великих арендаторів. Рівночасно повстают
великі прядильні і ткальні льну, де „освободжені“ наймаются
на роботу. Лен виглядає так само, як виглядав уперед.
Ані одно волоконце в нім не змінилося, але нова соціяльна
душа вступила в єго тіло. Тепер він становит часть постійного
капіталу панів мануфактуристів. Давнійше розділений
між множество дрібних витвірців, котрі го самі управляли
і пряли, він тепер згромадився в руках одного капіталіста,
котрий других заставляє для себе прясти і ткати. Виложена
в прядильни надвишка праці становила давнійше надвишку
доходу незлічених родин мужицьких, або також, за часів
Фрідріха II, йшла на extra-податки pour le roi de Prusse.
Тепер вона становит зиск немногих капіталістів. Веретена
і ткацькі станки, давнійше розсіяні широко по краю, тепер
стовпилися в кількох великих касарнях робучих, так само
й робітники, так само й сирий матеріял. І веретена і ткацькі
станки і сирі матеріяли зі средств незалежного прожитку
для прядильників і ткачів від тепер перемінюются в средства
командованя над ними і висисаня з них бесплатної
праці. По великих мануфактурах не видно того так, як по
\linebreak[4]
\makebox[\linewidth]{\dotfill}
\centerline{\emph{[На цьому уривається збережений рукопис Франка]}}

  % insert 2 blank pages to align total number of pages to 16
  \newpage\thispagestyle{empty}\mbox{}
  \newpage\thispagestyle{empty}\mbox{}
  \cleardoublepage
  \thispagestyle{empty}
\null\vfill

\noindent{\footnotesize Книжку із задоволенням зверстано в \LaTeX. Перелік використаних пакетів та вихідний код доступний за посиланням \underline{https://github.com/marx-in-ua/das-kapital}. Версія: \input{./common/version.tex}.}

\medskip
\noindent{\footnotesize При наборі використані гарнітури Alegreya ht (2011) та Alegreya Sans ht (2013), які спроектував аргентинський дизайнер Хуан Пабло дел Перал. Математичні знаки набрані гарнітурою STIX2Math, що розроблена в рамках ініціативи Scientific and Technical Information Exchange (STIX) font creation project.}


\cleardoublepage
\end{document}
