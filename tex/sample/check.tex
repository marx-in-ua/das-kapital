\documentclass{kapital}
%% should be a class option
\renewcommand{\parbreak}{\unskip\ignorespaces}
\renewcommand{\parcont}{\unskip\ignorespaces}

%% proper quote marks 
\newunicodechar{„}{«}
\newunicodechar{“}{»}

%% ditto mark
\renewcommand{\dittomark}{~}

%% overfull boxes
\vfuzz=12pt

% Том І. Книга перша
% Процес продукції капіталу
\newcommand{\VolumeNumber}{Том ІІ}
\newcommand{\BookNumber}{Книга друга}
\newcommand{\BookTitle}{Процес циркуляції капіталу}
\newcommand{\BookSource}{Переклад з другого німецького видання}
\newcommand{\BookAuthors}{за редакцією Д. Рабіновіча та С. Трикоза}
\newcommand{\SourceYear}{Харків, 1933}
\newcommand{\Biblio}{Т. І. — Кн. І: Процес продукції капіталу /
Пер. із 4 нім. вид.; За ред. Д. Рабіновича і~С.~Трикоза.}

\newcommand{\VolumeNumberDe}{Erster Band}
\newcommand{\BookNumberDe}{Buch I}
\newcommand{\BookTitleDe}{Der Produktionsprocess des Kapitals}
\newcommand{\AuflageDe}{Vierte, durchgesehene Auflage}
\newcommand{\HerausgegebenDe}{Herausgegeben von Friedrich Engels}
\newcommand{\PublisherDe}{Verlag von Otto Meissner}
\newcommand{\CityDe}{Hamburg}
\newcommand{\YearDe}{1890}
\usepackage{lua-visual-debug}

\begin{document}
  \pagenumbering{roman}
\frontmatter
\thispagestyle{empty}
\null\vspace{0.5cm}
\noindent\htitlespace{} {\large Карл Маркс}

\vspace{1.5cm}
\noindent\htitlespace{} {\HUGE\bfseries\letterspacefont\MakeUppercase Капітал}

\vspace{0.1cm}
\noindent\htitlespace{} {\large\bfseries Критика політичної економії }

\vspace{1.5cm}
\noindent\htitlespace{} {\large Том І. Книга І}

\noindent\htitlespace{} {\large Процес продукції капіталу}

\vfill
\noindent\htitlespace{} {маркс.укр • 2019}
\clearpage
\thispagestyle{empty}
{\footnotesize\noindent{\bfseries{}Маркс, Карл}

Капітал: Критика політичної економії / Заг. ред. А.~Потапова, І.~Зробок, Д.~Потапова. — \City, \Year. — \Biblio —
\lastpageref{pagesLTS.roman} + \lastpageref{pagesLTS.arabic} с.
(Готується до друку).
}
\vfill
\noindent{\footnotesize Друкується за виданням: Карл Маркс. Капітал: Критика політичної економії — \FullSource.}

\bigskip
\noindent{\footnotesize Цей твір ліцензовано на умовах Ліцензіїї Creative Commons Із Зазначенням Авторства — Поширення На Тих Самих Умовах 4.0 Міжнародна. Щоб ознайомитися з копією цієї ліцензії, завітайте на http://creativecommons.org/licenses/by-sa/4.0/ або направте листа за адресою Creative Commons, PO Box 1866, Mountain View, CA 94042, USA.}
\clearpage

\tableofcontents*
\cleardoublepage
\pagestyle{mypage}
\nonumsection{Андрію Річицькому}{}{}

Присвячується Андрію Річицькому (справжнє ім'я — Пісоцький Анатолій Андрійович), видатному українському вченому, перекладачеві та політичному діячу. В 1920-х роках він почав працювати над першим і єдиним українським перекладом «Капіталу» Маркса з німецької, проте встиг завершити лише перший том. Репресований у 1934 році, після чого його ім’я не згадувалося у наступних виданнях «Капіталу» зовсім, попри те, що вони базувалися на його роботі. Був реабілітований у 1990. 

\nonumsection{Подяки}{}{}

\noindent{}Особлива подяка \textbf{Миколі Климчуку} за те,
що видання вийшло охайним і гармонійним. 

\smallskip
\noindent{}Над виданням в рамках спільної ініціативи маркс.укр працювали:
\begin{itemize}[nosep]
\item \textbf{Ірина Зробок}
\item \textbf{Ернест Гук}
\item \textbf{Антон Потапов}
\item Юрий Латыш
\item Taras Bilous
\item Ivanna Kutsil
\item Max Starchevsky
\end{itemize}
\noindent{}а також
Dan Bogynski, 
Dmytro Zhelaha,
Богдан Бернадський,
Nina Garbo, Андрій Андросович, Liza Walther, Сергій Зінченко,
Mike A. Liakh, Stas Sergienko, Volodymyr Boiko, Andriy Panchenkov, Денис Кучеренко, Настя Авдоніна, Oleg Kavaler, Volodymyr Shostak, Ілля Токар, Vika Khomovska, Maxim Sokhatsky, Mariana Potapova, Danylo Yankovskyi, Anna Potasheva, Yaroslav Kovalchuk, Денис Панкратов, Роман Козлов, \textenglish{Yevhenii Mo\-nas\-tyr\-skyi}, Andrew Zukkermann, Олександр Брайко, Anton Potapenko, Oleksandr Lapchuk, Ліда Криштоп, Anton Stepankovsky, Anton Pechenkin, Wowhura Wowhura, Ann Kurovska, Надія Йовченко, \textenglish{Ksena Meyta, Taras Salamaniuk}, Сергей Алушкин, \textenglish{Kirill Kramskiy}, Eugenia Virlich, Valeriy Kuropyatnik, Наталка Чех, Artem Tidva, Snizhana Umanets, Liuba Kuibida, Andriy Pogasiy, Yulia Dukach, Дмитро Вершинін, Леонид Бегунов-Новиков, Галина Новосад, Artem Borysov, Oleksandr Nykolyak, Петро Садовий, Mikhail Khokhlovych, Oleksii Parfeniuk, Hlafira Titarenko, Володимир Гунько, Денис Яшный, Картина Мира, Валентин Германович Дупак, Classic Starr, Olena Martynchuk, Zakhar Popovych, Polina Vlasenko, Vasya Opechenik та Денис Потапов.



\cleardoublepage

  \pagenumbering{arabic}
  \mainmatter

  % your samples here
\chapter{Перетворення зиску в пересічний зиск}

\chaptertwoline{%
Перетворення додаткової вартості в зиск}{і~норми додаткової вартості в норму зиску}{%
Перетворення додаткової вартості в зиск}

\section{Витрати виробництва (kostpreis) і зиск}

\chapteronly{Перетворення товарного капіталу
і грошового капіталу в товарно-торговельний капітал
і в грошево-торговельний капітал
(купецький капітал)}

\addcontentsline{toc}{part}%
  {\protect\partnumberline{\thepart}Перетворення товарного капіталу}%

\addtocontents{toc}{\protect\vspace{-0.7em}}%
\addcontentsline{toc}{part}%
  {\protect\partnumberlineadd{грошового капіталу в товарно-торговельний}}%

\addtocontents{toc}{\protect\vspace{-0.7em}}%
\addcontentsline{toc}{part}%
  {\protect\partnumberlineadd{капітал і в грошево-торговельний капітал}}%

\section{Витрати виробництва (kostpreis) і зиск}

текст
\end{document}
