\documentclass{kapital}
%% should be a class option
\renewcommand{\parbreak}{\unskip\ignorespaces}
\renewcommand{\parcont}{\unskip\ignorespaces}

%% proper quote marks 
\newunicodechar{„}{«}
\newunicodechar{“}{»}

%% ditto mark
\renewcommand{\dittomark}{~}

%% overfull boxes
\vfuzz=12pt
\hfuzz=1pt

\newcommand{\Year}{2023}
% \newcommand{\Year}{2024}
% \newcommand{\Year}{2025}

\newcommand{\City}{Київ}

% Том І. Книга перша
% Процес продукції капіталу
\newcommand{\VolumeNumber}{Том ІІ}
\newcommand{\BookNumber}{Книга друга}
\newcommand{\BookTitle}{Процес циркуляції капіталу}
\newcommand{\BookSource}{Переклад з другого німецького видання}
\newcommand{\BookAuthors}{за редакцією Д. Рабіновіча та С. Трикоза}
\newcommand{\SourceYear}{Харків, 1933}
\newcommand{\Biblio}{Т. І. — Кн. І: Процес продукції капіталу /
Пер. із 4 нім. вид.; За ред. Д. Рабіновича і~С.~Трикоза.}

\newcommand{\VolumeNumberDe}{Erster Band}
\newcommand{\BookNumberDe}{Buch I}
\newcommand{\BookTitleDe}{Der Produktionsprocess des Kapitals}
\newcommand{\AuflageDe}{Vierte, durchgesehene Auflage}
\newcommand{\HerausgegebenDe}{Herausgegeben von Friedrich Engels}
\newcommand{\PublisherDe}{Verlag von Otto Meissner}
\newcommand{\CityDe}{Hamburg}
\newcommand{\YearDe}{1890}
\usepackage{lua-visual-debug}

\begin{document}
  \input{common/front.tex}

  \pagenumbering{arabic}
  \mainmatter

  % your samples here
\chapter{Перетворення зиску в пересічний зиск}

\chaptertwoline{%
Перетворення додаткової вартості в зиск}{і~норми додаткової вартості в норму зиску}{%
Перетворення додаткової вартості в зиск}

\section{Витрати виробництва (kostpreis) і зиск}

\chapteronly{Перетворення товарного капіталу
і грошового капіталу в товарно-торговельний капітал
і в грошево-торговельний капітал
(купецький капітал)}

\addcontentsline{toc}{part}%
  {\protect\partnumberline{\thepart}Перетворення товарного капіталу}%

\addtocontents{toc}{\protect\vspace{-0.7em}}%
\addcontentsline{toc}{part}%
  {\protect\partnumberlineadd{грошового капіталу в товарно-торговельний}}%

\addtocontents{toc}{\protect\vspace{-0.7em}}%
\addcontentsline{toc}{part}%
  {\protect\partnumberlineadd{капітал і в грошево-торговельний капітал}}%

\section{Витрати виробництва (kostpreis) і зиск}

текст
\end{document}
